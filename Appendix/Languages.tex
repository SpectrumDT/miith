
\chapter{Languages}
\target{Languages}
Here follows a brief description of some of the languages spoken in the \Miith{} universe. 
















\section{Etymologies}
Here follow some etymology trees showing the relationships between some of the languages and ethnic groups of \Miith{}. 









\subsection[Reptilian languages]{Reptilian languages}
The three great \caisith language families were \hr{Talaac}{Talaac}, \hr{Saphyr language}{\Saphyr} and \hr{Kush}{Kush}. 

% \Tree 
%   [.{\Draconic}
%     [.{Ancient \scathaese{} languages}
%       [.{\Tassian}
%         [.{\Mastheno}
%           [.{\Ortaican}
%             [.{Imetric} ] 
%             [.{\Ortic} ] 
%             [.{\Tepharin} ] 
%           ] 
%           [.{\Samurin} ] 
%         ] 
%       ]
%       [.{\Mekrii} 
%         [.{\Shurco}
%           [.{Rissitic} ] 
%         ] 
%       ]
%       [.{\Loi}
%       ]
%     ]
%   ]









\subsection{\Human{} languages}
\new
\Tree 
  %[.{\Resphan}
    [.{\Imrathic}
      [.{Ancient Vaimon}
        [.{Modern Vaimon \\(Redcor)} 
          {\Velcadian{}} ] 
        [.{Geican} ] 
      ] 
    ]
  %]
% 









\section{General spelling conventions}
Unless otherwise noted, all \Miithian{} words follow the following spelling conventions. 
Of course, this does not apply to real-world words, nor to any English or French names that appear. 

\begin{pronunciationenvironment}{}
  \pitem{ae}{[eI]}{ej}
  \pitem{\aediphthong}{[aE]}{aj}
  \pitem{ai}{[aI]}{aj}
  \pitem{c}{[k]}{k}
  \pitem{\c c}{[s]}{s}
  \pitem{ch}{[x]}{ch} (like in German)
  \pitem{\c ch}{[S]}{sh}
  \pitem{ph}{[f]}{f}
  \pitem{r}{[\rr]}{r}
  \pitem{y}{[j]}{j}
  \pitem{\yvowel}{[y]}{y}
  \pitem{\ydiphthong}{[aI]}{aj}
  \pitem{\zh}{[Z]}{zh}
\end{pronunciationenvironment}

A final \pex{e} is always silent \emph{unless} it is part of a diphthong or has a diacritic over it. 
For example, the \pex{e} in \pex{\scathaese} is silent.
The \pex{e} in \pex{\scathae} is \emph{not} silent because it is part of the diphthong \pex{ae}.
The \pex{\"e} in \pex{\Samure} is \emph{not} silent because it has a diacritic over it.









\subsection{Diacritic marks}
% A diaeresis over a letter (as in \"e or \"i) is used to signify that the letter in question should be pronounced as an independent vowel and syllable (not as part of a diphthong, nor silent). 
% (The \yvowel{} in the table above is a special case of this; the diaeresis distinguishes vowel Y from consonantal Y.) 
A diaeresis over \pex{\"i} or \pex{\"e} is used to is used to signify that the letter in question should be pronounced as an independent vowel and syllable (not as part of a diphthong, nor silent). 

Over other vowels (\pex{\"a}, \pex{\"o}, \pex{\"y}), the diaeresis acts instead as an \emph{umlaut}, changing the sound of the vowel. 
See the pronunciation tables. 















\section{Individual languages}









\subsection{\Durcaci}
\target{Rissitic language}
\target{Drukari language}
\index{\Durcaci language}
The dominant language of \Durcac. 
Descended from \hs{Kush}. 
Has its own unique writing system of glyphs. 
Also called \emph{Rissitic}. 

The \Durcaci language family also comprises a variety of related but distinct languages that were spoken by various tribes and nations in \Durcac. 
Most peoples in the region spoke \Durcaci languages, but some spoke other, unrelated languages. 

\Durcaci lacks the phoneme \txipa{[x]} (having only \txipa{[\c c]}). 
It might also lack \txipa{[l]}. 





\meta{%
  Rissitic is a rough, guttural language inspired by Danish, German and Dutch. 
}





\subsubsection{Writing}
Rissitic \hs{glyphs} resemble Egyptian hieroglyphs. 









\subsection{Imetric}
\target{Imetric language}
The language of the Imetrium. 
Descended from \Ortaican and written using the \Ortaican alphabet. 

Imetric lacks the phonemes \txipa{[b]} (approximated by \txipa{[v]}), \txipa{[T]}, \txipa{[x]}, \txipa{[\c c]}, \txipa{[S]}, \txipa{[z]} and \txipa{[Z]}. 




\meta{%
  Meant to resemble Latin or Greek.} 





\begin{pronunciationenvironment}{\subsubsection{Phonemes of note}}
  \pitem{a}{[a]}{ah}
  \pitem{r}{[\rollr]}{r}
\end{pronunciationenvironment}





\subsubsection{Writing}
\target{Ortaican letters}
Imetric is written with {\Ortaican{} letters}. 

\Ortaican{} letters resemble Greek or Cyrilic letters. 









\subsection{\Ortaican}
\target{Ortaican language}
A \scathaese language and language family. 
The ancestor of \hr{Tepharin language}{\Tepharin} and \hr{Imetric language}{Imetric}. 
Descended from \hr{Saphyr language}{\Saphyr}. 









\subsection{\Resphan}
\target{Resphan language}
The common tongue spoken by most \resphain{} (but not all). 

\Mystraacht, \CiriathSepher, \TiphredSerah{} and \Kezerad{} all speak the same language, but in different dialects. 
So do some \Baelzerach, but some of them speak completely different languages. 

The \Resphan{} tongue forms the plural of nouns either by adding an ending (e.g.: \ghobal{} $\mapsto$ \ghobaleth) or by changing the last syllable (e.g.: \resphan{} $\mapsto$ \resphain, \sathariah{} $\mapsto$ \satharioth). 

The phoneme \txipa{[T]} is common. 
The phoneme \txipa{[p]} does not exist. 




\meta{%
  Inspired by Hebrew, but only superficially (since I don't actually \emph{know} any Hebrew).} 





\begin{pronunciationenvironment}{\subsubsection{Phonemes of note}}
  \pitem{\adarkresphan}{[a]}{ah}
  \pitem{\ahresphan}   {[a:]}{aa}
  \pitem{\aflatresphan}{[\ae]}{\aumlaut}
  \pitem{r}{[\rollr]}{r}
\end{pronunciationenvironment}





\subsubsection{Dialects}
The \Mystraacht{} dialect differs from the above. 
It has retroflex on R and L. 

The \CiriathSepher{} sometimes add an extra demonstrative \quo{roll} to the R's and make them extra long, to clarify to everyone that they identify as \CiriathSepher. 





\subsubsection{Writing}
\Resphan{} script is a complex system of ideograms. 
In form they resemble Japanese hiragana. 









\subsection{\Saphyr}
\target{Saphyr language}
\index{\Saphyr}
A \caisith language. 
Spoke in \hr{Saphyrae}{\Saphyrae}. 









\subsection{Talaac (\Draconic)}
\target{Draconic language}
\target{Talaac}
\index{Talaac}
The language spoken by most \dragons and many \ophidians. 

Talaazu was a geographical region spanning many caisith kingdoms and city-states, mostly dominated by caisith of Talaac ethnicity. 

The Talaac were a Caisith ethnic group that spoke variants of the Talaac language and shared many cultural traits. 




\meta{%
  Meant to resemble the \quo{Enochian} language used in astrologer John Dee's writing.} 





\subsubsection{Classical Talaac}
\target{Classical Talaac}
\index{Classical Talaac}
\index{Talaac!Classical Talaac}
Classical Talaac was the language originally spoken by \hr{Sethicus}.
Millennia later, a similar language was still spoken by many \ophidians. 
This \quo{official} dialect of the language was still called Classical Talaac (to differentiate it from \hr{Mystic Talaac} and \hr{Vulgar Talaac}).





\subsubsection{Mystic Talaac}
\target{True Draconic}
\target{Mystic Talaac}
\index{\TrueDraconic}
\index{Mystic Talaac}
\index{Talaac!Mystic Talaac}
The formal, magical language of the \dragons and \ophidians.
Also called \emph{\TrueDraconic}. 
It was derived from \emph{Classical Talaac}, the language originally spoken by \hr{Sethicus}. 

Mystic Talaac was based on \hr{Ophidians discover spellwords}{a system of spellwords discovered by the \ophidians}. 

Mystic Talaac is an extremely complex language to learn, as the grammar is very cryptic. 
Even for \dragons{}, whose brains are superior to those of humanoids and pick up languages easier, it takes hundreds of years to learn correct Mystic Talaac. 
A good command of the language is a sign of status among \dragons{}. 

\target{True Draconic costs sanity}
Speaking \TrueDraconic was a sorcerous thing. 
Every word was a spell and a deep, possibly traumatic, cosmos-touching experience.
It was taxing and drained the sanity of mortals. 
Even formidable mortals like \Criseis only spoke \TrueDraconic when they must.

\meta{%
  I have used archaic English (Early Modern English) to represent \TrueDraconic.
  
  Note that Early Modern English had a convention that \quo{thou} was a familiar form and \quo{you} was a polite form.
  I will not be using this convention.
  In all uses of Early Modern English, \quo{thou/thee} is simply singular and \quo{ye/you} is simply plural.} 





\subsubsection{\CommonDraconic}
\target{Vulgar Talaac}
\emph{Vulgar Talaac}, also called \emph{\CommonDraconic}, was the collective term for various simplified, degenerated dialects of Talaac. 
Many mortal humanoids spoke it.
There were many variants of Vulgar Talaac, not all mutually intelligible. 

The full name for the language is \emph{Gemaltha N\^udrach Ruishaan} meaning \quo{Draconic daily speech}.
Its common name is \emph{Ruishaan} meaning \quo{daily}.





\subsubsection{Runes}
\target{Draconic runes}
\Draconic and the \quiljaaran languages were written with \quo{\ophidian runes}, also called \quo{draconic runes}.
They were a degenerate form of the ancient \ophidian written language.

\Sethicus had helped develop the runes. 
There was much potential for magic inherent in the runes. 
They were designed in accordance with \hs{occult geometry}. 





\subsubsection{Inspiration}
For inspiration for \Draconic words, I can look at the article on the \emph{World of Warcraft Wiki} about the Draconic language used in the \emph{Warcraft} world \cite{VideoGame:Warcraft}: 

\href{http://www.wowwiki.com/Draconic}{http://www.wowwiki.com/Draconic}









\subsection{\Tepharin}
\target{Tepharin language}
The language of old \Tephar, still spoken in much of southern \Velcad. 
Descended from Ortaican, but with much influence from \human{} tongues. 
Can be written with Ortaican or Vaimon letters. 

\Tepharin, and \Ortaican, lack the phoneme \txipa{[v]}. 
They have only \txipa{[w]} and \txipa{[hw]}. 

\Tepharin{} lacks \txipa{[z]} and \txipa{[S]}.
But it does have \txipa{[D]} and \txipa{[3]} (soft G, or 3). 





\meta{%
  \Tepharin{} is meant to resemble a mix of Latin and Celtic languages.} 





\subsubsection{Writing}
\Tepharin{} is written with \hr{Ortaican letters}{\Ortaican{} letters}. 









\subsection{Vaimon (archaic)}
\target{Archaic Vaimon language}
\target{Archaic Vaimon}
The language of the ancient \hr{Vaimon Caliphate}{\VaimonCaliphate}. 




\meta{%
  Like \Resphan, it is inspired (to some extent) by Hebrew.} 





\subsubsection{Numerology}
The Vaimon alphabet is tied to \hs{Vaimon numerology}. 





\subsubsection{Writing}
Vaimon script is flowing and cursive, resembling tengwar. 









\subsection{Vaimon (modern)}
\target{Vaimon language}
\target{Redcor language}
The modern dialect of the Vaimon language, as spoken by the Redcor. 

Most words are stressed on the \emph{last} syllable. 





%\subsubsection{Phonemes of note}
\begin{pronunciationenvironment}{\subsubsection{Phonemes of note}}
  \pitem{a}{[\ae]}{\aumlaut}
  \pitem{\^a}{[a]}{aa}
  \pitem{e}{[3]}{\oe{}} (Note: Silent if final.)
  \pitem{\'e}{[e]}{\'e}
  \pitem{\`e}{[E]}{e}
  \pitem{r}{[\gr]}{rh}
  \pitem{y}{[y]}{y} (as in German)
\end{pronunciationenvironment}



\meta{%
  Inspired by French.}











\subsection{\Velcadian}
\target{Velcadian language}
Once the official language of \hr{Great Velcad}{\theBelkadianEmpire} and still spoken in large parts of \Velcad. 
Written using the Vaimon alphabet. 





\meta{%
  Meant to resemble Celtic languages.} 











\subsection{Other languages}
Other languages included:
\begin{itemize}
  \item \hr{Pelidorian language}{Pelidorian}.
  \item \hr{Rungeran language}{Rungeran}.
  \item \hr{Vidran language}{\Vidran}.
\end{itemize}










