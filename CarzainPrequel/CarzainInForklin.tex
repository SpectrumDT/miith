\bookchapter{\Forklin}

\begin{comment}
\section{\Forklin}
\end{comment}

\begin{comment}
\subsection{The \Wylde{}}
\end{comment}

\stamp
  {\dateCarzainFirstInForklin}
  {Pelidorian army, west of \Forklin}

Carzain's journey from \Redglen{} to Malcur had been dull, dominated by dark forests. 
It had also been a hasty ride, and as an inexperienced \relcer{} he had been too busy managing the \relc, Arrow, to pay much attention to the land around him. 

The trip from Malcur to \Forklin{} was different. 
They had to maintain a slow walk so the infantry could keep up, which gave Carzain ample opportunity to admire his surroundings. 
And what surroundings. 
%Carzain remembers the great, majestic mountains, valleys and rivers that they passed on their way. 
They were on a well-kept, civilized path, protected from the dangers of the \Wylde{}. 
But even so, the \Wylde{} was overwhelming in its primal splendour. 
Immense mountains loomed around and above them, like the thrones of ancient gods, as if to remind the puny humanoids that for all their swords and armies and banners, they were still insignificant vermin. 
%, and that even the greatest armies they could field were still nothing. 
And there were green valleys and forests, so vivid, so full of life. 
And there were rivers, flowing with such force, such ferocity, such fury. 

Herds of giant \saurians{} grazed in the distance, of which the most imposing were the rare but humongous \tondras{}: 
Massive beasts, each perhaps as heavy as a thousand men, with barrel-shaped bodies, four pillar-shaped legs and a long, snaky neck and tail. 
They were \Wylde{} now, but once, thousands of years ago, the Vaimons and other great empires of the past had tamed these behemoths, harnessed their colossal strength... and reshaped the world. 

Great birds and pterans dominated the skies above. 
Once he saw a \quilrai, described in books as the world's largest pteran. 
Its feather-clad central body was little larger than that of a man, but 
its beaked and crest-crowned head was perhaps six feet long. 
%, yet its feather-clad central body was little larger than that of a man. 
Clad in white and azure, it soared on huge wings spread forty feet wide or more. 

Everything around them radiated size, power and beauty. 

After a long ride (or march, for the infantry) through the countryside they were now nearing the city of \Forklin{}. 
The city's castles and towers were visible in the distance. % and steadily growing. 
\Forklin{} was one of the largest and mightiest cities in Pelidor, rivalling Malcur in splendour. 
Indeed\dash Carzain recalled from the history lessons that his mother had insisted was an integral part of a Vaimon education\dash before the rise of the unified duchy of Pelidor, \Forklin{} had been the capital of an independent city-state, the equal rival of the city-state of Malcur. 
Both were conquered by the then-fledgling \GreatBelkade, and the Pelidor clan, who were awarded dukeship of the region, chose Malcur as their capital. Later, after the collapse of \theBelkadianEmpire, the dukes were quick to consolidate their power, and the region became the nation of Pelidor. 

Carzain's knowledge went no farther back in time than that, but he seemed to notice that, unlike Malcur, the city did not look Vaimon-built. 
Sure, those statues of angels and knights looked like Vaimon architecture. 
But the colours were different. 
Where Malcur was dominated by bright colours\dash white, ochre, yellow and sky blue\dash the walls and towers of \Forklin{} were much darker, painted in twilight gray and sapphire blue. 
As they drew closer, he noticed more differences. 
He could see battlements adorning every yard of the city wall. 
The tops of the towers were jagged as if beset with horns or teeth. 
Many of the statues were set in intimidating poses, weapons raised aloft and stern faces staring down, ready to pass judgement. 

\tho{Who built this city? 
     I've probably heard it, but I can't remember.}

Regardless, all in all, \Forklin{} conveyed an impression of fierce martial might; so strong, cold and hard that to Carzain's eyes it made regal Malcur look almost foppish in comparison. 

\tho{Oh, yeah. This is a city for men.}









\begin{comment}
\subsection{Moral support}
\end{comment}

\new
\placestamp{\Ishrah{} quarters, \Forklin}

The Pelidorian army was to stay in \Forklin{} the rest of the day and night to rest and resupply, setting out next morning. 
Despite the city's stark, martial exterior, Carzain was pleasantly surprised by the \ishrah{} quarters in \Forklin, which sported several soft beds and a large furry couch. 

\ta{Nice.} He threw himself down on the couch. 
\ta{What do you think this is?} he asked into the room, fingering the long black fur. 

\ta{Probably mammoth,} said \Sanyor, seating himself in the other end. 
\ta{From up north, maybe Belek.} 
\Sanyor, a \dax{} and one of the senior \ishrah{} members, was a Vaimon of Clan \Telcra. 
His bright purple scales struck a jarring contrast, in Carzain's eyes, to the drab and utilitarian multi-pocketed leather armour he always seemed to wear. 

Apart from Carzain, Curwen and \Sanyor{} there were three more \ishrah{} mages present, three \sphyles: 
Hicarro and her young apprentice, Fiorae, plus \ps{\Sanyor} apprentice, Thedoro. 
% Both were also \Telcras. 

Carzain had been told the \ishrah{} had three more members. 
There was Moro \Cornel, who had stayed behind to guard Malcur. 
She was the academic leader of the \ishrah\dash and the strongest, according to some\dash but it was Archibald Curwen who commanded the mages in battle. 
Another member, \Ambrose{} \Onatol{}, had been found guilty of treason and executed shortly before Carzain joined. 
He had an apprentice, Baernor, who was under suspicion and was being detained in Malcur. 

The six mages with the army were all Vaimons. 
This pleased Captain Curwen, because it made concerted efforts much easier. 
\Cornel{} and \Onatol{}, on the other hand, were \Rethyax{}, practitioners of Chaos magic. 
Evil magic. 
As \quo{black} as \Itzach{} may be, any Vaimon would agree that Chaos was blacker still. 
\tho{%
  We rogue Vaimons consort with dangerous things, but at least we don't make deals with \daemons. 
  Nothing good has ever come of that.}

The door swung open and Curwen strode in. 

\ta{Attention, gents!} barked Curwen. 
Some of the \sphyles{} sneered slightly at being labelled \quo{gents}. 
\tho{Could be worse,} Carzain thought. 
\tho{Could have been \quo{boys}.}

Curwen moved out of the doorway to admit four \humans: 
A middle-aged woman, a young woman and two men. 
\ta{%
  Clan Redcor has seen fit to send us a number of ambassadors to provide moral support,} he announced. 
\ta{%
  You may or may not end up working with them. 
  So meet them.} 

Carzain kept studying the younger woman. 
\tho{%
  That girl looks familiar. 
  Have I seen her before? 
  Or does she just look like someone I know?} 
She was around his age, taller than many women but\dash from what he had heard\dash not as tall as many Redcor women. 
Her chestnut hair was bound in a single braid, and her long brown-and-red dress was only moderately adorned. 
\tho{Probably low in the ranks, then.} 

Curwen addressed the Redcor. 
\ta{This is \Sanyor. Hicorra. Fiorae. Carzain \Shireyo.} 
Carzain thought he saw the girl's eyebrows twitch at the mention of his name. 

The older woman stepped forward to let her presence fill the room. 
Holding her head high and looking down her nose at them all, she seemed to instantly and quietly take control of the room, eclipsing Curwen as the officer in charge. 
Carzain's reaction was a mix of dislike and grudging respect. 

\ta{%
  I am \Chyrie{} \Esmerel{}, \Matron{} of \theSwanFaction{} of Clan Redcor,} 
the woman said in a melodious but stern alto. 
\tho{Ooo. \Matron. High rank.} 
\Mrs{} \Esmerel{} was nearly as tall as Carzain or Archibald Curwen, and her high heels and hairdo added several inches to that. 
She looked to be around fifty, but, being a Vaimon, she might be significantly older. 
Her fiery orange hair was held up by a bronze diadem and a multitude of pins, each adorned with a pearl or other gem. 
Her high-necked dress was white and beige, stylishly slashed with lines of yellow, and she wore necklaces, rings and embroidered gems aplenty, as befitting her rank. 
She ran her eyes over the \ishrah, taking in each of them and sizing them up, before nodding to the younger woman. 

\ta{%
  I am \Racel{} Galisetti, \soror{} of \theSwanFaction{} of Clan Redcor.} 

\tho{\Racel{} Galisetti.}

It took a second before that registered with Carzain. 

\tho{Wait, what?} 

One of the men introduced himself: 
\ta{\France{} \Perival{}, \gandierre, \Swan.} 
Carzain only heard him with half an ear. 

\tho{\Racel?} 

He studied the girl's face, hair hair, her eyes, her body. 
She noticed his staring and sent him an annoyed glare, but that did not stop him. 

\ta{\Isacc{} \Chiran{} of the \gandierre, of \theSwanFaction,} 
he dimly heard the other man say. 

\tho{By the \Sephiroth, it really is her!} 

\Racel{} Galisetti had been a childhood friend of his from \Redglen. 
The Galisettis, well-off merchants, had been friends of his parents, and it had been natural for their children to play together. 
The Galisettis had Redcor blood, if thin, and when a visiting Redcor discovered that their daughter had talent as a Vaimon, she was quickly whisked away to the \TopazChateau{} in \Redce{} for education\dash a great honour for her family. 
The Redcor had been eager to take Carzain as well, him being another talented Vaimon of Redcor descent, but his parents had declined. 

He and \Racel{} had been eleven or twelve, an age where girls still had cooties as far as he was concerned. 
As such, he had always thought of her as a somewhat annoying little girl. 
But now she was back. 
Twenty-three years old, she must be. 

\tho{All grown up, with curves in all the right places.}

He ran his eyes up and down her again. 

\tho{%
  No, wait. 

  Not really.} 

The dress was fairly impressive, and she exuded a certain aura of Redcor \emph{\noblesse}, but now that he got a closer look, the \quo{right places} actually left a bit to be desired when it came to curves. 
Compared to the stories of the young hero meeting his old childhood friend and finding her \quo{all grown up}, this reunion was proving somewhat anticlimactic. 
%\tho{Not really.} 
\tho{%
  I mean, she's pretty enough, I guess. 
  But I've seen better. 
  Heh. 
  I've slept with better.

  Besides, what am I thinking? 
  It's \Racel.} 
She was more like a sister to him\dash an annoying, nagging sister\dash than a \quo{real} woman. 
He laughed inwardly. 
\tho{She had cooties when I last saw her, and for all I know, still does.} 

In the meantime, with the introductions over, \Racel{} had approached him. 
\ta{Hello, Carzain.} 

\ta{Hello, \Racel.} 

\ta{You know, it's rude to stare like that.} 
She gave him a stern, lecturing look. 

Carzain almost fell for the look and mumbled apologetic explanation. 
\tho{\Qliphoth. She's got the Redcor gaze already.} 
At the last minute he pulled himself together and put up a grin: 
\ta{Sorry about that. 
Tell you what: 
To make amends I'll let you stare at me as long as you need to convince yourself that it's really me.} 

She smiled a measured smile. 
\ta{%
  %Still the naughty little boy. 
  You haven't grown, have you?} 

\tho{%
  Nah, I don't know. 
  I've known a few women who would dispute that...} 
% \tho{Except the \quo{naughty} part,}
% he almost added. 
% \tho{But no, that would be tacky.
% 
% Mustn't muddy down my style.} 

\ta{Racel,} said an imperious voice. 

It was \Esmerel. 
At this close range, the broad-skirted gown together with her height and commanding demeanour made her seem to fill the entire horizon. 
Before he could stop himself, Carzain's eyes widened in awe. 

\ta{\Matron{} \Esmerel,} said \Racel, politely bowing her head. 

\ta{Do you know this man?} \Esmerel{} demanded. 

\ta{Yes, \matron. We... grew up in the same town.} 

\ta{Did you now?} \Esmerel{} turned to him. 
\ta{\Mr{} Carzain \Shireyo?} 

\ta{Yes. Uh...} 
He searched his memory for those bits of Redcor etiquette his mother had taught him. 
\ta{%
  Carzain \Deracille{} \Shireyo. 
  How do you do, \Matron{} \Esmerel,} 
he said in the Vaimon language, doing a small bow with (he thought) all the correct flourishes. 

\ta{\Deracille?} said \Esmerel. 

\ta{%
  Yes, \matron. 
  My mother is \Roanne{} \Deracille, born to \theTulipFaction{} of Clan Redcor.} 

\ta{Hmmm. And you are a Vaimon?} 

\ta{Yes. Proficient in \Iquin{} and \Itzach{} alike.}

This produced in \Esmerel{} a frown that was ever so subtle and yet ever so disapproving. 
\ta{I see.} 

With a slight nod she dismissed him and turned away. 

It was not until after she walked away and Carzain had gradually gotten his own world back that he realized how much her presence had affected him. 
\tho{%
  Whoa. 
  And I thought \emph{\Racel{}} had the Redcor stare.} 
Thinking back now, he felt embarrassed by his own display. 
\tho{%
  She had me bowing and \quo{\matron}-ing and everything. 
  And that disapproving look at the end really got to me! 
  \Itzach, how did that happen?} 

Carzain fancied himself a strong-willed independent soul who would not be cowed by any authority. 
As such, being cowed by an authority was an unpleasant experience. 
He found both his dislike of \Chyrie{} \Esmerel{} and his grudging respect for her grow, and he was not altogether comfortable with the latter. 
Part of him wanted to vicously dismiss her as a stuck-up, tyrannical hag. 
But another part of him remembered the stories of the mighty Vaimons and their majesty, and that part recognized a sample of such Vaimon majesty when he saw it. 

He had grown up with a vague dislike for Clan Redcor. 
His mother had renounced her membership of the clan, and his father was Geican, age-old rivals of the Redcor. 
But, now that he thought about it, he realized that he did not really have anything on which to base it. 
His parents had always been reluctant to discuss the issue, and his mother had always dodged specific questions about her split with the clan. 
He had no good reason to dislike them. 

And as much as he hated to admit it, there was something fascinating about this \Chyrie{} \Esmerel. 









\begin{comment}
\subsection{Curwen}
\end{comment}

\new
\tho{\quo{Moral support}.} 

Archibald Curwen chewed on the word. 

\tho{Right. 
    Meaning you won't actually get off your asses to help us, but you want to get us indebted to you. 
    Meaning you're here to lobby your Redcor morals on us. 
    
    And to snoop around and spy on us. 
    
    I don't like those leading questions \Esmerel{} was asking me. 
    She's looking for something, but I can't figure out what.} 
Well, Curwen knew how to be evasive if asked inconvenient questions. 
He was not going to let anything slip that he didn't want the Redcor to know. 

\tho{I don't know this \Esmerel, but I don't like her.

    Spying on me, are you? 
    Fucking Redcor. 
    Well, two can play that game. 
    I'll ferret out what you're after.}







\begin{comment}
\subsection{\Racel}
\end{comment}

\begin{comment}
Carzain talks to \Racel. 
It turns out that \Racel{} was sent on this mission because she is Pelidorian. 

She dislikes his friends, calling them uncouth. 
Carzain laughs. 
\ta{This isn't the \TopazChateau. 
This is an army. 
Those guys are soldiers. 
Their purpose is to kill people. 
What did you expect?}
\end{comment}









\begin{comment}
\subsection{Practice}
\end{comment}
\placestamp{Army barracks, \Forklin}

\new
Carzain dodged a swipe and lashed out with his wooden sword. 
\Tsekkect{} dashed the sword away with her dirk, then performed some impossible spin and landed a savage roundhouse kick against Carzain's right flank. 

\ta{Oomph...} he groaned as all air was pushed out of his lungs. 
\Tsekkect{} did a hop and kicked him in the back. 
He fell flat on his face. 

\ta{Ow. All right, \Tsekkect. You win, you win.} 
Carzain climbed to his feet. 
%\ta{You fight really dirty, you know that?} 
\ta{I just don't get the way you fight.}

\ta{You try to make it sound like there's something wrong with me.} 
She put up a offended grimace which quickly melted away to a broad, toothy smile. 
% She gave him a broad, toothy smile. 
\ta{That's just \human{} provincialism. 
I fight like a Meccaran! There's a difference.} 

There was indeed a difference, Carzain admitted. 
His training and the books he had read had prepared him for fighting a \human{} or \scatha{} wielding a sword in his hands or the like. 
You would strike, parry, feint and get a blow in past the other fighter's parades. 

He was evenly matched with Delph. 
Carzain had a better reach, being taller, and more refined technique. 
Delph was not strong, but he was fast and very agile. 
And he fought dirty. 
He would use his fists and knees and even throw sand in his opponent's face. 
Still, he was a normal \human{}. 

\Tsekkect{} did not play by those rules. 
%And with good reason. 
She did not fence with a sword, but wielded only a slim dirk. 
Her main weapons were her feet. 
A \ps{\meccaran} arms were shorter and weaker than those of a \scatha{} or \human{}, but her legs were long, strong and dextrous. 
When he swung hight she would use her strong knees to crouch down and then trip him or stab him with the dirk. 
When he lunged at her she would leap aside and kick him in the side. 
She could even roll and stand on her hands while kicking with both legs. 

Carzain could handle Delph and won around half of their matches. 
But he could not beat the \meccaran. 
His training had not prepared him for this. 
% Granted, the Vaimons of legend had fought monsters and things

\ta{I think I've had enough fighting for today,} he said. 

\ta{What?} said Delph. 
\ta{It's not even late.} 

\ta{Well, I plan to go into the city and see if I can find myself a woman.} 

\ta{\quo{Find}? Carzain, let me let you in on a little secret.} 
Delph leaned in and lowered his voice. 
\ta{You know those reddish tents in the eastern corner of the army camp?}
he asked in a conspiratorial tone. 
\ta{%
  I have it on good authority that you can find \shout{whores} in there!} 
He gaped, feigning shocked delight. 

\ta{Yes, thanks. 
  But I prefer a \quo{real} woman.} 

\ta{What are you talking about?
  These girls are as real as they get. 
  They know the all tricks of the trade. 
  And I'm telling you, 'cause I know.}

\ta{It's just... the concept of going to a whore is repulsive to me. 
  For a woman, to get to sleep with me is supposed to be a privilege, an honour and a pleasure. 
  Not a chore to charge money for.
  In fact, if anything, the girl should be paying \emph{me}.}

Delph turned his eyes to the sky. 
\ta{You know... I've never thought of it that way. 
  %And why should I? 
  And I never will again! 
  Ha! 
  Oh, well. 
  Suit yourself. 
  More whores for me, then.} 

\ta{Yes. 
  Enjoy them.}

\ta{Oh, you fucking bet I will,} Delph laughed. 

\ta{Well...} Carzain rose. 
\ta{I will see you tomorrow.} 

And so Carzain went out into the city. 
Hunting. 









\begin{comment}
\section{Pretty Sally}
\end{comment}

\begin{comment}
\subsection{War room}
\end{comment}

\stamp
  {\dateCarzainVolunteers}
  {War room, \Forklin}

% Curwen is there, as are \Sethgal{} and \Dornaer{} and some more. 
% They are in a nice tower in the castle. 
% They plan to go to the eastern Pelidorian city of Dendrum. 
% They will park the army there and reinforce the city, holding it as a bulwark against the Rungerans. 
% They should be able to make it in time, but they are not sure. 
Marshal \Sethgal, \Dornaer, Curwen and other officers stood around a map spread over a large table in a room in Castle \Forklin. 

\ta{From what our intelligence tells us, they are on the Ucarn Road, somewhere near... here,}
said \Sethgal, pointing at the map. 
\ta{%
  The nearest major city is Dendrum. 
  My plan is to march to Dendrum and reinforce it. 
  With the army and the \ishrah{} we should be able to hold it against the Rungerans.}

\ta{But can we make it to Dendrum before them?} asked \Dornaer.
\ta{When you say \quo{somewhere near here}, you assume that means \emph{here}.} 
She pointed to a spot near the Nerim river at the eastern border of Pelidor. 
\ta{But it might just as well be \emph{here}.}
She pointed halfway between the Nerim and Dendrum. 
\ta{Our intelligence is not new nor precise enough to be sure. 
    We might arrive to find a city besieged... or taken.}

\ta{Do you have a better plan?} asked \Sethgal. 

\ta{Not yet. But we need do to more reconnaissance.}

% It is discussed that they should send someone ahead to scout on the Rungerans and report on their progress. 
% Curwen quickly volunteers for this task. 
% It is his idea to take the Rangers and sneak through the \Wylde{}. 

\ta{I agree,} said Curwen. 
\ta{Give me some \hr{Ranger}{\rangers} and a small contingent of troops.
    I will take the path through the \Wylde{}, through Kenshaer. 
    I can circle the Rungerans and spy on them.}

\ta{You want to outrun them by going through the \emph{\Wylde{}}?} asked another officer. 
\ta{That makes no sense!}

Curwen sighed. 
\ta{Trust me. I know magic.} 

% Someone else: Is it safe?
% Curwen: No. Little is safe in war. 
% Sethgal: How many men will you need, Captain?
% Curwen: A dozen at most. All mounted. 
% Sethgal: Very well. Pick your group and go. Light shine on you, Captain Curwen. 
% Curwen: It had better not. I'm going for stealth. 
% 
% His irreverence produces some frowns. 
% Curwen doesn't care. 
% He just goes. 

\ta{Is it safe to go through the \Wylde{}?} asked \Dornaer. 

\ta{No,} said Curwen. 
\ta{It's war. 
    It's not supposed to be safe.}

\ta{Can you do this, Curwen? 
    Circle the Rungerans, scout and return before they catch you?} asked \Sethgal. 

\ta{Yes,} said Curwen.
\tho{What a stupid question. What answer did you expect?}

\ta{How many men do you need?} 

\ta{A dozen at most.}

\Sethgal{} hesitated for a moment. 
\ta{Very well. 
    Pick your group and go. 
    Light shine on you, Captain Curwen.}

\ta{It had better not. 
    I'm going for stealth here.}
His religious irreverence brought scowls from some of the assembled as he rose to go. 
Curwen did not care. 
He had more important things to consider.  

% \tho{What are the Rungerans up to?}









\begin{comment}
\subsection{Lighten up}
\end{comment}

% \stamp
%   {\dateCarzainVolunteers}
%   {Pelidorian army camp\\Early morning}
\placestamp
  {Army barracks, \Forklin}

\ta{Come on, fancy-ass, lighten up,}
\Tsekkect{} was telling him. 
\ta{What's there to pout about?
    The Sun is shining and we're marching off to get ourselves killed.
    It's gonna be great.}

\ta{Yes. 
    You know, \Tsekkect, 
    I wish you'd stop calling me \quo{fancy-ass} already.}

\ta{Yeah, you'd wish a lot of things, wouldn't you?} 
she said with her usual foot-wide toothy smile. 

\ta{True,} he had to agree. 
% The \meccaranz{} rudeness was somewhat annoying, but also somehow endearing. 
\ta{Tell me, do you \meccara{} all call each other names like that?}

\ta{Stop saying \quo{you \meccara} like we're all the same, damn you. 
    You \scathae{} and \humans{} think you're the only races in the world with cultures, and all other races are just big crowds of identical people. 
    We come in different fucking kinds, too! 
    For instance, we \Thbatswa{} don't mix with the Ptatchukk. 
    'Cause they smell bad and talk ugly. 
    Also, they steal.}

\tho{Whoa. Talk about going off on a tangent,} 
Carzain thought. 
\ta{My. 
    What an enlightening and insightful monologue. 
    I now repent my wicked prejudice.} 

\ta{Yeah, you should.} 
They both laughed out loud. 

Annoying as the \ps{\meccaran} rudeness might be, it was also refreshing. 
Carzain had been in a bit of a sulky mood all morning. 
%ever since \Forklin. 
His efforts last night to find and seduce a woman in the city had failed. 
\tho{Another reminder that I'm fallible like any other man. 
     I always hate those.}

Just then, a messenger showed up. 
\ta{\Mister{} \Shireyo?} 

\ta{That's me,} said Carzain.

\ta{Captain Curwen wants you.} 





\begin{comment}
\subsection{Volunteered}
\end{comment}

\new
\placestamp{Archibald Curwen's tent}

\ta{Congratulations, \Shireyo,} 
said Curwen, sitting in his chair and puffing on his pipe. 

\ta{Thanks, Captain. What for?} 
said Carzain. 

\ta{You've just volunteered for a sally.} 
Curwen pointed at Carzain with the pipe's mouthpiece for emphasis. 

\ta{I have? 
    How gracious of me.} 

\ta{Yes. 
    I volunteered you.} 

\ta{How gracious of you, then, Captain. 
    So, who is this Sally? 
    She pretty?} 
Carzain asked with a grin. 

\ta{If you like the sight of dark \Wylde{}, 
    ravaged towns and possible fighting, 
    then yes, \quo{she} is quite hot.} 
Curwen's expression remained level. 
\tho{He's probably heard the joke before,} 
Carzain decided. 
\ta{I am leading the mission, and you're coming along,} Curwen continued.  
\ta{We're going to flank the Rungeran army and do some reconnaissance.}

\ta{Sounds like a job for scouts. 
    Or outriders, or whatever they're called. 
    Why us mages?}

\ta{Because we have to find out what magical weapons the Rungerans wield.} 

\ta{All right. 
    Why me?}

\ta{Because I tell you to, you smart-mouthed bastard.} 
Curwen took out the pipe and coughed for some moments.
His breath was foul. 
\ta{Now shut up and go get ready. 
    We leave in an hour.}

\ta{Yes, Captain.}







\new
Curwen sucked on his pipe and watched him go. 
\tho{I'm not so sure about that new kid,} he thought.
\tho{He talks tough, but is he really any good?
     Hm. 
     Well, here's his chance to prove whether he's an asset or a liability.}

He coughed again. 
\tho{Ah. 
     Nothing like a good smoke.
     
     I'll be watching you, \Shireyo.}



\begin{comment}
\subsection{No straight lines}
\end{comment}

\new
\placestamp{%
  Kenshaer Forest
  %\\
  %North of the main army's position
  \\
  Three hours later
}
% The sally is: Curwen, Carzain, Delph, \Tsekkect{}, two Rangers (\sphyles) and maybe four more soldiers. 
% Carzain requested to have \Tsekkect{} and Delph along. 
% The Rangers guide them along some hidden paths in the forest. 
% These two Rangers are locals and know these forests like the tips of their tongues.\footnote{\Scathaese{} idiom. A \scatha{} is presumed to know the tip of his tongue very well because he uses it to lick his eyes.} 
% They know all the shortcuts and, with their Ranger skills, can guide them through the \Wylde{} much faster than a whole army could travel. 
% 
% Sneaking through the \Wylde{} only works with small groups, not big armies. 
% The Ranger tries to explain it to Carzain, but he doesn't understand it. 
% She claims that it's not something that can be explained in regular language, because language is a thing of civilization, and civilization and its trappings are anathema to the \Wylde{}. 
% Carzain surmises that the Ranger doesn't really understand it herself and is just trying to hide it. 
Soon Carzain found himself deep in a shady, \Wylde{} forest. 
Apart from him the sally consisted of Archibald Curwen, Delph and \Tsekkect{} (included at Carzain's request), two guides and six more soldiers. 

The guides, \sphyles{} both, were \quo{\hr{Ranger}{\rangers}}; people skilled at journeying through the \Wylde{}. 
Such \rangers{} typically worked as pathfinders or hunters, and were often outcasts in regular society, allegedly due to being \quo{\Wylde{}} and \quo{bestial}, not truly civilized people. 
\tho{And no wonder,} Carzain thought. 
They were creepy. 
Especially one of them, Gwelthein. 
She was an older \sphyle, long and thin, with dusky blue scales. 
And there was something animalistic about her\dash her gait, her gaze, her speech, her fluid movements\dash as if she might at any moment drop down on all fours and start hissing and biting. 
Carzain was reminded of the boy Kreb, who had given off a similarly bestial vibe. 

The younger \ranger, Filcoi, seemed more like a normal \scatha. 
She was shorter and huskier than Gwelthein, dark green in colour. 
\tho{I wonder why she's green. Green \scathae{} are rare here.}
She wore leathers and carried a short, heavy sword, two daggers (that he could see) and a wooden sling. 

\ta{I did not quite understand what the captain said about our mission,} 
Carzain said to her. 
\ta{Why are we in the forest?
    How does that help us?}

\ta{Because we need to take a short cut and outflank the Rungerans,}
Filcoi explained. 

\ta{%
  Um... how is this a short cut? 
  According to the maps, the Ucarn Road is practically a straight line. 
  Surely the way through the woods must be a detour?}

\ta{%
  You should not put so much faith in the lines your maps. 
  They may work fine in settled lands and roads, but the \Wylde{} does not obey maps.}

\ta{%
  Well, granted, 
  the maps may be inaccurate when it comes what goes on in the \Wylde{}. 
  But this must still be a detour compared to the straight road...}

\ta{%
  No, I don't just mean the maps are inaccurate. 
  I mean what I said: 
  The \Wylde{} \emph{does not obey} maps. 
  There are paths out here that you can't even draw on a flat map.}

\ta{That doesn't make sense.}

Filcoi laughed. 
\ta{Welcome to the \Wylde{}.}

\ta{But, look. 
    In geometry you can prove that the shortest possible route between two points is a straight line.} 
Carzain drew lines and points in the air with his hands.
\ta{All right? 
    So you are saying you can find a route that is shorter than the shortest possible route?}

\ta{I'm no scholar. 
    I don't know what \quo{geometry} is. 
    But yes, I am saying that we can find a route shorter than the straight road. 
    At least, I hope we can.
    With luck.}

\tho{Sounds like total superstition to me,} 
Carzain almost said aloud. 
\tho{Still, it was Curwen who ordered this whole thing. 
     One would think he would know what he is doing. 
     This just doesn't make any sense...}

Filcoi smiled, noting the incomprehension on his face. 
She pointed with her thumb back the way they had come. 
\ta{Back there\dash the road, \Forklin, Malcur\dash%
    that's a world of straight lines and nice curves.}
Carzain smiled and almost giggled when she said \quo{nice curves}. 
\ta{%
  Out here is the \Wylde{}. 
  The \Wylde{} is not made of straight lines. 
  Out here, paths are crooked and winding. 
  They go places you don't expect them to. 
  They lead you places you didn't know you were going. 
  Places you didn't even know were there.
  Your eyes can deceive you. 
  If you know your way around, you can cross great expanses of \Wylde{} in no time. 
  If not, you can walk in a straight line for days and get nowhere.}
She fixed him with her gaze. 
\ta{%
  It's not that the maps are inaccurate.
  \emph{No map} can depict the \Wylde{}. 
  The \Wylde{} does not obey the laws of straight lines and flat surfaces that the civilized world does.} 

Carzain frowned. 
\ta{Really?}
\tho{%
  Damn. 
  I had heard stories about how the \Wylde{} twists around you and how it drives you mad because nothing makes sense. 
  But I always thought they were just superstition.
  Creepy stuff. 
  
  No wonder people shun these \rangers.}

He studied the \ranger. 
\tho{Hm. I still don't quite buy it.}

\ta{You don't believe me, do you?}
asked Filcoi. 

\ta{Well...}

She laughed softly. 
\ta{%
  You know, you mages always think you know everything. 
  I mean, sure, maybe you can zap people with lightning, or whatever it is you do, but do you know how the world really looks outside your towers and libraries?}

Carzain turned that around in his head. 
\tho{%
  Actually, it takes a great deal of understanding 
  of how the unseen worlds work before you can zap anyone with lightning.} 
He smiled.
\tho{%
  And moreover: 
  Overconfident? 
  Who, me?
  No way. 
  Among my many virtues, humility has always been the foremost.}

He studied the darkness of the wood again. 
\tho{Hm. I don't know. Could she be right?}

\ta{%
  My point is, \Miith{} is vaster and darker and more mysterious than most of us realize,} 
Filcoi concluded. 
Then she spurred her dark gray \relc{} on ahead, leaving him. 





\begin{comment}
\subsection{A fucked-up place}
\end{comment}

% The Rangers know this area better than the Rungerans do, so they should be able to outpace, outmaneuvre and outflank the Rungeran army. 
% Curwen wants to take a look at their trail and see what he can sniff up. 

Carzain looked around at the chaotic mass of branches and leaves surrounding them. 
\tho{No matter how I look at it, 
     this doesn't look like a shortcut to anything. 
     The way it looks to me, we are going nowhere, and slowly.}

\tho{I'll try Curwen again.
     He's an academic.
     He ought to speak my language.}
He sought out the captain. 

\ta{Captain,} said Carzain. 

\ta{What?} Curwen barked, not looking up from the pipe he was trying to stuff. 

\ta{Is it true that the \rangers{} can find a path 
    that's shorter than the geometrically shortest way possible?}

\ta{That's what we pay them for.
    And they didn't come cheap, either, so they had better.}

\ta{But how on \Miith{} is that possible?
    Do they use some kind of magic?}

\ta{It works because the \Wylde{} is a fucked-up place. 
    What do I know?}

Carzain turned his face away so he could roll his eyes without Curwen seeing it. 
\tho{Because the \Wylde{} is a \quo{fucked-up place}. 
     How enlightening.} 

He looked up.
They were deeper into the forest now. 
Branches enveloped them, all but blotting out the Sun, like many-fingered hands reaching out to crush them. 
He shuddered. 
The cramped twilight and the rustle of arms brought back memories of the last time Carzain was marooned in the \Wylde{} with a bunch of soldiers. 
Back in Heropond with the mutineers. 

\vizicar{But I slew them.} 

\tho{What?
     What was that?
     \Qliphoth. 
     Please tell me that was just my memories of last time. 
     Please tell me I am not hearing voices in my head.
     Again.}

% Carzain is disturbed by the spookiness of the forest. 
% He remembers the last time he was deep in a spooky forest. 
% Maybe Vizicar remembers, too. 









