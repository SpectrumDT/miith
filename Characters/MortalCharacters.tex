
% \part{Characters of \Miith: Mortals}
% \begin{comment}
\part[Scatha Characters]{\Scatha Characters}
% \end{comment}
\chapter{\Velcadians}















\section{Cuthran the Victorious}
\target{Cuthran the Victorious}
\index{Cuthran the Victorious}
King of \Vidra. 
The founder and first High King of \hr{Great Velcad}{\GreatVelcad}. 















\section{\Icor Pelidor}
\target{Icor Pelidor}
\target{Icor}
\Rayuth of Pelidor, husband and cousin of \hr{Tiroco}{\Tiroco{} Pelidor}. 















\section{Liocai Pelidor}
The sister of \Icor{} Pelidor. 









\subsection{Arsenal}
\subsubsection{Occult knowledge}
She knows a lot about \Malcur's mystic history. 
She is not a mage (\hr{Sorcerer-kings}{mage-clans rarely admit royalty}), but she knows quite a bit of the occult. 









\subsection{History}
\subsubsection{Family}
She is older than \Tiroco{} but remains unmarried and childless. 
She is still looking for a good political marriage. 















\section[Sethgal Pelidor]{\Rah{Sethgal} Pelidor}
\target{Sethgal}
\target{Sethgal Pelidor}
\index{Sethgal Pelidor}
\index{Pelidor!Sethgal Pelidor} 
Member of the \Malcuric{} branch of \hs{House Pelidor}. 
A half-cousin of \hr{Icor Pelidor}{\Icor{} Pelidor}. 









\subsection{Arsenal}
Sethgal spoke some Imetric, since he was politically active and often needed to confer with Pelidor's Imetrian allies. 









\subsection{Equipment}





\subsubsection{Sword}
\target{Sethgal's sword}
Sethgal's sword has a scabbard specially shaped to produce a nice, impressive \quo{shing} sound when drawn.









\subsection{History}
Sethgal was born (or laid and hatched) in his family's estate at \hs{Sulcanum}, a city north of \Malcur. 

He has been a soldier all his life. 

He once loved a commoner \sphyle{}. 
In his youth he even dreamed of marrying her. 
But he knew that wouldn't do. 
He wanted political power
So in the end he chose to go for a political marriage. 
His lover did not forgive him. 









\subsection{Personality}
Sethgal competed with \Icor{} for the ducal throne. 
In the end the family decided that due to their individual skillsets, \Icor{} was better on the throne and Sethgal better on the battlefield. 
So they made Sethgal a warlord instead. 
The fact that Sethgal was already married, whereas \Icor{} was available for a political marriage, influenced the family's decision.
Sethgal believes he should have been given the \rayuthship, and that only the marriage issue ended up deciding the matter. 

After \ps{\Icor} death, Sethgal desires to be \rayuth. 
He thinks he is the best man for the job. 
So he \manoeuvres himself into position as Marshal and intends to win glory by defeating the Rungerans. 
If he becomes enough of a hero, the House will have no choice but to make him \rayuth. 

Sethgal has a wife. 
She is noble and of an influential family. 
Sethgal has married politically and shrewdly. 
His wife does little useful work and in fact has few useful skills. 
That is just how Sethgal likes it: 
A weak wife gives him\dash, and by extension, House Pelidor\dash more power over their shared holdings. 

They have three children: 
Two daughters and a son. 
His oldest daughter, Quorae, is a soldier and an officer serving with him in the war. 
The other two are too young to fight. 
Sethgal has high ambitions for his children. 
He loves them and wants them to get ahead. 
That is one of the reasons he wants to be \rayuth. 

But the main reason he wants the throne is that Sethgal genuinely believes he is the best man for the job. 
He is harsh and logical, and in his view that is just the kind of leader the country needs. 
\Icor{} is too soft and sentimental in his view. 
Sethgal has long seen Runger as a threat and warned about it. 
He feels his warnings have not been heeded enough. 
Once he gets the throne he intends to tighten things up and make Pelidor the strong country it deserves to be. 





\subsubsection{Hobbies}
A favourite hobby of Sethgal's is dancing. 
He discovered the beauty of dance when he noticed the similarities with swordsmanship. 
As a soldier, he has of course practiced with swords a lot. 
Dancing, to him, has all the beauty of swordsmanship but without the violence and blood and grit. 





\subsubsection{Religion}
Sethgal was a faithful Iquinian and would often pray to the \sephiroth for guidance and aid and protection.
He was, after all, a knight. 
He could channel their power to a small degree. 
But he was pragmatic and weighed \quo{real} considerations higher than theological ones. 









\subsection{Politics}
\subsubsection{Archibald Curwen}
Sethgal does not like \hs{Archibald Curwen}. 
Sethgal himself is an accomplished leader and a bit of a manipulator, and he recognizes some of the same in Curwen. 
He feels the mage sticks his nose into things that shouldn't concern him. 
Curwen is a worm at court who has more influence than Sethgal would like. 
Meddling in the affairs of Pelidorian nobility. 
Curwen may be the son of some obscure nobleman in Belek, but he is not part of Pelidorian nobility, and he should not try to pull strings in matters that don't concern him. 

Consequently, Sethgal is hesitant to follow Curwen's advice. 
He doesn't want to dance to the mage's tune. 

He starting distrusting Curwen when he noticed that Curwen was ordering some soldiers around on his own business. 
Sethgal hasn't been able to see through Curwen's intrigues (Curwen is still more skilled than he is), so he doesn't suspect him of being a traitor. 
He just thinks Curwen is a highly ambitious man who wants to gobble up political power and influence. 
Perhaps, Sethgal reflects, Curwen is bitter because he tasted power growing up as a noble, but was then denied the throne. 
Curwen is a younger son, and Sethgal knows that Belek traditionally has the eldest son inherit the family title and estate. 
This makes him feel a kind of kinship with Curwen for a moment. 

The inheritance thing is silly to Sethgal: 
\tho{%
  Why not choose the most able person, like we do here in Pelidor? 
  Or, at least, like we ought to do. 
  If House Pelidor had truly selected the best person for the job, then I would be \rayuth now.
  
  And likely dead. 
  Assassinated. 
  Hm\prikker}





\subsubsection{Family}
Sethgal's parents were strict, but also loving. 
They often praised him for his accomplishments. 
This has given him a great deal of pride. 
He has a high opinion of himself and believes that he can do most things better than most people. 





\subsubsection{\Tiroco}
Sethgal sort of blames \hr{Tiroco}{\Tiroco} (or at least her marriage with \Icor) for cheating him of the \rayuthship. 
He feels a sort of resentment against her. 
But he knows this is unfair, so he tries to stop himself and be nice to her. 

She is pretty incompetent, but he can forgive that. 
His own wife is incompetent, too\dash that was why he married her. 
He recognizes that incompetent people have their place in the world, too. 









\subsection{Physique}
Sethgal was taller than average, but lean. 

















\section{\Tiroco{} Pelidor}
\target{Tiroco}
\Rinyuth of Pelidor, wife/widow and cousin of \hr{Icor}{\Icor{} Pelidor}. 























\chapter{Imetrians}















\section{Cassiron Drexis}
\target{Cassiron Drexis}
\index{Cassiron Drexis}
%\sectioncharunspec{Serpentin Curiet}{\scatha}{\male}
\index{Drexis!Cassiron Drexis}
A great Imetric hero that lived long before the \thirdbanewar. 















\section{Mannica Raeco}
\target{Mannica Raeco}
A \Retaxis{} in the Imetric army and a \nycaneer. 









\subsection{Politics}
\subsubsection{Family}
She is married to \hs{Telcastora Ilcas}. 
They have \hr{Ilcas' children}{four children}.















\section{Serpentin Curiet}
\target{Curiet}
\target{Serpentin Curiet}
\target{Curiet Serpentin}
\index{Serpentin Curiet}
%\sectioncharunspec{Serpentin Curiet}{\scatha}{\male}
\index{Curiet!Serpentin Curiet}
Imetric scholar, student of foreign language and culture. Also knows theology and magic theory (but cannot cast magic). 

\begin{description}
  \item[Name:] Serpentin Curiet. 
  \item[Race:] \Scatha{}. 
  %\item[Alignment:] Neutral good. 
  \item[Size:] 
  \item[Appearance:] Turquoise scales. 
\end{description}









\subsection{History}
In the end, {Curiet drinks Witchbane} and suffers permanent brain damage. 















\section{Telcastora Ilcas}
%\sectioncharlive{Telcastora Ilcas}{\scatha}{\male}
\target{Telcastora Ilcas}
\target{Ilcas Northstar}
\index{Telcastora!Telcastora Ilcas}
\index{Northstar!Ilcas Northstar}
%Male \scatha{} \birthtonow{Telcastora Ilcas}. 
Goes by the name \quo{Ilcas Northstar} in \Velcad{}. 

An Imetrian soldier, a \nycaneer{} and adventurer. His is a known hero in the Imetrium and bears the sword \hr{Telderain}{\Telderain}, the heirloom of the Telcastora (Northstar) clan. In the Imetric army he bears the rank of \Retaxis{}, but this is an \honorary title in acknowledgement of his deeds; he does not command troops as a \Retaxis{} normally would. 

Ilcas is married to Mannica Raeco and they have four children. His usual \travelling companions are the two \nycans{} Countess and Razor. 

% \begin{description}
%   \item[Name:] Telcastora Ilcas. 
%   \item[Race:] \Scatha{}. 
%   %\item[Alignment:] Lawful good. 
%   \item[Size:] 170 cm tall, 210 cm long. Very strong and muscular. 
%   \item[Appearance:] Dark blue scales. 
% \end{description}

At some point, he begins to suspect that the Imetrium is hiding something. 

Perhaps he has nightmares of \hr{The dark universe}{the dark universe}, connected to his sword, \hr{Telderain}{\Telderain}.










\subsection{Physique}
\target{Telcastora Ilcas's appearance}
Telcastora Ilcas was slightly taller than average.
Muscular and slightly heavy of build, but not so much.

His scales were cobalt blue. 

His most distinguishing feature was a wound on his snout that never quite healed. 
Both his lips still had a gap in them, and his tongue looked forked in an asymmetric manner (the wound was slightly to the left side). 

He has pronounced ridges above both eyes, but the right one is chipped, evidently a scar from a past battle.

Ilcas is slowly developing vision problems. Because of the nose wound, his scarred tongue is rougher than it should be, and it is grating against his eyes whenever he licks them, slowly eroding them. Now, his age (over 40 years) means that his vision is slowly going downhill overall, but the tongue wound only makes it worse. 

To compensate for this, he often uses his \hs{telepathy} to see through the eyes of his \nycan{} companions. 










\subsection{Arsenal}





\subsubsection{Badassitude}
\target{Ilcas badass}
Telcastora Ilcas was a somewhat aloof, decisive and badass type. 
He was scary and eerie to some because of his \hr{Ilcas Naga blood}{\naga blood} and his sinister Imetrian religion. 

Compare him to Tisamon from \cite{AdrianTchaikovsky:ShadowsoftheApt}. 




\subsubsection{Healing}
Ilcas was a skilled herbal healer. 
He had to learn this skill since he was so often out in the \Wylde{} on missions. 
Together with \hs{Razor's psionic healing}, the two could accomplish most the healing tasks they might need. 




\subsubsection{Pseudonym: Lucas Leppa}
\index{Lucas Leppa}
Ilcas sometimes uses the pseudonym \quo{Lucas Leppa} when \travelling in disguise. 
(This is interesting because \quo{Lucas Leppa} is an anagram of \quo{Claus Appel}.)




\subsubsection{Weapons}
In addition to \hr{Telderain}{\Telderain}, Ilcas also carries a slim smallsword in a scabbard in his belt. 
He uses the smallsword in battles that are less serious or if he does not want to kill. 
See, \hr{Telderain kills}{\Telderain{} kills}, and it is hard to prevent it from killing. 

When he finally unsheathes \Telderain, he is fucking wicked. 

Compare to Joscelin Verreuil from \cite{JacquelineCarey:KushielsLegacy}, who mostly fights with his daggers and only draws his sword in the biggest, baddest battles. 









\subsection{History}
In \yic{Mutiny}, Telcastora Ilcas meets Carzain and Nishain and helps escort them from \Scyrum{} back to \Redglen. 
He tells them he's on a holiday of sorts, but actually he has an important assignment after this. 





\subsubsection{\Naga blood}
\target{Ilcas Naga blood}
Telcastora Ilcas possessed some measure of \naga blood. 
He was a \hr{Naga hybrid}{\scatha-\naga hybrid}, a \hr{Demiscatha}{\demiscatha}. 
This made him a member of Pelidorian nobility, but also made him creepy and scary to outsiders. 

He had a fin or sail on his head. 
His tail was a bit too flexible, like that of a crocodile. 
Or, even worse, a fish. 





\subsubsection{Becomes \Retaxis}
\target{Ilcas becomes Retaxis}
\index{\Retaxis}
At some point after they pass near \Redglen{} (between \yic{Mutiny} and \yic{Runger war}), Northstar leaves the caravan to go on a secret mission to oppose a Rissitic scheme. 
It's a long quest, but in the end Ilcas succeeds. 
He is \honoured back home for his heroic accomplishments.
As a reward for this mission, he requests to be granted the \honorary rank of \Retaxis, so he now holds the same rank as his wife. 

The rank is granted to him by \Sarokash in person. 









\subsection{Personality}
Ilcas was proud of his achievements and openly introduced himself as a \quo{great hero} (as \hs{Imetric etiquette} allowed). 

But underneath that he was very conscious of his own weaknesses and the great responsibility he carried in the form of \hr{Telderain}{\Telderain}. 

Whenever Ilcas was discussing something important, he preferred to have \hs{Razor} nearby if at all possible. 
Razor was a much better telepath than Ilcas was. 

Ilcas liked children. 
He had four children of his own, but due to his career he had rarely been able to spend time with them. 
So whenever he was around other people's children, he would dote on them and play with them. 
His telepathic abilities helped. 
(Children were more easily reached with telepathy than adults due to being less Shrouded.)





\subsubsection{Leadership}
Ilcas is a good, inspiring leader because of his faith. 
He believes in his cause with great fervour, and this inspires people to follow him.
But he is no great strategist or commander. 
So he can make people follow him, but would not know where to lead them. 





\subsubsection{Religion}
Ilcas was a very devout Imetrian. 
He was intelligent, philosophical and well-versed in Imetric theology. 

Ilcas was a wandering Imetric missionary.
That wa part of the reason why he was allowed to go around and adventure: 
He had been known to spread the Imetric faith, or at least sympathy towards its cause, wherever he went. 

Compare him to Robert E. Howard's Solomon Kane.





\subsubsection{Racist}
\target{Ilcas' racism}
Ilcas was slightly racist. 
He believed \humans{} were inferior to \scathae, because he saw how cowardly and dishonest they were, prone to backstabbing each other all the time. 
He also believed the Imetrians were of superior breeding to other \scathae. 









\subsection{Politics}
\subsubsection{Family}
\target{Ilcas' children}
Ilcas is married to \hs{Mannica Raeco}. 
They have four children: 

\begin{dramatispersonae}
  \dramitem{Telcastora Cassili}{\scatha}{\female}, 
    their eldest child, Paladin-in-training 
  \dramitem{Telcastora Selcai}{\scatha}{\female}, 
    their second child, a \nycaneer{}
  \dramitem{Mannica Tarcus}{\scatha}{\female}, 
    their third child, a soldier
  \dramitem{Mannica Cicon}{\scatha}{\female}, 
    their fourth child, an apprentice mage
\end{dramatispersonae}









\subsection{Skills}





\subsubsection{Languages}
Ilcas, being well \travelled, spoke many languages.









\subsection{\Telderain}
\target{Telderain}
\index{\Telderain}
\Telderain{} is the sword wielded by Telcastora Ilcas. 
It is an heirloom of the Northstar family. 
It is 400 years old and enchanted. 

\Telderain{} is alive and has a will of its own. 
It is sentient, but not with humanoid-level intelligence. 

The sword does not need repair or maintenance. 
It keeps itself sharp and clean. 
It just needs blood. 

\Telderain{} is so swift and accurate because it has its own will and moves on its own. 
When Ilcas draws it, it seems to leap into his hand of its own volition. 
The weapon's speed is particularly effective because it is so surprising; \Telderain{} is large and heavy, so enemies assume it will be slow. 
They are surprised when they are sliced to bits before they even saw it coming.





\subsubsection{Anti-magic}
\Telderain{} has the power to protect against and dispel magic. 
It is primitive, for the sword and its wielder are not always (rarely, actually) clever and magically-skilled enough to be able to deftly counter spells. 
But the sword and the wielder can use their intuition and brute force. 
It offers some protection. 

This ability is activated with the spellword \word{cuspinum}. 





\subsubsection{Appearance}
\Telderain{} was a large bastard sword which could be wielded in one or two hands. 
It was of an archaic design, larger and heavier than most modern swords, and required great strength to wield. 
It was made of a \dragonsteel-iron alloy. 

It looked very old-fashioned, cumbersome and clumsy compared to the much slimmer modern swords and sabres. 
But was much faster than it looked. 
Ilcas exploited this deception to the full. 





\subsubsection{The history of \Telderain}
The sword \Telderain, the heirloom of the Northstar clan, is infused with the souls of captive \mdaemons, \xzaishann-spawned monstrosities. In order to contain the \mdaemons, several noble Northstars sacrificed themselves, letting themselves be killed with the sword and having their souls bound within it.

The \daemons{} were bound against their will and seek to corrupt the wielder and use the sword's power to cause destruction. (The \daemons{} have no goal in particular, and might be mostly mindless. They just lash out in anger.) 

The Nothstar souls seek to contain the \daemons{} and prevent them from fucking up the wielder, but the battle drains their strength. Therefore, the weapon must fight and kill, drinking blood to sustain itself\dash otherwise it might drive the wielder mad. It can keep stable for a while if given small offerings of the wielder's blood (or someone else's). 

Even so, the Northstar souls weaken with time. So each new wielder of the sword must also bind his soul to it, so that when he dies, his soul will join his forefathers in the sword, safeguarding their legacy. 





\subsubsection{Why Ilcas was chosen}
Ilcas was chosen as the wielder of \Telderain{} because (among other things) as a \nycaneer{} he already knows and understands the struggle between the humanoid and the bestial within him. The \nycan{} part is pretty easy to master, but the struggle gives him some very useful skills that can be used in holding the \daemons{} within the sword at bay. 





\subsubsection{The reason for his fanaticism}
The sword is one of the reasons why he is so fanatical, and why he follows his ethics with such uncompromising brutality: 
He has to cling tight to some firm beliefs. 
If he falters in his resolve, his willpower will weaken and the \daemons{} will be able to control him. 
Chaos will rule. 
Compare to the \quo{paths} that vampires follow in the RPG \emph{Vampire: The Masquerade}. 

He says: \quo{I cannot afford doubt.} 
Compare to Akiyama Ren from \emph{Kamen Rider \Ryuki}. 





\subsubsection{Wants to kill}
\target{Telderain kills}
The sword is a killer. 
It wants and needs to kill. 


At times, when Ilcas needs to draw great amounts of arcane power from the sword, it nearly takes over him. 

\lyricslimbonicart{In Abhorrence Dementia}{
  The dominions on Earth shall return to the beast,\\
  as the darkside awaits the capture and feast\\
  with dark surrounding illusions.
  
  Possesion is a passion, simplicity is intuitive.\\
  Native forces of violent misery.
}

It makes him doubt his inner strength. Will he be able to overcome it?

\lyricslimbonicart{In Abhorrence Dementia}{
  The soil in a man's heart is stonier.\\
  In stench of rot and sour ground.\\
  The obedient fall into cruelty\\
  where all arts of life shall be undone.
  
  A madness wells up in me\\
  as I swallow the pain.\\
  In the shell of the beast\\
  where unbounded evil reigns.\\
  In Abhorrence Dementia.
}

Another time when he unleashes the sword's power and is nearly consumed in the process. 
He is seized by niilism and a Chaotic passion for \quo{living in the now} and abandoning sense, thought and morality. 

\lyricslimbonicart{Under Burdens of Life's Holocaust}{
  Like a lonely candle burning,\\
  I blaze as the darkness emerging,\\
  and I yield for the art to bleed.
  
  When the moon is drained by all its light,\\
  and the stars they shine like serpent eyes,\\
  I'm feeling deaths desire for me.
  
  Baptised in esteem or arrogance.\\
  Raping all virtue and sweet romance.\\
  Bewitched by sins and lust.
  
  Emancipate the deceiving earth.\\
  Praise the sign in the sky for no new rebirth.\\
  Under burdens of life's holocaust.
}

\lyricslimbonicart{The Dark Paranormal Calling}{
  A serpent in my soul.
  
  In midnight's aura as a ghostly fire.\\
  The child prodigy of abysmal desire.\\
  As the deathlike silence pervades, \\
  the incubus now invades.\\
  Show yourself, unclean spirit. \\
  Tonight I give thy shadow life.\\
  Rise with me in darkest blessing.\\
  Thine demon force I feel possessing.\\
  A holotropic mind and spectral eyes divine.\\
  Through the shallow haze as the sky turns red\\
  I walk and dream among the dead.
}






\subsubsection{The purpose of \Telderain}
\target{Telderain in Imetric Matrix}
In the end, it turns out that \Telderain{} is a vital part of the Imetric master plan. The sword is actually a \vertex{} in the Imetric \matrixx. 















\section{Ulphon Nestor}
\target{Ulphon Nestor}
\index{Ulphon Nestor}
\index{Nestor!Ulphon Nestor}
Ulphon Nestor, a male \scatha, was an Imetric cleric and mage with the rank of \Ispan. 
He fought in the battle of \Forclin where \hr{Ulphon Nestor dies}{he was killed}. 























\chapter{Rissitics}
\section{\Dzasselid}
%\sectioncharunspec{\Dzasselid}{\scatha}{\male}
A Rissitic \Shessefkesad{}, \ps{\Shilred}{} master. Initially working to conquer a \Scyric{} city together with \ps{\Narkiza}{} army. Later drafted for \hr{Ishnaruchaefir's fake mission}{\ps{\Ishnaruchaefir} fake mission}. 

He is cool and has a background story. Inspired by Quick Ben from \emph{Malazan Book of the Fallen}. 

For the first while, we don't see things from his POV. We only see him from the outside through \Shilred. 









\subsection{Haunted by dark powers and feelings}
\Dzasselid{} is idealistic, but he knows too much of the terrible truth. 
At times, he is seized by misanthropy and nihilism. 

\lyricslimbonicart{When Mind and Flesh Depart}{
  The human way of life, \\
  an inferior state of mind.
  The third stone from the Sun\\
  had become a stillborn illusion.
}

When he unleashes his terrible magic, he finds it hard to stop again. 

\lyricslimbonicart{When Mind and Flesh Depart}{
  I'm under siege of anger and fury.\\
  There is no values, no faith or glory.\\
  Antagonising mortal flesh.\\
  Life on Earth I do oppress\\
  with infernal bleedings of bitterness.
  
  I have seen in my darkest dream\\
  through the astral gate in a vortex fate:\\
  Death coldening the warrior's fire.
}















\section{Geldashad}
%\sectioncharunspec{Geldashad}{\human}{\male}
\target{Geldashad}
Rissitic \Ashenoch{}. 

\begin{description}
  \item[Name:] Geldashad something\prikker 
  \item[Race:] \Scatha{}, \Ashenoch{}. 
  %\item[Alignment:] Chaotic evil. 
  \item[Size:] 
  \item[Appearance:] 
\end{description}















\section{\KarsaOrlong}
%\sectioncharunspec{\KarsaOrlong}{\cregorr}{\male}
\target{Karsa Orlong}
A barbarian warrior fighting for the Rissitic army under \Narkiza, alongside \Shilred{} and \Dzasselid. 

He is from a \Durcaci{} tribe only recently converted to Rissitism. He is born an infidel and still venerates his ancestral religion alongside \Nechsain. 

He is an extremely powerful elite fighter, a holy champion of his tribe, guided and strengthened by the spirits of his ancestors and the blood of his savage gods. \Shilred{} sees him fight and is terrified to see that his strength almost matches that of an \Ashenoch. 

Perhaps he is \Geldashad, the evil \Ashenoch. Or perhaps they are two different characters. If they are the same, then perhaps \Narkiza{} hates him because he is a racist, prejudiced against \ps{\KarsaOrlong} tribe. Perhaps \Narkiza{} has some childhood trauma against them\dash maybe he fought them in a war back when they were infidels and remembers their monstrous savagery and black magic. (Of course, the Rissitics use \quo{black} magic, too, but \Narkiza{} is used to that, while the sorcery of the tribesmen's witch doctors is alien to him. Perhaps \Narkiza{} later reflects on this and learns a valuable lesson.)
















\section{\Narkiza}
\target{Narkiza}
\target{Sesstra}
\index{\Narkiza}
%\sectioncharunspec{\Narkiza}{\scatha}{\male}
\Narkiza{} is an \Ashenoch{} and the commander of the Rissitic army that marches north into \Velcad{} to conquer. 
He is actually good, but he wars at times with the \shadowcreature{} inside him. 
(As an \Ashenoch, he is merged with a \shadowcreature.)

He needs to have some of the same badass attitude as Lord Soth (from \emph{Dragonlance} and \emph{Ravenloft}). 

We often see him through the eyes of \Kufur, a \sphyle{} and his subordinate officer. 
She describes how his voice is hollow and metallic. 
This is partly because of the metal mask/helmet he wears, but also partly because of his \mdaemonic{} \Ashenoch{} nature. 

\begin{description}
  \item[Name:] \Narkiza{} Iumyra Shurid'shestrae Rekkan-\Neftzaid{} \Ashenoch-\Hashkfed.
  \item[Race:] \Scatha, \Ashenoch{}. 
  %\item[Alignment:] Lawful good. 
  \item[Size:] 190 cm. 
  \item[Appearance:] 
\end{description}









\subsection{Names, titles and reputation}
The Rissitic name \quo{{\Narkiza}} was a variant of \quo{\hr{Settras}{\Settras}}. 
There was symbolism in this. 





\subsubsection{Dark warlord}
\target{Narkiza's dark warlord reputation}
\Narkiza was a dark warlord, \hr{Narkiza's dark warlord personality}{with matching personality}. 

Compare to the Chaos Warriors from \cite{RPG:Warhammer}. 









\subsection{History}





\subsubsection{Youth}
\Narkiza came from a tribe in the southern \Durcac.
He was ethnically \Durcaci and spoke High Rissitic, but his tribe was more savage and brutal than most. 

He went through some measure of \trope{TrainingFromHell}{Training From Hell} in his youth.
He worked very hard to become strong, and to be a leader.
That was what enabled him to become a hero while still mortal.

Later he was recognized by the church as something special and turned into an \Ashenoch.





\subsubsection{Uniting the tribes}
\Narkiza \hr{Narkiza unites tribes}{united the Rissitic tribes}. 





\subsubsection{Became an \Ashenoch}
When \Narkiza had performed enough heroic deeds to convince the gods that he was the real deal, he was chosen to undergo the \hr{Ashenoch ritual}{ritual that would transform him into an \Ashenoch}. 

He was killed and reborn as an \Ashenoch:
He knew he was chosen by the gods to be someone special.
To be a great conquering hero, the champion of his people.

\citebandsong{Nile:BlackSeedsofVengeance}{Nile}{
  The Black Flame
}{
  Open, For Me the Gates Shall Open\\
  Over the Fire of the Spirit, The Breath Drawn by the Gods. \\
  Arise Apophis Return, That I Might Return, \\
  Borne by the Flame Drawn by the Gods Who Clear the Way that I Might Pass.\\
  The Gods Which Sprang from the Drops of Blood \\
  which Dripped From the Phallus of Set\\
  That I might be Reborn\\
  For I am Khetti Satha Shemsu, Seneh Nekai\\
  And Will Become Set of a Million Years
  
  Akhu Amenti Hekau\\
  I shed My Burnt Skin and am Renewed
}





\subsubsection{Eating the gods}
\target{Narkiza's rise to power}
\Narkiza rose to greatness by destroying and devouring his enemies and (with the help of his own gods) even slaying and consuming their spirits and demigods.
He soon became a legend. 
After his first several acts of heroism he became an \Ashenoch. 
Since then he grew rapidly in power.

\citebandsong{Nile:InTheirDarkenesShrines}{Nile}{
  Unas, Slayer of the Gods
}{
  Poureth Down Water From the Heavens\\
  Tremble the Stars\\
  Quake the Bones of Aker\\
  Those Beneath Take Flight \\
  When They See Unas Rising

  The Akh of Unas Is Behind Him\\
  The Conquerer Are Beneath His Feet\\
  His Gods Are In Him\\
  His Uraei Are on His Brow\\
  The Words of Unas Protect Him

  Unas This Bull of The Heavens\\
  That Trusteth With His Will\\
  Living On Utterances of Fire \\
  From the Lake of Flame\\
  Unas That Devoureth Men and Liveth on The Gods
}

His own gods were with him.
With their power he cast down and ate the enemy gods. 
This was part of \Secherdamon's plan: 
To punish those gods who rebelled against him, and to invest his chosen champion with their power.

This was similar to \hr{Secherdamon's rise to power}{how \Secherdamon himself rose to power}. 

\citebandsong{Nile:InTheirDarkenesShrines}{Nile}{
  Unas, Slayer of the Gods
}{
  Behold Amkebu Hath Snared Them for Unas\\
  Behold Techer Tep F Hath Known Them and Driven Them Unto Unas\\
  Behold Her Tbertu Hath Bound Them\\
  Behold Khensu The Slaughterer of Lords\\
  Hath Cut Their Throats for Unas\\
  Behold Shesemu Hath Cut Them Up For Unas
}





\subsubsection{Invasion of \Velcad}
Approximately at the same time as the \hr{Runger war}{Pelidor-Runger war}, \Narkiza led an invasion of \Velcad.
He was maybe 100-150 years old at the time, and he had been steadily gaining strength all his life. 








\subsection{Personality}





\subsubsection{Dark warlord}
\target{Narkiza's dark warlord personality}
\Narkiza was a dark warlord, \hr{Narkiza's dark warlord reputation}{with matching reputation}. 

\Narkiza was a mighty warlord and also a religious holy figure.
He was the ravager of his enemies and the protector of his own.

\citebandsong{Nile:BlackSeedsofVengeance}{Nile}{
  Chapter for Transforming into a Snake
}{
I am a Crocodile immersed in Dread\\
  I am the Crocodile who Takes by Robbery\\
  I am the Great and Mighty Loathsome Reptile\\
  Who is in the Bitter Waters\\
  I am the Lord of those who Bow Down in Sekhem
  (See Karl Sanders' multiple translations of "Sekhem".)
}

\citebandsong{IcedEarth:SomethingWickedThisWayComes}{Iced Earth}{
  The Coming Curse
}{
  Saviour to my own, devil to some
}





\subsubsection{Sanity}
\Narkiza was a brave but scarred soul.
He tried to be as good as possible within his brutal warrior culture and religion. 
But he was losing sanity through dark sorcery, becoming and \Ashenoch and communing directly with \Nechsain and other dark powers. 





\subsubsection{Threats}
\Narkiza's threats when he attacks and besieges a city:

\citebandsong{Nile:BlackSeedsofVengeance}{Nile}{
  Defiling the Gates of Ishtar
}{
  Open the Gate that I may enter, open, lest I break down the walls\\
  Open the Gate, lest I cause the dead to outnumber the living\\
  Open the Gate, lest I cause the dead to rise and to devour the living
}









\subsection{Physique}
\Narkiza was a very large dax.















\section{\Shilred}
%\sectioncharunspec{\Shilred}{\scatha}{\female}
\Shilred{} is a Rissitic mage, a \Shessefkesad. She is apprenticed to \Dzasselid{}. In the beginning she is working with \Dzasselid{} to conquer a \Scyric{} city. Later she is drafted by \Ishnaruchaefir{} and sent on \hr{Ishnaruchaefir's fake mission}{his fake mission}. 

\Shilred{} is a racist and hates everyone who is not like her. During the story she gradually grows as a person. Also, through her eyes we get to see the cooler \Dzasselid{} and \Ishnaruchaefir{} from an outside perspective. 

But she is not a pure asshole. She may be a racist, but she is also an idealistic person who wants to help her empire and her religion because she believes they are good. She also wants to help her comrades and prevent harm from coming to them. But she is meddling and somewhat bitchy. Perhaps like Egwene from \emph{Wheel of Time}. 

\Shilred{} might die at the end of her mission. 









\subsection{Nerd}
\target{Shilred is a nerd}
\Shilred{} is something of a nerd. 
She is a strong and skilled mage, but very much a scientific type, and she doesn't know much of the cruel reality of the world beyond the temples. 
When \hr{Shilred-tachi reach the cave}{she sees glimpses of the \xss{} and other horrors}, she is shocked and traumatized. 









\subsection{Obsession with the \xss}
Perhaps she later develops an unhealthy obsession with the \xss.

\lyricslimbonicart{In Embers of Infernal Greed}{
  By dawn's early light \\
  I see no end to the dark obsession. \\
  I am swallowed by night's\\
  infernal dream profession. \\
  O' sea of fire, all hatred's desire. \\
  Thy abhorrent cremation. Sparks in my eyes. \\
  Forces generates from the bottomless pits\\
  of detestation.
}















\section{\TessHanith}
\target{Tess-Haanith}
\TessHanith was an agent of \Secherdamon. 
She acted as a Rissitic priest.
She was actually an undead immortal \hr{Psyrex' undeath}{like \LocarPsyrex}. 









\subsection{Physique}
\TessHanith was an undead \hr{Xul-Gann}{\XulGann}, \Lich-like or mummy-like. 
Her face was horribly scarred and disfigured from millennia of decay.
Her frail mortal body could not contain the vast sorcerous power invested in her, so it warped and changed shape.

\target{Tess-Hanith's undeath}
She was weaker than \LocarPsyrex, physically and metaphysically. 

She never walked but was always carried on a palanquin. 















\section{Texall}
\target{Texall}
Texall was a very skilled Rissitic mage and magic-researcher. 
He discovered and formulated the \hs{Texall Axioms}, which were a great breakthrough in magic theory and ended up becoming a large part of the foundation of later \hs{Rissitic magic} theory. 

He lived in the first century after the \hr{Fall of Ortaica}{fall of \Ortaica}. 

At the time when he lived, the Rissitic language was pretty close to \Ortaican. 
Therefore his name sounds quite different from modern Rissitic. 























\chapter{\Ortaicans}















\section{Kish Tulla}
\target{Kish Tulla}
\index{Kish Tulla}
Kish Tulla was a \dax of \Yormis.
He was a childhood acquaintance of \hs{Suthis Mephilex} and \hs{Jaslar Thul}. 

Tulla was no \rethyax, but he had great talent with weapons.
He became a soldier and aspired to be a weapon master.















\section{Jaslar Thul}
\target{Jaslar Thul}
\index{Jaslar Thul}
Jaslar Thul was a young \sphyle of Clan \hs{Jaslar}. 
She was a \rethyax in \Yormis. 

Her clan was an enemy of Clan \hs{Suthis}. 
Jaslar had a particular personal dispute with \hs{Suthis Mephilex}. 
Jaslar was two years older than Mephilex, and the two had always hated one another. 















\section{\MoroCobrel}
\target{Moro}
\target{Moro Cornel}
\MoroCobrel{} was a \sphyle{}. 
She was a \rethyactic mage and a prominent member of the \Malcuric{} \ishrah. 
At the time of the Pelidor/Runger war she was about 60 years old. 

She is something of an \quo{investigator}, in the style of the player characters from the RPG \emph{Call of Cthulhu}. She doesn't know nearly everything, but she feels herself drawn to investigating, while at the same time being repulsed by the morbid things she discovers. 









\subsection{Physique}
Moro's body is decayed and deformed. 
She uses magical healing to keep it in shape so she looks merely like an ugly old crone and not a monster. 
But her body is unstable and constantly decays, so she regularly has to reshape it. 
(Like once every month or so.)

She sacrifices animals to keep her body in shape. 

She might be able to heal herself more effectively, perhaps even permanently, by appealing to \hr{Nasshikerr}{\Nasshikerr} or researching some darker magic, or even coming crawling to a Vaimon and beg for help. 
But she refused to do that. 
She told herself it was out of pride, or that she did not need healing. 
But in truth, it was due to her self-loathing. 
She felt she deserved her disfigurement as just punishment for being such a bad person. 

At court she wrapped herself in illusions and presented herself as a handsome \sphyle{} in her prime. 

At night she skulked around the city in her true form. 
This had advantages: It let her diguise herself by doing nothing, since no one knew her true face. 
And even better, since her body constantly warped and must be reshaped, her true appearance constantly changed, throwing observers off guard. 









\subsection{History}
She has a dark past. 





\subsubsection{As a young girl}
When she was a young girl, her city was stalked by villains. She was raped by evil men, and later saw her family killed, perhaps by monsters from Beyond\dash torn apart in front of her eyes, splattering her with their blood and brains. 

She saw visions of the Beyond and was driven half-mad. 
But she was strong of mind, so she fought her way through it, and gained valuable insight in the process. 
She was even able to work some primitive magic after this.

(Who was her family? Is she nobly-born? Is \MoroCobrel{} her true name?)

Have flashbacks to where she discovers and develops her magic. 
She sees visions of \hr{The dark universe}{the dark universe and its wonders}. 

She tries very hard to make it good and nice, but external circumstances forcer her to use the dark aspects of her skill, and so those are the only aspects that are honed and developed. 
Her own traumata, present already at an early age, only make it worse, and her magic becomes like her: 
Misanthropic, hateful and full of pain. 
No matter how much she tries to make it nicer, it only becomes worse, darker and more evil.

She is very good at causing people pain and horror. 
She hates this power, but is forced to use it all the time. 
Like \SailorNothing. 

As she descends deeper into the darkness of Chaos magic, she feels like she can hear the \sephiroth{} crying, mourning her fall from grace.

\lyricslimbonicart{Beneath the Burial Surface}{
  The sky is darkening, soon the night befall.\\
  Righteously angels are weeping for my soul.\\
  All childhood dreams are soon to be lost,\\
  all innocence to be shattered.
  
  I am the fallen from grace.
  
  My face is a river.\\
  See my eyes as they drown in black.
  My sacred doom and nemesis\\
  beneath the burial surface\\
  To the final act of the immortal sin\\
  I am lead by funeral winds.
}





\subsubsection{Apprenticeship}
She lived on her own for some years as an orphaned outcast, a criminal and murderer, until a Chaos sorcerer, living in hiding in her city, discovered her talent. He took her in as his apprentice. 

Maybe have a scene where \hr{Chaos mage initiation}{Moro is initiated}. 

He had several other wards: 
Apprentices or servants. All children.

It turned out that he was a cruel and evil master. 
He abused her physically and emotionally and the other children. 
He even raped her, which was particularly traumatic, as it was a vile betrayal by a man she trusted to be her protector, a father figure. 
He was a strong-willed but deeply deranged man. 

He was also sneaky, turning the children against one another and making them bully each other. 
That way, they had no allies, and he was the only authority in their lives.

At last, Moro betrayed and killed him\dash tearing him apart with her horrible magic, or perhaps turning his own similarly horrible magic against him. 
Some of the other kids were loyal to their cruel master, and Moro had to kill them, too. 

Have traumatic flashbacks where we see Moro killing several of her fellow children. 
Perhaps she even went mad and went on a killing spree, killing everyone in sight: 
Enemies, friends, servants, random civilians. 

Compare to the \cite{Anime:ElfenLied}.

\lyricslimbonicart{In Embers of Infernal Greed}{
  By dawn's early light \\
  I see no end to the dark obsession. \\
  I am swallowed by night's\\
  infernal dream profession. \\
  O' sea of fire, all hatred's desire. \\
  Thy abhorrent cremation. Sparks in my eyes. \\
  Forces generates from the bottomless pits\\
  of detestation.
  
  Under the delusion of hatred. \\
  The becoming of a malignant tormentor. \\
  In a spell of symptomatic madness \\
  I identify with aggressor. \\
  This world will become infernal land, \\
  a violation of harmony. \\
  As I kill with a psychic command, \\
  destruction let all pain fly free. \\
  
  The metamorphosis from man to beast.
}





\subsubsection{Educated in \Yormis}
Moro was born, raised and educated in the ill-rumoured city of \hr{Yormis}{\Yormis}. 
There she took her first steps on her path to dark enlightenment.
Which she would later lament.





\subsubsection{Learns horrible occult secrets}
\target{Moro is an investigator}
\MoroCobrel knew much about the Beyond, the occult, the \Miith Mythos. 
She was a Cthulhu style investigator. 
In her youth she did much exploring, adventuring and research along with some adventurous and foolish comrades. 
Among other things, she learned of the existence of the horrid demigod \hr{Ubloth}{\Ubloth} that dwelt beneath Mount \Shrun. 
That was why she became so scarred and bitter and unhappy. 

The lesson: Being an adventurer does not pay on a cruel, horrible world like \Miith.

Moro knew of the \quo{Elder things} that would one day return to conquer \Miith and overthrow all mortal civilizations. 
But she did not know which creatures were on which sides. 
She knew the words \quo{\xs} and \quo{\bane}, but she was not quite sure what they referred to and what the distinctions were, let alone who opposed whom. 

Among other things, she \hr{Moro and QJ}{knew a little of the \quiljaaran}.





\subsubsection{\Ubloth cult}
\target{Moro and the Ubloth cult}
Moro was a part of the \hr{Ubloth}{\Ubloth} cult for a very short while, but soon left it again.
She never actually saw \Ubloth, but she felt its presence in magical rituals. 

And she saw and touched the \hr{Well in Yormis}{black well under \Yormis}. 

Later, Moro remembered the loathsome amorphous god \Ubloth that dwelt in the deep darkness beneath Mount \Shrun near \Yormis. 
She feared the \Primordials, for she remembered the feeling of \Ubloth, and it felt similar to the \Primordials.
The \Ubloth cultists believed that \Ubloth was kin to the \Primordials and therefore worthy of worship.
Moro believed that if the \Primordials were kin of \Ubloth, then they should be reviled and loathed and shunned just as \Ubloth should. 





\subsubsection{Drank \draconian blood}
Moro once drank \dragon blood in a misguided attempt to \hr{Dragon blood gives immortality}{gain immortality and power}.
As it turned out, the truth was more complicated than that.
She and her companions did not know how to prepare the \dragon blood with spells.
So it backfired and scarred her.





\subsubsection{Undead plot}
Moro was involved in \hr{Moro and Sperra story}{Suthis Mephilex's rise to power}. 





\subsubsection{The High King's court}
During her apprenticeship she became acquainted with the court of the \Velcadian{} High King. Ultimately, she became involved in the final days of \GreatVelcad, and had a hand in the events leading to its fall. 

She was tricked and forced to do all sorts of bad things, which she now angsts about.





\subsubsection{Pelidorian \ishrah}
Moro moved to Pelidor and joined the \ishrah. 





\subsubsection{Sacrifice to \Nasshikerr}
Moro regularly kills people to sacrifice to \hr{Nasshikerr}{\Nasshikerr}.
She does not like it, but she makes sure to kill only bad people, and by now she has grown numb and used to it.
Her killings are actually legal.
She knows Pelidorian law well. 
She can pull up paragraphs that give her, as leader of the \ishrah, the right to do this.
But she does her best to keep it secret anyway.
She does not want her reputation to get any worse, since that would make it harder for her to pull strings.
And she does not want the populace to know that the \rayuth's own \ishrah mages run around and kill people in the night to get blood for their dark gods.
Better to live in ignorance of that.
Ignorance is bliss.
Moro knows that all too well.
Nore that all the above is one of my attempts to make the whole world more dystopian.

Moro has contacts among the city guards.
She pays them and they cover up her shady deals.
Some of them owe her for magical favours she did them, like healing/necromancy or cursing their enemies.





\subsubsection{Investigating \Malcur}
Rian and Moro investigate and begin to uncover \Malcur's dark secrets. 

When Moro is out in the city, she notices that \maybehr{Malcur is going mad}{\Malcur is going mad}. 

\MoroCobrel knows the \xs exist. 
At some point she wonders about their identity and the connection to what is happening.
She recognizes something in the catastrophic fall of \Malcur that reminds her of what she knows of the \xs.
And she suspects her own \Nasshikerr knows more than he lets on. 
She suspects \Nasshikerr is in league with the \xs, or the vile alienists who employ their power.
(Remember, \hr{XS Taorthae}{some \taorthae are \xss}.)






\subsubsection{After \Malcur}
She tries to save \Malcur and prevent the resurrection of \Nithdornazsh. But she only succeeded in saving Rian and Neina. 

And not only that, she made the problem worse by \hr{Needle dies}{killing Needle}. If Needle had stayed alive, perhaps the Cabal would have been able to counter the \Nithdornazsh{} plot. 

Moro realizes this, and it becomes a new trauma for her. 

Near the end of the whole series, Moro does something heroic, sort of \quo{redeeming} herself. But she learns too much of the truth, and her sanity drops to zero (to use the terminology of the RPG \emph{Call of Cthulhu}). In the end she dies. Perhaps suicide, but probably not. 









\subsection{Personality}
Today, Moro lives in \Malcur, hiding the trauma of what happened at the High King's court, along with her other, more personal traumata. 

She is traumatized, but emotionally repressed. In order to deal with and control her terrible \chaos{} magic she has had to resort to drastic means to repress her natural revulsion against it, to keep her psyche in check and protect it, prevent herself from going raving mad. She has formed a mental shield around herself that is almost like a miniature \Sephirah. 

\lyrics{%
  She wanted to scream, but no sound would come. \\
  She desperately needed to cry, but no tears would come.
}

(The above is not a quote, it's my own.)

She is bitter, wounded, disillusioned and full of pain. She sees the universe as a horrible place and believes the worst of people and situations. For example, she encourages Rian to drop his girlfriend because she will only hurt him sooner or later anyway. 

She herself has not had sex in decades. Perhaps she finally gets some sex? (Compare to the story of Ganoes Paran and Tattersail in \emph{Malazan Book of the Fallen}.) 

Compare her to Jacqueline Ess from \authorbook{Clive Barker}{Jacqueline Ess: Her Will and Testament} (from \emph{Books of Blood II}). 

She is bitter and disillusioned, plagued by guilt over all sorts of things she has done, or been forced or tricked into doing. The part about being tricked to do horrible things is especially traumatic and has made her misanthropic and paranoid, not trusting anyone. 





\subsubsection{Gods}
Moro knew about the \taorthae and other gods that were much darker.

\citeauthorbook[p.137]{RobertEHoward:TheAltarandtheScorpion}{Robert E. Howard}{%
  The Altar and the Scorpion%
}{
  \ta{The real gods are dark and bloody!
    Remember my words when soon you lie on an ebon altar behind which broods a black shadow forever!
    Before you die you shall know the real gods, the powerful, the terrible gods, who came from forgotten worlds and lost realms of blackness.
    Who had their birth on frozen stars, and black suns brooding beyond the light of any stars!
    You shall know the brain shattering truth of that Unnamable One, to whose reality no earthly likeness may be given, but whose symbol is\dash the Black Shadow!}
    
    The girl ceased to cry, frozen, like the youth, into dazed silence.
    They sensed, behind these threats, a hideous and inhuman gulf of monstrous shadows.
}





\subsubsection{Hating her power}
Still, she tries to fight for what is right. She is disgusted by the dark, foul magic she uses. But she is stuck in the Shroud and has a psychological block that prevents her from learning other magic, however hard she's tried. Her traumata and self-hatred will not let her. 

Compare her to \SailorNothing. 





\subsubsection{Phobias}
Moro suffers from paranoia and fear of being touched. This is from sexual abuse and having betrayed or been betrayed by almost everyone she's loved. 

Maybe she has a phobia of blood and/or corpses as a trauma from all the people she herself has killed in a gruesome splatter of blood. 









\subsection{Politics}





\subsubsection{\QuilJaaran}
\target{Moro and QJ}
\MoroCobrel knew a little about the \quiljaaran (\quo{\serpentmen}). 
\hr{QJ in Yormis}{Some of them worked in \Yormis}. 

She knew a bit about their terrible secret. 
She had once seen a glimpse of them in \Yormis, \hr{QJ ride salamanders}{riding on their grotesque salamanders}.
But she only saw a brief glimpse, and what she saw scared her.

Before the event she had read about the \serpentmen.
It was scary reading.
The \serpentmen were almost-\scathaese Elder horrors.
She only knew small fragments of the full story and understood even less.
She wished she knew nothing at all. 
This knowledge was an \arcanum she had stolen. 
For many years she regretted having stolen it. 

It was gruesome to consider that the legends were true. 
So Moro tried to tell herself that the \serpentmen did not exist, that she had merely seen some normal \scathae and that her fanciful young mind had made them look like serpents.

Later, in \Malcur, Moro saw \quiljaaran again and was forced to admit that they were real. 

She wondered about them.
What was their nature, and what was their relation to the \taorthae?
She used to regret having stolen that old \arcanum, but now suddenly she wished she knew more, for now she might have to combat the \serpentmen.





\subsubsection{Servants}
Moro had a few servants whom she trusted. 
They helped her and Rian. 
But she did not tell them all her secrets, and she did not bring them out in the city to explore \Malcur.









\subsection{Skills and powers}





\subsubsection{Astrology}
\target{Moro has doubts about her astrology}
Among other duties, Moro acts as court astrologer. 
But she has doubts about her ability. 
She is experienced and has drawn many horoscopes, but she is not sure if she believes in it. 

In her own experience, foretelling anything about a single person is very difficult. 
Individual mortals are too small and insignificant for the stars to care about them, so asking the stars about them does not gain much. 
She follows the textbook guidelines on how to make horoscopes, and she does the best she can to give people something meaningful, but secretly she suspects that she is a fraud. 





\subsubsection{Magic}
\MoroCobrel studied chiefly the \arcanum of \Nasshikerr, \hr{Moro serves Nasshikerr}{whom she served}. 





\subsubsection{\Nasshikerr}
\target{Moro serves Nasshikerr}
Moro's patron god was the \Taortha \hr{Nasshikerr}{\Nasshikerr}. 

She sacrificed to him: 
Her own blood, and that of animals, and occasionally even humanoids. 
She went out into the slums and killed thieves and other scum. 

\Nasshikerr{} was a vain god and demanded that she crawl and beg. 
She struggled with this. 





\subsubsection{Social skills}
Moro has some skill at intimidating and impressing people. 
She makes them see her as a mystic, powerful and frightening character whom they very much want on their side. 
She is seen as powerful and a potentially great ally, or a potentially terrible foe. 
That has helped secure her status in the \ishrah{} (together with the fact that she helped the first \rayuths found Pelidor in the beginning). 

But Moro is actually very insecure about these things. 
She thinks she has no social skills. 
When she has to accomplish something specific and new that requires communication, she gets all nervous and afraid and fears she can't do it right. 





\subsubsection{\Vertex{} potential}
Young Moro was on her way to becoming a \vertex, since she discovered the Beyond and developed the ability to look and reach into it and change it. 
But she recoiled from the truth. 
She buried herself under a mountain of traumata and denial and so lost her chance at \vertex-hood (verticity?). 
She lost herself in the Shroud again. 

Today she is a talented mage and haunted by madness and visions, but certainly no \vertex. 















\section{Sebril Nezra}
\target{Sebril Nezra}
\index{Sebril Nezra}
Sebril Nezra was an \Ortaican \dax who lived in \Yormis.
He was the lover and later husband of \hs{Suthis Mephilex}.















\section{Sseju}
\target{Sseju}
\index{Sseju}
Sseju was a \scatha{} who lived in the \VaimonCaliphate. 

He became a religious leader of his people and helped shape the nascent \hr{Ortaica}{\Ortaican \Bacconate}, which emerged from the ashes of the \hr{Shurcarie}{\Shurcarie}.









\subsection{History}





\subsubsection{Sseju}
Sseju might be a \sphyle.





\subsubsection{Life}
He was \hr{Shurco}{\Shurco} or \hr{Ortaica}{\Ortaican} (depending on whether or not there existed an \quo{\Ortaican} people at that time). 

Sseju was a faithful but very critical Iquinian.
He argued for a religious reformation where \scathae were treated as equal citizens. 
But he did not want to abolish Iquinianism.





\subsubsection{Death}
He gained a rather large following. 
Eventually he was executed by \Caliph \VizicarDurasRespina for heresy. 
As a martyr, he gained even more followers. 





\subsubsection{Martyrium}
The Sentinels had not cared much about Sseju during his lifetime. 
Wannabe religious leaders were a dime a dozen.
But with his great posthumous popularity, the Sentinels realized that they could use him as a martyr.
So they took his name and twisted his story and teachings and used his popularity to build up their new \hr{Ortaican religion}{\Ortaican religion}. 









\subsection{Name}
The name \quo{Sseju} is an anagram of \quo{Jesus}.
There is no message in this.
It is just to fuck with the reader's head. 















\section{Suthis Cruan}
\target{Suthis Cruan}
\index{Suthis Cruan}
Suthis Cruan was a \rethyax \scatha of \Yormis.
He was the father of \hs{Suthis Mephilex}. 















\section{Suthis Dristan}
\target{Suthis Dristan}
\index{Suthis Dristan}
Suthis Dristan was a \rethyax \scatha of \hs{Yormis}{\Yormis}. 
He was a descendant of the \hs{Suthis} clan and the older cousin of \hs{Suthis Mephilex}. 

Dristan was a promising \rethyax, but he was unable to bear the revelation of the \hr{Suthis Innermost Arcana}{Innermost \Arcana}. 
He \hr{Suthis Dristan dies}{went insane and was killed}.















\section{Suthis Mephilex}
\target{Suthis Mephilex}
\index{Suthis Mephilex}
Suthis Mephilex was a \rethyax \scatha of \hs{Yormis}{\Yormis}. 
She was of the powerful \hs{Suthis} clan and a descendant of \hs{Suthis Ondra}, founder of \Yormis. 









\subsection{History}





\subsubsection{Breeding}
\target{Mephilex has Ophidian blood}
\target{Mephilex has XS blood}
\target{Mephilex is an experiment}
Suthis Mephilex had the blood of \ophidians in her, and the blood of \xss.
She was born as a result of the Suthis clan's \hr{Suthis research}{research and experiments}, and she was a very successful breeding experiment.





\subsubsection{Childhood}
Mephilex had been trained from birth by her family as a \rethyax. 
The big movers in Clan Suthis (including her father, \hs{Suthis Cruan}) knew that she was an \ophidian hybrid and a very special child.
This shone through.
Mephilex was always expected to do great things.
They were always pushing her forward. 

She had great support during her childhood, but there was also much pressure on her. 
She was praised and rewarded when she did things right, but whenever she failed at something the people around her (especially her father) would be extremely disappointed in her.

So Mephilex grew up anxious to succeed.
She was very afraid of her own weakness.

Her family was not understanding when she showed signs of weakness.
They were disappointed.
It was as if they expected her to be perfect, and blamed her whenever she proved flawed. 
Mephilex learned that it was important to give an impression of success and to hide all her flaws and weaknesses.
So whenever she uncovered an imperfection in herself she would cover it up, conceal it from everyone, and even repress it from herself. 
This meant that she was afraid to show fear, and that if she failed at something she always tried to retcon her intentions so as to make it look as though she had not really failed at anything.





\subsubsection{Siblings}
Especially Mephilex's father, Suthis Cruan, was hard on her and had great ambitions for her.
This was in part because she was his only child. 
Cruan had fathered other children, but all had died one way or another. 

Some had been experiments with \xs blood, like Mephilex. 
Most of these had died in their eggs or at a very young age.
Some were deformed in body or mind.
Cruan put these failed children to death, for he would not accept such failure from his own bloodline.

Cruan kept secret from Mephilex the fact that he had killed some of her siblings, but she heard hints and whispers from the adults around her, and children are smart, so she figured out the truth.
This made her afraid of her father.
She loved him, but she was deathly afraid to disappoint him.
As a child, she was sure he would kill her the moment she disappointed him.
As she grew up she realized that he was unlikely to just kill her, but the fear had taken root deep inside her, and it stayed. 
All her life there was a part of Mephilex's mind that told her: 
\hypota{If you disappoint your father, you will die.}





\subsubsection{Poetry}
Mephilex was always a lover of poetry.
She began reading poems when she was very young. 
Among her favourite poets were Naakil Shune (a \dax of \Ortaica), Heravada Zukir (a \sphyle of \Ortaica) and \hs{Aburun Dol Cuma} (a Vaimon of the \caliphate). 
She also loved the fragmentary \ophidian poems she found in the \rethyax libraries. 

Mephilex tried to write her own poems. 
But when, as a child, she showed her first efforts to her family, they laughed and dismissed them as foolish and childish. 
It was clear that she was expected to grow up and be serious and do real stuff, not write poetry.
This made Mephilex sad.
She kept reading and writing poems, but in secret. 
She wanted to express herself, but she dared not show her poetry to anyone, for fear that she might be rejected.
She had a great fear of rejection. 





\subsubsection{Friendship with Kish Tulla}
Mephilex's very first good friend was \hs{Kish Tulla} (a \dax of her own age). 
She played with him a lot when she was a toddler. 
But then Clan Kish fell out of political favour with Clan Suthis.
Relations between the clans grew cold. 
Then suddenly Mephilex and Tulla were forbidden to play together and forcibly separated. 
This was a hard blow to young Mephilex, for she was introverted by nature, and it had been somewhat hard for her to form such a close attachment to Tulla.
She had invested a lot in him. 
Now he was taken from her.
It seemed to the young child as if she was being punished for letting another into her trust. 
Subconsciously, Mephilex learned that it was dangerous to form attachments to others. 
She became more introverted and reserved. 





\subsubsection{Youth}
As a young apprentice mage, Mephilex studied under the \rethyaxes of her family as well as other great tutors in \Yormis and abroad.
She was a dilligent and skilled student, but she had her ups and downs.
When she encountered hardship she had a tendency to panic and freeze up, which would only make her performance worse. 

Mephilex's enemies learned of this weakness, and they exploited it.
Worst of these enemies was \hs{Jaslar Thul}. 
She was one of the few who were just as skilled as Mephilex, or better.
Thul resented having Mephilex as competition, so she antagonized Mephilex. 
Thul discovered Mephilex's weaknesses and used them against her.

Mephilex had her own skills, though. 
She knew where her own weak points were, and she realized that her enemies knew how to hit her where it hurt. 
So she began to study her enemies and find out what their weak points were. 
Through study and observation, young Mephilex gradually learned that she could manipulate others by prodding at their weaknesses and subtly appealing to their yearnings and fears. 
She began paying more attention to gossip and social undercurrents, because she realized these were sources of information which she could use against her enemies, in the everyday struggle that was her life. 





\subsubsection{First love}
When Mephilex was 13 or so, she met \hs{Kish Tulla} again.
Tulla was now a dashing young swordsman who aspired to be a great soldier.
In the meantime the clans Kish and Suthis had reconciled, so it was acceptable for Mephilex to associate with Tulla. 

Mephilex remembered her first friend.
While she still felt the pain of having lost him once, she also felt hope.
And she was slowly maturing as a sexual \sphyle.
She began to feel attraction for Tulla.

She and Tulla began to hang out again. 
She had high hopes.
She fell in love.

But Jaslar Thul also wanted Tulla.
And Thul also realized that Mephilex wanted him, and she wanted to take him away from Mephilex. 
So Thul set her rumour-mongering machine in motion.
She made sure that Tulla heard rumours that reviled Mephilex as a heartless manipulative bitch. 
She poisoned Tulla against Mephilex while seducing him herself.

Mephilex was trusting and loved to hang out with Tulla. 
But Tulla came to believe Thul's message: 
That Mephilex was just manipulating him for her own ends.
And the sad part was that it was partially true.
Mephilex had not learned how to make a true and close friend, but she had learned how to manipulate. 
She tried to bind Tulla to her by playing on his hopes and fears, because it was all she knew.

Tulla turned against her, and he turned to Thul.
Thul and Tulla became childhood sweethearts, and Tulla openly scorned Mephilex's advances and reprimanded her for her selfish treatment of him.
He told her she only wanted him for her own sake, to make her feel less miserable and alone and worthless.
To make her feel less of a failure. 
(Thul had told Tulla exactly how to hurt Mephilex.) 
Tulla hurt Mephilex deeply and humiliated her in front of others. 
For weeks she was a laughingstock.

This made Mephilex bitter and hateful. 
She quickly figured out that Jaslar Thul had been the mastermind. 
She had always resented Thul, but now she began to hate her with all her heart. 





\subsubsection{Study abroad}
Mephilex's family believed that that in order to become great, Mephilex should get outside \Yormis for a while and get some diversity in her education.
So they sent her away to study under other masters for some years.
This was her 15th and 16th year.
This was very giving for Mephilex.
She grew as a mage and a \sphyle. 
She even had her first love affair, although she was too perfectionistic and nervous to actually have sex. 





\subsubsection{Back in \Yormis}
Mephilex was 17 when she returned to \Yormis. 
She was now a very skilled young mage.
She mastered the basics and theory of magic, but she had not yet made pacts with any gods (for this had been forbidden her), so she lacked access to powerful magic. 

Here she encountered Jaslar Thul again. 
Thul was older, so in the meantime she had been introduced to the Higher \Arcana (for it was customary in all \Yormis to do this at a certain age.)
Thul had made pacts with her gods, so she was now a much more powerful sorceress.
Moreover, since Thul had lived in \Yormis all this time she had been able to fortify her social position.
She was now an alpha female with a strong social network of friends, supporters and sycophants. 
Meanwhile, Mephilex's network had atrophied while she had been away, so she found herself in a weaker position with fewer and more tenuous friendships. 

Mephilex and Thul still hated each other, of course. 
Mephilex tried to gain more social power in \Yormis through intrigue. 
She had some success.
Thul immediately recognized Mephilex as a threat and moved in to stop her.
They began a war. 

Mephilex' best friend was her cousin Dristan.
She leaned on him for support. 
Thul realized this. 
So when word got out that Dristan had died, Thul used it against Mephilex.
She mocked Mephilex savagely, hitting her where it really hurt. 





\subsubsection{Dristan died}
Some time after Mephilex returned to \Yormis from her studies abroad, her cousin \hs{Suthis Dristan} faced his initiation.
He failed and died. 

This struck Mephilex hard, for Dristan had been her closest friend and confidante. 

Moreover, she felt guilty.
She felt that she should have supported Dristan more.
Maybe if she had stood more behind him, he would have been stronger.
Then he would not have faltered and failed and died. 
Mephilex felt that she was leeching on him. 
That she was using him and sucking validation out of him while not giving enough back. 
She felt responsible for Dristan's death.





\subsubsection{Rulership}
At the time of the \thirdbanewar Suthis Mephilex was the most powerful \rethyax in \Yormis. 
She gained power after \hr{Mephilex rises to power}{a plot by some mages to conquer \Yormis with an army of the undead}.
She was a young \sphyle. 
(At the time \hr{Moro}{\MoroCobrel} left \Yormis, Mephilex was only around 25.)
But she had eaten the flesh of old and mighty mages, using the sorcery of some very deep \arcana.
This way she had subsumed their souls and gained their knowledge, just as these mages had done in their lives. 

So Mephilex possessed knowledge going back many centuries. 
This made her very powerful. 
She was the closest thing \Yormis had to a ruler. 

Mephilex served and worshipped \Thessulax and some undead \ophidians that slept in the tomb underneath \Yormis. 

And she and her cohorts worshipped the loathsome \Ubloth. 
Mephilex-tachi drank \hr{Effluvium of Ubloth}{the effluvium of \Ubloth}, trading sanity and humanity for longevity and power. 

Compare her to the mage Malygris in \cite{ClarkAshtonSmith:TheDeathofMalygris} and Namirrha in \cite{ClarkAshtonSmith:TheDarkEidolon}. 









\subsection{Name}
\quo{Suthis} was her family name and \quo{Mephilex} was her given name. 
She followed the old \Ortaican naming tradition and spurned the customs of the \Tepharites and other upstarts. 









\subsection{Politics}





\subsubsection{Monstrous slaves}
Mephilex and her cohorts had raised up some horrid quasi-humanoid things to serve them.
These were once slave races of the \ophidians and had been entombed with their masters.
Mephilex-tachi had resurrected some of them (having been taught the necessary spells and formulae by the \ophidians), but imperfectly.
Many of the things they raised came back as misshapen monsters. 
They were still obedient slaves, but they were crippled and awful to look upon, and some howled in constant pain. 





\subsubsection{Sebril Nezra}
\hs{Sebril Nezra} was Mephilex's lover and later husband.





\subsubsection{\Thessulax}
Mephilex served the \dragon/\taortha \hr{Thessulax}{\Thessulax}, Queen of the Underworld. 
She learned much of the lore of the dead from \Thessulax. 

And she served the \ophidian \hr{Ishtacca}{\Ishtacca}, who ruled the necropolis beneath \Yormis. 

And she worshipped the loathsome amorphous god \hr{Ubloth}{\Ubloth}. 









\subsection{Physique}
Mephilex was beautiful, but in a cold, unnatural way, like a statue of ice. 
This was due to her \hr{Mephilex has Ophidian blood}{\ophidian blood}.















\section{Suthis Ondra}
\target{Suthis Ondra}
\index{Suthis Ondra}
Suthis Ondra was a \rethyax \scatha. 
He was the founder of \hs{Yormis}{\Yormis} and one of the founders of \Ortaica. 

Ondra sired a great \hs{Suthis} dynasty. 
\hs{Suthis Mephilex} was a descendant. 















\section{\TulionSperra}
\target{Tulion Sperra}
\TulionSperra was a \sphyle of \Yormis. 

\quo{Tulion} was her family name and \quo{Sperra} her given name.

She was the lesbian lover of \MoroCobrel. 
In their relationship, Sperra was always the dominant and Moro the submissive. 

She and Moro \hr{Moro and Sperra investigate Mephilex}{investigated Suthis Mephilex's shady doings}. 
Eventually \hr{Tulion Sperra dies}{Sperra died}. 















\section{\UldraanKerross}
\target{Uldraan Kerross}
\UldraanKerross was a \dax, a \rethyax mage in \Yormis. 
He was once the most powerful and influential mage in \Yormis, the head of its \baccon. 

\Uldraan should be very evil. 
A \trope{CompleteMonster}{Complete Monster} whom the reader hates. 
A sadist, jerk and perhaps paedophile. 

\Uldraan worshipped the god \Ubloth and drank \hr{Effluvium of Ubloth}{its effluvium} to gain mystic power and prolong his life. 
It also affected his sanity and made him more depraved and evil.

At the time of Suthis Mephilex's rebellion, \Uldraan had served \Ubloth for almost 100 years. 
His body was decaying and melting into goo from overindulgence of the effluvium. 
But he was still very powerful and had terrible cosmic forces at his command, for he had communed with the forgotten slumbering gods of \Miith, of whom loathsome \Ubloth was but a fledgling. 























\chapter{Other}
\section{\Criseis}
\target{Criseis}
\index{\Criseis}
\Criseis{} is \ps{\Ishnaruchaefir} life-long servant. 

\hr{Psyrex}{\LocarPsyrex} is her cousin. 









\subsection{Arsenal}





\subsubsection{Languages}
\target{Criseis languages}
\hr{Ishnaruchaefir's languages}{Unlike \Ishnaruchaefir}, \Criseis{} did speak \Velcadian. 





\subsubsection{Senses}
\target{Criseis' senses}
\target{Criseis's senses}
\ps{\Criseis} senses are extremely sharp. 
This is due to both nature and nurture. 
She is an \uber-scatha, much closer to \draconic{} stock than later \scathae, so she and others of her generation have near-\draconic{} senses. 
And already from birth she had better senses than even her peers. 

Since then, she has trained herself to be super-perceptive. 
This makes her a great asset to \Ishnaruchaefir, \hr{Ishnaruchaefir's senses}{whose senses are dulled}. 

\Criseis is \ps{\Ishnaruchaefir} secret weapon. 
She is super-sensitive and can detect things that are supposed to be hidden from everyone.
That is her greatest skill, which she has spent millennia honing.
But it is fairly unknown, because \Criseis has rarely been a visible player in any major events.
\Ishnaruchaefir is infamous mostly for his deeds during the \secondbanewar.
That was enough to make him one of the famous and feared \dragons on \Miith (rivalled only by \Secherdamon), but since then he has done little of note, and \Criseis has done even less.
It is known that she acts as his spy, but no one knows just how skilled a spy and telepath she is.
Her skill also makes her a superb negotiator and manipulator.
She has built up an impressive spy network singlehandedly.









\subsection{History}





\subsubsection{Birth}
She was one of the very first \scathae{}. 





\subsubsection{Saved by \Ishnaruchaefir}
\target{Ishnaruchaefir saves Criseis}
There was some great war, and \Criseis{} was nearly killed. 
\Ishnaruchaefir, in a fit of compassion, decided to save her. 
From that moment she dedicated herself to him and vowed to serve him forever. 

This was long before the \secondbanewar, back when \Ishnaruchaefir{} was still idealistic. 
Before he became bitter and withdrawn. 

Her lord gifted her with \xzaishannic{} blood, granting her immortality and \daemonic{} powers. 

Compare to \bandsong{In Slaughter Natives}{As My Shield}. 





\subsubsection{Siblings died}
\target{Criseis's siblings}
\Criseis{} once had a brother and a sister. 
Both of them were hired by \Ishnaruchaefir{} along with her. 
For a while (thousands of years) they all three worked for him. 
But then some enemies (\resphain{} or \dragons) kidnapped one, hoping to use her as leverage against \Ishnaruchaefir. 

\Ishnaruchaefir{} did not respond to the bait. 
He let the enemies kill her. 
But he exacted a terrible revenge. 
He waged a centuries-long campaign of death and horror against those who had wronged him, and their entire families. 
He would hide out in his Mirage Asylum and strike from the shadows wherever he could do the most harm. 
His continued terrorism ultimately made his targets so frustrated and desperate that they started doing really stupid things. 
Eventually he was able to kill them. 
Then he was satisfied and retreated to his Asylum. 

He was ruthless and cruel, and caused immense suffering for innocent civilians. 
(\Resphan{} and \human{} lives are \emph{worthless} to him, remember.) 
But at the end, everyone knew not to fuck with him or his civilian servants. 

Then, later, the same happened with \ps{\Criseis} brother. 
Again \Ishnaruchaefir{} exacted his revenge. 

Now only \Criseis{} remained. 
It did not happen again. 









\subsection{Personality}





\subsubsection{Coping with immortality}
Being immortal is not always easy for \Criseis. 
\Dragons{} and the like are used to it, but she is a \scatha, and \scathae{} are not meant to be immortal. 
She gets too attached to her mortal companions and then sees them die. 

But \Criseis{} is strong. 
She copes and does not \trope{Wangst}{Wangst}
She has seen too many people die. 
She knows how horrible that can be. 
She has no desire to die herself. 





\subsubsection{Fears the \dragons}
\target{Criseis fears Dragons}
The \dragons \hr{Dragons have dark knowledge}{wielded some very dark knowledge}. 
Even \Criseis shuddered to merely think of the terrible Aenigmata that her master spent his centuries pondering.
She knew the Universe was vast and dark and cruel, but she did not want to think about it.
She was still just a \scatha, even though immortal.
She still thought like a mortal, and she wanted to keep it that way.
She knew if she tried to think like a \dragon it would destroy her.





\subsubsection{Happiness}
\Criseis was fairly happy because she had a special talent for inner peace. 
She said that she had met very few people who had her talent\dash her \emph{serenity}, as she called it.

She had read \Sethicus's warnings, and she suspected that the great polymath might have had the talent. 
But no other \dragon appeared to share it. 
Certainly not \Iscrafel or \Secherdamon or \Nzessuacrith. 





\subsubsection{Role}
At times, she will defend her lord against verbal attacks from others. 
\ta{%
  You know nothing of my lord. 
  You cannot imagine the suffering he has had to endure\prikker and the burden he still bears.} 

I can use her commentary to give insight into \ps{\Ishnaruchaefir} personality, while keeping him aloof and \trope{Badass}{badass}. If he had do say these things himself, it would come off as whining, \trope{Wangst}{Wangst} and \trope{Spikeification}{Spikeification}. 

She is worried whenever \Ishnaruchaefir{} starts talking to his dead \Triestessakhin, through the \hr{Ishnaruchaefir's glaive}{glaive} or otherwise. That is usually a bad sign, a sign that her lord is about to do something reckless. 





\subsubsection{Social skills}
\Criseis{} is one of the most skilled Sentinels there are. 
A great manipulator and diplomat. 
\Ishnaruchaefir, who \hr{Ishnaruchaefir nerd}{lacks the same social eptitude}, often leaves her in charge of matters that require social interaction. 





\subsubsection{Training}
\Criseis{} has always trained rigorously to make her body and mind stronger, smarter, more knowledgeable and more perceptive. 
For several reasons:
\begin{enumerate}
  \item She loves her master and wants to serve him as well as possible.
  \item She is often in danger when accompanying her master. 
    She does not want to be a burden that he has to worry about defending. 
    She wants to be able to take care of herself and take some burden off his shoulders. 
  \item She does not want to die, for more than one reason:
    \begin{enumerate}
      \item 
        She fears death, obviously. 
        She only has \hr{Lesser immortality}{\quo{lesser immortality}}, so if her body dies she is gone. 
      \item 
        Her master needs her and her positive influence. 
        She fears he will become more miserable and evil if she is not there to cheer him up and encourage him to be less evil. 
      \item 
        She fears that if she dies her master will go on a rampage, like he did when \hr{Criseis's siblings}{her siblings} died.
        So she is protecting not only her own life, but those of many innocents as well. 
    \end{enumerate}
  \item 
    She likes to train. 
    She has gotten used to it. 
    It gives her some measure of peace and confidence. 
    And it allows her to temporarily flee from and abstract from her lord's pain, which she feels through him. 
\end{enumerate}





\subsubsection{Tries to contain \Ishnaruchaefir}
\target{Criseis contains Ishnaruchaefir}
\Criseis{} fears \ps{\Ishnaruchaefir} destructive and evil behaviour. 
She tries her best to contain his destructive tendencies and prod him in a more compassionate direction. 
For this reason, she often asks to \hr{Ishnaruchaefir and Criseis}{come along with him}. 










\subsection{Politics}
\Ishnaruchaefir{} sees \Criseis{} as something akin to a daughter. 
\Nzessuacrith{} knows this, and she hates \Criseis{} for it. 
\Criseis{} fears \Nzessuacrith. 

\target{Master Quessanth}
She calls \Ishnaruchaefir{} \quo{Master \Quessanth}. 
He lets her call him by his first name because she is almost a daughter to her. 
Actually, he doesn't insist on the title, but she is submissive because obedience is bred into her \scathaese{} blood, flesh and bones. 





\subsubsection{Family}
\Criseis{} \hr{Criseis's siblings}{once had siblings, but they died}. 

She also has a bunch of descendants, and nephews (descendants of her siblings). 
See, in her youth she was fertile and gave birth to a bunch of children. 
But her immortality was flawed, and some parts of her body could not replenish themselves endlessly, so after some centuries she went infertile. 
Her children multiplied and lived and died. 
Now, after thousands of years, they are many generations from her, but they are still her descendants. 
Some of them work for \Ishnaruchaefir{}. 
These know and remember her and affectionately call her their \quo{matriarch}. 









\subsection{Physique}





\subsubsection{Appearance}
\target{Criseis's appearance}
She looked like a \scatha. 
An old, worn-out \scatha, but still a healthy, unbowed, living \scatha.
\hr{Psyrex' appearance}{Unlike \LocarPsyrex}, she was not undead, \hr{Criseis's immortality}{merely immortal}.










\subsection{Scenes}
\subsubsection{Musing}
Have a scene where \Criseis{} muses: 
\tho{%
  My lord is always alone. That is why he took me as his eternal companion. But even with me, he is still alone.}

She has guilt and angst about being unable to truly console \Ishnaruchaefir{} and heal his grief and pain. 










\subsection{Skills and powers}





\subsubsection{Immortality}
\target{Criseis's immortality}
\Criseis was immortal. 
She had \hr{Dragon blood gives immortality}{drunk her master's blood}, which had granted her immortality.

But \hr{Psyrex' undeath}{unlike \LocarPsyrex}, she was not pumped full of unnatural arcane power. 
\Criseis was a skilled mage, but \Psyrex was a demigod.
So she was not undead. 
She \hr{Criseis's appearance}{looked like a normal \scatha}.





\subsubsection{\TrueDraconic}
\Criseis could speak \TrueDraconic, but she only spoke it when she had to (usually only to cast spells).
It was \hr{True Draconic costs sanity}{hard on a mortal mind}. 















\section[Locar Psyrex]{\LocarPsyrex}
%\sectioncharunspec{\Psyrex}{\scatha}{\male}
\target{Psyrex}
\index{\LocarPsyrex}
\index{\Psyrex}
\LocarPsyrex{} is a \scathaese{} Sentinel \chaos{} sorcerer, a servitor of \Secherdamon. 

In \emph{\TwilightAngelRemember{}}, he is active in \Malcur, working to bring \Nithd{} to \Miith{}, pulling the strings of the \Malcuric{} Sentinels and the criminal underworld in order to set up the ritual. 

\Psyrex{} is a \vertex{} in \ps{\Secherdamon} \matrix. 

He should be portrayed as a \trope{Badass}{badass} \trope{MagnificentBastard}{Magnificent Bastard}. Compre to Haazeel Thorn from \FLuneNoire. 

%His true name is Locar Orchorod \Psyrex. 

\hr{Criseis}{\Criseis} is his cousin. 









\subsection{Species}
\LocarPsyrex might be a \scatha or he might be a \quiljaar. 









\subsection{Physique}





\subsubsection{Appearance}
\target{Psyrex' appearance}
\Psyrex was undead, \Lich-like or mummy-like. 
His face was horribly scarred and disfigured from millennia of decay.
His frail mortal body could not contain the vast sorcerous power invested in him, so it warped and changed shape.
It was still strong enough to be useful, but not very strong.

\index{beard!\LocarPsyrex}
\Psyrex{} wore a false beard (one of those long, thin Chinese beards) out of eccentricity. 








\subsection{Arsenal}





\subsubsection{Power}
\Psyrex{} is a very powerful archmage.

Of all \scathae, none is more \draconian{} in nature than \LocarPsyrex. 
So it is said. 





\subsubsection{Throne}
\target{Psyrex's throne}
In his home in his \daemonic{} halls, he sits on a great, monstrous throne, wreathed in mystic mists. There he lays his sinister plans and commands his \daemonic{} minions. 

Sometimes he deals with \daemons{} who are as powerful as he or greater. Then he must negotiate. 

\lyricsbs{Arcane Wisdom}{Maelstroms of Majestic Night}{
  I, Lord of all Winterstorms, \\
  reign opaque and serene \\
  from my sovereign throne o' ruthless might.
  
  Amongst the shadow of ev'ry raven wing.\\
  Obsidian black in the moon-misty sky.
}





\subsubsection{Undeath}
\target{Psyrex's undeath}
\target{Psyrex' undeath}
\Psyrex was undead.
He \hr{Psyrex' appearance}{looked \Lich-like or mummy-like}. 

He had \hr{Dragon blood gives immortality}{drunk his master's blood}, which had granted him immortality.
But this was not necessarily the nice kind of immortality. 
\ps{\Psyrex} frail mortal body could not contain the vast sorcerous power invested in him, so it warped and changed shape.
It was still strong enough to be useful, but not very strong.

He was \hr{Criseis's immortality}{far more powerful than \Criseis}, and even \hr{Tess-Hanith's undeath}{\TessHanith}. 
\Criseis was a skilled mage, but \Psyrex was a demigod.

Compare him to a \hr{Rissitic Liches}{Rissitic \Lich}. 









\subsection{History}
\subsubsection{Ancestry}
He might be a \rachyth, a descendant of \Secherdamon. Or he might be half \daemon. 

He might be an ancient servitor of the \dragons{}, like \hr{Criseis}{\Criseis}. 
He addresses \Criseis{} as \quo{sister}. She is not happy about it. 

Have more references to how \Psyrex{} allegedly has \draconian{} and/or \daemonic{} blood. The truth is unknown. Some people say \draconian, others \daemonic, others both. 





\subsubsection{He became \ps{\Secherdamon} immortal servant}
He was elated and ecstatic when \Secherdamon{} chose him to be his immortal servant. 

\lyricsbs{Hate Eternal}{Hell Envenom}{
  Hell Envenom as I enter the portal of flames.\\
  I descend upon the lord of chaos.\\
  Onward I now transcend.\\
  Upon thou art my master.
  
  My time has come to perish.\\
  I summon thee, redeemer of pain and despair.\\
  Thy infernal ways, I call upon you all.
  
  Hell Envenom as I bow down and revere the beast.\\
  The vast inferno now reveals thy path.\\
  Onward I now transcend.\\
  Cometh forth thy master.
  
  I fear not what lies before me.\\
  Forever black, the skies upon us are, \\
  here no nether more.\\
  For behind him I shall follow.\\
  I covet and embrace the flood of the eternal sea, \\
  here nor nether more.\\
  For behind him I shall perish,\\
  here nor nether more.
  
  Thus we cascade into the chasm.\\
  Thus we fall into the abyss.\\
  Thus we pass across the shadows.\\
  Thus we cometh before thy kingdom.\\
  I shall fall before you and kneel before thy throne.\\
  I present my self amongst the departed.\\
  All mighty one, may the flame burn through the ages.\\
  May thy flame burn!
}









\subsection{Personality}
\Psyrex{} may be something like a Lizard Mage from the Palladium RPG. 





\subsubsection{Bureaucrat}
\Psyrex was a powerful mage, but not as powerful as a \quiljaaran.
He was still just a glorified \scatha. 
His role was mostly that of bureaucrat and administrator of the Dark Crescent. 

There were \quiljaaran affiliated with the Dark Crescent. 
These worked with \Psyrex, but did not serve him.
They had their immortal pride. 





\subsubsection{Make him darker}
\target{Psyrex darker}
\Psyrex should be darker than in my initial portrayal.

Give him reputation for being sinister and malevolent and wicked.
Perhaps keep his master's identity secret. 
Have him address 

Compare him to Angsaar:

\lyricstitle{\href{http://www.bal-sagoth.co.uk/glossary.html}{Bal-Sagoth Glossary}}{
  \textbf{Lord Angsaar:}
  The Dark Liege of Chaos; an immortal sorcerer of awesome malevolent potency, twisted by dreams of galactic domination and vengeance. Also known as The Scourge of Lemuria, the Bane. of the Atlantean Kings, and the Arch-foe of the Immortals of Ultima Thule.
}









\subsection{Politics}
\subsubsection{The Dark Crescent}
\Psyrex{} is well-known and infamous as the leader of an \Azmith-spanning cult, the \hs{Dark Crescent} Order. The Order is widespread and has much power in certain parts of \Velcad{}, despite the best efforts of the Iquinian churches at suppressing it. The cult is capable of raising a quite impressive army when it needs to (but its enemies do not know this, only suspect). 

\lyricsflnv{4}{
  Wismerhill: \ta{Who in the world is he?}
  
  Hellaynnea: \ta{The Lord of the Black Moon. The most powerful man after the Emperor! An archmage and nearly a demigod.}
}















\section{\DulNepherRamas}
\target{Dul-Nepher-Ramas}
\index{\DulNepherRamas}
\DulNepherRamas was a \sphyle, a \hs{Mysteriarch} of \Shurco.























\part{\Human Characters}























\chapter{Scions}















\section{\Belzir}
\target{Belzir}
\index{\Belzir}
A Vaimon of \ClanGeican and the last \Calipha of the \VaimonCaliphate, \Belzir{} was also a Scion, an incarnation of the \Malach{} \Shiaraid. Her rule was controversial and she was considered by many a heretic and an evil menace. \Belzir{} had several husbands and dozens of lovers and bore 28 children. The most famous of these was Zacrias, who would eventually become Lord of \ClanGeican after her death. 

She kept herself alive and young far beyond a normal lifespan, even for Vaimons, using life-draining necromancy. Eventually this transformed her into an undead \Reaver. 

Her controversial acts caused many Vaimon nobles to plot and rebel against her. Ultimately, a great civil war broke out which became the undoing of the \VaimonCaliphate\dash the \darkfall. In the \darkfall{}, \Belzir{} was slain, but her soul managed to stay alive in the Dreamlands. 

After a thousand years, \Belzir{}, now a disembodied soul, learned to \hr{Telepathy}{telepathically} contact the living on \Miith{}. Planning her resurrection and return to the world of the living, she contacted Geican politicians and formed the Royalist Faction. Unbeknownst to her, the Faction became infiltrated by the Cabal. 

\begin{description}
  \item[Name:] {\Belzir} daughter of {Cormin} of \ClanGeican. She styles herself \Calipha of the \VaimonCaliphate and sometimes uses the title of Dark Prophet. 
  \item[Race:] Vaimon. 
  %\item[Alignment:] Chaotic evil. 
  %\item[Size:] 175 cm (tall for a Geican woman), average weight. Strong and fit, but not overtly muscular. 
  \item[Appearance:] 175 cm (tall for a Geican woman), average weight. Strong and fit, but not overtly muscular. \Belzir{} is exceptionally beautiful and very much aware of it. Her demeanour is one of power, arrogance and sexual allure. Her hair is black and wavy. 
  \item[Symbol:] She needs to have one\prikker 
\end{description}









\subsection{Arsenal}





\subsubsection{Binding souls}
\target{Belzir binding souls}
\index{\carcer!\Belzir}
\hr{Malachim binding souls}{As a \malach}, \Shiaraid{}/\Belzir{} had a \sephirah-like power to bind souls to her. 

At first she felt little of the bound souls (\hr{Ramiel binding souls}{unlike Ramiel}) because she had few of them yet. 
\Delphine{}, her first incarnation, \hr{Delphine binding souls}{had bound almost zero souls} (virtually only that of \Eryal). 
But \Belzir{} had a relatively long life and a large body count, so her \carcer{} got filled up with plenty of souls. 
But most of them were expended and destroyed when \Belzir{} fell. 
This eruption of power \hr{Belzir's Carcer unleashed}{helped cause the \HundredScourges}. 









\subsection{History}





\subsubsection{Reign}
\Belzir, as a \sathariah, was vulnerable to \hr{Curse}{\NexagglachelsCurse}. 
She succumbed to the curse. 
She indulged in hedonism and luxury, and hated the Cabal. 
She secretly wanted to destroy the Vaimons, who were an extension of the loathsome \banes. 

\lyricslimbonicart{Lycanthropic Tales}{
  When my inner evil confessed,\\
  the devil shaped me into form.\\
  In a perfect trance, a dark romance,\\
  in the silence of the night,\\
  I adapted to the beast in my soul.
  
  Lycanthropic tales.\\
  The dark demon prevails.\\
  Lycanthropia.\\
  A seed of moon dementia.\\
  The dark demon prevails.\\
  Lycanthropic tales.\\
  A call from the beginning of time.
}




\subsubsection{No one knew she was a Scion}
\target{No one knew Belzir was a Scion}
Due to \hr{Shiaraid's stealth}{\ps{\Shiaraid} stealth} no one, except \Belzir{} herself, knew \Belzir{} was a Scion. 
Not until late in her life when she reached her \Apotheosis. 
And even then, no one knew that she was the same as \Delphine. 

But maybe she told Zacrias near her death. 
Because Zacrias went on to name a daughter \quo{\Delphine} in her \honour. 





\subsubsection{\Apotheosis}
\Belzir was \hr{Belzir's Apotheosis}{the first Scion ever to achieve \Apotheosis}. 





\subsubsection{Death and imprisonment}
\Belzir discovered that there was a system of \bane ruins hidden beneath the Redcor's \TopazChateau.
She \hr{Belzir goes below Chateau}{went there}, hoping to gain its power, but failed and died and was imprisoned by the Redcor. 





\subsubsection{Afterlife}
After her death, \Belzir{} was not reborn like a normal \Malach. 
Why not? 
Because she had discovered too much about herself and was able to keep her soul afloat. 
Unfortunately, she was not able to fashion or inhabit a new body, so she was cast adrift in a Limbo-like void, where she lay for over a thousand years. 
She suffered horribly there and went mostly mad.

\lyricsbs{\Duana}{
  \href{http://www.necrobabes.org/duana/chrysallis.html}{Chrysalis}
}{
  In the dimension of a dream \\
  she floats through walls \\
  and empty places \\
  where hungry voids \\
  like mouths \\
  swallow, chew and spit her \\
  fragments to the solar winds \\
  forked tongues \\
  speak to her in flavors \\
  and whispered halos \\
  with breath as soft as \\
  pussy \\
  willows
  
  she dwells in black holes of \\
  non-existence \\
  on the edge of the \\
  event horizon \\
  slow-dancing with death \\
  and pressing his body close \\
  to fill her hungry void \\
  silent syllables of laughter \\
  ripple from her rose-petaled lips \\
  like ether and watery kisses \\
  like sewers and the plague
}

She lies \quo{dead and dreaming}. 

\lyricsbs{Arcane Wisdom}{Maelstroms of Majestic Night}{
  Black is my soul as it withers. \\
  Black is my heart as it dies. \\
  Black is my blood as it freezes. \\
  Black is my silhouette as it lies\prikker 
}





\subsubsection{The ghost of \Eryal}
\target{Shiaraid and the ghost of Eryal}
After \Shiaraid{} ate \Eryal, she \quo{sort of} lived on as a ghost in \ps{\Shiaraid} \carcer. 
From time to time, \Shiaraid{} would seek out \ps{\Eryal} ghost in order to find love and peace (if only momentarily). 
It worked, to some extent. 

\citebandsong{BeyondTwilight:FortheLoveofArtandtheMaking%
}{%
  Beyond Twilight%
}{%
  For the Love of Art and the Making%
}{
  Flesh to flesh, soul to soul\\
  Deeply bound, deaf to all the sound\\
  Slowly she goes to that place\\
  that allows, allows her to be free\\
  Love so pure it overwhelms me\\
  
  I will devour her in our \\
  sacred blood, I will capture her.\\
  Flesh to flesh, soul to soul\\
  Deeply bound, darkness in our eyes\\
  Love so pure it almost overwhelms me
}





\subsubsection{Gradually awakening}
\target{Belzir keeps in touch with Royalists}
Even dead and imprisoned, \Belzir was powerful enough to maintained a telepathic link to her most loyal followers. 
She was only rarely able to contact them, for she did not have much consciousness in her dead state. 
She was dreaming.

She could not contact people on her own initiative. 
In order to contact her you had to pray to her and hope for a reaction. 
To her followers she appeared as an aloof, distant god. 

Many left her, but some loyal ones still worshipped her. 
They became the \hs{Royalist} Faction. 

\target{Belzir awakening}
For hundreds of years the Royalists tried to free their queen, but they were too feeble to challenge the powerful Redcor. 
But gradually they were able to do small things to help their queen awaken. 
They founded a \hr{Cult in Redce}{cult in \Redce}. 
Through it they were able to weaken the Redcor barriers surrounding \Belzir's prison.

Shortly before Carzain \Shachar arrived in \Redce, the \hr{Royalist breakthrough in Redce}{Royalists achieved a breakthrough}. 
They had now drilled so many holes in \Belzir's prison that she could now also contact the non-Royalists in dreams.
She began to be able to use mind-controlling magic on the Redcor. 

This was why \hr{Redcor need Carzain}{the Redcor needed Carzain}. 





\subsubsection{Relationship with Carzain}
She tried to seduce Carzain and use him to resurrect her. 

\lyricsbs{Arcane Wisdom}{Maelstroms of Majestic Night}{
  Fogbound, through malevolent disguise, \\
  I dreamt in cold waters \\
  of ancient realms darkly enthron'd. \\
  Whilst cold ice-winds stormed the mountain, \\
  under a profane enchantment I sentenc'd all to live\prikker \\
  or perish. 
}









\subsection{Personality}





\subsubsection{Sorrow}
\target{Belzir's sorrow}
She feels great sorrow and pain in her incarceration Beyond, and it radiates from her. 
She has lines of tears under her eyes. 
Kind of like a stereotypical goth girl.

Part of her sorrow stems from guilt. 
She remembers that in her past life as \Delphine, \hr{Silqua dies}{she killed her beloved Silqua}.
She \hr{Delphine's guilt}{hated herself for it}, and still does. 









\subsection{Politics}





\subsubsection{Idolized and demonized}
There are legends about \Belzir that go both ways. 

Today, in Geica, there are \hs{myths} about \Belzir{} as a beautiful, sad queen who sleeps and suffers, waiting for a hero who is worthy of her to come and wake her. She was a heroic saviour figure overthrown by evil ones and now being unjustly punished and imprisoned\dash perhaps tortured, in a sexy way. 

Maybe she appears naked, impaled and nailed to a tree of thorns. 

\lyricsbs{Karl Sanders}{Of the Sleep of Ishtar}{
  Ishtar unlocked the seven gates. \\
  Covenant of Absu. 
  
  Impaled, naked and Bleeding. \\
  Hung left to die upon a Stake. 
  
  Ishtar, suffer in Darkness. \\
  Unheard you cry, \\
  cry for dawn,\\
  cry for dawn,\\
  cry for dawn.
}

\lyricsbs{Monolith Deathcult}{Den Ensomme Nordens Dronning}{
  Sleep, oh Majesty, surrounded by massive darkness. \\
  Her skin torn apart by sub-zero claws. \\
  Buried deep in thy ice dungeon. \\
  Sleep, oh Majesty, Lonely Queen of the North.
}

She lies betrayed, betrayer and bleeding. 
She waits for one to awaken her. 

\lyricsbs{Monolith Deathcult}{Den Ensomme Nordens Dronning}{
  The huntress from the land of Magog, \\
  manacles on her seabed. \\
  Mauled in the deep she waits \\
  for the Czar of Mesech and Tubal. \\
  The blood of the 118 drips \\
  from her glowering gaping mouth. \\
  She patiently bears suffering in the Barents abyss.
}

Some saw/see her as the proud flower of the \VaimonCaliphate, the finest representative of a people at the peak of civilization, a hero from the golden age of \humanity. 
And now the world's saviour. 

\lyricsbs{Monolith Deathcult}{Den Ensomme Nordens Dronning}{
  She is the bride of Gog, a post-Soviet Czarina. \\
  The Czardom's silver bow, \\
  CzArtemis leads a fleet in bloom.
}

She is romanticized as a sleeping princess. 
Even the democratic Geica idolizes her because she is a useful figurehead that can be used to channel the people's patriotism. 
This backfires\prikker 

\lyricsbs{Monolith Deathcult}{Den Ensomme Nordens Dronning}{
  In a solitude of the sea \\
  Deep from human vanity, \\
  And the Pride of Life that planned her, \\
  stilly couches she. 
  
  And now the Queen and Antaeus lie dead \\
  The twain forever converged on the seabed. 
}

Among non-Geicans, \Belzir{} is demonized as the most evil woman ever. 
She is a vampire, a monster, a boogey(wo)man used to frighten children.

\lyricsbs{Marduk}{Nightwing}{
  And the mantel of power should be shouldered by the firstborn\\
  The one who crave evil and all kinds of human feelings scorn\\
  He who drank his fathers blood and leaves his foes ripped and torn\\
  And which the king halls up high since long forlorn
  
  Nightwing\dash{}storm through eternity\\
  And rip asunder those who fall for the human mockery\\
  Within the massive castlewalls lurks the evil now again\\
  Which with wise men made a truse by giving the blood of gods best men\\
  He who possessess the gift which sends shivers down the spine\\
  And awaken people who step away from the mortal worlds decline\\
  He is the fierce creature which the angels fear to chase\\
  Who see pain as passion and lives at war with the mortal race
  
  Pulled from our frail existence by the claws of death\\
  To defy the scythe and feel the reapers breath\\
  To walk the tunnel backwards when you first life has been slain\\
  And when all mortal feelings has ceased to cause you pain\\
  For I am darkness and so you shall be\\
  As on the nightwing you ride with me
}

\lyricsbs{Emperor}{The Ancient Queen}{
  The Ancient Queen. \\
  Deadly hate, deadly love. \\
  Dark under the shadow. \\
  Death has been resisted. \\
  The Ancient Queen, \\
  ruler of the domain. \\
  Keeper of the fury, \\
  the Queen dwells in shadow.
}

\citeauthorbook[p.216--218]{RobertEHoward:HouroftheDragon}{Robert E. Howard}{%
  Hour of the Dragon%
}{
  \ta{You have heard of Princess Akivasha?} inquired the girl on the couch.
  
  \ta{Who hasn't?} he grunted. 
  The name of that ancient, evil, beautiful princess still lived the world over in song and legend, though ten thousand years had rolled their cycles since the daughter of Tuthamon had reveled in purple feasts amid the black halls of ancient Luxur.
  
  \ta{Her only sin was that she loved life and all the meanings of life,} 
  said the Stygian girl. 
  \ta{To win life she courted death. 
    She could not bear to think of growing old and shriveled and worn, and dying at last as hags die. She wooed Darkness like a lover and his gift was life-life that, not being life as mortals know it, can never grow old and fade. She went into the shadows to cheat age and death\dash}
  
  \prikker 
  
  And through his fear ran the sickening revulsion of his discovery. The legend of Akivasha was so old, and among the evil tales told of her ran a thread of beauty and idealism, of everlasting youth. To so many dreamers and poets and lovers she was not alone the evil princess of Stygian legend, but the symbol of eternal youth and beauty, shining for ever in some far realm of the gods. And this was the hideous reality. This foul perversion was the truth of that everlasting life. Through his physical revulsion ran the sense of a shattered dream of man's idolatry, its glittering gold proved slime and cosmic filth. A wave of futility swept over him, a dim fear of the falseness of all men's dreams and idolatries.
}














\section{\CarzainShachar}
%\sectioncharlivelong{Carzain \Deracille{} \Shachar}{\human}{\male}{Carzain}
\target{Carzain}
%Male \human{} \birthtonow{Carzain}. 
A Vaimon mage living in \Redglen{} in Pelidor. 
Her father is Nishain \Shachar{} and her mother is \Roanne{} \Deracille. 
She was taught magic by her parents. 

Carzain is actually a Scion, an incarnation of the \Malach{} Ramiel. 









\subsection{Equipment}





\subsubsection{Ring of Eryx}
\target{Ring of Eryx}
\index{Eryx!Ring of Eryx}
The Ring of Eryx was an ensorcelled ring that \CarzainShachar stole from the acolytes in the \hr{Fane of Kul-Yana}{Fane of \KulYana}.









\subsection{History}





\subsubsection[Taught by Weylon]{Taught by {\Weylon}}
As a child, Carzain befriended \hr{Weylon}{\rah[\Weylon]}, an old retired Tiger knight. 

\Weylon{} taught her to fight with swords. 
So Carzain had learned from childhood how to fight. 
This came in handy later. 





\subsubsection{Before the Runger War}
Carzain is not completely young and inexperienced in the Runger war. 
She has been to war before. 

The mutiny in Heropond happened several years earlier. 
Back then, \hr{Carzain joins the army}{Carzain joined the army} because she hoped it would help her find out what was happening to her, who she was and who Vizicar was. 

In the intermittent years, Carzain has \travelled the land as an adventurer, working her way towards \kenosis. 
She has often worked as a mercenary or a bounty hunter, killing monsters for rewards from nobles and wealthy citizens. 
Sometimes even helping those in need, if she feels like it. 

Several years later, Carzain heard the rumour that Runger was invading her homeland. 
She discusses this with the voice in her head, and they decide to join the war. 





\subsubsection{Youth and frustration and adventuring}
Throughout her youth Carzain was haunted by an undefinable frustration, a faceless dread, a hunger. 
She longed for a purpose. 
She felt that she was destined for greatness, but she did not know how.
She achieved many successes, but they all felt hollow. 

Hence she roams around as an adventurer for 15 years (from age 20 to 35). 





\subsubsection{\ZeethanKraal}
Carzain met the \caisith sorcerer \hr{Zeethan Kraal}{\ZeethanKraal} and became his apprentice.
Carzain would do tasks (quests) for Kraal, and in return Kraal would teach her sorcery and Arcana and Gnosis. 





\subsubsection{\Moongod mission}
\ZeethanKraal sent Carzain on a quest.
Here Carzain encountered \hr{Glithid}{\glithids} and a dread \hr{Moongod}{\moongod}. 

Kraal told Carzain that \hr{Moongods on Miith and on Visha}{the \moongods did not look on \Miith like they did on Visha}. 

Kraal asked Carzain to perform a magical ritual that would summon a \moongod because Kraal wanted to study it.
The \moongod destroyed a town.
Carzain confronted Kraal with this. 

\begin{prose}
  Kraal: 
    \quo{%
      Oh, was there a town there?
      Hm.
      It was not there last time I checked.
      Perhaps I should have investigated that.
      At times I forget how much you mortals move around in a couple of centuries.}
\end{prose}

Mortal lives meant little to Kraal.

When Carzain saw the \moongod, everything was a shapeless chaos.
Nothing made sense.
She could not find head or tail in anything. 
The \moongod was enormous, sickly white in colour. 
That much she knew.
But she could not comprehend its form. 
It was like a tree\dash no, like a giant humanoid without arms\dash no, like the moon of Visha itself\dash no, all three at once and yet none of them. 

The \glithids were a horde of separate living creatures, and yet they were all part of the \moongod\dash or they were a part of the forest floor, the background upon which the god danced. 
At times Carzain felt as if she herself were one of the dancers\dash as if she were \emph{one with} the dancers. 

This was because the Shroud was so thin.
The arcane ritual had temporarily weakened the Shroud. 
Only thus could the \moongod come to \Miith and dance.






\subsubsection{Lumica}
\target{Carzain and Lumica}
Carzain developed a rivalry with the sorceress \hs{Lumica}.
This was a problem, for Lumica was a queen; her husband was \hs{Morn Ossan}, a king. 

But Carzain called in favours with her old ally \hr{Zeethan Kraal}{\ZeethanKraal}.
And she snuck into Lumica's tower and seduced her slave girl \hs{Taya}.
With their help, Carzain unearthed evidence that Lumica was plotting behind her husband's back.
Thus Carzain was able to manipulate Morn Ossan and turn her against Lumica.
Eventually Carzain killed Lumica. 

Then Carzain took up residence in Lumica's tower.
This was what she wanted from the start. 






\subsubsection{Learned dark secrets}
At the time of the \hs{Runger war}, Carzain had learned some \trope{CosmicHorror}{Cosmic Horror} revelations.
Most of them were things from Vizicar's or Tydesmos's time which she had come to remember. 

Later she would learn many more. 





\subsubsection{The Fane of \KulYana and the Ring of Eryx}
\target{Fane of Kul-Yana}
\index{\KulYana!Fane of \KulYana}
\index{Eryx!Ring of Eryx}
A party of heroes wanted to slay the evil acolytes of \hr{Llorgul}{\Llorgul} that dwelt in the feared Fane of \KulYana. 
The acolytes kidnapped people and used them as sacrifices. 
Everyone nearby feared them, for the acolytes had sorcery and unearthly allies. 

\CarzainShachar joined the heroes because she coveted the ensorcelled \hs{Ring of Eryx} which the acolytes possessed.
The heroes all ended up dead, but Carzain escaped with the ring.






\subsubsection{Runger War}
At the beginning of the Runger war, Carzain is already close to full \kenosis. 
Maybe she \emph{has} achieved perfect \kenosis. 
She is ready to grasp for \apotheosis{} now.
Vizicar knows they are more than a \human, and he tells Carzain this. 
Vizicar wants to attain \kenosis{} and regain his body, power and memory. 
Then he wants to achieve \apotheosis{} and learn his true nature.
Then will they be a god. 

Carzain is in eastern Pelidor when she hears the news. 
She decides to go do some scouting/reconnaisance before riding to the army. 
So she heads further east and sniffs out the Rungeran army. 
She finds a destroyed village and smells evil magic there. 

Then she goes into the forest, finds the army camp and spies on \hr{Takestsha}{\Takestsha}. 
\emph{Maybe} Takestsha discovers Carzain. 
If so, Carzain gets to show her badassitude as she fights her way out of it. 

In these early Carzain chapters, have some references to the \hr{Myths of vanquished monsters}{myths of Iquinian heroes vanquishing inhuman Elder Races and monsters}. 
Carzain gets \quo{Elder} feelings from the \hr{Takestsha's Eresh-Kali magic}{\EreshKali{} magic}, just like \hr{Jirad Tantor}{Tantor} did. 
Carzain and Vizicar are unsure of those myths and have a feeling that the Elder Races still live, and that humanoids do not dominate the world as much as they think they do. 

Then she \hr{Carzain returns to Forklin}{goes to \Forclin} to meet \hs{Archibald Curwen} and rejoin the army, bringing news with her. 

When Carzain dreams, she \hr{Satharioth dream of Ophidian eyes}{dreams of \ophidian eyes}. 
The eyes of \Nexagglachel. 
Carzain vaguely remembers \hr{Nexagglachel's hatred}{\ps{\Nexagglachel} promise to destroy the \satharioth from the inside}. 






\subsubsection{Scenes and role in \TwilightAngelRememberEmph}
Rewrite the Carzain story thread. 
Make Carzain more mysterious. 

Kill a lot of the waking scenes with Carzain, Delph and \Tsekkect. 
In fact, kill most of the thread. 
The Carzain/Vizicar story thread should be a teaser for things to come.
It should not be the main story. 

At the beginning of Carzain's part in \TwilightAngelRememberEmph, have several \emph{short} Vizicar scenes to clarify that she has attained \kenosis. 







\subsubsection{Cannot remember \Tydesmos}
\target{Carzain cannot remember Tydesmos}
Carzain remembered far more of Vizicar than either of them remembered of \Tydesmos. 
\Tydesmos was \hr{Tydesmos power}{a mighty and wise dark mage who possessed much Mythos knowledge}.
Vizicar and Carzain both blocked these memories out because they were horrible. 

In \emph{\CarzainWithRedcorBook{}}, Carzain \hr{Carzain remembers Tydesmos}{gradually regained her \Tydesmos memories}. 





\subsubsection{Crippled}
\target{Ramiel crippled}
\target{Carzain crippled}
Carzain takes a serious wound in the battle of \Forclin and does not recover. 
She is maimed.
Everyone tells her she should rest and recuperate, but she keeps running around on adventures and using lots of magic. 
The \qliphoth wear her down.
Her body becomes decrepit and infirm.

She becomes more and more decrepit, all the way up until \hr{Ramiel's awakening}{Ramiel awakens back to her full \resphan power}. 
After then her body recovers but \hr{Ramiel is mad after awakening}{her sanity suffers instead}.





\subsubsection{Turns evil}
\target{Carzain turns evil}
Carzain \hr{Carzain's idealism}{used to be idealistic}. 
Gradually as the story goes on, Carzain becomes more and more disillusioned. 
She sees the supposedly good Redcor around her act like total bitches, and is more impressed by the more brutal justice represented by Telcastora Ilcas. 
Then later she grows disappointed with the Imetrium, too. And by that time, she is bitter with revenge and ambition, so she turns to evil. 

Carzain and Vizicar gradually realize that on \Miith{}, everything revolves around war and conflict. It is the nature of the world, and to fight against it is doomed to failure. 

Carzain encounters several different civilizations and their different moralities, and finds them all lacking\dash{}totalitarian, intolerant and contradictory. 
She has also read snippets of \WanderersInDarknessEmph and other forbidden books, and what she has read has strengthened her growing suspicion that good is futile. 
She is near to concluding that everyone is a hypocrite, that all morals are false and that the only truth is power. 
When she awakens and becomes Ramiel again, she becomes completely disillusioned and turns to evil completely. 
(Perhaps it is her already high degree of disillusionment at this point that lets her keep her sanity when confronted with the truth.) 











\subsection{Name}
\emph{\Shachar} is an old-fashioned Vaimon name. 
It is based on \emph{Heilel ben-Shachar}, which appears in the Hebrew version of the Bible. 
In Isaiah 14:12, King James version, it is rendered as \quo{Lucifer, son of the morning}. 
A more literal translation might be \quo{Venus, son of the morning}.









\subsection{Personality}





\subsubsection{Idealism}
\target{Carzain's idealism}
At the beginning of the story (in \emph{\TwilightAngelRemember{}} and \emph{\CarzainWithRedcorBook{}}), Carzain is quite idealistic. 
She wants to be a hero, not only for the glory and the sex, but also to save the world. 
She doesn't have a clear picture of what she wants to do, but she wants to do something great, something heroic. 

Vizicar is more cynical, since he has ruled an empire. 

Gradually as the story goes on, Carzain \hr{Carzain turns evil}{becomes more and more disillusioned}. 





\subsubsection{Madness}
From the beginning of \emph{\TwilightAngelRemember{}}, Carzain's sanity is already a bit low from all the dark \nieur{} magic she has studied. 
This gives her a mild form of sociopathy and a tendency towards violence and provocation. 

(The above terminology is that of the RPG \emph{Call of Cthulhu}.)





\subsubsection{Misses an extended family}
Carzain has only had her parents. 
No grandparents, uncles, aunts, siblings, cousings, nothing. 
She misses that. 





\subsubsection{Sephiroth-type}
\target{Carzain is Sephiroth}
Carzain was a type like Sephiroth from \cite{VideoGame:FinalFantasyVII}. 
In the beginning she was a badass hero, but idealistic.
She fought for \quo{the side of good}, helping people against monsters and evil. 

In this time, she had many of Vizicar's memories, but not all. 
Vizicar had some dark Mythos knowledge which they could not remember. 
Carzain knew there were some memories she missed, and she sought for those memories. 
But she was not so distressed. 
She was fairly happy and optimistic (at least, by the standards of a dark sorcerer on the cruel world of \Miith). 
She saw the search for her memories as a positive self-development project. 
She expected they would make her happier and more complete. 

She expected the memories to be nice and pretty and bring clarity and understanding and wisdom. 
Because that was what they had done earlier. 
Vizicar lived a happy life as a loved and successful \caliph. 
He was surrounded by beauty and opulence. 
In her youth, Carzain had been confused and feeling strange, as if something was missing inside her.
Gradually, as Vizicar's memories returned, she came to feel more and more whole and confident and happy. 
There were some memories from the earlier lives that were dark and traumatic, but they were overshadowed by good memories. 
She expected that the last missing memories would continue the same development and make her even more happy. 

\target{Carzain's Sephiroth epiphany}
Then, at the end of \TwilightAngelRememberEmph, she had an epiphany. 
A lot of dark memories came pouring back into her. 
Not just from Vizicar's life, but from earlier lifetimes as well. 
The memories were not clear, but a chaotic jumble of images and impressions. 

The memories were not nice as she had hoped, but dark and hideous. 
They did not bring clarity and understanding and wisdom.
They brought confusion and horror. 
She remembered scattered fragments of her life as Ramiel.
Ramiel had experienced many shaking revolutions in his life:
\begin{itemize}
  \item 
    In \Merkyrah, Ramiel found out that his whole culture was based on a lie.
    He learned that his true self was a terribly dark, bloody, murderous thing.
    
    He was evil, but it felt right.
    It felt good.
  \item 
    This new development estranged him from his old friends.
    They waged bloody war against one another (the \hs{Murder of the Dawn}). 
    He killed his own people in droves. 
    
    But it was right.
    It was what he had been born to do. 
  \item 
    It also estranged him from his lover (\hr{Shiaraid}{\Shiaraid}). 
    She ran from him into the arms of other \resphain and \resviel. 
    They grew to hate each other.
    They would come back together and go apart several times. 
  \item 
    His mother was killed in the war against \Merkyrah.
  \item 
    He and his people invaded \Tembrae and waged a war of bloody genocide against the locals.
    They killed the king of \Tembrae and dismembered him and drank his blood.
    They split his soul apart and ate it.
    
    It was evil. 
    But it was right.
  \item 
    His father (\Nathrach) was killed in the Incursion. 
  \item 
    His dynasty was betrayed and backstabbed by the other dynasties. 
    (That was the way they saw it, at least.)
    His mentor (\Zachirah) was killed by their betrayal.
    Ramiel himself was killed, but not fatally. 
  \item 
    He and his betrayed fellows tried their best to rebuild their glory.
    They began a new project that would make them greater than ever and be a huge breakthrough for their entire race.
    But again they were betrayed.
    He was cast down and lost his power. 
\end{itemize}

Carzain now felt like a victim in a cruel and evil world that was out to get her. 
But she was also a hero. 
A dark hero. 
She knew it is in her nature to conquer and destroy.
She \quo{knew} that the morals she had followed in her last couple of lives were illusions.
She was a much darker, bloodier character than that. 
She had been in denial but could now no longer hide her true self. 
She was a destroyer. 

This made her turn into a dark, ambitious anti-villain. 
She no longer believed mortal morals had any validity. 
She had to be an \Ubermensch and make her own morals. 
And destroy anything that got in the way of her self-realization.
That was what she had always done, and though it might be evil, it was right. 
She was something greater and more important than the mortals around her. 
She had to fight for her own self-realization and destroy her enemies.
That was her nature. 
Her purpose. 
And she had to be true to that. 

She turns evil. 

Maybe she should kill \Racel in a scene reminiscent of the one in \cite{VideoGame:FinalFantasyVII} where Sephiroth kills Aerith. 

See also the section on \hs{Carzain's idealism}. 





\subsubsection{Suspects the supernatural}
At the time of the \hs{Runger war}, Carzain knew a bit about the supernatural horrors of the world and suspected more. 

In the early Carzain chapters in \TwilightAngelRememberEmph, have some references to the \hr{Myths of vanquished monsters}{myths of Iquinian heroes vanquishing inhuman Elder Races and monsters}. 
Carzain gets \quo{Elder} feelings from the \EreshKali{} magic, just like Tantor did. 
Carzain and Vizicar are unsure of those myths and have a feeling that the Elder Races still live, and that humanoids do not dominate the world as much as they think they do. 








\subsection{Physique}
\begin{description}
  \item[Appearance:] 
    Slightly taller than average. 
    Slender of build but unusually muscular for a woman.
    Long black hair. 
\end{description}

\target{Carzain wants green eyes}
Carzain's eyes are brown (as are Nishain's). 
She wishes they were \hr{Geican green eyes}{green, in the true Geican manner}. 
In contrast, Vizicar's eyes are blue, which is pretty typical for Vizicar's \vclan. 





\subsubsection{\Demihuman}
\target{Carzain is a Demihuman}
Carzain had somewhat dark skin. 
Dark enough that some might call her a \demihuman.





\subsubsection{Stigmata}
Carzain had some \hr{Vaimon stigmata}{Vaimon stigmata}. 
She had some nasty sores on her body, scars of spells that had taken a great toll on her. 





\subsubsection{Strength}
\target{Carzain's strength}
Carzain could summon up superhuman reserves of endurance when she had to. 
Even when wounded and tired, she could fight on and cast more magic. 
She had strong ties to many \qliphoth, having used them for many years as Vizicar and Carzain. 
She could call on the \qliphoth and draw strength from them. 
She drew power from the bloody elder reaches of the Outer Darkness. 









\subsection{Politics}





\subsubsection{Geican}
Carzain fully considered herself a Vaimon of \ClanGeican, even before she had ever been to Geica.
She introduced herself as such. 





\subsubsection{Reputation}
\target{Carzain's reputation}
During her adventuring years (before the \hs{Runger war}), Carzain had a reputation as a monster who fought monsters. 

Carzain was searching for her \quo{soul}.
Rumours got out about this. 
She dropped hints here and there, but never bared her heart. 
People knew that she was looking for something, allegedly her \quo{soul}. 
Some said she had lost her soul (perhaps through her black magic) and wanted to find it again (perhaps having repented the black magic). 









\subsection{Skills and powers}





\subsubsection{Batman-like stealth}






\subsubsection{Carzain as Batman}
Have combat scenes at \Forclin with Carzain as a Batman-type dark hero. 
She comes out of the shadows and strikes down her prey. 

\lyricsbs{Hammerfall}{Renegade}{
  He stalks in shadow lands, soundless, with gun in hand\\
  Striking like a reptile, so fierce\\
  No chance to get away, no time for your last prayer\\
  When the prowler sneaks up from behind
  
  An outlaw chasing outlaws, the hunter takes his pray\\
  The law of the jungle he obeys\\
  Craving for the danger to even out the scores\\
  Face to face, once and for all
  
  Renegade, renegade\\
  Committed the ultimate sin\\
  Renegade, renegade\\
  This time the prowler will win
}




\subsubsection{Dark magic}
In all chapters where Carzain uses magic, remember to describe \hr{Itzach pain}{the pain and horror of \Itzach}, in graphic detail. 
And with blood! 

She knew a spell (\qliphah) that could stop people's hearts or otherwise kill them from the inside. 
And she can cause big, bleeding wounds to spring open. 
She can also rip a person's body apart, but the latter, while more effective, is much harder. 





\subsubsection{Languages}
Carzain, having \travelled extensively, spoke many languages. 
Vizicar's experience helped. 
(Dated though Vizicar's knowledge might be, it still helped give an understanding of the various language groups.)





\subsubsection{Power}
\target{Carzain's power}
At the time of the Runger war, Carzain was very powerful. 
Much more powerful than \hs{Archibald Curwen}. 

She is a type like Kane from \cite{KarlEdwardWagner:GodsInDarkness} and \cite{KarlEdwardWagner:MidnightSun}. 





\subsubsection{Poor healer}
Carzain has never been a very skilled healer. 
Her mother, who is very skilled, has taught her, but Carzain has never really gotten good at it. 
She has a bit of a mental block against it, because she sees healing as an effeminate, wussy skill. 
She would rather practice the kind of magic that'll let her kill people. 
(\hr{Mystraacht philosophy}{\Mystraacht{} ideology}? Hell no\prikker)





\subsubsection{Usurer}
Carzain knew a spell to summon a \hr{usurer}\dash a \daemon that could heal and strengthen her and feed on her slain foes. 





\subsubsection{Vizicar's voice}
Recall that Carzain is a Scion, an incarnation of the \Malach{} Ramiel. 
Occasionally, when she fights and kills, Ramiel awakens in her, and she hears the voice of \hr{Vizicar}{\VizicarDurasRespina}, the previous incarnation of Ramiel. 

This is inspired by, and similar to, the scenes in \authorseries{Robert Jordan}{Wheel of Time} where Rand al'Thor hears the voice of Lews Therin Telamon. But where Lews Therin is mad, annoying and even dangerous, Vizicar is sane and a powerful ally. 





\subsubsection{\Wylde skills}
Owing to her brief \hr{Ramiel incarnated as savage}{incarnation as a savage tribesman}, Carzain had learned how to live in the \wylde. 
She remembered this. 
She had important skills such as stealth and how to live off the land. 
Compare her to Robert E. Howard's Conan.















\section{\Delphine}
\target{Delphine}
\index{\Delphine}
A \human{} woman in \hr{Imrath}{\Imrath}, an ally of \hs{Cordos Vaimon} and one of the first \hs{Vaimons}. 
Actually a \hs{Scion} of \hr{Shiaraid}{\Shiaraid}. 









\subsection{Equipment}





\subsubsection{Castle Yeshimon}
\target{Yeshimon}
\target{Castle Yeshimon}
\index{Yeshimon}
\Delphine dwelt in Castle Yeshimon. 
It was a small castle. 









\subsection{History}
She was inspired by \hr{Kor-Rashad}{\KorRashad} and other \qliphoth. 
She became a \quo{dark prophet of \Itzach}. 





\subsubsection{\Sarun ancestry}
\target{Delphine is Sarun}
\Delphine was of \hr{Sarun}{\Sarun} ancestry.
Later ages painted her as a \Sarun queen and the consort of the evil \hr{Sarun sultan}{\Sarun sultan}. 
In reality she was only half \Sarun and had nothing to do with the sultan.





\subsubsection{Early history}
\Delphine{} was a mystic and sorceress. 
She belonged to the same people as \hs{Cordos Vaimon} and was an ally (and sometime lover) of his. 
But already back then she was somewhat controversial, and some mistrusted her. 
But this mystery and controversy only made her \emph{more} alluring and sexy. 





\subsubsection{Never knew she was a Scion}
\target{Delphine never knew she was a Scion}
\hr{Shiaraid's stealth}{\Shiaraid{} was a stealthy \Malach}, so no one ever discovered \Delphine{} was a Scion. 
Not even \Delphine{} herself. 
She died pretty young, so she never came anywhere near \Apotheosis. 





\subsubsection{Legacy}
She remained forever after a dark, controversial figure in Vaimon history and legend. 
Some revered her as a great thinker. 
Others (Iquinians) reviled her as an evil blasphemer, if not \hr{Delphine in mythology}{evil incarnate}. 

According to some later legends it was \Delphine who corrupted the Vaimons.
She was the first to call upon the \qliphoth and use them for magic and teach others to do the same, after Silqua had discovered the \qliphoth and realized how evil they were and forbidden anyone to call on them.
In reality:
\begin{enumerate}
  \item 
    Silqua did learn about the \qliphoth and reviled them as evil, but she never forbade anyone anything.
    She was too humble and submissive a girl.
  \item 
    It was Lestor Delain who \hr{Lestor uses Qliphoth}{began using the \qliphoth}. 
    Lestor was later \hr{Lestor washed white}{washed white}. 
  \item 
    \Delphine was no Vaimon and never called upon any \sephirah or \qliphah.
    She was a \Sarun sorceress.
\end{enumerate}










\subsection{Name}
Her name is inspired by Delcardes, a beautiful girl who appears in \cite{RobertEHoward:TheCatandtheSkull}. 

Other possibilities include Delphine, Delphias and Deshracca.









\subsection{Skills and powers}





\subsubsection{Binding souls}
\target{Delphine binding souls}
\index{\carcer!\Delphine}
\hr{Malachim binding souls}{As a \malach}, \Shiaraid{}/\Delphine{} had a \sephirah-like power to bind souls to her. 

\Delphine{} never used this power for much. 
\Delphine{} died relatively young and was never much in touch with her \carcer, so she didn't bind many souls. 
But \hr{Silqua dies}{she did bind} one important one: 
That of \Eryal. 















\section{\Iolivine}
\target{Iolivine}
\index{\Iolivine}
A Scion of the \hr{Malach}{\malach} \hr{Ishicah}{\Ishicah}. 
\Iolivine{} was a Vaimon woman of \ClanRedcor. 
She lived in the \Ortaican{} Age. 









\subsection{History}





\subsubsection{Notes on Scions}
\target{Iolivine's notes}
\Iolivine{} had awakened to some of her powers. 
She studied \hr{Vizicar's notes}{\VizicarDurasRespina's notes about Scions} and wrote her own notes referencing Vizicar's. 

\target{Redcor bogarted Iolivine's notes}
These notes were extremely detailed. 
But since they were written by a Redcor, \ClanRedcor was very possessive of them, and for centuries they did their best to secure or destroy every copy of them. 
After all, it was heretical knowledge, and the Redcor had to keep it safe and hidden. 





\subsubsection{Heresy}
Later in her life, she became a \hr{Vaimon heretics}{heretic} and tried to set herself up as a goddess. 





\subsubsection{Captured and enslaved}
She failed and was \hr{Ishicah enslaved}{taken captive by the \Ortaicans}, where she was enslaved an eventually destroyed. 

















\section{Silqua}
\target{Silqua}
\target{Silqua Delain}
\target{Silqua Vaimon}
\index{Silqua}
\index{\Delaen!Silqua \Delaen}
\index{Vaimon!Silqua Vaimon}
%\sectionchardead{Silqua}{\human}{\female}
%Female \human{} \birthtodeath{Silqua}. 
The first \hs{Vaimon} and the discoverer of \hr{Iquin}{\Iquin} and \hr{Itzach}{\itzach}. 
Called \quo{the Prophet} by the \hs{Iquinian Church}. 







\subsection{Sexuality}
Remember to explore Silqua's sexuality. 

Contrast her with \hr{Belzir}\Belzir, who also lived in another incarnation during her time.





\subsubsection{Sexual visions}
Some of her visions, and her \malach{} power, are inherently sexual. 





\subsubsection{Sexual fantasies}
She has sexual fantasies, and she's ashamed of them. 

There are metaphysical implications. Perhaps a \resphan{} who seduces her. He comes to her in dreams and gives her pleasure\dash a darker, forbidden kind of pleasure, the likes of which she never gets from Cordos, who is, after all, pretty traditional. 

\lyricsduana{nightsong}{Night Song}{
  at night he slips through my window \\
  cool air caressing me like a lover's moist tongue 
  
  I shiver new awakening \\
  moaning a rhapsody of vowels \\
  and consonants like crosses in a row \\
  his lips of bittersweet agony \\
  bruising tender flesh \\
  as I lay impaled by lust \\
  and hear the sound of my heart beating \\
  druming blood through my viens 
  
  his love is wild like wolf-song \\
  howling at the pregnant moon \\
  his voice is smooth as liquid velvet \\
  as he sings his soothing lullaby \\
  hush hush hush my pretty pretty one 
  
  until the dawn bleeds from darkness \\
  tangerine rays grazing my cooling flesh \\
  and he flees like the shadow of a dream \\
  leaving silence in his wake
}





\subsubsection{Rape}
Later in the story, she is raped. 
She develops sexual traumata. 
This is one of the main reasons why the Iquinians are so sexually repressed and represive.







\subsection{History}





\subsubsection{Birth}
She was born Silqua \Delaen, the daughter of Lord \hr{Maegon Delain}{Maegon \Delaen}. 
She was a princess of \hs{Calaan}.




\subsubsection{Childhood}
Silqua was a religious thing, raised in a priesthood. 
She was raised to be sexual, submissive and obedient. 

\citeauthorbook[p.448]{RobertEHoward:BlackColossus}{Robert E. Howard}{%
  Black Colossus (draft excerpt)%
}{
  On each birthday, up to her twentieth, Yasmela had been laid across the knees of the image in Ishtar's temple and birched soundly by a priestess, to teach her humbleness in the sight of the goddess.
}





\subsubsection{Marriage}
In \yic{Silqua married} she married \hs{Cordos Vaimon}, prince and later king of \hr{Imrath}{\Imrath}, and became Silqua Vaimon. 





\subsubsection{Suffering}
Silqua suffers a lot during the story. She is something of a \trope{TheWoobie}{Woobie}.%, if not an outright \trope{ChewToy}{Chew Toy}.





\subsubsection{Idealization}
In later myths, \hr{Cordos and Silqua in mythology}{Cordos and Silqua were idealized} completely out of proportion.









\subsection{Politics}





\subsubsection{Family}














\section[Tydesmos Gendar-in-Caphet]{\TydesmosGendarInCaphet}
\target{Tydesmos}
\index{\TydesmosGendarInCaphet}
%\sectioncharunspec{\TydesmosFull}{\human}{\male}
%Tydesmos Gendar-in-Caphet
%Male \human{}. 
\TydesmosGendarInCaphet was a Vaimon mage and a Scion of the \Malach{} Ramiel. 









\subsection{History}
\target{Tydesmos not from Azmith}
He did \emph{not} live in \hr{Azmith}{\Azmith}, but in some other \hs{Shrouded Realm}. 









\subsection{Skills and powers}
\target{Tydesmos power}
\Tydesmos was a mighty and wise dark mage who possessed much Mythos knowledge. 

\citeauthorbook[p.76--77 of 138]{KarlEdwardWagner:DarknessWeaves}{%
  Karl Edward Wagner%
}{%
  Darkness Weaves%
}{
  Kane stood in what appeared to be a cavern, stretching endlessly far beneath the earth. Jagged stalactites hung like black clouds from the cavern roof a mile above; about him the horizon vanished over a smoking plain of shattered rock and angry lava pits. 
  In this nightmare vision of Hell, Kane was not alone. 
  Dark creatures of blighted beauty stood around him\dash bizarre demons with leathery wings, and beautiful faces that glowed with evil wisdom. 
  They wheeled about Kane in attitudes half of menace, part curiosity. 
  Kane spoke earnestly to one who seemed to be their leader\dash a tall demonic figure of perfect beauty and consummate evil, whose eyes shone like yellow suns.
}















\section{\VizicarDurasRespina}
\target{Vizicar}
\index{\VizicarDurasRespina}
A Vaimon of \ClanDelain who grew up to become \VaimonCaliph. 
Vizicar was a Scion, an incarnation of the \Malach{} Ramiel, one of the last \VaimonCaliphs and a great conqueror and statesman. 









\subsection{Arsenal}
Vizicar has many skills. He is an expert statesman, diplomat and demagogue, and a skilled swordsman and horseman. 

He has also pursued a number of other skills. In his youth, he felt some pressure on him, as an Emperor, to excel at \emph{everything}. So he tried to study every skill in the world. He only succeeded in mastering a few of them. 

Eventually, he learned to enjoy doing things he sucked at, such as archery and certain musical instruments. 





\subsubsection{Art}
Vizicar was an atypical Ramiel. 
He indulged in lots of art, culture and sport, in an attempt to block out his nagging feelings of being insignificant, \hr{Ramiel is nothing}{\quo{nothing}}

Among other things, he played several musical instruments. 





\subsubsection{Fencing}
Vizicar is a master fencer. 
His best style is \chatresse, the art of wielding a \chandre{} (a medium-heavy sabre). 









\subsection{History}





\subsubsection{Family}
Vizicar's father was Crizifer Arcan Respina. 

Vizicar had one wife and several bound concubines. 
That was what the law allowed at the time. 

Vizicar had children, but they were mostly raised by nannies and nurses. 
He was always busy doing Emperor-things and could rarely play with them. 
So he developed little affinity with children. 

He was an only child himself. 
\hs{Scions are often only children}, he knew. 





\subsubsection{Crowning}
Vizicar succeeded \LucionRinOrcas{} as \VaimonCaliph. 





\subsubsection{\Delaen was secular}
In Vizicar's time, \hr{Delaen was secular}{\ClanDelaen was quite secular and non-religious}. 





\subsubsection{Harem}
Vizicar had a harem of wives and concubines, in keeping with the \hr{Vaimon Middle-East}{Middle-Eastern theme of the \VaimonCaliphate}. 





\subsubsection{\Apotheosis}
He strove for \hr{Apotheosis}{\Apotheosis} in his life, but never achieved it. 





\subsubsection{Studied under Shanix}
Vizicar was taught swordsmanship by the great master \hs{Daemien Iras Shanix}. 
Vizicar brags that he surpassed his mentor's skill, but privately to Carzain he confesses that the last part is a lie. 
He never surpassed or even equalled Shanix. 





\subsubsection{Notes on Scions}
\target{Vizicar's notes}
Vizicar found some of the writings of the ancient \hs{Vaimon heretics} and was inspired by them. 
But as an Emperor he was pretty constrained and could not experiment as he would like. 
He was more disciplined and orthodox than \Belzir.

Vizicar wrote some notes on Scions, discussing his discoveries about the nature of \Malachim{} and Scions and of the \kenosis{} and \apotheosis. 
\hr{Belzir reads notes}{\Belzir{} read these notes}. 

The notes drew influences from the writings of the \hs{Vaimon heretics}. 
Vizicar suspected that some of these heretics were secretly Scions, perhaps without knowing it. 

The notes were then lost in the \Darkfall. 
At Carzain's time only fragmented secondary literature existed. 
Carzain found references to the notes and a few quotes, but no fragments of the notes themselves. 

\citeauthorbook{RobertEHoward:TheValleyoftheWorm}{Robert E. Howard}{%
  The Valley of the Worm%
}{
  Each man on earth, each woman, is part and all of a similar caravan of shapes and beings. 
  But they cannot remember\dash their minds cannot bridge the brief, awful gulfs of blackness which he between those unstable shapes, and which the spirit, soul or ego, in spanning, shakes off its fleshy masks. 
  I remember. 
  Why I can remember is the strangest tale of all; but as I lie here with death's black wings slowly unfolding over me, all the dim folds of my previous lives are shaken out before my eyes, and I see myself in many forms and guises\dash braggart, swaggering, fearful, loving, foolish, all that men have been or will be.
}





\subsubsection{Knew about other Realms}
Vizicar became convinced from his memories that \hr{Tydesmos not from Azmith}{\Tydesmos{} was not from \Azmith}.

He and Tydesmos had no languages in common, so it was difficult for them to learn to communicate. 

Vizicar remembered at least one incarnation who lived \emph{in} the \VaimonCaliphate and \emph{before} \Tydesmos{}. 
This meant \Tydesmos{} must have lived during the time of the \VaimonCaliphate. 
But \Tydesmos{}, who was a highly educated man, had never heard of the \caliphate, nor any recognizable form of the Vaimon language. 
This meant \Tydesmos{} must have lived far away from the \caliphate. 
Thus proving that the world must be bigger than \Azmith. 

Interestingly, though, \ps{\Tydesmos} world \emph{did} have a religion that was eerily similar to the Iquinian church that Vizicar and his Vaimon predecessors knew. 
The details and rituals were different, everything had other names, and the story of Cordos and Silqua was replaced by something entirely different.
But the sixteen \sephiroth{} were easily recognizable, as was most of the central dogma. 

For Vizicar, who (due to his Vaimon studies) was very interested in theology and philosophy and metaphysics, this had extremely interesting ramifications. 
If a completely strange civilization had independently developed the same religion, then there must be something true about said religion. 





\subsubsection{Julius Caesar type}
At Vizicar's time, the \VaimonCaliphate was divided and in civil war. 
Vizicar fought long and hard to unite it.
He had to commit terrible crimes and betray his allies, but in the end he succeeded.
He conquered the entire \caliphate and brought peace to the world. 
He was seen as a bloody-handed conqueror in his time, but history recognized him as a great hero who did the world a service. 

Compare him to Julius Caesar.
In fact, read about Julius Caesar. 

Unfortunately it would all unravel but a few generations after his death, under the reign of \Belzir.





\subsubsection{\Ivesser}
Vizicar had problems with a woman, \Ivesser. 
She was a beautiful and manipulative seductress, of good family but not a Vaimon. 
She became Vizicar's lover, concubine and perhaps even wife. 
She plotted with his enemies against him and caused him much trouble. 

Afterwards \Ivesser was vilified as a slut, a whore and an evil tratoress and manipulatrix. 
But she had good reasons to oppose the conqueror Vizicar and act as she did.
She was trying to help her people, and she grieved greatly whenever she had to do something evil or cause someone hurt. 
She was actually a sweet and innocent girl, but since she was no Vaimon or warrior, she was forced to use her only great weapon: 
Her sexuality.
Vizicar hated and blamed her at times, but in the end he understood that she though she betrayed him she was not evil. 

Compare her to the historical Queen Cleopatra, \PhedreNoDelaunay from \cite{JacquelineCarey:KushielsLegacy} and the woman from \cite{MaelMordha:GealtachtMaelMordha}. 





\subsubsection{Death}
\target{Vizicar dies}
Vizicar died in a shipwreck.

Vizicar was an old man at this time. 
He tried to flee using flying magic, but failed. 
He liked to tell himself that he would have survived if he had been younger and fitter. 

Because of this, Vizicar/Ramiel developed a phobia of water, especially the sea.

\target{Vizicar fears Nag}
During the shipwreck his people tried many things to save themselves, including many \hr{Spells in Nag}{\naga spells in the tongue of Nag}. 
Vizicar later came to associate the sound of Nag with the shipwreck and his own panic and drowning and suffocation and cold and exhaustion and crushing water and darkness and death. 
Later on the sound of the language would cause him to panic.
(The association was extra strong because the words were magical words of power.)

There was much suspicion of foul play. 
Vizicar himself would later (in his other incarnations) suspect that evil sea gods or sea creatures (like \nagae) had conspired against him. 
But in truth, it was a perfectly normal storm and he was just unlucky.
When at last \hr{Ramiel's awakening}{Ramiel regained all his memories}, he \hr{Ramiel accepts Vizicar's death}{realized that there were no sea gods out to get him}. 

Vizicar was succeeded as \VaimonCaliph by Cordos Irildra. 









\subsection{Personality}
Vizicar is arrogant and confident. 
At times virtually a \trope{LargeHam}{Large Ham}. 
Compare him to Tony Stark from the movie \cite{Movie:IronMan}. 

The Vaimons dressed in bright, beautiful clothes, but it could not hide the inner darkness that dominated the empire, the evil that festered within, the corruption brought on by the infestation that was \iquin. 

Vizicar tried to fight against it, but he could not break free of the circle of evil, violence and tyranny. 





\subsubsection{Doubted Sseju quotes}
\target{Vizicar doubts Sseju quotes}
Vizicar doubted the veracity of the \hs{pseudepigraphic Sseju quotes} that the \Ortaicans used. 
They did not sound like the \hs{Sseju} he knew. 
Sseju lived in Vizicar's time, and Vizicar remembered him as a critical Iquinian, not a heathen. 
Certainly he had little in common with the \Ortaican religion. 
Vizicar suspects fraud and pseudepigraphy. 









\subsection{Politics}





\subsubsection{\Shurco}
In Vizicar's time \Shurco was the great and feared rival of the \VaimonCaliphate.
Vizicar dared not war against them, so he maintained a cautious peace with the \Shurco while he worked to bring all \human lands under his rule. 

The \hs{Mysteriarch} \hr{Dul-Nepher-Ramas}{\DulNepherRamas} was the most honoured guest and ambassador at Vizicar's court. 





























\chapter{Vaimon Age}















\section{Aburun Dol Cuma}
\target{Aburun Dol Cuma}
\index{Aburun Dol Cuma}
Aburun Dol Cuma was a Vaimon philosopher and poet of Clan \hs{Geican} who lived in the \VaimonCaliphate.

Aburun committed heresy against the \hs{Iquinian} faith.
He believed that the virtues of the \sephiroth were not the ultimate truth but merely one small fragment of it, and a potentially misleading one. 
Aburun believed that in order to understand the greater truth one had to meditate on the \qliphoth and the darkness. 

He believed that one could achieve a state of \quo{non-being} that is better than life, since life is so full of suffering. 
(Compare to the concept of \hs{True Death}.)
This enlightenment could be achieved only through insight, independent of one's moral deeds. 

Aburun did not reject the \sephiroth completely, but he demoted them to one among several equally valid paths. 
The church was not happy.

The church eventually killed Aburun.
His philosophy lived on as a small sect within Clan Geican.
\Belzir was \hr{Aburun inspires Belzir}{inspired by him}. 
Much later, after \Belzir, Aburun's philosophy became dominant among the Geicans.















\section{Arcan \Delaen}
\target{Arcan Delain}
\target{Arcan}
\index{\Delaen!Arcan \Delaen}
The firstborn son of \hr{Maegon Delain}{Maegon \Delaen}. 
Brother of \hs{Silqua}. 









\subsection{Physique}
\subsubsection{Appearance}
\index{beard!Arcan \Delaen}
Arcan is tall and big. 
He has long, blonde hair and beard. 















\section{Cordos Vaimon}
\target{Cordos Vaimon}
Cordos was the son of Belandos Vaimon and the king of \Imrath{}. 
He married Silqua \Delaen and, with her help, became one of the first \humans{} to channel Iquin and Nieur. 
He founded the Vaimon order and the \VaimonCaliphate and became the first \VaimonCaliph. 









\subsection{History}





\subsubsection{Idealization}
In later myths, \hr{Cordos and Silqua in mythology}{Cordos and Silqua were idealized} completely out of proportion.















\section{Daemien Iras Shanix}
\target{Daemien Iras Shanix}
A great and famous Vaimon swordmaster. 
He invented the \hr{chandre}{\chandre} sabre and the style of \hr{chatresse}{\chatresse}. 

He was remembered even long after the \Darkfall{} as one of the greatest swordsmen ever. 

He was \ps{\VizicarDurasRespina} mentor who taught him swordfighting. 

In later Redcor dialects his name was pronounced \DamianChanici. 















\section{Grith Ecallivan}
\target{Grith Ecallivan}
\index{Grith Ecallivan}
\index{Ecallivan!Grith Ecallivan}
A hero of the ancient \hr{Vaimon Caliphate}{\VaimonCaliphate}. 
He was a scoundrel, a rogue who went his own ways, but he was loyal to the Vaimon cause and served the \caliphate in his own ways. 
He was condemned as an outlaw in his lifetime, but was posthumously recognized and \honoured as the hero he was. 
The tales relay how he studied his enemies from the inside and always found out how to strike at their weaknesses and steal their secrets. 















\section{Lestor \Delaen}
\target{Lestor}
\index{\Delaen!Lestor \Delaen}
The second son of \hr{Maegon Delain}{Maegon \Delaen}. 
Brother of \hs{Silqua}. 









\subsection{History}
Lestor is a very dark antihero. 
He was one of the first Vaimons to embrace \Itzach, and he did so with hate in his heart and intent to destroy his enemies. 





\subsubsection{Invokes \qliphoth}
\target{Lestor uses Qliphoth}
Lestor used the \qliphoth. 
He was actually the first Vaimon ever to do so. 





\subsubsection{Rapes \Delphine}
He \hr{Lestor rapes Delphine}{raped \Delphine} after she had killed Silqua. 





\subsubsection{Washed white}
\target{Lestor washed white}
Lestor was a very dark and morally ambiguous character. 
In later legends he was washed white and became a knight in shining \armour.









\subsection{Physique}
\subsubsection{Appearance}
\index{beard!Lestor \Delaen}
Lestor is slightly shorter and considerably slimmer than his brother, \hs{Arcan}. 
His long hair is dark brown, almost black. 
He bears only a slim moustache and a little beard. 















\section{Maegon \Delaen}
\target{Maegon Delain}
\index{\Delaen!Maegon \Delaen}
Once a lord in the kingdom of \Imrath{} and the father of \hs{Silqua}. 
He also had two sons, \hs{Arcan} and \hs{Lestor}. 















\section{Uther the Tiger}
\target{Uther the Tiger}
Full name: High King Uther I, son of Patrick of House Belek. 
Once king of Belek. 
An ally of \hs{Cuthran the Victorious}. 

He took his name from the \hs{white tiger}. 
He founded the \hs{Tiger} order. 















\section{Zacrias}
\target{Zacrias}
\index{Zacrias}
%\sectioncharunspec{Zacrias}{\human}{\male}
%Male \human{}. 
A Vaimon of \ClanGeican and the son of \Belzir. 
Zacrias is the most famous of \Belzir's many children. 
After his mother's death, he went on to become Lord of \ClanGeican. 









\subsection{Children}
\target{Zacrias' children}
Zacrias had many children. 
Several were named after his mother. 
He had three daughters named \Belzir, \Delphine{} and \Shiaraid. 












































\chapter{\ClanRedcor}















\section{\Chyrie \Esmerel}
\target{Esmerel}
\index{\ChyrieEsmerel}
\index{\Esmerel!\Chyrie}
A Redcor \Matron{}. She finds \hs{Carzain} in Pelidor and convinces him to return with her to \Redce.









\subsection{Personality}
\target{Esmerel is a nerd}
\Esmerel{} is a theologian and scholar more than she is a field operative. 
She has dedicated her life to studying Scions and \Malachim. 

She is something of a nerd without much experience outside the Redcor academies. 
So she needs to have some more world-wise people with her when \travelling. 















\section{\Dominice}
\target{Dominice}
A Redcor \Matriarch. A very badass Vaimon mage. 

She is of \theTulipFaction{} and as such supposedly lives in celibacy
But in truth she keeps secret lovers. 
Despite the fact that she's an ancient crone, she is quite lively in bed.















\section{\Laetitia Lacquasse}
%\sectioncharunspec{Lacquasse}{\human}{\female}
Vaimon of \ClanRedcor. Warrior, Mistress-at-Arms. 

\begin{description}
  \item[Name:] 
    \Ryzin{} \Laetitia{} Lacquasse of \ClanRedcor. 
    (Older name: \Brizen{}.)
  \item[Race:] Vaimon. 
  %\item[Alignment:]  
  \item[Size:] 180 cm. Heavy for a woman, but due to muscle, not fat. 
  \item[Appearance:] Curly red hair cut short. Plain-looking, not particularly attractive (even intimidating). Wears trousers in the style of the \Ryzin. Her eyes are blue. 
\end{description}















\section{\PatriccoKimon}
\target{Kimon}
\index{\PatriccoKimon}
\index{\Kimon!\PatriccoKimon}
\PatriccoKimon was a male Vaimon of \ClanRedcor and something of a free-thinker and heretic. 
He believed the Redcor should study and explore the sinister presence that dwelt underneath the \TopazChateau. 

\Kimon believed the religious creation myth and many other myths were lies, blankets to cover the truth and lull people into complacency.
He believed the truth would be shocking but marvelous and eventually beneficial to know. 

\citeauthorbook[p.165--166]{TimCurran:Hive}{Tim Curran}{Hive} {%
  Though he could not honestly believe in some invisible, mythical god, he could understand religion now.
  He could understand that it was a security blanket men wrapped around themselves.
  Maybe it was dark and close uner that blanket and you couldn't see more than a few inches in any direction, but it was safe.
  God created Heaven and Earth and there was a serenity to that, now wasn't there?
  It was simple and reassuring.
  And if religion was indeed a sheltering blanket, then science was the cold hand which yanked it away, showing man his ultimate insignificance in the greater scheme of things, the truth about his origin and destiny.
  The very things man had tried for so many millennia to walk away from, to forget.
  A cage he had liberated himself from slowly and, even if a candle of truth still burned in the depths of his being, if he did not look at it, then it did not exist. 
  But now man had been thrown back into that cvage, had the door slammed shut in his face.
  And the truth, the real truth of who and what man was and where he'd come from, was staring him dead in the eye. 
  
  And with that in mind, Gundry knew now that enlightenment was the lamp that would burn men's souls to cinders and the truth was the beast that would devour him and swallow him alive. 
  
  For if those things down in the lake had their way, men would never be men again, but just appendages of a cold and cosmic hive-intelligence as it had been intended from the very beginning. 
}

In the end \Kimon discovered that the truth only brought horror and madness. 
The \quo{bright light} of the \quo{Sun} of truth reveals only hideousness. 
Ignorance is better. 

\citebandsong{BlindGuardian:ANATO}{Blind Guardian}{Precious Jerusalem}{
  I've gone beyond but there's no life\\
  And there is nothing how it seems\\
  I've gone beyond but there's no life\\
  There is no healing rain in Eden\\
  The empty barren wasted paradise\\
  Let's celebrate the dawning of the Sun
}

\citebandsong{BlindGuardian:ANATO}{Blind Guardian}{The Age of False Innocence}{
  Cut off the light, take a look\\
  There's nothing beyond but pain\\
  Suffer in the deepest void\\
  The flame of hope is gone\\
  What have I done?\\
  Denied the father and the son\\
  For a moment it seemed\\
  There's space beyond the spheres

  Don't believe in their eternity\\
  We're still held in blindness\\
  And I've been turned into a liar\\
  If there is no heaven there won't be release
}















\section{\Racel Galisetti}
\target{Racel Galisetti}
A \soror{} of \hr{Redcor}{\ClanRedcor}. 

She shouldn't be using the name Galisetti. 
That is her father's name. 
Most of \Velcad{} is patrilinear, but the Redcor are matrilinear, so as a Redcor she should be using her Redcor family name. 














\section{\Roanne{} \Deracille}
\target{Roanne Deracille}
%\sectioncharunspec{\Roanne{} \Delishe}{\human}{\female}
%Female \human{}. 
A Vaimon, previously a \Soror{} of \ClanRedcor, she has now abandoned the Redcor Church and lives in \Redglen{} as an apothecary and healer. Her husband is Nishain \Shachar{} and together they have one child, Carzain. 









\subsection{History}
\Roanne{} is a bit of a \hr{Failed Vaimons}{failed Vaimon}. 
When she was with the Redcor, her mistresses pushed her too hard and had too high expectations of her. 
She would always let them down and disappoint them. 
This ingrained a lot of guilt into her. 

She could not handle the pressure or the psychological indoctrination. 
She tried drawing more \iquin{} power than she could manage, tried coming closer to the \sephiroth{} than was healthy for her. 
It nearly drove her mad. 
She became frightened. 
Her superiors scolded her. 

She just wanted to heal people. 
She was content with knowing just a few \sephiroth. 
But her superiors saw potential in her, so they pushed her on to ever greater accomplishments. 
She could not keep up. 

She failed, but she did not go mad. 
She could have tried again\dash and they tried to make her do it\dash but she didn't want to.
She was afraid. 
This wasn't what she wanted. 

So she gave up, left the \vclan and ran away. 
She came to Pelidor, where \ClanRedcor does not have too much power. 

And she met Nishain \Shachar. 
They fell in love. 
He was smooth and a skilled pick-up guy, so he quickly got into her pants. 
But when she got pregnant, she forced him to marry her. 
She did not want to let her child be born in sin. 












































\chapter{Other Vaimons}















\section{Alistair Curwen}
\target{Alistair Curwen}
\index{Alistair Curwen}
\index{Curwen!Alistair Curwen}
Alistair Curwen was the son of Archibald Curwen and his wife, Brithanae. 
Like his father, Alistair was a \Telcra Vaimon and a Cabalist. 















\section{Archibald Curwen}
\target{Archibald Curwen}
\target{Curwen}
\target{Charcoal}
\index{Archibald Curwen}
\index{Charcoal}
\index{Curwen!Archibald Curwen}
A \human{} Vaimon mage serving in the \ishrah{} of the Pelidorian army, Archibald Curwen is actually a Cabalist of the \hr{Cabalist circles}{thirteenth circle} with the code name of \quo{Charcoal}. 

% He is around 200 years old. He kills and drinks blood and life-force to keep himself alive beyond the normal life-span of a \human. 
% 
% Perhaps he has even tasted \resphan{} blood\prikker stolen or given. 

He should be called \quo{Lord Curwen}, not \quo{Captain}. 
In fact, get rid of the military title. 









\subsection{Physique}
\target{Curwen's appearance}
\index{beard!Archibald Curwen}
He looked similar to the \JimushiJuubei{} from the anime \cite{Anime:Basilisk}. 
Or like the actor Billy Bob Thornton. 
But older and fatter and with a beard. 

His hair and beard were originally dark brown. 
But as he aged they faded to a steely gray. 
He had a bald patch on the top of his head. 

In his old age, Curwen would sometimes mourn the graying of his hair and beard. 
But the gray look also had advantages, he reflected. 
It gave him the look of iron: 
Cold, hard and unyielding. 

\target{Curwen's gun}
\target{Curwen's pistol}
He usually wears a sword and a single-shot pistol in his belt. 
The pistol is \hr{Rissitic economy}{Rissitic-made}. 





\subsubsection{Race and racism}
Curwen was a \truehuman and a racist.









\subsection{History}





\subsubsection{Youth}
\target{Curwen born son of James}
Archibald Curwen was born in Belek in the days of \hr{Great Velcad}{\GreatVelcad}. 
He was a younger son of a title-less Belek nobleman, James Curwen. 

He was apprenticed as a Vaimon early on and joined a \Velcadian{} \ishrah{} to fight for the High King. 






\subsubsection{Marriage}
Archibald Curwen married a civilian noblewoman named Brithanae. 
With her he had three children.
Two died of diseases at a young age. 
Only one, his son \hs{Alistair Curwen}, survived. 

He read and heard a lot of stories about the great Vaimon heroes of old, how noble and powerful and totally sweet they were. 
Gradually he became disgruntled with serving the High King. 
He was a mage, so why should he serve a mundane? 
Vaimons should rule the world like they once did, not just serve. 





\subsubsection{Killing his wife}
Then \hs{Wesrun}, a sexy female Vaimon Cabalist, approached him. 
They talked about how mages should rule. 
She agreed with him. 
She seduced him. 
He cheated on his wife. 
Slowly she introduced him to the Cabal. 

He joined the Cabal and committed evil deeds.

Brithanae found out that Archibald was cheating on her. 
She freaked out. 
They fought. 
He remembered his talks with his lover and other Cabalists: 
\quo{%
  What are these mundanes to us mages?
  What do they know? Nothing!
  What right have they to try and boss us around? None!
  They must be put in their place.}
He hits her. 
She scowls and curses him. 
He kills her. 

He was tried for murder, but the Cabal pulled some strings, and he was cleared of charges. 
From that point on he was in their debt. 
He was theirs. 





\subsubsection{Cabalist life}
Much of his life he \travelled with the \Velcadian{} \ishrah{}, fighting wars of conquest or quenching rebellions. 
By chance he was in Pelidor when things went wrong and \GreatVelcad started to collapse. 
As a Cabalist he knew he had to act quickly to maintain some semblance of Cabal control. 
So he did everything he could to secure order. 
He and his fellow Cabalists decided that \hs{House Pelidor} was a good bet. 
So he threw in his lot with them and helped the \rayuths secure their rule. 
As such, he was a valued member of the Pelidorian court from the very beginning, and a high-level \ishrah{} member. 
He is highly respected by the Pelidors. 





\subsubsection{Death}
Curwen was \hr{Curwen dies}{killed by \Takestsha/\Nzessuacrith} in the battle for \Forclin. 









\subsection{Personality}





\subsubsection{Goals}
His goal is to rule a kingdom. 
One he wanted to be a full-fledged sorcerer-king reigning in public. 
But no longer. 
Now he has only disdain for the \quo{public}. 
What matters it if the populace know who he is? 
Who are they?
They are nothing. 
What they think, feel and know means nothing to him. 
They are means to an end. 
And as such, Curwen is perfectly happy to reign behind the throne as an \quo{advisor}, pulling the strings of some weak ruler. 





\subsubsection{Fears}
Curwen's nightmare is to lose his magical ability and become a mundane. 
A worthless, mindless mundane cursed to live a drab and pointless life, enslaved to a great world he will no longer be able to see or influence. 

Fortunately this cannot happen. 
You can't simply \quo{lose your magical ability}. 
Magic isn't a limb you can just cut off. 
To lose his magic he'd have to lose his mind. 
Or die. 

Still, he dreams of it at night, and it scares him: To wake up one morning and have forgotten all he's ever learned of magic and the Realms Beyond.  





\subsubsection{Interests}
Apart from magic and power, Curwen's main passion is food, drink and smoke. 
He greatly enjoys to seek out and savour exotic foods, wines and weeds to smoke. 

He smokes the pipe. 
He smokes both weak and strong weeds. 
The weak weeds when he needs to think. 
The strong weeds when he wants to relax. 

He eats a lot, so he is pretty fat. 
But still strong. 

Another interest, naturally, is women. 
He likes to fuck many different women, preferably young and beautiful. 
He's perfectly willing to resort to sexual harassment, blackmail or even rape. 
He quickly loses interest in women and discards them, though. 
Curwen looks down on nobles. 
He knows that mages only serve them in name. 
In the real world, which the nominal rulers are too stupid to see, the mages rule. 





\subsubsection{Occupation}
He has worked in military \ishroth{} all his life. 
As a Cabalist he has also developed great skill as a spy, manipulator and interrogator. 
He has worked as an interrogator in Pelidor. 





\subsubsection{Superstition}
Curwen did not truly believe in the superstition of \hr{Gun fiends}{fiends that made guns fire}. 
But he was raised among soldiers who believed in the fiends, so it was hard even for him to make himself completely free of the superstition. 





\subsubsection{\Vertex{} status}
Curwen is skilled and cunning, but he is no \vertex.
He is too closed-minded, greedy, prejudiced and judgmental for that.









\subsection{Politics}





\subsubsection{Family}
Archibald Curwen \hr{Curwen born son of James}{was the son of James Curwen I}. 
Archibald was a younger son. 
It was his older brother Ducan who inherited James's title and estate. 
Later Ducan died and his son, James II (Archibald's nephew), inherited the estate. 
James II was a competent leader, but he was also young. 
His uncle Archibald was often able to talk or coerce James into doing his bidding. 





\subsubsection{\MoroCobrel}
One person he hates is \MoroCobrel. 
She is his senior in the \ishrah{} and thus has higher academic rank. 
She was also pivotally influential in the formation of the duchy of Pelidor from the ruins of \hr{Great Velcad}{\GreatVelcad}. 
She has nearly as much influence has he has. 
He hates her for it. 
He has suspected her of being a Sentinel, but hasn't been able to prove it. 
He has tried to have her killed, but has failed. 

She doesn't seem to use her influence as much as he uses his. 
This worries him. 
She also seems to know an awful lot. 
That also worries him. 
He doesn't understand what she's after. 
He doesn't understand her. 
Thus he fears her, and thus he hates her. 





\subsubsection{Pelidor clan}
Archibald Curwen is one of the most influential persons in all Pelidor. 
He holds a high military and academic rank and has the ears of many people at court, including \Icor{} and \Sethgal. 

\Sethgal{} respects Curwen because he is a tough warrior. 





\subsubsection{\Tiroco Pelidor}
\Tiroco{} knows that \Icor{} listened to Curwen's advice, and therefore she should do the same. 

Curwen kind of likes \Tiroco for some reason. 
He treats he with a kind of friendly condescension.
When she was a child he was occasionally nice to her and did her favours, like showing her impressive stuff he could do with his magic. 
She remembers him as an uncle-kind-of-character.
Nice, but also scary.
He still sees her as a bit of a child, even as \rinyuth.









\subsection{Skills and powers}





\subsubsection{Cabal knowledge}
Charcoal is not at the top of the Cabal hierarchy. He knows about the \hs{Shroud}, that it separates the Realms and blocks people's vision. 

But he does not know everything about the Shroud. He doesn't realize the extent to which it permeates the universe and all thinking. 





\subsubsection{Languages}
\target{Curwen's languages}
Curwen spoke a number of languages, including \hr{Vaimon language}{Vaimon}, \hr{Pelidorian language}{Pelidorian} and \hr{Rungeran language}{Rungeran}/\hr{Vidran language}{\Vidran}. 





\subsubsection{Mundane skills}
As a nobleman, Curwen learned from a young age how to fight, ride a \relc{} and fight from \relc-back. 















\section{Criel and Jael}
%\sectioncharunspec{Criel and Jael}{\human}{\female{} and \male}
Daughter and son of \hs{Hayad}. 
Twins. 
Warriors and Seducers (ninjas of a kind), members of the Cabal. 
The two are very close and may have an incestuous sexual relationship. 

\begin{description}
  \item[Name:] Jael son of Hayad and Criel daughter of Desmid. 
  \item[Race:] Vaimons. 
  %\item[Alignment:] Chaotic evil. 
  %\item[Size:] He is 170 cm, she is 165. Both are slender and light of build. 
  \item[Appearance:] He is 170 cm, she is 165. Both are slender and light of build. 
\end{description}
















\section{Hayad son of Kazzed}
\target{Hayad}
%\sectioncharunspec{Hayad son of Kazzed}{\human}{\male}
\index{Senator Hayad}
Vaimon of \ClanGeican, member of the Geican Senate. 
Hayad was a successful Seducer before he went into politics. He is also a member of the \hs{Royalist Faction} (the secret cult serving \Belzir). 

\begin{description}
  \item[Name:] Hayad \Freid of \ClanGeican. 
  \item[Race:] Vaimon. 
  %\item[Alignment:] Lawful evil. 
  \item[Size:] 170 cm. 
  \item[Appearance:] 
\end{description}















\section{\Jirad Tantor}
\target{Jirad Tantor}
Mage in the royal \ishrah{} of the king of Runger. 
Kinsman of the \scarv{} of Tantor. 
















\section{Nishain \Shachar}
%\sectioncharlivelong{Nishain \Shachar}{\human}{\male}{Nishain}
%Male \human{} \birthtonow{Nishain}. 
A Vaimon mage of \ClanGeican, now living in \Redglen{} in Pelidor and working as a scholar and pharmacist. His wife is \Roanne{} \Delishe{} and they have one child, Carzain. He is notable for having discovered the \Kliffah{} \Gavron{} in his youth as a scholar in Geica.

















\section{Shereid}
\target{Shereid}
\index{Shereid}
%\sectioncharlive{Shereid}{\human}{\female}
%Female \human{} \birthtonow{Shereid}. 
A Vaimon of \ClanGeican, member of the Seducers' Guild and the Royalist Faction. 

She works at the \TopazChateau{} as a spy. 



\begin{description}
  \item[Name:] {Shereid} \Kazzed{} of \ClanGeican. 
  \item[Race:] Vaimon. 
  %\item[Alignment:] Chaotic evil. 
  \item[Size:] 
    165 cm (average for a Geican woman, shorter than most \human{} women). Somewhat plump of build. 
  \item[Appearance:] 
    Short and plump with large breasts. Pretty but not beautiful. Her hair is brown and wavy, and she prefers to wear it loose. She wears green dresses and robes. 
    \hr{Geican green eyes}{Her eyes are green}. 
\end{description}









\subsection{Arsenal}
\subsubsection{Life-draining through sex}
\target{Shereid's sperm-eating spell}
Shereid knows a \Qliphah{} that lets her drain a man's life force through sex. The \Qliphah{} is invoked during sex. When the man finally ejaculates, the \Qliphah{} sucks out some of his life-force and soul with the sperm. If the sperm is then injected into Shereid's body (through her mouth, vagina, anus or other opening), she is able to consume the man's life force.

\index{\succubus}%
Some \hr{Succubus}{succubus}-monsters possess a more powerful version of this spell, that lets them suck out a man's soul and body \hr{Succubus sucking dick}{through his dick}. 









\subsection{History}
\subsubsection{Death}
Perhaps she will eventually die this way: She accompanies Ramiel on a mission. An enemy of Ramiel takes her hostage and does the \quo{one step closer and she dies!} thing. Ramiel advances. The enemy kills Shereid, and Ramiel avenges her, but lets her die. 

Or perhaps Ramiel initially feigns fear of letting her die, to lull his enemies into a false sense of security. They retreat, and later he strikes back and destroys them\dash letting them kill Shereid in the process. 

Perhaps the enemy taunts Ramiel: 
\ta{Your sympathy for lesser creatures is your weakness.}

Ramiel: 
\ta{I have taken that lesson to heart.}









\subsection{Name}
The name \quo{Shereid} is derived from \hr{Shiaraid}{\Shiaraid}, which is \ps{\hr{Belzir}{\Belzir}} true name. 
\Belzir{} knew the name and used it on occasion. 
After her death, \hs{Zacrias} named \hr{Zacrias' children}{one of his daughters} \Shiaraid{} after her. 
This name evolved, and after a thousand years' time it had become \quo{Shereid}. 









\subsection{Politics}
\subsubsection{Sentinel association}
Shereid might be a Royalist loyal to \Belzir, or she might be a Sentinel working behind \ps{\Belzir}{} back. If the latter, she serves \Psyrex{} and \Secherdamon{} and is tasked with leading Carzain into corruption. 















\section{Wesrun}
\target{Wesrun}
\index{Wesrun}
A female \human{} Vaimon and Cabalist. 
She was once the lover of \hs{Archibald Curwen}. 
She is his age and still lives. 
She was sexy and hot once, but now looks her age. 

She and Curwen still have some kind of feelings for one another. 












































\chapter{Yet Others}















\section[Cyri]{\Cyri}
\target{Cyri}
\index{\Cyri}
\Cyri was a \human girl.
She was taken as a slave by the evil sorcerer \hs{Thuza}.
He used her as a sex and torture toy, but also taught her something of sorcery.
Eventually \Cyri managed to slay her master.
Later she forged a career as a sorceress. 















\section{Delph}
\target{Delph}
Delph is a Pelidorian \human{} man of the same age as Carzain. \hr{Carzain meets Delph}{They meet in the army in \Malcur} and become friends. 

Delph is a soldier attached to the \ishrah{} as a bodyguard of sorts. He is inspired by Moonglum from the Elric of \Melnibone{} books. 







\subsection{History}
\target{Delph's history}
Delph is a \Tepharin{} guy from somewhere near \Malcur. 

He had a rough childhood. His parents died of disease when he was relatively young, leaving him to fend for himself. He was always a charming child, so he survived by becoming a swindler: He would charm someone into thinking he was their friend, then rob them and run off. He would always have things to hide from people, so he used humour as a defensive mechanism, both psychologically and to fend off bothersome questions from people. 

He's developed a degree of kleptomania, so he tends to steal even when not in dire need. He steals some stuff from Carzain. Maybe Carzain suspects him, sowing distrust in their friendship; maybe he doesn't. 

Delph keeps a pet rat, a large brown male named Royn (after his adoptive father). 

He is unemployed and a loser. 
He has lived as a beggar or thief from time to time. 
Now he's in the army. 







\subsection{Death}
Delph dies near the end of \emph{\TwilightAngelRemember{}}. 
%When that happens, Royn survives and runs off into the \Wylde{} to become a wild rat. 

Before his death, Delph made Carzain promise to take care of the rat if he should die. 

\target{Curiet becomes rat-keeper}
After his death, it turns out that \hs{Serpentin Curiet} is the only one who can handle the rat without getting bitten. So he becomes the rat-keeper. 















\section{\Dzerezdin}
%\sectioncharunspec{\Dzerezdin}{\human}{\female}
Rissitic \Ashenoch{}. Addicted to draining Shadow-energy. 

\begin{description}
  \item[Name:] \Dzerezdin{} something\prikker 
  \item[Race:] \Human, \Ashenoch{}. 
  %\item[Alignment:] Neutral evil. 
  \item[Size:] 170 cm. 
  \item[Appearance:] 
\end{description}
















\section{Evith}
\target{Evith}
\index{Evith}
Evith was a \hr{Naor}{\naor} bred by \hr{Teshrial}{\Teshrial}. 

Of all the \humans{} that served the \resphain, there were perhaps those who held higher ranks than the \naorim, but none were more sacred. 

Evith was a winged \hr{Demihuman}{\demihuman}. 
And striped like a zebra. 
















\section{Lica}
\target{Lica}
A \human{} girl of sixteen or so, living in a nation in eastern \Velcad{}, south of \Redce. 
She is able to see through the Shroud to some extent. 









\subsection{Penetrating the Shroud}
Lica has the ability to penetrate the Shroud. 





\subsubsection{Origin of her power}
Lica was a normal girl in her childhood. 
Like other children, she was less Shrouded than adults and could sometimes see through the Shroud. 
Like other children, she was supposed to grow out of this and learn to accept the lies of the Shroud. 

But she was unwilling to leave the \quo{world of childhood} behind and block out the truth that she knew she was seeing. 
She was braver than others and would not accept the lie; she wanted to keep the truth. 

This has made her into a \vertex, able to see into the Beyond and thus affect the Web of Realms. 
She is a very weak \vertex, though. 











\subsection{Personality}
\subsubsection{Afraid of darkness and solitude}
Lica has come to hate her power and wants to go back into the Shroud, but she can't. 
Her sanity has been permanently damaged. 

She is afraid of darkness, and afraid of being alone. 
\hr{Light and darkness}{Light} and company strengthens the Shroud (compare with the \hs{Shroud of Civilization}). 
Alone in the darkness you can see things. 

\lyricslimbonicart{Towards the Oblivion of Dreams}{
  Whatever you do, don't fall asleep.\\
  There are voices in the night\\
  leading you to places you don't want to be,\\
  make you see things you don't want to see.
}

\lyricslimbonicart{Legacy of Evil}{
  The night has a legacy of evil.\\
  Nocturnal dreamscapes.\\
  A wilderness of infinite disharmony\\
  in an isolated aspect eternally.\\
  A maze inside your brain\\
  leading into the insane\\
  abyss of fear, illusions and despair.
  
  Something ghastly is passing by.\\
  The full moon in the sky.\\
  See the coming hurricane of terror,\\
  ancient sadness and horror.\\
  A virulent syndrome of misanthropy,\\
  captured by the obscure mystery. 
}















\section{Lumica}
\target{Lumica}
\index{Lumica}
Lumica was a \human sorceress.
She was the wife of King \hs{Morn Ossan} and served him as his court sorceress. 









\subsection{History}
Lumica was an enemy of \hr{Carzain}{\CarzainShachar}. 
They \hr{Carzain and Lumica}{fought}.
She died.















\section{Morgan Runger}
\target{Morgan Runger}
\index{Morgan Runger}
\index{Runger!Morgan Runger}
The king of \hs{Runger}. 















\section{Morn Ossan}
\target{Morn Ossan}
\index{Morn Ossan}
Morn Ossan was a \human king. 
His wife was \hs{Lumica}. 















\section{\Piacet}
\target{Needle}
A slave. 
Personal handmaiden to \hr{Tiroco}{\Tiroco} Pelidor. 

She is also a Cabalist of the \hr{Cabalist circles}{\needlecircle} who goes by the codename \quo{Needle}. 
She is \hs{Charcoal}'s Cabalist second-in-command in \Malcur. 










\subsection{Arsenal}
\subsubsection{Magic}
Needle was trained in the use of magic since the time she joined the Cabal. 
She was taught by Charcoal and other Cabalists. 
She was a bit old when she started (15), and she was not able to devote much time to her studies, so she never became a great mage. 
But at the time of the \hs{Runger war} she had been studying for 15 years, so she was a half-decent novice Vaimon. 









\subsection{Physique}
\target{Needle's appearance}





\subsubsection{True \human}
Needle might have been poor, but at least she was a \truehuman. 
She looked down on those filthy \hr{Demihuman}{\demihumans}.








\subsection{History}
Needle was a slave. 

She was born a free woman, the daughter of a couple of poor city labourers. 
Her early years were happy enough. 
Her first traumatic memory was when she saw her older brother stabbed to death while resisting a robber. 
Young \Piacet, eleven years old at the time, watched from where she crouched hidden. 
It scarred her. 

\target{Needle hates the Black Plague}
The robber was a member of the \hs{Black Plague}. 
Since then she has hated the plaguers. 

Later, her mother died of disease, and her father soon after. 
The family was indebted, so \Piacet{} and her older sister Belya were both seized by the creditors and sold into slavery to pay off the debt. 
They ended up in a brothel. 
But at least the two sisters were together. 
That was their only consolation, for all they had was each other. 
\Piacet{} was 14 and Belya 16. 
As soon as they started working, they were subjected to \hr{Sterilization}{magical sterilization} to avoid any trouble. 
No brothel wanted to have its girls get knocked up. 

After a half year's time, Belya contracted a sickness from a customer, wasted away and died. 
Then Needle was all alone. 

Needle worked as a sex slave for a year. 
Then Charcoal showed up. 
He had a sharp eye and quickly saw the potential in her. 
He did not bed her. 
Charcoal knew that he was no seducer nor a great lover, so fucking a potential recruit would be counter-productive. 
He just talked to her. 
She was infatuated by this mysterious man. 

He pulled some strings and got her sold to the palace. 
It was a great step up for her. 
She was expected to work hard and given little thanks for it, but she was no longer a sex slave. 
She was safe and in a clean, beautiful place. 
It was the first time in her life that she was surrounded by anything but \squalour. 

Even more importantly, Charcoal introduced her into the Cabal. 
He had recognized a potentially very useful assistant in her. 
But he had needed to get her away from the society's bottom and nearer to the centre of power, where she could be of use. 

She herself had no love for the social order or the rulers. 
That is why she was eager to join the Cabal when given the chance. 
To get some measure of power of her life and those of others, and to pay back her enemies. 

Charcoal proceeded to mold her into a good Cabalist. 
She learned quickly and rose through the ranks, eventually being given to \Tiroco{} as her personal handmaiden. 
She also rose in the Cabal ranks, reaching the \needlecircle. 

Compare her to Serw\"e from \authorbook{R. Scott Bakker}{The Darkness That Comes Before}. 

\lyricsbs{Monolith Deathcult}{I Spew Thee Out of My Mouth}{
  Let me embrace my fate which was adrift and hung. \\
  I will be again the apple of his eye. \\
  Take me from among the doomed Laodiceans \\
  whose defamation thrilled the seething skies.
}









\subsection{Personality}





\subsubsection{Misanthropy}
Needle had little love for \humans{}. 
They had killed her brother and raped her and her sister, and they were the cause of this rotten world. 
\Scathae{} \hr{Needle's racism}{were even worse}. 

But she loved and worshipped the \resphain. 
They had never abused her directly, so she saw them as noble, superior gods. 
For them she would commit any crime against the mortals she hated. 

She disliked the nobles who ruled \Miith{} despite never having earned it. 
She wanted to get rid of them and instead have truly worthy rulers\dash the \resphain{}\dash take their place. 





\subsubsection{Motivation}
\target{Needle's motivation}
Needle wanted to change the world order. 
She was not evil. 
She was genuinely working for a better world. 
She believed in the \resphain.
She saw how evil the current world order was, and she was convinced a \resphan world order would be better. 

She would also like to be on top and in charge for a change. 
She has been a slave her whole life. 
Now she wants to be the mistress. 
She wants some respect. 
She does not think it is so much to ask for. 

She loves and worships the \resphain{} and longs for the glorious day when they will conquer \Miith{}. 
She looks forward to seeing the unworthy earthly rulers fall. 

She believes that the Cabal is fundamentally good but must work underground and employ harsh measures because the world is overrun with corrupt rulers and evil forces. 
They are a liberation movement, and the ends justify the means. 
The Sentinels, on the other hand, are evil. 

She is afraid of the dark magic the Cabal uses.
But she knows that \Miith is a cruel world, and you have to take drastic measures if you are to change it.
She does not like having to use dark magic or associate with hideous things like \banes, but she must. 
It is a necessary evil to combat a worse evil. 

Needle believed in an apocalyptic religion.
She worked for the advent of the divine realm on \Miith.
The \resphan kingdom, which would be a paradise.

She looked forward to the \hr{Second Advent of Lithrim}{Second Advent of \Lithrim} very much. 
Read also the section on \hs{Cabal eschatology} (really). 

Have plenty of religious stuff about her.
More religion!

The memory of her sister, Belya, is a reminder of how evil the world is, and how it is out to get her. 
When she lashes out against the nobles, it is partially as a revenge for Belya. 

\lyricsbs{Emperor}{My Empire's Doom}{
  Once our lord has come again, \\
  all black hearts will celebrate \\
  the darksome fiend \\
  in the kingdom of sorrow. 
  
  Nocturnal fiends, \\
  the beings of the night and shadows \\
  calling forth the pure spirit \\
  of the night.

  When he returns his spirits must choose purity.\\
  When he conquers kingdoms of death \\
  our dead shall all reign again. 
  
  Our reign may come secretly. \\
  Your Lord's castle is gone. \\
  His glory is no more, \\
  and nothing can stop the purifying. \\
  They remember how you conjured \\
  all those years of sorrow. 
  
  Awakened from the shadows to create. \\
  Riding with me and with all his chosen. \\
  Our lord to be has come again. \\
  Riding from the past. \\
  Little you know your life fades.
}





\subsubsection{Racism}
\target{Needle's racism}
Needle hated \scathae. 
Especially those filthy nobles who had enslaved her, and who were indirectly responsible for her family's poverty and tragedy. 
She longed for the days when they would be the slaves and she would be the mistress. 

Needle knew about the existence of \quiljaaran \serpentmen. 
In her eyes, \scathae were only one step less monstrous than the \serpentmen. 
The \serpentmen were, in turn, only one step less monstrous than the \dragons, who were manifestations of all evil. 









\subsection{Politics}
\subsubsection{Charcoal}
Needle doesn't dislike Charcoal. 
He is her mentor, and she owes him. 
But he is a jerk at times. 
She is tired of working under him. 
She longs to be promoted and be his equal, not his servant. 















\section{Rian}
\target{Rian}
\index{Rian}
Rian is a \human{} man of 17 years or so, at the time of the Pelidor-Runger war. 

Rian is young, sharp-minded, quick-thinking, open-minded and unafraid, not easily ensnared by anyone, not even the Shroud. As such, he can see quite a bit into the Beyond. This allows him to recognize \Ishnaruchaefir{} as being someone extraordinary (although Rian doesn't know the name \quo{\Ishnaruchaefir} nor the term \quo{\vertex}) and even lets him see bits of the \hr{Ishna fights Teshrial's monster}{fight between \Ishnaruchaefir{} and \ps{\Teshrial}} monster. 

Rian doesn't know what's going on, but he doesn't like what he is seeing, so he does what he can to sabotage it, thus antagonizing both the Sentinel and the Cabal side. 









\subsection{History}
\subsubsection{Birth}
Rian is not an orphan. 
But his real parents died of disease when he was in his teens. 


% He was found near the \hs{dead garden} in Malcur. 
% He fantasizes that his parents were great wizards and adventurers who had to leave their child beyond and will eventually come back for him. 
% But this is all daydreaming. 
% His parents were really just regular losers. 
% There is nothing mystical in his origin. 
% His being found near the dead garden has no significance, except perhaps psychologically. 





\subsubsection{Thieving days}
I don't know who took care of him in his childhood. 
Maybe \hs{Crazy old woman}{\Uswa} helped him. 
Maybe some of the nicer and older thieves took care of him. 

Anyway, he's lived his whole life in the slums. 
As a young boy he began to hang out with thieving gangs. 
He joins a gang called the Sneaks. 

He has a close bond to Dennick, a fifteen years older man who serves as a big brother kind of thing for the young Rian. 

The gang leader of the Sneaks is Badrick (real name: Patrick). 
At least at the time when Rian sees {\Ishnaruchaefir} slay a \ghobal{} in \Malcur. 

He has a few good friends of his own age, including Grim and Torm. 

He was once in love with a girl named \Mya. 
She was a tough action girl from another gang. 
She was a year older than he, and strong, and pretty. 
He pined over her a long time before he found out that she already had a lover, Garrick. 

When she and Garrick learned of Rian's infatuation they laughed at him. 
Garrick was like: 
\ta{Wazzat, little boy? You want some of this?}
Then he grabbed and fondled \ps{\Mya} ass. 

Rian was hurt and scared (Garrick was big) and ran away. 





\subsubsection{Friendly priest}
Rian had a nice priest or monk who had been his spiritual guide in the process of breaking away from his life of crime. 
It was he who taught Rian the meaning of his name and a lot of other religious stuff. 
Unfortunately, the old priest died shortly before Neina was kidnapped. 





\subsubsection{No more thieving}
Rian sees \QuessanthIshnaruchaefir{} kill a \ghobal{} in \Malcur. 
This triggers something inside the young thief. 
He is inspired by the greatness he senses in the \draconic{} \vertex. 
He wants to be something better himself, to rise above his thieving past. 

After all, Rian is a religious boy. 
He knows that thieving is wrong. 
He doesn't want to be a lowlife loser and a sinner his whole life. 
He wants to be good and obey the Light. 

There's a man, Bryon Carpenter, whom Rian has helped in the past. 
After some stuff happens, Bryon agrees to take Rian in as an apprentice. 
Rian gets a chance to change his thieving ways and learn an honest trade. 
He jumps at the chance. 
He abandons his gang and moves in with Bryon to become a carpenter. 





\subsubsection{Traumatized}
Rian became quite traumatized after witnessing all sorts of horrible supernatural things while investigating \Malcur and searching for signs of his missing girlfriend, Neina. 

Remember that Rian now has a religious trauma and fears that there is something wrong with the \sephiroth. 
He tries rationalizing it away, but a lingering dread remains and will return to haunt him throughout the story. 

\citebandsong{DeathspellOmega:FasIteMaledictiinIgnemAeternum}{%
  Deathspell Omega
}{
  The Shrine of Mad Laughter
}{
  The idea of God is pale next to that of perdition, \\
  but of this I could have no inkling in advance.
}

Remember that Rian has become scarred and terrified by what he has seen. 
He is harder and more bitter. 

Rian, shocked by what he has witnessed, begins to see more and more through the Shroud. 
He is haunted by monsters from the Beyond. 

\citeauthorbook{RobertEHoward:WhichWillScarcelyBeUnderstood%
}{%
  Robert E. Howard%
}{%
  Which Will Scarcely Be Understood%
}{
  Oh, little singers, what know you of those \\
  ungodly, slimy shapes that glide and crawl\\
  out of reckoned gulfs when midnights fall\\
  to haunt a poet's slumbering, and close\\
  against his eyes thrust up their hissing head,\\
  and mock him with their eyes so serpent-red?
  
  Conceived and bred in blackened pits of Hell,\\
  the poems come that set the stars on fire;\\
  Born of black maggots writhing in a skull \\
  men call a poet's skull\dash an iron bell \\
  filled up wh burning mist and golden mire.
}

Now, as he works with Moro, he learns \maybehr{The dark universe}{how dark the universe really is}. 
    
\maybehr{Rian is religious}{Make Rian more religious}.
Make sure he prays in every chapter and scene that he is in.
He prays to be delivered from \Isphet's evil. 
He has lingering existential/religious dread from the day when he saw the dark sorcerer slay the shining god (even though he was Shrouded and does not remember it all). 

In all the Rian chapters, whenever it is appropriate, have references to the \maybehr{Myths of vanquished monsters}{myths of Iquinian heroes vanquishing inhuman Elder Races and monsters}. 
When he encounters something supernatural, he fears that the wicked Elder monsters will conquer the world. 

Whenever appropriate, Rian thinks about \dragons, including that blackest \dragon of pure evil, \Isphet.
He imagines he hears the beat of black, leathern wings in the dark vaults of night and chaos. 
He imagines he smells the venom of \dragons on the wind. 
He feels the soul-devouring blazing fire of \dragons. 
In one such situation, Moro tells him that (according to her beliefs) \hr{Ortaicans reject Isphet}{\Isphet does not exist}. 





\subsubsection{Disillusioned after divination}
After the traumatic divination experience, Rian has nightmares of the \Sephirah{} soul prison.

Alternatively, it might be Carzain dreaming of the \Malach{} soul prison. 

Quotes from \cite[Dread]{CliveBarker:BooksofBlood:II} (about a guy who, as a boy, suddenly becomes deaf and suffers from fits of tinnitus): 

\lyricsbs{Clive Barker}{Dread (Books of Blood II)}{
  One moment his life had been real, full of shouts and laughter. 
  The next he was cut off from it, and the external world became an aquarium, full of gaping fish with grotesque smiles.
  
  His eyes would jerk open. 
  His body would be wet with sweat. 
  His mind would be filled with the most raucous din, which he was locked in with, beyond hope of reprieve. 
  Nothing could silence his head, and nothing, it seemed, could bring the world, the speaking, laughing, crying world back to him. 
  
  He was alone.
  
  That was the beginning, middle and end of the dread. 
  He was absolutely alone with his cacophony. 
  Locked in this house, in this room, in this body, in this head, a prisoner of deaf, blind flesh.
}

See also the section about \quo{\maybehr{The dark universe}{the dark universe}}. 





\subsubsection{Paranoid}
In the chapter \quo{The Bleeding Wood}, Rian sees through the Shroud and sees the ghastly Beyond for the first time.
He sees a \sphyle that has turned to wood).
After this Rian becomes paranoid. 

\citeauthorbook[p.41--43]{RobertEHoward:TheShadowKingdom}{Robert E. Howard}{%
  The Shadow Kingdom%
}{
  As [Kull] sat upon his throne in the Hall of Society and gazed upon the courtiers, the ladies, the lords, the states,en, he seemed to see their faces as things of illusion, things unreal, existent only as shadows and mockeries of substance.
  Always he had seen their faces as masks, but before he had looked on them with contemptuous tolerance, thinking to see beneath the masks shallow, puny souls, avaricious, lustful, deceitful; now there was a grim undertone, a sinister meaning, a vague horror that lurked beneath the smooth masks.
  While he exchanged courtesies with some nobleman or councilor he seemed to see the smiling face fade like smoke and the frightful jaws of a serpent gaping there.
  How many of those he looked upon were horrid, inhuman monsters, plotting his death, beneath the smooth mesmeric ilusion of a human face?
  
  Valusia\dash land of dreams and nightmares\dash a kingdom of the shadows, ruled by phantoms who glided back and forth behind the painted curtains, mocking the futile king who sat upon the throne\dash himself a shadow. 
  
  \prikker 
  
  After all, the priests of the Serpent merely went a step further in their magic, for all men wore masks, and many a different mask with each different man or woman; and Kull wondered if a serpent did not lurk under every mask.
}










\subsection{Personality}





\subsubsection{Dead garden}
\target{Rian knows the dead garden}
Rian knows the \hs{dead garden} well.
He can find his way in it even though it twists and mutates and paths are never the same.
He has developed an intuition and knows how to recognize safe paths.
He can also tell which parts are dangerous and where the bad monsters might lurk.
Even so, roots and branches grab for him.

The dead garden is twisting and crawling with chaos.
It is monstrous and haunted.
Rian knows it and is afraid.
He prays to Silqua to protect him from the \wylde.
And he stays close to the totems when possible.
He also has a totem talisman, which he clutches hard while praying.
    
He likes the dead garden somehow.
But he knows he should not.
It is a perverse, sinful, evil fascination. 
He should repress it and stay away.
It might corrupt him. 
But his fascination with the place keeps leading him here. 

Make this clear!
It is not because he feels safe in the garden, but because he savours the thrill of the forbidden, dangerous place. 
In his post-thieving days, he lives as an honest citizen.
He craves the excitement of his childhood when he was a thief. 
Being an apprentice does not have the same adrenaline rush.
So he goes to the dangerous, forbidden garden to sate his lust for adventure.

Besides, when he goes here he can feel like a bit of a hero.
An explorer.
A conqueror.
He can play at being a Vaimon.
And he likes that fancy.
If he were a great Vaimon, Neina would love him unconditionally and they would have great sex. 

He prays for forgiveness before, during and after each trip to the garden. 





\subsubsection{Goal}
Rian's goal is to have his own business (as a carpenter, as it turns out) and be his own man. 
Then he wants to marry a girl he loves and have children. 

And he wants to be a father. 
He never had a real father himself, so he wants to be a great father to his children and love them and bring them up so they won't be orphans but real children with a real family, so no one will look down on them. 





\subsubsection{Religion}
\target{Rian is religious}
Rian was a very religious boy. 
After he left his life of crime, he believed it was \iquin that had saved him from sin and given him a new chance at life. 

He prayed to Silqua and the \sephiroth every day for guidance, blessing and mercy, and forgiveness of his sins. 

Rian knew \Isphet, believed in him and feared him. 
He often prayed to be delivered from \Isphet's evil. 
    
Make sure he prays in every chapter and scene that he is in.
Make clear how grateful to the Light he is for how he has been freed from his life of crime and allowed to make a new, honest life for himself.

Whenever he saw something supernatural and evil, he would pray for deliverance from this great evil.

Rian was scarred and horrified to see the evil sorcerer slay the shining god. 
\Criseis Shrouded him and made him forget the details, but some measure of religius/existential dread remained with him. 
Remember that in all his later chapters.

In all the Rian chapters, whenever it is appropriate, have references to the \hr{Myths of vanquished monsters}{myths of Iquinian heroes vanquishing inhuman Elder Races and monsters}. 
When he encounters something supernatural, he fears that the wicked Elder monsters will conquer the world. 

Rian was happy that he had broken away from his life of crime. 
Now that he was a law-abiding citizen, he had come closer to the \hs{Kingdom of the One Light}.
He was closer to being one with the Light. 
By the grace of the \sephiroth he had broken some of the \hr{Iquinian fetters}{fetters of darkness} that tied him to \Itzach.
He knew he still had plenty of fetters. 
He was still a sinner.
But he was on the right track, and he believed that if he worked hard and prayed and was faithful, then the \sephiroth would bless him and enable him to break the rest of his fetters.
So that at last, when he died, they would \hr{Iquinian mercy}{show him mercy} and he would become one with them.

Rian should have one or two \sephiroth whom he especially strives to emulate. 

Rian was horrified whenever he saw the Beyond because \hr{The Beyond horrible to Iquinians}{it was nothing like the \Atziluth he imagined}. 
(Read that section!)









\subsection{Physique}





\subsubsection{\Tulan}
Rian was a \hr{Tulan}{\tulan} (as were Neina and her family).















\section{Taya}
\target{Taya}
\index{Taya}
Taya was a \human girl who served \hs{Lumica} as a slave.
She \hr{Carzain and Lumica}{helped \CarzainShachar kill Lumica} and later became \Shachar's slave. 















\section{Thuza}
\target{Thuza}
\index{Thuza}
Thuza was a male \human sorcerer. 
He captured the girl \hr{Cyri}{\Cyri} and made her his slave and apprentice.















\section[Weylon]{\Rah{\Weylon}}
\target{Weylon}
\index{\rah[\Weylon]}
A \hs{Tiger} knight.
In his old age he retired from fighting duty and went to live with his nephew (son?), Gaen Goldsmith, in \hr{Redglen}{\Redglen}. 

In \Redglen{} he was often called \quo{\Weylon{} Goldsmith}, even though he was not and had never been a goldsmith. 

Weylon befriended \hr{Carzain}{Carzain \Shachar} when the latter was a young girl.
Weylon taught Carzain something of swordsmanship. 















\section{Xerxes}
\target{Xerxes}
\index{Xerxes}
Xerxes was a shah, a king, who lived during the time of \VizicarDurasRespina. 
Xerxes was a rival of Vizicar, and they fought many wars. 
Many times Vizicar tried to conquer Xerxes' kingdom and bring it into the fold of his united \VaimonCaliphate, but always Xerxes thwarted him. 
The tides of war shifted many times, and each monarch managed to capture and imprison his foe at least once, but always the captive would escape and live to fight again. 

In the end Xerxes died of old age, at a time when he had the upper hand. 
Vizicar was then able to turn the tide again and defeat Xerxes' son (also named Xerxes).
But before he could conquer all of the shah's lands, Vizicar himself died. 












































% \begin{comment}
% \part{Nephilim and Humans}
% \end{comment}
\part{Other Mortal Characters}























\chapter{\Nephilim}















\section{\Eshayzal}
\target{Eshayzal}
\index{\Eshayzal}
A \nephil{} mage who was once advisor and companion to \hr{Semiza}{\Semiza}. 
He was \ps{\Semiza} brother or cousin. 

His name is inspired by Azazel, one of the Grigori. 








\subsection{History}
\subsubsection{Became a \Lich}
\target{Eshayzal is a Lich}
After being buried together with \Semiza, \Eshayzal{} became a \Lich. 
He now resembles the \Liches{} from the game \emph{Heroes of Might and Magic V}. 















\section{\Ilu}
\target{Ilu}
\index{\Ilu}
\Ilu{} was a \nephil{} princess, the daughter of sorcerer-king \hr{Semiza}{\Semiza}. 
She was chosen to be the mother of the \hr{Bane Messiah}{\banemessiah}, \hr{Thanatzil}{\Thanatzil}. 
She died giving birth to \Thanatzil. 

She has a bit of \PhedreNoDelaunay{} (from \authorseries{Jacqueline Carey}{Kushiel's Legacy}) in her, in that she is a weak and humble but still strong and brave heroine. 
She sacrifices herself for what she believes is right. 









\subsection{Her lover}
\target{Ilu's lover}
\Ilu{} has a secret lover. He is a warrior, perhaps a ninja-like stealth warrior, who would covertly sneak in through her window at night. They kissed often, and he fingered her a few times (perhaps with clothes on). But she only sucked his dick once. 















\section{\Morza}
\target{Morza}
\index{\Morza}
\Morza was a \nephil soldier in \hr{Numah}{\Numah}.









\subsection{History}
He was one of the soldiers charged with protecting the mothers of the \resphain.
When disaster struck and \Thanatzil failed, he \hr{Morza rescues survivors}{took command and led the mothers and the other survivors to safety}.

He led them to a promised land in \Nyx.

They found some passages in their traditional religious scripture, combined with \Semiza's new bane religion, which foretold that this would happen.
He became their chieftain.





\subsubsection{Saintly status}
For thousands of years after his death, \Morza would be revered as a great hero.
Compare him to Moses and other leaders from the Hebrew Bible.









\subsection{Personality}
\Morza was no great champion or hero. 
He was a regular soldier, a \quo{little guy} who stood up and did what he had to when there was no one else to do it. 















\section[Semiza]{\Semiza}
\target{Semiza}
\index{\Semiza}
\Semiza{} was the \nephilic{} sorceror-king of \hr{Numah}{\Numah}. 
He was responsible for summoning \Daggerrain{} and his \banes{} back to \Miith{} from \Nyx, sparking the \secondbanewar. 

His name is inspired by Shemyaza/Samyaza, one of the Grigori in Judeo-Christian tradition. 









\subsection{Physique}
\subsubsection{Appearance}
\Semiza{} is huge, fat and bloated. 
Compare him to the god Ulcis from \authorbook{Alan Campbell}{Scar Night}. 









\subsection{Skills and powers}





\subsubsection{Immortality as a \Lich}
\Semiza{} is immortal. 
He has eternal life, is very difficult to kill and will \hr{Immortality}{reincarnate} even if killed. 
And his soul is tough and difficult to destroy. 

This is because he is backed up by \ps{\Daggerrain}{} magic. 
The \banelord{} promised him eternal life and eternal rulership in return for helping the \banes, and \Daggerrain{} is not one to bail out on his promises. 

After the Murder of the Dawn, \hr{Semiza becomes a Lich}{\Semiza became a \Lich}.

He looks like the \Liches{} from the game \emph{Heroes of Might and Magic V}. 









\subsection{History}





\subsubsection{\Aryoth blood}
\target{Semiza has Aryoth blood}
\Semiza was no \aryoth. 
He was a mortal \nephil. 
But he did have lots of \aryoth blood in his veins. 
\hr{Aryoth-blooded rulers}{That was why he was royal}. 









\subsection{Myths today}
\Semiza{} is known in myths as \quo{Shemmazya}, the mad king. 






































\chapter[Meccara]{\Meccara}















\section{\Uswa}
\index{\Uswa}
\target{Uswa}
\target{Crazy old woman}
\target{crazy old woman}
A crazy old \meccaran{} who lives in or near the \hs{dead garden} in \Malcur. 

She is inspired by the undead girl Kettle who lives in the Azath cemetery in \cite{StevenErikson:MidnightTides}. 

She is mad, but also somewhat wise. She is able to see far into the Beyond. This is what drove her mad. She also has divination skill and can predict the future (to the extent that such a thing is possible in my universe). But no one ever believes in her predictions. In this regard, compare her to Cassandra from \authorbook{Homer}{The Iliad}.

\Uswa warns Rian that the trees are getting restless, that something is going to happen. She warns him of \hr{Teshrial's creatures}{the worms}.

\lyricsbalsagoth{%
  In the Raven-Haunted Forests of Darkenhold, Where Shadows Reign and the Hues of Sunlight Never Dance
}{
  Can you not see the coils of the worm all about you?\\
  Can you not hear the writhing of the worm beneath you?\\
  Can you not scent the breath of the worm riding the wind?\\
  Can you not touch the skin of the worm in all that surrounds you?\\
  Can you not taste the ichors of the worm upon your tongue?\\
  Do dreams of the worm not haunt your slumber?
}

Not only that. 
The crazy old woman tells long, cryptic stories of the alleged nature of the universe. Compare to the legend told by Redmask in \cite[p.340]{StevenErikson:ReapersGale}, or Feather Witch's Tile castings in \cite{StevenErikson:MidnightTides}, or Fiddler's Deck reading in \cite{StevenErikson:TheBonehunters}.














\section{\Tsekkect}
\target{Tsekkect}
A \meccaran{} woman serving in the Pelidorian army, a friend of Carzain and Delph. 

She hails from the Thbatswa tribe. 
She curses and swears by her own obscure Thbatswa gods. 

She left her tribe and came to the big city in search of glory and riches because she had heard stories of how the big city was full of glory and riches. 
That didn't work out well. 
She became an unemployed loser. 
No one trusted her because she was \meccaran. 

Then she joined the army. 
That worked better. 






































\chapter{Beasts}















\section{Belgrim}
\target{Belgrim}
\index{Belgrim}
%\sectioncharunspec{Belgrim}{Cortio}{\male}
\Narkiza's Cortio mount. His full name is Belgrim IV, since he is \Narkiza's fourth mount (all four have been male and named Belgrim). 















\section{Demores (Countess)}
\target{Demores}
\target{Countess}
\index{Demores}
\index{Countess}
%\sectioncharlive{Countess}{\nycan}{\female}
A companion of Telcastora Ilcas, a \nycan{} of the Dorlinum breed of the Secca race and of exceptional size and strength.



















\section{Vanix (Razor)}
\target{Razor}
\target{Vanix}
\index{Razor}
\index{Vanix}
%\sectioncharlive{Razor}{\nycan}{\male}
%Male \nycan{} of the Destran race. 
A companion of Telcastora Ilcas, a \nycan{} of the Mictzan breed of the Destran race and of singular intelligence. 

He is quite young, several years younger than Countess. 









\subsection{Arsenal}
Razor is highly intelligent and perceptive and an excellent telepath. 

He understands spoken language to some extent. 
He understands Imetric pretty well, and some \Velcadian, but not as good as he and Ilcas would like. 
His trainers taught him Imetric first and \Velcadian{} second, since Imetric was considered their first language. 
In retrospect this was a mistake, since Razor would go on to work as an agent in foreign lands. 

Razor is also very fast and agile. 
He cannot run as fast or as far as Countess in a straight line, but he is more agile and \manoeuvrable. 





\subsubsection{Healing}
\target{Razor's psionic healing}
Razor is skilled at psionic healing. 









\subsection{History}
After Telcastora Ilcas dies, Razor goes into the \Wylde{}. 
Here he encounters old and wise \nycans, and he learns magic. 
He returns having \trope{TookALevelInBadass}{Taken a Level in Badass}. 

Razor is, or becomes, a powerful \vertex. 
He meets/seeks out/is summoned by a loose council of \nycan{} elders/\vertices/demigods. 
They take him in as an apprentice and a member of their \matrixx. 
Via telepathy they teach him \hr{Nycan magic}{their magic}. 

Many other \nycans{} fear this magic. 
But Razor sees it and immediately goes: \ta{Hell yeah!}

This makes Razor one of the few heroes of my story that are young and completely mortal. 

There are also \cuezcans{} among the elders whom Razor meets. 
\Cuezcans{} are allied with \nycans, remember. 















