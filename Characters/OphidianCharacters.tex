
\part{\Ophidian Characters}























\chapter{\Dragons}
















\section{\AeocrithRystessakhin}
\target{Triestessakhin}
\target{Rystessakhin}
\index{\Rystessakhin}
\index{\AeocrithRystessakhin}
A \dragon, \ps{\Ishnaruchaefir} beloved and the mother of \Nzessuacrith. 

She opposed the scheme of the \hr{Shrouding}{\SecondShrouding}, which \Ishnaruchaefir{} advocated. 
They fought over this. 
Eventually \hr{Ishnaruchaefir slays his beloved}{he killed her}. 

\Ishnaruchaefir{} bound her soul in \hr{Ishnaruchaefir's glaive}{his glaive}, which became his \hs{weaving artifact}. 
\ps{\Triestessakhin}{} became condemned to an eternity as an unwilling \trope{BarrierMaiden}{Barrier Maiden}.

\lyricsbalsagoth{The Hammer of the Emperor}{
  A garland of newborn stars to adorn thee...\\ 
  the Permian Extinction, a parting gift.\\
  May your maleficent soul walk only in dark places.
}

\Triestessakhin{} is now condemned to an eternity as an unwilling \trope{BarrierMaiden}{Barrier Maiden}. 
She suffers. 

\lyricslimbonicart{Funeral of Death}{
  Blood is dripping as mind's tripping.\\
  In twilight sleep death is reaping.\\
  Blood stained, feels cold\\
  in solitude as night grows old. \\
  Death salvation, life capitulation, \\
  blood stained, feels cold, in the frozen soul.
  
  Life is a mandatory sacrifice \\
  for the eternal dream of Paradise. \\
  Make the time stand still, \\
  as silence is the last will.
}









\subsection{Consciousness or lack thereof}
\target{Rystessakhin's consciousness}
\Rystessakhin{} retained very little consciousness in her state. 
Her soul had been crippled and warped from the imprisonment and the immense strain that she carried in her efforts to uphold the Shroud. 
She had lost most of her personality and had no hope of regaining true consciousness ever again. 

\Ishnaruchaefir{} told himself that she lived and retained some measure of consciousness, and could even be contacted. 
But he deceived himself. 

Once in a while, he meditated on the glaive and sought out her soul on the astral plane (or whatever), where they would have \quo{sex}... of a sort. 
They clashed in a violent, sexual storm of love, hate, lust, pain, sorrow and conflict. 
When he returned from these seances, he is at once rejuvenated and aged. 
Haggard, but vitalized. 
To him, she remained within his reach (German: \emph{zum Greifen nah}), and yet forever lost to him. 

\lyricsduana{beached}{Beached}{
  \textbf{the storm} 
  
  white veins of light \\
  streak thru' clouds blackened by lust \\
  betray Earth \& Sea \\
  amidst their lovers' spat 
  
  waves crash dangerously \\
  'gainst jagged rock \\
  foaming \\
  a red passion 
  
  Midnite embraces them\dash{}still they rage \\
  the violent spray whips \\
  exposed leg of beach \\
  with icy fingers \& howl in climax
  
  \textbf{the calm} 
  
  scales\dash{}iridescent \& smooth \\
  (so like human fingernails) \\
  polished abalone \\
  cold grey flesh slicked with mucus 
  
  locked \\
  into a salty stare \\
  unblinking\dash{}that lovers' gaze \\
  tears frozen \\
  like crystals of ice \\
  distant \\
  and hidden \\
  beneath the promise \\
  of a kiss\dash{}blue lips \\
  the treasure of pearls lie \\
  cold \& mute 
  
  porpoise song \\
  (so erotically mournful) \\
  flows a liquid harmony \\
  \dash{}unheard\dash{} \\
  by their shell-like ears \\
  \& washed upon a sandy tomb 
  
  whisper \\
  timeless Ocean 
  
  silence\dash{}so final
}





\subsubsection{\Criseis{} disbelieves}
\Criseis{} is \skeptical. 
She belives that, at best, \Triestessakhin{} is locked away in permanent sleep and with no consciousness. 
More likely, she is forever dead and destroyed and exists only as an echo of her soul imprinted in the glaive, and as a memory in \ps{\Ishnaruchaefir} mind. 
That is why \Criseis{} is so worried whenever her master talks to the glaive: 
She feels he is indulging his mad delusion that \Triestessakhin{} is still alive, and fears that dwelling on it will only cause him to spiral further into madness and denial. 









\subsection{Personality}
\target{Rystessakhin's personality}
\Rystessakhin was something of a \quo{\trope{HighQueen}{High Queen}}.
She was noble and loving and good and loved all creatures.
When \Ishnaruchaefir preferred to look outward into the universe and had little interest in the affairs of mortals, \Rystessakhin looked inward and down. 
She cared deeply for the lesser creatures and was always advocating that the \dragons be more humane towards them.

Later, \Ishnaruchaefir, after killing her, would \hr{Ishnaruchaefir's compassion}{feel some obligation towards carrying on her will}. 















\section{\ApharesNesthra}
\target{Apep-Nesthra}
\index{\ApharesNesthra}
\ApharesNesthra was an \ophidian and one of the first generation of \dragons.
He was a consort of \Xserasshana and the father of \Secherdamon. 

\ApharesNesthra was a brutal, bloodthirsty god of war. 
He was awakened along with \Sethicus and others to fight in the \firstbanewar. 
He was slain by the \banes, but sort of survived by shattering his immortal spirit into many fragments. 

Using powerful spells and prayers, a \dragon could later call upon a fragment of \Nesthra and draw upon its power. 
\Nesthra's might, fury, hatred and bloodthirst lived on. 

Compare him to the Eldar god Khaine in \cite{RPG:Warhammer40000}. 















\section{Candrazor}
\target{Candrazor}
\index{Candrazor}
%\sectioncharunspec{Candrazor}{\dragon}{\male}
An ancient \dragon{} and a Sentinel Lord Questor. 
Another of his true names is Rochydanoss. 
He has a number of secret identities, including Lord Cassander. 










\subsection{Cassander}
%\sectioncharunspec{Cassander}{\human}{\male}
A \human{} lord and sorcerer living somewhere in \Velcad{}. Actually the \dragon{} Candrazor in disguise. 

He looks much like the Count of Monte Cristo as portrayed in the anime \cite{Anime:Gankutsuou}. 















\section[Cryocas Nzessuacrith]{\CryocasNzessuacrith}
\target{Nzessuacrith}
\index{\CryocasNzessuacrith}
\index{\Nzessuacrith}
A \dragon, the daughter of \QuessanthIshnaruchaefir and \AeocrithRystessakhin.









\subsection{History}





\subsubsection{\Takestsha}
\target{Takestsha}
\index{\Takestsha}
\target{Takestsha on the run}
\Takestsha, ostensibly a \human{} woman, is \Nzessuacrith{} in disguise. Allegedely, she is a mage on the run from her masters because she discovered something she was not supposed to learn. She has manipulated \hs{Morgan Runger} into \hs{Pelidor-Runger war}{waging war against Pelidor}. 

As part of \hr{Secherdamon wants Nithdornazsh}{\ps{\Secherdamon} \Nithdornazsh gambit}, \Takestsha led a \hr{Tantor's journal}{Rungeran expedition} to \hr{Eresh-Kal}{\EreshKal}.









\subsection{Names}
\Cryocas was her egg-name.
\Nzessuacrith was her adult name. 

To the \Ortaicans she was known as the \taortha \hr{Usherain}{\Usherain}. 









\subsection{Personality}





\subsubsection{Art}
\Nzessuacrith{} enjoys the arts and actively pursues them. 
She has written a great wealth of poetry and literature, including some long epic poems. 
Some suspect her of being the true identity of \hr{Melcryth}{\Melcryth}, the author of \emph{\hr{Wanderers in Darkness}{\WanderersInDarkness}}, but she denies this. 





\subsubsection{Likes beautiful things}
\target{Nzessuacrith likes beauty}
\Nzessuacrith{} kind of likes beautiful things, such as buildings. 
She has no compunctions against killing any number of people, but she hates destroying beautiful buildings and will go out of her way to avoid it.





\subsubsection{Sex}
She has worked undercover a lot and has had plenty of sex. 
She dislikes sex with \human{} men, because they tend to want a submissive woman, and submissiveness, needless to say, goes against her \draconic{} nature. 
She resents \hr{Morgan has sex with Takestsha}{being forced to have sex} with \hs{Morgan Runger}, who is bad in bed. 

Even so, it's not hard for her to do it. 
\Dragons{} have no sexual taboos, so having sex is no more controversial than engaging in conversation. 










\subsection{Physique}
In her true, \draconian{} form, she has scales shining the \colour of steel or silver. 
She has a vertical \quo{comb} of horns on her head and a sail that runs down the length of her spine. 
Maybe she has fins on her head. 









\subsection{Politics}





\subsubsection{\Vizsherioch}
\Nzessuacrith{} does not quite trust \hr{Vizsherioch}{\Vizsherioch}. 
She fears him. 
She is unnerved whenever he calls her \quo{cousin} (even though they \emph{are} cousins). 
He is soft-spoken and polite (by \draconic{} standards), and this just makes him even more creepy and sardonic. 
There is something very sinister within him. 

There is too much alien \xs{} nature in him. 
She fears that \Secherdamon{} has gone too far in breeding and moulding \Vizsherioch, and \hr{Nzessuacrith fears for Secherdamon's sanity}{she fears for \ps{\Secherdamon} sanity}. 

\Vizsherioch{} \hr{humanoid Vizsherioch}{often appears in humanoid form}. 
\Nzessuacrith{} believes this is to masquerade as a harmless \scatha{} and hide his true, horrible power. 









\subsection{Skills and powers}
\target{Nzessuacrith's stealth}
\Nzessuacrith was a master of stealth.
She had studied it idly even before the \secondbanewar, and after that she had spent thousands of years perfecting it. 
She could hide herself better than any other immortal, or so she claimed.

Because she was so stealthy, she was also very skilled at detecting others.









\subsection{\Usherain}
\target{Usherain}
\Nzessuacrith was a goddess of the \hr{Taortha}{\taortha} pantheon of \hr{Ortaica}{\Ortaica}. 
Here she was known under the name \Usherain. 
She held a portfolio overlapping with that of \hr{Nasshikerr}{\Nasshikerr}. 















\section{\DiorethRephexsar}
\target{Rephexsar}
\DiorethSethicusRephexsar was a \dragon.
She was the daughter of \ValcanSethicus and as such the half-sister of \RaemythNexagglachel.
She was the mother of \NarucOsanggrath. 









\subsection{History}
\Rephexsar awoke together with \Nexagglachel. 
The two became rivals and fought for domination of the world.
Epic and devastating wars were fought between their two fractions. 

\Rephexsar fought in the \secondbanewar. 
Here she was killed, not permanently but very badly. 
It took her a long time to resurrect and recover. 
By then she had been overtaken in political influence by \Secherdamon. 

In the \Scatha Age she was not very active in \Azmith, but held great power in other Realms.
She was still one of the mightiest beings alive on the planet. 















\section{Eresshakal}
Eresshakal was the oldest \dragon still around at the time of the \thirdbanewar\dash older even than \Iscrafel. 
She had spent most of that time in Durance.















\section{\IrocasSecherdamon}
\target{HriistD}
\target{Secherdamon}
\index{\Secherdamon}
\index{\IrocasSecherdamon}
\IrocasSecherdamon{} was the youngest of the three sons of {\Tiamat} and the brother of \Ishnaruchaefir and \Nexagglachel. 
His father was \hr{Apep-Nesthra}{\ApepNesthra}.

He always looked up to \Nexagglachel, and after his brother's death \Secherdamon{} assumed his role as \dragonking. He is one of the leaders of the Sentinels and one of the most influential \dragons{} in the world.

One of his secret identities was \HriistN. 
In this guise he led the Rissitic Empire of \Durcac. 

In a sense, \Secherdamon{} was a planner, schemer and creator. 









\subsection{Arsenal}





\subsubsection{Cannot tread on \Miith}
To gain all his power, \Secherdamon{} has absorbed too much \xzaishannic{} power. 
He has been warped into \xzaishann-like being. 
Now he is affected by the \hr{XS slumber}{sleepiness of the \xss} and cannot tread the world of the Shroud. Unlike \Ishnaruchaefir, who can still travel freely.

A vital part of his master plan is to open the path so that he himself may enter the world of the Shroud. \hr{Secherdamon wants Nithdornazsh}{\Nithdornazsh{} is a stepping-stone} on this path.

Soon, he will enter \Miith{} in all his glory.

\lyricsbalsagoth{
  Behold, the Armies of War Descend Screaming From the Heavens
}{
  [THE DISCIPLES OF ZAKUMAKURA:]\\
  Since before mankind hurled himself squamously from the sea we have awaited the awakening of great Zakumakura!\\
  Now... the Dragon-King shall at last rise to claim his earthly throne!\\
  Cast your gaze to the firmament and know fear, for His forces fill the sky!\\
  Behold, the armies of war descend screaming from the heavens!
}





\subsubsection{Happiness}
\Secherdamon{} knew some magic that could make him forever happy by flooding his brain with super-charged pleasure. 
And it would \emph{work}, with no nasty side effects. 

But he knew that if he started doing it, there was a chance he would keep doing it and neglect his quest and leave \Miith{} to its fate, which is something he would not do. 
So he sacrifices happiness for the sake of his war. 

Perhaps he regained this happiness \hr{Secherdamon dies}{when he died}. 





\subsubsection{Power}
At the time of the \hs{Unravelling}, \Secherdamon{} was one of the mightiest \dragons{} alive. 
In many respects he was more powerful than his brother, \Ishnaruchaefir. 
\Ishnaruchaefir{} might be able to best him in a duel, but on a larger scale \Secherdamon{} had nastier weapons at his disposal and far more political influence (although \hr{Ishnaruchaefir and the Sentinels}{\Ishnaruchaefir{} had more contacts than he let on}). 

In fact, \Secherdamon{} eventually gathered much more political power than \Nexagglachel{} ever did. 
See, in \ps{\Nexagglachel} time the \hr{Dragons disorganized}{\dragons{} were disorganized} because they had nothing to unite \emph{for}. 
But after the \secondbanewar{} they found a common, external enemy in the \resphain.
This helped \Secherdamon{} unite them. 
They had something to unite \emph{against}. 





\subsubsection{\Resphan slaves}
\target{Secherdamon's Resphan slaves}
\Secherdamon kept a number of \resphan slaves. 
These were captives that he and his Sentinels had caught over the millennia. 
He used them to create his \hs{Rissitic undead} (such as the \hr{Ashenoch}{\Ashenoch}), breed his \hr{Reptile Colossus}{\reptilecolossi} and power his \hr{Rissitic Matrix}{Rissitic \matrix}.

In \Durcac, there were rumours and legends that in the crypts deep beneath the pyramids, the undead priests kept immortal \human- or angel-like creatures captive, as wailing, weeping, maimed, tortured slaves. 









\subsection{History}





\subsubsection{One of the first pureborn \dragons}
\Secherdamon was born the son of \Tiamat and \ApepNesthra. 
He was one of the first \ophidians born to a pair of \dragons. 





\subsubsection{Friendship with \Nexagglachel}
Irocas was at first a hanger-on. He was a study buddy of \Quessanth and \Raemyth during their studies of sorcery, and they benefitted greatly from their friendship with the bookish and gifted Irocas. Irocas himself was extremely ambitious as a scholar and sorcerer\dash{}he craved knowledge. 
Curiosity, \hr{Irocas Exaltation}{he actually mellowed out after his draconian exaltation}.






\subsubsection{Seeking Gnosis}
\target{Secherdamon seeks Gnosis}
Already in his youth, before the \secondbanewar, \Secherdamon{} was a scientist who sought for Gnosis. 
He sought to understand the Aenigmata of the \xss. 
Then \hr{Ishnaruchaefir steals Secherdamon's research}{\Ishnaruchaefir{} stole his research}.

\target{Secherdamon becomes better when denied Gnosis}
This made him bitter. 
He felt betrayed. 
He had \emph{earned} that Gnosis much more than \Ishnaruchaefir{} had.
\Secherdamon{} had worked hard on it his entire life. 
\Ishnaruchaefir{} had emotionally blackmailed him into handing over all his centuries of research, and then stabbed him in the back, refusing to share the fruits of their joint labour. 

That was one of the reasons why he hated \Ishnaruchaefir{} so much. 

Later he finally \hr{Secherdamon gains Gnosis}{gained the Gnosis he sought}. 





\subsubsection{Exaltation}
\target{Irocas Exaltation}
Curiosity, he actually mellowed out after his draconian exaltation and began to devote himself to the arts instead of being so obsessed with the sciences and his personal greatness. 






\subsubsection{Early history}
Originally, \HriistD{} was only a weak \vertex, living in the shadow of his elder and more capable brothers. 

After the \hr{Second Banewar}{\secondbanewar} he was too weak to participate in the \hr{Shrouding}{\SecondShrouding}, and it pained him. 





\subsubsection{Takes the name \Nexagglachel}
\target{Secherdamon takes the name Nexagglachel}
In the process of developing his new \hr{Rissitic magic}{Rissitic magic theory}, \Secherdamon made many great discoveries and achieved much Gnosis.
Eventually, he had so much power that he could take another name. 

He took the name {\quo{\Nexagglachel}} (after his brother \hr{Nexagglachel}{\RaemythNexagglachel}, whom he admired). 

It was at the same time that he took up the mantle of \quo{\RissitNechsain}, \hr{Rissit is a saviour}{the saviour of the \Ortaican people}. 





\subsubsection{Gained his Gnosis}
\target{Secherdamon gains Gnosis}
\Secherdamon{} spent his entire life \hr{Secherdamon seeks Gnosis}{seeking the Gnosis of the \xss}. 
For thousands of years he laboured without result. 
He blamed \Ishnaruchaefir{} and his Shrouding for that. 
The Shroud made it much harder to do research. 
\Secherdamon{} could feel himself growing stupid as the Shroud closed in, like a noose tightening around his neck. 
He \hr{Secherdamon hates the Shroud}{hated the Shroud}. 

It was not until after \hr{Vizsherioch}{\Vizsherioch} was born that they, working together, finally found that Gnosis. 
\Secherdamon{} knew, though, that \Vizsherioch{} saw deeper and realized more far-reaching Gnoses, insights that he could not communicate to his father. 
\Secherdamon{} was saddened at the thought that he might never learn this Gnosis. 





\subsubsection{Breaking the Shroud}
\Secherdamon{} had his own plan to break out of the Shroud. 

It was the reason why \hr{Rissitic creativity}{the Rissitics could be such free-thinkers}. 

When at last he \hr{Secherdamon gains Gnosis}{found his Gnosis}, he was (almost) ready to invoke \hr{Naath-Kur-Ramalech}{\NaathKurRamalech} and destroy the Shroud. 





\subsubsection{The Dagger}
\target{Dagger}
\index{Dagger, the}%
\ps{\Secherdamon} goal in this whole Shroud-drilling business was not so much to allow his Rissitics to be creative. 
That was just a nice bonus.
His real objective was to shape a powerful \vertex{} that would be able to cut through all obstacles, pierce the Shroud and slash open the way for \xsic{} domination. 

In the beginning, \Secherdamon{} thought he himself was going to become the Dagger. 
Later he decided this was a blind alley, and began creating \Vizsherioch, intending for him to become the Dagger. 

Early on, \Secherdamon{} made sure to prepare a place in his \matrix{} for the Dagger. 
But for a long time this slot stood empty, because no one was powerful enough to claim the title. 

The resurrection of \Nithdornazsh{} was \hr{Vizsherioch and Nithdornazsh}{a vital step in the forging of the Dagger}. 

Finally, \hr{Vizsherioch becomes the Dagger}{\Vizsherioch{} became the Dagger}. 









\subsection{Names}
\Secherdamon's full name was \Irocas \Veldraxx \Nexagglachel \Secherdamon. 

He \hr{Secherdamon takes the name Nexagglachel}{took the name \quo{\Nexagglachel}} (after his brother \hr{Nexagglachel}{\RaemythNexagglachel}, whom he admired). 








\subsection{Personality}





\subsubsection{Behaviour and titles}
\Secherdamon{} might use the royal \quo{we} (\emph{pluralis majestatis}). 
Were he to speak Japanese he might use \emph{ore-sama} or a similar pronoun.

His servitors, such as \LocarPsyrex, call him \quo{Exalted Lord}. 





\subsubsection{Generosity}
\target{Secherdamon's reputation}
\Secherdamon{} is known as a generous benefactor to those who are loyal to him. 
He does care about his servants and allies. 

\Ishnaruchaefir{} has \hr{Ishnaruchaefir's reputation}{a different reputation}. 





\subsubsection{Leadership}
\Secherdamon{} tries to be a heroic and self-sacrificing leader, like \TyarithXserasshana{} and \Nexagglachel. 
A pillar of strength and a guiding beacon for his people. 
He does not always succeed, but he tries his best. 

Compare to Anomander Rake in \cite{StevenErikson:TolltheHounds} especially. 

\Secherdamon{} has fought hard and long to reach the position where he is. 

\citebandsong{Ihsahn:TheAdversary}{Ihsahn}{%
  And He Shall Walk In Empty Places
}{
  \quo{
    Remember this, you others.\\
    The fire and the fury,\\
    the strength and defiance,\\
    this you admire, this you desire.\\
    I had to win them for myself.}
}

\TyarithXserasshana{} was his great idol. 
He admires her passion, willpower and ruthlessness. 

\citebandsong{Ihsahn:TheAdversary}{Ihsahn}{%
  And He Shall Walk In Empty Places
}{
  In remembrance of the adversary\\
  I conjure up the lion will:\\
  Hungered Violent Solitary Godless.
}

He will reach his goal, no matter the cost.

\citebandsong{Ihsahn:TheAdversary}{Ihsahn}{%
  And He Shall Walk In Empty Places
}{
  And he shall walk in empty places,\\
  with a claim on destiny and self at hand.\\
  An endless journey towards the rising sun.
}





\subsubsection{Manipulation and ruthlessness}
\Secherdamon was more calculating but less cruel than his brother \Iscrafel. 
\Secherdamon was a master manipulator.
He was willing to sacrifice anyone for his plots and experiments. 
He actually did believe that mortal lives had moral value.
He just believed that his long-term plan was much more important than any mortal lives. 
It pained him in the beginning to sacrifice them, but after some centuries he became completely numb to the suffering of mortals. 
He could not let anything deter him from pursuing what he believed was the greater good for the races of \Miith (especially the \draconian race). 




\subsubsection{Megalomania}
\target{Secherdamon's megalomania}
\index{stewardship}
\Secherdamon{} sees himself as \Miith{}'s saviour and protector against the wicked \banes. He stood up and took upon himself the mantle of the planet's steward when no one else was willing and worthy of it. 

In his view, \resphain{} and \humans{} are a disease, an infestation which must be destroyed. 

\lyricsbalsagoth{%
  In the Raven-Haunted Forests of Darkenhold, Where Shadows Reign and the Hues of Sunlight Never Dance
}{
  I am the immortal King of the Deep Woods,\\
  servitor of the Old [\Firstgendragons].
}

\index{stewardship}
He has inherited and gained much of his power from the \firstgendragons. He sees it as a divine gift and a confirmation of his status as the \Miith{}'s steward, caretaker and ruler\dash the one responsible for the planet's future. 

To his eyes, \Tiamat{} and the \firstgendragons{} are the ultimate apotheosis: The combination of natural (\ophidian) and supernatural (\xzaishannic) power, the perfect harmony of Order and Chaos, intelligence and brute force. And he is their representative, their high priest, their heir... perhaps even their peer. 

\lyricsbalsagoth{%
  In the Raven-Haunted Forests of Darkenhold, Where Shadows Reign and the Hues of Sunlight Never Dance
}{
  I hear the whispered words of the [\daemons].\\
  Such ancient secrets they sing...
}

\target{Nzessuacrith fears for Secherdamon's sanity}
He is arguably going insane. 
He has been too close to the \xss, and his world view, goals and ethics have become gradually twisted. 
\Nzessuacrith{} fears for him. 





\subsubsection{Science}
\target{Secherdamon's science}
\target{Secherdamon's research}
\ps{\Secherdamon} greatest obsession is actually not power, but \emph{knowledge}. He thirsts to know everything, unravel all secrets of the universe. This obsession has made him perhaps the greatest scientist on \Miith{}, and his research has created enormous amounts of new knowledge of magic and other sciences. 

A notable example is his development of the Rissitic magic theory. 

Another is the creation of his son \hr{Vizsherioch}{\Vizsherioch}.

His research is not just for discovering new things, but very much also for rediscovering \hr{Dragons have forgotten}{the knowledge they once possessed}. 

\target{Secherdamon hates the Shroud}
Because of his scientific disposition, \Secherdamon{} \emph{hated} the Shroud and hated \Ishnaruchaefir{} for masterminding it. 
To him, the stupefying Shroud was an atrocity against \Miith, nature, the universe, and indeed the very concepts of Knowledge and Truth. 
Aenigma and Gnosis. 
The foundations of the greatness of the \dzraicchenosses. 





\subsubsection{Ideal: The old \dragonland}
\ps{\Secherdamon} ideal is the old fallen \dragonland, the ancient empire of his people in their glory days. 

\lyricsbs{Arcane Wisdom}{%
  Abyssic Wrath of the Death-Philosophy (Once a Proud and Ancient Civilization)
}{
  Bathing in ice-cold rivers by the starlit night. \\
  Galloping on the fronts by the shivering light. \\
  Brave Europa, with our kin shinning bright, \\
  united we stood when need appeared to fight.
  
  Abyssic Wrath of the Death-Philosophy. \\
  Abyssic Wrath of the Death-Philosophy.
  
  Tears blurring our vision. \\
  Lethe in our swords. \\
  To the Strife!
  
  A myriad centuries later, \\
  the need is now stronger. \\
  Clearly what we had is ours no longer. \\
  Time hast come to avenge their mourning souls. \\
  Take back what is ours, \\
  put to the sword the weak ghouls.
}









\subsection{Physique}
\Secherdamon{} was golden in \colour, with stripes of black and red. 
He conveyed an image of blazing fire with some black ash/coal/smoke interspersed. 
His eyes were pearly white. 
He had a crown-like formation of many horns. 

\Secherdamon was longer than \Ishnaruchaefir but lighter. 
Compared to the \hr{Dragon size}{standard \draconian proportions}, he was quite long and slim. 

He had multiple heads. 
He changed his body into a multi-headed hydra to emulate \Tiamat, his great ideal. 

\Secherdamon was terrible to look upon. 
Even more so than \Ishnaruchaefir.
\Secherdamon had gathered power and worked hard to make himself imposing and regal and dominant and godlike.

\citebandsong{Nile:Ithyphallic}{Nile}{
  As He Creates So He Destroys
}{
  No living creature can look upon his face\\
  And endure its terrible heat and black radiance\\
  That is like the reverberating unseen rays of molten iron\\
  Which strike and burn the skin of those who would dare\\
  Gaze into the countenance of the idiot god
}









\subsection{Politics}
Among \ps{\HriistD}{} allies is \Xarocchetsel, who has his own mystic agenda. And his son, \Vizsherioch, and \Nzessuacrith. 

He has all sorts of creatures, \Miithian{} and alien, serving him.

\lyricsbalsagoth{
  In the Raven-Haunted Forests of Darkenhold, Where Shadows Reign and the Hues of Sunlight Never Dance
}{
  Swaying serpents ring my oak-hewn throne,\\
  Night and Shadow are my hunting dogs.\\
  Ravenous, they howl to be unshackled,\\
  that their maws may be glutted \\
  (with the blood of my foes).
}

He is the closest thing the Sentinels of \Miith{} have to a supreme leader. 

\lyricsbs{Emperor}{I Am the Black Wizards}{
  Mightiest am I, \\
  but I am not alone in this cosmos of mine. \\
  For the black hills consists of black souls, \\
  souls that already dies one thousand deaths. \\
  Behind the stone walls \\
  of centuries they breed their black art. \\
  Boiling their spells in cauldrons of black gold. 
  
  Far up in the mountains, \\
  where the rain fall not far, \\
  yet the Sun cannot reach. \\
  The wizards, my servants, \\
  summon the souls of macrocosm.
}





\subsubsection{Plagued with corruption}
\target{Secherdamon plagued with corruption}
\Secherdamon's Sentinels were plagued by much corruption, infighting and inefficient bureaucracy. 
\Secherdamon had much trouble getting anything done. 
He could not be everywhere at once. 
And it was hard for him to determine which subordinates were faithful and which were twisting his will and running their own businesses behind his back. 
And he was dependent on them, so he could not just crack down on them. 

This is an aversion of the trope \quo{\trope{TheTrainsWillRunOnTime}{The Trains Will Run On Time}}.

This was why the Rissitic Dominion \hr{Rissitic tribes divided}{fell apart in \Narkiza's time}.









\subsection{\HriistN} 
\target{Rissit Nechsain}
\target{Rissit}
\index{\HriistN}
\index{Rissit}
The primary god of the Rissitic religion. 
\quo{\Hriist} is his true \emph{spirit name}, \quo{\Nechsain} is his title. 
Also called \quo{Rissit} by nonbelievers. 
%(\quo{\Hriist} is difficult to pronounce: The initial HR is pronounced as a [H] followed by an [RH] (guttural R). Both T's should be audible. Also note that in \quo{\Nechsain}, the CH is indeed a [CH], not a [KH] (unlike similar words like German \emph{nächste}, [NEKH-ste]]).)

Rissit's symbol is a dagger thrust into the ground with a cobra snake coiled around it, as if crawling down from the heavens. 

He was originally an \hr{Ortaican gods}{\Ortaican{} god}. 
His following \hr{Rissitics were Ortaican}{was an \Ortaican{} sect}. 
But he later split with the rest and formed his own religion. 

Unbeknownst to his Rissitic worshippers, he is actually the \dragon{} \hr{Secherdamon}{\IrocasSecherdamon}. 



\subsubsection{The Seven Scorpions}
There are seven powerful \daemonic{} demigods that serve \Nechsain. They take the form of scorpions or scorpion-like humanoids or monsters. Sometimes the seven merge together in one body, resembling a scourge/whip with seven tails, each shaped like the tail of a scorpion. 

When \Nechsain{} fights, the seven scorpions swarm around him and fight for him.















\section{Ixaeor Anakeron}
\target{Ixaeor}
\target{Anakeron}
Ixaeor Anakeron, a female \dragon, was a military commander. She was a friend of \Nexagglachel and \Iscrafel.

She fell during the Incursion War.














\section[Laccashyth]{\Laccashyth}
A female \dragon. 















\section{\QuessanthIshnaruchaefir}
%\sectioncharunspec{\Ishnaruchaefir}{\dragon}{\male}
\target{Ishnaruchyfir}
\target{Ishnaruchaefir}
\index{\Ishnaruchaefir}
\index{\QuessanthIshnaruchaefir}
The second of the three sons of \Tiamat and the brother of \Nexagglachel{} and \Secherdamon.
His father was \Iurzmacul. 

\Ishnaruchaefir{} was an explorer, unraveller and destroyer. 
He was \hr{Ishnaruchaefir's power}{immensely powerful}. 









\subsection{Arsenal}





\subsubsection{Power}
\Iscrafel, along with \Secherdamon, was the most powerful \dragon alive at the time of the \thirdbanewar.
This was because they had delved deeper into the mysteries of the universe than any others alive. 









\subsection{Glaive} 
\target{Ishnaruchaefir's glaive}
\target{glaive}
\ps{\Ishnaruchaefir} weapon of choice is a \quo{glaive}, named \Triestessakhin.

Within it, he has \hr{Ishna slays his beloved}{bound the soul of his beloved}, \Triestessakhin{}. 
It is also his \hr{Weaving artifacts}{weaving artifact}. 

\Triestessakhin{} has the form of a massive, scythe-like polearm with multiple blades sticking out in both directions. 
Heavily inspired by the one wielded by Crusnik in the anime \cite{Anime:TrinityBlood}.

The weapon holds tremendous emotions\dash all of the bruised emotions from \ps{\Ishnaruchaefir} and his love's relationship and mutual betrayal. 

It is clearly alive. 
It moans, howls, screams, cries and whispers to him. 
Its words are mostly nonsense. 
\Triestessakhin{} is quite insane after thousands of years of imprisonment, after all. 

She can also communicate with others, mostly in dreams. 
Perhaps she seeks out people (Ramiel?) and haunts them like a ghost. 

She loathes her own role as an instrument of the Shrouding. 
She is a good, loving mother-type character. 
\Ishnaruchaefir{} sadly comments that she was \quo{too good for this world}.

At times the glaive can be seen to weep tears or blood.

Perhaps at the end of the story, \hr{Ishnaruchaefir's glaive is destroyed}{the glaive will be destroyed}. 

He does not carry \Triestessakhin{} with him at all times. 
Most of the time it follows him around in the form of a maddened, grieving ghost which only he\dash and the especially clear-sighted\dash can see and hear. 
When he needs it as a weapon, he must shed some of his own blood in order to give it form. 
See an example of this {when he fights a \ghobal{} in \Malcur} in \TwilightAngelRememberEmph.

The glaive's blade is constructed using lots of \hs{occult geometry}. 
Thus the strange shape. 
It utilizes occult knowledge that he learned from some \hs{cosmic gods}. 





\subsubsection{Name}
Do not call the glaive \quo{\Rystessakhin}. 
Just \quo{the glaive}.
At least, not in the beginning. 





\subsubsection{Summoning the glaive}
\begin{prose}
  The \scatha{} held his right hand in a fist before his chest, touched it with his left, and Rian watched as he raked his sharp claws along the back of his own hand, drawing blood. There was a seething sound, and smoke seemed to rise from the wound... as if the black one's blood was boiling in his veins. He shook the wounded hand and the unearthly blood was freed upon the air, sizzling and bubbling, and then held the hand high in the air for the blood to drip down. 
  
  
  
  \new
  \index{\xzaishann}%
  With her supernaturally clear sight, \Criseis{} could see how her lord's spilled blood ate like acid through the weave of the Realms. 
  \tho{%
    The blood of a \secondgendragon. 
    Replete with the power of the \xzaishanns. 
    The power of \Chaos. 
    The power that makes and unmakes the worlds. 
    The blood that is life and death. 
    \emph{All} life. 
    And, perhaps, all death.}
  
  \QuessanthIshnaruchaefir{} then spoke spellwords in the potent and ancient \maybehr{True Draconic}{True \Draconic} tongue. Those arcane words of power never translated well into mortal tongues, but the meaning of the incantation was approximately this: 
  
  \index{\Rystessakhin}
  \index{\AeocrithRystessakhin}
  \index{\ophidian}
  \index{glaive}
  \draconicspell{%
    Come to me, o' glaive. \\
    By the blood of the \ophidian{} people, \\
    by the blood of the \xzaishanns, \\
    by the blood that is life I give thee form.\\
    I summon thee! \AeocrithRystessakhin! Arise!}
  
  As his speech veered into the True \Draconic{} tongue, his voice changed, becoming unnaturally deep; growling, thundering, as if resounding from the throat of a far vaster creature. 
  As if emerging from his true form. 
  
  The air reverberated with the sorcerous power of those words. 
  And the seething, hissing blood that spilled from his hand began to coalesce and solidify. 
  % Rian squinted again, watching as it took the shape of... \tho{of what?} Some oblong object. \tho{A branch? A rod? An... axe?} Nay, not an axe, but clearly a weapon. 
  Gradually it took the shape of an oblong object. 
  First it resembled a rough branch. 
  Then a rod. 
  Then an axe. 
  Then finally a scythe-like polearm, half again as long as \Ishnaruchaefir{} in his \scathaese{} form was tall, and crooked, and with not one edge but several wicked blades, pointing both forward and back, making the head almost half as wide from end to end as the shaft was long. 
  A scimitar-like spike jutted upward from the top, making the weapon look like a monstrous cross between a halberd and a scythe; another such blade protruded from the back end. 
  
  The flowing blood dried and hardened, taking on the texture of metal. 
  The shaft was dull black, but the blades appeared to shimmer between several \colours, and it was as if each \colour embodied a shade of emotion. 
  Glossy onyx black was a deep sadness, abandonment, solitude... emptiness. 
  Blood red was pain, agony like nails through the heart. 
  And a fierce, fiery yellow that was a great, superhuman passion, terrible and cruel in its searing intensity. 
  A passion for what might be hunger, or hate, or love. 
  
  Sinister power flowed in waves from the scythe. 
  It was the glaive \AeocrithRystessakhin, \ps{\Ishnaruchaefir}{} dreadful weapon. 
  \tho{%
    And the mark of his crime. 
    The source of much of his strength... and perhaps, much of his weakness.
    
    \Rystessakhin. 
    Summoned and given form by his own blood.
    Symbolic of the cruel and bloody sacrifice it once took to forge that fell weapon. 
    Although it was not \emph{your} blood that was spilled that day, was it, Master \Quessanth? 
    
    But then, in an indirect way, it was.}
  And \Criseis{} shuddered to recall that terrible day, when \QuessanthIshnaruchaefir{} had made his dreadful, irrevocable choice. 
  
  The black sorcerer lifted the glaive to his lips, so close he could have kissed it, and whispered soft words, in a \scathaese{} voice again. \ta{I have need of you again, my love. There is work for us...}
  
  \tho{Oh, no. Now he is talking to it.} 
  \Criseis{} fidgeted. 
  In her experience, whenever her master started talking to the glaive, it meant something nasty was about to happen. 
  % Then again, he rarely summoned the artifact in the first place unless he meant for something nasty to happen. 
  
  
  
  \new
  At first, the hapless thief could but stare in awe and terror at the astounding conjuration. But while the black one stood mumbling to himself for a long moment, the Rian's wits gradually returned. 
  
  \tho{A sorcerer! He is a sorcerer!} 
  
  It all made sense now, including the way the \scathae{} had been able to wander all over the city with no one noticing them. He had heard that wizards could do that. 
  \tho{An illusion. Or whatever it's called.} 
  Rian felt proud of himself for being, it would seem, the only one in the city to see through the illusion. 
  \tho{Not so all-powerful, now, are you?} he almost wanted to say. 
  
  But a quick glance at the sinister weapon made him eat those words again, even before he could speak them. 
  
  He peered closer. 
  
  \tho{Is that something... dripping from the blade? 
    Is it blood? Or... tears?} 
  The first guess was morbid, but the second was eerier still. Again Rian had to shudder. 
\end{prose}










\subsection{History}





\subsubsection{Early history}
\target{Iscrafel's early history}

In the olden days of the \draconian{} empire, \Nexagglachel was the leader of many \dragons. 

\Quessanth fought for his brother as a champion.
He was a martial artist and warrior-mage who hungered for glory and recognition. 
\Quessanth was nearly the reigning martial arts champion of the world. Only \hs{Zanshir} and \hs{Theraster} he considered his superiors. Zanshir was killed before Quessanth ever bested her. Theraster he only managed to defeat a few times in training matches. 

Young \Quessanth was a smith and weaponmaker as a warrior. 
In his free time, for fun he also studied the art of cooking and confectionery. 

In the beginning, \Ishnaruchaefir{} was top motivated and strove after knowledge, power, skill and mastery. 
This is one of the reasons why he became so \trope{Badass}{badass}. 

Later his priorities changed. 







\subsubsection{Death and rebirth as a \dragon}
\target{Ishnaruchaefir's death and rebirth}
Early on, \Ishnaruchaefir{} went through a \hr{Shaeeroth ritual}{ritual of death and rebirth}, in order to unlock his true power and turn him into a {\dragon}.

Compare to how \hr{Rissit's death and rebirth}{\Secherdamon later did the same}.





\subsubsection{Wanted to destroy \humans}
\target{Ishnaruchaefir wanted to destroy Humans}
\Ishnaruchaefir hated the \humans for their abhorrent religious practices and the \hr{Human trees}{\human trees} that they created. 
He argued that \dragonkind ought to destroy the vile creatures. 

Other \dragons, including \Rystessakhin, argued that the \humans had a right to live, even if their culture was barbaric and revolting. 
Yet others hated the \humans but did not think it was worth the effort to eradicate them. 





\subsubsection{Responsible for \ps{\Nexagglachel} fall}
\Ishnaruchaefir{}, despite his best efforts, was responsible for his brother's death and ignominy, and for gifting the \resphain{} with more power than they could otherwise have hoped for. 
\HriistD{} hated him for it.
He never quite forgave himself for it, either. 
This shaped \Ishnaruchaefir{} into a reluctant, desperate saviour-figure.





\subsubsection{Inherits stewardship of \Miith}
\target{Ishnaruchaefir's stewardship}
\index{stewardship}
In their last conversation together, \Nexagglachel{} \hr{Ishnaruchaefir gains stewardship}{entrusted to \Ishnaruchaefir{} the \quo{stewardship} of \Miith}. 
It would now be up to \Ishnaruchaefir{} to lead the \dragons{} and defend the planet. 

This heavy burden of guardianship shaped \ps{\Ishnaruchaefir} life. 
He refused to let himself fail his promise to \Nexagglachel, the brother he adored and admired. 
He promised to \quo{do what must be done}, no matter how hard, how painful, how much strength it would take. 
This oath he intended to keep, with fanatical devotion. 
He would betray \Rystessakhin, \Secherdamon, \Nzessuacrith{} and the entire planet before he betrayed \ps{\Nexagglachel} trust. 
\Nexagglachel{} made \emph{him} responsible, and it was \emph{his} burden to bear, no one else's. 
He would not let \Rystessakhin{} or \Secherdamon{} or anyone else shoulder the burden in his stead. 
That would be shirking his responsibility, which would be betrayal. 
His responsibility, his inheritance from \Nexagglachel, becomes his life. 





\subsubsection{Forced to mature}
\Ishnaruchaefir had to mature quickly now that \Nexagglachel was gone. 
Before, \Ishnaruchaefir had been the irresponsible, talented loner and dreamer (where \Nexagglachel was the wise, responsible elder and \Secherdamon was the hard-working, envious, ambitious, inferiority-complex-suffering younger brother). 

Suddenly \Ishnaruchaefir was the oldest and mightiest \dragon alive and had to lead his entire race against their dreaded ancestral foe. 
He had to grow with the task, and at record speed. 






\subsubsection{Leading the \dragons{} to war}
\Ishnaruchaefir{} fought against the \resphan{} menace, \hr{Ishnaruchaefir leads the Dragons to war}{leading his \draconic{} brethren and their massive armies into battle}.







\subsubsection{The \SecondShrouding}
\Ishnaruchaefir{} \hr{Ishnaruchaefir and the Shrouding}{was one of the masterminds behind the \SecondShrouding}. 

He saw \hr{Ishnaruchaefir chooses eternal war}{three possible futures}: 
Death, Chaos or eternal war. 
He chose eternal war, hoping to contain it and limit its damage. 







\subsubsection{Slaying his love}
\Triestessakhin, \ps{\Ishnaruchaefir} beloved, was against the \SecondShrouding{}. 
They argue and fight over it. 
Eventually \hr{Ishnaruchaefir slays his beloved}{\Ishnaruchaefir{} kills \Triestessakhin}. 

Mourning what he had done, \Ishnaruchaefir{} captured the soul of \Triestessakhin{} and bound it inside \hr{Ishnaruchaefir's glaive}{his glaive}, which he now always carries on him. 
He went on to use the glaive as his \hs{weaving artifact} as he and his fellows wove the \hr{Shrouding}{\SecondShrouding}. 

He remains to this day one of the pivotal \vertices{} keeping the Shroud in place.

But slaying his beloved was traumatic and became a turning point in his life. It distanced him further from Chaos, the \firstgendragons{} and his fellow \dragons.





\subsubsection{Angsting}
He angsted over having killed his love. 

\lyricsbs{Dreamsfear}{As Darkness Falls}{
  I am the master of my destiny.\\
  But this man in the mirror I see \\
  cannot be me.\\
  Seems just like yesterday:\\
  My life lay before me.\\
  But now I am reaching the end.\\
  What have I done?
  
  Once upon a time I had a dream, long ago,\\
  that everything I touched belonged to me and turned to gold.\\
  Now I know that dream can never be, it's far too late.\\
  Now I know that dream can never be, it's far too late.
  
  I just want to get the chance to be the best I can be. \\
  Looking to the future I can't see what's in store for me.\\
  Looking to the future I am blind, can't see the light.\\
  Looking to the darkness I am blind, can't see the light.
  
  Can't see the light. \\
  As darkness falls.
}

He thought back about \Triestessakhin{} and began to doubt. 
Was she right after all? 

\lyricsbs{Dreamsfear}{Burning Bridges}{
  Some things I took for granted, \\
  within which beauty lies.\\
  But I admit I never \\
  ever saw it in your eyes.\\
  Now, maybe I don't have the gift\\
  to see the things that you see,\\
  like when you said that you could see\\
  something beautiful in me.
}

But he remained stoic. 
He refused to repent or ask forgiveness. 

\citebandsong{BlindGuardian:NIME}{Blind Guardian}{The Curse of F\"eanor}{
  I will always remember their cries.\\
  Like a shadow which covers the light.\\
  I will always remember the time.\\
  But it's past, I cannot turn back the time.\\
  (I) don't look back.\\
  There's still smoke near the shore.\\
  But I arrived.\\
  Revenge be mine.

  I will always remember their cries.\\
  Like a shadow they'll cover my life.\\
  But I'll also remember mine.\\
  (And) after all I'm still alive.
}





\subsubsection{Hated as the Destroyer and Betrayer}
\target{Destroyer}
Since the \SecondShrouding, \Ishnaruchaefir{} was reviled by mortals and \resphain{} alike as the Destroyer: 
The wicked one who masterminded the \SecondShrouding{} which tore apart the world and laid waste to the Realms, killing millions, including many of his own race. 
He is the single greatest killer in history since the \firstbanewar. 
Everyone fears him for that. 

The title \hr{Ishnaruchaefir gains title of Destroyer}{actually began during the \secondbanewar}.

There are even \hr{Destroyer myth}{myths about him}. 

His fellow \dragons{} do not blame him for killing mortals, for few \dragons{} place much stock in mortal lives. 
But they blame him for killing and destroying \Rystessakhin{} and innumerable others of their kind. 
Destroying other \dragons{} is a serious crime among their race. 
(Temporarily killing them is OK.)








\subsubsection{Disappearance into the void}
\target{Ishnaruchaefir goes into the void}
After slaying \Triestessakhin, \Ishnaruchaefir{} is wracked with grief. 
He withdraws into himself. 

Seeking new answers, perhaps as an escape from himself or as a help in moving on from his grief, he begins to explore the cosmos beyond the known Realms. 

\lyricsbs{Emperor}{The Eruption}{
  ... and after years in dark tunnels\\
  he came to silence\\
  there was nothing...
  
  he realised that the cheering cries of worship\\
  were but echoes of his harsh outspoken word\\
  reflecting back at him from cold and naked walls\\
  in hollow circles fled illusions of wisdom he had heard
  
  \quo{From nothing came all I ever knew}
  
  and he beheld the ruins\\
  of an empire torn apart\\
  yet, no grief nor rage did bind him\\
  just silent and bewildered\\
  by the emptiness\\
  he stumbled off his throne
  
  suddenly, the walls around him cracked wide open\\
  and an endless void appeared in flickering, grey light\\
  \quo{What force, but silence, has deprived me of my coil?\\
  No trail to guide me. No point of reference in sight.}
  
  \quo{By nothing, resurrection will be pure.}
  
  and he beheld the ruins\\
  of an empire torn apart\\
  wiping dust off his shoulders\\
  just silent now in this emptiness\\
  leaving all behind
  
  step by step, past all past\\
  slowly he approached the surface\\
  nothing left to sacrifice\\
  the mirrors mocked him on the way
}

He flees into the outer universe, unwilling to face his own Aenigma. 

\citebandsong{Emperor:Prometheus}{Emperor}{He Who Sought the Fire}{
  And again he came to cherish,\\
  the comfort of mysteries.\\
  Inspiring and far away,\\
  timeless in the moment,\\
  they painted immortality.
}

He has a trauma which he must overcome before he can come to Gnosis of himself. 

\citebandsong{Emperor:Prometheus}{Emperor}{He Who Sought the Fire}{
  Forever drawn towards the centre\\
  of this ensorcelling flame.\\
  Tet, still in fear of the sacrifice\\
  it would take to know its name.
}

\target{Glaive must be destroyed}
The trauma has to do with \Rystessakhin. 
He must first free himself of his guilt and unhealthy attachment to her. 
He must make peace with her. 
In order to do that, \hr{Glaive is destroyed}{the glaive must be destroyed}. 

\citebandsong{Emperor:Prometheus}{Emperor}{He Who Sought the Fire}{
  \quo{Do not despair}, said the mystery.\\
  \quo{You will always have a friend in me.\\
    until the day you break my code.\\
    then I will be gone and you are free...\\
    ...to manifest another.}
}

\lyricsbs{Arcane Wisdom}{
  Seeker of All Wisdom and Knowledge (A Journey Beyond the Star-Realm)
}{
  Haunting memories of past deeds \\
  ablaze the mind, bind the senses. \\
  Yet the wisdom seeker flows on. \\
  High above mankind he soars.
}

\lyricsbs{Arcane Wisdom}{Of Skyfire Burning Deep}{
  My dark awakenings, abandoning my own self \\
  in search of other worlds. \\
  Not of sorrow and demise, yet of a star wintry cold; \\
  leaving me as dead to this outer parallel.
}

\lyricslimbonicart{Interstellar Overdrive}{
  Dark cosmic void, a neverending universe.\\
  The final frontier, so dark and mysterious.\\
  I saw the dying sun as I waited for the night.\\
  Now cold silence reigns, magic darkness domains.
  
  Unfold thy secrecy.
}

At some point, \Ishnaruchaefir{} disappears. 
In his exploration of the cosmos, he is sucked away and not heard of for a thousand years or more. 

\lyricslimbonicart{Dynasty of Death}{
  Through dark tunnels in levitation, \\
  black cosmic space in manifestation.\\
  I escape the earthly pandemonium\\
  into a vast nocturnal sanctum.\\
  A nemesis for all evil I confess.\\
  My soul bleeds by all the forces I possess\\
  A benediction of unholy wrath and sorrow.\\
  My heart is buried in dark catacombs of horror.
  
  Sucked into that hole, that deep black hole.\\
  Not for a thousand years will I manage to crawl\\
  out of this darkness,\\
  this supernatural darkness.\\
  All the vibes are insane\\
  in the wilderness of pain.
}

He feels like he is dead.  
Maybe he \emph{is} dead in a sense. 
(\KhothSell, remember.) 

\lyricsbs{Hate Eternal}{Path to the Eternal Gods}{
  On this journey into death, \\
  I am beside myself in tremendous bliss,\\
  For this allegiance has been made. \\
  Shall my sins be absolved, \\
  washed away by the blood of the sacred lambs? \\ 
  Yet I am not amongst the flock.
  
  The remnants of my life now become my vast illusions,\\
  whilst exiled from your grace. 
  
  With fortitude, with courage, 
  I face all my fears, \\
  on my path to the eternal gods. \\
  With my wrath, with my disdain, 
  I face all my fears, \\
  on my path to the eternal gods.\\
  As I cross over into my final place, 
  I face all my fears, \\
  on my path to the eternal gods.\\
  As I bear trials in my final resting place, 
  I face all my fears, 
  on my path to the eternal gods.\\
}

But at last he returns: 
Changed. 
Mightier. 
Wiser. 
Grimmer. 

He has communed with dark cosmic gods and found great power. 

\lyricsdimmuborgir{Dreamside Dominions}{
  Losing control in seductive madness.\\
  Spiritual revelations, apocalyptic hypnosis.\\
  Dead \colours appear within unshallow graves.\\
  Alone in awe I face abhorrence below.
}





\subsubsection{Takes the name \Tzeorossh}
\target{Ishnaruchaefir takes the name Tzeorossh}
In his self-imposed exile, \Ishnaruchaefir used the Gnosis he had gained from his sojourn into the void and took a new name:
\quo{\Tzeorossh}, meaning \quo{Exile}.

The title of \quo{\hs{Exile}} was originally given to him by \Secherdamon in hatred.
Eventually, as \Ishnaruchaefir secluded himself in the Mirage Asylum, he came to accept his status as an exile, so he accepted the name and even formally and metaphysically adopted it.





\subsubsection{Why do the cosmic gods live?}
Once, \Ishnaruchaefir asked a great cosmic god:
\ta{Why do you live? What is your purpose? What is your goal, your motivation?}

The god answered: \ta{We \emph{know}.}

\Ishnaruchaefir pondered that ever since. 
He was sure there was some great insight hidden in that answer. 
Perhaps because the cosmic gods \emph{knew}, they were content and happy and needed never strive for anything ever again. 
Perhaps the gods \emph{knew} the future and thus had no need of motivations, since they knew that everything they would ever do was already determined. 
Or maybe the truth was something else entirely.

\Ishnaruchaefir pondered that question till the end of his days. 
He never solved it.
Not at the time of the \thirdbanewar and not after it. 





\subsubsection{Since then}
In the millennia since then, \Ishnaruchaefir{} has opposed the \banes{} and \resphain{}, and has sometimes worked with his fellow \dragons, but he has always been a loner who went his own way and answered to no one, showing up whenever he chose to do battle and then disappear again. 

He has worked as a wandering warrior for an aeon, showing up when everyone least expects it to kick the butts of the Cabal or some other organization opposing him, his Bloodline or his moral principles. 

One of his reasons for waging war against his fellow \dragons{} is his belief that it actually strengthens them: A strong enemy breeds unity. If he can keep the \dragons{} organized in two warring factions, then they are much better off than if they were all just squabbling amongst themselves. Or so \Ishnaruchaefir{} reasons.

He has not always been fighting on the Sentinel side. At times he has directly opposed them, saving \resphain{} from \draconic{} attacks and waged war against \dragons. For this reason, \HriistD{} does not trust him, and the Cabal don't know what to think. 

He actually liked the idea of the \resphain{} being imbued with \draconic{} blood, because he belied they would inherit \ps{\Nexagglachel}{} hatred of the \banes{}, and they would plot against their creators. This was one reason why he let it happen. But it still somewhat pains him. He was correct, tho: Some of the \satharioth{} went on to form \Kezerad, \Mystraacht{} and \Baelzerach. 

%His fortress of \Nithdornazsh{} has fallen into disrepair. In fact, it has withered and died. But its soul lingers and can be reborn. It just needs to feed. (Resurrecting a living fortress has very rarely been done, so no one suspects that's what \Ishna{} is up to.)





\subsubsection{Maintaining the Shroud}
\target{Ishnaruchaefir maintains the Shroud}
\Ishnaruchaefir would like to be free of \Rystessakhin and the glaive, but he needed them in order to maintain some power over the Shroud.
Without him the Shroud might grow unstable, and he would certainly lose influence.

More importantly, the \hs{Mirage Asylum} would collapse, for it was maintained by his magic and could not survive without him.
And he could not live in peace or get any work done (scientifically and politically) without a sanctuary.

So he must keep \Rystessakhin enslaved as a bound wraith, however much it pained him and her.
After all, he still had his stewardship.
He had a responsibility for the future of the world, so he could not afford to lose his personal power nor his political and metaphysical influence on the world.

\Ishnaruchaefir must continuously work to maintain the Shroud. 
This forced him to \hr{Ishnaruchaefir's Nadir}{go into a periodic Nadir}. 





\subsubsection{Realized the truth about \Sethicus and \Tiamat}
Originally \Iscrafel protested against the hivemind plan that \Tiamat and later \Secherdamon had because \Iscrafel felt that the hivemind plan would only make them similar to the \banes. 

\target{Iscrafel learns that Sethicus is in the Threnody}
In the end (during the final war) \Iscrafel realized that \hr{Sethicus in the Threnody}{\Sethicus had become part of the Threnody}. 
This made him despair and lose faith in his ideals of enlightenment. 
He realized that \Sethicus, the most enlightened hero ever, did not achieve fulfillment or even oblivion, but instead eternal suffering and horror as a maddened and powerless ghost. 

\target{Iscrafel accepts Tiamat's hivemind}
\Iscrafel realized that the \hr{Tiamat's hivemind plan}{\Tiamat's hivemind plan} was the only option.  
Life as it was now was meaningless and worthless. 
The \quo{third option} of enlightenment that \Sethicus had sought turned out to be an illusion. 
\Secherdamon's plan was now the last option that offered any vestige of hope. 
\Iscrafel ultimately agreed that \Secherdamon's plan was better than the alternative, and so set about helping \Secherdamon. 











\subsection{Names, titles and reputation}
\target{Ishnaruchaefir's names}
\Ishnaruchaefir's full name was \Quessanth \Melechet \Nierzshah \Tzeorossh \Ishnaruchaefir.

\begin{itemize}
  \item 
    \quo{\Nierzshah} means \quo{\hs{Destroyer}}.
    He \hr{Ishnaruchaefir takes the name Nierzshah}{took this name during the \secondbanewar}. 
  \item 
    \quo{\Tzeorossh} means \quo{\hs{Exile}}.
    
    He \hr{Ishnaruchaefir takes the name Tzeorossh}{took this name after the \Shrouding}. 
    The title of \quo{\hs{Exile}} was originally given to him by \Secherdamon in hatred.
    
    When he, once in a while, introduced himself with his full name, he would make a sarcastic grimace when he said \quo{\Tzeorossh}.
\end{itemize}

\target{Ishnaruchaefir's titles}
\Ishnaruchaefir{} had a number of titles and kennings by which he was sometimes known in various stories and myths, most notably the epic poem \emph{\hr{Wanderers in Darkness}{\WanderersInDarkness}}. 
These include:

\begin{itemize}
  \item The \hs{Destroyer}.
  \item The \hs{Exile}.
  \item Wanderer in Darkness. 
\end{itemize}






\subsubsection{Mythical status}
\target{Myths about Ishnaruchaefir}
\Ishnaruchaefir{} was infamous among the Vaimons and other learned people. 
He was referred to as an \quo{Immortal} and an evil \chaos{} sorcerer. 
But most people don't know that he is a \dragon. 

Remember to have many references to him as a mythical, mystical figure. 

Among other things, he was known in a twisted version as the evil god \hr{Isphet}{\Isphet} in \hs{Iquinian mythology}.

\Ishnaruchaefir was \hr{Ishnaruchaefir in Ortaican mythology}{not mentioned at all} in \hr{Ortaican mythology}{\Ortaican mythology}.





\subsubsection{Reputation}
\target{Ishnaruchaefir's reputation}
\Ishnaruchaefir{} has a reputation for being dangerous, unreliable and a liar. 

He has a worse reputation than \hr{Secherdamon's reputation}{\Secherdamon{} does}. 










\subsection{Personality}





\subsubsection{Burdens}
\target{Ishnaruchaefir's compassion}
\Ishnaruchaefir carried a number of burdens:

\begin{description}
  \item[\ps{\Nexagglachel} stewardship:]
    \Ishnaruchaefir inherited from \Nexagglachel a \hr{Ishnaruchaefir's stewardship}{stewardship}, a responsibility to defend \Miith against invaders. 
  
  \item[\ps{\Rystessakhin} compassion:]
    \Rystessakhin was \hr{Rystessakhin's personality}{very kind and compassionate}
    After \hr{Ishnaruchaefir kills Rystessakhin}{killing her}, he felt that he had robbed the world of a great benefactor. 
    He felt some obligation to carry on her will now that she could not.
    So occasionally he would be prone to fits of compassion for lesser beings when he remembered how his beloved would have felt towards them.
    This was what happened when he \hr{Ishnaruchaefir saves Criseis}{saved \Criseis}, and again when he \hr{Ishnaruchaefir saves Rian and Neina}{saved Rian and Neina}.
  
  \item[Destroyer:]
    Though he tried to repress and hide it, \Ishnaruchaefir had some measure of guilt over \quo{\hr{Ishnaruchaefir destroys the world}{destroying the world}}. 
    Especially because he knew \Rystessakhin would not have wanted it.
  
  \item[Maintainer of the Shroud:]
    \Ishnaruchaefir was \hr{Ishnaruchaefir and the Shroud}{partially responsible for maintaining the Shroud}. 
    He possessed a \hs{weaving artifact} in his \hs{glaive}. 

\end{description}






\subsubsection{Code of \honour}
\target{Ishnaruchaefir's code of honour}
\Ishnaruchaefir{} has a certain code of \honour and is often willing to \honour truces and treat his enemies with a certain respect. 
But if prodded, he is prone to flying into a genocidal rage and commit horrible atrocities. 

The best example of this is the story of the deaths of \hr{Criseis's siblings}{\ps{\Criseis} siblings}. 

One should note that \resphan{} lives mean \shout{nothing} to him. 
\Human{} lives neither. 





\subsubsection{Contrast to Ramiel}
In a sense, \Ishnaruchaefir{} stands in contrast to Ramiel: 
\Ishnaruchaefir{} was born of Chaos but seeks to keep the Chaos within him in check. 
Ramiel, on the other hand, is born of the cold power of \Erebos, but actively embraces Chaos.

At times, they meet halfways and come to a sort of understanding as respecting enemies.





\subsubsection{Destroyer}
\Ishnaruchaefir eventually learned to enjoy causing destruction. 
It was something he was genuinely good at.
 
\Ishnaruchaefir was not his brothers. 
He tried to create and preserve, but with little success. 

It was in destroying that he found his true talent, his calling. 
It felt liberating, cleansing, cathartic. 
Long ago he learned to accept this and even enjoy it. 
He embraced his natural role as the Destroyer. 





\subsubsection{Fake weaknesses}
\target{Ishnaruchaefir's fake weakness}
Ever since the \secondbanewar, \Ishnaruchaefir{} pretended to be at once stronger and weaker than he really was. 
He was universally feared, which was good. 
But he knew it would also come in handy if people would underestimate him. 

So he cultivated an image as a \trope{XanatosGambit}{Xanatos Gambiteer} supreme, a schemer with the cunning and ruthlessness to utterly destroy anyone who tried to match wits with him. 

But at the same time, he downplayed his physical prowess. 
He was one of the mightiest \dragons{} ever to live, but he faked being weaker. 
Among other things, he planted fake \quo{Achilles' heels} in \emph{\hr{Wanderers in Darkness}{\WanderersInDarkness}}. 
And when he, once in a rare while, entered into physical combat, he would act weak. 
He fled from several engagements he could have won. 

Once or twice, he even let himself be killed. 
Especially if the circumstances of the battle in question could be connected to one of his fake Achilles' heels. 

Under circumstances connected to one of his \emph{genuine} weaknesses, however, he tried to the very best of his ability to hide it.
This was made easier by the fact that he, by default, was playing weak, and therefore had reserves of strength he could call upon to offset any imposed genuine weakness. 





\subsubsection{Fear of the darkness within himself}
\Ishnaruchaefir{} has some \quo{deep caves and tunnels} which he fears. 
These exist in the world around his Mirage Asylum, but they also reflect, and are shaped by, the darkness within himself. 
(Remember, the Shroud shapes our perception of the world around us, so a voyage through space can simultaneously be a voyage through one's own mind.) 

At times he needs to venture into them. 

Compare to Chia Black Dragon and her caves in \cite[p.120-150]{StephenMarley:ShadowSisters}. 

\Ishnaruchaefir{} fears the \xsic{} blood within him, and the madness and evil it brings. 
This is one of the reasons why he has done such research into the far cosmos: 
He is trying to escape from that which is inside him.

\lyricslimbonicart{Behind the Darkened Walls of Sleep}{
  Behind the darkened walls of sleep, \\
  as body rest and mind goes deep. \\
  A door opens in my heart. \\
  A dark euphoria.
  
  As twilight falls, the awakening 
  of the creature within me. \\
  We are bound, we are blessed 
  in supernatural darkness. 
  
  Behind the darkened walls of sleep.
}





\subsubsection{Free will}
In his old age, \Iscrafel often pondered the question of free will\dash inspired by \hr{Sethicus and free will}{\Sethicus' theorem}.
\Iscrafel asked himself whether he had had free will back in his youth when he made those fatal decisions that laid waste to the world. 





\subsubsection{Impulsiveness}
\target{Ishnaruchaefir's impulsiveness}
\Ishnaruchaefir{} often acted impulsive, volatile and unpredictable. 
But this was actually a ruse to throw his enemies off his trail, hide his true long-term plans and make his actions harder to predict. 

In truth, \Ishnaruchaefir{} expended all of his impulsiveness back in the \secondbanewar, where, during the \SecondShrouding, he devastated half of \Miith{} in a moment of irrational anger. 
He learned from that and never let his anger control him again.





\subsubsection{Insanity}
\Criseis feared \Iscrafel, for she knew that he was \hr{Vril shokraan}{\emph{vril shokraan}}\dash so wise as to be considered dangerously insane.
But she braves her fear and stands by him, for she believes that she keeps him slightly sane, and that he would be a worse menace to the world without her. 
He was saner back when \Criseis’s siblings were alive. 
Now she is all he has left. 

\Iscrafel is cruel and hateful and known to visit gruesome torment and long-term psychological terror on his enemies (and he counts many as his enemies, including the entire \resphan race). 
He was more cruel (but less calculating) than his brother \Secherdamon. 






\subsubsection{Loner}
\target{Ishnaruchaefir is a loner}
\Ishnaruchaefir{} was something of a loner. 
This was always true. 
Since he was young he always went his own ways and sought his own counsel. 
That is why he became a great sorcerer, philosopher and fighter, but not nearly the ruler, politician or diplomat that his brothers were. 

His solitary demeanour was one of the reasons why he decided to turn on \Triestessakhin{} and kill her. 
He was certain his own plan was best. 
He trusted himself more than he did her. 

Since then, he isolated himself more and became even more of a loner. 

\target{Ishnaruchaefir never apologizes}
\Ishnaruchaefir{} \emph{never} apologizes to anyone (except perhaps \Criseis, and then only indirectly). 
He will not ask anyone for forgiveness or love.
He gave that up \emph{long} ago. 

\target{Ishnaruchaefir nerd}
\Ishnaruchaefir{} also does not have great social skills. 
He is cunning and can be a great manipulator, but mostly from afar. 
He is a bit of a nerd. 
So he often leaves it to \Criseis{} to handle things involving social interaction. 





\subsubsection{Pessimism}
In his old age, \Iscrafel was a great pessimist. 
He believed that most lives were not worth living. 
But a few lives were worth it: 
Those who achieved enlightenment and free will. 
(\Iscrafel conflated enlightenment with free will, although philosophers differed on the question of how closely linked the two were.) 
\Iscrafel believed that the enlightenment and theosis of \Sethicus was such a glorious event that it just \emph{might} justify the suffering-filled lives of millions of \caisith before him. 

\Iscrafel wanted to help the \draconic race achieve enlightenment and freedom. 
That was his motivation. 
He hated life, including his own life, but he had a small hope that he might be able to achieve something that would make all his millennia of suffering worth it. 
He was very afraid of perishing before he had achieved his goal, because that would make his life and suffering\dash and the lives and sufferings of so many others\dash be in vain, for nothing. 
This desperate need drove him and gave him strength. 





\subsubsection{Shroud policy}
\target{Ishnaruchaefir and the Shroud}
Throughout the story, \Ishnaruchaefir{} helps several people break free of the Shroud. 
Some of the good guys get the impression that \Ishnaruchaefir{} opposes the Shroud and fights against it. 
But this is not true. 
\Ishnaruchaefir{} is pro-Shroud and pro-\hr{Charade}{\charade} and always has been (\hr{Ishnaruchaefir and the Shrouding}{he helped forge it}, after all). 
But he wants to help create some heroes that can fight for whatever causes he deems just. 
And often, a hero is better if he sees through the Shroud. 





\subsubsection{Sex life}
\Ishnaruchaefir{} ventures out of his \hs{Mirage Asylum} every now and then to get himself some tail. 
He usually seduces some \sphyle. 
He has few scruples about this and will gladly fuck very young girls or married \sphyles. 
He doesn't commit rape, but he can be \emph{very} persuasive. 

Occasionally he seeks out other \dragons{} for sex. 
He is hated and feared by many, but his legendary status, immense power and personal charisma means that many female \dragons{} go for him anyway. 

Sometimes he changes shape and has sex with a \resvil, \human{} or \nephil{} just for the Hell of it. 

He does not look for serious relationships. 
That did not work out last time, and he doesn't \emph{really} need one (unlike a \human{} or \scatha). 
He is, after all, \hr{Ishnaruchaefir is a loner}{a loner}. 

He does \emph{not} have sex with \hr{Criseis}{\Criseis}. 
She is more like a daughter to him. 





\subsubsection{Spouts inanities}
\target{Ishnaruchaefir's inanities}
To test people, \Ishnaruchaefir{} sometimes spouts lines that look profound at a glance but are really just inane platitudes. It's a test to see if the other guy accepts them as profound wisdom or recognizes them as inane and calls him out on it. 

As he says: 
\ta{Who needs profundity when you have a reputation?}





\subsubsection{Torpor}
\Ishnaruchaefir{} spent much of his time lying around in \hs{torpor} in the Mirage Asylum, pondering some of his many Aenigmata. 





\subsubsection{Understands \NexagglachelsCurse}
\target{Ishnaruchaefir understands the curse}
\Ishnaruchaefir{} \hr{Ishnaruchaefir eats Quelthah}{slew the \sathariah \Quelthah and ate his soul}, including his \hr{Fragments of Nexagglachel}{\Nexagglachel{} fragment}. 
From this, \Ishnaruchaefir{} {inherited a certain intuitive understanding} of \hr{Curse}{\NexagglachelsCurse} and the psychology of the \satharioth. 

This gave him a certain edge against the \satharioth, and against his fellow Sentinels. 
\Secherdamon{} knew much of the \resphain{} and had studied them much, but he did not have the same first hand experience and understanding of their psychological condition (how the Curse actually \emph{feels}) as \Ishnaruchaefir{} had. 









\subsection{Physique}
\Ishnaruchaefir{} was obsidian black with fiery red eyes. 
He had four long, slim, curved horns on his head. 

\Ishnaruchaefir was 25 metres long, with \hr{Dragon size}{standard proportions and weight} for a \dragon of his length. 





\subsubsection{\Human{} form}
If \Ishnaruchaefir{} were a human (or if he were to take \human{} form), he would look much like the Count of Monte Cristo as portrayed in the anime \cite{Anime:Gankutsuou}. 





\subsubsection{\Scathaese{} form}
In his \scathaese{} form his scales are onyx black. 
He wears black \armour, lined with edges of silver and blood red, and a black cloak. 

\lyricstitle{Draft excerpt from the chapter \quo{What Slithers Beneath}}{
  \tho{That is a strange-looking \dax.} 
  His scales were pure black, glossy like some kind of precious stone. 
  Rian had never seen a black \scatha{} before. 
  Blue, red, occasionally green, but never black. 
  \tho{How could I not notice this straight away?} 
  
  As he studied the pair closer, new details seemed to appear out of nowhere. The \dax{} was very tall, seven feet at least. His garb was all black with occasional lines of silver and blood red. Dressed in a long cape over what appeared to be metal armour, he looked the part of a warrior, but he bore no weapon that Rian could see. 
  The two ridges on his head were strangely elongated, almost like a pair of slim, backwards-curling horns. 
  
  %As Rian stood and stared, 
  Then, slowly, calmly, not breaking his stride, the black-scaled \dax{} turned his head slightly and, with a slight smile on his lips, fixed Rian with his gaze. It was only a glimpse out of the corner of his right eye, 
  but it so startled the young thief
  %but Rian was so startled 
  that he immediately, instinctively ducked out of sight and back into the crowd. 
  
  Rian silently cursed himself. 
  \tho{Dumbass! Dumbass! Dumbass!} Acting guilty was the worst thing a thief could do, anyone knew that. When a passer-by noticed you, you acted like nothing was out of place. The best way to invite suspicion was by \emph{being suspicious}. 
  
  %But the look in that eye had been more than the boy could take in. It spoke volumes about... something... abnormal. Inhuman. Larger than life. Unnatural. Wisdom, terrible power... and somehow, splendour. 
  %and awe-inspiring majesty. 
  %And the eyes themselves glinted red, as if a mystic fire burned behind them.  \tho{Who is he? \emph{What} is he?} 
  But that stare had been more than the boy could take in. 
  The eye glinted yellow. 
  Not just the a regular yellow eye\dash uncommon but not unheard of. 
  No, this eye shone. It \emph{burned}. 
  As if a mystic fire burned behind it. 
  
  And the \emph{look} in that eye! It had spoken volumes of cryptic meaning. Of something... abnormal. Inhuman. Larger than life. 
  
  Unnatural. 
  
  Wisdom, terrible power... and somehow, splendour. 
  
  \tho{Who is he? \emph{What} is he?} 
  
  \ldots 
  
  The mysterious pair became yet more mysterious by the moment. 
  The dark one's tail was strangely snaky, whipping back and forth like that of a cat. Certainly nothing like the mostly rigid tail of any normal \scatha. His snout was slender in the middle but widened slightly at the end, unlike the blunted triangle forms of ordinary \scathaese{} snouts. The ridges above his eyes were unnaturally elongated, tapering backwards almost like horns. And that glow in his eyes. Baleful, and yet fascinating. Rian half-imagined the black-scaled one to be some mighty \dragon{} or \daemon{} out of legend. 
  
  \ldots
  
  From his hiding place, not comprehending a word, Rian saw the tall \dax{} lift his nose to smell the air. 
  His tongue darted out. 
  And it was no \scathaese{} tongue! 
  A normal \ps{\scatha} tongue looked like a \ps{\human}. 
  This one's tongue was long and slim and forked, and he tasted the air with it as would a snake!
  
  The black one straightened and flexed the finger of his left hand. 
  Rian squinted, then gaped. 
  Those were not \scathaese{} fingers, either! Each of them sported a long claw! 
  And when he looked closer, the black one's smile revealed not \scathaese{} teeth (which were no sharper than \psp{\humans}), but fangs.
  Again, like a snake! 
}









\subsection{Politics}





\subsubsection{\Baelzerach}
\target{Ishnaruchaefir and Baelzerach}
\Ishnaruchaefir{} has a tribe of \Baelzerach{} \resphain{} allied with him who will help him if he needs it. 
Their chieftain is \hr{Najarod}{\Najarod}. 





\subsubsection{Cosmic gods}
\target{Ishnaruchaefir and cosmic gods}
\ps{\Ishnaruchaefir} power does not all stem from the \xss{}. He has also had dealings with the \hs{cosmic gods}. This is one of the reasons why he is so badass. But his dealings with these alien, incomprehensible powers have also driven him somewhat mad. He has seen glimpses of the truth beyond Chaos and Darkness. 

\target{Ishnaruchaefir and Zaz}
Among other cosmic gods, \Ishnaruchaefir had some dealings with the enigmatic pair \hr{Zaz}{\Zaz and \Urzaz}. 

There was a time when he \hr{Zaz denies Ishnaruchaefir}{appealed to them for aid and was violently denied}. 

Since then, he repaired his relations with them and learned how to better interact with them. 
They became \quo{allies} of his, and he could command much of their power.

\lyricsbalsagoth{Unfettering the Hoary Sentinels of Karnak}{
  What sublime power awaits the aspirant, the querent who dares seek answers in those shadowed places where men of lesser fortitude fear to gaze?
  
  The path to elucidation is seldom devoid of thorns, the road to knowledge rarely free of perils!
}

\lyricslimbonicart{The Dark Paranormal Calling}{
  I cross dimensions unseen \\
  to ride on the axis of dreams.\\
  As I drift on through the dark corridors of post mortem,\\
  the only light in the darkness\\
  is the flame that burns in my soul.
  
  I intend to follow \\
  the eternal flame of my secrecy.\\
  Emancipate the mortal world \\
  as minds redeem from the mortuary.\\
  The only life in the darkness\\
  is the force that yearns in my soul.
}





\subsubsection{\Criseis}
\target{Ishnaruchaefir and Criseis}
\Ishnaruchaefir{} holds \Criseis{} tight and has feelings for her almost like a daughter. 
She serves as a replacement for \Nzessuacrith, with whom he fell out. 

He frequently brings her along on his journeys, for a number of reasons: 
\begin{enumerate}
  \item 
    \hr{Ishnaruchaefir's senses}{Unlike him}, she \hr{Criseis' senses}{has sharp senses}. She can scout for him. 
  \item 
    She asks to come along. 
    She \hr{Criseis contains Ishnaruchaefir}{hopes to contain his destructive behaviour}. 
\end{enumerate}






\subsubsection{Family}
\target{Ishnaruchaefir's family}
\Ishnaruchaefir was the second son of \TyarithXserasshana.
His father was \hr{Iurzmacul}{\Iurzmacul}. 

\Ishnaruchaefir{} loved \Triestessakhin. 
They had one child together, \Nzessuacrith. 

He also had three sons (with other mothers). 
They were \hr{Ishnaruchaefir's sons die}{all killed in the \secondbanewar}. 
After this, he poured all of his love on \Nzessuacrith, and they were closely knit. 
This made it extra hard for both of them when they split (after \Ishnaruchaefir{} killed \Triestessakhin). 

\Ishnaruchaefir{} also had three grandchildren, the sons and daughters of his fallen sons. 
They were all black \dragons, and young enough to have only one name each. 
They were: 
\begin{itemize}
  \item \hr{Rathyon}{\Rathyon}. 
  \item \hr{Tentocoth}{\Tentocoth}. 
  \item \hr{Thiencaste}{\Thiencaste}. 
\end{itemize}





\subsubsection{Ineffectual}
In spite of his fearsome reputation, \Ishnaruchaefir was really something of an \trope{IneffectualLoner}{Ineffectual Loner}. 
Between the Shrouding and the \thirdbanewar he had virtually no impact on the \feud. 
He struck very hard when he did strike, once in a very rare while.
This made him seem like a great menace. 

In practice, \Secherdamon (with his subtler methods) achieved far more. 





\subsubsection{\NerrhanKoss}
\Ishnaruchaefir{} owes some measure of allegiance towards the \xs{} \hr{Nerrhan-Koss}{\NerrhanKoss}, who is sort of his mentor and patron. 
It was \NerrhanKoss{} who \hr{Glaive origin}{gave him his glaive, sort of}. 





\subsubsection{Sentinels}
\target{Ishnaruchaefir and the Sentinels}
After the \SecondShrouding, \Ishnaruchaefir{} was reclusive and seemed to have no contact with the world, hidden away in his Asylum as he was. 

There were persistent rumours that \Ishnaruchaefir{} had been cast out of the Sentinels of \Miith{} for his atrocities, being too evil even for them. 
The stories came from the fact that \Secherdamon, one of the most influential Sentinels, publicly denounced \Ishnaruchaefir. 

But this was not true. 
\Secherdamon{} had no authority to \quo{exclude} \Ishnaruchaefir{} from the Sentinels. 
In secret, \Ishnaruchaefir{} actually maintained more Sentinel contacts and covert influence than most people believed. 





\subsubsection{\Tiamat's hivemind}
\Iscrafel suspected that \hr{Tiamat's hivemind is like the Banes}{the \dragons were doomed to downfall like the \voyagers}. 
Late in his life \hr{Iscrafel accepts Tiamat's hivemind}{he came to accept this} and decided to help \Tiamat and \Secherdamon. 





\subsubsection{\Zaz and \Urzaz}










\subsection{Skills and powers}





\subsubsection{Dull senses}
\target{Ishnaruchaefir's senses}
\Ishnaruchaefir{} has a weakness. 
Because of the heavy burden he bears (\Rystessakhin), the chaos he carries with him and the pain he endures, his senses are dulled. 
They are still sharper than a mortal's, but by \draconic{} standards they are dull. 
This applies whether he physically carries the glaive or not. 
It applies to both his physical and metaphysical senses. 

This is one of the reasons why he so often drags \Criseis{} along with him: 
She \hr{Criseis' senses}{has sharp senses} and can scout for him. 





\subsubsection{Howling}
When \Ishnaruchaefir{} draws deep of his dark, cosmic magic, you can hear a faint howling, coming from infinitely far off, from eternally dark halls beyond the firmament. Like grim monstrosities mindlessly howling\dash blind, cruel, uncaring like the universe itself. Like the piping, dancing Outer Gods at Azathoth's court in \authorbook{\HPLovecraft}{The Dream-Quest of Unknown Kadath}. 

This has to do with the dark \hs{cosmic gods} whom \Ishnaruchaefir{} has studied to master his magic. 





\subsubsection{Languages}
\target{Ishnaruchaefir's languages}
\Ishnaruchaefir{} spoke several languages. 

He spoke an archaic form of Imetric. 
He had had dealings with the Imetrians, since \Sarokash was his kinsman. 

But he never learned \Velcadian. 
Before the \thirdbanewar{} he had not been to \Azmith{} for centuries, so he had never learned it. 
Fortunately, \hr{Criseis languages}{\Criseis{} did speak \Velcadian}, so he could use her as a translator. 





\subsubsection{Nadir}
\target{Ishnaruchaefir's Nadir}
There are periods where \ps{\Ishnaruchaefir} \vertex{} goes into a natural Nadir. 
This has to do with his status as one of the chief architects and \hr{Ishnaruchaefir maintains the Shroud}{maintainers of the Shroud} and the wielder of an important \hs{weaving artifact}. 

\Rystessakhin{} is a central keystone in the Shroud. 
Occasionally the glaive needs to be \quo{recharged}. 
This leaves \Ishnaruchaefir{} weary and weak. 
Also, he cannot wield \Rystessakhin{} in combat.
The glaive is a very powerful weapon, so not having it makes a significant difference. 

The Nadir happens semi-regularly and can be predicted astrologically. 
(Although the cycle varies, so you have to be close in time before you can predict it with any accuracy.)
It happens once every 30 years or so. 
(But do not mention any exact number in the books. I don't want to paint myself into a corner.)
The Nadir lasts about a week. 
He is weakest by the middle of that week. 

According to \WanderersInDarknessEmph, the Nadir occurs \quo{when the \hs{Exile} is engulfed by the briny waters}. 
This is an astrological sign. 
\WanderersInDarknessEmph also reveals that he is at his weakest in the middle of the period, and there are further astrological signs to mark when this happens. 

The Nadir is very hard and traumatic for \Ishnaruchaefir, not only metaphysically but also emotionally. 

Compare to two scenes in \cite{StevenErikson:TolltheHounds}:
The scene where Anomander Rake puts the sword Dragnipur away for a short while, demonstrating what an immense burden it is, and the scene where Rake is weak and vulnerable after having slain Hood and absorbed his wicked-powerful soul into Dragnipur. 

Also compare to Chia Black Dragon's \quo{dying time} in \cite{StephenMarley:SpiritMirror}. 

\target{Nadirs get worse}
The Nadirs are getting worse each time, because the Shroud is \hs{unravelling} and it takes more effort to hold it together. 

\target{Ishnaruchaefir bleeds in Nadir}
When \Ishnaruchaefir is in his Nadir, he gets bleeding wounds all over his body. 
And one can see a myriad long, aethereal tendrils radiating out from him, through which power drains out of him to sustain the Shroud and the glaive. 

They spring open on their own because of the power he has to expend in his weakened state.
He is paying \hr{The cost of magic}{the cost of magic}.
 
Compare him to Anomander Rake in Darujhistan in \cite{StevenErikson:TolltheHounds}. 

\Ishnaruchaefir's Nadir happened at times when the \quo{tides} of the Shroud were low, meaning that the Shroud was weak and permeable. 
At these times, \Ishnaruchaefir had to work hard to keep the Shroud stable. 
This hard spellword was what made him weak. 
He had to open himself up to the world in order to pull its strings, and this openness made him vulnerable. 

If he failed, he risked a nasty backlash against himself and his Mirage Asylum. 
The Asylum was unstable by nature and prone to collapsing or flying off into space if not maintained. 

It was unclear whether the Shroud itself might collapse without \Ishnaruchaefir's support. 
\Ishnaruchaefir himself tended to believe that his work was necessary to keep the Shroud alive.
His critics tended to think his work was of purely local significance. 







\subsubsection{Power}
\target{Ishnaruchaefir's power}
\target{Ishnaruchaefir's rage}
\target{Ishnaruchaefir's inner strength}
\Ishnaruchaefir was one of the mightiest \dragons in the history of \Miith. 
He was a tremendously powerful sorcerer and could call up hordes of monsters/demons to fight for him. 
He was also immensely strong physically and \hr{Ishnaruchaefir's inner strength}{possessed vast reserves of mental strength}. 

The sources of his power were manifold:

\begin{itemize}
  \item 
    Some of his metaphysical stemmed from the \xss. 
  \item 
    Some of his power \hr{Ishnaruchaefir and cosmic gods}{stemmed from the cosmic gods}, such as \hr{Zaz}{\Zaz and \Urzaz}, with whom he had \hr{Ishnaruchaefir and Zaz}{dealings}. 
  \item 
    Some of his power stemmed from pure rage.
    In his normal cold, badass state, \Ishnaruchaefir{} is a terrific enough opponent. 
    But when his rage is triggered, his unholy \xzaishannic{} blood boils, and he flies into a blind fury, unconsciously tapping deep into the \chaotic{} power which he otherwise keeps in check. 
    In this state, he can cause tremendous destruction\dash and often has.
    
    In his fury, he relives the moment where he slew his beloved, and the events that led up to that moment\dash the conflicts, the lies and betrayals.
  \item 
    Another source was his pure inner strength. 
    This was a combination of his passion, his motivation and his uncaringness. 
    He possesses immense determination when he needs something done, but at the same time he has a certain detachedness that makes him fearless and unwavering. This is important, because fearlessness is one of your most important weapons when trying to master the energy of Chaos. 
    
    \Ishnaruchaefir{} masters Chaos because he \emph{must}, and because he does not fear it.
    
    \hr{Secherdamon}{\Secherdamon}, for all his millennia of research and all his dark pacts, has never achieved the same level of raw power (although he has garnered a greal deal more political power). At his core, \Secherdamon{} still has too much neediness, greed and fear. He lacks the cool, calm inner strength of his brother. 
\end{itemize}









\subsubsection{\Vertex{} status}
\target{Exile}
\ps{\Ishnaruchaefir} \vertex{} is called \quo{the Exile}. 
It is visible as a dim cloud of darkness in the night sky (to the occult astrologer who is using her spiritual sight and knows what to look for). 

His function and status as a \vertex{} is connected to his dark past and his betrayals. 
They have shaped him into the \vertex{} that he is today. 

He is often called a \quo{rogue \vertex}, aligned with no \matrix. 
This is strictly not correct. 
He has his own \matrix, with which his grandchildren and \Criseis{} are also aligned. 
And he is loosely affiliated with the \hs{Pyre} and some of the other Sentinel \matrices. 
But the links are tenuous and hazy and difficult to interpret. 















\section{\RaemythNexagglachel}
\index{\RaemythNexagglachel}
\index{\Nexagglachel}
\target{Nexagglachel}
The eldest of the three sons of \Tiamat and the brother of \Ishnaruchaefir{} and \Secherdamon.

\Nexagglachel was the son of \Tiamat and \Sethicus, and thus the greatest of the three brothers.

\Nexagglachel{} was a ruler, leader and preserver of order. 









\subsection{History}





\subsubsection{Durance}
After \ps{\Sethicus} rebellion failed, \Nexagglachel and the other \dragons were entombed. 
\Nexagglachel was \hr{Nithdornazsh was Nexagglachel's tomb}{bound in \Nithdornazsh}. 

When he awoke, he used \Nithdornazsh as his citadel. 





\subsubsection{Ambition like a dark lord}
\Nexaggrael had great ambitions even from a young age. He believed he could build a better world, and for that he needed to rule. 

Young \Raemyth was something of an aspiring dark lord\dash{}but the noble kind. He saw a world horribly flawed and full of suffering, and he wanted to fix it. He believed that the ends justified the means. He wanted to rule and make the world a better place. 






\subsubsection{Friendship with \Quessanth and others}
\Quessanth and \hs{Ixaeor} believed him. 
\Quessanth was also eager to serve as his champion and win glory and mastery as a martial artist. 

Irocas was at first a hanger-on. 







\subsubsection{Almost rebuilt the \draconian{} civilization}
\Secherdamon{} and \Ishnaruchaefir{} believe that the \draconian{} people were well on their way to rebuilding all the glory they had lost in the \firstbanewar. 
\hr{Nexagglachel could not rebuild Ophidian civilization}{It was difficult, but they were making progress}. 
The \resphain{} destroyed all that when they assassinated him. 
His two brothers would hate the \resphan race forever for that crime. 





\subsubsection{Fall}
After the \resphan{} rebels had \hr{Rebels conquer Merkyrah}{conquered \Merkyrah}, they \hr{Fall of Nexagglachel}{captured \Nexagglachel}, killed him and \hr{Origin of Satharioth}{created the \satharioth} from his blood.

\target{Nexagglachel sacrifices himself}
It is hinted that \Nexagglachel{} had predicted what was coming, and that he willingly sacrificed himself, for two reasons:

\target{Nexagglachel makes Satharioth hate Banes}
\begin{enumerate}
 \item To save and spare his younger brother, \Ishnaruchaefir.
 \item To live on in the \resphain{} and instill in them \hr{Satharioth hate Banes}{a deep hatred of the \banes} that would eventually cause them to betray their masters. 
\end{enumerate}

It is also hinted that he predicted that his two brothers would, in time, grow more powerful than he. 





\subsubsection{Captivity}
\Nexagglachel{} was \hr{Fall of Nexagglachel}{captured by the \banes}. 
\hr{Nexagglachel in captivity}{He was tortured, but remained defiant}. 





\subsubsection{Sacrifice and victory}
In his \hr{Nexagglachel wants to be Tiamat}{striving to be as great as \TyarithXserasshana}, \Nexagglachel{} eventually accepted the cards that chance had dealt him. 
And he played those cards. 
If he were to die, then he would not die in vain. 
He would become a ghost, a plague, a \hr{Curse}{Curse} upon the \resphain{} and their \bane{} masters. 

\Nexagglachel \hr{Nexagglachel lives on in Satharioth}{did not perish but lived on inside the \satharioth}. 





\subsubsection{Victory}
\Nexagglachel set out to bring about the downfall of the \resphain, and he succeeded. 
When at last \hr{Daggerrain falls}{\Daggerrain{} was defeated}, it was very much due to \NexagglachelsCurse. 
He had done it. 
He had sacrificed himself for his people and brought ruin to the hated \banes, just like his mother had done. 
He was now her equal, and second to no \dragon. 





\subsubsection{Legacy}
After his death, \Nexagglachel{} was remembered fondly.

\citebandsong{Ihsahn:angL}{Ihsahn}{Threnody}{
  He lies quiet now, \\
  in the nothing.\\
  And there is no epitaph, \\
  no stone.
  
  Walker of barren paths. \\
  Seer of night.\\
  Friend of shadows.\\
  A carrier of light.
}

\Ishnaruchaefir{} and \Secherdamon{} both felt lost and alone without him to guide them. 

\citebandsong{Ihsahn:angL}{Ihsahn}{Threnody}{
  There are no promises\\
  in his solitary grave.\\
  There is no salvation.\\
  Only words.
}

They sought strength in his memory. 

\citebandsong{Ihsahn:angL}{Ihsahn}{Threnody}{
  But what then are these precious streams\\
  of coldness from the heights?\\
  They will never reach the fields below.
}

But he still lived on in the form of a hidden ally: The Curse. 

\citebandsong{Ihsahn:angL}{Ihsahn}{Threnody}{
  And his legacy flows\\
  like a river from ice.\\
  The hungry heart opens\\
  and drinks from this fountain.\\
  So cold.
}







\subsection{Names and reputation}
\target{Nexagglachel is a powerful name}
\quo{\Nexagglachel} was a very powerful name. 
The Sentinels who founded \Ortaican myth wanted to \hr{Ortaicans use Nexagglachel's name}{milk it for everything it was worth}. 

Later, \Secherdamon even \hr{Secherdamon takes the name Nexagglachel}{took the name \quo{\Nexagglachel} as one of his own}.




\subsubsection{\Mezzagrael}
In \hr{Ortaican mythology}{\Ortaican mythology}, \Nexagglachel was known under the name \quo{\hr{Mezzagrael}{\Mezzagrael}}. 









\subsection{Personality}
\Nexagglachel{} was one of the noblest of \dragons. Much better than his two brothers. \Ishnaruchaefir{} acknowledges this, and even angsts about it a bit, while \Secherdamon{} rejects the notion. 

\Nexagglachel{} was a great and brave hero in some of the ancient wars. 

%\target{Ishnaruchaefir wars against Nexagglachel}
%Perhaps he fought against \Nexagglachel, and attempted to rally \dragons{} to this conflict, in an attempt to create cohesion rather than division among the \draconian{} people. He hoped to split them into two warring factions instead of a dozen warring factions.
He waged wars against other \draconian{} kingdoms and attempted to rally \dragons{} to this conflict, in an attempt to create cohesion rather than division among the \draconian{} people. He hoped to split them into two warring factions instead of a dozen warring factions. \Ishnaruchaefir{} and \Secherdamon{} fought by his side. 

\target{Nexagglachel wants to be Tiamat}
\Nexagglachel{} hoped and strove to become \hr{Tiamat's personality}{as great a leader as \TyarithXserasshana}. 
He wanted to unite the \dzraicchenosses{} as she had done, but never succeeded. 
The political situation was against such a unification. 
\Nexagglachel{} was every bit as skilled and strong-willed as \Kserasshana{} and could have done it, but events conspired against him. 









\subsection{Physique}
\target{Nexagglachel's appearance}
\Nexagglachel was pearly white with black eyes. 









\subsection{Politics}





\subsubsection{Descendants}
\target{Nexagglachel's children}
\Nexagglachel had two daughters.
Both were destroyed in the \hs{Incursion}.
Thus perished \hr{Sethicus}{\Sethicus}'s bloodline, for \hr{Sethicus's children}{\Nexagglachel was his only child}. 














\section{\Rathyon}
\target{Rathyon}
\index{\Rathyon}
A \dragon. 
The youngest of \ps{\QuessanthIshnaruchaefir} three grandchildren. 









\subsection{Arsenal}
\subsubsection{Fast flyer}
\Rathyon{} was a fast flyer. 
The fastest \dragon{} alive, it was claimed. 









\subsection{History}
He was an egg during the (end of the) \Secondbanewar{} and the \SecondShrouding{}, where his parents both perished. 
\Ishnaruchaefir{} took him in. 

He was already named before he hatched. 
His mother had given him an egg-name. 

\Criseis{} was almost like a mother to him. 















\section{\Skelcurmaggra}
\target{Skelcurmaggra}
\index{\Skelcurmaggra}
\Skelcurmaggra was a \dragon. 
She was virtually not active in \Azmith at all, but she had much power in some other realms. 

















\section{\Tentocoth}
\target{Tentocoth}
\index{\Tentocoth}
A \dragon. 
The middle of \ps{\QuessanthIshnaruchaefir} three grandchildren. 
Younger brother of \Thiencaste, cousin of \Rathyon. 









\subsection{Personality}
\subsubsection{Music}
\Tentocoth{} was a musician and spent much of his free time composing music, \hr{Draconic music}{in the \draconic{} style}. 















\section{\Thessulax}
\target{Thessulax}
\index{\Thessulax}
\Thessulax was a \dragon. 
She was one of the Elder \Dragons who had lain in Durance. 

She had much power and influence in \hr{Kai-Leng}{\KaiLeng} and other \hs{Chthonic Realms}. 









\subsection{Equipment}





\subsubsection{Tomb and mummies}
The tomb in which \Thessulax lay in Durance lay deep beneath the earth, in \KaiLeng. 
When she awoke she chose to make the underworld her abode. 

\target{Thessulax's mummies}
With her were entombed great numbers of \ophidians who joined her in Durance as mummies. 
She planned to resurrect them as a great undead army one day. 
Some of them had already been resurrected, but it was not easy. 









\subsection{Politics}





\subsubsection{Family}
\Thessulax was the mother of \hr{Zessuruch}{\Zessuruch}. 





\subsubsection{\Taortha}
She was part of the \taortha pantheon.
Here she was considered the Queen of the Underworld. 
















\section{\Thiencaste}
\target{Thiencaste}
\index{\Thiencaste}
A \dragon. 
The oldest of \ps{\QuessanthIshnaruchaefir} three grandchildren. 
Older sister of \Tentocoth, cousin of \Rathyon. 















\section[Tyrasshana]{\TyarithXserasshana}
\target{Tiamat}
\index{\TyarithXserasshana}
\TyarithXserasshana{} was a lover of \Sethicus and one the very first \dragons. 

Her father was \hr{Hesod-Nerga}{\HesodNerga}. 







\subsection{Name}
Her taken name \quo{\Kserasshana} means \quo{\xs-like}. 
She took it after she became the founder of the \dzraicchenosses. 

She founded the tradition of \hr{Draconic names}{\quo{taken names}}, which the \dragons{} kept using for many thousands of years. 






\subsection{History}





\subsubsection{Worm heritage}
\Tiamat was originally of low status. 
She was born a \hr{Ophidian Worm}{Worm}, the \Caisith lower caste.





\subsubsection{Rebellion}
\Tiamat became twisted and insane and \hr{Tiamat mutates}{mutated into a monster}. 

Eventually \Tiamat \hr{Sethicus versus Tiamat}{turned against \Sethicus}. 

\Sethicus defeated \Tiamat and \hr{Tiamat imprisoned}{imprisoned her in a nightmare dimension}.





\subsubsection{\Xserasshana{} and the \firstgendragons}
\Tiamat{} now sleeps for millennia, 
because that is her nature as a Dreaming Queen of the \ophidians{}. She intends to awaken again some time, thousands of years in the future. But some \dragons{} fear that the \banes{} will reach the Heart of \Miith{} in her absence and gain the power to prevail, destroying even \Tiamat{} and her \firstgendragons. 

They attempt to awaken her.

Eventually, \Tiamat{} half-stirs and agrees to provide her spawn with some power with which to oppose the enemy. 





\subsubsection{Mythical status}
\Tiamat{} is a mythical, divine figure. 
Like the Shadow-King from Bal-Sagoth. 
She has a number of artifacts that are now highly sought-after relics. 

\lyricsbalsagoth{The Obsidian Crown Unbound}{
  [The Wizards of Vyrgothia:]\\
  Darkly bejeweled circlet of night, \\
  Crown of the Elder King,\\
  Unfettered at last the Trinity of Might, \\
  the Sceptre, the Sword and the Ring!
}





\subsubsection{Ungod of the Outer Darkness}
\target{Ungod}
\index{Ungod}%
After her defeat, \Tiamat became the \hr{Ungod} of the Outer Darkness. 

Her goal was to absorb all intelligent life on \Miith and unite them all in one great Concord. 
She found the \banes and conquered them and made them her slaves, part of her Concord hivemind. 
That was one of the reasons why the \banes were so horrid: 
They were not individuals. 
They were, in a sense, soulless\dash but also part of something greater and terrifying. 

\Tiamat masterminded the fall of Semiza and the rise of the \resphain.
She made Semiza corrupt the \resphain to her service. 
She made them invade \Miith and wage war against the \ophidian races. 

\Tiamat wanted to grow strong and escape from her awful prison. 
The advent of \Lithrim finally freed her. 
She was now free. 
Ramiel, \Azeraid and \Iscrafel managed to destroy \Daggerrain and the \bane hordes, but they could not stop the Mother. 
\Tiamat was now awake, and she would devour the world. 

(\Daggerrain was a servant of the \hr{Voidbringer}, the cosmic god of the \banes. 
The Voidbringer is separate from the Ungod, although they are often conflated by the ignorant.)

\target{Tiamat's hivemind plan}
But all was not completely bleak. 
\Tiamat's merger with the \banes made her more stagnant and decaying and destructive than she might otherwise have been.
The heroes had destroyed this, the worst part of her. 
And in their own quests they had grown strong.
They now represented a mighty and virtuous force of positive change. 
\Tiamat absorbed them, and in doing so she absorbed their virtue and strenght and made it part of her.
So they would live on in her, and their deeds and legacy would help steer the great Mother. 
Now \Tiamat would devour everything on the planet and then move on to conquer the universe. 

And the Mother spread her wings. 





\subsubsection{\Tiamat becomes a cosmic predator like the \banes}
\target{Tiamat's hivemind is like the Banes}
\hr{Banes are created}{The \banes were descendants of the \voyagers}. 
They were what the \voyagers inevitably had to evolve into if the \voyagers were to continue their progress. 
Just as the \voyagers had no choice but to become the destructive \banes, so \Tiamat's hivemind was a step in the same direction. 
With the hivemind, the people of \Miith (who were themselves seeded by the Voyagers) became a bloated, life-consuming horror that would eventually suck dry the Heart of \Miith and need to travel to other worlds to consume them. 
They would become a cosmic parasite (or a cosmic predator, depending on your view).

This was the only progress possible. 
The alternative was to stagnate and do nothing and eventually be overrun by some other civilization who had \quo{progressed} further up/down the ladder of cosmic evolution\dash in this case the \banes. 
So there were only two choices: 
Die, or live long enough to become the villain. 

\Sethicus railed agains this and hoped that a third option existed. 

\Iscrafel suspected the above, but only late in his life (during the \thirdbanewar) \hr{Iscrafel accepts Tiamat's hivemind}{did he come to fully understand it}. 










\subsection{Personality}
\target{Tiamat's personality}
\Kserasshana{} \hr{Tiamat's power}{was not as \uber-powerful as some think}. 
But she had one great asset: 
\Kserasshana{} was a great leader. 
She managed to unite the \dzraicchenoss{} people under her banner, a feat which no other leader before or after her ever accomplished. 
At the end of the \firstbanewar, she even led her \firstgendragons{} in a suicide attack to create the \hr{Crystal Sphere}{\CrystalSphere}. 
\Kserasshana{} was cruel, but also noble. 
She sacrificed herself for her people. 









\subsection{Physique}
\target{Tiamat's true form}
Among the \dragons{} she is depicted as a \dragon{} with multiple heads. 









\subsection{Politics}





\subsubsection{Children}
\Tiamat physically mothered the first born \dragons. 
She bore many children.
In the \hr{Sethicus versus Tiamat}{great war between her and \Sethicus}, most of her children took her side and were destroyed for it. 
Only three of her children survived to go into \hs{Durance}:
\Nexagglachel, \Ishnaruchaefir and \Secherdamon. 









\subsection{Skills and powers}





\subsubsection{\Dweomer status}
\target{Tiamat dweomer}
Eventually \Tiamat became an immortal god. 
Some believed that she became a \dweomer unto herself, a parasitic mother whose lifeforce was linked to the entire \draconian race. 
All souls that the \dragons consumed eventually passed to \Tiamat.
Thus she grew ever stronger in her nightmare prison beyond reality. 





\subsubsection{Power level}
\target{Tiamat's power}
Physically and metaphysically \Kserasshana{} was no mightier than \Secherdamon{} or \Ishnaruchaefir{} became (at Carzain's time). 
Her magic was strong, but also primitive and \naive. 
The idea that \Kserasshana{} was super-duper-\uber-powerful is a romanticized myth. 















\section{\ValcanSethicus}
\target{Sethicus}
\index{\Sethicus}
An \ophidian. 









\subsection{Arsenal}





\subsubsection{Mummified relics}
\target{Relics of Sethicus}
After the death of \Sethicus, it was possible to commune with the remnants of his spirit to some limited degree by using the mummified remains of his body. 
He could still be contacted to some extent via his mummified body.





\subsubsection{Tomb}
\target{Sethicus tomb}
\Sethicus's \quo{tomb} was a colossal temple. 
It was once his seat of power.
It became his tomb when he was imprisoned in \hs{Durance}. 
Some time after \hr{Sethicus dies}{his death}, his tomb was lost. 
It was later discovered in the \hs{Telluric Realm} of \hr{Neevrai}{\Neevrai}. 
Later the tomb was used to construct the \hs{Ark}. 









\subsection{History}





\subsubsection{Youthly piety}
\target{Valcan stuck to the Epitomes}
Young Valcan was law-abiding and stuck with the \hs{Epitomes}. Even then he was a prodigy and achieved incredible feats and innovations. Many were horrified when the genius Valcan began consorting with the Outer Darkness. 






\subsubsection{Youthly rebellion}
\Sethicus underwent some horribly disturbing revelations in his youth.
They led him on a path of mysticism.
He began to romance with the \xss more and more. 

Compare him to Sephiroth from the game \cite{VideoGame:FinalFantasyVII}.

\lyricstitle{
  \href
    {http://finalfantasy.wikia.com/wiki/Final_Fantasy_VII}
    {Final Fantasy Wiki: Final Fantasy VII}
}{
  Five years before the beginning of the game, Cloud and Sephiroth were sent to Cloud's hometown of Nibelheim to investigate the Mako Reactor there. Inside, Sephiroth found Jenova, a creature Shinra mistook as an Ancient and whom had been called Sephiroth's mother. Sephiroth begins to look deeper into his past and the Jenova Project from which he was born. It was led by Professor Gast and the deranged Professor Hojo. What he finds drives him insane. Believing himself to be the last Ancient, Sephiroth begins to take revenge on humanity by burning Nibelheim to the ground. Lost in the fires is Cloud's mother and Tifa's father. Cloud runs up to confront Sephiroth, but his recollection fails him before he can reach the end of the story.
}





\subsubsection{Rebellion}
He became an \ophidian{} leader with controversial ideas. 
He practiced black \xsic{} magic and waged wars. 

Compare him to Urizen from \authorbook{William Blake}{The Book of Urizen} and other works. 

\Sethicus criticized the \ophidian{} society. 

\lyricsbs{Emperor}{In the Wordless Chamber}{
  In the wordless chamber\\
  they feared death desperately.\\
  Thus they clustered to the fruits of the earth,\\
  craving diversion,\\
  as if to avoid knowing why.
  
  In the wordless chamber\\
  they feared life desperately.\\
  Thus they proclaimed any given truth\\
  and swallowed,\\
  as if to justify their fear.
}





\subsubsection{Durance}
His enemies united against him and conquered him. 
But \hr{Draconic immortality}{he was immortal as a \dragon}. 
So they \hr{Sethicus imprisoned}{imprisoned him}. 
He spent millennia in \hs{Durance}. 





\subsubsection{Followers}
Even in his \hs{Durance}, \Sethicus had followers.
Some \ophidians \hr{Ophidians follow Sethicus under Durance}{carried on his religion}. 





\subsubsection{Death and transcendence}
\target{Sethicus in the Threnody}
\Sethicus \hr{Sethicus dies}{died in the \firstbanewar}. 

But unbeknownst to most, a remnant of \Sethicus's powerful soul that lingered on as an undead, quasi-conscious presence.
When \Sethicus died, his occult mastery was so great and his will so powerful that his soul did not perish, but transcended from \Miith and became a disembodied god.
His spirit lived on, but in a horrid, maddened state. 

He became a new voice in the \hr{Threnody}\dash a tortured wraith, a mammoth \daemon doomed to howl in madness forever. 

\hr{Iscrafel learns that Sethicus is in the Threnody}{\Iscrafel was horrified to discover this.}

Some living \dragons and \ophidians believed that \Sethicus still lived as an enlightened, disembodied cosmic god.
Some of them began to worship him (and his \hr{Relics of Sethicus}{mummified relics}) as an absconded god.
But few suspected the truth, and those who did went insane still-living could not fully ascertain whether \Sethicus still existed in some form.





\subsubsection{Mythical archetype}
\target{Sethicus as archetype}
In \draconian metaphysics, \Sethicus had a status as a mythical primogenitor. 

He is featured heavily in \WanderersInDarknessEmph as a supreme founder god, from whom the three \quo{wanderers} derive their natures and powers. 

Compare to Adam Kadmon in \Cabbalist theory (whose body contains the ten \sephiroth) and Albion in William Blake's mythology (the primal man of whom the Four Zoas are mere fragments). 

\lyricswikipedia{Adam_Kadmon}{Adam Kadmon}{
  The conception of Adam Kadmon becomes an important factor in the later Kabbalah of Luria. 
  Adam Kadmon is with him no longer the concentrated manifestation of the Sefirot, but a mediator between the \emph{En-Soph} (\quo{Infinite}) and the \emph{Sephiroth}. 
  The \emph{En-Soph}, according to Luria, is so utterly incomprehensible that the older Kabbalistic doctrine of the manifestation of the \emph{En-Soph} in the \emph{Sephiroth} must be abandoned. 
  Hence he teaches that only the Adam Kadmon, who arose in the way of self-limitation by the \emph{En-Soph}, can be said to manifest himself in the \emph{Sephiroth}. 
}









\subsection{Name}
His original name was \quo{Valcan of Tan-Izul}. 

When he became a \dragon, he took the name \emph{\Sethicus}. 

To the \Ortaicans, \Sethicus was known as the \taortha \hr{Settras}{\Settras}.









\subsection{Personality}





\subsubsection{Addiction to knowledge}
\target{Sethicus addicted to knowledge}
Sethicus confessed that he feared his quest would only lead him to madness and doom. 
But he was addicted and had to move forward. 
\hr{Addiction to knowledge}{Enlightenment was highly addictive}. 





\subsubsection{Innovator}
\Sethicus himself was a brilliant scientist and \hr{Sethicus brought innovation}{made many exciting new discoveries} in various fields. 
He sped up the otherwise slow \ophidian society. 





\subsubsection{Mortals}
\Sethicus ascribed little value to the lives of mortals. 
He would gladly sacrifice any number of them to serve \hr{Sethicus plan}{his masterplan}. 
After all, they were \hs{False Life}. 
Their lives were short, full of suffering and mostly without purpose or meaning.
So what difference could a little more death and suffering make?





\subsubsection{Understands the \xss}
\target{Sethicus understands XS}
\Sethicus was the only \dragon to ever come close to truly understanding the \xss.
In the process of learning to understand them, his mind became twisted. 
He lost his \quo{humanity} and became alien. 
His fellow \ophidians could no longer understand him and began to call him insane.









\subsection{Philosophy}
\target{Sethican philosophy}
Where most \ophidians were \hr{Ophidian philosophy}{atheistic and nihilistic}, \Sethicus was a mystic and founded a religion. 





\subsubsection{Necrocosmos}
\Sethicus pioneered the theory of the \hs{Necrocosmos}. 





\subsubsection{Free will}
\target{Sethicus and free will}
\Sethicus revolutionized \caisith philosophy when he proved a theorem establishing that\dash under certain special conditions\dash \caisith individuals (and perhaps other races) could reach a state of \emph{null causation} (also called \emph{null position}) where truly free choice became possible. 





\subsubsection{Gods}
\Sethicus believed:
\begin{itemize}
  \item 
    That there existed a number of great gods who were immanent parts of the universe and thus worthy of idolization.
  \item 
    That there existed a mystic spiritual world connected to those gods.
  \item
    That great insight, happiness and power could be found by exploring said mystical world.
\end{itemize}

Some of \ps{\Sethicus} great \hs{cosmic gods} were entities well-known to the \ophidians.
Other \ophidians accepted them as existing and powerful beings but did not believe in their immanent, spiritual aspect. 

These cosmic gods were not \xss. 
\Sethicus knew that they were far older and mightier and \quo{higher} than the \xss. 
But he believed the \xss were a wiser and more spiritually advanced race than the \ophidians and that much could be learned from them. 
\Sethicus approached the \xss as mentors and patrons on his quest towards a higher goal. 
The \xss were not themselves the end goal. 





\subsubsection{Myth and poetry}
\target{Sethican myth}
Much of \Sethicus's science was written down in the form of myths and poetry. 
\Sethicus was an eccentric mystic who felt that mystical insight was more important than mundane knowledge. 
Besides, he was a power-hungry and power-jealous snob who was not keen on sharing his insights with the common herd, so he deliberately expressed himself in a cryptic insider manner. 

Many of his writings (especially his \hr{Sethican eschatology}{his eschatology}) was horrible to read and brought insanity. 

Many people rejected \Sethicus's more bizarre writings. 
They believed that \Sethicus simply turned mad and twisted near the end of his career.
That way they could pass his writings off as delusional ramblings instead of horrifying truth. 





\subsubsection{\Ophidians}
\Sethicus suspected that \hr{Ophidians related to XS}{the \ophidians were related the \xss}. 





\subsubsection{Personal development}
\Sethicus believed in personal spiritual development. 
His goal was not simply magical and worldly power for their own sake, but as stepping stones toward reaching a higher state of existence. 

The \dragons took a great leap in their development when, with the help of \KhothSell, they \hr{Draconic immortality}{gained their True Immortality}. 
\Sethicus himself had achieved so great a spiritual mastery that his soul was able to survive the permanent destruction of his body and \Sethicus \hr{Sethicus in the Threnody}{become a disembodied wraith}. 

\citeauthorbook[p.65]{VengerSatanis:CthulhuCult}{Venger Satanis}{Cthulhu Cult}{
  While crossing the Abyss, most of our personality disintegrates. 
  Our fragile and useless parts begin to die.
  Once the old self has been annihilated, the True Self can be born.
  Beyond what we know\ldots{} where rainbow hued spheres luridly shimmer and indulate, Cthulhu's Emerald Kingdom waits for the initiated.
  The undiscovered, sunken towers of R'lyeh call to us.
  Yes, the city of decay and morbid delights can be reached by desperate seekers.
  It pulsates and slithers; His asymmetric realm is a silent sepulcher of slime. 
}





\subsubsection{Planes}
\target{Sethican planes}
\Sethicus believed that the world was composed of a number of parallel \quo{planes}, each plane deeper than the next. 
The planes (arranged with the deepest plane first) were:

\begin{description}
  \item[\DaathKurZulNathla] was the plane of primal chaos. 
    It was blind, mindless, formless force of change, creation and destruction.
    It was the deepest plane which all others were built upon and emanated from. 
    
    Compare it to Azathoth from the Cthulhu Mythos. 
    
    \DaathKurZulNathla was also a plane where conflicting forces collided.
    \Sethicus believed that such conflict between opposing forces was the source of all life and motion.
    He saw sexual reproduction (where opposite genders collide to create new life) as one of the many manifestations of this phenomenon. 
    
    \Sethicus considered the \noggyal \hs{mother-mass} to be an entity of \DaathKurZulNathla, insubstantial and creative/destructive as it was. 
  
  \item[\Osserylloch] was the plane of the basic emotions or motivations, including fear, hunger, anger and lust. 
    
  \item[\Barbeloth] was the plane of materia. 
  
  \item[\YothUnXachtyon] was the plane of consciousness, intelligence. 
\end{description}

Note that in \Sethicus's world view, emotion (and thus spirit) preceded matter. 
This conflicted with traditional \hr{Ophidian philosophy}{\ophidian philosophy}. 
The \ophidians believed that life was random, mechanical, diverse, unrelated to each other, without any built-in soul. 
\Sethicus believed that all life sprang from a single uniform spritual source, namely the primal chaos. 

\Sethicus's theory inspired the \hr{Ortaican planes of existence}{\Ortaican theory of planes of existence}.






\subsubsection{Sound}
\target{Sethican sound mysticism}
\Sethicus had much mysticism regarding sound and words. 
He believed that words (spoken with will and intelligence behind them) possessed power that reached down into the \hr{Sethican planes}{deep planes}. 

Therefore \hs{Chaos magic} had such an emphasis on spoken incantations.

\Sethicus was one of those who developed/discovered the \hr{True Draconic}{\TrueDraconic} tongue. 






\subsubsection{\Voyagers}
\Sethicus had a low opinion of the \voyagers.
He saw them as lower creatures, usurpers who foolishly challenged the superior \xss (his adored masters and mentors) and tried to take their place.
The \voyagers believed themselves great enough to create new life, but it led to disaster and their downfall (in the form of the \hr{Noggyal}{\noggyaleth} and \hr{Bane}{\banes} which rebelled against the \voyagers). 




\subsubsection{\Sethican writings survive}
\target{Sethican writings}
After \Sethicus was forced into \hs{Durance}, some writings containing the original words and theories of \Sethicus regarding \hr{Sethican philosophy}{his philosophy} survived. 
These were treated as sacred by the surviving \Sethican cultists. 

Some of \Sethicus's earlier works were full of errors. 
For one, they confused the \xss with all sorts of other cosmic gods, more or less labelling all cosmic gods as \xss. 
For another, \Sethicus once believed that the \xss would very soon rise and reclaim the world. 
\Sethicus later learned the error of this, but some of his older writings still existed. 
And since those works contained many spells that did work, they were accepted as canonical by the cult, and those erroneous beliefs were perpetuated. 









\subsection{Politics}





\subsubsection{Descendants}
\target{Sethicus's children}
\Sethicus had only one child, \hr{Nexagglachel}{\Nexagglachel}. 
Thus \Sethicus's bloodline perished when \Nexagglachel and \hr{Nexagglachel's children}{his daughters} died.









\subsection{Eschatology}
\target{Sethican eschatology}
\index{eschatology!\Sethicus}
\Sethicus had written down a terrible prophecy of the end of the world. 
It was particularly terrible to those \quo{in the know} because they knew that \Sethicus knew what he was talking about, so if he believed all these things, they were probably true. 
Few dared read it, and fewer still dared try to interpret it.

Back when \Sethicus was alive, a few people had dared ask him about the prophecy. 
But \Sethicus, though a visionary genius, was no great teacher. 
When, once in a while, he successfully managed to explain it to someone, it tended to drive them mad. 
So after \Sethicus's death, no one knew what the prophecy truly meant. 

The prophecy was a continuation of the story of the \hs{dead universe}. 
It told that \Miith was a fresh, alluring, rotting piece of carrion in the dead universe. 
It attracted carrion feeders. 
That is why the planet would never know peace. 

There would come an apocalypse where \Miith would go under in a hell of death, blood and eternal suffering and torment. 
Sooner or later. 
If not by the hands of the \banes, then some other foe. 
It had happened before (before the \ophidians) and would happen again in cycles, leaving less life each time, until \Miith would at last be completely dead and all that was once life would be condemned to suffer in blindness and madness and darkness and cold forever. 

The only hope was to flee. 
This was \hr{Sethicus plan}{\Sethicus's masterplan}. 





\subsubsection{Masterplan}
\target{Sethicus plan}
\Sethicus believed that \hr{Sethican eschatology}{the world would go under in a hell of blood and suffering}.
He believed that the only hope was to flee.
So that was what he planned to do. 
He hoped to achieve \hs{True Life} for himself and the \draconian people (and perhaps the \ophidians too). 

Part of the motivation behind \Sethicus's cruel tyranny and his mad obsession with science and religion was his long-term plan to save his people. 
He knew he had to be cruel to be kind. 
The majority of his people would never understand the grandness of his vision nor believe in it. 
To carry out his plan he would need to be strong and to make tremendous sacrifices, even on other people's behalf. 

Compare to the movie \cite{Movie:Watchmen}. 

He wanted to engineer some huge global apocalypse and feed on the destructive energies unleashed. 
This was supposed to help his personal spiritual development and enable him to become a god. 
According to \hr{Sethican philosophy}{his philosophy}, destruction was a good thing because it brought renewal, evolution and new life. 
This principle was embodied by \hr{Satha}{\RuinSatha}.

Compare \Sethicus to Sephiroth from the game \cite{VideoGame:FinalFantasyVII}.

\lyricstitle{
  \href
    {http://finalfantasy.wikia.com/wiki/Final_Fantasy_VII}
    {Final Fantasy Wiki: Final Fantasy VII}
}{
  Sephiroth's plan is to use the Black Materia, a Materia so powerful that the Cetra hid it away to stop its use. The Black Materia contains the spell Meteor, the ultimate Black Magic. It can summon a giant meteor to crash into the surface of the Planet. Sephiroth's plan is to create a wound in the Planet so large that the Lifestream will need to be sent en masse to heal it. Here, Sephiroth will intercept the Lifestream and take complete control of the world.
}

In the midst of his long preparations to save his race, \Sethicus was suddenly and unexpectedly \hr{Sethicus betrayed}{betrayed by lesser beings} who did not understand his vision.
He was defeated and imprisoned in a tomb for a million years.
Then he was awakened and everyone expected him to sacrifice himself to save them from the \banes. 
This left \Sethicus bitter, so when his soul survived, he left \Miith and his race behind forever and sought salvation for himself alone. 

But he left behind many artifacts and many teachings (\hr{Sethican myth}{in the form of myths and poetry}). 
The \hs{Ark} is one such remnant. 















\section{\Vexstrasshin}
\target{Vexstrasshin}
\index{\Vexstrasshin}
A \dragon{}. 
She was active during \ps{\Semiza} lifetime and tried to stop him and \Thanatzil. 
She succeeded... sort of. 

She was eventually \hr{Vexstrasshin dies}{killed by \ps{\Daggerrain} trickery}. 















\section{\Vizsherioch}
\target{Vizsherioch}
\index{\Vizsherioch}
\Vizsherioch{} is a \dragon, the son of \Secherdamon. 
In him, his father invested not only \draconic{} power, but also stolen \bane{} and \resphan{} power, intending to create a \draconic{} counterpart to the \satharioth. 

Vizsherioch is the Son of Chaos, the messiah of the \dragons and the harbinger of the \xs and the hordes of Chaos. 

He is evil and badass. 
Compare him to Blackheart from the movie \cite{Movie:GhostRider}.









\subsection{Name and titles}
\Vizsherioch and probably some other names. 

One of his titles, which he used himself, was \quo{Chaos Incarnate}. 
Another title was \quo{Son of Chaos}.

Later, \hr{Vizsherioch}{\Vizsherioch} \hr{Vizsherioch takes the name Sethicus}{took the name \quo{\Sethicus}}. 









\subsection{Physique}
\target{humanoid Vizsherioch}
He often takes on a humanoid form: 
A \scatha{} with ivory white scales. 
This makes him seem at once innocuous and freakish. 

\Vizsherioch appears as a \dax in his prime, with pearly white scales, wearing a loose robe of white, silver and gold. 

He is fearful to look upon, even to mighty ones such as \LocarPsyrex. 
Where \Secherdamon is fiery bright, his son \Vizsherioch is dark and sinister. 
Not in \colour, but in feel. 
A vast darkness follows behind him and around him. 

His eyes are frightening, even for \Psyrex. 
\ps{\Secherdamon} eyes are terrible enough, but \Psyrex is used to them. 
There is passion, fervour and desire in the eyes of \Secherdamon, and anger and hate, too. 
But in \ps{\Vizsherioch} eyes there are hints of otherworldly madness. 










\subsection{History}





\subsubsection{Birth}
The newborn \Vizsherioch{} fancied himself a reborn \xs. 

\lyricslimbonicart{From the Shades of Hatred}{
  A thousand years time dimension \\
  in subconscious incarceration. \\
  My hatred to man, has transformed me \\
  into a habitation for demons. \\
  A devil incarnation. \\
  In the forgotten past, ages ago, \\
  beast became my alter ego.
  
  The god in me, infernal black divinity.\\
  After years of agony and pain \\
  hatred is all that remains.
}

It had taken \Secherdamon{} thousands of years of planning and research to finally create \Vizsherioch. 
Compare him to Set Abominae from \cite{IcedEarth:SomethingWicked}. 





\subsubsection{Prophecies}
With his \bane{} blood he was a super-\dragon. 
Some wise people foresaw that he was destined to conquer and rule. 

\lyricsbalsagoth{Naked Steel (The Warrior's Saga)}{
  Born beneath the thrice-cursed cromlech \\ 
  (destined for deeds of greatness),\\
  Three stars aligned to assauge thine (newborn) cries,\\
  Foretold, the hilt of Red-Tooth awaits thine hand \\
  (kingdoms shall fall before thee!),\\
  And in the Nine Scrolls thine death prophesized.
}

Some foresee that he will fail and die a tragic death at a young age.

\lyricsbalsagoth{Naked Steel (The Warrior's Saga)}{
  This heart that pounds like a hammer,\\
  this heart that pounds so strong,\\
  this heart that pumps a great warrior's blood,\\
  this heart will pound for half as long.
}

It would be cool to have him break this prophecy. 





\subsubsection{\Malcur}
The summoning of \Nithdornazsh in \Malcur is part of \ps{\Secherdamon} plan to bring \maybehr{Vizsherioch}{\Vizsherioch} into Ascendancy. 
\Nithdornazsh{} is to become \ps{\Vizsherioch} citadel, a \nexus{} from which he can grow strong and spread his tendrils (politically and metaphysically) into the Realm of the Shroud. 
This is a vital step in the forging of the \maybehs{Dagger}. 
When the \Nithdornazsh{} project is complete, \Vizsherioch{} is more Dagger-y than ever. 

Previously, \Secherdamon{} had kept \Vizsherioch{} sequestered and hidden. 
He is his only son and the fruit of thousands of years of hard work, so \Secherdamon{} is very protective and does not want to lose him. 





\subsubsection{Takes new names}
\target{Vizsherioch takes the name Sethicus}
When he turned into a \dragon he took two names. 
He took \quo{\hr{Sethicus}{\Sethicus}} as one of his names.
Maybe he took \quo{\Secherdamon} as the other. 





\subsubsection{Role}
He will become a pivotal player in the later books of \SentinelsofMith.

%Perhaps he will die at the end. Or perhaps \Secherdamon{} will die and \Vizsherioch{} will inherit his empire. I haven't quite decided who, but one of them should die. 
Eventually, \hr{Secherdamon dies}{\Secherdamon{} dies} and \Vizsherioch{} absorbs his power and soul into himself. 





\subsubsection{Extreme plan}
\ps{\Vizsherioch} plan was more extremist than \ps{\Secherdamon}. 
\Secherdamon{}, despite his dubious sanity, cared first and foremost about the \draconian{} race and wanted to save them. 
\ps{\Vizsherioch} loyalties, on the other hand, lay on the side of the \xss. 
He wanted to destroy the Shroud and let the \xss{} into the world, and dreamt of turning \Miith{} into an eternal Realm of Chaos. 

\Vizsherioch{} realized that the \banelords{} also wanted to unravel the Shroud. 
And so a doublecrossing race began, with both sides of the Feud trying to pull the right threads and so unravel the Shroud so it crumbled into a pattern that was in their favour and would let their own variant of \trope{CosmicHorror}{Cosmic Horrors} into the world. 





\subsubsection{Builds cult}

\Vizsherioch revelled in his megalomania and built a cult around himself.
This picked up speed after \hr{Secherdamon dies}{\Secherdamon died}, because \Vizsherioch now knew the responsibility lay with him alone.

He took the \hs{Dark Crescent} and reshaped it into a focused group that served him. 

\citebandsong{Nile:InTheirDarkenesShrines}{Nile}{
  Churning the Maelstrom
}{
  I Am the Uncreated God\\
  Before Me The Dwellers in Chaos are Dogs\\
  Their Masters Merely Wolves\\
  I Gather The Power\\
  From Every Place\\
  From Every Person\\
  Faster Than Light Itself
}









\subsection{Personality}
Have scenes with \Vizsherioch{} that show his personality and interests. What are his interests? Art? Science? Sex?




\subsubsection{He revels in evil}
\Vizsherioch{} revels in evil, death and destruction. 
He hopes it will bring him closer to his \xsic{} nature and fulfill his dream of becoming a \xs{} incarnation.

\lyricslimbonicart{The Ultimate Death Worship}{
  O' Darkness my master and mentor, \\
  witness the blood I shed. \\
  Victorious dreamlike death I enter, \\
  floating the streams so red. \\
  Destruction is the jewel of the black heart. \\
  To treat life as nothing holy. \\
  Hatred is the diamond in blasphemous art, \\
  as death you kiss infernally.
}





\subsubsection{Split personality}
\Vizsherioch{} has a split personality and changes back and forward between \quo{young mode} (a \dragon{} only few thousands or even hundreds of years old) and \quo{old mode} (when he feels like he's a \xs, millions of years old). 

When he speaks in his deep-pitched, magical voice, it is even deeper, darker and more foreboding than that of \Ishnaruchaefir{} or \Secherdamon. 





\subsubsection{Older than he looks}
In a sense, \Vizsherioch{} is older than he looks, having inherited some memories from the \resphan{} and \xzaishann{} blood in his veins.

He sees himself as the ultimate combination of the best aspects of the old and new generations of creatuers. 

\ta{%
  You think me young. 
  Fools. 
  The blood of the \xzaishanns{} runs stronger in me than in anyone else. 
  I AM [insert names of ancient \xzaishannic{} lords]. 
  
  This body is but temporary. 
  I am, I will be, and I have always been. 
  I am eternal. 
  \shout{I am Chaos}!}

\lyricsdimmuborgir{The Chosen Legacy}{
  I am the first creature of this Kingdom.\\
  I will be the One\\
  to out live His time\\
  with the triumph of free will.
}

He has memories going back millions of years. 

\lyricsbs{Monolith Deathcult}{Deus Ex Machina}{
  Atlantis was built when you amoebas crawled through filth. \\
  I am Holiness Divine, your Lord and Master, \\
  the Supreme God and Creator.
}







\subsection{Politics}

\Vizsherioch{} is very loyal and loves his father. Not only has he inherited genes of loyalty from his \bane{} blood, but his father has also been careful to treat him with love and respect and groom him to be a close ally and heir. 

He has been around \hr{Nzessuacrith}{\Nzessuacrith} for much of his life and sees her as a sister-figure of sorts. 
He calls her \quo{cousin} (which she is). 
This unnerves her. 
She secretly resents and envies him for having such a close and loving relationship with his father, whereas she herself has not gotten along with her father, \Ishnaruchaefir, in thousands of years. He knows this, and secretly holds her in contempt. 









\subsection{Skills and powers}





\subsubsection{Immortality}
\target{Immortal Vizsherioch}
\Vizsherioch{} is immortal. 
He can be killed, but he will come back to life. 

This is a state-of-the-art version of immortality, devised by \Secherdamon{} as an improvement on the \hr{Ophidian immortality}{skin-shedding}, \hr{Draconic immortality}{\KhothSell-based} and \hr{Malach immortality}{reincarnation-based} immortality techniques. 

Actually, he is more immortal than other immortals. 
He has a soul in him that is almost a \xsic{} soul, making it indestructible by any means known to the \Miithian s. 
If \Miith{} were destroyed by the \Voidbringer, \Vizsherioch{} would still live on as a \xs{} fledgling. 





\subsubsection{Weaving artifacts embedded in him}
\Secherdamon{} had \hr{Secherdamon steals weaving artifacts}{stolen several weaving artifacts}. 
When he created his son, he sealed and embedded the artifacts within \ps{\Vizsherioch} body, along with all of their power. 

This was impossible at the time of the \SecondShrouding{}, but \Secherdamon, being \hr{Secherdamon's science}{one of the greatest scientists and sorcerers of all time}, has researched and learned much since then, and his knowledge of magic is far deeper and more detailed now.





\subsubsection{\XzaiShannic{} weaknesses}
\target{Vizsherioch's XS blood}
\Vizsherioch{} has even more \xsic{} blood than his father, and suffers even more from their inherent weaknesses. 
He is prone to the \hr{XS slumber}{\xsic{} slumber}. 















\section{\Zessuruch}
\target{Zessuruch}
\index{\Zessuruch}
A young \dragon. 
At the time of the \thirdbanewar{} she was one of the youngest \dragons alive. 

\Zessuruch was the daughter of \Thessulax. 









\subsection{History}
She was \hr{Teshrial kills Zessuruch}{killed at one point} by \hr{Teshrial}{\Teshrial} and some other \resphain. 
But not permanently. 









\subsection{Personality}
In her youth, \Zessuruch was quite overconfident. 
After \hr{Teshrial kills Zessuruch}{she was killed in combat}, she became more careful. 























\chapter{Others}















\section{\Hesherritan}
\target{Hesherritan}
\index{\Hesherritan}
\Hesherritan was a female \ophidian.









\subsection{History}
\Hesherritan was born in the year \yic{Hesherritan birth}. 
At the time of the \thirdbanewar, she was the oldest non-Durance \ophidian in the world.















\section{\HesodNerga}
\target{Hesod-Nerga}
\index{\HesodNerga}
An \ophidian{}, the father of \hr{Tiamat}{\Tiamat}.
















\section{\Ishtacca}
\target{Ishtacca}
\index{\Ishtacca}
\Ishtacca was an \ophidian elder, born before the \firstbanewar. 
In the \Scatha Age he was the oldest and most powerful \ophidian in \hr{Yormis}{\Yormis}. 
He was an undead mummy.
He let his physical body lie protected in a sarcophagus deep beneath the city while he sent his mind out \hr{Ophidian possession}{to possess the bodies of the dead}.

\Ishtacca was often simply called \quo{the Master} by his mortal servitors. 









\subsection{History}
\Ishtacca was born in the year \yic{Ishtacca birth}.









\subsection{Politics}





\subsubsection{\Ubloth}
\target{Ishtacca and Ubloth}
\hr{Ubloth}{\Ubloth} used to dwell in the depths beneath the earth where it had been spawned when \Miith was young. 
It was \Ishtacca who had summoned \Ubloth from the deep and compelled it to come to the cavern under \Yormis where \Ishtacca's servitors could worship \Ubloth and make pacts with it. 
The apathetic \hs{God Beneath} seemed content to acquiesce.
















\section{\Iurzmacul}
\target{Iurzmacul}
\index{\Iurzmacul}
A \nagalord. 
Once an ally of \TyarithXserasshana. 

He might be the father of \hr{Ishnaruchaefir}{\Ishnaruchaefir}.



















\section{Maegon}
\target{Maegon}
\index{Maegon}
A \nagalord. 
One of the better known \nagalords, alongside \hr{Iurzmacul}{\Iurzmacul}. 

A sea god, known and feared by many coastal dwellers. 
Even the powerful sea god \hs{Shellagh} fears Maegon. 

He is worshipped in the Imetrium. 

Compare him to Dagon from \HPLovecraft's Cthulhu Mythos. 















\section{\Nasshikerr}
\index{\Nasshikerr}
\target{Nasshikerr}
An \hr{Ortaican gods}{\Ortaican{} goddess} of shadows, stealth and the hidden. 
Ofttimes a patron of thieves, spies and outcasts. 
Perhaps also a god of the underworld.

She was really an \ophidian.









\subsection{History}
\Nasshikerr was born in the year \yic{Nasshikerr birth}.









\subsection{Physique}
\Nasshikerr looked grotesque, \hr{Appearance of Ortaican gods}{as \Ortaican gods tended to do}.
He often took a form resembling a chameleon, with a \scatha-like face and recognizable facial expressions. 

Describe \Nasshikerr as slithering, writhing, loathsomely serpentine.









\subsection{Skills and powers}





\subsubsection{Shkormi Formulae}
\target{Shkormi}
\target{Shkormi Formulae}
\target{Greater Shkormi Formula}
\target{Lesser Shkormi Formula}
The Shkormi Formulae were a set of stealth spells that \Nasshikerr (and perhaps other gods) could grant. 

The Greater Shkormi Formula was a long and expensive and difficult spell that required a direct link to \Nasshikerr or another god with similar powers.
The Lesser Shkormi Formula was a much shorter and easier spell, but less powerful. 
It required only a weak link to \Nasshikerr, so it could be cast by any \rethyax.

The shkormi were a race of \daemons.
They could drag a person down into the Shroud, thus shielding him from the gazes of others. 
But spells were needed to keep the shkormi in check, lest they drag their subject too far down and destroy him. 















\section{\NathRamos}
\target{Nath Ramos}
\index{\NathRamos}
\NathRamos was an \ophidian.















\section{Racul the Necromancer}
\target{Racul}
\index{Racul}
Racul was an \ophidian necromancer. 
He lived 10.000s of years before the \hs{Draconian Ascendancy}{\Draconian Ascendancy}, before the civilization of \hs{Kush}.









\subsection{History}





\subsubsection{Worm background}
Racul started out as a hero of the lower classes. 
His mind was as sharp and as powerful as that of an \hr{Imperial Ophidian}{Imperial}, but in body he was a \hr{Worm Ophidian}{Worm}.
This meant that he was seen as a nobody. 





\subsubsection{War}
Racul invaded the world with hordes of the undead, including undead machines and formless horrors of flesh and blood and bone. 





\subsubsection{Fall}
His fall paved the way for the rise of Kush.















\section{Ramarxes}
\target{Ramarxes}
Ramarxes was a \caisith.
He was a forerunner and rival of \hr{Sethicus}{\Sethicus}. 
Eventually \hr{Sethicus kills Ramarxes}{\Sethicus slew him}. 















\section{\Sarokash}
\target{Sarokash}
\index{\Sarokash}
\Sarokash is the chief god of the Imetric Tribunal and the ruler of the Imetrium. 
She is actually a \dragon or \nagalord. 
\also{Imetrium, Tribunal}









\subsection{Physique}
In her true form, \Sarokash is a giant \nagalord. 
She often takes the form of a \scatha. 
In all forms, her scales are glittering emerald green. 















\section{Theraster}
\target{Theraster}
Theraster was a male \caisith martial artist. 
He was a mentor of \Quessanth and one of the few whom \Quessanth considered his superiors in the martial arts. 
\Quessanth thinks he only surpassed Theraster in the last few years before his Exaltation.

Theraster never became Exalted.















\section{Zanshir}
\target{Zanshir}
Zanshir was a female \caisith martial artist. 
She was one of the few whom \Quessanth considered his superiors in the martial arts, before her death. 
She died before \Quessanth felt he had surpassed her\dash indeed, Quessanth believed he only surpassed her after his Exaltation.

She never became Exalted.
















\section{\ZeethanKraal}
\target{Zeethan Kraal}
\index{\ZeethanKraal}
\ZeethanKraal was an \ophidian archmage that lived in the age of the \thirdbanewar. 
He ruled a city. 

Compare him to Nehkt Semerkeht from \cite{RobertEHoward:NehktSemerkeht}. 









\subsection{History}
\ZeethanKraal was born in the year \yic{Zeethan Kraal birth}.









\subsection{Personality}





\subsubsection{Studies}
\Kraal studied \quo{exceptional individuals} or \quo{heroes}\dash special people who were suspected to possess free will. 
\Kraal was convinced that he himself was \emph{not} such an individual, but he strongly suspected that \CarzainShachar was one. 
That was why \Kraal wanted to have \Shachar as his apprentice. 
\Kraal based his research on \hr{Sethicus and free will}{that of \Sethicus}. 









\subsection{Politics}





\subsubsection{\Caisith Autocracy}
\Kraal was a member of the \hr{Caisith Autocracy}{\Caisith Autocracy}. 





\subsubsection{Carzain Shachar}

\hr{Carzain}{Carzain \Shachar} knew \ZeethanKraal and had dealings with him. 
The two were occasional allies.
\Shachar once pretended to team up with a band of heroes who tried to slay \Kraal, only to betray them and help his ally \Kraal kill the heroes.

\Kraal taught Carzain many things, but he also kept much secret. 






























