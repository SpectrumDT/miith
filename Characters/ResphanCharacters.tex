
\part{\Resphan Characters}























\chapter{\CiriathSepher}















\section{\Azraid}
\target{Azraid}
\target{Gevural}
\target{Gepheral}
\index{\Azraid}
\index{\Gevural}
\Azraid was a \sathariah \resphan and one of the original \hr{Resphan rebellion}{\resphan{} rebels}. 
He was the \hr{High Lord of Kiriath-Sepher}{High Lord} of \hr{CS}{\CiriathSepher}, and a Cabalist of the \hr{Cabalist circles}{first circle}. 









\subsection{Equipment}





\subsubsection{Home}
\Azraid dwelt in the tower of \hr{Shaiphon}{\Shaiphon}. 





\subsubsection{Spear}
Maybe he wields a spear. 

\lyricsbalsagoth{When Rides the Scion of the Storms}{
  I see him... \\
  grim and noble astride his great winged steed, \\
  gleaming spear crackling in his grasp, \\
  beckoning me onwards to the next life... \\
  to ever more slaughter and carnage... \\
  Yes, a dour and brooding spirit he is, \\
  and in his burning eyes I see \\
  a great secret which I must discover,\\ 
  a powerful mystery I alone must solve.
}










\subsection{History}





\subsubsection{Name}
\Azraid was originally named \Gevural.
He forsook that name and took instead the name \Azraid, which means... something... in the High \Resphan{} tongue. 
(His two names are inspired, respectively, by the \Sephirah{} Geburah/Gevurah and the angel Azrael.)






\subsubsection{Priest}
Before the \hs{Delving}, young \Gevural{} was a \hs{theurge}-in-training. 
He was very much into philosophy and theology. 
He was a scientist and thinker, eager to learn everything and very fascinated by sorcery (especially \hr{Merkyran sorcery taboo}{that which was taboo}).

He was extremely intrigued when they discovered \Semiza{} and heard his story. 
\Gevural{} became just as obsessed with the darkness and the \banes{} as \Zachirah. 
But unlike \Zachirah, \Gevural{} was always very rational and critical. 

\citebandsong{DeathspellOmega:Kenose}{%
  DeathspellOmega
}{
  \Kenose
}{
  The pursuit of perversity, is it not but a mask\\
  on the search for meaning and knowledge?
}





\subsubsection{Slays \Damiarch and becomes \Azraid}
During the rebellion of \Merkyrah, \Gepheral \hr{Azraid coup}{slew his brother \Damiarch and became \Azeraid}. 





\subsubsection{Remained loyal}
After the \hr{Shrouding}{\Shrouding}, the \resphain{} \hr{Dynasties split}{splintered into dynasties}, and many fled from the alliance with the \banelords. 
\Azraid{} was one of few \resphan{} lords who remained \quo{loyal}. 





\subsubsection{\NeoResphain}
\Azraid{} was experimenting to create the new \hr{Neo-Resphan}{\neoresphain}: 
A stronger, more pure breed of \resphan; a superior merger of Chaotic and Entropic power. 

\target{Azraid never became Neo-Resphan}
\Azraid led and oversaw the \neoresphan project, but he never underwent the treatment himself. 
He took no chances. 
His life was too valuable to risk.
He needed to be there to stop the \banes.
\Azraid was very unlike \Sethicus, who was a brave pioneer who always went further than any other. 

\target{Azraid turns Malachim into Neo}
\Azraid also \hr{Azraid masterminds Malachim}{monitored the \malach project}.
It gave him much valuable understanding which he could use in implementing his own \neoresphan project. 
Also, he had a plan to turn the awakening \malachim into \neoresphain. 
He \hr{Ramiel becomes Neo}{eventually succeeded with Ramiel}. 









\subsection{Personality}





\subsubsection{Anti-\bane{} monologue}
\begin{prose}
  \ta{%
    The \banes{} call us their heirs, but they have never let us inherit anything. 
    They still use us as their servitors. 
    This is not betrayal. 
    I simply take what I was promised, what I am rightfully owed. 
    
    Despite how far we have come, we are still bound by the \banes{} and the Entropy and inevitable decline that they embody. 
    We must free ourselves from the \banelords{} in order to fullfill our potential. 
    In order to rise above their fate... and their damnation. 
    
    As for the \Voidbringer... I do not claim to understand its motives, but I very much doubt that the well-being of our people is a high priority in its mind.
    Ergo we must free ourselves from it.}
  
  \new
  \ta{%
    Of all \resphain, bar \Thanatzil, I stand closest to the \banes. 
    But I do not understand them.
    And that scares me.}
\end{prose}

\citebandsong{DeathspellOmega:FasIteMaledictiinIgnemAeternum}{%
  DeathspellOmega
}{
  A Chore for the Lost
}{
  God of terror, \\
  very low dost thou bring us, \\
  very low hast thou brought us...
}





\subsubsection{He is \ps{\Daggerrain}{} blind spot}
\Azraid{} is \hr{Daggerrain's blind spot}{\ps{\Daggerrain}{} blind spot}:
\Daggerrain{} saw weakness and transparency of the \nephilim, and later that of \humans, and so he underestimated the magnitude of raging inner conflicts that a \resphan{} can endure and still manage to conceal. 
(\Dragons{} do not hide their conflicts quite so well, because of their \chaotic{} nature.) 

The point is that \Daggerrain{} knew that Ramiel could not be trusted, and therefore took precautions against any betrayal by him. 
But \Daggerrain{} had never doubted \Azraid, the \psp{\banelords}{} most favoured and beloved son. 
\Azraid{} seemed almost a \banelord{} with a \human-like face, but he turned out to be much more treacherous, \chaotic{} and unpredictable than expected. 







\subsubsection{Obsession and pragmatism}
\Azraid{} is driven by a craving to understand and master Darkness... his own inner Darkness. 

He learns to love darkness, death, decay, pain and horror. He is a great artist, poet, philosopher and explorer.

\lyricsbalsagoth{The Power Cosmic: Epilogue}{
  Such power as was wielded by Zurra corrupted his heart, master.\\
  His quest for the Lexicon was not a desire born of the eternal search for cosmic enlightenment, but rather of a vain hope that such elucidation would allow him to understand the horrors which blighted his own immortal soul...
}

Among other things, he is obsessed with death. 
He is a necrophiliac and has built strange museums (musea?) filled with the dead, dying and undead. 
His palace is full of bones, corpses and pools of blood. 

He is fascinated by \Erebos{} and its decay and death... \hr{Erebos undead}{perhaps undeath}?

He seeks insight by challenging all standards of ethics and aesthetics and exploring all the forbidden alternatives.

As he likes to say: 
\ta{Beauty... and hideousness. In their exploration lies great wisdom.}

He admires \hr{Khoth-Sell}{\KhothSell} and her sons, because they, in some sense, embody death. 

\target{Azraid's pragmatism}
On the other hand, \Azraid{} can very pragmatic when it comes to other things. He does not really care about ideology, nor the \bane{} legacy, nor the war between \banes{} and \dragons. He only really cares about his own philosophy, and everything else is a chore to serve that end. This is why he \hr{Azraid adopts Merkyran imagery}{allowed \CiriathSepher{} to degenerate into something resembling \Merkyrah}, to which other \resphan{} lords objected. 

\lyricslimbonicart{The Black Hearts Nirvana}{
  Hallowed be the darkness that coronates my soul.\\
  Deep within its shelter I seek my highest goals.\\
  I shall release what is conquered \\
  from which that I now possess.\\
  All lifeforce is abandoned \\
  into the arms of death.
  
  Beyond the great dark adventures, \\
  in streams from the vast mysteries,\\
  limbonic spheres enclose me. \\
  My star is the death of memories.
}

\lyricslimbonicart{In Abhorrence Dementia}{
  I admire the spiritual force of evil:\\
  A pure supreme instinction in survival.\\
  Never underestimate the powers of hatred\\
  when the blackness overwhelming.
  
  With a hostile image against all living,\\
  the splendid visions of malignant breeding.
}

It is actually not just perversions, but a part of his millennia-spanning plan to overthrow the \banelords.

\lyricslimbonicart{The Black Hearts Nirvana}{
  Through lifetime I've reached for the candle\\
  in search for the legends of time.\\
  Cause how many moments isn't a century\\
  when everything dies behind the eyes?\\
  Cause how many moments isn't a century\\
  when everything inside just dies?
}

\lyricslimbonicart{Unleashed From Hell}{
  My art is a reflection, \\
  a mirror of tormented images.\\
  A labyrinth of morbid minds. \\
  Cathedral halls of stone.\\
  Abysmal ruins.
  
  Walking a path of putrefied flesh\\
  in the garden of rotting sculptures.\\
  Everything dwells in an aura of death\\
  and the presence of nocturnal vultures.
}







\subsubsection{Politeness}
\Azraid{} was mostly very calm and polite. 
Unlike many \resphain, he rarely resorted to insults and direct threats. 
He would be very nice and friendly to other \resphain{} and calmly \quo{request} things of them instead of rudely ordering them to.

Of course, everyone knew how powerful he was and how much he could fuck them over if they refused him. 
But even so, his good manners made him somewhat more likable, and he gained fewer enemies than he might have had he been more brutal. 





\subsubsection{Sex}
\Azraid{} does not have as much sex as most \resphain{}. 
Sex is life, after all, and \Azraid{} is more preoccupied with death. 

When he does have sex, necrophilia is among his perversions. 
Also sado-masochism and vore, but those are pretty common among the \resphain.  

He still has sex regularly, though. 
As the High Lord and the greatest of all \resphain, it is his duty to do his best to breed. 
But he has little success. 
He is infertile. 
His transformation into a freak with an \hr{dead hand}{evil hand} has harmed his fertility. 

He did manage to sire \ps{\Zereth} (\emph{after} he had become a freak), but that was it. 
He has never been able to produce another child. 

But he keeps trying. 
Not because he enjoys it. 
Indeed, he is quite bored with sex and has more important things to consider. 
But because it is his duty. 
He does want more children, after all. 

Once every two or three nights, he has a \resvil{} sent to his bed. 
There is no end to the number of \resviel{} willing to stand in line and wait to be allowed to have sight with the great \Azraid. 
He has servants charged with selecting the best and most fertile of them. 









\subsection{Physique}
\target{Azraid's appearance}
He has long white hair that hangs down in wisps. 
Almost aethereal-looking. 
Or like a spider's web. 
His feathers are a very pale gray. 
His skin is a bit pale, too.

He bears a white cloak that hides and covers his \hr{dead hand}{dead left hand}. 

He seems to shine with a light that somehow Shrouds everything else in darkness, making \Azraid{} the only thing that is bright and visible. 

He is not tall, only 200 cm, but his aura and charisma make him look taller. 

He looks weird. 
His skin has wrinkles, like the ones \nephilim{} and \humans{} are prone to get when they grow old. 
\Resphain{} don't develop wrinkles under natural circumstances. 
It is a side effect of the transformation that also gave him his \hr{dead hand}{evil hand}. 
The wrinkles are freakish, but somehow they don't make him seem like a weak mortal. 
Rather, they convey a sort of otherworldly dignity and authority. 
It is a proof of his macrocosmic wisdom and experience. 





\subsubsection{Dead hand}
\target{dead hand}
%He has a rotten, dead
His left hand is rotten and dead, and oversized like a bird's talon. 
Within it is contained much of his arcane power, which is based on death and decay. 
Compare with \hr{Entropy}{the \ps{\banes}{} Entropy}.
It is sickly brown in \colour. 

\Azraid keeps the hand hidden at most times so he can unveil it for greater effect at dramatically opportune moments. 

Remember to have a cool scene somewhere in the background story. 
\Azraid{} is empowered by an occult ritual and gains his dead hand. 
He also gains power over \hs{wraiths}. 









\subsection{Politics}





\subsubsection{\Banes}
\target{Azraid hates Banes}
\target{Azraid plots against Banes}
Ostensibly, \Azraid{} is very loyal to the \banes. 
He has resisted falling prey to \hr{Curse}{\NexagglachelsCurse}. 

But unbeknownst to everyone, \Azraid{} has secretly hated the \banelords{} and plotted against them for thousands of years, ever since they manipulated him into \hr{Azraid kills Damiarch}{killing his brother, \Damiarch}. 

He has spent his entire career planning and preparing so that he might eventually betray the \banes. 
His life and all his obsessions are built up around this. 

He has worked to forge a new power source for their people, independent of \Erebos. 
He intends to usurp and conquer \Nyx{} and sever it from \Erebos{} forever. 

He fears that the \banes{} will send another invasion force. 
But he hopes that if he free \Miith{} of \Iquin{} and \Daggerrain, then his people will have couple hundred or thousand years to evolve and grow stronger, so they will be able to stand against a new invasion. 

Near the end of \SentinelsofMiithEmph, it turns out that \Azraid{} has been \hr{Azraid stockpiles Erebean power}{stockpiling \Erebean{} power} for himself.





\subsubsection{\CiriathSepher}
\target{Azraid and Ciriath-Sepher}
The \CiriathSepher did not always love \Azraid.
But he was an extremely powerful and skilled \sathariah. 
Even the other dynasties acknowledged that and respected and even feared \Azraid for it.
So the \CiriathSepher were glad to have \Azraid.
He gave them prestige in front of the other dynasties.

\target{Azraid obscure}
\ps{\Azraid} role in the dynasty was rather obscure and unclear. 
This was intentional on his part. 
He liked to keep to the shadows and only exercise his power subtly. 
He played the part of the absent philosopher and artist rather than the active ruler. 

This had several advantages:

\begin{enumerate}
  \item 
    That way, he did not appear as a tyrant or a threat to anyone, so he made fewer enemies. 
  \item 
    By staying out of public sight he ensured that his enemies had little hold on him. 
    Little was known of his comings and goings, nor of his strengths and weaknesses, so those who wanted to attack him did not know how to get at him. 
\end{enumerate}

This \quo{keeping to the shadows} would never have worked in \Mystraacht. 
They would quickly lose respect for him and depose him. 
But it worked in \CiriathSepher{} because of his great social skills, the way he was able to pull strings and exercise subtle influence, and constantly keep up an image of mysterious power and allure. 
The sophisticated and artistic \CiriathSepher{} respected him for it. 





\subsubsection{Family}
\Azraid{} had one child: \hr{Zereth}{\Zereth}. 
He did not like to share power and was hesitant to spread his lifeforce around. 







\subsubsection{\Nexagglachel}
\Azraid{} was always the one of the \satharioth{} who understood \hr{Nexagglachel}{\Nexagglachel} the best. 
To this day he hears the voice of \Nexagglachel{} in his head. 
And to this day he needs to remind himself that it is not real. 
\Nexagglachel{} is dead. 
\Azraid{} is talking to himself, to a fragment of \ps{\Nexagglachel}{} memory and will that lingers on as a ghost in his mind ever since he \hr{Fragments of Nexagglachel}{consumed his brain}. 









\subsection{Skills and powers}





\subsubsection{High Telepath}
\Azraid{} is a highly skilled \hs{High Telepath}. 















\section[Firaxel]{\Firaxel}
\target{Firaxel}
\index{\Firaxel}
A \ketheran{} \resvil. 
\Teshrial{} was infatuated with her and wanted to have sex with her. 









\subsection{Physique}
She painted her face: 
Wavy white lines, like light or fire radiating out from her lovely, beautiful face. 

Sometimes she would treat her hair with some stuff so it would rise up in the air and then fall down, like a fountain. 

Her skin would be polished to blank onyx. 

Her hair and feathers were reddish brown.
She would often dye it purple. 
She disliked her natural \colour. 

She dressed in blue and violet in a manner reminiscent of and incorporating influences of \TiphredSerah. 
She was, after all, partially of \TiphredSerah{} descent. 









\subsection{History}





\subsubsection{Scientific career}
\target{Firaxel is a scientist}
\Firaxel{} is a scientist and researcher. 









\subsection{Politics}
\subsubsection{Family}
\Firaxel{} was descended from \hr{Shehizol}{\Shehizol} and \hr{Quelthah}{\Quelthah}, both high-ass \ketherain.
She was high status. 















\section{\Ganethed}
\target{Ganethed}
\index{\Ganethed}
A \thelyad{} \resphan{} of \CiriathSepher{} and a Cabalist of the \teshrialcircle circle. 
A colleague and rival of \Teshrial. 









\subsection{Physique}
\index{beard!\Ganethed}
\Ganethed{} was quite broad and stocky and powerfully built\dash a throwback to a more \nephil-like body form. 
Strong and muscular, but slightly fat. 
He wore a beard, unusual for \resphain. 

He wore his hair short. 

His hair and feathers were brown. 
He dyed them golden.









\subsection{Politics}





\subsubsection{\Urizeth}
\Ganethed was a kinsman of \hr{Urizeth}{\Urizeth}.















\section{\Harbeth}
\target{Harbeth}
\index{\Harbeth}
\Harbeth, called \quo{the Raven of the Battlefields}, was a \resphan{} lord, a \sathariah. 
He was charged with binding and \quo{safeguarding} the spirits of the fallen dead, protecting them from foreign powers who would steal them (such as the \hs{Worm Cult reapers}). 

\meta{%
  His name, title and image is inspired by the \Cabbalistic \Qliphah{} Hareb-Serap, and especially the portrayal of him in the RPG \emph{Kult}.} 









\subsection{History}





\subsubsection{Ethnicity}
\Harbeth was not of \Merkyrah.
He was a king of another \resphan nation.
When \CiriathSepher arose from the ashes of \Merkyrah, \Harbeth willingly allied himself with this new power. 
As reward he was made a prince of \CiriathSepher and became a confidante of \Azraid. 









\subsection{Politics}
\target{Harbeth is Azraid's heir}
In the \hr{CS order of succession}{\CiriathSepher{} order of succession}, \Harbeth{} was \ps{\Azraid} heir apparent. 
But \Harbeth{} was a brutal and scary hawk. 
Unpopular but feared. 
No one wanted him to be High Lord. 
So they preferred \Azraid. 
Even those who hated \Azraid{} tended to figure \quo{better the devil you know...}. 

This was a deliberate ploy from \ps{\Azraid} side. 
\Harbeth{} was a close and trusted ally of his. 









\subsection{Physique}
\Harbeth{} was extremely tall\dash 260 cm, one of the tallest of all \resphain.
He was also broad-shouldered.
But very gaunt, almost skeletal. 
He looked kind of like a living corpse. 

His hair and feathers were a drab gray.
He never dyed them.















\section{\Jeshred}
\target{Jeshred}
\index{\Jeshred} 
A \thelyad{} \resvil{} of \CiriathSepher.
Mother of \Zereth. 















\section{\Mehaloch}
\target{Mehaloch}
\index{\Mehaloch}
A \sathariah of \CiriathSepher. 
\Mehaloch was a cruel \quo{devourer}.
He killed and destroyed a lot in his hunger.
He was killed early on, during the \secondbanewar.
Compare to Darth Nihilus from \cite{VideoGame:KOTORII}.















\section{\Menessiaraid}
\target{Menessiaraid}
\index{\Menessiaraid}
A \ketheran{} \resphan{} of \CiriathSepher. 
He was a friend of \hr{Teshrial}{\Teshrial}; an older, more experienced \emph{sempai}-type. 
He had once been \ps{\Teshrial} teacher. 









\subsection{Equipment}





\subsubsection{Home}
\Menessiaraid dwelt in the tower of \hr{Tebethal}{\Tebethal}. 









\subsection{Physique}
\Menessiaraid{} was bald. 
He was as tall as \Teshrial, but more gaunt and less attractive. 

His feathers were pure black.
He dyed them with patterns of white amid the black. 









\subsection{Personality}





\subsubsection{Religions}
\target{Menessiaraid creates religions}
\Menessiaraid worked on fabricating and maintaining religions, dogma and mythology. 
But not on \Azmith. 
Have references to this. 
Make the reader be critical of religions. 















\section{\Morcariel}
\target{Morcariel}
\index{\Morcariel}
\Morcariel was a \sathariah \resphan of \CiriathSepher. 
He \hr{Morcariel betrays Damiarch}{betrayed and murdered \Damiarch} and \hr{Morcariel reigns}{made himself High Lord of \CiriathSepher}. 
Later \hr{Morcariel dies}{he was destroyed by \Secherdamon} during the \hs{Incursion}.

\Morcariel was pure evil. 
A \trope{CompleteMonster}{Complete Monster}. 
Compare him to the villains from the movie \cite{Movie:Avatar}. 















\section{\Shehizol}
\target{Shehizol}
\index{\Shehizol}
A \resvil{} of \CiriathSepher, a \sathariah. 
Ancestor of \hr{Firaxel}{\Firaxel}. 















\section{\Teshrial}
\target{Teshrial}
\index{\Teshrial}
A \resphan{} lord of the \CiriathSepher. 

He is the superior of \Achsah{} and Charcoal and oversees many affairs in \Malcur, \Scyrum{} and their general area. He is a Cabalist of the \hr{Cabalist circles}{third circle}. 








\subsection{Physique}





\subsubsection{Appearance}
\target{Teshrial's appearance}
\Teshrial{} had black skin, but he dressed in white.
His hair and feathers were a rather light gray.
He dyed them pure white. 

Maybe he wore all sorts of golden jewelry. 

%He carries a sceptre-like mace or morning star.
He carries a short rod that resembles a sceptre or mace. It can be used as such, but it is actually a \hr{Resphan technology}{high-tech weapon}, a gun of sorts. 

He wears a beautiful, crown-like circlet and a white robe adorned with gold, silver, obsidian, jade, onyx and so forth. 

\target{Teshrial is androgynous}
He has a somewhat androgynous \quo{\bishounen{} look}

His eyes are bright purple, close to pink. 
They are pretty, but he finds them a bit too effeminate. 
Maybe he wears some makeup to try and make the \colour look better. 









\subsection{Equipment}




\subsubsection{Crystal \armour}
\Teshrial{} wears \armour and wields weapons \hr{Resphan crystals}{made of crystal}. 





\subsubsection{Home}
\Teshrial dwelt in the tower of \hr{Tebethal}{\Tebethal}. 





\subsubsection{\Ruishagh: His demesne and manor}
\target{Ruishagh}
\index{\Ruishagh}
\Teshrial{} owns the estate of \Ruishagh (outside \Nyx). 
It has a central house, \Ruishagh{} Manor. 





\subsubsection{Weapons}
\target{Turishah}
\index{\Turishah}
At first, \Teshrial{} wielded the \hr{Senaan}{\senaan} \Turishah, forged specifically for him. 
It had a silvery blade. 
But this weapon was destroyed by \QuessanthIshnaruchaefir{} in his first battle with \Teshrial. 

\target{Ossiraith}
\index{\Ossiraith}
Later \Teshrial{} was entrusted with \Ossiraith, previously wielded by \Menessiaraid. 
\Ossiraith{} was green. 









\subsection{History}
\target{Teshrial's history}





\subsubsection{Origin}
\Teshrial{} was a \ketheran. 
\hr{Teshrial's family}{His family} were some very high-ranked \resphain. 

He is much younger than \hr{Achsah}{\Achsah}, but higher ranked in the Cabal than she. 
She \hr{Achsah hates Teshrial}{hates him for it}.





\subsubsection{Kills \Zessuruch}
\target{Teshrial kills Zessuruch}
At some point before the beginning of \emph{\TwilightAngelRemember{}}, \Teshrial and some of his brethren battled the \dragon \hr{Zessuruch}{\Zessuruch} and slew her. 
Not permanently, though. 

This was a great accomplishment for \Teshrial, who proudly styled himself \quo{\Dragon-slayer} after this. 





\subsubsection{Failures}
\target{Teshrial's failure}
After participating in the slaying of \Zessuruch, \Teshrial was popular. 
He was a rising star and thought he was well on the way to becoming the great hero of \CiriathSepher that he had trained and hoped to become all his life. 
But then things started to go wrong for him. 
His missions failed, he was beaten badly and had to flee many times. 
He was ridiculed and lost status. 

He had been well on his way to scoring \Firaxel, but now he was losing her. 
This was horrible for him.
He had tasted fame and power and popularity and had become addicted.
Now he wanted it back, and he was willing to take terrible risks to do that.






\subsubsection{Unfinished business}
\target{Teshrial's unfinished business}
In the years before the \thirdbanewar, \Teshrial wooed \hr{Firaxel}{\Firaxel}. 
He wanted to score her and have children with her. 
He gradually fell more in love with her and became more obsessed. 

\Firaxel showed signs of interest. 
They even kissed and cuddled. 
But \Firaxel was unwilling to make any promises.
She was not sure she wanted children with him. 

There was a rival whom \Firaxel was having sex with.
\Teshrial was worried about him.
He hoped that if he defeated \Ishnaruchaefir and advanced science in the process, \Firaxel would leave the rival and have children with \Teshrial.
He tried to coax some kind of reassurance out of \Firaxel, but she remained \blase and would promise nothing.
At the end, \Teshrial was hopeful but not certain. 

This is a dramatic reason for this:
If I want the reader to miss \Teshrial when he dies, he must have some unfinished business.
That way, there is tension because the reader expects that \Teshrial's life story will continue beyond his duel with \Ishnaruchaefir, because he has other conflicts he still needs to resolve. 
So it will be more of a surprise when he dies. 

Also, \Teshrial should have a personal development that is incomplete. 
He is slowly becoming less of a snob.
He learns to appreciate \Urizeth's friendship and even respects \Achsah. 





\subsubsection{Death}
Near the end of \emph{\TwilightAngelRemember{}}, \Teshrial{} was \hr{Ishnaruchaefir kills Teshrial}{killed and destroyed by \Ishnaruchaefir}.









\subsection{Personality}





\subsubsection{Culture}
\Teshrial{} knows about \hr{Bloodwine}{bloodwine} because he is supposed to as a noble. 
In secret, he prefers drinking raw blood from the vein. 
He knows this is uncultured, so he keeps it secret. 





\subsubsection{Motivation}
\hr{Teshrial's family}{His family} are some very high-ranked \resphain. 
As such, he has great pressure upon him to do great deeds. 
He wants to prove to the world that he is great, just as great as his family. 
This makes him zealous and impatient and overconfident. 

But he has also trained all his life and is psycho-skilled. 
The battle with \Ishnaruchaefir{} means everything to him. 
He cannot and must not lose. 

Make it clear that \ps{\Teshrial} motivation for fighting \Ishnaruchaefir{} and wooing \Firaxel{} is not purely selfish. 
He has many reasons for doing what he does:

\begin{enumerate}
  \item 
    He wants to gain sexiness in the eyes of the \resviel. 
    \Firaxel{} first of all.
  \item 
    He wants to win glory for himself and gain status and rank in the dynasty and the admiration of his fellow \resphain. 
  \item 
    He wants revenge for his humiliating defeat. 
  \item 
    But he also wants to do heroic things because he genuinely thinks they are right. 
    He wants to destroy \Ishnaruchaefir{} not just out of a personal vendetta, but because he is an evil menace\dash so evil and insane that even his own people, even his brother and daughter, revile him as anathema. 
    He wants to rid the world of the evil Destroyer, who has caused the Resphain and everyone else so much grief and harm. 
    And \Teshrial{} wants to avenge the many victims (warriors and civilians, mortal and immortal alike) whom the wicked Exile has slain over the millennia. 
  \item 
    He wants to have children, not just for his own vanity and legacy's sake, but because his race needs children. 
    He genuinely thinks he has good genes, and it is his duty to carry them on. 
\end{enumerate}

His hatred of \Ishnaruchaefir is as exhilarating and pleasurable as is his love for \Firaxel. 
Remember, \hr{Races love war}{violence is pleasure}.





\subsubsection{Sexuality and procreation}
\target{Teshrial's virginity}
\Teshrial{} has had sex with many people; male and female, \ketherain and \thelyadeth{}, \humans{} and \nephilim. 
Even an \ashenblood{} \resvil{} once. 
(He is ashamed of that last one and hides it, hoping no one will find out.)

But never a truly high-quality \ketheran{} \resvil. 
And \quo{high-quality} means \emph{fertile}. 
All the \resviel{} he has fucked have been infertile. 

\target{Teshrial wants children}
That pains him. 
He wants children. 
He hasn't been able to sire any yet, but through no fault of his own. 
He is certain that his seed is powerful and can sire many \resphan{} children. 
Strong children. 
\Ketheran{} children. 
He is sure he would be a great father, too. 
He would love his children and bring them up to achieve great and wonderful things. 

He is certain it is not his fault. 
He is fertile. 
He has impregnated several \human{} women (and eaten them) in his short life. 

But all the \resviel{} he has fucked have been barren. 
He knows because he has researched them. 
Of all those he has fucked, none has ever given birth, or even conceived. 
It is their fault. 
He is sure it is. 
\begin{prose}
\tho{Bah. 
  Fucking an infertile \resvil{} is just one step up from masturbating.} 
\end{prose}

\ps{\Teshrial} seed deserves better. 
He deserves someone great. 
A \ketheran, and one as talented and beautiful as he. 
He deserves \hr{Firaxel}{\Firaxel}. 
She is a miracle. 
Fertile. 
Full of life. 
She has given birth twice already. 
That is what makes her so sexy and irresistible. 
It also means she has suitors everywhere. 
\Teshrial{} knows this. 
He must do something great to be worthy of her. 







\subsection{Politics}
\subsubsection{Family}
\target{Teshrial's family}
\ps{\Teshrial} were some very high-ranked \resphain. 
His father was \hr{Tuerdal}{\Tuerdal}, a hero of the \hr{Kezeradi War}{\Kezeradi{} War}. 
His mother was \Zereth, the daughter of mighty \hr{Azraid}{\Azraid}. 





\subsubsection{\Ghobaleth}
\Teshrial{} had an ace up his sleeve: 
\hr{Teshrial's creatures}{A group of terrible \ghobaleth{} hiding beneath \Malcur.}

\target{Teshrial fears Noggyaleth}
\Teshrial was afraid of the \noggyaleth. 
Every time \Teshrial saw the \noggyaleth (or even thought of them), he was filled with fear and loathing. 
He was not ashamed to admit he fears them. 
Admit it to himself, that is.
He was still too ashamed to admit it in front of others.

In the \Malcur venture, it was \Urizeth who usually performed the spells to direct the \noggyaleth.
\Teshrial steered clear of learning those spells. 
Ostensibly because he thought it was nerd business and unsuitable for a \resphan like him; a gentleman of the finest breeding and the finest culture, and a master of martial arts, that noblest and most beautiful and artistic of pursuits. 
He used to joke that dealing with slimy, icky things like \noggyaleth was not his style. 
But the truth was that \Teshrial had always been afraid of \noggyaleth and unwilling to admit it. 

Later he was forced to learn the \noggyal spells. 




\subsubsection{Status}
\Teshrial{} is of the finest, most prestigious bloodline (that of \Azraid).
He is young and not highly experienced, but he is a rising star and already held in high regard. 
The dynasty has high hopes for him. 
He is expected to go on to do great heroic deeds in his life. 







\subsection{Skills and powers}





\subsubsection{Fighting skill}
\target{Weapon master Teshrial}
\Teshrial{} was an expert weapons master with both melee weapons, firearms and magical devices. 
He walked the \hs{Path of Ice}. 

\Teshrial was a rare prodigy.
Before his death he was young, but a better warrior than many \resphain much older than he.

\hr{Menessiaraid}{\Menessiaraid} once told him: 
\begin{prose}
  \Menessiaraid:
  \ta{%
    You are better than I, Teshrial. True, I win about half of our matches, but only because we have fought thousands of times and I know you so well. I have some chance to predict your style. No outsider could do that.}
\end{prose}

Compare him to Lucius from the \emph{Horus Heresy} books. 

\ps{\Teshrial} specialty is duelling, ie., single combat. 
Few \resphain{} can match him one-on-one. 

On the other hand, \Teshrial{} is not so great a commander or researcher or philosopher, or an enchanter of items or anything. 
He is also not a great \human{} breeder. 
He is proud of the \humans{} he breeds, but the breeding business is more a hobby than a profession. 

He wields a \hr{Senaan}{\senaan}. 





\subsubsection{Foreign languages}
\target{Teshrial speaks poor Draconic}
\hr{Resphain speak poor Draconic}{Like most \resphain}, \Teshrial spoke \Draconic only poorly. 















\section{\Tuerdal}
\target{Tuerdal}
\index{\Tuerdal}
A \ketheran{} \resphan{} of \CiriathSepher. 
\ps{\Teshrial} father. 
A hero of the \hr{Kezeradi War}{\Kezeradi{} War}. 















\section{\Urizeth}
\target{Urizeth}
\index{\Urizeth}
A \thelyad{} \resvil{} of \CiriathSepher. 
A historian. 

She compiled a big annotated \emph{\hr{Wanderers in Darkness}{\WanderersInDarkness}} that gathered several versions in one and compared them in detail, with analysis and interpretation. 









\subsection{History}





\subsubsection{Against \Ishnaruchaefir}
In \TwilightAngelRememberEmph, \Urizeth agreed to help \Teshrial in his quest to defeat \QuessanthIshnaruchaefir

\Urizeth was \hr{Ishnaruchaefir kills Urizeth}{killed by \Ishnaruchaefir} after he found out. 

\Urizeth's body was not destroyed.
It was retrieved by her family or friends, but it was blasted and burnt and in a bad shape.
When she revived she was hairless and feathless and horribly scarred. 
Her limbs were destroyed. 
She could neither walk nor fly. 

After a while, she regained the use of her hands to some extent.
But she still had to have servants assist her if she wanted to read or anything. 

She gradually recovered. 
Near the end of \TwilightAngelRememberEmph, \Urizeth could walk again, albeit unsteadily.
She had not many feathers yet and could not fly, but that would come.









\subsection{Personality}





\subsubsection{Eccentricity}
\target{Urizeth is eccentric}
Make \Urizeth really eccentric, even disturbingly so.
She is quite deranged after having studied WID and its terrible Aenigmata for so long.
\Teshrial is uncomfortable in her company, but he also knew that she was very knowledgeable and that her dark arts and dark insights would be of invaluable help to him.
After her death and imperfect rebirth she looked even more like a madman. 
Complete with mad cackling.
Maybe.

\Urizeth must have a nicely quirky and eccentric personality.
She is a nerd and does not have \Teshrial's social status or social skills.
She is somewhat mad, but she does have a strong will. 
She is older, wiser and more experienced than he and will not let him push her around.
She tries to be a wise mentor to \Teshrial, but does not succeed.
\Teshrial, on the other hand, knows she is a strange nerd.
He fears to have his reputation tainted by associating with her.
She is not only a \thelyad of TiphredSerah, but also very uncool.
\Teshrial is a huge snob, and so is his circle of friends.
He has to remind himself that working with this freak will pay off in the end.
But later in the story, he develops a genuine respect for \Urizeth, and later still even fondness.
She mourns him when he dies.









\subsection{Politics}





\subsubsection{Cabal}
\target{Urizeth is not a Cabalist}
\Urizeth was not a Cabalist at all. 
She was hired for the \hr{Malcur venture}{\Malcur venture} as an external consultant because of her great occult expertise. 





\subsubsection{\Ganethed}
\Urizeth was a kinswoman of \hr{Ganethed}{\Ganethed}.















\section{\Vesrai}
\target{Vesrai}
\index{\Vesrai}
A \ketheran{} \resvil{} of \CiriathSepher. 
Daughter of \hr{Zereth}{\Zereth}. 
% Mother of \hr{Teshrial}{\Teshrial}. 















\section{\Zereth}
\target{Zereth}
\index{\Zereth}
A \ketheran{} \resvil{} of \CiriathSepher. 
The only child of \hr{Azraid}{\Azraid}.
Her mother was \hr{Jeshred}{\Jeshred}. 

Had a number of children, including \hr{Teshrial}{\Teshrial} (fathered by \hr{Tuerdal}{\Tuerdal}). 

She was one of the runners-up in the \hr{CS order of succession}{\CiriathSepher{} order of succession} (but not the first after \Azraid). 

She was one of her father's closest allies. 
She was not so much a warrior, but an expert in all sorts of social relation management. 
She was very popular in \CiriathSepher{}: 
The princess, loved by all. 























\chapter{\Kezerad}


















\section{\Essenai}
\target{Essenai}
\index{\Essenai}
\Essenai was a \sathariah \resvil of \Kezerad.
She was very wise and full of insight.
As an expert sorceress he had much occult, cosmic insight. 

She wrote a translation of \WanderersInDarknessEmph into the \Resphan tongue.

She died in the \hr{Fall of Kezerad}{fall of \Kezerad} and became a \sephirah.

\hr{Sithiyacaan loves Essenai}{\Sithiyacaan loved her} and mourned her greatly.


















\section{\Eryal}
\target{Eryal}
\index{\Eryal}
A \thelyad{} \resvil{} of \Kezerad. 
She became a \malach{} and incarnated as \hr{Silqua}{Silqua \Delaen}. 









\subsection{Arsenal}





\subsubsection{Binding souls}
\target{Eryal binding souls}
\hr{Malachim binding souls}{As a \malach}, \Eryal{} had a \sephirah-like power to bind souls to her. 
This ability was especially strong in her, despite her being a lowly \thelyad. 
This was because of her charming and attractive personality, which attracted people to her. 
The souls stayed bound to her almost because they \emph{wanted to}, because the \emph{loved} her. 
Almost. 

The \hr{Banelords wanted to use Eryal's Carcer}{\banelords{} wanted to use \ps{\Eryal} \carcer} as a blueprint for their giant \carcer, \iquin. 









\subsection{History}





\subsubsection{Lover of \Shiaraid}
\Shiaraid{} and \Aryal{} \hr{Shiaraid and Eryal lovers}{used to be lovers}. 





\subsubsection{Hesitant to rebel}
In the \Merkyran{} rebellion, \Eryal{} did not want to rebel.
She was optimistic and assumed the best about the \Merkyran{} system. 
She was unhappy and afraid when she heard that \Shiaraid{} wanted to rebel. 
She tried to talk \Shiaraid{} out of it. 
Some of the other rebels caught wind of this and recommended that \Eryal{} be killed so she couldn't inform on them. 
But \Shiaraid{} would not hear of it. 
She knew \Eryal{} would never betray her. 

After much work, \Shiaraid{} finally convinced/bullied \Eryal{} into joining her. 
\Eryal{} felt guilty about it. 
But she wanted to be with her lover, and this was the only way. 





\subsubsection{Role in \SentinelsofMiithEmph}
What is she doing now? 
Probably something to atone for \hr{Silqua blames herself}{the evil she blames herself for having done}.

Perhaps she has something to do with \Sithiyacaan. 
Perhaps she is a voice in his head, pleading and begging him to return and help the world, instead of just trying to forget everything. 

She is not incarnated again. 
She sleeps deeper than most \malachim. 
This is because she is more good and therefore has more traumata to repress. 









\subsection{Politics}





\subsubsection{Family}
\Eryal was blood-kin to \hr{Sithiyacaan}{\Sithiyacaan}. 
Perhaps even his daughter. 





\subsubsection{Ramiel}
\hr{Ramiel resented Eryal}{Ramiel resented \Eryal}. 















\section{\Sevestris}
\target{Sevestris}
\index{\Sevestris}
\Sevestris was a \thelyad \resvil of \Kezerad. 
She and \Sithiyacaan were close lovers. 
\Sevestris was taken captive by \Secherdamon, enslaved and used to animate one of his loathsome \reptilecolossi.
\Sithiyacaan and his allies later had to destroy her soul in order to kill the monster.















\section[Sithiyacan]{\Sithiyacaan}
\target{Sithiyacaan}
\index{\Sithiyacaan}
\target{Last Kezeradi prince}
\Sithiyacaan is a \resphan lord, a \sathariah and the last surviving \hr{Kezerad}{\Kezeradi} prince. 

He is a terribly grim warrior, fighting for good but with brutal means. 
He is cruel, driven by revenge. 

Compare to Silchas Ruin from \cite{StevenErikson:ReapersGale}.









\subsection{History}





\subsubsection{Rebellion}
When \hr{Azraid coup}{\Azraid{} does his coup}, \Sithiyacaan{} realizes that the rebels are evil. He speak out, but is put back in his place. He stays and tries to pull the \hr{Kiriath-Sepher founded}{newly-formed \CiriathSepher} in a better direction

He takes part in the slaying of \Nexagglachel{} and becomes a \sathariah. But soon after that, he breaks with the other \resphain{} and goes off to found \Kezerad.

After the fall of \Kezerad, he is extremely bitter and disillusioned. In order to sever his \hs{Kezeradi telepathy}, he has to brutally suppress his emotions and his memories of home. This has forced him to perfect the technique of suppressing his true self and disguising himself as a harmless \human, but it has also driven him somewhat mad. 

He was somehow and to some extent responsible for the fall of \Kezerad, and he feels endless guilt over this. This is part of what drives him mad.

\lyricsbalsagoth{The Scourge of the Fourth Celestial Host}{
  [NORRIN-RADD:]\\
  I am the last scion of Zenn-La,\\
  Never more to embrace Shalla-Bal.\\
  I was born to soar beyond the stars...
  
  The edge of oblivion beckons...\\
  The blood of countless billions stains these silvern hands...\\
  but I must... I will endure!
}

\lyricsbs{Aeternus}{Denial of Salvation}{
  For all the thousands I've killed,\\
  for all the children I've tortured,\\
  for all the souls I've burned,\\
  hear a demon's cry.
  
  The darkened lusts\\
  you carved into me\\
  suffocates the remains\\
  of my human sanity. \\
  Why me? \\
  This I never desired.
  
  See me.\\
  Free me.\\
  Unchain me.
  
  This life eternal\\
  is not what you spoke off.\\
  Why this horned, grim face,\\
  this cold, black body?\\
  I want to be set free.
  
  Save me.\\
  Free me.\\
  Unchain me.
  
  I am a demon.\\
  I serve the wicked.\\
  I'm trapped eternally.\\
  Now I must return\\
  to dimensions of\\
  rage, pain, war, hate.\\
  Where is my god?
}





\subsubsection{Family}
He is blood-kin to \hr{Eryal}{\Eryal}. 
Perhaps even her father. 





\subsubsection{\Sithiyacaan's revenge}
\target{Sithiyacaan attacks dynasties after Fall}
After the \hr{Fall of Kezerad}{Fall of \Kezerad}, \Sithiyacaan \hr{Sithiyacaan despairs after Fall}{despaired and almost broke down}. 

Later he learned that \hr{Essenai}{\Essenai} was not destroyed as he had thought, but condemned to a fate worse than death:
Forever cursed and transformed into a monster, an abomination. 
\Sithiyacaan flew into a rage and led desperate attacks against the dynasties an in attempt to destroy \iquin and save her. 
His mad attack failed and he was forced to flee. 
This failure and cowardice gave him extra guilt, which made him extra mad. 





\subsubsection{Back in the war}
At some point (during \SentinelsofMithEmph) he resolves to get involved in the \secretwar{} again. 
But he is still mad and ineffectual much of the time. 
In his \human{} guise, he has occasional glimpses of insight and \quo{wise}, \quo{prophetic} outbursts.

As the story progresses, he learns to better control his mind and feelings and sheds his feeble, maddened persona.

He once faces Ramiel in combat and proves his equal.

Compare him to Silchas Ruin from \cite{StevenEriksonIanCameronEsslemont:MalazanBookoftheFallen}. 
But with his madness and split personality, he might be more of a Silchas Ruin, Udinaas, Kettle and Rhulad all wrapped into one.

With his switching between ineffectual and badass forms, he resembles Abel Nightroad (Crusnik 02) in the anime \cite{Anime:TrinityBlood}.





\subsubsection{In \Redce}
At the beginning of the \thirdbanewar, \Sithiyacaan \hr{Sithiyacaan in Redce}{dwelt in hiding in \Redce} in the guise of \hr{Herette}{\MoriceHerette}. 





\subsubsection{Awakening}
Finally \hr{Sithiyacaan awakens}{\Sithiyacaan awakened}. 









\subsection{Physique}
\Sithiyacaan{} was one of the physically largest \satharioth. 
He was almost as tall as \Harbeth{} and \Zachirah. 

His feathers and hair were blood red.









\subsection{\MoriceHerette}
\target{Herette}
At the beginning of the \thirdbanewar, \Sithiyacaan \hr{Sithiyacaan in Redce}{dwelt in hiding in \Redce} in the guise of {\MoriceHerette}. 
\Sithiyacaan was a maddened, traumatized wretch, almost powerless. 

\Herette was a crazy old \human man who painted horrible pictures and made hideous, blasphemous sculptures. 
Compare to images of Yog-Sothoth, or the statues of Cthulhu in \cite{HPLovecraft:TheCallofCthulhu} and the Oriuagor figures in \cite{PalleVibe:OriuagorsProfeti}. 
These were inspired by his repressed memories of the fall of \Kezerad and all its horror, and of the lost \beacons who were now the awful \sephiroth.
Once in a while the Redcor would recognize elements in his art that resembled the \sephiroth\dash something that looked like a blasphemous parody but was actually awful truth. 
This infuriated them. 

Many Redcor wanted to punish \Herette as a heretic and blasphemer, but there were some \Kezeradi agents in the Redcor \hs{Conclave}.
These protected \Herette because he was occasionally lucid enough to help them, and they knew they would need his help in the time to come. 

No one knew how old \Herette was. 
He was never born as \human, of course. 
(\Sithiyacaan was not a \malach, merely a mad \resphan who had locked himself up in \human form.)









\subsection{Split personality}
He has developed a split personality of sorts, so that in his guise\dash typically that of a mad old beggar\dash he does not remember more than fragments of his true past. He also tends to babble incoherently and be plagued by nightmares, but very vague ones. 

\target{Sithiyacaan's appearance}
But he still retains his inherent \resphan{} power. Perhaps he is even a \hr{Sathariah}{\sathariah}. In a dire emergency, his memory will return, and he casts off his \human{} shell and transforms into his true form.





\subsubsection{Fights to regain sanity}
\Sithiyacaan{} fights to regain his sanity. 
But he fears to regain his powers, in case his madness should return. 

\citebandsong{Ihsahn:angL}{Ihsahn}{Elevator}{
  All lights disperse\\
  and the devil takes me down.
  
  There is panic in my fascination.\\
  Like soothing wine is my despair.\\
  Gracefully I fall to pieces.
  
  Then lights disperse\\
  and the devil takes me down.
  
  The gears keep turning\\
  and the ropes stretch far\\
  in this world of hopelessness.
}

But he is making progress.

\citebandsong{Ihsahn:angL}{Ihsahn}{Elevator}{
  I have come a long way now.\\
  The fatal riddles beckon me.\\
  I have come a long way now.\\
  A leap from faith and gravity.\\
  I have come a long way now.\\
  To find the nest where treasures sleep.\\
  I have come a long way now.\\
  The fairest lies are hidden in the deep.
}

He still angsts.

\citebandsong{Ihsahn:angL}{Ihsahn}{Elevator}{
  There is vanity in my destruction.\\
  There is mockery in my ordeal.\\
  Indefinite is the course of my decent.
}





\subsubsection{His true form}
His true form is that of a terrible fallen angel, radiating fell power and hatred. His wings are tattered and broken, his features harrowed, his figure emaciated, his face furrowed by lines of shed tears. 

His eyes are dry and dead, all tears of compassion shed and dried millennia ago. Or... maybe he sheds tears of blood\dash in real-time, running down from his eyes. Especially when he's in combat and forced to wield the \nieur-based power that is hateful to him.

Contrast him with \hr{Ramiel's appearance}{Ramiel}, who revels in his \resphan{} glory.

In this form, he is prone to be carried away by rage\dash the rage he otherwise so brutally suppresses\dash and wreak great destruction. When he returns to his senses, he is disgusted by what he has done, and it drives him to repress his true self even further. He hates the wretched horror he has become. 





\subsubsection{Imprisoned within himself}
The true \Sithiyacaan{} is incarcerated in a prison in a repressed section of his own mind. A personal Hell, a mini-Realm within himself. There he lies in torture and hates himself. 

Compare to the Seraph Inarius from the manual to the game \emph{Diablo}. He is imprisoned in a hall of mirrors and forced to gaze upon his own misshapen form for eternity. 

When from time to time he awakens, he is wrapped in chains. His weapons are blades on chains that he swings around. Compare to the {Ghost Rider} from comics and a movie (\cite{Movie:GhostRider}) of that title. 

When he awakens, he tears his \human{} body to pieces, blood and guts spraying everywhere. When he falls asleep again, the Shroud rebuilds his \human{} form, but he's not feeling well. 

He mentally flees from all the suffering, trying to block it out. 
But it still speaks to him. 

\lyricslimbonicart{When Mind and Flesh Depart}{
  A cryptic message comes from the heart\\
  as I see and experience the bleeding art.\\
  The pain is only to avoid\\
  'cause the spirit enters a greater void.
}

Once in a while he lets rip, releasing all the suffering contained within him. 

\lyricslimbonicart{When Mind and Flesh Depart}{
  I cleanse within ascending steams.\\
  Arising shadow, the dark soul releases\\
  all its power.
  
  When mind and flesh depart.
}

\lyricslimbonicart{Infernal Phantom Kingdom}{
  A grim darkened spirit\\
  in a world of woe.\\
  Imprisoned evil beauty\\
  from the cold depths below.\\
  Linger in perpetual dreamstate,\\
  in the grip of a powerful rage.
  
  Summon the oblivion.\\
  Hear demons call from the dungeon.\\
  Light has forever abandoned this land.\\
  Life has forsaken this souls.\\
  Reaching out from the cold.\\
  A dark and hellish void,\\
  beyond the entrance of imagination.
}









\subsection{Skills and powers}





\subsubsection{Power}
\Sithiyacaan{} is one of the mightiest \resphain{} in the world. 
Stronger even than most \satharioth. 
Up there with \hr{Azraid}{\Azraid} and \hs{Ramiel} (even after \hr{Ramiel is overpowered}{Ramiel becomes \uber{} after eating \Belzir}). 

This is because of \ps{\Sithiyacaan}{} \hr{Madness}{madness}, which grants him dark insight and thus dark powers. 
But he is very unstable and only rarely able to utilize his full powers. 
\trope{SanityHasAdvantages}{Sanity Has Advantages}. 





\subsubsection{Vampirism}
Having no feelings of his own (because he represses them so badly), he has learned to steal the emotions of others and feed on them. Recall that \hr{Resphan vampirism}{\resphain{} are vampiric by nature}. This leaves the victim with an uncanny feeling of emptiness, loss and pointlessness. 

In extreme cases, \Sithiyacaan{} over-feeds, draining all emotion from his victims, leaving them mad, crippled or dead. He is prone to doing this in his \quo{fallen angel} form. Perhaps he even drains their soul. 

When he does it in his true form, he tends to smile or laugh and lick his lips. Compare him to the Crusnik from \emph{Trinity Blood}, who also does this.

In rare cases he will use his emotion-draining skill as a weapon.









\subsection{Politics}





\subsubsection{\Essenai}
\target{Sithiyacaan loves Essenai}
\Sithiyacaan loved \Essenai. 
He mourned greatly when she was turned into a loathsome \sephirah. 
Freeing his beloved from her vile prison was a very important motivation for him. 























\chapter[Mystracht]{\Mystraacht}















\section{\Cishiel}
\target{Cishiel}
\index{\Cishiel}
A \ketheran{} \resvil{} of \Mystraacht. 
Daughter of \hs{Ramiel}. 









\subsection{History}





\subsubsection{Youth}
\Cishiel{} was born \emph{after} Ramiel became a \sathariah. 
(This meant that she was a proper \ketheran. Which she would not have been had she been born before he actually became a \sathariah.) 





\subsubsection{Ramiel disappeared}
She was very young when her father went missing as a \malach. 
She herself \hr{Cishiel wanted to be a Malach}{wanted to be a \malach}, but Ramiel forbade it. 

Now that her father was gone, there were rivals who wanted to kill her, since as Ramiel's sole heir she was a contender to the throne of \Mystraacht. 
But she was lucky to have some strong and resourceful family members who protected her while she grew up. 

She grew into a very competent and deadly \resvil. 
When \hr{Ramiel returns to Mystraacht}{Ramiel returned}, she supported him. 
But she did not forgive him for abandoning her when she was a little girl. 





\subsubsection{Cabal rank}
At the time of the \thirdbanewar, \Cishiel was a Cabalist of the \cishielcircle circle. 









\subsection{Physique}




\subsubsection{Appearance}
\Cishiel usually dyed her hair and feathers fiery orange. 









\subsection{Politics}





\subsubsection{Children}
\Cishiel had a son or even two.
They were young and had accomplished little.
The younger (if there were two) was just a child, 100 years old at the time of the \firstbanewar.





\subsubsection{\Eryal}
\target{Cishiel hates Eryal}
\Cishiel{} hated \Eryal. 
She blamed her for driving Ramiel and \Shiaraid{} apart, which lead to their downward spiral and drove them to their fall from grace \malach{} fiasco. 
If it were not for \Eryal, \Cishiel{} would be a highly respected princess of \Mystraacht. 
Instead she had to fought for everything she gained, including her life.





\subsubsection{Mother}
\Cishiel's mother was not \Shiaraid but some other \resvil.
It was in one of those many and long periods where Ramiel and \Shiaraid were not on speaking terms.
\Cishiel's mother was killed in the \resphanwars, after the \malach fiasco but before the inception of the Cabal.





\subsubsection{Ramiel}
Ramiel was \Cishiel's father. 

\ta{Ramiel and Cishiel have sex}{Ramiel and \Cishiel may have had an incestuous relationship}.








\subsection{Skills}





\subsubsection{Martial arts}
\Cishiel walked the \hs{Path of Darkness}.















\section{\Dasteron}
\target{Dasteron}
\index{\Dasteron}
A \ketheran{} \resphan{} of \Mystraacht{} who coveted the throne of the \hs{Overlord}. 
He was \hr{Dasteron dies}{ultimately slain} and \hr{soul-eating}{devoured} by Ramiel. 









\subsection{Equipment}





\subsubsection{\Scaleron}
\target{Scaleron}
\index{technology!weaponsmithing}
\Scaleron{} was a \hr{Belthrad}{\belthrad} sword forged and wielded by \Dasteron{} and later taken from him by Ramiel. 
It was red in \colour. 

\ps{\Scaleron} design was inspired by \hr{Ascaril}{\Ascaril}. 
\Dasteron{} had never seen \Ascaril, as \hr{Ascaril destroyed}{it was destroyed} before he was born, but he studied descriptions and depictions of it, and even some of \ps{\hr{Lyorith}{\Lyorith}} old schematics and design notes, which she used to forge \Ascaril. 

But \Dasteron{} believes that \Scaleron{} is superior to \Ascaril. 
It is, after all, made with superior technology. 









\subsection{History}





\subsubsection{Cabal rank}
At the time of the \thirdbanewar, \Dasteron was a Cabalist of the \dasteroncircle circle. 









\subsection{Physique}
\target{Dasteron's appearance}
\index{beard!\Dasteron}
\Dasteron{} is slightly taller than Ramiel, and somewhat broader, too. 
But Ramiel has a broader wingspan (\hr{Resphan wingspan is important}{which is important}). 

He is bald and bearded, looking a bit like Anton Szandor LaVey. 









\subsection{Politics}





\subsubsection{Family}
\Dasteron{} was the son of \hr{Ozariel}{\Ozariel}. 

\target{Dasteron's cousins}
He had two cousins: 
\Sargamel and \Themirod. 
Both \thelyadeth, kin of \ps{\Dasteron} mother. 
Both were powerful and influential \Mystraacht lords. 
They were his close allies, and he could never have \hr{Dasteron becomes Overlord}{become Overlord} without their loyal support. 
After Ramiel's takeover \hr{Dasteron's cousins oppose Ramiel}{they became his staunch rivals}. 









\subsection{Skills}





\subsubsection{Combat skill}
\target{Dasteron's skill}
\Dasteron{} was an extremely skilled warrior and mage. 
He had to become an \uber-skilled fighter in order to rise in the \Mystraacht{} ranks and \emph{almost} make himself Overlord. 
He did not have the raw strength of a \sathariah, so he trained to compensate. 

\target{Dasteron's upbringing}
One reason why \Dasteron{} was \hr{Dasteron stronger than Ramiel}{stronger than Ramiel} was that \Dasteron{} had lived his entire life as a \Mystraacht{} warriors. 
He has been raised \trope{TheSpartanWay}{The Spartan Way}. 

All his skill, all his power and all his fame were things he had \emph{earned} by hard work, not just gotten handed to him by virtue of birth or \sathariah{} power. 

\target{Dasteron's paths}
He was a rare expert who mastered all three \hs{Paths}: 
\hr{Path of Ice}{Ice}, \hr{Path of Light}{Light} and \hr{Path of Darkness}{Darkness}. 
He used mostly Light because he wanted to foster a very \Mystraacht{} macho-image. 
He kept his mastery of Ice and Darkness secret so he could use it as a surprise in combat. 





\subsubsection{Knight of the Void}
\Dasteron was a \hs{Knight of the Void}.





\subsubsection{Smithing}
\target{Dasteron's smithing}
Apart from powermongering, \ps{\Dasteron} greatest passion was weaponsmithing. 
He forged the sword \hr{Scaleron}{\Scaleron} and many other useful magical items, which he would bear and use in combat. 
He was \trope{CrazyPrepared}{Crazy Prepared}. 















\section{\Dezruth}
\target{Dezruth}
\index{\Dezruth}
A \thelyad{} \resphan{} of \Mystraacht. 
An acquaintance and ally of \hr{Teshrial}{\Teshrial}. 









\subsection{Appearance}
\target{Dezruth's appearance}
He had chestnut hair and a matching moustache. 















\section{\Gilchad}
\target{Gilchad}
\index{\Gilchad}
A \thelyad{} \resphan{} of \Mystraacht. 
An accomplice of \Cishiel. 

















\section{\Lyorith}
\target{Lyorith}
\index{\Lyorith}
A \resvil{} of \Merkyrah{} and later \Mystraacht. 
Lover of \Nathrach. 
Mother of Ramiel. 









\subsection{Arsenal}
\subsubsection{\Ascaril}
\target{Ascaril}
\index{\Ascaril}
\index{technology!weaponsmithing}
A \hr{Belthrad}{\belthrad} sword, forged by \Lyorith. 
It was a marvel, creating utilizing bits of \hr{Bane technology}{\bane{} (ie, \voyager) technology}. 

It was first wielded by \Lyorith. 
When she died \Nathrach{} inherited it. 
When \hr{Nathrach dies}{\Nathrach{} died} Ramiel inherited it. 
Ultimately \hr{Ascaril destroyed}{the sword was destroyed} when Ramiel fell. 





\subsubsection{Weaponsmithing}
\index{technology!weaponsmithing}
\Lyorith{} was a great artisan, sculptor and smith. 
When she joined the rebels against \Merkyrah, she developed a taste for violence and bent her passion towards weaponsmithing. 
\Semiza{} and the \banelords{} taught the \resphain{} some secrets of technology. 
Using these new techniques, \Lyorith{} forged many powerful weapons. 
The greatest of them was the sword \hr{Ascaril}{\Ascaril}, which she wielded. 

She also made \hs{Ramiel's guns}. 









\subsection{History}
\Lyorith{} was persuaded to join \Mystraacht{} with \Nathrach{} and Ramiel. 

She perished during the \Merkyran{} war. 















\section[Nathrach]{\Nathrach}
\target{Nathrach}
\index{\Nathrach}
A \resphan{} of \Mystraacht, a \sathariah. 
Killed near the end of the \secondbanewar. 

He was a close friend of \hr{Zachirah}{\Zachirah} and the father of \hs{Ramiel}. 









\subsection{Personality}





\subsubsection{Ambition for Ramiel}
He places great pressure on his son, Ramiel. 
He was a good and loving father, but also a stern one. 
He applauded Ramiel's achievements, but also kept pushing him on to new, greater achievements. 
He kept telling Ramiel: 
\ta{\hr{Ramiel can do better}{You can do better.}}





\subsubsection{Death}
\hr{Nathrach dies}{\Nathrach died} trying to become a \sathariah.















\section{\Netzach}
\target{Netzach}
\target{Zachirah}
\index{\Zachirah}
A \resphan{} of \hr{Merkyrah}{\Merkyrah}.
The father of \hr{Shiaraid}{\Shiaraid} and the mentor of \hs{Ramiel}. 
He was one of the \hr{Delving}{\Delvers} and a \hr{Sathariah}{\sathariah}. 
He became the founder and first Overlord of \hr{Mystraacht}{\Mystraacht}. 








\subsection{Physique}
\target{Zachirah's appearance}
\index{beard!\Zachirah}
\Zachirah{} was huge: 
250 cm tall or more, and very muscular. 
Almost as massive as a \nephil. 

He wore a beard. 

His hair and feathers were almost black, but with a hint of red or brown. 









\subsection{Arsenal}









\subsection{History}





\subsubsection{Ambitious warrior}
Young \Netzach was a proud and mighty warrior, and ambitious. 
He was the scion of a glorious line of heroes and \hr{Resphan Warlord}{Warlords}. 
He felt cheated of the glory and power that he rightfully deserved, by birthright and by virtue of his heroic deeds. 
He wanted more.
He wanted power and fame and greatness. 

His father, \hr{Shadrach}{\Sharrath}, raised \Netzach to be a great warrior and theurge. 
\Sharrath loved his son and rold \Netzach that he was the greatest \resphan in the world and would one day become greater even than his father. 
This filled young \Netzach with pride and dreams of grandeur.

He was drawn to black magic, for he had fought against sorcerers and \hr{Merkyran cannibalism taboo}{eaters} and idolaters and learned much about them.
Sometimes he had even fought alongside them against worse villains or monsters.
He had found the sorcerers to be strong, fascinating and worthy opponents. 

Compare him to Malekith from \cite{GavThorpe:Malekith}. 





\subsubsection{Slew \Sartheron}
\target{Netzach slays Sartheron}
Eventually the great hero \Netzach slew the dark sorcerer \hr{Sartheron}{\Sartheron}. 
In the process, \Netzach was forced to do dark and terrible things that hurt his reputation. 
This meant that \Netzach did not get the recognition he felt he had deserved.
He grew dark and bitter. 





\subsubsection{Slew Lothomog}
\Netzach slew the dark god \hs{Lothomog}, who had corrupted an entire \resphan hold and was spreading out its tentacles of power to bring down \Merkyrah. 
This was perhaps his greatest act of heroism. 
It might have saved \Merkyrah.
It made \Netzach a popular and beloved hero among the people.
But the theurges were not happy because \Netzach had used dark magic in the battle. 





\subsubsection{Dark mage}
Before the \hs{Delving}, \Zachirah had dabbled in black magic. 
He believed the church of \Merkyrah was narrow-minded and blind.
He explored darker powers and communed with darker gods, including various cosmic gods and the occasional \xs. 

He had forged a number of magical items which held great magical power and enabled him to do great feats of sorcery. 
But the church was onto him.
His forbidden research was discovered. 
His magical items were confiscated. 
He was nearly imprisoned (\hr{Merkyran prison}{which was the worst penalty in \Merkyrah}). 
But \Zachirah was sly and had good social skills and connections. 
He convinced his friends \Nathrach and \Damiarch and \Gevural to intercede on his behalf and vouch for him. 
His friends were very crafty \resphain.
They pulled some strings and got him acquitted. 

Then they went on the Delving together. 

Later \Zachirah got his rings back. 
He had four rings, each depicting an animal: 
The serpent, the spider, the scorpion and the bat. 

\citeauthorbook[p.14--16]{RobertEHoward:ThePhoenixontheSword}{Robert E. Howard}{%
  The Phoenix on the Sword%
}{
  I was a great sorcerer in the south. 
  Men spoke of Thoth-amon as they spoke of Rammon. 
  King Ctesphon of Stygia gave me great honor, casting down the magicians from the high places to exalt me above them. 
  They hated me, but they feared me, for I controlled beings from outside which came at my call and did my bidding. 
  By Set, mine enemy knew not the hour when he might awake at midnight to feel the taloned fingers of a nameless horror at his throat! 
  I did dark and terrible magic with the Serpent Ring of Set, which I found in a nighted tomb a league beneath the earth, forgotten before the first man crawled out of the slimy sea.
  
  \ldots 
  
  Thoth grasped the ring in both hands, his dark eyes blazing with a fearful avidness.
  
  \ta{My Ring!} he whispered in terrible exultation. \ta{My power!}
  
  How long he crouched over the baleful thing, motionless as a statue, drinking the evil aura of it into his dark soul, not even the Stygian knew. When he shook himself from his revery and drew back his mind from the nighted abysses where it had been questing, the moon was rising, casting long shadows across the smooth marble back of the garden-seat, at the foot of which sprawled the darker shadow which had been the lord of Attalus.
  
  \ldots 
  
  \ta{Blind your eyes, mystic serpent,} he chanted in a blood-freezing whisper. 
  \ta{Blind your eyes to the moonlight and open them on darker gulfs! 
    What do you see, oh serpent of Set? Whom do you call from the gulfs of the Night? Whose shadow falls on the waning Light? Call him to me, oh serpent of Set!}
   
  Stroking the scales with a peculiar circular motion of his fingers, a motion which always carried the fingers back to their starting place, his voice sank still lower as he whispered dark names and grisly incantations forgotten the world over save in the grim hinterlands of dark Stygia, where monstrous shapes move in the dusk of the tombs.
}





\subsubsection{Delving}
\Zachirah went on the Delving with \Damiarch-tachi.
He hoped to gain some valuable insight into the \umbrae.
He believed he could learn from them and utilize secrets gleaned from those terrible beings to gain power for himself. 





\subsubsection{\ps{\Semiza}{} revelations made him evil}
\hr{Semiza shows tailored visions}{\Semiza{} shows tailored visions} to each of the \hr{Delving}{\Delvers}. 
\Zachirah{} is shown things that make him evil and hateful. 

He immediately adores and worships the \banes. 

\lyricsbs{Hate Eternal}{Behold Judas}{
  As I stand before thee, Master of the Arcane.\\
  Lest we forget your burden, father.\\
  From beneath the binding \\
  of the irreverent one who is shroud in darkness.\\
  Revealed! Heathen to all that is sacred!
  
  I serve myself unto thee, Master of the Labyrinth.\\
  Succumb we must to this promised design.\\
  From within you suffer, \\
  your knowledge of your transgressions.
  
  I bow down before you, master of the kingdom,\\
  under the guise of a holy existence.\\
  From the depths of all time \\
  is the man who claims to be our savior.\\
  Emerged! Traitor to all that is holy!
}

He was originally a critic of the \Merkyran{} system. 
\Semiza{} recognizes his \skepticism and disgruntlement and uses it as a lever to twist \Zachirah{} and make him hate every aspect of \Merkyrah{} and everyone in it. 
He turns to the opposite of \Merkyran{} values (pacifism and stuff) out of pure spite and hate. 

Furthermore, \ps{\Zachirah}{} ambition for glory, \honour and achievement is twisted into a cruel lust for power. 

\lyricsdimmuborgir{D\o{}dsferd}{
  I d\o{}dsdalens \o{}de skj\o{}nnhet\\
  har min sjel vandret vill.\\
  Mens m\o{}rket og sorgen r\aa{}det\\
  tentes hatets flammende ild.
}





\subsubsection{Going mad}
\target{Zachirah goes mad}
He slowly goes mad. 

\lyricsbs{Hate Eternal}{Sacrilege of Hate}{
  Struggling with conflict \\
  over my deep-seeded hate. \\
  Striving with constant angst. \\
  Breed all of my pain. \\
  Breed all of my pain. 
  
  Fighting with anger.\\
  Overcoming all that is sane.\\
  Writhing to be ordained. 
  
  Feeding on grief. \\
  Needing pain ever so deep. \\
  Reasoning through rational. \\
  Breed all of my pain. \\
  Breed all of my pain. 
  
  Seeking my vengeance.\\
  Feeding on pity and empathy.\\
  Searching for infamy.\\
  Blessed with blasphemy.\\
  
  I live off all of the meek.\\
  Your feebleness so weak.\\
  With greed I solidify your fate.\\
  I'll bleed you of your hate. 
}

He is gradually corrupted. 

\lyricsbs{Vital Remains}{Sanctity In Blasphemous Ruin}{
  Buried beneath the centuries, \\
  memories of horrific prophecies, \\
  the word of God forced upon the weak.
  
  I watch the blind in disgust in their cathedrals of God, \\
  as they pray for his return. \\
  We, Legion, shall reap and crush their hope. \\
  Our flag overshadows the worthless one, \\
  for all the world to see.
  
  My sanctum, this majesty of sin, \\
  these structures of malevolence inherit my darkened spirit. \\
  Storm the gates, spiral portal descends. \\
  I quicken with the burning of barren relics.
  
  Your beloved fixture of flesh and oak \\
  ingest the silhouettes from below. \\
  Growing blacker, blacker, blacker, blacker, \\
  ebony is the \colour of our salvation.
}





\subsubsection{Founds \Mystraacht}
He turns to evil and \hr{Zachirah founds Mystraacht}{founds the cruel \Mystraacht}. 

\lyricsbs{Hate Eternal}{Darkness by Oath}{
  I swore by oath of darkness, of legion, of one. \\
  I swore by the truth. \\
  I swore my allegiance to darkness, of legion, of many. \\
  I swore by my faith. 
}

He lives now to serve the \banes{} and to wreak evil. 
  
\lyricsbs{Hate Eternal}{Darkness by Oath}{
  By my strength I await the \\
  ever present overboding entities. \\
  Initiate my path of deceit. \\
  Initiate my reign of disdain.
}

He wants to free the \banes{}, whom he sees as his people's true gods and the world's rightful fathers and masters. 
  
\lyricsbs{Hate Eternal}{Darkness by Oath}{
  Bring forth centuries of misery. \\
  Return thyself to the throne. \\
  Emancipate thee from binding restraint. \\
  Bring forth centuries of pain. \\
  Return thyself to the eternal reign. \\
  Emancipate thee from binding restraint. 
}





\subsubsection{Slave \resviel}
\target{Zachirah's slave Resviel}
After he became a \sathariah{} and the Overlord of \Mystraacht, \Zachirah{} became massively sexy in the eyes of many. 
He embodied, in a sense, all the masculine power of the perfect \resphan. 

He was such a pick-up artist that even proud \resviel{} begged and fought for the chance to become his personal sex slaves. 
He took four such slaves. 
They were permanently chained half-naked to his throne and from that moment on lived as his property and playthings, dedicating their lives entirely to his pleasure. 

\hr{Resphain are possessive}{\Resphain{} are possessive} of their women, but no \resphan{} in recorded history had been able to enslave \resviel{} like \Zachirah{} did. 
This was part of the reason why he was so admired and respected. 
It showed off his immensely superior manliness. 
This \uber-manly frame was how he was able to rule \Mystraacht{} for so long. 
People dared not go up against so macho a \resphan. 

\Zachirah{} had absolutely no sexual taboos and would often have one of his slaves suck his dick while he was talking. 
He would also often beat his slaves when he felt like it. 
Most of all he enjoyed the status and the envy of the other \resphain{} (no one else was manly enough to make pureblood \resviel{} serve them as willing slaves). 

Once he became displeased with a slave. 
He ordered her fellow slaves to torture her to death, after which he ate her soul. 
The vacancy was filled in \emph{no time}, and with several applicants vying violently against each other for the position. 

Sometimes he took his throne and slaves into battle, where they would fight for him with magic. 

They \hr{Zachirah dies}{died with him}. 





\subsubsection{Death}
He was \hr{Zachirah dies}{killed during the \resphanwars}, victim of betrayal. 
After this, Ramiel and \Shiaraid{} (both \Mystraacht) were dis\honoured and turned to the \Malach{} experiment. 









\subsection{Personality}
\target{Zachirah's ambition}
\Zachirah{} was, at first, \honour{}able enough, but hard, stern and very ambitious. 
He desired power and status, for himself and for his bloodline. 





\subsubsection{Ambition for Ramiel}
He places great pressure on his pupil, Ramiel. 
He was a good and loving father, but also a stern one. 
He applauded Ramiel's achievements, but also kept pushing him on to new, greater achievements. 
He kept telling Ramiel: 
\ta{\hr{Ramiel can do better}{You can do better.}}





\subsubsection{Dark Lord}
\Zachirah{} became an evil dark lord. 

\lyricsdimmuborgir{Hunnerkongens Sorgsvarte Ferd Over Steppene}{
  Du levde i m\o{}rke.\\
  Du vandret i sorg.\\
  Du plyndret med st\aa{}l til hest.\\
  En stolt og stridig konge,\\
  som erobret hver en borg,\\
  for s\aa{} \aa{} heise fanen til fest.
  
  Attila, hunnernes konge.\\
  Krigenes herre, v\aa{}r far.\\
  Du hentet din styrke fra m\o{}rke,\\
  og p\aa{} tokt med deg jeg n\aa{} drar.\\
}









\subsection{Politics}





\subsubsection{Family}
\Netzach's father was \hr{Shadrach}{\Sharrath}. 
His daughter was \hr{Shiaraid}{\Shiaraid}. 





\subsubsection{Tribes}
\target{Zachirah has tribe contacts}
Before the Delving, \Zachirah was a very learned and wise \resphan.
He knew the nether secrets of \Nyx.
He had many sinister contacts, both \resphan and mortal and monstrous. 

He had contacts among the various tribes that warred against \Merkyrah. 
This was one of the reasons why he ended up playing such an important part in the War of Awakening.















\section{\Ozariel}
\target{Ozariel}
\index{\Ozariel}
A \ketheran{} \resphan{} of \Mystraacht. 
Son of \hr{Zachirah}{\Zachirah}. 
Father of \hr{Dasteron}{\Dasteron}. 















\section{Ramiel}
\target{Ramiel}
\index{Ramiel}
%A \Malach{} whose incarnations include \TydesmosFull{} and \VizicarFull. 
Ramiel is a \Malach, a \resphan{} lord who has left his \resphan{} body to incarnate again and again as a \human{} Scion. 
He was one of the original \satharioth{} who drank the blood of \Nexagglachel. 

%Each Scion of Ramiel retains some memories of his previous incarnation. Typically these memories are locked away at birth and only awaken later, triggered by some massively emotional event. 

In the middle days of the \VaimonCaliphate, Ramiel was incarnated as \hr{Tydesmos}{\TydesmosGendarInCaphet}, a Vaimon archmage obsessed with arcane knowledge and power. 

In the late days of the \caliphate, Ramiel incarnated into \hr{Vizicar}{\VizicarDurasRespina}, one of the last \VaimonCaliphs and a great conqueror and statesman. 

In the year \yic{Carzain birth}, Ramiel incarnated as \CarzainDeracilleShireyo, a rogue Vaimon in Pelidor. In the year \yic{Mutiny}, the dormant personality of Vizicar awakened in young Carzain's mind after Carzain fought his first battle to the death and killed his first man. 









\subsection{Physique}
\target{Ramiel's appearance}
In his \resphan{} form, Ramiel was a dark angel. 
His skin, hair and feathered were ebon black, as was the metal \armour he summoned to encase him. 
He wielded a sword, like a scimitar but as long as a claymore. 

He was dark and terrible, but beautiful and awe-inspiring. 
A true \sathariah{} lord in his best shape. 
Contrast him with \hr{Sithiyacaan's appearance}{\Sithiyacaan}, who had the appearance of a tragic fallen angel.

Ramiel was not extremely tall. 210 cm or so. 









\subsection{Arsenal}





\subsubsection{Guns}
\target{Ramiel's guns}
\index{\Currah}
\index{\Strith}
Ramiel originally had two powerful guns, \Strith{} and \Currah. 
They were made for him by \Lyorith. 
He lost them when he fell. 
Later \hr{Ramiel regains guns}{they were returned to him by \Cishiel}. 









\subsection{History}
\subsubsection{He can do better}
\target{Ramiel was a prodigy}
The young Ramiel was very skilled and was praised as a prodigy. 
He was also very happy and jolly, telling jokes and stuff. 

He also cared a lot about art and culture. 
He painted and wrote music. 

\target{Ramiel can do better}
His father, \Nathrach{}, would applaud his successes, but also drive him onward to new achievements. 
\Nathrach{} would always tell him: 
\ta{You can do better.} 
This line becomes one of Ramiel's driving forces: 
He wants to prove to everyone, especially himself, that he can always \emph{do better}. 





\subsubsection{\Semiza{}: You are nothing}
\target{Ramiel is nothing}
Ramiel was one of \hr{Explorers meet Semiza}{the explorers who found \Semiza}. 
%At the time when \hr{Explorers meet Semiza}{the explorers met \Semiza}, Ramiel was young. 
At this time, Ramiel was young. 
\Semiza{} showed them \hr{Semiza shows tailored visions}{terrible visions, tailored to each of them}. 
Ramiel was confronted with the true vastness, cruelty and indifference of the universe. 
He saw how small, weak, humnle, wretched, insignificant and conceited his people were, and it filled him with angst. 

Since \hr{Ramiel was a prodigy}{young Ramiel was praised as a prodigy}, this revelation struck him a doubly hard blow. 
All his youth he had thought he was something great and had something to be proud of; that \hr{Ramiel can do better}{he could do better}. 
So the realization of his worthlessness was so much greater shock and trauma. 




\subsubsection{Nightmare: He is nothing}
\target{Ramiel dreams of being nothing}
Ramiel's oldest, most recurring and most terrible nightmare is one where he is micro-small, powerlessly and aimlessly adrift in an infinitely huge, black, cold void, where colossal and humongous gods fly past without sparing him a glance or even noticing him. 
He fears being crushed like a fly by sheer chance, with nothing he can do to prevent this meaningless, ignominious end. 

He can do nothing. 

He is nothing. 




\subsubsection{Nightmare variant: A blade of grass}
In a variant of the \quo{nothing} nightmare, Ramiel dreams that he is a blade of grass. 
Lame, blind, shackled to the ground with roots that he will never be able to break because they are part of him. 
It is his very being that keeps him chained. 

Humongous dinosaurs and mammals trample around him. 
They eat the other straws around him. 
He trembles, hoping that he is not next. 

Insects gnaw at him. 
There is pain. 
And powerlessness. 

\lyrics{I have no mouth, and I must scream.}





\subsubsection{Ambition: To prove the converse}
\target{Ramiel's ambition}
Ramiel is never the same again (at least, not for thousands of years). 
A dark cloud hangs over him. 
He is no longer naturally jolly and happy\dash and when he tries it feels fake, forced. 
His humour is rougher, meaner. 

Some of the older \resphain{} worry about this change in young Ramiel. 
This includes \Sithiyacaan{} and some of the priests. 

From this point, his whole life becomes a struggle to prove the converse: 
To challenge the enormous, all-consuming cosmos. 
So since then Ramiel's driving force has been \emph{ambition}: 
The craving for greatness, to prove his worth. 
Not so much for the sake of power or luxury, but in order to feel that he is something great. 

He gives up his interest in the arts. 
They seem pointless: 
If humanoids are inherently worthless, then by extension their creations are worthless, too, so why bother? 





\subsubsection{Growing machismo}
\target{Ramiel grows macho}
Ramiel grew much more macho after he joined the rebellion. 
He abandoned \hr{Ramiel's manners}{his manners} and grew proud and ambitious. 
Almost fanatically so. 
He began to radiate masculinity and power, but trained eyes could detect an undercurrent of desperation, neediness, uncertainty. 





\subsubsection{Sadism}
\target{Ramiel develops sadism}
His obsession with being \quo{nothing} also caused him to develop a sexual sadism. 
He took pleasure in the submission of others, in exercising his power over them. 
He did not allow sadism to take over his life in other aspects, but in the bedroom he fully embraced it and practiced it. 
This worked great with \Shiaraid, who was a masochist.






\subsubsection{Life as a \sathariah}
Maybe Ramiel's past is intimately linked to \Cuezca. 
Which would explain his \hr{Dreaming of Cuezcan Spires}{dreaming of \Cuezcan{} spires}. 
Perhaps the founding of the \Mystraacht{} faction, or the turning it took, was linked to \Cuezca. 

Ramiel has always coveted the throne of the \Mystraacht{} \apex{}, and much of his life has been shaped by his striving for the throne. 





\subsubsection{Lovers}
\target{Ramiel scored as Sathariah}
As a \sathariah, Ramiel attracted more \resvil lovers than ever before.
This was partially because of \hr{Sathariah social status}{his social status as a \sathariah}, but also because of his inner demons and darkness and insanity.
The \resviel could feel the violent emotions swirling inside him, \hr{Races love war}{and it turned them on}.






\subsubsection{Died many times}
Ramiel was killed many times in the rebellion (the War of Awakening), the \secondbanewar and the \resphanwars. 
He was brave and reckless, as a true \resphan man.
\Shiaraid died much less often, since she was sneaky and kept to the background, as a true \resvil.





\subsubsection{Fall from grace}
\target{Ramiel's fall from grace}
Somehow Ramiel is disgraced. 
\hr{Mystraacht summon Daemons}{As the \Mystraacht{} were wont}, he conjured \daemons{} and commanded them in war, like the Warlord of Blood from \emph{Diablo}. 
But he had enemies, inside \Mystraacht{} and outside, who were willing to sell him out. 
During a battle, his \daemons{} deserted and turned on him and his allies. 
It was depicted as his fault for being too weak and overconfident, but in truth he had complete control of the situation until he was backstabbed by his own. 
He knew this, but was unable to prove it, and so he lost his \honour among the \resphain. 





\subsubsection{Becoming a \malach}
\target{Ramiel becomes a Malach}
%Perhaps it was because of betrayal, defeat and disgrace that Ramiel elected to become a \Malach, 
After \hr{Ramiel's fall from grace}{his fall from grace}, Ramiel volunteered for the dangerous but glorious \Malach{} experiment. 
To regain his \honour. 

Or perhaps his plan from the beginning was to become a \malach{} in order to be reborn, live many lives and thus acquire wisdom and skills and emerge reshaped as a wiser, mightier \resphan{} and \vertex. 





\subsubsection{Scion incarnations}
Ramiel had a number of incarnations as a \malach: 

\begin{enumerate}
  \item 
    A few early ones, including at least one who lived in the \VaimonCaliphate. 
  \item 
    \hr{Tydesmos}{\TydesmosGendarInCaphet}.
  \item 
    \hr{Vizicar}{\VizicarDurasRespina}.
  \item 
    \target{Ramiel incarnated as savage}
    A short-lived savage tribesman.
    In this life Ramiel learned how to live in the \wylde. 
    He remembered important skills such as stealth and how to live off the land.
  \item 
    A short-lived farmer. 
  \item 
    \hr{Carzain}{\CarzainDeracilleShireyo}
\end{enumerate}






\subsubsection{Haunted by dreams}
\target{Ramiel haunted}
\target{Ramiel dreams}
In his incarnations as Carzain and Vizicar, Ramiel dreams. 

Vizicar appears to Carzain as a benevolent ally, a wiser, more experienced mentor character. But all is not fine and dandy with him. Vizicar has his own traumata, and Carzain inherits them. 

Occasionally Carzain has dreams and hallucinations. 
He sees \hr{Ramiel dreams of being nothing}{the devouring emptiness of eternity}, and despairs at his own insignificance. 
He \hr{Ramiel is nothing}{fears being nothing}.

He also dreams of being haunted by people. 
These are people from Vizicar's life, his friends, associates and allies whom he betrayed. Somehow they are tied to him and continue to haunt him. This may be caused by his own fear and paranoia, perhaps even guilt.
On occasion, these ghosts are even able to take physical form and come into the physical world to attack him. 

The ghosts appear undead and rotting. 
Some of them are \resphain; now fallen carrion angelds. 
Like in \authorbook{Alan Campbell}{Scar Night}. 

These apparations may be created entirely by his mind, but I am more leaning toward the idea that they are the actual souls of the original people, kept chained by the \ps{\Malach}{} emotions and forced into a state of undeath. 

You see, while Vizicar maintains his composure and sounds perfectly sane and balanced, he suffers from control-mania and paranoia. When people come close to him, he starts to fear their influence, and he has killed many of his lovers and friends out of this fear. 

Or something. 

There is also a theme of Ramiel questing for the truth of his existence, his true origin and purpose, which might be purposefully kept from him, perhaps by the \banes. 
(This is inspired by the trailer I saw for the movie \cite{Movie:BourneUltimatum}.) 
This must be tied together with Vizicar's derangement somehow. 

Something Ramiel fears is his forgetfulness. (This needs to have a cool name: The Decay, the Fading, the Oblivion, something like that.) 
See, \Malachim{} are immortal, but their memories are not. 
They fade with each new incarnation. Even now, with Carzain, Tydesmos is little more than a faint voice in Vizicar's head, a memory of a memory, a ghost of a ghost. And Vizicar can feel the memory of his own life fading. He fears that when Carzain dies, he will fade away entirely. But he also believes that the fading of memory can be arrested, perhaps even reversed, if he can solve the riddle of his past and origin. 





\subsubsection{Scenes from a memory}
Ramiel's dreams are comparable to \cite{DreamTheater:ScenesfromaMemory}: Scattered fragments of past lives. 

And visions of the Metropolis: The endless, massive, decaying, haunted eternal city that is \Nyx. Compare to the \emph{Kult} RPG. 





\subsubsection{Grows to hate \dragons{} and \banes}
After he \hr{Shiaraid dies}{betrayed and slew \Shiaraid}, whom he loved, he became increasingly bitter. 
He \hr{Ramiel blames both sides for the tragedy}{blamed both \dragons{} and \banes{} for this tragedy} and grew to hate both sides. 
This was part of his motivation for betraying the \banelords. 





\subsubsection{Apotheosis and wisdom}
\target{Ramiel is wiser from walking the earth}
Ramiel \hr{Ramiel's final awakening}{finally achieved} the \hs{Apotheosis} he had sought for. 

When he awakens, Ramiel is stronger than ever before. 
Partially because he has spent thousands of years \trope{WalkingTheEarth}{Walking the Earth} as a Scion, which has given him a broad selection of useful experience. 
In total, this proves to have been a more enriching learning experience than it would have been had he lived these millennia as a \resphan. 
(And also less dangerous.
As a \malach he was very hard to kill.)

\target{Ramiel is traumatized from awakening}
He is traumatized and shaken. 
His sanity has suffered harsh blows.
But he is back with a vengeance and revels in his regained and newfound power.

He is wiser now. 
This helps him deal with and come to terms with his many mental problems and gain some semblance of sanity. 

\hr{Ramiel is mad after awakening}{Later, it will turn out} that Ramiel's sanity really has suffered badly from all his experiences and all his revelations.

\target{Ramiel is critical of Mystraacht ideology}
After his Apotheosis, he is more thoughtful and philosophical than his old self. 
He is more critical of \Mystraacht ideology. 
He still uses the traditional \Mystraacht macho ideals in his rhetorics (\hr{Ramiel uses macho rhetoric as Overlord}{as Overlord} and \hr{Ramiel uses macho rhetoric against Dasteron}{when competing for the throne}), but internally he questions it a lot, and in his actual policy he is often far more pragmatic than would be expected of a \Mystraacht (especially suprising to those who knew his old self). 
The \Mystraacht way is to charge in with all your macho bravado, and to worry about bravery and cowardice and reputation. 
Ramiel ends up being much more rational and careful.
He is also critical of \Zachirah's \quo{Religion of Evil}. 
Back in the day he followed \Zachirah's ideology unquestioningly and with great fervour, but the wiser Ramiel of today is smarter than that and recognizes the insanity and danger of the ideology. 





\subsubsection{Grows super-powerful}
\target{Ramiel is overpowered}
Now that he has regained his full \sathariah{} power, Ramiel is pretty \uber. 
See, he has eaten \Belzir, who was also a \sathariah. 
So now he has double \sathariah{} power. 
He has two \hr{Fragments of Nexagglachel}{fragments of \Nexagglachel}. 

\target{Ramiel is underestimated after Apotheosis}
This has never happened before, so people he encounters do not know what it entails or how to account for it. 
So they underestimate him. 





\subsubsection{Grows super-powerful}
\target{Ramiel is mad after awakening}
Ramiel \hr{Ramiel is overpowered}{may be powerful after his Apotheosis}.
His body may have recovered from its \hr{Ramiel crippled}{crippled mortal state}.
But having eaten \Shiaraid is not purely a blessing. 
In addition to her power, Ramiel has also inherited \hr{Shiaraid's curse}{\ps{\Shiaraid} curse} of self-destructive madness. 
In him it does not take the form of sexual masochism. 
That is too far from his true personality. 
In him it takes instead the form of recklessness and a willingness to take insane chances. 
This is merely an extrapolation of a trait he already had. 
It also makes him more sadistic and disdainful, thus inviting others to bring about his downfall. 

He also discovers a new hunger for power, souls and lives.
A new bloodlust. 
A new consuming emptiness emanating from the \bane essence within him, an emptiness that he must feed lest it consume him.
Compare to Sylar from the TV series \emph{Heroes}, and the Dark Eldar from \cite{RPG:Warhammer40000:DarkEldar}.

It takes a while for him to realize this. 
He fears it. 
He knows he is turning evil and mad. 
He fights it. 
But he never manages to entirely rid himself of it. 
\hr{Curse lives on}{\NexagglachelsCurse lives on}.

Moreover, though Ramiel tries to hide it, his sanity really has suffered badly from all his experiences and all the revelations during \hr{Ramiel's final awakening}{the process of Apotheosis}.
It is what turns him against the \banes and makes him take such mad chances.
And he is wracked with much anguish and guilt over slaying \hr{Shiaraid dies}{\Shiaraid} (and, to a later extent, \hr{Dasteron dies}{\Dasteron} and \hr{Ramiel kills Gilchad}{maybe \Gilchad}).

But he also feels an alienation from his fellow \resphain, as if all mortal and immortal life is really a bunch of worthless, insignificant specks of dust.

And he still feels a desperate need to \hr{Ramiel is nothing}{prove that he is more than nothing}.

\citebandsong{Nile:AnnihilationoftheWicked}{Nile}{User-Maat-Re}{
  User-Maat-Re, thou hast done nothing.
}





\subsubsection{Final closure}
At the end, he finally wins a victory of sorts over \Daggerrain, who is a representative of this dark, overpowering cosmos, \hr{Ramiel is nothing}{in comparison to which he is nothing}. 
This makes Ramiel realize all the above, giving him a birds-eye-view of his life. 
He finally understands what he has been struggling to grasp his whole life. 

Finally he gets some closure, finds some peace. 
He has moved the universe. 
He has made a difference. 
He has proven his worth. 

This makes him more good. 
He resolves to oppose \Azraid{} and try to lead his people in a better direction. 
It also leads him to understand \Ishnaruchaefir{} better. 

And he resumes his interest in the arts. 
He has now \quo{proven} that humanoids are not entirely worthless. 
Therefore, by extension, art might have some merit after all. 

Ramiel: 
\ta{20,000 years ago I started composing a symphony. 
  I still remember much of it. 
  Perhaps it is time I finished it...}










\subsection{Personality}
Like all the other \Malachim, Ramiel has lost his memory of his previous life as a \resphan{} lord, recalling only scattered fragments, typically in fever-like dreams and \deajvus. 

Ramiel years for his lost memory and makes it his great quest to rediscover his true identity.





\subsubsection{Afraid of water}
\target{Ramiel fears the sea}
Ramiel is somewhat afraid of the sea. 
He doesn't get seasick, but he gets uncomfortable and prefers to sequester himself below decks, certainly not going anywhere near the railing. 
He is scared he will fall in. 

This has to do with \hr{Ramiel defeated at sea}{a traumatic experience where he lost a naval battle}. 





\subsubsection{Fear of being nothing}
His greatest fear was that of \hr{Ramiel is nothing}{being nothing}. 
He could not stomach the thought of submitting to anyone. 
This was one of the reasons why he ultimately betrayed the \banelords. 

Under the influence of \hr{Curse}{\NexagglachelsCurse}, this fear occasionally surged to really hysterical levels. 





\subsubsection{Manners}
\target{Ramiel's manners}
In his youth, Ramiel was more polite. 
He would say \quo{Lord \Zachirah}, \quo{Lord \Damiarch} and so on. 

This stopped when he and the other rebels became convinced that the \Merkyran{} religion was an oppressive, destructive slave morality. 

Everyone \hr{Ramiel grows macho}{noticed his growing machismo}.





\subsubsection{Megalomania}
\target{Ramiel's megalomania}
Ramiel sees himself \hr{Secherdamon's megalomania}{in a similar light as \Secherdamon}: 
As an heir to divine power and dominion. 

But he is not content to pray and hope for gifts of power. 
He wants to usurp his creators and take it by force. 
And in \SentinelsFinalBook, he tries to do exactly that, by \hr{Ramiel betrays Voidbringer}{betraying the \Voidbringer}. 
He means to supplant the old and stagnant \banes{}, wipe away the failures of the past and bring \quo{new blood} to power. 

He fights for the survival and future of the \resphan{} and \human{} races, and for the proud legacy of the \banes{} and their progenitors, the \voyagers. 
He means to carry on that legacy, by any means possible.





\subsubsection{Willing to destroy and rebuild}
Unlike \Nzessuacrith, who has \hr{Nzessuacrith likes beauty}{a weakness for beautiful things}, Ramiel has no problem with destroying things around him. 
His prefers to fight with no reservations, and then, after he has won, rebuild things, stronger and more beautiful than ever. 









\subsection{Politics}
\subsubsection{Allies and enemies}
\target{Ramiel's enemies}
He has enemies among the \resphain, who do not wish to see him awaken. Among these are rival \Mystraacht{} princes. Yet other \resphain{} love him and want him back. Some of them actively support him\dash overtly or covertly. Some of these eventually backstab him. 

One of his primary enemies is a manipulative and beautiful \resvil{} who desires to be the \Mystraacht{} Overlord herself. Compare her to Azshara from the \emph{Warcraft: War of the Ancients} books by Richard Knaak. 

Have a Scabandari Bloodeye-like character who betrayed Ramiel. Today he is a mighty \resphan{} lord. 
He might even be \Teshrial, or \Azraid. 





\subsubsection{\Azraid{} and \Damiarch}
Ramiel respects \Azraid{} as a competent ruler and a great \resphan. But he is also somewhat repulsed by \Azraid, with all his strange perversions and obsessions. He is too bizarre, too mad.

In the beginning (during the \hr{Merkyran rebellion}{\Merkyran{} rebellion}), Ramiel was young, rash and impulsive. 
He greatly admired \Damiarch{} and looked up to him as a hero-figure to emulate. 
He has since accepted \ps{\Azraid}{} betrayal of \Damiarch, but never forgiven him. 




\subsubsection{\Eryal}
\target{Ramiel resented Eryal}
Ramiel never liked \Eryal. 
He looked down on her, seeing her as a soft, weak, \Merkyrah-sympathizing coward. 
Not \Mystraacht{} material by a long shot. 

Also, Ramiel was jealous because \Shiaraid{} loved \Eryal{} more than she did him. 
He could not fathom what \Shiaraid{} saw in the girl, but he resented \Eryal{} for it. 
\hr{Resphain are possessive}{\Resphain{} are possessive} of their women. 





\subsubsection{Family}
Ramiel was the son of \hr{Nathrach}{\Nathrach}. 
He had a daughter, \hr{Cishiel}{\Cishiel}, born shortly before he went missing as a \malach. 
She was \hr{Ramiel meets Cishiel in person}{there to greet him} when he \hr{Ramiel returns to Mystraacht}{returned to \Mystraacht}. 

\target{Ramiel and Cishiel have sex}
It should be hinted that Ramiel and \Cishiel may have an incestuous relationship. 
Once in a while he spanks her and says: \ta{Do not disobey me, Cishiel.}
She gets aroused and whispers: \ta{Yes, My Overlord Sathariah.}





\subsubsection{\Yurideth}
Ramiel claimed he had never had sex with a \hr{Yurid}{\yurid}.
He had on some occasions taken prisoners, made them into \yurideth and given them to his men, but he had never raped a \yurid himself, he claimed.
He found it distasteful.





\subsubsection{\Vizsherioch}
Ramiel sees \Vizsherioch{} as his great rival. 
He longs to challenge this half-blood \xs{} in order to \hr{Ramiel's ambition}{test his mettle against a representative of the dark, endless universe}. 
But he never manages to do this satisfactorily. 

After \hr{Daggerrain falls}{\ps{\Daggerrain}{} fall} he makes this into a \quo{New Year's resolution}: To one day confront and defeat \Vizsherioch. 
(At this point, \Vizsherioch{} goes dormant, but everyone knows that he is bound to return sooner or later.) 





\subsubsection{\Vorcanth}
\target{Ramiel's wolves}
Ramiel has a group of monsters that serve him, his faithful hounds. 
They are wolf- or hyaena-like monsters. 
They are the \hs{Moon-Wolves}. 

\lyricsbs{Bal-Sagoth}{
  Starfire Burning Upon the Ice-Veiled Throne of Ultima Thule
}{
  The mystic wolves of the frost-moon (slowly, silently) encircle me,\\
  Their eyes are blazing azure, \\
  and their fur is whiter than the sublime snows.
}

The wolves are connected with astrology, remember. 

\lyricsbs{Emperor}{Cosmic Keys to my Creations and Times}{
  They are the planetary keys to unlimited wisdom \\
  and power for the Emperor to obtain. \\
  (They) being the gods of the wolves \\
  whom upon they bark at night, \\
  requesting their next victim in thirst of blood. \\
  I enjoy those moments I may haunt with these beasts of the night.
}







\subsection{\Carcer}
\target{Ramiel binding souls}
\target{Ramiel's bound souls}
\index{\carcer!Ramiel}
\hr{Malachim binding souls}{As a \Malach{} (and a \sathariah, to boot)}, Ramiel possessed a \sephirah-like power to bind the souls of the slain to himself. 
He had three Scion incarnations, all of which were great warrior-mages who fought and killed many people, so he ended up with a big-ass \carcer{} full of souls. 

In his Carzain incarnation, his full \Malach{} powers were latent at first, he did not have conscious control over this ability. 
The result was that the souls of all the people he had killed or whose deaths he caused, as well as the souls of many of his allies (who were not killed by him, but were emotionally bound to him), were bound to him as ghosts. 
They haunted him at night, and sometimes even manifested as wraiths to attack him physically\dash perhaps dragging him into \Nyx{} with them. 
This was a major trauma of his. 

Compare this to Karsa Orlong from \cite{StevenEriksonIanCameronEsslemont:MalazanBookoftheFallen}, who has a horde of fallen souls bound to him by chains.

Gradually, during the story, as he comes to understand his power and his nature better, he learns how to manage these ghosts\dash how to dismiss and suppress them, how to call them forth and commune with them (they don't have much sanity left, but sometimes you can talk to them and get them to make sense). 
He even learns how to control them and channel their power as a magical weapon. 

\citebandsong{DeathspellOmega:FasIteMaledictiinIgnemAeternum}{%
  DeathspellOmega
}{
  The Shrine of Mad Laughter
}{
  Yet it is a world below \\
  and above and in all eternity, \\
  a gift of fever, \\
  the wind of death that sustains the life in me. \\
  Yes, the lightness of hovering in permanent anguish.\\
  I dared to borrow those words, \\
  to articulate them and to savour their turpitude, \\
  as I beheld the shrine of mad laughter.
}

Ramiel begins to realize the parallel between himself and the \Sephiroth, who also chain dead souls. He begins to entertain ideas of divinity. 

But ultimately, he needs to truly use this power as he was meant to. 
His personalities as Carzain, Vizicar, Tydesmos and others must be broken down and destroyed. 

He still harbours feelings of guilt and fear with regard to the ghosts. 
He is violently forced to confront these feelings when his tormentor sets the ghosts loose upon him. 
He fights them, they fight them, and they seem to destroy each other. 
It is pretty traumatic. 

In the end, his Scion personalities are destroyed and unravelled, and he devours and absorbs the ghosts into himself. 
He is now a mighty devourer of souls. 
He is now a \Malach. 
He is now the true Ramiel. 

\lyricsbs{Emperor}{Wrath of the Tyrant}{
  Carnage consumes the emptiness. \\
  Wait till my spirits come forth. \\
  Violate all his chosen ones. \\
  Drink the fires of death. 
  
  Carrying the deaths of his fallen warriors \\
  deep inside of him, in his eyes. \\
  Walk upon this Earth tonight, \\
  carrying the staff of cold souls.
}





\subsubsection{Wielding the bound souls as a weapon}
\target{Bound souls as a weapon}
It is possible to channel the power of the bound souls, momentarily release them into the world and thus wield them as a weapon. The souls are driven mad in their prison and filled with hate, bitterness and rage, a longing to share their terrible pain and sorrow with others. 

Thus unleashed as a torrent of necromantic powers, they slash and tear at the souls of any victims they can catch. They attack by infecting others with their own pain, despair and madness.

If their victim is mentally strong, he can resist the attack and suffer only some mental anguish. 

If he is weak-willed, the ghosts will tear through his mental defenses and be able to harm his physical body\dash slashing it as if with knives, or perhaps causing it to decay and decompose with \quo{instant leprosy}, in extreme cases causing it to collapse in a heap of misshapen, cancerous flesh. 

If a victim survives such an attack, he will likely be permanently scarred and maddened. 

\lyricsbs{Exmortem}{Bitter Discipline}{
  I enter the stage as an evil demagogue, \\
  demanding obedience and devotion \\
  from the legions of dead.
  
  Thousands of men \\
  wandering in darkness. \\
  Searching for their God. \\
  (Here I am.) \\
  Welcome to Hell.
  
  Worship pain, worship Death. \\
  Onto the dark side, let the damned rise. 
  
  Bitter discipline.
}





\subsubsection{Wearing the bound souls as \armour}
Ramiel also learns to shape the bound souls into a magical \armour encasing him. This is as potent as an \hs{Archon Ward}.

Compare to Wismerhill from \FLuneNoire, who can command the winds to encircle him like \armour. 















\section[Shiaraid]{\Shiaraid}
\target{Shiaraid}
\index{\Shiaraid}
A \sathariah{} \resvil{} of \Mystraacht, and later a\malach{}. 
Known for her incarnations as \hr{Delphine}{\Delphine} and \hr{Belzir}{\Belzir}.








\subsection{Arsenal}





\subsubsection{Binding souls}
\target{Shiaraid binding souls}
\index{\carcer!\Shiaraid}
\hr{Malachim binding souls}{As a \malach}, \Shiaraid{} had a \sephirah-like power to bind souls to her. 

\Delphine{}, her first incarnation, \hr{Delphine binding souls}{had bound almost zero souls} (virtually only that of \Eryal). 

\Belzir{} built up \hr{Belzir binding souls}{a more impressive \carcer}. 





\subsubsection{Stealth}
\target{Shiaraid's stealth}
\Shiaraid{} was one of the stealthier \Malachim. 
No one ever suspected that \Delphine{} was a Scion, \hr{Delphine never knew she was a Scion}{not even \Delphine{} herself}. 
And in her incarnation as \Belzir{}, it \hr{No one knew Belzir was a Scion}{did not come out until late in her life} that she was a Scion. 
She herself learned it far earlier than anyone else did. 








\subsection{History}





\subsubsection{\Semiza}
\target{Shiaraid develops sadomasochism}
It was her meeting with \Semiza{} that made \Shiaraid{} turn to sadomasochism. 
The urges had always been lying in her, but latent.
She repressed the urges as \quo{unnatural}, because of her repressive religious conditioning in \Merkyrah. 

\Semiza{} made her realize what she truly wanted. 
Once awakened, the urges would not go away and continued to haunt her. 
And the rebellion gave her the courage to accept and embrace her sexuality. 
Dually, her unfulfilled sexuality (and the realization that it would forever remain unfulfilled if she had to follow \Merkyran{} law) was a major part of her motivation for joining the rebellion. 

\Shiaraid{} learned much about herself when she gave herself away fully in sexual submission. 

\citebandsong{BeyondTwilight:FortheLoveofArtandtheMaking%
}{%
  Beyond Twilight%
}{%
  For the Love of Art and the Making%
}{
  On your knees\dash
  You start to gain a new perspective\\
  On your knees\dash
  Your eyes gazing upwards\\
  On your knees\dash
  Where you choose to be
}





\subsubsection{Unpopular}
\target{Shiaraid unpopular}
Before her \kenosis, \Shiaraid{} had lost a lot of her popularity among her fellow \resphain. 
\hr{Curse}{\NexagglachelsCurse} was affecting her really badly. 
It hit her worse than it did the other \satharioth. 
She was more susceptible to it somehow. 
So she was descending into madness, prone to fits of 
This ostracized and isolated her from her allies and friends. 

Near the end, nearly the only ones who stood on her side were Ramiel and \Eryal, and possibly \Cishiel. 
And even their affections were faltering. 
\Shiaraid{} was growing pretty desparate. 
Of course, that just made her even more mentally unstable, and she fell victim to the Curse even more. 
It was a vicious circle. 

After her \kenosis, most \resphain{} were glad to be rid of her and did not want her back. 
This meant that her \hs{Royalist Faction} had no allies among the \resphain. 









\subsection{Physique}
\Shiaraid's hair and feathers had an orange tint to them. 









\subsection{Personality}





\subsubsection{Dogs}
\target{Shiaraid's dogs}
\Shiaraid{} always liked dogs. 
She kept some of them in all her incarnations. 





\subsubsection{Sexuality and curse}
\target{Shiaraid's sexuality}
\target{Self-destructive Shiaraid}
\Shiaraid{} is a sado-masochist. 

She is a type like Melisande Shahrizai from \authorseries{Jacqueline Carey}{Kushiel's Legacy}: 
A dominatrix and manipulator supreme. 
But she also has a submissive side. 
She enjoys both dominating others and being dominated herself. 

This is partly because she inherited the \hr{Fragments of Nexagglachel}{fragment of \Nexagglachel} that let him endure pain and even take pleasure in it. 

\target{Shiaraid's curse}
It is also due to \hr{Curse}{\NexagglachelsCurse}. 
He is inside her and twists her mind, making her insane and self-destructive. 
It compels her to destroy herself and those she holds dear, even taking pleasure in it. 
She \hr{Satharioth despair at the curse}{fears this}, but cannot stop herself. 
She knows her lusts are dangerous to her. 

\begin{prose}
  \tho{It is madness. But it feels so sweet...} 
\end{prose}

\target{Shiaraid's tragedy}
And so, \ps{\Shiaraid} tragedy is the Curse, which forces her to destroy herself and those she loves: 
\Aryal, \Zachirah{} and Ramiel. 
Whenever she has once again caused ruin for herself and those she loves, she imagines she hears \Nexagglachel{} laughing inside her head, mocking him as she once mocked him in his captivity. 
She hates him, but still some part of her recognizes the justice in it. 










\subsection{Politics}





\subsubsection{Family}
\Shiaraid{} was the daughter of \hr{Zachirah}{\Zachirah}. 
She had two sons, born before or during the \secondbanewar. 
But after she became a \malach{} and went missing, her sons were both killed by contenders for the throne of \Mystraacht{} who did not want to have more of \ps{\Zachirah} heirs alive than necessary. 





\subsubsection{Lover of \Aryal}
\target{Shiaraid and Eryal lovers}
\Shiaraid{} and \Aryal{} used to be lovers. 
\Shiaraid{} was the dominant one and \Aryal{} the submissive. 

Compare to Melisande Shahrizai and \Phedre{} in \authorseries{Jacqueline Carey}{Kushiel's Legacy}. 





\subsubsection{Lover and friend of Ramiel}
\Shiaraid{} was a close friend and lover of Ramiel. 
The two were about the same age. 















\section{Saduraid}
\target{Saduraid}
A \resvil.









\subsection{History}





\subsubsection{Attack against the tree}
Saduraid was one of the leading military commanders during the \hs{Incursion}. 
She led the \hr{Resphain attack tree}{first attack against the \draconian world-tree}. 





\subsubsection{Could not become a \sathariah}
This means that she could not also participate in the \sathariah project. 
She could not become a \sathariah.

Later she became bitter because of this.
She felt cheated of the glory given to the \satharioth. 





\subsubsection{First \neoresphan}
Instead she made a desperate pact with the \banes.
She became the first \hr{Neo-Resphan}{\neoresphan}\dash mightier than a \sathariah, as mighty as a \dragon.
(It was possible to channel massive amounts of \bane energy through her because there was a better connection to Erebos back then.) 





\subsubsection{Death}
Saduraid led an attack against \Secherdamon and {his great invention}. 
She and her monstrous hordes slew several of \Secherdamon's beloved fellow \dragons.
Eventually \Secherdamon slew Saduraid. 






















\chapter{\TiphredSerah}















\section{\Dorzand}
\target{Dorzand}
\index{\Dorzand}
\Dorzand was a \sathariah of \TiphredSerah.
He was very mysterious but charming.
A great manipulator.
A \quo{Count Dracula} type character. 
(The romanticized Dracula, not the actualy Vlad III Draculea.)

He was wise and saw deep.
He was closer to the banelords than many.
\Azraid dealt with \Dorzand, but did not trust him.

His name is based on Dorozhand, a god from 
\cite[\quo{Of Dorozhand}]{LordDunsany:TheGodsofPegana}. 















\section{\Essaryn}
\target{Essaryn}
\index{\Essaryn}
\Essaryn was a \resvil{} of \TiphredSerah, a \thelyad and a \hr{Resphan Protector}{Protectress}. 















\section{\Ishicah}
\target{Ishicah}
\index{\Ishicah}
A \resvil{} of \TiphredSerah, a \sathariah{}. 









\subsection{History}
\Ishicah{} tried to stage a coup and make herself queen of \TiphredSerah. 
She figured that \CiriathSepher{} was better off with a strong ruler, and \Mystraacht{} had been likewise when \Zachirah{} lived, so she would be doing \TiphredSerah{} a favour. 

But she was thwarted and disgraced. 
So she joined the \hr{Malach project}{\Malach{} project}. 

She became a \Malach.

Later she was \hr{Ishicah enslaved}{captured and enslaved by the \Ortaicans} and died (permanently) in captivity. 















\section{\Lothagiel}
\target{Lothagiel}
\index{\Lothagiel}
A \ketheran{} \resphan{} of \TiphredSerah. 









\subsection{History}
\Lothagiel{} once tried to slay \Ishnaruchaefir{}. 
He failed. 
He did a great deal of research on \ps{\Ishnaruchaefir} Aenigma before facing him, and left behind a lot of useful research notes, which would later be \hr{Teshrial gets notes}{discovered by \Teshrial}. 

\Lothagiel{} studied both history and myth. 

The myths, including the poem \emph{\hr{Wanderers in Darkness}{\WanderersInDarkness}}, read like a prophecy of when \Ishnaruchaefir{} will fall. 
\Lothagiel{} did not believe in prophecies, but he believed the myths contained a core of truth. 
So he studied all the sources he could. 

It turned out there was good evidence that \Ishnaruchaefir{} had certain \hr{astrology}{astrological} weaknesses related to the \hr{Matrix}{\matrices}. 
(Remember, the \matrices{} are related to astrology.) 
Under certain patterns of stars, he was weakened, and to certain weapons he was vulnerable. 
\Lothagiel{} procured some of these weapons and plotted how to get to face \Ishnaruchaefir{} at the right time in the right place. 

\Lothagiel{} knew about \ps{\Ishnaruchaefir} Nadir, but he did not know about the \quo{Achilles' heel}. 
He did have suspicions about the heel, but he never managed to crack the mystery of it. 

In the end, \Ishnaruchaefir{} picked up word that \Lothagiel{} was researching him. 
He did not like to be investigated and decided to put a stop to it. 
So he sought out \Lothagiel, tricked him, lured him out, waylaid him and destroyed him. 
At this time, \Lothagiel{} was still in the preparation phase and not ready to fight him. 
And besides, \Lothagiel{} had never intended to face \Ishnaruchaefir{} in single combat. 

This served as a warning to all: 
\ta{Do not fuck with the Destroyer. 
  Do not plot against the Destroyer.
  Do not even \emph{think} of opposing the Destroyer.
  If you do, the Destroyer will fucking kill you (Steve Ballmer style).}















\section{\Nemuragh}
\target{Nemuragh}
\index{\Nemuragh}
A \thelyad{} \resphan{} of \TiphredSerah. 
Gay lover of \Lothagiel. 















\section{\Quelthah}
\target{Quelthah}
\index{\Quelthah}
A \sathariah \resphan{} of \TiphredSerah. 
Ancestor of \hr{Firaxel}{\Firaxel}. 









\subsection{History}
\Quelthah was \hr{Quelthah dies}{slain by \QuessanthIshnaruchaefir} in the Incursion.






















\chapter{\Merkyrah}















\section{\Berugiel}
\target{Berugiel}
\index{\Berugiel}
\Berugiel was a \resphan of \Merkyrah and one of the founders of the \Merkyran religion.















\section{\Damiarch}
\target{Damiarch}
\index{\Damiarch}
\Damiarch{} was a \resphan, a theurge of \Merkyrah.
He was the leader of the \hr{Explorers meet Semiza}{expedition that found \Semiza} entombed beneath \Merkyrah. 

\Damiarch{} was noble and good and wanted nothing more than to enlighten his people with the truth, and help them evolve, break the cycle of decay and become greater than they were. 
He \hr{Merkyran rebellion is necessary}{knew that he might have to use brutal methods to do so}, and he lamented this, but the ends justified the means. 
From the start it was \Gevural{} who was the extremist one, willing to go to any lengths for their cause, and with little remorse. 









\subsection{History}





\subsubsection{War against \Monara}
\Damiarch was once sent out by \hr{Vahaniel}{\Vahaniel} to destroy a group of rebels who dwelt in the tower of \hr{Sherem}{\Sherem}. 
This was \Damiarch's very first mission as a commander. 

During the attack \Damiarch cornered \hr{Monara}{\Monara}, the patriarch of \Sherem. 
\Monara was an old \resphan who would very soon die and had no strength to revive if he were killed. 
\Monara held a long sad monologue explaining why he and his peple had resorted to black magic even though it was forbidden in \Merkyrah.
He just desired life. 
He had eaten \resphan flesh to survive. 

\Monara knew he was beaten, so he surrendered to \Damiarch. 
As a token of submission he presented his two daughters to \Damiarch as slaves and asked him to treat them well, for they were innocent of any crime. 

\Damiarch slew \Monara and took his daughters back home as slaves.
But he thought a lot about \Monara's words. 
\Monara had told many uncomfortable truths and touched many sore points. 
\Damiarch's faith had been shaken. 









\subsection{Politics}





\subsubsection{Family}
\Damiarch was the older brother of \hr{Azraid}{\Gepheral/\Azraid}. 
Their father was \hr{Vahaniel}{\Vahaniel}. 















\section{\Dolsharra}
\target{Dolsharra}
\index{\Dolsharra}
\Dolsharra was a \thelyad \resphan of \Merkyrah.
He was the leader of the group that became the \hs{Lawbringers}, but was also \hr{Dolsharra dies}{the first Lawbringer to die}.















\section{\Tezrabal}
\target{Kezrabal}
\target{Tezrabal}
\target{propaganda minister}
\index{\Tezrabal} 
\index{propaganda minister}
A \resphan{} of \Merkyrah.
Known as the \quo{propaganda minister} of the \Merkyran{} rebellion. 









\subsection{History}
\subsubsection{Early history}
\Tezrabal{} was not on \ps{\Damiarch} expedition.
She joined the rebels soon after. 





\subsubsection{Rebellion}
She was one of the most fanatic, extremist and downright evil of the rebels. 
She hated the church. 
She was obsessed with rebellion, in any form. 
Shee quickly embraced all sorts of perversity and wickedness, reasoning that \quo{anything else is an improvement}. 
Preferably as \quo{else} as at all possible.  

\citebandsong{DeathspellOmega:SiMonumentumRequiresCircumspice}{%
  DeathspellOmega
}{
  Sola Fide
}{
  O Satan, I acknowledge you as the Great Destroyer of the Universe.\\
  All that has been created you will corrupt and destroy.\\
  Exercise upon me all your rights.\\
  I spit on Christ's redemption and to it I shall renounce.\\
  My life is yours Lord, let me be your herald and executioner.\\
  My actions shall lead other hearts away from salvation.\\
  All shall acknowledge Your sacred royalty and crawl in terrified devotion.
}

She regarded \Merkyrah{} and its religion with contempt. 

\citebandsong{DeathspellOmega:Kenose}{%
  DeathspellOmega
}{
  \Kenose
}{
  Observe \Merkyrah, the chariot of the glory of God\\
  Adrift and exiled, the Pilgrim of Light, grandiose and weeping\\
  Thine aura, compared, is but pale and frail, \\
  alike to the one of an ailing child...
}

She made herself a prophet of evil. 

\citebandsong{DeathspellOmega:SiMonumentumRequiresCircumspice}{%
  DeathspellOmega
}{
  Second Prayer
}{
  Oh Satan, you're the God before whom I stand\\
  Live your life in me,\\
  See how I erase my name from the lamb's book of life\\
  And reject the benefit of the holy wounds\\
  I will walk before thee, lord, in the land of the living\\
  For you teacheth my hands to war and my fingers to fight\\
  And sow seeds that do not proceed of the natural order\\
  They shall grow to columns of the holy lair\\
  That which harbours the dragon with seven heads.
}

\citebandsong{DeathspellOmega:SiMonumentumRequiresCircumspice}{%
  DeathspellOmega
}{
  \Hetoimasia
}{
  Hearken, thou, until I relate things\\
  that shall come to pass in latter ages of the world,\\
  for we are the seeds of the triumph yet to come...\\
  Only a few, in the multitudes upon earth,\\
  shall be aware of what they do, but all will\\
  court the assassination of Christ's redemption, again and again...
}

\index{propaganda minister}%
She \hr{Kezrabal becomes propaganda minister}{became the \quo{propaganda minister} of the rebellion}.
Her passion for rebellion made her a great demagogue. 
She had a way of making people follow her. 
They were seduced by the sheer faith and fervour that radiated from her.





\subsubsection{\Sathariah}
She became one of the leaders of \CiriathSepher{} after the rebellion. 
She tried to become a \sathariah, but \Nexagglachel{} destroyed her. 









\subsection{Personality}
\Tezrabal believed that an extremist ideology was more powerful than a moderate one.
Extremism had more power to foster fanaticism and zeal and fervour. 
A moderate ideology would just encourage moderation and thought and doubt and reason, which was not what the revolution needed.

Compare to Adolf Hitler's thoughts on this in \cite{AdolfHitler:MeinKampf}. 















\section[Monara]{\Monara}
\target{Monara}
\index{\Monara}
\Monara was a \resphan of \Merkyrah who rebelled against the \Merkyran theocracy and committed heresy.















\section{\Sharrath}
\target{Shadrach}
\target{Sharrath}
\index{\Sharrath}
\Sharrath was a \resphan of \Mystraacht. 
He was a great Warlord and a skilled theurge as well.
He saved \Merkyrah from possible destruction.








\subsection{Politics}





\subsubsection{Family}
\Sharrath's son was \hr{Netzach}{\Netzach}.















\section{\Vahaniel}
\target{Vahaniel}
\index{\Vahaniel}
\Vahaniel was a \resphan theurge of \Merkyrah.
He was the father of \hr{Damiarch}{\Damiarch} and \hr{Azraid}{\Gepheral/\Azraid}.















\section{Yaruel}
\target{Yaruel}
\index{Yaruel}
Yaruel was a \resphan of \Merkyrah. 









\subsection{History}
There was war between \Merkyrah and another tribe. 
Yaruel wanted to fight, but he was deemed too young and not ready. 
Instead Yaruel was given the task of guarding a group of \human labourers.
They were sent to an uninhabited tower to gather food and other resources. 

Yaruel's father told him that this task was just as important as the war.
Resources were scarce, especially with the war, so these foraging excursions were vital. 
And they had to delve deep in a haunted tower, so there would be plenty of danger. 

But it was not glorious.
Yaruel would win no glory on this trip. 

\Resphan manpower was also scarce, so Yaruel would go alone with the \humans.
This made it worse. 
He would not even have a \resphan companion.

They went to a tower near the Chasm of \hr{Oggra}{\Oggra}.
They were carried by many \carths.
It was difficult to transport \humans between distant towers, so they were supposed to stay here and harvest for many days and then get rescued. 

They went down into a lower level.
This level was windowless and pitch black.
Their only light was their torches.
Then suddenly the stairway collapsed on top of them.
Something fell and blocked the hole, so not even the flying Yaruel could get out.
They were all trapped in the darkness. 

Yaruel told the \humans to keep working and promised them that rescue would come soon enough.
But he knew he was lying. 
Given the war, his people could scarcely spare any \resphain. 
Rescue would not come any day soon. 
He bitterly reflected:
\tho{At least I will have plenty of \humans to feed on if it should come to that.}

As time goes on, Yaruel panicked more and more. 
He was young and inexperienced and did not know how to deal with emergencies like this. 
He wanted to come out of this as a brave hero, so he would be allowed to fight in the next battle.
He did not know how to do that.

Sooner or later, monsters came.
Yaruel had to go and fight them.
He chased the monsters away but lost his way himself.
The \humans were now alone.

In the end Yaruel came back. 
He was now a twisted undead wretch, possessed by the unspeakable horrors of the deep.























\chapter{Others}
\section{\Achsah}
\target{Achsah}
%\sectioncharunspec{\Achsah}{\resvil}{\female}
\Achsah{}, a \resvil{}, is \ps{\Teshrial} subordinate within the Cabal. 
She operates in \Malcur and the surrounding area. 

She is a Cabalist of the \hr{Cabalist circles}{sixth circle}. 







\subsection{Physique}
\Achsah{} is quite stocky of build. 
Heavier than \Teshrial. 
This is due to her \nephilic{} heritage. 







\subsection{History}





\subsubsection{\Merkyrah}
\Achsah{} was born in the city of Ishiin in \Merkyrah. 
She was one of the original \quo{\hr{Early Resphan fallen ones}{fallen ones}} who joined the rebellion and served the rebels as eager groupies. 

She is an \quo{\ashenblood}, born of a \nephilic{} mother. 
In \Merkyrah{} she was an outcast, despised by the noble \resphain. 





\subsubsection{First sex}
\Achsah could not remember her first sexual intercourse, but she knew it had happened in a deep stupor after she had been eating \hr{Glowmoss}{\glowmoss}. 
She was at an orgy with her rebellious underground \bezed punk friends in a secret underground hideout where misfits like her would gather to party and take drugs and have sex in a feeble attempt to rebel against the prevailing system of \Merkyrah. 
She was very young back then.
It was perhaps a century before the rebellion began. 





\subsubsection{Rebellion}
The rebels came and talked to her of how they would overthrow the oppressive order and create a new paradise where the \resphain\dash \emph{all} \resphain\dash would get all the things they were owed. 
They talked of awakening the true nature of their people, and how the outcasts were not bad but simply misunderstood by the narrow-minded and evil church. 
She loved it. 
Also, there were some sexy guys among them. 
Ramiel not least. 
So she joined them and willingly served. 
She was no great hero in those wars, but she served to the best of her ability. 





\subsubsection{Encountered \Ishnaruchaefir}
\target{Achsah met Ishnaruchaefir}
\Achsah once encountered \Ishnaruchaefir on the battlefield. 
She saw his visage, ablaze with fierce chaotic sorcery and hatred toward her kind. 
She felt his presence, radiating a silent promise of death and vengeance. 
And she had not stood her ground. 
She had not even let him come near her. 
She had fled from his path in panic. 
The rest of the battlefield had seemed like a sanctuary, then. 










\subsection{Arsenal}





\subsubsection{High Telepath}
\Achsah{} is a moderately skilled \hs{High Telepath}, but not an expert. 





\subsubsection{Rank}
\target{Achsah's rank}
\Achsah{} is a \bezed{}. 
Because of her lowly origin, she now belongs to the lower class, despised among the \resphain, despite her great skill and power. 

But she is very old and experienced and skilled, so she has climbed up the ladder and now holds a very high Cabal rank. 
She outranks many purebloods. 
This galls them. 
They must obey her, but they are not required to call her by any title. 
(But she is allowed to \emph{not} call them \quo{my lord \thelyad} and the like. Which she would otherwise have to with someone of equal or higher Cabal rank.)

As a commoner, she belongs to none of the \resphan{} dynasties. 









\subsection{Personality}





\subsubsection{Goals}
\Achsah{} dreams of being recognized and loved by her fellow \resphain. 
She has no wings, but she is still a capable, faithful and beautiful \resvil, and worthy of love and respect. 
She hates many \resphain{} because they look down on her and regard her as nothing due to her low birth. 

She doesn't suck up to \Teshrial{} or his ilk. 
She doesn't need their respect. 
She already hates them, and they hate her, so she doesn't try to change it. 
But it hurts her. 
Every reminder that she will never be a true \resvil{} in their eyes is painful and cuts like a knife. 
Even from \Teshrial. 

So when Ramiel comes along and treats her with some more respect, she pledges herself to him. 





\subsubsection{Interests}
\Achsah{} likes children and likes playing with them. 
She is envious of families because, as an \ashenblood, she is sterile and can never have children of her own. 
This makes her resentful of parents, but she likes the children. 
She has, at times, sought out work as a teacher or governess for other \ps{\resphain}{} children. 
It doesn't always work out so well, because they (parents and children alike) scorn her for her impure blood. 

She also likes mortal children and keeps several of them as slaves. 
She treats them well. 
Some think she has sex with them, but this is not true (mostly), and she is hurt by the accusation. 





\subsubsection{Philosophy}
\Achsah{} is a philosophical person. 
There are times when she questions and doubts her evil ways.

\lyricslimbonicart{Under Burdens of Life's Holocaust}{
  A veil of darkness rest upon my shoulders.\\
  I reign and serve and obedient beholder,\\
  'cause the shadows are my hearts domain\\
  and where I wander.
}






\subsubsection{Sexuality}
\target{Achsah's sexuality}
\Achsah{} is bisexual and plays her cruel, demented sex-games on men and women alike.

She likes rugged, manly men. 
She is one of the reasons why \hr{Achsah rewards Charcoal}{she has sex with Charcoal}. 
This is because of her half-\nephil{} heritage, and because she resents the pureblood \resphain{} who scorn her. 






\subsubsection{Wishes she had wings}
\Achsah{} has no wings, and it pains her. 
She often looks up at the sky and wishes she could soar free like a pureblood. 
She can fly using magic, of course, but it's not the same. 
She's not as fast nor as \manoeuvrable, and it just doesn't feel right. 
She always feels like a fake encroaching on the turf of the purebloods. 







\subsection{Politics}





\subsubsection{Attracted to Ramiel}
She was once a groupie of \hs{Ramiel}. She is attracted to him. 

When she learns that Ramiel is back, she helps him in hope of currying favour. Covertly at first, then overtly.





\subsubsection{She hates \Teshrial}
\target{Achsah hates Teshrial}
She hates \Teshrial, because he is younger and less experienced than she but placed above her.















\section{Kizmath}
\target{Kizmath}
\index{Kizmath}
Kizmath was a \hs{Morphous} \resphan, a \hs{High Telepath}. 
He was the oldest High Telepath known, having been a Morphous for 5000 years. 
His body was completely atrophied and mutated. 
He was a monster who never let anyone gaze upon his true form.
He was famous and revered as a wise oracle, but also shunned and despised for being such a freak. 















\section{\Lethiarch}
\target{Lethiarch}
\index{\Lethiarch}
\Lethiarch was a \bezed \resvil affiliated with no dynasty. 
She was one of the gods of \hr{Uruthar}{Uruthar} and later became the last goddess of the city beneath the hills of \hr{Uldor}{Uldor}. 









\subsection{History}
Gradually the \hs{gods of Uruthar died}, until \Lethiarch was the last one left.
She reigned alone in a vast empty city of death and decay and morbid degeneration.
For more than a thousand years she had been the only truly intelligent high-standing living creature in Uruthar. 
She took some of her more intelligent followers and raised them from the dead and turned them into undead mummies so she could still have their company.
But in time, one by one they incurred her displeasure and she destroyed them, until they were all gone and there were no worthy candidates left for her to resurrect.
Then she was alone. 

She suffered.
She wept and screamed and raged in her solitude. 
Her sanity suffered.

And her body grew weak as well, for the quality and quantity of available food declined. 
Her people grew weaker and more worthless.
Soon she could no longer sustain herself on the hillmen alone, for their souls had grown too broken and depraved.
She had to send her people out in the countryside to kidnap victims for her to devour.
Uldor once had its native industry and agriculture and hunting that kept the city alive, but in time all civilization atrophied and the ruins of Uruthar became a parasitic nation that survived only by raiding its neighbours. 

\target{Lethiarch's body decays}
She was exiled alone in darkness, surrounded by deformed vermin-men and eating the odd kidnapped villager.
This was not the life that a \resvil was made for. 
She still hungered.
Her sanity suffered. 
She began to lose willpower and control over herself.
She lost her humanity.
And her body decayed. 
She began to slowly melt into a foul, shapeless, slimy, \banespawn-like thing.





\subsubsection{She needed sex}
\target{Lethiarch wants sex}
One of the things that \Lethiarch missed most was sex.
She could not fuck the degenerate hillmen. 
They were too disgusting. 
Sometimes she had them go out and kidnap strong and handsome young men for her.
She would keep these men as sex slaves and make them service her until she tired of them and killed them.
But this was never really satisfying for her.
It was a poor substitute for real, manly lovers.
She ached for good sex. 









\subsection{Physique}





\subsubsection{Appearance}
To a \human \Lethiarch appeared as supernaturally beautiful and alluring and sexy.
But there was also something unnatural and ghastly and repellent about her.
Some taint of hideous inhuman evil, of aeons-old festering corruption that festered horribly in these loathsome blackened pits beneath the earth. 









\subsection{Skills and powers}





\subsubsection{Languages}
\Lethiarch did not speak the Vaimon tongue, but she did speak an archaic form of \hs{Sturiac}. 















\section{\Najarod}
\target{Najarod}
\index{\Najarod}
A \resphan{} of \Baelzerach. 
Chieftain of a tribe \hr{Ishnaruchaefir and Baelzerach}{allied with} \QuessanthIshnaruchaefir. 
A \hr{Daemoniac}{\daemoniac}.

\Najarod{} was one of the few people who could enter and leave the \hs{Mirage Asylum}. 















\section{\Osra}
\target{Osra}
\index{\Osra}
\Osra was a \bezed \resphan affiliated with no dynasty. 
He was one of the gods of \hr{Uruthar}{Uruthar}. 
He was eventually \hr{Osra dies}{slain by \Lethiarch}. 















\section{Rayne \Tarcharos}
\target{Rayne Tarcharos}
Rayne \Tarcharos was an early \resphan sorcerer who lived before the fall of \Merkyrah.
He was the founder of the evil empire of \hr{Tarcharos}{\Tarcharos}.















\section{\Sartheron}
\target{Sartheron}
\index{\Sartheron}
\Sartheron was a \resphan sorcerer who lived at the time of \Merkyrah. 
He was a terror for hundreds of years.
But he was subtle.
He did not conquer towers and holds; instead he remained hidden and pulled strings from the shadows.

\Sartheron himself always kept to the shadows.
He became a living legend. 
Everyone feared him, and every hero dreamt of slaying him. 





\subsection{Equipment}





\subsubsection{Immortality through life-seeds}
\Sartheron had discovered a way to create primitive \hr{life-seeds}{life-seeds}. 
He had been slain by heroes more than once but always came back to life.
He gained a reputation for being immortal. 











\subsection{History}





\subsubsection{Almost destroyed \Merkyrah}
\Sartheron allied himself with loathsome powers: The corrupting \hr{Nether Ones}{Nether Ones}. 
He came close to destroying \Merkyrah.





\subsubsection{Slain by \Netzach}
The hero \Netzach \hr{Netzach slays Sartheron}{slew \Sartheron}.









\subsection{Politics}

\Sartheron taught several apprentices over the years.
Some of these apprentices eventually left his side and became evil overlords of their own until they were slain by heroes or undone by their own sorcery. 















\section{Shessirar}
\target{Shessirar}
\index{Shessirar}
Shessirar was a \hs{Morphous} \resvil, a \hs{High Telepath}. 









\subsection{Physique}
Shessirar's body was quite grotesque, but she could still function and move with the aid of \shapen. 
She had not been a High Telepath for as long as \hs{Kizmath} had. 















\section{\Shiin}
\target{Shiin}
\index{\Shiin}
\Shiin was a \resvil of \Baelzerach.
She was the youngest lover of \hr{Najarod}{\Najarod} and an ally of \QuessanthIshnaruchaefir. 
















\section[Thanatzil]{\Thanatzil}
\target{Thanatzil}
\index{\Thanatzil}
\Thanatzil{} was the \banemessiah, the first \resphan{} ever and the founder of the \resphan{} race. 

His name is inspired by \href{http://en.wikipedia.org/wiki/Thanatos}{Thanatos}, the personification of Death in Greek Mythology. Interestingly, the Greek Thanatos was the son of \Erebos{} and \Nyx. 

He is a rather tragic \trope{AntiVillain}{Anti-Villain}. 

\lyricsdimmuborgir{Allegiance}{
  Cuddled through a cold womb he was, \\
  pitch black and without sunshine rays.\\
  Hell patiently awaiting him on blood spilled soil. \\
  A noble grief stirred heart, always ready to die.
  
  In sinister systematisation, submission is golden. \\
  As an apprentice to violence, slaughter and bloodshed. \\
  He was like an object (that is) being processed, \\
  a force-fed destructor ready for abomination.
  
  The vast solitude in him witnessed it all, \\
  those self afflicting eyes and their fear painted faces. \\
  Made out of utter discipline, failure unacceptable. \\
  Hosts to oblivion, exploring the darkest of places, \\
  Stench of rotten flesh breathing down his neck.
  
  Every day seemed like an endless night. \\
  When would he ever wake from this void? \\
  No other voice than his own will ever tell\\
  what was real and where he had been, \\
  what he had done.
  
  Life forever lost its innocence, \\
  Never to see the light of day again. \\
  He pondered his last few steps\\
  into the realms of Death \\
  with his hands bloodstained.
  
  Courage and consistency. \\
  Bravery and valor. \\
  Honor and pride. \\
  For what was it all worth?
}









\subsection{Personality}





\subsubsection{Serious}
\Thanatzil had to grow up very quickly. 
He had to be responsible and serious and not childlike.
In Japanese he would be using the pronoun \emph{\quo{watashi}} rather than \emph{\quo{boku}} even as a young boy. 































