\begin{comment}
\part{The World of \Miith}
\end{comment}









\begin{comment}
\chapter{Timeline}
\end{comment}

% I would like all my text commands to end in an \xspace, because then I don't have to put an explicit ``{}'' or ``\ '' after them to get a space.
% To save space I want to define a command-definer that automatically appends an xspace to the defined command. 
% I am adverse to creating a completely new command-definer, because then Kile (my IDE) will not understand that the new commands have been defined, and then autocompletion will not work. 
% So I reuse the name ``\providecommand'', which Kile already knows about.
% \renewcommand{\providecommand} [2]{\newcommand{#1}{#2{}\xspace}}













\begin{comment}
\section{Symbols}
\end{comment}



\begin{comment}
\subsection{Grammar}
\end{comment}

% Form regular English S-possessive of a plural noun. 
\newcommand*{\psp} [1]{#1'}
\newcommand*{\pps} [1]{\psp{#1}}
% Form regular English S-possessive of a noun. 
\newcommand*{\ps}  [1]{#1's}



\begin{comment}
\subsection{Non-alphabetical symbols}
\end{comment}

%A paragraph break with no indentation
%\newcommand{\new}{\subparagraph{}\noindent\ignorespaces}
%\newcommand{\new}{\subparagraph{}}
%\newcommand{\new}{\paragraph{}\noindent{}\ignorespaces}
\newcommand{\new}  {\bigskip\noindent}

\newcommand{\dash} {{}\thinspace---\thinspace}
\newcommand{\dah}  {\dash} % A common misspelling.



\begin{comment}
\subsection{Consonants}
\end{comment}

% Voiced alveolar fricative. 
\newcommand{\zh}{\v z}

% A soft D in Rissitic.
\newcommand{\rissdh}{dh}
% \newcommand{\rissdh}{\textipa{\dh}}

\begin{comment}
\subsubsection{R-variants}
\end{comment}

% Short guttural R.
\newcommand{\gr}    {\textinvscr}
% Long guttural R.
\newcommand{\grr}   {\textscr}
% Regular English R.
\newcommand{\rr}    {\*r}
% 'Rolling' R.
\newcommand{\rollr} {r}

% A guttural R in Kingstongue.
\newcommand{\Rhh}{\v R}
\newcommand{\rhh}{\v r}
% A long guttural R in Kingstongue.
\newcommand{\Rrhh}{\v R}
\newcommand{\rrhh}{\v r}



\begin{comment}
\subsection{Vowels}
\end{comment}

% Umlaut letters.
\newcommand{\Aumlaut} {\"A}
\newcommand{\aumlaut} {\"a}
\newcommand{\Uumlaut} {\"U}
\newcommand{\uumlaut} {\"u}

% Long vowel in a stressed syllable.
\newcommand{\Along}   {\=A} % Dark A
\newcommand{\along}   {\=a} % Dark A
\newcommand{\Ilong}   {\=\I} 
\newcommand{\ilong}   {\=\i} 
\newcommand{\Ulong}   {\=U} 
\newcommand{\ulong}   {\=u} 

% Round O.
\newcommand{\Oround}  {\^O}
\newcommand{\oround}  {\^o}

% Non-silent final E.
\newcommand{\finale}  {\"e}

\begin{comment}
\subsubsection{Resphan}
\end{comment}

% Flat A, dark A and long dark A in the Resphan tongue.
\newcommand{\Aflatresphan} {A}
\newcommand{\Adarkresphan} {A}
\newcommand{\aflatresphan} {a}
\newcommand{\adarkresphan} {a}
\newcommand{\Alongresphan} {\Along}
\newcommand{\alongresphan} {\along}
\newcommand{\ahresphan}    {ah} % Like \alongresphan, but final

% Long I in the Resphan tongue. 
\newcommand{\Ilongresphan} {\Ilong}
\newcommand{\ilongresphan} {\ilong}

% Rounded O in the Resphan tongue. 
\newcommand{\Oroundresphan} {O}
\newcommand{\oroundresphan} {o}

% AE diphthong in the Resphan tongue. 
\newcommand{\aeresphan}    {ae}

\begin{comment}
\subsubsection{Rissitic}
\end{comment}

% Flat A, dark A and long dark A in the Rissitic tongue.
\newcommand{\Aflatrissitic} {\"A}
\newcommand{\Adarkrissitic} {A}
\newcommand{\aflatrissitic} {\"a}
\newcommand{\adarkrissitic} {a}
\newcommand{\Alongrissitic} {Aa}
\newcommand{\alongrissitic} {aa} 

% Round O.
\newcommand{\Oroundrissitic}  {\Oround}
\newcommand{\oroundrissitic}  {\oround}

% Ø in the Rissitic tongue.
%\newcommand{\rissoe}{oe}
\newcommand{\rissoe}          {\o}

\begin{comment}
\subsubsection{Diphthongs}
\end{comment}

\newcommand{\aediphthong} {a\"e}



\begin{comment}
\subsection{Y-variants}
\end{comment}

\begin{comment}
\subsubsection{Consonantal}
\end{comment}

% Consonantal Y, as in English "yes".
\newcommand{\y} {y}
\newcommand{\Y} {Y}

% Consonantal Y in the Nephilic tongue. 
\newcommand{\yconsonantnephil} {\y}

\begin{comment}
\subsubsection{Vowel}
\end{comment}

% Vowel Y, as in Danish "tyv".
\newcommand{\yvowel} {\"y}

% Vowel Y in the Redcor tongue. 
\newcommand{\yvowelredcor} {\yvowel}

% Vowel and consonantal Y in the Rissitic tongue. 
\newcommand{\yvowelrissitic}     {\yvowel}
\newcommand{\yconsonantrissitic} {y}

\begin{comment}
\subsubsection{Diphthong}
\end{comment}

% Diphthong Y, as in English "fly". 
\newcommand{\ydiphthong} {\^y}



\begin{comment}
\subsection{Groups of letters}
\end{comment}

% "an" in Redcor, as in French "France". 
\newcommand{\anredcor} {\^an}







%%%%%%%%%%%%%%%%%%%%%%%%%%%%%%%%%%%%%%%%%%%%%%%%%%%%%%%%%%%%%%%%%%%%%%%%%
%%%%%%%%%%%%%%%%%%%%%       TIMELINE THINGS       %%%%%%%%%%%%%%%%%%%%%%%
%%%%%%%%%%%%%%%%%%%%%%%%%%%%%%%%%%%%%%%%%%%%%%%%%%%%%%%%%%%%%%%%%%%%%%%%%

\begin{comment}
\section{Commands}
\end{comment}


% ``Numeric value, basic''.
% Create a numeric value with key #1 and value #2. 
\newrobustcmd*{\nvb}[2]{%
  \csedef{NV #1}{\number\numexpr#2}%
}
% ``Numeric value, offset''. 
% Takes three arguments: New key, base key, offset. 
% Create a numeric value with key #1 and value #2+#3. 
\newrobustcmd*{\nvo}[3]{%
  \csedef{NV #1}{\number\numexpr\csuse{NV #2}+\number\numexpr#3}}

% Gives the difference between two numeric values.
\newcommand*{\difference} [2]{%
  \number\numexpr\csuse{NV #1}-\number\numexpr\csuse{NV #2}
}

\newrobustcmd*{\hiscb}[2]{\nvb{#1}{#2}}
\newrobustcmd*{\hisc}[3] {\nvo{#1}{#2}{#3}}

% `birthandage' specifies a person's name (`foo') birth year and his age at death. 
% It also introduces two hisc's: `foo birth' and `foo death'. 
% Birth year is given as a base hisc and an offset. 
% Death year is computed as birth year + age. 
\newcommand*{\birthandage}[4]{\hisc{#1 birth}{#2}{#3}\hisc{#1 death}{#1 birth}{#4}}
% `deathandage' is like birthandage, except you get to specify a year of death, and the year of birth is calculated. 
\newcommand*{\deathandage}[4]{\hisc{#1 death}{#2}{#3}\hisc{#1 birth}{#1 death}{-#4}}
% `birthliving' lets you specify the birth year of a person who still lives. 
\newcommand*{\birthliving}[3]{\hisc{#1 birth}{#2}{#3}}

% Timeline environment. 
\newenvironment{timeline}{\begin{description}}{\end{description}}
% Timeline item. 
\newcommand{\timeitem}[1] {\item[\yds{#1} (\yic{#1}):]}







\begin{comment}
\subsection{Stamp things}
\end{comment}

% These are `time stamps' and `place stamps' that are to be placed at the beginning of a chapter or section and indicate the time and/or place where the section takes place
\newcommand{\typesetstamp} [1]{\subsubsection*{\noindent #1}}
%\newcommand{\typesetstamp}[1]{\noindent\ignorespaces\textbf{#1}}

% Typeset date with link. 
% Takes four arguments: Year counter, day, month and label-to-link-to. 
\newcommand{\typesetdatelink} [4]{%
  Year \vaimonyear{#1}{} of the \hr{Vaimon Calendar}{\VaimonCalendar}
  \\ \nopagebreak
  #2 day of {#3}}

% Typeset date.
% Takes three arguments: Year counter, day, month. 
\newcommand{\typesetdate} [3]{%
  \typesetdatelink{#1}{#2}{#3}{#3}}

% Like \typesetdate, but uses the Draconian Supremacy calendar.
\newcommand{\typesetdraconiandate} [1]{%
  Year \dragonyear{#1} of 
  \hr{Draconian Supremacy calendar}{\Draconian{} Supremacy}
}
% Like \typesetdate, but uses the Black Dawn calendar.
\newcommand{\typesetresphandate} [1]{%
  Year \resphanyear{#1} of the \hr{Black Dawn calendar}{Black Dawn}
}
\newcommand{\typesetbeforevaimondate} [1]{%
  \beforevaimonyear{#1} years before the 
  \hr{Vaimon Calendar}{\VaimonCalendar}
}
\newcommand{\typesetdraconianvaimondate} [1]{%
  \typesetdraconiandate{#1}\\
  \typesetbeforevaimondate{#1}
}
\newcommand{\typesetdraconianresphanvaimondate} [1]{%
  \typesetdraconiandate{#1}\\
  \typesetresphandate{#1}\\
  \typesetbeforevaimondate{#1}
}

















\begin{comment}
\section{Dominators of \Miith}
\end{comment}



\begin{comment}
\subsection{\FirstbanewarBook}
\end{comment}

% The year where Tiamat first summons the Xzai-Shann. 
% Considered the beginning of known history. 
% \hiscb{XS}{0}
% \hisc{Sethicus creates Dragons}{XS}{-1000000}

\hiscb{Sethicus creates Dragons}{0}
\hisc{Durance begins}           {Sethicus creates Dragons}{9457}
\hisc{First Banewar begins}     {Durance begins}          {1362458}
\hisc{First Banewar ends}       {First Banewar begins}    {171}
\hisc{Saphyrae dominates}       {First Banewar ends}      {26942}
\hisc{Aryoth invasion}          {Saphyrae dominates}      {3843}
\hisc{Draconian Ascendancy}     {Aryoth invasion}         {320}
\hisc{XS}                       {Draconian Ascendancy}    {0}

\birthliving{Nexagglachel}      {XS}                  {28}
\birthliving{Ishnaruchaefir}    {XS}                  {139}
\birthliving{Secherdamon}       {XS}                  {951}
\hisc{Scathae created}          {XS}                  {287}



\begin{comment}
\subsection{\ThanatzilBook}
\end{comment}

\birthliving{Semiza}                 {XS}                  {2425}
\birthliving{Ilu}                    {Semiza birth}        {32}
\hisc       {Ilu pregnant}           {Ilu birth}           {16}
\birthandage{Thanatzil}              {Ilu pregnant}        {1} {21}
\hisc       {First Humans conveived} {Thanatzil birth}     {15}



\begin{comment}
\subsection{\MerkyrahBook}
\end{comment}

\hisc{Semiza found}            {Thanatzil death}{13738}
\hisc{Merkyrah falls}          {Semiza found}   {33}
\hisc{Murder of the Dawn}      {Merkyrah falls} {0}



\begin{comment}
\subsection{\SecondbanewarBook}
\end{comment}

\hisc{Satharioth}              {Merkyrah falls}{1489}
\hisc{Shrouding}               {Satharioth}    {116}
\hisc{Second Banewar ends}     {Shrouding}     {0}

\newcommand{\dateSecherdamonsVow} {%
  \typesetdraconianresphanvaimondate{Second Banewar ends}}



\begin{comment}
\subsection{\ResphanWarsBook}
\end{comment}

\hisc{Zachirah death}                {Shrouding}      {431}
\hisc{Malach fiasco}                 {Zachirah death} {286}



\begin{comment}
\subsection{\TheLieSublimeBook}
\end{comment}

\hisc{Cuezca rises}                  {Malach fiasco}       {261}
\hisc{Cuezcan Apocalypse}            {Cuezca rises}        {137}







\begin{comment}
\section{Archons of \Miith}
\end{comment}



\begin{comment}
\subsection{Cordos Vaimon's time}
\end{comment}

% The 0th year of the Imperial Calendar, the founding of the Vaimon Caliphate. 
\newcommand{\ic}{\value{VC}}
\hisc{VC}                               {Cuezcan Apocalypse} {94}
\hisc{IC}                               {VC}                 {0}
\hisc{Founding of the Vaimon Caliphate} {VC}                 {0}

% Cordos Vaimon
\birthandage{Cordos Vaimon}    {VC}                 {-51}{72}
% Belandos Vaimon II (Cordos' father)
\birthandage{Belandos}         {Cordos Vaimon birth}{-25}{55}
% Faegos Vaimon (Cordos' grandfather)
\birthandage{Faegos}           {Belandos birth}     {-20}{51}
% Silqua Delaen
\birthandage{Silqua}           {VC}                 {-35}{30}
\birthandage{Silqua Vaimon}    {Silqua birth}       {0}  {30}
% Silqua's two brothers. 
\birthandage{Arcan Delain}     {Silqua birth}       {-15}{64}
\birthandage{Lestor Delain}    {Silqua birth}       {-11}{82}

% The wedding of Silqua Delain and Cordos Vaimon
\hisc{Silqua wed}              {Silqua birth}{18}
\hisc{Silqua married}          {Silqua wed}{0}




\begin{comment}
\subsection{Darkfall time}
\end{comment}

% Grith Ecallivan, a Vaimon scoundrel hero. 
\birthandage{Grith Ecallivan}  {VC}{761}{81}

% Vizicar
\birthandage{Vizicar}          {VC}{1251}{61}

% Belzir
\birthandage{Belzir}           {VC}{1486}{73}

% The Darkfall, the fall of the Vaimon Caliphate. 
\hisc{Darkfall}{Belzir death}  {0}
\hisc{Hundred Scourges}{Darkfall}{0}



\begin{comment}
\subsection{Ortaican and Tepharin}
\end{comment}

\hisc{Fall of Ortaica}         {Darkfall}{178}
\hisc{Fall of Tepharae}        {Darkfall}{419}

% Texall discovers his axioms. 
\hisc{Texall Axioms}           {Fall of Ortaica}{87}

% The time of the Eresh-Kali civilization.
\hisc{Eresh-Kal flourished}    {VC}{1100}
\hisc{Eresh-Kal destroyed}     {VC}{1159}



\begin{comment}
\subsection{Uther the Tiger's time}
\end{comment}

% The year of the beginning of the Carzain story, where Carzain fights the mutinying mercenaries. 
\hisc{Founding of Belkade}     {Darkfall}           {577}
\hisc{Fall of Belkade}         {Founding of Belkade}{188}

\hisc{Founding of Imetrium}    {Mutiny}             {-848}

\birthandage{Cuthran the Victorious} {Founding of Belkade}{-32}{68}
\birthandage{Uther the Tiger}        {Founding of Belkade}{-35}{61}







\begin{comment}
\section{Sentinels of \Miith}
\end{comment}

\hisc{Mutiny}                      {Fall of Tepharae}   {129}
\hisc{Catrian dies}                {Mutiny}             {3}

% The year of the Pelidor-Runger war, where Carzain fights.
\hisc{Runger war}                  {Mutiny}{9} % In terms of year numbers it's two years, but in terms of actual days it's only slightly over one year.

% The year where the Rissitics conquer Fendor.
\hisc{Siege of Fendor}             {Runger war}{-1}

% The year where Giles Tantor's diary is written. 
\hisc{Tantor diary}                {Runger war}{-1}

% The year where Ishnaruchaefir slays a Ghobal in Malcur. 
\hisc{Ishnaruchaefir slays Ghobal} {Runger war}{-1}







\begin{comment}
\subsection{\RungerWarBook}
\end{comment}



\begin{comment}
\subsubsection{Vertex Awakening}
\end{comment}

\newcommand{\dateCatrianDies} {%
  \typesetdate%
    {Catrian dies}{21st}{\Thimared}}

\newcommand{\dateMutiny} {%
  \typesetdate%
    {Mutiny}{11th}{\Magaled}}
\newcommand{\dateDayAfterMutiny} {%
  \typesetdate%
    {Mutiny}{12th}{\Magaled}}
\newcommand{\dateIlcasNorthstarIntroduction} {%
  \typesetdate%
    {Mutiny}{14th}{\Magaled}}
\newcommand{\dateCarzainFucksMirai} {%
  \typesetdate%
    {Mutiny}{20th}{\Magaled}}
\newcommand{\dateCarzainBackInRedglen} {%
  \typesetdate%
    {Mutiny}{17th}{\Cushed}}



\newcommand{\dateIshnaruchaefirIntroduced} {%
  \typesetdate%
    {Ishnaruchaefir slays Ghobal}{13th}{\Teshiron}}
\newcommand{\dateIshnaruchaefirSlaysAGhobal} {%
  \typesetdate%
    {Ishnaruchaefir slays Ghobal}{13th}{\Teshiron}}



\begin{comment}
\subsubsection{The Invasion}
\end{comment}

\newcommand{\dateIcorDies} {%
  \typesetdate%
    {Runger war}{21st}{\Atzirah}}
\newcommand{\dateCurwenKillsOnatol} {%
  \typesetdate%
    {Runger war}{22nd}{\Atzirah}}
\newcommand{\dateCarzainJoinsTheArmy} {%
  \typesetdate%
    {Runger war}{2nd}{\Razilah}}
\newcommand{\dateCarzainGoesToMalcur} {%
  \typesetdate%
    {Runger war}{7th}{\Razilah}}
\newcommand{\dateIcorMeetsPsyrex} {%
  \typesetdate%
    {Runger war}{14th}{\Razilah}}
\newcommand{\dateCharcoalFucksNeedle} {%
  \typesetdate%
    {Runger war}{20th}{\Razilah}}




\begin{comment}
\subsubsection{The March}
\end{comment}

\newcommand{\dateTheArmySetsOut} {\typesetdate%
    {Runger war}{1st}{\Keshirah}}
\newcommand{\dateCharcoalReadsDiary} {\typesetdate%
    {Runger war}{4th}{\Keshirah}}
\newcommand{\dateIcorBuried} {\typesetdate%
    {Runger war}{6th}{\Keshirah}}
\newcommand{\dateAchsahAndNeedle} {\typesetdate%
    {Runger war}{9th}{\Keshirah}}
\newcommand{\dateCharcoalReadsMoreDiary} {\typesetdate%
    {Runger war}{11th}{\Keshirah}}
\newcommand{\dateNeinaCaptured} {\typesetdate%
    {Runger war}{16th}{\Keshirah}}
\newcommand{\dateCarzainFirstInForklin} {\typesetdate%
    {Runger war}{17th}{\Keshirah}}
\newcommand{\dateRianLooksForNeina} {\typesetdate%
    {Runger war}{18th}{\Keshirah}}
\newcommand{\dateCarzainVolunteers} {\typesetdate%
    {Runger war}{18th}{\Keshirah}}
\newcommand{\dateIcorHauntsTiroco} {\typesetdate%
    {Runger war}{20th}{\Keshirah}}
\newcommand{\dateRianAndTheCrazyOldWoman} {\typesetdate%
    {Runger war}{21st}{\Keshirah}}



\begin{comment}
\subsubsection{Jirad Tantor's diary}
\end{comment}

\newcommand{\dateTantorMeetsTakestsha}        {\typesetdatelink%
    {Tantor diary}{10th}{\Feazirah}{Feazirah}}
\newcommand{\dateTakestshaIntroducesEreshKal} {\typesetdatelink%
    {Tantor diary}{19th}{\Feazirah}{Feazirah}}
\newcommand{\dateTantorResearchesEreshKal}    {\typesetdatelink%
    {Tantor diary}{22nd}{\Feazirah}{Feazirah}}
\newcommand{\dateTakestshaSetsOut}            {\typesetdate%
    {Tantor diary}{2nd}{\Barion}}
\newcommand{\dateTakestshaReachesGedrock}     {\typesetdate%
    {Tantor diary}{14th}{\Barion}}
\newcommand{\dateWaythaneDayOne}              {\typesetdate%
    {Tantor diary}{15th}{\Barion}}
\newcommand{\dateWaythaneDayTwo}              {\typesetdate%
    {Tantor diary}{16th}{\Barion}}
\newcommand{\dateWaythaneDayThree}            {\typesetdate%
    {Tantor diary}{17th}{\Barion}}
\newcommand{\dateWaythaneDayFourTempleFound}  {\typesetdate%
    {Tantor diary}{18th}{\Barion}}
\newcommand{\dateEreshKalDayOne}              {%
  \dateWaythaneDayFourTempleFound}
\newcommand{\dateEreshKalDayTwo}              {\typesetdate
    {Tantor diary}{19th}{\Barion}}
\newcommand{\dateEreshKalLeftBehindDayOne}    {\typesetdate%
    {Tantor diary}{20th}{\Barion}}
\newcommand{\monthWhereTantorsDiaryEnds}      {%
  \Gamishiel}




\begin{comment}
\subsubsection{Into the Fray}
\end{comment}

\newcommand{\dateSortieFindsTown}           {\typesetdate%
    {Runger war}{23rd}{\Keshirah}}
\newcommand{\dateTirocoUndergroundWork}     {\typesetdatelink%
    {Runger war}{1st}{\Feazirah}{Feazirah}}
\newcommand{\dateRianAndJorgen}             {\typesetdatelink%
    {Runger war}{3rd}{\Feazirah}{Feazirah}}
\newcommand{\dateSortieInCombat}            {\typesetdatelink%
    {Runger war}{4th}{\Feazirah}{Feazirah}}
\newcommand{\dateRianSeesMagicRitual}       {\typesetdatelink%
    {Runger war}{6th}{\Feazirah}{Feazirah}}
\newcommand{\dateTirocoAndTheCrazyOldWoman} {\typesetdatelink%
    {Runger war}{7th}{\Feazirah}{Feazirah}}
\newcommand{\dateCarzainDreamsInWaythane}   {\typesetdatelink%
    {Runger war}{7th}{\Feazirah}{Feazirah}}
\newcommand{\dateTirocoAndMoro}             {\typesetdatelink%
    {Runger war}{7th}{\Feazirah}{Feazirah}}
\newcommand{\dateCarzainReturnsToForklin}   {\typesetdatelink% Five days after 'sortie in combat'!
    {Runger war}{9th}{\Feazirah}{Feazirah}}
\newcommand{\dateRianSeesRaid}              {\typesetdatelink%
    {Runger war}{10th}{\Feazirah}{Feazirah}}
\newcommand{\dateTirocoDreamsofWorms}       {\typesetdatelink%
    {Runger war}{11th}{\Feazirah}{Feazirah}}
\newcommand{\dateTirocoAsksMoro}            {\typesetdatelink%
    {Runger war}{12th}{\Feazirah}{Feazirah}}
\newcommand{\dateBilaComesToForklin}        {\typesetdatelink% 
    {Runger war}{18th}{\Feazirah}{Feazirah}}




\begin{comment}
\subsubsection{Spectre of the Fray}
\end{comment}

\newcommand{\dateRungeransBesiegeForclin}   {\typesetdate% 
    {Runger war}{1st}{\Barion}}
\newcommand{\dateRungeransBringUpCannons}   {\typesetdate% 
    {Runger war}{2nd}{\Barion}}
\newcommand{\datePelidorianSally}           {\typesetdate% 
    {Runger war}{2nd}{\Barion}}
\newcommand{\dateTakestshaStormsForclin}    {\typesetdate% 
    {Runger war}{3rd}{\Barion}}


\newcommand{\dateNeedleSummonsBanes}        {\typesetdatelink%
    {Runger war}{11th}{\Feazirah}{Feazirah}}

\begin{comment}
\subsubsection{Rissitic story thread}
\end{comment}

\newcommand{\dateSiegeofFendor} {\typesetdate%
    {Siege of Fendor}{18th}{\Hapheron}}























\begin{comment}
\chapter{Cosmic things}
\end{comment}


% The Universe. Is it capitalized or not?
\newcommand{\Universe}  {Universe\xspace}
\newcommand{\universe}  {universe\xspace}

% The Cosmos - probably a synonym for the Universe.
\newcommand{\Cosmos}  {Cosmos\xspace}
\newcommand{\Cosmic}  {Cosmic\xspace}
\newcommand{\cosmos}  {Cosmos\xspace}
\newcommand{\cosmic}  {Cosmic\xspace}










\begin{comment}
\section{Realms}
\end{comment}

% Miith. 
\newcommand{\Miithians}  {\Miithian{}s\xspace}
\newcommand{\Miithian}   {\Miith{}ian\xspace}
\newcommand{\Miith}      {Miith\xspace}

% The Planes. Another term for the Realms, but somewhat more general.
\newcommand{\Plane}      {Plane\xspace}
\newcommand{\Planes}     {Planes\xspace}

% Tembrae, the name of the "main" Realm of \Miith{} at the time of the Second Banewar.
\newcommand{\Tembraeans} {Tembraeans\xspace} 
\newcommand{\Tembraean}  {Tembraean\xspace}
\newcommand{\Tembrae}    {Tembrae\xspace}

% Makai, the Hellish underworld containing Inferno and Stygia.
\newcommand{\Makai}      {Machai\xspace}
\newcommand{\Machai}     {\Makai}
\newcommand{\Tuat}       {\Makai}
\newcommand{\Machaic}    {\Makai{}c\xspace}

% The Crystal Sphere, the seal that keeps the Banelords out. 
\newcommand{\CrystalSphere}    {Crystal Sphere\xspace}

% The world of the moth-things.
\newcommand{\Shuggon}          {Shuggon\xspace}











\begin{comment}
\section{The Web of Realms}
\end{comment}

% The Shrouding, previously called the "Second Shrouding". 
\newcommand{\SecondShrouding} {Shrouding\xspace}
\newcommand{\Shrouding}       {Shrouding\xspace}

% Vertex Spike: A sudden big Vertex reaction. 
\newcommand{\Spike}           {Spike\xspace}
\newcommand{\spiked}          {Spiked\xspace}
\newcommand{\spike}           {Spike\xspace}
\newcommand{\VertexSpike}     {\vertex{} \spike}
\newcommand{\vertexspike}     {\vertex{} \spike}

% Vertex: A person or item with the ability to profoundly affect the Web of the Realms. 
\newcommand{\Vertex}          {Vertex\xspace}
\newcommand{\Vertices}        {Vertices\xspace}
\newcommand{\vertex}          {Vertex\xspace}
\newcommand{\vertices}        {Vertices\xspace}

% Nexus: A geographical location especially important in the Web. 
\newcommand{\Nexus}           {Nexus\xspace}
\newcommand{\Nexi}            {Nexuses\xspace}
\newcommand{\Nexuses}         {Nexuses\xspace}
\newcommand{\nexus}           {Nexus\xspace}
\newcommand{\nexi}            {Nexuses\xspace}
\newcommand{\nexuses}         {Nexuses\xspace}

% Matrix: An organized group of Vertices. 
% A sufficiently strong Matrix may unravel much of the Web and re-weave it in their own image.
\newcommand{\Matrix}          {Matrix\xspace}
\newcommand{\Matrices}        {Matrices\xspace}
\newcommand{\matrixx}         {Matrix\xspace}
\newcommand{\matrices}        {Matrices\xspace}
% Renew the existing 'matrix' command, which I don't expect to be using. 
\renewcommand{\matrix}        {Matrix\xspace} 

% Apex: The Vertex that leads and controls a Matrix.
\newcommand{\Apex}            {Apex\xspace}
\newcommand{\Apexes}          {Apexes\xspace}
\newcommand{\apex}            {Apex\xspace}
\newcommand{\apexes}          {Apexes\xspace}

% Cardinal Points: Pivotal Vertices below the Apex.
\newcommand{\CardinalPoints}  {\CardinalPoint{}s\xspace}
\newcommand{\CardinalPoint}   {Cardinal Point\xspace}
\newcommand{\cardinalpoints}  {\CardinalPoint{}s\xspace}
\newcommand{\cardinalpoint}   {\CardinalPoint{}\xspace}

% Dweomer: A cosmic source of power. 
\newcommand{\Dweomers}        {Dweomers\xspace}
\newcommand{\Dweomer}         {Dweomer\xspace}
\newcommand{\dweomers}        {dweomers\xspace}
\newcommand{\dweomer}         {dweomer\xspace}

% Carcer: A soul prison. 
\newcommand{\Carcers}         {Carcers\xspace}
\newcommand{\Carcer}          {Carcer\xspace}
\newcommand{\carcers}         {Carcers\xspace}
\newcommand{\carcer}          {Carcer\xspace}










\begin{comment}
\section{Cosmic gods}
\end{comment}



\newcommand{\Zaz}              {Zaz\xspace}
\newcommand{\Urzaz}            {Urzaz\xspace}










\begin{comment}
\section{Chaos Magic}
\end{comment}

% Spellwords. 

% Guth'nyad is used by Ishnaruchaefir in "What Slithers Beneath". 
\newcommand{\Guthnyad} {G\ulong{}th'nyad\xspace}















\begin{comment}
\chapter{Wild things}
\end{comment}

% Wildfog, the foglike substance that obscures the Wild.
\newcommand{\Wildfog}  {Wylde-fog\xspace}
\newcommand{\wildfog}  {Wylde-fog\xspace}

% The Wild.
\newcommand{\Wylder}   {Wylder\xspace}
\newcommand{\Wylde}    {Wylde\xspace}
\newcommand{\wylde}    {Wylde\xspace}

% Eidola: Totems that keep the Wylde at bay.
\newcommand{\Eidolon}  {Eidolon\xspace}
\newcommand{\Eidola}   {Eidola\xspace}
\newcommand{\eidolon}  {Eidolon\xspace}
\newcommand{\eidola}   {Eidola\xspace}









\begin{comment}
\section{People}
\end{comment}

% The Goydens, a savage people living in the Wild in Pelidor. 
\newcommand{\Goyden}  {Go\y den\xspace}
\newcommand{\Goydens} {\Goyden{}s\xspace}

% Rangers, people with a special affinity for the Wild.
\newcommand{\Rangers} {Rangers\xspace}
\newcommand{\Ranger}  {Ranger\xspace}
\newcommand{\rangers} {rangers\xspace}
\newcommand{\ranger}  {ranger\xspace}

% Nycaneers. 
\newcommand{\Nycaneers} {\Nycaneer{}s\xspace}
\newcommand{\Nycaneer}  {\Nycan{}eer\xspace}
\newcommand{\nycaneers} {\nycaneer{}s\xspace}
\newcommand{\nycaneer}  {\Nycaneer}

% Melda, a Nycan's commanding officer.
\newcommand{\Meldae}    {Meldae\xspace}
\newcommand{\Melda}     {Melda\xspace}
\newcommand{\meldae}    {Meldae\xspace}
\newcommand{\melda}     {Melda\xspace}















% \begin{comment}
% \part{Cultures of \Miith{}}
% \end{comment}















\begin{comment}
\chapter{Azmith}
\end{comment}

% The `Known World', the continent on which stuff takes place. 
% Azmith means `all \Miith{}'
\newcommand{\Azmithian}   {Ezmithian\xspace} 
\newcommand{\Azmith}      {Ezmith\xspace} 
\newcommand{\Ezmith}      {Ezmith\xspace} 

% Kai Leng, the Underworld.
\newcommand{\KaiLeng}     {Kai-Leng\xspace}







\begin{comment}
\section{Imetrium}
\end{comment}




\begin{comment}
\subsection{Culture}
\end{comment}

% Imetric military ranks, in descending order.
\newcommand{\Deccor}   {Deccor\xspace}
\newcommand{\Retaxis}  {Retaxis\xspace}
\newcommand{\Salican}  {Salican\xspace}
\newcommand{\Vexstra}  {Vexter\xspace}
\newcommand{\Corphin}  {Corphin\xspace}
\newcommand{\Inclan}   {Inclan\xspace}
% Generic word for `soldier' in singular and plural.
\newcommand{\Rengos}   {Rengos\xspace}
\newcommand{\Rengoi}   {Rengoi\xspace}

% Imetric clerical ranks, in descending order.
\newcommand{\Laccorin} {Laccorin\xspace}
\newcommand{\Ispan}    {Ispan\xspace}
\newcommand{\Telphan}  {Telphan\xspace}
\newcommand{\Amra}     {Amra\xspace}
% Generic word for `priest' in singular and plural.
\newcommand{\Stracos}  {Stracos\xspace}
\newcommand{\Stracoi}  {Stracoi\xspace}




\begin{comment}
\subsection{Geography}
\end{comment}

% The Risvael Sea, the large sea between Belkade, Durcac and the Imetrium
\newcommand{\Risvael}     {Risvael\xspace}
\newcommand{\Risvaelsea}  {Risvael Sea\xspace}
\newcommand{\Risvalsea}   {Risvael Sea\xspace}
% The two fortress-towns on Fendor
\newcommand{\FendorSmall} {\c Cicora\xspace} % smaller, western one
\newcommand{\Cicora}      {\FendorSmall}
\newcommand{\FendorLarge} {Fendacor\xspace} % bigger, eastern one
\newcommand{\FendorBig}   {\FendorLarge}
\newcommand{\Fendacor}    {\FendorLarge}
% People of Cicora
\newcommand{\Cicoran}     {\Cicora{}n\xspace}
\newcommand{\Cicorans}    {\Cicoran{}s\xspace}

% The fortress-town on Tugan (the smaller island in Risvael)
\newcommand{\TuganTown}   {Pandex\xspace}





\begin{comment}
\subsection{Religion}
\end{comment}


% Setthias, an Imetric goddess
\newcommand{\Setthias}  {Settias\xspace}
% Nishi-Setthias, an Imetric goddess
\newcommand{\NishiS}    {Nishi-\Setthias}
% Hiothrex
\newcommand{\Hiothrex}  {Hiothrex\xspace}
% Telderain, the Northstar sword
\newcommand{\Telderain} {Teldraxxus\xspace}

% The Imetriad, the holy book of the Imetrium.
\newcommand{\Imetriad}  {Imetriad\xspace}















\begin{comment}
\section{Meccaran cultures}
\end{comment}


\newcommand{\Meccarans}  {Mecc\along rans\xspace} % Alternate plural. 
\newcommand{\Meccaran}   {Mecc\along ran\xspace}  % Singular.
\newcommand{\Meccara}    {Mecc\along ra\xspace}   % Plural.
\newcommand{\meccarans}  {Mecc\along rans\xspace} % Alternate plural. 
\newcommand{\meccaran}   {Mecc\along ran\xspace}  % Singular.
\newcommand{\meccara}    {Mecc\along ra\xspace}   % Plural.



\begin{comment}
\subsection{Tribes}
\end{comment}

\newcommand{\Thbatswa} {Thbatswa\xspace}















\begin{comment}
\section{Rissitics}
\end{comment}










\begin{comment}
\subsection{Geography}
\end{comment}

% Isshuri, the Durcaci continent.
\newcommand{\DurcacContinent} {Isshuri\xspace}
\newcommand{\Durcaccontinent} {\DurcacContinent}

% Durcac, homeland of the Rissitics
\newcommand{\Durcac}          {Drucar\xspace}
\newcommand{\Durcaci}         {Drucari\xspace}










\begin{comment}
\subsection{Gods and mythology}
\end{comment}

% Hriist'tet Nechsain, aka Rissit
\newcommand{\RissitNechsain}   {Rissit \Nechsain}
\newcommand{\Hriist}           {Hriist'tet\xspace}
\newcommand{\Nechsains}        {Nekhsains\xspace}
\newcommand{\Nechsain}         {Nekhsain\xspace}
%\newcommand{\Nechsain}        {Nechs\^ain{}\xspace}
\newcommand{\HriistN}          {\Hriist{} \Nechsain}

% Major gods
\newcommand{\Esphet}           {Es'phet{}\xspace} % Goddess of the wind and weather
\newcommand{\Rektet}           {Rektet{}\xspace} % Queen of Swarming Insects
\newcommand{\Tchesef}          {Tchesef{}\xspace} % God of earth and sand

% Maskim or Masqim, malevolent beings of the Shadow World in Rissitic mythology.
\newcommand{\Maskim}           {Maskim{}\xspace}
\newcommand{\Maskims}          {\Maskim{}\xspace}
\newcommand{\Masqim}           {\Maskim{}\xspace}
\newcommand{\Masqims}          {\Maskims{}\xspace}
\newcommand{\Neghir}           {Neg'h\ilong r{}\xspace} % % The major gods of the Rissitics, below Nechsain.
\newcommand{\Annunaki}         {\Neghir{}\xspace}
\newcommand{\Igigi}            {Igigi{}\xspace} % Collective term for minor spirits or gods in Rissitism.

% Creatures of the Body World.
\newcommand{\Asakku}{Asakku{}\xspace} 
\newcommand{\Asakkus}{\Asakku{}s{}\xspace}
\newcommand{\BodyCreature}{\Asakku{}\xspace}
\newcommand{\BodyCreatures}{\Asakkus{}\xspace}
\newcommand{\bodycreature}{\Asakku{}\xspace}
\newcommand{\bodycreatures}{\Asakkus{}\xspace}

% Creatures of the Spirit World.
\newcommand{\Utukku}{Utukku{}\xspace} 
\newcommand{\Utukkus}{\Utukku{}s{}\xspace}
\newcommand{\SpiritCreature}{\Utukku{}\xspace}
\newcommand{\SpiritCreatures}{\Utukkus{}\xspace}
\newcommand{\spiritcreature}{\Utukku{}\xspace}
\newcommand{\spiritcreatures}{\Utukkus{}\xspace}

% Creatures of the Shadow World.
\newcommand{\Ekimmu}{Ekimmu{}\xspace} 
\newcommand{\Ekimmus}{\Ekimmu{}s{}\xspace}
\newcommand{\ShadowCreature}{\Ekimmu{}\xspace}
\newcommand{\ShadowCreatures}{\Ekimmus{}\xspace}
\newcommand{\shadowcreature}{\Ekimmu{}\xspace}
\newcommand{\shadowcreatures}{\Ekimmus{}\xspace}











\begin{comment}
\subsection{Social things}
\end{comment}

% Priest caste. 
\newcommand{\Nyzlet} {N\yvowelrissitic zlet{}\xspace}
\newcommand{\nyzlet} {\Nyzlet{}\xspace}

% Knight caste. 
\newcommand{\Reken}  {Rek\'en{}\xspace}
\newcommand{\reken}  {\Reken{}\xspace}
\newcommand{\Rekkan} {\Reken{}\xspace}

% Craftsman caste.
\newcommand{\Bedhin} {Be\rissdh{}\ilong{}n\xspace} 
\newcommand{\bedhin} {\Bedhin}

% Labourer caste.
\newcommand{\Hoka}   {\Hok{}\xspace} 
\newcommand{\Hok}    {H\oroundrissitic{}k{}\xspace}
\newcommand{\hok}    {\Hok{}\xspace}

% Warrior caste.
\newcommand{\Kyth}   {K\yvowelrissitic th{}\xspace} 
\newcommand{\kyth}   {\Kyth{}\xspace}

% The \Keffoydh, the people of Nechsain.
\newcommand{\Keffoydh}  {Keff\rissoe \yvowelrissitic \rissdh{}\xspace}
\newcommand{\Keffoidh}  {\Keffoydh{}\xspace}
\newcommand{\Rissitic}  {\Keffoidh{}\xspace}
\newcommand{\Rissitics} {\Rissitic{}\xspace}
\newcommand{\Redhaifyn} {\Keffoidh{}\xspace} % The faithful

% Orders and titles among the Rissitics
\newcommand{\XulGann}          {Xul-Gann\xspace}
\newcommand{\RissLich}         {Xul-Gann\xspace}
\newcommand{\UrrGammosh}       {Xul-Gann\xspace}

\newcommand{\Faledh}           {Fale\rissdh{}\xspace} % Priestly subordinate
\newcommand{\Uriko}            {Uriko\xspace} % A person's secular superior

\begin{comment}
\subsubsection{Tsalt, the Priest Caste}
\end{comment}
% High Priest
\newcommand{\TsaltNyzleth}{Tsalt-N\yvowelrissitic zleth{}\xspace} 
% The three orders of Tsalt
% Scientists, theologists, mages
\newcommand{\Dzeyrgvin}{Dzeyrgvin{}\xspace} 
% Battle mages
\newcommand{\Sheshefkesad}{Shesshefk\'esad{}\xspace}
\newcommand{\Shesshefkesad}{\Sheshefkesad{}\xspace}
\newcommand{\Shessefkesad}{\Sheshefkesad{}\xspace}
% Administrator priests
\newcommand{\Tegiul}{Tegiul{}\xspace} 
% The four ranks of Tsalt
% Highest rank
\newcommand{\Kseinga}{Kseinga{}\xspace} 
% = Giisshef
\newcommand{\Ginfik}{Khiffesh{}\xspace}  
\newcommand{\Khiffesh}{\Ginfik{}\xspace}
\newcommand{\Ryzeyd}{R\yvowelrissitic ze\yconsonantrissitic d{}\xspace}
% Lowest rank
\newcommand{\Bryn}{Br\yvowelrissitic{}n{}\xspace} 

% Apprentice priests, not yet assigned to an order
\newcommand{\Hifrib}{Hifrib{}\xspace} 



\begin{comment}
\subsubsection{Rekkan, the Knight Caste}
\end{comment}
% The military ranks of the Rekkan
% Supreme commander of the Rissitic armies
\newcommand{\Efririm}    {Efr\ilong rim{}\xspace} 
% General
\newcommand{\Neftzaid}   {Nefts\adarkrissitic id{}\xspace} 
\newcommand{\Neftsaid}   {\Neftzaid} 
% Captain
\newcommand{\Ondmyst}    {Ondm\yvowelrissitic st\xspace} 
% Captain or Lieutenant 
\newcommand{\Kozud}      {Kozud\xspace} 

% Warrior mages
\newcommand{\Ashenoch}   {Ashenoch\xspace} 
% The three tiers of Ashenoch
% Highest
\newcommand{\Hashkfed}   {Hashkfed\xspace} 
\newcommand{\Kichodd}    {Kichodd\xspace}

\newcommand{\Fedza}      {Fedza\xspace} 

% Ninjas
\newcommand{\Gishorn}    {G\ilong sshorn\xspace} 
\newcommand{\Gisshorn}   {\Gishorn}
\newcommand{\Gisshorns}  {\Gisshorn}















\begin{comment}
\section{Scathaese cultures}
\end{comment}

% Baccon, a type of council
\newcommand{\Baccons}     {\Baccon{}s\xspace}
\newcommand{\Baccon}      {Pathyon\xspace}
\newcommand{\baccons}     {\Baccons}
\newcommand{\baccon}      {\Baccon}
% Bacconate, a nation ruled by a baccon.
\newcommand{\Bacconates}  {\Bacconate{}s\xspace}
\newcommand{\Bacconate}   {\Baccon{}ate\xspace}
\newcommand{\bacconates}  {\bacconate{}s\xspace}
\newcommand{\bacconate}   {\baccon{}ate\xspace}
% Raebar, a member of a Baccon. 
\newcommand{\Raebari}     {Raebari\xspace}
\newcommand{\Raebar}      {Raebar\xspace}
\newcommand{\raebari}     {Raebari\xspace}
\newcommand{\raebar}      {Raebar\xspace}










\begin{comment}
\subsection{Ortaica}
\end{comment}

% The Ortaican religion.
\newcommand{\Ortaicanism}       {Ortaicanism\xspace}

% The Ortaicans.
\newcommand{\Ortaicans}         {Ortaicans\xspace}
\newcommand{\Ortaican}          {Ortaican\xspace}

% Ortaica, a Scathaese Bacconate. 
\newcommand{\Ortaica}           {Ortaica\xspace}

% The Rethyax, the Ortaican order of Chaos mages. 
\newcommand{\rethyactic}        {Rethyactic\xspace}
\newcommand{\rethyaxes}         {\rethyax}
\newcommand{\rethyax}           {\Rethyax}
\newcommand{\Rethyactic}        {Rethyactic\xspace}
\newcommand{\Rethyaxes}         {\Rethyax}
\newcommand{\Rethyax}           {Rethyax\xspace}

% Arcana. 
\newcommand{\Arcana}            {Arcana\xspace}
\newcommand{\Arcanum}           {Arcanum\xspace}
\newcommand{\arcana}            {Arcana\xspace}
\newcommand{\arcanum}           {Arcanum\xspace}

% Leges, Ortaican basic laws.
\newcommand{\lex}               {Lex}
\newcommand{\leges}             {Leges}
\newcommand{\Lex}               {Lex}
\newcommand{\Leges}             {Leges}










\begin{comment}
\subsection{Other}
\end{comment}

% Masthenon, an ancient Scathaese people. 
\newcommand{\Masthenon}        {Masthenon\xspace} % Plural. 
\newcommand{\Mastheno}         {Mastheno\xspace}  % Adjective. 
\newcommand{\Masthen}          {Masthen\xspace}     % Singular.

% Tepharae, the Tepharin Bacconate. 
\newcommand{\Tepharae}         {Tepharae\xspace}

% Tephar, the capital city of the Tepharin people. 
\newcommand{\Tephar}           {Tephar\xspace}

% The Tepharites.
\newcommand{\Tepharites}       {Tepharites\xspace}
\newcommand{\Tepharite}        {Tepharite\xspace}
\newcommand{\Tepharins}        {Tepharites\xspace}
\newcommand{\Tepharin}         {Tepharin\xspace} % Adjective. 

% The Samurites.
\newcommand{\Samurite}         {Samurite\xspace}
\newcommand{\Samurites}        {Samurites\xspace}
\newcommand{\Samurin}          {Samurin\xspace} % Adjective. 
\newcommand{\Samure}           {Samur\finale{}\xspace} % The region.
\newcommand{\Samur}            {Samur\xspace} % The people as a whole. 

% The Ortics. 
\newcommand{\Ortican}          {Ortic\xspace}
\newcommand{\Orticans}         {Ortics\xspace}
\newcommand{\Ortic}            {Ortic\xspace} % Adjective. 

% The Shurco, a Scathaese civilization that existed during the Vaimon Caliphate.
\newcommand{\Shurcos}          {\Shurco}
\newcommand{\Shurco}           {Meshemku\xspace}
\newcommand{\Shurcarie}        {\Shurco}
%\newcommand{\Shurcarie}        {Shurcari\finale{}\xspace}

% Yormis, a city of sorcerers.
\newcommand{\Yormis}           {Yormis\xspace}
\newcommand{\Yormissian}       {Yormissian\xspace}

% Shrun, a mountain near Yormis.
\newcommand{\Shrun}            {Shr\ulong n\xspace}



















\begin{comment}
\section{Vaimon: General}
\end{comment}


% Iquin and Nieur, the forces of Light and Darkness.
\newcommand{\Iquin}            {Iquin{}\xspace}
\newcommand{\iquin}            {\Iquin{}\xspace}
\newcommand{\Nieur}            {\Itzach{}\xspace}
\newcommand{\nieur}            {\itzach{}\xspace}
\newcommand{\Itzach}           {Itzach{}\xspace}
\newcommand{\itzach}           {\Itzach{}\xspace}

% Clerics and Templars: Priest and warrior Vaimons. 
\newcommand{\clerics}          {\cleric{}s{}\xspace}
\newcommand{\cleric}           {cleric{}\xspace}
\newcommand{\Clerics}          {\Cleric{}s{}\xspace}
\newcommand{\Cleric}           {Cleric{}\xspace}
\newcommand{\templars}         {\templar{}s{}\xspace}
\newcommand{\templar}          {templar{}\xspace}
\newcommand{\Templars}         {\Templar{}s{}\xspace}
\newcommand{\Templar}          {Templar{}\xspace}









\begin{comment}
\subsection{Magic and religion}
\end{comment}

% Archon, the collective term for the various spirits of Vaimon metaphysics
\newcommand{\Archon}           {Archon\xspace}
\newcommand{\Archons}          {Archons\xspace}
\newcommand{\archon}           {Archon\xspace}
\newcommand{\archons}          {Archons\xspace}

% The Empyrean, the place where the Archons are said to dwell. 
\newcommand{\Empyrean}         {Empyrean\xspace}
\newcommand{\empyrean}         {Empyrean\xspace}

% Sephiroth, the Archons of Iquin
\newcommand{\Sephiroth}        {Sephiroth\xspace}
\newcommand{\Sephirah}         {Sephir\ahresphan{}\xspace}
\newcommand{\sephiroth}        {Sephiroth\xspace}
\newcommand{\sephirah}         {\Sephirah}

% Kliffoth, the Archons of Nieur
\newcommand{\Kliffoth}         {Qliphoth\xspace}
\newcommand{\Kliffah}          {Qliph\ahresphan{}\xspace}
\newcommand{\Qliphah}          {\Kliffah{}\xspace}
\newcommand{\Qliphoth}         {\Kliffoth{}\xspace}
\newcommand{\kliffoth}         {Qliphoth{}\xspace}
\newcommand{\kliffah}          {Qliphah{}\xspace}
\newcommand{\qliphah}          {\kliffah{}\xspace}
\newcommand{\qliphoth}         {\kliffoth{}\xspace}

% Malachim, the Archons who incarnate as Scions
\newcommand{\Malachim}         {Neshamoth\xspace}
\newcommand{\Malach}           {Nesham\ahresphan{}\xspace}
\newcommand{\malachim}         {Neshamoth\xspace}
\newcommand{\malach}           {Nesham\ahresphan{}\xspace}

% Kenosis, the process of a Malach becoming a Scion. 
\newcommand{\Kenosis}          {Kenosis\xspace}
\newcommand{\kenosis}          {Kenosis\xspace}

% Apotheosis, the process of a Scion regaining his Malach glory. 
\newcommand{\Apotheosis}       {Apotheosis\xspace}
\newcommand{\apotheosis}       {Apotheosis\xspace}

% "Angels".
\newcommand{\Angel}            {Angel{}\xspace}
\newcommand{\Angels}           {\Angel{}s{}\xspace}
\newcommand{\Angelic}          {\Angel{}ic{}\xspace}
\newcommand{\angel}            {angel{}\xspace}
\newcommand{\angels}           {\angel{}s{}\xspace}
\newcommand{\angelic}          {\angel{}ic{}\xspace}



\newcommand{\Atziluth}         {Atziluth\xspace} % Atziluth, the highest world.
\newcommand{\Gehinnom}         {Gehinnom\xspace} % Gehinnom, the physical world. 
\newcommand{\Isphet}           {Isphet\xspace} % Isphet, a dark god.
\newcommand{\Shechinah}        {Shechin\ahresphan{}\xspace}
\newcommand{\shechinah}        {\Shechinah} % State of being in touch with the Archons. 
\newcommand{\Tikkun}           {Tikkun\xspace} % The sacred task of restoring the world to unity. 
\newcommand{\tikkun}           {Tikkun\xspace}

% The four elements of Iquin. 
\newcommand{\Mor}              {Mor\xspace}
\newcommand{\Shiram}           {Shir\along m\xspace}
\newcommand{\Urisol}           {Urisol\xspace}
\newcommand{\Zumir}            {Zum\ilong r\xspace}







\begin{comment}
\subsubsection{Sephiroth}
\end{comment}

% The sixteen Sephiroth are divided into four groups, one group of four Sephiroth for each of the classical elements. 
% This is reflected in their names:
% Sephiroth of Fire end in `on'.
% Sephiroth of Ait end in `ah'.
% Sephiroth of Earth end in `ed'.
% Sephiroth of Water end in `el'.

% Air:
\newcommand{\Shatzirah} {Sh\aflatresphan tzir\ahresphan\xspace}
\newcommand{\Atzirah}   {Sh\aflatresphan tzir\ahresphan\xspace}
\newcommand{\Feazirah}  {F\'e\aflatresphan zir\ahresphan\xspace}
\newcommand{\Keshirah}  {Keshir\ahresphan\xspace}
\newcommand{\Razilah}   {R\aflatresphan zil\ahresphan\xspace}

% Fire:
\newcommand{\Barion}    {B\alongresphan rion\xspace}
\newcommand{\Hapheron}  {H\aflatresphan pheron\xspace}
\newcommand{\Sizion}    {S\ilongresphan zion\xspace} % He Who Smites With Flame
\newcommand{\Izion}     {S\ilongresphan zion\xspace} % He Who Smites With Flame
\newcommand{\Teshiron}  {Teshiron\xspace}

% Earth:
\newcommand{\Zumrad}    {Zumrad\xspace}
\newcommand{\Cushed}    {Zumrad\xspace}
\newcommand{\Tzadrued}  {Tzadrued\xspace}
\newcommand{\Hoshied}   {Tzadrued\xspace}
\newcommand{\Thimarod}  {Thim\adarkresphan rod\xspace}
\newcommand{\Thimared}  {Thim\adarkresphan rod\xspace}
\newcommand{\Magaled}   {M\alongresphan g\adarkresphan led\xspace}
\newcommand{\Yemared}   {M\alongresphan g\adarkresphan led\xspace}

% Water:
\newcommand{\Vamishiel} {V\aflatresphan mishiel\xspace}
\newcommand{\Gamishiel} {V\aflatresphan mishiel\xspace}
\newcommand{\Lishiel}   {Lishiel\xspace}
\newcommand{\Ishiel}    {Lishiel\xspace}
\newcommand{\Nomariel}  {Nom\alongresphan riel\xspace}
\newcommand{\Omariel}   {Nom\alongresphan riel\xspace}
\newcommand{\Yeziel}    {\Y eziel\xspace}



% Older names for Sephiroth, now obsolete and replaced
% Sezyron, a Sephirah of Air who creates bolts of lightning from your hands.
\newcommand{\Sezyron}   {\Razilah{}\xspace}
% Bihirai, a Sephirah of Healing
\newcommand{\Bihirai}{\Ishiel{}\xspace}
% A Sephirah of the Air, used to create controlled gusts of wind. 
\newcommand{\Brycorre}{\Keshirah{}\xspace}
% A Sephirah of the Earth, used to shape, move and Sculpt earthen objects, but can also be used to paralyze a person. 
\newcommand{\Curomon}{\Cushed{}\xspace}
% Feazin, used for flying
\newcommand{\Feazin}{\Atzirah{}\xspace}









\begin{comment}
\subsubsection{Qliphoth}
\end{comment}

\newcommand{\Abbath}           {Abbath\xspace}
\newcommand{\Bozchul}          {Banath-crul\xspace} % Brings death.
\newcommand{\Djerzad}          {Djerzad{}\xspace}   % Causes bones to break. 
\newcommand{\Dweryog}          {Dwerjeel\xspace}
\newcommand{\Gavron}           {Gavron{}\xspace}    % Used to cut stuff
\newcommand{\Horvaleth}        {Horvaleth{}\xspace} % The Cruel Winter. 
\newcommand{\Iachadion}        {Iachadion\xspace}
\newcommand{\Iomelech}         {Iomelech\xspace}
\newcommand{\Iphicoss}         {Iphicoss{}\xspace}  % The Treacherous Gale. 
\newcommand{\Kithvaz}          {Kithva\v z{}\xspace} % Attacks the mind.
\newcommand{\KorRashad}        {Kor-Rashad\xspace}  % The guide through the Empyrean.
\newcommand{\Micozalra}        {Micozalra\xspace}
\newcommand{\OmmonThul}        {Ommon-Tul\xspace}  % Slows victims.
\newcommand{\Oronigath}        {Oronigath\xspace}
\newcommand{\Ozorugai}         {Ozorugai\xspace}
\newcommand{\Nyxachel}         {N\yvowel xachel{}\xspace} % Lightning.
\newcommand{\Shabolan}         {Shabolan\xspace}
\newcommand{\Shurreem}         {Shurreem\xspace}    % Creates harmless black fire. 
\newcommand{\Zobbath}          {Zobbath\xspace}









\begin{comment}
\subsection{Calendar}
\end{comment}

% The Draconian Supremacy calendar. 
\newcommand{\DS}               {DS\xspace}

% The Black Dawn calendar. 
\newcommand{\BD}               {BD\xspace}

% The Vaimon Calendar.
\newcommand{\VaimonCalendar}   {Vaimon Calendar\xspace}
\newcommand{\ImperialCalendar} {Vaimon Calendar\xspace}
\newcommand{\IC}               {VC\xspace}
\newcommand{\VC}               {VC\xspace}

% The eight days of the week.
\newcommand{\Corjin}           {Corjin\xspace}
\newcommand{\Zetherab}         {Zetherab\xspace}
\newcommand{\Setherab}         {Zetherab\xspace}
\newcommand{\Rebecab}          {Rebecab\xspace}
\newcommand{\Arcab}            {Arcab\xspace}
\newcommand{\Norquin}          {Norquin\xspace}
\newcommand{\Tirjin}           {Tirjin\xspace}
\newcommand{\Kerzab}           {Kerzab\xspace}
\newcommand{\Siljin}           {Siljin\xspace}

% The last day of the year. 
\newcommand{\Camaire} {Camair\finale{}\xspace}









\begin{comment}
\subsection{Geography}
\end{comment}

% Imrath, the nation where Cordos Vaimon was king
\newcommand{\Imrathic}         {\Imrath{}ic\xspace}
\newcommand{\Imrathi}          {\Imrath{}i\xspace}
\newcommand{\Imrath}           {Imrath\xspace}
% Shi'in Merodar, the old capital city of the Vaimon Caliphate.
\newcommand{\ShiinMerodar}     {Shiin Merodar\xspace}
\newcommand{\Merodar}          {\ShiinMerodar}

\newcommand{\Vymorjans}        {\Vymorjan{}s\xspace}
\newcommand{\Vymorjan}         {\Vymorja{}n\xspace}
\newcommand{\Vymorja}          {V\ydiphthong morja\xspace}









\begin{comment}
\subsection{Historical things}
\end{comment}

% The Darkfall, the cataclysmic downfall of the Vaimon Caliphate.
\newcommand{\Darkfall}        {\HundredScourges}
\newcommand{\darkfall}        {\HundredScourges}
\newcommand{\HundredScourges} {Hundred Scourges\xspace}

% Dragoncraft, a term for black magic in ancient times.
\newcommand{\Dragoncraft}     {\Dragon{}craft\xspace}
\newcommand{\dragoncraft}     {\dragon{}craft\xspace}

% Vaimon Caliph and Caliphate. 
\newcommand{\VaimonCaliphate} {Vaimon Caliphate\xspace}
\newcommand{\VaimonCaliphs}   {Vaimon Caliphs\xspace}
\newcommand{\vaimoncaliphs}   {Vaimon Caliphs\xspace}
\newcommand{\VaimonCaliph}    {Vaimon Caliph\xspace}
\newcommand{\vaimoncaliph}    {Vaimon Caliph\xspace}
\newcommand{\Caliphate}       {Caliphate\xspace}
\newcommand{\Caliphas}        {Caliphas\xspace}
\newcommand{\Calipha}         {Calipha\xspace}
\newcommand{\Caliphs}         {Caliphs\xspace}
\newcommand{\Caliph}          {Caliph\xspace}
\newcommand{\caliphate}       {Caliphate\xspace}
\newcommand{\caliphas}        {Caliphas\xspace}
\newcommand{\calipha}         {Calipha\xspace}
\newcommand{\caliphs}         {Caliphs\xspace}
\newcommand{\caliph}          {Caliph\xspace}









\begin{comment}
\subsection{Other things}
\end{comment}

% Truesilver, the Vaimon super-metal.
\newcommand{\Truesilver}{Truesilver{}\xspace}
\newcommand{\truesilver}{\Truesilver{}\xspace}

% Ishrah, the term for an organized group of mages.
\newcommand{\Ishroth}  {Ishroth\xspace}
\newcommand{\Ishrah}   {Ishr\ahresphan{}\xspace}
\newcommand{\ishroth}  {ishroth\xspace}
\newcommand{\ishrah}   {ishr\ahresphan{}\xspace}



\begin{comment}
\subsubsection{Weapons and martial arts}
\end{comment}

% A medium-heavy sabre. 
\newcommand{\chandre}   {\c ch\^andre{}\xspace}

% The art of wielding a chandre. 
\newcommand{\chatresse} {\c ch\^atr\`esse{}\xspace}



















\begin{comment}
\section{Vaimon Clans}
\end{comment}

\newcommand{\VaimonClans}      {Vaimon clans\xspace}
\newcommand{\VaimonClan}       {Vaimon clan\xspace}
\newcommand{\VClans}           {Clans\xspace}
\newcommand{\VClan}            {Clan\xspace}
\newcommand{\vclans}           {clans\xspace}
\newcommand{\vclan}            {clan\xspace}
\newcommand{\ClanDelaen}       {Clan Delaen\xspace}
\newcommand{\ClanDelain}       {Clan Delaen\xspace}
\newcommand{\ClanGeican}       {Clan Geican\xspace}
\newcommand{\ClanRedcor}       {Clan Redcor\xspace}
\newcommand{\ClanZether}       {Clan Zether\xspace}
\newcommand{\ClanSether}       {Clan Sether\xspace}
\newcommand{\ClanTelcra}       {Clan Telcra\xspace}









\begin{comment}
\subsection{Irgel}
\end{comment}


% Clan Irgel. 
\newcommand{\Irgel}{Irgel{}\xspace}









\begin{comment}
\subsection{Redcor}
\end{comment}

\begin{comment}
\subsubsection{Orders}
\end{comment}

% Ryzin, a female Redcor Templar. 
\newcommand{\Ryzin}       {R\yvowelredcor \v zin{}\xspace}
\newcommand{\Ryzins}      {\Ryzin{}\xspace}
\newcommand{\ryzin}       {\Ryzin{}\xspace}
\newcommand{\ryzins}      {\ryzin{}\xspace}
% Gandierre, a male Redcor Templar.
\newcommand{\Gandierres}  {\Gandierre{}s\xspace}
\newcommand{\Gandierre}   {G\^andi\`erre\xspace}
\newcommand{\gandierres}  {\Gandierres}
\newcommand{\gandierre}   {\Gandierre}

\begin{comment}
\subsubsection{Geography}
\end{comment}

% Redce, the homeland of the Redcor
\newcommand{\Redce}           {Redc\'e{}\xspace}
\newcommand{\Redcean}         {\Redce{}an{}\xspace}

% The Topaz Chateau, from which the Redcor Conclave rules.
\newcommand{\Chateau}         {Ch\^ateau{}\xspace}
\newcommand{\TopazChateau}    {Topaz \Chateau{}\xspace}

\begin{comment}
\subsubsection{Factions}
\end{comment}

% The factions, or houses, or whatever.
\newcommand{\RedcorHouses}    {\RedcorHouse{}s{}\xspace}
\newcommand{\RedcorHouse}     {House{}\xspace}
\newcommand{\RedcorFaction}   {\RedcorHouse{}\xspace}

% The Swan Faction. 
\newcommand{\theSwanFaction}  {the \SwanFaction{}\xspace}
\newcommand{\TheSwanFaction}  {The \SwanFaction{}\xspace}
\newcommand{\SwanFaction}     {\Swan{} \RedcorFaction{}\xspace}
\newcommand{\Swan}            {Swan{}\xspace}
\newcommand{\theFoxFaction}   {the \FoxFaction{}\xspace}
\newcommand{\TheFoxFaction}   {The \FoxFaction{}\xspace}
\newcommand{\FoxFaction}      {\Fox{} \RedcorFaction{}\xspace}
\newcommand{\Fox}             {Fox{}\xspace}
\newcommand{\theTulipFaction} {the \TulipFaction{}\xspace}
\newcommand{\TheTulipFaction} {The \TulipFaction{}\xspace}
\newcommand{\TulipFaction}    {\Tulip{} \RedcorFaction{}\xspace}
\newcommand{\Tulip}           {Tulip{}\xspace}









\begin{comment}
\subsection{Telcra}
\end{comment}

% The 'Tiger Vaimons' are a new clan founded during the Vaimon Caliphate. 
\newcommand{\TigerVaimon}  {\Telcra{}\xspace}
\newcommand{\Telcras}      {\Telcra{}\xspace}
\newcommand{\Telcra}       {Telcra{}\xspace}









\begin{comment}
\subsection{Iquinian Church}
\end{comment}

% The name of the religion.
\newcommand{\Iquinian}   {Iquinian{}\xspace}
\newcommand{\Iquinians}  {\Iquinian{}s{}\xspace}
\newcommand{\iquinian}   {Iquinian{}\xspace}
\newcommand{\iquinians}  {\iquinian{}s{}\xspace}

% Iquinian clerical ranks. 
\newcommand{\Neophyte}    {Neophyte{}\xspace}
\newcommand{\Neophytes}   {\Neophyte {}\xspace}
\newcommand{\Soror}       {Soror{}\xspace}
\newcommand{\Sorors}      {\Soror{}s{}\xspace}
\newcommand{\Sorores}     {\Sorors{}\xspace}
\newcommand{\Frater}      {Frater{}\xspace}
\newcommand{\Fraters}     {\Frater{}s{}\xspace}
\newcommand{\Fratres}     {\Fraters{}\xspace}
\newcommand{\Pater}       {Pater{}\xspace}
\newcommand{\Paters}      {\Pater{}s{}\xspace}
\newcommand{\Patres}      {\Paters{}\xspace}
\newcommand{\Mater}       {Mater{}\xspace}
\newcommand{\Maters}      {\Mater{}s{}\xspace}
\newcommand{\Matres}      {\Maters{}\xspace}
\newcommand{\Patron}      {Patron{}\xspace}
\newcommand{\Patrons}     {\Patron{}s{}\xspace}
\newcommand{\Patriarch}   {Patriarch{}\xspace}
\newcommand{\Patriarchs}  {\Patriarch{}s{}\xspace}
\newcommand{\Matron}      {Matron{}\xspace}
\newcommand{\Matrons}     {\Matron{}s{}\xspace}
\newcommand{\Matriarch}   {Matriarch{}\xspace}
\newcommand{\Matriarchs}  {\Matriarch{}s{}\xspace}
\newcommand{\neophyte}    {\Neophyte{}\xspace}
\newcommand{\neophytes}   {\neophyte {}\xspace}
\newcommand{\soror}       {\Soror{}\xspace}
\newcommand{\sorors}      {\soror{}s{}\xspace}
\newcommand{\sorores}     {\sorors{}\xspace}
\newcommand{\frater}      {\Frater{}\xspace}
\newcommand{\fraters}     {\frater{}s{}\xspace}
\newcommand{\fratres}     {\fraters{}\xspace}
\newcommand{\pater}       {\Pater{}\xspace}
\newcommand{\paters}      {\pater{}s{}\xspace}
\newcommand{\patres}      {\paters{}\xspace}
\newcommand{\mater}       {\Mater{}\xspace}
\newcommand{\maters}      {\mater{}s{}\xspace}
\newcommand{\matres}      {\maters{}\xspace}
\newcommand{\patron}      {\Patron{}\xspace}
\newcommand{\patrons}     {\patron{}s{}\xspace}
\newcommand{\patriarch}   {\Patriarch{}\xspace}
\newcommand{\patriarchs}  {\patriarch{}s{}\xspace}
\newcommand{\matron}      {\Matron{}\xspace}
\newcommand{\matrons}     {\matron{}s{}\xspace}
\newcommand{\matriarch}   {\Matriarch{}\xspace}
\newcommand{\matriarchs}  {\matriarch{}s{}\xspace}













\begin{comment}
\section{Velcad}
\end{comment}

\newcommand{\Velcadians}    {\Velcadian{}s\xspace}
\newcommand{\Velcadian}     {\Velcad{}ian\xspace} 
\newcommand{\Velcad}        {Velcad\xspace}
\newcommand{\VelcadCountry} {Vidra\xspace}
\newcommand{\Vidran}        {Vidran\xspace} % Language. 
\newcommand{\Vidra}         {Vidra\xspace}



\begin{comment}
\subsection{Great Velcad}
\end{comment}

\newcommand{\theVelcadianEmpire}  {\GreatVelcad{}\xspace}
\newcommand{\GreatVelcad}         {Great \Velcad{}\xspace}
\newcommand{\theBelkadianEmpire}  {\GreatBelkade{}\xspace}
\newcommand{\GreatBelkade}        {\GreatVelcad{}\xspace}



\begin{comment}
\subsection{Pelidor}
\end{comment}

\newcommand{\Malcuric}         {\Malcur{}ic\xspace} % Adjective, belonging to Malcur.
\newcommand{\Malcurians}       {\Malcurian{}s\xspace}
\newcommand{\Malcurian}        {\Malcur{}ian\xspace} % The people of Malcur.
\newcommand{\Malcur}           {Malc\ulong r\xspace}

\newcommand{\CastlePelidor}    {Castle Pelidor\xspace}

\newcommand{\Greenhaven}       {Redglen\xspace}
\newcommand{\Redglen}          {Redglen\xspace} % Carzain's home town. 

\newcommand{\Forkliners}       {Forcliners\xspace}
\newcommand{\Forkliner}        {Forcliner\xspace}
\newcommand{\Forklin}          {Forclin\xspace}
\newcommand{\Forcliners}       {Forcliners\xspace}
\newcommand{\Forcliner}        {Forcliner\xspace}
\newcommand{\Forclin}          {Forclin\xspace} % A city in the northern Pelidor





\begin{comment}
\subsection{Scyrum}
\end{comment}


\newcommand{\Scyrics} {\Scyric{}s\xspace} % A kingdom near Pelidor and north of Martinum.
\newcommand{\Scyric}  {S\c c\ydiphthong ric\xspace}
\newcommand{\Scyrum}  {S\c c\ydiphthong rum\xspace}

\newcommand{\Pylandos} {P\ydiphthong landos\xspace} % The capital city of Scyrum. 

\newcommand{\Bryndwin}    {Br\yvowel ndwin\xspace} % A large town in eastern Scyrum. 
\newcommand{\Bryndwiner}  {\Bryndwin{}er\xspace}
\newcommand{\Bryndwiners} {\Bryndwiner{}s\xspace}





\begin{comment}
\subsection{Runger}
\end{comment}


% The lost Draconic temple in Runger where \Takestsha-tachi allegedly find their magic. 
\newcommand{\EreshKali}     {Eresh-Kali\xspace}
\newcommand{\EreshKal}      {Eresh-Kal\xspace}
\newcommand{\Rungertemple}  {Eresh-Kal\xspace}





\begin{comment}
\subsection{Pelidor continent}
\end{comment}


% The name of the continent containing Pelidor.
\newcommand{\Galessan}          {Galessan\xspace}
\newcommand{\Pelidorcontinent}  {Galessan\xspace}
\newcommand{\PelidorContinent}  {Galessan\xspace}








\begin{comment}
\subsection{Culture}
\end{comment}

% Take a name and a title. 
% If the name is empty, print the title. 
% If the name is non-empty, then append the title to it, italicized. 
\newcommand{\Nobletitle} [2]{%
  \ifempty{#1}{\MakeUppercase#2}{#1-\foreign{#2}}}
\newcommand{\nobletitle} [2]{%
  \ifempty{#1}{#2}{#1-\foreign{#2}}}

% Rayuth. Equivalent of a duke. 
\newcommand{\Rayuthships}      {Rayuth-ships\xspace}
\newcommand{\Rayuthship}       {Rayuth-ship\xspace}
\newcommand{\rayuthships}      {rayuth-ships\xspace}
\newcommand{\rayuthship}       {rayuth-ship\xspace}
\newcommand{\Rayuths}          {Rayuths\xspace}
\newcommand{\rayuths}          {rayuths\xspace}
\newcommand{\rayuth}      [1][]{\nobletitle{#1}{rayuth}\xspace}
\newcommand{\Rayuth}      [1][]{\Nobletitle{#1}{rayuth}\xspace}
% Rinyuth, the wife or husband of a Rayuth. 
\newcommand{\rinyuth}     [1][]{\nobletitle{#1}{linyuth}\xspace}
\newcommand{\Rinyuth}     [1][]{\Nobletitle{#1}{linyuth}\xspace}
% Rinyuth, the wife or husband of a Rayuth. 
\newcommand{\scarv}       [1][]{\nobletitle{#1}{scarv}\xspace}
\newcommand{\Scarv}       [1][]{\Nobletitle{#1}{scarv}\xspace}
% "-rah". A knightly title. Equivalent of "sir".
\newcommand{\Rah}         [1][]{\Nobletitle{#1}{rah}\xspace}
\newcommand{\rah}         [1][]{\nobletitle{#1}{rah}\xspace}









\begin{comment}
\subsection{Other places}
\end{comment}

% Marcil, a nation in northeastern Belkade. 
\newcommand{\Marcil} {Mar\c cil{}\xspace}

% Thyrin, a nation in eastern Belkade. 
\newcommand{\Thyrin} {Th\ydiphthong rin{}\xspace}

% Pylor, the river between Andras and Scyrum. 
\newcommand{\Pylor}  {P\ydiphthong lor{}\xspace}
















\begin{comment}
\section{Serpentines}
\end{comment}

% The Serpentines, the lands east of Belkade and west of Irokas
\newcommand{\Serpentine}  {Serpentine\xspace}
\newcommand{\Serp}        {Serpentine\xspace}
\newcommand{\Serpriver}   {Serpentine\xspace}
\newcommand{\Serpsea}     {Serpentine Sea\xspace}
\newcommand{\Serplands}   {Serpentines\xspace}
\newcommand{\Serpadj}     {Serpentine\xspace}
\newcommand{\Serpadjs}    {Serpentines\xspace}

% The Dragon Ridge, the mountain range between the Serpentines and Nom.
\newcommand{\DragonRidge} {\Dragon{} Ridge{}\xspace}
\newcommand{\Dragonridge} {\DragonRidge{}\xspace}















\begin{comment}
\section{Human cultures}
\end{comment}

\newcommand{\Saruns}           {\Sarun{}s\xspace}
\newcommand{\Sarun}            {Sar\ulong n\xspace}















\begin{comment}
\section{Other things}
\end{comment}

% The name of the secret war between Sentinels and Cabal.
\newcommand{\Secretwar}   {Feud\xspace}
\newcommand{\secretwar}   {Feud\xspace}
\newcommand{\Feud}        {Feud\xspace}
\newcommand{\feud}        {Feud\xspace}

% The Charade: The unspoken agreement that the Cabal-Sentinel feud must be kept secret.
\newcommand{\Charade}     {Charade\xspace}
\newcommand{\charade}     {Charade\xspace}

% Dun, the larger of \Miith{}'s two moons, called the Gray Moon. 
\newcommand{\Dun}         {D\ulong{}n\xspace}

% Ultima Thule, a land to the north.
\newcommand{\UltimaThule} {Thulaam\xspace}















\begin{comment}
\chapter{Neevrai}
\end{comment}

\newcommand{\Neevrai} {Neevrai\xspace}























\begin{comment}
\part{The Denizens of \Miith}
\end{comment}



















\begin{comment}
\chapter{Bane and Resphan things}
\end{comment}






\begin{comment}
\section{Cosmos}
\end{comment}

% Erebos, the Realm of Darkness, the homeworld of the Banes, called also the Baneworld.
\newcommand{\Erebos}       {Erebos\xspace}
\newcommand{\Erebean}      {Erebean\xspace} % Adjective.
\newcommand{\erebean}      {Erebean\xspace}
\newcommand{\Baneworld}    {\Bane world\xspace}
\newcommand{\baneworld}    {\Bane world\xspace}
% Father Erebos, the personification of Erebos.
\newcommand{\FatherErebos} {Father \Erebos}

% Nyx, the Realm of Twilight that lies between Erebos and \Miith{}.
\newcommand{\Nyxians} {\Nyx{}ians\xspace} % Adjective.
\newcommand{\Nyxian}  {\Nyx{}ian\xspace} % Adjective.
\newcommand{\nyxian}  {\Nyx{}ian\xspace} % Adjective.
\newcommand{\Nyx}     {N\yvowel x\xspace}
\newcommand{\Oggra}   {Oggra\xspace}   % Chasm in Nyx.
\newcommand{\Ullor}   {\Ulong luor\xspace}   % The abysses beneath Nyx.
\newcommand{\Ulorr}   {\Ullor}   % The abysses beneath Nyx.





\begin{comment}
\section{History}
\end{comment}

% The Banewar, the first great war between Banes and Dragons.
\newcommand{\Banewar}       {Banewar\xspace}
\newcommand{\Banewars}      {Banewars\xspace}
\newcommand{\banewar}       {Banewar\xspace}
\newcommand{\banewars}      {Banewars\xspace}
\newcommand{\FirstBanewar}  {First Banewar\xspace}
\newcommand{\Firstbanewar}  {First Banewar\xspace}
\newcommand{\firstbanewar}  {First Banewar\xspace}
% The Second Banewar, the second great war between Dragons and Banes and now also Resphain.
\newcommand{\Secondbanewar} {Second Banewar\xspace}
\newcommand{\secondbanewar} {Second Banewar\xspace}
% The Resphan wars, the great wars between the Resphan kingdoms.
\newcommand{\ResphanWars}   {\Resphan Wars\xspace}
\newcommand{\Resphanwars}   {\Resphan Wars\xspace}
\newcommand{\resphanwars}   {\Resphan Wars\xspace}
% The hypothetical Third Banewar. 
\newcommand{\Thirdbanewar}  {Third Banewar\xspace}
\newcommand{\thirdbanewar}  {Third Banewar\xspace}

% The Delving, the time when Damiarch-tachi found Semiza. 
\newcommand{\Delving}       {Delving\xspace}
\newcommand{\Delver}        {Delver\xspace}
\newcommand{\Delvers}       {Delvers\xspace}





\begin{comment}
\section{Banes}
\end{comment}

% Banes, the alien invaders.
\newcommand{\Banes}            {Banes\xspace}
\newcommand{\Bane}             {Bane\xspace}
\newcommand{\banes}            {Banes\xspace}
\newcommand{\bane}             {Bane\xspace}

% Sitra Achra, another name for the Banes.
\newcommand{\SitraAchras}      {Sitra Achra\xspace}
\newcommand{\SitraAchra}       {Sitra Achra\xspace}

% Castes of Banes.
\newcommand{\Banespawn}        {\Bane{}spawn\xspace}
\newcommand{\Banespawns}       {\Banespawn}
\newcommand{\banespawn}        {\Banespawn}
\newcommand{\banespawns}       {\banespawn}
\newcommand{\Lesserbane}       {Lesser \bane}
\newcommand{\Lesserbanes}      {\Lesserbane{}s\xspace}
\newcommand{\lesserbane}       {Lesser \bane}
\newcommand{\lesserbanes}      {\lesserbane{}s\xspace}
\newcommand{\Greaterbane}      {Greater \bane}
\newcommand{\Greaterbanes}     {\Greaterbane{}s\xspace}
\newcommand{\greaterbane}      {Greater \bane}
\newcommand{\greaterbanes}     {\greaterbane{}s\xspace}
\newcommand{\Baneknight}       {\Bane{}knight\xspace}
\newcommand{\Baneknights}      {\Baneknight{}s\xspace}
\newcommand{\baneknight}       {\Baneknight}
\newcommand{\baneknights}      {\baneknight{}s\xspace}
\newcommand{\Banelord}         {\Bane{}lord\xspace}
\newcommand{\Banelords}        {\Bane{}lords\xspace}
\newcommand{\banelord}         {\bane{}lord\xspace}
\newcommand{\banelords}        {\bane{}lords\xspace}
\newcommand{\Baneoverlord}     {\Bane{} Overlord\xspace}
\newcommand{\Baneoverlords}    {\Baneoverlord{}s\xspace}
\newcommand{\baneoverlord}     {\Baneoverlord}
\newcommand{\baneoverlords}    {\baneoverlord{}s\xspace}
\newcommand{\Baneking}         {\Bane{}king\xspace}
\newcommand{\Banekings}        {\Baneking{}s\xspace}
\newcommand{\baneking}         {\Baneking}
\newcommand{\banekings}        {\baneking{}s\xspace}
\newcommand{\BaneMessiah}      {\Bane{} Messiah\xspace}
\newcommand{\banemessiah}      {\Bane{} Messiah\xspace}
\newcommand{\HalfBane}         {Half-\Bane}
\newcommand{\halfbane}         {Half-\Bane}

% Baneblood, the white ooze that seeps out when Banes are wounded.
\newcommand{\Baneblood}        {\Bane{}blood\xspace}
\newcommand{\baneblood}        {\Baneblood}

% Stalkers, a type of Banes.
\newcommand{\Stalkers}         {Stalkers\xspace}
\newcommand{\Stalker}          {Stalker\xspace}
\newcommand{\stalkers}         {Stalkers\xspace}
\newcommand{\stalker}          {Stalker\xspace}

% Screamers, a type of Banes.
\newcommand{\Screamers}        {Screamers\xspace}
\newcommand{\Screamer}         {Screamer\xspace}
\newcommand{\screamers}        {Screamers\xspace}
\newcommand{\screamer}         {Screamer\xspace}






\begin{comment}
\section{Resphain}
\end{comment}

% Resphain, the Tiste Andii of \Miith{}.
\newcommand{\Resphain}  {Resph\adarkresphan in\xspace}
\newcommand{\Resphan}   {Resph\aflatresphan n\xspace}
\newcommand{\resphain}  {Resph\adarkresphan in\xspace}
\newcommand{\resphan}   {Resph\aflatresphan n\xspace}

% Resviel, the female Resphain.
\newcommand{\Resviel}  {Resviel\xspace}
\newcommand{\Resvil}   {Resvil\xspace}
\newcommand{\resviel}  {Resviel\xspace}
\newcommand{\resvil}   {Resvil\xspace}

% Neo-Resphain: The new and improved generation of Resphain.
\newcommand{\NeoResphain} {Neo-\Resphain}
\newcommand{\NeoResphan}  {Neo-\Resphan}
\newcommand{\neoresphain} {Neo-\Resphain}
\newcommand{\neoresphan}  {Neo-\Resphan}


\begin{comment}
\subsection{Social classes}
\end{comment}

% Satharioth, the Resphan lords who drank the blood of \Astorglax.
\newcommand{\Sathariah}        {S\adarkresphan th\adarkresphan ri\ahresphan{}\xspace}
\newcommand{\Satharioth}       {S\adarkresphan th\adarkresphan rioth\xspace}
\newcommand{\sathariah}        {\Sathariah}
\newcommand{\satharioth}       {\Satharioth}

% Ketherain, Resphan descendants of the Satharioth.
\newcommand{\Ketherain}        {Kether\adarkresphan in\xspace}
\newcommand{\Ketheran}         {Kether\adarkresphan n\xspace}
\newcommand{\ketherain}        {\Ketherain}
\newcommand{\ketheran}         {\Ketheran}

% Ruistheleth, descendants of the Satharioth from before they were Satharioth. 
\newcommand{\Ruistheleth}      {Ruistheleth\xspace}
\newcommand{\Ruisthel}         {Ruisthel\xspace}
\newcommand{\ruistheleth}      {Ruistheleth\xspace}
\newcommand{\ruisthel}         {Ruisthel\xspace}

% Thelyadeth: Purebloods, with pure Resphan ancestors. 
\newcommand{\thelyadeth}       {\thelyad{}eth\xspace}
\newcommand{\thelyad}          {\Thelyad}
\newcommand{\Thelyadeth}       {\Thelyad{}eth\xspace}
\newcommand{\Thelyad}          {Thely\adarkresphan d\xspace}

% Thelyadeth: Purebloods, with pure Resphan ancestors. 
\newcommand{\gessurim}         {Gessurim\xspace}
\newcommand{\gessur}           {Gessur\xspace}
\newcommand{\Gessurim}         {Gessurim\xspace}
\newcommand{\Gessur}           {Gessur\xspace}

% Yurideth: Pureblood Resviel kept as sex slaves. 
\newcommand{\Yurideth}         {Shiphchoth\xspace}
\newcommand{\Yurid}            {Shiphch\ahresphan{}\xspace}
\newcommand{\yurideth}         {\Yurideth}
\newcommand{\yurid}            {\Yurid}

% 'Ashen-blood', the lower classes among the Resphain. 
\newcommand{\bezedeth}         {\bezed{}eth\xspace}
\newcommand{\bezed}            {\Bezed}
\newcommand{\Bezedeth}         {\Bezed{}eth\xspace}
\newcommand{\Bezed}            {Bezjed\xspace}
\newcommand{\ashenblooded}     {\ashenblood{}ed\xspace}
\newcommand{\ashenbloods}      {\ashenblood{}s\xspace}
\newcommand{\ashenblood}       {ashen-blood\xspace}
\newcommand{\Ashenblooded}     {\Ashenblood{}ed\xspace}
\newcommand{\Ashenbloods}      {\Ashenblood{}s\xspace}
\newcommand{\Ashenblood}       {Ashen-blood\xspace}

\newcommand{\Daemoniacs}       {\Daemoniac{}s\xspace}
\newcommand{\Daemoniac}        {Daemoniac\xspace}
\newcommand{\daemoniacs}       {\daemoniac{}s\xspace}
\newcommand{\daemoniac}        {\Daemoniac}






\begin{comment}
\subsection{Dynasties}
\end{comment}

% Kiriath-Sepher, loyal to the banes. 
\newcommand{\KiriathSepher}    {Ciri\adarkresphan th-Sepher\xspace}
\newcommand{\CiriathSepher}    {Ciri\adarkresphan th-Sepher\xspace}

% Timnath-Serah, loyal to the banes. 
\newcommand{\TiphredSerah}     {Tiphred-Ser\ahresphan\xspace}
\newcommand{\TimnathSerah}     {Tiphred-Ser\ahresphan\xspace}

% Mystraacht, secretly plotting against the banes.
\newcommand{\Mystraacht}       {M\yvowel str\alongresphan cht\xspace}

% Kezerad, good guy Resphain. 
\newcommand{\Kezeradi}         {\Kezerad{}i} % The people of Kezerad.
\newcommand{\Kezerad}          {Kezer\adarkresphan d{}\xspace}

% Baelzerach, daemonic Resphain.
\newcommand{\Baelzerach}       {B\aeresphan lzer\adarkresphan ch\xspace}







\begin{comment}
\subsection{Nations}
\end{comment}

% Merkyrah, the old good Resphan empire.
\newcommand{\Merkyrans}        {\Merkyran{}s\xspace}
\newcommand{\Merkyran}         {Mer-kir\adarkresphan n\xspace}
\newcommand{\Merkyrah}         {Mer-kir\ahresphan{}\xspace}

\newcommand{\Tarcharos}        {T\adarkresphan rch\adarkresphan ros\xspace}







\begin{comment}
\subsection{Culture}
\end{comment}

\begin{comment}
\subsubsection{Economy}
\end{comment}

\newcommand{\Jal}              {\emph{Jal}\xspace}
\newcommand{\jal}              {\emph{jal}\xspace}

\begin{comment}
\subsubsection{Items}
\end{comment}

% Ethylshe, a kind of \resphan{} bloodwine. 
\newcommand{\ethylshe}         {eth\yvowel lsh\finale\xspace}

% Glowmoss.
\newcommand{\Glowmoss}         {Glow-moss\xspace}
\newcommand{\glowmoss}         {glow-moss\xspace}

\begin{comment}
\subsubsection{Languages}
\end{comment}

\newcommand{\Umaric}           {Umaric\xspace}

\begin{comment}
\subsubsection{Magic}
\end{comment}

\newcommand{\Cafir}            {C\aflatresphan fir\xpace} % Kezeradi dweomer.
\newcommand{\Esheram}          {Esher\adarkresphan m\xspace} % Merkyran dweomer.

\newcommand{\Beacon}           {Beacon\xspace}  % The Beacons of Kezerad.
\newcommand{\Beacons}          {Beacons\xspace} 
\newcommand{\beacon}           {Beacon\xspace} 
\newcommand{\beacons}          {Beacons\xspace} 

\begin{comment}
\subsubsection{Martial arts}
\end{comment}

\newcommand{\Eshethicor}       {Eshethicor\xspace}
\newcommand{\Rumicor}          {Rumicor\xspace}
\newcommand{\Shabacora}        {Sh\adarkresphan b\adarkresphan cor\adarkresphan{}\xspace}

\begin{comment}
\subsubsection{Weapons (individual)}
\end{comment}

\newcommand{\Ascaril}          {\Aflatresphan shc\adarkresphan \ilongresphan l\xspace}

% Ossiraith, wielded by Menessiaraid.
\newcommand{\Ossiraith}        {Ossiraith\xspace}

% Scaleron, a sword forged by Dasteron. 
\newcommand{\Scaleron}         {Sc\aflatresphan leron{}\xspace}

% Strith and Currah, Ramiel's guns. 
\newcommand{\Strith}           {Strith\xspace}
\newcommand{\Currah}           {Curr\ahresphan{}\xspace}

% Turishah, wielded by Teshrial.
\newcommand{\Turishah}         {Turish\ahresphan{}\xspace}

\begin{comment}
\subsubsection{Weapons (types)}
\end{comment}

% Belthrad, a type of sword. 
\newcommand{\Belthradeth}      {\Belthrad{}eth{}\xspace}
\newcommand{\Belthrad}         {Belthr\adarkresphan d{}\xspace}
\newcommand{\belthradeth}      {\MakeLowercase \Belthradeth{}\xspace}
\newcommand{\belthrad}         {\MakeLowercase \Belthrad{}\xspace}

% A gelveir, a special ceremonial knife used in the Communion. 
\newcommand{\Gelveirs}         {\Gelveir{}s{}\xspace}
\newcommand{\Gelveir}          {Gelveir{}\xspace}
\newcommand{\gelveirs}         {\MakeLowercase \Gelveirs{}\xspace}
\newcommand{\gelveir}          {\MakeLowercase \Gelveir{}\xspace}

% Ghijedeth, pistols.
\newcommand{\Ghijedeth}        {\Ghijed{}eth{}\xspace}
\newcommand{\Ghijed}           {Ghijed{}\xspace}
\newcommand{\ghijedeth}        {\MakeLowercase \Ghijedeth{}\xspace}
\newcommand{\ghijed}           {\MakeLowercase \Ghijed{}\xspace}

% Kilghain, wing vambraces. 
\newcommand{\Kilghain}         {Cilghain{}\xspace}
\newcommand{\Kilghan}          {Cilghan{}\xspace}
\newcommand{\kilghain}         {\MakeLowercase \Kilghain{}\xspace}
\newcommand{\kilghan}          {\MakeLowercase \Kilghan{}\xspace}

% Ruthil, a shorter, slimmer sword. 
\newcommand{\Ruthiel}          {Ruthiel{}\xspace}
\newcommand{\Ruthil}           {Ruthil{}\xspace}
\newcommand{\ruthiel}          {\MakeLowercase \Ruthiel{}\xspace}
\newcommand{\ruthil}           {\MakeLowercase \Ruthil{}\xspace}

% Senaan, a Ciriath-Sepher type of sword. 
\newcommand{\Senain}           {Sen\adarkresphan in{}\xspace}
\newcommand{\Senaan}           {Sen\alongresphan n{}\xspace} 
\newcommand{\senain}           {\MakeLowercase \Senain{}\xspace}
\newcommand{\senaan}           {\MakeLowercase \Senaan{}\xspace}



\begin{comment}
\subsection{Geography}
\end{comment}

\begin{comment}
\subsubsection{Holds}
\end{comment}

\newcommand{\Cathedon}         {C\aflatresphan thedon\xspace}
\newcommand{\Hyardes}          {Hyardes\xspace} % Towers in Nyx.
\newcommand{\Carcosa}          {\Hyardes}
\newcommand{\Surammas}         {Sur\adarkresphan mm\darkfall s\xspace}

\begin{comment}
\subsubsection{Towers}
\end{comment}

\newcommand{\Jazerubel}        {J\adarkresphan zerubel\xspace}
\newcommand{\Lamaruch}         {L\adarkresphan m\adarkresphan ruch\xspace}
\newcommand{\Roshmal}          {Roshm\adarkresphan l\xspace}
\newcommand{\Shaiphon}         {Shaiphon\xspace}
\newcommand{\Sherem}           {Sherem\xspace}
\newcommand{\Tebethal}         {Tebeth\adarkresphan l\xspace}
\newcommand{\Tirunad}          {Tirun\adarkresphan d\xspace}
\newcommand{\Tzarubal}         {Tz\adarkresphan rub\adarkresphan l\xspace}

\begin{comment}
\subsubsection{More}
\end{comment}

\newcommand{\Hoshiabalon}      {Hoshiabalon\xspace}



\begin{comment}
\subsection{Merkyrah}
\end{comment}

\newcommand{\Iod}              {Iod\xspace}
\newcommand{\iod}              {\Iod}
\newcommand{\Iai}              {Iai\xspace}
\newcommand{\iai}              {\Iai}
\newcommand{\Iath}             {Iath\xspace}
\newcommand{\iath}             {\Iath}








\begin{comment}
\section{Mortals}
\end{comment}



\begin{comment}
\subsection{Nephilim}
\end{comment}

% Nephilim, the ancestors of Humans. 
\newcommand{\Nephil}       {Nephil\xspace} % singular
\newcommand{\Nephilim}     {Nephilim\xspace} % plural
\newcommand{\nephil}       {Nephil\xspace} % singular
\newcommand{\nephilim}     {Nephilim\xspace} % plural
\newcommand{\Nephilic}     {Nephil\xspace}  % The adjective.
\newcommand{\nephilic}     {Nephil\xspace}  % The adjective.

% Gnomphilim, relatives of the Nephilim. 
\newcommand{\Gnomphil}     {Gnomphil\xspace} % singular
\newcommand{\Gnomphilim}   {Gnomphilim\xspace} % plural
\newcommand{\gnomphil}     {Gnomphil\xspace} % singular
\newcommand{\gnomphilim}   {Gnomphilim\xspace} % plural
\newcommand{\Gnomphilic}   {Gnomphil\xspace}  % The adjective.
\newcommand{\gnomphilic}   {Gnomphil\xspace}  % The adjective.

% Girigor, Semiza's land.
\newcommand{\Girigor}      {\Numah}
\newcommand{\Numah}        {Numah\xspace}



\begin{comment}
\subsection{Humans}
\end{comment}

% Humans.
\newcommand{\Humankind}    {Humankind\xspace}
\newcommand{\Humanity}     {Humanity\xspace}
\newcommand{\Humans}       {Humans\xspace}
\newcommand{\Human}        {Human\xspace}
\newcommand{\humankind}    {Humankind\xspace}
\newcommand{\humanity}     {Humanity\xspace}
\newcommand{\humans}       {Humans\xspace}
\newcommand{\human}        {Human\xspace}

% True Humans.
\newcommand{\Truehumans}   {True Humans\xspace}
\newcommand{\Truehuman}    {True Human\xspace}
\newcommand{\truehumans}   {True Humans\xspace}
\newcommand{\truehuman}    {True Human\xspace}
\newcommand{\Truemen}      {True Men\xspace}
\newcommand{\Trueman}      {True Man\xspace}
\newcommand{\truemen}      {True Men\xspace}
\newcommand{\trueman}      {True Man\xspace}

% Demihumans.
\newcommand{\Demihumans}   {Demihumans\xspace}
\newcommand{\Demihuman}    {Demihuman\xspace}
\newcommand{\demihumans}   {Demihumans\xspace}
\newcommand{\demihuman}    {Demihuman\xspace}

% Feeres, a race of demihumans.
\newcommand{\Feeres}       {\Feere{}s\xspace}
\newcommand{\Feere}        {Feere\xspace}
\newcommand{\feeres}       {\Feeres}
\newcommand{\feere}        {\Feere}

% Glunes, a race of demihumans.
\newcommand{\Glunes}       {\Glune{}s\xspace}
\newcommand{\Glune}        {Glune\xspace}
\newcommand{\glunes}       {\Glunes}
\newcommand{\glune}        {\Glune}

% Sheomirs, a race of demihumans.
\newcommand{\Sheomirs}     {\Sheomir{}s\xspace}
\newcommand{\Sheomir}      {Sheomir\xspace}
\newcommand{\sheomirs}     {\Sheomirs}
\newcommand{\sheomir}      {\Sheomir}

% Tulans, a race of demihumans.
\newcommand{\Tulans}       {\Tulan{}s\xspace}
\newcommand{\Tulan}        {Tulan\xspace}
\newcommand{\tulans}       {\Tulans}
\newcommand{\tulan}        {\Tulan}

\newcommand{\Shapens}      {\Shapen}
\newcommand{\Shapen}       {Shapen\xspace}
\newcommand{\shapens}      {\Shapens}
\newcommand{\shapen}       {\Shapen}

% Humanoids.
\newcommand{\Humanoid}     {Humanoid\xspace}
\newcommand{\Humanoids}    {\Humanoid{}s\xspace}
\newcommand{\humanoid}     {humanoid\xspace}
\newcommand{\humanoids}    {\humanoid{}s\xspace}

% Lithrim, the creature formed by all humans combined. 
\newcommand{\Lithrim}      {Lithrim\xspace}

% Hedrim, servants of the Resphain.
\newcommand{\Hedor}   [1][]{\maybeappend{Hedor}{#1}}
\newcommand{\Hedrim}  [1][]{\maybeappend{Hedrim}{#1}}
\newcommand{\hedor}   [1][]{\maybeappend{hedor}{#1}}
\newcommand{\hedrim}  [1][]{\maybeappend{hedrim}{#1}}

% Naorim, the elite food slaves of the Resphain. 
\newcommand{\Naorim}       {\Naor{}im\xspace}
\newcommand{\Naor}         {N\adarkresphan or\xspace}
\newcommand{\naorim}       {\naor{}im\xspace}
\newcommand{\naor}         {\Naor}




\begin{comment}
\section{Monsters}
\end{comment}

\newcommand{\Carth}        {Carth\xspace}
\newcommand{\Carths}       {\Carth{}s\xspace}
\newcommand{\carth}        {\Carth}
\newcommand{\carths}       {\Carths}

% Chthonians, the dwellers beneath the earth.
\newcommand{\Chthonian}    {Chthonian\xspace}
\newcommand{\Chthonians}   {\Chthonian{}s\xspace}
\newcommand{\chthonian}    {\Chthonian}
\newcommand{\chthonians}   {\Chthonians}

\newcommand{\Morkin}       {Morkin\xspace}
\newcommand{\Morkins}      {\Morkin{}s\xspace}
\newcommand{\morkin}       {\Morkin}
\newcommand{\morkins}      {\Morkins}

% Mothlain, Nether Ones.
\newcommand{\Mothlan}      {Mothl\aflatresphan n\xspace}
\newcommand{\mothlan}      {\Mothlan}
\newcommand{\Mothlain}     {Mothl\adarkresphan in\xspace}
\newcommand{\mothlain}     {\Mothlain}

% Soulreapers, the predators of Erebos who prey on Banes.
\newcommand{\Umbrae}       {Umbrae\xspace} 
\newcommand{\Umbra}        {Umbra\xspace}
\newcommand{\umbrae}       {Umbrae\xspace}
\newcommand{\umbra}        {Umbra\xspace}

% Gods of the Umbrae.
\newcommand{\Norganthus}   {Norganthus\xspace}
\newcommand{\Tzyaragoth}   {Tz\ydiphthong aragoth\xspace}

% Banerats, rat-like monsters.
\newcommand{\Grimrats}     {\Grimrat{}s{}\xspace}
\newcommand{\Grimrat}      {Grim-rat{}\xspace}
\newcommand{\grimrats}     {\grimrat{}s{}\xspace}
\newcommand{\grimrat}      {\Grimrat{}\xspace}
\newcommand{\Banerats}     {\Banerat{}s{}\xspace}
\newcommand{\Banerat}      {\Grimrat{}\xspace}
\newcommand{\banerats}     {\banerat{}s{}\xspace}
\newcommand{\banerat}      {\Banerat{}\xspace}

% Ghobaleth, enormous worm monsters from Erebos.
\newcommand{\Ghobaleth}    {Nogg-yaleth\xspace}
\newcommand{\Ghobal}       {Nogg-yal\xspace}
\newcommand{\ghobaleth}    {Nogg-yaleth\xspace}
\newcommand{\ghobal}       {Nogg-yal\xspace}
\newcommand{\Noggyaleth}   {Nogg-yaleth\xspace}
\newcommand{\Noggyal}      {Nogg-yal\xspace}
\newcommand{\noggyaleth}   {Nogg-yaleth\xspace}
\newcommand{\noggyal}      {Nogg-yal\xspace}

% Ophanim, many-eyed monsters that the Resphain use as mounts.
\newcommand{\Ophanim}      {Ophanim\xspace} 
\newcommand{\Ophan}        {Ophan\xspace}
\newcommand{\ophanim}      {Ophanim\xspace}
\newcommand{\ophan}        {Ophan\xspace}

% Flying polyps, monsters in the deeps of Erebos. 
\newcommand{\Flyingpolyps}     {\Flyingpolyp{}s\xspace}
\newcommand{\Flyingpolyp}      {Flying Polyp\xspace}
\newcommand{\flyingpolyps}     {\flyingpolyp{}s\xspace}
\newcommand{\flyingpolyp}      {\Flyingpolyp}

\newcommand{\Ozurian}       {Ozurian\xspace}
\newcommand{\Ozurians}      {\Ozurian{}s\xspace}
\newcommand{\ozurian}       {\Ozurian}
\newcommand{\ozurians}      {\Ozurians}


\begin{comment}
\section{Places}
\end{comment}

% Teshrial's mansion. 
\newcommand{\Ruishagh}  {Ruish\adarkresphan gh{}\xspace}





\begin{comment}
\section{The Cabal}
\end{comment}

\newcommand{\dasteroncircle} {first\xspace}
\newcommand{\cishielcircle}  {second\xspace}
\newcommand{\teshrialcircle} {third\xspace}
\newcommand{\achsahcircle}   {fourth\xspace}
\newcommand{\lelmachcircle}  {sixth\xspace}
\newcommand{\charcoalcircle} {seventh\xspace}
\newcommand{\needlecircle}   {eighth\xspace}
\newcommand{\briarcircle}    {ninth\xspace}









\begin{comment}
\section{Things}
\end{comment}

% The Book of Erebos, penned by Semiza.
\newcommand{\BookofErebos}             {\BathShemTorradjErebossha}
\newcommand{\SemizazBook}              {\BookofErebos}
\newcommand{\BathShemTorradjErebossha} {\BathShem{} Torr\adarkresphan dj\xspace}
\newcommand{\BathShem}                 {B\alongresphan th Shem\xspace}

% The Morbus, the evil disease planted by Daggerrain.
\newcommand{\Morbus}                   {Morbus\xspace}

\newcommand{\UmajaTablets}             {\Umaja Tablets\xspace}
\newcommand{\Umaja}                    {Umaja\xspace}















\begin{comment}
\chapter{Dragons}
\end{comment}










\begin{comment}
\section{First generation Dragons}
\end{comment}

% The first generation of Dragons.
\newcommand{\Firstgendragon}   {Ctherross{}\xspace}
\newcommand{\Firstgendragons}  {\Firstgendragon}
\newcommand{\firstgendragons}  {\firstgendragon}
\newcommand{\firstgendragon}   {\Firstgendragon}

% The Dominators, also called the Chaos Lords, the great Draconic gods.
\newcommand{\Dominator}   {Dominator{}\xspace} % singular
\newcommand{\Dominators}  {Dominators{}\xspace} % plural









\begin{comment}
\section{Second generation Dragons}
\end{comment}

% The second generation of Dragons.
\newcommand{\shaeeroths}        {\secondgendragons{}\xspace}
\newcommand{\shaeeroth}         {\secondgendragon{}\xspace}
\newcommand{\Shaeeroths}        {\Secondgendragons{}\xspace}
\newcommand{\Shaeeroth}         {\Secondgendragon{}\xspace}
\newcommand{\Secondgendragon}   {Shae'eroth{}\xspace}
\newcommand{\Secondgendragons}  {\Secondgendragon{}\xspace}
\newcommand{\secondgendragon}   {\Secondgendragon{}\xspace}
\newcommand{\secondgendragons}  {\secondgendragon{}\xspace}









\begin{comment}
\section{Places}
\end{comment}

% Dragonland, a name for Irokas.
\newcommand{\Dragonland}    {\Dragon{}land\xspace}
\newcommand{\dragonland}    {\Dragonland}

\newcommand{\Baltherium}    {Baltherium\xspace}

% Dathka, the capital city of the fallen Dragonland. 
\newcommand{\Dathka}        {Dathka\xspace}

% Nith'dornazsh, the Draconic city beneath Malcur. 
\newcommand{\Nithd}         {\Nithdornazsh}
\newcommand{\Nithdornazsh}  {Nith\rq dornazsh\xspace}
\newcommand{\Nithvezurun}   {Nith\rq vezurun\xspace}

\newcommand{\Nom}           {Nom\xspace}

\newcommand{\Sardathrion}      {Sardathrion\xspace} % Nexagglachel's city









\begin{comment}
\section{Creatures}
\end{comment}



\begin{comment}
\subsection{Dragons}
\end{comment}

% Draecchonosh, the Kingstongue word for Dragon.
\newcommand{\Draecchonosh}     {\Dzraicchenoss}
\newcommand{\draecchonosh}     {\dzraicchenoss}
\newcommand{\Dzraicchenosses}  {\Dzraicchenoss}
\newcommand{\Dzraicchenoss}    {Dzraic'chenoss{}\xspace}
\newcommand{\dzraicchenosses}  {\Dzraicchenosses}
\newcommand{\dzraicchenoss}    {\Dzraicchenoss}

% Dragons.
\newcommand{\Dragonkind}       {Dragonkind\xspace}
\newcommand{\dragonkind}       {Dragonkind\xspace}
\newcommand{\Dragons}          {Dragons\xspace}
\newcommand{\Dragon}           {Dragon\xspace}
\newcommand{\Draconic}         {Draconic\xspace}
\newcommand{\Draconian}        {Draconian\xspace}
\newcommand{\dragons}          {Dragons\xspace}
\newcommand{\dragon}           {Dragon\xspace}
\newcommand{\draconic}         {Draconic\xspace}
\newcommand{\draconian}        {Draconian\xspace}



\begin{comment}
\subsection{Ophidians}
\end{comment}

\newcommand{\Ophidian}         {Ophidian\xspace}
\newcommand{\Ophidians}        {Ophidians\xspace}
\newcommand{\ophidian}         {Ophidian\xspace}
\newcommand{\ophidians}        {Ophidians\xspace}
\newcommand{\TrueOphidian}     {True \Ophidian}
\newcommand{\TrueOphidians}    {\TrueOphidian{}s}
\newcommand{\trueophidian}     {\TrueOphidian}
\newcommand{\trueophidians}    {\TrueOphidians}

% Caisith, another name for the Ophidians.
\newcommand{\Caisith}          {Caisith\xspace}
\newcommand{\Caisiths}         {Caisith\xspace}
\newcommand{\caisith}          {Caisith\xspace}
\newcommand{\caisiths}         {Caisith\xspace}

% The Quil-Jaaran, an elder race. 
\newcommand{\QuilJaaran}       {\QuilJaar{}an{}\xspace}
\newcommand{\QuilJaar}         {Quil-Jaar{}\xspace}
\newcommand{\quiljaaran}       {\QuilJaaran}
\newcommand{\quiljaar}         {\QuilJaar}

% Quil-Jaaran nations. 
\newcommand{\KyanHweDin}       {Kyan-Hwe-Din{}\xspace}
\newcommand{\Okiriru}          {Okiriru{}\xspace}

% Saphyrae, a Quil-Jaaran nation that existed before the Dragons came. 
\newcommand{\Saphyraeans}      {\Saphyraean{}s\xspace}
\newcommand{\Saphyraean}       {\Saphyrae{}an\xspace}
\newcommand{\Saphyrae}         {\Saphyr{}ae\xspace}
\newcommand{\Saphyr}           {Saph\yvowel r\xspace}

\newcommand{\Asaathur}         {Asaathur\xspace}
\newcommand{\Haamon}           {Haamon\xspace}
\newcommand{\Aamon}            {\Haamon}

\newcommand{\UzulKaya}         {Uzul-Kaya\xspace}
\newcommand{\KulYana}          {Kul-Yana\xspace}

\newcommand{\Serpentmen}       {Serpent-men\xspace}
\newcommand{\Serpentman}       {Serpent-man\xspace}
\newcommand{\serpentmen}       {serpent-men\xspace}
\newcommand{\serpentman}       {serpent-man\xspace}
\newcommand{\Snakemen}         {Snake-men\xspace}
\newcommand{\Snakeman}         {Snake-man\xspace}
\newcommand{\snakemen}         {snake-men\xspace}
\newcommand{\snakeman}         {snake-man\xspace}



\begin{comment}
\subsection{Mortal humanoids}
\end{comment}

% Scathae.
\newcommand{\Scathaese}        {Sc\along thaese\xspace}
\newcommand{\Scathae}          {Sc\along thae\xspace}
\newcommand{\Scatha}           {Sc\along tha\xspace}
\newcommand{\scathaese}        {Sc\along thaese\xspace}
\newcommand{\scathae}          {Sc\along thae\xspace}
\newcommand{\scatha}           {Sc\along tha\xspace}

% Demiscathae.
\newcommand{\Demiscathae}      {Demisc\along thae\xspace}
\newcommand{\Demiscatha}       {Demisc\along tha\xspace}
\newcommand{\demiscathae}      {Demisc\along thae\xspace}
\newcommand{\demiscatha}       {Demisc\along tha\xspace}

% Subraces of Scathae.
\newcommand{\Tassians}          {Tassians\xspace}
\newcommand{\Tassian}           {Tassian\xspace} % Blue
\newcommand{\Mekriis}           {Mecrii\xspace} 
\newcommand{\Mekrii}            {Mecrii\xspace} % Red
\newcommand{\Lois}              {Loi\xspace}
\newcommand{\Loi}               {Loi\xspace}

% Male and female Scathae. 
\newcommand{\daxes}             {daxes\xspace}
\newcommand{\dax}               {dax\xspace}   % Male Scatha
\newcommand{\Daxes}             {Daxes\xspace}
\newcommand{\Dax}               {Dax\xspace} 
\newcommand{\sphyles}           {sph\ydiphthong les\xspace} 
\newcommand{\sphyle}            {sph\ydiphthong le\xspace} % Female Scatha
\newcommand{\Sphyles}           {Sph\ydiphthong les\xspace} 
\newcommand{\Sphyle}            {Sph\ydiphthong le\xspace} 

% Rachyth, Dragon/Scatha hybrids. 
\newcommand{\Rachyth}           {Rach\ydiphthong th\xspace}
\newcommand{\Rachyths}          {\Rachyth}
\newcommand{\rachyth}           {\Rachyth}
\newcommand{\rachyths}          {\Rachyths}

% Cregorr.
\newcommand{\Cregorr}           {Cusharan\xspace}
\newcommand{\Cregorrs}          {Cusharans} 
\newcommand{\Cregoradj}         {\Cregorr}
\newcommand{\cregorr}           {\Cregorr}
\newcommand{\cregorrs}          {\Cregorrs} 
\newcommand{\cregoradj}         {\Cregoradj}

% Troglodytes: Degenerate Scathae living underground.
\newcommand{\Troglodyte}        {Troglodyte\xspace}
\newcommand{\Troglodytes}       {Troglodytes\xspace}
\newcommand{\troglodyte}        {Troglodyte\xspace}
\newcommand{\troglodytes}       {Troglodytes\xspace}
\newcommand{\Tsutora}           {Tsu-Tora\xspace}
\newcommand{\Tsutoras}          {Tsu-Toras\xspace}
\newcommand{\tsutora}           {Tsu-Tora\xspace}
\newcommand{\tsutoras}          {Tsu-Toras\xspace}

% Locul, a pre-Scathaese servitor race. 
\newcommand{\Loculs}            {Loc\ulong{}l\xspace}
\newcommand{\Locul}             {Loc\ulong{}l\xspace}
\newcommand{\loculs}            {Loc\ulong{}l\xspace}
\newcommand{\locul}             {Loc\ulong{}l\xspace}



\begin{comment}
\subsection{Titles}
\end{comment}

% Title of the Dragon King or Queen.
\newcommand{\DragonKing}   {\Dragon{} King{}\xspace}
\newcommand{\DragonKings}  {\DragonKing{}s{}\xspace}
\newcommand{\Dragonking}   {\DragonKing{}\xspace}
\newcommand{\Dragonkings}  {\DragonKings{}\xspace}
\newcommand{\dragonking}   {\DragonKing{}\xspace}
\newcommand{\dragonkings}  {\dragonking{}s{}\xspace}
\newcommand{\DragonQueen}  {\Dragon{} Queen{}\xspace}
\newcommand{\DragonQueens} {\DragonQueen{}s{}\xspace}
\newcommand{\Dragonqueen}  {\DragonQueen{}\xspace}
\newcommand{\Dragonqueens} {\DragonQueens{}\xspace}
\newcommand{\dragonqueen}  {\DragonQueen{}\xspace}
\newcommand{\dragonqueens} {\dragonqueen{}s{}\xspace}

% Dragonlord.
\newcommand{\Dragonlord}   {\Dragon{}lord{}\xspace}
\newcommand{\Dragonlords}  {\Dragonlord{}s{}\xspace}
\newcommand{\dragonlord}   {\Dragonlord{}\xspace}
\newcommand{\dragonlords}  {\dragonlord{}s{}\xspace}









\begin{comment}
\section{History}
\end{comment}

% The war between the newly-arisen Draecchonosh and the traditional Ophidians.
\newcommand{\Draecchonoshwar}  {\Draecchonosh{} War{}\xspace}
\newcommand{\draecchonoshwar}  {\Draecchonoshwar{}\xspace}









\begin{comment}
\section{\Sethican philosophy}
\end{comment}

\newcommand{\DaathKurZulNathla} {Daath-kur-zul-Nathla\xspace}
\newcommand{\Osserylloch}       {Osserylloch\xspace}
\newcommand{\Barbeloth}         {Barbeloth\xspace}
\newcommand{\YothUnXachtyon}    {Yoth-un-Xachthyon\xspace}









\begin{comment}
\section{Things}
\end{comment}

% Dragonsteel, a powerful metal.
\newcommand{\Dragonsteel}   {\Dragon{}steel{}\xspace}
\newcommand{\dragonsteel}   {\Dragonsteel}

% Skekrathuin, Draconic arm blades.
\newcommand{\Skekrathuin}   {Skekrathuin{}\xspace}
\newcommand{\Skekrathuins}  {\Skekrathuin}
\newcommand{\skekrathuin}   {skekrathuin{}\xspace}
\newcommand{\skekrathuins}  {\skekrathuin}

% Zrekklakh, tail blades. 
\newcommand{\Zrekklakh}     {Zrekklakh{}\xspace}
\newcommand{\zrekklakh}     {zrekklakh{}\xspace}

% Colossi, the Dragons' Humongous Mecha.
\newcommand{\Colossus}      {Colossus{}\xspace}
\newcommand{\Colossi}       {Colossi{}\xspace}
\newcommand{\colossus}      {\Colossus}
\newcommand{\colossi}       {\Colossi}

% Wanderers in Darkness, a poem. 
\newcommand{\WanderersInDarknessEmph} {\emph{\WanderersInDarkness{}}\xspace}
\newcommand{\WanderersInDarkness} {Black Stars' Aenigmata\xspace}
% Melcryth, author of Wanderers in Darkness. 
\newcommand{\Melcryth}            {Melcr\ydiphthong th{}\xspace}

% Draconic languages.
\newcommand{\TrueDraconic}   {True Draconic\xspace}
\newcommand{\CommonDraconic} {Common Draconic\xspace}














\begin{comment}
\chapter{Other Peoples}
\end{comment}

% The Voyagers, the foes of the Krakens.
\newcommand{\Voyager}   {Voyager{}\xspace}
\newcommand{\Voyagers}  {\Voyager{}s{}\xspace}
\newcommand{\voyager}   {\Voyager}
\newcommand{\voyagers}  {\Voyagers}

% Cuezca, the lost, Atlantis-like primordial civilization.
\newcommand{\Cuezca}    {Cuezca{}\xspace}
\newcommand{\Cuezcan}   {\Cuezca{}n{}\xspace}
\newcommand{\Cuezcans}  {\Cuezcan{}s{}\xspace}
\newcommand{\cuezcan}   {\Cuezcan}
\newcommand{\cuezcans}  {\Cuezcans}

% The Cuezcan Apocalypse, the great catastrophe surrounding the destruction of Cuezca.
\newcommand{\CuezcanApocalypse} {\Cuezcan{} Apocalypse{}\xspace}

% The Moon-Wolves, the mystic wolves of the frost-moon. 
% Inspired by Bal-Sagoth and the Malazan Deragoth.
\newcommand{\Vorcanths}   {Vorcans\xspace}
\newcommand{\Vorcanth}    {Vorcan\xspace}
\newcommand{\vorcanths}   {\Vorcanths}
\newcommand{\vorcanth}    {\Vorcanth}
\newcommand{\MoonWolves}  {\Vorcanths}
\newcommand{\MoonWolf}    {\Vorcanth}
\newcommand{\moonwolves}  {\vorcanths}
\newcommand{\moonwolf}    {\vorcanth}

% The Rocolza. An elder race of giants. Related to Nephilim. 
\newcommand{\Rocolzas} {\Aryothim}
\newcommand{\Rocolza}  {\Aryoth}
\newcommand{\rocolzas} {\aryothim}
\newcommand{\rocolza}  {\aryoth}
\newcommand{\Aryothim} {\aryothim}
\newcommand{\Aryoth}   {\aryoth}
\newcommand{\aryothim} {Aryothim\xspace}
\newcommand{\aryoth}   {Aryoth\xspace}




\begin{comment}
\section{Chaos}
\end{comment}


% The force of Chaos.
\newcommand{\Chaos} {Chaos{}\xspace}
\newcommand{\chaos} {\Chaos}
% The adjective `Chaotic'.
\newcommand{\Chaotic} {Chaotic{}\xspace}
\newcommand{\chaotic} {chaotic{}\xspace}

% Daemons, the creatures of Chaos.
\newcommand{\Daimonic} {\Daemonic}
\newcommand{\Daimonia} {\Daemons}
\newcommand{\Daimon}   {\Daemon}
\newcommand{\daimonic} {\daemonic}
\newcommand{\daimonia} {\daemons}
\newcommand{\daimon}   {\daemon}
\newcommand{\Daemonic} {\Daemon{}ic\xspace}
\newcommand{\Daemons}  {\Daemon{}s\xspace}
\newcommand{\Daemon}   {Daemon\xspace}
\newcommand{\daemonic} {\daemon{}ic\xspace}
\newcommand{\daemons}  {\daemon{}s\xspace}
\newcommand{\daemon}   {daemon\xspace}

% Homunculi, quasi-intelligent creatures.
\newcommand{\Homunculi}  {Homunculi}
\newcommand{\Homunculus} {Homunculus}
\newcommand{\homunculi}  {Homunculi}
\newcommand{\homunculus} {Homunculus}

% Magical daemons: Intangible beings or forced that may be summoned to perform spells. 
\newcommand{\MDaemon}   {\Daemon{}\xspace}
\newcommand{\MDaemons}  {\Daemons{}\xspace}
\newcommand{\MDaemonic} {\Daemonic{}\xspace}
\newcommand{\mdaemon}   {\daemon{}\xspace}
\newcommand{\mdaemons}  {\daemons{}\xspace}
\newcommand{\mdaemonic} {\daemonic{}\xspace}

% Physical daemons: Material creatures that may be summoned to act or fight. 
\newcommand{\PDaemon}   {Devil{}\xspace}
\newcommand{\PDaemons}  {\PDaemon{}s{}\xspace}
\newcommand{\PDaemonic} {\PDaemon{}ish{}\xspace}
\newcommand{\Pdaemon}   {\PDaemon{}\xspace}
\newcommand{\Pdaemons}  {\PDaemons{}\xspace}
\newcommand{\Pdaemonic} {\PDaemonic{}\xspace}
\newcommand{\pdaemon}   {\Pdaemon{}\xspace}
\newcommand{\pdaemons}  {\pdaemon{}s{}\xspace}
\newcommand{\pdaemonic} {\pdaemon{}ish{}\xspace}

% Lemures, souls bound in the forms of misshapen \pdaemons.
\newcommand{\Lemure}   {Lemure{}\xspace}
\newcommand{\Lemures}  {\Lemure{}s{}\xspace}
\newcommand{\Lemureic} {\Lemure{}ic{}\xspace}
\newcommand{\lemure}   {lemure{}\xspace}
\newcommand{\lemures}  {\lemure{}s{}\xspace}
\newcommand{\lemureic} {\lemure{}ic{}\xspace}

% The Haskelek, the Rissitic daemon demigod.
\newcommand{\haskelek}  {\Haskelek{}\xspace}
\newcommand{\Haskelek}  {Haskelek{}\xspace}

% Balrogs, a race of daemons. 
\newcommand{\Balrog}  {Balrog{}\xspace}
\newcommand{\Balrogs} {\Balrog{}s{}\xspace}
\newcommand{\balrog}  {\Balrog{}\xspace}
\newcommand{\balrogs} {\balrog{}s{}\xspace}

% Nrekkloi, a race of daemons.
% A nrekkloi has a lobster-like abdomen with six legs and a hard carapace, and a fur-covered, vaguely humanoid torso with clawed arms. 
\newcommand{\Nrekkloi}  {Nrekkloi{}\xspace}
\newcommand{\Nrekklois} {\Nrekkloi{}\xspace}
\newcommand{\nrekkloi}  {\Nrekkloi{}\xspace}
\newcommand{\nrekklois} {\Nrekkloi{}\xspace}

% Succubi, pdaemons who tempt people with illusions and promises of sex, power or whatever.
\newcommand{\Succubus} {Succubus{}\xspace}
\newcommand{\Succubi}  {Succubi{}\xspace}
\newcommand{\succubus} {\Succubus}
\newcommand{\succubi}  {\Succubi}










\begin{comment}
\subsection{\XzaiShann}
\end{comment}

% Xzai-Shann, a race of daemons. 
% Inspired, to a limited extent, by the Thzan-Tzai from Stephen Marley's books about Chia the Black Dragon Sorceress. 
\newcommand{\XzaiShannic}      {Xzai-Shann\xspace}
\newcommand{\XzaiShanns}       {Xzai-Shann\xspace}
\newcommand{\XzaiShann}        {Xzai-Shann\xspace}
\newcommand{\xzaishannic}      {Xzai-Shann\xspace}
\newcommand{\xzaishanns}       {Xzai-Shann\xspace}
\newcommand{\xzaishann}        {Xzai-Shann\xspace}
\newcommand{\xs}               {Xzai-Shann\xspace}
\newcommand{\xss}              {Xzai-Shann\xspace}
\newcommand{\xsic}             {Xzai-Shann\xspace}

\newcommand{\Theratons}        {Theratons\xspace}
\newcommand{\Theraton}         {Theraton\xspace}
\newcommand{\theratons}        {Theratons\xspace}
\newcommand{\theraton}         {Theraton\xspace}



\begin{comment}
\subsubsection{Greater \XzaiShann}
\end{comment}

\newcommand{\Achamoth}         {Achamoth\xspace}

% Kyaethem Chrei'az, a Xzai-Shann.
\newcommand{\KyaethemChreiAz}  {Kyathem Chreiza\xspace}

% Khoth-Sell, the Draconic goddess of death.
\newcommand{\KhothSell}        {Khothsell\xspace}

% He who is the Key and the Gate. 
\newcommand{\NaathKurRamalech} {Naath-Kur-Ramalech\xspace}

% Nerrhan-Koss, a dark god.
\newcommand{\NerrhanKoss}      {Nerran-Koss\xspace}
\newcommand{\NerranKoss}       {Nerran-Koss\xspace}
\newcommand{\NehranKoss}       {Nerran-Koss\xspace}

% Ruin Satha who reigns from his dark throne.
\newcommand{\RuinSatha}        {\Ruin-\Satha}
\newcommand{\Ruin}             {Ruin\xspace}
\newcommand{\Satha}            {Satha\xspace}



\begin{comment}
\subsubsection{Lesser \XzaiShann}
\end{comment}

\newcommand{\HothNrul}         {Rurglax\xspace}
\newcommand{\Yolbaoth}         {Yolbaoth\xspace}















\begin{comment}
\section{Nagae}
\end{comment}


% Kraken.
\newcommand{\Krakens}          {Kraken\xspace}
\newcommand{\Kraken}           {Kraken\xspace}
\newcommand{\krakens}          {Kraken\xspace}
\newcommand{\kraken}           {Kraken\xspace}

% Moroch, the youngest of the Kraken and the creator of the Nagae and Vlekkesh'sala.
\newcommand{\Moroch}           {Moloch{}\xspace}

% Nagae, the Deep Ones of \Miith{}.
\newcommand{\Nagalords}        {\Nagalord{}s\xspace}
\newcommand{\Nagalord}         {\Naga{} Lord\xspace}
\newcommand{\nagalords}        {\Nagalords}
\newcommand{\nagalord}         {\Nagalord}
\newcommand{\Nagae}            {Nagae\xspace}
\newcommand{\Naga}             {Naga\xspace}
\newcommand{\nagae}            {Nagae\xspace}
\newcommand{\naga}             {Naga\xspace}

% Ichtyans, another name for Nagae.
\newcommand{\Ichthyans}        {Ichthyans\xspace}
\newcommand{\Ichthyan}         {Ichthyan\xspace}
\newcommand{\ichthyans}        {Ichthyans\xspace}
\newcommand{\ichthyan}         {Ichthyan\xspace}

% Linnorms, giant reshaped Nagae that serve as warriors.
\newcommand{\Linnorms}         {Linnorms\xspace}
\newcommand{\Linnorm}          {Linnorm\xspace}
\newcommand{\linnorms}         {Linnorms\xspace}
\newcommand{\linnorm}          {Linnorm\xspace}



\begin{comment}
\section{Animals}
\end{comment}



\begin{comment}
\subsubsection{Birds}
\end{comment}

% Grulcan (Diatryma/Gastornis).
\newcommand{\Grulcans}    {\Grulcan{}s{}\xspace}
\newcommand{\Grulcan}     {Grulcan{}\xspace}
\newcommand{\grulcans}    {\Grulcans}
\newcommand{\grulcan}     {\Grulcan}



\begin{comment}
\subsection{Dragon-like animals}
\end{comment}

% Drakes, Ophidian-like saurians. 
\newcommand{\Drakes}           {\Drake{}s{}\xspace}
\newcommand{\Drake}            {Drake{}\xspace}
\newcommand{\drakes}           {\MakeLowercase\Drakes}
\newcommand{\drake}            {\MakeLowercase\Drake}

\newcommand{\Varcals}          {\Varcal{}s{}\xspace}
\newcommand{\Varcal}           {Varcal{}\xspace}
\newcommand{\varcals}          {\MakeLowercase\Varcals}
\newcommand{\varcal}           {\MakeLowercase\Varcal}

\newcommand{\Lindworms}        {Lindworms\xspace}
\newcommand{\Lindworm}         {Lindworm\xspace}
\newcommand{\lindworms}        {Lindworms\xspace}
\newcommand{\lindworm}         {Lindworm\xspace}

% Mezolisk, aka Daggerdrake.
\newcommand{\Mezolisks}        {Mezolisks\xspace}
\newcommand{\Mezolisk}         {Mezolisk\xspace}
\newcommand{\mezolisks}        {\Mezolisks}
\newcommand{\mezolisk}         {\Mezolisk}

\newcommand{\Reptilecolossi}   {Reptilian Colossi\xspace}
\newcommand{\Reptilecolossus}  {Reptilian Colossus\xspace}
\newcommand{\reptilecolossi}   {Reptilian Colossi\xspace}
\newcommand{\reptilecolossus}  {Reptilian Colossus\xspace}

% Vreiid (Wyvern).
\newcommand{\Vreiiden}         {Vreiiden\xspace}
\newcommand{\Vreiid}           {Vreiid\xspace}
\newcommand{\vreiiden}         {vreiiden\xspace}
\newcommand{\vreiid}           {vreiid\xspace}
\newcommand{\Wyvern}           {\Vreiid}
\newcommand{\Wyverns}          {\Vreiids}
\newcommand{\wyvern}           {\vreiid}
\newcommand{\wyverns}          {\vreiids}



\begin{comment}
\subsection{Fish}
\end{comment}

% Caderyn, the huge monster fish.
\newcommand{\Caderyn}          {Cader\yvowel n{}\xspace}
\newcommand{\Caderyns}         {\Caderyn{}s{}\xspace}
\newcommand{\caderyn}          {\Caderyn{}\xspace}
\newcommand{\caderyns}         {\caderyn{}s{}\xspace}



\begin{comment}
\subsection{Mammals}
\end{comment}

\newcommand{\Belwans}     {\Belwan{}s{}\xspace}
\newcommand{\Belwan}      {Belwan{}\xspace}
\newcommand{\belwans}     {\belwan{}s{}\xspace}
\newcommand{\belwan}      {belwan{}\xspace}



\begin{comment}
\subsection{Pterans}
\end{comment}

% Pterans.
\newcommand{\Pterans}     {Pterans\xspace}
\newcommand{\Pteran}      {Pteran\xspace}
\newcommand{\pterans}     {pterans\xspace}
\newcommand{\pteran}      {pteran\xspace}

% Ravcor. A pterosaur. 
\newcommand{\Ravcors}     {Ravcors\xspace}
\newcommand{\Ravcor}      {Ravcor\xspace}
\newcommand{\ravcors}     {ravcors\xspace}
\newcommand{\ravcor}      {ravcor\xspace}

% Quil-rai. A pterosaur. 
\newcommand{\Quilrais}    {Quil-rai\xspace}
\newcommand{\Quilrai}     {Quil-rai\xspace}
\newcommand{\quilrais}    {quil-rai\xspace}
\newcommand{\quilrai}     {quil-rai\xspace}



\begin{comment}
\subsection{Saurians}
\end{comment}

% Saurians.
\newcommand{\Saurians}    {Saurians\xspace}
\newcommand{\Saurian}     {Saurian\xspace}
\newcommand{\saurians}    {saurians\xspace}
\newcommand{\saurian}     {saurian\xspace}



\begin{comment}
\subsubsection{Theropods}
\end{comment}

% Corgorah (Tyrannosaurus).
\newcommand{\Corgoroth}   {Corgoroth\xspace}
\newcommand{\Corgorah}    {Corgor\ahresphan{}\xspace}
\newcommand{\corgoroth}   {\Corgoroth}
\newcommand{\corgorah}    {\Corgorah}
\newcommand{\Gorgoroses}  {\Corgoroth}
\newcommand{\Gorgoros}    {\Corgorah}
\newcommand{\gorgoroses}  {\corgoroth}
\newcommand{\gorgoros}    {\corgorah}
\newcommand{\Cortios}     {\Corgoroth}
\newcommand{\Cortio}      {\Corgorah}
\newcommand{\cortios}     {\corgoroth}
\newcommand{\cortio}      {\corgorah}
% Rissitic name for a Corgorah. 
\newcommand{\Tashrek}     {Tashrek\xspace}
\newcommand{\tashrek}     {Tashrek\xspace}

% Nycan (dromaeosaurid).
\newcommand{\Nycan}       {N\ydiphthong{}can\xspace}
\newcommand{\Nycans}      {\Nycan{}s\xspace}
\newcommand{\Nycanese}    {\Nycan} % Adjective.
\newcommand{\nycan}       {\Nycan}
\newcommand{\nycans}      {\nycan{}s\xspace}
\newcommand{\nycanese}    {\nycan} % Adjective.

% Lotha (medium-sized theropod).
\newcommand{\Lothae}      {Lothae\xspace}
\newcommand{\Lotha}       {Lotha\xspace}
\newcommand{\lothae}      {Lothae\xspace}
\newcommand{\lotha}       {Lotha\xspace}



\begin{comment}
\subsubsection{Ornithischian herbivores}
\end{comment}

% Tondra. A sauropod. 
\newcommand{\Tondras}     {\Tondra{}s\xspace}
\newcommand{\Tondra}      {Tocondra\xspace}
\newcommand{\tondras}     {\tondra{}s\xspace}
\newcommand{\tondra}      {tocondra\xspace}

% Brukath. The largest sauropod on Mith.
\newcommand{\Brukaths}    {\Brukath{}s\xspace}
\newcommand{\Brukath}     {Brukh\aflatrissitic{}l\xspace}
\newcommand{\brukaths}    {\Brukath{}s{}\xspace}
\newcommand{\brukath}     {Brukh\aflatrissitic{}l\xspace}



\begin{comment}
\subsubsection{Saurischian herbivores}
\end{comment}

% Miksha (Gallimimus). 
\newcommand{\Mikshas}     {\Miksha{}s\xspace}
\newcommand{\Miksha}      {Mishkla\xspace}
\newcommand{\mikshas}     {\miksha{}s\xspace}
\newcommand{\miksha}      {mishkla\xspace}

% Mulgron (Triceratops). 
\newcommand{\Murocs}      {Murocs\xspace}
\newcommand{\Muroc}       {Muroc\xspace}
\newcommand{\murocs}      {murocs\xspace}
\newcommand{\muroc}       {muroc\xspace}
\newcommand{\Mulgrons}    {Murocs\xspace}
\newcommand{\Mulgron}     {Muroc\xspace}
\newcommand{\mulgrons}    {murocs\xspace}
\newcommand{\mulgron}     {muroc\xspace}

% Relc (hadrosaurian lizard horse). 
\newcommand{\Relc}        {Relc\xspace}
\newcommand{\Relcs}       {Relcs\xspace}
\newcommand{\relc}        {relc\xspace}
\newcommand{\relcs}       {relcs\xspace}

% Relcer (a person who rides a relc). 
\newcommand{\Relcer}      {Relcer\xspace}
\newcommand{\Relcers}     {Relcers\xspace}
\newcommand{\relcer}      {relcer\xspace}
\newcommand{\relcers}     {relcers\xspace}















\begin{comment}
\section{Plants}
\end{comment}

% Jiliba, a plant whose berries in a tea will help you regain strength. 
\newcommand{\jiliba}      {jiliba{}\xspace}
\newcommand{\Jiliba}      {Jiliba{}\xspace}
% Dvingen, a plant whose leaves can be used to make a healing poultice.
\newcommand{\dvingen}     {dvingen{}\xspace}
\newcommand{\Dvingen}     {Dvingen{}\xspace}















\begin{comment}
\section{Undead}
\end{comment}

% Liches, undead mages.
\newcommand{\Lich}             {Lich\xspace}
\newcommand{\Liches}           {\Lich{}es\xspace}
\newcommand{\lich}             {lich\xspace}
\newcommand{\liches}           {\lich{}es\xspace}

% Reavers, the Vampires of \Miith{}.
\newcommand{\Reaver}           {Reaver\xspace} % Singular
\newcommand{\Reavers}          {Reavers\xspace} % Plural
\newcommand{\Reaveric}         {Reaver\xspace} % Associated adjective
\newcommand{\reaver}           {Reaver\xspace} % Singular
\newcommand{\reavers}          {Reavers\xspace} % Plural
\newcommand{\reaveric}         {Reaver\xspace} % Associated adjective

% Leeches, the lesser Vampires of \Miith{}.
\newcommand{\Leech}            {Leech{}\xspace} % Singular
\newcommand{\Leeches}          {\Leech{}es{}\xspace} % Plural
\newcommand{\Leechic}          {\Leech{}\xspace} % Associated adjective















\begin{comment}
\section{Moon-things}
\end{comment}

\newcommand{\Glithid}          {Glithid\xspace}
\newcommand{\Glithids}         {Glithids\xspace}
\newcommand{\glithid}          {glithid\xspace}
\newcommand{\glithids}         {glithids\xspace}

\newcommand{\Moongod}          {Moon-god\xspace}
\newcommand{\Moongods}         {Moon-gods\xspace}
\newcommand{\moongod}          {Moon-god\xspace}
\newcommand{\moongods}         {Moon-gods\xspace}

\newcommand{\Moonthing}        {Moon-thing\xspace}
\newcommand{\Moonthings}       {Moon-things\xspace}
\newcommand{\moonthing}        {Moon-thing\xspace}
\newcommand{\moonthings}       {Moon-things\xspace}

\newcommand{\Shugul}           {Sh\ulong{}gul\xspace}
\newcommand{\Shuguls}          {\Shugul}
\newcommand{\shugul}           {\Shugul}
\newcommand{\shuguls}          {\Shugul}















\begin{comment}
\section{Others}
\end{comment}


% Monsters covered in blades. Perhaps insectoid.
\newcommand{\NerasKirishgaith} {Neras Kirish'gaith{}\xspace}
\newcommand{\bladedpeople}     {\NerasKirishgaith{}\xspace}


% Nymphs: Collective term for various feminine "spirits". 
\newcommand{\Nymphs}           {Nymphs\xspace}
\newcommand{\Nymph}            {Nymph\xspace}
\newcommand{\nymphs}           {Nymphs\xspace}
\newcommand{\nymph}            {Nymph\xspace}

% Naiads, water spirits.
\newcommand{\Naiads}           {\Naiad{}\xspace}
\newcommand{\Naiad}            {Ne\"ereth{}\xspace}
\newcommand{\naiads}           {\naiad{}\xspace}
\newcommand{\naiad}            {\Naiad{}\xspace}

% Sylphs, air spirits.
\newcommand{\Sylphs}           {Sylphs\xspace}
\newcommand{\Sylph}            {Sylph\xspace}
\newcommand{\sylphs}           {Sylphs\xspace}
\newcommand{\sylph}            {Sylph\xspace}

% Jinni, a mythological creature.
\newcommand{\Jinni}            {Jinni\xspace}
\newcommand{\Jinn}             {Jinn\xspace}
\newcommand{\jinni}            {Jinni\xspace}
\newcommand{\jinn}             {Jinn\xspace}

% Malgryph, a mythological creature.
\newcommand{\Malgryphs}        {Malgr\yvowel phs\xspace}
\newcommand{\Malgryph}         {Malgr\yvowel ph\xspace}
\newcommand{\malgryphs}        {Malgr\yvowel phs\xspace}
\newcommand{\malgryph}         {Malgr\yvowel ph\xspace}

% Sylphs, air spirits.
\newcommand{\Chimaerae}        {Chimaerae\xspace}
\newcommand{\Chimaera}         {Chimaera\xspace}
\newcommand{\chimaerae}        {Chimaerae\xspace}
\newcommand{\chimaera}         {Chimaera\xspace}

\newcommand{\FireSalamanders}  {Raxxians\xspace}
\newcommand{\FireSalamander}   {Raxxian\xspace}
\newcommand{\firesalamanders}  {Raxxians\xspace}
\newcommand{\firesalamander}   {Raxxian\xspace}

\newcommand{\feldraxes}        {\Feldraxes}
\newcommand{\feldrax}          {\Feldrax}
\newcommand{\Feldraxes}        {\Feldrax}
\newcommand{\Feldrax}          {Feldrax\xspace}

\newcommand{\werlocs}          {\Werlocs}
\newcommand{\werloc}           {\Werloc}
\newcommand{\Werlocs}          {\Werloc{}s}
\newcommand{\Werloc}           {Werloc\xspace}

\newcommand{\rulyamoths}       {\Rulyamoths}
\newcommand{\rulyamoth}        {\Rulyamoth}
\newcommand{\Rulyamoths}       {\Rulyamoth{}s}
\newcommand{\Rulyamoth}        {Rulyamoth\xspace}















\begin{comment}
\part{Characters of \Miith}
\end{comment}


% `Mister' and `Mistress', polite titles used in Belkadian. 
\newcommand{\Mister}           {Mister{}\xspace}
\newcommand{\Mr}               {\Mister{}\xspace}
\newcommand{\Mistress}         {Mistress{}\xspace}
\newcommand{\Mrs}              {\Mistress{}\xspace}
\newcommand{\Miss}             {Miss{}\xspace}
% `Dai-' and `Rao-', polite prefixes used in Imetric.
\newcommand{\Dai}              {Dai-{}\xspace}
\newcommand{\dai}              {dai-{}\xspace}
\newcommand{\Rao}              {Rao-{}\xspace}
\newcommand{\rao}              {rao-{}\xspace}









\begin{comment}
\chapter{Draconic Faction}
\end{comment}









\begin{comment}
\section{\Ophidians}
\end{comment}

\newcommand{\Sethican}         {Sethican\xspace}
\newcommand{\Sethicus}         {Sethicus\xspace}
\newcommand{\ValcanSethicus}   {\Valcan{} Sethicus\xspace}
\newcommand{\VardredSethicus}  {\Valcan{} Sethicus\xspace}
\newcommand{\Vardred}          {\Valcan}
\newcommand{\Valcan}           {Valcan\xspace}

\newcommand{\Hesherritan}      {Hesherritan\xspace}
\birthliving{Hesherritan}      {IC} {-231478}

\newcommand{\Ishtacca}         {Ishtacca\xspace}
\birthliving{Ishtacca}         {Durance begins}{23143}

\newcommand{\Nasshikerr}       {Nasshiker\xspace}
\birthliving{Nasshikerr}       {IC} {-91854}

\newcommand{\NathRamos}        {Nath Ramos\xspace}

\newcommand{\ZeethanKraal}     {Zeethan \Kraal}
\newcommand{\Kraal}            {Kraal\xspace}
\birthliving{Zeethan Kraal}    {IC} {-77231}








\begin{comment}
\section{\Dragons}
\end{comment}



\begin{comment}
\subsection{\Firstgendragons}
\end{comment}

% Hesod-Nerga, the first Dragon and the father of Tiamat.
\newcommand{\HesodNerga}                  {Chesod-Nerga\xspace}
\newcommand{\HesodN}                      {Chesod-Nerga\xspace}

% Tiamat, the terrible Dragon Mother. 
\newcommand{\Tiamat}                      {T\ydiphthong rasshana\xspace}
\newcommand{\TyarithXserasshana}          {\Tyarith \Xserasshana}
\newcommand{\Tyarith}                     {Dzairith\xspace}
\newcommand{\Xserasshana}                 {\Tiamat}
\newcommand{\Kserasshana}                 {\Tiamat}
\newcommand{\Tyrasshana}                  {\Tiamat}

% Typhon, the Dragon god of war. 
\newcommand{\Typhon}                      {T\ydiphthong phon\xspace}

% Iurzmacul, the Dragon god of the sea and of wisdom.
\newcommand{\Iurzmacul}                   {Iurzmacul\xspace}

% Apep-Nesthra, the Dragon god of sorcery and diabolism.
\newcommand{\ApepNesthra}                 {Aphares Nesthra\xspace}
\newcommand{\ApharesNesthra}              {Aphares Nesthra\xspace}
\newcommand{\Aphares}                     {Aphares\xspace}
\newcommand{\Nesthra}                     {Nesthra\xspace}



\begin{comment}
\subsection{\Shaeeroth}
\end{comment}

% Nexagglachel's curse. 
\newcommand{\NexagglachelsCurse}          {\ps{\Nexagglachel} Curse\xspace}

% Nexagglacheldraex, a Dragonlord, the eldest son of \KhothSell.
\newcommand{\Nexagglachel}                {Nexaggrael\xspace}
\newcommand{\RaemythKhivaashNexagglachel} {\Raemyth \Khivaash \Nexagglachel}
\newcommand{\RaemythNexagglachel}         {\Raemyth \Nexagglachel}
\newcommand{\Khivaash}                    {Khivaash\xspace}
\newcommand{\Raemyth}                     {Raem\yvowel th\xspace}

% The third and youngest son of \KhothSell, who is actually Hriist'tet Nechsain.
\newcommand{\IrocasSecherdamon}           {\Irocas \Secherdamon}
\newcommand{\Secherdamon}                 {Secherdamon\xspace}
\newcommand{\Veldraxx}                    {Veldr\"axx\xspace}
\newcommand{\Irocas}                      {Irocas\xspace}
\newcommand{\HriistD}                     {\Secherdamon}

% Ishnaruchyfir, a Dragonlord and Ashenclaw knight, the second son of \KhothSell. 
\newcommand{\QuessanthIshnaruchaefir}     {\Quessanth \Ishnaruchaefir}
\newcommand{\Quessanth}                   {Quessanth\xspace}
\newcommand{\Ishnaruchyfir}               {\Ishnaruchaefir}
\newcommand{\Iscrafel}                    {Iscrafel\xspace}
\newcommand{\Ishnaruchaefir}              {\Iscrafel}
%\newcommand{\Ishnaruchaefir}              {Ishnaruch\aediphthong fir\xspace}
\newcommand{\Melechet}                    {Melechet\xspace}
\newcommand{\Nierzshah}                   {Nierzshah\xspace}
\newcommand{\Tzeorossh}                   {Tzeorossh\xspace}

\newcommand{\Rephexsar}                   {Rephexsar\xspace}
\newcommand{\Dioreth}                     {Dioreth\xspace}
\newcommand{\DiorethSethicusRephexsar}    {Dioreth Sethicus Rephexsar\xspace}
\newcommand{\DiorethRephexsar}            {Dioreth Rephexsar\xspace}

\newcommand{\Skelcurmaggra}               {Skelcurmaggra\xspace}
\newcommand{\Thessulax}                   {Thessulax\xspace}

% Vizsherioch, the son of Secherdamon.
\newcommand{\Vizsherioch}                 {Vizsherioch\xspace}



\begin{comment}
\subsection{Other \Dragons}
\end{comment}

% Nisgzarchief, Cryocas' lover
\newcommand{\Nisgzarchief}                {Nisgzarchiev\xspace}

% Cryocas Nzessuacrith, Ishnaruchyfir's daughter.
\newcommand{\CryocasNzessuacrith}         {\Cryocas \Nzessuacrith}
\newcommand{\Cryocas}                     {Cr\ydiphthong ocas\xspace}
\newcommand{\Nzessuacrith}                {Nzessuacrith\xspace}

% Ishnaruchaefir's beloved. 
\newcommand{\Triestessakhin}              {R\yvowel stessakhin\xspace}
\newcommand{\Rystessakhin}                {R\yvowel stessakhin\xspace}
\newcommand{\AeocrithRystessakhin}        {\Aeocrith \Rystessakhin}
\newcommand{\Aeocrith}                    {Aeocrith\xspace}

% Ishnaruchaefir's three grandchildren.
\newcommand{\Tentocoth}                   {Tentocoth\xspace}
\newcommand{\Thiencaste}                  {Sincastra\xspace}
\newcommand{\Rathyon}                     {Rathyon\xspace}

\newcommand{\Laccashyth}                  {Laccash\ydiphthong th\xspace}
\newcommand{\Vaccashyth}                  {Laccash\ydiphthong th\xspace}

\newcommand{\Vexstrasshin}                {Vexstrasshin\xspace}

\newcommand{\Xarocchetsel}                {Xarocchetsel\xspace}

\newcommand{\Zessuruch}                   {Zessuruch\xspace}

\newcommand{\Naruc}                       {N\along ruc\xspace}
\newcommand{\Osanggrath}                  {Osanggrath\xspace}
\newcommand{\NarucOsanggrath}             {\Naruc{} Osanggrath\xspace}

% Leader of the Imetrium, previously known as Salacar.
\newcommand{\Sarokash}                    {Sarokash\xspace}







\begin{comment}
\section{Servants of the \Dragons}
\end{comment}

% Criseis, Ishnaruchaëfir's servant.
\newcommand{\Criseis}     {Cris\"eis{}\xspace}

% Psiotai, Rissit's \scathaese{} servitor.
\newcommand{\Psyrex}      {Ps\ydiphthong rex{}\xspace}
\newcommand{\LocarPsyrex} {\Locar{} \Psyrex{}\xspace}
\newcommand{\Locar}       {Locar{}\xspace}













\begin{comment}
\chapter{Erebean Faction}
\end{comment}









\begin{comment}
\section{\Banes}
\end{comment}

% Voidbringer, the Baneking.
\newcommand{\Voidbringer}      {Voidbringer{}\xspace}
\newcommand{\Morogamhoth}      {Morogamhoth\xspace}
\newcommand{\Zephron}          {Zephron\xspace}

% Daggerrain, the Bane Overlord of \Miith{}.
\newcommand{\Daggerrain}       {Daggerrain{}\xspace}
\newcommand{\Amothanaxur}      {Amothanaxur\xspace}

% See also the section about Qliphoth. Each Qliphah is actually a Banelord.



\begin{comment}
\section{\Resphain}
\end{comment}



\begin{comment}
\subsection{\CiriathSepher}
\end{comment}

\begin{comment}
\subsubsection{\Satharioth}
\end{comment}

\newcommand{\Azraid}       {\Aflatresphan zer\adarkresphan id{}\xspace}
\newcommand{\Azeraid}      {\Azraid}
\newcommand{\Gepheral}     {Gepher\adarkresphan l{}\xspace}
\newcommand{\Gevural}      {Gepher\adarkresphan l{}\xspace}
\newcommand{\Harbeth}      {H\adarkresphan sher\adarkresphan pel\xspace} % Harbeth, the Raven of the Battlefields.
\newcommand{\Morcariel}    {Morc\adarkresphan riel\xspace}
\newcommand{\Mehaloch}     {Men\aflatresphan{}loch\xspace} % Mehaloch, a devourer.
\newcommand{\Shehizol}     {Shehizol{}\xspace}

\birthliving{Azraid}       {Semiza found}{-608}

\begin{comment}
\subsubsection{\Ketherain}
\end{comment}

\newcommand{\Firaxel}      {F\ilongresphan r\adarkresphan xel\xspace}
\newcommand{\Menessiaraid} {Menessi\adarkresphan r\adarkresphan id\xspace}
\newcommand{\Teshrial}     {Teshri\aflatresphan l\xspace}
\newcommand{\Zeirath}      {\Teshrial}
\newcommand{\Tuerdal}      {Tu\"erd\adarkresphan l{}\xspace} % Tuerdal, Teshrial's father. 
\newcommand{\Vesrai}       {Vesr\adarkresphan i{}\xspace} % Vesrai, Teshrial's mother.
\newcommand{\Zereth}       {Zereth{}\xspace} % Zereth, daughter of Azraid. 

\birthliving{Zereth}       {Azraid birth}  {1364}
\birthliving{Vesrai}       {Zereth birth}  {2085}
\birthliving{Tuerdal}      {Zereth birth}  {2516}
\birthliving{Teshrial}     {Runger war}    {-1167}
\birthliving{Firaxel}      {Teshrial birth}{-480}

\begin{comment}
\subsubsection{\Thelyadeth}
\end{comment}

\newcommand{\Ganethed}     {G\aflatresphan{}nethed\xspace} 
\newcommand{\Jeshred}      {Jeshred{}\xspace} % Jeshred, mother of Zereth. 
\newcommand{\Paerzim}      {V\aeresphan rz\ilongresphan m\xspace} % Working for Teshrial in Malcur.
\newcommand{\Urizeth}      {Urizeth\xspace}

\birthliving{Ganethed}     {Teshrial birth}{-418}



\begin{comment}
\subsection{\Kezerad}
\end{comment}

\begin{comment}
\subsubsection{\Satharioth}
\end{comment}

\newcommand{\Sithiyacaan}      {Sithi\y \adarkresphan c\alongresphan n\xspace} % Last Kezeradi prince.
\newcommand{\MoriceHerette}    {Mori\c ce \Herette} % Sithiyacaan in disguise.
\newcommand{\Herette}          {H\'er\`ette\xspace}

\begin{comment}
\subsubsection{\Ketherain}
\end{comment}

\newcommand{\Essenai}      {Essen\adarkresphan{}i\xspace} % A wise one, later a Sephirah.

\begin{comment}
\subsubsection{\Thelyadeth}
\end{comment}

% Eryal, the Malach that became Silqua.
\newcommand{\Ariel}        {Er\y al\xspace}
\newcommand{\Aryal}        {Er\y al\xspace}
\newcommand{\Eryal}        {Er\y al\xspace}

\newcommand{\Sevestris}    {Sevestris\xspace} % Sithiyacaan's beloved. 



\begin{comment}
\subsection{\Mystraacht}
\end{comment}

\begin{comment}
\subsubsection{\Satharioth}
\end{comment}

\newcommand{\Nathrach}     {N\adarkresphan thr\alongresphan ch{}\xspace}
\newcommand{\Ramiel}       {R\adarkresphan miel{}\xspace}
\newcommand{\Shiaraid}     {Sh\ilongresphan\adarkresphan r\adarkresphan id{}\xspace}
\newcommand{\Zachirah}     {\Netzach}
\newcommand{\Netzachirah}  {\Netzach}
\newcommand{\Netzach}      {Netz\adarkresphan ch\xspace}

\birthliving{Zachirah}     {Semiza found}{-691}
\birthliving{Ramiel}       {Semiza found}{-490}
\birthliving{Tzerachiel}   {Semiza found}{-479}

\begin{comment}
\subsubsection{\Ketherain}
\end{comment}

% Dasteron, would-be Overlord of Mystraacht. 
\newcommand{\Dasteron}     {D\aflatresphan steron\xspace}
% Kishiel, daughter of Ramiel. 
\newcommand{\Cishiel}      {Cishiel\xspace}
\newcommand{\Kishiel}      {\Cishiel}
% Ozariel, son of Zachirah and father of Dasteron. 
\newcommand{\Ozariel}      {\Oroundresphan z\adarkresphan riel\xspace}

\begin{comment}
\subsubsection{\Thelyadeth}
\end{comment}

% Dezruth of Mystraacht. 
\newcommand{\Dezruth}      {Dezruth{}\xspace}
% Gilchad, an accomplice of Kishiel.
\newcommand{\Gilchad}      {Gilch\adarkresphan d{}\xspace}
% Sargamel, cousin of Dasteron. 
\newcommand{\Sargamel}     {S\adarkresphan rg\adarkresphan mel\xspace}
% Themirod, cousin of Dasteron. 
\newcommand{\Themirod}     {Themirod\xspace}



\begin{comment}
\subsection{\TiphredSerah}
\end{comment}

\begin{comment}
\subsubsection{\Satharioth}
\end{comment}

% Dorzand, a wise philosopher.
\newcommand{\Dorzand}      {Dorz\adarkresphan{}nd\xspace}
% Ishicah, a Resvil and Sathariah who was captured and enslaved by the Ortaicans. 
\newcommand{\Ishicah}      {Ishic\ahresphan{}\xspace}
% Quelthah. 
\newcommand{\Quelthah}     {Quelth\ahresphan{}\xspace}

\begin{comment}
\subsubsection{\Ketherain}
\end{comment}

% Lothagiel, who once tried to slay Ishnaruchaefir. 
\newcommand{\Lothagiel}    {L\oroundresphan th\adarkresphan giel\xspace}

\begin{comment}
\subsubsection{\Thelyadeth}
\end{comment}

\newcommand{\Essaryn}      {Ess\adarkresphan r\yvowel n\xspace}
\newcommand{\Nemuragh}     {Nemur\adarkresphan gh{}\xspace} % Lothagiel's gay lover.



\begin{comment}
\subsection{\Merkyrah}
\end{comment}

\newcommand{\Berugiel}         {Berugiel\xspace}
\newcommand{\Damiarch}         {D\aflatresphan mi\adarkresphan rch\xspace}
\birthliving{Damiarch}         {Semiza found}{513}
\newcommand{\Dolsharra}        {Dolsh\adarkresphan rr\adarkresphan\xspace}
\newcommand{\Kezrabal}         {\Tezrabal}
\newcommand{\Lyorith}          {Ly\oroundresphan rith\xspace}
\newcommand{\Monara}           {Mon\alongresphan r\adarkresphan{}\xspace}
\newcommand{\Shadrach}         {Sh\aflatresphan rr\adarkresphan th\xspace}
\newcommand{\Sharrath}         {\Shadrach}
\newcommand{\Tezrabal}         {Tezr\adarkresphan b\adarkresphan l\xspace}
\newcommand{\Vahaniel}         {V\adarkresphan h\adarkresphan niel\xspace}



\begin{comment}
\subsection{Others}
\end{comment}

\begin{comment}
\subsubsection{Ancients}
\end{comment}

\newcommand{\Sartheron}        {S\adarkresphan rtheron\xspace}
\newcommand{\Thanatzil}        {Th\aflatresphan n\aflatresphan tzil\xspace}


\begin{comment}
\subsubsection{\Bezedeth}
\end{comment}

\newcommand{\Achsah}       {\Adarkresphan chs\ahresphan\xspace}
\newcommand{\Lethiarch}    {Lethi\adarkresphan rch\xspace} % A god of Uruthar.
\newcommand{\Osra}         {Osr\adarkresphan{}\xspace} % A god of Uruthar.

\birthliving{Achsah}       {Murder of the Dawn}{-233}
% Lelmach, a Bezed serving Urizeth. 
\birthliving{Lelmach}      {Teshrial birth}{-284}


\begin{comment}
\subsubsection{\Baelzerach}
\end{comment}

% Najarod, an ally of Quessanth Ishnaruchaefir. 
\newcommand{\Najarod}      {N\adarkresphan j\adarkresphan rod\xspace}
\newcommand{\Shiin}        {Shiin\xspace}









\begin{comment}
\section{\Aryothim}
\end{comment}

\newcommand{\Gorgomon}         {Gorgomon\xspace}
\newcommand{\Klaad}            {Klaad\xspace}
\newcommand{\Murru}            {Murru\xspace}









\begin{comment}
\section{\Nephilim}
\end{comment}

% Semiza (Semyaza), the last god-king of the Nephilim
\newcommand{\Semiza}       {Sem\ilongresphan z\aflatresphan\xspace}

% Eshayzal (Azazel), Semiza's companion.
\newcommand{\Eshayzal}     {Esha\yconsonantnephil z\adarkresphan l\xspace}

% Ilu, Semiza's daughter. 
\newcommand{\Ilu}          {Ilu\xspace}

% Morza, a heroic warrior. 
\newcommand{\Morza}        {Morza\xspace}













\begin{comment}
\chapter{Mortals}
\end{comment}

% Carzain Shireyo is one of the most pivotal characters, so he is listed first. 
\birthliving{Carzain}{Mutiny}{-22}









\begin{comment}
\section{Belkadians (miscellaneous)}
\end{comment}

% The name of the last High King of the Belkadian Empire. 
\newcommand{\LastHighKing}     {Leopold\xspace}









\begin{comment}
\section{Geicans}
\end{comment}

\newcommand{\Belzhir}          {Bel\v zir\xspace} 
\newcommand{\Belzir}           {\Belzhir}
\newcommand{\Cormin}           {Cormin\xspace}% Cormin, the mother of Belzir.
\newcommand{\Freid}            {Freid\xspace} % Hayad's family name.
\newcommand{\Kazzed}           {Kazzed\xspace} % Shereid's family name.

\birthliving{Shereid} {Carzain birth}{-5}









\begin{comment}
\section{Imetrians}
\end{comment}

% Ilcas Northstar's military rank when he meets Carzain
\newcommand{\IlcasStartRank}   {\Salican}
\newcommand{\IlcSR}            {\IlcasStartRank}

\birthliving{Ilcas Northstar} {Mutiny}{-39}
\birthliving{Raeco Mannica}   {Mutiny}{-40}
\birthliving{Curiet Serpentin}{Mutiny}{-22}
\birthliving{Countess}        {Mutiny}{-10}
\birthliving{Razor}           {Mutiny}{-6}
\birthliving{Equin Mirai}     {Mutiny}{-20}
\birthliving{Ulphon Nestor}   {Mutiny}{-51}









\begin{comment}
\section{\Imrathi}
\end{comment}

\begin{comment}
\subsubsection{Humans}
\end{comment}

\newcommand{\Delaen}           {Delaen\xspace} % Silqua's family name.

%Scion of Shiaraid during Silqua's time. 
% Possible names: Delphine, Delcardi, Cassilda, Eryx Isherai, Iscariah. 
\newcommand{\Delphine}         {Iscari\ahresphan{}\xspace} 

\newcommand{\Zether}           {Zether\xspace}% Son of Cordos and Silqua.

\begin{comment}
\subsubsection{Scathae}
\end{comment}

\newcommand{\Byakun}           {Byak\ulong{}n\xspace}% A dark sorcerer-priest.









\begin{comment}
\section{Pelidorians}
\end{comment}

\begin{comment}
\subsubsection{House Pelidor}
\end{comment}


\newcommand{\Dornaer}          {Dornaer\xspace} % Icor's younger brother.
\newcommand{\Icor}             {Icor\xspace}   % The Duke of Pelidor, now deceased
\newcommand{\Tiroco}           {Tiroco\xspace} % The Duchess of Pelidor
\newcommand{\Sethgal}          {Sethgal\xspace} % Pelidorian Marshal.

\birthandage{Icor}          {Runger war}{-39}{38}
\birthliving{Tiroco}        {Runger war}{-27}
\birthliving{Sethgal}       {Runger war}{-46}
\birthliving{Liocai}        {Runger war}{-37}
\birthliving{Dornaer}       {Runger war}{-35}
\birthliving{Roric}         {Runger war}{-7}
\birthliving{Frico}         {Runger war}{-3}

\hisc{Icor becomes duke}{Roric birth}{-2}

\begin{comment}
\subsubsection{At the court in Malcur}
\end{comment}

\newcommand{\Risvet}           {Risvet\xspace}% Baroness Rispet Hemfork

\newcommand{\Minister}         {Minister\xspace} % Jasper Bartholomy's title.
\newcommand{\minister}         {minister\xspace}

\newcommand{\WimarNorden}      {Wulfwin Norden\xspace} % Telcra cleric. 

\newcommand{\Kintaer}          {Kintaer\xspace} % Kintaer, a noble family. 
\newcommand{\Theal}            {Th\"eal\xspace} % Earl Theal Kintaer. 
\newcommand{\Constance}        {Elfrin\xspace}  % Constance Kintaer.

\birthliving{Constance Kintaer}{Runger war}{-18}

\begin{comment}
\subsubsection{Commoners in Malcur}
\end{comment}


\newcommand{\Urban}     {Urban\xspace}     % A Sentinel mafia boss.
\newcommand{\Brittany}  {\Piacet}          % Charcoal's Cabalist subordinate.
\newcommand{\Piacet}    {Pia\c cet\xspace}

\newcommand{\Bryon}     {Br\ydiphthong on\xspace} % Bryon Carpenter.
\newcommand{\Mya}       {M\ydiphthong a\xspace}   % A girl Rian once had the hots for.
\newcommand{\Uswa}      {Uswa\xspace}             % The crazy old woman in Malcur. 

\birthliving{Rian}             {Mutiny}      {-16}
\birthliving{Badrick}          {Mutiny}      {-19}
\birthliving{Bryon Carpenter}  {Mutiny}      {-52}
\birthliving{Neina}            {Mutiny}      {-15}
\birthliving{Needle}           {Runger war}  {-29}
\birthandage{Belya}            {Needle birth}{ -2}{16}

\birthliving{Rod Baker}        {Mutiny}      {-39}
\birthliving{Dennick}          {Rian birth}  {-15}
\birthliving{Uswa}             {Mutiny}      {-51}
\birthliving{Jorgen}           {Rian birth}  {-9}

\begin{comment}
\subsubsection{In Redglen}
\end{comment}

\begin{comment}
\paragraph{Shireyo household}
\end{comment}

% Shireyo, Carzain's and Nishain's last name.
\newcommand{\Shachar}    {Shachar\xspace}
\newcommand{\Shireyo}    {Shachar\xspace}
\newcommand{\MrShachar}  {\Mr{} \Shireyo}
\newcommand{\MrShireyo}  {\Mr{} \Shireyo}
\newcommand{\DaiShireyo} {\Dai\Shireyo}

% Roanne/Rowena, Carzain's mother's first name. 
\newcommand{\Roanne}   {Roanne\xspace}

% Deracille, Carzain's mother's last name.
\newcommand{\Deracil}   {\Deracille}
\newcommand{\Delishe}   {\Deracille}
\newcommand{\Deracille} {D\'er\^a\c cille\xspace}

% Carzain's full name
\newcommand{\CarzainDeracilleShireyo} {Carzain \Deracille{} \Shireyo}
\newcommand{\CarzainShireyo}          {Carzain \Shireyo}
\newcommand{\CarzainShachar}          {Carzain \Shachar}

\birthliving{Nishain Shireyo}    {Carzain birth}            {-33}
\birthliving{Roanne Deracille}   {Nishain Shireyo birth}    {+8}

\begin{comment}
\paragraph{Others}
\end{comment}

\newcommand{\Symeon}  {S\ydiphthong m\"eon{}\xspace} % An officer.
\newcommand{\Weylon}  {Weylon{}\xspace} % The goldsmith's uncle

\birthliving{Denuis}             {Mutiny}                   {31}
\birthliving{Leopold Goldsmith}  {Carzain birth}            {0}
\birthliving{Frederick Goldsmith}{Leopold Goldsmith birth}  {3}
\birthliving{Gavin Goldsmith}    {Frederick Goldsmith birth}{21}
\birthliving{Sir Guy}            {Carzain birth}            {42}
\birthliving{Ethwar Carver}      {Runger war}               {43}
\birthliving{Theal Carver}       {Runger war}               {5}
\birthliving{Mesna Tailor}       {Runger war}               {48}
\birthliving{Symeon}             {Runger war}               {38}
\birthliving{Gregory Smith}      {Carzain birth}            {2}

\begin{comment}
\subsubsection{The Ishrah}
\end{comment}

\newcommand{\MoroCobrel}       {Moro Cobrel\xspace}
\newcommand{\Cobrel}           {Cobrel\xspace} % Moro Cornel, Pelidorian archmage. 
\newcommand{\Cornel}           {Cobrel\xspace}
\newcommand{\MissCornel}       {\Miss Cobrel\xspace}

\birthliving{Moro Cobrel}              {Runger war}               {-63}
\hisc{Moro Cobrel flees Yormis}        {Moro Cobrel birth}        {20}
\hisc{Suthis Mephilex learns Arcana}   {Moro Cobrel flees Yormis} {-7}
\birthliving{Suthis Mephilex}          {Suthis Mephilex learns Arcana} {-18}
\birthliving{Suthis Cruan}             {Suthis Mephilex birth}         {-39}

\newcommand{\Ambrose}          {Amras\xspace} % Ambrose Anatoli, a Pelidorian mage.
\newcommand{\Onatol}           {Onatol\xspace}
\newcommand{\Anatoli}          {\Onatol}


% A \scathaese \rethyax, Curwen's second-in-command 
\newcommand{\Sanyor} {Sanyor\xspace}

\birthliving{Archibald Curwen} {Runger war}{-60}
\birthliving{Onatol}           {Runger war}{-77}
\birthliving{Sanyor}           {Runger war}{-55}
\birthandage{Borg Zelab}       {Runger war}{-79}{77} % Borg Zelab, a Cabalist. 

\begin{comment}
\subsubsection{The Military}
\end{comment}

\newcommand{\Delph}    {Delph\xspace}    % A Human soldier, Carzain's friend. 
\newcommand{\Tsekkect} {Tsekkect\xspace} % A Meccaran soldier, Carzain's friend. 

\birthliving{Tsekkect} {Runger war}   {-18}
\birthliving{Delph}    {Carzain birth}{1}









\begin{comment}
\section{Redcor}
\end{comment}

\begin{comment}
\subsubsection{Vaimon Age}
\end{comment}

\begin{comment}
\subsubsection{Ortaican Age}
\end{comment}

% Iolivine, a Scion of Ishicah. 
\newcommand{\Iolivine} {Iolivine{}\xspace}

\begin{comment}
\subsubsection{Velcadian Age, Redce}
\end{comment}

% Raquel (or whatever), a Redcor Cleric and Carzain's childhood friend
\newcommand{\Raquel} {R\^ac\`el{}\xspace}
\newcommand{\Racel}  {\Raquel{}\xspace}
\newcommand{\Rakel}  {\Raquel{}\xspace}
\newcommand{\Raqel}  {\Raquel{}\xspace}

% Esmerel, the Redcor that discovers Carzain and takes him to Redce
\newcommand{\EsmerelFull}   {\ChyrieEsmerel{}\xspace}
\newcommand{\ChyrieEsmerel} {\Chyrie{} \Esmerel{}\xspace}
\newcommand{\Esmerel}       {\`Esmer\`el{}\xspace}
\newcommand{\Chyrie}        {\c Ch\yvowelredcor rie{}\xspace}

% Sylvie Dereine, a historian. 
\newcommand{\Dereine}  {Der\`eine{}\xspace}
\newcommand{\Sylvie}   {S\yvowelredcor lvie{}\xspace}

% Redcor Matriarch, `leader' of the Conclave.
\newcommand{\Dominice} {Dominic\finale{}\xspace} 

% Redcor Matriarch, big and fat.
\newcommand{\Brizide}  {Brizide{}\xspace} 

% Laetitia Lacquasse, warrior woman.
\newcommand{\Laetitia}{Laetitia{}\xspace}
\newcommand{\Brizen}{Bri\v zen{}\xspace} % Lacquasse's family name.

\newcommand{\PatriccoKimon} {Patrico Kimon\xspace}
\newcommand{\Patricco}      {Patrico\xspace}
\newcommand{\Kimon}         {Kimon\xspace}

\birthliving{Chyrie Esmerel}    {Runger war}   {-62}
\birthliving{Racel Galisetti}   {Carzain birth}{0}
\birthliving{France Perival}    {Runger war}   {-33}
\birthliving{Isacc Chiran}      {Runger war}   {-30}
\birthandage{Sylvie Dereine}    {Runger war}   {-335}{67}
\birthliving{Clarice Camilienne}{Runger war}   {-36}

\begin{comment}
\subsubsection{Pelidor}
\end{comment}

% Zacophine Vincerre, a Redcor emissary in Pelidor.
\newcommand{\Zacophine}        {\v Z\^acophine{}\xspace}
\newcommand{\Vincerre}         {Vin\c c\`erre{}\xspace}
% Clarice Camilienne, a Redcor emissary in Pelidor. 
\newcommand{\Clarice}          {Cl\^ari\c ce{}\xspace}
\newcommand{\Camilienne}       {Camili\`enne{}\xspace}
\newcommand{\ClariceCamilienne}{\Clarice{} Camilienne{}\xspace}

% France Perival, a Gandierre. 
\newcommand{\France}  {Fr\anredcor\c ce{}\xspace}
\newcommand{\Perival} {P\'erival{}\xspace}

% Isacc Chiran, a Gandierre. 
\newcommand{\Isacc}  {Is\^ac{}\xspace}
\newcommand{\Chiran} {\c Chir\anredcor{}\xspace}











\begin{comment}
\section{Rissitics}
\end{comment}

% Male Scathaese Ashenoch, general.
\newcommand{\Narkiza}  {Sesstr\adarkrissitic{}\xspace} 

% Narkiza's morning star.
\newcommand{\Femtu}    {Femtu{}\xspace} 
\newcommand{\Forshval} {\Femtu{}\xspace}

% Narkiza's assistant officer.
\newcommand{\Kufur} {Kufur{}\xspace} 

% An evil male human Ashenoch. 
\newcommand{\Geldashad}  {Geld\aflatrissitic sh\aflatrissitic d{}\xspace}

% A sexy female Human Ashenoch.
\newcommand{\Dzerezd}   {\Dzerezdin{}\xspace}
\newcommand{\Dzerezdin} {Dzerezdin{}\xspace}

% Cool Shessefkesad mage with a history. Inspired by Quick Ben from Malazan Book of the Fallen.
\newcommand{\Dzasselid}  {Dz\aflatrissitic sselid{}\xspace}

% Shesshefkesad mage.
\newcommand{\Shilred}  {Shilred{}\xspace}
\newcommand{\Filgzed}  {\Shilred{}\xspace}
\newcommand{\Filshed}  {\Filgzed{}\xspace}

% A barbarian warrior.
\newcommand{\KarsaOrlong}  {Kw\^al-shekh{}\xspace}

% The reigning Tsalt-Nyzleth. 
\newcommand{\TessHanith} {Tes'h\^anith{}\xspace}
\newcommand{\TesHanith}  {\TessHanith{}\xspace}
% A Draconic Urr-Gammosh
\newcommand{\Dasvedshiracht}  
  {D\aflatrissitic sv\'ed'shir\adarkrissitic cht{}\xspace}

% Narkiza's flagship
\newcommand{\MotherTiamat}    {\shipname{Mother Tiamat}{}\xspace} 









\begin{comment}
\section{Rungerans}
\end{comment}

\begin{comment}
\subsubsection{The King's Court}
\end{comment}

% Morgan's son and daughter. 
\newcommand{\Mathyas} {Math\ydiphthong as\xspace}
\newcommand{\Iselle}  {Is\`elle\xspace}

\birthliving{Morgan Runger}{Mutiny}{-55}

\begin{comment}
\subsubsection{The Ishrah}
\end{comment}

% Takestsha, a mage in Runger. Actually Nzessuacrith in disguise.
\newcommand{\Takestsha}  {Takestsha{}\xspace}

% Jirad (previously Giles) Tantor.
\newcommand{\Jirad}  {Jirad{}\xspace}

% His son. 
\newcommand{\Mycah} {M\ydiphthong c\ahresphan{}\xspace}

% Orla of Hemfork, Rungeran mage.
\newcommand{\Orla}  {Orla{}\xspace}

\birthliving{Jirad Tantor}    {Tantor diary}{-50}
\birthliving{Mycah Tantor}    {Tantor diary}{-15}
\birthliving{Orla of Fanshire}{Tantor diary}{-54}
\birthliving{Rosen Jaegwin}   {Tantor diary}{-20}









\begin{comment}
\section{Shurco}
\end{comment}

\newcommand{\DulNepherRamas}   {Dul-Nepher-Ramas\xspace}









\begin{comment}
\section{Vaimons}
\end{comment}

% Redcor pronunciation of Daemien Iras Shanix.
\newcommand{\DamianChanici} {Dami\anredcor \c Ch\^ani\c ci{}\xspace}

\newcommand{\Ivesser}                {Ivesser\xspace} % A woman of Vizicar's time.

% Vizicar's predecessor as Vaimon Caliph.
\newcommand{\LucionRinOrcas}         {\Lucion{} Rin Orcas{}\xspace}
\newcommand{\Lucion}                 {Lu\c cion{}\xspace}

% Tydesmos Gendar-in-Caphet.
\newcommand{\Tydesmos}               {T\ydiphthong desmos{}\xspace}
\newcommand{\TydesmosGendarInCaphet} {\Tydesmos{} Gendar-in-Caphet{}\xspace}

% Vizicar Duras Respina.
\newcommand{\VizicarDurasRespina}    {Vizicar Duras Respina{}\xspace}









\begin{comment}
\section{Yormis}
\end{comment}

\newcommand{\TulionSperra}     {Tulion Sperra\xspace}
\newcommand{\Sperra}           {Sperra\xspace}
\newcommand{\Tulion}           {Tulion\xspace}

\newcommand{\UldraanKerross}   {Uldraan Kerross\xspace}
\newcommand{\Uldraan}          {Uldraan\xspace}
\newcommand{\Kerross}          {Kerross\xspace}









\begin{comment}
\section{Others}
\end{comment}

% Catrian, who dies in the prologue. 
\birthandage{Catrian}{Catrian dies}{-41}{41}

\newcommand{\Cyri}             {C\ydiphthong ri\xspace}













\begin{comment}
\chapter{Others}
\end{comment}









\begin{comment}
\section{Chthonian Powers}
\end{comment}

\newcommand{\Malgaddon}        {Malgaddon\xspace}
\newcommand{\Ubloth}           {Ubloth\xspace}
\newcommand{\Yagnathul}        {Yagnathul\xspace}









\begin{comment}
\section{Ortaican gods}
\end{comment}


\begin{comment}
\subsection{Primordials}
\end{comment}

\newcommand{\Primordials}      {Primordials\xspace}
\newcommand{\Primordial}       {Primordial\xspace}
\newcommand{\primordials}      {Primordials\xspace}
\newcommand{\primordial}       {Primordial\xspace}

\newcommand{\Costorul}         {Costorul\xspace}  %\KhothSell).
\newcommand{\Kythraxas}        {Kythraxas\xspace} %\KyaethemChreiAz). 
\newcommand{\Nelxurra}         {Nelxurra\xspace}  %\NerranKoss). 
\newcommand{\Rammasul}         {Rammasul\xspace}  % \NaathKurRamalech). 

\begin{comment}
\subsection{Younger gods}
\end{comment}

% Taorthae: Collective term for the gods. 
\newcommand{\Taorthae}         {\Taortha{}e\xspace}
\newcommand{\Taortha}          {Taortha\xspace}
\newcommand{\taorthae}         {\Taortha{}e\xspace}
\newcommand{\taortha}          {Taortha\xspace}

\newcommand{\Daxian}           {Tlazcan\xspace}
\newcommand{\Isxae}            {Isxae\xspace}
\newcommand{\Llorgul}          {Llorgul\xspace}
\newcommand{\Mezzagrael}       {Mezzagra\"el\xspace}
\newcommand{\Settras}          {Settras\xspace}
\newcommand{\Usherain}         {Usherain\xspace}



















\begin{comment}
\part{Typesetting}
\end{comment}
















\begin{comment}
\chapter{Ordinary words}
\end{comment}

\begin{comment}
\section{Danish}
\end{comment}

% DJØF.
\newcommand{\DJOF}             {DJ\O F{}\xspace}



\begin{comment}
\section{English}
\end{comment}

\newcommand{\Armoured}         {Armoured\xspace}
\newcommand{\Armours}          {Armours\xspace}
\newcommand{\Armour}           {Armour\xspace}
\newcommand{\armoury}          {armoury\xspace}
\newcommand{\armoured}         {armoured\xspace}
\newcommand{\armours}          {armours\xspace}
\newcommand{\armour}           {armour\xspace}

\newcommand{\Clamour}          {Clamour\xspace}
\newcommand{\Clamoured}        {Clamoured\xspace}
\newcommand{\clamour}          {clamour\xspace}
\newcommand{\clamoured}        {clamoured\xspace}

\newcommand{\cliche}           {clich\'e{}\xspace}

\newcommand{\Coloured}         {Coloured\xspace}
\newcommand{\Colours}          {Colours\xspace}
\newcommand{\Colour}           {Colour\xspace}
\newcommand{\coloured}         {coloured\xspace}
\newcommand{\colours}          {colours\xspace}
\newcommand{\colour}           {colour\xspace}

\newcommand{\coordination}     {co\"ordination\xspace}
\newcommand{\coordinated}      {co\"ordinated\xspace}
\newcommand{\coordinate}       {co\"ordinate\xspace}

\newcommand{\cooperate}        {co\"operate\xspace}
\newcommand{\cooperating}      {co\"operating\xspace}

\newcommand{\fiancee}          {fianc\'ee{}\xspace}

\newcommand{\Favoured}         {Favoured\xspace}
\newcommand{\Favours}          {Favours\xspace}
\newcommand{\Favour}           {Favour\xspace}
\newcommand{\favoured}         {favoured\xspace}
\newcommand{\favours}          {favours\xspace}
\newcommand{\favour}           {favour\xspace}

\newcommand{\Honorary}         {Honorary\xspace}
\newcommand{\Honourable}       {Honourable\xspace}
\newcommand{\Honoured}         {Honoured\xspace}
\newcommand{\Honours}          {Honours\xspace}
\newcommand{\Honour}           {Honour\xspace}
\newcommand{\honorary}         {honorary\xspace}
\newcommand{\honourable}       {honourable\xspace}
\newcommand{\honoured}         {honoured\xspace}
\newcommand{\honours}          {honours\xspace}
\newcommand{\honour}           {honour\xspace}

\newcommand{\Laboured}         {Laboured\xspace}
\newcommand{\Labours}          {Labours\xspace}
\newcommand{\Labour}           {Labour\xspace}
\newcommand{\laboured}         {laboured\xspace}
\newcommand{\labours}          {labours\xspace}
\newcommand{\labour}           {labour\xspace}

\newcommand{\manoeuvrability}  {manoeuvrability\xspace}
\newcommand{\manoeuvrable}     {manoeuvrable\xspace}
\newcommand{\manoeuvred}       {manoeuvred\xspace}
\newcommand{\manoeuvres}       {manoeuvres\xspace}
\newcommand{\manoeuvre}        {manoeuvre\xspace}

\newcommand{\metres}           {metres\xspace}
\newcommand{\metre}            {metre\xspace}

\newcommand{\naivete}          {\naive te{}\xspace}
\newcommand{\naive}            {na\"ive{}\xspace}

\newcommand{\Squalour}         {Squalour\xspace}
\newcommand{\squalour}         {squalour\xspace}

\newcommand{\Skepticism}       {Skepticism\xspace}
\newcommand{\Skeptical}        {Skeptical\xspace}
\newcommand{\Skeptic}          {Skeptic\xspace}
\newcommand{\skepticism}       {skepticism\xspace}
\newcommand{\skeptical}        {skeptical\xspace}
\newcommand{\skeptic}          {skeptic\xspace}
\newcommand{\Scepticism}       {Skepticism\xspace}
\newcommand{\Sceptical}        {Skeptical\xspace}
\newcommand{\Sceptic}          {Skeptic\xspace}
\newcommand{\scepticism}       {skepticism\xspace}
\newcommand{\sceptical}        {skeptical\xspace}
\newcommand{\sceptic}          {skeptic\xspace}

\newcommand{\traveling}        {travelling\xspace}
\newcommand{\traveled}         {travelled\xspace}
\newcommand{\travelers}        {travellers\xspace}
\newcommand{\traveler}         {traveller\xspace}
\newcommand{\travelling}       {travelling\xspace}
\newcommand{\travelled}        {travelled\xspace}
\newcommand{\travellers}       {travellers\xspace}
\newcommand{\traveller}        {traveller\xspace}



\begin{comment}
\section{Fictional}
\end{comment}

% Melnibone, the homeland of Michael Moorcock's Elric.
\newcommand{\Melnibone}        {Melnibon\'e\xspace}
\newcommand{\melnibone}        {Melnibon\'e\xspace}



\begin{comment}
\section{French}
\end{comment}

\newcommand{\blase}            {blas\'e}
\newcommand{\boheme}           {boh\`eme\xspace}

% D\'ej\`a-vu. 
\newcommand{\Dejavu}           {D\'ej\'a-vu\xspace}
\newcommand{\Dejavus}          {\Dejavu{}s\xspace}
\newcommand{\Deajvus}          {\Dejavus} % Misspelling. 
\newcommand{\dejavu}           {d\'ej\`a-vu\xspace}
\newcommand{\dejavus}          {\dejavu{}s\xspace}
\newcommand{\deajvus}          {\dejavus} % Misspelling. 

% Facade.
\newcommand{\facade}           {fa\c cade\xspace}

% Faux pas. 
\newcommand{\fauxpas}          {faux pas\xspace}

% Melee. 
\newcommand{\melee}            {m\^el\'ee\xspace}

% Noblesse. 
\newcommand{\noblesse}         {noblesse\xspace}

\begin{comment}
\subsection{People}
\end{comment}

% Francois (French name).
\newcommand{\Francois}         {Fran\c cois\xspace}

% Phedre, the heroine from Jacqueline Carey's Kushiel's Legacy. 
\newcommand{\PhedreNoDelaunay} {\Phedre n\'o Delaunay\xspace}
\newcommand{\Phedre}           {Ph\`edre\xspace}



\begin{comment}
\section{German}
\end{comment}

% Über.
\newcommand{\Uber}             {\"Uber\xspace}
\newcommand{\uber}             {\"uber\xspace}
\newcommand{\Ubermensch}       {\Uber{}mensch\xspace}

% Götterdämmerung.
\newcommand{\Gotterdammerung}  {G\"otterd\"ammerung\xspace}



\begin{comment}
\section{Greek}
\end{comment}

% Hetoimasia.
\newcommand{\Hetoimasia}       {H\'eto\"imasia\xspace}

% Kenose.
\newcommand{\Kenose}           {K\'en\^ose\xspace}
\newcommand{\kenose}           {K\'en\^ose\xspace}



\begin{comment}
\section{Hebrew}
\end{comment}

\newcommand{\Cabbalistic}      {Cabbalistic\xspace}
\newcommand{\Cabbalists}       {Cabbalists\xspace}
\newcommand{\Cabbalist}        {Cabbalist\xspace}
\newcommand{\Cabbalah}         {Cabbalah\xspace}



\begin{comment}
\section{Japanese}
\end{comment}


\begin{comment}
\subsection{Anime}
\end{comment}

% Gankutsuou. 
\newcommand{\Gankutsuo}        {Gankutsu\=o{}\xspace}

% Ikari Gendou from Neon Genesis Evangelion.
\newcommand{\Gendou}           {Gend\=o{}\xspace}

% Jimushi Juubei from Basilisk. 
\newcommand{\JimushiJuubei}    {Jimushi J\=ubei{}\xspace}

% Urotsukidouji. 
\newcommand{\Urotsukidouji}    {Urotsukid\=oji{}\xspace}
\newcommand{\Juujinkai}        {J\=ujinkai{}\xspace}

\begin{comment}
\subsection{Tokusatsu}
\end{comment}

% Kamen Rider Ryuki.
\newcommand{\Ryuki}            {Ry\=uki{}\xspace}

% Ryuuoun from Go Go Sentai Boukenger, and his Jaryuu and Daijakuryuu.
\newcommand{\Ryuuoun}          {Ry\=u\=on{}\xspace}
\newcommand{\Jaryuu}           {Jary\=u{}\xspace}
\newcommand{\Daijakuryuu}      {Dai-Jakury\=u{}\xspace}

% Juuken Sentai Gekiranger
\newcommand{\JuukenSentaiGekiranger} {{\em J\=uken Sentai Gekiranger}{}\xspace}
\newcommand{\juukensentaigekiranger} {\JuukenSentaiGekiranger{}\xspace}

% RinJyuKen.
\newcommand{\Rinjuuken}        {Rinj\=uken{}\xspace}

\begin{comment}
\subsection{People}
\end{comment}

\newcommand{\Gou} {G\=o{}\xspace}

\begin{comment}
\subsection{Other}
\end{comment}

% Bishounen.
\newcommand{\Bishounen} {Bish\=onen{}\xspace}
\newcommand{\bishounen} {bish\=onen{}\xspace}

% Zanbatou, a very large sword. 
\newcommand{\zanbatou}  {zanbat\=o{}\xspace}

































\begin{comment}
\chapter{Environments}
\end{comment}

% Blurb environment. Used to typeset back-side blurb.
\newenvironment{blurb}{\begin{quote}}{\end{quote}}

% Changes environment. Used to list planned changes and revisions. 
\newenvironment{changes}{}{}
\newcommand{\changesitem}[2][] {
  \ifthenelse
    {\equal{#1}{}}
    {\subsubsection{#2}}
    {\subsubsection[#1]{#2}}}

\newenvironment{subchanges}{
  \begin{description}
}{
  \end{description}
}
\newcommand{\subchangesitem}[1] {\item[#1:]}

% Poetry environment. Used to quote bits of poetry, such as ``Wanderers in Darkness''.
\newenvironment{poetry}{%
  \begin{flushleft}
  \begin{verse}%
%   \begin{obeylines}
    \sl%
}{%
%   \end{obeylines}
  \end{verse}%
  \end{flushleft}%
}

% Diary environment. Used when Charcoal reads Tantor's diary. 
\newenvironment{diary}{%
  \begin{quotation}\noindent\sl%
}{%
  \end{quotation}\noindent}

% Dream environment. Signifies a dream sequence. 
\newenvironment{dream}{\em}{}

% An alternate kind of comment environment. 
% It is useful because "garbage" and "comment" can be nested inside one another. 
\excludecomment{garbage}

% Prose environment. Used in the notes to present a piece of action that is meant to appear more-or-less verbatim in the final prose.
\newenvironment{prose}{\begin{quote}}{\end{quote}}










\begin{comment}
\section{List things}
\end{comment}

% Command to typeset a list of Archons
\newenvironment{sephlist}{\begin{description}}{\end{description}}

% Command to typeset the entry of a Archons. 
\newcommand*{\seph}[1]{\item[#1:]}










\begin{comment}
\section{Dramatis Personae}
\end{comment}
\newenvironment{dramatispersonae}{%
  \begin{flushleft}%
  \begin{itemize}%
  \small
}{%
  \end{itemize}%
  \end{flushleft}%
}
\newenvironment{subdramatispersonae}{%
  \begin{itemize}%
}{%
  \end{itemize}%
}

% How to typeset an item. 
\newcommand*{\basicdramitem}    [1]{\item #1}
\newcommand*{\basicdramsubitem} [1]{\item #1}

% A "Dramatis item" has three mandatory arguments: 
% Name, race and gender. IN THAT ORDER.
% There are also two optional arguments: %
% Label and title.
% Will show birth year.
% ``dramdead'' also shows death year.
\newcommandtwoopt*{\dramitem} [5][][]{%
  \basicdramitem{\character{#1}{#2}{#3}{#4}{#5}}}
\newcommandtwoopt*{\dramdead}      [5][][]{%
  \basicdramitem{\chardead{#1}{#2}{#3}{#4}{#5}}}
\newcommandtwoopt*{\dramitemnoyear} [5][][]{%
  \basicdramitem{\characternoyear{#1}{#2}{#3}{#4}{#5}}}



\begin{comment}
\subsection{Characters}
\end{comment}

% Symbols for female and male, used with \gitemchar command
\newcommand{\female} {\Venus}
\newcommand{\male}   {\Mars}
\newcommand{\neuter} {$\emptyset$}

% Check if a character label #1 has a birth year associated with it.
% If birth year exists, do #2. Else do #3.
\newcommand*{\ifborn}[3]{\ifempty{\csuse{NV #1 birth}}{#3}{#2}}

% Gives a character's birth and death years, or just the birth year if still alive. 
% Argument must be a name label that corresponds to counters (see Timeline section). 
\newcommand*{\birthtodeath} [1]{\vaimonyear{#1 birth}{} to \yic{#1 death}}
\newcommand*{\birthtonow}   [1]{born \yic{#1 birth}}

% Typeset years of a live or dead character. 
% Takes two arguments: Label, status. 
% If "status" is "live" then birth year is shown. 
% If "status" is "dead" then birth and death years are shown. 
% If the character label has no birth year, then do nothing. 
\newcommand*{\typesetyears}[2] {%
  \ifborn{#1}{, %
    \ifthenelse%
      {\equal{#2}{dead}}%
      {\birthtodeath{#1}}%
      {\birthtonow{#1}}%
  }{}%
}

% Typeset years of a live or dead character. 
% Takes three arguments: Label, name, status. 
% The name is used to look up years, unless "label" is non-empty, in which case the label is used instead. 
\newcommand*{\typesetyearslabel}[3] {%
  \ifempty{#1}{\typesetyears{#2}{#3}}{\typesetyears{#1}{#3}}%
}

% A character. 
% Arguments are label, title, name, race and gender. In that order. 
\newcommand*{\character}    [5]{%
  \typesetcharacter{#1}{#2}{#3}%
    {\characterparenthesis{#1}{#3}{#4}{#5}{live}}%
}
% A dead character (shows birth and death years). 
\newcommand*{\chardead}     [5]{%
  \typesetcharacter{#1}{#2}{#3}
    {\characterparenthesis{#1}{#3}{#4}{#5}{dead}}%
}
\newcommand*{\characternoyear}    [5]{%
  \typesetcharacter{#1}{#2}{#3}%
    {\characterparenthesisnoyear{#1}{#3}{#4}{#5}{live}}%
}
    
% Typeset the paranthetical part of a character.
% Takes five arguments:
% Label, name, race, gender, status (live/dead). 
\newcommand*{\characterparenthesis}[5]{%
    {({#3} {#4}\typesetyearslabel{#1}{#2}{#5})}}
\newcommand*{\characterparenthesisnoyear}[5]{%
    {({#3} {#4})}}

% How to typeset a character, given label, title, name and parenthesis. 
\newcommand*{\typesetcharacter} [4]{%
  \def\typesetindex{%
    \ifempty{#2}%
      {\index{#3}}%
      {\index{#2}}%
  }%
  \ifthenelse%
    {\equal{#1}{}}%
    {\textbf{\maybehr{#3}{#3}{}} \textnormal{#4{}}\typesetindex}%
    {\textbf{\maybehr{#1}{#3}{}} \textnormal{#4{}}\typesetindex}%
}








\begin{comment}
\section{Glossary}
\end{comment}



\newenvironment{gloss}{\begin{description}}{\end{description}}
\newenvironment{subgloss}{\begin{gloss}}{\end{gloss}}



% A glossary item with title and description. 
\newcommand*{\glositem} [1]{\item[#1:]}



% Entry for a character. 
% Three mandatory arguments: Name, race and gender. 
% There are also two optional arguments: "Status" and label. 
% If "status" is "live" then birth year is shown. 
% If "status" is "dead" then birth and death years are shown. 
% The name is used to look up years, unless "label" is non-empty, in which case the label is used instead. 
\newcommandtwoopt*{\gitemcharacter}   [5][][]{%
  \ifempty{#1}%
    {\glositem{#3 \textnormal{({#4{}} {#5{}})}}\index{#3}}
    {%
      \glositem{%
        #3 \textnormal{{\characterparenthesis{#2}{#3}{#4}{#5}{#1}}}
      }%
      \index{#3}%
    }%
}



% Glossary item with the prefix "the". 
\newcommand*{\gitemthe}[1]{\glositem{The #1}\index{#1}}

% Short form for a glossary item.
% There are two optional arguments: 
% Plural form and alternate plural form. 
\newcommandtwoopt*{\gitem} [3][][]{%
  \ifthenelse
    {\equal{#1}{}}
    {\glositem{#3}\index{#3}} % If there is no plural form given.
    {\ifthenelse
      {\equal{#2}{}}
      {% If there is one plural form given.
        \item[\textbf{#3{}} 
          \textnormal{(plural \textbf{#1{}})}:]%
        \index{#3 (plural #1)}
      }
      {% If there are two plural forms given. 
        \item[\textbf{#3{}} 
          \textnormal{(plural \textbf{#1{}} or \textbf{#2{}})}:]%
        \index{#3 (plural #1 or #2)}
      }}%
}

% Short form for a glossary item.
% There are two optional arguments: 
% Plural form and alternate plural form. 
\newcommandtwoopt*{\gitemnoindex} [3][][]{%
  \ifthenelse
    {\equal{#1}{}}
    {\glositem{#3}}
    {\ifthenelse
      {\equal{#2}{}}
      {%
        \item[\textbf{#3{}} 
          \textnormal{(plural \textbf{#1{}})}:]%
      }
      {%
        \item[\textbf{#3{}} 
          \textnormal{(plural \textbf{#1{}} or \textbf{#2{}})}:]%
      }}%
}



% A glossary item with link. 
\newcommand*{\gitemlink} [2][]{%
  \ifempty{#1}
    {\glositem{\hs{#2}}}%
    {\glositem{\hr{#1}{#2}}}%
}



\begin{comment}
\subsection{Tags}
\end{comment}

%Empty lines. 
\newcommand{\emptylines} {

}
\newcommand{\glossarytag} [2]{\emptylines\textbf{#1{}} #2}
% \newcommand{\also}[1]{\emptylines\emph{See also:} #1.}
% \newcommand{\also}[1]{\paragraph{See also:} #1.}
\newcommand{\appearance} [1]{\glossarytag{Appearance:}{#1}}
\newcommand{\meta} [1]{\glossarytag{Meta:}{#1}}
\newcommand{\also} [1]{\glossarytag{See also:}{#1}}
\newcommand{\seee} [1]{See \emph{#1}.}











\begin{comment}
\section{Pronunciation guide}
\end{comment}


\newenvironment{pronunciationenvironment}[1]{%
  \begin{flushleft}%
  \begin{multicols}{2}[#1]%
  \begin{description}%
  \begingroup
  \small
}{%
  \endgroup
  \end{description}%
  \end{multicols}%
  \end{flushleft}%
}

% Pitem with label - in case the main word has strange symbols in it.
% So far it does nothing, just the same as a regular pitem. 
\newcommand*{\pitemlabel} [4]{\pitem{#1}{#2}{#3}}
\newcommand*{\piteml}     [4]{\pitemlabel{#1}{#2}{#3}{#4}}

% Pronunciation item, i.e., an entry in the pronunciation table.
% \newcommand*{\pitem} [3]{\item{#1:} \txipa{#2} (#3) \index{#1}}
\newcommand*{\pitem} [3]{\item{\bf{#1:}} #3\index{#1}}

% \newcommand*{\txipa}  [1]{\textipa{#1}}
\newcommand*{\txipa}  [1]{BLANK}

\newcommand{\longv} {\textlengthmark}

% Pronunciation example. Used to mark up a well-known word that is used as an example.
\newcommand*{\pex} [1]{\quo{#1}}
% Letter. Used to mark a letter or group of letters used for reference/comparison.
\newcommand*{\lett} [1]{#1}











\begin{comment}
\section{Timeline}
\end{comment}

\newenvironment{epoch}[1]{%
  \subsubsection{#1}\begin{tabular}{rp{10cm}}%
}{%
  \end{tabular}}

\newcommand{\bepo} {\begin{epoch}}
\newcommand{\eepo} {\end{epoch}}

% \dragonyear is `year DS', as in, a year number in the Draconian Supremacy calendar (reckoned from when Tiamat contacted the Xzai-Shann)
\newcommand*{\dragonyear} [1]{\difference{#1}{XS}}
\newcommand*{\yds}        [1]{{\dragonyear{#1}} \DS{}}

% \resphanyear is `year BD', as in, a year number in the Black Dawn calendar (reckoned from the Murder of the Dawn)
\newcommand*{\resphanyear} [1]{\difference{#1}{Murder of the Dawn}}
\newcommand*{\ybd}         [1]{{\resphanyear{#1}} \BD{}}

% \vaimonyear is `year \IC{}', as in, a year number in the Vaimon Calendar (reckoned from the founding of the Vaimon Caliphate)
\newcommand*{\vaimonyear} [1]{\difference{#1}{VC}}
\newcommand*{\yic}        [1]{{\vaimonyear{#1}} \IC{}}

% \beforevaimonyear is \vaimonyear but with the minus removed. 
\newcommand*{\beforevaimonyear}[1]{\difference{VC}{#1}}

% An event, with a year number (counter label only) and a description. 
\newcommand{\event}  [2]{\yic{#1} & #2 \\}
% Last event in an epoch. The only difference is that it lacks a line break at the end. 
\newcommand{\eventl} [2]{\yic{#1} & #2}















\begin{comment}
\chapter{Quote things}
\end{comment}



\begin{comment}
\section{Names of specific works}
\end{comment}

% Malazan Book of the Fallen.
\newcommand{\SEMalazan} {%
  \authorseries{Steven Erikson}{Malazan Book of the Fallen}}
% A specific Malazan Book. Takes two arguments: Title and number in the series.
\newcommand*{\MalazanBook}     [2]{\authorbook{Steven Erikson}{#1}}
\newcommand{\MalazanReapersGale} {\MalazanBook{Reaper's Gale}      {7}}
\newcommand{\SEDustofDreams}     {\MalazanBook{Dust of Dreams}     {9}}
\newcommand{\SETolltheHounds}    {\MalazanBook{Toll the Hounds}    {8}}
\newcommand{\SEReapersGale}      {\MalazanBook{Reaper's Gale}      {7}}
\newcommand{\SEBonehunters}      {\MalazanBook{The Bonehunters}    {6}}
\newcommand{\SEMidnightTides}    {\MalazanBook{Midnight Tides}     {5}}
\newcommand{\SEHouseofChains}    {\MalazanBook{House of Chains}    {4}}
\newcommand{\SEMemoriesofIce}    {\MalazanBook{Memories of Ice}    {3}}
\newcommand{\SEDeadhouseGates}   {\MalazanBook{Deadhouse Gates}    {2}}
\newcommand{\SEGardensoftheMoon} {\MalazanBook{Gardens of the Moon}{1}}

% Francois Marcela Froideval - Chroniques de la Lune Noire.
\newcommand{\FMFroideval}      {\Francois{} Marcela Froideval}
\newcommand{\LuneNoire}        {Chroniques de la Lune Noire}
\newcommand*{\FLuneNoireVol} [1]{%
  \authorseries{\FMFroideval}{\LuneNoire{} (Volume #1)}}
\newcommand{\FLuneNoire}{\authorseries{\FMFroideval}{\LuneNoire{}}}

% Sailor Nothing, complete with link.
\newcommand{\SailorNothing} {%
  \href{http://www.pixelscapes.com/sailornothing/}{Sailor Nothing}}

% Duana the Necrobabe (writes dark poetry and stuff). 
\newcommand{\Duana} {\href{http://www.necrobabes.org/duana/}{Duana}}

% Robert Chambers and The King in Yellow.
\newcommand{\RWChambers}   {Robert W. Chambers}
\newcommand{\KinginYellow} {The King in Yellow}
\newcommand{\RWCTKIY}      {\authorseries{\RWChambers}{\KinginYellow}}

% David Icke and The Biggest Secret.
\newcommand{\DIBiggestSecret} {%
  \authorbook{David Icke}{The Biggest Secret}}

% Poppy Z. Brite.
\newcommand{\PZBrite} {Poppy Z. Brite}

% Howard Phillips Lovecraft.
\newcommand{\HPLovecraft} {H.P. Lovecraft}

\newcommand{\simonnecronomicon} {Simon's \emph{Necronomicon}}



\begin{comment}
\section{Quotes from specific works}
\end{comment}

% Bal-Sagoth lyrics.
\newcommand{\lyricsbalsagoth}   [2]{\lyricsbs{Bal-Sagoth}{#1}{#2}}

% Limbonic Art lyrics.
\newcommand{\lyricslimbonicart} [2]{\lyricsbs{Limbonic Art}{#1}{#2}}
% Cite author-book. 
\newcommand{\citelimbonicart}   [3]{%
  \citebandsong{LimbonicArt:#1}{Limbonic Art}{#2}{#3}}

% Dimmu Borgir lyrics.
\newcommand{\lyricsdimmuborgir} [2]{\lyricsbs{Dimmu Borgir}{#1}{#2}}

% Duana poem.
\newcommand{\lyricsduana} [3]{%
  \lyricsbs{\Duana}{%
    \href{http://www.necrobabes.org/duana/#1.html}{#2}}{#3}}

% 'Lyrics' from Chroniques de la Lune Noire
\newcommand{\lyricsflnv}  [2]{\lyricstitle{\FLuneNoireVol{#1}}{#2}}

% Quotes from the Bible. 
\newcommand{\lyricsbible} [2]{\lyricstitle{The Bible (#1)}{#2}}

% xkcd webcomic. Takes two arguments: Strip number and text. 
\newcommand{\lyricsxkcd}  [2]{%
  \lyricstitle{\href{http://xkcd.com/#1/}{xkcd}}
              {#2}}

\newcommand{\lyricslimyaael}  [2]{%
  \lyricstitle{%
    \href{http://limyaael.insanejournal.com/#1.html}{Limyaael}%
  }{#2}
}

% Cite Wikipedia. 
% Arguments are URL, article title and text. 
\newcommand{\lyricswikipedia} [3]{%
  \lyricstitle{%
    \href{http://en.wikipedia.org/wiki/#1}{Wikipedia: #2}%
  }{#3}
}



\begin{comment}
\section{Types of quotation tags}
\end{comment}

% Artist-title.
\newcommand*{\artisttitle} [2]{#1 -- {\em #2}}

% Author-book. Used when referring to books or series.
\newcommand*{\authorseries} [2]{\artisttitle{#1}{#2}}
\newcommand*{\authorbook}   [2]{\artisttitle{#1}{#2}}

% Band-title. Used when referring to songs or pieces of music.
\newcommand*{\bandsong}  [2]{\artisttitle{#1}{#2}}
\newcommand*{\bandalbum} [2]{\artisttitle{#1}{#2}}

% Movie.
\newcommand*{\movie} [1]{{\em #1}}

% Lyrics environment. Used when quoting songs, books or other things.
\newcommand{\lyricstitle} [2]{%
    \paragraph{#1}
  \nopagebreak
  \lyrics{#2}}
\newcommand{\lyricsauthorbookpage}[4]{\lyricstitle{\authorbook{#1}{#2} p.#3}{#4}}
%\newcommand{\lyricsbs}[3]{\paragraph{\artisttitle{#1}{#2}}\lyrics{#3}}
\newcommand{\lyricsbs} [3]{\lyricstitle{\artisttitle{#1}{#2}}{#3}}
\newcommand{\lyrics}   [1]{%
  \begin{quote}%
    \begin{flushleft}%
      {\sl\small #1}%
    \end{flushleft}%
  \end{quote}%
}

% Citation with title. 
% Takes three arguments: Key, title, text. 
\newcommand{\citetitle} [4][]{%
  \lyricstitle{#3 \cite{#2}}{#4}}

% Cite movie. 
\newcommand{\citemovie} [4][]{%
  \citetitle[#1]{Movie:#2}{#3}{#4}}

% Cite band-song. 
% Takes four arguments: Key, artist, title, text. 
\newcommand{\citebandsong} [5][]{%
  \citetitle[#1]{#2}{\artisttitle{#3}{#4}}{#5}}
% Cite author-book. 
\newcommand{\citeauthorbook} [5][]{\citebandsong[#1]{#2}{#3}{#4}{#5}}














\begin{comment}
\chapter{Books}
\end{comment}


% The titles of the various books in my series. 



\begin{comment}
\section{Dominators of \Miith{}}
\end{comment}
\newcommand{\DominatorsofMithEmph} {\emph{Genesis of \Miith}\xspace}
\newcommand{\DominatorsofMith}  {Genesis of \Miith}
% \newcommand{\DominatorsI}{Chaos Empower}
% \newcommand{\DominatorsII}{Sons of Darkness Rise}
% \newcommand{\DominatorsIII}{The Lie Sublime}
\newcommand{\FirstbanewarBook}  {\Draconian{} Ascendancy\xspace}
\newcommand{\AvatarofEntropy}   {Entropy's Avatar\xspace}
\newcommand{\ThanatzilBook}     {\AvatarofEntropy}
\newcommand{\MerkyrahBook}      {The Murder of the Dawn\xspace}
\newcommand{\SathariahBook}     {Angels of Blood\xspace}
\newcommand{\SecondbanewarBook} {Sons of Darkness Rise\xspace}
\newcommand{\ResphanWarsBook}   {The \Resphanwars}
\newcommand{\TheLieSublimeBook} {The Lie Sublime\xspace}

\begin{comment}
\section{Archons of \Miith{}}
\end{comment}

\newcommand{\ArchonsofMith}         {Archons of \Miith}
\newcommand{\SilquaBookEmph}        {\emph{Bringer of the Light}\xspace}
\newcommand{\SilquaBook}            {Bringer of the Light\xspace}
\newcommand{\HundredScourgesBook}   {Bringer of the Darkness\xspace}

\begin{comment}
\section{Sentinels of \Miith{}}
\end{comment}
\newcommand{\SentinelsofMiithEmph}      {\emph{\SentinelsofMith{}}\xspace}
\newcommand{\SentinelsofMithEmph}       {\emph{\SentinelsofMith{}}\xspace}
\newcommand{\SentinelsofMiith}          {\SentinelsofMith}
\newcommand{\SentinelsofMith}           {Sentinels of the Eschaton\xspace}
\newcommand{\CarzainPrequelBookEmph}    {\emph{The Dreaming Predator}\xspace}
\newcommand{\CarzainPrequelBook}        {The Dreaming Predator\xspace}
\newcommand{\TwilightAngelRememberEmph} {\emph{Twilight Angel, Remember}\xspace}
\newcommand{\TwilightAngelRemember}     {Twilight Angel, Remember\xspace}
\newcommand{\RungerWarBook}             {\TwilightAngelRemember}
\newcommand{\TheKenosis}                {The \Kenosis}
\newcommand{\CarzainWithRedcorBook}     {\TheKenosis}
\newcommand{\HaskelekBook}              {\TheKenosis}
\newcommand{\LicaBook}                  {\CarzainWithRedcorBook}
\newcommand{\GeicaBook}                 {The Geican Awakening\xspace}
\newcommand{\ResurrectionBook}          {The Dark Queen's Resurrection\xspace}
\newcommand{\RamielsAwakeningBook}      {Theosis: My Name Is Legion\xspace}
\newcommand{\RamielAwakens}             {\RamielsAwakeningBook}
\newcommand{\SentinelsFinalBookEmph}    {\emph{The Eschaton}\xspace}
\newcommand{\SentinelsFinalBook}        {The Eschaton\xspace}
\newcommand{\ThirdBanewarBook}          {\SentinelsFinalBook}

% Text font used when typesetting a book title inside the body of the text. 
\newcommand*{\booktitle}[1] {\emph{\quo{#1}}}























\begin{comment}
\chapter{Spells}
\end{comment}


\newcommand*{\spelldra}[1]{\paragraph{#1}{\Draconic{} spell #1}{Spells!#1 (Draconic)}} 
\newcommand*{\spellim}[1]{\paragraph{#1}{Imetric spell #1}{Spells!#1 (Imetric)}} 
\newcommand*{\spellime}[1]{\spellim{#1}}
\newcommand*{\spellris}[1]{\paragraph{#1}{Rissitic spell #1}{Spells!#1 (Rissitic)}} 
\newcommand*{\spellsha}[1]{\paragraph{#1}{Shaman spell #1}{Spells!#1 (Shamanistic)}} 
\newcommand*{\spellvai}[1]{\paragraph{#1}{Vaimon spell #1}{Spells!#1 (Vaimon)}} 

% Rissitic spell with special label (if the name contains special characters)
\newcommand*{\spellrisl}[2]{\paragraphhh{#1}{Rissitic spell #2}{Spells!#1 (Rissitic)}}

% Section reference to spell.
\newcommand*{\refspelldra}[1]{\ref{\Draconic{} spell #1}}
\newcommand*{\refspellim}[1]{\ref{Imetric spell #1}}
\newcommand*{\refspellris}[1]{\ref{Rissitic spell #1}}
\newcommand*{\refspellsha}[1]{\ref{Shaman spell #1}}
\newcommand*{\refspellvai}[1]{\ref{Vaimon spell #1}}
% Page reference to spell.
\newcommand*{\pagerefspelldra}[1]{\pageref{\Draconic{} spell #1}}
\newcommand*{\pagerefspellim}[1]{\pageref{Imetric spell #1}}
\newcommand*{\pagerefspellris}[1]{\pageref{Rissitic spell #1}}
\newcommand*{\pagerefspellsha}[1]{\pageref{Shaman spell #1}}
\newcommand*{\pagerefspellvai}[1]{\pageref{Vaimon spell #1}}















\begin{comment}
\chapter{General}
\end{comment}



\newcommand{\prikker}{\ldots{}\xspace}













\begin{comment}
\section{Text fonts}
\end{comment}




\begin{comment}
\subsection{Hyper-references}
\end{comment}

% A hyperlink target. 
\newcommand*{\target}      [1]{\label{#1}\hypertarget{#1}{}}
% \newcommand*{\target}      [1]{\label{#1}}

% Hyper-reference.
\newcommand*{\hr}[2]{\hyperref[#1]{#2{}}}
% \newcommand*{\hr} [2]{\hyperlink{#1}{#2}}
\newcommand*{\hs} [1]{\hr{#1}{#1}}

% `Maybe' hyperlinks. Link if label exists, otherwise just print the text. 
\newcommand*{\maybehr}[2]{\eachlabelcase{{#1}{\hr{#1}{#2}}{#2{}}}}
\newcommand*{\maybehs}[1]{\maybehr{#1}{#1}}

% Glossary reference. Links to the glossary and adds an index entry. 
% Has two optional arguments: Label and index entry. 
\newcommandtwoopt*{\gref}[3][][]{%
  \ifthenelse
    {\equal{#2}{}}
    {\ifthenelse
      {\equal{#1}{}}
      {\index{#3}\hr{#3}{#3}}
      {\index{#3}\hr{#1}{#3}}}
    {\ifthenelse
      {\equal{#1}{}}
      {\index{#2}\hr{#3}{#3}}
      {\index{#2}\hr{#1}{#3}}}
}




\begin{comment}
\subsection{Speech and stuff}
\end{comment}

% The following commands are used to typeset special kinds of text. 

% Encase text in double quotes. 
\newcommand{\dquo} [1]{\mbox{}``\mbox{}#1\mbox{}''\mbox{}}
% Encase text in single quotes. 
\newcommand{\squo} [1]{\mbox{}`#1'\mbox{}}
% Quotes
\newcommand{\quo} [1]{\squo{#1}}
% Guillemets (angular quotes).
% \newcommand{\gui}[1]{\fg{}#1\og{}} 
\newcommand{\gui} [1]{\guillemotright{}#1\guillemotleft{}} 


% Nested talk, ie., a quote nested within a quote: `So I said: `foo''
\newcommand{\subtalk} [1]{\squo{#1}}
\newcommand{\subta}   [1]{\subtalk{#1}}

% Special, foreign words, such as `Nieur'. 
\newcommand{\foreign} [1]{\emph{#1}}
% A complete spoken line in a foreign tongue.
\newcommand{\tafo} [1]{\talk{\foreign{#1}}}
\newcommand{\word} [1]{\foreign{#1}}
% A sentence in a foreign tongue. 
\newcommand{\tongue} [1]{\foreign{#1}}
% Words in Draconic Kingstongue.
\newcommand{\kingstongue} [1]{\foreign{#1}}
\newcommand{\kingst} [1]{\kingstongue{#1}}

% Regular talk. 
\newcommand{\talk} [1]{\dquo{#1}}
\newcommand{\ta}   [1]{\talk{#1}}

% A longer spell in Draconic Kingstongue.
\newcommand{\draconicspell} [1]{\begin{verse}{\sl #1}\end{verse}}

% Something a character thinks to himself. 
\newcommand{\thought} [1]{{\it#1}}
\newcommand{\tho}     [1]{\thought{#1}}
% A telepathic transmission. 
%\newcommand{\telepathy}[1]{\dquo{\textsc{#1}}}
\newcommand{\telepathy} [1]{{\it \gui{#1}}}
\newcommand{\tele}      [1]{\telepathy{#1}}
% Something that is subcommunicated, conveyed through body language
\newcommand{\bodyl}     [1]{{\em {\dquo{#1}}}}
% Hypothetical talk, something that might be spoken but isn't
\newcommand{\hypotalk}  [1]{\bodyl{#1}}
\newcommand{\hypota}    [1]{\hypotalk{#1}}

% Shouting.
\newcommand{\shout} [1]{{\sc #1}}



\begin{comment}
\subsection{Characters}
\end{comment}

% Customized speech for various characters. 

\newcommand{\vizicar}       [1]{\telepathy{#1}}
\newcommand{\carzain}       [1]{\telepathy{#1}}
\newcommand{\daggerrain}    [1]{{\sc \gui{#1}}}
\newcommand{\secherdamon}   [1]{{\sc \gui{#1}}}
\newcommand{\ishna}         [1]{{\sc \gui{#1}}}



\begin{comment}
\subsection{Real life things}
\end{comment}

% The name of a ship... might be italicized
\newcommand*{\shipname}[1]{\emph{#1}}
% The Latin name of an animal species.
\newcommand*{\latinname}[1]{\emph{#1}}

% Trope: Link to the TV Tropes Wiki.
\newcommand*{\trope}[2]{\href{http://tvtropes.org/pmwiki/pmwiki.php/Main/#1}{#2}}

\begin{comment}
\subsubsection{Tropes}
\end{comment}

\newcommand{\XanatosGambit}{\trope{XanatosGambit}{Xanatos Gambit}}














\begin{comment}
\section{Mathematics}
\end{comment}

\newcommand{\Rp}{\mathbb{R}_+}                % Positive real numbers
\newcommand{\Rm}{\mathbb{R}_-}                % Negative real numbers
\newcommand{\R}{\mathbb{R}}                   % Real numbers
\newcommand{\Q}{\mathbb{Q}}                   % Rational numbers
\newcommand{\Z}{\mathbb{Z}}                   % Integers
\newcommand{\N}{\mathbb{N}}                   % Natural numbers
\newcommand{\ph}{\varphi}                     % Phi
\newcommand{\rh}{\varrho}                     % Rho
\newcommand{\e}{\varepsilon}                  % Epsilon
\newcommand{\fracs}[2]{\textstyle{\frac{#1}{#2}}} % Small fraction
\newcommand{\bimp}{\Leftrightarrow}           % Bi-implication
\newcommand{\imp}{\Rightarrow}                % Implication (`p only if q' or `p implies q')
\newcommand{\pmi}{\Leftarrow}                 % Implication right to left (`p if q')
\newcommand{\x}{\times}                       % Cartesian product
\newcommand{\ri}{\right}                      % Right delimiter
\newcommand{\lf}{\left}                       % Left delimiter
\newcommand{\rpar}{\right)}                   % Right parenthesis
\newcommand{\lpar}{\left(}                    % Left parenthesis
\newcommand{\rtub}{\right\}}                  % Right tuborg }
\newcommand{\ltub}{\left\{}                   % Left tuborg {
\newcommand{\abs}[1]{\left| #1 \right|}       % Absolute/numerical value
\newcommand{\set}[1]{\ltub #1 \rtub}          % Set
\newcommand{\supns}[1]{\sup\set{\abs{#1}}}    % Numerical supremum of a set
\newcommand{\supnsub}[2]{\sup\set{\abs{#1}| \ x \in #2}}
                                              % Numerical supremum of a subset
\newcommand{\bolle}{\circ}                    % Bolle - funktionssammensaetning
\newcommand{\ZMP}[1]{(\mathbb{Z}/#1)^\ast}    % Group of prime remainder classes modulo n
\newcommand{\alter}{(-1)^n}                   % Alternating sequence
\newcommand{\alterp}{(-1)^{n+1}}              % Alternating sequence n+1
\newcommand{\intoi}{\int_1^\infty}            % Integral from one to infty
\newcommand{\intni}{\int_n^\infty}            % Integral from n to infty
\newcommand{\intzi}{\int_0^\infty}            % Integral from 0 to infty
\newcommand{\intzo}{\int_0^1}                 % Integral from 0 to 1
\newcommand{\intoa}{\int_1^a}                 % Integral from 1 to a
\newcommand{\intab}{\int_a^b}                 % Integral from a to b
\newcommand{\sumoi}{\sum_{n=1}^\infty}        % Sum for n = 1...\infty
\newcommand{\sumzi}{\sum_{n=0}^\infty}        % Sum for n = 0...\infty
\newcommand{\sumon}{\sum_{n=1}^N}             % Sum for n = 1...N
\newcommand{\sumzn}{\sum_{n=0}^N}             % Sum for n = 0...N
\newcommand{\floor}[1]{\lfloor #1 \rfloor}    % N rounded down
\newcommand{\ceil}[1]{\lceil #1 \rceil}       % N rounded up
\newcommand{\hfloor}[1]{\floor{#1/2}}         % N/2 rounded down
\newcommand{\hceil}[1]{\ceil{#1/2}}           % N/2 rounded up
\newcommand{\hak}[1]{\left\langle #1 \right\rangle} % Cyclical group <g>
\newcommand{\half}{\frac{1}{2}}               % One half: 1/2 - big
\newcommand{\halfs}{\fracs{1}{2}}             % One half: 1/2 - small
\newcommand{\third}{\frac{1}{3}}              % One third: 1/3 - big
\newcommand{\thirds}{\fracs{1}{3}}            % One third: 1/3 - small
\newcommand{\quart}{\frac{1}{4}}              % One quarter: 1/4 - big
\newcommand{\quarts}{\fracs{1}{4}}            % One quarter: 1/4 - small
\newcommand{\tquart}{\frac{3}{4}}             % Three quarters: 3/4 - big
\newcommand{\tquarter}{\frac{3}{4}}           % Three quarters: 3/4 - big
\newcommand{\tquarts}{\fracs{3}{4}}           % Three quarters: 3/4 - small
\newcommand{\tquarters}{\fracs{3}{4}}         % Three quarters: 3/4 - small
\newcommand{\nth}{\frac{1}{n}}                % One n'th: 1/n - big
\newcommand{\nths}{\fracs{1}{n}}              % One n'th: 1/n - small
\newcommand{\pihalf}{\frac{\pi}{2}}           % Pi half: Pi/2 - big
\newcommand{\pihalfs}{\fracs{\pi}{2}}         % Pi half: Pi/2 - small
\newcommand{\toi}{\to \infty}                 % (something) -> \infty
\newcommand{\limi}[1]{\lim_{#1 \to \infty}}   % lim for #1 going to \infty
\newcommand{\limmi}[1]{\lim_{#1 \to -\infty}} % lim for #1 going to minus \infty
\newcommand{\limz}[1]{\lim_{#1 \to 0}}        % lim for #1 going to 0
\newcommand{\limni}{\limi{n}}                 % lim for n going to \infty
\newcommand{\limnmi}{\limmi{n}}               % lim for n going to minus \infty
\newcommand{\for}{\quad \textrm{for} \quad }  % Math mode: `for ...'
\newcommand{\forni}{\for n\to\infty  }        % Math mode: 'for n -> infty'
\newcommand{\comma}{\quad \textrm{,} \quad }  % Math mode: (formula) , (text)
\newcommand{\Yps}{\Upsilon}
\newcommand{\YPS}{\Upsilon}
\newcommand{\gaffel}[1]{\left\{ \begin{array}{ll} #1 \end{array} \right.}















\begin{comment}
\section{Chapters and sections}
\end{comment}

\newcommand*{\bookchapter}[1]{\chapter{#1}}

\newcommand{\placestamp} [1]{\typesetstamp{#1}}
\newcommand{\timestamp}  [1]{\typesetstamp{#1}}
\newcommand{\stamp}      [2]{\typesetstamp{#1 \\ \nopagebreak #2}}
\newcommand{\diarystamp} [1]{{\em\bf #1}}















\begin{comment}
\section{Command macros}
\end{comment}

% If #1 is the empty string then do #2, else do #3.
\newcommand{\ifempty} [3]{%
  \ifthenelse%
    {\equal{#1}{}}%
    {#2}%
    {#3}}

% Append #2 to #1 if #2 is non-empty. Otherwise just return #1
\newcommand{\maybeappend} [2]{\ifempty{#2}{#1{}\xspace}{#1-#2}}

% Do stuff in the preamble.
\newcommand{\stuffinthepreamble} {
  \author{%
    Claus Appel \\ 
    \href%
      {mailto:clausappel@protonmail.com}%
      {clausappel@protonmail.com}%
  }
  \makeindex
  \frenchspacing
  \bibliographystyle{plainnat}
  \pagestyle{headings}
%   \setcounter{tocdepth}{0}
}
\providecommand{\citet}[1]{\cite{#1}}

% Do maketitle and other stuff at the start of the document. 
\newcommand{\stuffatthebeginning} {
  \pagenumbering{alph}
  \maketitle
  \pagenumbering{arabic}
  \setcounter{page}{2}
  \renewcommand{\multicolumntoc}{2}
  { 
%     \begin{flushleft}
    \scriptsize
  %   \addcontentsline{toc}{chapter}{Contents}
    \tableofcontents
%     \end{flushleft}
  }
}

% Print index and other stuff at the end. 
% \newcommand{\stuffattheend} {
%   \bibliography{Mith.bib}
%   \printindex
% }















\begin{comment}
  \section{Indexing}
\end{comment}

\index{suppression of technology|see{repression of technology}}
\index{technology!repression|see{repression of technology}}
\index{calendar!Vaimon|see{\VaimonCalendar}}



















