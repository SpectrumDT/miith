2009-05-01

If Vizicar was dying in a shipwreck, why did he not just fly to safety? 
Clarify the limitations of Vaimon flight, and mention this explicitly.
Vizicar tried to fly away but could not, for several reasons, incl. weather and stamina.
Maybe flight is more like super-powered leaping, as seen in Asian martial arts movies.



2009-05-02

Læs om nazistisk mytologi og Thule-selskabet.
Ostara, et nazistisk-mytologisk blad.
Adolf Hitler\dash{}Mein Kampf
J.H. Brennan\dash{}Occult Reich

Listen to Wagner's operas!

Hav masser af nazisme og racisme ifm. \Aryothim og Nephilim.
Og for den sags skyld \resphain.
Hvorfor nedstammer \resphain ikke fra de overlegne \Aryothim?
\Semiza og Ilu har nok Rocolza-blod.

Læs Hindu-mytologi: Bhagavad Ghita.

Læs Dan Brown.

Have lots of old monuments built by Elder Races. Superior works of superior architecture that could only have been built by superior Elder Races.



2009-05-13

Clarify the definition of "sorcery".

Carzain should be brave and iron-willed in the face of adversity, like Lucian from Underworld: Rise of the Lycans. 
Maybe he should take over leadership of the Pelidorian army after Sethgal is killed. 
They still lose the war, but he saves many of them. 
Or something.



2009-05-15 (NT)

Greater \banes (\baneknights) can fly using a number of long limbs resembling insect antennae with feeling-hairs on them.

The \resphain have a number of texts on \bane sorcery.
These are dark and mysterious and feared.
They can cause madness even in \resphain.
They dread the \banes and all having to do with them.

Make \Urizeth really eccentric, even disturbingly so.
She is quite deranged after having studied \WanderersInDarknessEmph and its terrible Aenigmata for so long.
\Teshrial is uncomfortable in her company, but he also knew that she was very knowledgeable and that her dark arts and dark insights would be of invaluable help to him.
After her death and imperfect rebirth she looked even more like a madman. 
Complete with mad cackling.
Maybe.

The \dragons wielded even darker knowledge. 
They were a truly ancient race, the blood of \xs was in their veins and they saw deep into the nature of the Cosmos.
Even \Criseis shuddered to merely think of the terrible Aenigmata that her master spent his centuries pondering.
She knew the Universe was vast and dark and cruel, but she did not want to think about it.
She was still just a \scatha, even though immortal.
She still thought like a mortal, and she wanted to keep it that way.
She knew if she tried to think like a \dragon it would destroy her.

In war, (mortal) mages would bombard the enemy army with lightning and rain of fire, all the while trying to dispel and negate the enemy sorcerers' attempts to do the same. 

Power management for my Eee? I put it on standby all the time. 



2009-05-16

How to disable touchpad entirely with a hotkey?

\Cishiel had her hair and feathers permanently dyed fiery orange.

When Ramiel awakens he says: "Bring me my sword and my guns."
When Ramiel slays \Gilchad, he first criticizes him a bit. 
Then zaps him with dark lightning.
Then shoots him with his pistols.
Then blasts him repeatedly with sorcery and lightning.
Gradually he tears \Gilchad apart and, turning himself into a devouring, hungry vortex, sucks in \Gilchad and devours him body and soul.
It is like witnessing a black hole and its accretion disc.

It should be hinted that Ramiel and \Cishiel have an incestuous relationship. 
Once in a while he spanks her and says: "Do not disobey me, \Cishiel."
She gets aroused and whispers: "Yes, My Overlord \sathariah."

At first, Ramiel's motivation is just to get his powers and memory back. 
When he achieves that, he spends many days doing research, learning everything about the state of the world from records and from \Cishiel's Cabalist contacts.
He covets power and greatness and feels it is his right by succession to become Overlord.
But not only that.
He also doubts \Dasteron's ability, since \Dasteron apparenly cannot make himself Apex. 
He returns to \Mystraacht and studies \Dasteron in person.
He becomes convinced that, while \Dasteron is a good leader, he is not good enough.
He is a skilled politician and fighter, but he does not have the cosmic insight or Vertex strenght that \Mystraacht needs.
He finds out that \Dasteron shies away from all the darker writings and fears dealing with the \banes.
\Dasteron tends to obey the \banes quickly and with great fear.
\Dasteron is fearless when it comes to his fellow \resphain, but Ramiel fears \Dasteron is not capable of managing \resphan affairs in a larger cosmos.
This makes up Ramiel's mind.
He must dethrone \Dasteron and take the throne himself.

Some \xs ward runes work by fucking with the attacker's mind, projecting frightening visions of the \xs.
As in Andy Chambers\dash{}High Elves (1997):
"The Stone of Midnight exudes an impenetrable mist of darkness, and anyone trying to strike at the possessor will be confronted by his worst nightmares, vion of his own death and failure of all the works of his life."

\Rystessakhin was a High Queen.
She was noble and loving and good and loved all creatures.
When \Ishnaruchaefir preferred to look outward into the universe and had little interest in the affairs of mortals, \Rystessakhin looked inward and down. 
She cared deeply for the lesser creatures and was always advocating that the \dragons be more \humane towards them.
Later, \Ishnaruchaefir, after killing her, would feel some obligation towards carrying on her will, so occasionally he would be prone to fits of compassion for lesser beings when he remembered how his beloved would have felt towards them.
This was what happened when he saved \Criseis, and again when he saved Rian and Neina.
This responsibility only added to the heavy burden \Ishnaruchaefir carried (being Nex's heir and responsible for saving the world, and also the Destroyer).

When Ramiel finally awakens, he is uber-powerful because his many millennia as a Scion have given him a broad range of diverse forms of experience. 
In total, this proves to have been a more enriching learning experience than it would have been had he lived these millennia as a \resphan. 
(And also less dangerous.
As a \malach he was very hard to kill.)
When he awakens he remembers all the bits of cosmic knowledge he has gleaned over his many lives, the many Aenigmata whose Gnosis he has glimpsed. 
Back then he usually understood very little of it, but now, with the full knowledge of his many lives, he is able to see this knowledge from many exciting perspectives, and suddenly it all makes sense.
So immense amounts of horrid, sinister insight crash down upon his mind like a tidal wave.
With his multi-faceted experience he is able to understand much more than he otherwise would.
He attains Gnosis that no \resphan has ever held before him, save perhaps \Azraid.

And it is hard on him,
He screams in terror and anguish and has to fight a desperate and bloody battle against the many inner \daemons of his thoughts and fears to avoid losing his mind entirely.
His sanity is lashed like a vessel on a storm-wracked sea.
Afterwards he is traumatized and shaken. 
His sanity has suffered harsh blows.
But he is back with a vengeance and revels in his regained and newfound power.

\Cishiel is worried.
She fears her father has gone mad, and that his new power will make him even more mad and more dangerous.
But she knows it is too late to turn back.
She has cast in her lot with Ramiel and cannot hope to betray him, even if she would... which she would not.
He is still her father and she loves him. 
He saved her from the horrible fate that befell him.
She knows that had he not forbidden her to become a \malach herself, she would have had to undergo the same torture and madness that Ramiel has endured\dash{}or she might have been destroyed long ago, or still trapped in the body of an amnesiac \human.
No, she owes everything to Ramiel and still has more to repay him.
She feels more anguish for \Dasteron, her other ally, for she fears Ramiel will destroy him.
Ramiel looks terrible in his fury (when he eats \Gilchad), and \Cishiel fears no one will be able to stand against him, not even the formidable and resourceful \Dasteron.

Later, it will turn out that Ramiel's sanity really has suffered badly from all his experiences and all his revelations.
It is what turns him against the \banes and makes him take such mad chances.
And he is wracked with much anguish and guilt over slaying \Shiaraid (and, to a later extent, \Dasteron and \Gilchad).
But he also feels an alienation from his fellow \resphain, as if all mortal and immortal life is really a bunch of worthless, insignificant specks of dust.
And he still feels a desperate need to prove that he is more than nothing.
("User-Maat-Re, thou hast done nothing...")

\Sethicus was the first \dragon, and he created his fellow \dragons in an attempt to stage a coup against the existing \ophidian oligarchy, which he saw as decadent and ineffectual, something that was holding the \ophidian people back. 
He wanted to seize power and achieve "great things" (whatever that meant). 
After a long, violent insurrection, \Sethicus was defeated.
But his \dragons were immortal.
Far more so than the natural \ophidians.
The \ophidians lacked the sorcery to destroy them, or if they knew how, they refused to use it on principial grounds because it was destructive sorcery and would corrupt their people and lead them on a sliding slope towards the \xs, which would be the end of their enlightened civilization.
So they imprisoned \Sethicus-tachi instead.
For thousands of years. 
Then, when the \banes attacked, the \ophidians realized they would need the help of \Sethicus and his \dragons. 
So after much negotation (first amongst themselves, then with \Sethicus), the \ophidians released the \dragons, and they went to war against the \banes.
Eventually \Sethicus fell, but his corpse was preserved, and all the way up to the \thirdbanewar it was enshrined in \Dathka and worshipped as a dead god. 
Perhaps there was even some remnant of \Sethicus's powerful soul that lingered on as an undead, quasi-conscious presence.



2009-05-18 (NT)

Drinking \resphan blood is very healthy for \humans.
It heals wounds and rejuvenates and extends lifespan.
But it is also addictive.
Psychologically at first.
Repeated consumption over short time leads to physical addiction.
Once badly addicted, a \human will need it regularly or she will die.
Drinking \resphan blood also has a psychological effect of making the drinker instinctively devoted to the \resphan in question.
She will effectively fall in love and will feel an urge to serve him.

Some \ashenblood \resphain lived among mortals as vampires.
Reavers they were called. 
\Bezedeth did not need quite as much lifeforce to survive as purebloods. 
They could survive for a long time on mortal souls alone and only needed to consume immortal power once in a while or they would weaken.
Even if deprived of immortal sustenance for long, they would not die quickly but weaken, fall into torpor and slowly wither away over a course of decades.
Some would set themselves up as rulers of mortals\dash{}immortal vampire lords.

After the splintering of \Ortaica, the \rethyaxes found that it became somewhat harder to contact the \taorthae.
This had to do with the fact that the \taorthae were busy fighting wars on other Realms and amongst themselves at this time.
In fact, the "Absconding of the \taorthae", as it became known, preceded and caused the splintering of \Ortaica.
Post-\Ortaican \rethyactic scholars would interpret it as the other way around.
Scholars would also blow the Absconding out of proportion.
Myths arose that back in the "golden age" of \Ortaica, the gods had walked the earth amongst their followers and fought the enemies of the Bacconate in person.
The Vaimons' records had plenty of stories of \daemons fighting for the \Ortaicans, so it was not too hard to believe that the gods had been there. 
And now they were gone. 
The \rethyaxes mourned this.
Some lost faith in the \rethyactic religion, and the remnants of \Ortaica were weakened further.
This allowed the Imetrians and the \Iquinian Tepharins to take over.
In fact the Absconding was much more relaxed. 
The gods had never been very active in the first place.
They obeyed the Unspoken Covenant, after all.

Some of the \taorthae appeared in the form of \dragons. 
After the Absconding, \rethyactic theologians would discuss whether they actually were \dragons or whether they just took on their likenesses. 
Even the \rethyaxes doubted whether \dragons really existed.
They had not been seen since the \human Age began.
Perhaps the Vaimons had destroyed them all, or perhaps they had never existed.
Perhaps the Vaimons had made the \dragons up in order to make the accounts of their own deeds seem more impressive.

Forum: Can I give my dark angels sunglasses on without getting narmful?

\Ortaica should have a Meso-American theme. 
Redesign the language.
Redesign Imetric accordingly, OR provide a different etymological background. If Imetric sounds like Latin, it cannot be descended from \Ortaican. 

\Nasshikerr -> \Nasshikerr (with an S with hacek)
Rissit's original name -> H'risht'tetl

The war surrounding \Belzir and the ensuing Hundred Scourges were partially masterminded by the \taorthae, who were just emerging at this time. 
They had a long-term plan to destroy the Vaimons, and they realized they could employ this rebellious Empress to their own ends. 
According to \Ortaican myths, their gods brought down the wicked Vaimons and drove away their hideous \Archons. 



2009-05-21

Køb bøger:
Robert E. Howard\dash{}Bran Mak Morn
Steven Erikson\dash{}The Lees of Laughter's End
Alan Campbell\dash{}...
Horus Heresy
Brian Lumley\dash{}Necroscope

Clarify that \Ishnaruchaefir is a genuine and current threat to the \resphain, not just a past threat.
He keeps fucking up their schemes, and he is endangering a long-term scheme that is vital to \CiriathSepher if they want to rise to supremacy and realize their worth.
He preys on the \resphain and kills them and their servants. 
He has been passive long, but now he is becoming a serious menace, and the Cabal fear him.
And he is casting spells (storm beacons and the like) that prevent the Cabalists from doing their thing in Malcur.
They have a constructive goal in Malcur. 
Clarify that.
They think \Ishnaruchaefir is the biggest threat against that goal, but it turns out \Secherdamon is a worse threat. 

\Ishnaruchaefir is not so big a threat that all \resphain in the world are after him, though.
So far he is just endandering the Malcur gambit, which only a small part of \CiriathSepher really care about. 
(Although the ones that do care have very high expectations of this gambit and hope it will determine the future fate of \CiriathSepher, if not all \resphain. The ones not part of the gambit are more skeptical. \Azraid has hopes for the venture, but remains skeptical and aloof.)
(Make a section about the \CiriathSepher Malcur Venture.)

Anyway, there are several \resphain who want to do \Ishnaruchaefir in, but he is notoriously elusive.
But he has promised \Teshrial to give him a rematch, and told \Teshrial to contact him when he is ready.
In some very clear terms.
So the \resphain know that if they want to do \Ishnaruchaefir in, \Teshrial is their best bet.
When \Teshrial meets \Ishnaruchaefir, \Ishnaruchaefir reminds him of this. 
\Ishnaruchaefir makes \Teshrial promise him that he will fight only \Teshrial, no other \resphain. 
(\Teshrial smirks inside because he plans to ambush \Ishnaruchaefir with his non-\resphan allies. \Ishnaruchaefir anticipates this. \Ishnaruchaefir is smart enough to know that \Teshrial would never face him if he did not have a trap or several prepared.)
\Ishnaruchaefir also warns them that if \Teshrial should break his promise, \Ishnaruchaefir will refuse to fight, and return later to exact a terrible vengeance.
\Ishnaruchaefir reminds \Teshrial that he found out quickly and easily about \Urizeth, to remind \Teshrial how deep his Cabal spies go and how good his intelligence is.
\Ishnaruchaefir lets slip that he knows \Teshrial has feelings for \Firaxel, and hints that he will come after \Firaxel if \Teshrial betrays him. 
(This is a bluff. It was through luck that \Ishnaruchaefir was able to get at \Urizeth so quickly and efficiently. He doubts he would be able to do the same with \Firaxel.)
\Teshrial remembers the story of how \Ishnaruchaefir terrorized the \resphain after \Criseis' siblings were murdered, so he takes \Ishnaruchaefir's warning very seriously.
Later, when his fellow \resphain recommend they set a bigger ambush, \Teshrial very firmly declines, and explains his reasons.
They now have a chance to exploit the Destroyer's code of honour. 
They would be fools to let that go to waste. 
Instead, several powerful \resphain give \Teshrial gifts of magical items for him to use in the battle, including a sword, pistols, armour, amulets and wing braces.



2009-05-22

Make clear that \Miithian psychology differs from RL psychology.
In RL it is often asserted that all true happiness derives from \quo{love}, and that all other desires are hollow and unfulfilling in the end.
This may or may not be true in RL, but it is not true on \Miith.
\Resphain and \dragons alike derive genuine pleasure from conflict, from living out \quo{negative} emotions such as anger, hate and pride.
That is one reason why there has been so much war:
Peace is not happy for these warrior races.

In a sense, this means the \resphain and \dragons are inherently \quo{insane}, since they naturally tend towards behaviour that is destructive for them in the end.
They also have an unfortunate tendency to idolize and be attracted to such destructive, \quo{insane} behaviour.
That is one reason why the \satharioth were so popular despite their derangement (at times very obvious, as with \Zachirah). 
Ramiel for one attracted more \resviel because of his inner demons and darkness and insanity.
They could feel the violent emotions swirling inside him, and it turned them on.

In combat a \dragon is worth more than 10 \resphain. 
The average \dragon can take on at least 20 purebloods. 
If they are mounted on \umbrae then it only takes an average of 6 of them.
\Nzessuacrith, while no \shaeeroth, is far more than an average \dragon.
So when she shows up, \Achsah has to call in ALL available reinforcements.
The other \resphain are not happy about that, because they would like to keep some in reserve for \Ishnaruchaefir.
But \Achsah pleads her case, and \Teshrial agrees with her (because he fears breaking \Ishnaruchaefir's agreement).
(This happens before \Teshrial's duel has started.)
So a dozen \resphain hurry to \Forklin to fight off \Nzessuacrith. 
This is distressing and hurtful to their plans because they had not planned for such an eventuality. 
Like \humans they can be shortsighted, so if \dragon attacks almost never happen they will assume they will never happen, and hence not plan for them. 
\Nzessuacrith's appearance is a blatant breach of the Unspoken Covenant which none could have foreseen.
Also remember that the faction taking care of Pelidorian business is small. 
There are not many \resphain there, so they are not well-equipped to deal with such fierce \dragon attacks, from \shaeeroth and other elders.

\Achsah retains leadership of the delegation in \Forklin. 
She is highly talented and experienced, more powerful than many purebloods.
(Some of the others are \thelyadeth or Gessurim, others \bezedeth.)

It is known that \satharioth are more powerful than \ketherain and \thelyadeth, which in turn are more powerful than \bezedeth. 
It is commonly believed that \ketherain are more powerful than \thelyadeth, but this is a dubious assertion.

\Achsah gets suspicions that Malcur is a decoy at some point after \Urizeth's death and before her revival.
\Achsah asks \Teshrial. 
He is frantic at this point and spends his time reading \WanderersInDarknessEmph. 
He is especially distressed because it had been a long time since his last meeting with \Urizeth, and he is sure she had discovered many new things in the meantime, which he now cannot learn from her. 
He fears he has been set back terribly.
After all, he has good reason to fear that \Urizeth will succumb to \Ishnaruchaefir's terrorism and back out and refuse to help \Teshrial further.
If she stays, she might get destroyed, and \Teshrial knows \Urizeth is no warrior.
She has no obligation to stay, and her self-preservation might very well override her motivation to help him.
So far she has been helping \Teshrial mostly out of scientific curiosity, or so it seems to \Teshrial.
So he has good reason to fear \Urizeth is gone for good from his POV.
Moreover, he has now taken to reading \WanderersInDarknessEmph himself, despite having little occult experience. 
(He is first and foremost a martial artist. Stress this!)
This reading is taking a hard toll on his already frayed nerves and sanity.
So when \Achsah comes to him, \Teshrial is stressed and haggard and frantic.
He does not listen much to her. 
He just lets her go her way and do her thing.
\Achsah walks away feeling worried for \Teshrial. 
She does not like him, but she still feels compassion to see him in this disturbed and disturbing state.

\Resphain have naturally black skin, but they are not always completely black.
Their hair and feathers can be black, but they can also be other colours: Grays, browns, reds, or even (rarely) white, yellow or orange.
A \resphan's body hair and feathers will usually be of one single colour.
Assign colours to all my \resphan characters!

Skin colour also varies a bit, even among purebloods. 
It can be slightly gray, or bluish, or chocolatey brown, or very dark red.
Maybe look up colours on Wikipedia.



2009-06-04

The \resphain practice a lot of eugenics.
They want to improve and purify their race. 
To this end, they perform a lot of experiments on \humans:
Breeding and sorcery and medicine. 
They hope to find results from \human research that will carry over to the \resphain themselves. 
Compare to the Nazis and their race theory. 
Especially \Mystraacht had some strict ideals of physical purity, strength and health, tying in with their macho warrior ideology.



2009-06-06

Returned Ramiel is wiser than his old self, and as a result he is more thoughtful and philosophical. 
He is more critical of \Mystraacht ideology. 
He still uses the traditional \Mystraacht macho ideals in his rhetorics (as Overlord and when competing for the throne), but internally he questions it a lot, and in his actual policy he is often far more pragmatic than would be expected of a \Mystraacht (especially suprising to those who knew his old self). 
The \Mystraacht way is to charge in with all your macho bravado, and to worry about bravery and cowardice and reputation. 
Ramiel ends up being much more rational and careful.
He is also critical of \Zachirah's \quo{Religion of Evil}. 
Back in the day he followed \Zachirah's ideology unquestioningly and with great fervour, but the wiser Ramiel of today is smarter than that and recognizes the insanity and danger of the ideology. 

Some \dragons have fled \Miith and dwell in \Machai. 
This is their \quo{second homeland}, because they draw so much of their power and their being from there.
So some are content to dwell in \Machai and concern themselves with \Machaic affairs and remain aloof from \Miithian affairs. 
Late in the \thirdbanewar, \Secherdamon or \Ishnaruchaefir finally manages to rouse many of these aloof, retired \dragons.
They are awesome to behold as they return to \Miith in full force. 
Dozens of Elder \Dragons, several of the \shaeeroth.

Since the inception of the Shroud and the weakening of the Heart, almost no new \dragons were born.
Most \dragons alive in the \thirdbanewar were exceedingly old, being adult already when the first \resphain came to \Miith. 
In fact, \dragons were even longer-lived and slower-breeding than the \ophidians from which they were descended. 
Bu the aloof \dragons in \Machai had stopped caring.
They were powerful and immortal and had the luxury to wait a few thousands or even tens of thousands of years between procreating. Many of them were over 20K years old and saw the war with the \resphain as a temporary nuisance. 
\Ishnaruchaefir and \Secherdamon were similarly old and would likely have thought the same, but the war came horribly close to them when some of their close family members (\Nexagglachel and \Ishnaruchaefir's three sons) were destroyed by the \resphain.
Now, in the \thirdbanewar, \Secherdamon or \Ishnaruchaefir had to convince the aloof ones that the war was more than a nuisance and that its conclusion would have great repercussions for all \dragons, even the ones who had fled to \Machai.
Eventually some of the aloof ones became convinced. 
Not all, but enough.

The conclusion of the war WAS important.
The aloof ones could hide in \Machai, but they could not really get away. 
They were still bound by the Heart and would suffer its fate.
Either that, or they would have to evolve, abandoning their \Draconic nature entirely and becoming something different.
\Ishnaruchaefir and \Secherdamon knew this might be an option\dash{}to flee to \Machai and abandon \dragonkind\dash{}but it was not a palatable one. 
They both wanted the Draconian race to survive.
\Secherdamon wanted to win the war and was willing to enlist the \xs, with all the associated costs.
\Ishnaruchaefir did not like that idea.
So he looked outward into the universe for other ideas.

Actually, \Ishnaruchaefir and \Azraid had met and talked earlier. 
\Ishnaruchaefir hated \Azraid for being a \sathariah, but he also knew from \Nexagglachel's brief telepathic sending just before his death that \Azraid was one of the better ones of the \resphan lords, from a \Draconic POV.
So early in the war \Ishnaruchaefir and \Azraid secretly developed a faint empathy of sorts.
\Ishnaruchaefir had some suspicion that \Azraid had more up his sleeve than he let on, and that he had plans for \Miith different from those of the \banelords.
The two should meet or talk at the end of TAR or in the next book.
Gradually and ever so subtly through \quo{Sentinels of \Miith} \Ishnaruchaefir becomes convinced that \Azraid is on the level, and he also begins to suspect the general nature of \Azraid's plan. 
So when \Azraid finally approaches \Ishnaruchaefir and asks for his cooperation in a secret venture, \Ishnaruchaefir has a bit of a \quo{what took you so long?} attitude, and the whole interaction has a \quo{I know you know I know you know...} frame.
The two have each formulated ideas that can be combined into a useful whole.



2009-06-07

After his return to Overlordship, Ramiel broods over the doings of the \resphan race.
He is disgusted by their stupidity and hypocrisy. 
The \Mystraacht curse the \CiriathSepher for being shallow and dishonest, but the new, wiser Ramiel can now see that the \Mystraacht are no better themselves, with all their shallow machismo and their hollow talk of honour and bravery and whatnot.
He thinks: \quo{There are times when I wish the \banelords would just come and devour everything. Or whatever it is they plan to do.}
Even among the highest tiers of the \resphain who serve them, there is no one who really knows what the \banelords intend to do.
No \resphan understands the mind of a \bane.

The \resphain have almost no religion.
They see their own \matrices and the Heart as something great and powerful to swear and curse by. But they know how those things work (sort of), so it does not form the basis of a real religion (religions are based on ignorance and fiction and mumbo-jumbo). 
They also revere the \banelords, but not much.
They fear the \banes and do not like to talk or even think about them. 
They serve the \banelords in name, but in their everyday lives the \resphain would prefer to pretend the \banelords did not exist.
They certainly have no desire to worship them.

Before \Dasteron becomes Overlord, he has to fight several battles to the death against potential rivals.
Strength in combat is not the only virtue demanded of an Overlord, but it is an important one.
He brings \Cishiel as a spectator to those battles. 
She is dressed to resemble a slave: A bikini of brass and a necklace that sort of looks like a slave collar, and otherwise naked.

Rian feels great cosmic terror when he sees \Ishnaruchaefir, even in his \Scathaese form in \WanderersInDarknessEmph. 
But Rian also feels drawn towards him and cannot look away.
He has to follow, even though parts of his mind implore him to turn away and forget the strange sight.
Compare to Marius from Diablo II.

After Malcur's fall it is \Criseis who espies Rian.
She begs \Ishnaruchaefir to go down and help Rian by killing or scaring away the monsters that are about to eat him.
\Ishnaruchaefir finds sympathy for these small creatures and obliges her.
Rian and Neina cower in horror of the vast \dragon.
\Criseis then dismounts and leads the \humans to safety while \Ishnaruchaefir goes to talk to \Nzessuacrith or something.

The various gods of \Azmith were usually seen by their followers as vast, powerful and frightening, but still mostly benevolent.
But most of them were really horrible elder entities like \dragons. 
It was gruesome for the people to realize the dark, sinister, occult nature of their gods.
It made some people realize how little they really knew, how powerful the occult unknown was and how much they were the slaves of the powers of primordial darkness.
A few Vaimons and other sorcerers suspected the truth, having learned much about the dark universe through their sorcerous practices.
These people generally kept their suspicions to themselves.
They were having a hard enough time adjusting to it themselves, and they knew most people would not believe them, or even burn them as heretics. 
Besides, it was a terrifying truth that people were better off not knowing at all.
It is better not to speculate, for down that road lies heresy, horror, corruption, evil and madness.
Ignorance is bliss.

Daxian -> Tlazcan
Isxae -> Iscsei
Nerrhan-Koss -> Nerran-Koss

Some of the most powerful and well-known \xs, and their \Ortaican names:

Khoth-Sell: Incuitzal
Naath-Kur-Ramalech: Nazcóloc
Nerran-Koss: Mecoroshetl

The \xs names are known to a few sorcerers and sages.
Only few heretics dare speculate aloud the possiblity that the \Ortaican gods really are the \xs in disguise.
That possibility is too horrid to consider.

Other \Ortaican gods are \dragons or QJ: Tlazcan, Iscsei, \Nasshikerr.

Moro \Cornel knows the \xs exist. 
At some point she wonders about their identity and the connection to what is happening.
She recognizes something in the catastrophic fall of Malcur that reminds her of what she knows of the \xs.
And she suspects her own \Nasshikerr knows more than he lets on. 
She suspects \Nasshikerr is in league with the \xs, or the vile alienists who employ their power.

Imetric magic is very mystical and strange and occult. Chaotic in a way, but a cold way.
In contrast, Rissiic magic is logical and scientific and rational.
Rissitic magic can also be chaotic, but in a controlled way, even though it may be hot and fiery.
Rissitics learn to channel, control and utilize their emotions.
Imetric mages come off as inhuman, robotic, alien, devoid of emotions.

Imetric magic: Compare to \quo{Warhammer: Hordes of Chaos} from 2002 by Gavin Thorpe and Rick Priestly and Anthony Reynolds and Alessio Cavatore, p.26:
"Raising his staff high into the air, he roared at the night sky. 
Dark lightning flickered at the corners of his midnight eyes as the heavens answered his call, deep, rumbling echoes sounding over the plains.
The darkness gathered around him like a whispering cloak, shadows coiling out rom the surrounding gloom to hover behind him.
The twisting blackness reached out with amorphous tendrils, their icy touch sucking the warmth from the warriors' bodies, and making the fire shrink out of existence.

In darkness the Sorcerer spoke, his voice as though there were a dozen people speaking the same words from his throat."

High-ranking Imetric sorcerers and priests are often those with much Naga blood, for these are more highly trusted by the gods and the predominant Imetric intelligentsia.
As they grow in age, power and occult \quo{forbidden} knowledge, their minds twist and become un-\scathaese and more \naga-like. 
As they channel lots of magic, their bodies twist as well, as the \naga sorcery awakens their \naga blood and strengthens and brings out the \naga part of them. 
They mutate and come to resemble \nagae.
Those with enough \naga blood and enough skill and experience (and luck) eventuality become \nagae and go to join their gods in the sea.
The Imetric gods are highly aquatic! Make this clear!
The older mages are strangely mutated and look deformed.
They are highly honoured for this in Imetric society, but outlanders use this as proof that the Imetrians are monsters that worship evil things.

Daimonion -> Daimon (pl. Daimonia)

\Resphain should be even larger. 
\Teshrial is twice as tall as a \human.
That way I can better justify why a small number of \resphain can take down a \dragon, while it takes so fucking many of other races.
\Resphain are even taller than \aryothim, but not nearly as heavy.

The Imetrians and Rissitics both employ a wide variety of beasts and monsters in war.
The Imetrians especially use sea monsters, which makes them so powerful and dangerous in naval battles. 
Imetrian sailors and sea-mages are so renowned that they are often hired as mercenaries by others\dash{}those who dare associate with such dangerous sorcerers and their terrible monsters, that is.



2009-06-08

Imetrian mages not only feel inhuman, they also look it physically.
And they think inhuman. 
Their minds become \naga-like, and they lose their more \scathaese emotions.
Outsiders sometimes interpret this as \quo{insanity}, but it is really just a change of perspectives, the acquisition of a new mindset, new emotions, new motivations, new ways of thinking.

Imetrians typically dress in shades of blue and green, like the sea. With stripes of white or copper.
Rissitics prefer reds, yellows and browns. Tinged with black and gold.

\Vizsherioch is the Son of Chaos, the messiah of the \dragons and the harbinger of the \xs and the hordes of Chaos. 
After he becomes a \shaeeroth, he becomes feared by all.
He grows so much in power as to rival \Secherdamon and \Ishnaruchaefir. 
He is, after all, (sort of) an incarnation of the \xs. A \xs in draconic form.

\Ishnaruchaefir is a tremendously powerful sorcerer and can call up hordes of monsters/demons to fight for him. 
He achieved much of his terrorism this way.
He functions as a one-man army of Chaos. 
This is one reason why the \resphain dread him so. 

In TAR, the \resphain working in Malcur have some viewing stations and stuff set up in a Realm adjacent to \Azmith. 
\Ishnaruchaefir sends a horde of his \daemons to overrun one such station. 
It demonstrates to the \resphain how dangerous he is, and why he must be stopped.

Everyone on \Azmith knows that monsters and evil spirits and even evil gods lurk in the Wild and in the dark, forbidden places of the world.
\Jirad Tantor should not dismiss things as superstition.
Rather, they know they are venturing into a dangerous place full of monsters.
Still, the forest is not so huge, and it is in the middle of Runger, surrounded by churches and Light-fearing men, so what is the worst that could happen?

Be sure to demonize the Rissitics. 
Rumour turns them into legions of demons from Hell, inhuman wielders of dark sorcery and evil.
The fact that \Narkiza is known to be an \Ashenoch, and that the Rissitics have always relied on magic and monsters, only makes their image worse.
They are seen as the marauding Legions of Chaos.
Make this clear in TAR and in the later books.
\Tiroco hears reports of towns and cities in the south being overrun by screaming, wicked Rissitics.
"Behold, the armies of war descend screaming from the heavens!"

\Secherdamon's \matrix was very powerful and very ambitious.
He tried to build it up to be the vastest and mightiest \matrix ever.
This meant that the \matrix required much maintenance.
So he had enire pyramids and ziggurats and the like erected whose purpose was to channel magical energies and act as \nexi and keep the \matrix alive and stable.
These buildings required sacrifices and spells performed daily to maintain them and keep them running.
They would grow and hum and cracle with sorcerous energy at all times.
Compare to the College of Light whose purpose is to maintain the sorcerous Wind of Hysh, in "Warhammer: The Empire" from 2000 by Alessio Cavatore. 

The fall of Malcur: "The Empire" p.76:
"The seething Realm of Chaos swept over the city, engulfing it, and Praag was changed forever, its stone walls and buildings melding into hellish and inhuman shapes. Those citizens unlucky enough to still be alive were swept into the maelstrom, their living bodies fused into the walls of the city itself, so that it was no longer possible to tell man from stone. Distorted faces leered from the walls, agonized limbs writhed from the pavements and pillars of stone shrieked in madness with voices that once came from \human lips. Praag had become a living nightmare and a grave warning of what lay ahead should the Chaos armies conquer the land."



2009-06-09

Hyperborea was an Immortal Realm. 
It was cold and full of ice.
It was closer to the \CrystalSphere than most any other place in \Miith, and a particularly good place for the \banelords to exert their influence, and for \Miithians to contact them or to draw power from \Erebos.
The \resphain feared and disliked the place (just as they feared the \banelords), so while they had bastions there, they did not like to spend time there. 
It was to Hyperborea that Ramiel journeyed to regain his memory and power. 
His goal was a particular lost temple city of UltimaThule that had once been a sorcerous bastion (of some race), but was now abandoned and ruined.
Perhaps UltimaThule was even a \voyager citadel.

The nothern regions of \Azmith gradually transition to Hyperborea.
The lands are full of monsters and cold magic.
The inhabitants (\human or \scatha) are fierce barbarians who pray to dark gods. 
Not only the \banelords have influence, but also the \xs. 
The Shroud is thin in the north, because the land gradually transitions to an Immortal Realm. 
There are also \quiljaar that dwell there. 

In Hyperborea one could feel the cold winds of darkness from \Erebos.
They spoke of desolation, despair and entropy.

Arcturus\dash{}To Thou That Dwellest in the Night:
"Here is so desolate;
Times, they are dark
Words ceas'\dash{}to end as echoes rolling afar

Empathy arises
whilst thou drapest this world in black;
The only colour that can paint my soul

Clad in the shades of night
Thou reflects the pure of heart

Amidst all the grief this winter unfoldeth
The thorn in my side\dash{}thou retainst

Thy breeze maketh me shiver
Maimeth me with its frozen malice
Thou minglest with the dense night
I hearken to the voice of thy winds
They are the saddest of all sounds of thine
Never will I take leave from thy haunt

Hast thou ever desired me?
I receive no answer, thou letst it pass in silence..."

At the end of SOM, \Miith falls, the old world order crumbles. Ramiel is now the lord of the \resphain.

Raudt og svart:
"Deira dagar har m?rkna
Moder jord meld sitt fraafall
Men sverdet har gaat vidare
Til ein ny b?rar

Me er einsomme menn
Me b?r fenresulvens muspell
Skoll skal sluka sola

Me er djerve menn

Sola svartner
Der fjellvegg ramlar
Naa gjestar sorgen
Paa livets tre (Yggdrasil)

For byleits bror farar
Kva er det ikkje Kampfar klaren?"



2009-06-10

The Sentinels in Malcur eagerly await the advent of Nithdornazsh. 
They do not know they are to be sacrificed.
\Secherdamon-tachi mourn their loss, but it is necessary.

Divine Intent:
  Rise in Splendour\\
  Arms upheld unto the Sky\\
  Blindly I Follow\\
  Soon All will be Revealed

  Searing Transcendant\\
  Visions of Divine Intent\\
  Worshiping Obedient\\
  I Await the Burning

  In Supplication\\
  I turn my eyes toward the Sun\\
  It Burns

  I turn my eyes toward the Sun\\
  Its Flames, Oh God it Burns\\
  Refusing to look Away\\
  In Agony I become Divine

  Obeisance I Make\\
  Engulfed in Radiant Fury\\
  Amon-Ra, The Hidden One\\
  Takes my sight that I might See

  Blinded I cry Out In Gratitude\\
  Tears Streaming I Return unto the Sands

  Burnt, Charred\\
  I behold no God

  I Become Divine



After the \secondbanewar, \Secherdamon planned his revenge against the \resphain:

Black Hand of Set:
  Eaters of \human Flesh\
  Hath eaten unlawful Flesh\\
  Upon our Brethren\\
  They have Feasted\\
  Seed of our Father\\
  We must now Avenge

  In secret conclave we Gather\\
  To rain Destruction \\
  on those Whom We have Cursed

  With vile Black Arts\\
  And Tempestuous Rage\\
  We vent our Wrath\\
  Red Blood stains my Hands and damns my Soul

  You will drink the Black Sperm of my Vengeance

  The Mighty Voices of my Vengeance\\
  Smash the Stillness of the Air\\
  And stand as Monoliths of Wrath\\
  Upon a plan of writhing Sepents

  I call upon the messengers of Doom\\
  To slash with Grim Delight this Victim I hath Chosen\\
  Feed upon his Brain Pulp, Rend his Throat\\
  Pierce his Lungs with the strings of Scorpions\\
  Oh Kali\\
  Oh Sekhmet\\
  Oh Dagon



\Thanatzil could feel the pressure of the \banelords upon him, and it scared him.
He felt naked and vulnerable under their dread gaze.

\citebandsong{Nile:FestivalsofAtonement}{Nile}{
  Wrought
}{
  Nanna Father of the Zonei\\
  Eldest of the Wanderers\\
  A Shadow out of Time\\
  The Moon is calling Me\\
  The Breath of the Old Ones\\
  Whispers in my Ear\\
  With inhuman Impatience\\
  They Beckon to Me

  I call to the Moon and Sin\\
  I now possess the secrets of the Tides of Blood

  I have traveled on the Spheres\\
  And the spheres do not protect Me\\
  I have Walked in the Pit\\
  And the Pit does not protect Me

  The Lords of the Wind rush about Me\\
  The Lords of the Earth crawl about my Feet\\
  And are Angered\\
  A Wind has Risen \\
  The Dark Waters Stir
}


There should be some apocalyptic events before the last SOM book where much of \Azmith is devastated and the truth of the \xs, \Iquin and perhaps even the \banes is revealed to the mortal masses. 
Hordes of Chaos and/or Darkness invade \Azmith, like in Warhammer. 
The surviving mortals suffer.
They know their world has been destroyed and they are now living in a Hell on Earth.

\citebandsong{Nile:FestivalsofAtonement}{Nile}{
  Extinct
}{
  Paradise Lost\\
  Dreaming of Extinction\\
  We wander through the Walls of Sacrifice\\
  Sick winds brush against my Skin

  Power of Extinction\\
  Growth Intelligence No More\\
  Mud Rot Skeletal Earth\\
  Drown thy spirit of Kings

  Society's walls break Down\\
  \Humans Pound Down\
  Dig I must dig Out\\
  Surviving puts all tools in Place\\
  Only peace comes in Death

  As I sleep I dream of Death\\
  Only Peace comes in Death
}


2009-06-11

Neina goes mad and remains mad ever after.

\citebandsong{Nile:RamsesBringerofWar}{Nile}{
  Howling of the Jinn:
}{
  Fiendish Insects encircle Me\\
  Howling Wind Wraiths\\
  Surround my disembodied Ka

  Dulcarnon\\
  Hideous Unseen\\
  Speaking in Tongues\\
  Heard only by the Mad

  Shrieking Insects Swarm over Me\\
  Suffocate Me\\
  Suffocate my Soul

  Majnun I am Empty\\
  Crawling Reptiles Devour my Soul\\
  They utterly and completely Annihilate Me\\
  I can hear the Howling of the D\jinn\\
  Echoing in the mountains of Kaf
}


The war song of the \dragons, which they sung when they first arose with \Sethicus, and when they returned to fight in the FWB, and when they returned under \Xserasshana: 
They revelled in their power and fury, their ability to cause destruction and fear.

\citebandsong{Nile:RamsesBringerofWar}{Nile}{
  Ramses Bringer of War
}{
  Wretched Fallen one of Khatti\\
  Rise against the oppressing Sword\\
  Encircled Abandoned Alone\\
  I Smite the vile Hittite Foe

  My Father Amon what carest Thee\\
  For the Vile and Ignorant of God\\
  My Father Amon what carest Thee\\
  For these Effeminate ones\\
  At millions of whom I groweth not Pale

  Raging like Menthu like Baal in his Hour\\
  Lo the mighty Sekhmet is with Me\\
  I enter in among them even as a hawk striketh\\
  I slay I hew to pieces and cast to the ground\\
  The royal snake upon my brow\\
  Spits forth Fire in the face of mine enemies\\
  And Burneth their Limbs
}

\Xserasshana and \Sethicus both saw themselves as mighty conquerors:

\citebandsong{Nile:RamsesBringerofWar}{Nile}{
  Ramses Bringer of War
}{
  My Chariot Wheels trample the Fallen\\
  Cut to pieces before my Steeds\\
  And laying in their own Blood\\
  I Crush the Skulls of the Dying\\
  And Sever the hands of the Slain\\
  I Ramses\\
  Builder of Temples\\
  Usurper of Monuments \\
  Slayer of Hittites\\
  Bringer of War
}

The \aryothim used dark magic in their initial campaigns of genocide against the reptillian races.
In fact, it is an ongoing theme on \Miith that in a war, the side that is willing to employ the darkest sorcery is usually the side that wins. 
(The Vaimons' magic was disguised as something nice and good, but it was really even more insidious than \xs magic\dash{}more insidious than anyone had imagined, as it turns out in the end, with Lithrim.)
\Aryoth battle song:

\citebandsong{Nile:RamsesBringerofWar}{Nile}{
  Der Rache Krieg Lied Der Assyrische
}{
  Nergal, dread God of War and Plague\\
  Avenge the shades of our Fallen ones\\
  Blacken the sun with Fierce Winds\\
  Bring forth your Terrible Storms!\\
  Yea! Burn the flesh of our Enemies\\
  Gash their throats with Weapons of Iron\\
  Destroy Them utterly\\
  With Locusts and Disease
}


During the \thirdbanewar, epidemies of disease sweep all the Shrouded Realms.
It is a consequence of the life-drain inflicted by \Iquin, and also a side effect of the growing influence of the \banelords and their Entropy.
Disease is a manifestation of Entropy.

\citebandsong{Nile:AmongstheCatacombsofNephrenKa}{Nile}{
  Pestilence and Iniquity:
}{
  Heralds of Pestilence\\
  Blackest Plague Rusheth through the Land\\
  Burning Evil Winds\\
  Carry Sickness\\
  Invoking the Bitter Venom of the Gods

  Loathsome Sickening Stench of the Defiled\\
  A Cesspool Breeding the Unclean\\
  Hordes of Locusts\\
  Fiends of the South Winds\\
  Cleanse the Earth from the Impure

  The Daemon That Siezeth the Body\\
  The Daemon that Rendeth the Body

  Ruthless and Profane\\
  Lord of all Fevers and Plagues\\
  Grinning Dark Angel of the four Wings\\
  Spawn of Eng\\
  Horned God with rotting Genitalia\\
  Pazuzu
}


Recommend me some apocalyptic horror stories, where the apocalypse actually happens.

After the Awakening, the \resphain resurrected \Semiza as a Lich.
It was traumatic for him, and also for some of them, for it involved some pretty wicked magic. 
\citebandsong{Nile:AmongstheCatacombsofNephrenKa}{Nile}{
  Opening of the Mouth
}{
  I split open your Eyes for you\\
  I open your mouth for you\\
  With the Adze of Iron\\
  Which split open the mouths of the Gods\\
  The Iron which issueth from Set

  With the Adze of Iron\\
  I did clean your Bones\\
  With Iron Hook and Chisel\\
  I did scrape your Flesh\\
  And separate your Bones for you

  Sebau Fiends work Evil on the Body

  Raise yourself Asari\\
  Receive your Head\\
  Collect your Bones\\
  Gather your limbs Together\\
  Throw off the Earth from your Flesh\\
  Khnemet-Ur cometh to Thee
}



The Rissitics were seen as demonic hordes of evil, feared by the \Velcadians.
They used dark sorcery liberally and all the time, making them seem inhuman and supernatural monsters.
The Rissitics were not a unified nation, but a loose collection of tribes.
Durcac was the largest, most powerful Rissitic nation, but not the only one, and it controlled less than half of the Rissitic peoples.
They were united by the Rissitic religion, but Rissit/\Secherdamon was not the only Rissitic god, and he did not have complete control of them.
It was \Narkiza/ who united the Rissitic peoples and paved the way for \Secherdamon's planned invasion of \Velcad.
Everyone feared this super\human overlord who had subdued the fearsome Rissitics under his banner. 
He had to first convince everyone that Rissit and the other gods were with him.

The Imetrians were also feared for being dark and inhuman, but they were less menacing and aggressive than the Rissitics, and so the \Velcadians sometimes made deals and alliances with them for mutual protection against the Rissitics.
The Rissitics rarely tried to invade the Imetrium.

Check PROSAs kursus-kalender!
Find ud af, om jeg kan abonnere på den 

Have wing body language in ALL \resphan chapters!



2009-06-12

The Rissitic peoples did not all speak the same tongue.
There was a whole family of Rissitic languages. 
\quo{High Rissitic} was spoken in Durcac.
Some of the tribes spoke languages completely different from and unrelated to Rissitic.

The Rissitics do not have a caste system. 
They are more chaotic and disorganized than that.
Move the caste system to the Imetrians.

In the last many decades before SOM, \Secherdamon had been very much preoccupied with strengthening his \matrix, getting it into top shape before the coming \thirdbanewar.
And preparing \Vizsherioch to become the Dagger. 
He had little time and energy left to devote to political stability, and he realized too late that the Rissitic tribes were badly fractured.
Getting them all back together was a very non-trivial task, even for a god. 
Fortunately he had \Narkiza. 

\Secherdamon did not do all the organization of his followers personally.
He had a group of \quo{angels}, i.e., messengers and administrators and bureaucrats.
\LocarPsyrex was one of them.

I need to rethink the Dark Crescent. 
It was an underground religious organization of mages and barbarians and stuff.
Many minor \rethyaxes and petty mages were secretly part of the Crescent.
Many heathen tribes in \Velcad were affiliated with the Crescent.

\TessHaanith was an immortal messenger like \LocarPsyrex. 
\Psyrex was undead, lich-like or mummy-like. 
His face was horribly scarred and disfigured from millennia of decay.
His frail mortal body could not contain the vast sorcerous power invested in him, so it warped and changed shape.
It was still strong enough to be useful, but not very strong.
\TessHaanith, for example, never walked but was always carried on a palanquin. 

\Criseis felt disgust and sympathy and disbelief whenever she saw \Psyrex.
She herself looked like a \scatha. 
An old, worn-out \scatha, but still a healthy, unbowed, living \scatha.
\Psyrex looked like a monster.
Of course, she had none of his power.
She was a skilled mage, but \Psyrex was a demigod.

A \malgryph (with umlaut over the Y) was a mythological animal that looked like a giant \nycan with feathered wings, a pair of backward-curving horns and the head and tail of a snake.
It was said to be wise, possessing secrets of sorcery.
It was occasionally used in heraldry. 
It was feared as a terribly destructive monster and a bringer of evil omen.
It was an ominous thing to have on your banner.

\WanderersInDarknessEmph spoke of a mysterious pair of entities named \Zaz and \Urzaz. 
It was unclear whether these were \dragons, \xss, cosmic gods or even purely metaphorical entities, personifications of something abstract.
Compare to Gog and Magog from the Bible.

\Urizeth discovers that \Ishnaruchaefir apparently fears, and takes damage from, the \quo{body of \Zaz}. 
A clever reading of the poem reveals that the body of \Zaz is the same as the \quo{blood of the \Chimaera}. 
The \Chimaera is a creature, probably a metaphoric one. 
It is related to and perhaps identical to \Zaz and \Urzaz.
Perhaps it is the union of these two (possibly contrasting entities) that form the \Chimaera (a \chimaera is a crossbreed or mashup or combination). 
The blood of the \Chimaera is a physical substance.
\Urizeth remembers that there exists some practical research regarding this.
It was suspected that the \Chimaera's blood might have interesting arcane uses, so alchemists did a lot of research on its nature and composition and how to reproduce it.
\Urizeth searches in the archives and finds some material about it.
It turns out that the alchemists did indeed succeed in brewing some \Chimaera's blood.
It seemed to satisfy the properties described in the poem, so the alchemists were confident they had the right mixture.
Sadly, they failed to find any use for it, so the research project fizzled and was forgotten.
But now \Urizeth and \Teshrial have rediscovered it, and \Urizeth believes the \Chimaera's blood is vital to defeating \Ishnaruchaefir.
Some \WanderersInDarknessEmph passages describe how the \quo{body of \Zaz} is anathema to him.
\Urizeth believes that if they can brew some \Chimaera's blood and coat \Teshrial's weapons with it, then it can be used as a poison against \Ishnaruchaefir.
But it must be done stealthily. 
\Ishnaruchaefir already suspects what they are up to.
He knows they are researching him and reading \WanderersInDarknessEmph, so he may suspect they have uncovered this secret. 
That might in fact be the very reason why he killed \Urizeth.
So \Teshrial cannot simply walk up to \Ishnaruchaefir and stab him with \Chimaera's blood. 
He would catch on to that and take precautions\dash{}if he has not taken precautions already.
For example, he will likely be warded against bullets and arrows. 
Maybe the \Chimaera's blood can be smeared on a sword, but they are not sure about that. 
The substance seems to be difficult to keep stable.
It is prone to evaporating and melting away if exposed for too long.
But now they have something to work on, and that is good.

\Chimaera's blood -> \Chimaera's ichor

There was a constellation called the \Malgryph.
It had a very obscure meaning in ancient \draconian occultism.
It was never used in the \rethyactic tradition except in the vaguest of references, so \Urizeth-tachi had great difficulty researching it. 
They would have to consult a \dragon or \quiljaar sorcerer to learn what the \Malgryph meant.
(Maybe they tried contacting a \quiljaar, but \Ishnaruchaefir got to him first and coerced him into silence.)
But \Urizeth knows that the stars representing \Zaz and \Urzaz are part of the \Malgryph constellation.
It is possible that the \Malgryph IS the \quo{\Chimaera}.
A \malgryph is, after all, a mix of different beasts and hence a \chimaera of sorts.

There existed spells to conjure forth a \malgryph.
Not a \quo{real} \malgryph, of course. 
It was not a real, existing animal.
But chaos sorcerers could conjure daimonia and temporarily shape them in the form of a ghostly \malgryph that would fight for him for a time.
It was a powerful, deadly spell of alienism.
But it was known only to \dragons and a few other sorcerers.
It was not found in regular \rethyactic textbooks.

Actually, as it turns out, \Teshrial and \Urizeth do not need only the ichor but the \malgryph itself.
They have to research and discover the spell that lets them summon a \malgryph, and then unleash it upon \Ishnaruchaefir.
This is a clever ruse by \Ishnaruchaefir. 
He planted some clues saying that he feared the \malgryph (the \quo{body of \Zaz}), that he quailed before it, that it held the power to cast him down and destroy him.
It was based on some authentic \WanderersInDarknessEmph passages that tell how \Ishnaruchaefir feared \Zaz and \Urzaz and was cursed and cast out by them.
This was based on real events.
\Zaz and \Urzaz were real cosmic gods, albeit highly obscure ones. 
Compare them to Kur'oc and Gul'kor.
There was a time when \Ishnaruchaefir applied to them and tried to pull some favour from them. 
They denied him and punished him, and he was wounded and weakened by their attack.
Compare to "The Hound of Chaos Transcends the Barricades of Z'xulth".
\Ishnaruchaefir made up some more \WanderersInDarknessEmph verses that were very similar to these ones and embellished on them, telling a bogus story about how he was an enemy of \Zaz and \Urzaz and all their being. 
In reality he is an ally of sorts of those cosmic gods, and he can command much of their power.
When \Teshrial starts to conjure the \malgryph, \Ishnaruchaefir acts afraid and tries to stop the summoning.
\Teshrial realizes it will be more difficult than he thought to summon the \malgryph, so he uses his secret weapon and transforms into his monstrous form.
This gives pause to \Ishnaruchaefir, for it is an unexpected move. 
\Teshrial is able to push back \Ishnaruchaefir and overpower him for a while. 
Long enough to buy time for himself to summon his \malgryph.
(Make sure \Teshrial looks heroic and self-sacrificing here. He does not like turning into a monster, but he is noble and selfless and does it anyway. He thinks of his beloved and hopes she will forgive him for the way he has defiled his own body.)
But \Ishnaruchaefir laughs and casts his own spells.
He takes control of the \malgryph and turns it against \Teshrial and his worms.

When \Ishnaruchaefir is just about to kill \Teshrial, he tells him:
"So you sought to use the \malgryph against me, did you? Too late. That would have worked five thousand years ago. But I have grown stronger since then. I have overcome some of my old vulnerabilities."
After winning the battle, \Ishnaruchaefir confesses in private to \Criseis that this was a lie. 
It is \Criseis who pieces together the story and realizes that \Ishnaruchaefir has planted the fake \WanderersInDarknessEmph verses and thus masterminded \Teshrial's quest against him. 
It is she who tells the reader this.
She asks her master if what she suspects is true. 
He refuses to comment, just smiles to himself.

Have \WanderersInDarknessEmph passages at the start of each "part".

SOM book 1 -> Black Stars' Aenigmata
SOM book 2 -> Twilight Angel, Remember

\Urizeth must have a nicely quirky and eccentric personality.
She is a nerd and does not have \Teshrial's social status or social skills.
She is somewhat mad, but she does have a strong will. 
She is older, wiser and more experienced than he and will not let him push her around.
She tries to be a wise mentor to \Teshrial, but does not succeed.
\Teshrial, on the other hand, knows she is a strange nerd.
He fears to have his reputation tainted by associating with her.
She is not only a \thelyad of TiphredSerah, but also very uncool.
\Teshrial is a huge snob, and so is his circle of friends.
He has to remind himself that working with this freak will pay off in the end.
But later in the story, he develops a genuine respect for \Urizeth, and later still even fondness.
She mourns him when he dies.



2009-06-14

The Awakened \resphain convinced (some of) the other \resphan tribes to join them and help them destroy \Merkyrah.

\citebandsong{Nile:BlackSeedsofVengeance}{Nile}{
  Black Seeds of Vengeance
}{
  Dismemberment and slaughter shall you perform on them\\
  The mighty Sekhmet will devour them\\
  The chain of Sut is around their neck, Horus hammereth them\\
  Nepthys hacketh them to bits, the eye of Ra eateth into their faces\\
  Their carcasses will be consumed in the desert\\
  The seed of Amu with perish utterly\\
  Their filth shall never breed among you again\\
  We shall blot out the remembrance of Amalek from under the sky.
}

\Narkiza's threats when he attacks and besieges a city:
\citebandsong{Nile:BlackSeedsofVengeance}{Nile}{
  Defiling the Gates of Ishtar
}{
  Open the Gate that I may enter, open, lest I break down the walls\\
  Open the Gate, lest I cause the dead to outnumber the living\\
  Open the Gate, lest I cause the dead to rise and to devour the living
}

\Narkiza, when he was killed and reborn as an \Ashenoch:
He knew he was chosen by the gods to be someone special.
To be a great conquering hero, the champion of his people.
\citebandsong{Nile:BlackSeedsofVengeance}{Nile}{
  The Black Flame
}{
  Open, For Me the Gates Shall Open\\
  Over the Fire of the Spirit, The Breath Drawn by the Gods. \\
  Arise Apophis Return, That I Might Return, \\
  Borne by the Flame Drawn by the Gods Who Clear the Way that I Might Pass.\\
  The Gods Which Sprang from the Drops of Blood \\
  which Dripped From the Phallus of Set\\
  That I might be Reborn\\
  For I am Khetti Satha Shemsu, Seneh Nekai\\
  And Will Become Set of a Million Years
  
  Akhu Amenti Hekau\\
  I shed My Burnt Skin and am Renewed
}

Read the Papyrus of the Undying.
Read the Book of Am-Tuat.
Read the Book of Gates.
Find out the meaning of Symphony X\dash{}V and Paradise Lost and Dream Theater\dash{}SFAM and other concept albums and epic songs.

The immortality of the Rissitic Liches, including \LocarPsyrex:

\citebandsong{Nile:BlackSeedsofVengeance}{Nile}{
  Chapter for Transforming into a Snake
}{
  I am a Long Lived Snake\\
  I Pass the Night and Am Reborn Every Day\\
  I am the Snale which is in the Limits of the Earth\\
  I Pass the Night and am Reborn, Renewed and Rejuvenated Every Day
}

\Narkiza is a mighty warlord and also a religious holy figure.
He is the ravager of his enemies and the protector of his own.

\citebandsong{Nile:BlackSeedsofVengeance}{Nile}{
  Chapter for Transforming into a Snake
}{
I am a Crocodile immersed in Dread\\
  I am the Crocodile who Takes by Robbery\\
  I am the Great and Mighty Loathsome Reptile\\
  Who is in the Bitter Waters\\
  I am the Lord of those who Bow Down in Sekhem
  (See Karl Sanders' multiple translations of "Sekhem".)
}

\citebandsong{IcedEarth:SomethingWickedThisWayComes}{Iced Earth}{
  The Coming Curse
}{
  Saviour to my own, devil to some
}

Read the original behind "Nas Akhu Khan She En Asbiu": 
"Invocation to those who dwell in the lake of flames", spell 112 of the Sarcophagus Text of Nectanabus.

Ancient ruined monuments, temples and cities:

\citebandsong{Nile:BlackSeedsofVengeance}{Nile}{
  To Dream of Ur
}{
  Desolate and Forsaken \\
  Eerily Moaning Dark Winds\\
  Murmur Incantations\\
  Dusk Calls Forth Shadows

  Spirits of the Glorious Dead \\
  Lingering, Bound to this Place\\
  They Whisper of Untold Sagas, of Long Dead Cities\\
  the Seven Shining Cities Sacred to the Aphkhallu

  Of Ages Past when the World was Young\\
  When Babylon was Blessed of Marduk\\
  and the Sound of her Armies was the Blare of Ominous War Horns\\
  and the Clash of Immortal Cymbals\\
  of Bronze Gates Arrayed in Splendour\\
  and Magnificent Walls of Sunbaked Brick \\
  Of Temples of Marble and Bloodstained Altars

  Long Before the Jeweled Throne of Ur\\
  Fell Silent and Turned to Dust\\
  Beneath the Endless Shifting Sands\\
  and the Inevitable Vengeance of the Elements
}

Even \Jirad Tantor is impressed by the fallen temple of EreshKal and wonders what it was like when it was inhabited and at the height of its power and glory.
Copy the same quote.

Only pureblood \resphain possessed true immortality.
\Ashenbloods died permanently when they were killed.
This means \Achsah must be very brave and convince her fellow \bezedeth to be likewise when they had to go up against \Nzessuacrith.
In \Merkyrah, \bezedeth had an understandable reputation for being cowards, compared to the immortal purebloods. 
They were scorned by the \Merkyran church. 
Being mortal they were condemned to oblivion, while the purebloods would receive the mercy of God and live forever. 
This made them very bitter, so when the rebels came and preached to them the \bezedeth were all too eager to rebel against \Merkyrah.

\citebandsong{Nile:InTheirDarkenesShrines}{Nile}{
  The Blessed Dead
}{
  Looked Down Upong With Scorn\\
  We Work the Fields of the Masters\\
  And Share Not the Bounty of the Black Earth

  Destitute Servile Cast Out\\
  Affording No Tomb\\
  We Shall Be Buried\\
  Unprepared in the Sand

  We Shall Never Be The Blessed Dead

  Scorned By Asar\\
  Condemned at the Weighing of the Heart\\
  We are Exiled from the Netherworld\\
  Serpents fall Upon us Dragging us Away\\
  Ammitt Who Teareth the Wicked to Pieces

  Pale Shades of the UnBlessed Dead\\
  None Shall Enter Without the Knowledge\\
  Of the Magickal Formulas\\
  Which is Given to Few to Possess

  Not for Us to Sekhet Aaru\\
  Our Souls Will be Cut to Pieces with Sharp Knives\\
  Tortured Devoured\\
  Consumed in Everlasting Flames

  We Shall Never Be The Blessed Dead
}



The Rissitics use the practice of cursing people's magical names:

\citebandsong{Nile:InTheirDarkenesShrines}{Nile}{
  Execration Text
}{
  Mut The Dangerous Dead\\
  Trouble me No Longer\\
  I Inscribe Thy Name\\
  I Threaten Thee With The Second Death\\
  I Kill Thy Name\\
  And Thus I Kill Thee Again\\
  In The Afterlife

  Bau Terror of the Living\\
  Angry Spirits of the Condemned Dead\\
  I Write thy Name\\
  I Burn Thy Name In Flames\\
  I Kill Thy Name\\
  And Thus Thee Are Accursed\\
  Even Unto The Underworld

  Mut The Troublesome Dead\\
  Plague Me No Longer\\
  Thou Art Cursed\\
  Thy Name Is Crushed\\
  Thine Clay is Smashed And Broken\\
  Thy Vengeance Against The Living\\
  Shall Come to Naught
}


\Merkyrah had a plethora of false myths and spells meant to protect against evil and invoke God.
The rebels preached that all this was a lie. 
They, on the other hand, had real magic with real power.
\citebandsong{Nile:InTheirDarkenesShrines}{Nile}{
  Kheftiu Asar Butchiu
}{
  The Fire Which Is in the Serpent Khetti\\
  Shall Come Forth\\
  And Blaze Against The Enemies of Osiris\\
  Whosoever Knoweth How to Use\\
  These Words of Power\\
  Against the Serpent Shall Be As One\\
  Who Doth Not Enter Upon His Fiery Path
}


\Narkiza rose to greatness by destroying and devouring his enemies and (with the help of his own gods) even slaying and consuming their spirits and demigods.
He soon became a legend. 
He was maybe 100-150 years old at the time of the Pelidor-Runger war, and he had been steadily gaining strength all his life. 
After his first several acts of heroism he became an \Ashenoch. 
Since then he grew rapidly in power.

\citebandsong{Nile:InTheirDarkenesShrines}{Nile}{
  Unas, Slayer of the Gods
}{
  Poureth Down Water From the Heavens\\
  Tremble the Stars\\
  Quake the Bones of Aker\\
  Those Beneath Take Flight \\
  When They See Unas Rising

  The Akh of Unas Is Behind Him\\
  The Conquerer Are Beneath His Feet\\
  His Gods Are In Him\\
  His Uraei Are on His Brow\\
  The Words of Unas Protect Him

  Unas This Bull of The Heavens\\
  That Trusteth With His Will\\
  Living On Utterances of Fire \\
  From the Lake of Flame\\
  Unas That Devoureth Men and Liveth on The Gods
}

His own gods were with him.
With their power he cast down and ate the enemy gods. 
This was part of \Secherdamon's plan: 
To punish those gods who rebelled against him, and to invest his chosen champion with their power.

\citebandsong{Nile:InTheirDarkenesShrines}{Nile}{
  Unas, Slayer of the Gods
}{
  Behold Amkebu Hath Snared Them for Unas\\
  Behold Techer Tep F Hath Known Them and Driven Them Unto Unas\\
  Behold Her Tbertu Hath Bound Them\\
  Behold Khensu The Slaughterer of Lords\\
  Hath Cut Their Throats for Unas\\
  Behold Shesemu Hath Cut Them Up For Unas
}


\Secherdamon himself rose to power that way.
Not by eating his enemies directly, but by sacrificing them to the \xss and being granted power in return.
This amounted to the same thing, allowing him to dodge draconic taboos.
\Secherdamon remembered this similarity when he sponsored \Narkiza.

\citebandsong{Nile:InTheirDarkenesShrines}{Nile}{
  Unas, Slayer of the Gods
}{
  Unas Hath Ingested Their Spirits\\
  Hath Feasted On Their Immortality\\
  He Hath Consumed their Shadows\\
  Unas The Slayer of the Gods\\

  Unas The Sekhem Great\\
  The Sekhem of the Sekhemu\\
  Unas The Ashem Great\\
  The Ashem of the Ashemu\\
  Behold Orion\\
  Unas Riseth
}


\Narkiza -> \Narkiza



The \malachim became powerful by eating their fellow \resphain.
In this they remained true to the spirit and purpose of the \resphan race. 
More so than their enemies, they boasted.

\citebandsong{Nile:InTheirDarkenesShrines}{Nile}{
  Unas, Slayer of the Gods
}{
  Unas Hath Taken Possession of the Hearts of the Gods\\
  Unas Feedeth on their Entrails\\
  He Hath gorged on their Unuttered Sacred Words\\
  He Hath Assimilated the Wisdom of the Gods\\
  His Existence is Everlasting

  Behold The Souls of the Gods are in Unas\\
  Their Spirits are In Unas\\
  The Flame of Unas in Their Bones\\
  Their Shadows are With their Forms
  
  Unas is Rising\\
  Hidden, Hidden
}


Read "The Chapter for bringing Heka to those who Burn" (part of "The Book of Resurrecting Apophis", basis for "Churning the Maelstrom"). 



\Vizsherioch revels in his megalomania and builds a cult around himself.
This picks up speed after \Secherdamon dies, because \Vizsherioch now knows the responsibility lies with him alone.

\citebandsong{Nile:InTheirDarkenesShrines}{Nile}{
  Churning the Maelstrom
}{
  I Am the Uncreated God\\
  Before Me The Dwellers in Chaos are Dogs\\
  Their Masters Merely Wolves\\
  I Gather The Power\\
  From Every Place\\
  From Every Person\\
  Faster Than Light Itself
}


After Ramiel has regained his full \malach powers and become Overlord, he lets himself hail as the saviour and champion of the \resphan race. 
He has mastered his \carcer and the dark powers of the \banes.
He is the ideal of all \resphain who crave power and greatness and perfection.

The new \Thanatzil, even.

\citebandsong{Nile:InTheirDarkenesShrines}{Nile}{
  Churning the Maelstrom
}{
  Hail To He Who Is In The Duat, Who Is Strong\\
  Even Before The Servants of Serpents\\
  He Gathers The Power From Every Pit of Torment\\
  From They Who Hath Burnt in Flames\\
  From Words of Power Uttered By the Darkness Itself

  Hail To He in The Pit, Who Is Strong\\
  Even Before the Terrors of The Abyss\\
  Who Gathers The Power from the Wailing And Lamentations\\
  Of The Shades Chained Therein\\
  From He Who Createth Gods \\
  From The Silence Alone
}


\Ishnaruchaefir initially believes \Secherdamon's plan takes place in Malcur.
Then he falls for the \Forklin decoy when it is unveiled. 
He is impressed to learn that both were decoys. 

The \nephil necromancer-kings at \Semiza's time derived much knowledge and power from the dead.

\citebandsong{Nile:InTheirDarkenesShrines}{Nile}{
  I Whisper in the Ear of the Dead
}{
  I Dream of the Dead\\
  And Their Shades Showeth Me \\
  Visions Which No Living Man Can Know\\
  I Whisper in the Ear of the Dead\\
  And Mine is the UnWritten Knowledge\\
  That Lieth Under The Black Earth
}


Read "The River God" by Wilbur Smith.

When \Thanatzil fell, \Semiza was not just buried in the earth.
He fled underground into the crypt where mummies were kept.
Here he was entombed.
Maybe he even had a little time to instruct his servants to place him in a sarcophagus. 
Or maybe he crawled into a sarcophagus to seek shelter when the place was collapsing around his ears. 
He became trapped there.
He thirsted and hungered and his body wasted away.
He had no strength to free himself, and also realized that it would be pointless even if he could.
But he still retained his mental powers.
As a necromancer, he had great skill and experience in the art of sending his mind out to wander the spirit worlds.
So he was able to maintain some contact with \Daggerrain and with the \resphain.
He remained in his tomb for over a thousand years, until \Damiarch-tachi found his now-undead remains and freed him.

\citebandsong{Nile:InTheirDarkenesShrines}{Nile}{
  Hall of Saurian Entombment
}{
  Through Subterranean Labyrinths of Catacombs\\
  We Hath Crawled To Gather in this Dimly Lit Hall Of Colossal Proportion\\
  Which Few Ever See\\
  Along Black Walls\\
  Rise Tier after Tier of Carven Painted Sacrophagi\\
  Each Standing in a Niche in the Stone\\
  The Mounted Tiers Rising Up To Be Lost in the Gloom Above\\
  Thousands of Carven Masks Stare Down Upon Us\\
  We Who are Rendered Futile and Insignificant\\
  By This Vast Array of the Dead
}


\Sethicus had imperial ambitions back in the day, when he founded the \dragons.

\citebandsong{Nile:InTheirDarkenesShrines}{Nile}{
  Invocation to Seditious Heresy
}{
  And Here I Stand\\
  I who would be master of the Black Earth\\
  Have summoned you here secretly\\
  You who are faithful to me\\
  To share in the Black Kingdom that shall be\\
  Tonight we shall witness\\
  The breaking of the chains which Enslave us\\
  And the birth of a Dark Empire
}


\Sethicus fell and was seemingly destroyed in the \firstbanewar.
He overloaded his body with power and could not resurrect.
He was believed lost by all his followers.
Out of respect for his heroism they buried him in a great tomb in \Dathka. 
But unbeknownst to all, his spirit lived on, albeit in a shattered, weakened state.
A bit like the Emperor of Mankind in Warhammer 40,000.

\Xserasshana wanted to resurrect the secrets that \Sethicus had discovered.
So she found his tomb in \Dathka and tried to resurrect him.

\citebandsong{Nile:InTheirDarkenesShrines}{Nile}{
  Invocation to Seditious Heresy
}{
  And Here I Stand\\
  I who would be master of the Black Earth\\
  Have summoned you here secretly\\
  You who are faithful to me\\
  To share in the Black Kingdom that shall be\\
  Tonight we shall witness\\
  The breaking of the chains which Enslave us\\
  And the birth of a Dark Empire

  Who am I to know what powers lurk and and Dream\\
  in these murky Tombs\\
  They hold secrets forgotten for three thousand years\\
  But I shall Learn They shall teach me\\
  See how they sleep staring through their Carven Masks\\
  Priests Monks Acolytes \\
  Kheri Heb Rekbi Khet

  The Mummified Remains of the Sacrificial Whores\\
  of The Cannibalistic Serpent Cults of Thirty Centuries \\
  With Black Incantation and Foul Necromantic Art\\
  Propitiated with the Blood of the Living

  We will waken them from their long Slumber\\
  The Ancients knew Nay Commanded the Words of Power\\
  And shall teach them to Me\\
  I shall restore them to Life\\
  To Labour for my own Dark Imperial Desires\\
  I will Waken Them Will Rouse Them\\
  Will learn their forgotten Wisdom\\
  The knowledge locked in those withered Skulls\\
  By the Lore of The Dead\\
  We shall Enslave the Living

  Pharaohs and Priests long Forgotten\\
  Shall be our Warriors and Slaves\\
  Who will Dare to Oppose Us\\
  Out of the Dust shall Avaris Rise
}

When a \dragon (such as \Ishnaruchaefir or \Nzessuacrith) kills, he does it by invoking \xss and channelling their destructive power.

\citebandsong{Nile:InTheirDarkenesShrines}{Nile}{
  Destruction of the Temple of the Enemies of Ra
}{
  The Fire of the Eye of Horus is Upon You\\
  Searching You Consuming You\\
  Setting you on Fire Burning you To Ashes

  Unemi The Devouring Flame Consumes You\\
  Sekhmet The Blasting Immolation of the Desert Maketh an End of You\\
  Xul Ur adjugeth you to Destruction\\
  Flame Fire Conflagration Pulverize You

  Your Souls Shades Bodies and Lives Shall Never Rise Up Again\\
  Your Heads Shall Never Rejoin your Bodies\\
  Even The Words of Power of The God Thoth\\
  The Lord of Spells\\
  Shall Never Enable you to Rise Again
}


\Draconic/\ophidian coldness:

\citebandsong{KarlSanders:SaurianMeditation}{Karl Sanders}{
  Dreaming Through the Eyes of Serpents
}{
  After many trips with my son to the zoo, I noticed that crocodiles, monitors, snakes\dash{}and pretty much all the Reptiles\dash{}have this way of sometimes staying completely motionless when they want to with this deep, cold, timelessly penetrating glare. 
  It struck me as something like a trance state. It caused me to wonder what in the world would a snake meditate upon...
}



2009-06-15 

\Nexagglachel managed to destroy some of the would-be \satharioth.
He let them drain some of his essence, but also sent a portion of his will and malevolence into their bodies.
(A larger portion than into the others, that is.)
This way he was able to destroy them from within, causing them to tear themselves apart in pain, frenzy and madness.

\citebandsong{Nile:AnnihilationoftheWicked}{Nile}{
  Cast Down the Heretic
}{
  Horus Repulseth Thy Crocodile.\\
  Sut Defileth Thy Tomb.\\
  Nephthys Hacketh Thee in Pieces.\\
  The Sons of Horus Speareth Thee.\\
  The Gods Repulse Thee.\\
  The Flame of Their Fire is Against Thee.\\
  Cursed Art Thou, Impaled Thou Art, Flayed Art Thou.\\
  Heretic, Thou Art Cast Down.

  Blows are Rained upon Thee.\\
  Dismemberment and Slaughter are on Thee.\\
  Thy Crocodile is Trampled under Foot.\\
  Thy Soul is Wrenched from its Shade.\\
  Thy Name is Erased. Thy Spells are Impotent.\\
  Nevermore Shalt Thou Emerge from Thy Den.\\
  Thy City Armana Lays in Ruin.\\
  Damned Art Thy Accursed Soul and Shadow.\\
  Die O One, which Art Consumed.\\
  Thy Name is Buried in Oblivion.\\
  Silence Covereth Thee and Thy False One.\\
  Down upon Thy Belly.\\
  Be Drowned, Be Drowned, Be Vomited Upon.

  Khnemu Draggeth Thy Spawn to the Block of Slaughter.\\
  Sick Shalt Thou be at the Mention of Thine Own Name.\\
  Sekhmet Teareth Out Thy Bowels and Casteth Them into Flames.\\
  She Filleth Thine Orifices with Fire.\\
  Uadjit Shutteth Thee in the Pits of Burning.\\
  Nevermore Shall You Breathe or Procreate.\\
  Neither Thy House or Tomb Exist.\\
  Thou Shalt Drive Thy Teeth into Thine Own Body.\\
  Heretic, Thou Art Cast Down.\\
  Overthrown, Ended, Hacked in Pieces, Slaughtered, Butchered.\\
  Ra Hath Made Thoth to Slay Thee Utterly.
}


One of the things the \dragons lost after the \firstbanewar was the secret of immortality.
This was what \Xserasshana brought back to them from \Sethicus's tomb.
Alternately, maybe immortality was the gift of \Semiza and \Daggerrain to the rebel \resphain.
(But maybe not. Then it would be harder to convince the \ashenbloods to join them.)

\citebandsong{Nile:AnnihilationoftheWicked}{Nile}{
  Sacrifice Unto Sebek
}{
  Lord of the Temple of the Mount of Sunrise.\\
  Open the Way unto the Underworld.\\
  Cause the Dead to Rise to New Life.\\
  Bring the Child Horus upon the Throne of Osiris.
}

Praise and invocation of the \xs, such as \NaathKurRamalech:

\citebandsong{Nile:AnnihilationoftheWicked}{Nile}{
  Chapter of Obeisance Before Giving Breath to the Inert One in the Presence of the Crescent-Shaped Horns
}{
  Khensu Neter Hetef,\\
  who Possesseth Absolute Dominion \\
  over the Evil Spirits that Infest the Earth and Sky.\\
  He of the Silence of the Moon.\\
  Giver of Oracles. He that Must Forever Wax and Wane.\\
  Thou Art in Union with Thoth,\\
  the Excellent Tehuti of Truth and Time.\\
  Keeper of the Lunar Cycle,\\
  whose Hands are Able,\\
  whose Tongue is Mighty in Speech.\\
  Author of the Works of Knowledge.\\
  Writer of the Ancient Wisdom.\\
  Master of the Words of Power. 

  I am He Who Calleth Down Curses \\
  and Commandeth the Elements unto Darkness.\\
  I Hath Uttered the Hidden Words.\\
  Even unto the Divine Words which Art Written in the Book of Thoth.
}

Curses as a \dragon destroys his enemies in the names of the \xss:

\citebandsong{Nile:AnnihilationoftheWicked}{Nile}{
  Lashed to the Slave Stick
}{
  Ra Pronounceth the Formulae Against Thee.\\
  The Eye of Horus is Prepared to Attack Thee.\\
  Sekhmet Uttereth Words of Flame Against Thee \\
  and Pierceth Thy Breast.

  Abui, The Gods Who Burns the Dead.\\
  Shall Leave You Smoldering in Exile from the Netherworld.\\
  Abati, the Gorer, causes You to Howl like a Jackal in Anguish.
}

Curses when the \resphan rebels devour the souls of their enemies:

\citebandsong{Nile:AnnihilationoftheWicked}{Nile}{
  Lashed to the Slave Stick
}{
  Your Corruptible Bodies Shall be Cut to Pieces.
  Your Souls Shall have No Existence.
  Ye Shall Never Again See Ra as He Journeyeth in the Hidden Land.
  The Doom of Ra is upon You.
}

\Semiza's garden and tomb were located in a region of darkness, monster-haunted and feared by all \resphain.
\Damiarch-tachi were very brave to venture there.
They believed they would have to brave this evil wasteland in order to solve the mystery of the \umbrae and enable their people to defend themselves from this menace.

\citebandsong{Nile:AnnihilationoftheWicked}{Nile}{
  Annihilation of the Wicked
}{
  The Dominion of Seker.\\
  Barren Desert of Eternal Night.\\
  Shunned by Ra.\\
  Behind the Gate Aha-Neteru.\\
  The Wastelands of Seker.\\
  Eldest Lord of Impenetrable Blackness.\\
  Death God of Memphis.\\
  He of the Darkness and Decay of the Tomb.\\
  He of Rosetau, the Mouth of the Passage to the Underworld.\\
  Closely Guarded by Terrible Serpents\\
  who Careth Not for His Own Cult of Worshippers.
}

There they find the tomb of \Semiza.

\citebandsong{Nile:AnnihilationoftheWicked}{Nile}{
  Annihilation of the Wicked
}{
  Seker, Ancient and Dead,\\
  Primeval Master of the World Below,\\
  Remaineth Unwitnessed, Unseen.\\
  Hidden in His Secret Chamber.\\
  His Primitive Graven Image like as a Hawk-headed Man.\\
  Shrouded and Swathed in Tomb Wrappings.\\
  Standing Between a Pair of Wings \\
  which Issue Forth from the Back of a Monstrous Serpent,\\
  Having Two Heads, Having Two Necks \\
  and Whose Tail Terminates in a \human Skull.
}

Likewise \Sethicus's tomb in \Dathka. Which later also becomes \Xserasshana's tomb.

The aloof, retired \dragons still watch \Miith from the shadows:

\citebandsong{Nile:AnnihilationoftheWicked}{Nile}{
  Von Unausspechlichen Kulten
}{
  I Hath Dreamed Black and Grim, Desolate Visions\\
  of the Pre-Human Serpent Folk \\
  and Communed with Long-dead Reptiles.\\
  Silently Watching Through the Ages in Cold, Curious Apathy.\\
  The Unending Sorrows and Suffering of an Abysmal \humankind.
}

After the escape from Malcur, Rian and Neina are forever scarred from witnessing their home transforming into a nightmarish hell from within.

\citebandsong{Nile:AnnihilationoftheWicked}{Nile}{
  Von Unausspechlichen Kulten
}{
  I Dare Not Again Surrender to the Deep Sleep Which Ever Beckons Me.\\
  Lest I in Dread Shudder at the Nameless Things.\\
  That May at this Very Moment Be Crawling and Lurking.\\
  At the Slimy Edges of My Conciousness.\\
  Slithering Forth from the Bowels of Their Infernal Pits.\\
  Worshipping Their Ancient Stone Idols and \\
  Carving Their Own Detestable Likenesses \\
  On Subterranean Obelisks of Blood-soaked Granite.
}

As SOM progresses more and more people begin to suspect that the end is nigh.
The Dark Crescent actively prophesy the end and rejoice in it.

\citebandsong{Nile:AnnihilationoftheWicked}{Nile}{
  Von Unausspechlichen Kulten
}{
  I Await the Day When the Claws of Doom Shall Rise.\\
  To Drag Down in Their Reeking Talons \\
  the Weary and Hopeless Remnants of a Jaded, Decayed, War-despairing Mankind.\\
  Of a Day When the Earth Shall Open Wide \\
  and the Black, Bottomless, Yawning Abyss 
  Engulfs the Arrogant Civilizations of Man.\\
  Chthonic Retribution Shall Ascend Amidst Universal Pandemonium \\
  and Those Who Slither and Crawl Shall Rise Again \\
  Once More to Inherit the Earth.
}

Even in the \Human Age and the \Scatha Age, the ancient elder peoples still rule \Miith:

\citebandsong{Nile:AnnihilationoftheWicked}{Nile}{
  Sss'Haa Set Yoth
}{
  Lurking Among Us Hidden in Obscurity,\\
  Descended from the Dawn of the Ages,\\
  The Children of Yig And Set, Serpent Volk.\\
  Whose Civilaztions were Ancient, Long Dead And Forgotten\\
  Before The Eldest Pyramids were Built.\\
  Man Was Not the First to Walk Upright Upon the Earth.

  Cruel, Pitiless, of a Cunning Mind.\\
  Unwilling to Relinguish Dominion of the Earth.\\
  They Are Cold, Knowing No Remorse.\\
  They hath Made UnWitted Slaves of a Blinded Humanity.
}

The Shroud and the Unspoken Covenant were great tools for keeping mortals enslaved. 
Both sides of the \Feud benefited from this.
Mortals turned out to be more useful than the immortals had initially thought.
Back in the day when there were many immortals, they relied on brute force.
In the Age of the Shroud, the immortals relied on stealth, cunning, science and long-term planning to gain Ascendancy or their \matrices.

\citebandsong{Nile:AnnihilationoftheWicked}{Nile}{
  Sss'Haa Set Yoth
}{
  And thus they keep us Ever Subjugated, Hopeless, Fearful\\
  And despairing in the Deep Hidden Abysses of our Souls.\\
  Helpless, Unable to Escape the Unending Quiet Desperation.\\
  We have Been Broken down\\
  Conditioned to Accept Unconscious Slavery.
}

\Secherdamon was terrible to look upon. 
Even more so than \Ishnaruchaefir.
\Secherdamon had gathered power and worked hard to make himself imposing and regal and dominant and godlike.

\citebandsong{Nile:Ithyphallic}{Nile}{
  As He Creates So He Destroys
}{
  No living creature can look upon his face\\
  And endure its terrible heat and black radiance\\
  That is like the reverberating unseen rays of molten iron\\
  Which strike and burn the skin of those who would dare\\
  Gaze into the countenance of the idiot god
}

\Secherdamon, during his rise to power, called down curses and spells of destruction upon his enemies.
The \malachim did likewise during their (ultimately foiled) rise to power. 

\citebandsong{Nile:Ithyphallic}{Nile}{
  Ithyphallic
}{
  Let the shades of my fathers curse their faces\\
  Let the eye of Sekhmet\\
  Send the violence of the sun down upon their heads\\
  Let searing torrents of fire descend upon their brow\\
  Let flames immolate their places of sleeping

  Let the eye of Sekhmet\\
  cause their hearts to burst into flames

  Let my curses be heard\\
  Let my will be as Menthu the bull\\
  Potent to create\\
  And savage to slay those whom I hath cursed\\
  Let my wrath be terrible\\
  And my vengeance unmerciful
}


2009-06-17

\Narkiza came from a tribe in the southern Durcac.
He was ethnically Durcaci and spoke High Rissitic, but his tribe was more savage and brutal than most. 
He went through some measure of Training From Hell in his youth.
He worked very hard to become strong, and to be a leader.
That was what enabled him to become a hero while still mortal.
Later he was recognized by the church as something special and turned into an \Ashenoch.

Amarok: Complain about double-click feature

Read "N-body" report!

Telcastora Ilcas is not much taller than average.
Muscular and slightly heavy of build, but not so much.
\Narkiza is a very large dax.
Sethgal is taller than average, but lean. 

Sethgal's sword has a scabbard specially shaped to produce a nice, impressive "shing" sound when drawn.
He uses it in the "War is Coming" scene, to look impressive to his troops. 

\Urizeth is not a Cabalist.
Mention it in her introduction.

The Shrouded Realms where mortals live need to be more mystical, more supernatural, more sinister.
Malcur has to be more ghost-infested.
The menace of monsters from the \wylde must be greater, more obvious.
Perhaps there is an inner city, encircled by walls, which is mostly safe from monsters, and an outer city outside the walls, with only walls and \eidola for protection.
The slums are, of course, outside.
The inner city is very densely populated and crowded, since everyone who can afford it wants to live there.
So there are also slum-like places in there. 
And the dead garden is a small patch of \wylde in the middle of the city, which everyone tries to forget about (because otherwise they would be living in fear of it all the time).
For inspiration on supernatural things, consider The Deepgate Codex.

\Uswa is in bad shape due to malnourishment and hunger.
When Rian comes seeking her advice, he brings food.
\Criseis gives \Uswa some coins in sympathy.

As Rian's hunt goes on, he becomes hungry and his health suffers.
His life is hard enough to begin with, and with his hunt for Neina costs him much valuable time and energy, leaving him hungry and tired. 
When Moro \Cornel hires Rian, she feeds him.
He is very grateful for that. 
Rarely if ever has he had such good food. 

When Moro springs Rian from prison, he is worried: "Won't they come after me again?"
Moro: "No, don't worry. I will make you disappear in the paperwork."
Rian is scared. He does not know what paperwork means, but he associates writing with mages, so he assumes "vanish in the paperwork" is some magic. 
He is not sure he wants anything to do with that.

Have a scene in the beginning with Carzain as a mercenary.
He has been hired by a village or somesuch to fight a gang of bandits. 
There are maybe 10 bandits or so. 
Maybe one or a few leader bandits are big-ass motherfuckers with \nephil blood.
Half-ogres.
They are encamped in some spooky, possibly haunted place, like a tower ruin or a cemetery.
Carzain plays on their fears and perhaps even utilizes the magic of the place directly. 
First he hides in the woods around them and picks them off one by one.
Then he takes on the entire rest of the group singlehandedly.
Make Carzain something of a Batman type character who strikes from the shadows and uses stealth, cunning, technology and brute force.
He has to do some really impressive, flashy things that scare the shit out of the villains and impress the reader. 
Such as necromancy.
Don't make him toss fireballs.
Make his magic more sinister, like ripping people's hearts out of their bodies from a distance, or summoning spirits or ghosts to fight for him.

Maybe \Teshrial should wear glass armour in WSB. 
And in the duel. 
And the other fighting \resphain too.



2009-06-20 

After Razor sees stuff in Malcur, the scene ends.
It ends with Razor being horrified, as if waking from a nightmare. 
He tears himself away from the gruesome vision.
Later, after they have left Malcur, Razor thinks.
He thinks back to how he summoned the other \nycans, but the thing was gone when they got there.
The other \nycans were skeptical.
Countess tried to comfort him, assuring him everything would be all right.
Razor does not believe her. 
She does not know what she is talking about. 
The chapter ends with him resolving to talk to Ilcas about it.
Ilcas is wise. 
He will understand and believe him.



2009-06-23

When \Sethicus died, his occult mastery was so great and his will so powerful that his soul did not perish, but transcended from \Miith and became a disembodied god.
According to legend, he went on to ride with the \xss in eternity.
He could still be contacted to some extent via his mummified body.
His wraith maintained some fetters to the body.
Legend also said that \Xserasshana did likewise after her death, although that was more dubious.

\Dragons swore by or even prayed to the "dead gods": 
\Sethicus, \Xserasshana and others.

Rename first book to "Black Stars' Enigma"
Second book: TAR

Remodel the \umbrae after the wallpaper I have of a giant manta-like monster. 
\umbrae are huge and deadly, and even a few of them are a threat to a \dragon.

The Exile fears the "body of \Zaz" and "that which issueth forth from \Urzaz".
\Urizeth had long had trouble interpreting the latter.
It could be the very "being" or "aura" of \Urzaz, or it could be his breath or something else that emanates from him.
But now that \Teshrial specifically asks him to look into the problem, she remembers some old research she has done.
It did not lead to much back then, but now she digs it up and looks at it with renewed motivation.
She had been trying to nature of the mysterious \Zaz and \Urzaz, and had an idea they were connected with the \Chimaera.
But \WanderersInDarknessEmph is a huge poem, and she had not been looking in the right places.
Now her attention is turned towards the parts that deal with the Exile, and here there are some very clear indications that (once you thoroughly interpret them) strongly suggest that \Zaz and \Urzaz more or less ARE the \Chimaera.
These clues, though, are planted by \Ishnaruchaefir.
In reality, the nature of \Zaz and \Urzaz is more complex and ambiguous than that.



2009-06-24

Flying magic is difficult to learn and maintain, for mortals and immortals alike.
Most mages can manage only powered leaps.
Levitation is very hard.
\Achsah should not fly in WSB.

Green eyes were considered the hallmark of Geican \humans.
They were still rare among Geicans, but much more common than among \humans as a whole.
And a number of prominent figures in Geican history had those eyes, including \Belzir.
Shereid also had them.

The \resphain, to some extent, enjoyed anger and pain.
It was an affirmation that they were alive and active.
Pain reminded them of their bravery and daring.
Anger reminded them of their will, their goals and motivation.
Make this clear:
- Where \Teshrial revives
- Whenever \Teshrial remembers his defeat 
- Where \Urizeth revives

Consider the issue of \dragonic \quo{forgetfulness}. 

\Nzessuacrith was a master of stealth.
She had studied it idly even before the \secondbanewar, and after that she had spent thousands of years perfecting it. 
She could hide herself better than any other immortal, or so she claimed.
She thinks this to herself in the scene where Carzain spies on the Rungeran army.
Because she is so stealthy, she is also very skilled at detecting others.
So she can detect that a very powefrful mage is nearby.
(He is well hidden by his Kenosis, but some \vertex-ness shines through.)
So maybe she sends some folks out to kill him, and then she muses to herself when he fights them off.
Or maybe not.

When people use an overdose of magical power, their bodies suffer for it.
They get pain, bleedings\dash{}inner or outer\dash{}and scars; signs of disease and decrepitude.
They go around in pain and cough up blood.
This also happens to immortals (although they can heal more wounds than mortals can).
When \Ishnaruchaefir is in his Nadir, he has bleeding wounds that spring open on their own, because of the power he has to expend in his weakened state.
In his battle with \Teshrial, every spell he casts causes more wounds to spring open on him\dash{}he pays for his magic with blood and pain.
This also happens to Carzain and Curwen in the heat of battle.
Worst of all it happens to the Rungeran mages\dash{}all but Takestsha.



2006-06-26

The \resphain used ships a lot.
Sailors were often \bezedeth.
Purebloods had great pride in their wings, so they did not like to admit the fact that they were dependent on ships, and so they did not become professional sailors.
The \dragons depended less on ships because \dragons were great swimmers, so if they tired of flying they could just swim.

\Nasshikerr tells Moro:
"Contact me again in no less than 15 days, and bring me a sacrifice. Then I will tell you more."
When the time is up, Moro goes out into the city.
She finds two \human bandits mugging and beating up a defenseless \scatha.
"Perfect", she thinks.
She stands some way away, hiding, and casts her spells.
First she casts a dozing spell. 
It is slow to cast and slow to take effect, so it is not useful in combat (unless you are more skilled at it than Moro is), but it is useful here.
It makes all three of them sleepy and dull.
After 30 seconds' time, the two bandits are having trouble standing up.
The victim, who was already badly beaten up, has apparently fallen unconscious.
"How fortunate. One less spectator to worry about."
Then she moves in to attack. 
She casts some more spells to hurt, wound and cripple the two thugs.
She curses at them to their faces, tells them how evil they are, how they are \human filth, and how she will make them suffer for it.
She thinks: 
"I can only transport one. I have to get rid of the other. Should I kill him? I probably should. I don't want him spreading rumours about me."
She casts another glance at the victim. Still unconscious.
"But a quick, painless death is too nice for such a piece of shit."
She batters them some more with her spells.
She picks one to keep for a sacrifice and decides to kill the other.
She takes the one to kill and severs all his legs and arms and lets him bleed to death.
She asks the other: "Do you surrender?"
They are cowardly shits and scared shitless, so they do.
Now that he has surrendered, she can cast a weak mind control spell.
Now, with a homunculus possessing the surviving thug, she can (for a short period of time) manipulate him like a zombie and force him to move as she wills.
It is difficult work.
He moves slowly and awkwardly and bumps into things, and she has to maintain the spell.
Finally she gets him back to her secret entrance and into the castle with her, into her secret chambers.
There she ties him up securely.
She uses more magic to break his legs and arms so he can't fight even if he should break free.
Through the deal (in her chambers, when he is immobilized and free of her mind control) she talks to him and tells him how much she hates him and how he will suffer for his evil.
He weeps and whimpers and grovels and begs for mercy.
She tells him: "Did you show mercy to that \scatha earlier today when he begged? No, I did not think so. Spare me your hypocrisy, you shit. You deserve a worse fate than this. Then again, I strictly do not know what will happen to your soul when my god \Nasshikerr comes to claim it. Maybe you will be destroyed entirely and sent to oblivion. Maybe you will be tortured forever in some hell. Maybe you will be enslaved and made to serve \Nasshikerr as some horrid lemure."
(Lemures are things in mythology and metaphysics. Moro is not sure whether they exist and, if so, exactly what manner of things they are.)
At last she calls upon \Nasshikerr and kills the prisoner.
\Nasshikerr then tells her what he knows.

Curwen kind of likes \Tiroco for some reason. 
He treats he with a kind of friendly condescension.
When she was a child he was occasionally nice to her and did her favours, like showing her impressive stuff he could do with his magic. 
She remembers him as an uncle-kind-of-character.
Nice, but also scary.
He still sees her as a bit of a child, even as \rinyuth.

\Teshrial transforms into \neoresphan form.
Then, using his newly enhanced vampire powers, he absorbs the bodies of the living and dead \ghobaleth in order to grow to huge size.

\Menessiaraid is horrified by the sight of \Teshrial's new form, but also impressed.
He sees him as an awesome, magnificent angel, both terrible and beautiful.
A vision of the greatness, potential and future of the \resphan race.
Even after \Teshrial is dead and \Menessiaraid mourns him, he still goes away awestruck at his friend's courage and greatness, and with high hopes for the future and the Quest for Perfection.

The \jinn were a race of immortals.
They predated \humans and lived, among other things, in the lands southeast of Durcac. 
\Secherdamon's cult encountered the \jinn and the humanoids who served and worshipped them, and merged with them.
He waged wars with the \jinn, but eventually struck alliances with many of them, and eventually came to rule over them.
The lords of the \jinn and the greatest among them were the ifrits.
Still, some \jinn opposed Nechsain and would not serve him. 
These \quo{evil} \jinn were horrors that haunted the desert, feared by all.

\Ophanim were creatures that lived in \Nyx.
They were actually creations of the \banes that had migrated to \Nyx.
Here they came to be enslaved by the \resphain.
They were living chariots with wheels and lots of eyes.



2006-06-27

The first \dragons appeared more than a million years before the \banewars, pioneered by \Sethicus/S\Sethicus/\Sethicuss.
They were the first True Immortals.
They learned many things from the \xss, but they were not dependent on the \xss for their sorcery.
They learned immortality though a pact with \KhothSell.
But they did not worship the \xss.
They retained the atheism of the \ophidians.
They started a war that devastated the \ophidian civilization.
In the end the \dragons were defeated and imprisoned.
They lay dormant for a million years while the \ophidian civilization recovered around them.

\Xserasshana was one of these Elder \Dragons/Primordials. 
She vaguely remembered the time before the \firstbanewar, but the \banes' radiation had destroyed much of her mind.
It also left her somewhat mad.
She became obsessed with restoring her people.
She turned to the \xss for help.
Through the \xss the \dragons regained much, but not everything.
A true \dragon, with all its natural power unlocked, is greater than a \xs-worshipping one.

Add more Cosmic Horror to the \dragons.
Read "Iron Kingdoms" for Cosmic Horror \dragons.
See also TV Tropes: "Our \Dragons Are Different"

Have more Cosmic Horror surrounding \Ishnaruchaefir.
The Mirage Asylum should be darker and more horrible.
\Criseis should also be surrounded in more dark magic.
\Criseis does not stab the thug.
She uses mind control to make him stab himself.
(Her mind control only works on weak minds.)
Portray how the bandit struggles against the spell.
Describe his dread and horror as he is forced to kill himself.
She uses her uuuuber-sensitivity to read his mind, which is a quite rare feat, impossible to most immortals.

It is known (in myth at least) that \draconian blood conveys immortality.
Rian thinks this in WSB.
Everyone knows that tidbit.
Also mention that \Criseis is immortal because she drinks her master's blood. 
(It is not necessarily the nice kind of immortality, though. It can be undeath like \Psyrex's.)

When \Ishnaruchaefir moves in to kill Rian in WSB, he does not move.
He just raises a hand and is about to cast a spell.
\Criseis can see the \daemons moving in to answer her master's call, so she acts quickly.

Also, Rian is more horrified and less fascinated.
He is scared shitless by \Ishnaruchaefir.
He is certain \Teshrial is the good guy and \Ishnaruchaefir is a wicked, evil monster.

Get rid of the \ghobal in WSB.
\Ishnaruchaefir instead does something that threatens the \resphain's citadel or otherwise endangers their interests in Malcur.
Maybe he only passes through the dead garden and ends up fighting \Teshrial some other place.
Probably get rid of the long walk through the centre of the city.
But \Ishnaruchaefir's attack against something is actually a decoy.
While he is doing it, \Criseis digs deep with her aethereal senses.
She reaches down deep below Malcur, in the planes close to \Nyx, and she detects the \ghobaleth.
\Criseis is his secret weapon. 
She is super-sensitive and can detect things that are supposed to be hidden from everyone.
That is her greatest skill, which she has spent millennia honing.
But it is fairly unknown, because \Criseis has rarely been a visible player in any major events.
\Ishnaruchaefir is infamous mostly for his deeds during the \secondbanewar.
That was enough to make him one of the famous and feared \dragons on \Miith (rivalled only by \Secherdamon), but since then he has done little of note, and \Criseis has done even less.
It is known that she acts as his spy, but no one knows just how skilled a spy and telepath she is.
Her skill also makes her a superb negotiator and manipulator.
She has built up an impressive spy network singlehandedly.
At the end of WSB, have a super-short scene where \Ishnaruchaefir tells her to describe to him what she has learned.
She begins to tell him something very interesting.
A surprisign and bold plan from the \resphain.
She talks offscreen, and he listens with great interest and a sardonic smile.

When \Teshrial revives, make him genuinely distressed and fearful of this great evil.
\Teshrial should not just be angry and arrogant.

\Nexagglachel was born during the \firstbanewar.
\Ishnaruchaefir was born some centuries later, but before the Draconian Ascendancy.

In order to build and release the Ark, \Ishnaruchaefir requires the help of some aloof Elder \Dragons that sleep dormant.
Perhaps even some of the True \Dragons, the Primordial \Dragons from \Sethicus's time who have slept for a million years.
(Replace the scene with \Ishnaruchaefir-and-the-\xss with this.)
Or maybe it is \Secherdamon who contacts them.

In the end, a few True \Dragons manage to awaken (now that the Shroud is broken) and come to \Ishnaruchaefir's aid.
There exist not many sleeping Primordials, though. 
Five or six at the very most.

Reduce the number of years since the Hundred Scourges.

When \Ishnaruchaefir and the other \dragons lie still and think in the Mirage Asylum, they reach out into the Beyond with their minds.
Here they not only think and speculate and theorize, but also practice their magic and do their research and experiments.
Often you can see storms of power around them as physical symptoms and manifestations of their otherworldly magical experiments.
The inhabitants of the Asylum see this as proof of the \dragons' terrible divinity and cower down and worship them.

\Ishnaruchaefir would like to be free of \Rystessakhin and the glaive, but he needed them in order to maintain some power over the Shroud.
Without him the Shroud might grow unstable, and he would certainly lose influence.
More importantly, the Asylum would collapse, for it was maintained by his magic and could not survive without him.
And he could not live in peace or get any work done (scientifically and politically) without a sanctuary.
So he must keep \Rystessakhin enslaved as a bound wraith, however much it pains him and her.
After all, he still has his stewardship, so he has a responsibility for the future of the world, so he cannot afford to lose his personal power nor his political and metaphysical influence on the world.

\KhothSell is the goddess of life, death, rebirth and immortality to the \dragons.
And also to the \Ortaicans and Rissitics.

\Draconic and the \quiljaaran languages were written with \quo{\ophidian runes}, also called \quo{draconic runes}.
They were a degenerate form of the ancient \ophidian written language.

Make Rian more religious.
Make sure he prays in every chapter and scene that he is in.
When he sees \Ishnaruchaefir, he prays for deliverance from this great evil.
Make clear how grateful to the Light he is for how he has been freed from his life of crime and allowed to make a new, honest life for himself.

\Teshrial enjoys hating \Ishnaruchaefir.
It is just as strong a feeling as being in love.
At some point in the middle of the book, he thinks about \Firaxel and acknowledges to himself that he is in love.
(Have him thinking more about \Firaxel throughout. Ever since the party where she favours him, he falls in love and grows obsessed with her.)
His love for \Firaxel and his hate for \Ishnaruchaefir are his twin motivations, both driving him in the same direction.
And he loves both feelings.
They both affirm that he is alive, that he is a true \resphan who fights against his people's enemies (hate) and for the future and greatness of his bloodline and their race (love, sex, procreation).

Rian should not be an orphan.
But his real parents died of disease when he was 12 or so.
Mention this when he tells us how he is now an apprentice.

When Carzain spies on the Rungeran army, he sees their cannons and counts them.
Later he tells Curwen-tachi how many there were.
They are not happy about the news.

Maybe change the \quo{\Ishnaruchaefir in \Forklin} scenes to \quo{\Criseis in \Forklin}.

Banes dislike rain and fear hail.
It is painful to their telekinetic sense.
They will retreat out of hail and not go out in it unless they absolutely must.

Yurid -> Jurid

When Ilcas is in Malcur, \Tiroco sees the \nycans and compares them to \grulcans.
A \nycan is slightly lighter than a \grulcan, but probably as strong, and faster.
They mostly do not carry riders.
She does not understand how the handlers control them.
Telepathy, as far as she understands.
She has also heard that \nycans are as intelligent as humanoids, but she does not know if that is true.
She also compares them with the Allosaurus-like dinosaurs that the Rissitics, among others, are known to use in combat.
Then she starts worrying about the Rissitic invasion.
Is it wise to send almost all military in Pelidor east to fight the Rungerans when the Rissitics might come knocking on their southern border any day?
The military leaders seemed to think it was a sound plan, so they did it, but now \Tiroco becomes afraid. 
She fears the Rissitics.
They are evil and use black magic and monsters.
She hopes they are stopped by the Vaimons so they never reach Pelidor.



2009-06-29

The \CiriathSepher do not always love \Azraid.
But he is an extremely powerful and skilled \sathariah. 
Even the other dynasties acknowledge that and respect and even fear \Azraid for it.
So the \CiriathSepher are glad to have \Azraid.
He gives them prestige in front of the other dynasties.

\Ishnaruchaefir's full name was Quessanth Niershah Melechet Tzeorossh \Ishnaruchaefir.
\quo{Melechet} means Destroyer.
\quo{Tzeorossh} means Exile.
When he, once in a while, introduced himself with his full name, he would make a sarcastic grimace when he said \quo{Tzeorossh}.
It was given to him by \Secherdamon in hatred.
Eventually, as \Ishnaruchaefir secluded himself in the Mirage Asylum, he came to accept his status as an exile, so he accepted the name, and later he would sometimes use it himself.

When \Urizeth first talks about \Ishnaruchaefir, she lists his full name and explains all the parts she knows.
\quo{Quessanth} is an egg-name and has no meaning.
The meaning of \quo{\Ishnaruchaefir} is unknown.
The \dragons have never taught the \resphain the secrets of their language.

After the \firstbanewar, the \banes' radiation also destroyed most of the \ophidians' computers, leaving their robots as immobile and useless hulks of metal and plastic and whatever.
Their civilization was very dependent on computers and robots, so this was another catastrophe for them.



2009-07-09

The mortal underlings of a \resphan \dynasty were called \hedrim, singular \hedor. 
The \hedrim[\Mystraacht], for example, were the \hedrim that served \Mystraacht. 


Vizicar notices the statues in \Forklin (of \sephiroth and mortals). 
He looks at a statue of \Feazirah. 
She is depicted kneeling with her head bowed, since she is the \sephirah of Humility.
He concludes that they must be newer than the Empire's time. 
Vizicar did not know every piece of his empire (it was huge), so for all he knew, this place could very well have been the site of a major Vaimon city back then. 
The \Ortaican parts could be newer. 
And he does not have the architectural expertise to tell which is older, the \Ortaican or the Vaimon parts of the architecture. 
But one thing convinces him: 
The statues are clothed. 
In his time, this area was dominated by Clan Sether, who were among the least taboo-afflicted clans.
Their statues were naked. 
Nowadays, the area is dominated by Redcor and Telcra.
Redcor have always been afraid of sex. 
Telcra have inherited taboos from the Redcor. 
So their statues are clothed. 

\begin{prose}
  Carzain: 
  \ta{So... you look at a kneeling woman and get disappointed because she's not naked.
    And then you spin a long story of history and architecture to hide that.}
  
  Vizicar:
  \ta{Exactly.}
\end{prose}


\Ghobaleth were horrors from Beyond. 
Perhaps \hr{Horrors from the void}{from the void}. 
They were terribly dangerous even to the \banes who commanded them.
Compare them to the shoggoths from H.P. Lovecraft's "At the Mountains of Madness". 


\\Nexagglachel, \Ishnaruchaefir, and \Secherdamon were Elder Dragons, Antediluvians. 
They could not be roused during the \firstbanewar for some reason. 
Later they were awakened and conquered the world, taking it back from the \aryothim. 


All awakened \dragons died in the \firstbanewar and its aftermath.


\Isphet was an evil figure in Iquinian mythology. 
He was an enemy of the \sephiroth. 
Sometimes described as a \qliphah. 
A nameless \qliphah of the Midnight Circle. 
Sometimes he was considered the king of all \qliphoth and the master of all that is evil. 
(A few doubted that he was a \qliphah at all.)

\Isphet was in continual battle with the \sephiroth.
His titles included \quo{the Adversary}.
His name also appeared in the variant Iscraphet or Iscraphel. 
In legendary times, he had tried to destroy the world. 
The \sephiroth and their angels had attacked him in force. 
They fought against the Adversary and his legions of \qliphoth. 

"And there was war in heaven. Michael and his angels fought against the Dragon..."

The \sephiroth prevailed and cast out the Adversary. 
From then the craven villain would hide in his dark pit of evil and only rarely dare to venture forth into the world of mortals. 
To this day, the myth said, the \sephiroth were locked in a cosmic battle with \Isphet. 
That was why the \sephiroth did not show themselves in the mortal world. 
And their believers had to help them combat him. 
In churches they performed rituals at regular intervals that were meant to keep \Isphet at bay.
He was immortal and would not perish until the end of the world, but he could be wounded and weakened and mutilated.

Compare to Egyptian mythology, where there were spells to overthrow the dragon Apep. 

\quo{Isfet}, as far as I know, is an Egyptian word that means \quo{chaos} and is associated with the serpent Apep, the eternal enemy of Ra, the Sun god. 

\Isphet was described as black with burning eyes, and sometimes as wreathed in fire and smoke. 
He was based on a memory of \Ishnaruchaefir (who did sort of destroy the world), and to a lesser degree \Secherdamon.

Rian knows \Isphet, believes in him and fears him. 
WSB: Have no \Isphet references. We do not want the reader to suspect just yet. 
In the later Rian chapters: He prays to be delivered from \Isphet's evil. 
Have a scene with Rian in church where he attends prayer and mutilates \Isphet. 
There is an effigy of \Isphet there in the form of a black serpent. 
Every church-goer is handed a needle or stick with which to impale the monster. 
At last, the effigy is hacked into pieces and burnt. 

Mention \Isphet and the \qliphoth in the scene with \Icor's funeral. 

The myth of \Isphet was made up somewhere in the Vaimon Age. It was unknown in Cordos Vaimon's time. 



Tantor is certain that \EreshKal was not built by \meccaran hands. 
But his version of the story is very much coloured by his racism. 
He looks down on \meccara. 

Be sure to have a clear difference in Tantor's writing style before and after his son's death. 
Before it, he looks down on the \meccara as \quo{lower humanoids}. 
They are repulsive, but he also sort of pities them. 
And he is impressed and awed by the grandness of the temple. 

After his son's death, he comes to viciously hate \meccara. 
He curses them for their evil, stupidity, inferiority, ugliness, bad smell and everything he can think of. 
And he now feels horror rather than awe at the temple. 

Already out in the forest, he and the others see abhorrent things skulking and creeping at the edges of their camp. 
Ugly midget humanoids. 
Probably \meccara. 
Sometimes the soldiers shoot at them. 
One soldier hits. 
The thing shrieks and bleeds, but escapes, and no one wants to pursue the wretched thing out into the \wylde. 
But they can see the blood. 
They know it is a living creature that bleeds. 
That reassures them. 
A bit. 

Inside the temple they see more of these skulking shapes. 

On the last day, after the big battle, they hear slavering noises.
They realize there are more \meccara, and they have brought monsters with them. 
They fear it is their gruesome gods (whose images and statues they saw in the big chamber) which have now awakened and are hunting the interlopers. 
They flee out quickly.
Several are grabbed by monstrous horrors. 
Those that stop to try and help them are also killed. 
There is nothing to do but run. 
But in the end, many survive, and now they have the valuable plaques they came for. 



To keep the \wylde at bay, people used \eidola. 
And \eidolon was a magical totem blessed and enchanted to protect civilization and keep away the \wylde and its denizens. 
Every religion had its way of creating \eidola. 
Have them in every \wylde scene. 



There were no cameras on \Miith. 
They could not be built like on Earth because of the different way \Miithian physics worked. 
An eye picked up images from many dimensions and the brain sorted out most of it (because of the Shroud and other things), so that the mind saw only a few of them. 
A photograph would be a mishmash of all those images from all those dimensions. 
Without a bigger context, the eye would be unable to make sense of the images on a photograph, so it would just be unrecognizable static. 

The immortals had high enough technology that they might reasonably be expected to create cameras, but they never did. 
The \ophidians, though, could make holograms using magic. 
These holograms extended out into all dimensions and were just like physical objects. 



\Byakun (with a circumflex over the U) was a dark priest who lived in Silqua's time. 
He was a dark mage and very feared. 
Compare him to various Lovecraft figures. 
Silqua's people were afraid of him. 
At some point, Silqua (somehow) heroically volunteered to be sold to \Byakun as a slave. 
She did some hero work, and Cordos came in and saved her. 

Compared to the Mahrkagir in Kushiel's Avatar, who takes \Phedre as a slave. 



The Silqua gambit ended up going astray because the Cabal was still newly formed and the different factions fought much against each other. 
Plus, the Sentinels were working against them. 
This chaos gave Delphine free reins to have her love affair with Silqua, which became more and more depraved and ended with her killing Silqua. 



Moro \Cornel knew much about the Beyond, the occult, the \Miith Mythos. 
She was a Cthulhu style investigator. 
In her youth she did much exploring, adventuring and research. 
That was why she became so scarred and bitter and unhappy. 

The lesson: Being an adventurer does not pay on a cruel, horrible world like \Miith.

Moro knew of the \quo{Elder things} that would one day return to conquer \Miith and overthrow all mortal civilizations. 
But she did not know which creatures were on which sides. 
She knew the words \quo{\xs} and \quo{\bane}, but she was not quite sure what they referred to and what the distinctions were, let alone who opposed whom. 

When Moro kills bandits:
  She could use much more nasty magic on them if she really wanted, but she does not want to attract attention, so she uses only moderate magic. 

\Nasshikerr scenes:
  Moro has a sinister statue of \Nasshikerr which she uses to summon him.

  Make \Nasshikerr more frightening, more awesome and loathsome.
  Like Tsathoggua.
  He has big, rolling chamaeleon eyes. 
  Moro fears him, but she is smart enough to suspect that he knows more than he is telling her. 
  The gods always are. 
  She resents the fact that he is like that (arrogant, superior, unwilling to \cooperate with a mere mortal), but that is how gods are.



End chapters on a cliffhanger! 
All over the place.
Such as the one where \Urizeth dies. 
Have it end with the line: \quo{Urizeth is dead.}
(Maybe it is too short to justify a chapter, so make it just a short \Teshrial scene at the end of some other chapter.)



2009-07-13



\Vorcanths were dark, terrible, mysterious things like the Hounds of Tindalos from the Cthulhu Mythos. 
They were some of the horrors that lurked in the Realms Beyond and would sometimes prey on \Miithians. 
Even immortals were subject to their depredations. 
Even their \resphan allies never understood them very well. 


Visha's Realm was a mostly tranquil but eerie place. 
Haunted by terrible predators, the \vorcanths. 

Lin Carter's "The Necronomicon: The Dee Translation", 1989, p.175

  Aye, be thou warned, for in all such voyages and venturings of mind or soul or spirit there be very great and terrible dangers, by mortal men undreamt-of and unknown. 
  Beware then, lest thou penetrate too deeply into the blackest backward and depthless abysm of the womb of infinite time. 
  For beyond the very Beginning thereof, and on the Other Side thereof,there dwelleth That of which man suspecteth not; and there thou wilt find a strange and ominous Realm where hidden horrors lurk and naked Terror hunts unseen; which dim, uncanny bourn hath the seeming and the semblance of a pale, and grey, and indefinite shore, lapped by the sluggish waves of unmeasured and unthinkable Time.
  And it is eve there, in an awful Light that is beyond all darkness, amidst a profound Silence that shieketh beyond all sound, that \emph{They} slink and prowl in all their ghastliness, slavering with a loathsome and ana unspeakable hunger for all that is clean and whole and unsullied.



The \xss ruled a vast interstellar empire spanning thousands of planets, perhaps millions. 
They knew secrets of dimensional travel that few races had achieved.
The \banes did not have such a dimensional travelling skill. 
This technology was one of the reasons why the \xss had grown to be such a successful and powerful race. 


In the far future, after the exodus from \Miith, the \dragons and \resphain ended up ruling each their vast interstellar empire spanning dozens or hundreds of planets. 
They became super-technological and super-magical civilizations close to rivalling the \voyagers in technology, but far surpassing them in aggression and greedy ambition. 
The \voyagers were a peaceful (if decadent) race of scientists and explorers. 
The exiled \Miithians were warriors and conquerors at their very core.

At this time, the \dragons and \resphain had found new \dweomers to sustain them, so they no longer had to fight each other for possession of the Heart of \Miith. 
But they were warriors by nature and instinct, so they still fought one another. 



Have a scene late in the story with Moro \Cornel:

  Moro captures and interrogates a Black Plague gangster.
  From him she learns where the abductees are kept. 
  Rian is with her.
  He insisted on being there to help.

In the scene where Moro kidnaps a robber to sacrifice:

  She interrogates her victims before she kills one.
  She wants to know whether they are Plaguers and whether they may know something.
  It turns out they are members of another gang which is known to oppose the Plaguers (and, as far as Moro knowns, is not involed in anything supernatural). 
  She is disappointed.
  She had hoped to find a Plaguer to interrogate.
  She has long tried to find a Plaguer to interrogate, but taking bandits captive is difficult. 
  They are sneaky bastards.
  Since her suspicions began she has only managed to capture one Plaguer, and he was a low-level nobody who knew nothing. 
  
  Also make it clear that she usually just sacrifices animals to \Nasshikerr. 
  Only once in a while does he demand a humanoid.



The \firstbanewar nuked the planet to smithereens. 
After the \banes had been banished, \Miith was bathed in fallout, both natural and supernatural. 
And ravaged by summoned monsters (many of whom still dwelt in the dark corners of \Miith millennia later). 
The surviving \ophidians were few. 
They tried to rebuild their civilization, but they failed. 
Their empire declined over the next several thousand years until there was nothing left. 
To make it worse, they waged wars against each other, too. 
The great war had made the survivors more violent and xenophobic. 



\Sethicus did not awaken \Tiamat's sons when he himself awoke. 
\Sethicus suspected that he and his fellow awakened \dragons might very well perish in this war, and he wanted to be sure the \draconian race would be carried on. 
So instead he rigged some spells that would awaken \Nexagglachel some thousand years later.
He also set up some other powerful \dragons (whom he sort-of trusted) to wake up. 
He wanted to be on the safe side and not rely on \Nexagglachel alone.
For all \Sethicus knew, something nasty might happen to \Nexagglachel in mean time, so he wanted a backup plan. 

When \Nexagglachel awoke, he immediately went out to awaken his two brothers.
When \Ishnaruchaefir was awakened by \Nexagglachel, he immediately went out to awaken \Rystessakhin. 

The awakened \dragons could not rebuild the \ophidian civilization. 
\Nexagglachel tried his best, but it was hard. 
They were only a few individuals.
They had much scientific knowledge, but it was incomplete.
And the \quiljaaran were not so great. 
They were slothful and uncreative and unambitious.
After \Nexagglachel died, only \Secherdamon retained the ambition to rebuild the \ophidian empire any time soon. 

The \dragons did make some progress, though. 
They introduced a lot of technology among their followers.

The \aryothim developed technology that, in some areas, was superior to that of the \quiljaaran. 
The \aryothim did not have nearly as powerful magic, but they had high-quality guns and stuff.
Perhaps they even had 20th century technology. 

At the time of the \secondbanewar, technology on \Miith was high on both sides. 
\Miithians and \resphain each had some weapons that the other side lacked, making the first battles full of nasty surprises for both sides. 



Have many dark, unexplored \quo{here-there-be-\dragons} places in the \wylde. 
Even in \Velcad. 
In the \thirdbanewar period as well as in earlier periods. 
Compare to places from the Cthulhu Mythos:
\begin{itemize}
  \item The Vale of Pnoth.
  \item The Forest of Zoogs.
  \item The Peaks of Throk.
  \item The Vaults of Zin.
  \item The Tower of Koth. 
  \item Kadath in the Cold Waste.
\end{itemize}
Among other things, have a dark valley of naked, black basaltic pillars, inhabited by Gug-like monsters. 
And have places where the \quiljaaran live. 
The \serpentmen were known from legends and feared. 



Some \xss dwelt on \Miith itself, such as the one near \Yormis.
The Shroud made them increasingly sluggish and slothful. 



In the \wylde in \Azmith one can occasionally see giant \umbrae soaring high above. 
Amorphous fearful shapes that cast a vast, ominous shadows.
Hence the name \quo{\umbra}. 



2009-07-14

Make a general section about \xss and their hierarchy. In fact, merge all the \xss into one section. 

According to some traditions, \NaathKurRamalech was the greatest of the \xss, not \DzyrochNathla.
He was certainly the \pps{\dragons} most important ally when in came to protecting \Miith from the \banes. 

Some believed that \NaathKurRamalech was younger and weaker than \DzyrochNathla, but simply seemed more powerful because he was closer tied to \Miith and took slightly more interest in the doings of his \Miithian worshippers. 



When \dragons felt the need to express great emotion, they would often switch to True \\Draconic. 
Examples: 
  \Ishnaruchaefir and \Rystessakhin at the \Shrouding. 
  \Ishnaruchaefir and \Nzessuacrith and \Secherdamon in \TwilightAngelRemember.



Move Nemuragh and Lothagiel to \TiphredSerah. 



\Ishnaruchaefir's Nadir happened at times when the \quo{tides} of the Shroud were low, meaning that the Shroud was weak and permeable. 
At these times, \Ishnaruchaefir had to work hard to keep the Shroud stable. 
This hard spellword was what made him weak. 
He had to open himself up to the world in order to pull its strings, and this openness made him vulnerable. 

If he failed, he risked a nasty backlash against himself and his Mirage Asylum. 
The Asylum was unstable by nature and prone to collapsing or flying off into space if not maintained. 

It was unclear whether the Shroud itself might collapse without \Ishnaruchaefir's support. 
\Ishnaruchaefir himself tended to believe that his work was necessary to keep the Shroud alive.
His critics tended to think his work was of purely local significance. 

TAR: 
The time when \Secherdamon plans to resurrect \Nithdornazsh coincides approximately with \Ishnaruchaefir's Nadir.
This is not by chance. 
In this period, the Shroud is thin, so they have a better chance of succeeding, breaching the barriers between the worlds and bringing their citadel to \Azmith. 

The Cabal's Malcur venture is also mouthing out into some conclusion at this time. 
Or was supposed to. 



Compress the history of \Miith.
I.e., make the periods shorter. 
Fewer thousands of years. 
Now that the \draconic major characters are moved millions of years back, I have plenty of impressive big numbers, so I can relax on the more recent history. 

How old are the \dragons?
\Nzessuacrith was as old in the \secondbanewar as Rathyon was in the \thirdbanewar.
\Ishnaruchaefir was at least as old as that when \Nzessuacrith was born. 



2009-07-15

When \Ishnaruchaefir is in his Nadir he bleeds, and one can see a myriad long, aethereal tendrils radiating out from him, through which power drains out of him to sustain the Shroud and the glaive. 

Before WSB, when \Achsah detects \Ishnaruchaefir:

  They know that \Ishnaruchaefir is clearly up to no good.
  \Urizeth fears he has caught wind of their doings in Malcur and will try to interfere and stop their plan.
  That must not happen. 
  Malcur is vital to the future of \CiriathSepher (or so this splinter group likes to believe). 
  \Ishnaruchaefir could fuck up the \noggyaleth and everything. 
  That must not happen. 
  But he is an immensely powerful \shaeeroth \dragon.
  What can they do?
  They are despairing.
  
  Then \Teshrial steps forward. 
  He is brave and will go out to meet \Ishnaruchaefir. 
  If he cannot scare him away, \Teshrial can at least keep \Ishnaruchaefir at bay long enough for the others to cover their tracks, retract the \noggyaleth so he cannot fuck them up, and retract their own tendrils of \\vertex power so \Ishnaruchaefir cannot guess their plans..
  
  They are all impressed, but they also warn \Teshrial about how dangerous it is.
  \Teshrial is brave. 
  He stands his ground. 
  
  Portray \Teshrial as well-meaning and idealistic, but also arrogant and badass. 
  (BTW, in this chapter, make it clear that \Teshrial does not really like \Urizeth, nor she him.
  She sees him as a shallow glory hound.
  He sees her as a weird nerd.)
  
  Afterwards, they are confident they have kept \Ishnaruchaefir from learning what he should not learn and fucking up their plans. 
  When \Teshrial died, everything had been cleaned up.
  They are sure he did not learn anything.
  But they were not keeping an eye on \Criseis.
  They do not understand quite how keen her senses are.
  Under their noses she has snooped around and gained a good overview of what the Cabal are doing. 

At some point, \\Menessiaraid expresses doubt about whether \Teshrial's Malcur venture is really so great and important a thing as they like to believe. 
\Azraid:
  \Azraid thinks to himself. 
  He is not sure whether \Teshrial's Malcur venture is really so great and important a thing as \Teshrial-tachi like to believe. 


Thule -> Thulaan -> Thulaam


Gormur? Gomkor? Gomkur? Gnophil? Gnomphil?

The \gormurim were a race of simian humanoids related to \nephilim.
They were of similar size as \nephilim, but farther from \human form.
They had long fur and long, baboon-like faces (but no tail).
They had tribes and Stone Age technology.
They lived mostly in \Thulaan and similar arctic regions, where the weather was too cold for the less hardy \humans and \scathae.
They worshipped \xss godlings. 
Compare them to the Gnophkeh from Lovecraft. 



After the FBW:
Many \ophidians began to worship brutal \xs godlings that cared nothing for technology nor progress. 
Under the rule of these harsh gods, the \ophidians descended into barbarism.
They waged wars against one another and destroyed what little remained of their civilization and the planet's natural resources.
Compare to the Serpent People from the Cthulhu Mythos, who turned to the worship of Tsathoggua and were punished by their ancestral patron god Yig, causing them to degenerate. 

In the next millennia, \Miith was populated by a lot of different barbaric, \xs-worshipping peoples. 
They would wage war a lot and destroy each other.
Technology remained low.
But then one day, some \quiljaaran created the \aryothim: A race of super-powered \nephilim blessed with \quiljaar-level intelligence.
Then technology began to rise. 



Perhaps \Tiamat did not die in the \firstbanewar.
She awoke with \Nexagglachel. 
She had been driven mad and evil by a million years of imprisonment.
She conquered the world and reigned as a cruel tyrant.
Eventually \Nexagglachel decided she was too evil and rose up against her.
He converted many \dragons to his side, including his two brothers. 
They slew her. 



Once, \Ishnaruchaefir asked a great cosmic god:
\ta{Why do you live? What is your purpose? What is your goal, your motivation?}

The god answered: \ta{We \emph{know}.}

\Ishnaruchaefir pondered that ever since. 
He was sure there was some great insight hidden in that answer. 
Perhaps because the cosmic gods \emph{knew}, they were content and happy and needed never strive for anything ever again. 
Perhaps the gods \emph{knew} the future and thus had no need of motivations, since they knew that everything they would ever do was already determined. 
Or maybe the truth was something else entirely.
\Ishnaruchaefir pondered that question till the end of his days. 


Some \draconian philosophers speculated that when the different individual \xss seemed to have different powers and specialization (and came to be seen as gods of some \quo{portfolio}), it was perhaps not a result of the gods' actual powers, but rather their interests.
Perhaps \NerranKoss was just a philosophical god with an interest in history and such matters. 
So when people asked him such questions, he was more likely to yield a useful answer than most other \xss.
And thus he became seen as a god of occult knowledge. 
It was unknown if this theory was true, but it was accepted by several. 



2009-07-16


The Vaimon Empire was not built through conquest alone, but even more so through diplomacy and religious propaganda.
History and legend later emphasized both Cordos' heroism in war and conquest and Silqua's gentle charisma. 
The truth was much more complex than that. 


Rian is scarred and horrified to see the evil sorcerer slay the shining god. 
Criseis Shrouds him and makes him forget the details, but some measure of religius/existential dread remains with him. 
Remember that in all his later chapters.
Maybe even ask on a forum how to express this. 


Make clear in the first Carzain chapters that Carzain and Vizicar do not literally talk to each other. 
In reality, they have much more direct access to each other's thoughts and memories. 
Their thoughts and memories interact in a more fluid manner. 
It is just presented in the text as an inner dialogue for the reader's sake.


The unification of the Cabal took decades. 


Some myths said that the \nephilim were half-\humans, created when \humans interbred with primitive hairy monsters of the \wylde --- possibly the \gnomphilim. 
This was false. 
In fact, it was the \human race that came into being when \nephilim mated with monsters.


A \dragon's length was approximately one quarter body, one quarter neck and one half tail.
Wingspan was about equal to total length.
\Ishnaruchaefir was 25 metres long in all. He weighed about as much as an Allosaurus. 
\Secherdamon was longer than \Ishnaruchaefir, but slimmer. 


At the time of the \thirdbanewar there were 1000-3000 \quiljaaran worldwide. 
100-200 of them worked for \Secherdamon.
10-15 of these were active on \Azmith.
5-6 of them were among the Rissitics. 

Several \quiljaaran dwell in or near \Yormis in disguise. 
Some of them are part of the Dark Crescent.
Others are independent.
They worship the \xs that dwells there and practice their dark science and philosophy. 

Moro \Cornel once saw some \quiljaaran (\quo{\serpentmen}). 
She knew a bit about their terrible secret. 


The Cabalist Malcur venture, like \Secherdamon's plan, also coincides with \Ishnaruchaefir's Nadir. 
When \Urizeth finds out when the next Nadir falls:
  She remarks that it is no coincidence that the Nadir falls now.
  It falls at a time when the Shroud is in \quo{ebb}. 
  Their own plan is also scheduled around the ebb and flow of the Shroud.
  But, as far as she can calculate, the deep point of \Ishnaruchaefir's Nadir comes slightly before the climax of their plan. 
  Very likely \Ishnaruchaefir plans to weather his Nadir and then quickly come back in time to fuck up their plan.
  But if they are lucky, \Teshrial can arrange to fight \Ishnaruchaefir at the bottom of the Nadir, before the climax.
  Thus saving their asses.
At the time when \Teshrial fights \Ishnaruchaefir (for real):
  The Cabal plan is nearly complete. 
  The Cabalist Malcur venture, like \Secherdamon's plan, also coincides with \Ishnaruchaefir's Nadir. 
  The \noggyaleth have grown numerous and large and powerful.
Every time \Teshrial sees the \noggyaleth (or even thinks of them), he is filled with fear and loathing. 
He is not ashamed to admit he fears them. 
Admit it to himself, that is.
He is still too ashamed to admit it in front of others.
In the battle with \Ishnaruchaefir:
  The \noggyaleth grab on to \Ishnaruchaefir with their sucking mouths and grasping limbs/pseudopods.
  They drag him down and engulf and swallow him.
  Then they try to drown and crush and devour and digest him.

Half-\nagae walk more hunched-over than normal \scathae.
They have short legs and long tails. 
Their heads are strangely narrow and flat (in the vertical direction). 
Their bodies are flexible, and they seem to writhe and wiggle in a repulsively fluid manner.
And their tails are snakelike and prehensile, which is horrible to look at.



2000-07-17 

Make it clear that the Shroud was a patch. 
It was a hasty emergency solution to patch up the cracks in the much more well-designed \CrystalSphere. 
Everyone knowledgeable, including \Ishnaruchaefir, knew that the Shroud was only a temporary measure and could never last. 
Sooner or later it would collapse.
Everyone had better be ready when that happened. 


\Banes had no bones.
Their bodies were flexible and could squeeze through very small openings (albeit they would have to leave any solid items behind). 


Vaimon Emperor -> Vaimon Caliph
Vaimon Empire  -> Vaimon Caliphate


When Carzain reports to Curwen:
  Curwen offers Carzain a smoke.
  Carzain declines.
Afterwards:
  Carzain: Vizicar, remind me again why we did not accept that smoke. 
  Vizicar: In my days, smoking was a plebeian thing to do. We royals never touched it. And I am not about to start now. 


In a scene late in the book, after \Teshrial has acquainted himself with the \noggyaleth, \neoresphain and \WanderersInDarknessEmph:
  He flies above \Nyx.
  He sees the \bane-built spires. 
  Now that he gazes into the deep, he notices how twisted they are.
  They look really scary and wicked.
  He had always taken them for granted, but now that he has gazed deep on the dark mysteries of his people (and the even darker mysteries of the \banes), they scare him.
  There is something evil about the way they twist and bulge.
  They twist into alien dimensions (more alien that what he likes to consider). 
  Like they are appendages of some vast monster that tries to crawl and claw its way up from the deep where it belongs. 
  See also the section on dark ancient cities. 


\Draconian citadels were enormous. 
They sprouted huge, bulbous, misshapen spires.
They catered to the \xss and to the principles of Chaos. 
They were built in accordance with Chaotic occult geometry in order to more fully utilize the power of the \xss. 


Dark ancient cities: 
  \Draconian and \bane buildings alike violated the laws of three-dimensional geometry.
  They extended into the dark, hidden dimensions of the Beyond. 
  The shapes of their walls, towers, statues and ornaments stretched out into the Beyond in a way that was not only physically painful for the eye to follow, but also horrible because it drew attention to hidden things that no mortal wanted to consider. 

Fix BibTeX. Tell BibTex that the stories appear in this volume.
Chaosium: The Tsathoggua Cycle, 2005, edited by Robert M. Price. 
John Glasby - The Old One
p.133--141

  ... we caught a fragmentary glimpse of something which rose from those benighted depths, clawing up from the unseen floor. [...]
  
  To me, they held ineffable suggestions of a blasphemous structure and architecture utterly unlike anything I had ever seen. 
  [...] the searchlight beam only touched their topmost regions.
  But even this was enough to show the sheer alienness of their general outlines.
  Had they been mere conical towers, it would not have offended our sense of perspective to such a degree. 
  But there were bulbous appendages and truncated cones which intermeshed in angles bearing no relation to Euclidean geometry and I felt my eyes twist horribly as I tried vainly to take in everything I saw. 
  
  [...] 
  I could not help feeling there was something evil about those nightmarishly misshapen spires and pinnacles with their bizarre curves and planes; yet it was not an evil associated with Earth but rather with the endless gulfs of space and time, with dimensions other than those we know. 
  
  The majority were smashed and broken with harsh, gaping orifices showing blackly against the sickly grey.
  What beings had once moved within these structres it was impossible to visualize. 
  Certainly no hand of man had erected them and carved their cruel, hideous contours.
  [...]
  The obscure quality of menace in their weird symbolism made me shudder and long for the sanity and safety of the ship.
  
  [...]
  
  [I watched] for he first indication of the vast grey-stone city.
  And then I saw them for the second time, rising out of the slime of the ocean floor, clawing upward for hundreds of feet; row upon seemingly endless row of fantastically symmetrical columns, the nearer ones blindingly clear in the harsh actinic light, with countless others stretching away into the black immensity. 
  
  [...]
  
  The effect of that monstrous labyrinth which stretched away from us into inconceivable distrances was indescribable for it was apparent at once that whatever stood on this undersea plateau had never been fashioned by nature, even in her wildest and most capricious moments. 
  And it was equally obvious that whatever hands had erected these edifices had been far from human. 
  
  [...]
  
  I dreamed of the long-dead city under the sea.
  But before my dreaming gaze it now stood unbroken and untarnished by time on dry land and there was no sign of the ocean.
  On an incredibly ancient plateau, wreathed in clouds of steam and noxious vapours, the Cyclopean buildings streched away in all directions as far as the eye could see and high into the lowering clouds where the topmost spires were lost to sight.
  There was something terribly unhuman about the geometry of its massive grey-stone walls, and the mind-wrenching alienness of its angles and intermeshing structures went against all reaon, all known laws of mathematics, logic and architecture.
  I knew, by some weird instinct, I was seeing it as it had been perhaps several millions of years ago when it had been newly built by that race from the stars. 
  
  [...] now I saw the inhabitants, those hideous and, if the \emph{Book of K'yog} was to be believed --- artificially --- created abominations that had built it!
  I saw them as vague shapes ion the vast avenues and squares, saw them clinging limpet-like to the sides of the buildings or oozing jelly-like from the grotesque apertures and doorways.
  What insane blasphemy had bred those \emph{things} I could not conceive, but the mere sight of them woke me, yelling incoherently, from my dream. 

Link to the above from \Nithdornazsh and the Mirage Asylum. 
And link to the aforementioned sections from every section where they appear in the story. 

Ancient immortal cities reached out into the Beyond. 
They were built in a time before the Shroud, when the barriers between the Realms were much more permeable. 
Their streets and corridors and towers were built so they criscrossed the Realms. 
This was why \Nithdornazsh was such a useful gateway between. 

After the resurrection of \Nithdornazsh, the Sentinels have a lot of work ahead of them.
They have to restore the city and build new eidola to shape the Shroud and the gateways and facilitate the travel between the Realms that they want. 


When the \dragons were created, the \ophidians had only begun to build an interstellar civilization. 
So the \dragons had only limited knowledge of the \ophidian super-technology.
The \dragons and \ophidians would both continue to develop their knowledge in the coming millennia, but in different directions. 

\Firstbanewar:
  When the \banes invaded, the \ophidians had an interstellar civilization.
  They commanded dozens of planets. 
  The \banes likewise had an interstellar empire. 
  But \Miith was the important planet that they had been searching for all along. 
  It contained a Heart full of \voyager-tampered energy, and the \noggyal mother-mass, which was the \quo{other half} that the \banes wanted. 
  
  The war lasted centuries. 
  At last the \ophidians were overwhelmed by the \bane hordes, so they retreated to their homeworld and barricaded it with the \CrystalSphere. 
  The \banes conquered all their other planets. 
  
  Fortunately, the \banes had only primitive space travel, and they were un-creative.
  So their interstellar empire could spread only slowly, and not far.
  If the \bane swarm spread too far apart, the farthest parts withered and died.
  They needed the \noggyal mother-mass in order to truly expand their dominion. 
  
  The \banes had a hivemind. 
  The overmind was the \baneking \Voidbringer.
  There was only one \baneking. 

\Draconian Ascendancy:
  When the \dragons awoke, they had spent millennia pondering mysteries of science and magic. 
  They had gained much insight into the occult, so their magic and mystic skills were immense. 
  But they knew little of physical technology, because they had had no opportunity to experiment with building physical tools. 
  So when they awoke, they could not just establish a new technological civilization. 
  Furthermore, their skill was based on occult revelations and deep Gnosis which they could not easily teach to anyone else. 
  So they remained monoliths of arcane power instead of building a new civilization. 


The Mirage Asylum was like a half-open castle ruin or space hulk drifting afloat in the vast, empty void of space. 


\Nithdornazsh was the old tomb of \Nexagglachel. 
Out of respect for him it had been deserted after his death. 
It became a necropolis in his memory, instead of being taken over and used for some other purpose. 
Thus the \resphain forgot about it. 
It was deep in \draconic territory and seemed to have little strategic significance, so the \resphain felt they had better things to do that attempt to conquer or raid it.
Until, finally, \Secherdamon felt it was time to revive it. 


TAR prologue:
  \Nzessuacrith sees the wrecks of mighty cannons and vast war machines of metal and flesh. 
  She herself bears grim wounds. 
  She lost her right forearm in battle with a titanic \umbra; a soul-devouring monster from \Erebos, fully as large as she was.
  Call her \quo{Cryocas}.
  Cater to the dumb reader who cannot understand long names. 
  And get rid of those dead \dragons. At least, do not name them.  


WSB: 
  \Criseis does not know what \Ishnaruchaefir has in mind. 
  She knows what he has told her to do (i.e., reconnoitre for any supernatural presences infesting Malcur), but no more.
  She imagines he intends to draw out whatever is in the ground and destroy it.
  
  When \Teshrial arrives, he looks at the glaive, and \Criseis can see him thinking. 
  He is thinking: 
  \tho{He pulled that thing out shortly before I arrived. 
    He must have been prepared to do something nasty with it. 
    Looks like I came just in time.}
    
  \Criseis agrees. He probably did come just in time, although in time for what, she cannot say, only try to guess. 
  
  After the battle, the \noggyal presence has retracted. 
  \Criseis can still feel it, but only faintly, and only because she knows it is there.
  She asks her master if he intends to pursue.
  
  \Ishnaruchaefir:
  \ta{Nay. Let them hide.}


When the first \scatha meets a \human: 
Show the \scatha's revulsion at the sight of this ghastly, grotesque, naked... thing. 
Compare the \human to a naked mole rat. 
\Scathae tended to see mammals as disease-ridden vermin in the first place, and this naked mammal that stood upright and tried to mimic (\quo{ape}) the \scathaese form was horrid to behold. 
Besides, the \scatha could feel the Cosmic Horror of the \pps{\banes} influence on the \human. 
It would take centuries for the two races to become fully accustomed to each other and not feel this Cosmic Horror revulsion anymore. 


When \Criseis feels (in WSB) or sees (in the big battle) a \noggyal:
The Old One, p.147:
  Even in retrospect it is not possible to convey in words the nature of that monstrosity which squeezed its vast bulk through the gaping abyss.
  It held a hint of noxious plasticity, of writhing tentacles which changed their number and shape.
  But more than anything, I had the impression of gigantic size, that huge as that part of it looked where it almost completely blocked the opening, there was an infinitely greater bulk mercifully hidden from us.
  
  [...]
  
  But I know there was nothing imagniary of halucinatory about the black, coiling tentacle that seized Dorman around the waist and bore him, kicking and screaming frantically, into the gaping, beaked maw which appeared as if from nowhere beneath that single glaring red eye!


Compress the Rian storyline. 
It is unbelievable that they would keep Neina captive for so long. 


Read some Robert Howard before I write the scenes with Carzain and Tantor in the \wylde. 


The \ophidian civilization used up a lot of the planet's fossil fuels. 
The \quiljaaran and \aryoth civilizations used up most of the rest.
By the Human Age, there was almost no fossil fuels left.
This made it hard to develop any industry. 
So the mortals didn't. 

\Nasshikerr tells Moro that there are some evil people that are preparing some big, evil ritual of magic, and that it is connected to that which is destroying the city.
Moro naturally assumes that this means the evil ones are the ones destroying the city.
But \Nasshikerr does not tell them what the factions are or what their different plans are.
He tells them a bunch of details pertaining to the two plans, but gives them no overview of the big picture. 

Near the end of TAR:
  Moro and Rian have gotten some tips from \Nasshikerr (chiefly) and from the thug Moro has interrogated (secondarily).
  They know there are some evil people that are preparing some big, evil ritual of magic, and that it is connected to that which is destroying the city.
  The captive thug tells them where the ritual will take place. 
  \Nasshikerr has told them the time. 
  They go there to attack the ritual and stop them.
  At the very same time, Needle also attacks.
  Moro and Rian see Needle and the wicked \banes she commands. 
  They are both horrified --- especially Moro, because she knows what \banes are. 
  They do not know that Needle is there to do the same thing as they, so they assume she is their enemy (after all, she commands the \banes).
  So they kill her. 
  
  The \banes wreak havoc. 
  It was an unforeseen development.
  \Psyrex had not expected the Cabalists to loose \banes. 
  And these \lesserbanes are small enough to be difficult to detect for a mage, but still deadly enough to be a great menace. 
  \Psyrex has to go himself and fight the \banes. 
  
  Moro leads one \bane away and hurts it.
  She cannot kill it, but she can keep it at bay and occupy its attention long enough for \Psyrex to deal with the other two \banes and complete their ritual. 
  Then, when \Nithdornazsh rises, Moro gets separated from the \bane and manages to finally escape it.
  She is convinced it will go elsewhere.
  After all, it has no particular reason to want to kill her. 
  
  In the chaos that ensues after the \banes wreak havoc in the Sentinel camp, Rian slips in, stealthily knifes a few thugs (killing his first man in the process) and manages to find and rescue Neina.
  Perhaps he finds her naked in bleeding in a corner, having been recently raped. 
  Perhaps he catches a couple of rapists in the act and kills them (but not before they could take their pleasure from her). 
  Perhaps this rapist-killing is his first kill. 


The cult in \Redce:
  Have erotic scenes, like in Gary Myers - The Horror Show (in The Tsathoggua Cycle), where a hot girl is stripped naked, tied up, beaten halfway to death and then sacrificed. 
  In the end, she changes her mind and screams and begs for mercy, but in vain, and she is devoured. 
  All the while, the cult chant their songs to their dark gods. 















