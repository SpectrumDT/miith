2009-06-30

\Merkyran religious spell/ritual to ward off umbrae:


\citebandsong{Nile:Ithyphallic}{Nile}{
  Papyrus Containing the Spell to Preserve Its Possessor Against Attacks from He who is in the Water
}{
  Amun\
  Lord of the gods\
  Thou who art of the four rams heads upon thy neck\
  Thou standest upon the spine of the crocodile fiends\
  To thine sides are the dog headed apes\
  The transformed spirits of the dawn

  Drive away from me the lions of the wastes\
  The crocodiles which come forth from the river\
  The bite of poisonous reptiles\
  Which crawl forth from their holes

  Be driven back crocodile thou spawn of Set\
  Move not by means of thy tail\
  Work not thy feet and legs\
  Open not thy mouth\
  Let the water which is before thee\
  Turn into a consuming fire

  I possess the spell to\
  Preserve me from he who is in the water

  Thou whom the thirty seven gods didst make\
  And whom the serpent of Ra didst put in chains\
  Thou who wast fettered with links of iron\
  In the presence of Ra\
  Be driven back thou spawn of Set

  Drive away from me the lions of the wastes\
  The crocodiles which come forth from the river\
  The bite of poisonous reptiles\
  Which crawl forth from their holes
}


The Cult of the Worm worshipped worms and decay as a manifestation of \KhothSell or other dark gods of death and decay.
On a similar note: 
The \ghobaleth were mindless things that existed only to crawl, burrow and feed.
They were asexual and reproduced by budding.
By burrowing through the crust of \Miith they made the planet's barriers unstable. 
They worked to slowly undermine the \CrystalSphere and open a back door from \Miith to \Erebos.

\citebandsong{Nile:Ithyphallic}{Nile}{
  Eat of the Dead
}{
  The highest fulfillment of man\
  Is to become food for the crawling things\
  That burrow and slither in \human flesh\
  Unceasing in mindless hunger\
  Remorseless undefiled by reason\
  The worms of the tomb they are pure

  Their purity elevates them\
  Above the putrefying pride of our race

  The destiny of man is\
  Merely to be\
  The nourishment of the worm\
  Yet their excrement bestows higher wisdom

  From decay arises new life\
  Fill myself with that which rots\
  And I shall be reborn

  By writhing upon my belly like a mindless worm\
  I shall rise up in awareness of truth\
  I gnaw upon my own decaying flesh\
  And my mind is forever purged\
  Of the corruption of faith
}


When the ancient \dragons either were defeated or chose to \quo{die}, they were interred and bound in vast tombs.
There they lay for thousands or millions of years.
Sometimes attended by living or undead monstrous servants (like the ones in \Nithdornazsh).
Maybe they did not lose a war back in the day; they simply grew tired and disinterested and decided to retire into their tombs. The surrounding \ophidians were happy to see them go.
If they were defeated, on the other hand, then they had to be interred in vast tombs and bound with fearsome spells and seals to prevent them from rising and claiming their revenge.
When at last the \ophidians realized they needed the \dragons back, it took a lot of magic to resurrect them, for their souls had been away for a million years and were no longer accustomed to living in a physical body.
Some of the \dragons had wandered the Realms of the \xss. 
Others had simply lain there thinking, developing new philosophy and science and magic.
Some of the resurrected \dragons did not want to be resurrected and went mad.
Their minds had become twisted into in\human, \xs-like monsters.

\citebandsong{Nile:Ithyphallic}{Nile}{
  The Essential Salts
}{
  It is a great shock to the soul
  To tear it back from its resting place and reanimate it
  The resurrected are often insane and scream ceaselessly 
  or dash themselves into walls
}


The dead \dragons lived on in myth\dash{}and still did in the Age of the Shroud:

\citebandsong{Nile:Ithyphallic}{Nile}{
  What Can Be Safely Written
}{
  On the walls of lost cities\
  And in the carvings of madmen\
  Who have glimpsed him in their dreams\
  Is his image delineated\
  Within a tomb protected by great seals he lies in death\
  Under the weight of the dark waters of the deep\
  Yet he dreams still, and in his dreams continues to rule this world\
  For his thoughts master the wills of lesser creatures
}


\Nzessuacrith attacks her foes with dark curses invoking the \xss. 
Perhaps \Ishnaruchaefir does likewise.

\citebandsong{Nile:Ithyphallic}{Nile}{
  Laying Fire Upon Apep
}{
  Fire be upon thee Apep\
  Ra maketh thee to burn\
  Thou who art hateful unto him\
  Ra pierceth thy head\
  He cutteth through thy face\
  Ra melteth thine countenance\
  Lo your skull is crushed in his hand\
  Thy bones are smashed in pieces

  Burn thou fiend\
  Before the fire of the eye of Ra\
  The hidden one hath overthrown thy words\
  The gods have turned thy face backwards\
  Thy skull is ripped from thy spine

  The lynx hath torn open thy breast\
  The scorpion hath cast fetters upon thee\
  Maat hath sent forth thy destruction\
  Thou shalt burn

  [solo: Karl]

  The god Aker hath condemned thee to the flames

  Fire be upon thee Apep\
  Thou enemy of Ra\
  Let flames gnaw into thee\
  And sear thy flesh\
  Fall down Apep\
  I hath set torch upon thee\
  Taste thou death Apep\
  The burning is upon you\
  Thou art consumed\
  I hath lain fire upon thee\
  I hath smeared thy remains with excrement\
  I hath spat on thin ashes\
  Taste thou death
}


The Imetrium and Rissitics, and other religions, taught that the souls of the faithful dead would be reincarnated and return to the world, and that the best of them would be gathered unto the gods and given some blessed position.
The last part was true (although the \quo{blessed position} might be no more than a mindless niche in some \matrix).
The first part was also true, although perhaps not in the way people imagined it. 
The gods could shepherd souls and make sure they got recycled in the correct places, by means of prayers and spells cast on the dying (to catch the soul and store it in the \matrix) and on the conceiving, pregnant and newborn (to make sure a suitable soul would be injected into the mother and merge with the child). 
But this process was error-prone. 
Souls, however pious, might wander off and get lost, or get eaten by aethereal predators in the Beyond.
And souls are difficult to maintain in a disembodied state.
Great and powerful and skilled souls might be able to maintain themselves and even keep some measure of consciousness and power while disembodied.
But regular mortal souls lost consciousness and entered a dreamy state.
Here the souls became malleable and fluid.
They would break apart and merge into one another; lose soul-matter to the void and absorb new soul-matter from the void.
So an incarnated soul might not be an exact copy of any dead soul, but could be a mix of fragments of many souls.
Still, close enough.


Ishnaruchaefir physical
Religions reincarnation



\Dragons, with their superior immortality, were able to maintain consciousness even though their bodies were dead or destroyed. 
\Sethicus and \Tiamat did this after their deaths.
Even \resphain could not do that. (It was a monumental achievement for \Shiaraid.)

Maybe get rid of \Tiamat and merge her with \Sethicus.
Add her names to him.
If so, then \Nexagglachel was a \Primordial.
\Ishnaruchaefir might be a \Primordial, or he might be born during the FBW.

The voids between planets were filled by all sorts of horrors.
When the \voyagers settled \Miith hundreds of millions of years ago, they erected dimensional barriers around the Realms of \Miith to keep it safe from these horrors.
In a sense, the \xss were some of these \quo{horrors} that the \voyagers feared.
They were some of the greatest \quo{horrors}.
These dimensional barriers, known as the \Voyagers' Palisades, were an early Shroud-like construction, but much better made (by the super-advanced \voyagers), and so had none or few of the destructive side-effects that the later Shroud had.
The Palisades also kept the \banes out, but eventually the \banes managed to drill through.
The \ophidians had some limited understanding of the \voyagers' work, and they drew on it.
They built the \CrystalSphere to patch up the holes in the Palisades that the \banes had made.
(It is the Palisades that prevent space travel, not the \CrystalSphere.)

The Mirage Asylum is regularly attacked by horrors of the void.
Then the inhabitants need the magic that flows from the \dragons and their blood in order to repel the invaders and defend their home.
In especially bad cases, when large swarms of horrors attack, the \dragons themselves must arise and fight.
The inhabitants know this, so they worship the \dragons as their protectors.
And they willingly surrender some of their own as soul sacrifices to the \dragons when needed. 
This is pretty often needed.
In fact, the Asylum should be bigger.
It should have a population of 10,000 \humanoids, or even 50,000.
The \dragons also feast on those horrors of the void when they can.
The depredations of the horrors is a necessary evil.
The Asylum is located in a place far removed from the beaten paths of \Miith, so the \resphain and other \Miithians cannot find it.
But this also means the Asylum is partially outside the protective Palisades.
This means various mindless or intelligent creatures of the void can enter and attack them.
But another pro of this is that they can grow exotic plants and things by drawing on some occult, dark energy streams that flow in from the void (but are blocked inside the Palisades), and combining these with the life-giving energy that flows from the Heart.

The process of becoming a \shaeeroth involves lowering the walls of denial in one's mind and surrendering to the vast, cruel cosmos.
You have to realize that you are nothing next to the true forces of the universe.
Only then can you begin to acquire true power. 
Your ego and fears and wishes have to be broken down in order to be built up into something new and stronger.
Ramiel experiences something much like this when he awakens from Kenosis: 
He gains insight into many horrible truths concerning his people, the \banes, their \matrices and their purpose.
This is only possible because of the wisdom he has already attained, and it makes him even wiser and stronger, but also less sane, and it brings him a pain and a despair that will continue to haunt him.

\citebandsong{Nile:Ithyphallic}{Nile}{
  Language of the Shadows
}{
  Abandon hope\
  And I shall become free\
  And with freedom acquire emptiness

  With the mind cleansed and empty\
  There is the void known as despair\
  A gateway upon an emptiness endless and vast

  In despair the language of the shadows is intelligible\
  In madness all sounds become articulate

  Terror and despair they guide me\
  Into nightmares that follow one upon the other\
  Like windblown grains of sand

  [solo: Dallas]

  I have become as the wastelands\
  Of unending nothingness\
  Now shall the night things\
  Fill me with their whisperings\
  And the shadows reveal their wisdom
}


2009-07-01

Remember that cities must be designed with domestic dinosaurs in mind.
Factor this into the descriptions of \Forklin and Malcur.

The hippopotamus is the largest and most dangerous mammal in the world, as far as some people know.
Mention this when the Rissitic Hippopotami are mentioned.

\Iquinian knights have superpowers.
They are not Vaimons, but with the help of priests they can summon the \sephiroth and perform super\human feats of martial arts.
They also look and feel imposing and regal and commanding and holy, because they have the blessing of the \sephiroth\dash{}that blessing is what makes them knights.
Mention this when Sethgal appears (at court or with the army).

First in the Carzain-and-bandits chapter:
Carzain has just arrived to the village.
The people of the village are recovering from a recent attack.
Many are dead or violated.
They mourn and curse the \humans.
Make it sad and horrible, so the reader really hates the raiders.
The \scathae complain that this should not happen.
This is not the \Human Age anymore. 
\Humans do not rule \Miith.
Honest \scathae should not live in fear of their depredations.
It is unjust.
Besides, under normal circumstances \scathae are perfectly capable of defending themselves.
It is just this village that is small and peaceful and defenseless, while the \humans are armed and vicious.

Make clear that the village is in the middle of nowhere, an island surrounded on all sides by an ocean of \wylde.
They are part of Pelidor, but cut off from the outside.
A decade ago, the last Vaimon in the village died, and they were unable to find a replacement.
The border totems began to fail, and eventually the \wylde reclaimed the road to the outside.
Now they are alone, and neither the church nor the \rayuth does anything to help them.

Some of the bandits have \nephil blood.
These men have the strength of a \nephil and the petty cruelty of a \human.
The most dangerous traits of both races.
A nasty combination.

They mistrust and hate Carzain for being \human.
But he is a strong alpha male, so he still gets into the village and coerces them into telling him their story.
He is moved (but still sort of stoic).
He is low on supplies and needs the cooperation of the villagers, so he has to do something for them.
He promises that he will get them revenge and retrieve the boy the bandits have kidnapped.
They ask him why he would help them.
He tells them he has no love for his own race.
It is certainly no better than the \scathae.

The bandits hide in some ruins that still have \wylde totems.
It was once a part of the village, but the \wylde reclaimed it and the road to it.
But the totems still have some power.
They keep the worst things of the \wylde at bay, allowing only minor things through.
Minor things that the armed bandits feel confident they can defend themselves from.
(Have some such things, or at least mention them and have the bandits expect to see them. Maybe \nycans. 
The bandits are overconfident if they think they can take on \nycans.)

When he is fighting the last bandit, the bandit asks him why he, a \human, would side with the creeps.
Carzain says he has no particular love for \humans, and certainly not bandit scum like him.

The \Iquinian church contained a core of true Vaimons who made up the higher tiers of the priesthood.
Then there was a larger number of assistant priests and deacons who could not invoke the \sephiroth on their own, but who assisted the Vaimons in prayer and magical rituals.
These deacons and cantors and sextons only knew simple orisons.

In the Tantor chapters:
Tantor-tachi are afraid of the \wylde.
The \wylde is fucking dangerous.
It is a place of twisting, crawling chaos, where nightmarish horrors\dash{}giant animals and worse\dash{}lurk and feed on hapless travelers.
They have a number of Vaimons (Tantor-tachi) with them and several assistant priests.
They pray, sing hymns to \Iquin and Silqua and burn incenses and carry sacred totems.
All this is to keep the \wylde at bay.
The common soldiers join in the songs and sing the chorus lines.
(Have some songs like \quo{Gregoriansk Datalogi}.)

The dead garden is twisting and crawling with chaos.
It is monstrous and haunted.
Rian knows it and is afraid.
He prays to Silqua to protect him from the \wylde.
And he stays close to the totems when possible.
He also has a totem talisman, which he clutches hard while praying.

\Criseis also feels the \wylde in the dead garden.
She feels the chaos seep in from the Beyond, from the Immortal Realms.
It is not unpleasant to her, for she is used to it.
It is a peculiar sensation, compared to being deep in the Shroud.
Her senses are opened up to a much vaster and darker world.
Her ears are filled with unearthly noises
Her nose is bombarded with exotic smells (by \Azmithian standards).

\Tiamat was the mate and partner of \Sethicus.
\Nexagglachel was the son of \Tiamat and \Sethicus, and thus the greatest of the three brothers.
\Ishnaruchaefir was the son of \Tiamat and \Iurzmacul.
\Secherdamon was the son of \Tiamat and \ApepNesthra.

Make the whole world more dystopian!

A \resvil mother always died in childbirth.
Every new \resphan life was bought with death.
Usually it took only days for the mother to revive, though.

Girigor -> \Numah
Another nearby \nephil kingdom was Zibrid.

A \lotha (plural \lothae) was a theropod, like Megaraptor.
The Rissitics used them as beasts of war. 
Perhaps the Rungerans also did.
The Vaimons, on the other hand, relied mostly on herbivorous dinosaurs in war.
Such as sauropods.
And pachycephalosaurs.



2009-07-02

When Rian sees \Ishnaruchaefir summon Rystessakhin from his own blood, he thinks of the myths surrounding \dragon blood.

Moro once drank \dragon blood in a misguided attempt to gain immortality and power.
As it turned out, the truth was more complicated than that.
She and her companions did not know how to prepare the \dragon blood with spells.
So it backfired and scarred her.

Moro regularly kills people to sacrifice to \Nasshikerr.
She does not like it, but she makes sure to kill only bad people, and by now she has grown numb and used to it.
Her killings are actually legal.
She knows Pelidorian law well. 
She can pull up paragraphs that give her, as leader of the \ishrah, the right to do this.
But she does her best to keep it secret anyway.
She does not want her reputation to get any worse, since that would make it harder for her to pull strings.
And she does not want the populace to know that the \rayuth's own \ishrah mages run around and kill people in the night to get blood for their dark gods.
Better to live in ignorance of that.
Ignorance is bliss.
Moro knows that all too well.
Nore that all the above is one of my attempts to make the whole world more dystopian.

Moro has contacts among the city guards.
She pays them and they cover up her shady deals.
Some of them owe her for magical favours she did them, like healing/necromancy or cursing their enemies.

The WSB chapter (or the one before it) begins with Evith getting eaten.
This is our very first glimpse of \resphan culture.

In the siege on \Forklin, have a Rungeran \scatha with some POV scenes.
He is a great and brave warrior.
He is loyal to his king and fights for his country.
He wants to secure a good future for his children.
Runger is \human-dominated, so getting up in the world is non-trivial for a \scatha.
Fighting is his greatest talent, so that is what he does.
All for his children's future.
He leads a charge against the Imetrians and is brutally and swiftly killed by \nycans.

\Semiza needed a lot of old \aryoth relics in order to create \Thanatzil and imbue him with all the power of the greatest ones that \nephil-kind has to offer.
He took some old bones and skulls and mummies and destroyed them in the process.
The priests of the old religion were not happy at all, and neither were the people.
It was hard for \Semiza.
He cursed them for being so backwards and ignorant and unwilling to change.
They could not see that their old religion had failed them and that they needed to look elsewhere.
\Semiza was sad that he was one of the only \nephilim who saw with an open mind.



2009-07-03

\Nexagglachel killed many \resphain before he was killed and captured.
When he revived, he was surprised to see that several of the killed \resphain had also revived.
They possessed True Immortality.
He realized this was a foe to be reckoned with.

The \resphain did not learn the \draconic tongue before they began the Incursion. 
They learned it slowly afterwards, but only few ever learned it and none ever truly mastered it.
The \dragons were not interested in teaching it to them, but the \resphain learned quite a bit from some \quiljaaran whom they got persuaded or coerced into cooperating.

After \Nexagglachel was taken prisoner:
\Nexagglachel learned the \resphan tongue by telepathically attacking a \resphan.
This was a sneak attack that took his captors completely by surprise.
\Nexagglachel was able to kill, destroy and devour his victim before the \resphain could stop him.
He learned much from the \resphan's mind.
He took from him many secrets about their origins, their plans, their technology and magic and their language.

When the \resphain invaded from \Nyx, they and their mortal armies brought with them many new, nasty diseases from \Nyx.
And they, in turn, encountered \Tembraean diseases foreign to them.
The immortals were too resilient to succumb to mere mortal diseases, but the mortal \Tembraeans and \Nyxians were not so lucky.
They died in droves.
\Tembrae was ravaged by deadly epidemies, further compounding the horror and carnage of the \secondbanewar.
But the \secondbanewar lasted more than 100 years, and it was fought chiefly with immortal armies.
The epidemies were a great problem for logistics since they killed a lot of servants and food sources, but they were not as great a problem as some other aspects of that time.
When the \secondbanewar was over, the epidemies had run their course and the survivors had developed resistance.

\Cishiel's mother was not \Shiaraid but some other \resvil.
It was in one of those many and long periods where Ramiel and \Shiaraid were not on speaking terms.
\Cishiel's mother was killed in the \resphanwars, after the \malach fiasco but before the inception of the Cabal.

Ramiel claimed he had never had sex with a \jurid.
He had on some occasions taken prisoners, made them into \jurideth and given them to his men, but he had never raped a \jurid himself, he claimed.
He found it distasteful.

\Morza was a \nephil soldier in \Numah.
He was one of the soldiers charged with protecting the mothers of the \resphain.
When disaster struck and \Thanatzil failed, he took command and led the mothers and the other survivors to safety.
He led them to a promised land in \Nyx.
They found some passages in their traditional religious scripture, combined with \Semiza's new \bane religion, which foretold that this would happen.
He became their chieftain.
For thousands of years after his death, \Morza would be revered as a great hero.
Compare him to Moses and other leaders from the Hebrew Bible.

Carzain's raided village:
The village has only few guns.
The raiders have more guns.
The brave villagers managed to chase off the bandits, but with heavy losses.
They fear the raiders will come back and try to conquer the entire village.
The raiders also have \relcs.
\Relcs are very much military animals.
Armies have them, but few civilians. 
They have other and slower animals.
This means the villagers are hopelessly outmanoeuvred.

There is one female raider.
She fires guns too.
She is the leader's personal lover.
When Carzain attacks, she says: \quo{You wouldn't hit a girl, would you?}
He kills her without mercy or hesitation.



2009-07-04

At least two \sephiroth are specifically charged with safeguarding the sancta of civilization and keeping the \wylde at bay.
These are invoked and/or prayed to in all \wylde scenes.
And people walking in the \wylde almost always carry \wylde talismans blessed by these \sephiroth or other divine beings.
- Carzain (with bandits and when spying)
- The bandits
- Tantor
- Rian (in the dead garden)
- The Pelidorian and Rungeran armies (such as when Carzain spies on them)

Rian knows the dead garden well.
He can find his way in it even though it twists and mutates and paths are never the same.
He has developed an intuition and knows how to recognize safe paths.
He can also tell which parts are dangerous and where the bad monsters might lurk.
Even so, roots and branches grab for him.

\Nexagglachel did much to sow the seeds of doubt, fear, discord and rivalry among his \resphan captors, even when alive.
He would goad them on and manipulate them.
When they tried to interrogate him he would give them \trope{HannibalLecture}{Hannibal Lectures} instead.
He helped shape \Azraid into the very critical \resphan he later became.
And he shaped \Sithiyacaan into a rebel leader.
And \Shiaraid into a dangerous maverick.

The female bandit is named Faeni.
When Carzain kills the bandits, one of them escapes together with Faeni.
He has long lusted after her, but she has always treated him like shit.
Now he rapes her.
He finishes raping her, and she is lying crying and sobbing and shuddering on the ground in pain.
Then Carzain kills them.
It is hinted that he stayed and watched while she was raped.
Carzain tells her that is was her just punishment for being a bandit girl.

After Carzain has killed all the bandits, he is sad to have had to kill the girl.
She was pretty.
Well, semi-pretty.
But he could not forgive her crimes.
And he would never stoop to rape.
Not even when she deserved it.
Besides, watching that ugly rape sort of turned him off.
He is sad that it was a \scatha village and not a \human one he rescued.

Maybe drop the kidnapping story.
Maybe Carzain just comes back to the village with the heads of the bandits.

Have some POV with Faeni before her rape.
Make the reader really, really hate her.
She tortured those \scathae and loved it.
And she hates the guy who later rapes her.
Make her treat him like shit.

Alternately, maybe Carzain kills the guy when he has only groped and kissed and bound and stripped and beaten and humiliated her, but not stuck his dick in her.
Carzain: \ta{I considered letting him have you, as just punishment for your crimes. But it turns out I am a less cruel man that I thought. Lucky for you.}
Then he kills her.



2009-07-05

Ishrah -> \ishrah

In the \resphan chapters of TAR:
Mention that \Ishnaruchaefir hides in his Mirage Asylum, a secret small Realm which only he knows how to find.
It it one of his most valuable resources.
It is how he has kept himself hidden and alive all these millennia.
Such a coward.
If he would only come out of his hole and fight like an honourable warrior, the \resphain could have dealt with him millennia ago. 
Of course he knows this, which is exactly why he keeps his Asylum so secret and guarded.

Perhaps \Ishnaruchaefir needs not attack and destroy stuff in order to be a threat. 
I just need to clarify that if he is not stopped soon (chased away or preferably killed), he will wreck everything they have worked for in Malcur.
When he is at his full strength, he could attack in force and drive the Cabal out of Malcur entirely.
It is known that he takes an interest in Malcur, so he likely has long-term evil plans there.
He gave hints of that in WSB. (Make him give hints!)
That must not be allowed to happen.
Furthermore, even now that he is weak, he might be up to something.
If \Urizeth's conclusions are correct, then these Nadirs happen to him regularly, and if so, \Ishnaruchaefir must have learned long ago to live with them and still get stuff done.
One must not assume that he is harmless in his Nadir.
Maybe it is \Azraid who speculates the above to \Teshrial.

It was, for a large part, Cabal intrigue and treachery that drove the \Ortaicans apart and destroyed their empire.

\Ortaica lasted about 150 years after the death of \Belzir.
SOM takes place only 500 years after \Belzir.

Add years to everything!

Update Carzain's age and stuff.

There are full-fledged muskets.

There were mage-smiths, both on the Vaimon side and other sides.
They could make very powerful armour that protected even against muskets.
So those who could afford it still wore plate armour.
But lower grades of armour were uncommon.
(Fix the bandits to have guns and maybe swords but no armour.)

The Vaimons, especially the Redcor, had a taboo against science and technology.
They used it, but they opposed any innovation and research.
Research that challenged traditionally held views were taboo.
It reminded the Vaimons of \Belzir's heresy, and of heathenism and other wickedness.

The immortals had a printing press, both for paper and graph-glass.
They also had industry, which they used to produce many of their weapons, including most guns.
A lot of swords were still hand-crafted, though.
A mage-smith could pour much magic into a sword and it would make a huge difference.
Guns were easier to mass-produce, and handcrafting did not make so great a difference.
Still, some powerful guns were handcrafted.

The immortals made sure that their industry stayed in the Immortal Realms and did not trickle down into the Shrouded Realms.
They had to consider the Unspoken Covenant.

Maybe there were railroads.
These did not have steam trains on them, but they did have big wagon trains drawn by huge dinosaurs.
The Pelidorian and Rungeran armies had such trains.
Mention them a lot.

Re-design the geography of the northern lands. 
And the south, for that matter.
Make the north into more of a united land mass.
And splinter the Imetrium into an archipelago.

Read GURPS High-Tech.

At some point in the siege of \Forklin, Curwen finds himself in melee combat, where he has a hard time using his magic.
He gets pushed badly.
He realizes that he has not fought for his life with a sword in many, many years.
He is too old for it, now.
So he has to call for Carzain to come and rescue him.

Carzain's Kenosis book -> "Void Angel, Remember"
\Merkyrah book -> "The Murder of the Dawn"

\Dorzand was a \sathariah of \TiphredSerah.
He was very mysterious but charming.
A great manipulator.
A \quo{Count Dracula} type character. 
(The romanticized Dracula, not the actualy Vlad III Draculea.)
He was wise and saw deep.
He was closer to the \banelords than many.
\Azraid dealt with \Dorzand, but did not trust him.
His name is based on Dorozhand, a god from Lord Dunsany's "The Gods of Pegana" (1905), section "Of Dorozhand".

One \sathariah, \Mehaloch, was a cruel \quo{devourer}.
He killed and destroyed a lot in his hunger.
He was killed early on.
Compare to Darth Nihilus from \cite{VideoGame:KOTORII}.

Curwen enters the Ghost Tower:
The Ghost Tower is much larger on the inside than on the outside.
Curwen knows this is a Shroud phenomenon.
In the city, the repressive Shroud twists the mind and the eye and makes the tower look small, and takes a man along paths that make the tower look small.
But in here, the Shroud is weaker, so the true extent of the Tower reveals itself to him.
Or something like that.



2009-07-06

The \Iquinian church had longer, elaborate myths telling about the things that man did not know, and admonishing man that he ought not to know and should not try to find out.
Compare to "Yonath the Prophet" from "The Gods of Pegana".

At the beginning of every \quo{Part} (and possibly in the middle of \quo{Parts} too), have passages from \WanderersInDarknessEmph, \Iquinian myths, \Ortaican myth and perhaps other stuff.
If I can make up enough of it, have some in every chapter.
Make sure they wildly contradict each other.
Even the ones belonging to the same religion.

The Vaimon clans had each their prophets and founding fathers who claimed revelations from the \sephiroth and laid down laws and commandments and spiritual \quo{wisdom}.
Some of these revelations were genuine.
Others were induced by \qliphoth or other immortals.
The clans did not agree on which prophetic works and scriptures were canon, so there was plenty of division in the Vaimon religion already in the time of the empire.

The immortals could not easily unite the empire.
First of all, there was the Unspoken Covenant to consider.
Second, the Sentinels and Cabal were always fighting and trying to fuck each other's plots up.
Third, each faction had plenty of infighting and disagreements, so how could they be expected to prevent the same thing from happening in the Shrouded Realms?

\Ortaican buildings were often adorned with gargoyles.
They looked hideous to \human eyes, less to to \scathaese eyes.
\Forklin was an example of this.
Carzain or Curwen notices.

In the entire history of \Miith, there has existed less than 1000 \dragons put together.
When \Sethicus came to awaken the \dragons, he found that many had chosen to leave their bodies behind permanently and could not be resurrected.
Others had perished for various reasons.
Some refused \Sethicus's call and remained asleep.
But they were still able to pay attention to what happened in the world around them to a degree, through their \daemons and \homunculi.



2007-07-07

\Cishiel had a son or even two.
They were young and had accomplished little.
The younger (if there were two) was just a child, 100 years old at the time of the \firstbanewar.

\Essenai was a \resvil of \Kezerad.
Perhaps a \sathariah, perhaps just a \thelyad.
She was very wise and full of insight.
She wrote a translation of \WanderersInDarknessEmph into the \Resphan tongue.
Then she died.
Perhaps in the fall of \Kezerad.
Perhaps she became a \sephirah.

The \resphain did not form any formal religious beliefs because they feared to think about it.
If they began to ponder the nature of the universe and the meaning of life, they would have to factor in their own nature, and thus their own origin.
They knew their origin: They were engineered by the \banes.
Try as they might, they could not escape that fact.
Some few took to worshipping the \banes as gods, but most could not bring themselves to do that.
They feared the \banes, for the \banes were horrible monstrosities.
They did not want to talk about the \banes.
So the \resphain shied away from the subject of the meaning of life.
Therefore they never developed much in the way of religion and mythology and philosophy.
They had some kind of psychological need for religion, like mortals did, but they filled that need with superstition and informal magibabble such as invoking their \matrices as \quo{gods}, as if the \matrices could be prayed to.
They knew the \matrices did not answer prayers, but like some atheists do, they sometimes succumbed to the psychological urge to pray.

Ramiel's awakening book:
A theme of the latter part of the book is \Dasteron's tragedy.
Ramiel goes to \Mystraacht and meets \Dasteron.
Ramiel hangs out for a while and learns how thing work nowadays.
He greets \Dasteron with respect (of sorts), but no subservience.
\Dasteron reciprocates. (Ramiel is, after all, a \sathariah, and thus in a sense \Dasteron's superior.)
Ramiel comes to \Mystraacht intent on hating \Dasteron, this pathetic usurper who thinks he can be Overlord.
But Ramiel finds, much to his chagrin, that \Dasteron is a good \resphan, a worthy leader and even a potential friend.
This complicates matters.
Because Ramiel also knows that however good \Dasteron is, Ramiel has to dethrone him.
\Dasteron knows this as well.
Sooner or later, Ramiel will challenge him for the throne.
Here is the real tragedy:
Ramiel knows that if he loses their match, he will not submit to \Dasteron's rule.
He will challenge him again and again until he wins, for Ramiel must win.
He also comes to know \Dasteron well enough to know that \Dasteron will do the same if Ramiel wins.
They are both full of pride.
Each believes that he is the better leader, that only he is worthy and capable of leading \Mystraacht.
\Dasteron believes that Ramiel, for all his brute force, is too insane and unstable and dangerous to have as Overlord.
So they conflict is doomed to continue forever, until one of them perishes.
So Ramiel makes a very hard decision.
He will destroy and eat \Dasteron as soon as he wins (if he can, that is). 
This is tragic.
Ramiel will miss \Dasteron.
He is a great \resphan and a friend.
It will be a great loss for \Mystraacht and for the \resphan race, and for Ramiel, and for \Cishiel.
But it must be done.

The battle itself, despite what they try to show the spectators, is a grim, sad affair.
There is no hate or anger, so there is no enjoyment.
It is just a grim and gritty fight to the bitter end.

After this, Ramiel is very serious and determined and weighed down with responsibility.
Not at all the reckless adventurer he was in his youth.
Compare to William Adama from Battlestar Galactica.

Ramiel was killed many times in the rebellion (the War of Awakening), the \secondbanewar and the \resphanwars. 
He was brave and reckless, as a true \resphan man.
\Shiaraid died much less often, since she was sneaky and kept to the background, as a true \resvil.

The \resphain were driven to a crazed fury during the War of Awakening.
They were mad with bloodlust and with the desire for change.
It is what made them so insane and violent.
It is what made even sensible and compassionate \resphain commit terrible, bloody atrocities.
It is what caused them to embrace such a twisted Religion of Evil and wage a war of genocide against their own people and their hapless servitors.
Their true, chaotic power brought a terrible rush.
Only later as they got more used to their powers did they learn to get over this rush and still maintain a level head.
Many came to repent their evil deeds during the rebellion.
Many of these repenting ones ended up joining \Kezerad.



2009-07-09

The mortal underlings of a \resphan \dynasty were called \hedrim, singular \hedor. 
The \hedrim[\Mystraacht], for example, were the \hedrim that served \Mystraacht. 


Vizicar notices the statues in \Forklin (of \sephiroth and mortals). 
He looks at a statue of \Feazirah. 
She is depicted kneeling with her head bowed, since she is the \sephirah of Humility.
He concludes that they must be newer than the Empire's time. 
Vizicar did not know every piece of his empire (it was huge), so for all he knew, this place could very well have been the site of a major Vaimon city back then. 
The \Ortaican parts could be newer. 
And he does not have the architectural expertise to tell which is older, the \Ortaican or the Vaimon parts of the architecture. 
But one thing convinces him: 
The statues are clothed. 
In his time, this area was dominated by Clan Sether, who were among the least taboo-afflicted clans.
Their statues were naked. 
Nowadays, the area is dominated by Redcor and \Telcra.
Redcor have always been afraid of sex. 
\Telcra have inherited taboos from the Redcor. 
So their statues are clothed. 

\begin{prose}
  Carzain: 
  \ta{So... you look at a kneeling woman and get disappointed because she's not naked.
    And then you spin a long story of history and architecture to hide that.}
  
  Vizicar:
  \ta{Exactly.}
\end{prose}


\Ghobaleth were horrors from Beyond. 
Perhaps \hr{Horrors from the void}{from the void}. 
They were terribly dangerous even to the \banes who commanded them.
Compare them to the shoggoths from H.P. Lovecraft's "At the Mountains of Madness". 


\Nexagglachel, \Ishnaruchaefir, and \Secherdamon were Elder \Dragons, Antediluvians. 
They could not be roused during the \firstbanewar for some reason. 
Later they were awakened and conquered the world, taking it back from the \aryothim. 


All awakened \dragons died in the \firstbanewar and its aftermath.


\Isphet was an evil figure in \Iquinian mythology. 
He was an enemy of the \sephiroth. 
Sometimes described as a \qliphah. 
A nameless \qliphah of the Midnight Circle. 
Sometimes he was considered the king of all \qliphoth and the master of all that is evil. 
(A few doubted that he was a \qliphah at all.)

\Isphet was in continual battle with the \sephiroth.
His titles included \quo{the Adversary}.
His name also appeared in the variant Iscraphet or Iscraphel. 
In legendary times, he had tried to destroy the world. 
The \sephiroth and their angels had attacked him in force. 
They fought against the Adversary and his legions of \qliphoth. 

"And there was war in heaven. Michael and his angels fought against the \Dragon..."

The \sephiroth prevailed and cast out the Adversary. 
From then the craven villain would hide in his dark pit of evil and only rarely dare to venture forth into the world of mortals. 
To this day, the myth said, the \sephiroth were locked in a cosmic battle with \Isphet. 
That was why the \sephiroth did not show themselves in the mortal world. 
And their believers had to help them combat him. 
In churches they performed rituals at regular intervals that were meant to keep \Isphet at bay.
He was immortal and would not perish until the end of the world, but he could be wounded and weakened and mutilated.

Compare to Egyptian mythology, where there were spells to overthrow the \dragon Apep. 

\quo{Isfet}, as far as I know, is an Egyptian word that means \quo{chaos} and is associated with the serpent Apep, the eternal enemy of Ra, the Sun god. 

\Isphet was described as black with burning eyes, and sometimes as wreathed in fire and smoke. 
He was based on a memory of \Ishnaruchaefir (who did sort of destroy the world), and to a lesser degree \Secherdamon.

Rian knows \Isphet, believes in him and fears him. 
WSB: Have no \Isphet references. We do not want the reader to suspect just yet. 
In the later Rian chapters: He prays to be delivered from \Isphet's evil. 
Have a scene with Rian in church where he attends prayer and mutilates \Isphet. 
There is an effigy of \Isphet there in the form of a black serpent. 
Every church-goer is handed a needle or stick with which to impale the monster. 
At last, the effigy is hacked into pieces and burnt. 

Mention \Isphet and the \qliphoth in the scene with \Icor's funeral. 

The myth of \Isphet was made up somewhere in the Vaimon Age. It was unknown in Cordos Vaimon's time. 



Tantor is certain that \EreshKal was not built by \meccaran hands. 
But his version of the story is very much coloured by his racism. 
He looks down on \meccara. 

Be sure to have a clear difference in Tantor's writing style before and after his son's death. 
Before it, he looks down on the \meccara as \quo{lower \humanoids}. 
They are repulsive, but he also sort of pities them. 
And he is impressed and awed by the grandness of the temple. 

After his son's death, he comes to viciously hate \meccara. 
He curses them for their evil, stupidity, inferiority, ugliness, bad smell and everything he can think of. 
And he now feels horror rather than awe at the temple. 

Already out in the forest, he and the others see abhorrent things skulking and creeping at the edges of their camp. 
Ugly midget \humanoids. 
Probably \meccara. 
Sometimes the soldiers shoot at them. 
One soldier hits. 
The thing shrieks and bleeds, but escapes, and no one wants to pursue the wretched thing out into the \wylde. 
But they can see the blood. 
They know it is a living creature that bleeds. 
That reassures them. 
A bit. 

Inside the temple they see more of these skulking shapes. 

On the last day, after the big battle, they hear slavering noises.
They realize there are more \meccara, and they have brought monsters with them. 
They fear it is their gruesome gods (whose images and statues they saw in the big chamber) which have now awakened and are hunting the interlopers. 
They flee out quickly.
Several are grabbed by monstrous horrors. 
Those that stop to try and help them are also killed. 
There is nothing to do but run. 
But in the end, many survive, and now they have the valuable plaques they came for. 



To keep the \wylde at bay, people used \eidola. 
And \eidolon was a magical totem blessed and enchanted to protect civilization and keep away the \wylde and its denizens. 
Every religion had its way of creating \eidola. 
Have them in every \wylde scene. 



There were no cameras on \Miith. 
They could not be built like on Earth because of the different way \Miithian physics worked. 
An eye picked up images from many dimensions and the brain sorted out most of it (because of the Shroud and other things), so that the mind saw only a few of them. 
A photograph would be a mishmash of all those images from all those dimensions. 
Without a bigger context, the eye would be unable to make sense of the images on a photograph, so it would just be unrecognizable static. 

The immortals had high enough technology that they might reasonably be expected to create cameras, but they never did. 
The \ophidians, though, could make holograms using magic. 
These holograms extended out into all dimensions and were just like physical objects. 



\Byakun (with a circumflex over the U) was a dark priest who lived in Silqua's time. 
He was a dark mage and very feared. 
Compare him to various Lovecraft figures. 
Silqua's people were afraid of him. 
At some point, Silqua (somehow) heroically volunteered to be sold to \Byakun as a slave. 
She did some hero work, and Cordos came in and saved her. 

Compared to the Mahrkagir in Kushiel's Avatar, who takes \Phedre as a slave. 



The Silqua gambit ended up going astray because the Cabal was still newly formed and the different factions fought much against each other. 
Plus, the Sentinels were working against them. 
This chaos gave \Delphine free reins to have her love affair with Silqua, which became more and more depraved and ended with her killing Silqua. 



Moro \Cornel knew much about the Beyond, the occult, the \Miith Mythos. 
She was a Cthulhu style investigator. 
In her youth she did much exploring, adventuring and research. 
That was why she became so scarred and bitter and unhappy. 

The lesson: Being an adventurer does not pay on a cruel, horrible world like \Miith.

Moro knew of the \quo{Elder things} that would one day return to conquer \Miith and overthrow all mortal civilizations. 
But she did not know which creatures were on which sides. 
She knew the words \quo{\xs} and \quo{\bane}, but she was not quite sure what they referred to and what the distinctions were, let alone who opposed whom. 

When Moro kills bandits:
  She could use much more nasty magic on them if she really wanted, but she does not want to attract attention, so she uses only moderate magic. 

\Nasshikerr scenes:
  Moro has a sinister statue of \Nasshikerr which she uses to summon him.

  Make \Nasshikerr more frightening, more awesome and loathsome.
  Like Tsathoggua.
  He has big, rolling chamaeleon eyes. 
  Moro fears him, but she is smart enough to suspect that he knows more than he is telling her. 
  The gods always are. 
  She resents the fact that he is like that (arrogant, superior, unwilling to \cooperate with a mere mortal), but that is how gods are.



End chapters on a cliffhanger! 
All over the place.
Such as the one where \Urizeth dies. 
Have it end with the line: \quo{\Urizeth is dead.}
(Maybe it is too short to justify a chapter, so make it just a short \Teshrial scene at the end of some other chapter.)



2009-07-10 

Abonner på PROSAs arrangement-kalender.



2009-07-11

In \Iquinian mythology (and in the view of many \humanoids of the \Human Age and Scatha Age), \dragons were godlike Elder horrors who ruled \Miith with terror in the dark days of chaos before the \sephiroth took action and created the Vaimons to throw them out.
The \dragons were the spawn of chaos, the kin of loathsome alien gods and wielders of the blackest sorcery ever conceived.
According to myth, \Iquin created the world long before.
The \dragons and other forces of evil also existed, for good cannot exist without its opposite.
The first generation of mortals (pre-\humans) were sinful and displeased the One Light.
So the \sephiroth abandoned them and allowed the \dragons and other evils to overthrow the mortals, overrun the world and rule it with terror for an Age.
After an Age of the World had passed, the \sephiroth took mercy on the world, and so they once again shown themselves.
They revealed themselves to Silqua and made her create the Vaimon order, so that \humans (having been punished enough) would now drive out the forces of Elder evil and rule the world again.
Those Elder mortals were \humans in some versions of the story.
Others held that they were the \nephilim, thus justifying hate and persecution against the \nephil race.
This myth is based, to some limited extent, on the true story of how the \aryothim once ruled the world before the return of the \dragons.

In truth, \dragons were not a dying race.
Their number had remained almost constant since the end of the \secondbanewar.
Remember, it took an assload of \resphain to take down just a single \dragon.

Forum: How to clarify that myths (like the above, presented in fragment in the text) are not necessarily true?
Perhaps make it clear that the text we see is just one of several possible translations, and not considered canon by all (perhaps even heretical by some).

\quo{Light}, in the sense of \quo{\Iquin}, should be something else.
\quo{The flame}, \quo{the beacon}, \quo{the blaze}, something less cheesy.

Rian thinks of \dragons:
He thinks of how horrible and evil \dragons are, and how lucky it is that they no longer exist.
He does not doubt that they once existed, though.
He is a religious boy and believes in the myths.

WSB:
Rian feels great terror when he sees \Ishnaruchaefir.
He wants to flee, but curiosity compels him to stay, follow and watch.
He curses himself and prays to the Light.
He knows he should not be curious, for curiosity is not a virtue.
But he cannot help himself.
He prays for strength to resist this temptation.
When that fails, he prays for forgiveness.
And he prays to be safeguarded when he sees the terrible battle.

Carzain comes to \Forklin:
He sees traces of \Ortaican art and depictions of \dragons.
The \Ortaicans liked \dragons and even worshipped them.
The \Iquinians see \dragons as evil incarnate.
Carzain himself does not know what to think.
But he does think.

The introduction with \Nzessuacrith:
Make the \dragons darker, more demonic, more sinister.
Less sympathy and Woobie-ness. More badass and Cosmic Horror.
Emphasize their coldness and alien-ness.
\Nzessuacrith gazes out over the world with cold \ophidian eyes.

It was possible to encounter the sleeping \dragons when dreaming.
Like Cthulhu, they reached out to touch the minds of mortals.
And they could use mind control.

First\banewar: 
\Sethicus lay entombed, but not unconscious.
He realized the \banes were invading.
So he reached out telepathically and touched the minds of \ophidians.
He was deeply buried and bound with fearsome spells, so it was not easy for him.
He could only send unclear visions and messages.
But his mental power was formidable, and he was able to compel the \ophidians to release him.
This would have been impossible at any other time, but now they knew they needed him and his \dragons, so it was easy to convince many powerful \ophidians to come and release his seals.
\Sethicus then went to work releasing his fellow \dragons.
He awakened many, but not all.
Some slumbered too deeply and could not be reached.
Others refused the call.
Yet others he let sleep.
This included his three sons.
It is possible that \Sethicus knew there was a great risk that the \dragons who fought in this war would be destroyed, and that he wanted them to live on so they could awaken at a later date and reclaim \Miith, thus carrying on his legacy.
It is known that he did some work to weaken the seals on their tombs so that the brothers would later awaken on their own.
\Sethicus was wise and saw far.

The Mirage Asylum was built on fragments of \Ishnaruchaefir's old tomb in which he had slept for a million years.

The imprisonment and entombment of the \dragons was also called the \quo{Durance of \Sethicus}, occasionally the \quo{Durance of the \Dragons}.

Køb Drager og Dæmoner.
Køb Call of Cthulhu.
Gå evt i Fantask og lignende.
Check out: Bibliotek.dk RSS feed?
Get the Lin Carter Necronomicon.

Undersøg muligheden for Open Source job

Change geography:
\quo{\Velcad} refers only to \Galessan.
There were several \Velcadian languages, all of them descended from \Tepharin (which, in turn, borrowed heavily from both \Ortaican and Vaimon).
Pelidorian and Rungeran were different but related languages.

Get rid of the \Velcadian Empire.
Replace it entirely with \Tepharae.
Get rid of the Tigers.
Maybe replace them with a \Tepharin order named after a dinosaur.
Maybe the \Tepharins came from the isle of \Velcad or something like that.
But rename the isle.

\Yormis was an \Ortaican city.
It lay in northern \Velcad, near Pelidor.
After the fall of \Ortaica and the rise of \Iquinian \Tepharae, \Yormis did not surrender and remained a bastion of rethyaxes that maintained the \Ortaican religion.
It became known as a dark city of sorcerers, and there were many unwholesome rumours about them.
\Yormis lay near the mountain of \Shrun (with a circumflex over the U), under which dwelt the \xs godling \Ubloth.
\Yormis was ruled by a bit of an anarchy of cults and mage-schools.
Some worshipped the \Ortaican gods.
Others worshipped darker powers.
A few knew that the \Ortaican gods were \xss.
The Dark Crescent had much power in \Yormis.

Moro Cornel came from \Yormis and was educated there.
There she took her first steps on her path to dark enlightenment.
Which she would later lament.

Ghobal -> Nogg-yal, pl. Nogg-yaleth.

The \noggyaleth were truly ancient creatures, older than the \ophidians.
They were native to \Miith and possibly kin to the \xss.
Some \draconic myths described them as the spawn of \RuinSatha or \KyaethemChreiAz, but this is very vague and uncertain.
Perhaps they were even the first life on \Miith.
When the \voyagers came to \Miith, they took the \noggyaleth and altered them to make a race of intelligent, powerful, versatile slaves.
They were originally completely mindless, but the \voyagers gave them intelligence.
Perhaps the \noggyaleth rose up against the \voyagers and drove them out.
Perhaps the \voyagers were driven out by the \xss and the \noggyaleth just remained.

The \ophidians did not know the \noggyaleth's true nature, although a few occultists suspected.
The \ophidians feared the \noggyaleth just as they did the \xss, but the \ophidian magic was mostly powerful enough to keep the chaotic, bestial \noggyaleth at bay or even destroy them.
In earlier days, the \ophidians waged great wars against the \noggyaleth and drove them underground.

The \noggyaleth were deeply and tightly connected to the Heart of \Miith, being the first life.
All other life, even \ophidians, somehow descends from the horrid \noggyaleth.
They embodied creative chaos, but not intelligence and planning.

\Banes were related to \noggyaleth.
When the \voyagers created the \banes, they used some \noggyal matter to make them.
The \banekings learned of the \noggyaleth's existence.
They realized that there lay the path to their future.
If the two races could assimilate each other, they would be greater than their creators the \voyagers in all things.
They would possess the best of both worlds: 
Cold intelligence and wild creative chaos.
When the \banes destroyed the \voyagers, they quickly made plans to go to \Miith and unite with the \noggyaleth.
But the \ophidians, who were the heirs to the \voyagers on \Miith, would have none of that.

Individual \banes could not just merge with individual \noggyaleth.
The \baneking himself must come to \Miith and absorb into himself the great mother-mass of the \noggyaleth, which lay seething and bubbling deep beneath the planet's surface, feeding directly on the Heart.
In a sense, the \noggyal mother-mass was an extension of the heart.

\Noggyaleth looked shoggoth-like:
An amorphous, ever-twisting mass of slime that only vaguely resembled flesh.
A \noggyal was flexible and chaotic and could reshape its body at will.
Often they would be covered with eyes and other sensory organs.
They had little individuality but could join together to increase their intelligence.

  And then, at length, there flashed upon my vision one glimpse of a depth and of an abomination more horrible than any that I had glimped before. I looked upon a foul black pit, with a carven rim of beslimed rock about it, all drowned in Plutonian gloom, litten only by the vile phosphorescence of the primal white jelly of the proto-Shoggoths... and amidst the hideous slime and the obscene stench I saw the bubbling, quivering plastic horrors, those shuddering towers of gelatinous, liquescent filth, studed with naked and protruding and staring eyeballs... and I shrieked, and fled, back down the pathways of space and time and dimension, knowing in that last, soul-blighting glimpse the nodding flowers that blossomed in the scummed shallows of that lake of bubbling filth\dash{}\emph{and shrieked, and fled, knowing at last where the Black Lotus bloomed, and upon what unspeakable slime it feeds.}
}

Compare \noggyaleth to beholders from D&D.
They stank hideously.
Their smell was alien and unnatural, but yet familiar in a way that was so deeply disturbing that no mortal could bear to consider it.

The \noggyaleth were great burrowers.
With their acidic secretions they could burrow holes and tunnels through the ground as well as through the barriers between the Realms.

Get rid of the \quo{kraken} or merge them with the \noggyaleth and \xss.

\Teshrial did not seek out \Urizeth late in the story.
She was working with him all along on the Malcur gambit.
The purpose of the gambit was to bore a hole from \Nyx and into the deeps of the planet \Miith.
They would build a conduct to the life-giving Heart and provide life and fertility to their race, which had long been dwindling.
It was a great and glorious undertaking.
Many did not believe it would succeed, but \Teshrial and his associates were enthusiastic.
He looked very much forward to it.
When the great bridge was complete, he would take \Firaxel to it, and they would have wonderful sex, and she would conceive, and he would be a father and a hero.

The \resphain employed \noggyaleth, but they did not understand them.
They thought of the \noggyaleth as bestial, semi-intelligent things, tamed and subdued with the power of spells.
They did not suspect the true extent of the \noggyaleth's intelligence and power, nor their ties to the \banelords' long-term plan.
They assumed they had achieved power over the \noggyaleth, but in truth the \noggyaleth had more freedom and plans of their own than the \resphain realized.

There were some \noggyaleth under Malcur.
They were part of the Cabal's plan.
They would drill the hole through the dimensions and pave the way for the bridge.
\Urizeth was there.
She was the party occultist and charged with dealing with the \noggyaleth.
There was one of several \noggyaleth.
It did not quite make sense to try to distinguish between individual \noggyaleth, for they would merge and split apart when they willed\dash{}or when their masters told them to.
\Teshrial feared the \noggyaleth and did not understand them.
He left it up to \Urizeth to manage them.

Merge \Urizeth with Selguin.
She is \CiriathSepher.

The \banes had a darker plan for Malcur.
It was really a part of their master plan for opening the way from \Erebos to the Heart of \Miith and the \noggyal mother-mass.
The \noggyaleth played a dark, terrible role that the \resphain did not suspect.
The \resphain just used the monsters, stayed away from them and refrained from asking questions.
They feared both \noggyaleth and \banes and did not want to know any more than they already did.

\Urizeth's death is a great setback for the Malcur project.
But there are deputy occultists who can take over and manage the \noggyaleth until she gets better.
When she returns, she can probably still manage them.
And if she chooses to back out entirely, they can find another \noggyal-handler.
The astrology problem is worse.
\Teshrial has no backup astrologer.
And he is convinced she will back out of their project to kill \Ishnaruchaefir.

At first, \Teshrial did not like \Urizeth.
Later he warmed to her, after her display of loyalty and bravery and determination, enough to rival his own. 
(He is, after all, much physically stronger than she is, and so in less danger of being destroyed. He is a ketheran, male and a great martial artist. She is \thelyad, female and a nerd.)
She also warms to him.
At first she sees him as an arrogant, self-important, condescending snob.
But later she realizes he is a fairly decent \resphan with noble motives.
And he learns to lose some of his snobbery.

Read: Richard Tierney - The Winds of Zarr (Yog-Sothoth is unleashed at the time of Moses)
and other Tierney

Moro Cornel -> Cobrel

The \jinn were disembodied horrors of the void.
They haunted the empty deserts, howling. 

\Dasteron does much good work to clean up \Mystraacht.
It had degenerated into a gangster-like den of petty greed and brutality. 
Barbarism.
\Dasteron has a long-term vision.
He unites \Mystraacht and restores it to honour, glory, dignity and purpose.

Remember that \KaiLeng was also a Realm.
There was more than one such underground Realm. The Chthonic Realms.
Two or three in all.
And three or four sea Realms. The Aquatic Realms or Pelagic Realms.
And up to 10 Shrouded land Realms. These ten were called the Tellurian Realms.
The underground Realms were thinly populated, and most immortals attributed to them less importance than they should.
Only the wise \dragons and the \banelords and a few \resphain understood the significance of the underground, Chthonic Realms.

Lin Carter Necronomicon, 2.1:

\citeauthorbook{LinCarter:TheNecronomiconTheDeeTranslation}{Lin Carter}{The Necronomicon: The Dee Translation (part II.I)}{
  Knowest thou this, that of all the arts and crafts and sciences whereunto may mortal men aspire, supremest and most potent of them all be the practice of Magic.
  Yet indeed, as \emph{Ibn Shoddathua} sayeth in his commentaries upon the Papyruses of Mum-Nath:
  Many are they who lust for the Mastery thereof, but few indeed are they who ucceed therein.
  For the wise magician is the Master of Nature and the archpriest of all her Mysteries; at his command there openeth forth the Grave of Sod or the shutten Sepulchre of Stone, to admit forth they who slumber therein; before the utterance of his will shall storms becalm themselves, and floods retreat back into the secret fountains of the Deep, and conflagrations extinguish their fiery flames.
  
  Aye, and verily can he call down from beyond the stars That which abideth in the dark and freezing spaces of the Void, or forth from the Pit may he summon That which resideth in the black and frightgul abyss; spells and enchantments may be cast upon even the holiest of men or they that be purest of heart. 
  I say unto thee that such power may the accomplished Initiate command that nations shall grovel before his awesome might, and that the very Kings and Princes of the Earth shall flock to do him homande and obeisance.
  Even the very life of the Sorerer may be by his Art extended far beyond the ordinary limitations set upon mortal men, aye, and verily, for untold centuries mat he thrive, untouched by Time.
  For, behold! doth he not wield the keys of Life and Death? wherefore shall all mere mortal men exalt the Master thereof, and grovel at his feet, \emph{i\"a Nyarlathotep}! 
  The Wise Magician is a very mighty god.
}

Forum: To what extent does there exist a Cthulhu canon?

\KhothSell is a goddess of life also.
Merge her with Shub-Niggurath.

\RuinSatha was a \xs. 
He was a chaotic dominator, said to reign from a basaltic throne at the seething and fiery centre of Chaos.
Some called him the greatest of the \xss, but such claims were ever uncertain.



2009-07-13

\Vorcanths were dark, terrible, mysterious things like the Hounds of Tindalos from the Cthulhu Mythos. 
They were some of the horrors that lurked in the Realms Beyond and would sometimes prey on \Miithians. 
Even immortals were subject to their depredations. 
Even their \resphan allies never understood them very well. 


Visha's Realm was a mostly tranquil but eerie place. 
Haunted by terrible predators, the \vorcanths. 

Lin Carter's "The Necronomicon: The Dee Translation", 1989, p.175

\citeauthorbook[p.175]{LinCarter:TheNecronomiconTheDeeTranslation}{Lin Carter}{
  The Necronomicon: The Dee Translation (part I.VII.III)
}{
  Aye, be thou warned, for in all such voyages and venturings of mind or soul or spirit there be very great and terrible dangers, by mortal men undreamt-of and unknown. 
  Beware then, lest thou penetrate too deeply into the blackest backward and depthless abysm of the womb of infinite time. 
  For beyond the very Beginning thereof, and on the Other Side thereof,there dwelleth That of which man suspecteth not; and there thou wilt find a strange and ominous Realm where hidden horrors lurk and naked Terror hunts unseen; which dim, uncanny bourn hath the seeming and the semblance of a pale, and grey, and indefinite shore, lapped by the sluggish waves of unmeasured and unthinkable Time.
  And it is eve there, in an awful Light that is beyond all darkness, amidst a profound Silence that shieketh beyond all sound, that \emph{They} slink and prowl in all their ghastliness, slavering with a loathsome and ana unspeakable hunger for all that is clean and whole and unsullied.
}


The \xss ruled a vast interstellar empire spanning thousands of planets, perhaps millions. 
They knew secrets of dimensional travel that few races had achieved.
The \banes did not have such a dimensional travelling skill. 
This technology was one of the reasons why the \xss had grown to be such a successful and powerful race. 


In the far future, after the exodus from \Miith, the \dragons and \resphain ended up ruling each their vast interstellar empire spanning dozens or hundreds of planets. 
They became super-technological and super-magical civilizations close to rivalling the \voyagers in technology, but far surpassing them in aggression and greedy ambition. 
The \voyagers were a peaceful (if decadent) race of scientists and explorers. 
The exiled \Miithians were warriors and conquerors at their very core.

At this time, the \dragons and \resphain had found new \dweomers to sustain them, so they no longer had to fight each other for possession of the Heart of \Miith. 
But they were warriors by nature and instinct, so they still fought one another. 



Have a scene late in the story with Moro \Cornel:

  Moro captures and interrogates a Black Plague gangster.
  From him she learns where the abductees are kept. 
  Rian is with her.
  He insisted on being there to help.

In the scene where Moro kidnaps a robber to sacrifice:

  She interrogates her victims before she kills one.
  She wants to know whether they are Plaguers and whether they may know something.
  It turns out they are members of another gang which is known to oppose the Plaguers (and, as far as Moro knowns, is not involed in anything supernatural). 
  She is disappointed.
  She had hoped to find a Plaguer to interrogate.
  She has long tried to find a Plaguer to interrogate, but taking bandits captive is difficult. 
  They are sneaky bastards.
  Since her suspicions began she has only managed to capture one Plaguer, and he was a low-level nobody who knew nothing. 
  
  Also make it clear that she usually just sacrifices animals to \Nasshikerr. 
  Only once in a while does he demand a \humanoid.


\Miith -> Atziluth?
         Beriah/Briyah?
         Yetzirah?
         Assiah/Asiyah?
Also read more about the worlds of the ten Sephiroth (mentioned in Lin Carter's Necronomicon). 

Carzain Shireyo -> Morningstar or Lightbringer or something that smacks of \quo{Lucifer}
Give Carzain an Arabic- or Hebrew-sounding last name that means something like \quo{light-bringer} or \quo{light-bearer} or \quo{morning star}.
Perhaps the original Hebrew name for \quo{Lucifer}. Find out what that is!
Ask Marianne.
Read Wikipedia about the myth of Lucifer.
And see the \quo{Jewish Encyclopedia}.
\quo{Carzain ben Shahar}?


The \firstbanewar nuked the planet to smithereens. 
After the \banes had been banished, \Miith was bathed in fallout, both natural and supernatural. 
And ravaged by summoned monsters (many of whom still dwelt in the dark corners of \Miith millennia later). 
The surviving \ophidians were few. 
They tried to rebuild their civilization, but they failed. 
Their empire declined over the next several thousand years until there was nothing left. 
To make it worse, they waged wars against each other, too. 
The great war had made the survivors more violent and xenophobic. 



\Sethicus did not awaken \Tiamat's sons when he himself awoke. 
\Sethicus suspected that he and his fellow awakened \dragons might very well perish in this war, and he wanted to be sure the \draconian race would be carried on. 
So instead he rigged some spells that would awaken \Nexagglachel some thousand years later.
He also set up some other powerful \dragons (whom he sort-of trusted) to wake up. 
He wanted to be on the safe side and not rely on \Nexagglachel alone.
For all \Sethicus knew, something nasty might happen to \Nexagglachel in mean time, so he wanted a backup plan. 

When \Nexagglachel awoke, he immediately went out to awaken his two brothers.
When \Ishnaruchaefir was awakened by \Nexagglachel, he immediately went out to awaken \Rystessakhin. 

The awakened \dragons could not rebuild the \ophidian civilization. 
\Nexagglachel tried his best, but it was hard. 
They were only a few individuals.
They had much scientific knowledge, but it was incomplete.
And the \quiljaaran were not so great. 
They were slothful and uncreative and unambitious.
After \Nexagglachel died, only \Secherdamon retained the ambition to rebuild the \ophidian empire any time soon. 

The \dragons did make some progress, though. 
They introduced a lot of technology among their followers.

The \aryothim developed technology that, in some areas, was superior to that of the \quiljaaran. 
The \aryothim did not have nearly as powerful magic, but they had high-quality guns and stuff.
Perhaps they even had 20th century technology. 

At the time of the \secondbanewar, technology on \Miith was high on both sides. 
\Miithians and \resphain each had some weapons that the other side lacked, making the first battles full of nasty surprises for both sides. 



Have many dark, unexplored \quo{here-there-be-\dragons} places in the \wylde. 
Even in \Velcad. 
In the \thirdbanewar period as well as in earlier periods. 
Compare to places from the Cthulhu Mythos:
\begin{itemize}
  \item The Vale of Pnoth.
  \item The Forest of Zoogs.
  \item The Peaks of Throk.
  \item The Vaults of Zin.
  \item The Tower of Koth. 
  \item Kadath in the Cold Waste.
\end{itemize}
Among other things, have a dark valley of naked, black basaltic pillars, inhabited by Gug-like monsters. 
And have places where the \quiljaaran live. 
The \serpentmen were known from legends and feared. 



Some \xss dwelt on \Miith itself, such as the one near \Yormis.
The Shroud made them increasingly sluggish and slothful. 



In the \wylde in \Azmith one can occasionally see giant \umbrae soaring high above. 
Amorphous fearful shapes that cast a vast, ominous shadows.
Hence the name \quo{\umbra}. 



2009-07-14

Make a general section about \xss and their hierarchy. In fact, merge all the \xss into one section. 

According to some traditions, \NaathKurRamalech was the greatest of the \xss, not \RuinSatha.
He was certainly the \pps{\dragons} most important ally when in came to protecting \Miith from the \banes. 

Some believed that \NaathKurRamalech was younger and weaker than \RuinSatha, but simply seemed more powerful because he was closer tied to \Miith and took slightly more interest in the doings of his \Miithian worshippers. 



When \dragons felt the need to express great emotion, they would often switch to True \Draconic. 
Examples: 
  \Ishnaruchaefir and \Rystessakhin at the \Shrouding. 
  \Ishnaruchaefir and \Nzessuacrith and \Secherdamon in \TwilightAngelRemember.



Move Nemuragh and Lothagiel to \TiphredSerah. 



\Ishnaruchaefir's Nadir happened at times when the \quo{tides} of the Shroud were low, meaning that the Shroud was weak and permeable. 
At these times, \Ishnaruchaefir had to work hard to keep the Shroud stable. 
This hard spellword was what made him weak. 
He had to open himself up to the world in order to pull its strings, and this openness made him vulnerable. 

If he failed, he risked a nasty backlash against himself and his Mirage Asylum. 
The Asylum was unstable by nature and prone to collapsing or flying off into space if not maintained. 

It was unclear whether the Shroud itself might collapse without \Ishnaruchaefir's support. 
\Ishnaruchaefir himself tended to believe that his work was necessary to keep the Shroud alive.
His critics tended to think his work was of purely local significance. 

TAR: 
The time when \Secherdamon plans to resurrect \Nithdornazsh coincides approximately with \Ishnaruchaefir's Nadir.
This is not by chance. 
In this period, the Shroud is thin, so they have a better chance of succeeding, breaching the barriers between the worlds and bringing their citadel to \Azmith. 

The Cabal's Malcur venture is also mouthing out into some conclusion at this time. 
Or was supposed to. 



Compress the history of \Miith.
I.e., make the periods shorter. 
Fewer thousands of years. 
Now that the \draconic major characters are moved millions of years back, I have plenty of impressive big numbers, so I can relax on the more recent history. 

How old are the \dragons?
\Nzessuacrith was as old in the \secondbanewar as Rathyon was in the \thirdbanewar.
\Ishnaruchaefir was at least as old as that when \Nzessuacrith was born. 



2009-07-15

When \Ishnaruchaefir is in his Nadir he bleeds, and one can see a myriad long, aethereal tendrils radiating out from him, through which power drains out of him to sustain the Shroud and the glaive. 

Compare him to Anomander Rake in Darujhistan in \cite{}

Before WSB, when \Achsah detects \Ishnaruchaefir:

  They know that \Ishnaruchaefir is clearly up to no good.
  \Urizeth fears he has caught wind of their doings in Malcur and will try to interfere and stop their plan.
  That must not happen. 
  Malcur is vital to the future of \CiriathSepher (or so this splinter group likes to believe). 
  \Ishnaruchaefir could fuck up the \noggyaleth and everything. 
  That must not happen. 
  But he is an immensely powerful \shaeeroth \dragon.
  What can they do?
  They are despairing.
  
  Then \Teshrial steps forward. 
  He is brave and will go out to meet \Ishnaruchaefir. 
  If he cannot scare him away, \Teshrial can at least keep \Ishnaruchaefir at bay long enough for the others to cover their tracks, retract the \noggyaleth so he cannot fuck them up, and retract their own tendrils of \vertex power so \Ishnaruchaefir cannot guess their plans..
  
  They are all impressed, but they also warn \Teshrial about how dangerous it is.
  \Teshrial is brave. 
  He stands his ground. 
  
  Portray \Teshrial as well-meaning and idealistic, but also arrogant and badass. 
  (BTW, in this chapter, make it clear that \Teshrial does not really like \Urizeth, nor she him.
  She sees him as a shallow glory hound.
  He sees her as a weird nerd.)
  
  Afterwards, they are confident they have kept \Ishnaruchaefir from learning what he should not learn and fucking up their plans. 
  When \Teshrial died, everything had been cleaned up.
  They are sure he did not learn anything.
  But they were not keeping an eye on \Criseis.
  They do not understand quite how keen her senses are.
  Under their noses she has snooped around and gained a good overview of what the Cabal are doing. 

At some point, \Menessiaraid expresses doubt about whether \Teshrial's Malcur venture is really so great and important a thing as they like to believe. 
\Azraid:
  \Azraid thinks to himself. 
  He is not sure whether \Teshrial's Malcur venture is really so great and important a thing as \Teshrial-tachi like to believe. 


Thule -> Thulaan -> Thulaam


Gormur? Gomkor? Gomkur? Gnophil? Gnomphil?

The \gnomphilim were a race of simian \humanoids related to \nephilim.
They were of about the same size as \nephilim, but farther from \human form.
They had long fur and long, baboon-like faces (but no tail).
They had tribes and Stone Age technology.
They lived mostly in \Thulaan and similar arctic regions, where the weather was too cold for the less hardy \humans and \scathae.
They worshipped \xss godlings. 
Compare them to the Gnophkeh from Lovecraft. 



After the FBW:
Many \ophidians began to worship brutal \xs godlings that cared nothing for technology nor progress. 
Under the rule of these harsh gods, the \ophidians descended into barbarism.
They waged wars against one another and destroyed what little remained of their civilization and the planet's natural resources.
Compare to the Serpent People from the Cthulhu Mythos, who turned to the worship of Tsathoggua and were punished by their ancestral patron god Yig, causing them to degenerate. 

In the next millennia, \Miith was populated by a lot of different barbaric, \xs-worshipping peoples. 
They would wage war a lot and destroy each other.
Technology remained low.
But then one day, some \quiljaaran created the \aryothim: A race of super-powered \nephilim blessed with \quiljaar-level intelligence.
Then technology began to rise. 



Perhaps \Tiamat did not die in the \firstbanewar.
She awoke with \Nexagglachel. 
She had been driven mad and evil by a million years of imprisonment.
She conquered the world and reigned as a cruel tyrant.
Eventually \Nexagglachel decided she was too evil and rose up against her.
He converted many \dragons to his side, including his two brothers. 
They slew her. 



Once, \Ishnaruchaefir asked a great cosmic god:
\ta{Why do you live? What is your purpose? What is your goal, your motivation?}

The god answered: \ta{We \emph{know}.}

\Ishnaruchaefir pondered that ever since. 
He was sure there was some great insight hidden in that answer. 
Perhaps because the cosmic gods \emph{knew}, they were content and happy and needed never strive for anything ever again. 
Perhaps the gods \emph{knew} the future and thus had no need of motivations, since they knew that everything they would ever do was already determined. 
Or maybe the truth was something else entirely.
\Ishnaruchaefir pondered that question till the end of his days. 


Some \draconian philosophers speculated that when the different individual \xss seemed to have different powers and specialization (and came to be seen as gods of some \quo{portfolio}), it was perhaps not a result of the gods' actual powers, but rather their interests.
Perhaps \NerranKoss was just a philosophical god with an interest in history and such matters. 
So when people asked him such questions, he was more likely to yield a useful answer than most other \xss.
And thus he became seen as a god of occult knowledge. 
It was unknown if this theory was true, but it was accepted by several. 



2009-07-16


The Vaimon Empire was not built through conquest alone, but even more so through diplomacy and religious propaganda.
History and legend later emphasized both Cordos' heroism in war and conquest and Silqua's gentle charisma. 
The truth was much more complex than that. 


Rian is scarred and horrified to see the evil sorcerer slay the shining god. 
\Criseis Shrouds him and makes him forget the details, but some measure of religius/existential dread remains with him. 
Remember that in all his later chapters.
Maybe even ask on a forum how to express this. 


Make clear in the first Carzain chapters that Carzain and Vizicar do not literally talk to each other. 
In reality, they have much more direct access to each other's thoughts and memories. 
Their thoughts and memories interact in a more fluid manner. 
It is just presented in the text as an inner dialogue for the reader's sake.


The unification of the Cabal took decades. 


Some myths said that the \nephilim were half-\humans, created when \humans interbred with primitive hairy monsters of the \wylde\dash{}possibly the \gnomphilim. 
This was false. 
In fact, it was the \human race that came into being when \nephilim mated with monsters.


A \dragon's length was approximately one quarter body, one quarter neck and one half tail.
Wingspan was about equal to total length.
\Ishnaruchaefir was 25 metres long in all. He weighed about as much as an Allosaurus. 
\Secherdamon was longer than \Ishnaruchaefir, but slimmer. 


At the time of the \thirdbanewar there were 1000-3000 \quiljaaran worldwide. 
100-200 of them worked for \Secherdamon.
10-15 of these were active on \Azmith.
5-6 of them were among the Rissitics. 

Several \quiljaaran dwell in or near \Yormis in disguise. 
Some of them are part of the Dark Crescent.
Others are independent.
They worship the \xs that dwells there and practice their dark science and philosophy. 

Moro \Cornel once saw some \quiljaaran (\quo{Serpent Men}). 
She knew a bit about their terrible secret. 


The Cabalist Malcur venture, like \Secherdamon's plan, also coincides with \Ishnaruchaefir's Nadir. 
When \Urizeth finds out when the next Nadir falls:
  She remarks that it is no coincidence that the Nadir falls now.
  It falls at a time when the Shroud is in \quo{ebb}. 
  Their own plan is also scheduled around the ebb and flow of the Shroud.
  But, as far as she can calculate, the deep point of \Ishnaruchaefir's Nadir comes slightly before the climax of their plan. 
  Very likely \Ishnaruchaefir plans to weather his Nadir and then quickly come back in time to fuck up their plan.
  But if they are lucky, \Teshrial can arrange to fight \Ishnaruchaefir at the bottom of the Nadir, before the climax.
  Thus saving their asses.
At the time when \Teshrial fights \Ishnaruchaefir (for real):
  The Cabal plan is nearly complete. 
  The Cabalist Malcur venture, like \Secherdamon's plan, also coincides with \Ishnaruchaefir's Nadir. 
  The \noggyaleth have grown numerous and large and powerful.
Every time \Teshrial sees the \noggyaleth (or even thinks of them), he is filled with fear and loathing. 
He is not ashamed to admit he fears them. 
Admit it to himself, that is.
He is still too ashamed to admit it in front of others.
In the battle with \Ishnaruchaefir:
  The \noggyaleth grab on to \Ishnaruchaefir with their sucking mouths and grasping limbs/pseudopods.
  They drag him down and engulf and swallow him.
  Then they try to drown and crush and devour and digest him.

Half-\nagae walked more hunched-over than normal \scathae.
They had short legs and long tails. 
Their heads were strangely narrow and flat (in the vertical direction). 
Their bodies were flexible, and when they walked they seemed to writhe and wiggle in a repulsively fluid manner.
Their tails were snakelike and prehensile, which was horrible to look at for a normal \scatha.



2000-07-17 

Make it clear that the Shroud was a patch. 
It was a hasty emergency solution to patch up the cracks in the much more well-designed \CrystalSphere. 
Everyone knowledgeable, including \Ishnaruchaefir, knew that the Shroud was only a temporary measure and could never last. 
Sooner or later it would collapse.
Everyone had better be ready when that happened. 


\Banes had no bones.
Their bodies were flexible and could squeeze through very small openings (albeit they would have to leave any solid items behind). 


Vaimon Emperor -> Vaimon Caliph
Vaimon Empire  -> Vaimon Caliphate


When Carzain reports to Curwen:
  Curwen offers Carzain a smoke.
  Carzain declines.
Afterwards:
  Carzain: Vizicar, remind me again why we did not accept that smoke. 
  Vizicar: In my days, smoking was a plebeian thing to do. We royals never touched it. And I am not about to start now. 


In a scene late in the book, after \Teshrial has acquainted himself with the \noggyaleth, \neoresphain and \WanderersInDarknessEmph:
  He flies above \Nyx.
  He sees the \bane-built spires. 
  Now that he gazes into the deep, he notices how twisted they are.
  They look really scary and wicked.
  He had always taken them for granted, but now that he has gazed deep on the dark mysteries of his people (and the even darker mysteries of the \banes), they scare him.
  There is something evil about the way they twist and bulge.
  They twist into alien dimensions (more alien that what he likes to consider). 
  Like they are appendages of some vast monster that tries to crawl and claw its way up from the deep where it belongs. 
  See also the section on dark ancient cities. 


\Draconian citadels were enormous. 
They sprouted huge, bulbous, misshapen spires.
They catered to the \xss and to the principles of Chaos. 
They were built in accordance with Chaotic occult geometry in order to more fully utilize the power of the \xss. 


Dark ancient cities: 
  \Draconian and \bane buildings alike violated the laws of three-dimensional geometry.
  They extended into the dark, hidden dimensions of the Beyond. 
  The shapes of their walls, towers, statues and ornaments stretched out into the Beyond in a way that was not only physically painful for the eye to follow, but also horrible because it drew attention to hidden things that no mortal wanted to consider. 

Fix BibTeX. Tell BibTex that the stories appear in this volume.
Chaosium: The Tsathoggua Cycle, 2005, edited by Robert M. Price. 
John Glasby - The Old One
p.133--141

  ... we caught a fragmentary glimpse of something which rose from those benighted depths, clawing up from the unseen floor. [...]
  
  To me, they held ineffable suggestions of a blasphemous structure and architecture utterly unlike anything I had ever seen. 
  [...] the searchlight beam only touched their topmost regions.
  But even this was enough to show the sheer alienness of their general outlines.
  Had they been mere conical towers, it would not have offended our sense of perspective to such a degree. 
  But there were bulbous appendages and truncated cones which intermeshed in angles bearing no relation to Euclidean geometry and I felt my eyes twist horribly as I tried vainly to take in everything I saw. 
  
  [...] 
  I could not help feeling there was something evil about those nightmarishly misshapen spires and pinnacles with their bizarre curves and planes; yet it was not an evil associated with Earth but rather with the endless gulfs of space and time, with dimensions other than those we know. 
  
  The majority were smashed and broken with harsh, gaping orifices showing blackly against the sickly grey.
  What beings had once moved within these structres it was impossible to visualize. 
  Certainly no hand of man had erected them and carved their cruel, hideous contours.
  [...]
  The obscure quality of menace in their weird symbolism made me shudder and long for the sanity and safety of the ship.
  
  [...]
  
  [I watched] for he first indication of the vast grey-stone city.
  And then I saw them for the second time, rising out of the slime of the ocean floor, clawing upward for hundreds of feet; row upon seemingly endless row of fantastically symmetrical columns, the nearer ones blindingly clear in the harsh actinic light, with countless others stretching away into the black immensity. 
  
  [...]
  
  The effect of that monstrous labyrinth which stretched away from us into inconceivable distrances was indescribable for it was apparent at once that whatever stood on this undersea plateau had never been fashioned by nature, even in her wildest and most capricious moments. 
  And it was equally obvious that whatever hands had erected these edifices had been far from \human. 
  
  [...]
  
  I dreamed of the long-dead city under the sea.
  But before my dreaming gaze it now stood unbroken and untarnished by time on dry land and there was no sign of the ocean.
  On an incredibly ancient plateau, wreathed in clouds of steam and noxious vapours, the Cyclopean buildings streched away in all directions as far as the eye could see and high into the lowering clouds where the topmost spires were lost to sight.
  There was something terribly un\human about the geometry of its massive grey-stone walls, and the mind-wrenching alienness of its angles and intermeshing structures went against all reaon, all known laws of mathematics, logic and architecture.
  I knew, by some weird instinct, I was seeing it as it had been perhaps several millions of years ago when it had been newly built by that race from the stars. 
  
  [...] now I saw the inhabitants, those hideous and, if the \emph{Book of K'yog} was to be believed\dash{}artificially\dash{}created abominations that had built it!
  I saw them as vague shapes ion the vast avenues and squares, saw them clinging limpet-like to the sides of the buildings or oozing jelly-like from the grotesque apertures and doorways.
  What insane blasphemy had bred those \emph{things} I could not conceive, but the mere sight of them woke me, yelling incoherently, from my dream. 

Link to the above from \Nithdornazsh and the Mirage Asylum. 
And link to the aforementioned sections from every section where they appear in the story. 

Ancient immortal cities reached out into the Beyond. 
They were built in a time before the Shroud, when the barriers between the Realms were much more permeable. 
Their streets and corridors and towers were built so they criscrossed the Realms. 
This was why \Nithdornazsh was so useful as a gateway between \Machai and \Azmith. 

After the resurrection of \Nithdornazsh, the Sentinels have a lot of work ahead of them.
They have to restore the city and build new eidola to shape the Shroud and the gateways and facilitate the travel between the Realms that they want. 


When the \dragons were created, the \ophidians had only begun to build an interstellar civilization. 
So the \dragons had only limited knowledge of the \ophidian super-technology.
The \dragons and \ophidians would both continue to develop their knowledge in the coming millennia, but in different directions. 

\Firstbanewar:
  When the \banes invaded, the \ophidians had an interstellar civilization.
  They commanded dozens of planets. 
  The \banes likewise had an interstellar empire. 
  But \Miith was the important planet that they had been searching for all along. 
  It contained a Heart full of \voyager-tampered energy, and the \noggyal mother-mass, which was the \quo{other half} that the \banes wanted. 
  
  The war lasted centuries. 
  At last the \ophidians were overwhelmed by the \bane hordes, so they retreated to their homeworld and barricaded it with the \CrystalSphere. 
  The \banes conquered all their other planets. 
  
  Fortunately, the \banes had only primitive space travel, and they were un-creative.
  So their interstellar empire could spread only slowly, and not far.
  If the \bane swarm spread too far apart, the farthest parts withered and died.
  They needed the \noggyal mother-mass in order to truly expand their dominion. 
  
  The \banes had a hivemind. 
  The overmind was the \baneking \Voidbringer.
  There was only one \baneking. 

\Draconian Ascendancy:
  When the \dragons awoke, they had spent millennia pondering mysteries of science and magic. 
  They had gained much insight into the occult, so their magic and mystic skills were immense. 
  But they knew little of physical technology, because they had had no opportunity to experiment with building physical tools. 
  So when they awoke, they could not just establish a new technological civilization. 
  Furthermore, their skill was based on occult revelations and deep Gnosis which they could not easily teach to anyone else. 
  So they remained monoliths of arcane power instead of building a new civilization. 


The Mirage Asylum was like a half-open castle ruin or space hulk drifting afloat in the vast, empty void of space. 


\Nithdornazsh was the old tomb of \Nexagglachel. 
Out of respect for him it had been deserted after his death. 
It became a necropolis in his memory, instead of being taken over and used for some other purpose. 
Thus the \resphain forgot about it. 
It was deep in \draconic territory and seemed to have little strategic significance, so the \resphain felt they had better things to do that attempt to conquer or raid it.
Until, finally, \Secherdamon felt it was time to revive it. 


TAR prologue:
  \Nzessuacrith sees the wrecks of mighty cannons and vast war machines of metal and flesh. 
  She herself bears grim wounds. 
  She lost her right forearm in battle with a titanic \umbra; a soul-devouring monster from \Erebos, fully as large as she was.
  Call her \quo{Cryocas}.
  Cater to the dumb reader who cannot understand long names. 
  And get rid of those dead \dragons. At least, do not name them.  


WSB: 
    \Criseis does not know what \Ishnaruchaefir has in mind. 
    She knows what he has told her to do (i.e., reconnoitre for any supernatural presences infesting Malcur), but no more.
    She imagines he intends to draw out whatever is in the ground and destroy it.
    
    When \Teshrial arrives, he looks at the glaive, and \Criseis can see him thinking. 
    He is thinking: 
    \tho{He pulled that thing out shortly before I arrived. 
      He must have been prepared to do something nasty with it. 
      Looks like I came just in time.}
      
    \Criseis agrees. He probably did come just in time, although in time for what, she cannot say, only try to guess. 
    
    After the battle, the \noggyal presence has retracted. 
    \Criseis can still feel it, but only faintly, and only because she knows it is there.
    She asks her master if he intends to pursue.
    
    \Ishnaruchaefir:
    \ta{Nay. Let them hide.}


When the first \scatha meets a \human: 
Show the \scatha's revulsion at the sight of this ghastly, grotesque, naked... thing. 
Compare the \human to a naked mole rat. 
\Scathae tended to see mammals as disease-ridden vermin in the first place, and this naked mammal that stood upright and tried to mimic (\quo{ape}) the \scathaese form was horrid to behold. 
Besides, the \scatha could feel the Cosmic Horror of the \pps{\banes} influence on the \human. 
It would take centuries for the two races to become fully accustomed to each other and not feel this Cosmic Horror revulsion anymore. 


When \Criseis feels (in WSB) or sees (in the big battle) a \noggyal:
The Old One, p.147:
\citeauthorbook[p.147]{JohnGlasby:TheOldOne}{John Glasby}{The Old One}{
  Even in retrospect it is not possible to convey in words the nature of that monstrosity which squeezed its vast bulk through the gaping abyss.
  It held a hint of noxious plasticity, of writhing tentacles which changed their number and shape.
  But more than anything, I had the impression of gigantic size, that huge as that part of it looked where it almost completely blocked the opening, there was an infinitely greater bulk mercifully hidden from us.
  
  [...]
  
  But I know there was nothing imagniary of halucinatory about the black, coiling tentacle that seized Dorman around the waist and bore him, kicking and screaming frantically, into the gaping, beaked maw which appeared as if from nowhere beneath that single glaring red eye!
}

Compress the Rian storyline. 
It is unbelievable that they would keep Neina captive for so long. 


Read some Robert E. Howard before I write the scenes with Carzain and Tantor in the \wylde. 


The \ophidian civilization used up a lot of the planet's fossil fuels. 
The \quiljaaran and \aryoth civilizations used up most of the rest.
By the \Human Age, there was almost no fossil fuels left.
This made it hard to develop any industry. 
So the mortals didn't. 

\Nasshikerr tells Moro that there are some evil people that are preparing some big, evil ritual of magic, and that it is connected to that which is destroying the city.
Moro naturally assumes that this means the evil ones are the ones destroying the city.
But \Nasshikerr does not tell them what the factions are or what their different plans are.
He tells them a bunch of details pertaining to the two plans, but gives them no overview of the big picture. 

Near the end of TAR:
  Moro and Rian have gotten some tips from \Nasshikerr (chiefly) and from the thug Moro has interrogated (secondarily).
  They know there are some evil people that are preparing some big, evil ritual of magic, and that it is connected to that which is destroying the city.
  The captive thug tells them where the ritual will take place. 
  \Nasshikerr has told them the time. 
  They go there to attack the ritual and stop them.
  At the very same time, Needle also attacks.
  Moro and Rian see Needle and the wicked \banes she commands. 
  They are both horrified\dash{}especially Moro, because she knows what \banes are. 
  They do not know that Needle is there to do the same thing as they, so they assume she is their enemy (after all, she commands the \banes).
  So they kill her. 
  
  The \banes wreak havoc. 
  It was an unforeseen development.
  \Psyrex had not expected the Cabalists to loose \banes. 
  And these \lesser\banes are small enough to be difficult to detect for a mage, but still deadly enough to be a great menace. 
  \Psyrex has to go himself and fight the \banes. 
  
  Moro leads one \bane away and hurts it.
  She cannot kill it, but she can keep it at bay and occupy its attention long enough for \Psyrex to deal with the other two \banes and complete their ritual. 
  Then, when \Nithdornazsh rises, Moro gets separated from the \bane and manages to finally escape it.
  She is convinced it will go elsewhere.
  After all, it has no particular reason to want to kill her. 
  
  In the chaos that ensues after the \banes wreak havoc in the Sentinel camp, Rian slips in, stealthily knifes a few thugs (killing his first man in the process) and manages to find and rescue Neina.
  Perhaps he finds her naked in bleeding in a corner, having been recently raped. 
  Perhaps he catches a couple of rapists in the act and kills them (but not before they could take their pleasure from her). 
  Perhaps this rapist-killing is his first kill. 


The cult in \Redce:
  Have erotic scenes, like in Gary Myers - The Horror Show (in The Tsathoggua Cycle), where a hot girl is stripped naked, tied up, beaten halfway to death and then sacrificed. 
  In the end, she changes her mind and screams and begs for mercy, but in vain, and she is devoured. 
  All the while, the cult chant their songs to their dark gods. 



2009-07-18

There existed a number of \quo{\demiscatha} and \quo{\demihuman} races. 
The Imetric half-breeds were considered \demiscathae.
Some also thought of the \nagae as \demiscathae, but those who had actually seen and dealt with the \nagae knew better.
The \nagae were a far more ancient race than the \scathae. 
As old as the \dragons and older. 

\Demihumans were far more widespread before the \VaimonCaliphate. 
Back then they were not \quo{\demihumans}, but simply different variants of \humans.
There was no one race that was considered \quo{real} \humans, of which the others were deviations. 

The racist Vaimons exterminated many of the other \human races and subjugated or drove away the rest. 
Later, the race which the Vaimons represented came to be thought of as \quo{real} \humans and all the others as \demihumans. 

The \scathae were more tolerant of their different kin, but after millennia of interbreeding there had emerged a race of \quo{standard} \scathae, and other variants came to be called \quo{\demiscathae}. 
Rethink the idea of the different \scatha ethnicities!

On other Realms than \Azmith, other \human variants dominated.
Evith should be some exotic \human variant. 
Some had tails or wings or horns. 
Some had exotic skin colours. 
Evith is striped like a zebra. 

Have plenty of \demihuman slaves in Pelidor and stuff. 
Perhaps nobles (such as the Rungeran court and \ishrah) keep hot, exotic \demihuman girls as sex slaves. 
Perhaps \Takestsha takes the form of a strange \demihuman to add to her mystery and allure.

The \nagae were also called \quo{\ichthyans}. 


Carzain has dark skin.
He is deviant enough that some might call him \demihuman.
Curwen is a purebred true \human, and also a racist. 
He respects Carzain in spite of his strange breeding. 


Rian is a pure \human. 
He mistrusts \demihumans. 
Have him meet some in the beginning of WSB.
He is about to steal from a \demihuman merchant. 
He thinks to himself:
\ta{He is probably rich and dishonest. A swindler. Those sneaky \demihumans always are.}


A description of a \dragon. 
Can be used where Rian sees \Ishnaruchaefir for the first time, and when \Ishnaruchaefir and \Nzessuacrith assume their true forms. 
Add this to the \quo{\dragon} section and link to it from the relevant story sections.

The Tsathoggua Cycle p.214
\citeauthorbook[p.214]{HenryJVesterIII:TheResurrectionofKzadoolRa}{Henry J. Vester III}{
  The Resurrection of Kzadool-Ra
}{
  The being prokected an aura of incalculable age and wisdom, and of powers gained on worlds long lost in space and time.
}



In the \Scatha Age, Vaimons were admired and feared. 
They were larger-than-life figures, touched by \Iquin (or worse, \itzach). 



\Dragons did not just give birth naturally like other creatures (even \resphain) did. 
For a pair of \dragons, to create an offspring was a long, very arduous and very arcane process. 
They must invoke the \xss and draw much power and life-energy from \KhothSell and from the Heart of \Miith, and deliberately and permanently invest much of their personal power in the child. 
Then the female would get pregnant and, later, lay an egg. 

In the age of the Shroud, this process was fraught with danger. 
The Heart was difficult to reach, so it took a lot of work and wasted energy to reach it and coax any new life out of it. 
It took more energy than usual, and the \draconic parents-to-be risked permanent bodily harm.
On top of that, it was uncertain. 
The egg might easily die without hatching, or the hatchling might be sickly and die young.
Most \dragons decided that the permanent investiture of personal power was not worth the risk, so they simply stopped having children.
After all, they were immortal, so they did not \emph{need} to replenish their race. 


\Draconian immortality was powerful. 
A \dragon killed by normal means would retain quite some consciousness and even a minimum of sorcerous power. 
And even after its body was permanently destroyed and its magical power broken or eaten, the \dragon's consciousness would still live on. 
It would have little power (at least in \Miith and in familiar Realms), but it would retain its memories and wisdom and would be free to journey to other worlds and build for itself a new life. 

After his permanent death, \ps{\Sethicus} soul lived on. 
He was the wisest of all \dragons and knew much Gnosis that no other \dragon had attained, so even when he had no more power in \Miith, he was still worshipped as a spiritual guide. 

After \Nexagglachel's death, his soul lived on in fragments inside the \satharioth.
The \resphain never understood this phenomenon. 
They never understood that \draconian immortality was something greater and more profound than their own immortality.
They did not understand the idea that a soul could be destroyed and devoured and yet live on and retain consciousness. 
The \resphain had no Gnosis of the Aenigmata that \KhothSell had taught to her followers.

See also the section on \hr{Draconic immortality}{\draconic immortality}. 


All immortals were voracious carnivores, if not soul-devourers. 
The \aryothim did not eat souls per se, but they did require large amounts of freshly killed animal flesh. 
Some religious rituals performed during the killing (akin to halal slaughter) helped and made the flesh more nutritious.
The \aryothim built a lot of religion on this. 


The \ophidians were generally long-lived, slow-living and slothful, so they developed new ideas only slowly. 
Hence their culture lasted for millions of years.

\Sethicus's time was characterized by very fast development. 
\Sethicus was a very impatient and ambitious \ophidian.
He craved more than his race's slow, complacent lifestyle.
He craved action, excitement and renewal.
He was a great scientist who made many inventions.
He inspired many followers to live fast like him. 
In those decades and centuries, technology developed just as quickly as it might for a short-lived race.

\Sethicus himself was a brilliant scientist and made many exciting new discoveries in various fields. 
Alienist sorcery was his favourite subject and later came to dominate his life and career. 

\Sethicus was a primus motor when it came to pushing the boundaries of space travel and establishing an interstellar \ophidian empire. 
Before him, they had been a spacefaring race for millennia, but rarely ventured beyond their solar system, and they had done no colonization beyond the occasional research station in the solar system. 
\Sethicus made them found settlements on many planets and moons in the solar system and even beyond it, and they began terraforming processes. 

After \Sethicus's fall, \ophidian progress once again slowed to a crawl. 
The colonies were either abandoned or became sleepy, unambitious things. 

Not all \dragons agreed with \Sethicus's aggressive, warlike manner and his plans of conquest. 
\Nexagglachel was one of his critics.
\Nexagglachel was always calm and benevolent and thought of the future of the people (\ophidians as well as lesser races), whereas \Sethicus was mostly focused on his own greatness. 
When \Sethicus and \Tiamat were defeated and entombed, many of the remaining \dragons, convinced by \Nexagglachel, willingly stepped down and let themselves be entombed.
There they could lie and think and explore. 
(Many of the \dragons were scientists and philosophers who had been attracted by \Sethicus's free thinking.)

They felt sure that their time would come again, some day in the future. 


After the \firstbanewar, the \ophidians lost control of many of their monsters. 
They ran amok and caused much havoc and destruction.
This was one of the many reasons why the \ophidian civilization declined and would not rise again for tens of thousands of years. 


The \aryothim developed technology that ran on fossil fuels. 
After they depleted all fossil fuels on the planet, they began relying on dark magic for their energy needs. 
They broke their old taboo against alienist sorcery, which had lasted for many hundreds if not thousands of years and formed the core of all their major religions. 

The \aryothim developed many religions. 
All the major ones had the taboo against sorcery as a central tenet. 
(Not that it was not broken from time to time.)


The \aryothim and their \quiljaaran enemies developed nuclear weapons. 
They waged war. 
There was at least one nuclear holocaust in those days. 
After each major war, the \aryothim were quicker to get back on their feet than the slow-living \quiljaaran were, so they slowly gained the upper hand. 

Perhaps it was one such war of mass destruction that roused \Nexagglachel-tachi from their sleep. 
\Nexagglachel felt the catastrophic loss of life and decided he had to do something to fix it. 
So he spent the next century or more slowly picking the locks of his tomb, until finally he was able to rise from the dead and wake his brothers. 


\Nexagglachel saw the horrible destruction that high-tech weapons had caused.
He decided that \humanoids were not intelligent enough to handle technology or powerful magic responsibly, so he repressed technology and kept it in the hands of the wise elite. 
After his death, the remaining \dragons respected this aspect of his philosophy.
It was part of the inspiration for the Unspoken Covenant. 


It was well-known in mythology and legend that the planets and moons were worlds with their own inhabitants. 
There dwelt special \demihuman and \demiscatha races on the two Lunar Realms.

Remember that the moons had low gravity. 


Have quotes from myth and history that wildly contradict each other.
Have each side openly demonize the other, so as to make it clear to the viewer that myth and history should not be taken as canon. 



2009-07-21

\Nephilim had tails like monkeys.

Some \demihumans had tails like \nephilim.
Some had feathers like \resphain.
A few rare breeds even had wings like \resphain. 
They could perform powered leaps with the wings, but they could not fly. 
And the wings were fragile and did not regenerate. 
Still, the winged \humans were considered by the \resphain to be the highest of all \humans because they resembled resphain. 
Evith was a winged \human. 
And striped like a zebra. 

\Thanatzil book:
  \Thanatzil began having sex when he was 15.
  He soon impregnated his first \nephil women.
  Soon after that the first \humans were born.

  The first \humans were diverse. 
  There were several exotic strains that quickly died out. 
  Also, remember that \humans quickly mixed with \nephilim. 
  So the first \humans might very well have looked much less \quo{\human} than the later ones. 
  
  \Morza saved several \human children. 
  (The oldest \humans were as much as six years old.)

\Thanatzil had to grow up very quickly. 
He had to be responsible and serious and not childlike.
In Japanese he would be using the pronoun \quo{watashi} rather than \quo{boku} even as a young boy. 

Races of \demihumans:
  \Sheomir (singular \sheomir) had a fox-like furry tail and a stripe of fur running down the spine. 
  They were very rare in \Velcad, where they were often kept as exotic slaves (possibly sex slaves). 
  They were more common in the Imetrium.
  They are inspired by the Sheovins from Nyki Blatchley - "Kaydana and the Staff of Ishlun" (2009). 

  \Tulans (singular \tulan) had skin with a reddish complexion, slightly sharp teeth, a pointed nose and face, and (for men especially) facial hair that looked a bit like whiskers. 
  Racists said they looked like rats. 
  They were the most common minority in Pelidor, where they were a subjugated lower class. 
  
  Rian was a \tulan (as was Neina and her family).
  
  Standard \humans were called \highhumans, \highmen, \highwomen.
  
  Needle might have been poor, but at least she was a true \human. 
  A \highwoman. 
  She looked down on those filthy \demihumans.

\Teshrial and \Menessiaraid are gay lovers. 
They kiss and have sex. 

When Rian saves Neina: 
  Have a long, tension-filled scene where we hop back and forth between Neina and Rian. 
  There is a thug, Blon, who wants to rape Neina.
  He reaches into her cell and gropes her. 
  She shies away and hates him and suffers. 
  But he dares not rape her because the sorcerers forbid it. 
  
  Later, when all hell is breaking loose, Blon decides he might as well go for it. 
  So while the rest are running around confused, he goes to Neina. 
  He gropes her and undresses her. 
  She screams and fights.
  She manages to kick him painfully in the nuts.
  He retreats a bit to get over the pain. 
  This gives her a short reprieve.
  Then he comes back in and beats her savagely. 
  He rips her clothes off.
  She screams and cries. 
  She is desperate. 
  She does not want to believe that this is happening to her. 
  At last, he penetrates her and completes his violation of her. 
  She loses her virginity to an evil, ugly, foul-smelling crook whom she hates. 
  
  All the while, we hop back and forth between Rian and Neina.
  He runs around frantically looking for her. 
  At last he reaches her.
  She is being raped by Blon. 
  Rian attacks.
  Blon pulls out.
  Rian kills him.
  This is the first person Rian has ever killed. 
  As Blon dies, sperm pours out of his dick onto the ground.
  Neina has been raped, but they can take some small consolation in the fact that Blon did not come inside her. 
  
  Rian has saved her, but at a terrible price. 
  Everything else is going to hell around them. 
  Moro wished he was with her.
  Together they might have been able to save more people.
  But no.
  Rian had only eyes for Neina.
  He insisted to Always Save the Girl.
  It was a horribly bad choice from a utilitarian point-of-view, and he did not even succeed. 
  Karma by proxy! 

At the time of the Pelidor/Runger war, Carzain has learned a lot of Cosmic Horror revelations.
Most of them are things from Vizicar's or Tydesmos's time which he has come to remember. 

Archibald Curwen is about 60 years old at the time of the Pelidor/Runger war. 
Vaimons did not live longer than regular people. 

Maybe change "Light" to "Will" (in the sense of "\iquin"). 

Iquinian dogma:
  The Iquinian church preached that \quo{we are all one}. 
  All humanoids were part of \iquin, even though bound in fetters of \itzach.
  They just had to realize it. 
  Commoners could contact the rest of \iquin (and, thereby, other humanoids) through prayer and mass. 
  The Vaimons were wiser than commoners and had access to more knowledge. 
  They learned how to break down the walls that separated them from the rest of the \iquin and thus draw upon the awesome power of \iquin to cast magic. 
  
  Iquinian myth held that \iquin had always existed. 
  The nature of \itzach was more ambiguous. 
  Some interpretations said \itzach had always existed as the twin of \iquin.
  Others said \itzach was an emanation of \iquin and existed at its mercy. 
  
  The physical world was held to be a consequence of \itzach. 
  \Iquin represented unity, but \itzach represented diversity. 
  \Iquin was the essence of all things, but it was \itzach that gave those things shape and thus made them into \quo{things}.
  In the beginning, the world was in balance.
  But then \itzach invaded \iquin, under the leadership of the dark god \Isphet. 
  The hordes of \itzach brought chaos. 
  \Itzach took the essence that emanated from \iquin and forced it into unnatural shapes, thus creating the false illusion that the world was made of separate \quo{things} and \quo{individuals}.
  
  The physical world was Defiled because it was created by the mingling of \iquin and \itzach. 
  This Defiled world was called \Gehinnom. 
  All living creatures were part of \Gehinnom and were thus also themselves Defiled; impure, unworthy, lowly sinners. 

  In reality (the church said), all things and beings were one, made from the same indivisible One Light. 
  All shapes and all individuality was a result of the fetters that \itzach cast upon the world. 
  The task of all living beings was to live in virtue and not sin.
  Virtue would dissolve the fetters, but sin would forge new fetters. 
  
  Iquinians believed that humanoids had \quo{fetters} of darkness that bound them to \itzach. 
  These fetters were made of sin. 
  All humanoids were born with fetters, for they were imperfect and flawed beings in a flawed material world. 
  All every time they sinned, they forged more fetters. 
  An Iquinian strove to break the fetters of darkness through prayer and by following the virtues of the \sephiroth. 
  Many people chose one \sephirah or a few, and then strove to emulate them and submit completely to the virtues they stood for. 
  When a person broke all his fetters, he would be free of \itzach and able to transcend into the One Light. 
  But this was only possible for the great Vaimon saints like Silqua and Cordos. 
  It was held to be impossible for regular people to break all their fetters. 
  They were flawed creatures, after all.
  But when a person died with any fetters left, he could still appeal to the mercy of the \sephiroth, and they might help him and bless him and free him of his last fetters so he could at last be one with the One Light. 
  This would only happen to those who were truly faithful, though. 
  The souls deemed unworthy would be cast out into the Outer Darkness to suffer forever in the chaos of \itzach.
  
  The Divine Realm, the Kingdom of the One Light, was the abstract \quo{place} where the \sephiroth dwelt.
  And it was the place where believers hoped to go when they died.
  They wanted to \quo{go into the One Light} and become one with \iquin. 
  
  The Divine Realm might be the same as the \empyrean. 
  Or perhaps the \empyrean is just the vestibule of the Divine Realm.
  But no, I think they are the same. 
  But a Vaimon cannot move into the innermost circles of the \empyrean as long as he has a physical body.
  The body fetters him to the lower, material world.
  He cannot become one with the \sephiroth, but he can touch them. 
  
  But if so, then why do the \qliphoth also dwell in the \empyrean? 

  Iquinian prayers were seen as a simple form of magic. 
  Praying really changed the world. 
  It brought the Divine Realm and the spiritual union closer. 
  
  Iquinian eschatology held that there would come a Doom's Day, an Apocalypse.
  On this day, the \sephiroth would descend to \Miith and all fetters would be broken. 
  All those souls that were found to be virtuous would be taken in and become one with the One Light.
  All those sould that were found to be sinners would be cast down and bound in \itzach, where they would dwell in pain and chaos and suffering forever more. 

Iquinian numerology:
  The Vaimon alphabet had sixteen consonants.
  Every \sephirah's name began with a different consonant. 
  Thus the letter corresponded to the \sephirah and its virtue. 
  Each consonant also had a number, the same as the number of the month of the corresponding \sephirah.
  The consonants are: 
    Bal
    C/Ker/Q (pronounced the same, romanized differently for aesthetic effect)
    Chaid
    Fir/Ph
    Luth
    Mor
    Nuz
    Raith
    Sum
    Shil
    Ten
    Thul
    Tzad
    Vod
    Zod
    Yith
    
    Other consonant sounds were spelled as \quo{voiced} versions of the base consonants:
      Tzad became J
      Chaid became H
      Ker became G
    Qu was spelled as Ker and the vowel U.
  
  Each vowel also needs to have some meaning...
  
  Every person's name thus had a theological meaning. 
  Rian's name contained the consonants of Razilah and Nomariel.
  

Rian was happy that he had broken away from his life of crime. 
Now that he was a law-abiding citizen, he had come closer to the Kingdom of the One Light.
He was closer to being one with the Light. 
By the grace of the \sephiroth he had broken some of the fetters of darkness that tied him to \Itzach.
He knew he still had plenty of fetters. 
He was still a sinner.
But he was on the right track, and he believed that if he worked hard and prayed and was faithful, then the \sephiroth would bless him and enable him to break the rest of his fetters.
So that at last, when he died, they would show him mercy and he would become one with them.

Rian should have one or two \sephiroth whom he especially strives to emulate. 



2009-07-22

The tranformation from \resphan to \neoresphan was a metamorphosis, like what insects and amphibians do. 
In a sense, the \resphain were neotenic, like axolotls: 
They had the natural potential to metamorph into a more powerful mature form (the \neoresphan form), but they were inhibited and unable to leave their immature, undeveloped \human-like forms. 
The \neoresphan project was intended to bring the \resphain out of their childhood, allow them to break their weak shells.

\Azraid led and oversaw the \neoresphan project, but he never underwent the treatment himself. 
He took no chances. 
His life was too valuable to risk.
He needed to be there to stop the \banes.
\Azraid was very unlike \Sethicus, who was a brave pioneer who always went further than any other. 

Only very few \resphain ever mastered the \Draconic tongue. 
Partially because it was very hard, and partially because it was hard to find anyone willing to teach them. 
No \dragon ever taught a \resphan language. 
But the \resphain were able to find a few \quiljaaran who spoke \Draconic and could be persuaded or coerced into teaching. 
These \quiljaaran did not speak perfect \Draconic, though. 
And no \resphan ever learned True \Draconic. 

\hr{Resphain speak poor Draconic}{Like most \resphain}, \Teshrial \hr{Teshrial speaks poor Draconic}{spoke \Draconic only poorly}. 
So when \Ishnaruchaefir speaks \Draconic, \Teshrial has to make an effort to understand it. 
\Criseis can tell how \Teshrial is listening intently, all the while trying to hide it.
He does not want to show his own shortcomings. 

In \draconian metaphysics, \Sethicus had a status as a mythical primogenitor. 
Compare to Adam Kadmos in Cabbalist theory (whose body contains the ten \sephiroth) and Albion in William Blake's mythology (the primal man of whom the Four Zoas are mere fragments). 
He is featured heavily in \WanderersInDarknessEmph as a supreme founder god, from whom the three \quo{wanderers} derive their natures and powers. 

WSB:
  Consider how to make \Ishnaruchaefir more mysterious, more unknown, less seen. 
  I do not want to de-mystify him. 
  
  Maybe \Ishnaruchaefir should be seen only dimly.
  He appears as a sinister force that tries to break through the Shroud.
  The \resphain show up and try to block him.
  
  Describe \Ishnaruchaefir as a dark \draconian god. 
  
  Only \Criseis is there in the flesh. 
  They threaten her, but she tells them how bad an idea it would be for them to hurt her. 
  
  Maybe there should be more \resphain than just \Teshrial there.
  The rest retreat when \Teshrial dies. 
  
  Compare him to Anomander Rake in his very first scene in Gardens of the Moon. 
  Here, Rake fights the Malazans, but from a high vantage point. 
  He is not seen clearly. 
  He is a distant force borbarding them. 
  And we see how superior and invulnerable he is.

When \Ishnaruchaefir learns of \Urizeth:
  Do not describe \Ishnaruchaefir clearly.
  Cut out after his first line and let his conversation with \Criseis take place offscreen.
  Compare to the descriptions of Shai'itan in Wheel of Time. 

The scenes with \Ishnaruchaefir in \Azmith:
  Get rid of them. 
  Replace \Ishnaruchaefir with \Criseis. 

In the scene where \Urizeth dies:
  \Urizeth does not see \Ishnaruchaefir.
  She only feels him approach as a tremendous \vertex, a behemoth tremour in the Web of Realms. 
  She tries feebly to defend herself, but he blasts her out of the sky and kills her before she even sees him.

Archibald Curwen was the son of James Curwen I. 
Archibald was a younger son. 
It was his older brother Ducan who inherited James's title and estate. 
Later Ducan died and his son, James II (Archibald's nephew), inherited the estate. 
James II was a competent leader, but he was also young. 
His uncle Archibald was often able to talk or coerce James into doing his bidding. 

Before \Lithrim:
  Have a longer story where the world goes mad. 
  More people discover the truth about \iquin, but by then it is too late, and their souls are too deeply bound to \iquin to break free. 
  
  They also discover that there are even older and more malevolent things in the world than \iquin.
  Things perhaps even more inimical to them.
  So that even if \iquin is a parasitic soul prison, it is perhaps better than the alternative. 
  
  Compare to the movie \cite{Movie:IntheMouthofMadness}. 

\Itzach was raw \bane power, whereas \iquin was filtered \bane power. 

The voids between the Realms were fucking dangerous. 
There existed blind, mindless, slavering things of which even the \resphain lived in fear. 
Whenever \resphain had to travel between the Realms, they could not just travel through the Beyond.
They only travelled along well-known pathways and had to have magic to keep them safe (more cosmic versions of the \eidola and \wylde charms that mortals used).

Malcur plan:
  The Sentinels plan to make Malcur ready for the great change and the Resurrection by gradually seizing control of the Shroud over the city and subtly changing people's beliefs. 
  When they have enough Shroud power, they can mind control the people and channel their spells through them. 
  The people of Malcur themselves will facilitate the Resurrection. 
  
  The ordinary thugs do not know about the Resurrection. 
  They only know about some great upcoming event called \quo{the Change}. 
  The Black Plague thug whom Moro interrogates only knows about \quo{\hr{The Change of Malcur}{the Change}}. 
  The sorcerers whom Rian see talk about how \quo{\hr{The Change of Malcur}{the Change}} is coming up. 

In Malcur:
  Have more horror with humanoids that turn into monstrous forms.
  Then they go mad and become wholly converted by whatever evil side has transformed them.
  They laugh and taunt the remaining mortals (such as Rian and Moro) and tell them that there is no escape, that the whole world will be transformed into a nightmarish hell of madness and chaos. 
  
  Have Moro and Rian as a kind of \quo{Only Sane Men}.
  They see Malcur go mad around them.
  They are badly affected, too.
  They are going crazy with fear because of all the horrors they see.
  Moro only manages to keep her sanity because she has seen things like this before. 
  And because she has monitored the process of Malcur's going to hell, and so is less surprised by it than the regular folks who just see it happening overnight. 
  Rian does not hold up well. 
  All this evil tears his world view asunder. 
  The \sephiroth seem to be powerless to prevent it. 
  And this evil is not like he had imagined it.
  He had expected winged devils with pitchforks, but not this. 
  Not people turning into warped monsters before his very eyes. 
  
  Rian tries to interpret it all in an Iquinian way to make it fit his world view. 
  These people must be caught in fetters of \itzach. 
  But his rationalization fails. 
  This is all too alien, too wrong. 
  It is nothing like the priests have described. 
  Sinners are supposed to be cast out into the Outer Darkness. 
  They are not supposed to mutate like this. 
  It is wrong. 
  Rian's world breaks down.
  
  Moro slaps Rian up and tells him to pull himself the fuck together. 
  She forces himself to shape up.
  She is the main reason why he keeps on going and doesn't break down and become a babbling lunatic. 

Rian sees ritual and \banerats:
  The scene where Rian where sees a \sphyle that is turning into wood should be expanded.
  I want more \hr{Humanoid horror}{humanoid-based horror}, remember. 
  The \sphyle comes alive and grabs Rian. 
  He screams.
  She talks to him, or tries to.
  It is hard for her to speak, but Rian understands more than he wants to. 
  
  Maybe it is a \human, not a \scatha. 
  
  Moro is very interested when Rian tells her of the tree-\sphyle. 
  
Rian in the prison cell:
  It is very dark in the cell.
  There is only a tiny window high above, and it is full of dirt, so barely any light can get through.
  He can barely see anything. 
  
  While he is in the cell he experiences more madness. 
  He feels around and discovers that one corner gives way to soft, wet, oozing flesh. 
  He gasps out load and crawls away. 
  He backs over to a wall and lets himself tumble backward into the wall.
  Only to find that the wall has turned to flesh behind him. 
  He panics and flails around.
  It feels like he is sinking into that wall of flesh.
  As if the wall is sucking him in. 
  He flees from it. 
  Stumbles around. 
  Falls. 
  The floor ruptures under his hands and they sink deep into slimy, lukewarm flesh.
  Ooze or blood wells up and soaks his hands and his clothes. 
  He screams. 
  The madness continues.
  He imagines he hears voices from the walls. 
  He imagines he sees fragments of faces on the walls.
  Not full faces, just hideous, all-too-\human mouths that whisper blasphemy to him. 
  He can only make out small parts of what they are saying, but the little he understands fills him with dread. 
  He shouts and claps his hands over his ears to block them out.
  He feels the slimy blood on his hands. 
  Now that his hands touch his head, the slime seems to seep into his hears. 
  He panics even more. 
  Finally he collapses and loses consciousness.
  
  He awakes when Moro comes into his cell and kicks him.
  He is still sticky.
  His clothes are half-wet with some disgusting substance. 
  They quickly flee from the cell. 
  Now that light comes in from Moro's torch, the walls are solid stone again.
  Rian tries to tell himself he was just hallucinating last night, but the goo is still on him. 
  He goes away horrified. 
  Moro is interested when she hears his story.
  She does not know how much of it to believe. 

Rian is rescued:
  Rian has gotten Neina out, but they are both hopeless.
  They are sure it is the end of the world.
  Rian feels he can only pray for salvation before he inevitably dies. 
  But then \Criseis comes and saves them. 
  She takes them out of the city.
  
  Rian asks \Criseis: 
  \ta{But where are we supposed to go form here? 
    If it's the end of the world...}
  
  \Criseis: 
  \ta{It is not the end of the world.}
  She looks thoughtful and worried for a moment.
  \ta{At least, not just yet.}
  
  Rian: \ta{But the city...}
  
  \ta{It is the end of Malcur, but not the world. 
    Listen.
    I have done what I could for you.
    You may not be unscathed, but at least you are alive.
    You have a chance to make a new beginning.
    Good luck.}

Move \Isphet to Iquinian mythology.

According to some Iquinian myths, the world was originally nicely delimited into a Dark half and a Light half. 
But then \Isphet, Lord of the Outer Darkness, attacked and invaded the light. 
He and his forces polluted the Light with Darkness. 
The essence of \iquin and the fetters of \itzach intertwined to create the material world. 

Other titles of \Isphet are Lord of Chaos and Lord of the Outer Darkness. 

In later Iquinian theology, Cordos and Silqua were highly exalted holy characters. 
They were the founders of modern mankind. 
Compare them to Adam and Eve from Judeo-Christian mythology.
Or Adam Kadmon from Cabbalah.
Or Albion from William Blake's mythology.



To make clear to the reader that Iquinian myths are not necessarily true: 
Have ideas and imagery that contradicts Earth mythology. 
For example, let snakes or horned/hoofed humanoids be good instead of evil. 
Actually, maybe certain \resphain should have horns. 


There dwelt flying polyps beneath \Nyx and \Erebos, where they burrowed and slithered.
Even the \banes feared these monsters.
As did the \resphain.
They were some of the monsters which even the \resphain feared. 
Replace the \ghobaleth with them. 



2009-07-25

\Rethyax/\Ortaican theology: 

  The \Ortaican/\rethyax religion was one of the dominant non-Iquinian religions on \Azmith.
  The Rissitic and Imetrian religions had inherited quite a bit of their theology from \Ortaica.

  A regular follower of the religion was just called an \Ortaican. 
  Originally, the term referred to an ethnic/national group, but later it came to be a purely religious term.

  The \rethyax gods were divided into two categories: 
  The \Primordials and the Younger Gods. 
  
  The \Primordials were \xss.
  They were credited with the creation of the world and living creatures.
  They had since absconded and now slept and kept themselves aloof from the world. 
  But they were still sources of great power. 
  The \Primordials were alien and incomprehensible to mortals. 
  
  There were four \Primordials:
    \Costorul (\KhothSell).
    \Kythraxas (\KyaethemChreiAz). 
    \Nelxurra (\NerranKoss). 
    \Rammasul (\NaathKurRamalech). 
  The \Ortaicans did not know of \RuinSatha. 
  The \quo{true} names of the \Primordials (the names the \dragons used) were deep and secret \arcana.
  
  The Younger Gods, the \Taorthae, were descendants of the \Primordials.
  While the \Primordials slept, the \Taorthae had inherited rulership of \Miith.
  The Younger Gods were more humanoid in aspect and easier to worship. 
  It was they who were charged with the daily maintenance of the world and with protecting their worshippers from malevolent powers.
  
  The existence of the \Primordials was an \arcanum known only to the \rethyaxes. 
  The masses only knew of and prayed to the \Taorthae. 
  The \rethyaxes believed that the \Primordials were so dark and so dangerous that no commoner should interact with them at all. 
  No uninitiated mortal should desire to look upon the faces of the \Primordials or hear their voices, lest his very mind be blasted and destroyed by the sheer awesome power of the eldest gods. 
  The \Primordials dwelt in the Elder Darkness far beyond the fields of mortals, for their visages were too great and terrible for the world to behold. 

  \citebandsong{Nile:Ithyphallic}{Nile}{
    As He Creates So He Destroys
  }{
    No living creature can look upon his face\\
    And endure its terrible heat and black radiance\\
    That is like the reverberating unseen rays of molten iron\\
    Which strike and burn the skin of those who would dare\\
    Gaze into the countenance of the idiot god
  }
  
  According to \rethyax mysticism, \scathae were created by the \Primordials.
  Each \Primordial created a part of the body and and a part of the \scathaese psyche. 
  Each was also associated with a part of the world. 
  \Rammasul was associated with the firmament (sky), the cranium and the firm, righteous thinking mind. 
  \Costorul was associated with the earth, the groin and sexuality. 
  \Nelxurra was associated with the stars and the questing, contemplative, philosophical mind. 
  
  The \Primordials came from the stars and shaped the world into its current form. 
  Except \Nelxurra, who never set foot on \Miith but stayed among the stars. 
  
  The \rethyax religion was designed with \scathae in mind.
  It was very anti-\human at first. 
  Later, theologians made concessions and allowed that the \Primordials might also dwell in \humans.
  
  In truth, \scathae were created by the \dragons using bio-magic learned from the \xss.
  The process of creating the \scathe was very much inspired by the way the \dragons had created themselves.
  The \scathae were designed as \dragons in miniature.
  As such, \scathae genuinely did have more individual potential for greatness than \humans did.
  \Humans were worthwhile only in a big mass.
  Each \scatha could reach great spiritual advancement and become a demigod.
  Look at \Criseis and \Psyrex. 
  This was also why drinking \draconian blood worked better for \scathae than for \humans. 
  
  Unlike Iquinians, \Ortaicans were expected to strive for greatness. 
  Humility was a virtue, but it was not given nearly as great emphasis as it was in Iquinian theology. 
  An \Ortaican was expected to know his own limitations.
  He must strive for purity before attempting anything that was out of his league.
  But he should always hope for more. 
  
  The \Ortaican religion was formed in the days of the \VaimonCaliphate. 
  As such, it was very much a reaction against the Vaimon faith, and many of its teachings were devoted to telling how bad Iquinianism was and how it should be opposed.
  
  \Ortaican mysticism was more naturalistic than Iquinian (and closer to the truth). 
  It taught that the world was a natural thing. 
  It existed and ought to exist. 
  It was the soul's own responsibility and choice to seek enlightenment and greatness. 
  Every mortal soul (at least, every \scathaese soul) had potential for greatness, but only if they approached it with virtue.
  
  A central dogma was that of the \arcana (singular \arcanum). 
  There existed a great number of \arcana. 
  An \arcanum was a piece of secret knowledge. 
  Every \rethyax spell was part of an \arcanum.
  Each god presided over one or more \arcana and could teach them to its followers. 
  
  Enlightenment was only for the select few, not the masses.
  The \arcana were secret for a reason.
  Knowledge was very dangerous in the wrong hands. 
  Only a strong soul could manage such knowledge. 
  Glancing an \arcanum prematurely could lead to disaster for you or your surroundings. 
  \Ortaican mythology was full of fables where someone stole an \arcanum that they were not ready for and came to a gruesome end as result. 
  
  Only a \rethyax should even try to learn any \arcana.
  Commoners should leave the esoteric mysteries to their betters.
  Some \arcana were deeper than others and had prerequisite \arcana. 
  
  \Arcanum theory was a simplification of the Aenigma theory the \dragons used. 
  Each \arcanum was a piece of well-known Gnosis.
  
  Some basic rules and principles of the \Ortaican religion were:
  
    \item 
      Only the strong and wise should try to be \rethyaxes.
      The \rethyaxes should be the leaders of society.
      Many non-Iquinian nations were ruled by \rethyaxes.
    \item 
      The strong and wise had a duty to protect the weak and ignorant.
      Not least, they had to be protected from themselves.
      This meant that the wise had to lead and create laws to keep the masses safe. 
      And prevent the masses from learning what they should not. 
    \item 
      The world was full of evil and horrors that hunger for the souls of mortals. 
      They would devour all the souls of the dead if they could and condemn them to everlasting suffering. 
      \Iquin was one such soul-devouring horror.
      (Have juicy myths about these horrors.)
      Mortals could not hope to defend themselves from such monstrosities.
      So they must serve the gods. 
      In return the gods would safeguard their souls from the horrors so that they could pass through the void unharmed. 
      The souls would then reincarnate to new life and be given a new chance to rise to greatness and enlightenment. 
    \item 
      Everyone had the potential for greatness in them, but not everyone should strive for greatness. 
      Some people's lot in life was to serve. 
      They should just live a virtuous life.
      If they did that, they would be reborn into a better life.
      Their chance would come soon enough if only they honoured the gods and their betters.
      One day, in another life, they might themselves become \rethyaxes. 
    \item 
      The masses should serve the \rethyaxes and the \rethyaxes should serve the gods.
      The gods, in turn, would grant power and knowledge to the \rethyaxes, and the \rethyaxes would protect the people and keep them safe. 
      A cosmic feudalism. 
    \item 
      No one should ever try to steal an \arcanum. 
      Instead, people should strive to be virtuous.
      Knowledge of \arcana did not make a soul better.
      Only virtue made a soul better. 
      Those who were worthy would be granted access to the \arcana in due time. 
  
  The basis of \Ortaican morality was a set of \leges (singular \lex) or pillars. 
  These were basic laws. 
  A religious constitution of sorts.
  \Ortaican dogma did not lay out much morality beyond the bare necessities.
  For example, there was no general law against theft.
  Instead, \Ortaican dogma merely dictated a social order.
  It would be up to secular rulers to formulate fitting laws.
  The Iquinians used this as an argument to show that \Ortaicanism was evil, since there was huge amounts of evil behaviour that the religion did not forbid and thus tacitly allowed and thus encouraged.
  (Notice the nice straw man argument.)
  
  The \rethyaxes sought to become closer to the gods and eventually become gods or demigods themselves.
  This quest, however, was in itself a secret \arcanum.
  The masses knew nothing of this ambition.
  
  The First \Arcanum was the \rethyax code that laid down the basic rules:
    \item How a \rethyax should interact with gods, commoners and other \rethyaxes.
    \item The basics (descriptive and moral) principles of magic.
    \item The rules for deference between \rethyaxes. 
    \item 
      The First \Arcanum also taught that the world was composed of a number of layers or planes. 
      Commoners could only perceive the most shallow layer, the Material Plane. 
      \Daimonia dwelt in the deeper planes. 
      \Rethyaxes learned how to feel and interact with the deeper planes. 
      They needed the help of the gods to do so. 
      
      The mere existence of the planes was an \arcanum.
      You had to learn other \arcana in order to actually feel the planes. 
      
      The \Primordials alone knew all the planes, and how many there were. 


Many non-Iquinian nations were ruled by \rethyaxes.
A \baccon was a \quo{council of the wise}, and wise meant mages, \rethyaxes.
Each nation in the \Serplands was ruled by a \rethyax \baccon or a sorcerer-king. 
It was only in the Iquinian lands that sorcerer-kings were prevented. 

What about the \Shurco?


Moro \Cornel studied chiefly the \arcanum of \Nasshikerr. 
Mention this. 


Moro was a part of the \Ubloth cult for a very short while, but soon left it again.
She never actually saw \Ubloth, but she felt its presence in magical rituals. 

Later, Moro remembered the loathsome amorphous god \Ubloth that dwelt in the deep darkness beneath Mount \Shrun near \Yormis. 
She feared the \Primordials, for she remembered the feeling of \Ubloth, and it felt similar to the \Primordials.
The \Ubloth cultists believed that \Ubloth was kin to the \Primordials and therefore worthy of worship.
Moro believed that if the \Primordials were kin of \Ubloth, then they should be reviled and loathed and shunned just as \Ubloth should. 


\Dragon blood and immortality:
  \Dragon blood worked best for \scathae, because \scathae were designed with a potential for greatness in mind. 
  
  For \humans and others, drinking \draconian blood was very dangerous.
  It had to be prepared with lots of special spells in order to be just drinkable.
  Deriving any power from it required lots of more spells. 
  An ordinary \human who drank \draconian blood would either die in horrible agony or mutate into a monster (and become mindless or raving mad in the process). 


Iquinians:

  Some Vaimons believed that \Iquin was all-powerful and could vanquish \itzach any day. 
  \Iquin refrained from doing so because it was disillusioned and disappointed by the sin and wickedness of living creatures, who gave themselves over to \itzach when they should worship \iquin alone. 
  Therefore \iquin allowed the Defiled world of \Gehinnom to exist.
  People had to be good and really deserve it before \iquin would grant them salvation.
  Those Vaimons who believed in omnipotence had a slightly more laissez-faire morality.
  After all, in the end everything would be all right. 
  The One Light would save the world. 
  Some of these even believed that \iquin was all-knowing and that everything was predestined. 
  
  Some Vaimons believed that \Iquin was \emph{not} all-powerful, that \iquin and \itzach were equal foes. 
  These endorsed a stricter morality.
  If \iquin was not all-powerful, then \itzach was a real menace and might even one day win.
  Then it was every mortal's duty to do his utmost to fight evil in all its forms. 
  The non-omnipotentialists saw the omnipotentialists as a threat because they embraced false complacency. 
  The non-omnipotentialists all denied the idea of predestination. 
  
  The issues of omnipotence and predestination were perhaps the two greatest points of contention in \iquinian theology.
  Wars had been fought between clans (and civil wars fought within clans) over such religious differences.
  Place each clan along these lines!
  What do the Redcor, Geicans and \Telcras believe?
  Remember that each clan may have changed its mind over the course of history.
  
  This was also a point of contention among the \resphain who masterminded the Iquinian church.
  Some of them believed the dogma of omnipotence was a great propaganda trick.
  Others believed it was too outrageous a lie, and that the followers would not keep buying it. 
  
  \ClanTelcra believed in omnipotence. 
  They believed that \itzach was no real threat and would always be overcome by \iquin.
  They spoke highly of the value and power of pure faith. 
  Therefore, they felt it was safe to invoke \itzach. 
  According to their scriptures, \itzach existed at the mercy of \iquin as its slave. 
  It was thus also given to the Vaimons, bound by the supremacy of \iquin to do their bidding. 
  Channelling \itzach was not without risks, of course, but sometimes it was needed, and times of need you had to do things that would otherwise be sinful. 
  Killing, for example, was normally a sin, but if you were fighting a just war, killing was acceptable if not required. 
  
  The Redcor, by contrast, did not believe in omnipotence. 
  They were more paranoid and ascetic than the \Telcras. 
  They feared \itzach and never invoked it. 
  
  Some of those who accepted omnipotence believed that the \qliphoth had no life of their own.
  They had to steal some essence from the One Light in order to live and exist. 
  They were hollow, empty shells; undead things with the semblance of life but lacking any true essence. 
  Some believed that the One Light, in its endless mercy and compassion, had allowed even these wretched half-beings to exist.
  Not all \Telcras believed this, but they did, overall, believe in an \Iquin more forgiving than the \Iquin the Redcor believed in. 
  
  The \Telcras had more popular appeal than the Redcor.
  The \Telcras preached that the One Light would triumph and everything would be good.
  Their \iquin was gentle and forgiving and peaceful. 
  The Redcor preached doom and danger and asceticism, which was not popular. 
  Their \iquin was stern and embattled. 
  The Redcor were more realistic, but the masses chose hope over realism. 
  
  The \Telcra/\Tepharin takeover was remarkably peaceful by \Miithian standards. 
  There were battles and wars fought, but it was mostly a quasi-peaceful religious reformation followed by lots of diplomacy. 
  For a century or two, the \Tepharin people dominated a large part of \Azmith. 
  
  The core message of Iquinianism was to follow the sixteen virtues.
  (Add this to the glossary.)
  
  The Iquinians believed that the diverse physical world was an evil.
  It was Defiled, called \Gehinnom. 
  It was a false world, unlike the true world of \Atziluth. 
  The duty of Iquinians was to purify \Gehinnom and return it to its state of purity and oneness.
  This holy duty to cleanse and thus unmake \Gehinnom was called \tikkun.
    \item 
      They sought to purify their own souls through righteous living and thus bring themselves back to \iquin.
    \item 
      They sought to spread the true faith so that other souls would be freed from \Gehinnom and return to \iquin.
    \item 
      They killed heathens so that their sould would return to \itzach where they belonged, and the heathens would not be able to spawn descendants nor spread their evil beliefs, which would otherwise condemn more souls and more spritual matter to maintain \Gehinnom. 
    \item 
      They fought evil religions to prevent their evil gods from keeping Defiled souls bound in \Gehinnom. 
  
  Animals were also Defiled, but they were not intelliget enough to be able to save themselves, so humanoids must do it for them.
  By butchering an animal with the correct spells and prayers and then eating its flesh, its essence would be freed from \Gehinnom and returned to \iquin. 
  So slaughtering and eating animals was a sacred, religious act, transforming Defiled matter into pure spiritual oneness. 
  The eater would become \quo{one} with the animal, thus bringing the world a bit closer to complete oneness.
  
  People would also say prayers before eating and drinking anything, even if not butchering it.
  (Add this when Rian drinks with Dennick.)
  
  The Iquinian religion was also called the \quo{Vaimon religion}.
  Post-\caliphate Geicans were not happy about this, since they had abandoned much of the religion but were still Vaimons.
  
  The \rethyaxes and others believed that Iquinianism was dangerous.
  It was an apocalyptic religion whose declared goal was the destruction of the world.
  It should be obvious to anyone that they were dangerous and should be fought.
  The Vaimons commanded powerful magic. 
  Who knew if the Vaimons might one day succeed in their crazy dreams to destroy the world?
  Have some quotes near the beginning of TAR telling of this theological conflict. 
  
  It was part of the sacred duty of \tikkun to fight the \wylde.
  The \wylde was a manifestation of Defilement.
  It sought to encroach on the habitations of humanoids and create even more defilement.
  It must be held back. 
  The Iquinians also believed that taking materials (such as wood and metal) from nature and using them to build things (such as houses, tools and weapons) was a holy thing. 
  The process took materials of pure Defilement from the \wylde and shaped them into things with a purpose, thus bringing them and the world closer to \iquin. 


Rayuth -> Layuth



2009-07-26

Ramiel's awakening
Throughout the book, the upcoming conflict between Ramiel and \Dasteron looms. 
It is a source of tension. 
Everyone knows it will happen and is waiting for it to erupt. 
Compare to the promised battle between Karsa Orlong and Rhulad in Reaper's Gale. 


Iquin:
  Iquinians believed that the true world, \Atziluth, which lay beyond \Gehinnom, was bright and pure and good and beautiful. 
  Therefore it came as such a dreadful shock when they glanced into the Beyond or otherwise learned that there existed true worlds Beyond that were far worse than \Gehinnom. 
  Worse even than \itzach.
  The traditional image of \itzach was an \quo{Outer Darkness}; distant and abstract and unreal. 
  The Beyond, once you got to know it, was very close, full of horrors awfully physical and slimy and smelly and slavering and \emph{real}. 
  
  Rian experiences this every time he sees something scary. 


\Iquin was the One Light. 
\Itzach was the Outer Darkness. 
Add this to Changes.


The One Light had kept itself hidden since the Defilement. 
Only select holy individuals had been allowed to see with the vision of the One Light.
These included Silqua and Cordos. 


At the time of the \thirdbanewar, \Cishiel was a Cabalist of the second circle. 
At the time of the \thirdbanewar, \Dasteron was a Cabalist of the first circle. 


\Urizeth was not a Cabalist at all. 
She was hired for the Malcur venture as an external consultant because of her great occult expertise. 
\Ganethed was the local occultist, but he was not good enough to do it all alone. 
He was a kinsman of \Urizeth, so he brought her in. 
\Achsah was also assigned to the project because of her occult experience, and because she was a High Telepath. 


\Menessiaraid worked on fabricating and maintaining religions, dogma and mythology. 
But not on \Azmith. 
Have references to this. 
Make the reader be critical of religions. 

Have references to \Menessiaraid's work with religions.
He tells \Teshrial about what he does, but not on \Azmith.
Make the reader be critical of religions. 

The \sephiroth should \emph{not} be exposed as evil in the first book. 
They should only be mildly suspected.
Overall they should be portrayed as pure and good, albeit not all powerful. 

\Psyrex tells someone: 
\ta{Your \sephiroth cannot save you, no matter how much they might \emph{love} you.
  Not any longer.
  It is too late for salvation for Malcur.}


The \quiljaaran under \Yormis rode on grotesque salamanders. 
(\QuilJaaran often rode.)
Moro \Cobrel had seen them do it. 


\RuinSatha was the first \xs with whom \Sethicus made a pact. 
\Sethicus would go on to make many important pacts with him. 
\Sethicus saw \RuinSatha as the most important \xs, and so he was represented as such in \draconian mysticism. 
But that might just have been \Sethicus's subjective opinion. 



The \VaimonCaliphate oppressed \scathae. 
In \Belzir's time there was a \scathaese (\Ortaican) uprising that had been brewing for centuries. 
Everything \Belzir did interacted with this rebellion. 
The rebellion was as important as \Belzir, if not more so, in bringing about the end of the \caliphate. 
But afterwards \Belzir was demonized because the surviving Vaimon clans wanted a scapegoat. 

One of the great successes of \ClanTelcra was integrating the \scathae. 
They preached that the Vaimons of the past had been unjust oppressors and heretics, and that in truth, \scathae were the equals of \humans. 
They demonized previous Vaimons and rejected the validity of the \caliphate (at least at first).
The Redcor were not happy about that. 


Vaimon clan -> \Vaimonclan. 



2009-07-27

In Rissitic mythology, Rissit \Nechsain was seen as a saviour figure. 
Their myths had it that \Nechsain had been sent by the eldest, true gods to free the world of the false gods that claimed it.
Unto \Nechsain the true gods had bestowed the power to overthrow the false gods and the authority to inherit the world. 

Rissitism was a breakaway faction from \Ortaica. 

Rissitism appeared after a religious/political schism among the \Taorthae. 
In Rissitic mythology this became a myth of a battle between the gods. 
Compare to the Egyptian myths about Set, Osiris and Horus, and how the various cults used the myths to demonize each other. 

Note that there was no war between the Rissitics and the orthodox 

\Nzessuacrith was one of those who opposed \Secherdamon in the schism.
Later, after \Ortaica had dissolved, \Secherdamon persuaded her to ally herself with him again. 



2009-07-28

Parts of a \scatha's consciousness felt like different gods or monsters. 
This was one of the \quo{truths} behind the myths that the various gods created various parts of the \scathaese mind. 
\Rethyax magic employed these mind aspects. 
A \rethyax had to activate his inner snake/worm/whatever to cast spells. 

It was horrifying when people recognized what other dark things were connected to things inside their minds.
Including various grotesque \xss such as \Ubloth.


Rian had a nice priest or monk who had been his spiritual guide in the process of breaking away from his life of crime. 
It was he who taught Rian the meaning of his name and a lot of other religious stuff. 
Unfortunately, the old priest died shortly before Neina was kidnapped. 

Remember to fix up the Rian chronology. 
The \quo{What Slithers Beneath} chapter is now closer to the rest of the story. 
Maybe there is not enough time to fit in Rian's rehabilitation between them. 



2009-07-30

Delphine -> Deshracca

Why \Ishnaruchaefir kills \Rystessakhin:
  \Ishnaruchaefir needed a sacrifice to complete his spell. 
  \Rystessakhin knew that, so she made sure to fuck his plan up.
  There was no one else around, and time was running out. 
  That was what \Rystessakhin meant when she said she would stop him with her life.
  She was certain he would not kill her. 
  
  She was wrong. 
  
  \Ishnaruchaefir was desperate.
  He knew time was running out.
  He had to act now.
  \Rystessakhin's refusal to see reason was making him furious.
  She had ruined all other options.
  
  \Ishnaruchaefir could not fail. 
  He had a duty to \Nexagglachel.
  He had failed him once before (when \Nexagglachel got captured instead of him). 
  He was willing to pay \emph{any} price in order to not fail \Nexagglachel again. 
  
  Just before this, \Ishnaruchaefir had sworn an oath not to back down no matter what. 
  No matter the cost to him or to anyone else. 
  
  He could sacrifice himself, but no. 
  He needed to be alive to maintain the spell and mop up afterwards. 
  He had to stay alive. 
  There were no other sacrifices available. 
  There was only \Rystessakhin. 
  
  So he killed her.


\Secherdamon's Sentinels were plagued by much corruption, infighting and inefficient bureaucracy. 
\Secherdamon had much trouble getting anything done. 
He could not be everywhere at once. 
And it was hard for him to determine which subordinates were faithful and which were twisting his will and running their own businesses behind his back. 
And he was dependent on them, so he could not just crack down on them. 

This is an aversion of the trope \quo{\trope{TheTrainsWillRunOnTime}{The Trains Will Run On Time}}.

This was why the Rissitic Dominion fell apart.


In the centuries before the \thirdbanewar, the Rissitic Dominion, which should have been a unified state, had practially split apart into a number of smaller rival states. 
This was because of corruption among \Secherdamon's Sentinels and other things. 

\Secherdamon realized the problem too late. 
He could not solve it in a top-down manner. 
He had to do some bottom-up work, too.
So he found \Narkiza.

\Narkiza was already an \Ashenoch and a recognized hero.
\Secherdamon spoke directly to \Narkiza and bade him conquer and unite all the Rissitic tribes. 
It was hard work, but \Narkiza did it.
\Secherdamon heaped many rewards on \Narkiza for it. 


\Narkiza was a brave but scarred soul.
He tried to be as good as possible within his brutal warrior culture and religion. 
But he was losing sanity through dark sorcery, becoming and \Ashenoch and communing directly with \Nechsain and other dark powers. 


Speaking True \Draconic was a sorcerous thing. 
Every word was a spell and a deep, possibly traumatic, cosmos-touching experience.
It was taxing and drained the sanity of mortals. 
Even formidable mortals like \Criseis only spoke True \Draconic when they must.

\Criseis only spoke True \Draconic when she had to.
It was \hr{True Draconic costs sanity}{hard on a mortal mind}. 


Pre-WSB: 
  Mention that \Ishnaruchaefir comes from his hidden dark citadel in the endless, twisting void. 


WSB:
  
  From \Criseis' POV: 
    \Criseis walks through the city.
    She carries \Ishnaruchaefir with her.
    She is connected to him and so provides a conduit that will help him break into the Tellurian Realm of \Azmith. 
    
    \Ishnaruchaefir is a terrible dark lord; remote, alien and frightening. 
    Even \Criseis fears him. 
    She feels his presence in her mind.
    He talks to her. 
    
    \Ishnaruchaefir speaks True \Draconic all the way through.
    \Criseis speaks Lowtongue to him.
    
    Be sure Rian describes \Criseis as old but healthy. 
  
  Dead garden: 
    There was a great hole in the middle of the dead garden.
    Simply called \quo{the hole}. 
    Not capitalized. 
    The dead garden and things in it did not have names. 
    
    The hole appeared to be bottomless and just went on down in the darkness. 
    It was a passage into the Beyond. 
    
    In the foremath of the \thirdbanewar, \Ishnaruchaefir tried to use the hole as a passage into \Azmith. 
    
    Improve the \quo{dead garden} section and link to it from all relevant sections. 
  
  From the \resphain's POV:
    \Ishnaruchaefir is a Cosmic Horror, a menace from Beyond that tries to break through the Shroud and surface into Malcur.
    The \resphain come to stop him. 
    They arrive to the garden before he does. 
    
    Rian sees them. 
    He wonders:
    \tho{What are these magnificent angels/gods doing?
      They appear to be waiting for something.
      What are they waiting for?}
    
    The \resphain talk about how to do it. 
    They are scared of him, and it clearly shows from their dialogue and behaviour. 
    
    \Achsah is the only one of the \resphain present who has encountered \Ishnaruchaefir before. 
    And she only saw him at a distance, wreaking havoc on her fellow \resphain.
    She fled in terror long before she could get a close look at him, much less fight him. 
    
    The purebloods present like to tell themselves they are not as cowardly as \Achsah. 
    But in the frightful calm before the battle, they lose their nerve and become less sure of themselves. 
    
    The \resphain discuss why \Ishnaruchaefir is being so overt. 
    It seems stupid.
    If he had something important to do here, surely he would be more sneaky about it.
    It does not add up.
    He must have some more sinister intention.

    They meet \Criseis.
    She is a dark sorceress. 
    She addresses them in the \Resphan tongue. 
    She warns them not to kill her.
    They can tell she is afraid of them, but she stands her ground even against several \resphain.
    \Teshrial admires her bravery. 
  
  \Ishnaruchaefir arrives: 
    He comes up out of the hole. 
    Or tries to.
    The \resphain try to hold him off while, from a distance, \Urizeth is working to hide their works and their \noggyaleth so he cannot fuck up their plan. 
    \Ishnaruchaefir bombards them from down in the hole. 
    They hear his voice.
    He kills \Teshrial.


There was a great war between the \ophidians and the \noggyaleth shortly before the \firstbanewar.
The \ophidians won and the \noggyaleth were forced underground.

The \noggyaleth had remained passive in the \firstbanewar. 
They were not highly intelligent. 
It took a long time for the \banes to get through to the \noggyaleth and form some kind of alliance. 
Until then, the \noggyaleth stayed hidden. 

Besides, the \noggyaleth were not stupid.
\Banes and \ophidians were bombarding the surface of the planet.
The \noggyaleth had a sense of self-preservation and did not want to expose themselves to that inferno.
Especially after having recently lost a major war. 

After the \firstbanewar, the \noggyaleth finally became convinced that it was in their interest to work with the \banes. 
They now attacked the remaining \ophidians. 
This was another factor that destroyed the \ophidian civilization. 

Since then and up till the \thirdbanewar, the \noggyaleth would hunt \ophidians.
Whenever the \ophidians tried to build greater cities and make any effort towards rebuilding their civilization, the \noggyaleth would attack. 
The \noggyaleth were insidious and cunning.
No matter how much the \ophidians tried to keep them out, they would worm their way in and find some way to fuck the \ophidians up:
  \item Dig underneath the city and make it collapse.
  \item Destroy roads and disrupt logistics. 
  \item Poison water or food supplies.
  \item Dig tunnels and allow vermin to enter. 
  \item Generally make life hell. 

Throughout the ages, the \ophidians tried to rebuild their civilization, but always the \noggyaleth would ruin it.
Over the millennia the \ophidians lost many of their records to \noggyal attacks and other conflicts, and it was hard for them to replenish or replace them.
Their numbers were dwindling and there were few left who remembered the old \ophidian civilization. 

Mostly, the \ophidians remained underground in small groups.
There they worked on long-term plans and conducted research into science and sorcery. 
They still did at the time of the \thirdbanewar. 

Most \ophidians were affiliated with the Sentinels, but many had goals of their own as well.

The \ophidians learned the art of shapeshifting even before the Shroud came. 
They used it to infiltrate many things, including the \aryothim and possibly even the \resphain. 
This was a great asset for the Sentinels. 
An \ophidian was a much smaller \vertex than a \dragon, so it could more easily hide.
This would be impossible for a \dragon in the long run (even a master of disguise such as \Nzessuacrith). 

An \ophidian could masquerade as a \bezed. 
This was not widely advertised among the \resphain.
Few such spies existed, and much fewer had ever been caught. 
The \resphan leaders were afraid of the spies, but they were not willing to admit that they existed.
They did not want to cause a widespread panic (like what happened in Battlestar Galactica).

\Saphyrae was the closest the \ophidians came to a new empire. 
It was a grand city.
But it was never completely thriving. 
The \noggyaleth besieged it from the shadows. 
So the \ophidians had little surplus to rebuild or innovate. 
Just keeping the city up and running was hard enough. 

When the \dragons awoke, they took control of \Saphyrae. 
\Nexagglachel set about rebuilding the \ophidian empire, and he was doing well. 
For two thousand years or so. 

\Saphyrae was destroyed in the \secondbanewar. 


\Ophidians looked not just serpent-like, but also \dragon-like. 
Their heads were much like those of \dragons (but without the horns). 
They had bony ridges on the top of the head and down along the spine and tail. 


Improve the sections about monsters in \Nyx and \Erebos. 
Have lots of cross-links.
Tell readers of one section to also read the other. 
Refer to the \flyingpolyps. 


The \flyingpolyps were some of the horrors whom even the \resphain feared.


Describe \Nasshikerr as slithering, writhing, loathsomely serpentine.


\Screamers were a kind of \banes. 
Their forms were designed to better combat \dragons. 

Biology:
  \Screamers were regular \banes reshaped into special forms. 

Physique:
  A \screamer looked kind of like the Xenomorphs from Alien and Alien versus Predator. 
  They had wings and could fly, like Zerg Mutalisks from Starcraft. 
  And great blades, like Zerg Hydralisks from Starcraft. 
  They were extremely fast and agile, built to evade a \dragon's melee and spell attacks. 

Psychology:
  \Screamers possessed humanoid intelligence, but they did not learn spells. 
  They were supported by \lesserbane, \baneknight or \banelord sorcerers. 


\Resphan swords and other melee weapons were covered with fields of shimmering energy.
Like forceblades from GURPS and vibro-blades from Rifts.
In all sorts of fantastic colours. 

Remember that different weapons are connected to the different Paths of \resphan martial arts. 

Link to the above from the relevant sections. 

Have lots of technology that looks like magic. 
Such as graph-glass.


There were myths about \quo{merfolk}: 
\Scatha-like creatures that dwelt in the sea and rivers. 
These were based on stories about sightings of \nagae. 

Mention the above under "Sea". 

\Nagae preferred to live in the salt water of the oceans, but they sometimes came into rivers. 



2009-07-31

Fix up the name of this \xs:
\Kythraxas is a name used by the \dragons.
\KythemCreiza is an \Ortaican name.
He was also worshipped by the \nagae under a third name. 



2009-08-01

Ideas:
  Maybe \Teshrial meets \Criseis and sets up date with her.
  \Ishnaruchaefir possesses her and speaks through her.
  He fakes fear of the Nadir and the Achilles Heel.
  
  It is a shaking experience for \Criseis.
  And also unsettling for \Teshrial. 



















