
\part{Other Immortal Characters}























\chapter{\Banes}
\section{\Daggerrain}
%\sectioncharunspec{Daggerrain}{\bane}{\neuter}
\target{Daggerrain}
Daggerrain is the \baneoverlord, the lord of all \banes{} on \Miith{} and surpreme leader of the Cabal. He serves the \baneking{} \Voidbringer, who still reigns on \Erebos. Daggerrain was the one who first contacted \Semiza{} after the \dragons{} had invaded the kingdoms of the \nephilim{} and is at least ten thousand years old. 

In all his thousands of years on \Miith{}, Daggerrain has never seen combat, for as the \baneoverlord{} he considers his person far too valuable to risk by entering battle. He often contacts high-ranking cabalists using \hr{Telepathy}{telepathy}, but other than \banelords{} and the lords of the \resphain, almost no one has physically met him. His residence is unknown.

Daggerrain's goal is not only his people's survival, but also a striving for evolution and perfection. He wants his people to become perfect, the greatest of all creatures of the universe. In this regard, compare him to the Zerg Overmind from the game \cite{VideoGame:Starcraft}.









\subsection{Name}
The name \Daggerrain{} is a rendering of his true name in the \hr{Bane telekinesis}{\ps{\banes}{} tactile language}. 
It was formulated by \Semiza, who felt \ps{\Daggerrain}{} terrible presence and likened it to being stranded amid an endless plain where sharp knives rain from the sky.  









\subsection{Master plan}
\target{Daggerrain's master plan}
\ps{\Daggerrain}{} long-term goal is to re-open the way to \Erebos, the so-called \quo{Gates of Apocalypse}. Compare to the song \bandsong{Mistigo Varggoth Darkestra}{The Key to the Gates of Apocalypse}. 

His master plan is brilliant and fool-proof. Remember to have scenes emphasizing how immensely skilled \Daggerrain{} is at running his \trope{XanatosGambit}{Xanatos Gambits}. 

Even with \ps{\HriistD}{} plotting and \ps{\Ishnaruchaefir} determined effort, the \dragons{} cannot match \ps{\Daggerrain}{} twenty thousand years of planning. 

He is the mastermind behind the wicked \hr{Sephirah plan}{\sephirah{} plan}. 





\subsubsection{The \firstbanewar}
Perhaps \Daggerrain{} never intended to win the \firstbanewar. Perhaps he just needed some fighting to distract everyone from the fact that he was building \Nyx{} as a gateway. He expected to lose the war, but retain the gateway. 





\subsubsection{Blind spot}
\target{Daggerrain's blind spot}
\Daggerrain{} has a blind spot: 
One parameter that he never quite understood and was never able to fully factor into his calculations. 

\begin{itemize}
  \item 
    Perhaps it is \hr{Ramiel betrays Banes}{Ramiel and his betrayal}. 
  \item 
    Perhaps this factor is \hr{Nexagglachel is Daggerrain's blind spot}{\Nexagglachel{} and his curse}.
  \item 
    Perhaps it is \hr{Azraid}{\Azraid} and \hr{Azraid hates Banes}{the hatred he secretly harbours} against the \banes.
\end{itemize}















\section{\Voidbringer}
\target{Voidbringer}
The \baneking{} whom \Daggerrain{} and all his \banes{} serve. 

\Voidbringer{} is a terrible \trope{CosmicHorror}{Cosmic Horror}, a \trope{SealedEvilInACan}{Sealed Evil in a Can} who waits to be unleashed upon the world. 

Compare him to Cthulhu from \authorbook{\HPLovecraft}{The Call of Cthulhu}.

The \Voidbringer{} is the eternal hunger, the endless devouring emptiness, a manifestation of \Bane{} \hs{Entropy}. If it ever comes to \Miith{} it will absorb and consume the \hr{Heart}{Heart of \Miith} and leave the planet a dead husk. Already when people on \Miith{} start summoning it, creatures all over \Miith{} can feel that something is horribly wrong, that some evil force is leeching life force from the Heart.

The \Voidbringer{} is mightier than even the \xss, powerful enough to almost be a \hs{cosmic god}. 

\citeauthorbook[p.122--124]{RobertEHoward:TheScreamingSkullofSilence}{Robert E. Howard}{%
  The Screaming Skull of Silence%
}{
  Silence!
  Utter and absolute!
  Throbbing, billowing waves of still horror!
  Men opened their mouths and shrieked but there was no sound!
  
  The Silence entered Kull's soul; it clawed at his heart; it sent tentacles of steel into his brain. 
  He clutched at his forehead in torment; his skull was bursting, shattering.
  In the wave of horror which engulfed him, Kull saw red and colossal visions\dash the Silence spreading out over the earth, over the Universe!
  Men died in gibbering stillness; the roar of rivers, the crash of seas, the noise of winds faltered and ceased to be.
  All Sound was drowned by the Silence.
  Silence, soul destroying, brain shattering\dash blotting out all life on earth and reaching monstrously up into the skies, crushing the very singing of the stars!
  
  And then Kull knew fear, horror, terror\dash overwhelming, grisly, soul-killing.
  Faced by the ghastliness of his vision, he swayed and staggered drunkely, gone wild with fear.
  Oh gods, for a sound, the very slightest, faintest noise!
  
  \ldots 
  
  The silence surged wrathfully about him.
  
  Mortal, who are you to oppose me, who am oldetr than the gods?
  Before Life was I was, and shall be when Life dies.
  Before the invader sound was born, the Universe was silent and shall be again.
  For I shall spread out through all the cosmos and kill Sound\dash kill Sound\dash kill Sound\dash kill Sound! 
  
  The roar of Silence reverberated through the caverns of Kull's crumbling brain in abysmal chanting monotones as he struck on the gong\dash again\dash and again\dash and again!
}























\chapter{Cosmic Gods}
\section{Achernar}
An evil star. 
The name is taken from \authorbook{Clark Ashton Smith}{The Isle of the Torturers}. 















\section{Targoros}
A mythical evil god. There is a celestial body, a large red nebula near the North Star, called the Eye of Targoros. 















\section{\XzaiShanns}
\target{Individual XS}














\subsection{\KhothSell}
\target{Khoth-Sell}
\index{\KhothSell}
%A \firstgendragon, the \draconic{} goddess of Death. 
A \firstgendragon, a \xs. 
Often depicted as female.

She was worshipped by the \dragons, \Ortaicans/\rethyaxes and Rissitics.
She was seen as a goddess of life, death, rebirth and immortality. 

\lyricsbs{Hate Eternal}{Catacombs}{
  Lord of Mictlan, land of the dead, \\
  deity of death and darkness.\\
  That which lies upon the graves. 
}

She was monstrous, alien and remote. 
But primal. 
She \emph{was} Death in a very real sense... somehow.
See also the section on \hr{Draconic immortality}{\draconic immortality}.

She was also a goddess of life and fertility. 
Compare her to Shub-Niggurath, the \quo{Black Goat of the Woods with a Thousand Young} from the Cthulhu Mythos. 

\lyricsbs{Emperor}{Wrath of the Tyrant}{
  He is the wind, He is the storm. \\
  He is the woods, He is the roots. \\
  Nobody will escape the wrath of the Tyrant. \\
  Forever the Beast shall wander the Earth.
}

In a sense (according to \hr{Sethican philosophy}{\Sethican philosophy}), \KhothSell was present inside all life.

\citeauthorbook[\quo{First Thought in Three Forms}, p.86--100]{%
  BentleyLayton:TheGnosticScriptures%
}{%
  Bentley Layton%
}{%
  The Gnostic Scriptures%
}{
  It is I who am the sound that was shown forth by my thinking. \\
  For it is I who ma the conjoined.\\
  I am called the thinking of the invisible.\\
  I am called the unchangeable voice.\\
  I am called she-who-is-conjoined. \\
  I am unique, incorruptible.\\
  It is I who am the mother of the sound:\\
  I speak in many ways; I complete the entirety; acquaintance (\emph{gnosis}) exists within me\dash acquaintance with the endless.\\
  It is I who speak in every creature; and I have been recognized by the entirety.\\
  It is I who impart the voice of the sound into the ears of those who have become acquainted with me, who are children of the light. \\
  And I came, for a second time, in the manner of a woman; and I spoke with them.\\
  And I shall instruct them about the coming end of the realmsAnd I shall instructed them about the beginningo fht e coming realm, which does not expereince change, and in which our appearance will change.\\
  They shall become purified within the aeons, in which I showed myself forth in the thinking of the image of my masculinity.\\
  I have put myself witin those who are worthy in the thinking of my unchangeable eternal realm.\\
  For I shall tell you a mystery of this realm,\\
  And I shall instruct you about the agencies that are within it.\\
  Birth is the production of an echo. For hour engenders hour, day engenders day, months produce months [...]. \\
  In such terms, this realm has become complete. \\
  And it has been reckoned, and is slight.\\
  For finger has loosened finger, \\
  and bond has been bound by bond. 
}

















\subsection{\KyaethemChreiAz}
\target{Kyaethem Chrei Az}
A \xzaishann.

\target{Kyaethem Chrei Az associated with air or water}
\KyaethemChreiAz was associated with the sea. 

He was a Cthulhu type, a Kraken, a Leviathan. 
Complete with a huge staring eye. 


















\subsection{\NaathKurRamalech}
\target{Naath-Kur-Ramalech}
\index{\NaathKurRamalech}
\NaathKurRamalech{}, is a \hr{XS}{\xs} who \emph{is} the dimensional barriers around \Miith{}. The Shroud is a part of him and his body, which he has allowed the \dragons{} and \banes{} to shape and twist. 

The gate to \Erebos{} is also a part of him.

Compare to the Outer God Yog-Sothoth from the Cthulhu Mythos by H.P. Lovecraft and others. 

When \NaathKurRamalech{} is invoked people nearby can feel the cosmic darkness stirring and vibrating deep in their bodies and souls. 

\lyricsbalsagoth{The Obsidian Crown Unbound}{
  The Ogre-Mage and the Swordmaster began to utter fearsome words in a tongue which was ancient ere the gleaming stars shifted upon the fathomless countenance of the distant heavens, words which in truth were not words, but rather a resonant key which would aspire to unlock a dire power which had reposed shackled since the fall of the legendary Shadow King himself, whose ebon circlet's power they even now sought to thwart.\\
  The incantation they gave voice to in the midst of that sanguineous turmoil which engulfed them was not so much heard by those within earshot as perceived, sensed as a vague disturbance in the fabric of reality, as fuliginous ripples on the surface of a hitherto still and placid pool, growing ever larger and more far reaching; an unnerving and unnamable sense of change which insinuated itself into the mind of the listener and suggested with a cold and disturbing quasi-certainty that something of preternaturally ineffable magnitude was transpiring, as surely as a festering and gangrenous corpse would split to spill its noisome gore.\\
  And as that maddeningly implacable incantation reached its resounding climax, a momentary silence enshrouded the battlefield, swathing the vista of chaos in an aura of noiselessness more pure and untainted than the tranquility of the boundless and stygian void.\\
  It was as if time itself had halted for one immemorial moment.\\
  And it was in that oddly immeasurable instant that the dark and peerless power unfettered by those grim pseudo-words finally, ultimately, made itself known before the sundered gates of ancient Gul-Kothoth...
}





\subsubsection{Philosophical role}
In \hr{Sethican philosophy}{\ps{\Sethicus} original philosophy}, before the \hr{Crystal Sphere}{\CrystalSphere}, \NaathKurRamalech was given little importance. 
He had only a small role in theology and philosophy.
This changed with the forging of the \CrystalSphere.
The \dragons and \ophidians realized how powerful \NaathKurRamalech was and how greatly they needed him.
He grew in importance in their later philosophy. 

Prayers to \NaathKurRamalech:

\citeauthorbook[p.92]{BentleyLayton:TheGnosticScriptures}{Bentley Layton}{
  The Gnostic Scriptures
}{
  Next, the perfect cild showed itself unto its eternal realms (aeons), which had come into existence for its sake.\\
  It showed them forth and bestowerd glory upon them, and gave them thrones.\\
  It stood at rest within the glory by which it had glorified itself.\\
  They praised the perfect child, the anointed (Christ), the deity, the noly-begotten, and they glorified it, saying:\\
  
  It exists! It exists!\\
  O child of god! O child of god! \\
  It is this that exists!\\
  O eternal realm (aeon) of the eternal realms,\\
  You who gaze at the eternal realms that you have engendered!\\
  For, you have engendered by your will alone.\\
  Therefore [we] glorify you.\\
  Ma! M\=o!\\
  You are omega, omega, omega! You are alpha! You are being!\\
  O eternal realm of the eternal realms! O eternal realm that gave itself!
}





\subsubsection{Status}
\target{Naath is the greatest}
According to some traditions, it was not \hr{Satha is the greatest}{\RuinSatha who was the greatest of the \xss}, but \NaathKurRamalech. 
He was certainly the \pps{\dragons} most important ally when in came to protecting \Miith from the \banes. 

Some believed that \NaathKurRamalech was younger and weaker than \RuinSatha, but simply seemed more powerful because he was closer tied to \Miith and took slightly more interest in the doings of his \Miithian worshippers. 














\subsection{\NerrhanKoss}
\index{\NerrhanKoss}
\target{Nerrhan-Koss}
\target{Nerran-Koss}
A \xs. 
One of the more mysterious of them, \NerrhanKoss{} is an alienist and occultist among the \xss{} themselves and consorts with \hs{cosmic gods} that are as gods even compared to the \xss. 

\NerrhanKoss{} had some dealings with \QuessanthIshnaruchaefir{} and \hr{Glaive origin}{sort of gave him his glaive}. 















\subsection{\RuinSatha}
\index{\RuinSatha}
\target{Ruin Satha}
\target{Satha}
\RuinSatha was a \xs. 
He was a chaotic dominator, said to reign from a basaltic throne at the seething and fiery centre of Chaos.





\subsubsection{Physique}
\target{Ruin Satha and fire}
\RuinSatha was considered the \xs god of fire. 
The \dragons and \rethyaxes derived much of their \hr{Ruin Satha fire magic}{destructive fire magic} from him. 

He was represented as an inferno of fire. 
He was depicted in varying \colours.
Sometimes yellow like regular flames. 
Sometimes green, black, gray or purple. 
Sometimes many \colours at once. 

Make sure he is a sickly and horrid flame. 
Not just a regular flame. 

\lyricsbalsagoth{The Scourge of the Fourth Celestial Host}{
  [UATU:]\\
  And lo, the Exterminator,\\
  the Destroyer of Worlds,\\
  the Purifier of Galaxies...
}





\subsubsection{Philosophical role}
\target{Ruin Satha philosophy}
\RuinSatha was the perfect, all-consuming, purifying flame. 
As such, \hr{Sethican philosophy}{\Sethican philosophy} considered him the ur-example of all life: 
A primal force that devoured and destroyed in order to live. 
A cosmic predator. 
\RuinSatha demonstrated that destruction could bring creation and innovation and new life, which became an important element in \ps{\Sethicus} philosophy. 

In a symbolic sense, \RuinSatha existed inside all living creatures.
He was the hunger, the aggression, the driving force. 

\target{Sethicus believes Ruin Satha created Ophidians}
\Sethicus believed the \ophidians were \hr{Ophidians related to XS}{were related to the \xss}.
He imagined they had been imparted hunger/will/motivation by \RuinSatha, and then physical life by \KhothSell. 
In that order. 
\RuinSatha represented, to \Sethicus, a deeper kind of life, a spiritual life. 
\KhothSell represented physical, bodily life, which was also important, but secondary to spiritual life. 





\subsubsection{Status}
\target{Satha is the greatest}
\RuinSatha was the first \xs with whom \Sethicus made a pact. 
\Sethicus would go on to make many important pacts with him. 
\Sethicus saw \RuinSatha as the greatest and most important of the \xs, and so he was represented as such in \draconian mysticism. 

\Sethicus believed that \RuinSatha dwelt in, and extended out through, the plane of \DaathKurZulNathla, the deepest plane of primal chaos. 
\RuinSatha's fire was a thing of \DaathKurZulNathla's chaos.
As such, \Sethicus believed \RuinSatha had achieved greater spiritual perfection than any other \xs. 
Therefore \Sethicus considered \RuinSatha the greatest of the \xss.
This need not imply that \RuinSatha was the most powerful or ruled over the other \xss, merely that he was the most enlightened of them. 

Not all later cultists (including \dragons and the \hr{Ophidians follow Sethicus under Durance}{\ophidians that idolized \Sethicus under the Durance}) understood this subtlety of \Sethican mysticism. 
In their watered-down theology, \RuinSatha became the \quo{king} of the \xss.

Yet others claimed that it was not \RuinSatha but \hr{Naath is the greatest}{\NaathKurRamalech who was the greatest of them}. 













\subsection{Lesser \XzaiShanns}





\subsubsection{\HothNrul}
\target{Hoth-Nrul}
\index{\HothNrul}
A \theraton. 

At one point it was \hr{Hoth-Nrul summoned}{summoned to \Miith}. 





\subsubsection{\Ubloth}
\target{Ubloth}
\index{\Ubloth}
\Ubloth was a minor \xs or \theraton godling that dwelt in a cave in \hr{Kai-Leng}{\KaiLeng} beneath Mount \hr{Shrun}{\Shrun} near \hr{Yormis}{\Yormis}. 

\target{Ubloth cult}
\Ubloth was worshipped by some of the \hr{Cults in Yormis}{\rethyax cultists in \Yormis} and gave them magical power in return. 
The \Ubloth cultists believed that \Ubloth was kin to the \Ortaican \hr{Primordial}{\Primordials} and therefore worthy of worship.
\hr{Moro and the Ubloth cult}{Moro \Cornel believed otherwise}. 

\Ubloth was a fairly simpleminded and unambitious creature.
It took the form of a huge, amorphous thing. 
Compare to Abhoth from \cite{ClarkAshtonSmith:TheSevenGeases}.

\Ubloth could be contacted through the \hr{Well in Yormis}{black well under \Yormis}. 

\Ubloth was traditionally considered neuter (neither male nor female). 





\subsubsection{\Yolbaoth}
\target{Yolbaoth}
\index{\Yolbaoth}
\Yolbaoth was a minor \xs or \theraton. 
He was quite willing to communicate with mortals and appeared rather eager to receive worship and give magical power in turn. 
Many of the darker \hr{Rethyax}{\rethyaxes} worshipped \Yolbaoth. 

Because of \Yolbaoth's great willingness to make pacts, he was invoked in many \draconic spells. 

\Yolbaoth dwelt in \hr{Kai-Leng}{\KaiLeng} underneath \Azmith. 















\section{\Zaz and \Urzaz}
\target{Zaz}
\index{\Zaz}
\index{\Urzaz}
\WanderersInDarknessEmph spoke of a mysterious pair of entities named \Zaz and \Urzaz. 
It was unclear whether these were \dragons, \xss, cosmic gods or even purely metaphorical entities, personifications of something abstract.
Compare to Gog and Magog from the \emph{Bible}.

\hr{Urizeth thinks Zaz and Urzaz are the Chimaera}{\Urizeth thought that \Zaz and \Urzaz} were the same as the \quo{\hr{Chimaera}{\Chimaera}}. 
But their true nature was more complicated and ambiguous than that. 

\Zaz and \Urzaz were real cosmic gods, albeit highly obscure ones. 
Compare them to Kur'oc and Gul'kor.
There was a time when \hr{Zaz denies Ishnaruchaefir}{\Ishnaruchaefir appealed to them for aid}. 

The stars in the sky that represented \Zaz and \Urzaz lay in the \hr{Malgryph constellation}{\Malgryph constellation}.






































\chapter{Imetric Gods}















\section{Dessali}
\index{Dessali}
\target{Dessali}
An \hs{Imetric} goddess who represents reason and knowledge. 

She was originally a Naiad, a water-dwelling spirit or demigod. 
She was a very inquisitive mind and discovered the truth of the Realms and the \feud{} through her research. 
She maintained her sanity by holding on to \quo{reason} as her guiding principle: If her axioms failed her, then she must discard them and seek new axioms. 
This has since then defined her existence and turned her into the badass thinker that she is today. 















\section{Eoncos}
\index{Eoncos}
\target{Eoncos}
Imetric god of war, strength and bodily health. 
His symbol is a \nycan, and he sometimes takes the shape of one. 

He might be related to the \hr{Ortaican gods}{\Ortaican{} gods}. 

He was originally a \nycan{} god worshipped by the \nycaneer{} tribes that lived side-by-side with the \Ortaicans. 
He succeeded in tying all (or most of) the tribes together and \hr{Imetric-Nycaneer alliance}{allying them with the nascent Imetrium}. 















\section{Hiothrex}
\index{Hiothrex}
\target{Hiothrex}
Imetric god of vengeance. 
He is really a \hs{Thorn Angel}. 















\section{\NishiS}
\index{\NishiS}
\target{Nishi-Settias}
Imetric goddess of life and death. 
Might originally be some sort of dryad-thingy. 






































\chapter{\Taorthae}















\section{\Daxian}
\index{Daxian}
\target{Daxian}
An \hr{Ortaican gods}{\Ortaican{} god}, associated with weather and the \Wylde{}. 
His partner and wife was \hr{Isxae}{\Isxae}. 

\Daxian was known as the Lord of All Storms, a dark and barbaric god. 
\hr{Llorgul}{\Llorgul} was his crueler kinsman. 













\section{\Isxae}
\index{Isxae}
\target{Isxae}
An \hr{Ortaican gods}{\Ortaican{} goddess}, associated with law and rulership. 
Her partner and husband was \hr{Daxian}{\Daxian}. 















\section{\Llorgul}
\index{\Llorgul}
\target{Llorgul}
\Llorgul of the Howling Wind was an \Ortaican god of the air and of northern cold and winter. 
Aloof and alien, he was known as the crueler, more monstrous kinsman of \Daxian.















\section{\Nasshikerr}
\index{\Nasshikerr}
\target{Nasshikerr}
An \hr{Ortaican gods}{\Ortaican{} god} of shadows, stealth and the hidden. 
Ofttimes a patron of thieves, spies and outcasts. 
Perhaps also a god of the underworld.

He was really a \quiljaar. 









\subsection{Physique}
\Nasshikerr looked grotesque, \hr{Appearance of Ortaican gods}{as \Ortaican gods tended to do}.
He often took a form resembling a chameleon, with a \scatha-like face and recognizable facial expressions. 

Describe \Nasshikerr as slithering, writhing, loathsomely serpentine.















\section{\NerrhanKoss}
\index{\NerrhanKoss}
Maybe \hr{Nerrhan-Koss}{\NerrhanKoss} the \xs{} is also considered a \hr{Taortha}{\Taortha}. 
An \hr{Ortaican gods}{\Ortaican{} god} of the afterlife and the occult. 
One of the darkest and most frightening of the \Ortaican{} gods. 
He is actually a \hr{front-end}{front} for a \xs. 















\section{Shellagh}
\target{Shellagh}
\index{Shellagh}
An \hr{Ortaican gods}{\Ortaican{} god}, associated with the sea. 
He helps people pass the sea safely. 
His symbol is a shark. 

He might be the son of \hs{Isxae} and/or \hs{Daxian}.
He might be the brother of \hs{Eoncos}. 

He fears \hs{Maegon}, a more powerful sea god who is actually a \nagalord. 
For this reason (and others), the Imetrians look down on Shellagh and see him as a weak and cowardly god. 





























\chapter{Others}















\section{\Haskelek}
\target{Haskelek}
\target{Haskelek story}
The \Haskelek{} is an old ally of the Sentinels and a terribly evil \daemon. 
It was brought to \Miith{} once about 1000 years ago (after Vizicar's reign but before \ps{\Belzir}) where it conquered and controlled a kingdom in central \Velcad{}, current day Pelidor and surroundings. 
But after a prolonged war, the \Haskelek{} was defeated by the Vaimons. 

%Unable to destroy it, the Vaimons imprisoned the \Haskelek{} in his temple, located in the middle of a huge, wild forest south or southeast of Pelidor. There it lay dormant, but its evil seeped out and corrupted the beasts and people in the surrounding area. The forests around its temple became dark, twisted and hostile (even more so than the regular \Miithian{} nature, which can be harsh enough) and the humanoid tribes to become degenerate savages who now worship the \Haskelek, or the memory of it. 

They were unable to destroy it, but they somehow managed to split its essence into several parts. Each part was now weak enough that they could imprison and entomb each of them in separate places. 
% and thus were able to imprison and entomb each of the weakened parts. 

In \emph{\TwilightAngelRemember{}}, the Rissitic and Rungeran armies \hr{Rissitics start Haskelek plan}{capture some strategic spots} that are vital in order to effect the resurrection of the \Haskelek, but they don't \hr{Raising the Haskelek}{begin the actual resurrection ritual} before \emph{\CarzainWithRedcorBook}. 







\subsection{One part is in \Redce}
One of the parts of the \Haskelek{} is hidden in \Redce, in a sacred place which the Redcor have been guarding for over a thousand years. 

This is a nice way to have Carzain fight the \Haskelek, since \hr{Carzain goes to Redce}{he is in \Redce}.









\subsection{Hatred}
When the \Haskelek{} awakens, it is hateful and filled with scorn against mortals. It longs to lash out at them.

\lyricslimbonicart{Behind the Mask Obscure}{
  In hellfire and damnation \\
  my undead soul walks the land.\\
  Throught the endless mist of time.\\
  I`m born to darkened adventures, \\
  retaliated in life obscure.
  
  To seek vengance for my pains,\\
  to serve the hatred in my veins.
  
  In centuries I have wandered,\\
  with deaths shadows drifting faithfully.\\
  In the darkest forests in man's domain\\
  I received my strenght and sorcery.\\
  My demon search for a doorway to be free,\\
  for once again to desecrate the hearts serenity.\\
  Transcend mortality, live throught eternally\\
  and feast upon all misery\\
  that is gathered here in life.
}









\subsection{The \Haskelek{} race}
The \Haskelek{} race is a people of great scientists and sorcerers. 

\lyricslimbonicart{Behind the Mask Obscure}{
  I`m born to darkened adventures, \\
  retaliated in life obscure.
}

The imprisoned one was mightier than most of its kind. 
So mighty that it almost rivaled a \dragonlord. 
Compare to Raest, the Jaghut Tyrant from \cite{StevenErikson:GardensoftheMoon}. 



















\section[Melcryth]{\Melcryth}
\target{Melcryth}
The alleged author to whom the poem \emph{\hr{Wanderers in Darkness}{\WanderersInDarkness}} is attributed. 
\quo{\Melcryth} is believed to be a pseudonym. 
His real identity, or even his race and gender, is unknown. 

Some suspect \hr{Nzessuacrith}{\CryocasNzessuacrith} of being \Melcryth, but she denies this. 































