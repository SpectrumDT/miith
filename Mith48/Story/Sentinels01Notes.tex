
















\section{Overview}
\begin{quote}
  \Miith{} is a cruel world. 
  
  \new
  For thousands of years a war has raged between the \dragons{} and the \resphain\dash a race of dark angels\dash and their mortal agents and pawns. 
  
  \new
  \VizicarDurasRespina, a long-dead sorcerer, awakens and finds himself a ghost trapped in the body of a strange man. 
  Vizicar must reconcile himself with his new body\dash and with a world that has changed in the many centuries since his death\dash before he can hope to complete his life's quest: 
  To discover who he is and why he is condemned to die and be reborn. 
\end{quote}




\begin{comment}
\bookchapter{Dramatis Personae}



% \section{Immortals and their servants}
\section[Dragons]{\maybehr{Dragons}{\Dragons}}
\begin{dramatispersonae}
  \dramitem[Sethicus]{\VardredSethicus}{\dragon}{male}, 
    founder of the \draconian race, 
    deceased
%     perished before the \secondbanewar{}
  \dramitem[Tiamat]{\TyarithXserasshana}{\dragon}{\female}, 
    ancient \dragon queen,
    deceased
%     perished before the \secondbanewar{}
  \dramitem[Nexagglachel]{\RaemythNexagglachel}{\dragon}{\male}, 
    \maybehr{Shae'eroth}{\shaeeroth}, 
    deceased
%     perished before the \secondbanewar{}
%     first son of \Tiamat{} 
  \dramitem[Ishnaruchaefir]{\QuessanthIshnaruchaefir}{\dragon}{\male}, 
    \maybehr{Shae'eroth}{\shaeeroth},
    called the Destroyer and the Exile
%     second son of \Tiamat{} 
  \dramitem[Secherdamon]{\IrocasSecherdamon}{\dragon}{\male}, 
    \maybehr{Shae'eroth}{\shaeeroth}
%     third son of \Tiamat{} 
  \dramitem[Nzessuacrith]{\CryocasNzessuacrith}{\dragon}{\female}
%   \dramitem{Kelthasserai}{\dragon}{\male}, 
%     fallen in the \maybehr{Second Banewar}{\Secondbanewar}
%   \dramitem{Skelcurmaggra}{\dragon}{\female}
%     fallen in the \maybehr{Second Banewar}{\Secondbanewar}
%   \dramitem{Izvathorn}{\dragon}{\male}
%     fallen in the \maybehr{Second Banewar}{\Secondbanewar}
%   \dramitem[Vaccashyth]{\Vaccashyth}{\dragon}{\female}
%   \dramitem[Dasvedshiracht]{\Dasvedshiracht}{\dragon}{\male}
\end{dramatispersonae}

\subsection{Servitors}
\begin{dramatispersonae}
  \dramitem[Psyrex]{\LocarPsyrex}{\scatha}{\male}, 
    leader of the \maybehs{Dark Crescent}
  \dramitem[Criseis]{\Criseis}{\scatha}{\female}, 
    servant of \Ishnaruchaefir
\end{dramatispersonae}




\section[Resphain]{\maybehr{Resphan}{\Resphain}}
\begin{dramatispersonae}
  \dramdead[Thanatzil]{\Thanatzil}{\resphan}{\male},
    the first \resphan
%     perished before the \secondbanewar{}
  \dramitem[Azraid]  {\Azraid}{\resphan}{\male}, 
    High Lord of \maybehr{CS}{\KiriathSepher}
  \dramitem[Teshrial]{\Teshrial}{\resphan}{\male}, 
    \maybehr{Ketheran}{\ketheran} of \CiriathSepher
    \begin{subdramatispersonae}
      \dramitem[Zereth]{\Zereth}{\resphan}{\female}, 
        daughter of {\Azraid}, mother of {\Teshrial}
      \dramitem[Tuerdal]{\Tuerdal}{\resphan}{\male}, 
        father of {\Teshrial}
    \end{subdramatispersonae}
  \dramitem[Firaxel] {\Firaxel}  {\resphan}{\female},
    \ketheran of \maybehr{Tiphred-Serah}{\TiphredSerah}
  \dramitem[Urizeth] {\Urizeth}  {\resphan}{\female}, 
    \maybehr{Thelyad}{\thelyad} of \CiriathSepher
  \dramitem[Ganethed]{\Ganethed} {\resphan}{\male},   
    \maybehr{Thelyad}{\thelyad} of \CiriathSepher
  \dramitem[Achsah]  {\Achsah}   {\resphan}{\female}, 
    \maybehr{Beuzed}{\bezed}
  \dramitem          {Lelmach}   {\resphan}{\female}, 
    \maybehr{Beuzed}{\bezed}
\end{dramatispersonae}

\subsection{Servitors}
\begin{dramatispersonae}
  \dramitem{Duma}{\human}{\female}, \naor matron
  \dramitem{Evith, Jirin, Luria}{\human}{\female}, young \naorim
\end{dramatispersonae}



\section[Pelidorians]{\maybehr{Pelidor}{Pelidorians}}
\subsection[House Pelidor]{\maybehr{House Pelidor}{House Pelidor}}
\begin{dramatispersonae}
  \dramitem[Icor]  [\Icor{} Pelidor]
    {\Rayuth[\Icor] Pelidor}{\scatha}{\male}, 
    ruler of Pelidor
    \index{Pelidor!\Icor{} Pelidor}
  \dramitem[Tiroco][\Tiroco{} Pelidor]
    {\Rinyuth[\Tiroco] Pelidor}{\scatha}{\female}, 
    his wife
    \index{Pelidor!\Tiroco{} Pelidor}
%   \begin{subdramatispersonae}
%     \dramitem[Roric]{Roric Pelidor}{\scatha}{\male}, 
%       their son
%       \index{Pelidor!\Tiroco{} Pelidor}
%     \dramitem[Frico]{Frico Pelidor}{\scatha}{\female}, 
%       their daughter
%       \index{Pelidor!\Tiroco{} Pelidor}
%     \dramitem[][egg]{An unnamed egg}{\scatha}{?}, 
%       their third child
%   \end{subdramatispersonae}
%   \dramitem[Liocai]{Liocai Pelidor}{\scatha}{\female}, 
%     \ps{\Icor}{} younger sister
%     \index{Pelidor!Liocai{} Pelidor}
  \dramitem[Sethgal][\Sethgal{} Pelidor]
    {\Rah[\Sethgal] Pelidor}{\scatha}{\male}, 
    \ps{\Icor}{} cousin
    \index{Pelidor!\Sethgal{} Pelidor}
  \dramitem[Dornaer][\Dornaer{} Pelidor]
    {\Rah[\Dornaer] Pelidor}{\scatha}{\female}, 
    \ps{\Tiroco}{} elder sister
    \index{Pelidor!\Dornaer{} Pelidor}
\end{dramatispersonae}

% \subsection{At the court in \Malcur}
% \begin{dramatispersonae}
%   \dramitem[Wulfwin Norden][\WimarNorden]{\Pater{} \WimarNorden}{\scatha}{\male}, 
%     \maybehr{Telcra}{\Telcra} \maybehr{Cleric}{\cleric} 
%   \dramitem{Iasper Bartholin}{\scatha}{\male}, 
%     ducal treasurer
%   \dramitem[][Graenell]{Baron Graenell}{\scatha}{\male}
%   \dramitem[Risvet Hemfork][\Risvet{} Hemfork]
%     {Baroness \Risvet{} Hemfork}{\scatha}{\female}
%   \dramitem[][Osphal Turmalin]
%     {Viscountess Osphal Turmalin}{\scatha}{\female}
%   \dramitem[Theal Kintair][\Theal{} \Kintaer{}]
%     {Earl \Theal{} \Kintaer{}}{\human}{\male}
%   \begin{subdramatispersonae}
%     \dramitem[Constance Kintaer]{\Constance\ \Kintaer}{\human}{\female}, his maiden daughter
%   \end{subdramatispersonae}
%   \dramitem{Charcoal}{\human}{\male}, 
%     Vaimon, 
%     Cabalist of the \charcoalcircle{} circle
%   \dramitem[Needle]{\Piacet\ (Needle)}{\human}{\female}, 
%     \ps{\Tiroco}{} handmaiden slave, 
%     novice Vaimon, 
%     Cabalist of the \needlecircle{} circle
%   \begin{subdramatispersonae}
%     \dramdead{Belya}{\human}{\female}, 
%       \ps{\Piacet}{} sister 
%   \end{subdramatispersonae}
%   \dramitem{Weyra}{\scatha}{\female}, 
%     \ps{\Tiroco}{} handmaiden slave 
%   \dramitem{Duen}{\scatha}{\female}, 
%     \ps{\Tiroco}{} handmaiden slave 
%   \dramitem{Sevac}{\scatha}{\female}, \ps{\Tiroco}{} bodyguard
%   \dramitem{Nobb}{\human}{\male}, a dungeon guard
% \end{dramatispersonae}

% \subsection{Commoners in \Malcur}
% \begin{dramatispersonae}
%   \dramitem{Rian}{\human}{\male}, a thief
% %   \dramitem[Badrick]{Patrick (\quo{Badrick})}{\human}{\male}, a thief
%   \dramitem[Bryon Carpenter]{\Bryon{} Carpenter}{\human}{\male}, 
%     an aging, childless man
%   \dramitem{Rod Baker}{\human}{\male}, a large man
%   \begin{subdramatispersonae}
%     \dramitem{Neina}{\human}{\male}, the baker's daughter
%   \end{subdramatispersonae}
%   \dramitem      {Dennick}{\human}{\male}, a thief
%   \dramitem[Uswa]{\Uswa}{\meccaran}{\female}, a drunken fortune-teller
%   \dramitem      {Jorgen}{\human}{\male}, \maybehs{Black Plague} gangster
%   \dramitem      {Ornen (Briar)}{\human}{\male}, 
%     Cabalist of the \briarcircle{} circle
% \end{dramatispersonae}

\subsection[The Ishrah]{The \maybehr{Ishrah}{\Ishrah}}
\begin{dramatispersonae}
  \dramitem[Moro Cornel]{Moro \Cornel}{\scatha}{\female},
    \maybehr{Rethyax}{\rethyax},
    academic head of the \ishrah{}
  \dramitem[Archibald Curwen][Archibald Curwen]
    {Lord Archibald Curwen}{\human}{\male}, 
    \maybehr{Telcra}{\Telcra} \maybehr{Templar}{\templar}, 
    military head of the \ishrah{}
  \dramitem[Onatol]{\Ambrose\ \Anatoli}{\scatha}{\male},
    \maybehr{Rethyax}{\rethyax}
%   \begin{subdramatispersonae}
%     \dramitem{Baernor}{\scatha}{\male}, \ps{\Onatol} apprentice
%   \end{subdramatispersonae}
%   \dramdead{Borg Zelab}{\human}{\male}, {\Telcra} {\templar}
%   \dramitem[Sanyor]{\Sanyor}{\scatha}{\male}, {\Telcra} \templar{}
%   \begin{subdramatispersonae}
%     \dramitem{Thedoro}{\scatha}{\female}, \ps{\Sanyor} apprentice
%   \end{subdramatispersonae}
%   \dramitem{Hicarro}{\scatha}{\female}, {\Telcra} \templar{}
%   \dramitem{Fiorae}{\scatha}{\female}, {\Telcra} \templar{}
\end{dramatispersonae}

% \subsection{The military}
% \begin{dramatispersonae}
%   \dramitem[][Gemadon]
%     {Captain Gemadon}{\scatha}{\male}, in \maybehr{Redglen}{\Redglen}
%   \dramitem[][Nimloc]
%     {Lieutenant Nimloc}{\scatha}{\male}, a scribe
%   \dramitem{Adrian Testor}{\human}{\male}, 
%     a wealthy \maybehr{Miksha}{\miksha} rider from \Redglen 
%   \dramitem{Rory}{\human}{\male}, a bladesmith from \Redglen 
%   \dramitem{Kreb}{\human}{\male}, a \Goyden{} boy
%   \dramitem[Tsekkect]{\Tsekkect}{\meccaran}{\female}, 
%     of the \maybehr{Thbatswa}{\Thbatswa} tribe 
%   \dramitem{Delph}{\human}{\male}, of \maybehr{Tepharin}{\Tepharin} descent
%   \dramitem{Gwelthein}{\scatha}{\female}, a \maybehr{Ranger}{\ranger}
%   \dramitem{Filcoi}{\scatha}{\female}, a \maybehr{Ranger}{\ranger}
%   \dramitem{Egian}{\scatha}{\female}
% \end{dramatispersonae}

\subsection{Others}
\begin{dramatispersonae}
  \dramitem{Claedd}{\scatha}{\female}, leader of Derwael
  \dramitem{Erowol}{\scatha}{\male}, an elder of Derwael
  \dramitem{Tur}{\scatha}{\male}, a youth of Derwael
  \dramitem{Bila}{\scatha}{\female}, a girl of Lunum
  \dramitem[Theal Kintair][\Theal \Kintaer]
    {\Rah[\Theal] \Kintaer}{\scatha}{\male}, noble and knight
\end{dramatispersonae}



\section{Imetrians}
\begin{dramatispersonae}
  \dramitem[Ilcas Northstar][Telcastora Ilcas]
    {\IlcSR{} Telcastora Ilcas \quo{Northstar}}
    {\scatha}{\male}, 
    \maybehr{Nycaneer}{\nycaneer} 
    \index{Ilcas!Telcastora Ilcas}
%   \dramitem{\Retaxis{} Raeco Mannica}{\scatha}{\female}, 
%     a \nycaneer{} and Ilcas' wife
%   \dramitem{Cassili Northstar}{\scatha}{\female}, 
%     their eldest child, Paladin-in-training 
%   \dramitem{Selcai Northstar}{\scatha}{\female}, 
%     their second child, a \nycaneer{}
%   \dramitem{Tarcus Mannica}{\scatha}{\female}, 
%     their third child, a soldier
%   \dramitem{Astor Mannica}{\scatha}{\female}, 
%     their fourth child, an apprentice mage
  \begin{subdramatispersonae}
    \dramitem{Countess}\nycan\female, companion of Telcastora Ilcas
    \dramitem{Razor}\nycan\male, companion of Telcastora Ilcas
  \end{subdramatispersonae}
  \dramitem[Ulphon Nestor][Ulphon Nestor]
    {\Ispan{} Ulphon Nestor}{\scatha}{\male}, 
    priest and mage
%   \dramitem{Equin Mirai}{\human}{\female}
\end{dramatispersonae}



\begin{comment}
\section[Rissitics]{\maybehs{Rissitics}}
\begin{dramatispersonae}
  \dramitem{\HriistN}{immortal}{\male}, supreme god (called Rissit by outsiders)
  \dramitem{\TesHanith\ \TsaltNyzleth}{\scatha}{\female}, high priestess
\end{dramatispersonae}
\subsection{In \FendorSmall}
\begin{dramatispersonae}
  \dramitem{Bantoyn \Rekkan-\Ondmyst}{\scatha}{\male}, 
    expedition leader (codename: Barrud).
  \dramitem{\Filgzed\ Hedrail Tsalt-\Sheshefkesad-\Ginfik}{\scatha}{\female}, 
    a mage (codename: Filiza).
  \dramitem{Dzavish Tsalt-\Sheshefkesad-\Bryn}{\human}{\female}, 
    a mage, her assistant (codename: Javiz)
  \dramitem{Br\^om}{\scatha}{\male} 
    (codename: Boruman), a sailor
  \dramitem{Dorm}{\human}{\male}, a Hazidi sailor
  \dramitem{Gelgein}{\human}{\male}, a Hazidi sailor
  \dramitem{Mamnik}{\human}{\male}, a trader (codename: Mamrim)
  \dramitem{Ashta}{\scatha}{\female}, a soldier (codename: Aisha)
  \dramitem{Juktat}{\scatha}{\male), a soldier (codename: Yudai) 
  \dramitem{A \Gisshorn\ agent}{\human}{\male}, codename: Mestos)
  \dramitem{Another \Gisshorn\ agent}{\human}{\male}, codename: Semphai)
  \dramitem{Ragev, Kirm and Noll}, \Filgzedz{} servants
\end{dramatispersonae}
\subsection{Near Fendor}
\begin{dramatispersonae}
  \dramitem{\Narkiza\ \Rekkan-\Neftzaid\ \Ashenoch-\Hashkfed} (\scatha{} \male), 
    general, wielder of the morning star \Femtu{}
  \dramitem{Belgrim}{\Cortio}{\male}, \Narkizaz{} mount
  \dramitem{Kufur \Rekkan-\Ondmyst}{\scatha}{\female}, 
    \Narkiza's assistant officer
  \dramitem{Geldashad \Rekkan{}-\Kozud{} \Ashenoch-\Fedza}{\human}{\male}
  \dramitem{\Dasvedshiracht{} Tsalt-\Shesshefkesad-\Ryzeyd{} \UrrGammosh}{\dragon}{\male}, 
    an undead \dragon{} mage
  \dramitem{Vekhtet Tsalt-\Shesshefkesad-\Kseinga]
  \dramitem{\Dzeredz}{\human}{\female}
\end{dramatispersonae}
\end{comment}



\section{Rungerans}
\begin{dramatispersonae}
  \dramitem[Morgan Runger][Morgan Runger]
    {King Morgan Runger}{\human}{\male}
%   \begin{subdramatispersonae}
%     \dramitem[Mathyas][Prince]  {\Mathyas}{\human}{\male}, 
%       Morgan's eldest son and heir
%     \dramitem[]       [Prince]  {Zacrias}{\human}{\male}, 
%       Morgan's younger son
%     \dramitem[Iselle] [Princess]{\Iselle}{\human}{\female}, 
%       Morgan's daughter  
%   \end{subdramatispersonae}
  \dramitem[Andros][Andros]
    {\Pater{} Andros}{\scatha}{\male}, 
    \maybehr{Telcra}{\Telcra} \cleric{} 
  \dramitem[Jirad Tantor]{\Jirad\ Tantor}{\human}{\male}, 
    \maybehr{Telcra}{\Telcra} \templar, \ishrah mage 
  \begin{subdramatispersonae}
    \dramitem[Mycah Tantor]{\Mycah{} Tantor}{\human}{\male}, 
      his son and apprentice
  \end{subdramatispersonae}
  \dramitem[Takestsha]{\Takestsha}{\human}{\female},
    \maybehr{Rethyax}{\rethyax}, \ishrah mage
  \dramitem[Orla of Fanshire]{\Orla{} of Fanshire}{\human}{\male}, \ishrah mage
%   \begin{subdramatispersonae}
%     \dramitem{Rosen Jaegwin}{\human}{\female}, his apprentice
%   \end{subdramatispersonae}
  \dramitem[Garog son of Otonn][Captain]{Garog son of Otonn}{\scatha}{\male}, soldier
  \dramitem[Enthon][\Frater]{Enthon}{\scatha}{\female}, 
    \maybehr{Telcra}{\Telcra} \cleric{} 
  \dramitem{Murein}{\human}{\female}, wise woman in \maybehs{Gedrock}
\end{dramatispersonae}




\section{Vaimons}
\begin{dramatispersonae}
  \dramitem[Carzain]{\CarzainDeracilleShireyo}
    {\human}{\male}, 
    rogue 
    \begin{subdramatispersonae}
      \dramitem{Arrow}{\relc}{\male}
    \end{subdramatispersonae}
%   \dramitem[Nishain Shireyo]{Nishain \Shireyo}{\human}{\male}, 
%     Carzain's father, \maybehs{Geican} Vaimon
%   \dramitem[Roanne Deracille]{\Roanne\ \Deracille}
%     {\human}{\female}, 
%     Nishain's wife, previously \maybehs{Redcor} Vaimon
%   \dramitem[Zacophine Vincerre][\Mater]
%     {\Zacophine{} \Vincerre}{\human}{\female}, 
%     Redcor, emissary to the court in \Malcur
%   \dramitem[Clarice Camilienne][\Soror]
%     {\Clarice{} \Camilienne}{\human}{\female}, 
%     Redcor, assistant to \Vincerre{}
  \dramitem[Chyrie Esmerel][\Matron]
    {\Chyrie\ \Esmerel}
    {\human}{\female}, 
    Redcor
%   \dramitem[Racel Galisetti][\Soror]
%     {\Racel Galisetti}
%     {\human}{\female}, Redcor, from \maybehr{Redglen}{\Redglen}
%   \dramitem[France Perival]
%     {\France\ \Perival}{\human}{\male},
%     Redcor \maybehr{Gandierre}{\gandierre}
%   \dramitem[Isacc Chiran]
%     {\Isacc\ \Chiran}{\human}{\male},
%     Redcor \maybehr{Gandierre}{\gandierre}
%   \dramitem{Soror Iselle}{\human}{\female}, Esmerel's assistant.
  \dramdead[Sylvie Dereine][\Sylvie\ \Dereine]{\Soror{} \Sylvie\ \Dereine}
    {\human}{\female}, Redcor historian
  \dramdead{Silqua Vaimon}{\human}{\female}, the first \maybehs{Vaimon}
  \dramdead{Cordos Vaimon}{\human}{\male}, 
    the first \maybehr{Vaimon Caliphate}{\VaimonCaliph}
%   \dramdead[Arcan Delain]{Arcan \Delain}{\human}{\male}, founder of \ClanDelaen
%   \dramdead[Lestor Delain]{Lestor \Delain}{\human}{\male}
%   \dramdead{Grith Ecallivan}{\human}{\male}
  \dramdead[Vizicar]{\VizicarDurasRespina}{\human}{\male}, Delain, once \caliph
\end{dramatispersonae}



\section{Others}
\begin{dramatispersonae}
  \dramdead{Catrian}{\human}{\female}, a weaver's wife in Bendaire
  \dramitem[Dorian]{Dorian}{\scatha}{\male}, in Bendaire
  \dramitem{Gaston}{\human}{\male}, a citizen of Bendaire
  \dramitem{Grum}{\human/\nephil half-breed}{\male}, bandit leader
  \dramitem{Rogg}{\human}{\male}, bandit 
  \dramitem{Ivar}{\human}{\male}, bandit 
  \dramitem{Gawl}{\human}{\male}, bandit 
  \dramitem{Faeni}{\human}{\female}, bandit 
  \dramitem[Shiaraid]{\Shiaraid}{\malach}{\female}, a dormant \maybehr{Vertex}{\vertex}
%   \dramitem{An angel}{?}{?}
%   \dramitem[Nasshikerr]{\Nasshikerr}{\Taortha}{\male}, \Ortaican god of shadows and stealth 
\end{dramatispersonae}








\end{comment}









\subsection[Malcur]{\Malcur}
Much of the book is set in or revolves around the city of \hr{Malcur}{\Malcur}, the capital city of \hs{Pelidor} and a powerful \hr{Nexus}{\nexus}.





\subsubsection{Sentinel plan}
\target{Secherdamon wants Nithdornazsh}
\target{Ghost Tower ploy}
The \hs{Sentinels} have a plan: 
\hr{Secherdamon}{\Secherdamon} wants to resurrect the \draconic{} fortress of \hr{Nith'dornazsh}{\Nithdornazsh}, and he wants to do it in \Malcur. 

The overall plan is this:

First they go for the \hs{Ghost Tower} near \hr{Forclin}{\Forclin}, besiege it and send in their Rissitic agents. 
The Cabalists immediately smell that something nasty is brewing, so they send in agents to stop them. 
They even send \banes. 

It comes to a major confrontation between Rissitic and Cabal agents. 
\hr{Nzessuacrith}{\Nzessuacrith}, disguised as \hr{Takestsha}{\Takestsha}, is forced to change to \draconic{} form to fend of the \bane{} attack. 

The Cabalists immediately detect the \dragon{} and send loads of reinforcements to the Ghost Tower, bent on fighting off the Rissitics (Sentinels?) at all costs. 

%But this was all part of the Sentinels' plan. Having drained Pelidor of Cabal resources\dash every Cabalist that matters off to defend the Ghost Tower\dash Sentinel agents in \Malcur now proceed to the cemetery. 
\target{Rissitics start Haskelek plan}
Now, \Nzessuacrith{} genuinely does want the Ghost Tower, because it contains one of the parts of the \hr{Haskelek}{\Haskelek}. But her whole attack is a diversion. She works for, and is unwittingly being manipulated by \Secherdamon, who has his own plans. Having drained Pelidor of Cabal resources\dash every Cabalist that matters off to defend the Ghost Tower\dash his agents in \Malcur now commence with their arcane ritual, to achieve the resurrection of the fortress of \hr{Nith'dornazsh}{\Nithdornazsh}. 

\target{Nithdornazsh is a useful gateway}
\hr{Dark ancient cities}{Ancient immortal cities} reached out into the Beyond. 
They were built in a time before the Shroud, when the barriers between the Realms were much more permeable. 
Their streets and corridors and towers were built so they criscrossed the Realms. 
This was why \Nithdornazsh was so useful as a gateway between \Machai and \Azmith. 

The Sentinels plan to make \Malcur ready for the great change and the Resurrection by gradually seizing control of the Shroud over the city and subtly changing people's beliefs. 
When they have enough Shroud power, they can mind control the people and channel their spells through them. 
The people of \Malcur themselves will facilitate the Resurrection. 

\target{The Change of Malcur}
The ordinary thugs do not know about the Resurrection. 
They only know about some great upcoming event called \quo{the Change}. 





\subsubsection{\ps{\Secherdamon} role}
\Secherdamon{} should be portrayed as a distant, mystic dark lord. 

Keep his name secret for a long time and have people refer to him only by various ominous titles. 

Compare him to Lucifer from \FLuneNoire. 






\subsubsection{The summoning ritual}
%There is a \vertex{} in the Shroud somewhere in \Malcur. The Sentinels want to draw a thread through this \vertex{} to form a connection to \Machai{} and pull \Nithdornazsh{} into the world. 
To this end, they need a magical ritual. \ps{\Secherdamon} servitor \hr{Psyrex}{\Psyrex} is responsible for this. He pulls the strings of the \Malcuric{} Sentinels and the criminal underworld in order to set up the ritual. 

The plan utilizes the \hr{Occult geometry}{occult geometry} that was used in the building of \Malcur.

The ritual to pull \Nithdornazsh{} from \hr{Machai}{\Machai} into \Miith{} requires \quo{beacons} to be set up at strategic locations. 
Each of these beacons must be secured and prepared for the ritual, including humanoids captured and ready to be sacrificed. 
The Sentinel-led mafia, masterminded by \Psyrex, can set up the beacons, but they need help diverting the town guards and the church. 
To do this, \Psyrex{} recruits \hr{Tiroco}{\rinyuth[\Tiroco]}. 





\subsubsection{\Secherdamon versus \noggyaleth}
I am not sure if \hr{Secherdamon}{\Secherdamon} is aware of the \noggyaleth. 
If not, then \hr{Ishnaruchaefir tells Secherdamon of the Ghobaleth}{\Ishnaruchaefir{} tells him at the end} and thus forces \Secherdamon{} to \cooperate{} with him.

If \Secherdamon{} does know, then he must be planning for it. 
He knows that the \noggyaleth{} can fuck up his resurrection ritual. 
They can even use their Shroud-weaving power to take control of the ritual and use it for Cabal purposes\dash but only if \Teshrial, their master, or \Achsah, his servant, is there to guide and control the mindless abominations. 
Thus the \hs{Ghost Tower ploy} to lure \Achsah{} out of \Malcur.
Then \LocarPsyrex{} and his Cabalists ought to be able to handle whatever \Teshrial{} can whip up. 

But the \noggyaleth{} are more strongly present than \Secherdamon{} believes. 
They have taken root and entrenched themselves much deeper than he thinks. 
So they can disrupt the spell, with catastrophic consequences for everyone. 
This fucks up \ps{\Secherdamon} plan. 

But \Ishnaruchaefir{} knows this, because only he is crazy and reckless enough to seek out the \noggyaleth{} and fight them in \melee{} combat, which \Secherdamon{} or \LocarPsyrex{} would never dream of doing. 
\Ishnaruchaefir{} finally \hr{Ishnaruchaefir tells Secherdamon of the Ghobaleth}{warns \Secherdamon{} about the problem}. 

Then \Ishnaruchaefir{} fights the \noggyaleth{} and holds them off. 
They are not all slain, but held off and weakened enough for \hr{Psyrex-tachi invoke the ritual}{\Psyrex-tachi to complete the ritual}. The sorcerers now have enough control of the local Shroud to be able to trap the \noggyaleth, and so turn the worms' Shroud-weaving power against them and their masters and use it to \hr{Nith'dornazsh rises}{resurrect \Nithdornazsh}. 

In this interpretation, the \noggyaleth{} are a vital part of the Sentinels' plan as well.





\subsubsection{\QuilJaaran in \Malcur}
\target{QJ in Malcur}
Maybe there are \quiljaaran in \Malcur, working on the side of \Secherdamon's Sentinels to further their dark plan.
They are the leaders of the Sentinel mages and the ones overseeing the dark ritual. 

Moro and Rian each see a glimpse of a \quiljaaran.
He sees through the Shroud for a moment, and its disguise slips. 
It used to just be a \scatha, but suddenly he sees it as a serpent thing. 
It is horrible to behold. 
He only sees it for a moment.
Then it slinks back into its tunnel/cellar or behind a door or whatever. 

\target{Moro and Rian rationalize QJ}
Both Rian and Moro block it out and try to deny it and rationalize it away.
They did not see a snake.
It was just a regular \scatha that looked weird in the light. 
Nothing more. 

But when they \hr{Moro and Rian realize QJ exist}{meet and compare stories}, they realize the snakes are real.





\subsubsection[Ishnaruchaefir's stealth]{\ps{\Ishnaruchaefir} stealth}
\target{Exile intersecting with Pyre}
\Ishnaruchaefir{} is stealthy. 
He helps \Secherdamon-tachi, but not too directly. 
He does not combine his powers with theirs in a straightforward manner. 
If he did that, his alliance would be astrologically detectable; astrologers would be able to see in the sky that the \hs{Exile} was intersecting with the \hs{Pyre}. 
He doesn't want that. 
So he merely helps them out without directly aiding them. 

Later \hr{Azraid muses on Exile and Pyre}{\Azraid{} will muse on this}. 





\subsubsection{Twist ending: Evil wins}
At the end, evil wins. 
\Malcur falls and \Nithdornazsh{} rises. 

\target{stop the evil}
This must be a surprise for the reader. 
Remember to have tons of references to how the heroes want to \quo{stop the evil}, so the reader thinks they will succeed. 










\subsection[Cabal plan for Malcur]{Cabal plan for \Malcur}
\target{Cabal plan for Malcur}
\target{Teshrial's creatures}
\target{Teshrial's monsters}
\target{Malcur gambit}
\target{Malcur venture}
The \hs{Cabal} have their own plans for \Malcur. 
\hr{Teshrial}{\Teshrial}, the leader of the Cabal in the Pelidor region, has brought several of the dreaded \hr{Ghobal}{\noggyaleth} to \Malcur. Perhaps these \noggyaleth{} have been there for thousands of years, or have gradually dug their way to \Malcur from Erebos over the course of thousands of years. 

Anyway, since \Malcur is a powerful \nexus, \Teshrial{} intends to use the \noggyaleth{} to open a portal to \hr{Erebos}{\Erebos} through \Malcur. 

The worms are hiding beneath the city. 
If the Sentinels try anything major, he is confident that the \noggyaleth{} can handle it. 

The purpose of the Cabal's \Malcur venture was to bore a hole from \Nyx and into the deeps of the planet \Miith.
They would build a conduct to the life-giving Heart and provide life and fertility to their race, which had long been dwindling.
It was a great and glorious undertaking.

The \resphain involved have very high expectations of this gambit and hope it will determine the future fate of \CiriathSepher, if not all \resphain. 
The ones not part of the gambit are more \skeptical. 
\Azraid has hopes for the venture, but remains \skeptical and aloof.
Many did not believe it would succeed, but \Teshrial and his associates were enthusiastic.
\Teshrial looked very much forward to it.
When the great bridge was complete, he would take \Firaxel to it, and they would have wonderful sex, and she would conceive, and he would be a father and a hero.

\target{Cabal stations near Malcur}
The \resphain working in \Malcur have some viewing stations and stuff set up in a Realm adjacent to \Azmith. 
Later \hr{Ishnaruchaefir attacks viewing station}{\Ishnaruchaefir attacks one such station}. 

There were some \noggyaleth under \Malcur.
They were part of the Cabal's plan.
They would drill the hole through the dimensions and pave the way for the bridge.
\Urizeth was there.
She was the party occultist and charged with dealing with the \noggyaleth.
There was one of several \noggyaleth.
It did not quite make sense to try to distinguish between individual \noggyaleth, for they would merge and split apart when they willed\dash{}or when their masters told them to.
\Teshrial feared the \noggyaleth and did not understand them.
He left it up to \Urizeth to manage them.
It is \Urizeth who performs the spells to direct the \noggyaleth, because \Teshrial is too scared to learn them. 

Read about how the \noggyaleth \hr{Noggyal corruption}{corrupt the planet}. 

\Teshrial \hr{Teshrial fears Noggyaleth}{is afraid of the \noggyaleth}. 

The \banes had a darker plan for \Malcur.
It was really a part of their master plan for opening the way from \Erebos to the Heart of \Miith and the \noggyal mother-mass.
The \noggyaleth played a dark, terrible role that the \resphain did not suspect.
The \resphain just used the monsters, stayed away from them and refrained from asking questions.
They feared both \noggyaleth and \banes and did not want to know any more than they already did.

The time when \Secherdamon plans to resurrect \Nithdornazsh coincides approximately with \Ishnaruchaefir's Nadir.
This is not by chance. 
In this period, the Shroud is thin, so they have a better chance of succeeding, breaching the barriers between the worlds and bringing their citadel to \Azmith. 
The Cabal's \Malcur venture is also mouthing out into some conclusion at this time. 
Or was supposed to. 

\hr{Urizeth is not a Cabalist}{\Urizeth was not a Cabalist at all}. 
She was hired for the \Malcur venture as an external consultant because of her great occult expertise. 
\Ganethed was the local occultist, but he was not good enough to do it all alone. 
He was a kinsman of \Urizeth, so he brought her in. 
\Achsah was also assigned to the project because of her occult experience, and because she was a High Telepath. 





\subsubsection{\Ishnaruchaefir is a menace}
Perhaps \Ishnaruchaefir needs not attack and destroy stuff in order to be a threat. 
I just need to clarify that if he is not stopped soon (chased away or preferably killed), he will wreck everything they have worked for in \Malcur.
When he is at his full strength, he could attack in force and drive the Cabal out of \Malcur entirely.
It is known that he takes an interest in \Malcur, so he likely has long-term evil plans there.
He gave hints of that in WSB. (Make him give hints!)
That must not be allowed to happen.
Furthermore, even now that he is weak, he might be up to something.
If \Urizeth's conclusions are correct, then these Nadirs happen to him regularly, and if so, \Ishnaruchaefir must have learned long ago to live with them and still get stuff done.
One must not assume that he is harmless in his Nadir.
(Maybe it is \Azraid who speculates the above to \Teshrial.)










\subsection[Tantor's journal and Eresh-Kal]{Tantor's journal and \EreshKal}
\target{Tantor's journal}
The Sentinels, who are secretly supporting Runger, have planted a diary (written by Rungeran mage \Jirad{} Tantor) among the possessions of \Ambrose{} \Onatol, a Pelidorian \ishrah{} mage. 

%Tantor relates how he, together with a Rungeran expedition, finds a forgotten temple that appears to be many thousands of years old, built with odd shapes that twist the eye and seem to defy geometry and reason, and adorned with carvings of horrific monsters. 
Tantor relates how he is sent on a secret expedition, led by \Takestsha, King Morgan's new advisor. \Takestsha{} is looking for the lost temple of \Rungertemple, in which is supposedly hidden a vast wealth of mystic knowledge. 

After a harrowing trek through the \Wylde{} they find the temple. It appears to be many thousands of years old, built with odd shapes that twist the eye and seem to defy geometry and reason, and adorned with carvings of horrific monsters. 

In this tower they find a wealth of mystic knowledge: Books, scrolls and artifacts. All in all a tremendous source of magic. They bring it back to the king. 

Not only do they have to hide this knowledge from the Iquinian church, but some mysterious people also appear and try to steal the magic and kill the finders. They use magic, and from their technique and behaviour Charcoal estimates that they are very likely Sentinels. But they are defeated or out\manoeuvred and the magic is brought to the king. Tantor has his reservations, but he follows the leader, a very strong-willed mage and a loyal Rungeran (perhaps a relative of the king). 

Tantor fears that Morgan Runger wants to use this terrible magic to conquer. He warns \Anatoli{}. 

It turns out that the discovery occured a year or a few years ago. Tantor and the other participants were sworn to silence, and he only recently gathered up the courage to write this. 






\subsubsection{The truth} 
The whole story about how Morgan's people discover a cache of magical knowledge is a hoax. It's meant to cover up the fact that the Rungerans are armed with Rissitic/Sentinel-provided magic. The Sentinel-looking characters that try to stop the Rungerans are completely fictional. They are there for the specific purpose to convince a Cabalist reader that the whole thing is \emph{not} a Sentinel plot (which it is). 

%The tower is real, however. Its location and nature are known to both the Sentinels and the Cabal. But it's guarded by nasty things, and the Rungerans were never there. (Or were they?) 
\Rungertemple{} is real, however, and so is the expedition. \Takestsha-tachi really were there and picked up some occult tablets and scrolls, but these are not as valuable as they are eld to believe. The actual mystic knowledge was provided by \Takestsha, who is really the \dragon{} \Nzessuacrith. 

Giles Tantor is a real person and \emph{did} correspond with \Anatoli{}. His account is real, although he was being mind-fucked to some extent by \Nzessuacrith. 
%But this story is not his work. The real Tantor was killed by the Sentinels, and his \quo{journal} was
A few fragments are concocted by the Sentinels and inserted afterwards. Tantor was later killed by the Sentinels, and his edited journal was sent to \Anatoli{} in \Malcur, ostensibly to warn him, but in reality it was meant to be leaked to Charcoal. 
Then \Anatoli{} was killed so he wouldn't spill the wrong beans. 

The \EreshKali{} magic is Rissitic magic, but not standard issue. 
It is a new experimental kind. 
\Secherdamon{} intends to kill two birds with one stone by field-testing his new spells and winning the Runger war at the same time. 
Also, the Rungeran mages are much more expendable than precious Rissitic ones. 
\Secherdamon{} loves his own people. 









\subsection{Rissitic invasion}
\target{Rissitics and Runger}
Shortly before the Rungeran attack against Pelidor, the Rissitic Empire launched invasion forces in southern \Galessan. 
They are especially targetting \hs{Sumian}. 
Or maybe \Scyrum. 

This is part of a bigger long-term strategy. 
\Secherdamon{} wants a bigger foothold in the world. 
And it also serves as a diversion, a means to distract the world from the Runger-Pelidor war. 
With Rissitics attacking from the south, they will draw all the eyes to them, and the world will not pay much attention to a little border dispute between two unseeming little kingdoms. 
The Redcor Church and the Imetrium will be far too busy to respond to Pelidor's call for aid. 

But in reality the Runger-Pelidor war is more important than the Rissitic invasion, because it sets the stage for the resurrection of \Nithdornazsh. 

Remember to make it clear that Durcac = Rissitics! 
Write this multiple times! 

And remember to advertise \hr{Narkiza}{\Narkiza}. 
Namedrop him and tell everyone how skilled and feared and powerful and cool he is. 





\subsubsection{Imagery}
Some people do not understand the fuss because they have heard that Durcac is relatively small and its warriors few in number.
The wiser people know that it is the Rissitics' superior technology and powerful sorcery that make them dangerous.

The Pelidorian court should be discussing the \hr{Rissitics and Runger}{Rissitic invasion of \Scyrum/Sumian} and what it means for all \Galessan. 
Make it clear that the Redcor Church is supporting the \Scyrics/Sumianese, and the Imetrium may throw its weight in as well. 
Everyone is very disturbed by the Rissitic invasion. 

The Rissitics are feared. 
Have references to the dreaded, mighty warlord \Narkiza, who is more than a \scatha. 
An immortal warrior mage. 
Almost a demigod. 
An \hr{Ashenoch}{\Ashenoch}. 

Be sure to \hr{Rissitic reputation}{demonize the Rissitics}. 
Rumour turns them into legions of demons from Hell, inhuman wielders of dark sorcery and evil.
The fact that \Narkiza is known to be an \Ashenoch, and that the Rissitics have always relied on magic and monsters, only makes their image worse.
They are seen as the marauding Legions of Chaos.
\Tiroco hears reports of towns and cities in the south being overrun by screaming, wicked Rissitics.

\lyricsbalsagoth{
  Behold, the Armies of War Descend Screaming From the Heavens
}{
  Behold, the armies of war descend screaming from the heavens!
}

Among other things, the Rissitics have an \quo{Order of the Hippopotamus}. 
The hippopotamus is the largest and most dangerous mammal in the world, as far as some people know.
\Tiroco has not seen a hippopotamus, of course, but she has heard of them.
She sees it as a loathsome symbol of bestial brutality. 
















\section{The Dreaming Predator}
\subsection{The \SecondShrouding}
Have a flashback to somewhere around the time of the \hr{Shrouding}{\SecondShrouding}. 
Probably seen through the eyes of \Nzessuacrith. 

We see her flying over a vast battlefield, strewn with the corpses of mortals, \dragons{}, \cuezcans, \banes{} and \resphain. 

She meets \Secherdamon{} and talks to him. 
He spews his hate for \Ishnaruchaefir{}, since this is shortly after the tragic \hr{Fall of Nexagglachel}{fall of \Nexagglachel}. 

She also talks to others, who comment on the conflict between \Ishnaruchaefir{} and \Secherdamon. 

This is before \hr{Secherdamon's rise to power}{\ps{\Secherdamon} rise to power}, but \Nzessuacrith{} or someone else\dash probably an \ophidian\dash predicts that the youngest brother is bound for greatness\dash a terrible kind of greatness. 
You can see it in his eyes, the bitterness, the avarice and the determination. 





\subsubsection{\Nzessuacrith{} flying over \Machai}
Maybe have a scene where \Nzessuacrith{} flies over the plains of \Machai{}. 

\lyricslimbonicart{Twilight Omen}{
  My soul had black wings and triumphant I did fly.\\
  I rode the storms and the midnight sky.\\
  I saw the thousand lights from cities underneath me,\\
  as I ventured deeper into the night,\\
  to that sacred place beyond the twilight zone.
}





\subsubsection{The fall of the Heart}
They talk about how most of the immortals are now dead, and of how the Heart was been weakened and is no longer strong now to bring them back to their glory days.
It cannot support such a population in its current state, with both factions pulling and tearing at the Heart. 









\subsection{Silenced}
Catrian, a Sentinel woman living somewhere near \Redce, sees a \bane{}. She is killed by the Cabal. 

During her flight, she submerges into \Nyx. She sees the \hr{Black stars of Nyx}{black stars}. 

Despite being a Sentinel, Catrian has never seen a \dragon. She has, however, seen some of the \draconian{} people, the \rachyth. 







\subsection{Wanderer in Darkness}
Have a scene where the Cabalists in the Pelidor region discuss their plans, and fear who might interfere. The company includes \Teshrial, \Achsah{} and Charcoal, who has been summoned. A few more unnamed Cabalists (perhaps named), a few lesser \banes{}. Maybe they are in communion with \Azraid{}, although he is not present in person. 



 






\subsection{What Slithers Beneath}
\target{Ishnaruchaefir attacks Teshrial's creature}
\target{Ishna fights Teshrial's monster}
\Ishnaruchaefir comes to \Malcur. 
\Teshrial comes to fight him off. 

\Criseis does not know what \Ishnaruchaefir has in mind. 
She knows what he has told her to do (i.e., reconnoitre for any supernatural presences infesting \Malcur), but no more.
She imagines he intends to draw out whatever is in the ground and destroy it.

\Ishnaruchaefir is keeping the \resphain busy.
\Criseis has taken shelter and is cowering there.
Or so it seems. 
In reality, \Criseis digs deep with her aethereal senses.
She reaches down deep below \Malcur, in the planes close to \Nyx, and she detects the \noggyaleth.

This was his plan all along. 
\Criseis is his secret weapon. 
She is \hr{Criseis's senses}{super-sensitive} and can detect things that are supposed to be hidden from everyone.
But few people suspect, because she is just a lowly \scatha. 

\target{Ishnaruchaefir fakes ignorance about Ghobaleth}
\Ishnaruchaefir{} says something like: 
\ta{What a shame that your scheme is now ruined.}
This is to fool \Teshrial{} into thinking \Ishnaruchaefir{} doesn't suspect the full extent of the \noggyal{} plan. 
\hr{Teshrial does not know that Ishnaruchaefir knows}{\Teshrial{} takes the bait}. 

\Teshrial boasts:
\begin{prose}
  \Teshrial:
  \ta{I have killed \dragons before, you know.}
  
  \Ishnaruchaefir: 
  \ta{Have you now?}
  
  \Teshrial:
  \ta{In the battle of $X$, $Y$ years ago.}
  
  \Ishnaruchaefir:
  \ta{Aye, I heard about that. So that was you.
    Hm. 
    So you killed \Zessuruch. 
    One of our youngest. 
    Huh. 
    I was not surprised. 
    \Zessuruch was prone to overconfidence.
    I hear she has grown wiser since her death.
    So you did our people a favour. 
    Well done, \dragon-slayer.}
\end{prose}

\hr{Resphain speak poor Draconic}{Like most \resphain}, \Teshrial \hr{Teshrial speaks poor Draconic}{spoke \Draconic only poorly}. 
So when \Ishnaruchaefir speaks \Draconic, \Teshrial has to make an effort to understand it. 
\Criseis can tell how \Teshrial is listening intently, all the while trying to hide it.
He does not want to show his own shortcomings. 

\Teshrial{} should not be too overconfident. 
He knows he is in over his head and does not really expect to win. 

Eventually, \Ishnaruchaefir is repelled. 
But before he leaves, he sends a challenge to \Teshrial.
He is impressed by \Teshrial's bravery.
He accepts \Teshrial's challenge and is willing to face him again.
He invites \Teshrial to find a time and place for a new duel.

\ta{\Resphain! Carry this message to your fallen comrade, \Teshrial.
  When he returneth to life, tell him this:
  Thy display of bravery hath left me impressed.
  If thou darest face me once more, then send unto me a rematch challenge, and I will humour thy request.}

\Ishnaruchaefir believes he can use \Teshrial as a \trope{XanatosSucker}{Xanatos Sucker}.
With some luck, he can manipulate everyone so that \Teshrial duelling him at the same time that \Secherdamon will cause \Malcur to come crashing down.
The whole \trope{XanatosGambit}{Xanatos Gambit} is not fully crystallized in \Ishnaruchaefir's head yet, but he has a good idea.

Afterwards, the \resphain are confident they have kept \Ishnaruchaefir from learning what he should not learn and fucking up their plans. 
When \Teshrial died, everything had been cleaned up.
They are sure he did not learn anything.
But they were not keeping an eye on \Criseis.
They do not understand quite how keen her senses are.
Under their noses she has snooped around and gained a good overview of what the Cabal are doing. 

After the battle, the \noggyal presence has retracted. 
\Criseis can still feel it, but only faintly, and only because she knows it is there.
She asks her master if he intends to pursue.

\Ishnaruchaefir:
\ta{Nay. Let them hide.}

Rian is scarred and horrified to see the evil sorcerer slay the shining god. 
\Criseis Shrouds him and makes him forget the details, but some measure of religius/existential dread remains with him. 
Remember that in all his later chapters.
Maybe even ask on a forum how to express this. 

When \Criseis{} insists they let Rian live, \Ishnaruchaefir{} goes:
\ta{Sigh. [Smile.] 
  I fear one day your sentimentality will be the death of us, \Criseis.
  But very well.
  Now come.}

At the end of the chapter, have a super-short scene where \Ishnaruchaefir tells \Criseis to describe to him what she has learned.
She begins to tell him something very interesting.
A surprisign and bold plan from the \resphain.
She talks offscreen, and he listens with great interest and a sardonic smile.

Afterwards, the \resphain are confident they have kept \Ishnaruchaefir from learning what he should not learn and fucking up their plans. 
When \Teshrial died, everything had been cleaned up.
They are sure he did not learn anything.
But they were not keeping an eye on \Criseis.
They do not understand quite how keen her senses are.
Under their noses she has snooped around and gained a good overview of what the Cabal are doing. 

Both \Ishnaruchaefir and \Criseis had to be here.
He could not just send \Criseis alone. 
He had tried that.
She has been in \Malcur before and has only managed to pick up hints.
She never dared delve too deep, for fear of detection. 
\Ishnaruchaefir had to be there for two reasons:
\begin{enumerate}
  \item 
    To scare the Cabalists and stir them up, make them do something drastic that could be detected. 
    The \noggyaleth are easier to detect when they are being moved than when they are just quietly burrowing around.
  \item 
    To take all the attention.
    If \Criseis just delves into the deep on her own, she would be detected. 
    If there is a big-ass fight going on in the dead garden, no one will notice \Criseis feeling around. 
\end{enumerate}
%Have a scene earlier on where \Ishnaruchaefir{} fights one of them and only barely prevails. Maybe this is why \Ishnaruchaefir{} knows about them. 





\subsubsection{Rian is traumatized}
Rian remembers the horrible black stars that hung in the heavens. 

\lyricsbs{Robert W. Chambers}{The Repairer of Reputations}{
  \ta{I\ldots{} wept and laughed and trembled with a horror which at times assails me yet. 
  
  This is the thing that troubles me, for I cannot forget Carcosa where black stars hang in the heavens; where the shadows of men's thoughts lengthen in the afternoon, when the twin suns sink into the Lake of Hali; and my mind will bear forever the memory of the Pallid Mask.}
}

He was also scared shitless from seeing \Ishnaruchaefir{} fight. He has lost sanity points (to use the terminology of the RPG \cite{RPG:CallofCthulhu}). 





\subsubsection{\Psyrex{} and \Secherdamon{} see it}
\Psyrex{} and \Secherdamon{} notice the fight and comment it. 
They remark on the fact that it is a blatant breach of the \charade. 















\section{The Runger War}
\target{Runger war}
\subsection{Daggers and \Daemons}
We go to \Malcur (capital city of Pelidor), where \rayuth[\Icor] is assassinated by the Sentinels. 
But it has to be well-concealed, so it's not obvious that it's the Sentinels that did it. 
It must look like it's just the Rungerans. 

At a ducal ball, an assassin sneaks up to the \rayuth during the dance and knifes him. She has the backing of Sentinel agents. It's their magic that lets her get close to the \rayuth. But she is brainwashed and thinks it's all thanks to her own cleverness and stealth and the aid of a few contacts. 

\Icor gets killed by a servant of \Onatol's who has been mind-controlled. 
\Onatol himself is likewise under mind control. 
The spell does not control \Onatol's entire mind, it just makes him slightly crazy and aggressive, so that he fights and gets himself killed when the soldiers come to arrest him. 
Being a mage, \Onatol has a strong mind that is hard to control. 
The servant's mind is easy to control, so the Sentinels have full control over him. 

After \Onatol dies, have a scene where \LocarPsyrex contacts his local agent. 
The agent tells \Psyrex that everything went well. 
The agent then slinks away.
The agent is one of the sorcerers who will later be responsible for raising \Nithdornazsh.

At the end of the book, make sure to reveal that \Onatol was an innocent guy, framed as a part of a circuitous \trope{XanatosRoulette}{Xanatos Roulette} with the purpose of leaking Tantor's diary to Charcoal and making him think he had discovered a vital clue. 

\textbf{Alternately}, get rid of the journal entirely.
The Tantor chapters are just told in flashback, or are relegated to a prologue part. 
In that case, Curwen does not learn of the magic and \Takestsha until Carzain comes to \Forclin and tells him about it.

\Icor's widow, \rinyuth[\Tiroco], suddenly becomes regent of Pelidor. 
We follow her and the general \hs{Sethgal}, who must lead the nation against the Rungeran invasion. 









\subsection{Moro \Cobrel}
\target{Moro feels Ishnaruchaefir and Teshrial}
Moro \Cobrel{} feels the battle between \Ishnaruchaefir{} and \Teshrial{} in the dead garden. 
(She does not feel the \noggyal. It is too deeply Shrouded and concealed.) 

A bit later she goes to the garden to investigate. 
She cannot find anything.
\ta{My spiritual senses are too dull.} 

She touches the ground and concentrates as much as she can. 
For a brief moment she feels the presence of some gigantic creatures, crawling around her or under her. 
But only for a brief moment, and very indistinctly. 

\tho{It might be my imagination.
  It would not be the first time my paranoia and traumata have had me see things. 
  
  If only I were more sane\ldots{}}








\subsection{War is Coming}
\target{Pelidor-Runger war}
The king of Runger plans to invade Pelidor after the assassination of \rayuth[\Icor]. 





\subsubsection{In Dormina}
Flash to Dolmina, the capital city of Runger. 
Insert a scene before \Icor{} dies: 

\hr{Takestsha}{\Takestsha} agent talks to King Morgan and and tells him that everything is in motion; soon \Icor{} will be dead. 
The king is pleased and issues the order to mobilize the armies and march on the Pelidorian border.

Flash forward to after \Icor{} is dead. 

We follow some character in King Morgan's court. 
Perhaps a commander, perhaps a mage, perhaps even a Rissitic or Sentinel. 
In any case, someone close to the king. 
We follow them on their march and see them attack and conquer some towns near the border. 
The garrisons are slaughtered and we hear hints that the king and his mages have yet to unleash their most deadly weapon. 









\subsection{Beyond the Veil}
In \Malcur they hold a funeral for \Icor. \Tiroco{} wants his spirit laid to rest so she can concentrate on moving on. 

We follow \Icor{} from the point of his death. He gets stabbed, then bleeds to death. At first he can still speak, but he quickly grows numb and can only lie there. He can still hear \Tiroco{} and feel her touch, but even that fades. 

Camilienne comes to heal him. He feels the \Sephiroth{} enter his body. He's fought wars and been healed by the \Sephiroth{} before, and it's always felt nice, but this time it's different. Death lifts the Shroud a little bit and lets us see some of the things we don't want to see. And now the dying \Icor{} vaguely sees the \Sephiroth{} as what they truly are: Horrors created by the \banes{} to enslave \humanity{} and enforce the Lie. He fears and despairs. 

After a while, the \Sephiroth{} dissipate. Camilienne proclaims him dead. He finds that he still has a bit of feeling for the living world. He can still feel \Tiroco{} a little bit. Which is bad, because he feels her pain and despair. 

Then he starts hearing the \daemons{} of the void. There are the occasional \daemons{} who have learned \Miithian{} languages and torment the dead to pass time. He is plagued by some of them. 

The \daemons{} go away and he floats around for a while in complete solitude, unable to perceive anything. He fears he is losing his mind, not wanting to spend eternity as a mad ghost. 






\subsubsection{\Icor{} sees vision of soul prison}
In the afterlife, \Icor{} sees a vision of a prison of souls: Thousands, perhaps millions of souls, chained and enslaved, screaming and weeping, bound within a horrible cage. 

This is actually the \Sephiroth{} that he sees. 

He sees souls of the dead being sucked into this Hell, into eternal suffering and torment. Some of them beg for help. A few are able to see him. They reach out to him, imploring him to save them. But he can't. Not only has he no idea of how he could help them; he is also paralyzed by fear. When one of the dead moans its pain and stretches its ghastly fingers toward him, he can only recoil in disgust, and feel guilt at his own cowardice, weakness and cruelty towards the tortured souls. 

Maybe he hears some dead souls talking. They are halfway or wholly mad and babble nonsensically. 






\subsubsection{\Psyrex{} approaches \Icor}
Why does he see it? 
Well, he is dead, so he is drifting towards it, until he is rescued by \Psyrex. 

\Psyrex{} comes to him in the guise of a child-like Cherub. 
 
\ps{\Icor} vision of the Realms is a bit clearer than normal, because he already had cause to suspect the \Sephiroth{} due to the events surrounding his death. 
And maybe \Psyrex{} helps him, showing him visions to scare him. 
(Of course, \ps{\Psyrex}{} interpretation of what \Icor{} sees is far from reality, and \Psyrex{} knows it.)

\Psyrex{} warns him that what he sees is \Itzach, where the souls of sinners go, and that \Icor{} must help him if he is to be considered among the righteous and avoid being thrown into \Itzach{} himself. 
\Icor{} doesn't quite believe \ps{\Psyrex}{} claims of holiness, but he sees no better choice. 







\subsection{Needle}
Charcoal goes to \hs{Needle}. Since Charcoal is going off to war, he gives her instructions to take over the local Cabal in his absence. 

Charcoal is certain that something fishy is going on in \Malcur. He suspects Sentinel activity, mages preparing something of \quo{cataclysmic proportions}. He is also convinced that they know of the Cabalist presence in \Malcur, since he has had agents killed and spells broken. 

He already has men out searching for them and wants Needle to take over the investigation. If they discover anything, they should destroy them, if possible. Show them that the Cabal means business. 

Even striking without really knowing what they're up against might be a good idea, because the Cabal is usually always careful and not happy to take chances, so if they come in with full force, it might convince their enemies that they know more than they do, and thus frighten them into being more reluctant, buying the Cabalists valuable time. Plus, if their enemies are afraid, they get stressed, and when stressed they are more liable to make mistakes that give them away. 

When done giving instructions, he fucks her. 









\subsection{For War and For Glory}
\Tiroco{} and her court have received word that war is coming. 
\hs{Sethgal} is appointed Marshal and leads the armies into battle. 
\Cobrel, Curwen, \Dornaer{} and many nobles are off to war, leaving \Malcur somewhat depopulated. 

\Tiroco{}, not a trained warrior, remains in \Malcur to rule. 
With her are \Vincerre{} (Camilienne being sent with the army), Norden and Bartholomy. 

\target{Malcuric city gates}
The Pelidorian army sets out from \Malcur, marching towards the Rungeran border to fight. 
They march through the \Malcuric{} city gates, which are enormous. 
Wide enough for an entire army to march through, almost in one broad line. 











\subsection{The Mystery of \EreshKal}
Charcoal reads Giles Tantor's journal. 

The whole thing might be written in some cipher code, but it's one Charcoal can easily decode. 

Just before Tantor reaches the lost temple, Charcoal is interrupted by a soldier who comes knocking on his tent flap. (For no specified reason. Curwen just has duties as an officer that require his attention.)







\subsection{Veils that Divide} 
\Icor{} comes back as a ghost. 
He finds his own funeral. 

He tries to communicate with \Tiroco, but fails.





\subsubsection{\Tiroco{} as regent}
\Tiroco{} and \Icor{} were married in an attempt to reconcile two branches of the Pelidor clan, the \Malcurians{} and the \Forcliners. 
Now that \Icor{} is dead, the clan fights over who should be the next \rayuth. 
They can't decide and vote while there's a war going on, so they postpone the decision till after the war. 
This means that \Tiroco{} is stuck as regent until then. 

This is part of \ps{\Psyrex}{} plan. 
\Tiroco{} is weak and easy for him to manipulate, especially when he has \Icor{} as leverage. 







\subsection{A Dark Angel's Gift}
Needle is sitting in her room, setting her hair, putting on makeup and making herself pretty. She is happy that Charcoal is gone. She hates Charcoal, and she loathes being his sex slave. 

Charcoal is cruel and evil and does all he can to make the sex painful to her. In the beginning he took her often, because she was beautiful. She caught on and started deliberately making herself ugly, so he would be turned on by her and thus rape her less often. He may or may not know this. Once in a while he beat her as punishment for being ugly, but he did fuck her less often. Needle accepts the trade: She would choose his whip over his dick any day of the week. 

But now Charcoal is gone and she can be beautiful again. Like any girl, she much prefers being sexy. 

She hasn't become pregnant with him, for which she is very grateful. It is the magic, she knows. All that spellcasting \quo{sucks all the power and life out} of his sperm, leaving it ashen and dead. His cum tastes bitter and dead, too. The cum of a real man, while it may not exactly taste good, at least feels alive, energetic. 





\subsubsection{\Achsah{} seeks out Needle}
\hr{Achsah}{\Achsah} seeks out Needle and gives her new instructions. 

When Needle flinches, \Achsah{} assures her: 
\ta{No need to fear, girl. I am not going to rape you.} 
She pauses, then adds: 
\ta{I have higher standards than that.}

She gives Needle a gift: She teaches her a spell that lets her summon \Achsah{}, to communicate or to send reinforcements. 







\subsection{The Gods of \EreshKal}
Charcoal reads more in Tantor's journal. Tantor-tachi reach the temple of \Rungertemple. Inside they find tablets of arcane scribblings. 

The tables show fragments of the cosmic paths that connect \Miith{} and \Machai; the paths that were \hr{Star-Maps of the Ancient Cosmographers}{mapped out by the founding \dragons millennia ago}. 









\subsection{Captured}
\target{Rian's missing girlfriend}
Introduce Rian. He is a \Malcurian boy of maybe 16-17, an orphan who used to be a thief. But at some point in his teens he helped a carpenter and his family. As thanks, the carpenter took him in as an apprentice, to help him get out of a life of crime and into a decent, \honourable career. 

Now, Rian has a girlfriend, namely Neina the baker's daughter. They are engaged to be married in a year or so, but Rian is a horny youth and can hardly wait. We meet them an evening when he is walking her home for some reason. They are kissing in a corner, and he moves in under her blouse to grope her breasts, but she stops him. He relents and kisses her goodnight, and they part ways. 

She walks to her house, but she is intercepted by a thug working for the Sentinels. He knocks her out and kidnaps her, intending to use her as a sacrifice for the \Nithdornazsh{} ritual. 







\subsection{\Forclin}
Several days have passed since the Pelidorian army set out from \Malcur. 

On their march north, the army passes through \Forclin. They camp there. 





\subsubsection{Curwen meets \Esmerel}
\target{Esmerel in Pelidor}
The Pelidorian army, led by \hs{Sethgal}, is marching out from \Malcur. Soon after, they are joined by Redcor \Matron{} \Esmerel. 

\Esmerel{} is in Pelidor not to search for Scions, but to oversee and monitor the war. 
She is there to provide healing and \quo{spiritual guidance}, which in practice means she is a spy and a lobbyist trying to make people owe her a debt of gratitude. 
She intends to help heal the wounded. 
And maybe intervene and give more direct help in combat if it should be required. 

The Redcor do not trust Morgan Runger. 
He has long been unwilling to heed (read: obey) his Redcor advisors. 
They suspect him of using black magic and being disloyal toward the Iquinian faith. 
They are unhappy about this. 
They feel Morgan might need to be taught a lesson about disrespecting the Redcor. 

She meets Curwen and asks him questions. 
Curwen doesn't trust her, so he's evasive. 

\Esmerel{} might have \Racel{} with her as an assistant. 
This is because \Racel{} is Pelidorian.
It is pure coincidence that Carzain happens to know her. 





\subsubsection{The Pelidorian officers discuss strategy}
The Pelidorian army is now near the Rungeran border. 
The officers discuss strategy. 
Marshal \hs{Sethgal} is there, as are Curwen, \Sanyor{} and \Dornaer. 





\subsubsection{Carzain goes on a sortie}
The Pelidorian army sends out a small force to scout ahead and raid the Rungerans a bit. 

Curwen is going, and decides to take Carzain along. 
\hs{Delph} comes with them as a bodyguard. 

Also, remember to tell us how big the sortie is. How many men?







\section{The Cancerous City}







\subsection{Trinity of Pestilences}
\target{Rian sees the Morbus}
\hs{Rian} is in the slums of \Malcur, searching for \hr{Rian's missing girlfriend}{his missing girlfriend}. 

The slums are filled with death, poverty, destitution, despair, hunger, hopelessness. 
The poor, degenerate people are almost living dead. 
They live as pitiful scavengers. 
Some of them even eat \human{} and \scathaese{} flesh. 
Some of them suffer from the \hr{Morbus}{\Morbus}. 

Compare to Calcutta in \cite{PoppyZBrite:CalcuttaLordofNerves}. 







\subsection{Where Angels Fear to Tread}
\target{Tiroco contacted in dreams}
After the funeral \Tiroco{} is visited by \ps{\Icor} ghost, first in her dreams. 
\Icor{} mourns and complains that he is trapped in limbo and unable to go into the Light. He begs \Tiroco{} to help him by doing something in the crypt.

% (Is there a closed crypt or only an open cemetery?)





\subsubsection{\ps{\Icor} point of view}
\ps{\Icor} mind is hazy from being a ghost. 
He tends to forget \ps{\Psyrex}{} instructions, and even the mere fact that he's a ghost.

He has a hard time getting through to \Tiroco{} due to the Shroud. 
He tries to appear in shadows, in mirrors, behind curtains and stuff\dash places where people's Shrouded perception of reality is weakened, where they are more open to the possibility of strange things happening. 
But if he is behind a curtain and talking to her, and she pulls the curtain aside, then he \quo{vanishes} and she will see right through him\dash the Shroud separates them. 

He is frustrated by what he perceives as \Tiroco{} being stupid. 
He is gradually going mad. 

\lyricsbs{Hate Eternal}{Two Demons}{
  I am diversity. So weary of the angst, 
  so weary of what I have become.\\
  Through this dread I will retain 
  this penance that still haunts me.
  
  I am complexity. Within these walls that hold me, \\
  my past and my present collide.\\
  In this state I must sustain
  this morphing that becomes me.
}

But he also wants to \hs{stop the evil} that is festering in \Malcur. 

It should be noted that \Icor{} sees the regular, Shrouded world and not the true world. 
This is due (in part) to a Shrouding spell cast on him by \LocarPsyrex. 





\subsubsection{\ps{\Tiroco} point of view}
In the beginning \Tiroco{} doesn't believe in it, thinking that it's just her sorrow and wishes that manifest in dreams. But the dreams become more insistent, and she begins to see \Icor{} even when awake (mostly at evening or otherwise in the dark, because the Shroud is more easily penetrable in darkness). 

She begins to believe that the visitations are real. She visits a priest of some sort, probably William Norden, to seek spriritual advice. Norden tells her that ghosts should not be listened to. For the most part, ghosts are not real, but illusions conjured by the \Qliphoth{} to trick people. And if the ghosts are real, they must be mad and twisted, and their pleas should not be heeded.

So \Tiroco{} tries to ignore the ghost, but \ps{\Icor} visitations become more frequent, and he begs and pleads, then blames her for betraying him. At last he guilts her into complying. 

The concrete things he wants her to do are quite complex and take a while to complete. She needs to consort with disagreeable people in the \Malcurian underworld and do furtive, suspicious things. At the same time, \Icor{} tells her that she must not reveal anything to the churches. He says they won't understand and will try to stop her. What she must do is suspicious and dark in nature, consorting with powers normally considered forbidden. But, \Icor{} assures her, it's the only way to free him. If the church's dogmas are to be observed, he will be trapped in limbo forever. Already, he says, he is losing strength, and soon he will be unable to contact her, doomed to float forever in pain and solitude in the cold void. 

So \Tiroco{} gets to work, contacting people and acquiring components necessary for \ps{\Icor} \quo{exorcism} ritual. 

But she also wants to \hs{stop the evil} that is festering in \Malcur. 









\subsection{The Terror of \EreshKal}
The sortie, that includes Carzain, pass by a Pelidorian village. 
It has been destroyed by the Rungerans' evil magic (as a test of their spells). 
They are horrified by the foul sorcery at work. 








\subsection{The \Qliphoth{} Lie Ever in Wait}
\Tiroco{} must move the guards around and divert the attention of the church and authorities. \Icor{} tells her that \quo{they would not understand}, that they, in ignorance, would sabotage their undertaking, dooming him. (Is it only \Icor{}, or are there more ghosts? I think he tells her that she is saving not only him, but many ghosts.) 

%Here's an idea: \Tiroco{} goes on a campaign against \quo{superstition}. She 
In order to do her undergound work, \Tiroco{} officially goes on a campaign against superstition. She approaches the church and convinces them to help her \quo{calm down} the population and suppress nasty rumours of ghosts, black magic and sinister crime. 
Fear is rampant in the country, after all, since the \rayuth has just been murdered and the king of Runger has declared war. So the peasants and citizens need to be reassured, need to forget their fears. 
And this is handy, because if stories of supernatural activity are suppressed and dismissed as superstition, it makes it much easier for \Tiroco{} to do her furtive work. 

And then we hear the story from the perspective of some normal citizen who sees horrible, unnatural things happening, while the church and authorities tell him to forget it, that it's all delusions. This is \hs{Rian}, the ex-thief with the missing girlfriend, Neina. 

This is good for two reasons: 
First, it's a clever move by Tiroco, which helps give her some personality. Until now she's been a very tame character with no real skills, so she needs this. 
Second, it gives me the opportunity to write some horror parts featuring some \quo{common} people who can be genuinely horrified (as opposed to the mages and heroes who are otherwise my main characters, who are less easily frightened). 

What she is doing is a cruel, brutal oppression of her people. Needless to say, she does not respect modern \human{} rights in \emph{any} form. Like the War on Terror in RL. But see it from her point-of-view, where she justifies her crimes. Her love for \Icor{} is weighed against her obligation to her country. 





\subsubsection{She wants to stop the evil}
\Tiroco{} hopes to stop the mystic evil that is festering in \Malcur. 
She hopes she is doing the right thing. 









\subsubsection{Needle notices \Tiroco}
Needle notices that \Tiroco{} is acting strange: 
Furtive, paranoid, jumpy, and keeping lots of secrets. 
Needle suspects that she is up to something. 






\subsubsection{Rian is threatened by a thug}
Rian snoops around in the dark corners of the city, trying to sniff out clues about his girl, Neina. 
He sees something suspicious, 
which might be some of the Sentinel agents, out doing their shady work. 

Then he is jumped by a thief. The thief turns out to be an old \quo{friend}, so for old times' sake he does not kill Rian. He merely slaps him up and tells him to go home and stop snooping around. 









\subsection{The Thirsty Nether}
On his snooping through the city, Rian overhears some thieves (in Sentinel employ) talking about their wicked plan. 
Possibly \Psyrex{}, or at least some of his top-level henchmen. 
He sneaks into a suspicious building. 
Here, he is witness to a terrible ritual of dark magic, involving humanoid sacrifice. 
He sees into \Machai, terrible vistas of \daemonic{} landscapes and stuff. 

He is discovered, and the mages send thugs after him. 
He runs for his life. 
He has been given a warning, so he knows that this time they will surely kill him if they can catch him. 
He nearly gets caught and killed, but Moro \Cobrel{} arrives to pull him out. 

He talks to her. She warns him that he shouldn't be snooping around in places where he might get killed. He tells that that he must, for the sake of his beloved. 

Rian fears the supernatural evil that is taking place in \Malcur. 
He hopes someone will stop it. 





\subsubsection{Needle is approached by a spy}
Needle is approached by a spy of hers. He tells her that he and his fellows have discovered a lair of people who use chaos magic and seem to be up to something nasty. (This spy was at the same place that Rian was. Perhaps he actively sabotaged Rian's infiltration attempt, using him as a distraction, allowing himself to escape unseen. 

Needle begins to prepare a raid. 









\subsection{Dark Crypts of the Mind}





\subsubsection{\Tiroco{} talks to the crazy old woman}
\Tiroco, on a covert trip through the city, passes through the \hs{dead garden}, where is addressed by the \hs{crazy old woman}. The old woman is clingy and mad, desperately clutching \ps{\Tiroco} sleeve and babbling about worms, blood, death and the destruction of the city. 

\Tiroco{} herself is shaken, since she is already weak from the things she's witnessed, so her bodyguard has to shake off the old woman and drag a shocked \Tiroco{} away. But afterwards, \Tiroco{} finds herself seeing worms at every turn.

\lyricsbalsagoth{%
  In the Raven-Haunted Forests of Darkenhold, Where Shadows Reign and the Hues of Sunlight Never Dance
}{
  Can you not see the coils of the worm all about you?\\
  Can you not hear the writhing of the worm beneath you?\\
  Can you not scent the breath of the worm riding the wind?\\
  Can you not touch the skin of the worm in all that surrounds you?\\
  Can you not taste the ichors of the worm upon your tongue?\\
  Do dreams of the worm not haunt your slumber?
}

She tells stories of death, destruction and the end of the world.

\lyricslimbonicart{Legacy of Evil}{
  The night has a legacy of evil.\\
  Nocturnal dreamscapes.\\
  A wilderness of infinite disharmony\\
  in an isolated aspect eternally.\\
  A maze inside your brain\\
  leading into the insane\\
  abyss of fear, illusions and despair.
  
  Something ghastly is passing by.\\
  The full moon in the sky.\\
  See the coming hurricane of terror,\\
  ancient sadness and horror.\\
  A virulent syndrome of misanthropy,\\
  captured by the obscure mystery. 
}





\subsubsection{\Tiroco{} talks to Moro}
%What about Moro \Cobrel? I don't think she has seen enough yet to truly act. But remember to have her in some scenes. 
Remember to have \Tiroco{} talk to Moro \Cobrel{} a few times in the course of the book.

Moro has not yet seen enough to truly act. But remember to have her in some scenes. 

Moro hopes to \hr{stop the evil}{stop the mystic evil} that is festering in \Malcur. 
She is investigating, trying to find out what is going on so she can stop it. 








\subsection{The Bleeding Wood}





\subsubsection{Rian sees a raid}
\target{Rian sees a raid}
Rian sneaks back to a place near the one he was last time. 
He spies on some Sentinel-employed thieves and their evil talk. 

Then some Cabal agents arrive to crash the party. 
Needle is there, surrounded by bodyguards. 
And around her, more-or-less hidden in \Nyx, are \banerats{}. 
Rian is frozen in terror and just crouches still, watching the Cabal and Sentinel mages, warriors and creatures \rayuth it out. 
He sees into \Nyx{} and other realms and is scared even more. 


When he finally runs away, he stumbles into \Nyx. 
He panics, but at last sits down and tries madly to convince himself that what he is seeing is not real, that he is hallucinating. 
And it works. 
His vision readjusts to the Shroud and he sees his old World again, and manages to stumble back into it. 

He goes away shaken to the core. 





\subsection{Rian meets a living building}
Already early on, during the preparations for the great summoning of \Nithd{}, some people have begun degenerating into mutants, or living buildings. 

Rian discovers a building that has living humanoids magically merged into the foundation. Compare this to Bootstrap Bill or Wyvern, who are becoming part of the \shipname{Flying Dutchman} in the movies \cite{Movie:PiratesoftheCaribbean:II} and \cite{Movie:PiratesoftheCaribbean:III}. 












\section{Spectre of the Fray}








\subsection{Imetrians join the war}
Telcastora Ilcas and some other Imetrians approach \Tiroco. 
They offer their military aid. 

Ilcas carries a gun.

The Imetrians have many soldiers spread out over the country, to protect their temples and their pilgrims. 
200 men at least, and several of them veterans. 
They are allowed this due to some agreements made with the Pelidorian \rayuths. 

They offer their aid. 
But in return they want more privileges for their religion in Pelidor. 
Their aim is clear: They want to proselytize. 

\Tiroco{} and her advisors acquiesce, albeit reluctantly. 
\Tiroco{} wants them, but she fears it is against her \quo{crusade}, and therefore it will look suspicious if she just invites them in. 
Fortunately, Moro and others come to the rescue and vouch for the Imetrians. 
The Vaimons argue. 
Eventually, an agreement is made. 
Only minimal concessions are given. 





\subsubsection{Ilcas and Moro}
Before the Imetrians take off, Ilcas approaches \Tiroco. 
He tells her he wants to talk to an \Ishrah{} mage and asks her if there is one she trusts. 

\Tiroco{} hesitates due to her own falling-out with Moro. 
But she collects himself and reminds herself that Moro is good. 
So she directs Ilcas to Moro. 

Ilcas approaches Moro. 
He tells her about the visions Razor has been having. 
He warns her about the evil in the city. 
He and Razor give all the details they can. 
Moro thanks him. 
But she is suspicious: Why do this? 
Ilcas is not required to by their agreement. 

But Ilcas is a hero. 
Since Razor is sure this alien force is \quo{evil}, Ilcas-tachi feel it is their duty as good Imetrians to inform the people who are allies of their people. 

Moro learns some useful stuff. 





\subsubsection{Imetrians ride away and talk}
When it is settled, the Imetrians ride from \Malcur towards \Forclin. 
They ride \relcs. 
You can't ride \nycans{} over long distances (they \hr{Nycan endurance}{lack the endurance}), and \mulgrons{} are too slow. 

Only minimal concessions were given. 

The Imetrians did not drive so hard a bargain. 
They know it will be bad for them if Pelidor falls to Runger. 
Runger has never let many Imetrians in. 
The Imetrians don't like Runger. 
To top it off, the Imetrians have long suspected Runger of having covert dealings with Durcac, albeit without proof. 

Ilcas talks to his mage-priest companion about this. 
They discuss the fact that Runger is possibly in league with Durcac. 

\begin{prose}
  Ilcas: 
  \ta{Should we not have told the \rinyuth about our fears?}
  
  Mage: 
  \ta{No. Our intelligence is our own. 
    Dessali teaches that knowledge is precious.
    We cannot just give it out for free.}
\end{prose}

Ilcas is still not convinced. 
If it were up to him he would have told them. 
But at least he told them about his fears regarding the evil in \Malcur. 
He tells the mage about that. 
The mage is not happy that Ilcas blurted it out. 
Ilcas defends himself saying that it was not Imetric intelligence but his discovery. 
Razor's discovery, actually. 
So the knowledge was Razor's property to dispense as he saw fit, and Razor saw fit to tell the Pelidorian mage. 









\subsection{Carzain in \Forclin}
\target{Carzain returns to Forklin}
\target{Carzain rejoins the army}
\target{Carzain sees the Morbus in Forklin}
Carzain \Shireyo{} has heard news of the war against Runger. 
He now comes to rejoin the army, and his old acquaintance Archibald Curwen. 

Carzain comes to rendezvous with the army in or near \Forclin. 

Vizicar is with him in his head and comments on the look of the city. 
He knew the early \Ortaicans{}, so he recognizes the design a little bit. 
But the \Ortaicans{} evolved a \emph{lot} in the centuries after Vizicar's time. 
Have Carzain comment on how the \Ortaican-built city of \Forclin looks very different from Vaimon-built ones such as \Malcur. 
\hr{Vaimon Middle-East}{Vaimon things look Middle-Eastern}. 
Vizicar is very interested in Forclin, its design and fortifications. 
It was built after his time, in \ps{\Ortaica} glory days.

Carzain reaches \Forclin. 

Carzain strides into the city and demands to be taken to Archibald Curwen, his old acquaintance. 
He \hr{Carzain rejoins the army}{rejoins the \ishrah}. 

In the scene where Curwen meets Carzain, have a scene from Curwen's point-of-view where he envies Carzain his thick, blair hair. 
Curwen has mixed feelings about \hr{Curwen's appearance}{his own gray hair}. 

It has been ten years since they last saw each other. 
Carzain is \hr{Carzain's power}{more powerful than Curwen now}, and Curwen knows it. 

Carzain brings valuable reconnaisance. 
The Pelidorians have scouted, of course, but they have not sent out any skilled mages (they have few of them to spare), so Carzain brings them things they don't know. 

When Curwen hears Carzain's story, he is very interested. 
It fits what he has been reading in Tantor's diary. 

Armed with Carzain's information, Curwen convinces Sethgal that it is best to stay holed up in \Forclin{} and meet the Rungerans there. 
See, Carzain's findings further confirms Curwen's suspicion that the Rungerans want \Forclin{} for mystic reasons. 

Carzain later walks around in the city. He sees the \hr{Morbus}{\Morbus} at work, and how it transforms people into half-dead and later undead abominations. 

He talks to people and hears \hr{Haskelek myth}{myths about the Ghost Tower and the \Haskelek}. 

Remember that the \ishrah{} mages wear plate \armour. 





\subsubsection{Vixal Pelidor}
Vixal Pelidor is the countess of \Forclin. 
She is the highest-ranking member of the \Forcliner{} Pelidors. 
Like the \rayuth, the count or countess is elected. 
She rules for life. 









\subsection{The Ghost Tower}
Sethgal hears that Dendrum has fallen.
Curiously, the Rungeran \ishrah did \emph{not} use their fabled doomsday weapon against Dendrum. 
The real reason is that \Takestsha does not want to draw the Cabal's attention too early.
She wants to drag the Cabal to \Forclin, but not before \Psyrex is ready to raise \Nithdornazsh.
So she has to wait for the opportune moment. 





\subsubsection{Curwen looks at the Ghost Tower}
When they return to \Forclin, Curwen sees the Ghost Tower. 
He thinks about it.
Only later \hr{Charcoal guesses Ghost Tower plan}{will he realize its importance}. 









\subsection{Rungerans besiege \Forclin}
The Rungeran army besieges \Forclin. 
Carzain marvels at the \hr{Glorious armies}{size and splendour of the army}. 

\target{Rungeran super-cannon}
The Rungerans have a big enchanted super-cannon. 
With mage support it can shoot really far and hard. 

Some approximate numbers:
The big super-cannon fires 600 m. 
Pelidor has 500,000 inhabitants. 
10--20\% of those live in the cities, the rest in the country.
(Some sources claim that 1300s England had an urbanization of around 20\%.)
Forclin has 12,000 to 20,000 inhabitants. 

See the sections about:
\begin{itemize}
  \item \hr{TBW weapon ranges}{weapon ranges in the age of the \thirdbanewar}. 
  \item \hr{Forclin}{\Forclin}, for population data. 
\end{itemize}

There are about 20,000 Rungerans. 
Mostly \humans. 
The heavy infantry is made up of \scathae. 
The cavalry consists largely of the Rungeran nobility, which is \human-dominated. 
Conversely, the Pelidorian nobility and cavalry was \scatha-dominated, whereas most of the \humans are found in the light and medium infantry and among the gunners and archers. 

The main body of the cavalry is the 4000 \relcers. 
The hard core is 60 or so \murocs. 
They have no \grulcans, though. 
\Grulcans are mainly a \Galessan thing.

There are also some 200 \mezolisks. 
These are actually imported from Durcac. 
But they are not badly conspicuous. 
Runger has been known to use \mezolisks{} before. 
The Rissitics have just helped them beef up the number of them. 

Perhaps most gruesome of all, the Rungerans have perhaps 300 \nephil ogres. 
But they are hidden, so the Pelidorians can't see them yet. 

In contrast, the Pelidorian army is smaller, only about 12,000 men. 
They have mostly \scathae. 
Including 1500 \relcers.
And 40 \murocs. 
And a fearsome elite cavalry riding \grulcans.
600 of them. 
Sethgal wonders how the \grulcans will fare against the Rungeran \mezolisks. 

The Pelidorians also have 9,000 infantry, gunners and archers. 







\subsection{The Cannonade}
The Rungerans bombard \Forclin.
Carzain and others sally out to stop them. 

Ramiel's dark, \draconic{} blood awakens, and the death-ravening black fury fills him. 
He fights with reckless abandon and a savage laugh on his lips, with the ferocity of a \dragon{}. 

When I am to write a major battle-scene, I should remember to re-read \bandsong{Bal-Sagoth}{To Dethrone the Witch-Queen of Mytos K'Unn (The Legend of the Battle of Blackhelm Vale)}. 





\subsubsection{Prayers}
Both sides have priests leading them in \hs{prayers against disease}. 

And the Iquinians pray for deliverance from \hr{Isphet}{\Isphet} and his evil.

Remember the \hs{Iquinian clerical hierarchy}. 





\subsubsection{Siege cannons}
The Rungerans use cannons in the siege. 





\subsubsection{Knights}
Both sides have Iquinian \hs{knights}, who \hr{Knights have superpowers}{have superpowers}.





\subsubsection{Charcoal guesses Ghost Tower plan}
\target{Charcoal guesses Ghost Tower plan}
Charcoal guesses that the Sentinels want the Ghost Tower.





\subsubsection{Imetrians arrive}
A small army of about 500 Imetrians fight at \Forclin. 
Not many. 
But these are not conscripts.
They are all skilled fighters.
And they have great \saurians with them. 
And a mage.
And a seasoned hero.
There are about 300 cavalry, 100 \nycaneers and 100 \nycans. 
All in all, they are a formidable fighting force, much more so than their numbers might suggest. 

Describe how skilled, disciplined and fearless they are, with their Imetric gods giving them courage and strength. 

They are supernaturally strong and effective.
As one of their priests says: 

\begin{prose}
  Priest: 
  \ta{We fight an important battle, brethren.
    There are few of us, but the power of our gods will run through us all the stronger for it.}
\end{prose}

And it is true. 
You can see their heathen gods are with them. 

We don't see this from the Imetrians' POV. 
After all, the Imetrians are just a minor player in the story so far. 
They should not get a big, dramatic role until I have had the time to develop them into something cooler, more badass, more background-rich, more well-rounded. 

We see it from Sethgal's POV. 
He is one of the only people present who speaks some Imetric. 
He overhears someone (Ilcas or an Imetric priest) giving the soldiers a peptalk. 

\begin{prose}
  Imetric priest: 
  \ta{If we die on this day, we shall live again!}
  
  Sethgal: 
  \tho{I know the Imetrians believe that they reincarnate when they die.
    I wonder if that is true.
    It is certainly not true for us Iquinians. 
    We die and go into the Light.
    But for them\ldots{} who can say?}
\end{prose}

The Imetrians scare Sethgal. 
They fight with great zeal, \hr{Imetrian coldness}{but their fervour is\ldots{} cold}. 
Calculating. 
Reptilian. 
He is quite disturbed.

(Maybe make a footnote about how the warm-blooded \scathae{} do not see themselves as \quo{reptiles}. At the very least, mention this in the glossary.)

Compare them to Haldir's Elves at Helm's Deep in the movie \cite{Movie:LordoftheRings:II} (not present in the book \cite{JRRTolkien:LordoftheRings:II}). 

Remember to read about \hs{Telcastora Ilcas} and the \nycans{} before writing this. 





\subsubsection{Sethgal and Ilcas}
After the battle, have a scene where Sethgal and Ilcas talk about leadership and strategy and experience. 

Ilcas is a good, inspiring leader because of his faith. 
He believes in his cause with great fervour, and this inspires people to follow him.
But he is no great strategist or leader. 
So he can make people follow him, but would not know where to lead them. 

Sethgal is the other way around. 
He is highly skilled as a general, but he does not have quite the same charisma, the same passion. 
His men admire and respect him, but they do not \emph{love} him. 
This is one of his weaknesses. 

Perhaps, Sethgal reflects, this is one of the reasons why he was not elected \rayuth{}. 

Also, remember to display \hs{Ilcas' racism}. 
He accidentally mentions that he and his men are of superior breeding than the Pelidorians. 
Then, when Sethgal is offended, he tries to back down and mitigate what he said. 

Sethgal then looks out over the battlefield.
He is grimly determined to rout these Rungerans. 

\lyricsbs{Bal-Sagoth}{
  And Lo, When the Imperium Marches Against Gul-Kothoth, Then Dark Sorceries Shall Enshroud the Citadel of the Obsidian Crown
}{
  A seething forest of blackened blades.\\
  A churning sea of ebon war-chariots.\\
  A searing storm of flaming shafts.\\
  All this havoc and more shall I unleash against my foe...\\
  Into battle! The Legion shall march... the fall of Gul-Kothoth is nigh!
}

See also \cite{RobertEHoward:KingsoftheNight}. 





\subsubsection{Ulphon Nestor dies}
\target{Ulphon Nestor dies}
Telcastora Ilcas's Imetrians have one mage with them: 
\hs{Ulphon Nestor}. 
He dies quite early on in the battle, killed by some brave Rungerans. 

Later, the Imetrians \hr{Ilcas-tachi attack the Rungeran Ishrah}{mount an attack against the Rungeran \ishrah}.
Here they need a mage for artillery support, so they ask Carzain to join them. 









\subsection{Morgan Runger sees what he has done}
We follow Morgan Runger. He has made an alliance with the Rissitics. Originally, he was thrilled at the prospect of rising to power as an ally of Durcac, but of late he has come to doubt, for two reasons. One, he fears that \Nechsain{} will subjugate him as a vassal rather than a true ally. Two, and perhaps more importantly, he beginst to ponder the ethical consequences of his actions. 

He oversees \Takestsha{} and her fellow sorcerers as they invoke the terrible spells gleaned from the tablets recovered from \Rungertemple, and he witnesses firsthand the horrible destruction they cause. 

He imagines that the \sephiroth{} are weeping, mourning his fall from grace.

\lyricslimbonicart{Beneath the Burial Surface}{
  The sky is darkening, soon the night befall.\\
  Righteously angels are weeping for my soul.\\
  All childhood dreams are soon to be lost,\\
  all innocence to be shattered.
  
  I am the fallen from grace.
  
  My face is a river.\\
  See my eyes as they drown in black.\\
  My sacred doom and nemesis\\
  beneath the burial surface\\
  To the final act of the immortal sin\\
  I am lead by funeral winds.
}

He sees the terrible wickedness of the \EreshKali{} magic.

\lyricslimbonicart{Darkzone Martyrium}{
  Black energies in the twilight space\\
  come shivering through the shallow haze.\\
  Into darkness so impure divine.\\
  A bloodshed emotion to evil wine.
}

Cannibalize the scenes with \Takestsha from \quo{The \Caliph Inviolate}! 
(Look in the \quo{Carzain Prequel} folder.)

\begin{prose}
  Morgan Runger had approached her from behind and was few steps away now. 
  She had smelled and heard him approach from a mile off, of course. 
  \ta{They were Pelidorian scouts,} she told him, not turning. 
  \ta{We have chased them off. I say we let them run.}
  
  \ta{Mhm. Someone will deal with it,} said the king. 
  Untroubled. 
  Morgan had seen the fire, as had everyone, but evidently it did not scare him. 
  He was confident that his bolstered \ishrah{} could counter any sorcerous attack. 
  Just as she wanted him to be. 
  
  He came up close behind and wrapped his arms around her. 
  Began to grope her body. 
  \tho{%
    Heh. Delegating responsibility so you can have your pleasure? How very kingly.}
  Morgan had grown quite shameless about their affair, despite the fact that, as a nominal \Iquinian, he was theoretically obliged to be faithful to his wife. 
  But he was a king and could do whatever he wanted. 
  Do \emph{whomever} he wanted. 
  \tho{%
    And, as any man might, he wants to remind everyone how beautiful a woman he is fucking.}
  
  % She doesn't mind, of course. 
  % As a \dragon{} she has no sexual shame. 
  \Takestsha{} reflected that perhaps she ought to fake some shame and modesty to make her guise as a \human{} woman more believable. 
  But then again, she was playing the role of the mysterious and erotic sorceress, and part of her allure was her rejection of conventional morals. 
  
  \tho{%
    Oh, yes. 
    I suppose I had better have sex with Morgan. 
    Have to keep my pet king compliant. 
    
    Tee-hee.
    The discovery of the Scion has made me in a good mood. 
    Who knows? 
    I might even enjoy it tonight.}
\end{prose}






\subsubsection{The evil magic}
\target{Rungeran temple magic}
The \Rungertemple{} magic might be defiler magic (as in \emph{Dungeons and Dragons: Dark Sun}), sucking life out of the world and leaving it gray, dusty and dead. Alternately, it might be bestial, destructive and chaotic magic, apalling in its sheer hate, ferocity and inhumanity. 

The magic involves the conjuration and binding of terrible \daemons{} from \Chaos. These should be as horrible, inhuman and Cthulhu-like as possible. The \daemons{} are the source of their power; they're the ones wreaking the destruction. 

It is hinted that the \daemons{} are not really bound; they are just playing along for their own unfathomable reasons, and may decide to turn on the mages any time. Have at least one scene where the sorcery suddenly backlashes on one of the mages and he dies a horrible death, his flesh boiled and burnt and his soul consumed by the \daemon{} he unleashed. 

Make it clear that the foolish \humans{} are playing with powers far beyond their understanding. 

It is very hard, taxing and traumatic work for the mages. The sorcerers, being ill-informed and ill-educated in the use of this great power, are twisted by it. Their bodies become warped and misshapen, and they go more and more mad. Compare to Hannan Mosag and his K'risnan in \cite{StevenEriksonIanCameronEsslemont:MalazanBookoftheFallen}.

\lyricsbs{Monolith Deathcult}{%
    1917 - Spring Offensive (Dulce Et Decorum Est)
}{
  Creeping like a snake from a can, \\
  the slithering stench of yellow death.\\
  Chemical flame of decay\\
  burning skin and intestine.\\
  Regurgitating the bloody guts.\\
  Spewing last life from a wretched soul.
}






\subsubsection{Morgan angsts}
Morgan grieves and is plagued with guilt and doubt? Should he continue along this path? But what else can he do? 

\target{Morgan has sex with Takestsha}
He has sex with \Takestsha{} (he has few compuctions against cheating on his wife, who is at home in Runger). \Takestsha{} comforts him and assures him that he is doing the right thing. 

Morgan has converted to the Rissitic faith, but he keeps it secret from people for the time being. He still catches himself praying to the \Sephiroth{} out of habit. 





\subsubsection{Morgan's history}
Morgan has a history. His father, Uther Runger, had something to do with High King \LastHighKing{} and the fall of \hr{Great Velcad}{\GreatVelcad}. 

Later (after \TwilightAngelRememberEmph), Morgan will turn against the Rissitics and attempt to right what he has wronged. 















\section{The Immortals}







\subsection{The immortals and their background}
Remember, in chapters featuring the immortals, to have references to their history and relationships. 
And flashbacks, and dream sequences.

The above chapter with \Achsah{} is a good example of where I might do this. 

Does \Achsah{} or \Teshrial{} share a back story with \Ishnaruchaefir? 
That might be cool. 

Compare to the many flashbacks and cryptic back story references in \cite{StevenEriksonIanCameronEsslemont:MalazanBookoftheFallen}. 









\subsection{\Achsah{} and \Teshrial}
Remember to read about \hr{Achsah}{\Achsah} and \hr{Teshrial}{\Teshrial} before writing this! 

Have one or more scenes with \Achsah{} and \Teshrial. She is subordinate to him in rank, so she humbly knees before her master. But inside \hr{Achsah hates Teshrial}{she despises him}. 

Maybe she tells him about new developments with \Tiroco, which she's heard from Needle. 

\target{Brains will triumph over brawn}
We see \Teshrial{} training with various weapons. 
He is an \hr{Weapon master Teshrial}{expert weapons master} with both melee weapons, firearms and magical devices. 
Therefore he wants to test his mettle against the infamous \Ishnaruchaefir. 
He wants to prove that technique triumphs over brute force ({unlike} last time). 
Some \resphain{} doubt this. 
He points out that he has fought \dragons{} before (perhaps even killed some), and he has fought the \Baelzerach{} and their \daemons. 
One of them thinks back to the time after \ps{\Teshrial} first battle with \Ishnaruchaefir. 
\Teshrial{} had been so badly wounded that he had had to drain 50 \humans{} dry of blood and consume all their lifeforce to heal himself and grow his wings and legs back.  

He remembers the humiliation of that day as if it were yesterday. 
It fires his pride. 

He thinks about {\ps{\Ishnaruchaefir} rematch offer}. 

\Teshrial{} admits to himself (if not to her) that he was overconfident and careless when fighting \Ishnaruchaefir{}. 
He promises himself to be better prepared next time. 
When he thinks back to the fight, he can see how reckless and stupid it was. 
What chance did he think he stood against the Destroyer? 
But he rationalizes it away, telling himself that he didn't go in to win. 
Rather (allegedly), the fight was a sort of reconnaisance. 
He did it to test \Ishnaruchaefir, see how he fought, so that he could better figure out how to counter it. 
It was just warm-up to make him better prepared for the real fight that is to come. 

\Achsah{} is somewhat disturbed to hear him talk of \quo{next time}. 
She would have thought that one battle against the terrible \Ishnaruchaefir{} must be enough. 
She is worried by the obsession her lord has developed. 

When \Teshrial{} mocks \ps{\Achsah} birth, she thinks to herself: 
\tho{%
  I was born like \Thanatzil, our founder. 
  In blood. 
  The same blood of life which, in other parts of our lives, we hail as something great and sacred.}
  
If \Thanatzil{} is mentioned, then remember to add him to the Dramatis Personae. 

Remember to read the sections about \hr{Teshrial}{\Teshrial} and \hr{Achsah}{\Achsah} before writing the chapter!








\subsection{\ps{\Teshrial} farm}
\target{Teshrial's farm}
Remember to read about \hr{Teshrial}{\Teshrial} before writing this! 

\Teshrial{} breeds \humans. 
Not just slave \humans, no. 
Top quality \humans{}, for the \hs{Communion}. 
He breeds them for beauty, health, obedience and good taste. 

\tho{%
  Most \resphain{} don't understand \humans.
  To them, a \human{} is just an automaton that serves their needs.
  They do not understand all the care and hard work that does into breeding them. 
  
  \Humans{} must be bred with love.}

And \Teshrial{} loves his \humans. 

He has a number of farms (breeding villaes) where he breeds \humans. 
He flies to one such farm to inspect it. 

Mention that \hr{Resphain enjoy flying}{\resphain{} enjoy flying}. 

\begin{prose}
\Teshrial: 
\tho{Maybe mortals would appreciate walking more if only nobles had legs.

  Brr. 
  What a grotesque and repulsive idea. 
  Legless \humans{} crawling around.
  Poor things.}
\end{prose}

He arrives at the farm. 

He inspects the newborn and gives them his blessing. 
He touches them with the tips of his wings to bless them, caressing them with his soft feathers. 

He looks at a young girl of around 15. 
Touches her. 
She giggles and squirms when he tickles her with his feathers. 
But she is brave enough to meet his gaze. 

\ta{%
  Look at you. 
  How fine you are. 
  How perfect. 
  As beautiful as a \resvil.}

Maybe the girl is Evith. 
Almost ripe. 

\ta{%
  Very soon, my beautiful one. Very soon.}

But he thinks to himself: 
\tho{%
  Sigh.
  If only it \emph{were} a \resphan{} child.}
There are very few \resphain{} being born these days. 
\Teshrial{} has barely ever seen a \resphan{} child. 
He has only heard stories of the golden age (\hr{Shroud harms fertility}{before the Shroud}) when his people were numerous and fertile. 
It has something to do with the Shroud, he knows. 
He does not understand the details of \dweomer{} theory. 
It has not been his specialty. 

\tho{%
  What I would not give \hr{Teshrial wants children}{to have a son or daughter of my own}. 
  Not a \bezed. 
  Not a surrogate.
  A pureblood.}






\subsubsection{Punishing lovers}
There are two young \humans{} who have broken the rules. 
They have become lovers and had sex, and the girl has gotten pregnant. 
This is a crime, because they should only breed as they are ordered. 
It is a sin, for a \ps{\human} flesh and sex is a sacred thing (\resphan{} property) not to be thus profaned. 
They must be punished. 

\Teshrial: 
\ta{%
  Know this: 
  You have caused me great sorrow. 
  Your womb and your seed are sacred treasures, but you have violated and defiled yourselves by your act of fornication.
  I would that this sin could be forgiven and washed awau, for I grieve to see my beloved children punished. 
  But some sins cannot be washed away. 
  You must die, and your flesh shall feed the beasts, for you have been tainted and can no longer achieve the blessing of the Communion.}









\subsection{\Ishnaruchaefir wreaks havoc}
Clarify that \Ishnaruchaefir is a genuine and current threat to the \resphain, not just a past threat.
He keeps fucking up their schemes, and he is endangering a long-term scheme that is vital to \CiriathSepher if they want to rise to supremacy and realize their worth.
He preys on the \resphain and kills them and their servants. 
He has been passive for a long time, but now he is becoming a serious menace, and the Cabal fear him.

And he is casting spells (storm beacons and the like) that prevent the Cabalists from doing their thing in \Malcur.
They have a constructive goal in \Malcur. 
Clarify that.
They think \Ishnaruchaefir is the biggest threat against that goal, but it turns out Secherdamon is a worse threat. 

\Ishnaruchaefir is not so big a threat that all \resphain in the world are after him, though.
So far he is just endandering the \hr{Cabal plan for Malcur}{\Malcur venture}, which only a small part of \CiriathSepher really care about. 
(Although the ones that do care have very high expectations of this gambit and hope it will determine the future fate of \CiriathSepher, if not all \resphain. 
 The ones not part of the gambit are more \skeptical. 
 \Azraid has hopes for the venture, but remains \skeptical and aloof.)

(Make a section about the \CiriathSepher \Malcur Venture.)

\target{Teshrial is their best bet}
Anyway, there are several \resphain who want to do \Ishnaruchaefir in, but he is notoriously elusive.
But he has promised \Teshrial to give him a rematch, and told \Teshrial to contact him when he is ready.
In some very clear terms.
So the \resphain know that if they want to do \Ishnaruchaefir in, \Teshrial is their best bet.





\subsubsection{\Ishnaruchaefir attacks viewing station}
\target{Ishnaruchaefir attacks viewing station}
The \resphain working in \Malcur \hr{Cabal stations near Malcur}{have some viewing stations and stuff} set up in a Realm adjacent to \Azmith. 
\Ishnaruchaefir sends a horde of his \daemons to overrun one such station. 
It demonstrates to the \resphain how dangerous he is, and why he must be stopped.









\subsection{\ps{\Teshrial} banquet}
\Teshrial{} holds a small banquet and invites some \ketheran{} acquaintances. 
Among them is \Firaxel, a \resvil{} whom he has the hots for and wants to score. 
She is high status, capable and beautiful. 
A real catch. 
\Teshrial{} wants to lose his \quo{\hr{Teshrial's virginity}{pureblood virginity}}. 

Read about \hr{Teshrial}{\Teshrial} and \hr{Firaxel}{\Firaxel} before writing this chapter!

For the party, \Teshrial{} has dyed his white hair with a single stripe of magenta, fitting the pink of his eyes. 
He also wears a magenta sash. 

Another guest is \hr{Dezruth}{\Dezruth} of \Mystraacht. 
(Remember to describe \hr{Dezruth's appearance}{his appearance}.) 
He comes bringing two naked girl-slaves with him (\hr{Resphan slave livery}{\Mystraacht{} slaves often go naked}). 
This provokes some of the \CiriathSepher{} present. 
\Dezruth{} himself thinks he is being very modest and respectful of his host's culture by bringing only two slaves, where he would have liked to bring more. 
The girls kneel by his side most of the time, only occasionally doing something active to service him. 
\Dezruth{} mostly relies on the local slaves when he needs to be serviced. 
His own slaves are mostly there to show his status. 

He woos \Firaxel. 
She teases him. 
(\Resviel{} are expert teasers.)
But she is a bit impressed and aroused when she hears the story of how he stood his ground alone against the dreaded \Ishnaruchaefir{} and had the courage to challenge him. 
And even to fight to the death! 
(\Ishnaruchaefir{} has a reputation for destroying souls, so the fact that \Teshrial{} has survived with his soul intact is considered an accomplishment.
The \hs{Shroud prevents soul destruction}, but \Ishnaruchaefir{} is strong enough that he \emph{could} have eaten \ps{\Teshrial} soul if he \emph{really} wanted to.
\Menessiaraid{} later warns him about this.)

\Teshrial{} sees her breasts heavy ever so slightly in arousal. 
He knows he has won a small victory and is one step closer to his goal. 

He tells about his ambition to defeat \hr{Ishnaruchaefir}{\Ishnaruchaefir}. 
He hopes that it will win him status in the \ps{\resviel}{} eyes, especially \Firaxel. 
He thinks about \hr{Teshrial's family}{his father and mother and their great deeds}. 

\tho{I deserve \resvil{} pussy. 
  I have earned it. But I will earn it even more.}

He and \Firaxel{} softly caress each other with the tips of their feathers. 

\Teshrial{} mentally comments to himself how un-beautiful \Achsah{} is compared to \Firaxel. 
(Contrast this to how beautiful \Achsah{} appears to Needle.) 

At the end of the party, she kisses him on the lips. 
That is another victory for \Teshrial, but no guarantee yet. 
She gives him one of her feathers as a gift. 
This is arrogant of her, and he worries that he has lost value by supplicating and treasuring it. 





\subsubsection{\ps{\Teshrial} quest}
\target{Teshrial's quest}
\target{Menessiaraid's advice}
\Teshrial{} tells the others about the quest he has undertaken to bring down the mighty \Ishnaruchaefir. 
He knows it is risky and dangerous, but he wants to take risks and be bold. 
The \resviel{} love a hero. 
And he notices how \ps{\Firaxel} eyes light up when he tells about it, subtly implying his own daring. 

At first, he simply talks big about vanquishing \Ishnaruchaefir, but has yet to formulate any real plan for how to do so. 

At the party is \hr{Menessiaraid}{\Menessiaraid}, a friend of \Teshrial{} who wants him the best of success in his wooing of \Firaxel. 
He recommends that \Teshrial{} study the myths. 

There are people who believe that \WanderersInDarknessEmph holds the key to slaying the mighty \dragonlord. 
But not much research has been done. 
It is not really worth it. 
The have \resphain{} very rarely been given the opportunity to confront the elusive \Ishnaruchaefir{} on anything but his own terms\dash if at all. 

But \Teshrial{} now has such a chance. 
When they fought, \Ishnaruchaefir{} {offered to accept \ps{\Teshrial} challenge to a rematch} should he issue it. 

Given that he has this extraordinary chance, \ps{\Teshrial} friends encourage him to do research, prepare himself and lay traps that fully utilize \ps{\Teshrial} strengths and \ps{\Ishnaruchaefir} weaknesses. 
\quo{Solve his \hs{Aenigma}}, so to speak. 

A few have believed to have found the answer and then gone off to fight \Ishnaruchaefir{}. 
None returned to tell the tale. 
\Menessiaraid{} knows of one such would-be hero who challenged \Ishnaruchaefir{} and fell: 
\hr{Lothagiel}{\Lothagiel}. 
\Menessiaraid{} recommends that \Teshrial{} seek out \hr{Nemuragh}{\Nemuragh}, who was a close friend or family member of \Lothagiel. 
\Teshrial{} later \hr{Teshrial seeks out widow}{does}. 





\subsubsection{\ps{\Teshrial} motivation}
Make it clear that \ps{\Teshrial} motivation is not purely selfish. 
He wants to win glory for himself, but he also wants to do heroic things because he genuinely thinks they are right. 
He wants to destroy \Ishnaruchaefir{} not just out of a personal vendetta, but because he is an evil menace\dash so evil and insane that even his own people, even his brother and daughter, revile him as anathema. 
So evil that he was even \hr{Ishnaruchaefir and the Sentinels}{cast out of the Sentinels} (or so \Teshrial{} believes). 
And \Teshrial{} wants to avenge the many victims (warriors and civilians, mortal and immortal alike) whom the wicked Exile has slain over the millennia. 

He wants to have children, not just for his own vanity and legacy's sake, but because his race needs children. 
He genuinely thinks he has good genes, and it is his duty to carry them on. 





\subsubsection{Called back by \Achsah}
While out on his farm, \Teshrial{} receives word from \Achsah{} (who is a \hs{High Telepath}) that \Ishnaruchaefir{} has appeared in \Malcur. 
He hurries back home. 









\subsection{\Teshrial goes to \Urizeth}
\subsubsection{\Teshrial{} reads portents of \Ishnaruchaefir}
Have lots of astrology involved when \Teshrial{} researches the Aenigma. 

Fearing \Ishnaruchaefir, \Teshrial{} reads the stars in an attempt to understand the weave of the \matrices. 
He finds \ps{\Ishnaruchaefir} star\dash the \hs{Exile}\dash shining bright, which is a sign that \Ishnaruchaefir{} is once more active as a \vertex, after having been mostly dormant for hundreds of years (unlike \Secherdamon, who is always plotting). 

\target{Teshrial does not know that Ishnaruchaefir knows}
\Teshrial{} is confident that \Ishnaruchaefir{} does not suspect the true extent of the \noggyal{} plot. 
After all, \Ishnaruchaefir{} said: 
\ta{%
  \hr{Ishnaruchaefir fakes ignorance about Ghobaleth}{%
    What a shame that your scheme is now ruined.}}
(Actually \Ishnaruchaefir{} lied. 
 He knew full well that there were more \noggyaleth.)

\Achsah{} also fears the return of the mystic immortal. 
She never knows if \Ishnaruchaefir{} is a direct enemy, a third-party hindrance, or even a temporary ally. 
She doesn't understand him. 

\target{Achsah and Teshrial worry about Secherdamon and Ishnaruchaefir}
\Achsah{} and \Teshrial{} also worry about \Secherdamon.
They know they both have business in \Malcur, so they suspect a connection. 
Even though the two are known to be archfoes, this coincidence is conspicuous. 
But they study the constellations, and the Exile is nowhere near any of \ps{\Secherdamon} \matrices, so they cannot be in league with one another. 
The two are presumably just both drawn toward \Malcur because it is a \nexus. 





\subsubsection{\Nemuragh}
\target{Teshrial seeks out widow}
\Menessiaraid{} \hr{Menessiaraid's advice}{once recommended} that \Teshrial{} seek out \hr{Nemuragh}{\Nemuragh}, who was a close friend and gay lover of \Lothagiel, a would-be hero who challenged \Ishnaruchaefir{} and fell. 

\target{Teshrial gets notes}
\Teshrial{} seeks out \hr{Nemuragh}{\Nemuragh}. 
\Nemuragh{} tells him of his friend's quest and his findings. 
After some negotiation, he lets \Teshrial{} have \ps{\Lothagiel} notes. 
This gives him a lot of insight into the Aenigma. 

Read about \hr{Lothagiel}{\Lothagiel}!

\ps{\Lothagiel} notes are not complete. 
There are fragments missing, presumably destroyed. 
Besides, they were written for himself to understand, not for outsiders. 
They were not compiled into an easily readable form. 
So \Teshrial{} has to piece it together and try to make out what \Lothagiel{} was thinking. 

\Nemuragh{} tells him that \Lothagiel{} was studying \WanderersInDarknessEmph. 
\Teshrial{} later brings this up with \Menessiaraid. 

\begin{prose}
  \Teshrial{} wants to scoff. 
  \tho{Poetry? 
    You want me to prepare for battle by reading poems?}
  It seems ridiculous to him. 
  
  But \Menessiaraid{} likes the idea.
  
  \Teshrial: 
  \ta{Please explain.} 
  
  \ta{\WanderersInDarknessEmph is powerful.
    It is widely believed that it holds the answers to many important questions within its riddles and symbols.}
  
  \WanderersInDarkness is an epic poem, Teshrial knows.
  It deals with Ishnaruchaefir and his brothers, among other things. 
  Written after the Incursion by \Melcryth, some mad \dragon.
  Or so it was believed.
  The poem is in Draconic, but the name \quo{\Melcryth} appears to be a pseudonym, and no one knows who hid behind it. 
  
  \Teshrial{} is inclined to disbelieve. 
  But he knows \Menessiaraid{} is not stupid.
  \Menessiaraid{}'s area of study is religion and mythology.
  He has helped shape the dogma and world-view of more than one puppet religion in the Shrouded Realms.
  He knows mythology is serious business.
  So he might have a point.
  
  \Menessiaraid:
  \ta{There are many who study it.
    Not just mystics and philosophers, but also magic scholars.
    And Matrix theorists. 
    The poem is difficult to understand, but its study has yielded some remarkable insights over the centuries.
    You should not discard it.
    Nor other pieces of myth and poetry like it.}
  
  \Teshrial: 
  \ta{Hm. 
    I am \skeptical, but I know you know what you are talking about.
    Very well.
    I will do as you say and look into the myths.}
\end{prose}

After his initial defeat, \Teshrial{} has developed an obsession with \Ishnaruchaefir{} (just as \Ishnaruchaefir{} wanted him to) and now studies the records about him. 
He reads them on \hs{graph-glass}. 

He reads \hr{Urizeth}{\ps{\Urizeth}} annotated \emph{\hr{Wanderers in Darkness}{\WanderersInDarkness}}. 

There are many different versions and fragments of the epic. 
Some of them are considered apocryphal by many. 
\Teshrial{} reads them anyway. 
They are artistically successful and evocative, so \Teshrial, fancying himself an artist, likes them and believes there must be some truth in them. 






\subsubsection{\Urizeth}
\Teshrial{} does not figure out all this on his own. 
He seeks out \Urizeth{} and talks to her in person. 
He asks for her help in interpreting the Achilles Heel and the Nadir. 
She provides much help. 
He shows her \ps{\Lothagiel} notes. 
She did not previously know of it, so she is very fascinated and interested. 
\Teshrial{} learns much. 

\Urizeth{} asks to keep copies of \ps{\Lothagiel} notes and study them some more. 
She tells \Teshrial{} to come back later, and she will discuss her findings with him. 





\subsubsection{Nadir}
From \ps{\Lothagiel} notes, \Teshrial{} learns a very interesting observation: 
Apparently \hr{Ishnaruchaefir's Nadir}{\Ishnaruchaefir{} has a period of Nadir} at more-or-less regular intervals. 

From regular \matrix{} theory it is unsurprising that this would happen. 
\Ishnaruchaefir, after all, is a big-ass \vertex, and the wielder of a powerful \hs{weaving artifact} to boot. 
But apparently no one had mapped the Nadir cycle before. 
\Lothagiel{} managed this. 
He researched some passages of \WanderersInDarknessEmph (which, as it turns out, were authentic, not planted) and picked up some hints. 
He studied \matrix{} theory, interpreted the symbols and kennings and connected the dots. 

According to \WanderersInDarknessEmph, the Nadir occurs \quo{when the \hs{Exile} is engulfed by the briny waters}. 
This is an astrological sign. 
\WanderersInDarknessEmph also reveals that he is at his weakest in the middle of the period, and there are further astrological signs to mark when this happens. 
\Teshrial{} and \Lothagiel{} have no means of verifying this last part, so they have to take the poet's word for it. 

Do not mention any exact numbers regarding the period. 
Just \quo{a number of years}. 
I don't want to paint myself into a corner.

Then he formulated a hypothesis and went about gathering empirical data from \quo{sightings} of \Ishnaruchaefir. 
These were few and far in between, of course, but still, with seven thousand years of history to draw from, he was able to dig up enough sightings for a pattern to emerge. 
And \Lothagiel{} was happy, for the observations supported his hypothesis: 
There were few to no sightings in the periods where \Ishnaruchaefir{} was supposed to be in Nadir. 
And when once in a while he was forced into combat, he seemed uncharacteristically reluctant, weak and prone to fleeing\dash and he did not wield his glaive. 

\ta{And without \Rystessakhin. 
  It\dash or should I say \emph{she}\dash is otherwise perhaps his most powerful weapon.}









\subsection[Criseis in Malcur]{\Criseis in \Malcur}
\Criseis goes to \Malcur to see how things are going. 
She alights on the roof of a high tower and gazes out over the city. 
She can clearly feel that something metaphysical is afoot. 





\subsubsection{Talks to a thug}
\Criseis{} espies a thug who smells like one aligned with the Sentinels. 
She catches him and forces him to tell her what he knows, using compulsion magic. 
Then she Shrouds the dude to make him forget. 

No, she kills him. 
She feels a bit bad about killing him, but reminds herself that he is an evil thug who preys upon the weak. 
She stabs him with a dagger, then ruffles through his pockets and steals some copper coins to make it look like a realistic mugging. 

Later she dumps the coins to a beggar. 





\subsubsection{Visits \Uswa}
\Criseis{} goes to the \hs{dead garden} and talks to \hr{Uswa}{\Uswa}. 
She recognizes that, though mad, \Uswa{} knows more than most people credit her for. 

First we see \Uswa{} alone. 
She is mumbling. 
She communes with \quo{things in the ground} and \quo{things in the air}. 
She knows that there is more than one \quo{tribe} of \quo{things} vying for dominance. 
She can see the chains leading down into the deep, and the spiritual chains that tie people to \Nyx{} and the Cabal \Matrix. 

Then \Criseis{} comes. 
\Uswa{} tells her stuff. 

\Uswa{} sees through \ps{\Criseis} mortal guise and sees that she is really immortal. 





\subsubsection{Meets \Teshrial}
\Teshrial{} meets \Criseis. 

She responds to him in the \Resphan{} tongue. 

He tells her he accepts {\ps{\Ishnaruchaefir} rematch offer}. 
He \quo{makes an appointment}. 

\Criseis{} is afraid. 
From the date \Teshrial{} requests, she can deduce that he has figured out \hr{Ishnaruchaefir's Nadir}{\ps{\Ishnaruchaefir} Nadir}. 
But she doesn't know that this is part of \ps{\Ishnaruchaefir} \trope{XanatosRoulette}{Xanatos Roulette}, so she begins to really fear \Teshrial, for her master's sake. 

And, come to think of it, also for her own sake.
She can tell that he means her ill. 

\begin{prose}
  \Criseis: 
  \ta{I see the way you are looking at me, Lord \Teshrial. 
    Are you planning to use me as leverage to get to my master?
    I think you will regret it if you do. 
    So far Master \Quessanth{} has maintained a certain diplomatic code of conduct. 
    A certain mutual respect. 
    This will cease the instant you try to harm me.
    And I do not think that is what you want, my lord.}
  
  \Teshrial: 
  \ta{Are you threatening me, \scatha.}
  
  \Criseis: 
  \ta{%
    [The story of \hr{Criseis's siblings}{her sister and brother}.] 
    And this is not a myth, Lord \Teshrial. 
    If you doubt me, I suggest you look up the incident in the \CiriathSepher{} archives.}
\end{prose}

\Teshrial feels she is being rude. 
She is a mere mortal and he is a \ketheran.
She has no right to bargain with him or try to \quo{warn} him. 
So he threatens her. 

\begin{prose}
  \Teshrial: 
  \ta{Are you afraid, \Criseis?}
  
  \Criseis:
  \ta{Yes. But \emph{you} should be more afraid, Lord \Teshrial.
    I hear you breed \humans.
    How would you like to have a village of them suddenly consumed in a fireball?
    That is the least of what my master might do to you if you harm me.
    Please know that this is not my threat. 
    I would not condone the slaughter of defenseless \humans, but I cannot answer for my master's anger.}
\end{prose}

\Teshrial{} knows about \ps{\Ishnaruchaefir} rages and terrorism, so he wasn't going to harm her. 
(At the beginning or at the end of the chapter we need to see \ps{\Teshrial} POV and show that he does know the story of \ps{\Ishnaruchaefir} terrorism.)
But her tale still chills him a bit. 
He thinks of how horribly evil the Destroyer is, that he would be so cruel as to let cute, defenseless \humans{} suffer for his wrath. 
This makes \Teshrial{} hate him.
He must die. 

\begin{prose}
  \Teshrial: 
  \tho{Such a monster. He must be destroyed.}
\end{prose}


As \Teshrial{} discovers, \Ishnaruchaefir{} \hr{Ishnaruchaefir's code of honour}{does have a certain code of \honour, but can go berserk if prompted}. 









\subsection{\Urizeth dies}





\subsubsection{\Ishnaruchaefir{} hears of \Urizeth}
Have a scene in the \hs{Mirage Asylum} with \Criseis{} and \Ishnaruchaefir. 

\begin{prose}
  \Criseis:
  \ta{Be careful with this \Teshrial, Master \Quessanth. 
    He is clever and resourceful. 
    Do you remember \Lothagiel? 
    I have learned from my Sentinel contacts that 
    \Teshrial{} has procured \ps{\Lothagiel} notes and is researching you. 
    Moreover, he is studying \WanderersInDarknessEmph, and even consulting an expert on the poem.
    A certain \Urizeth{} of \TiphredSerah.}
  
  \Ishnaruchaefir{} (smiling diabolically): 
  \ta{Is he now? 
    I will have to do something about that.
    \Urizeth, you say\ldots{}?}
\end{prose}

\Ishnaruchaefir{} is happy. 
This is exactly what he wants \Teshrial{} to do. 
But he wants to fake that he fears this research. 






\subsubsection{\Ishnaruchaefir kills \Urizeth}
\target{Ishnaruchaefir kills Urizeth}
\Ishnaruchaefir{} seeks out \Urizeth{} while she is out in a Shrouded Realm, visiting a demesne. 
He sneaks up on her as close as he can while avoiding detection (which is not very close). 
Then he charges. 
She immediately senses a huge-ass \vertex{} coming straight at her, so she tries to flee. 
But she is unprepared. 
Besides, she is no athlete and hence no fast flyer. 

He catches her. 
He blasts her with an attack spell. 
It does not kill her, but it shreds her wings, grounding her. 
Now she cannot escape. 

\begin{prose}
  \Urizeth: 
  \ta{You\ldots{} \Ishnaruchaefir!}

  \Ishnaruchaefir: 
  \ta{\Urizeth.
    It has come to my attention that you are\ldots{} researching me.
    I must interpret this as a challenge. 
    And I am not pleased.
    I fear I must make an example.
    To you and all your kind.}
\end{prose}

Then he kills her. 
She defends herself, but she knows she has no chance. 

When she dies, he makes a mock attempt at destroying her soul. 
She fights back, using all the \TiphredSerah{} stealth and cleverness at her disposal. 
She succeeds, and her soul survives and eludes his grasp.
He lets her think she outsmarted him, but in reality he wanted her to do it. 
Maybe he might have been able to destroy her, but he let her go. 
(He admits, though, that she was good. Slippery like an eel. He is not sure he could have destroyed her. Maybe he tells this to \Secherdamon{} or \Nzessuacrith{} or \Menessiaraid{} or \Criseis.) 
He wants her to go back to \Teshrial{} and redouble her efforts to help him. 

And above all, \Ishnaruchaefir{} wants the \resphain{} to think he fears them and their research. 
He wants them to think they are on the right track and redouble their efforts to divine his weaknesses from \WanderersInDarknessEmph. 

This is a gamble from his side. 
He knows there is a chance they will discover some true weaknesses, but he is willing to take that chance. 
He thinks it more likely that they will discover red herrings and thus play into his hands. 





\subsubsection{\Teshrial{} hears of it}
When \Teshrial{} returns to \ps{\Urizeth} place to discuss their research, he is dismayed when another \resphan{} there tells him that \Urizeth{} has been killed. 
She has not yet regained consciousness, so her fellows only know that she is dead but not destroyed. 
They do not know who did it. 

\Teshrial{} walks away dismayed. 
He has a good idea of who might have killed \Urizeth{}. 









\subsection{\Achsah goes to \Forclin}





\subsubsection{\Achsah{} thinks about Needle}
Needle is very handy. 
The Sentinels know about Charcoal, but they don't know about her. 
She is all close to \Tiroco{} and can spy on her and have her shadowed, and the Sentinels don't know it. 

\Achsah{} kind of likes Needle. 
She loves her for her imperfections. 
\Achsah{} herself is imperfect, too, being \bezed-born. 




\subsubsection{\Achsah{} in \Malcur}
Remember to have more scenes with \Achsah{} in \Malcur. 

She has doubts about Needle's competence, so she takes action more directly. 

Remember to portray her as cool. 





\subsubsection{\Achsah{} suspects a trap}
\target{Achsah suspects that Malcur is a decoy}
\Achsah{} is attempting to unravel what \Ishnaruchaefir{} and \Secherdamon{} are up to. 
Insert some musing about the terrible dark lord \Secherdamon, and the mystic immortal \Ishnaruchaefir, whose motives no-one knows, but whose badass-ness is universally feared. 

Late in the story, \Achsah{} becomes convinced that there is something wrong with the whole \Malcur deal. 
It is too obvious. 
The Sentinels are leaving too many clues out in the open. 
She has been a Sentinel and opposed \Secherdamon{} for thousands of years. 
She knows that being this overt is not his style. 
He is smarter and more stealthy than this. 

She becomes convinced that there is something deeper going on. 
She suspects that \Malcur is a decoy. 

This is deliberate. 
\Secherdamon{} has made his \Malcur ploy so obvious as to convince everyone that it must be a decoy. 
He draws all eyes to him and assures them that there is nothing to see. 
That way, when \Nzessuacrith{} attacks \Forclin, everyone will assume that \Forclin{} is the real deal and rush off to there to try and stop him. 
This will leave \Malcur wide open. 

So she goes to \Forclin. 









\subsection{\Urizeth revives}
\Urizeth{} is thoroughly killed, so it takes several days for her to come back. 
Six days or so after her death, she sends word to \Teshrial{} and asks her to come see her. 
He does. 

When he sees her, she is in terrible shape. 
She looks like a mummy; a shrivelled, mutilated husk of a \resvil. 
She can talk and move her arms, but her legs and wings are not yet fully regrown, so she is confined to a wheelchair for the time being. 
\Teshrial{} is grossed out, but he fully sympathizes. 
He remember how badly shape he himself was in after \Ishnaruchaefir{} had killed him. 

\Teshrial is impressed at the level-headed manner in which she bears her injuries. 
It must be a \TiphredSerah thing, he concludes. 
A \CiriathSepher would be devastated and lock himself up and let no one but his most trusted confidants see him until he was fully healed. 
\Teshrial knows; that was what he did.
A \Mystraacht, on the contrary, would probably display his wounds with pride, as a testament to his bravery or somesuch. 
The \TiphredSerah, as far as he understands, have a philosophy that \quo{looks can be deceptive} and that appearances should not be given much weight. 
\Teshrial supposes this mentality is the reason why \Urizeth is able to bear her wounds with so little emotion. 

\Urizeth{} tells him the story of how she was killed, and the warnings and threats \Ishnaruchaefir{} gave her. 
But \Urizeth{} refuses to give in to his threats. 
She also refuses to wait for her body to recover. 
She wants revenge on the evil \dragon. 
She wants to get back to work as soon as possible, so she can help \Teshrial{} devise a way to rid the world of this cruel monster for good. 






\subsubsection{They discover weaknesses}
With the help of \Urizeth{} and other wise, elder immortals (perhaps even non-\resphain, such as \quiljaaran), \Teshrial{} manages to figure out \ps{\Ishnaruchaefir} weaknesses. 
He studies records of \Ishnaruchaefir{} in battle and maps \hr{Ishnaruchaefir's fake weakness}{the weaknesses he displays}. 

But in truth, the weaknesses are all fake. 
They are lies that \Ishnaruchaefir{} himself planted for this exact reason: 
To fool his enemies, lure them out and destroy them. 
It wasn't in the original \emph{\hr{Wanderers in Darkness}{\WanderersInDarkness}}, so he has planted some edited and unfaithful copies with his own fake myths in them.
(Some bits he wrote himself, others he had his close allies compose.) 

\lyricsbs{William Blake}{%
  The Four Zoas (Night the Second, 24:10-24:4)
}{
  \begin{tabular}{cl}
    FZ2-23.11; E313 & 
    First he beheld the body of Man pale, cold, the horrors of death 
    \\
    FZ2-23.12; E313 & 
    Beneath his feet shot thro' him as he stood in the Human Brain 
    \\
    FZ2-23.13; E313 & 
    And all its golden porches grew pale with his sickening light 
    \\
    FZ2-23.14; E313 & 
    No more Exulting for he saw Eternal Death beneath 
    \\
    FZ2-23.15; E313 & 
    Pale he beheld futurity; pale he beheld the Abyss 
    \\
    FZ2-23.16; E313 & 
    Where Enion blind \& age bent wept in direful hunger craving 
    \\
    FZ2-23.17; E313 & 
    All rav'ning like the hungry worm, \& like the silent grave 
    \\
    &
    \\
    
      
    FZ2-24.1;   E314 & 
    Mighty was the draught of Voidness to draw Existence in 
    \\
    
      
    FZ2-24.2;   E314 & 
    Terrific Urizen strode above, in fear \& pale dismay 
    \\
    FZ2-24.3;   E314 & 
    He saw the indefinite space beneath \& his soul shrunk with horror 
    \\
    FZ2-24.4;   E314 & 
    His feet upon the verge of Non Existence; his voice went forth   
  \end{tabular}
}





\subsection{The Achilles Heel}
\target{Ishnaruchaefir's weakness}
There is a myth in \WanderersInDarknessEmph that \Ishnaruchaefir{} can only be killed under certain conditions. 
He has an Achilles Heel. 





\subsubsection{\Zaz and \Urzaz}
\WanderersInDarknessEmph spoke of a mysterious pair of entities named \hr{Zaz}{\Zaz and \Urzaz}. 
It was unclear whether these were \dragons, \xss, cosmic gods or even purely metaphorical entities, personifications of something abstract.
Compare to Gog and Magog from the \emph{Bible}.

\Urizeth discovers that \Ishnaruchaefir apparently fears them and takes damage from them.  
Some \WanderersInDarknessEmph passages describe how the \Zaz and \Urzaz are anathema to him.

The Exile feared the \quo{body of \Zaz} and \quo{that which issueth forth from \Urzaz}.
Some clues said that he feared the \malgryph (the \quo{body of \Zaz}), that he quailed before it, that it held the power to cast him down and destroy him.

\Urizeth had long had trouble interpreting \quo{that which issueth forth from \Urzaz}.
It could be the very \quo{being} or \quo{aura} of \Urzaz, or it could be his breath or something else that emanates from him.
But now that \Teshrial specifically asks her to look into the problem, she remembers some old research she has done.
It did not lead to much back then, but now she digs it up and looks at it with renewed motivation.

\target{Urizeth researches Chimaera}
She had been trying to nature of the mysterious \Zaz and \Urzaz, and had an idea they were connected with the \quo{\Chimaera}.
The \Chimaera is a creature, probably a metaphoric one. 
It is related to and perhaps identical to \Zaz and \Urzaz.
Perhaps it is the union of these two (possibly contrasting entities) that form the \Chimaera (a \chimaera is a crossbreed or mashup or combination). 

\target{Urizeth thinks Zaz and Urzaz are the Chimaera}
\Urizeth suspected a link from \Zaz and \Urzaz to the \Chimaera. 
But \WanderersInDarknessEmph is a huge poem, and she had not been looking in the right places.
Now her attention is turned towards the parts that deal with the Exile, and here there are some very clear indications that (once you thoroughly interpret them) strongly suggest that \Zaz and \Urzaz more or less \emph{are} the \Chimaera.

\target{Urizeth researches Malgryph constellation}
In connection to all this, there was \hr{Malgryph constellation}{a constellation called the \Malgryph}.
It had a very obscure meaning in ancient \draconian occultism.
It was never used in the \rethyactic tradition except in the vaguest of references, so \Urizeth-tachi had great difficulty researching it. 
They would have to consult a \dragon or \quiljaar sorcerer to learn what the \Malgryph meant.
(Maybe they tried contacting a \quiljaar, but \Ishnaruchaefir got to him first and coerced him into silence.)
But \Urizeth knows that the stars representing \Zaz and \Urzaz are part of the \Malgryph constellation.
It is possible that the \Malgryph \emph{is} the \quo{\Chimaera}.
A \malgryph is, after all, a mix of different beasts and hence a \chimaera of sorts.





\subsubsection{The \chimaera and the \malgryph}
A clever reading of the poem suggests that the body of \Zaz is the same as the \quo{\Chimaera's ichor}. 

The \Chimaera's ichor is a physical substance.
\Urizeth remembers that there exists some practical research regarding this.
It was suspected that the \Chimaera's ichor might have interesting arcane uses, so alchemists did a lot of research on its nature and composition and how to reproduce it.

\Urizeth searches in the archives and finds some material about it.
It turns out that the alchemists did indeed succeed in brewing some \Chimaera's ichor.
It seemed to satisfy the properties described in the poem, so the alchemists were confident they had the right mixture.
Sadly, they failed to find any use for it, so the research project fizzled and was forgotten.
But now \Urizeth and \Teshrial have rediscovered it, and \Urizeth believes the \Chimaera's ichor is vital to defeating \Ishnaruchaefir.

After some more reading and interpretation, \Urizeth believes they need not merely the ichor.
They need the \malgryph itself.
They have to research and discover the \hr{Malgryph summoning}{spell that lets them summon a \malgryph}, and then unleash it upon \Ishnaruchaefir.
\Teshrial is \skeptical, since he believes to know that \malgryphs do not truly exist. 
\Urizeth tells him how that works. 

\Urizeth believes that if they can brew some \Chimaera's ichor, they can use it to summon the \malgryph and have it fight \Ishnaruchaefir.
But it must be done stealthily. 
\Ishnaruchaefir already suspects what they are up to.
He knows they are researching him and reading \WanderersInDarknessEmph, so he may suspect they have uncovered this secret. 
That might in fact be the very reason why he killed \Urizeth.
So they need to summon the \malgryph in some stealthy, sneaky manner. 





\target{It is all fake}
The \malgryph idea is actually a clever ruse by \Ishnaruchaefir. 

In reality, \hr{Zaz}{\Zaz and \Urzaz} \quo{are} not the \malgryph. 
Their nature is more complex and ambiguous than that.

\Ishnaruchaefir planted the clues saying that he feared the \malgryph.
It was based on some authentic \WanderersInDarknessEmph passages that tell how \Ishnaruchaefir feared \Zaz and \Urzaz and was cursed and cast out by them.
This was based on real events.
\hr{Zaz}{\Zaz and \Urzaz} were real cosmic gods, albeit highly obscure ones. 
There was a time when \hr{Zaz denies Ishnaruchaefir}{\Ishnaruchaefir appealed to them for aid}. 
They denied him and punished him, and he was wounded and weakened by their attack.
\Ishnaruchaefir made up some more \WanderersInDarknessEmph verses that were very similar to these ones and embellished on them, telling a bogus story about how he was an enemy of \Zaz and \Urzaz and all their being. 
In reality \hr{Ishnaruchaefir and Zaz}{he is an ally of sorts of those cosmic gods}, and he can command much of their power.





\subsubsection{Heel is inaccessible}
But it is difficult for \Teshrial{} to exploit the Achilles Heel using the skills, weapons and techniques at his disposal as a regular \resphan. 

So \Menessiaraid{} advises him to seek out the \quo{research department}. 
Who knows? 
They might have an \hr{Teshrial's experimental weapon}{experimental weapon} for him\ldots{}

Maybe it is not \Menessiaraid{} but \Azraid{} \hr{Teshrial talks to Azraid}{who suggests this}. 








\subsection{\Teshrial talks to \Azraid}
\target{Teshrial talks to Azraid}
Have a scene where \Teshrial{} talks to \hr{Azraid}{\Azraid}. 
He remarks on how \hr{Azraid's appearance}{\Azraid{} is much shorter than he}, but looks taller because of his great presence. 

Remember to read about \hr{Azraid}{\Azraid} before writing this. 

\Azraid{} is very interested in the \vertexspike{}. 
He asks \Teshrial{} for details. 
\Teshrial{} regretfully informs him that he knows little. 
It is \ps{\Achsah} table. 
\Azraid{} \hr{Azraid learns of spike}{later learns more}. 

\Teshrial{} talks about his plan to slay \Ishnaruchaefir. 
He has several trump cards:
\begin{itemize}
  \item The Shroud.
  \item Astrology. 
  \item \Noggyaleth.
  \item The Achilles Heel. 
\end{itemize}

\Azraid{} is \skeptical about \ps{\Teshrial} plan to challenge \Ishnaruchaefir.
\Azraid{} has encountered and fought \Ishnaruchaefir{} and knows \Teshrial{} is clearly no match for the \dragon. 
\Teshrial{} assures him he has a plan. 
He will lay a clever ambush, use his \noggyaleth{} and employ the secrets of the prophecies to attack all of \ps{\Ishnaruchaefir} weaknesses. 
\Azraid{} is still doubtful, but the idea has potential. 
\Azraid knows that \Ishnaruchaefir{} is inquisitive and willing to take silly chances or be chivalrous for the sake of his curiosity. 
\Teshrial can take advantage of this. 
It can buy him time to pull off some of his gambits and traps. 
\Azraid gives \Teshrial{} his blessing to continue. 

\begin{prose}
  \Teshrial: \ta{I will lay a trap with my \noggyaleth.}
  
  \Azraid: 
  \ta{But \Ishnaruchaefir{} has already killed one of your \noggyaleth.}
  
  \Teshrial: 
  \ta{Yes, my High Lord \Sathariah, but he believes that was the end of it.
    He does not suspect the true extent of my \noggyal{} plot.
    And even if he does, the one he slew was a runt. 
    I have much bigger ones.
    And he knows naught of the cunning trap I have laid.
    I will utilize his Achilles Heel, as described in the prophecies.}
  
  \Azraid: 
  \ta{Prophecies. 
    You do not believe in such superstition, do you?}
  
  \Teshrial: 
  \ta{No, my High Lord \Sathariah, not literally, of course.
    But myths are often truth shrouded in poetry.}
  
  \Azraid: 
  \ta{True.
    \emph{\hr{Wanderers in Darkness}{\WanderersInDarkness}} has proven itself full of remarkable insight in the past, hidden in symbolism. 
    So you may be right. 
    But I hope you are interpreting this correctly, \Teshrial. 
  
    Besides\ldots{} 
    \Ishnaruchaefir{} is known to take reckless chances.
    You just might fool him. 
    Very well. 
    You have my blessing. 
    
    However, as well prepared as you are, I still fear that it may not be enough. 
    I have an idea\ldots{}}
\end{prose}

\Azraid{} is not sure he believes in the Achilles Heel. 
(This is foreshadowing of the fact that the heel is fake, so that does not seem like an \trope{AssPull}{Ass Pull}.) 

\begin{prose}
  \Azraid: 
  \ta{%
    Can it really be true, that \ps{\Ishnaruchaefir} greatest weakness was just under our noses all this time?
    It seems to good to be true.}
\end{prose}


\Teshrial{} comments (inside his head) on the fact that \hr{Azraid's appearance}{\Azraid{} has wrinkles}. 
And his monstrous hand, which he always keeps hidden\dash with good reason, as \Teshrial{} knows. 
Have subtle references to the evil hand in all \Azraid{} chapters. 

\Azraid{} asks whether \Ishnaruchaefir{} might be in league with \Secherdamon. 
But \Teshrial{} assures him he is not. 
\hr{Achsah and Teshrial worry about Secherdamon and Ishnaruchaefir}{He and \Achsah{} have previously examined this}. 

Perhaps \Ishnaruchaefir needs not attack and destroy stuff in order to be a threat. 
I just need to clarify that if he is not stopped soon (chased away or preferably killed), he will wreck everything they have worked for in \Malcur.
When he is at his full strength, he could attack in force and drive the Cabal out of \Malcur entirely.
It is known that he takes an interest in \Malcur, so he likely has long-term evil plans there.
He gave hints of that in WSB. (Make him give hints!)
That must not be allowed to happen.
Furthermore, even now that he is weak, he might be up to something.
If \Urizeth's conclusions are correct, then these Nadirs happen to him regularly, and if so, \Ishnaruchaefir must have learned long ago to live with them and still get stuff done.
One must not assume that he is harmless in his Nadir.
(Maybe it is \Azraid who speculates the above to \Teshrial.)

\Azraid thinks to himself. 
He is not sure whether \Teshrial's \Malcur venture is really so great and important a thing as \Teshrial-tachi like to believe. 





\subsubsection{Experimental weapon}
\target{Teshrial's experimental weapon}
\Azraid{} explains about some of the experiments that his scientists are working on. 
There is a mutation spell in their arsenal. 
\Teshrial{} is hesitant. 
But also excited.
\Azraid{} is good at marketing the thing. 
\Teshrial{} will bring extra glory to himself if he does his people this further service and field-tests their secret weapon. 

\Teshrial{} agrees to take this weapon (but only as a last-resort backup plan). 
\Azraid{} tells him to go to some Cabal scientists and get it from them. 

He does. 
There is a \banelord{} among them. 
Remember to make the \banelord{} use very archaic language. 

They give him this horrible parasite that crawls onto his body. 
When he activates it, it will burrow into his body and transform him into a monster. 

This tool is laboratory-tested but not field-tested\ldots{} and only tested at a part of its potential strength, not at full power. 
They fear that the full-powered version will have unhealthy side-effects. 

\Teshrial{} sees the mutation process demonstrated (offscreen) and shudders. 

He takes the weapon, but makes no guarantees. 
He intends to hold it back as a last-resort back-up weapon, 
He fears it and will not use it unless he has to. 

The weapon is part of the \hr{Neo-Resphan}{\neoresphan} project. 





\subsubsection{\NeoResphan{} treatment}
The below is an optional spin on the whole \quo{experimental weapon} deal. 

\Teshrial{} is not only handed a device. 
He also undergoes some alchemical and magical treatment that is meant to transform and strengthen his body. 
It works. 
When the day of the duel arrives his body has been strengthened, so he is more powerful than normal even in his \resphan{} form. 

But there are side effects. 
He gets attacks of pain and dizziness. 
Sometimes he wakes up at night with nausea. 
One day after a meal, where he had been drinking lots of blood, he finds himself bleeding afterwards when he's alone. 
The blood he has ingested is seeping out of him. 

I am not sure about the above. 
It might be too cheesy. 
Remember, I am going for \quo{heroic} side effects, not \quo{villainous} ones. 

He is afraid. 
But he bears the pain and fear. 
For his quest's sake. 
For \Firaxel. 





\subsubsection{\Teshrial must undergo a quest}
Perhaps \Teshrial must undergo some kind of quest or test or training before he is deemed ready and worthy to receive the \neoresphan treatment\dash or he must undergo training before he is able and ready to use it. 





\subsubsection{\Achsah{} fears the experimental spell}
\Teshrial{} mentions to \Achsah{} the possibility of using this experimental spell. 
\Achsah{} is frightened by the consequences of that. 
She does not like \Teshrial, but this mad sacrifice on his part is still frightening to her. 

\Achsah{} understands \ps{\Teshrial} motivation. 
He wants children. 
\Achsah{} can never have children. 
She wishes she could at least have the vain hope of once being a mother. 
So she tries to wish the best of success for him, at least. 





\subsubsection{\Teshrial sees how ugly \Nyx is}
In a scene late in the book, after \Teshrial has acquainted himself with the \noggyaleth, \neoresphain and \WanderersInDarknessEmph:
  
He flies above \Nyx.
He sees the \bane-built spires. 
Now that he gazes into the deep, he notices how twisted they are.
They look really scary and wicked.
He had always taken them for granted, but now that he has gazed deep on the dark mysteries of his people (and the even darker mysteries of the \banes), they scare him.
There is something evil about the way they twist and bulge.
They twist into alien dimensions (more alien that what he likes to consider). 
Like they are appendages of some vast monster that tries to crawl and claw its way up from the deep where it belongs. 
See also the sections on \hr{Nyx}{\Nyx}, \hr{Resphan architecture}{\resphan architecture} and \hs{dark ancient cities}. 









\subsection{\Achsah in \Forclin}
\Achsah is now in \Forclin. 
She alights on the roof of a high tower and gazes out over the city. 
She can clearly feel that something metaphysical is afoot. 
She believes she is right. 
The Sentinels are up to something here. 
Probably \Secherdamon. 
She will find out what. 

Se looks at the \hs{Ghost Tower} and comments that it is \hr{Ghost Tower history}{definitely \resphan-built}. 
Despite what the common folk may think. 
Her people once had a city here.
It was destroyed in the \resphanwars. 
Or something. 
(Read about the \hs{Ghost Tower}. 
 If no information is there, make it up and add it.)

She thinks about \Ishnaruchaefir. 
She is not comfortable with \Teshrial's idea of confronting \Ishnaruchaefir in battle. 
She thinks \Teshrial will get himself killed. 
\Teshrial is a vain and condescending snob, but he is not a bad \resphan.
He is no worse than most of \CiriathSepher. 
She cannot expect better treatment. 
She is \ashenblooded, after all. 
Her mother was a hairy, brutish \nephil. 

\Teshrial does not deserve to be destroyed by \Ishnaruchaefir. 

She remembers \Ishnaruchaefir when they met him in the dead garden in \Malcur. 
When \Ishnaruchaefir{} approached the city, \Achsah{} \hr{Detecting Vertices}{could feel him from miles away}. 
He was a behemoth \vertex. 
It felt like a humongous 200 ton sauropod stomping through the city. 
He had not only his own \vertex{} power, but also that of the glaive sending out tremours through the Shroud. 

\Achsah{} described \Teshrial{} in a way to make him seem madly overconfident. 
This is what he intends. 
He hopes to provoke \Ishnaruchaefir{} and goad him into coming to fight him, to \quo{teach him a lesson}. 

\Achsah{} lets slip that \Teshrial{} intends to use the astrological situation as a weapon against \Ishnaruchaefir. 
This is part of \ps{\Teshrial} plan: 
Leak information about one of the traps. 
That way \Ishnaruchaefir{} will be overconfident and not suspect the other traps. 

Make it clear how arrogant, dominant and strong \QuessanthIshnaruchaefir{} is. 
He all but commands \Achsah{} around. 
She realizes that if he gave her an order, she would likely obey him. 
She doesn't like that thought. 
She lived during the Incursion, after all. 
Or the \secondbanewar, as the Sentinels call it. 
She remembers his atrocities. 

\Achsah once \hr{Achsah met Ishnaruchaefir}{encountered \Ishnaruchaefir on the battlefield}. 
She saw his visage, ablaze with fierce chaotic sorcery and hatred toward her kind. 
She felt his presence, radiating a silent promise of death and vengeance. 
And she had not stood her ground. 
She had not even let him come near her. 
She had fled from his path in panic. 
The rest of the battlefield had seemed like a sanctuary, then. 

\begin{prose}
  \tho{I had no courage to face him. 
    But \Teshrial\ldots{} he stood his ground. 
    He met the Destroyer in single combat, and he fought to the death.
    \Teshrial{} is brave, I must give him that. 
    Much braver than I.} 
\end{prose}



She loathes, fears and despises \Ishnaruchaefir. 
But at the same time, it is hard not to be captivated by the force of his personality. 
Such will. 
Such power. 
Such glory. 
Such ferocity. 

There is raging bloodlust and savagery in him, she knows. 
But at other times, such as this, there is mournful feeling. 

He is an outcast, despised and condemned by his own people, feared and reviled by all. 
Yet he stands proud like a king. 
She can feel great pride and confidence radiating from him.
\Ishnaruchaefir is every bit the \dragonking{} that his brother \Nexagglachel{} was. 
She feel some kind of kinship with him in that moment. 
She is also an Exile of sorts herself, being \ashenblooded. 

(Not too much kinship, mind you. 
 I do not want to dilute the Cosmic Horror.
 Perhaps she sees him as an ideal or possible mentor-figure instead of a kindred spirit.
 An example of how awesome you can be as an Exile.)

\begin{prose}
  \tho{If only I had his strength.
    His disdain.
    His pride.} 
\end{prose}








\subsection{\Teshrial and \Criseis and \Ishnaruchaefir}
\Criseis is in \Malcur again. 
Again she carries \Ishnaruchaefir's presence with her, like she did the first time. 
This is deliberate, to lure the \resphain out. 
\Ishnaruchaefir suspects that \Teshrial is ready to challenge him, and he wants to force \Teshrial's hand. 
\Ishnaruchaefir does not like to be reactive. 
He wants to be proactive. 

\Teshrial is distressed. 
He comes down to \Forclin as expected to confront \Criseis. 
He brings some other \resphain with him who happened to be nearby. 
Perhaps \Menessiaraid. 

On the way down, \Teshrial thinks about the situation.
The \vertex signature is more obvious this time. 
Last time, only the High Telepath, \Achsah, was able to detect it. 
Now all their measuring apparatus can feel it.
The significance of this is not lost on \Teshrial. 
\Ishnaruchaefir is deliberately being obvious.
He wants to lure the \resphain out. 
They are doing exactly what \Ishnaruchaefir wants them to do. 
But they still have to respond.
Who knows what he will do if they do not come?

This can replace the scene where \hr{Ishnaruchaefir attacks viewing station}{\Ishnaruchaefir attacks a viewing station}. 
I just need to have something that clarifies what a great menace \Ishnaruchaefir is. 

The \resphain arrive in \Malcur and find \Criseis. 

\Criseis makes a peace sign.
She is afraid of \Teshrial.
He is more bitter and hateful towards her than last time. 
But he does not harm her. 

\Criseis tells him she comes with a message from her master.

\Ishnaruchaefir then comes up to possess \Criseis's body and speak through her.
\Criseis goes into a trance. 
\Teshrial can recognize the feeling of \Ishnaruchaefir.
He can see the suggestion of a vast, black, \draconian form around her. 
He feels much Cosmic Horror. 

They exchange some taunts. 
\Teshrial{} tells \Ishnaruchaefir{} the date when he wants his rematch. 
It is at the center of \ps{\Ishnaruchaefir} Nadir. 

\Teshrial{} acts overconfident and arrogant. 
Like a whelp who thinks he is something big. 
This is what he intends. 
He hopes to provoke \Ishnaruchaefir{} and goad him into coming to fight him, to \quo{teach him a lesson}. 

\Teshrial lets slip that he intends to use the astrological situation as a weapon against \Ishnaruchaefir. 
This is part of \ps{\Teshrial} plan: 
Leak information about one of the traps. 
That way \Ishnaruchaefir{} will be overconfident and not suspect the other traps. 

\Teshrial tells \Ishnaruchaefir the time and place where he wants to fight him.
\Ishnaruchaefir accepts. 

\Ishnaruchaefir{} fakes suicidal bravado (i.e., fakes that he is trying to fake that he is not afraid, ultimately to make \Teshrial{} think he \emph{is} afraid).
He laughs.

\begin{prose}
  \Ishnaruchaefir: 
  \ta{I see you have done your research. 
    You and that scribbler, \Urizeth.
    But very well.
    I will not renege on my word\ldots{} not today, at least. 
    You are smaller than I, so it is only fair to grant you this concession.
    My legend attributes to me many flaws, but cowardice is not one of them.}
  
  \Teshrial:
  \ta{Actually, it is.}
  
  \Ishnaruchaefir:
  \ta{Hm. Yes, now that you mention it, it actually is.
    But be that as it may.
    I accept your challenge, \resphan.}
\end{prose}

\Ishnaruchaefir{} makes \Teshrial{} think that he (\Ishnaruchaefir) thinks that \Teshrial{} knows only about the Nadir and not the Achilles Heel. 
He fakes being afraid that \Teshrial{} might discover the Heel, but still cocky and confident that \Teshrial{} probably won't find the Heel. 

\Ishnaruchaefir reminds him of the fact that if the Cabal ever want to kill \Ishnaruchaefir, \hr{Teshrial is their best bet}{\Teshrial is their best bet}. 
\Ishnaruchaefir makes \Teshrial promise him that he will fight only \Teshrial, no other \resphain. 
(\Teshrial smirks inside because he plans to ambush \Ishnaruchaefir with his non-\resphan allies. \Ishnaruchaefir anticipates this. \Ishnaruchaefir is smart enough to know that \Teshrial would never face him if he did not have a trap or several prepared.)

\Ishnaruchaefir also warns them that if \Teshrial should break his promise, \Ishnaruchaefir will refuse to fight, and return later to exact a terrible vengeance.
\Ishnaruchaefir reminds \Teshrial that he found out quickly and easily about \Urizeth, to remind \Teshrial how deep his Cabal spies go and how good his intelligence is.
\Ishnaruchaefir lets slip that he knows \Teshrial has feelings for \Firaxel, and hints that he will come after \Firaxel if \Teshrial betrays him. 
(This is a bluff. It was through luck that \Ishnaruchaefir was able to get at \Urizeth so quickly and efficiently. He doubts he would be able to do the same with \Firaxel.)

\Teshrial remembers the story of how \Ishnaruchaefir terrorized the \resphain after \Criseis' siblings were murdered, so he takes \Ishnaruchaefir's warning very seriously.















\section{Malcur Thread}









\subsection{Rian in church}
Have a scene with Rian in church where he attends prayer and \hr{Rituals against Isphet}{mutilates \Isphet}. 
There is an effigy of \Isphet there in the form of a black serpent. 
Every church-goer is handed a needle or stick with which to impale the monster. 
At last, the effigy is hacked into pieces and burnt. 

In the church, Rian prays for guidance from the \sephiroth, and for their help with his quest to free Neina, and perhaps even \hs{stop the evil}. 









\subsection[Malcur is going mad]{\Malcur is going mad}
\target{Malcur is going mad}
The city of \Malcur is gradually going mad, as a result of the Sentinel ritual gaining strength. 

Mages can see energy seeping, bleeding out of holes in the Shroud, like cosmic blood. 

The people are affected by this chaos, too. They go mad and irrational. Violent riots become more and more frequent, resulting in lynches, arson, gang wars and bloody murder for no reason at all. 

Chaos is unleashing people's natural, inherent viciousness and bestial savagery. 
Have some \trope{RapeTheDog}{Rape the Dog} moments where the people of \Malcur show what kind of scum they really are at heart. 
Then the reader will feel less sorry for them when they are massacred. 

For our main characters, one advantage of the chaos that is engulfing \Malcur is that it is easier to be covert. 
Moro \Cobrel can skulk through the streets, cowled and furtive, with a zombie-like man trailing behind her, and not even look very suspicious. 
No one has time to notice. 
In fact, people seem to be trying hard to deny the fact that it is happening. 
This makes it easier to operate, but it is also very worrying.
Why are people insane and denying it?
(Moro may have seen it before, though. She knows \Ubloth, remember.)

Maybe, at the end, have a scene where a bunch of guys are lynching an innocent victim. Then some supernatural horror comes along and the bullies all die a gruesome death, while the victim is lucky and slinks away. 

\lyricslimbonicart{Towards the Oblivion of Dreams}{
  With the underworld's \\
  subconscious darkness I am allied. \\
  Deeper aspects, forces of nature,\\
  mind can now see the unseen. \\
  Ancient land hidden from man. \\
  An esoteric dream in the desert sand. \\
  Shimmering sparks in the darkness, \\
  as death overtakes the soul. \\
  Loose the body, and earthly conscience. \\
  The freedom of the spirit must be total.
}





\subsubsection{Mystic night}
Many of the important scenes take place at night.
Shroud-weaving magic is more potent at night because the Shroud is weaker here. 

As we approach the end, the \Malcuric night becomes more mist-shrouded and mystic. 
Read some Bal-Sagoth.





\subsubsection{\Nithdornazsh peeks out}
Have scenes where \Nithdornazsh{} peeks out of the Shroud. 
Various mortals (possibly Rian) see glimpses of living (or undead) buildings made of flesh and metal. 
Walls suddenly transform into bleeding, oozing flesh for a short instant. 
Preferably in hidden places\dash dark alleys, the slums, behind boards.

Rian sees this more than once. 





\subsubsection{The horror of \Nithdornazsh}
Someone (perhaps Moro \Cobrel) sees the \maybehr{Blood-Red Vaults of Nith'dornazsh}{Blood-Red Vaults of \Nithdornazsh} and is horrified. 
She is almost mentally pulled in, but she manages to steer away. 

\lyricsbalsagoth{Beneath the Crimson Vaults of Cydonia}{
  This red charnel pit of primal horror, \\
  howling black ecstasies to the void.\\
  Ancient and divine, older than the hidden Icosahedron, \\
  now rebirthed beyond the chaosphere.\\
  Rise\ldots{} rise and destroy!\\
  Hatred, carnage, slaughter, havoc, chaos, murder!\\
  I am become the devourer of all life!
}


 


\subsubsection{\Humanoid-based horror}
\target{Humanoid horror}
Have more horror with humanoids that turn into monstrous forms.
Then they go mad and become wholly converted by whatever evil side has transformed them.
They laugh and taunt the remaining mortals (such as Rian and Moro) and tell them that there is no escape, that the whole world will be transformed into a nightmarish hell of madness and chaos. 

(This should also happen earlier in the story.)

Have Moro and Rian as a kind of \quo{Only Sane Men}.
They see \Malcur go mad around them.
They are badly affected, too.
They are going crazy with fear because of all the horrors they see.
Moro only manages to keep her sanity because she has seen things like this before. 
And because she has monitored the process of \Malcur's going to hell, and so is less surprised by it than the regular folks who just see it happening overnight. 
Rian does not hold up well. 
All this evil tears his world view asunder. 
The \sephiroth seem to be powerless to prevent it. 
And this evil is not like he had imagined it.
He had expected winged devils with pitchforks, but not this. 
Not people turning into warped monsters before his very eyes. 

Rian tries to interpret it all in an Iquinian way to make it fit his world view. 
These people must be caught in fetters of \itzach. 
But his rationalization fails. 
This is all too alien, too wrong. 
It is nothing like the priests have described. 
Sinners are supposed to be cast out into the Outer Darkness. 
They are not supposed to mutate like this. 
It is wrong. 
Rian's world breaks down.

Moro slaps Rian up and tells him to pull himself the fuck together. 
She forces himself to shape up.
She is the main reason why he keeps on going and doesn't break down and become a babbling lunatic. 





\subsubsection{Weak souls succumb}
\target{Weak souls go mad in Malcur}
Weak souls are easily affected by \quo{the Change}. 
They go mad and/or mutate into monsters. 
Strong souls are less susceptible. 
Moro and Rian are strong souls (that's why they are main characters).
But Neina is weak. 





\subsubsection{Comparison with Carcosa}
The following scenes from \cite{RPG:CallofCthulhu:GreatOldOnes}, a supplement to the RPG \cite{RPG:CallofCthulhu}, may give an idea of the atmosphere I want to evoke. 

\lyricstitle{\emph{The Great Old Ones} p.70-72}{
  [The prisoner of Carcosa] is free to do anything while awaiting rescue or madness in dark Carcosa. The alienness of this city of towering black buildings costs 1/1D10 SAN per day. 
  Fill the prisoner's time in the city with odd occurences:
  
  \begin{itemize}
    \item 
      a keening voice wailing a lonely dirge, the source of which can never be found;
    \item 
      intermittent wingbeats of great unseen things in the thick clouds overhead; 
    \item 
      a slithering wave of fog which tirelessly pursues the prisoner through the damp empty streets;
    \item 
      occasional footsteps or whispering voices in the streets of the abandoned city;
    \item 
      a glimpse of a shadowy figure down the street, where no one can be found;
    \item 
      nigthmarish splashing in the waters of the lake;
    \item 
      noises whose sources elude vision because of the thich fog;
    \item 
      a glowing Yellow Sign in the waters of the lake. 
  \end{itemize}
  
  [\ldots{}]
  
  As the cultists chant the ritual, thick waves of fog roll in from the lake, then the lake itself swells and grows larger, and the water takes on an oily sheen. 
  The ground gently quakes and stretches. 
  Suddenly the investigators find themselves standing on the outskirts of an alien city, at the edge of a lake much larger than the one they had been observing. 
  The swamp has vanished. 
  The night sky is dull white, and in it black stars shine in unfamiliar patterns. 
}









\subsection{\Nasshikerr returns to Moro}
\target{Nasshikerr returns to Moro}
\hr{Nasshikerr}{\Nasshikerr} returns to Moro. 
He has found out some stuff about what is going on in \Malcur. 
He tells her about how it all comes together. 

Moro goes into action. 
She joins up with \Tiroco, to a limited extent. 
They do all they can to \hs{stop the evil}. 

They find out when the evil ritual will take place, and do all in their power to stop it. 





\subsubsection{\ps{\Nasshikerr} POV}
\Nasshikerr{} does not think any real harm will come from telling Moro about \ps{\Secherdamon} plan. 
He is a rival of \Secherdamon. 

Many Cabalists and especially Sentinels are more interested in their internal power squabbles than the \hs{Feud} itself. 
They do not expect the balance of power in the Feud to change within polynomial time\ldots{} at least not change asymptotically. 
Millennia of pointless squabbling and minor balance-of-power fluctuations have made them complacent. 
They do not suspect that the end is near. 
Few are as keen-sighted as \Secherdamon{} or \Azraid, and few suspect the scope or nature of their \trope{XanatosGambit}{Xanatos Gambits}. 

Besides, \Nasshikerr{} does not actually believe that Moro will be able to accomplish anything real. 
She is up against the formidable \hr{Psyrex}{\LocarPsyrex}, after all. 
But \Nasshikerr{} feels it could be interesting to see her try. 
(But remember to not make \Nasshikerr{} too evil.) 

Moreover, \Nasshikerr{} hopes Moro will be able to figure out \ps{\Secherdamon} plan. 
He wants to know more about what \Secherdamon{} is up to in \Malcur. 
He cannot figure it out on his own. 
He makes Moro promise to report her findings to him. 

\Nasshikerr{} kind of likes Moro, even though he is rough with her. 
She is a bit short-sighted, and her traumata make her narrow-minded and blind, but she does her best, and he respects her for it. 





\subsubsection{Moro sacrifices to {\Nasshikerr}}
Moro must sometimes sacrifice humanoids to \Nasshikerr. 
She is ashamed of it. 
She hides it from Rian, since she remembers how horrified Rian was after he had seen the Sentinels sacrificing humanoids. 

Whenever she must, she takes some thieves or other lowlife scum that no one will miss. 

\begin{prose}
  \tho{Thieves and scum. That could have been Rian, before I knew him.}
  The thought made Moro shudder.  
\end{prose}

When Moro is out in the city, she notices that \hr{Malcur is going mad}{\Malcur is going mad}. 
    
\hr{Rian is religious}{Make Rian more religious}.
Make sure he prays in every chapter and scene that he is in.
He prays to be delivered from \Isphet's evil. 
He has lingering existential/religious dread from the day when he saw the dark sorcerer slay the shining god (even though he was Shrouded and does not remember it all). 

In all the Rian chapters, whenever it is appropriate, have references to the \hr{Myths of vanquished monsters}{myths of Iquinian heroes vanquishing inhuman Elder Races and monsters}. 
When he encounters something supernatural, he fears that the wicked Elder monsters will conquer the world. 









\subsection{Needle summons \banes and \banerats}
\target{Needle summons Banes}
Needle needs to go on a raid against some Sentinel agents. 
She is only a minor mage, but she has been taught a spell that will invoke \Achsah{}, so she can send minions to her location. 





\subsubsection{She reports to \Achsah}
Some stuff happens. 

Needle is told to go into the crypt. 

\Achsah: \ta{Go into the crypt and summon \banes.}

Needle: \ta{Gasp! The ancient Vaimon crypt?}

\Achsah: 
\ta{%
  Tee-hee (very feminine laugh). I can share this secret with you: That crypt is older than the Vaimons.}

When Needle is down there\dash \emph{deep} down\dash she thinks about this, but it doesn't mean much to her. 
She knows nothing of architecture or ancient history. 





\subsubsection{She goes into the dark crypt}
She goes into the Vaimon-built crypt beneath \CastlePelidor. 
These are dark, disused and have an \quo{antediluvian} feel to them. 
It is a secret place built by Cabalists in the \VaimonCaliphate to serve as a gateway to \Nyx. 
Everwhere, she feels as if the eyes of wicked \Qliphoth{} are upon her. 
As if her audacious steps cause sleeping powers to stir in anger. 
She's afraid. 





\subsubsection{The \banes{} need \human{} hosts}
The \banes{} cannot come to \Miith{} directly, but need to possess a \human{} host. 
So Needle brings three prisoners with her to use. 
All of them total losers. 
They're probably dungeon inmates. 
She has contacts in the dungeon and can easily make some less important prisoners \quo{disappear} in the papers. 





\subsubsection{The summoning ritual}
The ritual is dark and mildly erotic. 
She must circumvent the \Sephiroth{} and invoke \Qliphoth{}, and even nameless powers. 

Is she naked? 
She might be, but only if I can come up with a good excuse for stripping her. 
Perhaps she was simply told that she should be naked because the \resphain{} like the sight of a \human{} prostrate, naked and begging\dash completely humiliated and submissive. 
Perhaps \Achsah{} sexually abuses her a bit. 

Needle is pretty afraid, for she has never had to deal with major issues like this alone before. 
Nor has she ever seen a \bane{} or \banerat{} up close. 
She has seen \resphain, but the more horrid creatures she has seen only from a distance. 
She has always shied away from them and averted her eyes, never having to deal with them directly. So she is frightened at having to take command of them. 

Needle mentions the Shroud. 
She doesn't really understand what the Shroud is. 
But is proud that she knows it exists. 





\subsubsection{She remembers how she first became a Cabalist}
Needle is reminded of the day when she first became a Cabalist, where she was first confronted with \itzach, \Nyx{} and the horrors that dwell there. 
She feels like she is re-living it. 

\lyricsdimmuborgir{Grotesquery Conceiled Within Measureless Magic}{
  For thy presence made pleasure of pain,\\
  and thy madness turned sanity into vain.\\
  Profoundly wicked owner of souls.\\
  The mysteries of thy creation beheld by ghouls.
  
  Diabolically disguised heavenly bodies,\\
  and its atrociously desired primordial elements.\\
  Plunging through the confused beart of sulphur.\\
  In all this darkness, how can a man see?
}

She originally thought that mages were specially gifted supermen with inborn magical powers beyond the ken of the common herd. 
She was wildly surprised and more than a little scared when Charcoal-tachi told her that \emph{she} herself could learn magic. 
According to Charcoal, all it took to cast magic was a worthwhile brain and the willingness to see beyond the lies with which the common folk surround themselves (code for the Shroud) and face the dark truths of the universe. 
And then grab this truth with both hands and coerce it to spit out what you want. 





\subsubsection{Read up on \banes}
Remember to read up on \banes{} in my notes before writing this chapter. 





\subsubsection{She is afraid of the \banes}
She manages to summon two or three \lesserbanes{} and maybe some \banerats. 
The \banerats{} are not so powerful and act mostly as scouts and bloodhounds. 
But the \banes{} are terrible. 
Needle is scared shitless at commanding them. 
(She notices that they wear splint-mail-like \armour\ldots{} maybe.)

The \banes{} approach her and seem very menacing. 
She can't control them. 
They come ever closer. 

They seem to be directing some mental attack her at her, which she can't defend herself against. 

Needle panics, thinking that they are coming to eat her. 
She freezes up, or falls to her knees. 
She begins crying and whimpering. 
She whines and begs: 
\ta{Stay back! Go away! Don't hurt me\ldots{}}
She pleads with them to spare her life. 

\Achsah{} contacts her telepathically and slaps her up, telling her to get a hold of herself. 

Needle realizes that the \banes{} never intended to hurt her. 
They were simply menacing because that is how \banes{} appear to \humans: 
Hideous, cold, alien, evil. 

It turns out that the \pps{\banes}{} \quo{mental attack} was simply an attempt to communicate. 
They spoke to her in their telekinetic language and told her their names. 
\Bane{} names are not words but kinetic sensations. 
They are terribly frightening to unsuspecting mortals. 
\tho{So that's their way of speech, is it?}
To Needle it felt as if they were carving their names with daggers on her bare skin. 

Slowly she gets the hang of commanding them. 
But she is still afraid of them. 
Perhaps this gradually drives her mad. 

The three \pps{\banes}{} names are Quicksand, Undercurrent and Gallow. 
They look completely alike to her eyes, but their telepathic \quo{feel} is different. 

Seeing the \banes{} and fearing their terrible power makes Needle admire the \resphain{} even more, since they are mighty enough to control such fell, wicked beings. 
She comes to love them even more. 





\subsubsection{\Achsah{} laughs}
\Achsah{} has been observing. 
Now she sits back and laughs. 
There was no real need for Needle to be naked during the ritual. 
It was just a practical joke \Achsah{} pulled on her. 

She jokingly scolds herself for it. 
\tho{%
  That was unfair of me. 
  I shouldn't be doing that. 
  Playing tricks on poor, unsuspecting minions. 
  Naughty, wicked \Achsah.}





\subsubsection{What happens later}
The \banes{} kill many before they are finally overcome by \ps{\Psyrex}{} mages. 
Maybe Moro \Cobrel{} slays one of them\dash after a difficult fight!





\subsubsection{Quotes}
\lyricslimbonicart{Moon in the Scorpio}{
  In an atmosphere supreme\\
  forces dwell in domancy.\\
  The essence of its spirit is evil,\\
  as a curse upon thy name.
  
  Midnight is the shepherd of mysterious powers\\
  and moving shadows in the corner of the eye.\\
  Moon's blazing intuition\\
  contains what death requires.
  
  Cleanse the doors of perception.\\
  See things appear in its true art.\\
  The cold hands of divinity\\
  will tear thy soul apart.
  
  Behold the sky above \\
  when the moon is in the Scorpio.\\
  A cold bleak light. 
}










\subsection{Moro and Rian and Needle}





\subsubsection{Needle finds the Sentinel lair}
Needle and her servants succeed in locating the Sentinel lair in \Malcur. She begins to make preparations to storm them. 

Needle is growing ecstatic and hysterical, with the new dark powers at her command and the terrible \banes{} that follow her.

\lyricslimbonicart{Darkzone Martyrium}{
  I perish in my own desire.\\
  I burn within lusting hate.\\
  Destructively the minds inspire\\
  the soul to terminate.
  
  I ride the ancient overture\\
  as life is torn astray.\\
  I glance the illusive spectrum\\
  and all light that fades away.
  
  Black energies in the twilight space\\
  come shivering through the shallow haze.\\
  Into darkness so impure divine.\\
  A bloodshed emotion to evil wine.
}





\subsubsection{Moro and Rian find the Sentinel lair}
Moro and Rian have gotten some tips from \Nasshikerr (chiefly) and from the thug Moro has interrogated (secondarily).
They know there are some evil people that are preparing some big, evil ritual of magic, and that it is connected to that which is destroying the city.
The captive thug tells them where the ritual will take place. 
\Nasshikerr has told them the time. 

Moro and Rian go through \Malcur and look for the place which \hr{Nasshikerr returns to Moro}{\Nasshikerr} and/or \hr{Rian and Moro interrogate a thug}{their captive thug} described to them. 
This is dangerous and traumatic for them, because \hr{Malcur is going mad}{\Malcur is going mad}. 

Rian and Moro investigate and begin to uncover the city's dark secrets. 

When Moro is out in the city, she notices that \hr{Malcur is going mad}{\Malcur is going mad}. 

Finally they find the Sentinel hideout where the ritual takes place. 
They go there to attack the ritual and try to stop them.

At the very same time, Needle also attacks.
Moro and Rian see Needle and the wicked \banes she commands. 
They are both horrified\dash{}especially Moro, because she knows what \banes are. 
They do not know that Needle is there to do the same thing as they, so they assume she is their enemy (after all, she commands the \banes).
So they decide to try to kill her. 





\subsubsection{\Banes wreak havoc}
The \banes wreak havoc on the Sentinel camp. 
They slaughter many Sentinels. 

It was an unforeseen development.
\Psyrex had not expected the Cabalists to loose \banes. 
And these \lesserbanes are small enough to be difficult to detect for a mage, but still deadly enough to be a great menace. 
\Psyrex fears he will have to go in himself and fight the \banes. 





\subsubsection{Rian and \Cobrel{} kill Needle}
\target{Needle dies}
Near the end of the story, Needle has managed to sniff out that \Tiroco{} is working for the Sentinels, and she is tracking her moves. 
Needle has her Cabalists poised to break in and interrupt the Sentinel ritual\dash which would FUBAR \ps{\Secherdamon} plans. 

\Psyrex{} suspects the raid and is trying hard to pin it down, but failing. 
Needle is not stupid. 

But Rian and \Cobrel{} are onto Needle. 
Before Needle can launch her final raid, they strike against her. 
\Cobrel{} keeps her (few) defenders (mortal or supernatural?) at bay. 
Rian distracts Needle and her companions by throwing knives at them (he has acquired skill in knife-throwing in his thieving days). 
Then Moro goes in and kills Needle. 

Remember that Rian needs to put his carpentry skills to use! 
    
\hr{Rian is religious}{Make Rian more religious}.
Make sure he prays in every chapter and scene that he is in.
When he sees \Ishnaruchaefir, he prays for deliverance from this great evil.
He prays to be delivered from \Isphet's evil. 
He has lingering existential/religious dread from the day when he saw the dark sorcerer slay the shining god (even though he was Shrouded and does not remember it all). 

In all the Rian chapters, whenever it is appropriate, have references to the \hr{Myths of vanquished monsters}{myths of Iquinian heroes vanquishing inhuman Elder Races and monsters}. 
When he encounters something supernatural, he fears that the wicked Elder monsters will conquer the world. 





\subsubsection{\Banes go out of control}
Now that Needle is dead, the \banes have no one to give them orders. 

\Psyrex arrives in person to fight the \banes. 
They are hard, even for him. 

Moro leads one \bane away and hurts it.
She cannot kill it, but she can keep it at bay and occupy its attention long enough for \Psyrex to deal with the other two \banes and complete their ritual. 
Then, when \Nithdornazsh rises, Moro gets separated from the \bane and manages to finally escape it.
She is convinced it will go elsewhere.
After all, it has no particular reason to want to kill her. 








\subsection{Needle dies}





\subsubsection{Rian is going mad}
Rian is slowly being driven mad by the things he has witnessed. 

\lyricslimbonicart{The Yawning Abyss of Madness}{
  Behind the sealed door to imagination \\
  I sense the voices of devastation. \\
  Dementia praecox. 
  
  A cascade of dark emotions. \\
  An ominous silence imprisons me\\
  with disfigured landscapes.
}

Remember that Rian now has a religious trauma and fears that there is something wrong with the \sephiroth. 
He tries rationalizing it away, but a lingering dread remains and will return to haunt him throughout the story. 

\citebandsong{DeathspellOmega:FasIteMaledictiinIgnemAeternum}{%
  Deathspell Omega
}{
  The Shrine of Mad Laughter
}{
  The idea of God is pale next to that of perdition, \\
  but of this I could have no inkling in advance.
}






\subsubsection{They sneak up on Needle}
Rian and Moro sneak into the place where Needle is. 
They are afraid of the dark powers around them\dash the \banes{} worst of all. 
They feel the darkness and evil surrounding them, like a palpable force. 

\lyricslimbonicart{A Demonoid Virtue}{
  The night has predatorial eyes,\\
  drifting in a plae of disguise.\\
  Beneath the spelling moon\\
  the spirit rises out of darkness.\\
  The spirit rises\ldots{}
  
  Voices call for my soul.
}





\subsubsection{Why do you do this?}
Before Needle dies, Rian asks her: 
\ta{Why do you do this?}

Needle: 
\ta{For survival. 
  For the future of my people. 
  \emph{Our} people, damn you! 
  For \human kind's survival!}

Just before she dies, Needle's last thoughts are: 
\tho{Great \resphain.
  \Achsah.
  \Teshrial.
  Etc.
  Forgive me for failing you. 
  But I have loved you and served you with all my being.
  Truly I have. 
  
  I have no regrets.
  
  No.
  That's not true. 
  There is something I regret.
  I regret\ldots{}}

And then she dies. 
We never get to hear what she regrets. 

Remember that Rian now has a religious trauma and fears that there is something wrong with the \sephiroth. 
He tries rationalizing it away, but a lingering dread remains and will return to haunt him throughout the story. 

\citebandsong{DeathspellOmega:FasIteMaledictiinIgnemAeternum}{%
  Deathspell Omega
}{
  The Shrine of Mad Laughter
}{
  The idea of God is pale next to that of perdition, \\
  but of this I could have no inkling in advance.
}





\subsubsection{A \bane hunts them}
One of \hr{Needle summons Banes}{the three \banes{} that Needle summoned} comes after Moro and Rian. It is too powerful for them to kill, but Moro has enough magic to be able to hold it at bay and give them time to escape. 

Moro uses the spell \word{\hs{khestni}}. 
It hurts the \bane, but does not kill it. 
It is traumatic for Moro to cast. 
It almost hurts her more than the \bane. 

Whenever Moro draws deep of her terrible magic, she fears that perverse powers might gain control of her. 

\lyricslimbonicart{A Demonoid Virtue}{
  Voices call for my soul.
  
  The cunning serpents kiss I taste.\\
  Baptised beside the ancient takes of\\
  fire, fire burning higher.\\
  Unite with me in dark desires.\\
  I perish in bliss of cruelty.\\
  Tormented souls will never rest in peace.
}

But time and time again she must risk it\dash such as when fending off the \bane. 

\lyricslimbonicart{A Demonoid Virtue}{
  In the flames an omen blaze,\\
  enforcing throught the cosmic haze.\\
  To cross the line and dare to glance,\\
  and enter cold void where death romances\\
  in mysteries.
}

After some time, it stops pursuing and goes after more important targets. Moro is relieved, because she was at her limit and could not hold it off much longer. 











\subsection{Moro's research}






\subsubsection{Astrology}
After having failed to extract much information from \Tiroco, Moro tries to do the best she can with the descriptions \Tiroco{} provided her before storming out. 
So she tries to \hr{Astrology}{read the stars} for \Tiroco, to see if that gains her any insight. 

\hs{Moro has doubts about her astrology}. 
It is hard to read anything about a single person. 
It is somewhat easier to read about an entire city. 
She knows there is something wrong with \Malcur as a whole. 
She has some descriptions and clues. 
She tries to use these and read the stars about them. 

It doesn't entirely work. 
So she contacts \Nasshikerr{} instead. 





\subsubsection{\Nasshikerr}
\target{Moro and Nasshikerr}
Moro \Cobrel{} wanted to find out more. 
She could not puzzle out what was going on. 
So she called on her patron god, \hr{Nasshikerr}{\Nasshikerr}, whom \hr{Moro serves Nasshikerr}{she served}.
He appeared in his guise of an chameleon.  
She asked him for advice and knowledge. 

\Nasshikerr{} was a vain god who saw no reason to help Moro any more than necessary, so she had to plead and beg and bargain. 
Eventually \Nasshikerr{} agreed. 
He could not tell her what was happening right now, but she should call him again later, and he would have something to tell her. 
And he warned her that she had better have a humanoid sacrifice ready next time; her credit was all but spent. 

\Nasshikerr{} does not admit that he doesn't know. 
He gives evasive answers and promises to return. 
But Moro sees through him. 
\tho{Stupid, arrogant god. 
  Just admit that you don't know.}

Afterwards, \Nasshikerr{} sits and wonders. 
Should he inform \Secherdamon? 
Nah. 
He is no close ally of \ps{\Secherdamon}. 
He will play his own game. 

\Nasshikerr{} is surprised at how much Moro knows, though. 
Given how traumatized she is, she should not be able to see this much. 
The Shroud must be \hs{unravelling}. 

\Nasshikerr{} resolves to read the stars some more. 
He is an expert astrologer. 

Remember to read the section about \hr{Nasshikerr}{\Nasshikerr} before writing the chapter!

And remember that \Nasshikerr{} needs to be in the glossary. 
List his race as \quo{\Taortha}. 











\subsection{\ps{\Tiroco} visons get worse}
After the interrupted divination ritual with Moro \Cobrel, \Tiroco{} suffers from an \quo{unfinished gestalt}, and her mind is left in a vulnerable state, unnaturally divorced from the Shroud. 

Her visions start to get really horrible and bloody at this point. She gains the ability to see into the future. (Of course, seeing the true future is impossible no matter what, and her \quo{ability} is exceptionally unreliable.)

\lyricsbs{Bal-Sagoth}{%
  The Splendour of a Thousand Swords Gleaming Beneath the Blazon of the Hyperborean Empire
}{
  The land awash with spilled blood \\
  and viscera torn forth from the sundered dead.\\
  Gorge the Earth with flesh darkened by the claw and fang of war,\\
  rent open to the ravenous maws of worms.
}

\Tiroco{} is traumatized.

She starts to realize that her visions are related to \Icor{} and \Psyrex.

Later it will turn out that it is the nocturnal visitations by \Icor, enabled by \Psyrex, that has slowly opened her mind to the Beyond. 
Made her see through the Shroud a little bit. 
Her encounter with \Uswa{} left her with a lot of images in her mind. 
Dwelling on those memories, her subconscious mind was able to see, in the Beyond, the \emph{true} images that inspired \ps{\Uswa} speech. 
Those images now appear in her dreams. 







\subsection{Rian seeks out \Tiroco}
Rian tries to tell the city guards about the incidents, but they throw him out. He goes to the church, and they call him a heretical madman. 

Eventually he resolves to go directly to the \rinyuth. 

So he sneaks into \ps{\Tiroco} castle, corners her and talks to her about it. 
She realizes that this must not get out, and so, while it pains her to betray a subject, she does what she must: She calls the guards and tells them to throw Rian in prison and not listen to a word the madman says. 

At some point, Needle shows up. Rian freaks out, saying that it's her, that she is one of the evil sorcerers causing the whole thing. Everyone laughs, and the soldiers slap him silly. 

Needle begins to suspect that \Tiroco{} knows more than she lets on, so she begins spying on her mistress. 









\subsection{Moro springs Rian from prison}
Rian is thrown in the dungeon and sentenced to death. 

Before he is to be hung, however, Moro \Cobrel{} sneaks in and frees him. 

He tells her that Needle is one of the bad guys. 
She is intrigued by what he has discovered and agrees to let him work with her. 

Remember that Rian needs to put his carpentry skills to use! 

Moro carries a pistol. 





\subsubsection{They talk about \quiljaaran}
\target{Moro and Rian realize QJ exist}
Moro and Rian exchange their stories. 
They have both seen glimpses of the \hr{QJ in Malcur}{\quiljaaran in \Malcur}, but denied it.
Now they know they both saw them. 
So they must be real. 
This is a relief, but also a horrible discovery.

For Rian, \quiljaaran are a remnant of the \hs{Age of Chaos}.
They are Elder monsters that should by all rights have been \hr{Myths of vanquished monsters}{wiped out by Cordos Vaimon}. 

For Moro, they are a cruel, painful reminder of the horrors that lurked beneath \Yormis. 










\subsection{Rian sees vision of soul prison}
Rian asks Moro \Cobrel{} for magical help in finding Neina. Moro fears that Neina is dead and cannot bring herself to show the boy his girlfriend's horrible fate, so she refuses at first. But Rian pleads and begs, and at last Moro relents. 

So she uses her necromantic magic to search for Neina. She warns him first that this spell is not highly reliable and might not show anything useful. 

How does the spell work? What kind of divination? Astrology? Aquamancy? Pyromancy? 

After looking for her in the Shroud and the \Wylde{} (which can be frightening enough) to no avail, Moro expands her search to the soul prison of the \Sephiroth{}. Rian is horrified by what he sees; souls chained, tortured and crying out in pain. 

Rian somehow recognizes the feeling of the \Sephiroth. See, he is a very religious young man. A kind priest and a lot of religious discipline were highly instrumental in bringing him out of his life of crime and converting him into a respectable citizen. Bottom line, Rian prays a lot and knows the Light and the \Sephiroth{} well. So he recognizes their feeling in the soul prison. He is horrified. He tries rationalizing it away, but a lingering dread remains and will return to haunt him throughout the story. (Maybe at the end he dies and goes to the wicked \Sephiroth?) 

However, all is not bleak. He and Moro were unable to find Neina in the soul prison, and that is good news. But where is she then? Rian assumes that means she is alive, and is relieved. Moro is less optimistic, and though she tries to hide it from him, he picks up on her unease and is in turn made uneasy. 

He walks away with twin feelings of fear: \tho{What was that Hell I witnessed? And what was that feeling? No, it cannot be\ldots{} No, I will not think on it!}

\tho{Neina. Be alive. Don't be dead. Please be alive. Please, \Sephiroth, let her be alive.}

\tho{The \Sephiroth\ldots{} oh, Light\ldots{}}





\subsubsection{Moro's motivation}
Moro knows that something is seriously wrong in \Malcur, and she believes that the Rian/Neina case may be a lead. 
She intends to follow this lead and hopefully find some information on what is going on. 

She helps him search the Realms for Neina, but she is afraid she might lose herself in the process. 

\lyricslimbonicart{A Void of Lifeless Dreams}{
  I close my eyes and transcend.\\
  beyond the light to a dark world without end.\\
  The spirit escapes the temple of flesh,\\
  as seductive winds of madness\\
  absorbing me into the cryogenic system.
  
  A void of lifeless dreams.
  
  Enter a galactic domain,\\
  frozen in time and space.\\
  A new infinity of serenity.\\
  Interstellar voyage.\\
  The spirit escapes the temple of flesh\\
  into the sphere of mystified gloom.\\
  A cosmic funeral of memories. 
  
  Sarcophagus panorama.
  
  A mania to explore the enigma.\\
  Isolated celestial corpse.\\
  Astral embryonic life form.\\
  Invocation of the dormant realms.
}
  
  
  






\subsection{Rian and Moro interrogate a thug}
\target{Rian and Moro interrogate a thug}
Moro and Rian go out in the city.
They hope to find a Black Plague gangster whom they can capture and interrogate.
It takes several nights of trying before they find some. 


They drag the gangster into a nearby alley. 
The gangster is a sphyle. 
They interrogate her. 
Moro uses some subtle magic to make the gangster talk. 

From the gangster they learn where the abductees are kept and where the Plague have their hideout.
At least one hideout. 
There are others, but this thug does not know where they are.
She is only a low-lewel grunt. 
She is not in on the secrets of how the organization operates. 

The Black Plague thug does not know what the big plan is about. 
He only knows about \quo{\hr{The Change of Malcur}{the Change}}. 

Rian wants to storm in and rescue Neina now that they know where she is, or at least have a good idea where. 
Moro urges caution. 
Moro is not satisfied. 
The thug did not really know anything. 
They have learned an address and the descriptions of some people.
But they have not learned details nor overview of the Black Plague's plans. 

Moro wishes \Nasshikerr would come back to her with more information.
Later \hr{Nasshikerr returns to Moro}{he does}. 















\section{The Fall of Pelidor}









\subsection{Extreme multiprogramming}
When I get to the climactic scenes, there is a technique I need to remember to use: 
\emph{Change POV often}. 
It is employed in \cite{StevenEriksonIanCameronEsslemont:MalazanBookoftheFallen}, and to great and terrific effect. 
It prolongs the tension and draws attention to the bigness and epicness of the whole thing. 

All these things should happen more-or-less in parallel:
\begin{itemize}
  \item The fall of \Forclin. 
  \item The duel between \Teshrial{} and \Ishnaruchaefir. 
  \item Moro and Rian's underground work in \Malcur. 
  \item \Takestsha-tachi's spellcasting in \Forclin. 
  \item \Psyrex-tachi's spellcasting in \Malcur. 
  \item Moro's assassination of Needle. 
  \item Rian's rescuing Neina.  
  \item The battle for the Ghost Tower. 
\end{itemize}

End chapters on a cliffhanger! 
All over the place.









\subsection{Various people pick up on the \Sephiroth}
Gradually throughout the story, more than one person gradually comes to suspect the \Sephiroth{} and unravel the truth about them. 
Rian is obviously one. 
\Tiroco{} may be another. 

At the end, strong hints will have been thrown that the \Sephiroth{} are not what they are made out to be. We still don't know much, tho, other than the fact that they are connected with the soul prison. 

Remember to have loads of prayers and religious references, where the \Sephiroth{} are described as good, indeed, as the source of all good. 

\citebandsong{DeathspellOmega:FasIteMaledictiinIgnemAeternum}{%
  Deathspell Omega
}{
  The Shrine of Mad Laughter
}{
  The idea of God is pale next to that of perdition, \\
  but of this I could have no inkling in advance.
}

\paragraph{But:} 
The \sephiroth should \emph{not} be exposed as evil in the first book. 
They should only be mildly suspected.
Overall they should be portrayed as pure and good, albeit not all powerful. 

\Psyrex tells someone: 
\ta{Your \sephiroth cannot save you, no matter how much they might \emph{love} you.
  Not any longer.
  It is too late for salvation for \Malcur.}









\subsection[Locar Psyrex]{\LocarPsyrex}
In the notes below I have three scenes with \Psyrex: 
One with \Vizsherioch, one with \Nzessuacrith and one with \Ishnaruchaefir. 

Perhaps I should merge them all into one, having just \Nzessuacrith.
And maybe \Vizsherioch, felt as a distant telepathic presence. 





\subsubsection[Psyrex and Vizsherioch]{\Psyrex and \Vizsherioch}
\target{Vizsherioch and Nithdornazsh}
\index{Dagger, the}%
The summoning of \Nithdornazsh{} is part of \ps{\Secherdamon} plan to bring \hr{Vizsherioch}{\Vizsherioch} into Ascendancy. 
\Nithdornazsh{} is to become \ps{\Vizsherioch} citadel, a \nexus{} from which he can grow strong and spread his tendrils (politically and metaphysically) into the Realm of the Shroud. 
This is a vital step in the forging of the \hs{Dagger}. 
When the \Nithdornazsh{} project is complete, \Vizsherioch{} is more Dagger-y than ever. 

Previously, \Secherdamon{} had kept \Vizsherioch{} sequestered and hidden. 
He is his only son and the fruit of thousands of years of hard work, so \Secherdamon{} is very protective and does not want to lose him. 

\Vizsherioch approaches \LocarPsyrex.
\Vizsherioch appears as a \dax in his prime, with pearly white scales, wearing a loose robe of white, silver and gold. 

\Psyrex fears him. 
Where \Secherdamon is fiery bright, his son \Vizsherioch is dark and sinister. 
Not in \colour, but in feel. 
A vast darkness follows behind him and around him. 

\Psyrex fears to look into his eyes. 
\ps{\Secherdamon} eyes are terrible enough, but \Psyrex is used to them. 
There is passion, fervour and desire in the eyes of \Secherdamon, and anger and hate, too. 
But in \ps{\Vizsherioch} eyes there are hints of otherworldly madness. 

\Vizsherioch asks \Psyrex about his progress. 
\Psyrex tells him that there have been some setbacks. 
The Cabalists are sneaky.
He does not know who leads them, now that Charcoal is gone from the city. 
But whoever is in charge must be someone capable. 
They have managed to screw up some of his operations and kill some of his important Sentinels. 
But it is not so bad. 
He has planned for some amount of Cabal interference and taken precautions. 
He is importing more manpower. 
He will be ready on time. 

\begin{prose}
  \Vizsherioch: 
  \ta{The beacons are in place? Show me.}
  
  \Psyrex shows him. 
  \Vizsherioch sees the slender aethereal tendrils, grown from the Pyre \matrix.
  They reach up from \Nithdornazsh to twist around the ley lines and converge upon the \nexus point in Pelidor, where they grab on and hold fast, holding the \nexus in a constricting iron grip. 
  He sees the \matrix subtly reaching out to fasten upon the souls of the mortals in the city. 
  Binding them.
  They will be part of the invocation, he thinks. 
  
  \Vizsherioch: 
  \ta{Aye, I see it.
    You have done well, \Psyrex.}
  He smiles to himself.
  \ta{My father has confined me to our Realms for too long.
    I long to at last set foot in my new citadel.
    And to exert my power in \Azmith; supposedly the most pivotal of the Shrouded Realms.}
  
  \Vizsherioch becomes distant.
  \ta{The constellations are falling into place.
    I can feel the tension in the Pyre. 
    The Dagger is taking shape.
    Very soon now\ldots{}}
  He becomes present again.
  \ta{%
    What of the \resphain?}
  
  \Psyrex:
  \ta{I have detected some \resphan activity in both \Malcur and \Forclin.
    I still have every reason to believe they will swallow the bait.
    But of course, it will depend on \Nzessuacrith and her task.}
  
  \Vizsherioch:
  \ta{She will not fail us.}
  
  \Psyrex:
  \ta{Yes.
    It is not \Nzessuacrith nor the \resphain that make me uneasy.}
  \Psyrex pauses and hesitates. 
  
  \Vizsherioch: 
  \ta{You mean the Exile.}
  
  \Psyrex:
  \ta{Yes. He has been uncharacteristically\ldots{} \emph{active} recently.
    In the Pelidor region.}
  
  \Vizsherioch:
  \ta{But he has not antagonized us?}
  
  \Psyrex:
  \ta{Not that I can determine. 
    And that worries me.
    So far he seems to have acted only against the \resphain, but I cannot guess his motives.
    The Exile cannot be trusted.}
  \Psyrex looks at his star-charts. 
  Concentrates to shift his vision. 
  Looks up, bypassing the roof, at the real stars. 
  Stares at the star representing the Exile. 
  It is nowhere near the Pyre. 
  Nor any other known \matrix. 
  He is made uneasy by the thought of the rogue \vertex. 
  A wanderer in darkness who can appear anywhere at any time. 
  
  \Vizsherioch:
  \ta{\QuessanthIshnaruchaefir.}
  \Psyrex is taken aback. 
  He had never heard \Vizsherioch speak the Exile's name before. 
  He thought \Vizsherioch shunned the name like his father did.
  
  \Vizsherioch: 
  \ta{Called Exile and Destroyer.}
  To \Psyrex: 
  \ta{You fear him.}
  
  \Psyrex:
  \ta{Yes, I fear him. 
    Any creature lesser than a \dragon has cause to fear the Exile.}
  \tho{And many a \dragon should fear him, too.}
  
  \Vizsherioch:
  \ta{Hm.
    He is in the Pelidor region.
    So it is possible he has caught wind of our doings.
    He must not be allowed to interfere.}
  
  \Psyrex:
  \ta{The \resphain feel likewise.
    Have you heard, Lord \Vizsherioch, about this \resphan, \Teshrial, who talks about confronting the Exile?}
  
  \Vizsherioch:
  \ta{Yes. 
    Allegedly the Exile has promised to face him.}
  
  \Psyrex:
  \ta{I wonder what will come of this challenge. 
    Will the Exile really return to meet the \resphan?
    I do not think \Teshrial is a fool.
    He has a plan.
    But will it be enough?}
  
  \Vizsherioch:
  \ta{Interesting prospect, this duel.}
  He becomes distant.
  \ta{Perhaps I should seek out \Ishnaruchaefir.
    After all, I have never met my uncle face to face\ldots{}}
  
  \Psyrex tries to imagine such a confrontation.
  Would it end in violence?
  \Vizsherioch was powerful, but young. 
  Would he be able to stand before \Ishnaruchaefir?
  
  \Vizsherioch reads his thoughts.
  \ta{Fear not, \Psyrex.
    I will not challenge the Exile to single combat.}
  Distant.
  \ta{No, that would be wise.
    Not at this time.
    Not at this time\ldots{}}
\end{prose}





\subsubsection[Nzessuacrith visits Psyrex]{\Nzessuacrith{} visits \Psyrex}
Remember to have scenes where \Nzessuacrith{} visits \Psyrex{} at his Dark Crescent throne. 

Remove any direct references to the \Malcur \nexus. 
Just refer to \quo{the \nexus{} in Pelidor}. 
We don't want to tell the reader that \Forclin{} is a decoy. 





\subsubsection{\Ishnaruchaefir visits \Psyrex}
Then he goes to see \Psyrex. 
% Maybe have a scene where \Ishnaruchaefir{} approaches \Psyrex{} in \Malcur. 
Like the scenes in \cite{StevenErikson:GardensoftheMoon} where Anomander Rake visits Baruk in Darujhistan. 

Remove any direct references to the \Malcur \nexus. 
Just refer to \quo{the \nexus{} in Pelidor}. 
We don't want to tell the reader that \Forclin{} is a decoy. 

\Psyrex{} is not happy that \Ishnaruchaefir{} can simply barge in, bypassing his guards and spells. 

\begin{prose}
  \Ishnaruchaefir: 
  \ta{So. I see you are attempting to resurrect \Nithdornazsh.}
  
  \Psyrex: 
  \ta{Yes. Do not try to stop us!}
  
  \Ishnaruchaefir: 
  \ta{I would not dream of it. Are you aware of the \noggyaleth?}
  
  \Psyrex: 
  \ta{Of course. Here is how we plan to deal with them\ldots{}}
\end{prose}


\Psyrex{} tells how the resurrection of \Nithdornazsh{} is a \quo{pivotal step, a decisive battle in the eternal war. One step on the long ladder of bringing the \matrix{} of our kind to ascendancy}.  

\Psyrex{} berates \Ishnaruchaefir{} for being too obvious and attracting unwanted attention. 
He feels somewhat weird yelling at an immortal who is close being the equal of his Exalted Lord, \Secherdamon. 
But \Psyrex{} is proud and knows his worth, so he sees himself as almost \ps{\Ishnaruchaefir} equal. 
So he dares treat him like one. 
Still, he calls him \quo{\Ishnaruchaefir} and not \quo{Exile}. 
\Psyrex{} is not confident enough to refuse him his name. 

Make it clear how arrogant, dominant and strong \QuessanthIshnaruchaefir{} is. 
\Psyrex{} is proud to have stood his ground. 
\Ishnaruchaefir{} tried to pry his plan out of him.
\Psyrex{} resisted, but \ps{\Ishnaruchaefir} frame was super-strong, and he was forced to give up much.
But he did not reveal as much as he could have done. 
He has not totally lost, and he takes pride in that. 
Few can stand against such force of personality, and \Psyrex, for all his skill and wisdom, is only a \scatha. 
He rarely admits that, but in this hour, in a room with the dreaded Exile, he feels the truth of it.
Immortal though he may be, he is only a \scatha. 

Compare \Ishnaruchaefir{} to Lord Asriel in \cite{PhillipPullman:NorthernLights} when he twists Iofur Raknisson around his little finger. 





\subsubsection{The Blood-Red Vaults of \Nithdornazsh}
\target{Blood-Red Vaults of Nith'dornazsh}
Perhaps they look at the underworld beneath \Malcur. 
Like the {haunted dungeons beneath \CastlePelidor}, the underworld is a place where the Shroud is thin. 
It is a gateway to the Blood-Red Vaults of \Nithdornazsh. 

\lyricsbalsagoth{Beneath the Crimson Vaults of Cydonia}{
  Ancient and divine, older than the hidden Icosahedron, \\
  now rebirthed beyond the chaosphere.\\
  Rise\ldots{} rise and destroy!
  
  Hatred, carnage, slaughter, havoc, chaos, murder!\\
  I am become the devourer of all life!
  
  Phobos, Deimos! \\
  The moons' rays liquefied in these blood red pyramids.\\
  In the shrines of abomination, black tongues rapt with blasphemy.\\
  Chaosphere, watchtowers, genesis, Cydonia\ldots{}\\
  The Abyss yawns wide!\\
  Spirit of the carrion-thronged battlefield, open wide thy gate!
  
  Unruly evil!\\
  Colossal shapes etched against the moons, \\
  supine obeisance 'fore the mound,\\
  Accursed fiends, hail the Slitherer, \\
  abhorrent jaws drooling lunacy.
}





\subsubsection[Psyrex fears Ishnaruchaefir]{\Psyrex{} fears \Ishnaruchaefir} 
{\Psyrex} fears \Ishnaruchaefir. 
He uses divination and discerns signs that \Ishnaruchaefir{} is active as a \vertex, and also near him. 
He frets, afraid of what the enigmatic immortal might do.

Remove any direct references to the \Malcur \nexus. 
Just refer to \quo{the \nexus{} in Pelidor}. 
We don't want to tell the reader that \Forclin{} is a decoy. 









\subsection{\Teshrial's trap}
\Teshrial and \Urizeth go to \Malcur to plan and set up a trap for \Ishnaruchaefir. 
Maybe it was \Azraid who suggested this. 





\subsubsection{\Teshrial{} sees \Ishnaruchaefir{} as the worst threat}
Have a scene with \Teshrial{} where he contemplates potential threats to \Malcur. 
He sees \Ishnaruchaefir{} as the worst threat, because you never know what that meddling rogue immortal might do. 
\Secherdamon{} is less of a threat, because there is no way he would enter combat personally (it's a trauma he has from back when his brother was killed by \resphain). 
But he fears \Ishnaruchaefir{} and takes precautions for his return. 





\subsubsection{Looks at his \humans}
\Teshrial{} looks at some of home of his home-bred \humans. 
He caresses a beautiful youth with his feathers. 
\Teshrial{} loves his \humans{} and will not let the evil \Ishnaruchaefir{} harm them. 
He was shocked to learn that \Ishnaruchaefir{} would be so cruel as to let cute, defenseless \humans{} suffer for his wrath. 
This makes \Teshrial{} hate him.
He must die. 





\subsubsection{\Teshrial and \Urizeth inspect \noggyaleth}
\Teshrial and \Urizeth inspect their secret weapon: 
The \noggyaleth{} hiding beneath \Malcur. 
See section \ref{Teshrial's creatures}. 

\Teshrial \hr{Teshrial fears Noggyaleth}{is afraid of the \noggyaleth}. 


\Teshrial{} is overconfident. After \ps{\Ishnaruchaefir} initial attack on \Malcur (see section \ref{Ishnaruchaefir attacks Teshrial's creature}), \Teshrial{} was left with the impression that \Ishnaruchaefir{} believes that there was only one \noggyal{}, and that the way to \Malcur is now open. But \Teshrial{} has several more, and he believes that \Ishnaruchaefir{} doesn't know about them. 

\Teshrial{} is confident in their ability to defend the city. Maybe he knows that \Ishnaruchaefir{} knows that he has one of the \noggyaleth{}, but he thinks the rest are hidden. But \Ishnaruchaefir{} sees through him. He knows that \Teshrial{} must have several more \noggyaleth{}. 

Perhaps he needs to perform an occult ritual to unchain the \noggyaleth{}. 
%Perhaps he must needs invoke powers greater than he\dash\resphan{} kings, or the \banelords, or even \Voidbringer.
He needs to invoke greater powers: 
His people's ancient pacts with the \banelords{} and their unholy might.

\lyricsbalsagoth{%
  In the Raven-Haunted Forests of Darkenhold, Where Shadows Reign and the Hues of Sunlight Never Dance
}{
  I stand now at the anvil,\\
  adamantine hammer in my hand.\\
  In thunder-song the steel I smite,\\
  a clarion heard throughout this land.
  
  Ablaze upon the Altar of Stone,\\
  the Sigil of An-rayuth, the summoning.\\
  Folk of the Mist, Dwellers in Shadow,\\
  the thrice-blessed wand of the Wood-Gods is beckoning.
  
  At the aeon-swathed Shrine of the Oak I kneel.\\
  O' Oracle of the Great Forest, hear me this night\ldots{}
}

Maybe this chapter should be called \quo{The Conqueror Worm}. 
Or some other chapter.

\lyricsbalsagoth{%
  In the Raven-Haunted Forests of Darkenhold, Where Shadows Reign and the Hues of Sunlight Never Dance
}{
  Swaying serpents ring my oak-hewn throne,\\
  Night and Shadow are my hunting dogs.\\
  Ravenous, they howl to be unshackled,\\
  that their maws may be glutted \\
  (with the blood of my foes).
}









\subsection{\Ishnaruchaefir contacts \Secherdamon}
\target{Ishnaruchaefir tells Secherdamon of the Ghobaleth}
Read about \hs{Chaos magic}, and remember to invoke \Sethicus and \Tiamat. 

\Ishnaruchaefir{} contacts \Secherdamon{} and tells him about what he is doing. 

% Or maybe not \Secherdamon.
% Maybe only \LocarPsyrex. 
% Maybe he and \Secherdamon{} are not on speaking terms, so \Psyrex{} has to mediate. 
\Ishnaruchaefir visits {\LocarPsyrex} in his inner sanctum at the Dark Crescent. 
\Psyrex is distressed at how easily \Ishnaruchaefir broke in. 

\Ishnaruchaefir is not physically there. 
But he is close.
He is projecting his presence into \Psyrex's mind. 
\Psyrex feels a mammoth presence.

\Psyrex first calls \Ishnaruchaefir \quo{Exile}. 
But later he crumbles under the \ps{\dragonlord} piercing stare and starts calling him by his real name. 

He tells \Psyrex{} of the \hr{Teshrial's creatures}{\noggyaleth}. 
They are a greater threat than \Psyrex{} realized, and can seriously fuck up his plans. 

\Psyrex{} curses at him for not having told him this before.

\Ishnaruchaefir{} bargains. 
He is willing to help \Psyrex, but he wants something in return. 
(I don't know what yet.) 
\Ishnaruchaefir{} will lead \Teshrial{} and his \noggyaleth{} away from the city. 
He has planted an obsession in \ps{\Teshrial} mind and manipulated him with his false myths, so he believes \Teshrial{} will be obsessed enough to leave his precious city semi-unguarded and rush off to face \Ishnaruchaefir. 
But due to the nature of the myths, on which \Teshrial{} relies, this must be done at a strategically exact point in time. 

\Psyrex{} listens while \Ishnaruchaefir{} describes his plan. 
(Offscreen, of course. 
 \trope{UnspokenPlanGuarantee}{Unspoken Plan Guarantee} and all that.)
Then:
\begin{prose}
  \Psyrex: \ta{This is madness!} 
  
  \Ishnaruchaefir: 
  \ta{
    My last gambit\dash the one for which I am now infamous\dash was also madness. 
    I will do this.
    I will do this favour to my race.
    To \Nexagglachel.
    Yes, even to \Secherdamon.
    And when I do, you will stand ready.}
\end{prose}

Remember to stress how dangerous this is for \Ishnaruchaefir.
\Psyrex knows about the Nadir. 
He knows \Ishnaruchaefir will be at his very weakest. 
Ostensibly he will only be fighting a single \resphan, but \Psyrex is too smart to underestimate the \resphain.
\Ishnaruchaefir will be going into a trap, and he knows it. 

\Ishnaruchaefir{} cannot take on all of \ps{\Teshrial} \noggyaleth{} on his own. 
Therefore, the showdown must be planned to coincide with the ritual summoning \Nithdornazsh. 
This will confuse the \noggyaleth, and they will be forced to struggle against \Nithdornazsh, which will weaken them and slow them down. 

This means \Psyrex{} must speed up his plans. 

\Psyrex{} curses some more, then ultimately agrees, knowing that it is too late to do anything else. 

This scene is inspired by the conversation between Ganoes Paran and Shadowthrone in the middle of \cite{StevenErikson:TheBonehunters}. 

Later in the scene, \Secherdamon appears, in vision form if not in physical form. 
He curses the Exile, but ultimately agrees to his demands. 

\Criseis is with \Ishnaruchaefir. 
She greets \Psyrex. 
They call each other \quo{cousin}. 

Remove any direct references to the \Malcur \nexus. 
Just refer to \quo{the \nexus{} in Pelidor}. 
We don't want to tell the reader that \Forclin{} is a decoy. 





\subsubsection{\Psyrex tells \Nzessuacrith to hurry}
\target{Psyrex tells Nzessuacrith to capture Forklin quickly}
\Psyrex passes on the message to \Nzessuacrith, telling her to hurry up and \quo{capture} the Ghost Tower already. 
He knows that \Achsah{} is resourceful and might salvage everything even if \Teshrial{} falls.
He fears \Achsah more than he fears \Teshrial, actually. 

\Psyrex also relays to \Nzessuacrith a message from \Secherdamon.
\Secherdamon has said that if it becomes necessary, \Nzessuacrith should not hesitate to break the Unspoken Covenant.
Blatantly.

\Nzessuacrith smiles. 
She understands very well what the last part means.
It means that if need be, she should transform to \draconian form. 





\subsubsection{Nadir begins}
\hr{Ishnaruchaefir's Nadir}{\ps{\Ishnaruchaefir} Nadir} is coming, and he feels its onset. 
He prepares for it. 
Sets \Rystessakhin{} in a special place where she can \quo{recharge}. 

\Ishnaruchaefir \hr{Ishnaruchaefir bleeds in Nadir}{bleeds and looks terrible} when he is in the Nadir. 

Make clear that this is of vital importance for the Shroud and to protect \Miith{} from alien menaces (the \banes). 
This makes \Ishnaruchaefir{} dreadfully vulnerable, and \Criseis{} and the grandchildren know that he runs a terrible risk by going into battle at such a time. 

But to \Ishnaruchaefir, it is worth it. 
The resurrection of \Nithdornazsh{} would be a monumental victory for the Sentinels. 
He does not tell the people around him about his plan, of course. 

He leaves instructions with \Criseis{} and \Thiencaste-tachi about what to do if he dies. 
How to take care of \Rystessakhin{} and all that. 

\Criseis{} is very worried when she sees him put down \Rystessakhin. 
\hr{Nadirs get worse}{His Nadirs are getting worse} every time. 
Something must be profoundly wrong with the Shroud. 
She has tried to pry out of him what is wrong, but he is not talking much. 
When he puts down the glaive he stands tall and arrogant and does not let anything show. 
But \Criseis{} has served him for ten thousand years and knows him better than any other. 
She can detect all the little telltale signs: 
the way he hesitates for the briefest moment; the way he stands up a few millimetres taller afterwards when he no longer carries the glaive's burden. 
It all hints of a terrible, excuciating pain that would be unendurable to any lesser person\ldots{} even a lesser \dragon. 

She feels his pain. 
It is unfair, she thinks, that he has to carry this terrible burden and danger, and get nothing but scorn in return from the world. 
And she fears for the future. 
\hr{Nadirs get worse}{The Nadirs are getting steadily worse}. 
What if some day the burden gets so heavy that even he can no longer shoulder it? 

Before his great duel, \Ishnaruchaefir{} draws some energy from \Rystessakhin \dash as much as he dares\dash and uses it to empower his \hs{ward runes}. 
(Remember to read about \hs{ward runes}.)

Maybe \Criseis does not see \Ishnaruchaefir up close. 
She fears to get close to him.
He is radiating powerful and dangerous energy.
A lash with one of those whirling energy threads might kill her.
So she stays far away.
She contacts him with telepathy. 
(This way, the reader also only sees \Ishnaruchaefir as a blur.)

\Criseis warns him:

\begin{prose}
  \Criseis:
  \ta{\Teshrial{} is well-prepared this time. 
    He has studied your strengths and weaknesses.
    I believe he has studied \WanderersInDarknessEmph.}
  
  \Ishnaruchaefir:
  \ta{Has he now?}
  (Smug, mysterious smile.
  He knows \Teshrial{} has taken the bait and fallen for the story of his alleged Achilles Heel.)
  
  \Criseis:
  \ta{Do not do this, master!
    I beg you.
    It is obviously a trap.
    And you are at your weakest.
    You may fall!}
  
  \Ishnaruchaefir{} (looking wistful, contemplating the possibility that he might die):
  \ta{I might.
    But know this, \Criseis:
    This battle will be of pivotal importance.
    This I predict.
    A mighty storm is brewing on the Pelidorian horizon.
    This storm will herald a \thirdbanewar.
    And that is a war I intend for my race to win.
    It is a risk, but I am willing to take it.
    For \Nexagglachel.
    For our people.
    Even\ldots{} even for \Secherdamon, perhaps.}
\end{prose}





\subsubsection{\Secherdamon{} thinks}
\target{Secherdamon thinks Ishnaruchaefir will sacrifice himself}
Shortly before \ps{\Ishnaruchaefir} duel is scheduled, have a scene with \Secherdamon. 
He ponders his brother's involvement, and what \Ishnaruchaefir{} intends. 
Then he realizes what is about to happen. 

\begin{prose}
  \Secherdamon;
  \tho{Fuck.
    \Ishnaruchaefir{} is going into battle.
    Now.
    Right in the middle of his deep Nadir.
    (\Secherdamon{} knows about the Nadir\dash superficially, at least.)
    
    And he is walking straight into \ps{\Teshrial} trap!
    Surely he must realize this. 
    What is he doing?
    He might actually be in danger!
    
    Why is he doing this?
    And he is helping me.
    Why?
    
    Does he intend, at last, to sacrifice his life to pay for his crimes?
    That cannot be.}
\end{prose}

For a short moment, this line of thought makes \Secherdamon{} respect \Ishnaruchaefir{}. 
He thinks and visualizes \ps{\Ishnaruchaefir} name, something he has not done for many centuries. 












\subsection{\Forclin falls}
The war lasts about a year. Eventually, Runger escalates the war by bringing in more and more mages and supernatural aid. 
The Imetrians begin to suspect that Runger has allied with Durcac and gained the aid of Rissitic mages and troops (which is true). 
The Imetrians are not certain, but these suspicions sway them to finally send some aid to Pelidor. 

The mages draw so much magical power that they \hr{Magic overdose}{overdose and damage their bodies}. 
Including Carzain and Curwen. 
And worst of all the Rungeran mages (all but \Takestsha). 





\subsubsection{Push comes to shove}
The Pelidorians are holed up in \Forclin. 
It is a strong city, and they believe they can hold off the Rungerans for long enough to hopefully summon allies from neighbouring kingdoms, or perhaps the Imetrium. 
As long as they can keep the Rungerans out, they have a chance. 

\Takestsha{} was also content to keep the siege up for a while. 
But then she receives news from \LocarPsyrex. 
He tells her that she will have to speed up her plans. 
She curses, but agrees. 
So instead of just waiting and holding the city besieged, she now forces Morgan to go all-out on the offensive and make every effort to take the city. 





\subsubsection{Rungeran reinforcements}
\Takestsha{} has lost a few mages, so she calls in some more \quo{recruits} to the \ishrah. 
These are actually Rissitic mages, disguised as rogues. 
Morgan knows this, but the rest of the \ishrah{} and army doesn't, although they may suspect. 

They have also brought on some companies of mercenaries, which are Rissitic soldiers or \Durcaci{} tribesmen in disguise. 
These mercenaries bring in big, frightening monsters: 
Stegosaurs or ankylosaurs or sauropods. 
And a few of the warriors are \cregorrs!
Those guys are fucking wicked. 

Ilcas observes the new recruits to the Rungeran army. 
No one knows the mages are new (and they don't cast Rissitic magic, so they are difficult to pinpoint as being odd-men-out). 
But scouts have seen the new mercenary bands arrive. 
They are also noticeably different from the rest. 
It is no secret that they are not native Rungerans. 
Ilcas suspects that they are \Durcaci{} tribesmen. 
And if they are \Durcaci, then it is likely that they are also Rissitic. 
And this fits nicely with the theory that the Rungerans are allied with the Rissitics. 
Ilcas goes to tell Sethgal about it. 
He also does his best to send a message to the Imetrium and inform them that there is almost definitely Rissitic involvement. 

This is true, of course. 
The tribesmen \emph{are} Rissitic, or at least Rissitic-allied. 
They were only brought to the battlefield at this late hour, because \Takestsha{} did not want everyone to know that Runger was in league with \Durcac{} too soon. 





\subsubsection{Carzain}
Near the end of the war, Carzain is stationed in \Forclin{}, a strategically important city in the northern Pelidor. There is a lengthy siege and the Rungerans bring in Rissitic reinforcements. 

Also, Redcor \Matron{} \Esmerel{} is there. \Esmerel{} suspects that a Scion has been incarnated and dwells in Pelidor. The Redcor have determined this via astrology or something. \Esmerel{} comes to investigate, together with \Racel{} and two Gandierre, \France{} \Perival{} and \Isacc{} \Chiran. \Racel{} is there because she is Pelidorian\ldots{} I think. 
\Esmerel{} investigates and ultimately determines that Carzain is the Scion. 

They fight bravely, but ultimately the city falls.





\subsubsection{Carzain must be awesome}
Carzain should be brave and iron-willed in the face of adversity, like Lucian from \cite{Movie:UnderworldIII}. 
Maybe he should take over leadership of the Pelidorian army after Sethgal is killed. 
They still lose the war, but he saves many of them. 
Or something.





\subsubsection{Carzain-tachi fight for survival}
After the fall of Forklin, Carzain ends up together with Telcastora Ilcas and a few other Imetrians. 
Carzain also has a few of his soldier buddies along. 
Then they meet up with \Esmerel{} and her Redcor. 
They slink around and try not to be killed by the Rungeran forces, but they are forced into a fight, and several of them are killed. 
Only Carzain, Ilcas, \Esmerel{} and \Racel{} survive. 
They slink around some more and realize that \Forclin{} is lost. 





\subsubsection{Carzain and Delph discuss archetypes}
Carzain says to Delph: 
\ta{Y'know, this is almost like a scene from the legends.
  With the heroes standing in the rubble of the fallen city and swearing revenge and all.}

Delph:
\ta{I hope we're not in a legend, 'cause if we are, you're doomed.
  'Cause I'm the hero, and you're the hero's faithful companion. 
  And the companion always dies before the end.}

At the end, he is dying and thinks: 
\tho{I can't die.
  I'm not supposed to die.
  Damn you, Carzain. 
  You're the sidekick. 
  You were supposed to die, not me.}





\subsubsection{\Tsekkect{} runs away}
\Tsekkect{} is alone. 
Delph is dead, and she has been separated from Carzain-tachi. 
She thinks:
\tho{Fuck it. I've had it with these stupid nobles and their stupid war.
  I'm going back to my tribe.
  They'll take me back.
  They have to.
  They will.
  I hope.}

This is the last we ever see of her. 
I don't need her, I just want to avert the \trope{SortingAlgorithmOfMortality}{Sorting Algorithm of Mortality} by not killing the nonhuman companion. 





\subsubsection{Carzain-tachi encounters \vorcanths}
\target{Vorcanth help Ramiel}
Have one very mysterious scene where Carzain dreams. 
His traumas and \hr{Ramiel's bound souls}{the bound souls that haunt him} take on physical shape and attack him. 
But \hr{Moon-Wolves help Ramiel in dreams}{white wolves appear and dispel them}, saving him. 
Only to disappear again, casting enigmatic glances at him. 
He senses that he knows them, and feels conflicting feelings of loss, guilt, joy and relief.

When Carzain first sees them, he compares them to wolves. But Vizicar says hyaenas. Carzain hasn't seen a hyaena, but he recognizes them from Vizicar's memories. (Vizicar is more well-\travelled, since he is not only older, but also a king.)

Already in \TwilightAngelRememberEmph, Ramiel is receiving dreams about \vorcanths. 
It is a \hr{Carzain dreams of Moon-Wolves}{distress call from a \vorcanth{} in trouble}. 

Remember to read the section about \hr{Vorcanth}{\vorcanths} before writing the chapter!

Immediately after this scene with the \vorcanth{}, have a scene where some astrologer reads the stars and notices that Visha (associated with the \hr{Vorcanth Matrix}{\vorcanth{} \matrix}) is \hs{intersecting} with the Midnight Bat (the \hs{Mystraacht Matrix}{\Mystraacht{} \matrix}). 
Or maybe this doesn't happen until later, where Ramiel is contacted by a badass \vorcanth{} leader. 

It is important to stress that it is the \vorcanths{} who take the initiative and help Ramiel out first. After that, \emph{he owes them}. 









\subsection{\Teshrial prepares to battle \Ishnaruchaefir}
\Teshrial{} suspects that \Ishnaruchaefir{} will soon return. 
He prepares himself to face him. 

\Teshrial{} is laying a trap for \Ishnaruchaefir. 
He plans to meet him in a place parallel to \Malcur, but farther away from the Shroud (perhaps through the \hs{dead garden} again). 
He has his \noggyaleth{} lying in wait and intends to call them out to ambush \Ishnaruchaefir{} while fighting him. 

\Teshrial{} has several trump cards:
\begin{itemize}
  \item The Shroud.
  \item Astrology. 
  \item \Noggyaleth.
  \item The Achilles Heel. 
\end{itemize}





\subsubsection{\Teshrial refuses ambush}
\target{Teshrial fears to break agreement with Ishnaruchaefir}
The other \resphain tell \Teshrial that they should set a bigger ambush for \Ishnaruchaefir, have some more \resphain lie and wait and attack him. 
\Teshrial very firmly declines and explains his reasons.

\Ishnaruchaefir has promised to fight \Teshrial alone.
If \Teshrial breaks their agreement, \Ishnaruchaefir will refuse to fight and exact a terrible vengeance later. 
They now have a chance to exploit the Destroyer's code of \honour. 
They would be fools to let that go to waste. 

Instead, several powerful \resphain give \Teshrial gifts of magical items for him to use in the battle, including a sword, pistols, \armour, amulets and wing braces.





\subsubsection{\Achsah contacts \Teshrial}
\Achsah contacts \Teshrial. 
She tells him about the situation in \Forclin.
He only listens half-heartedly.
She can tell that he is obsessed with his upcoming duel. 

\Achsah fears for him. 
She tries to dissuade him. 
\Teshrial brushes her off and berates her for her cowardice.

Afterwards, \Teshrial is a bit ashamed. 
He should treat \Achsah better, he reflects. 
She is worried about him.
He should respect that.
He resolves that after the battle he will try to be nicer to his \bezed subordinates.
This is a part of \hr{Teshrial's unfinished business}{\Teshrial's unfinished business}. 




\subsubsection{His brethren fear for him}
Make it clear \ps{\Teshrial} brethren fear for him and want him to succeed. 
\Menessiaraid, his close friend, comes to see him off. 

\Menessiaraid{} gives \Teshrial{} a gift: 
His own \senaan, \hr{Ossiraith}{\Ossiraith}. 
He knows \ps{\Teshrial} favourite weapon, \hr{Turishah}{\Turishah}, was destroyed. 
He entrusts his friend with \Ossiraith. 
\Teshrial{} is \honoured and vows to put it to good use. 

In addition to \Ossiraith, \Teshrial{} carries into battle two \hr{Ghijed}{\ghijedeth}. 
Before going into battle he looks down at his pistols. 
He takes them with him just in case, but he does not expect so much from them. 
\Ishnaruchaefir{} will probably wear \hs{ward runes}, and as such will be protected against guns. 
Guns are not so great in single combat with \dragons{} overall.
They are good if you can outnumber or out\manoeuvre your enemy. 
But in a fight like this, \melee{} is the best way to go. 

\Teshrial dresses for the battle so as to match \Ossiraith. 
Read about \hr{Ossiraith}{\Ossiraith} and dress \Teshrial in the same \colour. 

\Urizeth also fears for him.
She is sad that she will be nowhere near him during the battle.
She feels a bit cowardly for it.
\Teshrial is also warming towards her.
This is part of his personal development and \hr{Teshrial's unfinished business}{his unfinished business}. 





\subsubsection{Stewardship of \Malcur}
\target{Teshrial leaves Bezed in charge}
\Teshrial{} knows that \Malcur is terribly important and that it would be bad if it fell into the Sentinels' hands. 
He must be careful and guard it against \Psyrex-tachi while he himself is off fighting. 
He knows the time of his duel is coming up, so he has contacted his superiors and requested to have more aid sent to him. 
Some more \resphain{} come to relieve him. 

When he feels that \Ishnaruchaefir{} is near, he takes off and leaves some low-ranking \resphain{} in charge of \Malcur. 
At this point \Achsah{} is already in \Forclin. 

\Teshrial{} thinks he is manipulating \Ishnaruchaefir{} and forcing him to face him at the right astrological moment. 
But he is actually being strung along. 

\lyricslimbonicart{In Mourning Mystique}{
  Darkness!\\
  I seek the silence that you bring.
  
  Grant me thy sacred gifts,\\
  bestow my soul thy offerings.\\
  Let the ancient forces of nature rule.\\
  Take my blood as the sacrifice,\\
  a symbolic faithful bond of truth.\\
  I kneel in front of thy altar black.
  
  When you look into an abyss, the abyss also looks into you.
  
  Tonight I enter into obscure dreams.\\
  In darkness shelter, I am unseen.\\
  With the esoteric gifts I possess\\
  I bring damnation by enforcing death.\\
  In the beginning of the storm \\
  I'll come forth.
  
  An arrival into a twilight reverb\\
  as just a shadow of the former self.
  Sorrow is my name.\\
  My true essence is pain
  
  Hear the mourning of the mendacious\\
  from the empty halls and shafts\\
  of false blinding light.\\
  Prepare the last sacrifice (on the altar)\\
  in the temple of decay.
  
  Please spare me from the final agony of shame.
  
  I am evil from the moment of conception.
}





\subsubsection{\Achsah{} wishes him well}
As \Teshrial{} departs, he contacts \Achsah{} telepathically and briefs her. 
She expresses her fears and concern for him. 
I need to paint some measure of friendship between the two. 
She does not like him, but she does not want him to die a painful death, or mutate permanently into a monster. 
She asks him to be careful and wishes him luck. 
He thanks her. 

This is small measure of making-up for them. 
They have never liked one another.
Here at the last minute, they take steps toward each other. 
It should be a bit touching, and tragic in retrospect. 





\subsubsection{\Firaxel{} visits him}
\target{Teshrial's date}
\Teshrial{} is visited by \Firaxel. 
He has specifically asked her to come see him off to his duel. 
He wants to see her one more time before he risks his life and soul. 
She is attracted to him and wants to see him, too. 
They have an intimate, romantic date. 

\hr{Firaxel is a scientist}{\Firaxel{} is a scientist}. 
He learns that she greatly admires people who do something to advance science, especially if they take great risks to do so. 

\begin{prose}
  \Firaxel: 
  \ta{Our race exists for a purpose: 
    To improve ourselves, to bring ourselves closer to perfection.
    And our best and finest tool in this quest is science.}
\end{prose}

\Teshrial{} wants to impress her. 
So he tells her of how he has spoken with cutting-edge researchers and is participating in the \neoresphan{} project and testing a dangerous experimental weapon. 
It is dangerous, but he does it. 
So he is fighting and risking his life not only to prove himself and for the war and rid the world of an evil menace and avenge all the victims of the wicked \Ishnaruchaefir, but also for science and the betterment of their people. 

This turns her on. 
He can see it. 
Her eyes glow with desire. 
Her breasts heave. 
She licks her lips with hunger. 

They \emph{almost} have sex. 
They certainly kiss and grope. 
She promises him sex (in unmistakable terms) when he returns triumphant. 
The best sex of his life. 
And he can believe it. 
\Resviel{} are great lovers. 

He walks away happy. 
His dreams are coming true. 
Seeing her has given him courage. 
Before he was unsure if he dared use the experimental weapon. 
He was planning on holding it back as a desperate last resort, and even so he was unsure if he would have the courage to use it. 
But now he knows how much \Firaxel{} will love him for it. 
Now he dares use it. 
For \ps{\Firaxel} sake he is willing to take that risk. 

Have some from \ps{\Firaxel} POV. 
\Firaxel{} has developed genuine feelings for him and is afraid for him. 
She admires him. 

\begin{prose}
  \tho{What a \resphan. 
    What a brave, noble, beautiful soul. 
  
    But my heart aches to see you go. 
    I fear to let you do this. 
    Silly warrior \resphan. 
    Why must you risk your life and your very soul? 
    Your bravery alone was enough to bring me into his bed. 
    I can't wait to have sex with you. 
    
    But what if you fail?
    What if you die? 
    I might never see you again. 
    This might be our last moment together.
    What then? 
    
    I want to have sex with you here and now.
    If only I could. 
    But I can't. 
    I have a duty. 
    I cannot give myself to anyone less than a true hero. 
    Otherwise I would be a slut.
    I have no choice but to wait.
    
    Damn all these rules.
    
    Come back safely to me.
    My brave, beautiful hero. 
    My sweet, beloved prince.}
\end{prose}






\subsubsection{He is nervous}
\Teshrial{} is nervous and afraid to enter his battle. 
After all, he is just a \ketheran{} facing a terrible \shaeeroth. 
But he thinks of \ps{\Firaxel} kiss. 
And her sexiness, her hot and lusty eyes, and her soft, whispered promises of erotic bliss to come. 
This gives him courage and strength. 
It is \trope{ThePowerOfLove}{The Power of Love}. 

There is also a little bit of unexpected \trope{ThePowerOfFriendship}{Power of Friendship} from \Achsah. 

Thinking of these bright things, \Teshrial{} takes heart. 
He rights himself and is now ready to face \Ishnaruchaefir. 









\subsection{The Power of \EreshKal}
In this section, \Takestsha storms \Forclin. 




\subsubsection{Curwen goes to the Ghost Tower}
\target{Charcoal at the Ghost Tower}
Throughout much of the story, Archibald Curwen (Charcoal), supposedly the sneaky master Cabalist, is duped, manipulated and played for a fool by his enemies. Sentinels and other agents seem to run circles around him. 
He's been played for a \trope{XanatosSucker}{Xanatos Sucker} the whole book. 
But at the end he finally realizes what's going on around him and strikes back. 

Near the end\dash perhaps after having discovered one or more of the people who have been cheating him, such as \Sanyor{}\dash Charcoal shows what a formidable agent he truly is. 

Curwen realizes that \Takestsha and her \ishrah are too powerful and dangerous.
He cannot just use conventional tactics against them.
Instead, he gets an idea. 
He devises a master plan to dispel the Rungerans' \EreshKali magic and strike a hard blow to their forces. 
But knows that will be hard. 
He cannot do it without help. 

So he gets another idea. 
He can use the Ghost Tower.

Curwen has been in \Forclin before. 
He knows the Ghost Tower and has even been inside it. 
He knows it is a conduit to the Realm where the \resphain live (though he has not been there). 
He formulates a plan to use the Tower in his counterspell against \Takestsha by channelling energy through the Tower's \nexus, or something like that. 

So he leaves Carzain in charge of the \ishrah and departs the battlefield, heading for the Ghost Tower. 





\subsubsection{Curwen contacts \Achsah}
On the way to the Tower, Curwen recites an orison to contact \Achsah and ask for her advice. 

\begin{prose}
  \Achsah:
  \ta{What? Do you have urgent news about the Sentinels?}
  
  Curwen:
  \ta{Well\ldots{} no, my Lady \Resvil. But\ldots{}}
  
  \Achsah:
  \ta{Then do not pester me.
    Figure it out yourself.
    I am busy.}
\end{prose}

\Achsah is unwilling to help. 
She has her hands full. 
She is stressed. 
She is sure the Sentinels are up to something really nasty here.
She tries to figure out what it is. 
She has no time to advise Charcoal. 
He is on his own.
She is sure he can solve his own problems. 
He is a skilled mage and a high-circle Cabalist. 





\subsubsection{\Achsah suspects \Takestsha}
\Achsah, who is in \Forclin, looks at the \quo{\EreshKali} spells cast by \Takestsha-tachi. 
\Achsah{} wonders when she first observes the \EreshKali{} magic. 
It does not feel like anything she would expect them to have. 
It also does not feel like what she would expect ancient \meccara{} to have. 
It is new to her, and it makes her suspicious. 
But what it \emph{does} smell like is Rissitic magic. 
That makes her even more suspicious. 
(She has heard Charcoal's account of Tantor's diary, but Charcoal has never seen Rissitic magic, so he cannot draw the connection.) 

Then she realizes that those spells are actually meant to tear the Shroud and reach into the Beyond, where it can summon\ldots{} stuff. 
And the Ghost Tower, which is in close promixity now that the Rungerans have breached \Forclin, acts as a catalyst. 
Those spells are tearing at the very fabric of the Shroud. 
Something fucking nasty is breaking through, or so \Achsah thinks. 
In reality the summoning spell at \Forclin is a smokescreen. 
It is meant to warp the Shroud and look big and impressive, but it doesn't actually \emph{do} anything. 
It's just meant to attract attention and convince everyone that the real stuff is happening in \Forclin, near the Ghost Tower. 

\Achsah now strongly suspects that \Malcur is a decoy and that the Sentinels' real goal is \Forclin. 

But still she keeps watching.
She does not intervene. 
Partially because she must remain ready and keep a bird's eye view of the action and cannot afford to commit herself to any narrow battlefield action.
Partially because of the Unspoken Covenant.
She will not be the first to break it. 





\subsubsection{\Takestsha attacks}
\Takestsha and her mages have to start their big attack spell. 
\Takestsha knows it is risky. 
Her mages are not holding up as well as they should. 
They are weaker than she had hoped. 

If it were up to her, \Takestsha would be patient and lay a prolonged siege. 
But \hr{Psyrex tells Nzessuacrith to capture Forklin quickly}{\Secherdamon and \Psyrex have asked her to make haste}. 

\target{Takestsha will not become Nzessuacrith too soon}
\Takestsha \emph{knows} what \Secherdamon's plan is.
She knows her own attack is a decoy.
Her mission is to attract the \resphain's attention and fool them into coming to fight her. 
Conquering \Forclin is just a means to that end. 
She can, of course, assume \draconian form right away.
But she will not break the Unspoken Covenant for no reason.
She will first go as far as she can in her \human guise.
Only when she absolutely has to will she break her disguise.
If she were to take \draconian form too early, it would be suspicious, and the \resphain might not be fooled. 
It must look like she was forced to unveil herself. 

So she has to act in haste. 
She decides she must take some risks she would otherwise not have taken. 

So she and her \ishrah begin their great spell that will bring down \Forclin. 
It may be foiled, but \Takestsha hopes it will work. 

\Takestsha must push the Rungeran mages really hard. 
Push them to their limit and beyond it. 
Her master spell is a colossus on feet of clay, and she knows it. 

\citeauthorbook[p.248]{RobertEHoward:HouroftheDragon}{Robert E. Howard}{%
  Hour of the Dragon%
}{
  He glanced up at the sky, and he glanced down at the slim white figure [of a captive virgin girl] on the dark stone. And lifting a dagger inlaid with archaic hieroglyphs, he intoned an immemorial invocation:

  \ta{%
    Set, god of darkness, scaly lord of the shadows, by the blood of a virgin and the sevenfold symbol I call to your sons below the black earth! Children of the deeps, below the red earth, under the black earth, awaken and shake your awful manes! Let the hills rock and the stones topple upon my enemies! Let the sky grow dark above them, the earth unstable beneath their feet! Let a wind from the deep black earth curl up beneath their feet, and blacken and shrivel them\dash}
}





\subsubsection{Curwen begins his counterspell}
Archibald Curwen is at the Ghost Tower. 

When Curwen enters the Ghost Tower, he finds that it is much larger on the inside than on the outside.
Curwen knows this is a Shroud phenomenon.
In the city, the repressive Shroud twists the mind and the eye and makes the tower look small, and takes a man along paths that make the tower look small.
But in here, the Shroud is weaker, so the true extent of the Tower reveals itself to him.
Or something like that.

Have a \quo{moon-shrouded crystal} or the like inside the Tower. 

\lyricsbs{Bal-Sagoth}{%
  Enthroned in the Temple of the Serpent Kings%
}{
  Deep within the glacial, ice-veiled temple,\\
  ancient enchantments summon the shades of the dreaming Serpent Kings,\\
  and the Ophidian Throne once again draws power from the Moon-shrouded crystal.
}

He begins his counterspell. 

\lyricslimbonicart{Solace of the Shadows}{
  I set the stones for invoking ceremonies.\\
  In the twilight zones arise abstract galaxies.\\
  The magic eye unveils the blackened skies.\\
  A new horizon begins to each one that dies.
}

He is frightened by the magnitude of the powers he unleashes. 
But also thrilled, exhilarated.

\lyricslimbonicart{Solace of the Shadows}{
  The desolation makes me feel\\
  so dark, so cold, the silence.\\
  So dark, so cold, the emptiness.\\
  Solace of the shadows.
  
  Night surrounds and embraces me.\\
  Darkness holds the secrets of man's fears.\\
  It captures my heart as the purgatory sears.\\
  I cast now the spell, as I cross through raging flames,\\
  into darkness, cursing names.
}





\subsubsection{Ilcas-tachi attack the \ishrah}
\target{Ilcas-tachi attack the Rungeran Ishrah}
Sethgal and his forces are strained to the utmost.
He has to hold back the mundane Rungeran army, which is enough of a problem already.
It is twice as large as his own, and the walls of \Forclin are crumbling under the Rungeran cannonade. 

Meanwhile, Telcastora Ilcas leads his Imetrians in a brave attack against the Rungeran \ishrah.
This is something Sethgal has asked them to do. 
The Imetrian mage, Ulphon, \hr{Ulphon Nestor dies}{has been killed}, so Ilcas asks Carzain to cover them. 
He does. 

\target{Ilcas injures Takestsha}
The \ishrah have massive ranks of soldiers protecting them.
But the Imetrians are fearsome fighters, and Carzain is a badass mage.
They break through and kill several mages. 
Ilcas and his \nycans even manage to seriously injure \Takestsha. 
She would have died if she were an ordinary mortal. 

This attack is won primarily by the Imetrians. 
Emphasize the courage and superhuman skill of Ilcas and his \nycans. 
They are awesome forces of destruction. 
Carzain also fights well, but he has a secondary role. 
Carzain gives them some artillery support and protects them against enemy magic, but he does not fight in \melee himself.
Carzain does not kill much.
He is mostly a distraction. 
This is the Imetrians' hour of triumph. 
Carzain's moment of glory comes later when he \hr{Carzain fights Takestsha alone}{fights \Takestsha alone}. 

\Takestsha knows this is bad.
Her spell is strained as it is.
If these interlopers kill too many of her mages, her spell will certainly fail. 

So she diverts her attention from the spell and unleashes some nasty spells against the attackers.
As nasty as she can make them in her current state. 
(She is in a weakened humanoid form, deep in the Shroud, and she is a bit exhausted, and she is stressed because she has so many things on her mind and must maintain so big and complex a spell.)

She kills several Imetrians and forces the rest to retreat.
Then her soldiers are able to close their ranks.
Carzain-tachi are overwhelmed.
They have no chance but to flee to save their own hides. 





\subsubsection{The \EreshKali magic backfires}
\target{Eresh-Kali magic backfires}
Curwen is working on his counterspell.
When Carzain-tachi attack \Takestsha, he sees the opening he needs. 
He strikes with the full force of his counterspell. 

The spell catches \Takestsha-tachi at the worst possible time. 
The Rungeran \ishrah is reduced. 
Several mages have been killed in the \hr{Ilcas-tachi attack the Rungeran Ishrah}{Imetric attack}. 
The remaining ones cannot endure all the stress and strain.
The \EreshKali spells backfire on them. 
This kills the entire Rungeran \ishrah{}.
Only \Takestsha survives, and she is badly wounded. 

This buys the Cabalists time to send in reinforcements, including \banes{} and perhaps \resphain, which forces \Nzessuacrith{} to breach the \charade{} and assume her \draconic{} form to fight them off. 

By now \Forclin{} is pretty doomed, but now \Nzessuacrith{} is weakened enough for \Achsah{}, her fellow \resphain{} and their \hr{Umbra}{\umbrae} to have a fighting chance against her. 





\subsubsection{Curwen dies}
\target{Curwen dies}
\Takestsha is not pleased to be thus thwarted. 
In the midst of the destruction, she reaches out and strikes back through the counterspell.
She grabs hold of Curwen's spells and twist them against him.
She kills Curwen.

He fought well and bravely.
He made a difference.
But he was just a mortal against an immortal, so he paid for that difference with his life. 









\subsection[Malcur must be a decoy]{\Malcur must be a decoy}
The Cabalists suspect that the Sentinels (specifically, \Secherdamon) are trying to open some kind of portal some place in Pelidor, but they don't know exactly where. 

The Cabalists believe the ploy is to create a \emph{portal} or \emph{conduit} to \Nithdornazsh{} or something like that. 
They don't imagine that the Sentinels actually intend to \emph{resurrect} \Nithdornazsh{} and bring the entire city to \Azmith, straight into the deep Shroud. 

At first, the Cabalists suspect \Malcur. 
But \ps{\Takestsha} desperate, rushed attack against \Forclin{} convinces them that the target is \Forclin{} and the Ghost Tower, and that \Malcur was a decoy. 
\Secherdamon{} is known for very elaborate decoys like this. 
\Achsah, who has long \hr{Achsah suspects that Malcur is a decoy}{suspected that \Malcur was a decoy}, is the first to believe the \Forclin{} ruse. 





\subsubsection{\Nzessuacrith shows her true \colours}
\Takestsha was badly shocked when \hr{Eresh-Kali magic backfires}{Curwen's counterspell struck} and caused her \quo{\EreshKali} master-spell to backfire. 
She had to suddenly use a lot of powerful magic just to keep alive. 
(Her \human body is much weaker than her \draconian body, and \hr{Ilcas injures Takestsha}{she was wounded by Telcastora Ilcas}.) 

In her haste, she accidentally lets her stealth slip.
Tremours of her true \vertex signature spill forth. 

\target{Takestsha bleeds power}
\Takestsha realizes her cover is blown. 
But she is also hurt. 
She is leaking arcane blood so badly that even Shrouded humanoids can see it.
She is trailing tendrils of power, and she bleeds blood that sizzles and burns and evaporates. 
Her own Rungeran soldiers flee screaming from her path.

\target{Takestsha retreats to heal}
\Takestsha has to retreat. 
She must take a little while to heal. 
But she promises that after some minutes of rest, she will back.
And this time she will hold nothing back.
The \resphain must surely know by now that she is here, so there is no point in playing any more games of deception.
As soon as she is fresh, she will break her humanoid guise and attack in her true form. 

\begin{prose}
  \Takestsha: 
  \ta{Mark my words. 
    I will not be thwarted by mere mortals.
    \Forclin will fall this day by my hand.
    Or my claw, if need be.}
\end{prose}






\subsubsection{\Achsah detects \Nzessuacrith}
\target{Achsah calls for help}
\Achsah detects it when \Nzessuacrith lets her stealth slip. 
She smells a \ps{\dragon}{} presence. 

It is a frightening \trope{CosmicHorror}{Cosmic Horror} revelation for \Achsah when she perceives and recognizes \Nzessuacrith.
\Nzessuacrith may not be a \shaeeroth, but she is great a great and terrible \dragon. 

Now \Achsah{} is really frightened. 
She realizes that she can't handle \Nzessuacrith{} on her own. 
Moreover, if a \dragon{} is here, then it must be \trope{SeriousBusiness}{Serious Business}. 
And it is well known that \Nzessuacrith{} works for \Secherdamon, so \Achsah{} is convinced that she is the spearhead of his plan. 

She also becomes convinced that this attack has been deliberately timed to coincide with \ps{\Teshrial} duel, so everyone's eyes would be fixed on \Malcur. 
So \Secherdamon{} must have learned of the duel and is using it as a smokescreen. 
No one in the Cabal imagines that \Ishnaruchaefir{} is \emph{actively} and directly helping \Secherdamon, because he hasn't been doing that for millennia now. 






\subsubsection{\Achsah calls for help}
\Achsah{} needs help. 
An average \dragon is \hr{Dragons vs Resphain in power}{a terrible opponent}, even \hr{Umbra power}{if you have \umbrae} on your side.
And \Nzessuacrith, while no \shaeeroth, is far more than an average \dragon. 
\Achsah will need \emph{many} reinforcements if she is to take down \Nzessuacrith. 
She contacts her fellow \resphain who are working on the \hr{Malcur venture}{\Malcur venture} and requisitions reinforcements. 

The other \resphain are not happy about that, because they would like to keep some in reserve for \Ishnaruchaefir.
But Achsah pleads her case, and \Teshrial agrees with her (because he \hr{Teshrial fears to break agreement with Ishnaruchaefir}{fears to break \Ishnaruchaefir's agreement}).
(This happens before \Teshrial's duel has started.)

So a dozen \resphain hurry to \Forclin to fight off \Nzessuacrith. 
This includes the \resphain{} whom \Teshrial{} \hr{Teshrial leaves Bezed in charge}{left in charge of \Malcur}. 
There are now only a few left to guard \Malcur. 

This is distressing and hurtful to the Cabal's plans because they had not planned for such an eventuality. 
Like \humans, \resphain can be shortsighted.
\Dragon attacks \emph{almost} never happen.
Therefore, many \resphain tend to assume \dragon attacks will \emph{never} happen, and hence they will not plan for them. 
\Nzessuacrith's appearance is a blatant breach of the Unspoken Covenant which none could have foreseen.

Also remember that the faction taking care of Pelidorian business is small. 
There are not many \resphain there, so they are not well-equipped to deal with such fierce \dragon attacks, from \shaeeroth and other elders.

Only one \resphan{} now remains to guard \Malcur: \Paerzim. 
\hr{Psyrex kills Paerzim}{That ends badly}. 





\subsubsection{\Achsah fetches \umbrae}
\target{Achsah fetches Umbrae}
When \Achsah{} has realized \Takestsha{} is a \dragon{} (and perhaps even the mighty \Nzessuacrith), she reckons she still has a few hours of time before the shit hits the fan. 

So she calls on Charcoal (who is in \Forclin) and gives him responsiblity for keeping control of \Forclin{} for some hours until reinforcements arrive from \Malcur. 

\Achsah{} then quickly submerges and goes to \Nyx{} to fetch one or more \umbrae. 
She believes she and her companions will need them if they are to confront the \dragon\dash especially if it really is \Nzessuacrith, as she fears. 

She has asked \Teshrial{} to send \emph{one} \resphan{} directly to \Forclin{} (to relieve poor Charcoal) and send the others to \Nyx{} to rendezvous with her there. 

Remember to have a cool, dark, evocative, mystic scene where they summon the \umbrae{} from the endless dark deep. 

\Achsah{} is in awe and fear when the \umbra{} shows up. 
She is old enough to remember the days in \Merkyrah{} when \hr{Merkyrans fear Umbrae}{the \resphain{} lived in fear of the \umbra}. 
She remembers the terror and awe she felt when \hr{Rebels awestruck by tame Umbrae}{she first saw the rebel leaders riding \umbrae}, back when she was just a rank-and-file rebel acolyte. 

Nowadays she has ridden an \umbra{} countless times. 
But she still remembers the fear of them and knows to be careful. 
Many younger \resphain{} do not know this, she reflects. 
They are overconfident around \umbrae.
They do not show the terrible monsters the respect they deserve. 
They have not lived back in the days when, at any moment, an \umbra{} might swoop in from the dark sky or the dark deep and kill and eat a half-dozen \resphain. 
(Remember, \resphain{} were weaker back then.)









\subsection[Psyrex wins]{\Psyrex wins}





\subsubsection[Psyrex-tachi invoke the ritual]{\Psyrex-tachi invoke the ritual}
\target{Psyrex-tachi invoke the ritual}
\Psyrex-tachi begin the sorcerous ritual that will summon and resurrect \Nithdornazsh. 

\Psyrex, the mad sorcerer, leads the ritual. 
Several other mages support him, and loads of servants stand by to perform menial tasks, such as procuring living prisoners to be sacrificed. 

\Psyrex rejoices. 

\citeauthorbook[p.137]{RobertEHoward:TheAltarandtheScorpion}{Robert E. Howard}{%
  The Altar and the Scorpion%
}{
  \ta{The real gods are dark and bloody!
    Remember my words when soon you lie on an ebon altar behind which broods a black shadow forever!
    Before you die you shall know the real gods, the powerful, the terrible gods, who came from forgotten worlds and lost realms of blackness.
    Who had their birth on frozen stars, and black suns brooding beyond the light of any stars!
    You shall know the brain shattering truth of that Unnamable One, to whose reality no earthly likeness may be given, but whose symbol is\dash the Black Shadow!}
    
    The girl ceased to cry, frozen, like the youth, into dazed silence.
    They sensed, behind these threats, a hideous and inhuman gulf of monstrous shadows.
}

They invoke mystic, forbidden names. 

\lyricsbs{Bal-Sagoth}{%
  As the Vortex Illumines the Crystalline Walls of Kor-Avul-Thaa%
}{
  By Klatrymadon and Zuranthus,\\
  such ancient secrets we discovered \\
  within these sinistrous, worm-worn pages,\\
  Etched with darksome glyphs and sigils, \\
  bound with fearsome spells, \\
  An eldritch tide of stygian sorceries \\
  unfettered by the forbidden Tome of Shadows\ldots{}
  
  Now thunderous cataclysm befalls the gleaming Kor-Avul-Thaa \\
  (The mystic gate stands open!) \\
  The Xytaxehedron held to the stars\ldots{} \\
  the incantation uttered with eager tongues\ldots{} 
}

\ps{\Ishnaruchaefir} is one of the names. 

\lyricsbalsagoth{Invocations Beyond the Outer-World Night}{
  Invocations and ideograms (dreams of the Xytaxehedron?),\\
  Conjuration of the inner world's (tenebrous) denizens,\\
  And their star-spanning progenitors, \\
  spawned beyond the outer-world night.
}

They invoke Chaos. 

\lyricsbs{Arcane Wisdom}{Symphonia Chaos}{
  Chaos, ruler of Time. \\
  Chaos, infinity is Thine. \\
  Shadowy inner essence, \\
  cosmic tapestry and sparkling \\
  spheres of a circular reason. \\
  Chaos, ruler of Time. \\
  Chaos, infinity is Thine. 
}

\lyricsbs{Emperor}{Moon Over Kara-Shehr}{
  Our time is upon us. \\
  Master! Appear! \\
  Over the nocturnal sky \\
  from the mountains of black we ride. \\
  Fly! 
  
  All thy servants fly \\
  though the serpent's darkened sky. \\
  Hears the opponent cry, \\
  ravaged by his terror. 
  
  Master! We ride with the storm \\
  in his name, the sire, wolves' king.\\ 
  Enter the power coursing \\
  through veins of the night. 
  
  Who gathers winds, summons thunder, \\
  summons rain, summons might?
}

Describe how \pdaemons{} fly screaming through the sky and blood boils up through the ground. 

\lyricsbs{Bal-Sagoth}{
  Dreaming of Atlantean Spires
}{
  The sky is black with Chaos-fiends,\\
  spellcraft rides the witch-storm's wings.\\
  Beneath the vaults of time-lost tombs\\
  sorcerers summon the shadow-kings.\\
  The Topaz Throne is beckoning,\\
  the jewelled sword awaits my grasp.\\
  The dreaming gods now grimly brood\\
  in the silence of Atlantean Spires.
}

Also, describe the mages' awe and fear as they invoke the names of gods, and then see those gods actually appear!

\lyricsbs{Bal-Sagoth}{
  As the Vortex Illumines the Crystalline Walls of Kor-Avul-Thaa
}{
  What long-shackled powers of the elder dark have our conjurings loosed?
  
  By Klatrymadon and Zuranthus, \\
  the vortex blackens the stars above,\\
  A vast plague of amorphous horrors \\
  descends to rend with fang and talon,\\
  (As with torrents of blood the crystalline walls run red?)\\
  And in the glooming chambers of our shadowed sanctum, \\
  we wait, half-mad with terror,\\
  To reap the slaughterous harvest which we have sown\ldots{}
  
  [The Chronicler of the Cataclysm:]\\
  And beyond the vortex, the churning black waters of the void did disgorge the Dwellers in Eternal Shadow. \\
  And upon a horde of winged horrors, brandishing swords of ebon flame, they rode out from the Gate\ldots{} \\
  And a terrible silence fell upon Kor-Avul-Thaa\ldots{}
  
  [The Echoes of the Oracle:]\\
  The sky rent asunder, \\
  black winged devils surge forth from the void\ldots{}\\
  A maelstrom of crimson fire burns above us\ldots{} \\
  what carnage has thou wrought?
  
  By Klatrymadon and Zuranthus, \\
  in Kor-Avul-Thaa, darkness reigns eternal\ldots{}\\
  Nevermore shall the city glimmer, \\
  for now the crystalline walls gleam black\ldots{}\\
  Ever black\ldots{}
}





\subsubsection{\Paerzim killed by \Psyrex}
\target{Psyrex kills Paerzim}
After \hr{Achsah calls for help}{\Achsah{} has summoned everyone to \Forclin}, only one measly \resphan{}, \Paerzim, remains in \Malcur. 
Needle is \ps{\Paerzim} second-in-command. 

\Paerzim{} is competent enough to give \Psyrex{} trouble. 

But then Needle is killed by Moro and Rian. 
This means \Paerzim{} has to take an active role in everything and divert his attention around. 
This is just what \Psyrex{} needs. 
He is now able to out\manoeuvre \Paerzim{} and kill him. 

Now the Cabal in \Malcur is all but destroyed, and the way for \Psyrex-tachi is opened. 
\Achsah-tachi are busy in \Forclin. 
\Teshrial{} is busy with his duel. 
None of them are in a position to come back and stop \Psyrex. 
He can freely complete his summoning ritual. 
\Nithdornazsh{} will rise. 





\subsubsection{The Cabal are thwarted}
With Needle dead, the Cabal raid is thrown into disarray. 
\Psyrex{} finds their trail and figures out their moves. 
When the Cabalists finally get their act together, \Psyrex{} has his men and magic prepared for them. 
The Cabalists fight bravely and sneakily, but \Psyrex{} just manages to stitch together enough makeshift defenses to fend them off. 

\Achsah, who might be able to put a stop to the Sentinels' dastardly plan, is up near the Ghost Tower. \ps{\Secherdamon} clever plan is working. 

Just before all Hell breaks loose\dash literally!\dash\Psyrex{} appears before Rian and Moro. He congratulates them for their hard work and their success, and thanks them for the help. If not for them, he says, Needle-tachi would have taken him by surprise, he and his companions would have been slain and their gambit would have crumbled. 

%Do Rian and Moro die? I think Rian at least dies. Moro is cool enough that she might escape, but she might also die. If she dies, I might want to portray her as a bitchy Aes Sedai who bullies Rian, so the reader is not sad to see her die. 

%I need to portray Rian as semi-cool but not so cool that the audience will be outraged when I kill him. And I need to be very sure that \Psyrex{} is portrayed as kickass, so people will like it when he wins. 





\subsubsection{\Tiroco{} feels the end is near}
\Tiroco{} feels the end is near. 

\lyricsbs{Marduk}{
  Cold Mouth Prayer
}{
  Quickened lumps of Earth,\\
  A feast for fowls and greedy worms.\\
  Count your sins, the Snakes within.\\
  An eyeless leap into the Bosom of Decay.\\
  Cold Mouth Prayer.\\
  Cold Mouth Prayer.
  
  A flash, a minute, a winter's dust.\\
  A choir of fingers sings of putrefaction.\\
  Cold Mouth Prayer. Cold Mouth Prayer.\\
  A smile left to rot in the Sun of Despair.
}









\subsection{Rian finds Neina}





\subsubsection{Rian chooses to save the girl}
Rian has the opportunity to stave off \Malcur's fall for a little while and save many more lives, but that would seriously endanger Neina. 
Rian is a dumbass and chooses to \trope{AlwaysSaveTheGirl}{Always Save the Girl}. 
Moro berates him for it to his face, and later (after he has gone) curses him in retrospect for being selfish and choosing what was obviously the worse choice.




\subsubsection{Rian finds Neina and escapes}
\target{they find Neina}
In the chaos that ensues after the \banes wreak havoc in the Sentinel camp, Rian slips in and manages to find and rescue Neina.

Have a long, tension-filled scene where we hop back and forth between Neina and Rian. 
There is a thug, Blon, who wants to rape Neina.
He reaches into her cell and gropes her. 
She shies away and hates him and suffers. 
But he dares not rape her because the sorcerers forbid it. 

Later, when all hell is breaking loose, Blon decides he might as well go for it. 
So while the rest are running around confused, he goes to Neina. 
He gropes her and undresses her. 
She screams and fights.
She manages to kick him painfully in the nuts.
He retreats a bit to get over the pain. 
This gives her a short reprieve.
Then he comes back in and beats her savagely. 
He rips her clothes off.
She screams and cries. 
She is desperate. 
She does not want to believe that this is happening to her. 
At last, he penetrates her and completes his violation of her. 
She loses her virginity to an evil, ugly, foul-smelling crook whom she hates. 

All the while, we hop back and forth between Rian and Neina.
He runs around frantically looking for her. 
At last he reaches her.
She is being raped by Blon. 
Rian attacks.
Blon pulls out.
Rian kills him.
This is the first person Rian has ever killed. 
As Blon dies, sperm pours out of his dick onto the ground.
Neina has been raped, but they can take some small consolation in the fact that Blon did not come inside her. 

Rian has saved her, but at a terrible price. 
Everything else is going to hell around them. 
Moro wished he was with her.
Together they might have been able to save more people.
But no.
Rian had only eyes for Neina.
He insisted that he had to \trope{AlwaysSaveTheGirl}{Always Save the Girl}.
It was a horribly bad choice from a utilitarian point-of-view, and he did not even succeed. 
Karma by proxy! 

Neina, being a weak soul, \hr{Weak souls go mad in Malcur}{is affected by the Change}.
She may be half-mad. 
She comes somewhat to her senses when she gets out of Malcur, but she will never be quite sane again. 





%At the last minute, however, they manage to escape. 
Rian finds Neina and they run away together. 

The sight of the two together, their love and happiness at being finally reunited, reaches Moro's heart, and she is both touched and envious. It softens her a little bit, and she gains a bit of hope for the future and the world.





\subsubsection{Neina's story}
\target{Neina's story}
Neina was supposed to have been killed a while ago. But some of the Sentinel-hired thugs screwed up, and she was left to rot in a dungeon for many weeks. 

A few times she was dragged to a ritual to be sacrificed, but then not sacrificed anyway and returned to the dungeon. 
\ps{\Psyrex}{} thugs are not the most organized people in the world. 

When Rian comes to rescue her, she is half-mad and hysterical, babbling about monsters and black magic. She is quite useless. 





\subsubsection{Rian and Moro almost eaten by living house}
\target{Rian and Neina are separated from Moro}
Rian, Neina and Moro huddle inside a house. Then suddenly the Shroud opens up and the house reveals itself as a monstrous living entity. It makes a grab for them with its claws or tentacles, intending to devour them. 

Compare to a scene in the anime \cite[episode 1]{Anime:IczerOne}, where a house comes alive.

They escape, but in the chaos, Rian and Neina get separated from Moro. Now the young lovers are in deep shit, because it was Moro's skill and sorcery that was keeping them alive.

Rian develops a nasty case of claustrophobia after being almost eaten by a living house, having seen the walls literally close in around him and turn into monstrous mouths. 





\subsubsection{Maybe Rian dies}
\target{Maybe Rian dies}
Maybe I will \hr{Remove Tiroco story thread}{remove the \Tiroco story thread}. 
If so, I might kill Rian and Neina instead.

\begin{itemize}
  \item 
    Maybe Moro tries to convince Rian to help her do some real good and save as many people as they can.
    Rian instead runs off, frantically intent on saving Neina, whatever the cost to everyone else. 
    He is quickly killed as punishment for his stupidity.
    \quo{\trope{AlwaysSaveTheGirl}{Always Save the Girl}} is not a good idea on \Miith. 
    
  \item 
    Maybe Rian does find Neina.
    He rescues her while she is still a virgin.
    They run out. 
    Then they are killed. 
\end{itemize}

Maybe Rian is killed by a \bane before Moro's eyes. 
Moro gets traumatized. 
But at least she manages to do some good and save some other people. 

Maybe it is Moro who gets rescued by \Criseis in the end. 















\section{\Nithdornazsh Rising}







\subsection{Battle for the Ghost Tower}
The Sentinels and Cabal battle for control of the Ghost Tower. 

Note that \hr{Immortals inside the Shroud}{the Shroud suppresses the immortals' powers}. 
I need to deal with that somehow. 

Read about: 
  \hr{Resphan}{\resphain},
  \hr{Dragon}{\dragons},
  \hr{Umbra}{\umbrae},
  \hr{Resphan equipment}{\resphan equipment},
  \hr{Weapons}{weapons},
  \hs{technology}. 
Read about \hs{Chaos magic}, and remember to invoke \Sethicus and \Tiamat. 

\Achsah retains command of her group of \resphain. 
The \resphain who come to help \Achsah \hr{Achsah's rank}{stand below \Achsah{} in rank} and must obey her commands. 
This galls them, for some of them are purebloods. 
But \Achsah is highly talented and experienced, more powerful than many purebloods.
(Some of the others are \thelyadeth or \gessurim, others \bezedeth.)

\hr{Ashenblood lesser immortality}{\Bezedeth do not possess True Immortality}, so if they die, they are gone for good. 
This means \Achsah must be very brave and convince her fellow \bezedeth to be likewise when they have to go up against \Nzessuacrith.
    
Maybe the \resphain wear \hr{Glass armour}{\armour made of glass or crystal}.

\Nzessuacrith attacks her foes with dark curses invoking the \xss. 
See the section on \hr{Magic visuals}{magic visuals}, especially \hr{Curses of destruction visuals}{curses of destruction}.





\subsubsection{Carzain attacks \Takestsha}
\target{Carzain fights Takestsha alone}
\Takestsha \hr{Takestsha retreats to heal}{has retreated from the battle to heal}. 
But Carzain is not done with her.
He fights his way through the Rungeran ranks using might and stealth.
He has left he Imetrians behind now. 
He is alone. 
This is his moment of glory. 

He tracks \Takestsha and attacks her. 
With the great sorcerous power he and she unleash (and the way \hr{Takestsha bleeds power}{\Takestsha bleeds arcane power}), no Rungeran soldier dares come anywhere near. 
They fight. 

Fortunately, Carzain is \uber-powerful for a mortal. 
Read about \hs{Carzain's strength}. 
\Takestsha keeps underestimating him and slapping him with too little power, and he keeps climbing to his feet again. 

Carzain is overpowered, but he fights bravely. 
He was wounded already when he approached her (having taken wounds in the charge against the \ishrah) and sustains many more wounds in the battle, but by heroic willpower he keeps himself alive and fights on. 
He pushes himself to the utmost, and in his hour of need, on the brink of death, he manages to unlock some of the dark power that lies sleeping within him. 
He gets closer to his true self. 

Carzain does not understand why \Takestsha is so powerful.
He is willing to swear that her magic is \rethyactic in nature.
But she is too fast. 
Conventional wisdom has it that in close combat, a Vaimon will always defeat a \rethyax. 
The \rethyax's magic may be more powerful, but the Vaimon's magic is faster. 
But not her. 
She is fast as fuck. 

Finally she pushes him away and transforms. 
She breaks her \human form and begins to mutate and grow. 
She begins to return to her \draconian form.

This gives pause to Carzain. 
He starts to suspect he has gaped over more than he can handle. 
But he does not back away.
He keeps up his attack. 

She fights on while she is mutating into her true form. 
She badly wounds him. 
Then his \malach self begins to awaken. 
Somehow the savage battle triggers something in him.
It is the first time in all his Scion lives that Ramiel has faced a \dragon this close. 
He has many strong memories of fighting against \dragons, including \Nzessuacrith herself. 
Now they come back to him. 
The fact that his body is badly wounded helps. 
It is more traumatic, and it forces his desperate mind to dig deeper for reserves of power. 
He has to fight for his life, harder and more desperately than he has ever fought before. 
Besides, he is surrounded by immense power that rends the Shroud in tatters. 
This makes it easier for him to see through his own inner Shroud and access powers and memories that he did not know he had. 

And Carzain indeed finds new reserves of power. 
He manages to unleash sorcery beyond any \human. 
A trace of his \sathariah power that is awakening. 

This is his \quo{\hr{Carzain's Sephiroth epiphany}{Sephiroth epiphany}}. 
As part of his epiphany, Carzain sees visions of \Mystraacht.
Also read about \malachim, \carcers, Carzain, Vizicar and Ramiel.

\citeauthorbook[p.243]{RobertEHoward:KingsoftheNight}{Robert E. Howard}{%
  Kings of the Night%
}{
  The sun was sinking into the western sea; all the heather swam read like an ocean of blood.
  Etched in the dying sun, as he had first appeared, Kull stood, and then, like a mist lifting, a mighty vista opened behind the reeling king.
  Cormac's astounded eyes caught a fleeting gigantic glimpse of other climes and spheres\dash as if mirrored in summer clouds he saw, instead of the heather hills stretching away to the sea, a dim and mighty land of blue mountains and gleaming quiet lakes\dash the golden, purple and sapphirean spires and towering walls of a mighty city such as the earth has not known for many a drifting age.
}

\citebandsong{DarkEmpire:DistantTides}{Dark Empire}{A Soul Divided}{
  Since my birth, haunting visions have occurred to me. \\
  A strange power controlling my destiny. \\
  Burned in the fire, but my blackened soul has remained.\\
  My dark desire will infect this mortal plane.
  
  Lying here, left for dead. Bandaged and alone.\\
  A separate half I'll create to undermine their throne. \\
  Feel your aggression. The taste of my hate inside of you\\
  feeds my obsession, until the time you set me free. \\
  This was meant to be
  
  Remember me, your true self. Its who you must become. \\
  You and I, are one and the same, the Chosen One. \\
  Rock turn to rust. My minions rise and humanity\\
  all turns to dust. None will ever stop my insanity.
  
  Banish it to the void. Seal your fate at once. \\
  Let the light take you in. See it through. \\
  I am immortal. Even if I'm wiped away,\\
  I can't be stopped, no. My influence has spread.
  Your kind is dead
}

\citebandsong{DarkEmpire:HumanityDethroned}{Dark Empire}{Eyes of Defiance}{
  Shadows of the past are consuming all I see.\\
  Breaking through the darkness there's another chance for me.\\
  Once again inside me breathes the energy of life. \\
  Wash away my sins, look into my eyes.\\
  I will defy!
}

He wounds \Takestsha a bit. 
She curses and draws back. 
\Takestsha blasts him again. 

This time he cannot get back up.
He is struck down to the ground, writhing in pain and helpless. 

Remember to have great mind-shattering revelations of cosmic magic when \Nzessuacrith assumes her true form. 
Compare to \cite{StephenMarkRainey:Signals}. 

\Takestsha is stunned a bit and draws back to think, and to complete her transformation. 
She feels the \sathariah power radiating from him. 
(Do not mention the word \quo{\sathariah}.) 
This surprises and startles her.
She realizes that he is more than a mere \human. 
He is a Scion. 
But in a way, this also reassures her.
It would be embarassing for her to live with knowing that a mere \human could cause her such trouble. 
Now that she knows he is a Scion. 
She understands better. 

While Carzain is down, \Takestsha completes her transformation.
She now stands before him in her full, terrible glory as \Nzessuacrith. 

She is just about to finish off this insolent mortal.
Then needles of shimmering energy lance into her.
She looks up.
It is \Achsah and her \resphan kin.
They have finally arrived at \Forclin to repel her.
\Nzessuacrith welcomes them.
She forgets about the high-powered \human and leaps into the air, eager to confront her hated foes. 

Carzain \hr{Ilcas rescues Carzain after fight with Takestsha}{falls unconscious and is rescued by Ilcas}. 





\subsubsection{\Nzessuacrith in \draconian form}
\Nzessuacrith{} has finally been forced to leave the body of \Takestsha{} and assume her true, \draconic{} form and enter the fray. 

\Nzessuacrith{} wears \hs{ward runes}. 
(Remember to read about \hs{ward runes}.)

Before entering combat, \Nzessuacrith{} casts a spell that makes her natural weapons poisonous. 

Read about \hr{Nzessuacrith}{\Nzessuacrith}, \hr{Achsah}{\Achsah}, \hr{Dragons}{\dragons}, \hr{Resphan}{\resphain} and \hr{Umbra}{\umbrae} before writing this section. 
And read some RPG books for inspiration on spells and weapons and magical items they might use. 

\Nzessuacrith{} is surprised that she is able to take \draconic{} form so deep in the Shroud. 
She realizes that her own \EreshKali spells have significantly weakened the Shroud (locally and temporarily, that is).
As she spells backfired, this effect could potentially have become even stronger and more out-of-control. 
She speculates that the Ghost Tower might be exterting a further destabilizing influence on the Shroud. 
But is she not quite convinced by this explanation. 
She fears the Shroud is \hs{unravelling}. 

\begin{prose}
  \tho{The barriers between the Realms are breakdown down.
    The great cosmic Seals are leaking.
    Like water seeping through holes in a rotted dam.
    What happens if\dash when\dash the river breaks through?
    A \thirdbanewar? 
    Can \Miith{} survive a \thirdbanewar?
    
    And what is causing it?
    Is it the weakening of the Heart?
    The conflict of the \matrices?
    What?}
\end{prose}

Have some mortals who are stricken with terror and awe at the sight of the \dragon{}. 
Perhaps Carzain and his party. 
They have heard \hs{myths} of \dragons, but their true power and majesty is downplayed in the Iquinian myths. 
The sight of an actual \dragon{} blows away all the myths and faerie tales. 

Delph dies in this final, climactic battle. 
But his rat lives.

Carzain sees the \hr{Umbra}{\umbrae} that \Achsah-tachi summon. 
He likens them to bats, but \hr{Umbra like bat}{different}. 
They also \hr{Umbra sounds}{hear the \umbrae{} howl}. 






\subsubsection{\Achsah meets the challenge}
The sight of \ps{\Nzessuacrith}{} unmasked has drawn \Achsah{} $100\%$ from \Malcur to the Ghost Tower. 
Charcoal's plan\dash aided by Carzain\dash has bought the Sentinels enough time to call in reinforcements and ultimately fight off the Sentinels. 

We see \Achsah{} at the Tower. 
Perhaps she is riding a terrible but splendid monstrous steed.

\lyricsbalsagoth{When Rides the Scion of the Storms}{
  I see him\ldots{} \\
  grim and noble astride his great winged steed, \\
  gleaming spear crackling in his grasp, \\
  beckoning me onwards to the next life\ldots{} \\
  to ever more slaughter and carnage\ldots{} \\
  Yes, adour and brooding spirit he is, \\
  and in his burning eyes I see \\
  a great secret which I must discover,\\ 
  a powerful mystery I alone must solve.
}

Does she actually wield a spear?

\begin{prose}
  \Achsah{} mocks \Nzessuacrith{}: 
  \ta{I had expected more. I had feared I would be facing \QuessanthIshnaruchaefir.} 
  
  \Nzessuacrith{} mocks \Achsah{} in turn: 
  \ta{I had expected a \ketheran. 
    Not some pathetic half-\human{} scum. 
    Without the stolen \draconic{} blood you \resphain{} are worthless. 
    You, \Achsah, are nothing but a swarthy \human{} with delusions of grandeur.} 
  
  \Achsah: \ta{A \human, am I? We will see about that.}
  
  They fight. 
  Then, a bit later: 
  
  \Achsah: 
  \ta{If you did your research, \Nzessuacrith, you would know that I am of \Merkyrah, and as such not not half \human{} but half \nephil.} 
  
  \Nzessuacrith: 
  \ta{Hah! 
  Feebly trying to defend the last shreds of your dignity? 
  Very well, I take back my last insult. 
  You, \Achsah{}, are nothing but a pathetic, swarthy \nephil{} with delusions of grandeur.} 
\end{prose}

Notice that \Nzessuacrith{} is proven right. 
\Achsah{} is soundly beaten and only prevails when Ramiel, a \sathariah, shows up to help her. 





\subsubsection{They battle}
Enter a great battle between \Nzessuacrith{} and \Achsah. 
This battle is long, hard, bloody, brutal and dirty. 

\Achsah{} draws up her full power, which is considerable. 
What she lacks of inborn gifts she makes up for in age and experience. 

\lyricslimbonicart{Beyond the Candles Burning}{
  I am a dark star rising on the raveous bleaky sky,\\
  a black diamond slunning so deep within the night.\\
  Maliciously I dwell in a bluish shaded beam\\
  with a stonecold heart into the core of my being.
}

Perhaps they fight in humanoid form first before going into their monstrous forms. 

\Achsah{} summons a bat- or Balrog-like monster to ride. 
This might or might not be an \hr{Umbra}{\umbra}. 
Compare to the monster that Durza summons in the movie \cite{Movie:Eragon} (not present in the book \cite{ChristopherPaolini:Eragon}). 

Describe the awesome forces of magic unleashed.

See the section on \hr{Magic visuals}{magic visuals}.

\Nzessuacrith{} is weakened, but still a badass motherfucker. 
She kills more than one \resphan{} (but not permanently). 

\Nzessuacrith{} does \emph{not} use the spell \word{\hs{khestni}}. 
She has nastier weapons at her disposal. 

\Nzessuacrith{} uses magic to enhance all of her moves. 
She growls words to power to punctuate every attack or parade. 
Unlike \Ishnaruchaefir. 

\Nzessuacrith summons monsters to do her bidding.

\citeauthorbook[p.344]{ClarkAshtonSmith:TheDarkEidolon}{Clark Ashton Smith}{%
  The Dark Eidolon%
}{
  Yea, the undying worms of fire and darkness have come forth like an army at thy summons, and the wings of nether genii have risen to occlude the sun when you called them.
}





\subsubsection{An \umbra escapes}
An \umbra{} escapes from the battlefield after its handler is killed. 
\Achsah{} reflects that it will probably go into the \Wylde{} and live off whatever mortals it can catch. 

There it will live out the rest of its days. 

The rest of its days\ldots{} how much is that? 
Are \umbrae{} immortal? 
\Achsah{} does not know. 

And do they reproduce in the \Wylde{}?
How do they reproduce? 
Do they reproduce at all? 
\Achsah{} does not know. 

The \banelords{} probably know, she reflects. 
But she has never been in a position to ask a \banelord{} a question. 
And she would not dare even if she could. 
She is not too proud to admit that the \banelords{} make her shit herself with fear. 
She remembered feeling their evil presence during the Incursion. 
And the presence of the dreaded \Voidbringer. 
The fear has not decreased in her memory, even after all these millennia. 
She still remembers it vividly. 

That is another thing the young \resphain{} do not understand. 
They do not fear the \banes{} as they should. 
The mortals have the right idea here, she thinks. 
Many mortals see their gods as terribly frightening, alien powers around which one must tread very carefully.

\begin{prose}
  \tho{We \resphain{} could learn from that.
    We like to think of ourselves as being on top of that ladder of power.
    But we are not.
    There are things out there far older, far darker and far more powerful than we.}
\end{prose}





\subsubsection{\Nzessuacrith flees}
\target{Ramiel scares Nzessuacrith}
\Nzessuacrith was badly wounded even before she assumed \draconian form.
She is fighting several powerful \resphain mounted on their terrible \umbrae. 
She is holding up, but it is a hard fight for her. 

\Achsah is skilled. 
Eventually her determination, bravery and good combat tactics win the day. 
\Nzessuacrith is overpowered forced to flee. 

\Nzessuacrith rationalizes it, telling herself that \Nithdornazsh must be about to rise. 
The \resphain cannot return to \Malcur in time to stop the ritual now. 
So she permits herself to flee from the field of battle. 

The \resphain are themselves badly wounded.
She has killed some of them (non-fatally).
They are in no condition to pursue. 





% \Nzessuacrith{} could defeat the \resphain, but \hr{Charcoal at the Ghost Tower}{Charcoal's spell antagonizes her}, and ultimately she loses and must retreat.
% 
% Now, Charcoal's magic alone would not be sufficient to make any difference, since \Nzessuacrith{} is a powerful \dragon. But Carzain/Vizicar is there. The \nieur{} ritual that Charcoal unleashes somehow causes the \sathariah{} within Carzain to stir and awaken a little bit. Acting instinctively, Ramiel adds his power to the ritual. 
% 
% Perhaps Ramiel senses that \Nzessuacrith{} is family. 
% She is, after all, kin to \Nexagglachel, the sire of the \satharioth. 
% 
% Now, Ramiel doesn't have much power available. 
% He is still in deep sleep. 
% But he makes his \emph{presence} known. 
% \Nzessuacrith{} feels the smell of a \sathariah, and it distracts and alarms her. 
% She had not counted on the innocuous Scion being a \sathariah. 
% It throws her off balance for a moment, and this is enough for \Achsah-tachi to wound her and drive her off. 





\subsubsection{Conclusion}
At the end of the day, the Tower is in Cabalist hands. 

The battle has wrought destruction of Godzilla-like proportions to the surrounding countryside. The Tower itself is untouched, because it is built with the technology of the ancients. \Forclin{} is laid in ruins, except for the castle and certain walls and towers, who are superhuman. 

\Nzessuacrith{} (\hr{Nzessuacrith likes beauty}{who likes beautiful things}) mourns the devastation wrought on the beautiful \Forclin. 

Now \Achsah{} and \Nzessuacrith{} know who Carzain is\dash to an extent, at least.
\Achsah{} is $100\%$ sure the \vertex{} is a \sathariah{} aligned with the \hs{Midnight Bat}. 
Which means it must be a Scion. 
Which means there are only two possibilities. 
She reports this to \Azraid, who \hr{Azraid learns of spike}{muses about it}. 

\Nzessuacrith{} and \Ishnaruchaefir{} are less certain about this. 
They have not studied the \vertex{} (or the theory of \malachim) as much as \Achsah{} has. 

Some of the Cabalists muse over how they underestimated Charcoal. 
They believed to have him figured out, but he still managed to thwart their plans. 

Meanwhile, in \Malcur, \hr{Ishnaruchaefir kills Teshrial}{\Ishnaruchaefir is about to kill \Teshrial}, and \hr{Nith'dornazsh rises}{\Nithdornazsh is rising}.





\subsubsection{Consequences for mortals}
Sethgal still lives. 
Much of \Forclin has been devastated, but all is not lost. 

It has gone worse for the Rungerans. 
Carzain had chased \Takestsha deep behind the Rungeran lines before she transformed.
So the battle of the immortals took place (initially) in the middle of the Rungeran army. 
In the process, the army took bad casualties, and its morale is broken. 
The remnants have scattered. 

The war with Runger is probably over. 

Carzain has vanished.
Maybe so have the Imetrians. 
Sethgal now has to try to rebuild. 





\subsubsection{Morgan Runger}
Morgan Runger sees his army panic and begin to scatter while the immortals battle. 
He realize the invasion is lost. 
He orders his remaining forces to retreat. 

He rides away in his howdah.
Awestruck he watches the destruction behind him while absently fondling the breasts of one of his naked concubines. 

Maybe Morgan's party is hit by a stray fireball and killed. 
Maybe he escapes back to Runger. 










\subsection{Carzain heads to \Redce}





\subsubsection{Ilcas rescues Carzain}
\target{Ilcas rescues Carzain after fight with Takestsha}
After his fight with \Takestsha, Carzain falls unconscious and lies bleeding to death. 
But Ilcas Northstar has seen his heroic battle against the \dragon-sorceress. 
Ilcas fights his way through the panic and the destruction caused by the warring immortals. 
He reaches Carzain's unconscious body and hauls it to safety. 

Ilcas cannot heal Carzain himself. 
But he finds (or is found by) \Esmerel, who offers her help.
She heals Carzain's grievous wounds and broken limbs. 
She saves his life. 
It takes an awful toll on her.
She looks ten or twenty years older afterwards. 
But she does what she feels she has to do.
She suspects this man is something really special, and she will sacrifice much to secure him for \ClanRedcor.
Besides, now he owes her. 

\Esmerel convinces Ilcas to help her transport Carzain to \Redce. 
They set out while Carzain is still unconscious. 

Carzain later awakens.
He learns that \Esmerel has saved his life, at great cost to herself. 
This means he owes her.
In exchange she coerces him into promising to go with her to \Redce and help \ClanRedcor against a dark enemy. 
He does not want to serve the Redcor, but he lets her coerce him.
He lets her think he is cowed by guilt and obligation and will now serve them faithfully. 
But in truth, he has his own plans. 





\subsubsection{Razor is more wary of Carzain}
Have a scene from Razor's POV. 

Since their last meeting, Carzain has grown darker. 
Vizicar has awakened again and is now closer to the surface. 
Therefore, more of his dark, wicked \sathariah{} nature shines through. 

Razor notices this immediately, and he is now more wary of Carzain than before. 
Does Razor hide this, or does he openly display his unease? 
He probably hides it. 
Razor is a sneaky bastard. 

Carzain notices the creepy lizard staring at him. 
He distrusts it. 
Cannibalize the \quo{creepy \human}/\quo{creepy lizard} scene I wrote once. 





\subsubsection{To \Redce}
\Esmerel{} wants to flee back to \Redce{} and tries to persuade Carzain to come with her, tempting him with romises of glory and greatness if he allies himself with the Redcor. 
Also, \Esmerel{} promises him that they can help him master his \hr{Kenosis}{\Kenosis}. 
He wants this, since in his current state he is an unstable madman and a danger to everyone. 
Also, he hopes if he allies with the Redcor, he can get his hands on \hr{Iolivine's notes}{\ps{\Iolivine} notes on Scions}, which \hr{Redcor bogarted Iolivine's notes}{the Redcor are bogarting}. 

Ilcas agrees to accompany them, because the Imetrians and Redcor want to temporarily join forces to drive back the Rissitics. Carzain agrees, partly influenced by Ilcas' decision. 

I need to stress the fact that Carzain does not agree to serve the Redcor. He is somehow tricked and coerced into coming with them, and thus can claim to be a prisoner of sorts in the Topaz \Chateau. 

How exactly does this work?

They slip out of \Forclin{} amid the chaos and make their way north to \Redce{}. 
But their way goes past the Ghost Tower, so Carzain becomes involved in the events there. 

Remember that the Redcor should refer a lot to their scripture and their historical heroes, such as Silqua and Rebecca Redcor. 





\subsubsection{Carzain is like Sephiroth}
Remember that \hr{Carzain is Sephiroth}{Carzain is supposed to be like Sephiroth} from \cite{VideoGame:FinalFantasyVII}. 

At the end of \TwilightAngelRememberEmph, he has a \hr{Carzain's Sephiroth epiphany}{Sephiroth-style epiphany} and turns evil. 

The epiphany has to do with Curwen (\hr{Charcoal at the Ghost Tower}{his plan to use the Ghost Tower}) and \Takestsha.
Carzain comes into combat with \Takestsha and realizes that she is \Nzessuacrith. 
She uses dark magic against him. 
This is traumatic. 

This is the first battle of this scale he has fought in this life.
He has fought smaller battles as Carzain, and bigger ones as Vizicar.
After a battle, he is usually loath and tired of all the slaughter and bloodshed and glad to see it end. 

After this battle, Carzain realizes that he is not loath of the slaughter as usual.
Rather, he finds that he has relished it as never before. 
He realizes that his true nature is a dark angel of battle.
He feels he has taken a great step forward to finding his soul. 





\subsubsection{Carzain espies Morgan Runger}
On the battlefield, Carzain espies \hs{Morgan Runger} in the distance. 

\vizicar{%
  Feh. Kings who are not mages themselves are mere upstarts. Such a coward. He commands his sorcerers to work their dire magics, but dares not do it himself.}

Vizicar is quite the pro-mage nazi. 







\subsubsection{Carzain sees reapers on the battlefield}
\target{Carzain sees reapers near Forklin}
After the battle of \Forclin, Carzain sees some \quo{reapers} on the battlefield. They are \hs{Worm Cult reapers}, as well as \hr{Crows and ravens}{crow- or raven-like men}. 

They appear as unclear ghosts. They might be an apparition created in his mind from the fog, smoke, animal and corpses. He is in doubt: \tho{Am I seeing things?}

They pick up some dead. Carzain seems to see corpses\dash or their souls\dash rise to wander, limp or crawl away into the Beyond. 

Perhaps the Worm Cult reapers fight the Ravens. 





\subsubsection{Carzain and Vizicar talk}
Carzain and Vizicar talk. 
Vizicar theorizes that it was he who \hr{Vizicar drives Carzain to war}{subconsciously imparted to Carzain a craving to fight and seek glory}, thus causing him to go off to war. 





\subsubsection{Carzain-tachi encounter a \bane}
Carzain-tachi encounter a \bane. 
Ilcas Northstar holds up an Imetric holy symbol and recites a prayer/\hs{orison} to ward off evil and keep it at bay. 
It seems to work for a moment, but then the \bane{} advances again. 





\subsubsection{Ilcas kills prisoners}
There is a scene where Telcastora Ilcas, together with some Redcor, have taken some enemies prisoner. 
These are Pelidorian soldiers who have deserted and turned into bandits. 

Ilcas is about to kill them. 

\begin{prose}
  \Racel/\Esmerel: 
  \ta{No! Stop. If you kill them, you are no better a man that they.}
  
  Ilcas: 
  \ta{What are you talking about? Fuck that!} 
  He kills the men. 
  \ta{Of course I am better than they. 
    They were not only brigand scum, they were also traitors. 
    They were paid and armed by their kingdom and entrusted to protect their people from enemies. 
    They betrayed that trust and turned against their own people. 
    They were the worst kind of filth. 
    I am doing the world a favour.}
\end{prose}

Ilcas truly hates these men's guts. 
Traitors and malfeasants are pieces of shit. 

\begin{prose}
  \Esmerel: More whining. 
  
  Ilcas: 
  \ta{We are not under the jurisdiction of Redcor law!} 
  
  \Esmerel: 
  \ta{Nor are we under Imetric law!}
  
  Ilcas: 
  \ta{Look around you. 
    Pelidor has fallen. 
    This place is \Wylde{}. 
    We are under no law. 
    When protected by no law it falls upon each of us to act on our morality as best we can. 
    I did exactly that.}
\end{prose}

Carzain stands next to Ilcas and admires the manliness. 

Maybe Ilcas concludes with something like this: 

\begin{prose}
  Ilcas: 
  \ta{%
    Maybe we \scathae{} have a more rational outlook on death and killing than you \humans{} do. 
    I don't know.}
\end{prose}

Perhaps the last part is said in private to Carzain out of earshot of the Redcor. 

Ilcas notices that during the killing, Carzain stood by and looked on with a bloodthirsty grin. 
This disturbs Ilcas. 
He had good reasons for killing them, but he did not \emph{enjoy} it. 
And now that he thinks about it, Carzain has been overfond of killing for a long time. 
He was also too proud for his own good when he told of the mercenaries he had killed in Heropond last year. 
Ilcas fears Carzain is turning into a bloodthirsty maniac. 





\subsubsection{Ilcas wants to feed his sword}
In truth, Ilcas killed the prisoners not just for the sake of punishment, but also to feed his sword. 
\Telderain{} hungers for blood and soul energy (although it does not eat entire souls; it just leaves them somewhat drained), otherwise it goes crazy and tries to drive Ilcas crazy, too.  

But he doesn't tell this to the Redcor. 
He doesn't want them to know that he wields a black magic sword filled with blood-drinking \daemons. 





\subsubsection{Later Ilcas lets a hostage die}
Later, Ilcas lets a hostage die. 
Perhaps a child, or an egg. 
He knows that sacrificing the hostage to kill the villain is preferable to letting the villain get away. 
But some of the dumbass Redcor, such as \Racel, are unwilling to see that. 





\subsubsection{Ilcas cares for his \nycans}
At some point, Ilcas Northstar endangers the rest of his companions for the sake of one of the \nycans. 

Some of the Redcor are butthurt about this.

\begin{prose}
  \Esmerel: 
  \ta{%
    You would endanger all of our lives for the sake of an \emph{animal}?}
  
  Ilcas: 
  \ta{\Matron, maybe I owe it to you to make our position clear. 
    Let me give you a run-down of my loyalties, in descending order of priority. 
    One: the Imetrium. 
    Two: the Telcastora clan.
    Three: my wife and children.
    Four: the \nycans.
    Five: you people.}
  He looks at Curiet and jokingly adds.
  \ta{Six: Serpentin.}
  
  Curiet:
  \ta{What? Why I am at the bottom?
    All of them are Vaimons! They can defend themselves. 
    I am a defenseless civilian!}
  
  Ilcas looks at him.
  \ta{Point taken. 
    Correction. 
    Five: Serpentin. 
    Six: you Vaimons.}
\end{prose}









\subsection{\Teshrial and \Ishnaruchaefir}
\target{Ishnaruchaefir kills Teshrial}
\Ishnaruchaefir{} has taken \ps{\Teshrial} \quo{bait}. 
He arrives in the dead garden in answer to the challenge. 

\Teshrial{} has several trump cards:
\begin{itemize}
  \item Astrology. 
  \item The Achilles Heel. 
  \item The Shroud.
  \item \Noggyaleth.
  \item \NeoResphan metamorphosis. 
\end{itemize}
    
\Ishnaruchaefir{} uses the spell \word{\hs{khestni}} once during the fight. 
It hurts \Teshrial, but does not kill him. 

\Ishnaruchaefir{} wears \hs{ward runes}. 
(Remember to read about \hs{ward runes}.)

Maybe the \resphain wear \hr{Glass armour}{\armour made of glass or crystal}.

Read about: 
  \hr{Resphan}{\resphain},
  \hr{Dragon}{\dragons},
  \hr{Teshrial}{\Teshrial},
  \hr{Ishnaruchaefir}{\Ishnaruchaefir},
  \hr{Resphan martial arts}{\resphan martial arts} (\Teshrial{} walks the \hs{Path of Ice}),
  \hr{Resphan equipment}{\resphan equipment},
  \hr{Weapons}{weapons},
  \hs{technology}. 
Read about \hs{Chaos magic}, and remember to invoke \Sethicus and \Tiamat. 

\Ishnaruchaefir \hr{Ishnaruchaefir bleeds in Nadir}{bleeds and looks terrible} when he is in the Nadir. 
Every spell he casts causes more wounds to spring open on him\dash{}he pays for his magic with blood and pain.

\Ishnaruchaefir uses some magic, but not so much. 
His magical power is depleted during his Nadir; his physical strength is much more intact (although not quite intact). 
So he mostly fights using his physical strength. 

The Cabal plan is nearly complete. 
The Cabalist \Malcur venture, like \Secherdamon's plan, also coincides with \Ishnaruchaefir's Nadir. 
The \noggyaleth have grown numerous and large and powerful.

Throughout the section, have throwaway references to \hr{Mystic names}{mystic names and places}, like Shung. 




\subsubsection{\Menessiaraid looks on}
\Teshrial{} brings one spectator to the fight: 
His good friend \Menessiaraid. 

He only brings one spectator. 
Otherwise he fears \Ishnaruchaefir, suspecting an ambush, would refuse to fight. 
On the other hand, they both knew they could not fight in private. 
Rumours of this fight have been going all over the place among the dynasties. 
They would need to have \emph{someone} there to witness it. 

So \Menessiaraid{} is there alone, and he does not fight. 
He is a skilled telepath, and his head is crowded full of powerful \resphain{} who want to see the fight through his eyes. 

This means there are fewer \resphain{} left to keep an eye on \Forclin{} and \Malcur. 
Which is exactly what \Ishnaruchaefir{} and \Secherdamon{} want. 





\subsubsection{\ps{\Teshrial} perspective}
The battle should be seen from \ps{\Teshrial} perspective. 
He tries every means at his disposal to defeat \Ishnaruchaefir, and more than once he thinks he has succeeded, but \Ishnaruchaefir{} keeps getting back up. 
Describe \ps{\Teshrial} anguish when he dies. 





\subsubsection{They fight in humanoid form}
\Ishnaruchaefir{} arrives. 
He does not carry his glaive, \Rystessakhin, but only a pair of \skekrathuins. 

They meet in a fairly tightly Shrouded layer of the Realm. 
They are forced to fight wearing \quo{\hs{Masks}}. 
This means that \Teshrial{} has something of an upper hand. 
He is \hr{Immortals inside the Shroud}{not as weakened as \Ishnaruchaefir{} is by the Shroud}. 

First they fight in humanoid form. 
\Teshrial{} challenges \Ishnaruchaefir{} to come down and face him in humanoid form. 
\Ishnaruchaefir{} accepts. 

\Teshrial{} is comparatively young and inexperienced. 
He has never seen \Ishnaruchaefir{} in combat before and underestimates him. 
He mocks \Ishnaruchaefir, calling him a decayed, outdated relic of the far past. 
He believes he can out\manoeuvre the old, bitter, set-in-his-ways \shaeeroth.
He underestimates him. 
Otherwise, \Teshrial{} would have realized how fucked he was, and would have fled. 

Even so, the battle is non-trivial, even for \Ishnaruchaefir. 

They duke it out. 
\Teshrial{} loses. 

At the beginning of the fight \Ishnaruchaefir{} is cool and calm. 
He keeps up this \facade{} as long as he is fighting in humanoid form. 
But when he reverts to \draconian{} form to battle the \noggyaleth{}, and later Mutant-\Teshrial, he lets loose all his \draconian{} fury and lets his hatred against the \resphain{} guide him. 

\Ishnaruchaefir{} does not use so much magic. 
His physical prowess in humanoid form is enough to defeat \Teshrial{}. 
His body is wicked-psycho-tough and durable on its own. 
It is part of his tactic to rely on physical strength as long as possible, thus conserving his magical reserves for when he really needs them. 
He doesn't go all-out magic until mutant-\Teshrial{} shows up. 





\subsubsection{\Noggyaleth}
\Teshrial{} is losing. 
So he pulls out one of his trump cards: 
He summons his \noggyaleth. 

The \noggyaleth have been burrowing through the ground beneath \Malcur, corrupting it.
Read about how the \noggyaleth \hr{Noggyal corruption}{corrupt the planet}. 

Now the \noggyaleth, having long hidden beneath the earth, burst forth.

When \Criseis sees a \noggyal:
\citeauthorbook[p.147]{JohnGlasby:TheOldOne}{John Glasby}{The Old One}{
  Even in retrospect it is not possible to convey in words the nature of that monstrosity which squeezed its vast bulk through the gaping abyss.
  It held a hint of noxious plasticity, of writhing tentacles which changed their number and shape.
  But more than anything, I had the impression of gigantic size, that huge as that part of it looked where it almost completely blocked the opening, there was an infinitely greater bulk mercifully hidden from us.
  
  [\ldots{}]
  
  But I know there was nothing imagniary of halucinatory about the black, coiling tentacle that seized Dorman around the waist and bore him, kicking and screaming frantically, into the gaping, beaked maw which appeared as if from nowhere beneath that single glaring red eye!
}

\Ishnaruchaefir{} fights them. 
\Teshrial{} stands back and supports them with spells. 

Have a description of the monsters that swarm out to attack \Ishnaruchaefir. 

The \noggyaleth cause the very earth to rise and attack \Ishnaruchaefir.

\citeauthorbook[p.290]{DavidDrake:ThanCursetheDarkness}{David Drake}{%
  Than Curse the Darkness%
}{%
  Pulsing, rising, higher already than the giants of the forest ringing it, the fifty-foot-thick column of what had been earth dominated the night.
  A spear of false lightning jabbed and glanced off, freezing the chaos below for the eyes of any watchers. 
  From the base of the main neck had sprouted a ring of tendrils, ruddy and golden and glittering overall with inclusions of quartz. 
  They snaked among the combatants as flexible as silk; when they closed, they ground together like millstones and spattered blood a dozen yards up the sides of the central columns.
}

But \Ishnaruchaefir is prepared. 
He has prepared spells that breaks their hold over the earth. 
He forces them to come out in the flesh, without the protection of the earth, and fight him naked. 
They do. 

See the section on \hr{Magic visuals}{magic visuals}, especially \hr{Summoning magic visuals}{summoning}.

\Teshrial{} has been fooled into believing that his \noggyaleth{} are a secret trap which \Ishnaruchaefir{} doesn't suspect. 
He keeps them hidden and only summons them in the last minute when he really needs them. 
But unbeknownst to him, \Ishnaruchaefir{} has predicted this move. 
And the fight has been planned so it coincides with \Psyrex-tachi's summoning of \Nithdornazsh. 
This means that the \noggyaleth{} are disoriented and weakened and slow to answer \ps{\Teshrial} call. 
This gives \Ishnaruchaefir{} plenty of time to prepare for them and pick them off one by one when they arrive. 
They cannot lie in wait and ambush him as \Teshrial{} wanted them to. 

\Ishnaruchaefir{} assumes his true, \draconian{} form to fight them. 
Insert an epic description of the mighty \dragon{} here, a la \bandsong{Bal-Sagoth}{Black Dragons Soar Above the Mountain of Shadows}. 

\Teshrial{}, desperately intent on vanquishing \Ishnaruchaefir{}, summons reinforcements, bleeding the city dry of Cabal monsters. 
This is what \Ishnaruchaefir{} wants, because it leaves \Malcur vulnerable to \ps{\Psyrex}{} spell. 

When \Ishnaruchaefir{} fights, he has bound \daemons{} that whirl around his body and act as \armour and weapons. 
They are black, gray, white, purple and blood red. 
Compare to the battle between Fulgrim and Ferrus Manus in \authorbook{Graham McNeill}{Fulgrim}. 

The \noggyaleth grab on to \Ishnaruchaefir with their sucking mouths and grasping limbs/pseudopods.
They drag him down and engulf and swallow him.
Then they try to drown and crush and devour and digest him.

\lyricstitle{Draft excerpt from the chapter \quo{What Slithers Beneath}}{
  %The crushing, drowning sensation had passed, and Rian felt like he was thinking clearly again. Yet the scene before his eyes still seemed more like a hazy dream than reality. 
  Rian was not sure if he was awake or dreaming. 
  \tho{I hope I am dreaming.} 
  Some veil like dark smoke obscured the garden, hiding the black one from view. %, giving him only vague glimpses of the black one. 
  
  Then there came a monstrous sound. 
  
  Roaring. 
  
  Shrieking. 
  
  Groaning. 
  
  Rattling. 
  
  From not one throat, but many. If, indeed, things capable of making such sounds had anything as familiar as throats. 
  
  It came from deep beneath the earth. It came from all around him. It came from inside his head. But above all else, it came from the darkened garden. From the centre of the opaque haze. 
  
  He heard a voice, then. Deep, growling, inhuman. But definitely a voice, speaking powerful words in an alien tongue. 
  %It spoke 
  \tho{The voice of the sorcerer.} Rian could not remember how he knew this, but he though that un-\scathaese{} voice was somehow that of the dark-scaled sorcerer. 
  
  And a faint, distant howling, barely audible above the monstrous roaring. \tho{Or did I imagine it?}
  
  There! Within the cloud, a flash. The glinting of metal. 
  
  \tho{The scythe\ldots{}} 
  
  The groaning noises intensified. 
  
  The battle had begun. 
  
  %And there! A glipse of what he thought was the black one
  Terrible sounds of combat could be heard. 
  The swish of steel through the air. 
  The grinding of huge jaws. 
  The clash of massive bodies. 
  Grunts of pain. 
  And terrible words in no \human{} tongue. 
  
  And through the fog, through the blur that veiled everything, he saw glimpses. Flashes of light\ldots{} \tho{Lightning? Fire?} 
  The writhing of towering things\ldots{} \tho{A worm? Worms?}
  And black scales. Immense dark wings. Claws. Teeth. Horns. Blades. 
  The splattering of alien blood. 
  
  Rian did not know how long the battle went on. It could have been heartbeats, or a whole afternoon. \tho{Or a whole night, if I am dreaming.} 
  
  Then suddenly, a high shriek ripped through his ears like a knife. 
  
  The shriek became a rattle. 
  
  The thrashing of a huge body, or bodies.
  
  Seething, like boiling water. 
  
  Thumping. 
  
  Then silence. 
  
  A long moment passed. 
  
  \tho{%
    Is the battle over? 
    
    Who won?}
  
  The silence stretched. Again he had to struggle to stay awake. The world swam before his eyes. \tho{Must not doze off. Must not.}
  
  Then, movement. 
  
  Rian tried to focus his blurred eyes. At the edge of the garden, something. Something was emerging. 
  
  Black. With a great bladed weapon. 
  
  \tho{The sorcerer. He prevailed.}
}





\subsubsection{\Teshrial summons \malgryph}
When \Teshrial starts to conjure the \malgryph, \Ishnaruchaefir acts afraid and tries to stop the summoning.
\Teshrial realizes it will be more difficult than he thought to summon the \malgryph, so he uses his secret weapon and transforms into his monstrous form.
This gives pause to \Ishnaruchaefir, for it is an unexpected move. 
\Teshrial is able to push back \Ishnaruchaefir and overpower him for a while. 
Long enough to buy time for himself to summon his \malgryph.
(Make sure \Teshrial looks heroic and self-sacrificing here. He does not like turning into a monster, but he is noble and selfless and does it anyway. He thinks of his beloved and hopes she will forgive him for the way he has defiled his own body.)
But \Ishnaruchaefir laughs and casts his own spells.
He takes control of the \malgryph and turns it against \Teshrial and his worms.





\subsubsection{\Teshrial mutates}
The \noggyaleth{} are vanquished. 

\Teshrial{} now uses his last resort: With an experimental spell he absorbs his servant monsters into himself and merges with them, thus mutating into a colossal monster. The spell is dangerous and quite likely irreversible\dash it might drive him mad, prevent him from changing back and dooming him to spend the rest of his life as an insane abomination. 

We see \Teshrial{} from the inside. He knows the danger and fears it, but he is consumed by his eagerness to slay the mythical \Ishnaruchaefir{} and prove his worth, so he can advance in the hierarchy of the \ketherain. 

The mutation thing is the only one of \ps{\Teshrial} traps which \Ishnaruchaefir{} failed to anticipate. 
It startles him and throws him off balance for a moment. 

\Teshrial{} suggets that they drop their \quo{\hr{Masks}{masquerade}}. 
This surprises \Ishnaruchaefir{}. 
It is rare for a \resphan{} to be the first to suggest to \quo{drop the masquerade}. 

\Teshrial begins his mutation spell. 
It is a tremendous and powerful spell. 
It is mentally traumatic for him. 

\citeauthorbook[p.254--255]{HPLovecraft:TheBlackTomeofAlsophocus}{H. P. Lovecraft}{%
  The Black Tome of Alsophocus%
}{%
  Again I made the five concentric circles of fire on the floor, and standing in the innermost one, invoked powers beyond all imagining with an incantation so inconceivably terrible that my hands trembled as I made the mystic passes and symbols.
  The walls dissolved and the great black wind swept me away through dark gulfs of space and grey regions of matter.
  I \travelled faster than thought, past unlit planets and vistas of unknown realms which swirled and shifted across immensurable distances; the stars flashed by so rapidly that they appeared as gossamer-fine threads of brightness interlaced across the universe, minute shooting stars of brilliance shining against black aether that was darker than the fabled depths of Shung.
}

The mutation takes time. 
But \Ishnaruchaefir{} does not take advantage of this. 
He politely stands back and gives \Teshrial{} time and room to transform. 
He is curious and wants to see this new weapon in action and test its mettle against his own. 
Knowledge is power, and he is willing to take even foolish chances to gain knowledge. 
He knows that if he has underestimated \Teshrial, he may well perish. 
But he chances it. 

\Teshrial{} mutates. 
He becomes a \neoresphan and comes to \hr{Neo-Resphan appearance}{look like one}. 
Then, using his newly \hs{increased vampire powers}, he absorbs the bodies of the living and dead \noggyaleth in order to grow to huge size.

\Menessiaraid gazes upon \ps{\Teshrial} mutated form. 
\Menessiaraid is horrified by the sight, but also impressed.
He sees him as an awesome, magnificent angel, both terrible and beautiful.
A vision of the greatness, potential and future of the \resphan race.

\lyricslimbonicart{Beyond the Candles Burning}{
  I am a dark star rising on the raveous bleaky sky,\\
  a black diamond slunning so deep within the night.
  Maliciously I dwell in a bluish shaded beam\\
  with a stonecold heart into the core of my being.
  
  Beyond the candles burning, beyond all minds eye.\\
  A vast emperic enigma awaits me as I die.\\
  In a graceful dance obscene, in a ring of fire,\\
  I obtain my majesty as flames caressing higher.
  
  Release my spirit, unleash my soul.\\
  From the darkest dungeon, oblivion calls.\\
  In the phallic halls of ancient forlorn\\
  a cold sanctuary in doom is born.
}

\lyricslimbonicart{Solace of the Shadows}{
  I require the solace of the shadows,\\
  so the night can be redeemed.\\
  As the winds of darkness whispers my name,\\
  a kiss of death I receive.
  Nocturnal enchanter, to thine art I yield.\\
  Within the candlelight a rapture is now revealed.
}

In his \neoresphan{} form, \Teshrial{} gains \hs{increased vampire powers}. 
He can now drain \ps{\Ishnaruchaefir} life force even from a distance. 
\Ishnaruchaefir{} realizes this and becomes more careful. 
He figures out that \Teshrial{} now also stands a better chance of absorbing his soul if he should win. 

\Ishnaruchaefir{} realizes that \Teshrial{} \quo{knows} about his Achilles Heel. 
So he pretends to fear this and makes certain to guard the Achilles Heel, to goad \Teshrial{} on. 

\Teshrial{} fights bravely and savagely. 
A battle of Godzilla-like proportions ensues. 
Perhaps \ps{\Teshrial} mutated form resembles the monster Orga from the movie \cite{Movie:Godzilla2000}. 

\target{Ishnaruchaefir impaled by spines}
\Ishnaruchaefir{} has his body impaled by two or three huge spear-like spines. 
One of his hearts is impaled, but \hr{Dragons have three hearts}{\dragons{} have three hearts}, so he can bear it. 

He can fight on fine despite the pain. 
But he doesn't play stoic. 
He growls and pants with pain of his many wounds. 
\Teshrial{} thinks he is finished, so he gets overconfident. 
He steps closer to finish off \Ishnaruchaefir{} and consume his soul to gain his power (he knows he can, with his increased vampire powers). 

\tho{%
  This will make me as great as a \sathariah. 
  Nay, greater!
  With this power I will be greater than even \Azraid!
  All glory will be mine!
  All \resviel{} will be mine!}

But this was what \Ishnaruchaefir{} wanted him to do. 
He is not dead, and now he leaps up and kills \Teshrial. 





\subsubsection{\Teshrial dies}
When \Ishnaruchaefir is just about to kill \Teshrial, he tells him:

\begin{prose}
  \ta{So you sought to use the \malgryph against me, did you? 
    Too late. 
    That would have worked five thousand years ago. 
    But I have grown stronger since then. 
    I have overcome many of my old vulnerabilities.}
\end{prose}

\Menessiaraid hears the above. 
That is deliberate from \ps{\Ishnaruchaefir} side.
He does not want to arouse suspicion. 
He does not want anyone to suspect that the relevant \WanderersInDarknessEmph passages are fakes that he planted. 
So he uses the above as a cover story. 

Have a sad scene from \ps{\Teshrial} POV as he dies. 
He thinks of his beloved, of \hr{Teshrial's date}{the amazing sex she promised him} and of the life they could have had together. 





\subsubsection{\Ishnaruchaefir and \Criseis after the battle}
After the battle, \Ishnaruchaefir heads off to \Malcur.

\Ishnaruchaefir admits to \Criseis that he is displeased with the fact that the \resphain have now learned of his Nadir and how to map it.
\Urizeth still lives, after all.
And she is not fool.
She has probably taken measures to ensure that her discoveries will live on even if she is destroyed. 

But it it of no matter. 
The \resphain were bound to find out sooner or later. 
Now that the \thirdbanewar is looming, \Ishnaruchaefir will have to take a much more active role than he has done previously. 
He could not hope to keep his Nadir cycle secret forever. 
He has taken a great chance and gone into combat during his Nadir this once.
It will not happen again. 

\Ishnaruchaefir confesses in private to \Criseis that the part about \Ishnaruchaefir's \quo{overcoming his old vulnerabilities} was a lie. 
It is \Criseis who pieces together the story and realizes that \Ishnaruchaefir has planted the fake \WanderersInDarknessEmph verses and thus masterminded \Teshrial's quest against him. 
It is she who tells the reader this.
She asks her master if what she suspects is true. 
He refuses to comment, just smiles to himself.





\subsubsection{\Menessiaraid after the battle}
\Menessiaraid goes away. 
He is very sad that his friend has died, and horrified to witness the power and cunning of the Destroyer. 
But he is also sort of hopeful. 
If \Teshrial can turn into such a magnificent monster, then it bodes well for the potential of the \resphan race. 
He goes away awestruck at his friend's courage and greatness, and with high hopes for the future and the Quest for Perfection.









\subsection{\Nithdornazsh rises}
\target{Nith'dornazsh rises}
With Needle killed and with \Achsah{} and all other high-up Cabalists fighting for the Ghost Tower, there is no one to stop \Psyrex{} and his \Malcurian Sentinels.
They successfully invoke their ritual. \Nithdornazsh{} rises into the world of \Miith{}, tearing the Shroud apart around it, and the city with it. 
\Secherdamon{} comes to claim his prize. 

The city is made of living flesh and metal. It gorges itself on the bodies of the inhabitants of \Malcur, adding their flesh to its own monstrous body. Perhaps some modicum of their consciousness remains, leaving them to bleed and moan while imprisoned as part of the city walls. 

In fact, \Nithdornazsh{} is dead, and it is resurrected by absorbing the flesh and souls of the people of \Malcur. It is born of \Malcur and devours its own mother. Remember to have lots of horror about this. Perhaps witnessed by Moro \Cobrel, who escapes at the end. 

During the resurrection ceremony, the people of \Malcur are seized by a madness. The more susceptible among them become insane mutants who run amok, hunting down other people and eating their flesh. 

A witch-storm rips the sky. \Pdaemons{} and \dragon-spawn fly on the winds, raining down terror from above. 

Have a scene with some normal \Malcuric{} citizen or citizens. Possibly Rian. He runs through the city, literally seeing Hell erupt around him. The stones and the very earth become alive underneath him, coming up to swallow him. 

Hordes of warriors and \pdaemons{} swarm out of thin air\dash out of the Beyond, which is no longer Beyond, but right here! They rampage through the city, slaughtering and enslaving the people. 

\lyricsdimmuborgir{Architecture of a Genocidal Nature}{
  Emerged from the depths of the Earth, gasps.\\
  It rages against mankind, to annihilate the Earth and worse.\\
  It spills the blood like rain. The beauty of Death it represents.
}

\citeauthorbook[p.76]{RPG:Warhammer:TheEmpire}{Alessio Cavatore}{
  Warhammer: The Empire
}{
  The seething Realm of Chaos swept over the city, engulfing it, and Praag was changed forever, its stone walls and buildings melding into hellish and inhuman shapes. 
  Those citizens unlucky enough to still be alive were swept into the maelstrom, their living bodies fused into the walls of the city itself, so that it was no longer possible to tell man from stone. 
  Distorted faces leered from the walls, agonized limbs writhed from the pavements and pillars of stone shrieked in madness with voices that once came from \human lips. 
  Praag had become a living nightmare and a grave warning of what lay ahead should the Chaos armies conquer the land.
}

\lyricsbalsagoth{Witch-Storm}{
  The skyqueen of the dead rides forth,\\
  black storm-borne steeds, immortal blood.\\
  Hark to the striking of the winds, \\
  the moon burns black as slaughter reigns.\\
  Witch-Storm!
}





\subsubsection{\Malcuric{} bullies die}
Have a scene with some \Malcuric{} thugs mugging a poor guy and being mean to animals. Then the \daemons{} come and give them a horrible, painful death, absorbing their souls into even more torment. 

The victim and animal run away and survive. 

Compare to the anime \cite[episode 6]{Anime:TokyoMajin} and \cite{Anime:ElfenLied}. 





\subsubsection{Comparison with Carcosa}
The following scenes from \cite{RPG:CallofCthulhu:GreatOldOnes}, a supplement to the RPG \cite{RPG:CallofCthulhu}, may give an idea of the atmosphere I want to evoke. 

\lyricstitle{\emph{The Great Old Ones} p.70-72}{
  [The prisoner of Carcosa] is free to do anything while awaiting rescue or madness in dark Carcosa. The alienness of this city of towering black buildings costs 1/1D10 SAN per day. 
  Fill the prisoner's time in the city with odd occurences:
  
  \begin{itemize}
    \item 
      a keening voice wailing a lonely dirge, the source of which can never be found;
    \item 
      intermittent wingbeats of great unseen things in the thick clouds overhead; 
    \item 
      a slithering wave of fog which tirelessly pursues the prisoner through the damp empty streets;
    \item 
      occasional footsteps or whispering voices in the streets of the abandoned city;
    \item 
      a glimpse of a shadowy figure down the street, where no one can be found;
    \item 
      nigthmarish splashing in the waters of the lake;
    \item 
      noises whose sources elude vision because of the thich fog;
    \item 
      a glowing Yellow Sign in the waters of the lake. 
  \end{itemize}
  
  [\ldots{}]
  
  As the cultists chant the ritual, thick waves of fog roll in from the lake, then the lake itself swells and grows larger, and the water takes on an oily sheen. 
  The ground gently quakes and stretches. 
  Suddenly the investigators find themselves standing on the outskirts of an alien city, at the edge of a lake much larger than the one they had been observing. 
  The swamp has vanished. 
  The night sky is dull white, and in it black stars shine in unfamiliar patterns. 
}





\subsubsection{\Secherdamon{} rejoices}
\Secherdamon{} rejoices at the fruition of his grand plan. 

\lyricsbalsagoth{Beneath the Crimson Vaults of Cydonia}{
  This red charnel pit of primal horror, \\
  howling black ecstasies to the void.
  Ancient and divine, older than the hidden Icosahedron, \\
  now rebirthed beyond the chaosphere.\\
  Rise\ldots{} rise and destroy!
  
  Hatred, carnage, slaughter, havoc, chaos, murder!\\
  I am become the devourer of all life!
  
  Phobos, Deimos! \\
  The moons' rays liquefied in these blood red pyramids.\\
  In the shrines of abomination, black tongues rapt with blasphemy.\\
  Chaosphere, watchtowers, genesis, Cydonia\ldots{}\\
  The Abyss yawns wide!\\
  Spirit of the carrion-thronged battlefield, open wide thy gate!
}

Monstrous \daemons{} who worship \Secherdamon{} appear and attack.

\lyricsbalsagoth{Beneath the Crimson Vaults of Cydonia}{
  Unruly evil!\\
  Colossal shapes etched against the moons, \\
  supine obeisance 'fore the mound. \\
  Accursed fiends hail the Slitherer, \\
  abhorrent jaws drooling lunacy.
}

\Secherdamon{} himself manifests as a ghost-like avatar. 

\lyricsbalsagoth{Beneath the Crimson Vaults of Cydonia}{
  The Abyss yawns wide\ldots{} Claws sharpened on the dead.\\
  The Abyss yawns wide\ldots{} Ensanguined fangs agleam.
  
  Great shadow, awaken and eclipse the suns of a thousand worlds\ldots{}\\
  Slumbering 'neath these crimson vaults, \\
  behold the majesty of the Outer Darkness!\\
  Praise the Z'xulth!
}


\lyricsbalsagoth{Beneath the Crimson Vaults of Cydonia}{
  Fell Worm of the Black Galaxy, \\
  awaken and descend without pity upon the Tellurian sphere!\\
  Destroy the flaccid priests of the newborn usurper faiths.\\
  Sweep away the thralls of the cruciform stave!\\
  Crush the lackeys of the corrupted hexagram!\\
  Devour the slaves of the eastern crescent!\\
  Crush them, grind them, slay them all!\\
  Plague-blessed, flay them alive!
  
  Now, behold in terror what waits beneath the crimson vaults of Cydonia\ldots{}
}







\subsection[Psyrex thanks Tiroco]{\Psyrex{} thanks \Tiroco}
\target{Secherdamon lets Tiroco and Icor choose their fate}
\Psyrex{} approaches \Tiroco{}. He thanks her for her help and gives her an offer to serve her. Knowing that he is a Hellish enemy of the Light who betrayed her and her entire city, she refuses and curses him. He expected as much. Still feeling that he owes her a favour, he lets her choose her fate. She has proven herself skillful enough that he cannot simply let her go, so if she will not join him, she must die. %He explains that he has shown mercy to capable enemies in the past and always lived to regret it. 

He remarks that: 
\ta{%
  Were I \Ishnaruchaefir, I might fall prey to sentimentality and spare you. 
  But I am wiser than that. 
  Show mercy to a capable enemy and you will live to regret it.}

But since she did it all for \ps{\Icor} sake, she deserves at least to be with him. So he grants her a choice: He can kill her and release \ps{\Icor} soul, letting them go into the Light as they wish. Or he can destroy them permanently, so that they will know peace. He advises them that the Light is not what they believe it to be, and that he would recommend annihilation rather than a return to slavery. 

%But she does not want to listen to him and chooses the Light. He is somewhat sad to oblige her. 
\Tiroco{} does not want to listen and chooses the Light. But \Icor{} objects. He has done some thinking and come to the conclusion that the \Sephiroth{} are not as good as people think. So he votes for annihilation. After a while, \Tiroco{} consents. 

Have a sad yet somewhat happy scene where the two drift off into oblivion in each other's arms. 

Then \Psyrex{} seats himself on his throne, surveying his new conquest. He muses that the \charade{} is broken. It is also interesting that \Ishnaruchaefir{} has returned to the \secretwar. 

Afterwards, \Psyrex{} comments to \Secherdamon{} (or \Nzessuacrith, or another) that it was difficult to save them. 
The \sephiroth{} had a strong grip on them. 
Moreover, the \quo{river} of \Iquin{} has grown stronger in recent years. 
Wilder, more raging. 
As if it is breaking out of control. 
(The Shroud is \hs{unravelling}.) 

\Psyrex{} fears the \sephiroth{} and \iquin{} and Iquinianism. 
Perhaps it should just be explained and made to seem as if he dislikes the Iquinian religion and does not want to give it any more souls and thus more power. 
Just keep it as a subtle undertone that he really believes an afterlife in \iquin{} would be painful for them\dash a fate he would not wish upon them, \honourable folk as they are. 
(After all, I don't want the reader to know that \Iquin{} is evil just yet.)





\subsubsection{Why do you do this?}
Before she dies, \Tiroco{} asks \Psyrex: \ta{Why do you do this?}

\Psyrex: 
\ta{For survival. 
  For the future of my people. 
  And your people as well, \scatha. 
  For our kind's survival, against the \humans{} and their creators.}

\Tiroco{} wonders who the \pps{\humans} \quo{creators} might be. Then she dies. 







\subsection[\Ishnaruchaefir returns to Malcur]{\Ishnaruchaefir returns to \Malcur}





\subsubsection{\Criseis meets \Psyrex}
Perhaps \Criseis meets \Psyrex in \Malcur.
They might call each other \quo{cousin} (if \Psyrex is a \scatha, that is). 





\subsubsection{Saves Rian and Neina}
\target{Ishnaruchaefir saves Rian and Neina}
Rian and Neina \hr{Rian and Neina are separated from Moro}{have gotten separated from Moro}. They flee and huddle together in the collapsing, mutating city. 

Then \Ishnaruchaefir{} arrives in \Malcur to bear witness to the rise of \Nithdornazsh. 

It is \Criseis who espies Rian.
She begs \Ishnaruchaefir to go down and help Rian by killing or scaring away the monsters that are about to eat him.
\Ishnaruchaefir \hr{Ishnaruchaefir's compassion}{finds sympathy for these small creatures} and obliges her.
Rian and Neina cower in horror of the vast \dragon.
\Criseis then dismounts and leads the \humans to safety while \Ishnaruchaefir goes to talk to \Nzessuacrith or something.

Rian sees something that he recognizes in \ps{\Ishnaruchaefir} eyes, although he isn't sure what. 
He notices that the \dragon{} has huge, bleeding wounds that go all the way through his body (after being \hr{Ishnaruchaefir impaled by spines}{impaled on huge spines}). 

\Ishnaruchaefir{} descends, lands near Rian and assumes humanoid form. Rian recognizes him: It is the mighty warrior-mage who {once inspired him to become a better man}. 
\Ishnaruchaefir is aware of this to some extent, and is glad to see that Rian has indeed become a better man. 

\Ishnaruchaefir{} is wise and can read much of Rian's story in his eyes. He sees immediately that the young man has fought bravely to save his beloved. This touches him, for he wishes he \hr{Ishnaruchaefir slays his beloved}{could have done same}. He also feels some kind of responsiblity for them, since he was part of the plan that destroyed their city. 
So, in a \trope{PetTheDog}{Pet the Dog} moment, he decides to save Rian and Neina. 

Some \daemons{} or evil warriors approach. \Ishnaruchaefir{} sends them packing with a single word. He then saves Rian and Neina, sending them through a portal out of the city and into an area that should be semi-safe. 

He sees that Neina is \hr{Neina's story}{badly traumatized}. So he casts a spell on her, erasing much of her memory of the last months. This has the side effect of making her permanently dumber than before. \Ishnaruchaefir{} warns Rian that he has done this, and that many memories have been lost other than those of her captivity (this kind of magic is inaccurate and risky, and it is not \ps{\Ishnaruchaefir} specialty). 

After the spell, Neina is not quite sure who Rian is, but some part of her remembers him and the love she had for him. He is hopeful and trusts that he can nurture her back to health. 

They thank \Ishnaruchaefir{} of all their hearts. Then they go through, hand in hand. 
We see them kissing and embracing, and it is implied that, despite whatever hardships cruel \Miith{} will throw at them, they will live happily every after. 

After the escape from \Malcur, Rian and Neina are forever scarred from witnessing their home transforming into a nightmarish hell from within.

\citebandsong{Nile:AnnihilationoftheWicked}{Nile}{
  Von Unausspechlichen Kulten
}{
  I Dare Not Again Surrender to the Deep Sleep Which Ever Beckons Me.\\
  Lest I in Dread Shudder at the Nameless Things.\\
  That May at this Very Moment Be Crawling and Lurking.\\
  At the Slimy Edges of My Conciousness.\\
  Slithering Forth from the Bowels of Their Infernal Pits.\\
  Worshipping Their Ancient Stone Idols and \\
  Carving Their Own Detestable Likenesses \\
  On Subterranean Obelisks of Blood-soaked Granite.
}

Neina goes mad and remains mad ever after.

\citebandsong{Nile:RamsesBringerofWar}{Nile}{
  Howling of the Jinn:
}{
  Fiendish Insects encircle Me\\
  Howling Wind Wraiths\\
  Surround my disembodied Ka

  Dulcarnon\\
  Hideous Unseen\\
  Speaking in Tongues\\
  Heard only by the Mad

  Shrieking Insects Swarm over Me\\
  Suffocate Me\\
  Suffocate my Soul

  Majnun I am Empty\\
  Crawling Reptiles Devour my Soul\\
  They utterly and completely Annihilate Me\\
  I can hear the Howling of the Djinn\\
  Echoing in the mountains of Kaf
}





\subsubsection{Rian and Neina Rescued}
Rian has gotten Neina out, but they are both hopeless.
They are sure it is the end of the world.
Rian feels he can only pray for salvation before he inevitably dies. 
But then \Criseis comes and saves them. 
She takes them out of the city.

Rian asks \Criseis: 
\ta{But where are we supposed to go form here? 
  If it's the end of the world\ldots{}}

\Criseis: 
\ta{It is not the end of the world.}
She looks thoughtful and worried for a moment.
\ta{At least, not just yet.}

Rian: \ta{But the city\ldots{}}

\ta{It is the end of \Malcur, but not the world. 
  Listen.
  I have done what I could for you.
  You may not be unscathed, but at least you are alive.
  You have a chance to make a new beginning.
  Good luck.}





\subsubsection{Saves Moro}
Maybe \Criseis saves not only Rian and Neina, but also Moro \Cobrel.

Moro has not been idle.
After realizing that the battle for the city is lost, she has worked hard to save as many people as possible. 
She has herded a lot of innocent people together and used her magic to protect them as best she could. 
Perhaps she is rejoined by Rian and Neina (if they are not dead).

But Moro is getting desperate.
She cannot get out of \Malcur. 
She fears it is only a matter of time before they all suffer a horrible death. 
But then \Criseis shows up and shows compassion. 

\Criseis cannot save them all alone.
She has to appeal to \Ishnaruchaefir (her master and god) and pray for him to grant her power so that she can open a tunnel and let them all escape. 
\Ishnaruchaefir, in a fit of compassion (a \trope{PetTheDog}{Pet the Dog} moment), indulges her and saves the people. 





\subsubsection{Meets \Nzessuacrith}
He meets \Nzessuacrith{} in \Malcur. 
She has just returned from the Ghost Tower, where she was hurt in combat and scared off by a wave of \sathariah{} power. 

She is in a bad shape after her fight. 
She has big, open, bleeding wounds. 
Her silvery scales are blackened and dirty. 
Many of her \hs{ward runes} are depleted. 
She is so wounded that she dares not go into the Shroud. 
Its constricting influence would cut off the flow of life-giving \xsic{} energy that is currently sustaining her, leaving her with a masked body too weak to cope with the bleeding wounds. 
That might (temporarily) kill her. 

\Nzessuacrith: 
\ta{The only power at their disposal that can match ours. 
    Stolen from us as it is.} 

But she is ashamed of the fact that a little \sathariah{} smell was enough to scare her so and force her to flee. 
She does not openly admit it to \Ishnaruchaefir, and only to \Secherdamon{} after treading water for a while. 

The \dragons \hr{True Draconic signifies emotion}{spoke \TrueDraconic out of emotion}. 

They do not \hr{Dragon violence}{snap and lash out at each other}, as a friendly pair of \dragons{} would. 
They just stare coldly. 
They don't like each other. 

\Ishnaruchaefir{} is intrigued by the \sathariah{} part. 
He flies out to the Ghost Tower general area to investigate. 
He is wiser than she and can see \matrices{} and \vertices{} more clearly. 
He tracks down Carzain to say hello. 

Maybe \Ishnaruchaefir{} is the one who tells \Nzessuacrith{} about \ps{\Secherdamon} true plan, and how she was just a diversion. 

In this scene, \Nzessuacrith{} for the first time calls \Ishnaruchaefir{} \quo{father}. 

\Nzessuacrith{} and \Secherdamon{} call \Ishnaruchaefir{} \quo{Exile}. 
They refuse to afford him the \honour of pronouncing his name. 

It is mentioned that the whole thing is a major breach of the \charade. 

\Ishnaruchaefir initially believes \Secherdamon's plan takes place in \Malcur.
Then he falls for the \Forclin decoy when it is unveiled. 
He is impressed to learn that both were decoys. 
When he talks to \Nzessuacrith, he admits this. 

\Secherdamon tells \Nzessuacrith that she took her damn time.
She only assumed her true form at the very last minute. 
She \hr{Takestsha will not become Nzessuacrith too soon}{was well aware of this}.
She retorts that she is not one to flaunt the Unspoken covenant. 

Remember to explain the role of \Ambrose \Onatol and \Jirad Tantor and the diary. 





\subsubsection{He meets Carzain}
While Carzain-tachi are making their way through the \Wylde{}, \Ishnaruchaefir{} appears. 

Maybe he secretly sends monsters to attack them, then to come in and rescue them. 

He talks to them.

Remember that \Ishnaruchaefir{} does not speak \Velcadian. 
He does, however, speak Imetric and Vaimon. 

\Ishnaruchaefir{} is known and infamous to the Redcor. 
\Esmerel{} recognizes him and scorns him as a evil sorcerer. 
Northstar recognizes his name, but is uncertain as to whether he is good or evil. 

(Why? 
What is \ps{\Ishnaruchaefir} back-story? 
Who knows him? 
Who likes him? 
Who hates him? 
Who fears him?)

% \Ishnaruchaefir{} then somehow leads them back to the Ghost Tower. 
% He recognizes that Carzain is the \vertex{} they are all looking for, and since he is on the Pelidorians' side, \Ishnaruchaefir{} figures that Carzain will stir up some ruckus at the Ghost Tower, disturbing \Nzessuaz{} mission and forcing her to bring out the big guns. 
% Which will attract more attention to her, drawing it away from \Malcur. 
% (\Ishnaruchaefir{} wants to help \Secherdamon{} resurrect \Nithdornazsh, and will happily sacrifice the Ghost Tower to that end.)

Carzain: 
\ta{Who are you? 
    Whose side are you on? 
    Are you good or evil?}

\Ishnaruchaefir{}: 
\ta{I\ldots{} simply am what I am.}

Carzain: (Thinks about it for a moment.) 
\ta{What a stupid answer. 
    What a feeble way of dodging the question!}

\Ishnaruchaefir: \ta{Hm. True.} Smiles. 
\ta{Very well. 
    The truth: 
    I do not know. 
    I fear you will have to judge for yourself.} 

See, this was a test. 
\Ishnaruchaefir{} \hr{Ishnaruchaefir's inanities}{sometimes does this}; spouting inanities to see if the other guy accepts them as profound wisdom or recognizes them as inane and calls him out on it. 

\Ishnaruchaefir{} tells Carzain: 
\ta{I predict that you and I shall meet again. 
    Perhaps as allies, but more likely as enemies. 
    The world is cruel, do you not agree?}

After Carzain-tachi depart, \Criseis{} asks him: 
\ta{Why did you not kill him, my Lord, knowing what he is?}

\ta{What excitement would there be left in life if I were to slay all my enemies when they be still small?}

\Ishnaruchaefir{} thinks to himself: 
\tho{%
  He has the \sathariah{} smell. 
  I ought to deeply hate him. 
  And I do.
  
  Maybe \Criseis{} is right. 
  Maybe I should kill him. 
  But what the Hell?
  If I kill him now, there will just be another Scion in a couple of centuries. 
  Might as well let him grow up and deal with him now.} 







\subsection{Epilogue}
\subsubsection{Carzain}
Carzain and his party finally set out for \Redce. 





\subsubsection{Charcoal is rewarded}
\target{Achsah rewards Charcoal}
Charcoal is rewarded by \Achsah{} for \hr{Charcoal at the Ghost Tower}{helping to guard the valuable Ghost Tower}.

%If it's \Achsah, then perhaps she rewards him by giving him the best sex he has ever had.
Perhaps \Achsah{} rewards him by giving him the best sex he has ever had.

Inside, Charcoal is thinking: 
\tho{I'm glad she doesn't mention the things I fucked up in \Malcur. 
  But, after all, I wasn't present. I can wipe the blame off on Needle\dash and \Achsah{} herself, and even \Teshrial.}





\subsubsection{\Achsah{} curses the \dragons}
\Achsah, on her throne in \Nyx, curses at the audacity of the \dragons. The resurrection of a \Machaic{} fortress on \Miith{} has never been attempted before and is a blatant breach of the \charade. 

She is also baffled by \ps{\Ishnaruchaefir} ingenuity. She had thought that the mystic warrior remained mostly aloof from the \secretwar{} and had not thought that he would interfere so directly, much less to the aid of his brother. And least of all did she expect him to see through the plan with the \noggyaleth{} beneath \Malcur. 

\ta{Damn \Secherdamon, that cheat. And damn \Ishnaruchaefir, that anarchist.}

Still, she is grateful that \Ishnaruchaefir{} has killed \Teshrial, whom she hated. 





\subsubsection{Some people wonder about the \quo{survival}}
Some people, both \human{} and \scathaese{}, speculate on the things they were told about how the feuding factions are allegedly fighting for: 
\quo{Survival. 
  The survival of our race. The \emph{purpose} of our race. We are waging the war that we were \emph{born} to wage.} 





\subsubsection{\Azraid{} muses}
\target{Azraid muses on Exile and Pyre}
\Azraid{} sits and muses about \Ishnaruchaefir{} and his involvement. 
\Azraid{} is the \apex{} of his \matrix, so he detected it immediately when \Teshrial{} was destroyed. 
But he doesn't know the details of the fight. 

He thinks about \Ishnaruchaefir{} helping \Secherdamon. 
\hr{Exile intersecting with Pyre}{The Exile did not intersect with the Pyre}. 
And yet, it still did. 
In intent and effect, if not in the formal, metaphysical sense. 

Remember to have subtle references to \ps{\Azraid} evil hand in all \Azraid{} chapters. 

\begin{prose}
  \tho{But it is the thought that counts. 
    Let that be a lesson for all astrologers.
    The Star-Maps of the Cosmos can reveal much, but it cannot tell us all things that are worth knowing.}
\end{prose}

\target{Azraid learns of spike}
He also thinks about the \vertexspike{}. 
\Achsah{} is $100\%$ sure the \vertex{} is a \sathariah{} aligned with the \hs{Midnight Bat}. 
Which means it must be a Scion. 
Which means there are only two possibilities. 

\Azraid{} is very interested. 
He wants Ramiel back. 
\Azraid{} himself \hr{Azraid dies}{is preparing to die}, so he wants to set up an heir to take over leadership of the \resphan{} race when he is gone. 
Ramiel is a good candidate. 

So \Azraid{} \hr{Azraid protects Carzain}{assumes the role of Ramiel's \quo{guardian angel}}, subtly pulling threads and protecting him from harm while he grows towards his \apotheosis. 
(This is necessary. Many forces conspire to kill Ramiel's Scion.)

\Azraid{} is distressed about the resurrection of \Nithdornazsh, but only in a detached, relaxed way, 
It is an annoyance that will make his work more difficult in the future. 






\subsubsection{\Secherdamon{} and \Nzessuacrith}
\Secherdamon{} and \Nzessuacrith{} talk. 
She mopes over being defeated. 
He had previously promised to come to her assistance if she needed it, so she berates him for not coming. 
He explains that he did indeed come to her assistance, for his actions in \Malcur accomplished more than the capture of the Ghost Tower could possibly do. 
He thanks her for her help, for it was the failure of her initial, more covert plan which forced her to go in with full force. 
This in turn drew all Cabal eyes to her, which allowed him to carry out his plan. 
All of which he had planned in advance. 

He apologizes for having used her. 
\Secherdamon{} and \Nzessuacrith, of all people, understand very well the pain of being betrayed by their own family. 
They are close allies and should be able to trust each other. 
They are more sensitive and considerate than the irreverent \Ishnaruchaefir. 

It is mentioned that the whole thing is a major breach of the \charade. 

\Secherdamon{} muses about \ps{\Ishnaruchaefir} unpredictability. 
He also reveals that \Ishnaruchaefir{} is his brother. 
He is greatly disappointed that \Ishnaruchaefir{} did not die. 
\Secherdamon{} \hr{Secherdamon thinks Ishnaruchaefir will sacrifice himself}{had hoped \Ishnaruchaefir{} would sacrifice himself}. 

After the resurrection of \Nithdornazsh, the Sentinels have a lot of work ahead of them.
They have to restore the city and build new eidola to shape the Shroud and the gateways and facilitate the travel between the Realms that they want. 

The book closes with \Secherdamon{} musing:
\tho{The Unspoken Covenant is broken. 
  Open war is upon us. 
  Perhaps even a \thirdbanewar.
  Yes. 
  The final war is upon us.
  And it has only just begun\ldots{}}















\section{Changes}










\subsection{The Dreaming Predator}
\begin{changes}
  \begin{comment}\paragraph{Prologue}\end{comment}
  \changesitem{Prologue} 
    \Nzessuacrith is horrified and saddened to find that she cannot contact \Secherdamon telepathically like she could before the Shrouding. 
    She can only talk to him using words.
  
  \begin{comment}\paragraph{Wanderer in Darkness}\end{comment}
  \changesitem{Wanderer in Darkness} 
    
    Have throwaway references to \hr{Mystic names}{mystic names and places}, like Shung. 
  
    Remember that the \humans should be various exotic \demihuman varieties.
    
    Have some more \hs{astrology}. 
    
    Mention this:
    \quo{Not even mighty \sathariah warriors like \Nathrach, \Netzachirah or even the great \Morcariel, founder of \CiriathSepher, were able to stand against the wrath of the Destroyer.}  
    
    Allan Balsgaard has the following comment for the chapters \quo{Wanderer in Darkness} and \quo{What Slithers Beneath}: 
    \ta{%
      Jeg ved godt tyskeren i dig protesterer voldsomt, men m\aa{}ske kunne du sk\ae{}rer ned p\aa{} antallet af navne. 
      Det er ikke n\o{}dvendigt at navngive hovedpersonens bror eller lign. 
      Det kan du g\o{}re i appendikset. 
      Husk at ogs\aa{} almindelige mennesker skal kunne l\ae{}se det.
      Desuden. 
      Kan de udf\o{}rlige beskrivelser af guderne, deres relationer og deres kampe ikke d\ae{}mpes lidt? 
      Hvis du virkelig ikke kan spare dig, s\aa{} lav det helst som en fort\ae{}lling om dem uden replikker. 
      Jeg er bange for at guderne bliver for opbrugte og trivielle, n\aa{}r du skal til at skrive bog nr. 4. 
      Jeg tror du risikerer at de p\aa{} et tidspunkt bliver afmystificeret og dermed kedelige for l\ae{}seren.%
    }
    
  
  
  \begin{comment}
    \paragraph{What Slithers Beneath: To Do}
  \end{comment}
  \changesitem{What Slithers Beneath: To Do} 
    Update the reference to when \Achsah met \Ishnaruchaefir.
    Add more details and make it more epic and ominous. 
    Do likewise for the reference to when \Teshrial fought \Zessuruch. 
    
    Do not show \Secherdamon, but do mention him a lot. 
    He is a terrible dark lord (the Serpentine Lord), and \LocarPsyrex is his almost-as-terrible immortal archmage. 
    Make \Psyrex more badass, more menacing, more terrifying. 
    Even the \resphain hesitate to fight him.
    
    Overall, namedrop \Psyrex and \Secherdamon more. 
    
    Have a scene with \Psyrex, the dark sorcerer.
    He thinks about \Ishnaruchaefir and the \resphain.
    He contacts his ally, who in these days goes by the name of \Takestsha. 
    
    He tells her:
    \ta{I have made a discovery.
      I have news that will greatly interest you.
      News of \emph{someone} who greatly interests you\ldots}
\end{changes}









\subsection{On the Wings of \Dragons Unseen}
\begin{changes}

  \begin{comment}
  \paragraph{Screaming in the Dark}
  \end{comment}
  \changesitem{Screaming in the Dark}
    Change the \scathae into \humans.
    
    In the end, Carzain and Vizicar resolve to go find a cute girl for the night. 
  
  \begin{comment}\paragraph{The Mystery of \EreshKal}\end{comment}
  \changesitem{The Mystery of \EreshKal}
    Look at \href{http://limyaael.livejournal.com/167123.html}{Limyaael's \quo{Sounds of the Jungle} rant} for ideas and advice on the forest scenes. 
    
    Replace \quo{forest} with \quo{\hr{Jungle}{jungle}}. 
    
    Read some Robert E. Howard before I write the scenes with Carzain and Tantor in the \wylde. 
    
    Tantor's diary is written in \hr{Rungeran language}{Rungeran}.
    Fortunately \hr{Curwen's languages}{Curwen speaks Rungeran}.
    
    Or maybe it is written in the Vaimon tongue, as Vaimons are wont to do.
    
    Curwen quickly figures out that \hr{Jirad Tantor}{\Jirad{} Tantor} must be a kinsman of the \scarv{} of Tantor. 
    \hs{Most mages are nobles}, after all.
    
    Perhaps \Takestsha takes the form of a strange \hr{Demihuman}{\demihuman} to add to her mystery and allure.
    
    Remember that there are \hr{TBW railroads}{railroads}. 
    Tantor-tachi go by railroad to Gedrock. 
    
    Tantor-tachi have one or more (small) sauropods as draft animals. 
    
    There is a \hr{Wylde border}{\Wylde{} border} around Gedrock, which Tantor passes by. 
    
    Tantor needs to be more racist. 
    When he hears there are \meccara{} in \EreshKal, he scoffs. 
    \Meccara{} are \quo{lower humanoids}, clearly less intelligent and civilized and worthwhile than \humans{} and \scathae. 
    He does not expect much of them, and he certainly does not fear them. 
    
    Charcoal does not call him out on this. 
    Charcoal is a racist himself.

    Everyone on \Azmith knows that monsters and evil spirits and even evil gods lurk in the Wild and in the dark, forbidden places of the world.
    \Jirad Tantor should not dismiss things as superstition.
    Rather, they know they are venturing into a dangerous place full of monsters.
    Still, the forest is not so huge, and it is in the middle of Runger, surrounded by churches and Light-fearing men, so what is the worst that could happen?
    
    When they are in Gedrock, make it clear that life in a village is dangerous. 
    They have \eidola, but even so, the \wylde is always close and threatening.
    The local priests have much work to do in keeping the \eidola blessed and keeping the \wylde out. 

    Remember to see the section about the \hr{Wylde}{\wylde}, and especially the one on \hr{Travelling through the Wylde}{\travelling through the \wylde}. 
    
    Tantor is certain that \EreshKal was not built by \meccaran hands. 
    But his version of the story is very much \coloured by his racism. 
    He looks down on \meccara. 
    
    Be sure to have a clear difference in Tantor's writing style before and after his son's death. 
    Before it, he looks down on the \meccara as \quo{lower \humanoids}. 
    They are repulsive, but he also sort of pities them. 
    And he is impressed and awed by the grandness of the temple. 
    
    After his son's death, he comes to viciously hate \meccara. 
    He curses them for their evil, stupidity, inferiority, ugliness, bad smell and everything he can think of. 
    And he now feels horror rather than awe at the temple. 
    
    Already out in the forest, he and the others see abhorrent things skulking and creeping at the edges of their camp. 
    Ugly midget \humanoids. 
    Probably \meccara. 
    Sometimes the soldiers shoot at them. 
    One soldier hits. 
    The thing shrieks and bleeds, but escapes, and no one wants to pursue the wretched thing out into the \wylde. 
    But they can see the blood. 
    They know it is a living creature that bleeds. 
    That reassures them. 
    A bit. 
    
    Inside the temple they see more of these skulking shapes.
    
    It is important that I hint at more than I show. 
    I want to evoke something like \cite{HPLovecraft:AttheMountainsofMadness}, \emph{not} something like \cite{JohnGlasby:TheBroodingCity}. 
    It is mostly Curwen who sees hints of dark things, for he knows much more of the occult and the dark, forbidden myths than Tantor does.
    He recognizes things from the sinister prehistory of the \scatha/\meccaran race.
    Curwen knew the truth (or parts of it) about how the \meccara degenerated. 
    (See the section about \hr{Meccaran Scathae}{\meccaran \scathae}.)
    
    Allan Balsgaard has this comment:
    \ta{%
      Rigtig fedt. 
      Dog kunne man godt g\o{}re det mere utydeligt hvad templet indeholdt.
      Ankomsten til skoven kunne m\aa{}ske godt v\ae{}re strukket lidt, s\aa{} man f\aa{}r en fornemmelse af rejse og afstand (lidt opvarmning). 
      Hvad med lidt naturbeskrivelser, beskrivelser af almindelige mennesker, b\o{}nner, vagabonder og andre rejsende??
      Samtalen med den gamle kone er rigtig god, dog kunne man godt v\ae{}re blevet sparet lidt for de mange detaljer. 
      Igen antydningens kunst.
      Takestsha er virkelig spooky, hun sover ikke...%
    }
  
  
  \begin{comment}
    \paragraph{The Terror of \EreshKal}
  \end{comment}
  \changesitem{The Terror of \EreshKal}
    Carzain should say something about how evil Morgan Runger is.
    Remember, I want the reader to really hate Runger and root for Pelidor. 
    
    Carzain muses.
    Maybe Morgan just wants to conquer and rule like any other king.
    But his \ishrah is evil. 
    They mean to unleash forces of hideous, world-consuming evil. 
    Carzain fears them. 
    
    Compare to how evil the Wasp Empire is in \cite{AdrianTchaikovsky:ShadowsoftheApt}. 
    
    
  \begin{comment}
    \paragraph{The Gods of \EreshKal}
  \end{comment}
  \changesitem{The Gods of \EreshKal}
    
    Read some Robert E. Howard before I write the scenes with Carzain and Tantor in the \wylde. 
    
    Since the last time he read in Tantor's diary, Curwen has done some research.
    He has found out that this \Jirad Tantor is indeed a real person and a member of King Morgan's \ishrah.
    \Jirad is a younger son of a nobleman, much like Curwen himself. 
    Many mages are like that in countries where the eldest child inherits stuff.
    Firstborn sons often shy away from magic because it is scary and they would rather just rule stuff.
    Most people, even though they may worship their \Archons or gods, fear close contact with magic. 
    Younger sons sometimes decide that magic is their best chance of power and glory.
    Some of them manage to get apprenticed, and some of these manage to become true mages. 
    
    Maybe change Tantor into a \scatha. 
    If so, mention it here.
    
    Replace \quo{forest} with \quo{\hr{Jungle}{jungle}}. 
    
    As soon as Tantor steps inside the entrance to \EreshKal{} he imagines he feels the dust and darkness of uncounted millenia. 
    He believes the building is extremely old. 
    From before the \Human{} Age. 
    Older than Cordos Vaimon. 
    
    It is well-known that the \meccara{} are a younger race than \humans. 
    But Tantor believes that these \EreshKali{} are the caretakers of a tradition of sinister, forbidden wisdom far older than they themselves. 
    A legacy that reaches back to the races and empires that reigned immemorial aeons ago and were forgotten before the first \humans{} set foot on \Miith. 
    
    \EreshKal{} is the ruins of what was originally a \quiljaaran{} city. 
    Tantor refers to the \hr{Myths of vanquished monsters}{myths of Iquinian heroes vanquishing inhuman Elder Races and monsters}. 
    He sees \EreshKal{} as a leftover from the mythical \hs{Age of Gods}. 
    He daydreams of the ancient wars when \humans{} vanquished the serpent men. 

    Even \Jirad Tantor is impressed by the fallen temple of EreshKal and wonders what it was like when it was inhabited and at the height of its power and glory.
    
    The buildings are huge, being \quiljaaran-built. 
    (Read about \hr{QJ architecture}{\quiljaaran architecture} and \hr{Ophidian architecture}{\ophidian architecture}.) 

    \citebandsong{Nile:BlackSeedsofVengeance}{Nile}{
      To Dream of Ur
    }{
      Desolate and Forsaken \\
      Eerily Moaning Dark Winds\\
      Murmur Incantations\\
      Dusk Calls Forth Shadows
    
      Spirits of the Glorious Dead \\
      Lingering, Bound to this Place\\
      They Whisper of Untold Sagas, of Long Dead Cities\\
      the Seven Shining Cities Sacred to the Aphkhallu
    
      Of Ages Past when the World was Young\\
      When Babylon was Blessed of Marduk\\
      and the Sound of her Armies was the Blare of Ominous War Horns\\
      and the Clash of Immortal Cymbals\\
      of Bronze Gates Arrayed in Splendour\\
      and Magnificent Walls of Sunbaked Brick \\
      Of Temples of Marble and Bloodstained Altars
    
      Long Before the Jeweled Throne of Ur\\
      Fell Silent and Turned to Dust\\
      Beneath the Endless Shifting Sands\\
      and the Inevitable Vengeance of the Elements
    }
    
    Charcoal laughs at Tantor. 
    He knows that the \quo{Age of Gods} is still going on and that the Elder Races are very much alive and kicking. 
  
    Describe the sounds and smells inside the \EreshKali{} temple. 
    The smell is musky and very old. 
    
    There are ruins of inhuman machines. 
    It is clear to see that the primitive \meccara are not the original inhabitants/founders/builders. 
    
    \citeauthorbook[p.76 of 138]{KarlEdwardWagner:DarknessWeaves}{%
      Karl Edward Wagner%
    }{%
      Darkness Weaves%
    }{
      Shambling man-sized creatures, who looked like monstrous hybrids of man and frog, stood watching Kane in the shattered chamber of some colossal prehuman structure. 
      Great bronze swords were clutched in webbed fists as they waited in the shadows of the cracked and leaning walls. 
      Slimy water covered much of the floor, and fleshy vines stole through jagged apertures to enshroud looming machines of unguessable nature. 
      A gigantic crystal filled the center of the chamber\dash a sullen dome nearly a hundred yards across, composed of a substance that resembled bloodstone. 
      The scarlet veins of the crystal suddenly seemed to glow with life. 
      Blinding flashes of coruscant energy burst from long-slumbering pillars of machinery, driving the amphibian creatures back in fear. 
      An eerie light of green, veined with red, shot forth from the depths of the awakened crystal and bathed Kane in its fire.
    }
    
    \citeauthorbook[p.49]{LeeClarkZumpe:PassagetoOblivion}{%
      Lee Clark Zumpe%
    }{%
      Passage to Oblivion%
    }{
      Spread before them, luminous monoliths towered, their apexes scratching at the starless night skies.
      Gargantuan statues\dash too abstract to be identifiable in nineteenth century terms\dash invoked idols of undreamt religions, while altars festooned with incadescent crystals surged with curious power. 
    }
    
    The Rungeran soldiers in Eresh-Kal wield guns. 
    (Read about \hs{guns}.)
    Tantor has a gun which he uses once to kill a Meccaran.
    Tantor tells: 
    \begin{prose}
      Tantor: 
      \ta{%
        These mongrel savages jump at the sound of a musket. I imagine they might never have seen guns before and think them to be sorcery.}
    \end{prose}
    
    Tantor needs to be more racist. 
    Emphasize how \quo{filthy} and \quo{degenerate} the \EreshKali{} are, even by \meccaran{} standards\dash and that is saying a lot, because even a regular \meccaran{} is pretty dirty and stupid and ugly and inferior. 
    Tantor describes how his people win because they are of superior races. 
    (Some of his men are \scathae. 
     They are, of course, not as good as \humans, but still clearly better than \meccara.
     \Scathae{} are \quo{higher humanoids}, although not the highest there is.)
    
    Tantor believes that \meccara are small (smaller than \humans and \scathae) because they are a lower race and less well-developed. 
    It seems to him that these \EreshKali are even more stunted and backward than the average \meccarans.
    They are particularly small and degenerate and weak. 
    
    Also write in the Glossary about \quo{lower} and \quo{higher} humanoids. 
    
    When \Mycah{} dies, Tantor thinks it is the ancient curse, cast upon this loathsome place by wicked pre-\human{} gods, that has caused it. 
    The old gods are bitter to have been displaced and hate the \humans{} that have since swarmed all over the world that was once theirs. 
    So they want revenge. 
    Encroaching on their territory can only end badly, Tantor muses. 
    
    Tantor finds some writing in the temple that he cannot read. 
    The symbols certainly look nothing like Vaimon nor \Ortaican{} script. 
    They are reminiscent of Rissitic glyphs, but Tantor is pretty sure that is not it, either. 
    He has reproduced some of them in the diary. 
    Charcoal cannot read them, either. 
    
    Charcoal wonders. 
    Can it really be true that there lay such a treasure trove of magical wealth hidden in Runger, right under everyone's noses, where any moderately-armed warband could just waltz in and claim it? 
    If so, how did \Takestsha{} learn of it, and why did no one else learn of it before her\dash especially if it really is as old as Tantor guesses? 
    It seems incredible. 
    The Cabal and/or Sentinels ought to have sniffed out this treasure and looted it long ago. 
    Something does not add up here\ldots{} 

    On the last day, after the big battle, they hear slavering noises.
    They realize there are more \meccara, and they have brought monsters with them. 
    They fear it is their gruesome gods (whose images and statues they saw in the big chamber) which have now awakened and are hunting the interlopers. 
    They flee out quickly, pursued by monstrous horrors. 
    They do not see the things clearly, but they get an impression of gray, slimy tentacles. 
    And they can hear their repellent high-pitched piping, mewling, whistling noises.
    And they can smell them.
    
    This ultimate horror is never seen, perhaps never even glimpsed. 
    Only hinted at.
    Compare to that undescribed ultimate horror that lies beyond the highest peaks in \cite{HPLovecraft:AttheMountainsofMadness}. 
    
    Several stragglers are grabbed by the monsters.
    But anyone who stops or turns around to try and help them are also killed. 
    There is nothing to do but run. 
    
    In the end, many survive, and now they have the valuable plaques they came for. 
    
    At least one soldier turns around and sees the monsters, but is rescued and survives. 
    He is driven stark raving mad. 
    
    Get rid of the sex scene with \Takestsha and Tantor. 
    It serves no purpose. 
    And brutally cut down on the part after the escape from \EreshKal. 
    
    Tantor now warns \Onatol. 
    King Morgan has his sights on Pelidor, and Tantor believes \Takestsha is exerting some considerable influence on the king. 
    She is after something.
    Something of great mystical significance.
    Tantor believes this something is to be found in the city of \Forclin.
    Some powerful artifact of the elder ages.
  
  \begin{comment}
    \paragraph{\Forclin}
  \end{comment}
  \changesitem{\Forclin}
    Explain what a \bacconate is.
    
    When Carzain sees the gargoyles, he should speculate that they are probably magical.
    They are monstrous and scary-looking.
    
    When Carzain sees the Ghost Tower, he sees the suggestion of shapes around it.
    Winged fiends circling about it. 
    
    Curwen should not go to Carzain.
    Curwen should say: \quo{Send him in.}
\end{changes}









\subsection{Spectre of the Fray}
\begin{changes}
  \begin{comment}
    \paragraph{The Ghost Tower}
  \end{comment}
  \changesitem{The Ghost Tower}
    In the Bila scene, remember to mention that the couriers from Dendrum came by boat on the Ucarn river. 
  
    Curwen thinks back to Tantor's diary, where Tantor claimed \Takestsha was after some occult target in the city of \Forclin.
    What could he have referred to? 
    Is there some artifact buried beneath the city that Curwen does not know about?
    
    Carzain stands at the Tower.
    He looks up but cannot see the winged fiends that he saw in the chapter \quo{\Forclin}. 
    But he imagines he hears their shrill howls. 
    
  \begin{comment}
    \paragraph{A Machine of Flesh and Iron}
  \end{comment}
  \changesitem{A Machine of Flesh and Iron}
    The Rungerans start bringing in their cannons. 
    The cannons come in by railroad. 

    But the Pelidorians have disabled the railroads. 
    They have taken down the rails and dug up the ground so the rails cannot be easily repaired. 
    They have also taken down some \eidola along the Ucarn road. 
    As a result, the roads have become less safe.
    The \wylde is slowly creeping in. 

    Destroying the roads is controversial.
    The religious people are not happy about it. 
    The Iquinians see the maintenance of roads as a religious duty.
    By destroying roads, Sethgal is allowing the evil of the \wylde to seep into the world and disrupt civilization.
    It is bad etiquette.
    But Sethgal does it anyway.
    He is pragmatic. 
    This can slow the Rungerans down, and that is a good thing. 
    It gives his people more time to prepare. 

    The Rungerans are forced to haul in their cannons over the rough land. 
    They use \nephil ogres to do some of their heavy work.

    The Rungerans have to fight their way through land that is gradually becoming \wylde. 
    This is not very dangerous, for the Rungerans have a large army, but it does slow them down.
    Besides, they have to devote resources to actively maintaining the road behind them so they have supply lines. 
    This weakens their combat forces. 
    Sethgal is sneaky. 

    But the Rungerans will not be deterred. 
    They send up forces first to encircle and besiege the city.
    Then they stand there. 
    They are encamped outside cannon range from the walls. 
    Gradually they fill it up with more men. 
    At last they bring in their cannons. 
    They have ogres that serve as heavy muscle for the cannons, to haul them back into place after each shot. 
    
    Sethgal sees the ogres. 
    They are revolting things\dash half-\human monsters, monstrous but also pitiable. 
    
    Remember that the Rungerans also bring in supplies by ship, from the river. 
    Also mention that there is a naval battle going on on the river, which might be very important. 
    Just mention it a few times, do not actually show the battle.
    Both sides try to bring in supplies by sea, all the while trying to sink or capture the enemy's ships and rob them of their supplies. 
    The fortune had shifted back and forth several times, no side seeming to gain naval superiority for long.
    It looked as though this battle would be decided on land. 
\end{changes}









\subsection{Malcur Thread}
\begin{changes}
  \begin{comment}\paragraph{Beyond the Veil}\end{comment}
  \changesitem{Beyond the Veil}
    Revise the funeral rites. 
    Think up a better funeral tradition. 
    See \href{http://limyaael.livejournal.com/179073.html}{Limyaael's death rant} for ideas. 

  \begin{comment}\paragraph{Veils that Divide}\end{comment}
  \changesitem{Veils that Divide}
    Mention \Isphet and the \qliphoth in the scene with \Icor's funeral. 


  
  \begin{comment}\paragraph{A Dark Angel's Gift}\end{comment}
  \changesitem{A Dark Angel's Gift}
    Maybe Needle is not offered command over \banes. 
    Maybe that doesn't happen until much later. 
  
  \begin{comment}\paragraph{Captured}\end{comment}
  \changesitem{Captured} 
    
    In all the Rian chapters, whenever it is appropriate, have references to the \hr{Myths of vanquished monsters}{myths of Iquinian heroes vanquishing inhuman Elder Races and monsters}. 
    When he encounters something supernatural, he fears that the wicked Elder monsters will conquer the world. 
    
    \hr{Rian is religious}{Make Rian more religious}.
    Make sure he prays in every chapter and scene that he is in.
    Make clear how grateful to the Light he is for how he has been freed from his life of crime and allowed to make a new, honest life for himself.
    He has lingering existential/religious dread from the day when he saw the dark sorcerer slay the shining god (even though he was Shrouded and does not remember it all). 
  
  \begin{comment}\paragraph{Trinity of Plagues}\end{comment}
  \changesitem{Trinity of Plagues} 
    Change title to \quo{Trinity of Plagues}. 
    
    In all the Rian chapters, whenever it is appropriate, have references to the \hr{Myths of vanquished monsters}{myths of Iquinian heroes vanquishing inhuman Elder Races and monsters}. 
    When he encounters something supernatural, he fears that the wicked Elder monsters will conquer the world. 
    
    \hr{Rian is religious}{Make Rian more religious}.
    Make sure he prays in every chapter and scene that he is in.
    Make clear how grateful to the Light he is for how he has been freed from his life of crime and allowed to make a new, honest life for himself.
    He prays to be delivered from \Isphet's evil. 
    He has lingering existential/religious dread from the day when he saw the dark sorcerer slay the shining god (even though he was Shrouded and does not remember it all). 
    
    \Uswa is in bad shape due to malnourishment and hunger.
    When Rian comes seeking her advice, he brings food.
    
    Dennick has been like an older brother to Rian ever since Rian's parents died. 
    
    Dennick has a wife. 
    She is not well.
    She is badly sick.
    Dennick fears she has the \hs{Disease}. 
    She lies in the next room.
    They let her lie in peace.
    
    Dennick does not offer beer. 
    Rian has a decent-paying job.
    So he brings beer to Dennick as a gift. 
    
    Dennick is a thief, but an honest thief. 
    Rian is sad that Dennick is still criminal, \hr{Iquinian prostitution metaphor}{prostutiting his soul} for material gain.
    Rian prays for Dennick's good but sinful soul every day. 
  \begin{comment}
  \paragraph{The \Qliphoth Lie Ever in Wait}
  \end{comment}
  \changesitem{The \Qliphoth Lie Ever in Wait}
    
    In all the Rian chapters, whenever it is appropriate, have references to the \hr{Myths of vanquished monsters}{myths of Iquinian heroes vanquishing inhuman Elder Races and monsters}. 
    When he encounters something supernatural, he fears that the wicked Elder monsters will conquer the world. 
    He prays to be delivered from \Isphet's evil. 
    
    \hr{Rian is religious}{Make Rian more religious}.
    Make sure he prays in every chapter and scene that he is in.
    Make clear how grateful to the Light he is for how he has been freed from his life of crime and allowed to make a new, honest life for himself.
    He has lingering existential/religious dread from the day when he saw the dark sorcerer slay the shining god (even though he was Shrouded and does not remember it all). 
  
  \begin{comment}
  \paragraph{The Thirsty Nether}
  \end{comment}
  \changesitem{The Thirsty Nether}
    Remember to read about the Shroud. 
  
    The sorcerers whom Rian see talk about how \quo{\hr{The Change of Malcur}{the Change}} is coming up. 
    
    Rian sees one of the \hr{QJ in Malcur}{\quiljaaran in \Malcur}. 
    (Read about them.) 
    Moro also sees it. 
    
    Needle knows she is a poor reader. 
    But she doesn't understand that it's because she learned it as an adult. 
    She thinks it's because she is lowborn, and that nobles simply have talent for reading and other fine arts by virtue of being nobles. 
    (Maybe ask on a forum how to best represent this.)
    
    In the chapter where \Tiroco meets \Uswa, mention that \ps{\Tiroco} bodyguard carries a sword and a pistol. 
    
    In the first scene where Rian flees: 
    He has seen something of the Beyond. 
    He still suffers from aftershock. 
    His mortal mind (otherwise driven by denial) has not fully recovered. 
    He is able to see through the Shroud to some small extent. 
    The streets look different.
    Houses and landmarks which he remembers are suddenly gone, replaced by new, unfamiliar ones. 
    Nothing is as he remembered it.
    He panics. 
    This should not be happening.
    He knows these streets. 
    He has been here many times to spy. 
    He runs around frantically and becomes hopelessly lost. 
    
  
  
  \begin{comment}
  \paragraph{The Dark Crypts of the Mind}
  \end{comment}
  \changesitem{The Dark Crypts of the Mind}
    Read about \hr{Moro}{Moro \Cobrel} and \hr{Nasshikerr}{\Nasshikerr}. 
    
    Read about \hr{Ortaica}{\Ortaican mysticism} and \hr{Rethyax magic}{\rethyactic magic}. 
  
  
  \begin{comment}
    \paragraph{The Bleeding Wood}
  \end{comment}
  \changesitem[The Bleeding Wood]{\hr{Rian sees a raid}{The Bleeding Wood}}
    \begin{itemize}
      \item 
        Read about the Shroud. 
      \item 
        Have a scene before the main chapter where Rian goes to the hideout. 
        He sees some people in the shadows. 
        By a trick of the light, the people look like monsters to him for a second. 
        He is already somewhat paranoid, and he is beginning to see through the Shroud and see strange things that look monstrous and freakish to him.
        So his nerves are on edge. 
        
        \citeauthorbook[p.27]{StephenKing:CrouchEnd}{Stephen King}{Crouch End}{
          Standing on the corner beside their parked motorcycles were three boys in leathers.
          They looked up at the cab and for a moment\dash the setting sun was almost full in her face from this angle\dash it seemed that the bikers did not have \human heads at all.
          For that one moment she was nastily sure that the sleek, flat and sloping heads of rats sat atop those black leather jackets, rats with beady black eyes staring at the cab. 
          Then the light shifted just a tiny bit and she saw of course she had been mistaken; there were only three boys in their late teens there, smoking cigarettes\ldots
        }
      \item 
        Mention that \hs{Needle hates the Black Plague}. 
      \item 
        Rian should not see quite so much. 
        He knows that there is black magic and wickedness and conflict, but he doesn't know who is who. 
        He sees Needle, but doesn't know which side she is on, so he just categorizes her as \quo{evil}.
        
        Rian just sees Needle-tachi near the building, armed and menacing. 
        Then a \grimrat{} startles him and chases him. 
        He sees no more. 
        He does not see the raid at all, so he does not know there are two different factions fighting each other. 
        He just knows there are evil mages and monsters. 
        
        Maybe he sees one of the \hr{QJ in Malcur}{\quiljaaran in \Malcur}. 
        (Read about them.) 
      \item 
        In all the Rian chapters, whenever it is appropriate, have references to the \hr{Myths of vanquished monsters}{myths of Iquinian heroes vanquishing inhuman Elder Races and monsters}. 
        When he encounters something supernatural, he fears that the wicked Elder monsters will conquer the world. 
        He prays to be delivered from \Isphet's evil. 
      \item 
        \hr{Rian is religious}{Make Rian more religious}.
        Make sure he prays in every chapter and scene that he is in.
        He has lingering existential/religious dread from the day when he saw the dark sorcerer slay the shining god (even though he was Shrouded and does not remember it all). 
      \item 
        There is a \grimrat{} that crawls up a beam and runs across a crossbeam to leap down on its victim. 
      \item 
        Skip the entire battle and go straight to the aftermath with Needle contemplating. 
      \item 
        Needle is horrified after seeing the \grimrats{} tear people to shreds. 
        \ta{But it had to be done.}
        
        She is also squeamish about killing the plaguers. 
        But she must. 
        She has to remind herself that thugs such as these deserve no mercy. 
        She thinks back to the beginning of her family's tragedy, where her brother was killed by someone who might very well have been a plaguer. 
        A \scatha{} dressed in black, with a hood. 
        
        \tho{Yes. It was a plaguer. It must have been. I hate them.}
        She clings to this hate. 
        It makes her job easier. 
      \item 
        Afterwards, \Psyrex{} sits on his throne and muses. 
        \begin{prose}
          \ta{I have lost one of my mages. 
            I will probably have to import a new mage to \Malcur.
            
            Sad business. 
            Still, they did their job. 
            The plan is still on schedule.}
        \end{prose}
      \item 
        The scene where Rian where sees a \sphyle that is turning into wood should be expanded.
        I want more \hr{Humanoid horror}{humanoid-based horror}, remember. 
        The \sphyle comes alive and grabs Rian. 
        He screams.
        She talks to him, or tries to.
        It is hard for her to speak, but Rian understands more than he wants to. 
        
        \citeauthorbook[p.100-102]{RobertEHoward:TheScarletCitadel}{Robert E. Howard}{%
          The Scarlet Citadel%
        }{
          Within these bars lay a figure, which, as he approached, he saw was either a man, or the exact likeness of a man, twined and bound about with the tendrils of a thick vine which seemed to grow through the solid stone of the floor. It was covered with strangely pointed leaves and crimson blossoms\dash not the satiny red of natural petals, but a livid, unnatural crimson, like a perversity of flower-life. Its clinging, pliant branches wound about the man's naked body and limbs, seeming to caress his shrinking flesh with lustful avid kisses. One great blossom hovered exactly over his mouth. A low bestial moaning drooled from the loose lips; the head rolled as if in unbearable agony, and the eyes looked full at Conan. But there was no light of intelligence in them; they were blank, glassy, the eyes of an idiot.
          
          \ldots 
          
          \ta{%
            He pent me in here with this devil-flower whose seeds drifted down through the black cosmos from Yag the Accursed, and found fertile field only in the maggot-writhing corruption that seethes on the floors of hell.}
        }
        
        Maybe it is a \human, not a \scatha. 
        
        Afterwards Rian runs out.
        Then he sees \emph{everyone} as wood-people.
        Or he sees some of them slowly mutating into demonic monsters.
        
        Maybe he runs into the Beyond and sees everyone as as chained and faceless things.
        In fact, maybe take the Catrian chapter and move it here, with Rian instead of Catrian. 
        
        Rian sees loathsome, creeping, half-undead, chained people. 
        Compare to Eallal from \cite{RobertEHoward:TheShadowKingdom}. 
        
        \citeauthorbook[p.39]{RobertEHoward:TheShadowKingdom}{Robert E. Howard}{%
          The Shadow Kingdom%
        }{
          The glow merged into a shadowy form.
          A shape vaguely like a man it was, but misty and illusive, like a wisp of fog, that grew more tangible as it approached, but never fully material.
          A face looked at them, a pair of luminous great eyes, that seemed to hold all the tortures of a million centuries.
          There was no menace in that face, with its dim, worn features, but only a great pity\dash and that face\dash that face\dash
          
          \ldots 
          
          The phantom came straight on, giving them no heed; Kull shrank back as it passed them, feeling an icy breath like a breeze from the arctic snow.
          Straight on went the shape with slow, silent footsteps, as if the chains of all the ages were upon those vague feet; vanishing about a bend of the corridor.
        }
    \end{itemize}
\end{changes}










\subsection{General}
\begin{enumerate}
  \item 
    Update the pronunciation guide!
    Fix the pronunciation of Imetric and Resphan/Vaimon words. 
    Read the \hs{Languages} appendix first. 
    
  \item 
    Fix \hs{noble titles}. 
    Update the glossary with them. 

  \item 
    Have more descriptions. 
    Describe not only looks, but also sounds and smells. 
    Describe: 
    \begin{itemize}
      \item \ps{\Tiroco} room. 
      \item Needle's room. 
      \item The temple of \EreshKal. 
      \item Sethgal's command tent. 
      \item The Pelidorian army camp. 
      \item The Rungeran army camp. 
    \end{itemize}
  
  \item 
    Rewrite the \Archon-summoning scenes to show the \hr{Visualizing Archons}{personalized feel of each \Archon}. 
  
  \item 
    Have and mention examples of \hr{Rissitic economy}{Rissitic export goods}. 
  
  \item 
    Go over the Dramatis Personae and add more titles to names. 
    Such as \quo{\Symeon{} Clerk}. 
  
  \item 
    The translation of \quo{Iquin} $\to$ \quo{Light} is cheesy. 
    Change it to \quo{One Light}. 
    
    Similarly, change \itzach to \quo{Outer Darkness}.
  
  \item 
    Make a appendix about \matrices{} and \hs{astrology}, entitled \quo{Map of the Stars} or \quo{Pattern of the Cosmos} or \quo{Star-Maps of the Ancient Cosmographers} or somesuch. 
    Read some Bal-Sagoth for inspiration. 
    
    And draw an actual map of the stars and mystic constellations, \hr{Vertices in the sky}{with \vertices{} shown}. 
    
    Remember to read the sections about \hr{Matrix}{\matrices} and \hs{astrology} before doing this. 
    
    Remember that \nexi{} such as \Malcur and \Nithdornazsh{} should also be represented in the Star-Maps. 
    
    Compare to the Deck of Dragons in \cite{StevenEriksonIanCameronEsslemont:MalazanBookoftheFallen}. 
  
  \item 
    Read the \hs{Languages} appendix and fix all names and words so they don't contain illegal phonemes. 
  
  \item 
    There are way too many of those dreams/visions of doom in \Malcur.
    Get rid of some of them. 
  
  \item 
    Get rid of elephants, rhinoceroi, dogs, wolves and lions. 
    \Miith{} is \hr{Saurian-dominated}{\saurian-dominated}. 
  
    A \belwan{} is not a mammal but a small, hornless ceratopsian. 
    
  \item 
    Have references to the \quo{\hs{Ages of the World}} early on, and in the Glossary. 
  
  \item 
    Maybe the entire first \quo{Part} should be a Prologue \quo{Part}. 
  
  \item 
    Add a \hr{Vaimon Middle-East}{Middle-Eastern style to the Vaimons}. 
  
  \item 
    Make clear in the appendix that there is a translation convention in effect.
    For example, the planet is not really called \quo{\Miith} in all languages.
  
  \item 
    Remember that cities must be designed with domestic dinosaurs in mind.
    And big-ass dinosaur-drawn trains. 
    Factor this into the descriptions of Forclin and \Malcur.

  \item 
    Make the whole world more dystopian.

  \item 
    End chapters on a cliffhanger! 
    All over the place.

  \item 
    Get rid of the \quo{kraken} or merge them with the \noggyaleth and \xss.
  
  \item 
    Have plenty of \hr{Demihuman}{\demihuman} slaves in Pelidor and stuff. 
    Read about \hr{Demihuman}{\demihumans}.
  
  \item 
    Rethink the idea of the different \scatha ethnicities.
    Now we have the concept of \hr{Demiscatha}{\demiscathae}. 
    
  \item 
    Replace all occurrences of \quo{chaos sorcerer/sorcery/magic} with \quo{\rethyax}. 
  
  \item 
    Replace \quo{\bane} with \quo{\hr{Sitra Achra}{\SitraAchra}} in \resphan contexts. 
  
  \item 
    Remember to keep track of \hr{Languages in the Scatha Age}{the languages spoken in the \Scatha Age}. 
  
  \item 
    Remember that the chapter \quo{What Slithers Beneath} has to be far enough removed in time from the main story that \Teshrial has ample time to revive and heal.
    Look how long time it took \Urizeth to heal. 
    But then, \hr{Ketherain heal faster}{\ketherain heal faster than \thelyadeth}. 
  
  \item
    Replace all forests with \hr{Jungle}{jungles}. 
  
  \item 
    Maybe get rid of \GreatVelcad and merge it with \Tepharae. 
    We are in the \Scatha Age, remember.
    
  \item 
    Maybe get rid of \ClanTelcra and merge it with \ClanZether. 
    
  \item 
    Maybe rename \Miith to Shetiyah.
    Means "foundation stone" in Hebrew; "the rock over the abyss around which the whole world was built". 
    
  \item 
    Maybe \LocarPsyrex is a \quiljaar.

  \item 
    Get rid of \ClanTelcra. 
    Merge it with \ClanZether. 
  
  \item 
    Make sure the \scathae appear as powerful as \humans and not as inferior pussies. 
    
    Koit: \ta{the scathae seemed big but a bit dumb, however usually kind-hearted.}
    
    The chapter \quo{Screaming in the Dark} has a bunch of \scathae that come off as dumb and helpless. 
    I should have a chapter before this that makes them look cooler and more well-developed and intelligent.
    In fact, maybe I should turn the villages in \quo{Screaming in the Dark} into \humans.
    Seems they cause too much trouble in the current form. 
    That way I would lose the racist aspect, but I can live with that. 
    I can fit in plenty of racism later.
    
    The villagers should be hairy \demihumans.
    Some of them comment that Carzain is a purer \human than any of them. 
    But then, so were some of the bandits, and that did not make the bandits any less evil. 
    
    \ta{an even better racist touch actually. 'why are our fellow humans attacking us, if they could take their toll on the nearby scatha village'}
  
  \item 
    Drop hints all over the place about the back histories of \dragons, \resphain and \banes.
    Especially in the Curwen and \Cobrel chapters. 
    See the section about \hr{Hints}{hints}. 
  
  \item 
    I should flesh out the idea of \qliphah \hs{nebulae} and refer to it in the story where appropriate. 

\end{enumerate}





\subsubsection{Characters}
\begin{changes}
  \begin{comment}\paragraph{Archibald Curwen}\end{comment}
  \changesitem{Archibald Curwen} 
    Rewrite Archibald Curwen. 
    He is boring because he is not emotionally involved in anything that goes on. 
    He is unlikable because he is grumpy and brutish. 
    Make him more dastardly, more sardonic. 
    Make him bitter in a sarcastic way rather than a grumpy way. 
    He should smile more. 
    He should say \quo{\Mister \Shireyo}. 
    
    He shouldn't be so fat. 
    Change \quo{lose that belly of yours} to \quo{lose some weight}. 
  
  \begin{comment}
  \paragraph{Carzain \Shachar}
  \end{comment}
  \changesitem{Carzain \Shachar} 
    Give Carzain more humour.
  
  \begin{comment}
    \paragraph{\LocarPsyrex}
  \end{comment}
  \changesitem{\LocarPsyrex} 
    Make him darker.
    \hr{Psyrex darker}{Read about him}.
  
  \begin{comment}
  \paragraph{Moro \Cobrel}
  \end{comment}
  \changesitem{Moro \Cobrel} 
    Have more scenes with Moro \Cobrel.
    \begin{itemize}
      \item 
        Including \hr{Moro feels Ishnaruchaefir and Teshrial}{one in the dead garden in the beginning}. 
      \item 
        And the scene with \hr{Moro and Nasshikerr}{Moro and \Nasshikerr}. 
    \end{itemize}
  
    Drop hints that she sacrifices humanoids to \Nasshikerr.
    Usually she just sacrifices animals. 
    Only once in a while does he demand a \humanoid.
    
    Maybe scratch the above. 
    Make Moro nicer, more heroic.
    Remember, she is the biggest \scatha character in the book.
    I do not want to give the reader the impression that \scathae are evil. 

  \begin{comment}\paragraph{Rian}\end{comment}
  \changesitem{Rian} 
    Rewrite Rian.
    Make him more detective-like.
    Right now he depends too much on blind luck. 
    
    And compress the whole Rian story thread. 
    It is unrealistic for Neina to be held captive and alive for so long. 
    
    Remember to fix up the Rian chronology. 
    The \quo{What Slithers Beneath} chapter is now closer to the rest of the story. 
    Maybe there is not enough time to fit in Rian's rehabilitation between them. 
    
    The dark parts of \Malcur, where Rian dares to go, feel more like nature than city.
    They are decayed and corroded so much that they have almost become \wylde. 
    The \eidola in these slums and mafia quarters have been allowed to decay, partially because all the crime makes the Vaimons and monks afraid to come in there to maintain the \eidola.
    The Shroud-weaving alienist sorcery of the Sentinels and Cabalists also has to do with it. 
    The sorcery unravels the Shroud and tears holes into the Beyond from which occult power can seep. 
    This also causes the Mask of Civilization to unravel and makes the civilized settlement areas slowly decay into \wylde.
    
    These semi-\wylde places awakens Rian's \hr{Human racial memory}{racial memory} and makes him remember his people's dark and bloody past.
    
    \citeauthorbook[p.204]{RobertEHoward:TheLittlePeople}{Robert E. Howard}{%
      The Little People%
    }{
      The horrid things that pursued her were closing in upon her.
      They would reach her before I.
      God knows the thing was horrible enough but back in the recesses of my mind, grimmer horrors were whispering; dream memories wherein stunted creatures pursued white limbed women across such fens as these.
      Lurking memories of the ages when dawns were young and men struggled with forces which were not of men. 
    }

    Maybe turn Rian into a girl, named Ria.
  
  \begin{comment}\paragraph{Telcastora Ilcas}\end{comment}
  \changesitem{Telcastora Ilcas} 
    Have more focus on Ilcas Northstar and the Imetrians. 
    Show Ilcas and how he is sent to Pelidor because the Imetrians fear Rissitic involvement. 
    (And have a better explanation of why the Imetrians and Rissitics are such rivals.)
  
  \begin{comment}
    \paragraph{Sethgal}
  \end{comment}
  \changesitem{Sethgal} 
    Mention more than once that Sethgal really hopes some Imetric reinforcements get here.
    As soon as the war started, Sethgal wanted to begin negotiations with the Imetrians. 
    He is quite pragmatic when it comes to religion.
    But other (more religious) forces are working against him and the Imetric alliance.
    This makes Sethgal angry and sad.
    They need those Imetrians. 
  
  \begin{comment}
    \paragraph{\Teshrial}
  \end{comment}
  \changesitem{\Teshrial}
    \Teshrial needs to have more weaknesses.
    See the section about \hr{Teshrial's failure}{\Teshrial's failure}. 
\end{changes}





\subsubsection{Geography}
\begin{enumerate}
 
  \item  
    Draw a map of Pelidor/Runger/\Scyrum. 
    And one of \Malcur. 
    
  \item  
    Mark the regions of the map as \quo{\scatha-inhabited}, \quo{\human-inhabited}, \quo{\wylde} etc.
    
  \item  
    Maybe countries should not have explicit borders. 
    After all, borders are difficult to police. 
    And most of the country is \wylde anyway. 
    See also \href{http://limyaael.insanejournal.com/373476.html}{Limyaael's \quo{International relationships rant}}. 
    
  \item  
    Have plenty of \quo{wasteland} and dark, unknown regions. 
  
  \item  
    Re-design the geography of the northern lands. 
    And the south, for that matter.
    Make the north into more of a united land mass.
    And splinter the Imetrium into an archipelago.
  
  \item
    Change geography:
    \quo{\Velcad} refers only to Galessan.
    
    There were several \Velcadian languages, all of them descended from \Tepharin (which, in turn, borrowed heavily from both \Ortaican and Vaimon).
    Pelidorian and Rungeran were different but related languages.
  
  \item 
    Get rid of \GreatVelcad.
    Replace it entirely with \Tepharae.
    Get rid of the Tigers.
    Maybe replace them with a \Tepharin order named after a dinosaur.
    Maybe the \Tepharins came from the isle of \Velcad or something like that.
    But rename the isle.

  \item 
    \hr{Unexplored places}{Have many dark, unexplored \quo{here-there-be-\dragons} places in the \wylde}.

  \item 
    Remember that some places are \hr{Baccon}{\bacconates} ruled by mages. 
\end{enumerate}





\subsubsection{Glossary}
\begin{enumerate}
  \item 
    Add the names of all \xss{} used to the glossary. 
    \Ishnaruchaefir{} invokes some in \quo{What Slithers Beneath}. 
    
  \item 
    Fix the glossary entry for \quo{\hs{sorcery}}. 
  
  \item 
    Glossary: 
    \Sethicus was the mythical founder of the \draconian race (sometimes called the \quo{race of \Sethicus}).
    
    \Thanatzil was the mythical founder of the \resphan race (sometimes called the \quo{race of \Thanatzil}).
    
  \item 
    Rewrite the glossary and remove all references to the \quo{present time}. 
    Replace by explicit years:
    \quo{In year $n$ VC, \Icor was the \rayuth of Pelidor\ldots{}}
  
  \item 
    Write the glossary from a \resphan point of view. 
    For example, \quo{Incursion} is an item.
    \quo{\Secondbanewar} redirects to \quo{Incursion}.
    \Sethicus and the Durance are myths.
    \Thanatzil is a legend, but not a myth.
  
  \item 
    Make a remark about Carzain and Vizicar.
    Their internal dialogue is not literally what happens.
    It is a figurative representation of how the two minds access each other's thoughts and memories. 
  
  \item 
    Fix the name: 
    \Rystessakhin{} $\to$ \AeocrithRystessakhin. 
\end{enumerate}





\subsubsection{Quotes and mythology}
\begin{enumerate}
  \item 
    At the beginning of every \quo{Part} (and possibly in the middle of \quo{Parts} too), have passages from WID, \Iquinian myths, \Ortaican myth and perhaps other stuff.
    If I can make up enough of it, have some in every chapter.
    
    Make sure they wildly contradict each other.
    Even the ones belonging to the same religion.
    Have each side openly demonize the other. 
    
  \item 
    Clarify that myths are not necessarily true. 
    Perhaps make it clear that the text we see is just one of several possible translations, and not considered canon by all (perhaps even heretical by some).
    Hint to the viewer that myth and history should not be taken as canon. 

  \item 
    To make clear to the reader that Iquinian myths are not necessarily true: 
    Have ideas and imagery that contradicts Earth mythology. 
    For example, let snakes or horned/hoofed humanoids be good instead of evil. 
    Actually, maybe certain \resphain should have horns. 
    
  \item 
    At the beginning, have some quotes that tell about the \hr{Ortaican-Vaimon relationship}{theological conflict between \Ortaicans and Iquinians}. 
    (Read the section immediately!)
    
  \item 
    Have \Ortaican \hs{pseudepigraphic Sseju quotes} that condemn Iquinianism. 
  
  \item 
    Quote the poem \quo{Gods in Darkness} from \cite{KarlEdwardWagner:DarknessWeaves}:
    
    \citeauthorbook[p.2 of 138]{KarlEdwardWagner:DarknessWeaves}{%
      Karl Edward Wagner%
    }{%
      Darkness Weaves%
    }{
      In their castle beyond night\\
      gather the Gods in Darkness,\\
      with darkness to pattern man's fate.
    }
  
  \item 
    Just before or after the chapter \quo{What Slithers Beneath}, I should have some quotes from Iquinian scripture where \Isphet is reviled for his great evil.
    But present him in a way as to make him look like a total stud; an awesome dark lord. 

\end{enumerate}






\subsubsection{\TeX nicalities}
\begin{enumerate}
  \item
    Sort the index properly! 
    
    Find out how to do \quo{see also} in the index. 
  \item   
    Fix the counter-related problem with years! 
    And make a command that checks whether a number is defined.
  \item 
    Make a macro that capitalizes a word. 
    See \authorbook{Donald Knuth}{The \TeX book} example 20.19 p.217. 
  \item 
    Reduce line spacing in Dramatis Personae and Pronunciation. 
  \item 
    Typeset the index in a smaller font.
  \item 
    Typeset the table of contents in a smaller font and in two columns. 
  \item 
    Add \texttt{$\backslash$gref} commands for \emph{all} Glossary items (incl. people), at relevant places in the text. 
    Replace all \texttt{$\backslash$hr} and \texttt{$\backslash$hs} with these. 
  \item 
    Make a global switch that determines whether a \texttt{$\backslash$gref} should act as a \texttt{$\backslash$hr} or a \texttt{$\backslash$maybehr}. 
\end{enumerate}





\subsubsection{Limyaael-inspired ideas}
\begin{enumerate}
  \item
    Maybe \Esmerel{} doesn't recognize the signs of a Scion immediately. 
    Maybe she has to go research a lot of writings and signs first (in the big-ass library at the \TopazChateau). 
    Then, later, when she is sure that it was a Scion she saw, she tells the Conclave and receives their blessing to go off and search for it. 
  
  \item 
    Have references to poems, religious dogma, history records and whatnot. 
    See \emph{Limyaael}'s \href{http://limyaael.livejournal.com/357919.html}{Non-linear narrative rant}.
\end{enumerate}





\subsubsection{Koit's ideas}
\begin{enumerate}
  \item 
    Make Carzain stronger\dash but keep up a threat that he might crumble under the power. 
    \begin{itemize}
      \item 
        \quo{%
          Why I didn't like Carzain? No one is that overconfident in his nonexistant abilities\ldots{} At least, that's how I saw him.}
    \end{itemize}
  \item Trim the scene with Tiroco and her egg. 
  \item Give \hs{Sethgal} more of a personality. 
  \item Write foreword and reading guide
  \item Fix \quo{Involate} $\to$ \quo{Inviolate}. 
  \item 
    Tiroco: 
    \quo{%
      She seems to be emotionally disabled from her current paragraphs. 
      The behaviour seems inhuman (and she's not human). 
      A typical human would likely do everything he or she could to spend more time with someone who has left (Icor), but she tries to flee from him when she sees him. 
      Perhaps it's right, but it would deserve a bit of mention why it's soo. 
      religious taboo set by Light? 
      Ghosts are perhaps of dark? 
      then I could understand her}
  \item 
    Make sure it is \quo{Charcoal} and not \quo{Curwen} who reads the diaries. 
  \item Give some indication that Curwen recognizes Takestsha. 
  \item Make Curwen's secret identity more secret. 
  \item Add more references to history and historic figures. 
  \item Have some Pelidor-Runger skirmishes before the main battle. 
  \item 
    \quo{have you devised any specific military tactics either side is well known of or feared for?}
  \item 
    It looks like \Ishnaruchaefir{} finds \Triestessakhin{} in the garden. It shouldn't. Fix it. 
  \item Maybe Iasper Bartholin's title shouldn't be \quo{Mister}. 
  \item Draw a map of \Malcur. 
  \item 
    page 11, paragraph 5: 
    \ta{The captain may be,} said Carzain, 
    \ta{but has less savoury characters among his men.}
  \item 
    page 11, last paragraph: \quo{even alien and uneartly minerals} I would think that 'unearthly' and 'alien' have quite the same connotation, especially when there's no spacetravel (or if you don't mean that the minerals are from some other dimension)
  \item 
    page 11, last: 
    \quo{It was a journey of many days from \Redglen{} to Martinum}.  Shouldn't it be the opposite\dash they are coming from Martinum
  \item 
    page 12, third: 
    \quo{They were five days' travel out of Martinum and still ten days from Redglen,} 
    
    really, a travel of 15 days on foot and with a cart is rather quick and means a shor distance.
  \item 
    p. 13, fourth: \quo{warring kingdoms and baronies}, You mention a 'baronry' while you'll be dealing with a 'duke' for the most of the come time forward\ldots{} 
  \item 
    p. 16, third: \quo{At the last minute} seriously, the guy's been running at him for *ages*
  \item 
    p. 16: the mace is above Carzain's head, if the mace falls, the head is below it.
  \item 
    p. 185 - how many moons? two? then it's 'larger' instead of 'largest' at paragraph 2 ending
  \item 
    Carzain: If you mention that he just came back from a trip to a faraway land, it's illogical that he *never* again brigns it up. 
  \item 
    The beginning of Northstar was really weird, I don't know anyone *That* arrogant (Show those Qliphoth who's boss). 
  \item 
    Add a disclaimer to the front of all the note documents:
    \begin{quotation}
      These notes are rough drafts and not fully updated. 
      They are full of anachronisms, contradictions and residues of discarded ideas. 
      They should most definitely not be considered canon. 
    \end{quotation}
  \item 
    Regarding the portrayal of \resphain in \quo{Wanderer in Darkness} and \quo{What Slithers Beneath}: 
    \ta{I like the joke you've got in it (respectable occultist. :D). well, they seem honour and respect based in many ways. that being said, we don't really learn much of them this quickly. the resurrection part is a bit puzzling... the part about allying with other Resphan would take ages in interesting, and I suppose it might carry some weight, but without going into the problem sooner or later, you've just brought up an empty detail.
    we don't really get to see the Resphan outside their conference or whatever. I already said that knowing more of their battle would be nice. }
    
    So, explain the \quo{ally with other \resphain} part better.
    There are all sorts of political reasons why it will not work.
  \item 
    Regarding the portrayal of travel through the Beyond int \quo{What Slithers Beneath}:
    It seems like the \resphain come from the sky. 
    It is very unclear what the Beyond is.
    I should explain this better: 
    They come out of nowhere, out of a parallel dimension. 
  \item 
    Regarding Communion: 
    \ta{%
      I suppose you'll be describing some other communion later on, where you describe exactly what the human is hoping for? to add the info here would stall it too much, but it would be good to know, I believe}
    
    It is unclear what Evith expects will happen.
    The short answer is, she wants to go to heaven.
    And she does\ldots{} sort of.
    I need to clarify that.
  \item 
    \ta{Well, as I said, the duel sequence [\quo{What Slithers Beneath}] could use a bit of touch... I like the Prologue of Dragons. A bit more destruction might be useful though... Then I liked how Carzain slew the bandits [\quo{Screaming in the Dark}]... What was perhaps inexplicable/bad was how Carzain actually travelled, or how long he did travel to the burned village, that he had the luxury to say that he had lots of time to spend there}
\end{enumerate}










\subsection{Split into two books}
If the book ends up being too long, I could split it in two halves.
Not by cutting it in the middle, but by splitting it geographically.

The first half takes place chiefly in \Forclin.
Main characters are Carzain, Curwen, \Achsah and \Takestsha.
It ends immediately after the great boss battle with \Achsah versus \Nzessuacrith.
\Forclin is in Cabal hands, but something ominous looms in \Malcur. 

Maybe have the opening of the conflict between \Teshrial and \Ishnaruchaefir already in the beginning of the first book.
Have \Achsah refer to it as a mystical, faraway conflict, but still very important. 
And then end the first book on a cliffhanger: 
\Achsah-tachi have secured \Forclin, but then they get a distress call from \Malcur:
\ta{\Malcur is under attack! By both \Ishnaruchaefir and the Sentinels!}

The other half takes place in \Malcur. 
Main characters are \Teshrial, Moro \Cobrel, Rian and maybe \Tiroco. 










\subsection{Removing story threads}





\subsubsection{\Tiroco story thread}
\target{Remove Tiroco story thread}
I may decide I have too many story threads and need to cut some of them.
If so, the \Tiroco/\Icor thread is a prime candidate for cutting. 

I will let the story thread live for now. 
I can always cut it later.

\begin{enumerate}
  \item 
    \Tiroco is supposed to die at the end of her story thread. 
    If \Tiroco is removed, then I should consider killing someone else instead. 
    I could \hr{Maybe Rian dies}{kill Rian and Neina}. 
    
  \item 
    If the \Tiroco thread dies, then the \quo{Daggers and \Daemons} chapter should be reduced to a footnote.
    All that the reader needs to know is this:
    \begin{itemize}
      \item The \rayuth has been assassinated.
      \item \Ishrah mage \Ambrose \Onatol was implicated in the assassination.
      \item \Onatol resisted arrest and was killed by Archibald Curwen.
      \item Charcoal has obtained some papers from \Onatol. 
    \end{itemize}
    
    Ostensibly, the murder of \Icor was to weaken Pelidor and make it easier to invade and conquer. 
    In reality, the assassination was a ruse.
    It was just there in order to make the Rungeran invasion look more secular in motivation, to hide the Sentinel involvement. 
\end{enumerate}
























