\section{The Siege of Fendor}
%\Narkiza{} and his army lay siege to the isle of Fendor, controlled by the Imetrians. They conquer it. 

\stamp
  {\dateSiegeofFendor}
  {\Risvalsea, west of Fendor}
%\dateplace{Carz}{something'th}{some month}{FendorSmall{}  harbour, Fendor}
%\dateplace{Siege of Fendor}{18th}{\Hapheron}{\Risvalsea, west of Fendor}
%
%
\tho{Perfect. They have swallowed the bait.} 

From his vantage point, soaring high above the sea,  \Dasvedshiracht{} could clearly see the Imetric ships, dozens of them, sailing west from Fendor toward Tugan, leaving the island's defenses decimated. 

The ruse had been elaborate: A party of moles had been snuck into Pandex, the fortified port-town on Tugan, with instructions to attack the town from within using their magic, supposedly to coincide with a fleet attack intended to invade the island. But unbeknownst to the infiltrators, a tip had been leaked to the Imetrians through a double agent concerning the upcoming Rissitic attack. When the sorcerers in Pandex began casting their spells, the Imetrian mages, notified about the threat, would easily detect and dispatch them. 

But that was a detail. More importantly, the Imetrians expected a major fleet attack on Pandex, and as a consequence\dash{}as he and \Narkiza{} had anticipated\dash{}they had called for reinforcement, and dozens of ships were now being pulled from Fendor to rush to the succour of Tugan's defenders. 

Awaiting an attack that would not come. 

For the prize that the Rissitics sought was not Tugan, but Fendor. True, there was a token force of Rissitic ships moving towards Tugan, deployed to make the feint more convincing in case the Imetrians were to scout ahead. But the bulk of the fleet was bound Fendor, where the gullible Imetrians would soon feel the overwhelming fury of \Nechsainz{} warriors. 

Yet before the true battle for Fendor would begin, there was another part of brave Rissitic agents whose mission would be crucial to the whole endeavour. \Dasvedshirachtz{} parched throat produced a rattling hiss, which those who knew him would recognize as a laugh. Once again he felt delight at the audacity of this idea of his: To use the exact same ploy twice as part of the same strategem; one as a feint intended to be discovered and fail, the other a matter of utmost secrecy, its success essential to the entire plan. It was madness, he knew, but also genius. 

\tho{Naturally, the moles in Pandex must be sacrificed. \Narkiza{} objected against the necessity of this, as I knew he would. He is a splendid commander, having deserved all his honours twice over, but he is soft of heart. I made him see reason, of course. It is a small sacrifice, all for \Nechsainz{} glory. Besides, the ones sent on that expedition were all expendable; I saw to that. Unlike the other expedition. Ah, \Filshed{}. I have high expectations of you. Do not let me down.}

With a powerful blow of his tremendous wings, the great undead \dragon{} turned and made his way back to the fleet. \tho{There is no point in risking discovery. After all, I take up quite a lot of space, in this world as well as the worlds beyond...}



\placestamp{Rissitic flagship \shipname{Mother Hydra}\\
Near Fendor's southern shore}
\ta{Grrrrrrr,} Belgrim growled in impatience and irritation, and again all seamen near the Cortio's mouth shied away and cowered. 

\ta{Easy, Belgrim,} \Narkiza{} soothed, scratching the huge beast on the head with \Femtu, his morning star. \ta{Easy. We are almost there now. See, there is land ahead.} Belgrim did not understand Rissitic, of course, but \Narkiza{} had worked with Cortios for over two hundred years now, and his experience told him that the beasts understood much more than one might think. Besides, he had developed a habit of talking to them. 

He stood up in Belgrim's saddle to gaze out over the waters. \tho{No Imetric ships in sight. It seems the raiders have done their job well.} 

After having lured much of the Imetric fleet away to Tugan, they had sent raiders in fast ships to perform hit-and-run attacks along the Fendoran shore. The raiders fled back to their ships and slipped away as soon as the Imetric forces came to engage them, but did enough damage that the Imetrians would feel compelled to pursue them. This was nothing out of the ordinary, of course, for raiding and harassing were parts of every war. But today, with much of the fleet gone to fight ghosts at Pandex, the harbours of \Cicora{} and \Fendacor{}, the twin fortress-towns of Fendor, would have few ships to spare, and with several of them now chasing the fleeing raiders, the remaining defenders would be spread thin indeed. 

And in this opening, \Narkiza{} and his army would strike. For now a fleet of forty-two Rissitic warships advanced toward the Fendoran shore, led by the mighty \MotherTiamat{}. Named after the terrible \Dragon{} Mother of myth, nearly forty strides in length and carved and adorned in the likeness of a many-headed \dragon, the great flagship was fearsome and awe-inspiring to behold. She bore three great masts and was armed with several heavy ballistae, four on each side and two at the bow, each of the ten pointing out of a carved \dragon's mouth. 

\ta{\Ondmyst{}!} he called. \ta{Get the men ready.}

\ta{Yes, \Neftzaid{},} said \Kufur{}, his assistant. He turned to bark orders to the soldiers. \ta{You heard the \Neftzaid, get up! You'd better have slept well today, because we will be marching and fighting through the night. Now get off your soft pillows and into your armour!} The last part was an exaggeration; obviously they did not have soft pillows on the boat. 

The shore was close, so close. This was too almost easy. Surely their landing would not go uncontested? 

As if on cue...

\ta{Ships ahoy!} the watchman shouted from the top of the mast. \ta{Ships to port! Imetric flag!}

\Narkiza{} turned to the port side and, shadowing his eyes with his hand, gazed over the waves. In the mist he could barely make out the shapes of ships. The fog had been a blessing until now, hiding their advance from the Imetrian scouts. A blessing, as if sent by the \Annunaki, and while many of the men might think so, \Narkiza{} knew, of course, that it was not so. The gods could easily have conjured such a mist using their magic power, but they could scarcely use it to cover their movements, for the Imetrian mages would surely detect their work and suspect. No, the fog had been a whim of nature, a blessing at first, but with a hidden thorn. For cover worked both ways, and now the Imetric fleet had snuck up on them. 

\tho{No matter. We would have to fight their navy sooner or later no matter what.} \ta{All ships turn port! Prepare to engage!} %A rash deicision, perhaps, since he did not yet know the size of the Imetric fleet. 

...

The Imetric fleet is significantly smaller. There are mages. The Imetrian mages cut holes in the hulls of a couple of ships. \Narkiza{} reaches out and uses a Body magic spell to kill one of the mages. (This is difficult at such a long range, but \Narkiza{} is very strong.) Then the \Shesshefkesad{} unite in a circle and slay the rest of the mages using nasty Shadow spells, perhaps even invoking the \Maskim. The Imetric mages try to fight back, but the glyphs warding the \MotherTiamat{} protect them and deflect the hostile spells. 

...

\ta{Pull up alongside that ship!} \Narkiza{} called. \ta{And prepare to board!} 

\ta{Aye, \Neftsaid{},} said \Kufur, then turned to the crew. \ta{Prepare to pull in the starboard oars! Ready those grappling hooks! Archers into formation!}

The Imetric ship was too close to escape, and her officers evidently realized it, for at their shouted orders many Imetrian sailors threw down their tools and readied their weapons. Crossbowmen were already arrayed and ready. 

An Imetrian commander gave the order, and the sound of a hundred crossbows loosing at once was like an explosion, even above the roar of the waves and the din of battle surrounding them. 

\ta{Incoming volley!} \Kufur{} called. \ta{Shields up and take cover!} 

\Narkiza{} himself voiced a spell that, with luck, would deflect quarrels that would otherwise strike him or Belgrim. \ta{\foreign{Keshur da hi ved yshnaed.}} As he spoke he drew glyphs in the air with his fingers. \ta{\foreign{Hekh� ned is sof�za.}} With his words and signs he called out to the \BodyCreatures{} of the Body world, and he could dimly perceive them flocking about him, like a swarm of gulls or crows. 

The killing wave descended, each bolt whistling a piercing shriek as it plunged toward its target. Some Imetric crossbow bolts were specially crafted to produce this noise as they flew, swooping down like a horde of \daemons{} screaming for blood. A devious contrivance to dishearten the enemy, and it did: \Narkiza{} saw many warriors cower, and he saw fear in their eyes; some even turned and ran. 

Several bolts tore through the sails, and time seemed to slow as the wave closed in. The \Ashenoch{} suppressed the instinct to avert his eyes and instead repeated his spell. \ta{\foreign{Keshur da hi ved yshnaed. Hekh� ned is sof�za.}} The \BodyCreatures{} drew closer around him, so that he could literally and physically feel them. 

He held his breath. 

Even Belgrim held his breath. 

The bolts impacted. \quo{Thunks} sounded where they struck wood, and screams and grunts sounded where they struck flesh. \Narkizaz{} \BodyCreatures{} did their work well, and Belgrim was struck only a glancing blow that deflected harmlessly off his chain mail barding. 

The commander saw that many were not so lucky. A score soldiers and more were hit, and of them several were down and not moving. Near his right a young man sprawled, a quarrel lodged in the side of his head, blood and brains seeping out. \tho{My spell might have caused that... sent a quarrel flying off to the side...} 

He was given no time to grieve. \ta{Another volley!} \Kufur bellowed. \ta{Shields up!}

\tho{\Maskim{} take the Imetrians and their double crossbows.} The Imetric crossbowmen were feared for a good reason. Not only was the range of their weapons tremendous, but each also held not only one but two quarrels, and thus could fire twice before having to reload. 

The Imetrians fired blindly, of course, for the \MotherTiamat's sides were higher, so that they could not see the deck, but even so, the ship was tightly packed enough for them to need no aim. 

Again the killing rain fell like wailing \daemons, and again the Rissitics hid beneath their shields and shuddered. \Narkiza{} repeated his incantations to keep the \BodyCreatures{} in their places. It was possible that his spell would send bolts whirling to strike his own people, but, reluctant as he was to accept it, he acknowledged that the exchange was worth it. His life was worth more than theirs. 

The volley struck home. Again the cries, the blood... and the guilt. Which he forced down with long-practiced efficiency. No time for that. Not on the battlefield. This much he knew. 

\ta{Take heart, warriors!} he yelled at the top of his voice. \ta{They cannot fire again now, and we are almost within range!}

\ta{Archers into formation!} \Kufur{} ordered. \ta{Prepare to loose!} The bowmen drew up in lines, notching arrows. 

The \MotherTiamat{} closed on her prey. 

\ta{Loose!} yelled \Kufur{}. 

Rissitic arrows hissed through the air, and now it was the Imetrians who cowered and ran for cover. 

\Kufur{} wasted no time. \ta{Oarsmen, get those oars in! Archers, another volley! Notch! Draw! Fire!}

...

They were now almost within boarding range. \ta{Throw hooks at will!} \Kufur{} ordered. \ta{All hands, prepare for battle!}

Grappling hooks were thrown, and the two ships were pulled close. 

\ta{This is it!} shouted the \Ashenoch. \ta{This is where we show the infidels what we are made of! Fight, \Keffoydh! For \HriistN!} 

\ta{\Nechsain! For \HriistN!} came the shouts in answer from his troops. 

The ships were now close enough for the first boarders from the \MotherTiamat{} to begin swinging over. A few Imetrians attempted to swing over as well, but with greater difficulty, the Rissitic flagship being taller. 

\Narkiza{} took \Forshval{}, his great morning star, in his left hand. \ta{\Kufur! I am going to board. You are in charge of the \MotherTiamat{}.}

\ta{Aye, \Neftsaid!}

\ta{All men, BRACE YOURSELVES! Go, Belgrim. Jump!}

Belgrim backed up a few steps, then ran toward the rail and jumped off. The great flagship rocked dangerously as the giant Cortio pushed off, and many warriors either grabbed the rigging or tumbled and fell. 

Clutching his saddle tight, \Narkiza{} hung in midair above the Imetric ship for a moment. 

Then the wooden deck flew up toward him. He braced himself for the impact. 

A loud crack as wooden planks splintered. A frustrated roar from Belgrim as he tumbled, his foot punching through the deck. Crouching down and hugging the reins close, the \Ashenoch{} barely kept his saddle. 

Belgrim roared again as he squirmed and struggled to rise. \Narkiza{} looked up and saw that some Imetrian spearmen, at first startled to flight at the appearance of the monster, had regained their courage and moved in to attack. Luckily it was Belgrim's right foot that had gone through so that they were lying on their right side, leaving \Narkizaz{} left hand free\dash his left had always been stronger than his right. Swinging \Forshval{} above his head, he spoke spellwords \ta{\foreign{Ro n� jegg�sh!}}, and the morning star's spiked head sprang off on a long chain, forming a terrible flail. 

\ta{\foreign{He id giss'ezba!}}

...
\new
They meet a few defenders on the shore, including a wooden sentry tower. Archers fire from the tower, but the Rissitics bombard the tower from the ships and soon the few defenders are forced to flee. They go ashore. The ships move south. 

...

\new
\ta{Into formation! Get in line, you creeps!} \Kufur{} and the other officers shouted orders, and \Narkiza{} waited while they got the troops arranged into formation. Finally they got the men organized, and \Narkiza{} called his officiers together and addressed them.

\ta{Remember the plan, everyone. We move east to \Fendacor, but slowly. And \emph{no} havoc against civilians save where I specifically order it!} He noticed Geldashad disdainfully looking away and moved closer to loom over him. \ta{That means you, Geldashad. Am I clear?}

\ta{Yes, \Neftzaid,} said Geldashad, yet his tone and stance screamed contempt. 

\Narkiza{} scowled. \ta{You will stay close to me at all times, \Fedza.} 

\ta{Pining for my charming company?} Geldashad asked mockingly. 

\ta{And you will be silent!} The \Ashenoch-\Fedza{} displayed no reaction to this, and \Narkiza{} repressed an angry shiver. \tho{\Maskim, I hate him.}

\ta{Any questions? Good. Move out.}





\placestamp{\FendorSmall{} harbour, Fendor}
%
%\Filgzed{} is a Rissitic sorcerer (female \scatha{}). She, with a few assistants and a squad of soldiers for protection, are to infiltrate \FendorSmall{}, the smaller of the two fortress-towns on Fendor (the other is \FendorLarge{}). She is equipped with a number of shade pearls and Glyph heads. Her mission is to conjure a number of \daemons{} to attack the Imetric city from within. Then they are to signal the fleet, which will attack simultaneously from without. Thus they aim to conquer \FendorSmall{}. 
%
%Shade pearls, gray and two centimetres in diameter, are taken from the rare Okokwang oysters that are found in the seas of the Far Orient. They are not inherently magical but useful components in a spell that lets you `shroud' your use of magic, so that it is difficult for other mages to detect that you are casting spells. 
%
%Glyph heads are the dismembered heads of humanoids, embalmed with herbs and spices and enchanted with arcane Glyphs. Each head contains a prepared spell (possibly more than one) and is charged with magical energy to release the spell. The stored spell can then be released by mage using only a shorter, simpler, easier spell. 
%
%\Filgzed{} and her retainers enter \FendorSmall{} in the guise of merchants selling oriental spices. (These spices also serve the secondary purpose of masking the stench of the Glyph heads.) They are to meet with a mole, a Rissitic agent masquerading as a regular Imetric citizen of \FendorSmall{}, and use his house as their base to perform the summoning ritual. They are to time the ritual so that the \daemons{} are unleashed exactly at noon. 
%
%About ten minutes after the \daemons{} have begun their attack, the fleet moves in to attack from the west. The plan is that the Imetrians will first mobilize their forces to fend off the \daemons, then send what they have left to counter the fleet attack. After this, the Rissitics will send yet another fleet to attack from the South. 
%
%The ships use catapults.
%
\index{shade pearl}
\index{okokwang}
\Filgzed{} produced the shade pearls to study them one more time. Procured at great cost and with great difficulty from the Far Orient, the pearls were supposedly created by the very rare okokwang oyster that lived in the deep, dark waters at the bottom of some far eastern sea, a deep allegedly aswarm with vicious, poisonous beasts and sinister, diabolical creatures. And looking into the pearls, \Filgzed{} was willing to believe many of the fantastic tales about their supernatural origin. 

They were perfectly round and smooth, a half to a full thumb in diameter, dark gray in colour, and they seemed to be somehow alive, smoke and shadows swirling below the surface, as if the pearls held imprisoned a horde of ghosts or devils. She could not determine whether these colours were real or an optical illusion. Once more she lamented the fact that she would not get the opportunity to study the pearls in detail, since they would have to be expended. 

They were beautiful. Beautiful and mysterious. In a sense, they captured the essence of magic itself. \Filgzed{} could almost weep to let them go. 

\ta{\Ginfik, we are nearing the harbour. Might I suggest you hide those... things?} Bantoyn was \Rekkan{}-\Ondmyst{} and the military leader of the expedition, and as such she was under his command, but she was still of the Tsalt caste, and the man was\dash{}understandably\dash{}hesitant to give her orders. 

%The speaker was Bantoyn, the military leader of the expedition. 

\ta{Aye, you are right.} She gave the shade pearls a last, affectionate look, then wrapped them up and hid the pouch away under her shirt. 

Not long after, they reached the merchants' harbour, and Dorm and Gelgein went to tie up the boat among the other merchant boats. The two, along with Br�m, were the sailors of the team; they were here to sail the boat and to help make their guise of as seafaring merchants more convincing. 
As soon as \Filgzed{} was convinced that the boat was stably secured, she eagerly climbed onto the wharf. Unlike the three men, she was by no means a sailor, and more than glad to leave the boat behind. Yet even on the wharf, for what seemed a long while it seemed to her that she could still feel the dreadful rocking, that the world was still swimming around her. 

When, after a minute, she finally got her stomach to settle down, she found that the crew had finished disembarking, and that a trio of harbour guards were moving up to accost them.

The lead guard was a aging \sphyle, Tassian with azure scales. \ta{State your names and business,} she demanded. 

\ta{My name is Barrud,} said Bantoyn in halting Imetric, with an affected Hazidi accent to mask his true, Rissitic accent. He indicated \Filgzed{}. \ta{This is my wife, Filiza. We are spice merchants from Hazid.} In order to pass as Hazidi they had, of course, had to assume Hazidi names. 

Remembering her role, \Filgzed{}\dash{}Filiza\dash{}folded her hands in her lap and assumed what she hoped was a demure pose. Having been born into the Tsalt caste, \Filgzed{} was not much accustomed to humility, but in the Hazidi culture a wife was supposed to be subservient to her husband, so she would have to try to fake it. %Her role as wife also meant that she would not be expected to speak in public, which was fortunate, for \Filgzed{} spoke no Hazidi, and while she spoke passable Imetric and Belkadian, she would not be able to hide her Rissitic accent. 
\Filgzed{} spoke no Hazidi, and while she spoke passable Imetric and Belkadian, she would not be able to hide her Rissitic accent, so to hide this, she had agreed to this pose as \quo{Barrud's} wife, since as such \quo{Filiza} would not be expected to speak in public. 

The guard eyed them all, then approached the boat. She studied the sacks. \ta{And this is your wares?} 

\index{telf}
\index{peggra}
\index{pamuro}
\ta{Yes, yes,} said Bantoyn. \ta{Fine spices from the east.} He indicated various sacks and bags. \ta{This is telf and peggra, ofst, niekki and bigolgol, pamuro and kibbishin...} \Filgzed{} was fairly convinced that some of the names\dash{}especially \quo{bigolgol}\dash{}were made up on the spot. 

The guard cut him off. \ta{Open those sacks.} 

\ta{At once.} He gestured at Gelgein and Dorm and spoke something in Hazidi. The two, while faithful Rissitics, were native Hazidi and spoke only Rissitic and their native tongue. At his command they began opening the sacks. 

The guard sniffed around and stuck her hand into several sacks to feel the contents. \Filgzed{} suppressed a wince as the guard sniffed the sack of pamuro, but luck stayed her hand and she did not investigate further. \Filgzed{} stifled a sigh of relief, for had the Imetrian stuck her hand into the bag she might well have discovered the glyph heads hidden at the bottom. 

\ta{Alright,} said the guard at last. \ta{You will have to visit the customs office before you will be allowed to set up shop, and it will cost you half a ducata for this inspection.} The stare she gave them was imperious and impatient. 

\ta{But of course} Bantoyn produced some coins from a pouch. \ta{Will Hazidi crescents do?} 

The lead guard gave a questioning glance to one of her companions, a younger man. The man nodded, then moved forward to take the coins. \ta{Seven crescents,} he said, then proceeded to bite each of the coins. 

Satisfied, the younger guard nodded at his officer. \ta{Good,} she said. \ta{Have a good day and stay out of trouble!} With that she spun around and strode away. 

\Filgzed{} sneered silently at her back. \tho{\Maskim{} take those greedy Imetrians. Accosting honest traders and having the nerve to charge a fee for it!} She smiled at herself. \tho{Well, strictly speaking not so honest. But they have no reason to assume that!} 

\ta{Alright,} said Bantoyn in Hazidi. \ta{you know what to do. Mamrim, you take care of the shop here. Boruman, Dorm, Gelgein, you assist him.} Boruman was Br�m; Gelgein and Dorm, being Hazidi, had no need of false names. \ta{Aisha, Yudai, you come with us and carry the \quo{goods}.} He smirked at \Filgzed{} and gestured for her to follow. \ta{Come, wife.} 

\ta{Do not get carried away,} she hissed, but she smiled while sayin it, and she obeyed, gesturing in turn to her assistant, Dzavish, to follow suit. \tho{What a barbaric custom, that one sex should be subservient to the other. A \human{} idea, no doubt.} Although having learned to appreciate the occasional individual, on the whole \Filgzed{} had no love for \humans{}. 

\new
And so the five\dash{}\Filgzed{}, her \quo{husband}, her assistant and the two soldiers\dash{}they made their way through the town towards their destination, the \quo{Thieves' Guild} at the northern edge of the market square by the harbour. 

\Filgzed{} followed closely behind Bantoyn, giving her ample opportunity to study the man. \tho{He is rather handsome, now that I come to think of it.} He was a \Mekrii{} \scatha{} like herself, his scales bronze-coloured and almost with a metallic gleam. He had several scars of battle, but in his case they worked in his favour, making him appear more tough, more intriguing. His Niccas were perhaps a bit too broad, but his legs were nice and strong. \tho{Who knows... if he were Tsalt, he just might \emph{be} my \quo{husband}. He still might... for a night, at least.}

Bantoyn had been in \FendorSmall{} before, so they found the spot easily enough. The Thieves' Guild was a tavern, the name presumably some Imetric joke which \Filgzed{} did not get. A large sign above the door depicted a man walking in a crouched position, wearing a half mask and carrying a small bag in each hand\dash{}evidently a thief. Whatever the etymology, the tavern was the designated spot where they were to meet with their contact, a Rissitic mole in \FendorSmall{}. 

\ta{You three stay out here,} Bantoyn ordered. \ta{Filiza and I will go in.}

The Thieves' Guild turned out to be a large and quite well-kept inn. There were at least a score of tables, a long bar and a musician\dash{}a young \sphyle\dash{}playing cheerful tunes on a lute. The innkeeper was a fat woman with red hair\dash{}\tho{must be a foreigner,} she thought\dash{}who gave a rather fierce impression. \Filgzed{} was scanning the room and studying the various patrons when Bantoyn poked her to get her attention, then pointed. Near the corner left of the door sat a man, a short, stout \human{}, dark-skinned with dark brown hair falling past his ears and a short beard on his chin. He wore a black-and-white chequered scarf wrapped around his head\dash{}the agreed signal that was to mark the mole. 

The man sat at a table drinking a mug of beer. They approached, and Bantoyn addressed him in Imetric. \ta{Pray tell me, good man. How does the ale taste here on Fendor?} 

\ta{Better than water,} said the man, \ta{but it pales compared to the dark beer brewed in Breccanum.} 

\Filgzed{} smiled. \tho{The correct, designated code response.} They sat down. 

\ta{Do not get too comfortable,} said the mole. \ta{We will not stay long. We cannot talk in this place.} He raised his beer-mug to his mouth. \Filgzed{} gave him an annoyed look, a look that said: \bodyl{Do not waste our time, fool.} Yet the man took his time draining the mug. At last he set it down and stood up. \ta{Come,} he said, with a gesture and tone much too imperious for \Filgzedz{} liking. \tho{I am Tsalt-\Ginfik, by \Nechsain{}, and this man will show me my due respect!} But she could scarcely proclaim her caste and rank here in the inn, so she and Bantoyn had no choice but to follow. 

They emerged into the street, where their three companions waited for them. \Filgzed{} immediately gestured for them to follow, eager as she was to reassert her authority. 

As they walked, \Filgzed{} made a point of walking beside the mole, not behind him. The mole led them through the street and into an alley. When she was satisfied that there were not too many potential eavesdroppers, \Filgzed{} drew closer to the man and whispered: \ta{You should mind your behaviour around me, \human{}. I am \Filgzed{} Hedrail Tsalt-\Sheshefkesad-\Khiffesh. Who are you?}

\ta{I am \Gisshorn,} said the mole, \ta{and you should not spew titles and Spirit-names so casually. You may call me Mestos.} 

\Filgzed{} was silent. \tho{A \Gisshorn! Yes, given the nature of the mission I suppose I should have suspected.} The secretive agents of the Spider Order were rarely encountered\dash{}at least, they rarely revealed themselves as such. Answering directly to the \TsaltNyzleth, they held great authority and could, in the high priestess' name, supersede the command of even Tsalt, and as such, they were feared by many and even resented by some, especially the Tsalt. The \Gisshorn{} were supposed to use their authority only when `the good of the empire' demanded it, but only the \TsaltNyzleth{} could impeach them for it, so people were naturally wary of them. \tho{And on a top secret military mission such as this, it would be easy for him to invoke `the good of the empire',} \Filgzed{} mused. So she followed in silence. 

\new
After a while the mole halted by a door. He opened the door and entered. \ta{This is my house. Come in.} They entered, and Mestos led them through a medium-sized living room into a larger, barn-like room, sparsely furnished with a few chairs and a table. There were several people in it, and \Filgzed{} recognized Ragev, Kirm and Noll, her three servants, here to assist her in the mundane parts of the spell ritual. \tho{I see the second party has arrived first. Good. I would hate to be forced to wait.} 

In addition to the servants there were three more people in the room: Seated at a table two armed \scathae{}, clearly soldiers, and standing near the wall one woman. She was young, of average height and build, with red hair reaching to her shoulders. She was dressed in inconspicuous civilian garb, a pair of bright red sashes crossing her chest the only aspect of her clothing that caught the eye, and she appeared unarmed. Still, something subtle in her stance, the way she lounged by the wall, relaxed yet somehow seeming ready to pounce at any minute, that made \Filgzed{} suspect that she was more than a mere civilian. More likely she was the other mole, Mestos' companion and perhaps another \Gisshorn. 

\ta{I trust this room is sufficient for your needs, \Sheshefkesad?} said Mestos, now in Rissitic. 

\ta{Hm. It will do.} \Filgzed{} did not dare try to assert authority over the \Gisshorn, but she could at least answer his rudeness in kind. %\ta{Has our second party arrived yet?}
\ta{How long till Sundown?}

The woman by the wall answered. \ta{An hour. Slightly less.} 

\Filgzed{} regarded her. \ta{And you would be...?}

The woman's answer was curt. \ta{\Gisshorn. Codename Semphai.}

\tho{As I thought.} \Filgzed{} drew herself up. \ta{We must prepare the summoning, and we have little time to waste. From now on I am in command.} She paused, eyeing the two \Gisshorn{} as if daring them to challenge her. Neither moved, nor seemed to react at all. Semphai regarded her with a cool, arrogant look whereas Mestos stared into space, appearing lost in thought. She turned to her servants. \ta{Boys, you know what to do. Set up the materials. Bantoyn, have the soldiers watch all entrances to the house.} The soldier bowed his assent and made the sign of the dagger: a flat hand held up before his face, then the hand moving down to become a fist touching his chest.

%a flat hand held up before his face with the thumb near his face and the edge pointing forward, then the hand moving down to become a fist touching his chest, fingers inward. It was the formal Rissitic greeting, an imitation of \Nechsainz{} symbol: A...

She turned to the \Gisshorn{}. \ta{You two. I assume you have the prisoners. Bring them.} 

Semphai did not move. Mestos stood unmoving a long moment, his eyes flicking to his companion, then he gave a small nod and turned to leave the room. Semphai followed, her eyes not leaving \Filgzedz. The sorceress stared back, refusing to be the first to break eye contact, until finally Semphai turned her head and left through a door. 

\ta{Noll. Writing aids,} she commanded. The servant provided her and Dzavish with chalk and coal, and the mages began drawing preliminary glyphs and diagrams while the servants set up candles and braziers of incense and laid out the styli, staves, wands, amulets and ritual daggers that the \Sheshefkesad{} would need. 

And, of course, the most important ingredient in the spell: the precious glyph heads. They were the embalmed and mummified heads of humanoids (two \human{} and two \scathaese{}) enscribed with occult glyphs and holding powerful spells, each of the four heads containing the prepared spell to summon one \daemon. Even using the powerful glyph heads, spells such as these could be held in reserve only for a short time\dash{}a few hours at most\dash{}so they had to act quickly. If the heads could last indefinitely, they could have created them many days in advance and had plenty of time to infiltrate \Cicora{} and prepare for the summoning. As it was, \Filgzed{} and her fellow \Sheshefkesad{} had performed the spell to contact the \daemons{} and bind them to the glyph heads the same afternoon, aboard the flagship, \shipname{Mother Tiamat}, after which they had immediately disembarked for \FendorSmall{} in their merchant boat. The spell components had been hidden in the sacks of spices in which they allegedly traded, the glyph heads buried deep in the sack of strong-scented pamuro in order to mask their charnel smell.

The last accessory that they needed Filgzed kept on her person. And an important one it was: The shade pearls. They were usable as a focus for a spell that, if successful, would create a mystic `shroud' that would impede magical detection, thus hiding their conjurings from the Imetric mages. As such, the pearls were absolutely crucial to their mission, for the summoning ritual would be length, and, naturally, if an Imetrian mage were to catch the scent of Rissitics brewing some tremendous spell, they would be quick to send soldiers and mages of their own to thwart them. Unfortunately, the very rare pearls would be consumed in the spell, so this opportunity was unique. They could not afford to fail. \emph{She} could not afford to fail. Her honour and career depended on it. 
%This was a unique opportunity to mount a sneak attack from behind the enemy's lines

A door opened behind her, and out of the corner of her eye she saw the two \Gisshorn{} returning from the cellar, leading the prisoners. There were six of them, four \humans{} and two \scathae{}. 

\ta{You need four,} Semphai stated in a matter-of-fact, almost lecturing tone, \ta{but we brought all six anyway.}

\Filgzed{} ignored her, finishing the diagram she was drawing before finally turning to them. She studied the prisoners. The \humans{} were a young woman and man, a somewhat older woman and an old man; the \scathae{} were both \daxes{} in their middle years. All six were stripped naked and securely bound and gagged. Some had bruises, but they seemed mostly unharmed. Physically, that is. Presumably, they were random \Cicorans{} that the \Gisshorn{} had kidnapped. She did not know how long they had been kept prisoner, nor did she care to find out. 

\ta{Hm, yes,} she said. \ta{I will take the four \humans{}. Keep the \scathae{} in reserve.} As a general rule, she would kill a \human{} before a \scatha{}. \tho{Well, of course, the \scathae{} will have to die, too. But their deaths will be more merciful.} \ta{Boys, tie them down.}

\ta{Be careful with them,} Mestos warned the servants. \ta{We have kept them hale and fresh as requested, and especially Icum here}\dash{}he indicated the young man\dash{}\ta{can be rather... uncooperative.} 

And true, Icum proved uncooperative. He struggled fiercely, as did the older woman, and took the servants quite some effort to get them securely tied down on the floor, placed strategically amid the glyph diagrams. The young woman and the old man were too terrified to put up much of a fight and were easily secured. 

At length, \Filshed{} looked up from her work, having completed the last of the preliminary diagrams. Despite the occult appearance of the scene\dash{}mystic symbols, braziers, candles and the ghastly glyph heads\dash{}no magic had been worked yet, only preparations. They would not begin the actual spell until sundown, at which point they would need the shade pearls. 

\ta{How long till sunset now?} she asked aloud. 

\ta{Half an hour,} Semphai answered. 

The middle-aged female prisoner squirmed and tried to cry out under her gag. Filshed turned to her.

\ta{Do not be impatient, \human{},} she lectured in Imetric. \ta{Your time will come. Soon.}



%When the ritual proper begins, the braziers are lit and \Filgzed{} and Dzavish carve glyphs in the floor and in the victims' flesh, in the victims' own blood. And of course, they chant words of power. 

%Before the chapter begins, contact has already been made with four Beth Ky'osh \daemons. The spells to summon the \daemons{} have been prepared, and much energy has already been invested. These spells are contained in the glyph heads, awaiting an unlocking ritual. (The process of enchanting a glyph head binds the soul of the head's owner as a prisoner of the spell, and the soul is kept bound to the body and used to contain the spell.) In the ritual, the glyph heads are each placed on or near a prisoner and mystic power is channelled through them. Ultimately, the glyph heads are shattered, the power is unleashed and the spells are completed. The Beth Ky'osh tear through the Veil of the Worlds and consume the bodies and souls of the prisoners. 



\placestamp{Near the centre of Fendor}
%
\ta{\Neftsaid!} 

\Narkiza{} turned to the messenger. \ta{What?}

\ta{Vekhtet Tsalt-\Kseinga{} requests your attendance, \Neftsaid-\Hashkfed.} 

\ta{Very well. Come, Belgrim.} He pulled the reins and turned Belgrim around, moving back from the front of the army to make his way towards the centre, where the priests were riding. Spotting Vekhtet, he moved in to ride beside her. 

\ta{\Kseinga. You needed to speak with me?}

\ta{Yes. The Imetrians have sent out an army from \Cicora{} and are moving against us.}

\ta{Good. Can you determine how many?} 

\ta{Not quite a thousand, but close.}

\ta{Excellent. Thank you, Tsalt-\Kseinga.}
