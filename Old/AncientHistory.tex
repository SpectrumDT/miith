\chapter{The History of Mith - one version}

Billions of years ago:

Nether beings are native to Mith, called Dark Elders. 
Dark Elders dwell in the oceans and dominate Mith. 
Dark Elders go dormant. 
A number of dynasties of gods rule Mith. 



- 1 million:

Mith inhabited by different cultures and gods, called Old Gods. 
Mighty Nether powers come to Mith, called Invaders. 
There are 10 Invaders. They are led by Hyrakht. 
Each Invader represents a force of evil. 
Hyrakht = oppression, Ezzeyrath = envy. 
Others: Fear, anger, ignorance, apathy, arrogance. Last three unknown. 
Invaders conquer Mith, Old Gods flee or perish. 
Invaders create Tyrant Worms (Unn-Hathr) to enslave population. 
Worms are powerful psionics. Use mind control to rule. 
Resistance is formed, led by Eagle Dragon King Rei-Canix. 
Resistance endures for hundred thousands of years under the unbroken dynasty of the House of Rei-Canix. 
Worms fear water. Worm mind control does not work through water. 
Life goes on beneath the seas. 

Rebellion is in hiding. Seems weakened. 
Tyrant Worms rebel against Invaders. 
Invader Ezzeyrath betrays his fellows and sides with Worms. 
Great Cataclysm ensues. 
Ezzeyrath slain by Hyrakht. 
Rebellion emerges from hiding. 
Five other Invaders perish. 
Hyrakht curses Worms, so that Sunlight kills them. 
Hyrakht and two other Invaders imprisoned on Mith. 
One Invader flees. 
99\% of Worms die. 
Rebellion triumphs, Worms forced underground. 
Eagle Dragon species exterminated in the war, only survivors are a few royal eggs.
Only one egg survives and hatches, becoming Princess Iza-Lyranis. 

Golden Age ensues, lasting for approximately 2000 years. 
No gods live on Mith, only distant alien gods are worshipped. 
Mith in peace, with the Princess as a global symbol of peace. 
Mith people develop high technology. 
Reach TL 8/9, develop primitive space travel. 
Princess Iza-Lyranis, last Eagle Dragon on Mith, dies at age 2066. 

With the Princess' death, religious conflicts escalate.
War of Oblivion begins. Lasts for about 50 years.
Most sentient life wiped out, Mith bombed back to before the Stone Age.

-4,000:
New gods arrive or appear on Mith.

-3,000:
Hriist and Salacar come into being on Mith. 
They begin breeding the Rachyth and Scatha species.

-2,000:
Nishi joins Salacar.

-1,000:
Scathae and Rachyth begin using tools.
Eoncos and Thaemas join Salacar.

-500:
Rachyth develop hieroglyphic writing. 
Rachyth begin to enslave the D�nac. 

-200:
Scathae develop alphabetic writing. 
Scathae begin to use bronze.

0: 
Salacar unites Imetric empire. Scathaese language spreads. 
Dessali and Hiothrex join the Imetrium. 

1:
Salacar declares year 1 of the United Imetrium. 

100:
First great war between Imetrium and Hriistites. 

400:
Rachyth develop green bronze. 

600:
Scathae discover iron. 



\chapter{The History of Mith - yet another version}
The history of the planet Mith has been shaped by several great wars, known as Cataclysms. There have been four such Cataclysms in recorded history, and there may have been more before that. Mithian history is divided into four Ages, with each Cataclysm marking the end of an Age.\footnote{Some people may feel that cataclysms are cheesy, because everyone and their mother has them in their fictional world. I don't care. My goal is not to be original and innovative at all costs, but simply to create a cool world.} 



\section{The Five Ages}
Here follows a description of the five Ages and the four Cataclysms. 

\subsection{The First Age: The Age of Beauty}
\label{Age of Beauty}
\index{Age of Beauty}
The first known Age is the Age of Beauty. 'Known' is a strong word, however; the Age ended many thousands of years ago, and no records of it survive. The Age is known only from ancient myth, and it is likely that most such myths, if not all, are pure fiction. 

The Age of Beauty is commonly believed to have spanned many thousands of years. Some believe that the Age was infinitely long, dating back to the mythical 'beginning of time', infinitely far back, and up to the Invasion. The latter is completely false, since the planet Mith has a finite age of around five billion\footnote{I will use a 'billion' to mean $10^9$, a thousand millions. As far as I understand, this is the default practice in English.} years, but the Mithians do not necessarily know that. 

According to myth, the Age of Beauty was a time of bliss and happiness. Everything on Mith was perfect and beautiful, all creatures lived in harmony, and war and evil were unknown. 

All this changed with the First Cataclysm, known as the Invasion. 

\subsection{The Second Age}
\subsubsection{The Invasion}
\label{Invasion}
\index{Invasion, the}
At some point, several thousand years ago, a race of alien creatures from another planet arrived at Mith and decided to colonize the planet. These creatures, now known as the Invaders, possessed vastly superior knowledge of technology and magic (probably TL10) and were able to subjugate or exterminate all native Mithians with ease.\footnote{At least, this is the story most often heard. Other myths claim that the native Mithians were advanced and able to defend themselves, and that Mith only succumbed after a long, devastating war.} After this, the Invaders reigned supreme and built great cities all over the planet. 

\subsubsection{The Age of Darkness}
\label{Age of Darkness}
\index{Age of Darkness}
The reign of the Invaders is now known as the Age of Darkness. Its length is not known. It is believed to have been at least 1000 years, but it may have been tens or even hundreds of thousands of years. All stories portray the Invaders as extremely monstrous and evil, keeping the native Mithians cruelly enslaved and oppressed. 

The leader of the Invaders was named Hyrakht\footnote{[HAJ-raakht]}\index{Hyrakht}. Another of the great Invader Lords was Ezzeyrath\footnote{[EZ-zej-raath]}\index{Ezzeyrath}. Some myths claim that each of the Invader Lords was the personification of some universal force of evil. Hyrakht is said to represent Oppression while Ezzeyrath embodied Envy. This interpretation is likely to be pure fiction, though. (Hyrakht and Ezzeyrath are the names that appear in the most common myths. They are also known by many other names.) 

The Invaders had several races of monstrous creatures for their servants, of which the best known are the Glekkyat\footnote{[GLEK-kj�t]} Worms\index{Glekkyat}. The origin of the Glekkyat is disputed. They may have been native Mithians that existed before the Invasion, they may have been bred and modified from native creatures, or they may have been alien creatures brought here by the Invaders. 

%Something about Hyrakht and Ezzeyrath
%Something about Tyrant Worms

\subsubsection{The Insurrection}
\label{Insurrection}
\index{Insurrection, the}
A Mithian resistance existed. They fought the Invaders and their minions using stolen Invader technology and magic and with the aid of mighty alien gods known as the Star-Gods\index{Star-Gods}. Still, the Mithians could do little against the Invaders. The tables did not turn before the Glekkyat started to rebel against their masters. Even so, the Insurrection might have failed, if not the Invader Lord Ezzeyrath had decided to betray his fellows and side with the rebels. 

The restult was a great and terrible war, lasting several decades if not centuries. During this war, the surface of Mith was devastated and entire continents were destroyed as others arose. Myths tell that the conflict culminated in a mighty duel between Hyrakht and Ezzeyrath, in which both slew each other. This duel is believed to be fictitous, but it is known that the Invaders were ultimately defeated. At the end, the Invaders managed to activate a mighty weapon that wrought immense destruction among their foes, killing millions if not billions. Especially the Glekkyat were hit hard by this final strike of the Invaders and were all but wiped out. Some kind of curse was cast upon the suriving Glekkyat, with the effect that could no longer endure the light of the Sun but were forced to flee underground. The vast majority of the Glekkyat were destroyed, but a few survived and may still exist in the Modern Age. The Invaders themselves are believed destroyed, though some tales claim that they (or at least some of them) were merely banished or imprisoned. 

Dragons are known to have been among the native Mithians that fought in the Insurrection. Fighting alongside them were several other races that are now extinc, wiped out in the war or its aftermath. Some tales also feature Vamons, Scathae and Humans in the Insurrection, whereas other myths claim that these races did not rise until millennia later. 



%But their reign did not last. Stories disagree whether the Resistance\index{Resistance, the} had survived unbroken since the Invasion or whether it arose at some later point (the latter is generally considered much more likely), but it is known that at some point, a native Mithian Resistance had gathered a great following. Armed with stolen Invader technology and magic and with the aid of mighty gods from the stars\footnote{Gods from the Stars}. The result was a great and terrible war, lasting several decades, now known as the Liberation. During the Liberation, the surface of Mith was devastated and entire continents were destroyed as others arose. 


\subsection{The Third Age}
\subsubsection{The Draconic Age}
\label{Draconic Age}
\index{Draconic Age}
The Insurrection left Mith bombed back to the Stone Age. New civilizations slowly rose from the ashes. Among the survivors of the war, the most powerful were the Dragons. Over the course of millennia, they slowly developed their own culture. The great leader of the Dragons during the Insurrection was Queen Tyndarea\footnote{[tin-DAA-ree-a]}\index{Tyndarea}, so in her honour, the entire race named themselves the Tyndarean\footnote{[tin-DAA-ree-an]}\index{Tyndarean Dragons} Dragons. 

Under the leadership of House Endarex\footnote{[en-DAA-reks]}\index{Endarex}, the descendants of Queen Tyndarea, they built a great empire, centered at the land of Nom, to the east of what is now Irokas. House Endarex became the first dynasty of Dragon Kings. The Dragons of Nom retained their belief in the Star-Gods who had aided them during the Insurrection. These are the gods still worshipped in Irokas today. 

Technology progressed only slowly. The Draconic Age lasted thousands of years, perhaps 20,000 years. There were several civil wars and several dynasties of Dragon Kings. 

\subsubsection{The Fall of Nom}
\label{Fall of Nom}
\index{Fall of Nom}
A great civil war. Dark, forbidden magic was used. It resulted in a magical catastrophe. Terrible monsters were summoned from the Baneworld of Erebos and other places. The land of Nom was laid waste. In the ensuing chaos and strife, House Irokas seized power and become the new Dragon King dynasty. The throne was moved to Mount Irokas and a new kingdom was named after them. But the kingdom of Irokas would not rise to the same power and glory as Nom, at least not for thousands of years. 



\section{The Age of Dragons}



\section{The Days of the Empire}
\label{Days of the Empire}
\index{Days of the Empire}
The new Dragon Kingdom of Irokas is greatly weakened. The Dragons are decimated, with more than 75\% of their number slain in the war. The remaining Draconic Houses are splintered. Many are not loyal to the King. Irokas therefore has little power. The Dragons only rule the land that is now known as the Kingdom of Irokas. On the rest of Mith, humanoid cultures rise to power. 

Because of their infertility and long lifespans, it takes many millennia for the Dragons to replenish their numbers. Humanoids, even Vaimons, reproduce faster. Therefore, they increase in numbers and power. Especially the Vaimons are powerful, because they discover Iquin-Nieur channeling magic. 

About 1000 years after the Fall of Nom, the Vaimon Empire is founded. It is ruled by a Vaimon Emperor from the Rainbow Throne. There are six Vaimon clans, each led from a magical throne of crystal: Diamond (Quaerin), Ruby, Sapphire, Emerald (Geican), Topaz (Redcor) and Onyx (Yrzhell). This Empire lasts for around 2000 years and reaches TL6. 

The Emperor is selected astrologically: %When the Emperor dies, Imperial astrologers search for signs of the new Emperor's coming
Imperial astrologers read the stars to ascertain the birth and location of an Imperial Scion. When a Scion is found, he or she (while a young infant) is taken to the Palace to be trained as the current Emperor's successor. 

\subsubsection{The Fall of the Empire}
\label{Fall of the Empire}
\index{Fall of the Empire}
The Fall of the Empire has to do with Bel'zhir\index{Bel'zhir}, the Dark Queen\index{Dark Queen, the}. She is a Vaimon woman of Clan Geican who becomes Empress, by perfectly normal means. She is evil and does not want to relinquish power. She uses dark magic to extend her life beyond its natural span. Perhaps she becomes a Nieur vampire? 

Anyway, she is ousted from power. But she managed to escape with her life. She plots to regain the throne. She allies herself with Clan Quaerin, in addition to her own Clan Geican. A great war ensues. 

Bel'zhir is slain, but her soul is not destroyed. Using powerful magic, her enemies successfully banish her soul from Mith (into the darkness of Nieur, as the Redcor tell the story), but they fail at destroying her. In time, she learns to communicate with people on Mith again and plots to return. 

Anyway, the defeat of the Queen does not end the war. Weapons of mass destruction are used, including TL6 atomic bombs and various magical doomsday devices. The Empire is destroyed and Mith bombed back several TLs. Thus end the Fourth Age. 

\subsection{The Fifth Age: The Modern Age}
\label{Modern Age}
\index{Modern Age}

