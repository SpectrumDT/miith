\section{Characters on Mith}

\subsection{Morals and Alignment}

\subsubsection{The concepts of good and evil}
In my world, moral concepts such as good and evil are \emph{not} objective. These terms cannot be defined objectively; they remain cultural phenomena and personal opinions. Persons and civilizations can only be properly described by their values, ideals and opinions, not by some archetypal `aligment'. This represents my actual view of moral philoshopy. 

% Unlike in some RPGs, such as Dungeons and Dragons and the Palladium RPGs, people are not defined by an `alignment'. This represents my own world view. Concepts of good and evil cannot be defined objectively; they remain cultural phenomena and personal opinions. Therefore, persons and civilizations can only be properly described by their values, ideals and opinions, not by some archetypal `aligment'. 

This being said, I do use the concepts of good and evil. I just need to stress that when I use these terms, they represent my own subjective moral judgement. For instance, I tend to describe the Imetrium as `good', but others may find that such a centralistic theocracy with communist leanings is by no means `good'. 

At any rate, good and evil are not physical or metaphysical concepts in any sense. Therefore, it is not possible, for example, to cast a spell that damages evil creatures only. It is also not possible to cast a spell that harms only creatures that adhere to a particular ideology or religion, since `adhering' to a particular view is not well-defined and detectable.\footnote{Even if it can be considered objective in some sense, to design a spell that would detect it is completely unrealistic at TL3, since such a spell would have to measure brainwaves, which is very poorly understood even at TL7. Perhaps at high TLs this will be possible, but in a medieval setting, forget it.}
It is, however, possible to have a spell affect, for instance, only those who are under the blessing of Rissit Nechsain, because such a blessing is an objective and detectable metaphysical phenomenon.\footnote{A simple spell of this type would only be able to detect a certain set of `standard blessings' that Rissit might use. A more advanced spell might be able to detect any blessing bestowed by Rissit by detecting his magical `signature'. It might be possible, however, for Rissit (or his priest) to `mask' his magical signature in order to fool such a spell.}

\subsubsection{Determining morals and personality}
So, what determines moral outlook and `alignment'? What determines whether a given person is good or evil? 

Well, I tend to support the theory of free will. This means that any living creature is able to make (and responsible for) its own choices. The mind is affected by external and internal forces, but only to a limited degree. The person always has the final say. Only in extreme situations (and depending on the person's willpower) may very strong emotions (such as fear) be able to overcome the rational mind and take control of the body. 



\begin{comment}
This section will be heavily opinionated and influenced by my own view of moral philosophy. If you, as a GM, have a different world view, you may discount my rantings here. But this is the world view on which I have based my world. 

A basic concept is the axiom of free will: Any living creature has a free will. The mind is non-deterministic. The choices that a creature makes are not (entirely) controlled by outside forces and cannot be predicted (with certainty). 

The mind is affected by external and internal forces, however. The mind is subject to causality, but not entirely. Such causal forces affect the statistical likelihood that a certain person, in a certain situation, will do a certain thing, but only statistically. The individual person always retains his free will to choose between alternatives. 

So, what does all this have to do with anything? Well, it has to do with how a personality is formed, and how a person becomes good or evil. 

In my world, the forming of a person's personality is determined by three factors: Genetic, social and personal. 
\end{comment}




But a personality is not shaped by the person alone. In my system, there are three factors to consider regarding the development of a person's mind. 



\begin{itemize}
  \item \textbf{The genetic factor} is determined by the brain in which the mind resides. Different species are evolutionally conditioned to think in a particular way, to have certain insticts and intuitions. This affects a person's choices, including moral choices. For example, Scathae are by nature social creatures, so a Scatha would be genetically inclined to embrace `social' ideologies, such as utilitarianism or elitism\footnote{`Elitism' is to be understood as the view that a certain group is more important, more worthy, than others, and has a natural right to rule. The Rissitic religion is elitist. An example of an elitist ideology in RL is nazism. I generally consider elistism to be evil.}. Dragons, on the other hand, are more solitary by nature, and as such are more likely to embrace individualistic world views such as anarchism or nihilism. 
  \item \textbf{The social factor} is how a person is raised and educated. It is determined by the surrounding society and the people in charge of raising the person (parents, school, church). For instance, a person born and raised in the Imetrium will be indoctrinated to think in terms of Imetric values, such as the individual being prepared to make sacrifices for the greater good of the community. Such a person is less inclined to develop an anarchistic, independent attitude. Social indoctrination sometimes backfires, however, creating rebels who purposely reject the values of their society. 
  \item \textbf{The personal factor} is determined by the individual. The personal factor can be further divided into two: 
  \begin{itemize}
    \item \textbf{Free will:} This constitutes the conscious choices that the person makes in his life. 
    \item \textbf{Residual personality:} Keep in mind that in my world, an infant is not a blank slate, since it is an incarnation of a soul that has lived before. The soul will carry over some degree of personality from his previous life. This `residual personality', in turn, is determined by genetic, social and personal factors from the person's previous life. 
  \end{itemize} 
\end{itemize} 



I will not go into details about the relative importance of the three factors. I will generally assume that the personal factor is stronger than the genetic one, ie., that any intelligent creature has the ability to understand and adopt any moral code, even one far removed from the typical behaviour of the species.\footnote{This may not apply to creatures that are very `monstrous' and `alien'.} This means, for instance, that even a Balrog (normally very evil) may choose to be sincerely gentle and good. This just happens very rarely. 




