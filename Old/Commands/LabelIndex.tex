%%%%%%%%%%%%%%%%%%%%%%%%% DISABLED %%%%%%%%%%%%%%%%%%%%%%%%%
% Give pronunciation as a footnote. 
%\newcommand{\pronunfoot}[1]{\footnote{[#1]}}
\newcommand{\pronunfoot}[1]{}
% Create tuple of (word, pronunciation). At the moment, it is just printed as a footnote. In the future, I hope to be able to automatically compile a pronunciation index to print at the end of the book. 
\newcommand{\pronunciation}[2]{\pronunfoot{#2}}
% Pronunciation, but with an added comment. 
\newcommand{\pronunciationlong}[3]{}%{\footnote{[#2]. #3}}

% Print a word and give pronunciation. 
\newcommand{\pronun}[2]{#1\pronunciation{#1}{#2}}
% Print a word in emphasisis and give pronunciation. 
\newcommand{\pronune}[2]{\emph{#1}\pronunciation{#1}{#2}}
% Print, pronuciation and add word to index. 
\newcommand{\pronuni}[2]{#1\pronunciation{#1}{#2}\index{#1}}
% Print emphasized, pronun and index entry. 
\newcommand{\pronunei}[2]{\emph{#1}\pronunciation{#1}{#2}\index{#1}}
% Print word and give pronunciation with an added comment. 
\newcommand{\pronunlong}[3]{#1\pronunciationlong{#1}{#2}{#3}}
\newcommand{\pronunelong}[3]{\emph{#1}\pronunciationlong{#1}{#2}{#3}}
\newcommand{\pronunilong}[3]{#1\pronunciationlong{#1}{#2}{#3}\index{#1}}
\newcommand{\pronuneilong}[3]{\emph{#1}\pronunciationlong{#1}{#2}{#3}\index{#1}}

% Introduce a term: Print the word, make a label of it and add it to the index. 
\newcommand{\introduce}[1]{#1\index{#1}\label{#1}}
% Intro: Short for `introduce'
\newcommand{\intro}[1]{\introduce{#1}}
% Intro-double: Introduce, and add a second or variant name as label and index. 
\newcommand{\introd}[2]{\introduce{#1}\index{#2}\label{#2}}
% Intro-emph: Introduce, but print it emphasized. 
\newcommand{\introe}[1]{\emph{\intro{#1}}}
% Intro-label: Introduce, but with a customizable index entry. 
\newcommand{\introi}[2]{#1\index{#2}\label{#1}}
% Intro-label: Introduce, but with a customizable label (useful if the name contains special characters). 
\newcommand{\introl}[2]{#1\index{#1}\label{#2}}
% Intro-pronun: Introduce and also give pronunciation. 
\newcommand{\introp}[2]{\intro{#1}\pronunciation{#1}{#2}}
% Intro-the: Introduce, and add a `, the' at the index entry. 
\newcommand{\introt}[1]{#1\index{#1}\label{#1}} 
\newcommand{\introthe}[1]{\introt{#1}} 
% Emphasized with an added `the'. 
\newcommand{\introet}[1]{\emph{#1}\index{#1}\label{#1}} 
% Customizable label, pronunciation. 
\newcommand{\introlp}[3]{\introl{#1}{#2}\pronunciation{#1}{#3}}
% Emphasized, pronunciation and an added `the'. 
\newcommand{\intropt}[2]{\introt{#1}\pronunciation{#1}{#2}}
% Emphasized, custom label. 
\newcommand{\introel}[2]{\emph{#1}\index{#1}\label{#2}}
% Emphasized, pronunciation. 
\newcommand{\introep}[2]{\introe{#1}\pronunciation{#1}{#2}}
% Emphasized, pronunciation and an added `the'. 
\newcommand{\introept}[2]{\introet{#1}\pronunciation{#1}{#2}}
% Emphasized, custom label and pronunciation. 
\newcommand{\introelp}[3]{\introel{#1}{#2}\pronunciation{#1}{#3}}
% Introduce and give `long' pronunciation (allow for a comment in the pronunciation entry). 
\newcommand{\introplong}[3]{\intro{#1}\pronunciationlong{#1}{#2}{#3}}
\newcommand{\introlplong}[4]{\introl{#1}{#2}\pronunciationlong{#1}{#3}{#4}}
\newcommand{\introeplong}[3]{\introe{#1}\pronunciationlong{#1}{#2}{#3}}
\newcommand{\introelplong}[4]{\introel{#1}{#2}\pronunciationlong{#1}{#3}{#4}}



