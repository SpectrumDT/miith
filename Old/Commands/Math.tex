\begin{comment}
\part{Mathematics}
\end{comment}

\newcommand{\Rp}{\mathbb{R}_+}                % Positive real numbers
\newcommand{\Rm}{\mathbb{R}_-}                % Negative real numbers
\newcommand{\R}{\mathbb{R}}                   % Real numbers
\newcommand{\Q}{\mathbb{Q}}                   % Rational numbers
\newcommand{\Z}{\mathbb{Z}}                   % Integers
\newcommand{\N}{\mathbb{N}}                   % Natural numbers
\newcommand{\ph}{\varphi}                     % Phi
\newcommand{\rh}{\varrho}                     % Rho
\newcommand{\e}{\varepsilon}                  % Epsilon
\newcommand{\fracs}[2]{\textstyle{\frac{#1}{#2}}} % Small fraction
\newcommand{\bimp}{\Leftrightarrow}           % Bi-implication
\newcommand{\imp}{\Rightarrow}                % Implication (`p only if q' or `p implies q')
\newcommand{\pmi}{\Leftarrow}                 % Implication right to left (`p if q')
\newcommand{\x}{\times}                       % Cartesian product
\newcommand{\ri}{\right}                      % Right delimiter
\newcommand{\lf}{\left}                       % Left delimiter
\newcommand{\rpar}{\right)}                   % Right parenthesis
\newcommand{\lpar}{\left(}                    % Left parenthesis
\newcommand{\rtub}{\right\}}                  % Right tuborg }
\newcommand{\ltub}{\left\{}                   % Left tuborg {
\newcommand{\abs}[1]{\left| #1 \right|}       % Absolute/numerical value
\newcommand{\set}[1]{\ltub #1 \rtub}          % Set
\newcommand{\supns}[1]{\sup\set{\abs{#1}}}    % Numerical supremum of a set
\newcommand{\supnsub}[2]{\sup\set{\abs{#1}| \ x \in #2}}
                                              % Numerical supremum of a subset
\newcommand{\bolle}{\circ}                    % Bolle - funktionssammens�tning
\newcommand{\ZMP}[1]{(\mathbb{Z}/#1)^\ast}    % Group of prime remainder classes modulo n
\newcommand{\alter}{(-1)^n}                   % Alternating sequence
\newcommand{\alterp}{(-1)^{n+1}}              % Alternating sequence n+1
\newcommand{\intoi}{\int_1^\infty}            % Integral from one to infty
\newcommand{\intni}{\int_n^\infty}            % Integral from n to infty
\newcommand{\intzi}{\int_0^\infty}            % Integral from 0 to infty
\newcommand{\intzo}{\int_0^1}                 % Integral from 0 to 1
\newcommand{\intoa}{\int_1^a}                 % Integral from 1 to a
\newcommand{\intab}{\int_a^b}                 % Integral from a to b
\newcommand{\sumoi}{\sum_{n=1}^\infty}        % Sum for n = 1...\infty
\newcommand{\sumzi}{\sum_{n=0}^\infty}        % Sum for n = 0...\infty
\newcommand{\sumon}{\sum_{n=1}^N}             % Sum for n = 1...N
\newcommand{\sumzn}{\sum_{n=0}^N}             % Sum for n = 0...N
\newcommand{\floor}[1]{\lfloor #1 \rfloor}    % N rounded down
\newcommand{\ceil}[1]{\lceil #1 \rceil}       % N rounded up
\newcommand{\hfloor}[1]{\floor{#1/2}}         % N/2 rounded down
\newcommand{\hceil}[1]{\ceil{#1/2}}           % N/2 rounded up
\newcommand{\hak}[1]{\left\langle #1 \right\rangle} % Cyclical group <g>
\newcommand{\half}{\frac{1}{2}}               % One half: 1/2 - big
\newcommand{\halfs}{\fracs{1}{2}}             % One half: 1/2 - small
\newcommand{\third}{\frac{1}{3}}              % One third: 1/3 - big
\newcommand{\thirds}{\fracs{1}{3}}            % One third: 1/3 - small
\newcommand{\quart}{\frac{1}{4}}              % One quarter: 1/4 - big
\newcommand{\quarts}{\fracs{1}{4}}            % One quarter: 1/4 - small
\newcommand{\tquart}{\frac{3}{4}}             % Three quarters: 3/4 - big
\newcommand{\tquarter}{\frac{3}{4}}           % Three quarters: 3/4 - big
\newcommand{\tquarts}{\fracs{3}{4}}           % Three quarters: 3/4 - small
\newcommand{\tquarters}{\fracs{3}{4}}         % Three quarters: 3/4 - small
\newcommand{\nth}{\frac{1}{n}}                % One n'th: 1/n - big
\newcommand{\nths}{\fracs{1}{n}}              % One n'th: 1/n - small
\newcommand{\pihalf}{\frac{\pi}{2}}           % Pi half: Pi/2 - big
\newcommand{\pihalfs}{\fracs{\pi}{2}}         % Pi half: Pi/2 - small
\newcommand{\toi}{\to \infty}                 % (something) -> \infty
\newcommand{\limi}[1]{\lim_{#1 \to \infty}}   % lim for #1 going to \infty
\newcommand{\limmi}[1]{\lim_{#1 \to -\infty}} % lim for #1 going to minus \infty
\newcommand{\limz}[1]{\lim_{#1 \to 0}}        % lim for #1 going to 0
\newcommand{\limni}{\limi{n}}                 % lim for n going to \infty
\newcommand{\limnmi}{\limmi{n}}               % lim for n going to minus \infty
\newcommand{\for}{\quad \textrm{for} \quad }  % Math mode: `for ...'
\newcommand{\forni}{\for n\to\infty  }        % Math mode: 'for n -> infty'
\newcommand{\comma}{\quad \textrm{,} \quad }  % Math mode: (formula) , (text)
\newcommand{\Yps}{\Upsilon}
\newcommand{\YPS}{\Upsilon}
\newcommand{\gaffel}[1]{\left\{ \begin{array}{ll} #1 \end{array} \right.}

