\chapter{Creatures of Mith}



\section{Species vs. race}
\index{Species}
\index{Race}

Fantasy literature usually calls different humanoid beings \quo{races}. From a technical, biological point of view, usually the term \quo{species} would be more correct. 

The most commonly used biological definition of a species is that two creatures are of the same species if they can interbreed and produce viable, fertile offspring. In this sense, the Men and Elves of Tolkien's world, for example, are a single species, since Half-Elves and Quarter-Elves exist. In Dungeons and Dragons, half-orcs exist, so orcs must belong to the same species as \humans. 

Race is a more narrow term. A race is a subset of a species. Two creatures of different species are automatically also of different races, but two creatures of different races might belong to the same species. 

On Mith, in the majority of cases, two creatures of different \quo{types} \emph{cannot} interbreed. There are, for instance, no \scatha{}/\meccaran{} hybrids. Thus, \scathae{} and \meccara{} are different species. In a few cases, however, two \quo{types} are actually the same species. \Dragons{} have been known to polymorph into \scatha{} form and mate with \scathae{} to produce hybrids called \rachyth{} (see section \ref{\Rachyth{}}). 
%\Humans{} and Vaimons, for instance, belong to the same species and can interbreed. 
%For instance, \dragons{} are divided into two sub These are obviously different and have different attributes, but they can interbreed without difficulty. These three kinds of \dragons{} are different races belonging to the same species. (In contrast, there exist other species of \dragons{} on Mith, including the Quetzalcoalt and the \linnorms.) 

I will often use \quo{race} where I ought perhaps to use \quo{species} instead. In those cases where there is reasonable cause for confusion, I will make a special note. 



\section{Categories of creatures}
\indexx{Categories of creatures}
\index{Taxonomy}
The various creatures can be divided into a number of \quo{taxonomical} categories. The categories that I use here are mostly vague and subjective, and are not meant to resemble actual biological taxonomy.\footnote{In fact, my categorization is independent of taxonomy: A mammal, for instance, may be an \quo{animal}, a \quo{humanoid} or a \quo{monster}, depending on its attributes. A reptile likewise. So these categories are \quo{orthogonal} to biological taxa.} 

\paragraphh{Humanoids}
A \quo{humanoid} is an intelligent creature that has a body shape similar to that of the dominant races, such as the \scathae{}. It is generally used in a quite broad sense. %The centauroid Tchacolda are considered humanoids, but the spidery Mlishur are generally not. \Dragons{} are also not humanoid. 

Non-intelligent creatures are not humanoids. A non-intelligent creature of humanoid shape is probably a \quo{monster}. 

Humanoids include \humans{}, \scathae{}, \meccara{}, \nagae{} and \cregorrs{}. 

\paragraphh{Monsters}
A \quo{monster} is a creature that is non-humanoid and somehow \quo{monstrous}. Usually, a monster has some supernatural powers and is fierce and frightening (otherwise, it is probably an \quo{animal}). 

Monsters can be intelligent or non-intelligent. An intelligent creature that is non-humanoid but not particularly supernatural is probably a monster. 

Monsters include \dragons{}, Manticores, \wyverns{}, Gryphons and Rocs (although the last two could also be considered simple \quo{animals}). 

\paragraphh{Animals}
An \quo{animal} is an unintelligent creature with no or few supernatural aspects or powers and resembling RL animals. (In this context, animals is not the entire kingdom of Animalia, since that would include many intelligent creatures also.) 

\paragraphh{Demons}
A \quo{demon} is a supernatural creature that is somehow \quo{evil}. Usually, demons are more monstrous and more powerful than monsters, but the line between the two is very blurry. Many demons are inorganic in nature. 

Sometimes, any creature from \Tuat{} is considered a demon. 

Balrogs and Imps are clearly demons. Djinn are usually also called demons. 

\paragraphh{Angels}
An \quo{angel} is a supernatural creature that is somehow \quo{good}. Typically, supernatural creatures that serve a particular god or religion are called angels, at least by the adherents of the religion in question. A creature might be a \quo{demon to some, angel to others}. %Angels may be minor gods on their own. 

Thorn Angels and Wraithworms are examples of creatures that might be called angels. Both might also be considered demons by those not of their religion. 

\paragraphh{Gods}
A \quo{god} is a creature with great supernatural powers who is typicalled worshipped by lesser creatures. Usually, gods have the ability to channel their magical power through a servitor (a priest), enabling the priest to perform miracles or cast spells.%\footnote{There is no real difference between \quo{miracles} and \quo{spells}, but generally, \quo{miracle} is a religious term while \quo{spell} is a scientific term.} 

Some demons and angels are so powerful that they might also be called gods. Sometimes, the gods of an enemy religion are classified as demons. 

Examples of gods are Salacar (section \ref{Salacar}), Rissit Nechsain (section \ref{Rissit}) and Nerrhan-Koss (section \ref{Nerrhan-Koss}). Rauthor (section \ref{Rauthor}) is an example of a demon god. 





\newpage
\section{Format of creature descriptions}
\subsection{Creature name}
%The name is sometimes given in singular, somtimes in plural. 

This opening section gives a superficial description of the creature and its role in the world of Mith. 

\subsubsection{Name}
This section may seem overblown, but it serves a useful purpose. It gives the singular and plural forms of the name and tells you what language the given name is in. (Sometimes, the listed name (and its declination) is in a Mithian language, like Imetric or Clictua or \Draconic{} Kingstongue. Other times it is in an RL language, such as English or Latin.) 

The \quo{name} section also gives the adjective associated with the name of the creature. For instance, the adjective associated with \quo{Elf} is \quo{Elven} (or \quo{Elvish}). The adjective associated with \quo{\dragon{}} is \quo{\draconic{}}, and so on. 

\subsubsection{Physique}
Describes the creature's appearance and size and some of its attributes. There is overlap between this and the \quo{biology} section. \quo{Physique} contains the more \quo{combat-oriented} data, ie., the information that will be most relevant for an individual creature in a given situation. 

\subsubsection{Biology}
This contains biological data pertaining more to the species than to the individual. This includes information on reproduction, lifespan and role in their natural ecology. 

\subsubsection{Psychology}
Some generic information about the personality of creatures of this race and how they differ from other races (such as \humans). Also describes briefly the typical societies/cultures that the race tends to form. 

\subsubsection{Attributes}
\begin{description}
	\item[Horror effect:] This describes how \quo{horrible} the creature is to meet. See \quo{supernatural horror} in the rules section. 
\end{description}





\newpage
\section{Humanoids}

\subsection{\Human{}}
\label{Human}
Mithian \humans{} are like Earth \humans{}. They are widespread especially in Belkade and in the Far North. 

There are a number of races/ethnic groups of \humans{} on Mith, who correspond to various ethnic groups on Earth. These include: 

\bd
  \item[\introi{Belkadians}{Belkadians (race of \humans{})}] have pale skin and look like the Germanic people of northern Europe (Germany, England, Scandinavia). They are the most common \human{} race, dominating most of Belkade and the Northern Kingdoms. 
  \item[\introi{Fraens}{Fraens (race of \humans{}}] have somewhat dark skin and resemble people from southern Europe (Greece, Italy). They are common in the Imetrium, Durcac and southern Belkade. 
  \item[\introi{Kohons}{Kohons (race of \humans{})}] have dark brown or black skin and negroid features (corresponding to Africans/Negroes). They are common in Durcac and the Far South. 
  \item[\introi{Hoyds}{Hoyds (race of \humans{})}] have brown skin and look like Arabs. They are common in southeastern Belkade (nations such as Geica) and the Orient. 
\ed

\subsubsection{Name}
%Singular \pronune{\human{}}{HJOO-m�n \emph{or} KHJOO-m�n}, plural \emph{\humans{}}. This declination is English. 
As in English. The associated adjective is \emph{\human{}}. 

The races: \pronune{Belkadian}{bel-KEJ-dee-an}, \pronune{Fraen}{FREJN}, \pronune{Kohon}{KOW-hon (Belkadian) \emph{or} K�-hon (Rissitic)} and \pronune{Hoyd}{HOJD} all form plural with an S, and the adjectives are the same as the singular forms. 

%Yes, I know this section is trivial, but I'll be damned if I violate protocol on this one. 

\subsubsection{Physique}
An average Mithian \human{} man is 170-175 cm tall and weighs 70 kg. A woman is 165-170 cm tall and weighs 55 kg. 

\subsubsection{Biology}
The average lifespan of a \human{} is 70 years, 90 in exceptional cases. 

\subsubsection{Psychology}



\subsection{\Meccaran{}}
\label{Meccaran}
\label{Meccara}
\index{\meccaran{} (plural \meccara{} or \meccarans)}
\Meccara{} are amphibian humanoids. They are widespread especially in the warmer climates of the South. 

\subsubsection{Name}
Singular \pronune{\meccaran{}}{me-KAA-ran}, plural \pronune{\meccara{}}{me-KAA-ra} or \emph{\meccarans{}}. %\emph{\Meccara{}} is generally used when referring to the race as a whole, while \emph{\meccarans{}} refer to a specific group of individuals. This grammar is Imetric. 
The word is Imetric. \emph{\Meccara{}} is Imetric declination, \emph{\meccarans{}} is English declination. I will use both forms synonymously. 
The associated adjective is \emph{\meccaran{}}. 

\subsubsection{Physique}
\Meccara{} look like humanoid frogs. They have large, strong hind legs and are fast and agile runners, leapers and swimmers. Their forearms are dextrous, but not very strong. In combat, \meccarans{} rely more on speed than on brute force. \Meccaran{} skin is smooth and no tougher than human skin. 

A \meccaranz{} mouth is filled with sharp teeth that curve backwards. Adapted (among other things) to catch fish, \meccaran{} teeth are effective for biting and holding. \Meccaran{} necks are short, however, so biting is difficult in combat. 

A average, full-grown \meccaran{} female weighs 60 kg and stands about 130 cm tall in her natual posture. The male is smaller than the female, about 50 kg and 120 cm in height. \Meccarans{} do not stand fully erect but in a \quo{crouched} position with the legs bent outwards. They can increase their height by up to one third by stretching out, but this is an unnatural position and can not be maintained for long. \Meccaran{} arms are short, only around 50 cm. 

\Meccaran{} vision is inferior to that of \humans. They are near-sighted and cannot see well at great distances (this is a result of adaptation to a life in dense forests). At close range they see as well as \humans. Their sense of smell is acute - not as good as that of a dog, but far better than a human's. Their other senses are like those of \humans. 

\Meccara{} have the ability to regenerate lost limbs. The time this takes depends on size: A lost hand or foot can be regrown in a month, an arm in four months and a leg in six months. The new limb is generally as good as the old one, but sometimes the regeneration is faulty, making the new limb smaller, weaker and/or less agile than the old one.\footnote{Make some kind of saving throw to avoid permanent attribute loss (and possible limping, if a leg grows back too short). If the same limb has been lost and regrown more than once, the chance of faulty regeneration increases each time (so it is generally not worth the risk to cut off the new limb and hope for a better one next time). A faulty limb cannot \quo{cured} by regular magical healing, since it not an injury but the natural state of the new limb. \quo{Fixing} a faulty limb is just as hard as it would be to strengthen the original limb.} At TL7 and above, limb transplants between individuals is generally easy. Non-fatal injuries to internal organs (even the brain) can also be healed. \Meccaran{} regeneration is slow, however, and not useful in combat. 

As a special feature, \meccara{} are left-handed by default. A minority (20\%) are right-handed. 

\subsubsection{Biology}
\Meccara{}, naturally evolved from large, freshwater-dwelling, predatory frogs, are carnivorous and semi-aquatic. They are able swimmers but cannot breathe water. They prefer subtropical to tropical climates and high humidity. Their natural habitats are jungles and swamps. They naturally live as hunters. Some of the more advanced \meccaran{} communities also raise livestock for food and tools, with farms to support the animals. (This is a science they have learned from other races. The naturally carnivorous \meccara{} would be unlikely to discover farming on their own.) 

%\Meccara{} prefer to bathe in fresh water regularly (once per day at least, more in hotter or drier areas). Drying out is painful, but not fatal. 

\Meccara{} must bathe in fresh water regularly. If a \meccaran{} cannot immerse (or at least splash) himself with fresh water every day, he will weaken and die. 

\Meccara{} have two genders and reproduce by external fertilization: The female lays her eggs in water and the male fertilizes them. A female can lay several dozen eggs at a time, most of which die. The eggs are spherical, 2-3 cm in diameter. After six months, the surviving eggs hatch into tadpoles. A hatchling tadpole is only few cm in length and has very little intelligence. Gradually they tadpoles develop legs and the ability to breathe air. After around five years, they are about 50 cm long, have legs as well as a tail, and are as intelligent as a \human{} child of two years or so. At this point they mostly land-living. After one or two more years, they lose the tail and the ability to breathe water. At the age of around 10, \meccara{} are sexually mature, about 80-90 cm tall and as intelligent as a \human{} teenager. They are full-grown adults around the age of 15. \Meccara{} live to be up to 50-60 years old, 70 in exceptional cases. The females outnumber the males, making up 60 percent of the population. 

%The \meccaran{} species is closely related to the Fitteran species. It is possible (albeit rare) for the two species to crossbreed. \Meccaran{}/Fitteran hybrids are larger than \meccara{} but more intelligent than Fittera. They are sterile, like mules, and usually considered ugly freaks by both species. There are, however, communities with \meccarans{} and Fitterans living together. In these communities, crossbreeds tend to be accepted as normal members of society. 

\Meccara{} are generally not monogamous and do not mate for life. Culturally, they have few sexual taboos and tend toward promiscuity\footnote{Promiscuity in the biological sense, meaning that any two individuals in a tribe may mate.}. Indeed, some tribes are known to indulge in collective sexual orgies, often under the influence of certain drugs brewed by their witch-doctors. Bi- and homosexuality is rather common and accepted in most cultures. 

\subsubsection{Psychology}
\Meccara{} tend to crave independence and freedom. As such, \meccaran{} tribes are rather loosely organized and laws are few and simple. \Meccara{} are curious and inquisitive by nature and have little fear of change. Some tribes have evolved into settled farmers and built towns or even cities, but their natural lifestyle is nomadic. 

\Meccaran{} adventureres are rather common. \Meccara{} tend to be very active and energetic, with laziness being seen as abnormal. 

All \meccara{} suffer from a mild dyslexia. The part of their brain that allows them to understand writing is not as well-developed as that of most races. They can learn to read and write, but they do it less well than other creatures of equal intelligence. Of course, dyslexia is not understood at TL3, so this is usually interpreted as stupidity, which leads to \meccarans{} sometimes being looked down on as barbarians. 

\subsubsection{Habitat}
\Meccara{} are most widespread in the tropical lands of Uzur and in the Far South. They are relatively common in the Imetrium and Durcac and uncommon in Belkade. They are rare in the North and East. Most \meccara{}-dominated communities lie in Uzur, but there are \meccaran{} tribes scattered across Belkade and other places as well. 



\subsection{\Naga{}}
\label{Naga}
\label{Nagae}
The \nagae{} are aquatic fish/reptile humanoids. They are long-lived and maintain their cities and kingdoms beneath the sea. They are the oldest humanoid species on Mith. 

All \nagae{} in the waters near \KnownWorld{} belong to the kingdom of Nag and speak the language known as Nag. 

\subsubsection{Name}
Singular \pronune{\naga{}}{NAA-ga}, plural \pronune{\nagae{}}{NAA-gej}.\footnote{The form \quo{\nagae{}} is not etymological justified, since \quo{\naga{}} is not Latin or Greek but from some Indian language, I believe. But this is the declination I use, because Latin grammar is cool.} 

The associated adjective is \emph{\naga{}}. 

\pronune{Nag}{NAAG} is a kingdom of \nagae{}. The language is also \emph{Nag} and the adjective is \pronune{Nagan}{NAA-gan}. 

\subsubsection{Physique}
A \naga{} is only vaguely humanoid, with an elongated snake- or eel-like body with a long tail and four limbs. They have what resembles a mix of fish and reptile characteristics. 

\Nagae{} are found in all kinds of colours and patterns, but in Nag, various shades of green are most common. 

A typical \naga{} is 1.5 to 2 meters long and weighs 40-80 kg, but some are much larger, growing as long as 5 meters and weighing up to a ton. Their arms and legs are short. They can walk on land, but they are not very agile. In the water, on the other hand, they are very fast and maneuverable, swimming with their legs and tail (rather like a seal). 

\Nagae{} can breathe both water and air, but they need to immerse themselves in water regularly. If they are kept out of the water for much more than 24 hours, they will dehydrate, weaken and die. 

\Nagae{} can regenerate lost limbs, but slowly. More slowly than \meccaran{} regeneration. An arm or leg takes about a year to regenerate. 

\Nagae{} fight with weapons. Most of their weapons are daggers, spears and javelins. \Naga{} technology is low, so many of their weapons are primitive, made of stone or bone. The \nagae{} cannot forge iron or bronze, but some of their weapons are made from exotic metals unknown on the land. Occasionally, a \naga{} may be encountered wielding a weapon made by land-dwellers. (Of course, iron weapons will rust quickly underwater, but \dragonsteel, \truesilver{} and certain enchanted weapons are immune to rust.) 

About \naga{} out of every twenty is a mage. \Naga{} magic is alien and very different from most land-dweller magic, but bears similary to \draconic{} magic. 

\subsubsection{Biology}
\Nagae{} live very long. A typical \naga{} has a lifespan of over 1000 years, and exceptional individuals may reach 2000 years or more. 

\Nagae{} near \KnownWorld{} speak the language of Nag. \Dragons{} can pronounce Nag flawlessly, and \scathae{} can learn it to some extent, but it is nearly unpronounceable to \humans{}. A few \nagae{} learn other tongues. A \naga{} encountered on the land is 20\% likely to know \Draconic{} Lowtongue and 5\% likely to know Rissitic, Imetric or some ancient \scathaese{} tongue. A \naga{} is also 5\% likely to understand a bit of Belkadian or Vaimon, but they can pronounce \human{} and Vaimon tongues only with great difficulty. (Those \nagae{} who do understand land-dweller tongues are usually the leaders and mages.) 

\subsubsection{Psychology}

\subsubsection{Habitat}
\subsubsection{Attributes}
\begin{description}
	\item[Horror effect:] Minor. 
\end{description}



\subsection{\Resphan}
\label{Resphan}
\label{Resphain}
\index{\resphan{} (plural \resphain)}
The \resphain{} are the progenitors of \humans{}, created by the \banes{} from \Nephil{} stock to be the heirs of the \banes{}. 

\subsubsection{Name}
Singular \emph{\resphan}, plural \emph{\resphain}. The adjective is \emph{\resphadj}. 

\subsubsection{Physique}
\Resphain{} look like \humans{}, but they are taller and their skin is a glossy onyx black. 

An adult male \resphan{} averages 210 cm but can grow as tall as 250 cm. The females are little taller than \humans{}, only 180 cm on average. 




\subsection{\Scatha{}}
\label{Scatha}
\label{Scathae}
The \scathae{} are reptillian humanoids. They are one of the dominant humanoid species on Mith, especially common in the Imetrium and in the Rissitic Dominion. 

\Scathae{} are very common and widespread. %In a sense, they are the 'humans' of Mith. 

\subsubsection{Name}
Singular \pronune{\scatha{}}{SKAA-dha}, plural \pronune{\scathae{}}{SKAA-dhei}. This grammar is Imetric. The associated adjective is \pronune{\scathaese{}}{skaa-dha-EEZ}. 

%The word for \quo{\scatha{}} in the Rissitic language is \emph{\rachyth{}}\footnote{[RA-chyth]}. In Rissitic, this form is both singular and plural. 

\subsubsection{Physique}
\Scathae{} are large, bipedal reptiles. An average adult \scatha{} is 2 meters long, 160 cm tall and weighs 80 kg. Males and females are the same size. They have a tail and do not stand fully erect. The tail accounts for about 40\% of the length and is not prehensile. They have five fingers on each hand and four toes on each foot. 

As they age, \scathae{} continue to grow in height and length, but not much in weight. Very old \scathae{} tend to be long and gaunt. 

\Scathae{} are strong for their size, but not very dextrous nor fast. They have thick scales, especially on the back, shoulders and overarms, which provide protection in combat. Their teeth and nails are small, like \humans{}', and not useful in combat. 

\Scathaese{} senses are like those of \humans, except the sense of touch, which is rather dull (due to their thick scales). %\Scathaese{} eyes are covered by a fully transparent but solid membrane that protects from sandstorms and the like. The outermost layer of this membrane is shed once every two months or so (similar to how a snake sheds its skin). They also have normal eyelids over this. As a special feature, 
\Scathae{} blink upwards with their lower eyelids, unlike \humans, who blink downwards with their upper eyelids. A \scatha{} has a long tongue and can lick his own eyes. 

\Scathae{} are diurnal and do not see well in the dark. 

The \scathae{} can be divided into a number of subraces, each with their characteristic scale colour and features. The Tassians (singular \intro{Tassian}) are coloured in shades of blue or violet (occasionally teal or turquoise) and are common in the Imetrium and southern Belkade. The \intro{\Mekrii} are red or reddish brown and most common in Durcac and the Orient. The \intro{Loiz} are green or greenish brown (occasionally black) and most common in the Serpentines. 

%\Scatha{} scales vary in color. In the Imetrium, blue and violet colors are common; red and brown colors dominate in Durcac and the East. Green colours are also found. 
Males are generally dark in color whereas females have brighter scales. Males are recognizable by two bony ridges on the top of the head (one over each ear). Females have a single ridge in the middle of the head, generally larger than that of the males. 

In combat, \scathae{} use weapons, such as swords or spears. They have good eyesight and make good archers or gunners. 

\subsubsection{Biology}
%\Scathae{} are warm-blooded reptilians. They are Archosaurs, related to dinosaurs and birds. The species is a result of natural evolution and has a few closely related (but unintelligent) species. They prefer temperate to subtropical climates with low humidity. They are omnivores with feeding habits similar to those of \humans{}. (Today, most \scathae{} are civilized and live off farming.) The \scathae{} evolved in the sandy deserts of the East, and the violent sandstorms are one of the reasons why they developed their thick, protective scales and eye-membranes. 

\Scathae{} are warm-blooded reptillians. They did not evolve naturally, but were the creations of the early \dragonlords{} who mixed \dragon{} and \naga{} genes with those of land-dwelling reptiles (including \nycans{}). They are herbivores. 

\Scathae{} have two genders and reproduce by internal fertilization. The female lays a small cluster of eggs (typically one or two). Males and females are born in equal numbers. 

\Scathae{} have a life expenctancy of around 75 years at TL3. Ancient \scathae{} of up to 100 years are exceptional but not unheard of. They reach sexual maturity in their early teens. They are usually considered adult around the age of 20 (varying depending on culture). 

Among \scathae{}, males and females are very similar, physically as well as psychologically, and the genders usually equal in society. Because of this, homo- and bisexuality is rather common. \Scathae{} are usually monogamous, but they may or may not mate for life.  

\subsubsection{Psychology}
The \scathae{} are naturally evolved from animals that were both predators and prey. Where many other creatures of the same areas developed great size and horns or spines to defend themselves, the proto-\scathae{} instead developed pack tactics, intelligence and hands. 

As the descendants of pack-living prey animals, \scathae{} have a strong sense of commitment to their fellows. This makes the typical \scatha{} loyal and patriotic, but also xenophobic. \Scathae{} are fiercely devoted to their community (family, clan, nation), slow to trust outsiders and zealous about punishing betrayers and criminals. 

\Scathae{} value security and a well-ordered life. They prefer to know their place in the pack/community and function well in a class system. \Scathae{} make good craftsmen and soldiers. They have little craving for excitement and rarely go out to \quo{adventure}. \Scathaese{} adventurers are typically on some kind of quest for their community, rather than simply seeking fame and wealth. 

\subsubsection{Habitat}
\Scathae{} are among the most widespread humanoids on Mith. They are common in all lands except the Northern Kingdoms, but especially dominant in the Imetrium, Durcac and Irokas. 

%They are especially common in Col, Geica and the Rissitic dominion, but also in the Middle Lands. They are uncommon in the North and in Uzur. 



\begin{comment}
\subsection{Vaimon}
%Vaimons are bird humanoids. They are a \quo{fallen} race that once ruled great empires spanning much of the continent. Now, only small communities are actually ruled by the Vaimons. Some Vaimons are reclusive and keep to themselves in their own Vaimon communities. Others live freely among other races. 

%Vaimons are very long-lived and maintain a legacy of \quo{ancient wisdom}. They can be thought of as the 'Elves' of Mith. 

The Vaimons are a race of super-\humans{}. Physically they resemble \humans{}, but they have superior magical talent and longer lifespans. 

Traditionally, the Vaimons are divided into several peoples, called clans. In the days of the Vaimon Empire (see section \ref{Vaimon Empire}), there were seven clans, but three of them have been destroyed so that only four remain today. Each clan traditionally had its own homeland and kingdom, of which three now remain: \Redce{}, Geica and Zoitan (homeland of the Quaerin). Clan Yrzhell have lost their original homeland and are now a scatteded, wandering people. The clans have some culture, beliefs and traditions in common, but each clan also has its own culture. 

Some Vaimons are Clanless, swearing allegiance to no clan. These are typically people (or descendant of people) who reject the philosophy of their ancestral clan, or wish to hide their heritage to escape persecution (the latter applies especially to those of Clan Geican). Clanless Vaimons usually live (alone or in small groups) among other peoples rather than with their fellow Vaimons. 

Vaimons are closely related to \humans{} and share many characteristics (physical and psychological). The term \introe{Hominids} is used to refer to \humans{} and Vaimons collectively. 

\subsubsection{Name}
Singular \pronune{Vaimon}{VAJ-mon}, plural \emph{Vaimons}. This is English grammar. The associated adjective is \emph{Vaimon}. 

%\pronune{Hominid}{HO-mi-nid} becomes \emph{Hominids} in plural, and the adjective is \emph{Hominid}. 

The names of the clans: \pronunei{Redcor}{rh�d-KO(RH)} is both singular, plural and adjective. The same is true for Clan \pronunei{Quaerin}{kwe-REEN}. \pronunei{Geican}{GEJ-k�n} becomes \emph{Geicans} in plural, and the adjective is \emph{Geican}. 

\subsubsection{Physique}
%Vaimons are humanoid birds. They look similar to storks or herons, but with oversized heads\footnote{Since their brains are as big as those of humans.}, and walk on long, thin legs. Their wings have become arms, and they can not fly. Their body (except the legs) is covered with feathers. Vaimons always wear clothes. 

%Vaimons are tall, but not very massive, because the long, thin legs make up much of their height. A full-grown male Vaimon is 180-190 cm tall and weighs around 60 kg. The female is somewhat smaller, 170-180 cm tall and weighing 55 kg. Vaimons have a straight, stork-like beak around 30 cm long. The legs are black and the beak is orange. The skin is light gray and the feathers are patterned in white and black. 

%Vaimons are rather fragile and not very strong compared to other humanoids. A Vaimon civilian is unlikely to defeat a \scathaese{} or \human{} civilian in combat. Vaimon warriors rely on longevity: Since they live longer and age slower, they are able to gather more experience. Therefore, a Vaimon warrior encountered will often be a very skilled fighter, using speed, reflexes and cunning to make up for his lack of physical strength. 

%Vaimon warriors prefer light, swift weapons. Their typical weapon of choice is the sabre (a long, light, single-edged sword, somewhat curved; similar to a rapier but heavier). Staves are also sometimes used. Bows are favoured, but guns are generally scorned. 

%Vaimon senses are like those of humans. 

%Vaimons are the only creatures known to be able to cast Vaimon magic (see \ref{Vaimon magic}). Approximately one out of three Vaimons have some magic talent (and of those, some are strong while some are weak). 

Vaimons look like \humans{}. Each Clan has its own distinctive facial and bodily features, but often it is possible to a Vaimon to masquerade as a \human{}. 

The Redcor are generally tall (175-185 cm) and slender and have blonde or red hair. Especially the Redcor women are tall, often as tall as the men. The Geicans are shorter (160-170 cm) and plumper, with brown or black hair. 

Physically, Vaimons are no stronger or more capable than \humans{}, but a Vaimon warrior will often be much more skilled than a \human{} warrior, simply because Vaimons live longer and thus have more years to practice and gain experience. 

\subsubsection{Biology}
%Vaimons may have evolved naturally on Mith, or they may have been created or have originated from a different planet. No closely related species are known to exist. They are carnivorous and prefer small lizards and amphibians (but today, these are eaten only as a delicacy, as it is difficult to breed enough of them to feed a Vaimon population). 

%Vaimons reproduce sexually. A female is fertile only in the mating season, which is one month each year. After some months of pregnancy, she will lay one or two eggs, which hatch after some more months. 

%Vaimons are long-lived, with life spans of around 300 years. They are sexually mature after 25 years and full-grown after 50 years. 

%Vaimons are monogamous and mate for life. Homosexuality is usually not acceptable. 

Vaimons are descended from \humans{}, who slowly mutated due to channeling Iquin and Nieur. See section \ref{Origin of Vaimons} for their history. 

They reproduce as \humans{} do, and a woman is pregnant for nine months. They are less fertile, and a mating will only occasionally cause impregnation. 

They mature as fast as \humans{}, but after reaching adulthood they age more slowly, about half as fast as \humans{}. They typically live about 150 years, but may reach 200. (In practice, the average Vaimon lives more than twice as long as the average \human{}, because Vaimons often belong to the wealthy upper class.) 

The Vaimons are few in number. In all of \KnownWorld{} there is approximately one Vaimon per 100 \humans{}. (The Vaimon/\human{} ratio is of course higher in certain kingdoms, such as \Redce{} and Geica.) 

Vaimons belong to the same species as \humans{} and can crossbreed with them. Such a coupling most commonly produces a \human{} child and only rarely (10\% chance) a Vaimon. 

Physically, Vaimons are not easily distinguishable from \humans{}. Metaphysically, however, they are in touch with the forces of Iquin and Nieur in a way that \humans{} are not. This arcane attunement can be detected by an Iquin-Nieur mage after careful study (but not at a glimpse), and it is what defines the Vaimons and separates them from \humans{}. 

\subsubsection{Psychology}
%Vaimons are social creatures, committed to their community. 

Psychologically, Vaimons are much like \humans{}. 

Vaimons have a strong creative urge and make great craftsmen and artists. Vaimon craftsmanship tends to be not only highly functional but also beautiful and ornate. Especially notable in this regard are architecture and sculpture. Vaimons often build great towers, palaces, statues and the like. An example of a marvel of Vaimon architecture is Martinum Bridge (\ref{Martinum Bridge}). 

Vaimons have an urge to affect and dominate other creatures. They are charismatic and often make capable leaders, but they can also often be tyrannical and/or manipulative. 

\subsubsection{Habitat}
Vaimons are uncommon to rare in all lands of the Old World, save Uzur and Irokas, where they are very rare. 
\end{comment}



%\section{Rejected creatures}



\begin{comment}
\section{Species and Races}

\subsection{Generally on species}
Needless to say, species is a very important aspect of a character. Mith is inhabited by multiple intelligent species, many of which should perfectly well as player characters. 
\end{comment}

\begin{comment}
\section{No humans}
\label{Humans}
\index{Humans}

There are no humans on Mith. 

Most fictional universes have humans in them. But my problem with this is that the systems invariably become human-centered. Humans are always the dominant race, and many other creatures are based on humans. This problem appears in all kinds of fantasy as well as RL mythology. Examples: An elf is a human with pointy ears. A dwarf, gnome, hobbit or halfling is a small human. A centaur is a human torso on a horse body. A medusa is a human with snakes for hair. A harpy is a birdlike human. A manticore is a lion with a human face. And so on. 

The problem is that humans have a special status above other creatures. They are central to the world in a way that orcs and lizardmen can never be. Damn it, where is the lizardman/crocodile centaur? Where is the serpent monster with an orc head? No, only human hybrids are allowed. Even the undead are always humans, with few exceptions. (In D\&D (2nd edition), there is \emph{one} undead creature called an `Undead Dwarf'. Go figure...) 

When designing Mith, I realized that the temptation to do this is very real. As soon as humans exist, it is just so easy to make other creatures that look like humans, or monsters that have human parts. I did not want to fall in this trap, so I decided that the best course would be to outlaw humans alltogether. 

So no humans. 

There are also no demihumans - no elves, dwarves or the like. And while I was at it, I figure it would work best if I went all the way and used only my own humanoid races. This means there are also no orcs, no trolls, and so on. Instead, Mith is populated by Scathae, Meccara and other creatures of my own creation. 

To say that I `went all the way' is a bit of an exaggeration. There are still classical mythological/fantasy monsters on Mith. There are Dragons, and there are Manticores and Medusae. But those monsters that have human parts in the original context have been altered and these aspects taken out. 

%\subsection{But I use them anyway...}
Despite the fact that humans do not exist on Mith, I may still refer to them. I do so only for comparative purposes or as hypothetical examples. I will say things such as `Scathae have eyesight as good as that of humans', `Meccaran have shorter arms than humans', etc. 
\end{comment}



\begin{comment}
\subsectionn{Fittera}
The Fittera are a race of amphibian humanoids, closely related to the Meccara. Fittera are adapted to cooler, drier climates, while the Meccara prefer warmer, more humid climates. Fittera are larger and stronger than Meccara, but less intelligent and less civilized. 

\subsubsection{Name}
Singular \emph{Fitteran}\footnote{[FI-te-ran] (Imetric word). Note that the stress is different from that in `Meccara'}, plural \emph{Fittera}\footnote{[FI-te-ra]} (Imetric declination) or \emph{Fitterans} (English declination). %The distintion here is the same as with \emph{Meccara}/\emph{Meccarans} (see under Meccara, p. \pageref{Meccara}). This grammar is Imetric. 
The associated adjective is \emph{Fitteran}. 

\subsubsection{Physique}
A Fitteran looks like a larger Meccaran. An adult female stands 160-170 cm tall and weighs around 80 kg. A male is 150-160 cm tall and weighs 70 kg. They are colored in shades of brown. 

Fitteran senses are like those of Meccara. 

They have the same regenerative abilities as Meccarans do. 

\subsubsection{Biology}
Fittera belong to the same genus as the Meccara but a different species. Their biological properties are mostly the same as those of the Meccara, but Fittera are less dependent on water and do not need to bathe nearly as often. Their lifespans are the same as the Meccara's. 

\subsubsection{Psychology}
Fittera are less intelligent than Meccara and most other intelligent creatures. They rarely understand science or intellectual art. In multiracial societies, Fittera usually live as peasants, labourers or soldiers. Fitteran mages are very rare. There are communities where Meccarans and Fitterans live together. In these tribes, the leaders will usually be Meccaran.\footnote{This is for Darwinian reasons: A tribe led by a stupid Fitteran chieftain is more likely to be wiped out by enemy tribes with more competent leaders.}

The Fitterans' communities are primitive barbarians, typically living as TL0-TL1 hunters. Tribes may be led by chieftains or shamans. Some of these tribes are peaceful `noble savages', living in harmony with nature. Other tribes are aggressive and may conduct raids on other communities.\footnote{Warlike Fitteran tribes may be used as a kind of `Orcs', if needed.} 

\subsubsection{Habitat}
Fittera are most common in the Middle Lands and not uncommon in the Rissitic Dominion, Geica and Uzur. 
\end{comment}



\begin{comment}
\subsection{Kinsari}
\label{Kinsari}
\index{Kinsari}
The Kinsari are simian humanoids. They originate from the isle of Erul, where they are common and have their own civilization. They are widespread across many other lands, especially in the Imetrium, but are uncommon all other places than Erul. They are especially noted as practitioners of psionics. 

\subsubsection{Name}
\emph{Kinsari} is both singular and plural. This grammar is the Kinsari language (as spoken in Erul). The associated adjective is \emph{Kinsari}. 

\subsubsection{Physique}
Kinsari are large monkeys. They stand around 160 cm tall and weigh around 55 kg, with the female being slightly smaller than the male. They stand erect on two feet and are covered in short fur. Their color is dark brown to black, but the belly and head are white. 

Their feet are very dextrous and may double as hands. They have a long, prehensile tail that can be used as a primitive `arm'. Kinsari are able climbers and are at home in the treetops. 

Kinsari senses are like those of humans. 

\subsubsection{Biology}
\subsubsection{Psychology}
\subsubsection{Habitat}
\end{comment}



\begin{comment}
\subsection{Lupin}
\label{Lupin}
\index{Lupin}
\subsubsection{Name}
\subsubsection{Physique}
\subsubsection{Biology}
\subsubsection{Psychology}
\end{comment}



\begin{comment}
\subsection{Gargoyles}
\label{Gargoyle}
\index{Gargoyle}
The Gargoyles are monstrous humanoids

\subsubsection{Name}
\subsubsection{Physique}
\subsubsection{Biology}
\subsubsection{Psychology}
\subsubsection{Habitat}
\subsubsection{Attributes}
\begin{description}
	\item[Horror effect:] Minor. 
\end{description}
\end{comment}



\begin{comment}
\subsection{Tchacolda}
\label{Tchacolda}
\index{Tchacolda}
\subsubsection{Name}
\subsubsection{Physique}
\subsubsection{Biology}
\subsubsection{Psychology}
\subsubsection{Habitat}
\end{comment}










\newpage
\section{\Dragons{} and \dragon-kin}
\quo{\Dragon-kin} is a collective term for creatures related to \dragons{}. All \dragon-kin are descended from \nagae{}, as described in section \ref{Origin of Dragons}. Strictly, the \nagae{} and \scathae{} also belong in this section, but I decided to place them under \quo{Humanoids} instead. 

%Dragon-Kin are a whole category of creatures that deserve their own section. There exist many different races of \dragons{} and \dragon-kin on Mith. 

%\Dragons{} are large, intelligent, reptillian creatures possessed of great physical and supernatural power. All \dragons{} are intelligent, many can fly, and some can breathe fire or other things. 

\subsection{\Dragon{}}
\label{\Dragon}
\Dragons{} are mighty, flying reptiles. They are highly intelligent and skilled in magic. 

There are two subraces of \dragons{}: Fire \dragons{}, who can breathe fire, and Ice \dragons{}, who can breathe a cone of cold and ice. 

\subsubsection{Name}
%\emph{\Dragon{}} is a normal English word. 
As in English. The associated adjective is \emph{\draconic{}}. 

\subsubsection{Physique}
A \dragon{} is a large quadrupedal reptile with a long neck and tail. A \dragon{} walks on straight legs like a mammal, not with the legs spread outward like a lizard or crocodile. The tail is prehensile and makes up about 40\% of its length. The body and neck each make up another 30\% of the length. A ridge of small spikes runs down its spine, from the head to the end of the tail. 

The forelegs and hindlegs are the same length. The hindclaws are larger and more bestial, whereas the foreclaws are more fine and dextrous, almost as fine as \human{} or \scatha{} hands. 

\Dragons{} have long, sharp teeth for biting and cutting, but also some flat teeth (at the back of the mouth) for chewing. They usually have a number of horns on the head, usually pointing backward. The shape and number of these horns varies a lot between individuals. 

A \dragon{} has a pair of great bat-like wings. The \dragonz{} wingspan is about the same as its length. There is innate magic in the wings allowing them to fly. (The wings are too small to carry an adult \dragon{} without magical aid.) 

\Dragons{} come in many different colours. Among the Fire \dragons{}, red, green and black colours are most common. Among Ice \dragons{}, white, gray and blue colours are more common. Most \dragons{} have scales in one dominant colour and patterns (stripes, lines or more complex patters) in one or two other (not too dissimilar) colour. For example, Princess Thiencaste is bright green with yellow and orange stripes. 

An adult \dragon{} is typically 6-9 meters long and weighs 500 to 1500 kg, with older individuals reaching up to 12 meters and 3 tons. \Dragons{} of particularly strong bloodlines may grow to 10-13 meters and 2-4 tons in adulthood and reach 16 meters and 7 tons in old age, or even 18 meters and 10 tons for exceptional individuals. Hatchling \dragons{} are about 50 cm long, and young \dragons{} may be anything from 1 to 6 meters. Males and females are of equal size. 

(Sample sizes: King Noreoccyrias was 16.5 meters long and weighed 9 tons at his death, 1000 years old. King Tentocoth is 500 years old, 12 meters long and weighs 3 tons. %He is longer and lighter than his father, and likely to grow to 18 meters in length if he lives 1000 years. 
Lord Ishnaruchyfir is 15 meters long and weighs 6 tons at 900 years. Princess Thiencaste is 3.3 meters long and weighs 90 kg at 150 years. Princess Criocas is 4 meters long and weighs 100 kg at 180 years.) 

%A typical adult (6-9 meters long) weighing 1.5 to 3 tons. A 12 meter \dragon{} might weigh 5 tons and a 15 meter \dragon{} 8 tons. 

A Fire \dragon{} can breathe a cone of fire. An innate magical ability sparks the fire, but the flames themselves are natural and work like normal fire. An Ice \dragon{} breathes a cone of extreme cold, filled with small shards of ice. (The ice shards are frozen drops of spit and slime from the \dragonz{} mouth and throat.) Every \dragon{} has one breath weapon, not both. The range of the breath weapon is about twice the \dragonz{} length (this can be increased with skill and practice). 

\Dragons{} have some resistance to both heat and cold. Fire \dragons{} are somewhat resistant to cold and highly resistant (but not immune) to fire and heat, and Ice \dragons{} vice versa. 

\Dragons{} have good regenerative abilities. They heal wounds quickly even in combat and can heal wounds most creatures can't. They can't regrow limbs naturally, but with the aid of spells, a \dragon{} can reattach severed limbs or even regrow new ones. 

\subsubsection{Biology}
\Dragons{} are descended from \naga{} lords who transformed themselves using bio-technology and magic, as described in section \ref{Origin of Dragons}. 

\Dragons{} reproduce by internal fertilization and lay eggs. They are very infertile. A mating only rarely causes impregnation. A female lays one to three eggs at a time (rarely four or five), but even under good conditions, only half of these eggs survive to hatch, and many of the young die early on. An average female can lay 10-15 eggs in her lifetime if she is sexually active thoughout her fertile life (of these, 2-4 young can be expected to survive). The female carries the eggs for slightly less than a year, and after she lays them they take about 1� year to hatch. If a female is impregnated, she will know it a few (3-5) days after the mating (but even at this stage, miscarriages do occur).  

The \dragons{} of Irokas have few sexual taboos and mate with whomever they choose. Having many descendants is a symbol of high status, so most \dragons{} are highly sexually active. A strong and attractive male may have dozens of children. (King Noreocchyrias had over 30.) 

A \dragon{} is sexually mature at an age of about 50 years. They typically live 600-700 years, but \dragons{} of strong bloodlines may reach 1000 years. A female stops being fertile at an age of  400-500. Males become gradually less fertile as they age, but have no set limit. 

Using magic, \dragons{} may transform themselves into other forms. (Many people believe that this is an innate ability, but it is in fact done with spells.) Since they are related to \scathae{}, a \dragon{} in \scatha{} form can actually interbreed with \scathae{}, producing \rachyth{} (see section \ref{\Rachyth{}}). It is possible that they can also interbreed with \nagae{} and other \dragon-kin, but this is not known. 

Fire \dragons{} and Ice \dragons{} are one species and can interbreed. The Fire genes are dominant and the Ice genes are recessive. This means that the offspring of a Fire and an Ice \dragon{} will most often be a Fire \dragon{}, but occasionally, a child of two Fire \dragons{} will be born an Ice \dragon{}. Overall, the Fire \dragons{} make up $\tquarts$ of the population and the Ice \dragons{} $\quarts$. 

There are about 3500 \dragons{} in \KnownWorld{}, of which the 3000 live in Irokas (see chapter \ref{Irokas}). 

\subsubsection{Psychology}
\Dragons{} are creatures of strong passions. They have the same emotions as \humans{} (love, hate, lust, pride...), but on a larger scale. They are very individualistic creatures and, as a rule, consider their own needs and desires first and their fellows second.	\Dragons{} are often ruthless and arrogant, seeing humanoids as inferior, unworthy savages (though, of course, some \dragons{} are gentle and kind). 

\Dragons{} are proud and aggressive. They instinctively seek to dominate other creatures and do not welcome being given orders. This is the main reason why the \draconic{} kingdoms of Nom and Irokas have always been plagued by bloody wars (even more so than humanoid kingdoms): \Dragons{} are not genetically disposed to follow a King, so they tend to splinter and rebel. 

\subsubsection{Habitat}



\subsection{Basilisk}
Large reptillian monster. Its gaze is magical and drains lifeforce. (This energy is not transferred to the Basilisk.) 



\subsection{\Leviathan}
Enormous sea-dwelling monster. A reshaped \naga{}. The \Leviathans{} are the kings of the \nagae{}. 



\subsection{\Linnorm}
Long, snakelike \dragon-kin. Dwells in water. \Linnorms{} are reshaped \nagae{} and serve the \leviathans{} as fighters. They are less intelligent than \nagae{} and never learn magic. 



\subsection{\Rachyth{}}
\label{\Rachyth}
\index{Raccaid}
\Rachyth{} are half-\dragons{}, \dragon{}/\scatha{}-hybrids, the result of a mating between a \scatha{} and a metamorphosed \dragon{} in \scatha{} form. 

Each \rachyth{} has a fraction of \dragon{} blood. \Rachyth{} can be described by their generation: A 1st generation \rachyth{} has a \dragon{} and a \scatha{} parent, so he is half \dragon{}. A 2nd generation \rachyth{} is one quarter \dragon{}, and so on. The \dragon{} genes dillute quickly, so after four generations ($\fracs{1}{16}$ \dragon{} blood), the \rachyth{} is a normal \scatha{} in all respects except perhaps appearance. 

\subsubsection{Name}
\pronune{\Rachyth{}}{RAA-chajth} is both singular and plural. The word is \Draconic{} Kingstongue (see section \ref{Kingstongue}). The associated adjective is also \emph{\rachyth{}}. 

\subsubsection{Physique}
\Rachyth{} look like \scathae{}, but have some distinctively \draconic{} traits. A \rachyth{} looks more or less \draconic{}, depending on generation. \Draconic{} traits include a longer, more narrow snout, a longer neck, a prehensile tail, spikes along the spine. \Rachyth{} also often have long, sharp teeth and claws. (Despite this, they are still herbivores like normal \scathae{}.) 

\Rachyth{} are taller and longer than \scathae{}. A first generation \rachyth{} may reach three meters or more in length and more than 210 cm in height. This increase in length is chiefly due to a longer tail and neck and a more elongated body; in mass, \rachyth{} are only slightly heavier than \scathae{}. 

\Rachyth{} have better regenerative abilities than \scathae{}, but they cannot regrow entire limbs. 

They have no wings and cannot fly, but some \rachyth{} with at least 50\% \dragon{} blood can breathe fire or ice. 

\subsubsection{Biology}
\Rachyth{} are the result of a crossbreeding between a \scatha{} and a \dragon{} transformed into \scatha{} form. In the majority of such crossbreedings, the \dragon{} is the father. The reason is that the mother must remain in the same form throughout the gestation period. If the mother changes form, it will kill the child and possibly the mother as well (if the pregnancy is far along). Most \dragon{} women are unwilling to spend several months in \scatha{} form to lay an egg. Most of the time, if a female \dragon{} has sex while in \scatha{} form, she will either use some kind of contraception or use magic to destroy the foetus if she becomes impregnated. 

\Dragon{}/\scatha{} unions are very infertile and will only rarely produce an egg (about 2\% chance per mating in average). A \scathaese{} mother impregnated by a \dragon{} will never lay more than a single egg, but a \draconic{} mother in \scatha{} form may lay up to two or three eggs (albeit rarely). Many of these eggs (2/3) die, but the \rachyth{} that do hatch are healthy and strong. 

\Rachyth{} can breed with each other, with \scathae{} and with \dragons{} in \scatha{}/\rachyth{} form. \Rachyth{} are also infertile, like \dragons{}, and usually only manage to produce a few children in a lifetime. 

The generation system is not perfect, since a \rachyth{} needs not have a single \draconic{} ancestor and \scatha{} for the rest. \Rachyth{} may mate with \dragons{} and with each other, so it is possible to have a \rachyth{} that is $\fracs{3}{4}$ \dragon{} or $\fracs{5}{16}$ \dragon{} or whatnot. As a general rule, if a \rachyth{} has less than $\fracs{1}{10}$ \dragon{} blood, he is, for all practical purposes, a normal \scatha{}. There are also the rare zero-generation \rachyth{}, the offspring of two \dragons{} mating in \scatha{}/\rachyth{} form. 

\Rachyth{} live longer than \scathae{}. A zero-generation or first generation \rachyth{} lives up to 400-500 years, a second generation perhaps 300 years, a third generation 150-200 years and a fourth generation 75 years (like a normal \scatha{}). 

%Regardless of whom they mate with, \rachyth{} (male and female) are much less fertile than \scathae{}. Only rarely will a mating produce an egg, and many of the eggs die. 

\subsubsection{Psychology}
\Rachyth{} psychology is a mix of \draconic{} and \scathaese{} behaviour (depending on the amount of \dragon{} blood an individual \rachyth{} has). 



\subsectionn{\Wyvern{}}
Flying \dragon-like monster. 





\newpage
\section{Monsters}
\subsectionn{Manticore}



\subsectionn{\Naiad{}}
\label{Naiad}
\Naiads{} are water-dwelling \quo{spirits}. They are mostly benevolent. \Naiads{} are one of several types of \quo{Nymphs}. They are usually considered female, although this is biologically incorrect. \Naiads{} have some innate telepathic and telekinetic powers. 

\subsubsection{Name}
Singular \pronune{\naiad{}}{NAJ-�d}, plural \emph{\naiads{}}. The word is originally Greek but the declination is English. The associated adjective is \emph{\naiad{}}, I guess. (Naiadese? Naiadean? Naiadic?) 

\subsubsection{Physique}
A \naiadz{} natural element is water. In the water, she appears as a barely visible patch of vaguely shimmering water. Closer inspection will reveal a network of long, spindly tendrils. These tendrils are the \naiadz{} actual body, a brain of some sort. In the water, a \naiad{} will typically spread out over an area 2-3 meters across. She is not massive and other creatures can pass through her body unharmed, perhaps without even noticing her. 

A \naiad{} must be surrounded by water to survive. On dry land, she must bring along water of her own using her telekinesis. To preserve her strength, she usually brings along only a small body of water. When travelling over land, the easiest and most confortable form to assume is that of an amorphous amoeba that slithers along. With practice and effort, a \naiad{} is able to shape her watery body into a humanoid form (or sometimes a humanoid torso with an amoeboid lower body, which is easier to control than legs). \Naiads{} are not very strong, so they will rarely drag along much more than 10-20 liters of water. So if she assumes humanoid form, it will be that of a very small humanoid. 

Cold will numb and slow a \naiad{}. If frozen, she will go into torpor but not die. If she is frozen and then shattered, she will die. Fire damage will quickly kill a \naiad{} if it can boil or evaporate the water surrounding her. Electricity does no damage. Acid does damage if it directly hits her tendrils, but this is rarely fatal unless there is a \emph{lot} of acid. Death magic has its normal effect. 

\subsubsection{Biology}
\Naiads{} are actually asexual and reproduce by budding. 

\subsubsection{Psychology}
\subsubsection{Habitat}



\subsectionn{Nightwing}
A giant bat, semi-intelligent. 



\subsectionn{Pegasus}
Winged horses. Bred by the Vaimons. Pegasi are creatures of Iquin and can channel Iquin to some extent. They are opposed to Nieur. They react badly to a person \quo{possessed} by Nieur: Mages who regularly channel Nieur and people dominated by Nieur-related emotions. Only a \quo{pure}, Iquin-aligned person will be allowed to ride a Pegasus. 



\subsectionn{Peryton}
Dark Pegasi. Creatures of Nieur. 



\subsectionn{Salamander}
Two species: Fire Salamander and Frost Salamander. 



\subsectionn{Umbra}
A monster of darkness, looks like a flying manta ray. Feeds on lifeforce. 





\newpage
\section{Animals, wild and tame}
\subsectionn{\Cortio{}}
\label{Cortio}
A \cortio{} is a great predatory dinosaur, similar to a Tyrannosaurus. \Cortios{} are wild beasts, but they can be tamed, albeit with difficulty. 

\subsubsection{Name}
Singular \pronune{\cortio{}}{KOR-tee-ow}, plural \emph{\cortios{}}. This declination is English. The adjective is \emph{\cortio{}}. 

\subsubsection{Physique}
\Cortios{} are huge reptiles, similar to dinosaurs like Tyrannosaurus or Allosaurus. 

\subsubsection{Biology}
\subsubsection{Psychology}
\Cortios{} may be tamed, but they are difficult to control. They are savage, aggressive creatures and must be brutally dominated if they are to be kept loyal. 

\subsubsection{Habitat}
\Cortios{} are most common in Durcac, southern Irokas and the Orient, but some exist in southern Belkade and the Imetrium. 



\subsectionn{Daggerdrake}
Large \dragon-like monster with great spikes. 



\subsectionn{Gryphon}
Gryphons can be tamed. The Imetrium has an order of paladins called Gryphonlords who ride Gryphons. 



\subsectionn{Lion}
These are like RL lions. 
%Two species of lions exists: Brown lion (or \quo{golden lion}) in the Rissitic Dominion and such places (similar to RL lions) and black lion further North (with dark gray or black hide). 



\subsectionn{\Nycan{}}
\label{Nycan}
\Nycans{} are large predatory reptiles resembling real-world dinosaurs like Deinonychus or Velociraptor. 

This creature is based partially on the actual dinosaurs, partially on the Jurassic Park movies and partially pure fantasy. 

\subsubsection{Name}
Singular \pronune{\nycan{}}{NAJ-kan}, plural \emph{\nycans{}}. The word is Imetric but the declination is English. The associated adjective is \emph{\nycan{}}. %(The word is derived from \quo{nychus}, as in \quo{Deinonychus}. This is originally a Greek word for \quo{claw}.)

\subsubsection{Physique}
A \nycan{} is a slender, bipedal dinosaur with a long, stiff tail. Its forearms are long and strong and have sharp claws. Each hind leg has not one but two oversized claws. \Nycans{} are covered in feathers and coloured in shades of brown, red and yellow.\footnote{There is evidence to show that some RL dinosaurs of this type were in fact feathered.} 

In combat, \nycans{} slash with the great claws on their hind legs if possible, and also attack with their foreclaws and bite. 

\Nycans{} are extremely effective predators. They are fast and agile runners and leapers and can sprint like cheetahs, reaching tremendous speeds over short distances. They are also highly intelligent, hunting in well-organized packs and displaying great cunning in their hunting behaviour. 

\Nycans{} have vision like that of \humans{} and hearing and smell like that of dogs. They are diurnal and do not see well in darkness. 

In the wild, \nycans{} are 2 to 2.5 meters in length and weigh around 55-70 kg. In captivity, beasts up to 4 meters long and weighing over 200 kg have been bred. The female is slightly larger than the male. 

\subsubsection{Biology}
\Nycans{} are warm-blooded dinosaurs and a result of natural evolution. Only one species of \nycan{} is known to exist - they are so effective that they have driven all closely related species into extinction. 

A \nycan{} female lays a small cluster of one to five eggs. Equal numbers of males and females are born. \Nycans{} live in packs, each pack led by an alpha female. Males and females both participate in hunting and raising the young. They are not monogamous. 

A \nycan{} is sexually mature after 7-9 years. Their average lifespan is about 20 years in the wild and 30-35 years tame, although they can reach 50-60 years. 
%They can live up to 50 years tame, but rarely more than half that in the wild. 

\subsubsection{Psychology}
\Nycans{} are used to living in a pack and can be tamed and taught to recognize a non-\nycan{} as pack leader. Once tamed, \nycans{} are very loyal, and there is no risk of them \quo{going wild}. 

\Nycans{} are as intelligent as apes and can learn a wide variety of skills. They can learn to understand language well enough to understand simple sentences and maintain a rather large vocabulary of both simple and more abstract concepts. A skilled \nycaneer{} (see section \ref{Nycaneer}) can give his \nycans{} quite complex instructions. 

Among themselves, \nycans{} communicate using a primitive telepathy. This telepathy, combined with smell, allows them to detect basic emotions in other creatures (such as fear or anger). Secondarily, they also commicate using sound. \Nycan{} voices are hoarse, screeching and high-pitched. They can scream (to warn fellows of danger), yelp (if in pain), hiss (to warn or intimidate foes) and purr (when friendly). 

%Among themselves, \nycans{} communicate using sound and body language, and also possess a limited empathy which allows for primitive communication and lets them detect basic emotions in other creatures (such as fear or anger). \Nycan{} voices are hoarse, screeching and high-pitched. They can scream, yelp, hiss and purr. 

\Nycans{} fear and hate the unnatural. This includes such things as undead and demons, as well as \quo{black} magic, but not all magic. As a general rule, anything that has a Horror Effect will qualify. When faced with \quo{unnatural} things, \nycans{} will become enraged and want to attack it, or flee, if the menace is perceived as too powerful or horrible. A \bane{}, for instance, will cause \nycans{} to flee, but they might stand and hiss at it from (what they perceive to be) a safe distance. Something with only a Slight Horror Effect, like a dark mage (such as a Nieur channeller) or a Rissitic Dominus, will cause the \nycans{} to hiss threateningly, but not immediately attack. An undead creature (such as a Rissitic Immortal Priest) is likely to be savagely attacked. If the \nycans{} are tame and well-trained, a \nycaneer{} may be able to restrain them\footnote{This would require a skill roll of some kind.}, but the instinctive hatred is strong. 

%Note that what is \quo{unnatural} is not entirely instinctive. A Rissitic Dominus, for instance, has a Slight Horror Effect, but a \nycan{} raised by Rissitic trainers and accustomed to Rissitic culture and magic 

\subsubsection{Habitat}
\Nycans{} prefer warm climates and open grasslands. They are predators and prey on all sorts of animals. 

A millennium ago, \nycans{} were widespread over much of the Old Continent. Being very dangerous creatures who may attack livestock and people, the \nycans{} were hated and feared by most intelligent creatures, and as a result, they were hunted into extinction in most lands by humanoids armed with weapons and magic. 

The only place in the West where \nycans{} survived was the land that is now the Imetrium, and centuries later, the Imetrians discovered how to domesticate them. Tame \nycans{} are now an invaluable asset in Imetric society, taking the place of dogs in RL. \Nycans{} are stronger and more intelligent than dogs, however. In the military, \nycans{} are used as trackers, battlefield shocktroopers, and even assassins. 

\subsubsection{\Nycaneers{}}
\label{Nycaneer}
\label{Nycaneers}
\index{\nycaneer}
%A \introep{\nycaneer{}}{naj-ka-NEER}\index{\nycaneer{}} (plural \emph{\nycaneers{}}) is a person who commands tame \nycans{}. In the Imetrium, being a \nycaneer{} is a highly respected position, and they are employed for many military or civilian tasks. 

%Not everyone can be a good \nycaneer{}; it takes a special talent, a certain feeling for \nycans{}. This is a magical ability, a type of empathy. In GURPS terms, this advantage might be called \quo{Animal Empathy: \Nycaneer{}}. A person with the \nycaneer ing talent has an intuition for recognizing the thougts and feelings of \nycans{} and communicating with them. \Nycans{} react well to such people, but may of course still be hostile, depending on the situation. 

A \nycaneer{} (plural \emph{\nycaneers{}}) is a \scatha{} who has a special empathy with \nycans{}. \Nycaneers{} share the innate telepathy that \nycans{} have and is able to communicate with them. \Nycans{} react well to such people (but may, of course, still be hostile, depending on the situation). 

%With training, a \nycaneer{} can learn to communicate with \nycans{} just as well as he communicates with other humanoids. 

Only \scathae{} and \rachyth{} have the \nycaneer ing talent. The reason is that \scathae{} are genetically related to \nycans{} (see section \ref{Origin of Scathae}). The talent is found in varying degrees, but even a weak \nycaneer{} can develop his skill, eventually being able communicate with \nycans{} almost as well as he communicates with other humanoids. \Humans{} and other races never have the \nycaneer ing talent. A non-\nycaneer{} can still learn to communicate with \nycans{} using simple commands, such as those used with dogs (but more advanced, since \nycans{} are more intelligent than dogs). About one in 40 \scathae{} is a \nycaneer{}. 

The Imetrium actively searches for young \nycaneers{} and maintains a formal \Nycaneer{} Academy. 

The \nycaneer{} that commands a given \nycan{} is that \nycanz{} \introep{\melda}{MEL-da} (Imetric word, plural \pronune{\meldae}{MEL-dej}). 

\subsubsection{Attributes}
\begin{description}
	\item[Horror effect:] None. 
\end{description}





\newpage
\section{Alien and Supernatural Creatures}

\subsectionn{Balrog}
\subsubsection{Name}
\subsubsection{Physique}
\subsubsection{Biology}
\subsubsection{Psychology}
\subsubsection{Habitat}



\subsectionn{\Bane{}}
\label{Bane}
\Banes{} are terrible alien creatures. They are not native Mith but come from the world of \Erebos, sometimes called the \Baneworld. 

\Banes{} are very rarely seen on Mith, and most people do not know they exist. They are sometimes summoned by evil mages or gods, but only the most mighty and depraved (and often dangerously mad) alienists dare attempt this, since merely communicating with a \bane{} is a horrifying and metally scarring experience. 

The \banes{} and their world are described in the infamous Book of Nom. It divides them into several types, called \banespawn, \baneknights, \banelords{} and \banekings. The \banekings{} are the mightiest of them, likely equal in power to the greater gods of Mith, if not superior. No \baneking{} is ever known to have come to Mith. \Banelords{} are as powerful as minor gods, and only very few of them have ever been summoned to Mith. 

The \banes{} most commonly summoned are the \baneknights{}. For this reason, they are typically referred to simply as \quo{\banes{}}. \Banespawn{} are rarely summoned, because they are difficult to communicate with and control, but a summoned \baneknight{} may sometimes bring along \banespawn{} as its own servants. 

In the following, \quo{\bane{}} will refer to the \baneknights{}. 

\subsubsection{Name}
Singular \pronune{\bane{}}{BEJN}, plural \emph{\banes{}}. The word and grammar is English. 

\subsubsection{Physique}
Seen from a distance, a \bane{} looks like a humanoid in a long, flowing robe, frayed and tattered at the edges. The robe is actually the creature's body, and the tattered egdes are small pseudopods used for walking. It has a head, torso and a number of arms (see below), but no legs, only a number of pseudopods or tentacles (each 20-50 cm long).  

Typically, a \bane{} will have two arms, but some have more, and they seem able to sprout new arms from their body as needed and retract them again when not needed. It is believed that growing additional arms and maintaining them is a psychically taxing, so that the \bane{} does not use it more than it has to. \Banes{} with only one arm have been sighted, but they typically maintain at least two. A newly sprouted arm will be weaker and less dextrous than normal\footnote{Perhaps a 50\% penalty to strength and a 30\% penalty to dexterity} and will grow to full strength gradually over a few minutes, perhaps 2-5 minutes. A \banez{} arms do not have elbows; they are fully flexible and should perhaps be called tentacles. At the end an arm splits into a number (variable, but usually around four) of small \quo{fingers}. \Banes{} can also sprout new fingers and other appendages at will, and create bladed claws from their fingers. 

\Banes{} crawl only slowly on their tentacles. Maximum running speed is like a fast walk for a \human, and typical walking speed is like a slow walk. They can use magic to leap, fly or teleport, but this costs them valuable mana and casting time, so they only use it if necessary. 

\Banes{}' height varies from 170 to 250 cm. Their body is fully corporeal and solid and slightly denser than the flesh of most Mithian creatures. Weight varies from 120 to 250 kg. 

The head has no recognizable features, including eyes. Sometimes you can feel the \quo{gaze} of the \bane{} upon you, but this is actually a psychic sensation of the \bane{} telepethically peering into your mind - a very unsettling experience, if not outright terrifying. The \banez{} head does seem to contain vital organs, however, as a powerful blow to the head will sometimes kill the creature. 

The \banez{} skin is tough and leathery on the body, head and major appendages. On the smaller or more recently generated limbs, the skin is more soft and supple. The creatures sometimes wear visible armor, made of an alien metal and adorned with grotesque and hideous patterns and symbols. Some people have reported markings on a \banez{} armor resembling stylized drawings of known alien monsters. 

\Banes{} are coloured in pure black, but wounds sometimes reveal blue-gray \quo{flesh} inside. A cut also causes a thick, pale white vapour to pour forth. This \quo{\baneblood} is heavier than air and will spill onto the ground. Large amounts of \baneblood{} spilt will leave a pale gray stain and kill vegetation in a small area. \baneblood{} smells foul but is not known to have any harmful effect on animal life. The blood may have magical properties if collected, but this is only speculation. 

In combat, \banes{} will typically strike in close combat with weapons. The weapons they typically wield are wicked swords and daggers, maces studded with spikes and blades, and (less often) strange weapons resembling a whip or flail. They seem to prefer close combat; if possible, a \bane{} will always strike its killing blow in melee. It is speculated that they drain life energy from their opponents this way. \Banes{} are sometimes seen pulling weapons seemingly out of their bodies, including large weapons that it ought not be possible to conceal. If disarmed, a \bane{} will pull out a new weapon if it has one, or strike with its bare tentacles (forming claws at the end). A \bane{} will never pick up and use a Mithian weapon, with the rare exception of some magical artifact. 

\Banes{} very much prefer to fight in close combat. All \banes{} are mages and can attack at a distance using magic, but they are reluctant to do so. They may do it in order to kill or subdue a fleeing victim if vitally necessary, but otherwise, \banes{} will only strike at range in order to defend themselves. They never wield ranged weapons such as bows. 

What senses do \banes{} have? They don't have smell and taste. Do they have vision and hearing? Do they have telekinetic and telepathic sense? Perhaps even more exotic senses? 

\Banes{} are somewhat resistant to heat/fire and cold attacks, and with their alien biology, they are immune to Mithian poisons and diseases. Electricity and acid do full damage, as do attacks with physical weapons. They do not breathe, and as such, cannot be strangled or suffocate. 

\subsubsection{Biology}
\Banes{} are from an alien world called \Erebos. Only a very few people from Mith have ever visited \Erebos, so little is known of it. The Chronicler of Nom did not visit the \Baneworld{} (as far as is known), but he did communicate with \banes{} and learn something of it. He speaks of tremendous cities, both hideous and beautiful at the same time, of colossal monolithic towers and castles, and of massive armies of \banes{} and monsters fighting terrible wars, apparently against other \quo{nations} of \banes{} (nations presumably ruled by \banekings) as well as other monstrous races. He even speaks of dark temples where the \banes{} worship alien gods in grotesque rituals. 

\Banes{} do not eat, drink or breathe. It is unknown what they live on, if anything. It is believed that they drain life energy from other creatures, for a summoned \bane{} will sometimes attack and kill living creatures, preferably intelligent creatures, for no apparent reason. 

\Banes{} do not seem to have genders. It is unknown how they reproduce, but the Book of Nom states that all \banes{} are born as \banespawn{} and slowly advance through the ranks. The Book seems to hint that the lower ranks of \banes{} (possibly only the Spawn) have finite life spans while the higher ranks are immortal. 

When slain, a \bane{} will dissipate into a pool of putrid \baneblood, leaving a dark and tainted stain of many square meters. On such a \bane{} \quo{tomb}, no wholesome vegetation will grow for decades (although it is sometimes infested by hideous, abnormal fungi and mosses), and there is a distinct atmosphere of unnatural evil. Seeing a \bane{} tomb has a Minor Horror Effect. 

\subsubsection{Psychology}
Little is known about the \banes{}' mindset. They are alien creatures not like anything on Mith. 

\Banes{} have no voices and cannot speak, but they can hear and understand speech.\footnote{Or can they? Maybe they can't hear, but use telepathy to scan people's minds.} If a \bane{} needs to communicate, it will do so telepathically. \Banes{} encountered on Mith are always able to understand one or more Mithian languages, typically Kingstongue and/or Ancient Vaimon. It is believed that the majority of \banes{} know little to nothing of Mith, and that those successfully summoned are always \quo{learned} \banes{} who have studied Mith and its creatures. 

When summoned, a \bane{} must be bargained with. No magic is known for enslaving \banes{} and controlling them against their will, and attempts to do this typically result in the summoner dying a gruesome death. Typically, the \bane{} agrees to perform some service in exchange for a gift. Suitable gifts are live sacrifices (large numbers of intelligent creatures), arcane knowledge and magical items. Especially prized are items made to combat \banes{}. It is believed that the \banes{} covet these items partly to prevent others from using them against them, and partly to use them in their wars against fellow \banes{} on \Erebos. A \bane{} may also demand that the summoner perform some obscure magical ritual for it (typically refusing to reveal the purpose of it). Sometimes, a \bane{} will crave strange things in return, such as a large quantity of a particular mineral, or a specific person to be captured alive. 

Communicating with a \bane{} is a horrifying experience. People have reported that hearing the telepathic voice of a \bane{} inside your head feels like a terrible violation of the mind. One person described it as the equivalent of \quo{being pinned down, stripped naked and having someone carve their message in runes on your chest with a knife}. Such an experience will have a Moderate Horror Effect. That being said, \banes{} do not communicate much. They will not, for instance, comment, taunt nor threaten their opponents in combat. The only ones likely to ever hear the \banez{} voice are the summoner and his allies. 

If asked questions, \banes{} will often refuse to reply. They have no recognizable concepts of politeness, so if asked a question it will or can not answer, the \bane{} will typically not explain but simple ignore it. Even if it deigns to answer, asking questions about the \banes{} themselves and their homeworld is risky business, as the \bane{} will tend to show rather than tell, telepathically showing images of the \baneworld{} and the \banes{}' life. Such an experience will have a Major to Extreme Horror Effect. (It should be noted that the \bane{} cannot use this as an attack. For the \bane{} to \quo{explain} something to you, you must be willing to accept the \quo{transfer}.) 

\Banes{} have no recognizable morals and principles. A \bane{} can be stealthy if ordered to, or has a reason to fear being discovered, but they will kill and destroy anyone and anything if they must, and sometimes will for no discernible reason. 

It is unknown how the \banes{} communicate among themselves. It might be telepathically, or using some strange senses that Mithians do not have. Their own names cannot be translated into spoken languages. If it needs to communicate with Mithian creatures, a \bane{} will sometimes adopt a spoken name for others to use, or allow its summoner to give it a name. Rissit Nechsain typically gives his \banes{} such names as Direfrost, Illwinter or Coldscar. 

\Banes{} wield great magical power, but it seems that they do not use it to its full effect. For example, they can attack at a distance using magic, but will not do so unless absolutely necessary. Some scholars believe that \bane{} magic is simply mana-expensive, so that they are not able to use it very often. Others believe that the \banes{} are bound by some alien code of honour or religion (perhaps superstition) that restricts their actions. Yet others speculate that perhaps the \banes{} fear to use their magic because it is unnatural and frightening to them, like the way many Mithian mages feel about their magic. 

\subsubsection{Habitat}
\Banes{} may be found in any terrain. 

\subsubsection{Attributes}
\begin{description}
	\item[Horror effect:] Moderate to see the \bane{} close up, Minor if seen from a distance. Minor to see a \bane{} \quo{tomb} (see under Biology). Moderate to hear the \banez{} telepathic voice. Major to Extreme to listen to a \bane{} explaining about the \baneworld. 
\end{description}



\subsectionn{Thorn Angel}
A race of creatures, probably not native to Mith. There exist only a score or so of them. They form a single band. Their leader is \Hiothrex{}, and they all serve the Imetrium. 

\subsubsection{Name}
\emph{Thorn Angel} is English. There is no associated adjective.\footnote{\quo{Thorn Angelic} just sounds lame...} 

\subsubsection{Physique}
\subsubsection{Biology}
\subsubsection{Psychology}
\subsubsection{Habitat}



\subsectionn{Weaver}
%An enormous monster, the shape of a dark cloud with a number of legs or tentacles. Looks vaguely like a giant spider. 

A Weaver is an alien monster from the \baneworld. They are enormous and extremely dangerous monsters, feared even by \dragons{}. 

%It looks like a great cloud of dark fog sprouting a number of legs and tentacles. Some find that a Weaver looks somewhat like a bloated giant spider. 

%\subsubsection{Name}
%\emph{Weaver} is English. There is no associated adjective. 
%As in English. 

\subsubsection{Physique}
A Weaver looks like a great cloud of dark fog sprouting a number of legs and tentacles. Some find that a Weaver looks somewhat like a bloated giant spider. 

They are huge in size, easily reaching 10 meters in diameter. Small Weavers down to 4 meters have been encountered. The average Weaver in Threll will be 7-8 meters in diameter, up to 10 at most. Larger one have been sighted in Nom. There are reliable reports of Weavers up to 30-40 meters in diameter, and a few explorers tell of behemoths growing as large as 100 meters. They are quite massive; a 10 meter Weaver weighs an estimated 20 tons, and the huge ones may weigh hundreds of tons. 

Weavers cannot fly or jump; they can only crawl, but they do this rather fast (slightly faster than a running \human{}). They are roughly symmetrical in all directions (having no front and back) and can move in any direction. 

A Weaver's central body is not visible because the creature excretes a cloud of opaque gray gas. This gas is somewhat lighter than air and constantly rises up from the creature like smoke; presumably exhaled as the Weaver breathes. The gas is poisonous if inhaled. 

A Weaver gives off a horrible stench, the reek of an unnatural and loathsome abomination. It can be smelled over a hundred meters away (depending on its size). The monster constantly emits hissing and groaning noises (believed to be incidental). 

Only the legs and tentacles are visible beyond the cloud of gas. The legs are segmented and somewhat spider- or insect-like, rhe tentacles soft and flexible and up to 20 meters long. Leg length up to is about $\fracs{1}{3}$ of the diameter, tentacle length is up to about $\fracs{3}{2}$ times the diameter. A Weaver has about 10-15 legs and 20-30 tentacles. The tentacles frequently sprout smaller sub-tentacles along the way. Some of the tentacles end in a sharp sword-like claw (these can be long, up to $\fracs{1}{10}$ diameter). The creature is covered in very tough, leathery skin with small spikes (centimeters in length) irregularly jutting out. The legs are more heavily armoured than the tentacles. 

A strong wind (natural or artificial) may blow away the dark fog to reveal the body inside. It will become apparent that there is no central hub, only a mass of tentacles splitting off from and rejoining each other. The tentacles are much thicker near the centre of the creature (up to $\fracs{1}{5}$ of diameter). 

Scattered about on the creature's body are dozens of mouth-like orifices. The size of these holes varies from few centimeters to over a meter near the centre of the monster. The dark gas rises out of these holes, and it will shove captured victims into them. Once inside the Weaver, a victim will be slowly digested. This is a slow process, however; a swallowed victim is more likely to die from breathing the poisonous gas, or from suffocation. A swallowed victim can be rescued if (a part of) the monster is sliced open. 

In combat, a Weaver's primary weapon is to lash out with its tentacles. It will attempt to grab victims, pull them closer to its body, grab them with more tentacles, then pull them apart while slashing them with their claws. If the victims (or the parts of dismembered victims) are small enough, it will attempt to swallow them. The Weaver can also use its legs as bludgeoning weapons to kick or stomp. 

If the Weaver cannot easily grab its opponents with its tentacles, it will shoot its \quo{web} at them. This \quo{web}, from which the Weavers derive their name, is a mass of long strands of tough, sticky, pale white material. Each strand is 1-3 cm thick, and the Weaver can fire a score of them at a time towards the same target. The strands are fired from small holes in the tentacles (different from the mouth openings described above), and they have been known to fire them at ranges of 50 meters or more. The web is nearly impossible to tear, but it can be cut or burnt, and great cold causes it to freeze and shatter. A Weaver has a large supply of the web, and will generally never run out of it during a single battle. 

Weavers have some resistance to magic. Spells cast directly upon the creature tend to be absorbed and dispelled. Spells cast indirectly upon the monster (like a fireball) work normally. 

Slaying a Weaver is a monumental task. Once slain, the creature's body will thrash and convulse for a minute or so, then lay still. It will also exhaling the dark gas, but it still smells. After some hours, its flesh will dry out and crumble away. If the creature is cut up, a lot of web can be gathered (up to $\frac{1}{200}$ of the creature's weight). 

\subsubsection{Biology}
Almost nothing is known of the Weavers' biology. Presumably they eat the flesh of creatures they swallow. It is unknown how (or whether) they reproduce. It is assumed that the larger Weavers are older, but this is conjecture. 

\subsubsection{Psychology}
It is unknown how intelligent Weavers are. No one has ever successfully communicated with one, and they rarely show signs of intelligent behaviour, but some believe that they are extremely intelligent. Attempts to communicate with them using telepathy usually result in serious damage to the telepath's mental health. 

Weavers are always aggressive when encountered and will attack pretty much every creature they detect. If its prey flees, the Weaver will pursue for a short time, but it will not move far from its original position. The Weaver itself will never flee, however; if pressed, it will fight to the death. 

Weavers are solitary; no one has encountered more than one at a time. It is unknown how and why they avoid each other and what would happen if two were to meet, but it is believed that each Weaver maintains a territory (this also explains why they rarely move far from the place they are encountered). 

\Banes{} greatly fear Weavers and will flee before them. 

Rarely, \banes{} may be encountered together with a Weaver, sometimes even hunting together. Such \banes{} can often be seen casting spells and performing strange rituals involving the Weaver. It is believed that these are religious rituals and that the renegade \banes{} serve and worship the Weaver. %These may be compulsion spells to keep the Weaver under their control, but it is also possible that the rites are of a religious nature and that the renegade \banes{} serve and worship the Weaver. 

\subsubsection{Habitat}
Not native to Mith, the Weavers originate from \Erebos. Most likely, the Weavers on Mith were summoned during the \banewar. On Mith, they are known to exist only in Nom and Threll. All weavers encountered in Threll have been fairly small, no larger than 10 meters. In Nom, enormous specimens have been encountered. 

Where do they dwell? In caves or out in the open? In valleys or in mountains? 

The total number of Weavers is unknown, but Threll is estimated to home between 100 and 1000 of them. Nom is huge and unmapped, so no one can reliably estimate the number of Weavers there. 

\subsubsectionnn{Weaver web}{Weaver web}{Weaver!Weaver web}
Weaver web may be gathered, during a fight or when the Weaver is dead. It dries in a couple of minutes, after which it is no longer (very) sticky. 

The chief use of Weaver web is that \banes{} fear it. \Banes{} seem to have a superstitious fear of Weavers and will flee from anything that smells like a Weaver, including web. One kg of web can be smelled by the \banes{} up to 30 meters away, and they will be hesitant to approach any closer than that. (Only \banespawn{} and \lesserbanes{} fall for this. \Banelords{} are not fooled. If led by a \banelord{}, \lesserbanes{} will attack, but will be at a disadvantage because they still fear the web.) 

The web dries up completely in a matter of days if left unprotected. Wrapped up and sealed, it can last up to a month, and preservation spells are known that will make it last several months. The older it is, the less it smells, and the less effective it is. When it dries up, it stops smelling and crumbles. 

The chief disadvantage of the Weaver web is that while it repels \banes{}, it also repels everyone else. The stench of it is loathsome and sickening, and animals and humanoids instinctively hate it. The easiest way to destroy the web is to burn it (it will produce some awful brown smoke, but then it will be gone).

\subsubsectionnn{Weaver maggots}{Weaver maggots}{Weaver!Weaver maggots}
Weaver maggots are small creatures that crawl around on a Weaver and inside its mouths. They are gray, wormlike, 20-50 cm long (with thickness at the middle equal to about $\fracs{1}{3}$ of the length) and weigh 1-4 kg. They have dozens of small lumps scattered over their bodies which they use for slowly crawling about. A maggot has no top and bottom, it can crawl on any side. It has a front end with several (3-7) toothless mouths. It has tough, leathery skin covered by a thin layer of sticky slime, like a slug. They also exhale the dark gas, but in much smaller amounts, too little to cause more than coughing and irritation. 

The maggots are harmless, although they will try to eat organic matter that doesn't move. They look and smell revolting, however, so many people will feel a strong dislike of them and an urge to kill them. Killing them is easy; they can be crushed or cut in half. They can also be easily captured. There is no known use for them, and they rarely live more than a week in captivity, but nevertheless, researchers are sometimes willing to buy them to study. 

Every Weaver has plenty of maggots crawling about on it, 6-10 of them per ton of weight. When the Weaver is killed, the maggots will immediately begin to eat the flesh of their host. They are cannibalistic and will eat other dead maggots. Presumably, the maggots are a symbiotic or parasitic species, or possibly immature Weavers. 



\subsectionn{Wingworm}
Wingworms are loathsome monsters from \Erebos resembling flying worms (hence the name). They are unintelligent beasts and the \banes{} sometimes use them as mounts. 

%\subsubsection{Name}
%Singular \emph{}
\subsubsection{Physique}
A Wingworm looks like a huge worm. 

\subsubsection{Biology}
\subsubsection{Psychology}
\subsubsection{Habitat}





\newpage
\section{Undead}
\subsectionn{Death Knight}
The Rissitic Death Knights are a kind of Wights. 
%\subsubsection{Name}
%\subsubsection{Physique}
%\subsubsection{Biology}
%\subsubsection{Psychology}
%\subsubsection{Habitat}



\subsectionn{Ghost}
\subsectionn{Ghoul}

\subsectionn{Immortal Priest}
The Rissitic Immortal Priests are a kind of Liches. 
%\subsubsection{Name}
%\subsubsection{Physique}
%\subsubsection{Biology}
%\subsubsection{Psychology}
%\subsubsection{Habitat}



\subsectionn{\Leech}
\Leeches{} are not to be confused with the similarly-named Liches, nor with regular leeches (non-capitalized), wormlike animals that live in swamps and the like and drink blood. \Leeches{} (capitalized) are a lesser form of \Reavers{} (see section \ref{\Reaver{}}): Intelligent undead who drain the life-force of others to survive. 

A \Reaver{} is created more-or-less voluntarily when a mage uses life-draining magic extensively. A person cannot turn himself into a \Leech{}. \Leeches{} are created by \Reavers{} as lovers, companions or servants (or all three). Most \Leeches{} serve a \Reaver, and many dream of becoming \Reavers{} themselves. 

\subsubsection{Physique}
\Leeches{} look like living people of their race. There is nothing to put your finger on, but sensitive people will notice a certain savage, bestial aura about the \Leech. 

\Leeches{} have most of the powers of \Reavers{}, but weaker. They are supernaturally strong, fast and agile and can regenerate almost all wounds. They can see in almost total darkness and see auras around living creatures. A \Leech{} may take seemingly-mortal wounds yet rise again. If the brain is separated from the heart (usually by severing the head) or either brain or heart is smashed or torn apart, the \Leech{} is permanently destroyed. 

They also have the vulnerabilities of \Reavers{}, albeit to a lesser degree. Their vulnerability to wood is the same, but they are more resistant to sunlight. A \Leech{} will be destroyed in a few minutes if fully exposed to daylight, but normal shade, such as that provided by a broad-brimmed hat or umbrella, is enough to preserve them from harm. Even if exposed to sunlight, they can suffer it for up to a minute with no more than moderate pain and no visible damage. This makes them well-suited as agents for the \Reavers.

Unlike \Reavers, a \Leech{} does not drain life by spells but by drinking the blood of her victim. The victim must be alive or very recently killed for their blood to have any value to the \Leech. 

A \Leech{} retains all her former skills, including magic. But unlike \Reavers, \Leeches{} need not be mages, and most know no magic. 

\subsubsection{Biology}
To create a \Leech, a \Reaver{} must drain a victim to the brink of death and then cast a special spell on her, called the \quo{\Reaverz Kiss}, that transfers a portion of his life force to her, in effect causing her to drain some of his energy. The spell is also an enchantment that grants the \Leech-to-be a limited ability to drain life force without the use of spells, namely by drinking their blood. 

The \Leech-to-be will awaken with a tremendous thirst for blood. She can drink the blood of the \Reaver, but he is unlikely to allow her to sate her thirst on his own blood. Rather, he will send her to feed on mortals (perhaps keeping prisoners or servants ready, perhaps sending her off to hunt her own). 

At this point, the prospective \Leech{} has the supernatural powers of a \Leech{} to some extent and a great psychological craving for blood, but she is not fully one of the undead and not yet physically dependent on blood to survive. Resisting the desire to feed is difficult, but it can be done. If she refrains from drinking blood, the effects of the spell will gradually wear off. She will become sick and weak, but most likely she will survive, and after 5-7 days the spell has worn off completely and she will be back to normal. (In such a case, however, the \Reaver{} is likely to feel betrayed and attempt to reclaim or punish her.) 

If, on the other hand, the would-be-\Leech{} gives in to her desire and feeds on blood, she will find a great erotic pleasure in doing so, and she will find herself gradually transforming into a full-fledged \Leech{}. \Leeches{} can sustain their strength by drinking mortal blood, but they are not true \Reavers, and their method of draining life force is imperfect. Consequently, they cannot survive on this fare alone, but must also drink the blood of a \Reaver. At least once a month, the \Leech{} must feed on \Reaver{} blood or she will perish. Drinking mortal blood solidifies the \Leechz undead state, but drinking \Reaver{} blood cements it. A \Leech who has drunk \Reaver{} blood once (beyond the initial Kiss) can still fight the curse of undeath and return to life, although it becomes much harder, and even after having drunk twice the curse may be broken, although at this point magical healing and exorcism will be needed. After drinking \Reaver{} blood thrice after the Kiss, the \Leechz fate is sealed. She is now fully of the undead and can never return to life. 

As with a \Reaver, she needs not kill her victims, but unlike a \Reaver, a \Leech{} cannot absorb her victim's soul. The blood of another \Leech{} is particularly savoury and nourishing, but no substitute for \Reaver{} blood. 

The goal of many a \Leech{} is to become a \Reaver{} herself. To do this, she must learn spells of life-draining or acquire an enchanted item that duplicates the effect. This will let her drain true life force on her own, eliminating her need to feed on her master and thus eliminating her dependence on him. After draining life with magic for a while (usually a matter of months, since she will likely be feeding a lot), the \Leech{} will transform into a true \Reaver, with all the powers and weaknesses described in section \ref{\Reaver{}}. 

Like \Reavers{}, \Leeches{} do age, but they can arrest and reverse aging by drinking blood, thus living forever. 

\subsubsection{Psychology}
Some \Leeches{} love their \Reaver{} master and serve him willingly. \Reavers{} often turn their lovers into \Leeches{}. A \Reaver{} may have a single \Leech{} lover and treat her as an equal or near-equal, or he may have a whole harem of them vying for his affection. The \Reaver{} may, of course, also turn non-lovers into \Leeches, but only those people he believes he can trust. A \Reaver{} who truly cares for his \Leech{} might even teach her magic so that she may become a \Reaver{} herself. 

A \Reaver{} has no supernatural control over his \Leeches{}, beyond the fact that they need his blood to survive. Usually, he will dominate his \Leech{} servants through physical and magical power and sheer force of will. After all, the \Reaver{} will usually be not only much older and more experienced, but also a great mage, whereas most \Leeches{} know no magic. 

Still, a \Leech{} might hate her \Reaver{} master and plot against him. Of course, unless she also wants to end her own existence, she cannot simply kill him. But there are ways of dealing with this dependence:

First, a \Leech{} must drink \Reaver{} blood, but it needs not be that of the \Reaver{} that created her, so a disgruntled \Leech{} who encounters another \Reaver{} might defect. 

Second, a \Leech{} might be able to overthrow her master, keeping him as a prisoner on whom to feed at leisure. Usually, the \Leech{} will be no match for her \Reaver{} master (he will see to that), but she might be exceedingly cunning. Alternatively, if a \Reaver{} has several \Leeches{} under his control, they might band together to overthrow him. For this reason, a \Reaver{} who does not fully trust his \Leech{} slaves might play them against each other, turning them into rivals and enemies, ensuring they they do not ally against him. 

Third, the \Leech{} might learn of the fact that it is possible for her to become a \Reaver{} herself. If she manages to gain access to life-draining magic, she is likely to flee. 

To ensure her obedience, a \Reavers{} will often lie to his \Leech{} slave, convincing her that only his blood (not that of another \Reaver) will keep her alive and keeping her ignorant of how she might become a \Reaver. 

A \Reaver{} may create any number of \Leeches{} he desires, but each of them must drink his blood every month or die, and letting the \Leech{} drink his blood temporarily weakens the \Reaver, so he will not want too many mouths to feed. More importantly, a large group of \Leeches{} may mutiny against their master (as described above), so a \Reaver{} will not want to create more of them than he can control. 

\subsubsection{Habitat}
\Leeches{} always live near their \Reaver{} master (or they won't live long). But since they are less vulnerable to sunlight, they are more easily able to lead normal-seeming lives and not arouse suspicion. 



\subsectionn{Lich}
Liches (not to be confused with \Leeches, who are weaker undead) are arguably the most powerful type of undead. A Lich is a powerful mage who has transformed himself into one of the undead by means of an occult spell of terrible power. 

What characterizes the Lich is that it is fully self-sustained and immortal. The spell that creates the Lich opens a conduit to a source of dark energy somewhere in the Beyond. From now and evermore, this power sustains the Lich. Thus, the Lich needs to external power source. It requires no magical rituals to sustain it, nor must it feed on the life-force of others, like \Reavers{} do, but will sustain itself and exist forever.

A Lich is fully intelligent and retains all the skills and knowledge it possessed in life, including magic, and it will continue honing its skills and learning more magic throughout its immortal unlife. A mage must be of formidable skill and knowledge in order to become a Lich in the first place, and will grow only stronger as the centuries pass, so an old Lich is a mighty creature indeed. 

\subsubsection{Name}
Singular \emph{Lich}, plural \emph{Liches}. 

\subsubsection{Physique}
Liches retain the form they had in life...

%\subsubsection{Biology}
%\subsubsection{Psychology}
%\subsubsection{Habitat}



\subsectionn{\Reaver{}}
Using dark magic, some people are able to drain the life force of others to empower themselves. This is a very potent ability, but it carries a price: Repeated use of life-draining magic is addictive, physically as well as psychologically, and the mage will develop a growing craving for it. As the addiction grows, the mage will gradually transform into one of the undead. Such people are called \Reavers{}. 

\Reavers{} are usually powerful mages, but non-mages who rely on enchanted items to drain life force are also affected and may become \Reavers{}. 

Becoming a \Reaver{} is a gradual process. As a person succumbs to the addiction, his \Reaveric{} traits will grow stronger and more pronounced. The transformation is considered complete when the \Reaver{} can no longer sustain himself by natural means (food and drink) and must live off the life force of others alone. 

%\Reavers{} are people who 

\subsubsection{Name}
%\emph{\Reaver{}}, plural \emph{\Reavers{}} as in English. 
As in English. 
%The adjective is \emph{\Reaveric{}}. 

\subsubsection{Physique}
\Reavers{} look much like living people of their race. In time, their skin will grow somewhat pale, but they can still pass for normal people. Only people or creatures with special empathic skills will be able to detect the \Reaver{} at a glance. 

As a \Reaver{} grows in power, he will learn to drain more life force than he needs, storing it in his body to make himself supernaturally strong and fast. Powerful \Reavers{} are able to perform astonishing feats of superhuman agility and strength. Such power, however, is costly to maintain, and a \Reaver{} who wishes to remain strong must drain great amounts of life. \Reaver{} mages can use their drained life force to power their spells, making them very formidable spellcasters. 

\Reavers{} gain the ability to see in darkness, their vision unnaturally sharp even in minimal light. They cannot see in total darkness, but they see as well by starlight (under a clear sky) as in broad daylight. They also gain the ability to see the life force auras surrounding living creatures. This sense is similar to infravision, except that it shows life, not heat. It \emph{will} work in complete darkness and can see through up to 10-15 cm of earth, wood or stone or 1-2 cm of metal. This sense will \emph{not} detect undead or artificial constructs. It may or may not show alien creatures, depending on how alien they are. 

As a \Reaverz{} mastery grows, he can drain energy from opponents by a mere touch without having to cast an elaborate spell. Some \Reavers{} know spells that let them drain life through clothes or armour, through a weapon or even at a distance. 

A \Reaverz{} greatest weakness (apart from his need to feed on life) is that he cannot abide the light of the Sun. Direct sunlight chars the \Reaverz{} flesh like fire. This vulnerability grows gradually as the \Reaver{} undergoes his transformation. A fully transformed \Reaver{} will die and crumble to ashes within a minute if exposed directly to full daylight (a few minutes if the he is large, like a \dragon{})\footnote{Only the body parts exposed to sunlight will burn. If the \Reaver{} is naked, his entire body will burn. But even if only the head is exposed, the \Reaver{} will still die when his head burns and crumbles.}. Light cast back from a strongly reflecting surface (such as a mirror or a lake) is almost as dangerous as direct sunlight. Light reflected from a bright surface (like snow, white marble or shiny metal) is less dangerous, wheras light cast back from dark surfaces (like wood or earth) is mostly harmless and can be endured for minutes with little harm. 

\Reavers{} mostly avoid going outside by day and travel only by night. If a \Reaver{} must travel by day, he will cover his entire body in clothing, using a hood, hat or mask to cover his head. There are subtle spells that offer some protection, but none are known that let a \Reaver{} withstand full daylight. A \Reaver{} unafraid of detection may use spells to cover himself in a cloud of darkness, providing further protection. 

Most non-sunlight (moonlight, bonfires etc.) is harmless to \Reavers{}. As for magical light, the spell description will sometimes state that the light works like sunlight. In this case, it will burn and can kill \Reavers{} as described above. 

\Reavers{} are vulnerable to weapons made of wood. Wooden weapons that hit bare skin or thin clothes will cause extra damage. Wood that strikes a \Reaver{} wearing armour has no special effect. If a piece of wood pierces a part of the \Reaverz{} body, that body part will be paralyzed until the wood is removed, and numb for a while even then. If wood pierces the head or torso  (not necessarily through the heart), the \Reaverz{} entire body will be immobilized. This will work even with a wooden arrow with a metal head. 

\Reavers{} have the ability to heal almost any wound, reattach severed limbs or even regrow them from scrath. Burns, such as from sunlight, and wounds from wooden weapons can also be healed, albeit slower than most damage. Using powerful necromancy, \Reaver{} mages can sometimes even resurrect themselves after being killed and mutilated. This will leave the \Reaver{} drained and weak, however, and he must quickly feast upon great amounts of life or perish forever. If the \Reaverz{} body is burned (in fire or in sunlight), destroyed with acid or eaten and digested, it is destroyed beyond hope of resurrection. (Such a \Reaver{} might, however, still be raised as one of the incorporeal undead, such as a Wraith.) 

Certain animals can detect \Reavers{}. This includes all canines, all small cats (but not all large cats) and \nycans{}. They do this by a combination of smell and empathy. These animals will fear and hate the \Reaver{} and are likely to either attack or flee. 

It is possible for persons to learn the mystic, empathic skill of recognizing \Reavers{} and other undead. Imetric Paladins and the priests of \NishiS{}, certain Vaimon Templars and Clerics and the Ashenclaw knights of Khoth-Sell are all taught this ability. 

\subsubsection{Biology}
All races can become \Reavers{}. There are different ways to become a \Reaver{}, because there exist several varieties of life-draining magic. It is often said that \Reavers{} drink the blood of their victims, but in fact this is only one of several means of draining energy. %Some \Reavers{} know different ways to drain life, but some know only one. A \Reaver{} who only knows to drain life by drinking blood is called a Blood \Reaver{} by scholars. Other varieties are the Shadow \Reaver{} (using Rissitic Shadow magic), Nieur \Reaver{} (using Vaimon magic) and Chaos \Reaver{} (using \draconic{} Chaos magic). The different types of \Reavers{} will differ in their array of spells and skills, and likely also in culture and habits. 

The Rissitic Ashenoch cannot become \Reavers{}. Using life-draining magic, the Ashenoch will develop a physical and mental addiction, but they will not transform into undead or gain any of the \Reaveric{} traits described above, no matter how much energy they drain. 

Contrary to popular belief, those drained and killed by \Reavers{} do not rise as \Reavers{} themselves. A \Reaver{} cannot create other \Reavers{}. A person can only become a \Reaver{} through use and abuse of life-draining magic. However, it is possible for a \Reaver{} to create \Leeches{}, undead creatures similar to \Reavers{} but weaker (see section \ref{\Leech}). 

Unlike most undead, \Reavers{} do age, and their bodies will decay and weaken. In fact, \Reavers{} age much faster than the living and may age a decade in a single month. However, a \Reaver{} can use drained life force to rejuvenate himself, thus staying young and immortal potentielly forever. The \Reaver{} can regulate how much he wants to rejuvenate the surface of his body. Most \Reavers{} choose a certain age and maintain their appearance to fit this age. He may change this at any time by allowing himself to age or spending more power to rejuvenate himself. (This is not immediate, but may take several months.) Thus, regardless of his true age, a \Reaver{} may have the look of a youth or an ancient man. Whatever their skin looks like, however, all \Reavers{} make sure to keep their inner body healthy and strong, so the \quo{old man} \Reaver{} may be just as physically strong and agile as the \quo{young man} \Reaver{}. Only when a \Reaver{} is deprived of energy to drain will he begin to weaken. 

A \Reaver{} will die after 10-50 days of not feeding, depending on his strength and how much he exerts himself. But they prefer to feed every day. A \Reaver{} needs not kill his victim, but draining a victim provides much more energy for the \Reaver{} to absorb than merely draining her to the brink of death, because then the \Reaver{} is able to absorb her very soul. A victim thus absorbed is destroyed forever and can never be resurrected. 

\Reavers{} particularly savour the life force of \Leeches{}, or better yet, other \Reavers{}. (These are the only undead that can be drained.) Draining another of the undead is exceptionally nourishing and enjoyable to the \Reaver, but moreover, draining another undead to destruction and absorbing her will cause the \Reaver{} to gain a portion of her power, thus making himself permanently stronger. Occasionally, this has the side effect that the \Reaver{} will adopt some bits of the absorbed one's personality. 

A \Reaver{} cannot eat food nor drink normal drinks. He can swallow it, but his atrophied digestive system rejects it, but will have to regurgitate it soon after (up to ten minutes at most). 

\subsubsection{Psychology}
It is sometimes said that \Reavers{} can feel no pleasure or happiness and are forever tormented by their foul, unnatural state. It is true that there are some \Reavers{} who come to hate what they have become and succumb to self-loathing and depression. 

But \Reavers{} can actually feel lots of pleasure. Unlike most undead, \Reavers{} retain their sexual drive and can still have and enjoy sex in their undead state (although a fully transformed \Reaver{} is sterile). Draining energy is, to some extent, viewed as a sexual act, and \Reavers{} prefer to drain attractive people of their own race and the opposite sex (or whatever attracts them). Reavers can live off unintelligent animals if they must, but it is much less nourishing and distasteful to them. Draining the life of a beast gives none of the pleasure that comes with draining an intelligent creature. 

%As mentioned, they enjoy sex just as they did in life, and 

Most other passions from life remain in undeath as well. They can also taste food and drink, but this pleasure is marred by the fact that they'll have to vomit it up again. 

%Unlike most undead, \Reavers{} retain their sexual drive and can still have and enjoy sex in their undead state. Once the transformation into undead is complete the \Reaver{} is sterile, though. 

It is uncommon but not unheard of for \Reavers{} to have relationships with other \Reavers{}. More commonly, a \Reaver{} will create \Leeches{} to act as his companions and servants. 

\subsubsection{Habitat}
Some \Reavers{} live in civilization, where they find ways to avoid the sunlight and lead ordinary-looking lives. Others live removed from civilization and only seek out other people when they must feed. 

%\subsubsection{Myths}
%The story of the \Reaver{} is well-known throughout Mith, and myths and superstition about them about in all lands. The vast majority of the storytellers who spread such myths have never met a \Reaver{} (though they may claim otherwise), so it should come as no surprise that much of what is \quo{known} about \Reavers{} is pure superstition. 

%According to the tales, \Reavers{} cannot cross running water, cast neither shadow nor reflection and cannot enter a home without invitation. Garlic or a holy symbol of Good repels them. Wolves, rats and bats serve the \Reavers{}, and they can assume the form of any of these animals. 



\subsectionn{Skeleton}
\subsectionn{Spectre}



\subsectionn{Wight}
Wights are a class of corporeal, intelligent undead who are created by magic. 



\subsectionn{Wraith}
\subsectionn{Zombie}



\bc
\newpage
\section{Lycanthropes}
Do they exist on Mith? If so, here are some ideas for them: 

\begin{itemize}
	\item Were-Lion
	\item Were-\nycan{}
	\item Were-Manticore
	\item Were-Mammoth? 
	\item Were-\cortio{}
	\item \Draconic{} Lycanthropes
\end{itemize}

\end{comment}



