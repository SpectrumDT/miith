\chapter{Old overview}
\section{The First \Dragons}
The \leviathans{} live in the sea. 
\HesodN{} urges his fellows to explore beyond the sea. 
They do, bringing along hordes of Nagae. 

When they go ashore, \HesodN{} proclaims that he and his fellows are no longer merely \leviathans{}, but that this day sees the founding of a completely new race. He names their people in the True Tongue, calling them \draecchonosh{}. It is a word that does not translate easily into Nag or any other language, but even those not learned in the true tongue can readily hear, when \HesodN{} speaks it with passion, that it is a word that speaks promises of strength, power, beauty and dominion. 

On land, \HesodN{} and his fellow \leviathans{} encounter impressive and inspiring creatures, such as \nycans, \cregorr{} and \wyverns{}. \HesodN{} shows his fellows these creatures and proclaims that he and his people will draw on the strength and beauty of these creatures and so become greater than any \leviathan. 

In a process that takes many years, \HesodN{} and his people absorb and assimilate the power and being of many land-living creatures, gradually transforming their own bodies, taking the best of what nature has to offer in the shape of various animals, and adding that to their own bodies. 
%They draw on the power and being of these creatures to transform their own bodies, taking the best of what nature has to offer in the shape of various animals, and adding that to their own bodies. 
\TyarithXserasshana{} is the first to pattern herself after a \wyvern{} and grow wings. 
%\HesodN{} and his closest companions transform themselves into magnificent \dragons{}, showing off their greatness and power and inspiring their Naga followers. 
%When they have transformed enough, \HesodN{} proclaims that they are no longer \leviathans{} but \draecchonosh{} - \dragons. 

The nascent \draconic{} kingdom establishes several cities near the shore. It turns out that the Nagae are ill suited to life on dry land\dash many Nagae are killed by wandering monsters and beasts. They encounter Cortios, and the \dragons{}, impressed by the majesty of these monsters, start creating the \cregorr{} in their image, as a race of warriors and protectors. 

The \cregorr{} are a great success and quickly become a large, respected warrior caste. With the \cregorr{} armies the \dragons{} move inland, claiming more territory from the monsters, \nephilim{} and other creatures that may inhabit the land. 








\section{Prince Semiza}
We also follow Semiza part of the time. 
Semiza is the eldest son of the \nephil{} sorcerer-king of Jocain. 
Early on we see him passing some kind of \quo{manhood test}, marking him as a full-fledged sorcerer and a worthy heir to the throne. 

At an early point, Semiza meets some girl that he likes. 
Either right after his rite of manhood or some time after, he is betrothed to her. 

Some years pass. 
The Jocainites hear the tales of war, of some terrible invaders from the sea devouring entire kingdoms. 
The king and mages are concerned, but Semiza is still young and untroubled. 

Then comes the day of Semiza's wedding. 
On this, supposedly the greatest day of the young prince's life, the \dragons{} invade Jocain. 
It becomes massacre and young Semiza sees his bride and his entire family slaughtered before his eyes. 

Semiza and a few survivors flee to a system of underground caverns. 
Somehow they manage to survive the \draconic{} army's onslaught. 
Semiza assumes the mantle of king and leads his people in exile. 







\section{The \Dragon{}-\Leviathan{} War}
Meanwhile, tension develops between the \dragons{} and the \leviathans{} living in the sea. At this point, the early \scathae{} are developed. The \leviathans{} decide that they have \quo{stock} in \ps{\HesodN}{} terrestrial kingdom and demand tribute. Some \leviathans{} also believe that the \dragons{} and their creations are abominations that should be destroyed. \HesodN{} refuses to pay any tribute, and the \leviathans{} attack. 

The \dragons{} are outnumbered, but they have done experimentation and research and discovered new, more daring and dangerous forms of Chaos magic and \daemon{} summoning. Among the greatest of the \draconic{} sorcerers are \HesodN, \TyarithXserasshana{} and her consorts. Summoning forth \daemonic{} aid from the Beyond, they manage, after a hard struggle, to turn the tide of the war and vanquish the \leviathans{}, who slink back into the sea. 







\section{Semiza, king-in-exile}
Meanwhile, in this great civil war, the \nephil{} survivors have been forgotten. Hiding in his underground lair, Semiza nurtures his hatred and dreams of revenge against the \dragons{}. 

Somehow Semiza reaches out into the Cosmos with his mind. Perhaps he is deliberately searching for allies; perhaps he takes mind-expanding drugs. By whatever means, he contacts Daggerrain. See, the \baneking{} \Voidbringer, Daggerrain's master, had felt the cosmic ripples caused by the \dragons{}' spellwork. 
%Tiamat's spellbinding. 
So \Voidbringer{} ordered his servant Daggerrain to investigate. Daggerrain probed with his mind toward \Miith{}, and shortly after, he found a \Miithian{} mind that was open for him to contact: that of Semiza, the \nephil{} king-in-exile. 

Daggerrain penetrated the \ps{\nephil} vulnerable mind and probed around in it. He took from the sorcerer-king's mind his language and his entire memory. Then he introduced himself to the \nephil{} king, offering him the power to take his revenge, destroy the \dragons{} and restore his kingdom to rule forever. At this point, Semiza asks \ta{Who are you?}, and in response Daggerrain sends a vision where Semiza finds himself alone and naked amid an endless, featureless plain, with a cruel and unrelenting hail beating down upon him with its sharp shards of ice, like a rain of a million daggers. Semiza interprets the \banelord{}'s name as \quo{Daggerrain}. 

If he is to help Semiza and his people, Daggerrain needs the \nephil{} sorcerer to bring him to \Miith{}. He instructs the \nephil{} in the spells to use, and Semiza goes to work procuring occult ingredients and sacrifices (mostly \scathae) and recruiting and training mages to assist in the spellwork. The preparations take perhaps several years, but at last they are ready to perform their summoning ritual. 







\section{The \Banewar}
The gateway to \Erebos{} is opened and Daggerrain's legions swarm forth into \Miith{}: millions of \bane{} warriors and beasts, and hundreds of \banelords, all led by Daggerrain himself. They invade and attack the \draconic{} kingdoms. 

The \dragons{} are hard pressed. Then \TyarithXserasshana{} steps forth. She is the daughter of \HesodN{} and a daring and talented sorceress, having conducted much research into forbidden magic together with her two consorts, Urzmagond and \ApepNesthra. She promises that she will be able to summon forth aid to vanquish their foes, but to do that she will need more power. She asks that the \dragons{} lend her their strength, pour their energy into her body in an arcane ritual, that she might bind the \daemons{} of the Beyond to her will, to serve her people forever for the glory of all \Dragonkind. 

She has her supporters and detractors. As the war with the \leviathans{} continues, however, the \dragons{} continue to lose ground and support for her plan grows. Soon enough \dragons{} acquiesce to support her. She begins her ritual, assisted by Urzmagond and \ApepNesthra, draining strength from her fellow \dragons{} and taking it for herself. With her augmented power, \Xserasshana{} reaches deep into the Beyond and spellbinds multitudes of \daemons. Her bindings lay the foundations of Chaos magic. She discovers Kingstongue and uses it to command the \daemons, and her spells of binding enslave the \daemons{} to her spellwords, so that future sorcerers can call upon those same \daemons{} using \ps{\Xserasshana}{} words. 

With her newly developed magic, \Xserasshana{}, flanked by her consorts, lead the \draconic{} legions into battle. Her newfound power is tremendous, and after a long and bloody war, even the terrible \bane{} legions are ultimately driven back. With enormous effort, the \dragons{} manage to close the gate to \Erebos{}. This cuts the \banes{} off not only from reinforcements, but from the source of their magical power. After this climactic strike, the remaining \banes{}, while they still number in the hundreds of thousands, are rounded up and destroyed. Only Daggerrain and a small force survive, and they hide underground with Semiza and his \nephilim. 
\Xserasshana{} is hailed for this glorious victory. 

Now, I have to devise a good explanation for this, but somehow \Xserasshana{} has only \quo{borrowed} her power from the \dragons{}. They expect her to relinquish it again when the battle is won. I expect they have some way of forcing her to relinquish it\dash or at least think that they do. But \Xserasshana{} has deceived them: When they try to reclaim their power, she denies them and keeps it for herself. Instead, she lashes out with it and forces the \dragons{} to obey her, proclaiming herself the supreme god of the new world. When her father, \HesodN, confronts her, she slays him, claiming that the supreme god can have no father. 

\Xserasshana{} lets her closest lovers and progeny share in her great power, and they become the \Dominators. (I don't know yet if all the \Dominators{} are born and full-grown at this point, or if some come later.) 
Many \dragons{} rebel against \Xserasshana{}, but enough support her, and eventually she destroys the rebels and rules supreme. 

\Xserasshana{} has a third lover, Saphlethcurir. He opposes her gambit for power and fights against her. Ultimately she slays him. But I should make him a major character, so the reader cares when he dies. 







\section{\ps{\Xserasshana}{} reign}
\Xserasshana{} rules the kingdom of the \dragons{}. With her command of the Beyond this evolves into a trans-dimensional empire spanning not only \Miith{} but reaching into the dimensions of Chaos bordering it. 

This kingdom endures for a thousand years or two. 

But during all this time, the \banes{} still exist. Daggerrain and the surviving \banelords{} survive together with Semiza's \nephilim. They do research and experimentation, partly to reopen a gateway to \Erebos{}, and partly to create the heirs of the \banes{}, merging their own \erebean{} power with the changefulness of Chaos. 

(I need to think this through better, but the idea is that the \banes{} are very ordered, structured creatures. As such, they lack innovation and progress and are thus doomed to a slow decline, unable to renew themselves. \Voidbringer{} knew this, and he recognized \Miith{} as a source of Chaotic energy... or a gateway to Chaos, or something, closer to Chaos than \Erebos{} is. So they come to \Miith{} in order to harness the power of Chaos to create a future for their people.)

They try and fail many times, and many times they are nearly discovered, surviving only through trickery and \draconic{} overconfidence, and often sacrificing many \nephilim{} to cover up their retreat. 

The \banes{} also work on shaping the dimension of \Nyx{}. Its purpose is threefold: First, it is meant to be a new source of power for them, similar to the power of \Erebos{} that they know. Second, it is intended as a bridge to \Erebos{}. Third, it is a hiding place where the \dragons{} cannot easily detect them. 







\section{The \Resphain{} and the \Secondbanewar}
Eventually, after more than a thousand years, the \banes{}' experiments bear fruit and their hybrid creatures are born: The \resphain, bred from \nephilic{} stock, endowed with \erebean{} minds and infused with the life-giving energy of Chaos. 

The \resphain{} settle in \Nyx{} and in hidden places on \Miith{} and proceed to build an empire, with themselves as the aristocracy, the \nephilim{} as the common folk and the \banes{} as their sinister affiliates and masterminds. For another thousand years or so, the \banes{} and \resphain{} labour to open a conduit from \Nyx{} to \Erebos{}. 

At long last they succeed and the way to the sinister \baneworld{} stands open once more. Daggerrain contacts his king, and again legions of terrible \banes{} pour forth. Another terrible war ensues, the \secondbanewar. \Dragons{}, \cregorr{} and \scathae{} battle against \banes{}, \resphain{} and \nephilim. 

After years of bloody conflict, the \draconic{} side manage to send a heroic party on a suicide mission into \Erebos{}, their quest to somehow destabilize the gate from the other side. However, the \banes{} deliver a mighty counterstrike, and the \draconic{} \Dominators{} are crippled and sent into dormancy/exile from \Miith{}. Mortal \dragons{} now vie for control. Simultaneously, the \banes{} have again been mostly wiped out, bu the fertile \resphain{} thrive\dash decimated, but determined and able to rebuild. 







\section{The Age of Nom}
Mortal \dragons{} eventually found the \Dragon{} Kingdom of Nom. Initially the \dragons{} revile \Xserasshana{} and her cruel rule. But after a while they discover that even in their dormancy they can still draw some measure of power from the \Dominators. This turns into a religion. In order to justify this they invent glorious lies about the \Dominators{} and their race's history, in which \Xserasshana{} is the heroic liberator rather than the kinslaying usurper and tyrant. 

This is called the Age of Nom, but Nom does not reign supreme. In fact, much of the time the \Dragonking{} barely reigns at all. The \dragons{} fight each other as well as the \resphain{}. \Resphain{} also fight each other, for many of them turn from the \banes{}' rule. 







\section{The \Cuezcan{} war}
\target{Cuezca}
\index{\Cuezca}
After the end of the great \dragon-\resphan{} war, both sides are badly weakened, and the Veil is formed, leaving room for new cultures to blossom. After a thousand years or two, one civilization rises to greatness: \Cuezca. (I don't know what race the \cuezcans{} are. They might be renegade \resphain.) 

\Dragons{} and \banes{} alike recognize that \Cuezca{} is a power factor, so they both try to sway them to their cause. They woo the \cuezcans{} very directly, using all means. They also teach their magic and science to the \cuezcans{}. The result is a catastrophe: With their newly acquired power, the \cuezcans{} cause immense destruction, tearing great expanses of the Realms apart. \Cuezca{} is destroyed in the process. 

The \dragons{} and \banes{} realize that open war\dash especially open war using mortal pawns, armed with power they can't wield responsibly\dash is no good, and that if they continue this way, they will destroy the world, leaving nothing to rule. So by unspoken consent, they \quo{agree} that from now on, their war will be fought in the shadows, using unwitting pawns. 

The Cabal and the Sentinels of \Miith{} are formed and the Unspoken Covenant is adopted. The Covenant dictates that the Cabal and Sentinels strive to keep mortals ignorant of the true world that surrounds them. They must not know of the secret organizations' existence, the underground war, their own origins nor the nature of the Realms. Also, the progress of science (including magic) must be suppressed and kept in check, and the Cabal and Sentinels must not teach their arcane knowledge to outsiders. 















\section{Writing style}
An idea: The book will shift back and forth between two different styles. Half of it is in biblical/mythologial style, describing the origin of the \dragons{} as if a myth. The other half is in more normal style, more normally describing the events from the POV of lesser characters, such as Semiza and the Nagae. 



% \chapter{\FirstbanewarBook}
% \section{The Exodus from the Deep}
It is said that once, many millions of years ago, a race of alien creatures came to Mith. Out of the sky they descended, riding in their shining chariots of iron and borne upon wings of fire, and they were the Voyagers, and they came to despoil Mith and claim her for their own. 

Unto the bosom Mith did they descend, and in her lands they came to dwell. In the valleys they dwelt, and in the hills, and in the mountains, and upon the seas they did sail. And the Voyagers founded kingdoms and built cities, and vast were their cities and castles, and their great towers of steel and glass did reach into the sky and aspired to touch the stars. 

And the alien ones did spawn their young, and filled up the world with their get, and they did breed slaves to serve them and their spawn. And so the beings of Mith were brought low, and they were enslaved to serve the Voyagers and their spawn, and all Mith lay under the yoke of foreign rule. 

But beneath the seas, beneath the howling winds and the crashing waves and miles upon miles of dark waters there dwelt the \krakens. In their shining halls far beneath the waves, their glorious halls where no mortal may ever tread, they lay in deep sleep. For aeons and aeons had they slept, for they were the true gods of Mith and had been for ever and ever, the oldest and greatest of all being that exist. 

And the \krakens{} did awaken, and they saw that the Voyagers from the stars had occupied the lands of Mith and made it their own, and they saw that they had made of the denizens of Mith their slaves and conquered all that lay before them. And the \krakens{} were angered, and they were enraged. 

And the \krakens{} rose forth from the sea and came to the land where dwelt the Voyagers, and they did battle with them. And there was war on Mith, and the valleys and the hills and the mountains were ablaze with war. The Voyagers fought agains the \krakens; in their chariots of iron borne upon wings of fire they fought, and their spawn fought, and their slaves. And the \krakens{} fought against the invaders; with their tremendous power and wisdom and magic they fought, and Mith was ravaged and lain waste. 
The skies were blackened and the land was set afire and the seas were filled with blood. 

The \krakens{} fought against the invaders, and the Voyagers fought against the \krakens{}, but they prevailed not. The \krakens{} triumphed and were victorious, and they destroyed the Voyagers and all that they had made. They lay waste to the cities of the alien ones, and their towers of steel and glass they razed to the ground, and they slew the brood of the Voyagers, and they slew those who had become their slaves. Every single one of the Voyagers they destroyed, and nowhere was one of the aliens to be found alive. 

Thus did the \krakens{} vanquish the alien Voyagers who had come to despoil Mith and claim her for their own, and Mith was made pure again. 

Seeing that their work was done, the \krakens{} were pleased, and they returned to their forbidden homes in the deepest, darkest oceans. To their shining halls far beneath the waves, their glorious halls where no mortal may ever tread, they returned, and in their sacred halls they lay down and slept. And for aeons and aeons they slept, and they sleep still, for they are the true gods of Mith, the oldest and greatest of all being that exist. Before all things they were, and when all things come to an end they shall be, for they are the greatest of all gods, and they are eternal. 

But \Moroch{} was the youngest of the \krakens{}, and he slept not, but ventured forth from his splendid halls. \Moroch{} came into the great ocean, and he said unto himself: \ta{To ensure that Mith is never again enslaved by an alien power I shall leave behind guardians to guard her. I shall create children, and they shall fill the seas, and they shall be the custodians of Mith.}

And so \Moroch{} made his children and set them in the sea, and he said unto them: \ta{Ye are my children, and the name of your race shall be called the \nagae. Ye shall dwell in the seas, and ye shall multiply, and ye shall rule the seas forever, and ye shall protect Mith forever. And ye shall remember the \krakens, and ye shall remember me, for I am \Moroch, your creator.}

And \Moroch{} returned to his marvelous halls in the deep and lay down together with his kin, and he slept. 

And the \nagae{} remembered his words, and they remembered the \krakens, and they remembered \Moroch, their creator. And they did multiply, and they dwelt in the seas and built marvelous cities of coral, and they were the custodians of Mith. And the greatest and wisest of the \nagae{} grew huge in size and tremendous in power, and they were the \leviathans. And the \leviathans{} were the rulers of the \nagae{}, the keepers of wisdom and the legacy of the \krakens{}. And always the \leviathans{} remembered \Moroch, and always the \nagae{} remembered \Moroch, and they worshipped him, and he was their greatest god. They remembered his legacy, and the \nagae{} were the custodians of Mith for a million years.

Now, in the sea whose name is called Nag there lieth the coral reef whose name is Gyth'lrao'sribish. Vast is Gyth'rao'srimish, and it goeth on for miles upon miles, and in Gyth'rao'srimish is built the city of Vash'hythrai. A marvel is Vash'hythlai, reaching from the darkened depths to the sunny shallow waters, and her homes are decorated with blossoms of a hundred thousand colours. 

In Vash'hythrai there dwelt of yore Kiba'zenwar. A \leviathan{} he was, great in wisdom and power, and few were they that were his peers. One night as he slept in his house in Vash'hythrai, built of shining many-coloured coral, he dreamt, and in his dreams he heard a voice. And the voice spake unto Kiba'zenwar, saying: \ta{Kiba'zenwar, Kiba'zenwar of the \leviathans, hearken unto me. Know that I am \Moroch, thy creator, thy god.} And Kiba'zenwar hearkened, and he knew that the voice spake true, and that it was \Moroch{} of the \kraken{}, creator of the \nagae{} and the \leviathans. 

The voice spake unto him, saying: \ta{Kiba'zenwar, I have chosen thee, for thou art great in wisdom and power, and thou wilt serve me well. Thou must speak unto thy people, and any and all who will follow thee, them shalt thou bring with thee. And thou shalt go forth from the sea and unto the dry land, and there thou shalt dwell, together with them who will follow thee. 

And thou shalt forge an empire. Into the valleys shalt thou go, and thy people shall go with thee and build there cities, and they shall dwell in the valleys. Into the hills shalt thou go, and thy people shall go with thee and build there kingdoms, and they shall dwell in the hills. Into the mountains shalt thou go, and thy people shall go with thee and build there castles, and they shall dwell in the mountains. 

Thus I charge thee: Thou shalt forge an empire, and thy people shall fill the earth, and thou shalt rule over the valleys and the hills and the mountains, and thou shalt be the king of the land and the sky, and thou and thy people shall be the guardians of the land.}

And Kiba'zenwar awoke and rose from his sleep and knew that \Moroch{} his creator had spoken unto him. And he called together his fellows and bade them assemble. And when he saw that they had hearkened and come unto him, Kiba'zenwar addressed addressed the assembly of his fellows and spake unto them, and said: \ta{Hear me, ye \leviathans{} and \nagae{}, for I, Kiba'zenwar, have heard the voice of \Moroch{} our creator. This very night, in my dreams, \Moroch{} hath spoken unto me, and this he said: 

\subta{Thou must speak unto thy people, and any and all who will follow thee, them shalt thou bring with thee. And thou shalt go forth from the sea and unto the dry land, and there thou shalt dwell, together with them who will follow thee. Thus I charge thee: Thou shalt forge an empire, and thy people shall fill the earth, and thou and thy people shall be the guardians of the land.} 

Thus spake \Moroch{} our god unto me, Kiba'zenwar, and I have hearkened unto him. Thusly, I shall go forth unto the land, and I shall carve out a kingdom, and whosoever will follow me shall partake in that kingdom, and we shall rule as the guardians of the land.}

Now Daath'kwilthveir heard his words, and she was a \leviathan, one of the few who were Kiba'zenwar's equals in power. Now she rose...

...

\ta{For ages and ages uncounted have we dwelt in the sea, and for ages have we mastered it. All manner of creatures that swim and crawl in the sea bow before us and know that we are their kings.

But it is not right that we should reign only beneath the sea. This I say to you: Of Mith's creatures we are the greatest, and we should be kings not only of the sea, but of the land, and of the sky. I say this: We, the \leviathans, should ascend from our kingdom beneath the waves. We should rise into the sky, and we should conquer the sky and make it our domain. And we should descend upon the land, and we should conquer the land and make it our domain. For we are the greatest of all Mith's children, and all Mith should belong to us, for such is our birthright. This I say to you. I, \HesodN{}, have spoken.} 
% \section{Tiamat's wings}

Imnath'uri threw back her long neck and shrieked in pain as mystic power rippled through her body. 
But she faltered not, and though her body and mind did rail, urging her to release the magic and slink back into the water, she did not give in.
%But she quickly regained her composure, and, 
%repressing the urge to release the forces of magic she was weaving and slink back into the water. 
Imnath'uri kept her spells in check, and lifting her eyes again toward the stars she resumed her incantations. %, speaking words of power in the True Tongue. 

\ta{Nash kweg\rhh ash se'rom Ashfaleth, ram pur churdzad shor Ashfaleth khel...} Imnath'uri chanted the words of the True Tongue, and the \daemons{} of the night did hearken. Urnazregond could see them, swarming about her, flowing through her. Savage they were, like a school of ravenous piranha fish. With fury they ripped into Imnath'uri's body, as if they lusted to tear her to bloody shreds, and indeed blood stained the rock all around where she perched, and scales torn loose together with red lumps of flesh littered the ground at her feet. But Imnath'uri held the imps in a grip of iron, and they struck only where she bade them strike, tore only where she bade them tear. 

Beside him, Jeshrasab stared in silence, mesmerized by the marvel of what they witnessed. But while Urnazregond marveled also, his sense of wonder was tinged with fear. He turned to \HesodN. \ta{Great one, surely this is not right? See how they rip and bite into her body! We must interfere!}

\ta{Patience, young one. Wait and see.} \HesodNz{} eyes were fixed on Imnath'uri, not even flickering as he spoke; \HesodN{} the elder, who had once been Kiba'zenwar, and whom Urnazregond had not yet grown accustomed to call by his new name. 
He studied her as intently as did Jeshrasab, but his was not awe but appraisal... appraisal and approval. \ta{She has them in grip yet. I would see what she intends.}

Urnazregond turned again to Imnath'uri. She had already changed much, in imitation of \HesodN, her large and heavy body having elongated into a sleeker, more smooth form. Her tail, which had been high and flat and ending in a great vertical fin as on any Vlekkesh'sal, had become long and slim, tapering to a whiplike point. The long fin along her back had become a row of sharp spines, and her webbed hands had given way to delicate fingers. Her head had grown a long, narrow snout, and great horns had sprung forth and now curved back along her snaking neck. \tho{She is fair. Barely like unto a Vlekkesh'sal at all, but so beautiful.}

Now she dug her claws into the earth, her tail lashing the ground behind her, as tirelessly she chanted the words of power that bound the imps to her will. The ritual of reshaping had gone on for hours; it would not be long till sunrise. Their magic was of the True Tongue, the tongue of the stars, and so relied on the guidance of the stars, and as such they worked their spells more easily at night when the stars could be clearly seen. Urnazregond looked into her eyes and saw that they were empty, staring into space, into the Worlds Beyond\dash the worlds of the stars, and of the \daemons. 

\ta{Sheshem'bracul Tashrun Kal naero tzuerimachir, naero xirashassed, naero xerifoss...} She invoked the names of stars and \daemons, her spell slowly rosing to a terrible crescendo, and Urnazregond could hear the wailing of the imps, like the music of a choir infernal. 

\ta{She is almost done...} announced Kiba'zenwar\dash\HesodN. Beside them Jeshrasab mouthed words, but no sound came out. 

Now Imnath'uri rose on her hind legs, stretching her body and neck up as if reaching for the sky. She threw wide her forearms and screamed aloud, a scream that carried dreadful pain, but also sublime ecstasy. There was a cracking, ripping and splattering sound, and twin fountains of blood erupted from her back, near her shoulders. Urnazregond shouted in alarm and would have rushed to her side, had not \HesodN{} called out to him. 

\ta{Stay. Do not interfere,} the elder commanded. 

And so Urnazregond beheld in horror this last stage of his beloved's transformation\dash horror, but also awe. Stumps of flesh and bone now grew forth from the wounds in Imnath'uri's back, like a second pair of arms. They grew and grew, stretching longer than her arms and legs, and they split and grew many ways at once, branching off into several \quo{fingers} near the end. Still they grew until two limbs together spanned as wide as her entire length, if not wider still. And now skin appeared, connecting the spindly fingers to form two immense...

\ta{Wings...} Jeshrasab breathed. \ta{She is forming wings!} 

And all of them watched in amazement as blood and life flowed into Imnath'uri's new wings. At long last she fell silent, the \daemons{} slowly dispersing around her. 

The spell of metamorphosis was complete. 

Imnath'uri slowly opened her eyes to gaze upon her new form. In wonder she studied her new wings, flexing them and spreading them wide for the first time. Carefully testing all her limbs and joints, she crouched down, all muscles tensed. She set off, jumping many yards into the air. Then she threw wide her wings and gave a tremendous blow\dash and soared! Again she flapped her wings, lifting her body higher into the air. Delighted, she let rip a triumphant roar as she propelled herself yet further skywards. 

\ta{Magnificent,} said \HesodN, the two younger Vlekkesh'sala stunned into silence. 



But he saw that her eyes were empty, staring into space, into the Worlds Beyond, and he suspected that she would...

Draecchonosh

