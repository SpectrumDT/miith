\chapter{Factions}
\section{\Baelzerach}
\label{\Baelzerach}
The \daemonic{} \resphain{} who dwell in \Machai{}. 
They have severed all ties to the \banes. 
Sometimes they wage war against the \dragons, sometimes they work together. 

The \Baelzerach{} have reshaped their bodies into \daemonic{} forms, with fiery red skin, horns, wings, tails and/or goat-like legs. 
The \resphanesses{} are more \human-looking than their males, tho. 

\Ishna{} played a part in the founding of \Baelzerach{} and has a place in their mythology as a legendary figure, half hero and half \devil. 

The \Baelzerach{} enjoy to hunt, kill and eat in the \draconic{} manner (see section \ref{Dragon diet}), rather than \hr{Resphan diet}{eating submissive slaves like other \resphain{} do}. 
This is one of the things that causes other \resphain{} to label them as barbarians. 

\Baelzerach{} alone of the great dynasties has no \satharioth{} among its numbers, since they branched off from \KiriathSepher{} before \hr{Origin of Satharioth}{the \satharioth{} were created}. 
They do have \ketherain, since there has been interbreeding among the dynasties, but they are usually not referred to as such among the \Baelzerach, who reject the notion of a \KiriathSepher{}-based aristocracy. 















\section{\Banes}
\label{Bane}
\label{Banes}
% The \banes{} were originally one of several races that warred for control of \Erebos. They won in the end by being the coldest, most ruthless and most efficient, almost machine-like. 
% 
% They have cruelly exploited their homeworld until \Erebos{} is now a dying husk, almost all its life energy drained away. The \banes{} know that they cannot evolve further while still tied to \Erebos\dash indeed, \Erebos{} and the \banes{} it are doomed to decay and extinction. So the \banes{} are searching for a new planet to leech from, so that they may survive and grow in power. (See also section \ref{Fighting for survival}.)







\subsection{History}
\subsubsection{The \secondbanewar}
The \hyperref[Resphan rebellion]{rebellious \resphain} win and summon the \banes{} again. The \hyperref[Second Banewar]\secondbanewar{} begins. 

Eventually \hyperref[End of the Second Banewar]{everyone loses}. The \banelords{} are gone again. 



%\subsubsection{\Cuezca{} and the Shrouding}
%Or wait. Maybe the \banes{} didn't go into dormancy until later, after the \hyperref[Cuezcan Apocalypse]{fall of \Cuezca} and the \hyperref[Second Shrouding]{weaving of the Shroud as it exists today}. (This is also the point where the Cabal and Sentinels were formed. Or did they already exist before, manipulating the \Cuezcans?)

%See also section \ref{The Shrouding} about the Shrouding. 





\subsubsection{\Banes{} in the \cosmos}
Maybe the \banes{} have a \cosmos-spanning empire that has been expanding for thousands of years, conquereddozens of planets, absorbed the local life and used it to evolve themselves, and then having moved on, leaving behind a barren, dead ruin, infested with wretched, degenerate survivors. 

But according to \Daggerrain, they have never seized a planet as good as Mith. Mith is a great prize to them, and they will do everything to get it.





\subsubsection{Why the \banes{} want Mith}
\label{why the \banes{} want Mith}
They want Mith because it was settled and populated by the \voyagers, their own creators. 






\subsubsection{The return of the \banelords}
\label{Return of the \banelords}
\label{return of the \banelords}
Recently (only decades or centuries before Carzain's age), the \banelords{} have returned and resumed command of the Cabal, demanding that all \resphain{} bow to them. Not all \resphan{} lords are happy about this. 

The \dragons{} have long feared that the \banelords{} might return. 

\lyricsbalsagoth{Invocations Beyond the Outer-World Night}{
But, it is here written that one day, when even the War of the Lexicon and the cataclysmic Great Chaos War have faded to naught but distant memory, a great conflict shall be waged between the forces of Order and the dread avatars of the Z'xulth. \\
Vile fiends of the Outer Darkness, They-Who-Lurk-And-Breed-In-Limbo, the Dwellers in Eternal Shadow unleashed through The Gate to That Which Lies Beyond! \\
The Black Galaxy disgorges its malignant horrors! \\
Mankind shall suffer inestimably at the hands of these sinistrous black titans of maleficent Chaos!}









\subsection{A force of Entropy}
\label{Bane Entropy}
\label{Bane parasitism}
The \banes{} are a force of Entropy. This is a major theme. The \banes{} are inherently destructive and parasitic. They can sustain themselves and grow stronger only by feeding upon their own kind. 

This means that the \banes{} are doomed to stagnation and decline. 

The \banes{} are creatures of the cold, darkness and emptiness. They steal, absorb and swallow everything. This is unlike the \dragons, who \hr{Dragons radiate life}{radiate life and light alike}. 

\hs{\Daggerrain} and his master, the \hs{\Voidbringer}, understand this. They seek to improve their race and let the \bane{} people evolve, to achieve perfection. To do this, they must look outwards and find new sources of life, new cosmic forces with which to infuse their race. 

This is why they have set their eyes on Mith. Mith has the \hs{Heart of Mith}, a powerful source of life, ultimately based on Chaos. Drawing on the life-giving power of the Heart, and working together with \hyperref[Semiza]{\Semizaz} sorcerers, \hyperref[Semiza designs Resphain]{they were able to create} the \hyperref\resphain, who were intended as the new generation of \banes, empowered with the creative force of the Chaotic Heart. The \hs{\satharioth} took this a step forward and stole the Chaotic \xzaishannic{} power from the \dragons, making themselves even greater. They see themselves as the future of the \bane{} race, the heirs of the \bane{} legacy.







\subsection{Sex}
\hyperref\Daggerrain{} believes that one way to escape the Entropy that afflicts the \bane{} race is by adopting and mastering the art of sex. See, sexual reproduction is a method that utilizes Chaos (random selection of genes, random mutations) to improve life, taking the best of what exists and adding potential improvements. As such, it is superior to the \banesz{} method of asexual reproduction, where they can create only clones that need to feed on their own kind in order to grow. 

The \hs{Heart of Mith} is inherently tied to life, sex and reproduction. Sex is an inherently chaotic force/phenomenon. The \banes{} intend to harness sex and use it to evolve themselves towards perfection. 

\Daggerrain{} sees the \hs{\resphain} as the way to go, but \humans{} are useful guinea pigs. 

The \banes{} and \resphain{} conduct all sorts of sexual experiments. Those overseen by the \banelords{} are well-thought-through and for the benefit of their science. Those conducted by the \resphain{} and \resviel{} are often just depraved pleasure disguised as science. 

Remember to have evil, wicked, sadistic sex-experiments! 







\subsection{Physique}
\subsubsection{Horror}
The \banes{} are a \trope{CosmicHorror}{Cosmic Horror}. Even the \resphain{} fear them. Even \lesserbanes{} are frightening and loathsome, and \resphain{} shy away from them.





\subsubsection{Life cycle}
\label{Bane cannibalism}
The \banes are cannibalistic. 
There are several tiers of them, from the lowly \banespawn{} through \lesserbanes{} to \banelords. \Banes{} can rise in the ranks only by devouring the souls of other creatures, including (mandatorily!) other \banes. New \banespawn{} are created by sacrificing a \lesserbane{} or a \banelord. 





\subsubsection{Cold aura}
\label{Banes are cold}
\Banes{} are cold as death. Colder, in fact. Their body temperature is very low, below room temperature, near the freezing point. They spread deathly cold around them that seems to sap the life of creatures around them. 

This cold is one of the tell-tale signs that a \bane{} is nearby. When the \bane{} is submerged in \Nyx{} the cold cannot be felt, but as it surfaces, the cold spreads out like a draught. 





\subsubsection{Shrouded in darkness}
Because of their connection to \Nyx{} and \Erebos, when they appear in Mith \banes{} are always \hs{Shrouded} in an unnatural aura of darkness. 

\lyricsbalsagoth{Six Keys to the Onyx Pyramid}{
The fiends seemed inexplicably to
be an extension of the night, as if their misshapen bodies were actually
somehow composed of the darkness itself. Even as I gazed directly at them,
I found I could not truly focus on their stygian forms... their bodies
appearing to shimmer and shift like the ripples of a heat-haze upon an arid
plain.}





\subsubsection{Giant form}
\Banelords{} are able to transform into a gigantic, monstrous \quo{combat form}\dash useful if you need to fight \dragons{} or other large monsters. 

The monstrous form looks bat- or ray-like. Like the \quo{Balrog} in the movie \emph{Eragon}. 







\subsection{Bane technology}
\label{Bane technology}
\label{\Bane{} technology}
The \banes{} command large reserves of spaceships and other technological artifacts salvaged from the \voyagers{} on \Erebos, and perhaps also other cultures whom the \banes{} have conquered. These were remnants of the \hyperref[High-tech civilization]{interstellar civilization that once existed}.

During the \hyperref[First Banewar]{\firstbanewar}, an important edge that the \banes{} had over the Mithians was their superior technology. 
%They commanded reserves of technological artifacts salvaged from the \voyagers{} on \Erebos, and perhaps also other cultures whom the \banes{} had conquered. 

%, since the \banes{} themselves were afflicted with stagnation and had great difficulty creating truly new things. 


The \banes{} do not understand the technology. They are unable to create new things, or even replicate existing ones, and are hard pressed to just repair and maintain what they have. 
This is partly because the \voyagers{} were infinitely more advanced than their \bane{} spawn, and partly due to their being \hyperref[Bane Entropy]{a force of Entropy, afflicted with stagnation and decay}. 
After all, 
\hs{\Semiza} had to 
\hyperref[Semiza designs Resphain]{help them design and create} their 
\hyperref[Resphain]\resphain{}.

The \hyperref\resphain{} are \hr{Resphan technology}{much more creative}.















\section{The Cabal}
\label{The Cabal}
\label{Cabal}
\index{Cabal, the}
\index{Cabal, the!Cabalist}
The Cabal is an  underground organization that secretly serves the \resphain{} and the \banes. Its members are called Cabalists.

%The Cabal is an underground order that run by and serving the \banes{} and their loyal \resphain.

The Cabalists are arranged in a strict hierarchy. They can and do backstab each other, but will do so only after careful planning and conspiration. Diplomacy and social relations are very important when one means to advance.

In contrast, the Sentinels are far more chaotic and violent. Their structure is more anarchistic, more overtly based on bullying and intimidation. Brute force and violence means more, and open combat between rivals is more common.









\subsection{Circles}
\label{Cabalist circles}
The hierarchy of the Cabal is arranged in \quo{circles}. The innermost circles are the highest ranks. 

The first circle counts as members only the very highest ranking \resphain{} and the \banelords. All \satharioth{} are of the first circle. 

Almost everyone in the first seven circles are \resphain{} or \banes. Almost all \resphain{} belong to the first nine circles. 









\subsection{Ordo ab Chao}
According to \DIBiggestSecret, the Freemason Society (who are, allegedly, run by the reptillian Babylonian Brotherhood) have as their motto \quo{Ordo ab Chao}, \quo{Order from Chaos}.

Maybe the Cabal should have a similar motto. 









\subsection{Lictor-like people}
\label{Lictors}
Some people who serve the \banes{} decay and shrivel, coming to resemble living corpses, dead and rotting for many years. Or pale, bloated maggots. Or sickly lepers. 

Compare to the hearse-driving man in \authorbook{\RWChambers}{The Yellow Sign}, or the Lictors in the RPG \emph{Kult}, or the priests in the movie \emph{300}.

Their voices are hoarse and inarticulate, almost bestial.

Note that they only look like this in their true, un-\hs{Shrouded} forms. The Shroud makes them look like healthy, normal \humans.

These Lictor-like people are thoroughly loathsome and repulsive. 

Maybe they degenerate because they are weak of mind and \hr{The price of magic}{cannot bear the pure energy of \nieur} that they channel.





\subsubsection{Charcoal hates them}
\hs{Charcoal} has to deal with the Lictor-men. He hates them and is disgusted by them. But some of them are of as high a rank as he. 

He thinks to himself: \tho{\Qliphoth, I'm glad I haven't degenerated into one of them. I'm not weak like those people.}









\subsection{The Missionarium}
The Missionarium is the division of the Cabal charged with controlling people's thoughts, beliefs, morals and emotions. They manipulate religions and the like. 

Compare to the Bene Gesserit Missionaria Protectiva from \authorbook{Frank Herbert}{Dune}. 









\subsection{Encouraging heroes}
The Cabal actively encourages would-be heroes to go out and do heroic deeds. See section \ref{Cabal encourages heroes}. 







\subsection{Fake resistance within the Cabal}
For new Cabalists, there is a period of \quo{trial membership}, which can last for several years. 

Also, to weed out potential traitors, there exists a fake \quo{resistance} within the Cabal. They allegedly work against the Cabal and try to foil its evil plans and do good instead. But in reality, the resistance is a hoax. It is controlled by the Cabal and used to lure out and identify potential dissidents within the Cabal, who might consider joining or forming an actual resistance. They are introduced to the fake resistance as a test, and if they join, or just neglect to inform their superiors, they are usually judged unreliable and killed off. 

Compare to the fake resistance in \authorbook{George Orwell}{1984}.















\section{Cosmic gods}
\label{Cosmic gods}
\label{Cosmic god}
\label{cosmic gods}
\label{cosmic god}
\label{Klatrymadon}
\index{cosmic gods}
I need to have some immensely powerful, mystic forces. Enigmatic gods whose motives are unknowable, but whose names are invoked in spells, and who will sometimes answer. Kind of super-\Qliphoth. 

They should be invoked early on in the story, side-by-side with \Ishnaz{} name, to give him mythical status. 

Compare them to Klatrymadon and Zuranthus, Kur'oc and Gul-kor from the Bal-Sagoth mythology.

\lyricsbs{Bal-Sagoth}{Summoning the Guardians of the Astral Gate}{
Ka-kur-ra, I summon thee.\\
Zul'tekh Azor Vol-thoth.\\
Mighty Xuk'ul, arise.\\
Kur'oc, Gul-Kor, come forth.

The threshold looms, \\
(the star-way between dimensions stretches before me...) \\
The Gate To That Which Lies Beyond yawns wide... \\
Unspeakable forces gibber and pulsate in the Outer Darkness... \\
Elder horrors dwell here, things which were ancient and revelled in sublime galactic malevolence when even Xuk'ul was naught but a bloated cosmic maggot, writhing and suckling at the breast of its amorphous mother... \\
They-Who-Lurk-And-Breed-In-Limbo... \\
the squamous sovereigns of the elder void!}

Perhaps the cosmic gods have a Nyarlathotep-like figure who is their representative. 









\subsection{A class above the \xss{} and \voyagers}
The \xss{}, \banelords{} and \voyagers{} are well above \dragons{} and \resphain{} in power, but they remain within the bounds of imagination\dash for the master races, not necessarily for mortals. 

But the cosmic gods are a class above them. They are as far above the \xss{} as the \xss{} are above \dragons{} and the \dragons{} are above \scathae. 

Comparing with the Cthulhu Mythos by H.P. Lovecraft and others:

\begin{itemize}
  \item Cosmic gods correspond to Outer Gods.
  \item Voyagers correspond to Elder Gods. 
  \item \Xzaishanns{} correspond to Great Old Ones. 
\end{itemize}








\subsection{Creating the \banes}
Possibly, the \banes{} were twisted by the design of a cruel cosmic god. See section \ref{Cosmic god creating the Banes}















\section{\Cuezca}
\label{\Cuezcan}
\label{\Cuezcans}
The \cuezcans{} saw their civilization destroyed by the war between the \dragons{} and \banes{}, and they hold a grudge. They work to play the Cabal and Sentinels against each other, killing and destroying as much of each other as possible. In the end, the surviving \cuezcans{} want to revive their empire and destroy the Shroud. 


















\section{The Cult of the Worm}
\label{Cult of the Worm}
The Cult of the Worm is a cult that worships a form of \hs{\KhothSell}, the \firstgendragon{} goddess of Death. The cult is underground but not exactly secret. Everyone knows it exists. Governments, the Iquinian church and the Cabal try to suppress it because it is subversive. 

Worshippers of the cult at times sacrifice their own limbs and organs, mutilating themselves for \quo{salvation}. This is probably misguided, since \hs{\KhothSell{} is very alien}. 

The cult knows about \hs{the Disease} and seek to twist and use it for their own purposes. 









\subsection{Reapers}
\label{Worm Cult reapers}
The priests of the cult function as \quo{reapers}. They haunt battlefields, slums, \hs{asylums} and other places where people die or suffer, and offer solace and salvation in exchange for loyalty and worship. Thus they prey on and recruit from the ranks of the dead and dying. The dying are easy to enlist. The dead less so, because they are harder to communicate with and less able to think clearly\dash but then, enlistment needs not necessarily be voluntary. Dead souls can be bound and enslaved.

Carzain \hr{Carzain sees reapers near Forklin}{sees some of these} after the battle at \Forklin.

If they must fight, the reapers kill with disease. Like the Silver Brethren from Stephen Marley's books \emph{Spirit Mirror} and \emph{Mortal Mask}. 


















\section{\PDaemons}
\Pdaemons{} are physical creatures from \Machai{} or other \chaotic{} realms. 





\subsection{Mighty \daemonic{} races}
Have some mighty races of \daemons{} that can only be contacted and bargained with, rarely bound. Compare to the \quo{star-spawn of Cthulhu} in the RPG \emph{Call of Cthulhu}. 





\subsection{Bat-like \daemon}
Have some minor \pdaemons{} with a mouth filled with a broad row of dagger-like teeth\dash a truly wicked smile. Inspired by Clive Barker's \emph{Hell's Event} (from \emph{Books of Blood II}). 

Also, it has batlike ears. 















\section{The Dark Crescent}
\label{Dark Crescent}
A an \Azmith-spanning underground order or cult led by \Psiotai. 

It is less covert than the Cabal or Sentinels. Most people know the order exists. The Order is widespread and has much power in certain parts of Belkade, despite the best efforts of the Iquinian churches at suppressing it. The cult is capable of raising a quite impressive army when it needs to (but its enemies do not know this, only suspect). 

Compare it to the organization of the Black Moon in \FLuneNoire, led by Haazeel Thorn. 

Is the Dark Crescent connected to the actual \hs{moons} of Mith? 







\subsection{The physical Dark Crescent}
The Dark Crescent is actually a physical thing. It is a flying ship shaped somewhat like a crescent moon. \hs{\Psyrexz{} throne} is located on it. 

Perhaps it is shaped like an ashen claw, giving rise to the name of the Dark Crescent knights. 

Sometimes the Dark Crescent can be seen in the sky. But only with difficulty, since it is black. 







\subsection{Dark knights}
\label{Dark Crescent Knights}
Have some sinister knights of the Dark Crescent. Like \hr{Dark knights of Mystraacht}{the dark knights of Mystraacht}. 

Compare to the Lords of Negation in \FLuneNoire. Maybe give them a similar title. 

Maybe they are the \quo{Ashenclaw knights}. 

Maybe they ride \hs{\vreiiden}.

Maybe there are undead knights who ride undead \vreiiden.

Maybe they resemble Revenants from the game \emph{Warcraft III}. 























\section{\Dragons}
\label{Dragons}
\subsection{Factions among the \dragons}
Remember to have fractions among the \dragons{}, and Sentinels in general. Their history must be as diverse and as bloody as that of the \resphain!

Speaking of which... how much of the Sentinel organization is controlled by \dragons? There are also \rachyth, remember. And the \Baelzerach, they might be Sentinel-allied, too. And there are groups within the Sentinels who do not work for \dragons{} at all, but worship their own gods, and are just in because they want to get rid of the \banes{} and \resphain{} (this is a strong argument for any \scatha{} or \rachyth). 

And there are cults who seek to resurrect the \thzantzais. Perhaps there are people with \thzantzaic{} blood. People, or descendants of people, who partook in \Tiamat-tachi's original ritual of summoning the \thzantzaic{} power. These people were meant as sacrifices, but a few of them survived and were imbued with \thzantzaic{} power. They fled and hid, but their descendats live on, as does their hatred for the \draecchonosh. 

And then there are the \ophidians{} and the \nagae{}, who may sometimes work with the Sentinels. 









\subsection{What they represent}
\label{Dragons radiate life}
The \dragons{} are a force of Chaos: Destruction and creation alike. They are associated with light, fire, storm and lightning. They radiate life, energy and passion, but also fury and destructiveness. Like \hs{Nature}. 

This is in contrast to the \banes{} and \resphain, who are \hr{Bane parasitism}{dark and parasitic}. 









\subsection{\Draconic{} diet}
\label{Draconic diet}
\label{Dragon diet}
\Dragons{} are natural predators, used to eat anything that moves. But their eating habits differ from those of the \resphain{} (see section \ref{Resphan diet}). 

\Resphain{} are cannibals, eating \resphan{} meat alongside that of all other intelligent creatures they can catch. In contrast, \dragons, despite their chaotic nature and their tendency to war amongst themselves, have a universal taboo against eating the flesh of other \dragons{}, \ophidians{} or \rachyth. A cannibal \dragon{} would be seen as an abomination by his fellows, and would be hunted down and slain. 

Where \resphain{} love to eat enslaved creatures who willingly submit and let themselves eat, \dragons{} live for the thrill of the hunt. They want their prey to flee and fear for its life. When they catch their prey, they want to be able to taste its fear, pain and anger, for such feelings are manifestations of life itself, and \dragons{} eat life, whereas \resphain, in a sense, eat death. 







\subsection{Appearance}
\label{Draconic appearance}
With their immortality and Chaos-born power, \dragons{} radiate an impression of ancient, alien might. Their \ophidian{} eyes are cold, baleful and unfathomable. 

\lyricsbs{Steven Erikson}{Reaper's Gale}
{... we looked to the east, and there saw, rising vast and innumerable on the cloud-bound horizon, \dragons. Too large to comprehend, their heads above the Sun, their folded wings reaching down to cast a shadow that could swallow all of Drene. This was too much, too frightening [...] for their dark eyes were upon us, an alien regard that drained the blood from our veins, the very iron from our swords and spears.}







\subsection{Physiology}
\label{Dragons have three hearts}
\Dragons{} have three hearts. 







\subsection{Immortality}
\label{Draconic immortality}
\label{\Draconic{} immortality}
\Dragons{} are immortal, but their immortality is different from \hyperref[Ophidian immortality]{that of the \ophidians}, which is based on the shedding of skin. 

This has something to do with \hyperref\KhothSell, who styles herself the \draconic{} goddess of Death. But how does that really work?

See also \hyperref[Malach immortality]{\Malach{} immortality}. 







\subsection{Forgetfulness}
\label{Dragons have forgotten}
In the \hr{Ophidian golden age}{Golden Age of the \ophidian{} civilization},
the \dragons{} were even greater than they are today. But the traumatic and cataclysmic \hs{Second Shrouding} struck even their great minds a hard blow. They have forgotten much of what they once knew, remembering only fragments of their tremendous science and magic.  

Many of the \daemons{} that once served as the \dragonsz{} slaves have turned on them, and the \dragons{} have forgotten then spells to command them. 

All of \hyperref[Secherdamon's research]{\Secherdamon's research} is not just for discovering new things, but very much also for rediscovering the knowledge they once possessed. 




\subsubsection{\Ophidian{} ruins}
There exist some ancient ruins from the \hs{\ophidian{} civilization}. The \dragons{} study them in hope of rediscovering something of their own past. 

But they are guarded.

\lyricsbalsagoth{Unfettering the Hoary Sentinels of Karnak}{
The Coptic papyrus states that, upon the walls of the pyramids and the temple were inscribed the mysteries of science, astronomy, geometry and physics; inscriptions of unknown peoples and lost civilizations whose lore was carved into the stone to preserve it from the ravages of the great deluge.\\
The surviving knowledge of long forgotten antediluvian races!\\
Aye, prudent Surid, heeding the warnings of his priests, erected certain repositories of long forgotten knowledge to withstand the first great flood, and then an all-consuming fire which was prophesied would come from the sky.\\
Masoudi, in the tenth century, described automata; titanic guardians of stone and metal which were placed to guard the treasures and the entombed lore, and which were tasked to destroy all those deemed unworthy, all those who dared enter the chambers unbidden.\\
I see them!\\
The hoary sentinels of Karnak are unfettered!\\
Rising from their sandy tombs to smite the intruder, the raider and the interloper with righteous fury!\\
And what is this... was there once a glimmer of life within the sightless stone eyes of the Theban guardian?\\
Does the silent watcher of Giza even now descend from its granite dais to once more stalk the shifting sands on carven claws?
}









\subsection{Madness}
Even \Tiamat{} and \ApepNesthra{} feared \hr{Tiamat maps the way to Machai}{the cosmic truths that they discovered}. 
And still today, the \dragons{} know that they are balanced on the edge of an abyss of \hs{madness}. 

\lyricsbs{Limbonic Art}{The Yawning Abyss of Madness}{
The yawning abyss of madness.\\
A cryptic slaughter by hate.\\
Darkness is the only survivor\\
as evil dominion terminates.\\
The yawning abyss of madness.
}

\lyricsbalsagoth{The Ghosts of Angkor Wat}{
I have concluded that these
perceived parallels and their possible significance carry me ever closer to
the centre of this great global web, the strands of which I have been
traversing in my long quest for enlightenment, and yet I now fear that the
owner of this web has surely felt the tremblings I have caused along its
delicate filaments, and may well feel compelled to investigate the
disturbance...
}

Some of them, such as \Ishnaruchaefir, meet this fear with a \quo{Devil-may-care} daredevil attitude. 









\subsection{Ancient homeland}
There exists \hr{fallen \dragon-land}{a place that once used to be the capitol and homeland of the proud \draconic{} empire, but now lies in ruins}.









\subsection{\Rachyths}
The \rachyths{} are a race of \draconian{} humanoids. 

\begin{itemize}
  \item Perhaps they are \dragon/\scatha{} hybrids. 
  \item Perhaps they are descendants of the old \hs{\ophidian{} humanoids}. 
  \item Perhaps they are an attempt to recreate the old \ophidian{} humanoids, who are nowadays extinct.
  \item Perhaps they \emph{are} the old \ophidian{} humanoids, who are thus not extinct at all. 
\end{itemize}









\subsection{Religion: Dead gods}
\label{\Dragons{} worship dead gods}
The \dragons{} worship the corpses of dead gods. These are both the \hr{Dead \xss}{dead \xss} and the \firstgendragons{} (\Tiamat-tachi). According to myths, these corpses will one day awaken to new life. 















\section{Geicans}
%\section{Clan Geican}
\label{Clan Geican}
\label{Geican}
\label{Geican philosophy}
The Geicans (or, at least, some Geicans) believe that philosophy and science should be free and unhindered by morality. 

Some of them believe that philosophy and science \emph{should} be dark, scary and boundary-crossing. Only by abandoning everything known and safe and throwing yourself out in the deep water of the unknown, in the vast, dark emptiness of the inhuman universe, can you gain new insight. 

They believe that if you do not confront the frightening and unknown, you will continue to live in its shadow, on its mercy. As its slave or as its prey. 

Or are they a rare example of the triumph of mortals? Then again, the Geican democracy is a pretty corrupt, mafia-like nepotist system.









\subsection{Free thinking: An embattled kingdom}
\label{Geica is embattled}
Why are the Geicans allowed to be such free thinkers? %Are they under someone's protection? Perhaps \Kezerad{} or the \Cuezcans? 

Well... Geica is embattled by the master races. Both factions seek to control the clan \hr{Master races seek to control magic}{and its magic}, and they antagonize each other, so that in the end no one controls Geica. This is one of the reasons why there is so much free thinking going on there. 









\subsection{The Royalist Faction}
\label{Royalist Faction}
The underground faction that serves \hr{Belzir}{\Belzir} and aims to restore her to life and power. 

The faction is actually being manipulated by the \hs{Sentinels}. 

Notable members include \hs{Hayad}, \hs{Shereid} and, ultimately, \hs{Carzain \Shireyo}. 















\section{\Ghobaleth}
\label{Ghobaleth}
The \ghobaleth{} (singular \ghobal) are a race of monsters native to \Erebos. They are enormous worms, growing as much as hundreds of metres long. 

The name is inspired by the \Qliphah{} Golab, in Cabbalah mysticism.

Compare them to the Dholes in \authorbook{H.P. Lovecraft}{The Dream-Quest of Unknown Kadath} or, to a lesser extent, the sandworms of \authorseries{Frank Herbert}{Dune}. 

The \ghobaleth{} are servitors of the \banes. In this aspect, compare them to the shoggoths from \authorbook{H.P. Lovecraft}{At the Mountains of Madness}. 

In fact, they are reshaped \banes. Or perhaps many \banes{} merged together in a single monstrous body. 

They are, to some extent, a manifestation of the \hyperref[Bane Entropy]{decay, corruption and Entropy} of \Erebos{} and the \banes. It is their kind that have carved out \Erebos, leaving only a mass of twisted spires and no ground. The also dug out \Nyx. And now they are in the process of undermining Mith, turning it into a dead husk, an empty shell. 

\Zeirathz{} monsters in Malcur (see section \ref{Zeirath's creatures}) are \ghobaleth. 

An idea is that the \ghobaleth{} feed on magical power, so \Zeirathz{} \ghobaleth{} are only truly formidable if the Sentinels show up with some great power. Or maybe they feed on \vertices. That might be evil. 

I need to think more on how exactly this works. 

They are worm-shaped, but can create pseudopods/tentacles when needed, or even halfway dissolve their own bodies. This makes them appear like a writhing mass of dozens of worms rather than a single worm. Especially when seen through the Shroud. 

Perhaps some \ghobaleth{} are free, not under Cabal control. These wild ones grow much larger than their \quo{tame} brethren, perhaps hundreds of metres long. Like \emph{Dune} worms, or dholes. 















\section{Gods}
\subsection{\Human{} gods}
Have a race of \human{} gods. They might be called the \quo{Titans} or something like that. 

They are actually descendants of the \Kezeradi. They may or may not know of their own origins, and they may or may not still have ties to \Kezerad. 

Daxian and Isxae are the foremost among their number. The Imetric god Eoncos is also one of them. 







\subsection{\Scathaese{} gods}
And remember to have some \Ortaican{} and \Shurco{} gods. (Turco? Thurco? Sturco? Durco?)















\section{\Gorgoroses}
\Gorgoroses{} are huge reptiles, terrible monsters from the ancient past. A \gorgoros{} resembles a large theropod dinosaur like \latinname{Tyrannosaurus} or \latinname{Allosaurus}, but with far larger and stronger forearms ending in huge, wicked claws, like the Behemoths of the \emph{Heroes of Might and Magic} games. 

Perhaps they are related to \dragons{} in some way. 

They have great mouths filled with tons of long, sharp, wicked teeth. Almost like a deep sea fish. 

Compare to Godzilla. 















\section{\Humans}
\Humans{} were originally created as a slave race by \Semiza-tachi. 
The \hr{Humans fail}{experiments failed}, and the test subjects were supposed to be killed off. 
But after \Thanatzil{} was slain the \humans{} escaped into the wild and bred true. 
The \resphain{} would later rediscover them and adopt them as their servant race. 

Being a failed slave race, \humans{} are sucky and measly creatures, and few of them are worth anything. 
The Cabal has tried much to improve on the \humans{} over the millennia, but with little success. 















\section{\Iquinian{} Church}
\label{\Iquinian{} Church}
\label{Iquinian Church}
A religion that worships \iquin{} as the source of all that is good. The \hs{Redcor} form one branch of the Iquinian Church. 









\subsection{Destroying information}
The Church is fond of destroying books and other materials dealing with the occult and other \quo{evil} things. They portray it as bad knowledge that people were not meant to know, and whose existence can only cause harm. But in reality, the Church is being manipulated by the Cabal, who wants \hr{Destroying information}{to keep people ignorant}.















\section{\Kezerad}
\label{Kezerad}
\label{Kezeradi}
A \resphan{} faction who, out of moral scruples, defied the \banes{} and created their own kingdom, supposedly one based on justice and good. 

Their traditional colours are gold and bronze. 







\subsection{History}
\Kezerad{} was once a beautiful, happy, almost utopian kingdom of \resphain, \humans, \nephilim{} and possibly other creatures. They created their own pure and good energy source, \iquin, which was independent of \Erebos{} and Chaos alike and instead drew upon the natural energy of Mith and its Heart. Perhaps they had help from elder \ophidians{} and/or other wise creatures in establishing this. 

\label{Fall of Kezerad}
But their fellow \resphain{} were unwilling to accept them, so they waged war on \Kezerad. 

Was this before or after the final \hs{Shrouding} and the \hyperref[Cuezcan Apocalypse]{\CuezcanApocalypse}?

Perhaps they were already at war with the \Baelzerach{} or another foe, and their fellow \resphain{} backstabbed them.

Anyway, \Kezerad{} was invaded by \resphan{} and \bane{} armies and was destroyed. Compare this to the destruction of the Blue Dragonflight in the \emph{Warcraft: War of the Ancients} books by Richard Knaak. 

Their leaders were transformed into the terrible \Sephiroth{}, and their \iquin{} was corrupted and twisted into the loathsome soul prison it is today.

%They fought against their former masters and their fellow \resphain, and eventually they were destroyed. Their leaders were transformed into \Sephiroth. 







\subsection{Connection to Silqua}
Maybe the fall of \Kezerad{} occurred very late, during \hs{Silqua}'s lifetime. 

The \hyperref[Silqua feels angels]{angels who contacted Silqua} where the \Kezeradi. They knew that she was an amnesiac \malach, but they also knew that she was not fully evil and had been somewhat sympathetic to their cause. They hoped to sway her to their cause and, through her, create a good empire on Mith, where peace and goodness could reign. 

The \Sephiroth{} that Silqua felt were the original \Kezeradi{} \Sephiroth, before they were corrupted by the other \resphain. 

But the plan was interrupted. The evil \resphain{} invaded and destroyed \Kezerad, twisted the \Sephiroth{} into abominations, and usurped the \Kezeradiz{} gentle quest. They seized control of the Vaimons and used them to create a \resphan-backed evil empire. 

This happened late in Silqua's life. Before, she had had premonitions, evil dreams and visions that this was going to happen\dash although she didn't understand it, since she didn't quite know who the angels were. She only knew that there was a conflict between light and dark angels. And she was afraid of her premonitions. 

When \Kezerad{} finally fell and the \Sephiroth{} were corrupted and enslaved, Silqua went mad. Perhaps this happened while she was captured by the enemy, or perhaps her madness \emph{made} her fall into enemy hands. After this, she was raped, tortured and finally killed. 







\subsection{Telepathy}
\label{Kezeradi telepathy}
Back in the day, the \Kezeradi{} shared a telepathic bond and were all intimately bound to each other. They even shared this bond, to some extent, with their \human{} and \nephilic{} subjects. This empathy is one of the reasons why \Kezerad{} was such an enlightened realm. 

This could also be a weakness in war, since the bond meant that they shared pain, so the massacre of one \Kezeradi{} village or city would create a mental backlash that could demoralize the rest. (Compare with the destruction of the planet Alderaan in \emph{Star Wars IV: A New Hope} and Obi-Wan Kenobi's remark that: \ta{It is as if a million voices suddenly cried out in pain, and then fell silent.})

And, even worse, when the \Kezeradi{} lords were captured and transformed into the horrid \Sephiroth, the bond persisted, giving the \Sephiroth{} great mental power over the remaining people of \Kezerad. The invaders had planned this well indeed. 

After the fall of the \Kezeradi{} civilization, the conquerors utilized the psychic bond to hunt down the remaining \Kezeradi{} and eradicate them. In order to survive, those who escaped had to deaden their telepathy and block out all empathic feelings from their minds. (Compare this with the Protoss Dark Templar from the game \emph{Starcraft}, who allegedly severed their nerve endings to forever off themselves off from the telepathy of the Protoss race.) This gradually turned the survivors into bitter husks, desperately longing for intimacy but knowing that to give in to feelings is to be destroyed. 

%The \Kezeradi{} still have a deep connection to the \Sephiroth, since the \Sephiroth{} are each forged around a core of a \Kezeradi{} soul. See, back in the day, the \Kezeradi{} shared a telepathic bond and were all intimately bound to each other. 







\subsection{\Kezeradi{} today}
There are surviving \Kezeradi. They are in an uneasy alliance with the \cuezcans, since both want to see the \bane{} faction destroyed. \Saynor, the \scathaese{} chaos sorcerer who is Curwen's second-in-command but will betray him, is a \cuezcan{} in disguise and working to free the \Sephiroth. 

The \Kezeradi{} have an image of \quo{fallen angels} about them. Not \quo{fallen} in the usual sense of having turned to evil, but in the sense that they were once an idealistic people, believing in good and beauty, but have become hardened, bitter and disillusioned. They look angelic, but harrowed: Bright-coloured skin and great feathered wings, but the wings are tattered and torn, their skin deathly pale or blotched and discoloured, and their once beatific visages are grim, contorted from millennia of grief, pain and rage. 







\subsection{\Sithiyacaan: The last \Kezeradi{} prince}
\hs{\Sithiyacaan} is a great hero who is the last surviving \Kezeradi{} lord. 







\subsection{New \Kezerad}
Some surviving \Kezeradi{} have built a new \Kezerad, a new beautiful, harmonious kingdom. It is much smaller than the original \Kezerad, but good. Once in a while they are able to rescue someone and bring them there. It's almost a Tanelorn-like place (as in Michael Moorcock's stories). This is one of the silver linings of the story. 

\hs{\Sithiyacaan} knows about the place, but he does not know its location, and he can never go there. It would make him remember too much, and the \Sephiroth{} would gain access to his mind and learn the place's location and come to destroy it. 

This causes him great distress. 















\section{\KiriathSepher}
\label{\KiriathSepher}
\KiriathSepher{} is the oldest and most traditional of the \resphan{} factions. 
They are loyal to the \banes. 

\label{High Lord of Kiriath-Sepher}
%They are ruled by a High Lord (like a king). 
The High Lord of \KiriathSepher{} is \hyperref[Azraid]\Azraid. 

The traditional colours of \KiriathSepher{} are white and silver. 
After the fall of \Kezerad, \KiriathSepher{} has also adopted the colours of gold and bronze. 

%They typically dress in white, silver and golden colours. 
The other factions criticize them for having adopted too much of the \Merkyran{} imagery and aesthetics, and perhaps even their philosophy. This is a result of \hr{Azraid adopts Merkyran imagery}{\Azraidz{} policy}.

%The armies of \KiriathSepher{} are clad in white, silver and gold. 















\section{\Malachim}
%In Vaimon metaphysics, the \Malachim{} are a class of \Archons{} that may incarnate as \humans{}. An incarnation of a \Malach{} is called a Scion. 
The \Malachim{} are \resphan{} lords who has left their \resphan{} bodies to incarnate again and again as \humans{}. 







\subsection{Immortality}
\label{Malach immortality}
The \Malach{} model of immortality through rebirth is intended as an improved form of the \hyperref[Ophidian immortality]{\ophidian{} shedding of skin}. It's an attempt to achieve immortality without stagnation. It's also inspired by \hyperref[Draconic immortality]{\KhothSellz{} project of \draconic{} immortality}. 

They derive sexual power from this, too, because whenever a \malach{} is born anew, he gains some new, improved lifeforce from his parents. Compare this with the Xenomorphs of the \emph{Alien} movies, who improve themselves by stealing the genes of their hosts. 

Many women die in giving birth to a Scion, because the \malach{} drains too much lifeforce from its host. 







\subsection{Vaimon view}
In Vaimon metaphysics, the \Malachim{} are considered a class of \Archons{}. 







\subsection{Scions}
\label{Scion}
\label{Scions}
A \human{} incarnation of a \Malach{} is called a Scion. Each Scion retains some memories of his previous incarnation. Typically these memories are locked away at birth and only awaken later, triggered by some massively emotional event. (For example, Carzain \Shireyo, a Scion of Ramiel, has his predecessor, \VizicarFull, awaken when Carzain fights his first battle to the death and kills a man.) 

From that point, the Scion has a split personality with two distinct persons inhabiting the same body. If the two personalities get along, they will absorb traits of each other and eventually merge into a single character. If the two do not get along, the Scion will degenerate into a mood-swinging maniac.

%There is a small number of \Malachim{} known. Ten or so. They include Ariel, Sachiel and Ramiel. 

The names of the \Malachim{} are Ariel, Nelchael, Ramiel, Sachiel, \Belzirmalach{} and more. A Scion does not necessarily remember his true name, but the Vaimons possess the knowledge to research a Scion's mind and ascertain his \Malach{} identity. 







\subsection{History}
Some of the \Malachim, including Ramiel, are of the \bloodresphain. 

%He was one of the original \bloodresphain{} who drank the blood of \Astorglax. 

I don't know exactly what the original purpose of the \Malachim{} was, but somewhere, the process went wrong, and the \Malachim{} all had their memories of their previous lives as \resphain{} erased. They now recall only scattered fragments of their old lives, typically in fever-like dreams and \deajvus. 

%Like all the other \Malachim, Ramiel has lost his memory of his previous life as a \resphan{} lord, recalling only scattered fragments, typically in fever-like dreams and \deajvus. 

The \Malachim{} were betrayed by \TimnathSerah, who were opposed to the project. There were \Timnaths{} among the \Malachim, but these were tokens to be sacrificed so as to not arouse suspicion. The \TimnathSerah{} leaked secrets of the spell to the Sentinels, enabling them to fuck it up and corrupt the \Malachim. 





\subsubsection{The \Malachim{} lose their memory}
The \malachim{} lose their memory and descend into amnesia and madness. 
They feel great pain, anguish, sorrow and hate over this betrayal.

\lyricslimbonicart{Purgatorial Agony}{
Experimental malediction.\\
Destructive minds.\\
Seeking places no living can find.\\
Awaiting darkness, transcending, venture into the night.

I hide within places unknown.\\
There was nothing else to do.\\
With sorrow and hatred burning inside,\\
the suffering was endless.\\
Mementos undivine.\\
The bleeding scenarios,\\
The reservoirs of shame.\\
Damned in misanthropic fires,\\
the soul dwelt.

The feelings for mankind were gone,\\
and so were the values of life.\\
There was only a final wish:\\
Death before a living hell.
}















\section{\MoonWolves}
\label{Moon-Wolves}
The \MoonWolves{}, the \quo{mystic wolves of the Frost-Moon}, are an ancient race of powerful, somewhat wolf-like creatures. They possess \human-level intelligence, but different, so they cannot easily communicate with humanoids. They are Wild creatures.

Remember to give them a proper name. 

They are inspired by the song \bandsong{Bal-Sagoth}{Starfire Burning upon the Ice-Veiled Throne of Ultima Thule}, and by the Deragoth (Hounds of Darkness) in Steven Erikson's \emph{Malazan Book of the Fallen}. 

They are an ancient race, older than the \nephilim, and used to live side by side with the \ophidians. They used to be among the masters and rulers of the Beast Realm, but as the \dragons{} and \resphain{} have gained territory, the \moonwolves{} have declined, and there are now few of them left.

Perhaps they look down upon tame animals who denigrate themselves to serving lowly humanoids, as the slaves of slaves.







\subsection{Appearance}
%Maybe they are not wolves. Maybe they are hyaenas, or some even more exotic, alien animal.
The \moonwolves{} are typically likened to wolves, but they are not wolves and not closely related to them. A somewhat closer match in appearance are hyaenas. But even that is pretty far from the mark.

What are they really? Perhaps they resemble some ancient mammal from the Tertiary epoch, or even the synapsids of the Palaeozoic.

At any rate, they are alien-looking creatures, as if they've stepped out of the ancient pre-history. Compare them to Sag'Churok and Gunth Mach, the two K'Chain Che'Malle that follow Redmask in \MalazanReapersGale.

When Carzain first sees them, he compares them to wolves. But Vizicar says hyaenas. Carzain hasn't seen a hyaena, but he recognizes them from Vizicar's memories. (Vizicar is more well-travelled, since he is not only older, but also a king.)








\subsection{Association with Visha}
\label{Moon-Wolves and the Moon}
\label{Moon-Wolves and Visha}
They are called \quo{the mystic wolves of the Frost-Moon}, and their power is connected to \hs{Visha}, the Frost-Moon, the Mystic Moon. 







\subsection{Association with dreams}
They have the ability to \quo{fade into the mist} and travel through hidden planes, journeying to secret \quo{folds} of the Realms, which the \dragons{} and others do not understand and where they cannot follow. 

Compare to the Hounds of Tindalos from a Cthulhu story by... Frank Belknap Long, I believe. 







\subsection{Enmity with \dragons}
\label{Moon-Wolves dislike Dragons}
The \moonwolves{} dislike \dragons, their ancient rivals who almost wiped them out. They root for the \resphain\dash although they fear the \banes. 

They see Ramiel as a saviour of sorts.







\subsection{Association with Ramiel}
Ramiel is a friend of the \MoonWolves{}. In the past, Ramiel helped out one of the great, venerable alpha male leaders of the \moonwolves. As thanks, a pack of them chose to remain by his side as his personal allies, following him as their alpha.

But this was before Ramiel became a \Malach{} and lost his memory (see section \ref{Ramiel becomes a Malach}). Since then, the wolves have \hyperref[Carzain dreams of Moon-Wolves]{tried to contact him in dreams}, but he has been entangled in the Shoud, and they have not been able to readily communicate with him. 





\subsubsection{Coming to his aid}
\label{Moon-Wolves help Ramiel in dreams}
The Shroud prevents them from just popping into the physical world. Also, if they did, the Sentinels and Cabal might hunt them down. Still, they come to his aid from time to time, in the world of dreams, at least. When he is besieged by monsters of his own imagination, at times he sees great white wolves that come and dispel or destroy the apparations. 

He senses that there is something he should know, but he doesn't understand it. But they awaken something in him, and they help him remember fragments of his past life. He realizes that the wolves are a vital clue in his quest to discover his past.

%He contacts them in dreams, and they help and guide him.
Perhaps one of his allied \moonwolves{} is sick or hurt, so the pack is questing for him, helping him and seeking his help in return. So they come to him in dreams, beckoning him to come to them. At last, he somehow manages to find the wounded wolf and help it. 

It pledges itself to him and becomes his permanent companion for the rest of the story. Ramiel now has his personal \moonwolf{} companion. At this point, he still does not understand the creatures and his link to them. He doesn't fully realize that until \hyperref[Ramiel's awakening in the temple]{his awakening in the temple}. 







\subsection{Worshippers}
There are some people who worship the \moonwolves{} as gods or demigods. Compare them to the wolf-worshipping Grey Swords and others in \SEMalazan.

\lyricsbs{Bal-Sagoth}{Naked Steel (The Warrior's Saga)}{'Neath the Moon-Wolf's gaze we shall slake our steel.}















\section{\Mulgrons}
\label{\mulgron}
A species of large ceratopsian dinosaurs. Similar to \latinname{Triceratops}.















\section{\Mystraacht}
The fiercest, most chaotic, most belligerent of the \resphain{} (barring the \hs\Baelzerach). Ramiel belongs to them. 







\subsection{Aesthetics}
Their traditional colours are black and shades of red, especially blood red. They glorify their martial prowess and often dress in the trappings of war: Metal armour, or robes designed to remind of armour. 

One of the clan's symbols is a black bat with blood-red fangs, eyes, claws and ears. Some of their warriors wear bat-like masks. 

Contrary to what some believe, the black bat has no deep metaphorical significance. 
The \Mystraacht{} do not believe in metaphors and deep meaning. 
They believe in power and fear. 
The bat was simply chosen because it looked physically impressive and could be used to inspire fear.

\lyricsbalsagoth{
The Splendour of a Thousand Swords Gleaming Beneath the Blazon of the Hyperborean Empire 
- Part III: 
Cry Havoc for Glory, and the Annihilation of the Titans of Chaos
}{
... none who gaze in awe beyond the mists and are blessed to behold it shall ever forget the splendour of a thousand swords gleaming beneath the blazon of [the Black Bat of \Mystraacht].
}







\subsection{Politics}
The \Mystraacht{} are the most overtly evil of the \resphan{} clans. They openly embrace both of their heritages: The \bane{} legacy, with all its parasitism, betrayal and patricide, and the \chaotic{} legacy, with all its violence and savagery. 

Ostensibly they are loyal to the \banes, but in secret they have their own \matrixx{} and are plotting against their creators. 

They once had a Overlord, but now they have only Princes.

Currently, the \Mystraacht{} are somewhat without direction, and have been stagnating for some thousand years. They need an \apex{} to their \matrixx, but all their worthy candidates fight amongst themselves and antagonize each other. 







\subsection{Ramiel's return}
\label{\Mystraacht{} \matrix}
But one day, Ramiel returns to claim the throne of \Mystraacht. 

\lyrics{The Topaz Throne is beckoning,\\
the jewelled sword awaits my graps.\\
The Dreaming Gods now grimly brood\\
in the silence of Atlantean Spires.}

Ramiel has always desired the \Mystraacht{} throne. Before he became a \malach, however, there was anothe \Mystraacht{} lord who outranked him. He has been slain and has no obvious heir. 

Ramiel has sinister dreams about the \Mystraacht{} \matrix{} and the throne. Insert a scene with Vizicar/Carzain dreaming about it. Really Bal-Sagoth-esque. 

\lyrics{Torches glow in silver cressets\\
in the Temple of the Serpent.}

Do the \Mystraacht{} have \ophidian{} connections?







\subsection{Dark knights}
\label{Dark knights of Mystraacht}
Have some sinister dark knights in \Mystraacht. Like \hr{Dark Crescent Knights}{the Dark Crescent knights}. 

Compare to the Lords of Negation in \FLuneNoire. Maybe give them a similar title. 

Maybe they resemble Revenants from the game \emph{Warcraft III}. 















\section{\Nagae{}}
\label{Nagae}
\label{Vlekkesh'sala}
The \nagae{} are a species of sea-dwelling reptillian/ichthyic humanoids. 

The more powerful of their race grow to huge size and call themselves the \vlekkeshsala.

Interesting tidbit: According to \DIBiggestSecret, the word \quo{Naga} means, in some Indian language, \quo{those who do not walk, but creep}. 







\subsection{History}
The \vlekkeshsala{} and \nagae{} are related to the \hyperref\ophidians. They were once \hyperref[Ophidians and Nagae were one species]{one species}. 







\subsection{Role in the conflict}
The \nagae{} and \leviathans{} work against the Cabal and Sentinels. They want to prevent either faction from controlling the Heart of Mith. 

What is their end goal? Is it good or evil? 

But on the whole, they don't do much, \hyperref[Ophidians today]{like the \ophidians}. 

The \nagae{} use an energy source slightly different from that of the \dragons. Some fear that if the \banes{} conquer the \dragons, they will come after the \nagae{} and their energy source next. Others don't believe it, or believe that the \banes{} will never succeed. So they won't help in the war against them. 





\subsection{Power over water and ice}
The \nagae{} and their \vlekkeshsal{} lords wield great power over water and ice. They live near the poles in the summer and migrate to the equator in the winter. 

Some of the greatest \vlekkeshsal{} lords have frozen themselves into thrones of ice, where they lie asleep for thousands of years at a time. 







\subsection{\Scatha/\naga{} hybrids}
\label{Naga-Scatha hybrids}
\label{Scatha-Naga hybrids}
In some coastal/island communities there dwell \scathae{} that interbreed with \nagae{}. The \nagae{} are more horrible, more primal kin to the \scathae, and the \scathae{} view them with fear, horror, loathing and awe. 

At times, children are born/hatched who are atavistic and look like monstrous \nagae. A particularly loathsome trait is their prehensile, snaking tails (alien to the \scathae, whose tails are rigid).

Compare to the Shake people in \SEReapersGale{} p.360-361, or the people of Innsmouth in \authorbook{H.P. Lovecraft}{The Shadow Over Innsmouth}, or the people of Imboca in the movie \movie{Dagon}. 







\subsection{A spear of ice}
Have a \naga{} lord\dash or a \vlekkeshsal{} in humanoid form\dash who wields a spear of ice. 















\section{\Nephilim}
What about the \nephilim?

In some places, humans keep \nephilim{} as slaves. 







\subsection{History}
\subsubsection{Chariots}
Back in the day, the \nephilim{} used chariots in war, because the \nephilim{} were large, and they knew of no mounts that were both fast and big enough to carry them. So they rode chariots, drawn by horses or other things.







\subsection{Ogre-Magi}
A certain group of \nephilic{} sorcerers are known to others as \quo{Ogre-Magi}. There are very few of them, but they are pretty powerful.














\section{\NerasKirishgaith}
\label{\NerasKirishgaith}
The \NerasKirishgaith{} are a race of mostrous semi-humanoid creatures with blades all over their bodies. Compare to the eponymous creatures from the anime \emph{Gilgamesh}. 

They might be \banes, but they might also be native Mithians. 

Perhaps they have great, insect-like wings. Perhaps only some breeds have wings. 

They have hive-like societies with queens, warriors and drones. They mind-control humanoids and use the bodies of these to interact with humanoids. 

They are the ancient rivals of the \ophidians, descended from a group of terrible \quo{Progenitors}, who were some of the ones that created all life on Mith, millions and millions of years ago. Compare to the Great Old Ones of H.P. Lovecraft's Cthulhu Mythos.















\section{\Nycans}
\label{Nycans are frightening}
Describe how the \nycans{} are terrible and frightening, with their cold, reptillian eyes that know too much and hide an unnatural, inhuman intelligence. 

This is a toned-down version of \hyperref[Draconic appearance]{\draconic{} appearance}. 

Compare them to Sag'Churok and Gunth Mach, the two K'Chain Che'Malle that follow Redmask in \MalazanReapersGale.

















\section{\Ophidians}
\label{Ophidians}
%\gitempl{\Ophidian}{\Ophidians}

%Before the \dragons{} there were the \ophidians: Intelligent snakes. They are snake-shaped and use telekinesis. They have powerful telepathy and a kind of racial memory. 

%Occasionally, the \ophidians{} ruled over the \nephilim{} as gods, but at some point the \nephilim{} began to hate them and waged a war of genocide against them.

%\subsubsection{\Ophidians{} as \dragons}
%Or perhaps the \ophidians{} are not a distinct species. Perhaps they \emph{are} the \dragons. In that case, the \dragons{} were originally more peaceful and uncaring, but \Tiamat{} allied them to \chaos{} and they become more violent, more cruel. 
The \ophidians{} are an intelligent race of snake-like creatures native to the Beast Realm. They are the ancestors of \dragons. 



The original \ophidians{} were intelligent snakes. They had no arms and legs but used telekinesis instead. They also had powerful telepathic abilities. Perhaps they have some kind of racial memory. Or perhaps they are simply immortal. 

%In ancient times, the \ophidians{} were one of the ruling races on Mith. Sometimes they ruled over the \nephilim. 

%Maybe they were the native rulers of the Beast Realm. 
The \ophidians{} are an ancient race, having existed for many millions of years. They have seen the rise and fall of many civilizations of lesser beings (most of which destroyed themselves out of folly). 

They fulfilled a role as guardians of Mith, and sometimes rulers. Occasionally, \ophidians{} would enthrone themselves as lords of the \nephilim{} or other lesser creatures\dash at times with evil intent. They may be something like the Jaghut in the \emph{Malazan Book of the Fallen} books. 

\label{Ophidian power source}
The \ophidians{} wielded\dash and wield\dash powerful magic. This magic is not of Chaos, but born of the native, natural power of Mith and her Heart. It is similar to the Wild power that the \hs{druids} use. The \hyperref[Kezeradi Iquin]{old incarnation of \iquin} used by the \hs{\Kezeradi} was a modified, idealized version of this.

In times of great need, such as when facing the \xzaishann, the \ophidians{} could also ask for help from a pantheon of aloof, mysterious cosmic gods. (See section \ref{Cosmic gods}.)

The \ophidians{} and the \nagae/\vlekkeshsala{} are related sister races sharing a common ancestry. They have had peaceful relations, but also their share of conflicts.

\lyricsbalsagoth{Into the Silent Chambers of the Sapphirean Throne (Sagas from the Antediluvian Scrolls)}{Winged dragon coiled in thrice,\\
bane of flame in shadowed ice.
Flooded by the bloated Moon,\\
the ivory worm now sleeps entombed.}

They are intrinsically bound to Mith, its life and future. Their blood is Life itself. And their venom is Death. 

Have some philosophy about how one bodily fluid from the \ophidians{} gives life while another takes life away. 







\subsection{Evil}
Note that while the \ophidians{} are less chaotic and violent than \dragons, this does not mean that they are not evil. They are cold, calculating, emotionless and aloof. Being immortal, they possess inhuman patience and perspective, and tend to see the short lives of lesser creatures as expendable. \Dragons{} are likewise, but more violent and passionate. 

\lyricsbalsagoth{A Tale From the Deep Woods}{The orm-garth awaits me, darkly astir with ophidian malice...}

Some \ophidians{} even hate all humanoids and \dragons{} and want to exterminate them, to make Mith clean again.







\subsection{Immortality}
\label{Ophidian immortality}
\label{\Ophidian{} immortality}
\Ophidians{} are immortal. They shed their skin periodically to renew their youth and rebirth themselves into new life. %, thus renewing themselves into new life. 
%Maybe they shed their skin to renew their youth, thus achieving immortality. In that case, 
The shed skin of an \ophidian{} may have mystic power. 

\hyperref[Draconic immortality]{\Draconic{} immortality} is different. See also \hyperref[Malach immortality]{\Malach{} immortality}. 







\subsection{The war against the \ThzanTzais}
%At some point, there was a war against the \thzantzais. I don't know where they came from. Maybe they were native to Mith, or maybe the \ophidians{} made the mistake of invading the \thzantzaic{} homeworld (\Machai?), opening a doorway to Mith for them. 

%\end{comment}
%The \ophidians{} warred against the \thzantzais, possibly alongside other Mithian races. Eventually, some of the \thzantzai{} lords betrayed their kind and sided with the Mithians, giving them the edge they needed to banish the \thzantzais{} from Mith. \HesodNerga{} and his daughters \Tiamat{} and \KhothSell{} were heroes in this war. After the war, the Mithians turned on the turncoat \thzantzai{} lords and slew them. After all, who trusts a traitor?
Once, many millions of years ago, the \ophidians{} waged a war against the \hyperref[\ThzanTzai]{\thzantzais} and prevailed.







\subsection{Rise of the \draecchonosh}
%Then some \ophidians{} craved more power. Their leader was \Tiamat. They reawakened the fallen \thzantzaic{} lords and absorbed their power into themselves, \hyperref[Origin of Draecchonosh]{transforming into \draecchonosh}. Or, in modern tongues, \dragons.
%(See section \ref{Origin of Draecchonosh}.)

Some \ophidians{} craved more power. Their leader was \Tiamat. They \hyperref[Origin of Draecchonosh]{transformed themselves into \draecchonosh}. 

It came to a \hyperref[Draecchonosh war]{conflict between the \draecchonosh{} and the traditional \ophidians}.

Later, the term \quo\dragon, derived from \quo\draecchonosh, came to be used of all \ophidians, regardless of whether they possessed \thzantzaic{} blood or not. 







\subsection{\Ophidians{} today}
\label{Ophidians today}
Perhaps the old, wise \ophidian{} lords and gods were drained and depleted after the harrowing war against the terrible \thzantzais{} and went into dormancy. They are less violent than the wicked \draecchonosh, but cold, inhuman and alien. 

Perhaps many of them were slain by the nascent \draecchonosh. 

Perhaps the survivors sleep and dream in dark places beneath the earth. 

Perhaps they are weakened by the \dragon/\bane{} war, or the Shrouding. 

Maybe they just sleep and figure: \ta{Those \draecchonosh{} think they're so tough. Let them deal with the \banes.} If the \banes{} were to conquer the \dragons, the \ophidians{} and \vlekkeshsala{} \emph{might} be able to defeat them. They themselves believe they could. Others are less sure. 

Perhaps the \ophidians{} believe, \hyperref[Ishnaruchaefir chooses eternal war]{like \Ishnaruchaefir}, that eternal war is preferable to what would happen if one of the races won. They've seen what happened when the \draecchonosh{} won, after all, and it wasn't pretty. 







\subsection{Cults}
There are serpent cults that worship the old \ophidians. They are mostly aloof and merely seek wisdom. Only occasionally do they involve themselves in the \feud. 







\subsection{\Ophidian-\resphan{} connection}
Perhaps some of old \ophidians, repulsed by their \draconic{} brethren and their violent behaviour, have sided with the \resphain. Perhaps they helped found \Mystraacht. 

Perhaps \Ishna, being less evil and chaotic than many \dragons, has dealings with the traditional \ophidians{}, and with \Mystraacht. 















\section{\Resphain}
\label{\resphain}
\label{\resphain{}}
The \resphain{} are divided into a number of fractions, kingdoms, clans, whatever. 

Some of them are loyal to the \banes. Some claim loyalty but secretly plot against the \banes, and others have forsaken their creators entirely. Some of the rebels are the \bloodresphain{} who drank the blood of \Astorglax{} and inherited his greed, his ambition, and above all his hatred of the \banes. These \resphain{} see themselves as true Mithians, the ultimate heirs to the legacy of Mith and \Erebos{} alike, possessing both \nephilic, \draconic{} and \bane{} blood. 

The word \quo{\resphan} is also used to refer specifically to the male of the species. The female is called a \resvil, plural \resviel. 







\subsection{Physique}
\subsubsection{Appearance}
\Resphain{} look like \humans, but taller and more beautiful, more perfect. Most markedly, they sprout a pair of great feathered wings on their backs. On some \resphain{} these wings are tattered and torn\dash why? Obvious candidates for this are the \Kezeradi{} survivors, and maybe Ramiel, if he is able to assume \resphan{} form at times, before he has regained his full memory (and, with it, his full perfection). 

By nature, \resphain{} have ebon black skin and hair. But many dye their hair (white or silvery colours being most common), and some even dye their skin. The \Kezeradi{} traditionally dyed their skin in bright colours: Pearly white, silvery or golden. 





\subsubsection{Procreation}
\Resphain{} can be born of \resvil{} or \human{} mothers, but the father is always a \resphan. 

The children born of a union between a \resphan{} and a \human{} woman will always be a \resphan{} or \resvil, and as full-blooded as the child of a \resphan{} and a \resvil. 

A \human{} man cannot impregnate a \resvil; her body will accept nothing less than \resphan{} seed. 

A \human{} woman who gives birth to a \resphan{} child will invariably die in childbirth as the parasitic \resphan{} sucks all life-force out of her. She will also suffer while carrying the wicked child. 

%A \resvil{} will not die in childbirth, but she will be very weak
A \resvil{} will be very weak for a while after giving birth, but she will not die in childbirth, and in time she will recover fully. 

Whenever a \resvil{} has sex with a \resphan{} (or otherwise has \resphan{} sperm injected into her) can choose to accept it, impregnating herself, or reject it. Unlike \human{} women, a \resvil{} remains fertile throughout her entire life. Her body grows new eggs periodically, so she can bear a theoretically unbounded number of children. 

\Resphain{} are born more well-developed than \humans{}. This is because they have drained so much life-force from their mother. For the same reason, they need not drink milk. They start feeding on flesh, blood and souls immediately. 

\label{Resphan sexual maturity}
Immediately when they are born, they resemble \human{} children of about five years, if perhaps somewhat smaller. They grow to physical adulthood quickly, and are sexually mature when they are less than ten years old. 





\subsubsection{Purebloods and \ashenbloods}
\label{Pureblood \Resphain}
\label{\ashenblood}
\label{\ashenbloods}
\Resphain{} born of a \human{} mother are considered inferior by their full-blooded brethren. They lack wings, often have gray skin rather than pure black, and are generally smaller and weaker. They are disparagingly called \quo{\ashenbloods}\dash a reference to how they kill their mothers during birth. 

\Resphain{} born of a \resvil{} mother are full-fledged \resphain{} and have wings. This applies even if one or both of their parents was \ashenblooded. Wings or no, \resphain{} with much \human{} blood are still considered inferior. 

\Resphain{} with no \human{} ancestors in recorded history call themselves purebloods. 

The \resphan{} nobility are the \ketherain, who descend from the \satharioth. 





\subsubsection{Perfection}
To \human{} eyes, the \resphain{} seem superhumanly perfect and beautiful\dash which they are, for they are the superbeings of whom \humans{} are but a shallow copy. 





\subsubsection{Monstrous \resphain}
There exist some \resphain{} who are not fully \human-looking, but monstrous, \bane-like, with semi-blank \bane-like faces. 

Who are they? Are they the very first batch of experimental \resphain, or are they a new, stronger variant? Perhaps even rivalling the \satharioth{} in power? If the latter, then maybe they can change back and forth between \human-like and \bane-like forms. 

Compare to the woman on the cover of \bandalbum{Hour of Penance}{The Vile Conception}.







\subsection{Morality}
Some \resphain{} believe that they truly are the noble and good angels that they style themselves to be. They see the \dragons{} and their spawn as dangerous, vicious savages, and themselves (and, perhaps, their \bane{} sires) as a superior civilization with an inborn right to rule. So they truly believe that what they are doing is morally right. Remember, \trope{UtopiaJustifiesTheMeans}{Utopia Justifies the Means}.

Others are more amoral in their outlook and simply see the \secretwar{} as a struggle for survival between two peoples who cannot and will not coexist. (See section \ref{Fighting for survival}.)







\subsection{Vampirism and cannibalism}
\label{Resphan vampirism}
\Resphain{} are vampiric creatures. They must consume the blood, flesh, life-force and souls of other creatures in order to sustain themselves and hold at bay the devouring \hyperref[Bane Entropy]{Entropy} within them, a curse which they inherited from their \bane{} sires.
%that must drain the lifeforce of others to sustain themselves. This is tied to the \hyperref[Bane Entropy]{Bane Entropy}. 

The \hyperref[Good Resphan Empire]{\Merkyrans{}} refused this, and that made them weak, easily overcome by the \hyperref[Resphan rebellion]{rebels}. 

\label{Resphan cannibalism}
\label{Resphan diet}
Most \Resphain{} happily eat \human, \scatha{} or even \resphan{} flesh\dash or \dragon{} flesh, whenever they can get their hands on it. They gain life energy this way, especially if it's powerful creatures like other \resphain{} or \dragons. 

Like their \bane{} sires, the \resphain{} are naturally cannibals who need to consume the flesh of their own kind in order to grow in power. \Draconian{} blood may be more potent, but the most delicious treat a \resphan{} knows is the blood and flesh of another \resphan{}. The mightier the better. Ideally, a \hs\sathariah.




\subsubsection{Sexual connotations}
Some of them develop a \quo{vore} fetish, take a sexual delight in eating flesh. Some take this to masochistic extremes, letting their own flesh be eaten and then using magic to regrow it. (The magic to regrow limbs is extremely expensive luxury. A \resphan{} needs to sacrifice something like three \humans{} just to regenerate his little finger.) 

Occasionally, a \resphan{} will bloodlet himself and drink his own blood as a kind of masturbation.





\subsubsection{Power and hunger}
\label{Resphan power and hunger}
The mightier a \resphan{} is, the hungrier he is, and the more he must feed. Otherwise he risks losing power, permanently. 

The \Malachim{} do not lose power permanently if they fail to feed. That is one of their superpowers. 







\subsection{Slaves of the \resphain}
There are some sickly, degenerate \humans{} who work as the \resphainz{} slaves. They do menial labour in foundries, mines, the bowels of ships and the like, far out of the sight of their masters, who do not want to see these ugly wretches. 

There are also some \quo{elite} slaves, bred to be beautiful and healthy, and groomed and nurtured to preserve them. They serve the \resphainz{} immediate needs as manservants, butlers, sex slaves, and even food. These upper slaves often command slaves of their own. 

\label{Resphan food slaves}
The food slaves are considered almost holy among other slaves and have very high status. They are religiously dedicated to what they see as a sacred duty to their living gods, and are happy to die and serve the \resphain{}, to be devoured with body and soul by their beloved masters. 







\subsection{Racial memory}
The \resphain{} have inherited some degree of racial memory from their \bane{} forebears. At least, the oldest, greatest, most pureblooded of the \resphain{} have. They remember vague impressions of the \voyagers{} (see section \ref{Voyagers}). 







\subsection{Language}
The \Resphan{} tongue is based on Hebrew. 

It is worth noting that this language uses a Spanish-style \quo{rolling} R. 







\subsection{Technology}
\label{Resphan technology}
The \resphain{} have inherited much of the \hr{Bane technology}{\banesz{} technological artifacts and knowledge}. 

They had them \hr{Merkyran technology}{already in \Merkyrah}. Back then they didn't know what to do with it, but since then they have learned much. 
The \resphain{} are much more creative than their \bane{} sires and have been able to create new inventions. But they are still far from the level of the \voyagers, and there is still much of their stolen \voyager{} technology that they don't understand. 









\subsection{\Satharioth}
\label{Blood Resphain}
\label{Satharioth}
\label{Sathariah}
The \bloodresphain{} are a group of exalted \resphan{} lords, empowered with \draconic{} blood \hr{Fall of Astorglax}{stolen from \Astorglax{} of the \secondgendragons}. 

Ramiel and \Belzirmalach{} belong to this group.





\subsubsection{Demography}
Originally there were thirteen \satharioth, of which three were \resviel. 
Since then, four or so have been permanently killed. 
Three \satharioth{} have become \malachim: \hs{Ramiel}, \hs{\Tzerachiel} and \hs\Aryal. 
One, \hs{\Sithiyacaan} has vanished. 
This leaves five \satharioth{} alive, active and kicking. 

Of the original thirteen, five remained loyal to \KiriathSepher, four defected to \Mystraacht{} and two each to \TiphredSerah{} and \Kezerad. 
\Baelzerach{} alone of the great dynasties has no \satharioth{} among its numbers. 





\subsubsection{\Ketherain}
The \ketherain{} are the descendants of the \satharioth. They, too, carry the \draconian{} blood, but diluted. 





\subsubsection{\Nexagglacheldraexz{} curse}
%\subsubsection{Hatred of the \banes}
\label{\Nexagglacheldraexz{} curse}
\label{Nexagglacheldraex's curse}
\label{Satharioth hate Banes}
The \satharioth{} and \ketherain{} harbour a deep-seated, irrational hatred of their \bane{} masters. The reason for this is twofold:

\begin{enumerate}
  \item 
    It is a a part of their heritage from the \banes, who are \hyperref[Bane cannibalism]{cannibalistic and patricidal by nature}. 
  \item
    It is amplified by the soul of \Nexagglacheldraex. He is dead, but he lives on in some form in the blood of the \satharioth. It is hinted that he \hyperref[Nexagglacheldraex sacrifices himself]{willingly sacrificed himself} in order to \hyperref[Nexagglacheldraex makes Satharioth hate Banes]{sow hatred} between the \resphain{} and their \bane{} progenitors.
\end{enumerate}

This hatred has made the \satharioth{} \hyperref[Satharioth betray Banes]{betray the \banes} and other \resphain{} several times. 

The \resphain{} still fear \Nexagglacheldraex{} and his corrupting influence. Perhaps they perform religious rituals to keep him at bay, so they can freely use his power. Compare to the Egyptian rituals used to pacify the \dragon{} Apep.





\subsubsection{Perhaps \Nexagglacheldraex{} is \Daggerrainz{} blind spot}
\label{Nexagglacheldraex is Daggerrain's blind spot}
But \Nexagglacheldraexz{} ghost is clever, and he eludes and manipulates them. 
Perhaps he is \hs{\Daggerrainz{} blind spot}, the factor that \Daggerrain{} never managed to understand and enter into his calculations. 
Or perhaps the entire \feud{} is actually very much a contest of sneakiness between \Nexagglacheldraex{} and \Daggerrain, with the \dragonlord{} hiding from \Daggerrain, using the knowledge of the \banes{} gained from merging with the \resphan{} soul to hide from the \banelord{}, gain insight into his plans and subtly subvert them. 
This may also be what eventually allows \hr{Ramiel betrays Banes}{Ramiel to betray \Daggerrain{} and overthrow him}.





\subsubsection{The curse causes them to descend into madness}
\label{Madness of \Nexagglacheldraexz{} curse}
The curse causes the \satharioth{} to slowly descend into madness and dementia. 

\lyricslimbonicart{Dynasty of Death}{
In the dark caves of oblivion\\
bad blood rises from the Mega Therion [\Nexagglacheldraex].\\
From vast stalactite halls so undivine,\\
through the misty corridors of time.\\
As one lives one shall die.\\
Ad noctum.
}

\lyricslimbonicart{The Yawning Abyss of Madness}{
Again I drift the halls of wondering. \\
The black castle of solitude. \\
On the very edge of sanity \\
in mental cryogenic interludes. \\
I have slipped into the seventh, \\
the seventh circle of Hell, \\
in realms where deadly shadows \\
infest every cell. 

Internal ceremonies in ritual death. \\
External bleedings for the demon of madness. \\
Hide from the torture of the dazzling light. \\
The demolition voice shall speak tonight. 

While I'm staring down into the darkest pit, \\
an ocean black as the night, \\
so infinite deep and consuming, \\
it swallows all life force with might. 

Again I drift the halls of wondering, \\
as I focus for the darkness to come. \\
In anguish minds uplift the conquering \\
to cross the line of death beyond.

An abstract reality and bottomless insanity. \\
To search for the powers to please\\
the subconscious spirit of disease. \\
Time found no remedy, \\
cause winds of darkness was stealing me. 

The yawning abyss of madness. \\
A cryptic slaughter by hate. \\
Darkness is the only survivor \\
as evil dominion terminates. \\
The yawning abyss of madness.
}













\section{\Scathae}
The \scathae{} were made to serve the \dragons. 

When? Probably during \Tiamatz{} reign. 

Despite this, they were not slaves to the same degree as \humans. They were bred from \naga{} stock and as such, they were free citizens to a higher degree than \humans{} ever were. They had more capability to think for themselves, and they understood more of the world around them, and of their masters. After the \hs{Second Shrouding}, though, they became more enslaved, as their minds succumbed to the Shroud.

Perhaps they were created to replace the old \hs{\ophidian{} humanoids} when they became extinct. 









\subsection{\Scathaese{} senses}
\label{Scathaese colour vision}
\Scathae{} have colour vision different from that of \humans. They can distinguish between some nuances that \humans{} cannot tell apart, and vice versa. \hr{Curiet's colour vision}{This is brought up} by \hs{Curiet Serpentin}.









\subsection{The first \scathae}
The first generation of \scathae{} to be created were more powerful and had more \draconian{} and \xsic{} blood than modern \scathae. A few of these prototypes survive, including \hr{Criseis}{\Criseis} and \hs{\PsyrexFull}. 









\subsection{\Troglodytes}
\label{\troglodytes}
\Troglodytes{} are a degenerate subrace of \scathae{} (or several superficially similar subraces) that dwell in caves underground. They are often found worshipping the \daemonic{} monsters of \hs{Kai Leng}. 

To the eyes of other \scathae, the \troglodytes{} are frightening and loathsome abominations. They remind the \scathae{} of some primal, primitive aspect of their own nature and origin, something which the \scathae{} would rather forget. 















\section{Sentinels of Mith}
\label{Sentinels}
\label{Sentinels of Mith}
\subsection{Summoning the \xss: A deadly balance}
\label{Summoning the \xss: A deadly balance}
The Sentinels, led by \Secherdamon{} and \Vizsherioch, among others, seek to awaken the \xss, but only partially. Enough to draw upon the \xssz{} power and use it to vanquish the \banes. But they don't want them to awaken so much that they will take control and subjugate or wipe out their \draconic{} spawn. 

To this end, the Sentinels \hr{To break or preserve the Shroud}{work to weaken the Shroud}. 

\label{\Dragons{} want to usurp the \xss}
In the best of worlds, the Sentinels hope to \quo{revive} the \xss{} by usurping them\dash transforming themselves into new \xss. \Vizsherioch{} comes closer to this than anyone else.

This is a delicate balance. Some, such as \Ishnaruchaefir, believe that the Sentinels will fail, and therefore \hr{Ishnaruchaefir fights to preserve the Shroud}{strive to preserve the Shroud}.









\subsection{Fractions among the Sentinels}
There are various fractions among the \dragons. Each \draconic{} \bloodline{} fights against the other \bloodlines, seeking to further its own goals. 

I don't think there's any \dragonking. Although \Bloodline{} Irokas does exist. 

Near the end, \ApepNesthra{} might show up and assume rulership of the \dragons. 

The Sentinels have a dark council who work to ensuer the \quo{Imperial Court of \Draconian{} Sovereignty}. This title is taken from \DIBiggestSecret{} p.44.















\section{\Sephiroth}
\label{Sephiroth}
The sixteen \Sephiroth{} were originally created from the souls of sixteen rebellious \resphan{} lords, each merged with the soul of a sacrificed \banelord. 

These \resphain{} were the lords of the rebellious \Kezeradi. After the rebels' defeat, the \banes{} (or \bane-allied \resphain) decided that using their leaders as \Sephiroth{} would not only be an effective and useful tool, but also a suitable punishment: They would live on to see the virtues they fought for corrupted and perverted, and their own power would be used to achieve it. 

A theme throughout the series is that some of the characters gradually discover the nature of the \sephiroth. Some work actively to destroy the \sephiroth, thus freeing the souls of the captive \Kezeradi{} lords. 









\subsection{Plan}
\label{\Sephirah{} plan}
The \sephiroth{} are part of a sinister long-term plan. Probably has to do with \hs{the Disease}. 









\subsection{Soul prison}
The \sephiroth{} are a soul prison. 

\lyricslimbonicart{Grace By Torments}{
Underneath the ice of winter\\
there is a stream so powerful.\\
A dark river that will swallow your soul.

As it twists into form,\\
you dwell in a cold dark void.\\
Frozen and lifeless dreaming,\\
deep down in the darkness screaming.

When pain comes to power,\\
all the sorrows of the heart\\
leading your soul astray\\
to a realm of agony.

All you have left is death's desire for you.

A black hole.\\
Well of souls.\\
Icy breath.\\
Stone cold death.\\
Darkness calls\\
as Heaven falls to Earth.
}

\lyricslimbonicart{Seven Doors of Death}{
Asylum of neurotic minds.\\
Devour all consciousness.\\
Insanity is there to find.\\
Dominion of darkness.
}









\subsection{Elements of the mind}
The \Sephiroth{} are divided into four \quo{elements}. These are four integral components of the mind, and, if I can manage it, of the universe as a whole. It needs to be things that can be hyped up in propaganda as being virtues, the source of all good, but that can also easily be subverted and twisted into a source of evil. 

My current ideas are: 

\begin{description}
  \item[Passion:] 
    The deep, heartfelt belief in a cause and the burning desire to fight for it. 
  \item[Eye/Vision:] 
    The Eye lets you perceive and interpret the world: In a \quo{true}/\quo{good} way, or as a veil of lies, seeing only what you want to see (or what your masters want you to see).
  \item[Voice:] 
    The Voice that lets you communicate with your fellow beings, to offer wisdom and comfort, or to condemn and spread hate and lies. 
  \item[Tears:] 
    Tears of joy, or tears that release sorrow and help you deal with it. Or tears that confirm your sorrow and only serve to pull you deeper into a mire of suffering. 
\end{description}

Another schema might be that of four basic emotions:

\begin{description}
  \item Anger: Defense, expansion.
  \item Fear/pain: Self-preservation.
  \item Hunger: Self-improvement.
  \item Lust: Procreation. This one can be sublimated into an urge to create other things than progeny, such as art, or an empire.
\end{description}








\subsection{\Vertex{} status}
The \Sephiroth{} are not \vertices{} individually, because they are half-mindless slaves, lacking the force of personality to become true \vertices. Rather, \iquin{} as a whole is one immense \vertex{}. 









\subsection{List of \Sephiroth}
% Command to typeset a list of Archons
\newenvironment{sephlist}{\begin{description}}{\end{description}}
% Command to typeset the entry of a Archons. 
\newcommand{\seph}[1]{\item[#1:]}



\subsubsection{\Sephiroth{} of Air}
\begin{sephlist}
  \seph{\Atzirah{}}
    The carrying wind. Used for lifting objects or flying. He embodies the virtue of Honour. His purpose is to keep people inside the system for fear of dishonour while encouraging them to strive to please their masters and the Church. 
  
  \seph{\Feazirah{}}
    The gentle wind, represents Humility. She is charged with keeping people pacified and keeping them from complaining. 
  
  \seph{\Keshirah{}}
    The powerful wind. Used to create controlled gusts of wind. She embodies Dilligence, the virtue that makes people work hard for their masters. 
  
  \seph{\Razilah{}}
    Used to create lightning. He represents the virtue of Righteousness and opposes any kind of heresy, blasphemy and unorthodoxy, and also crusades against the heathens. 

\end{sephlist}

\subsubsection{\Sephiroth{} of Fire}
\begin{sephlist}
  \seph{\Barion{}}
    His virtue is Courage, that which encourages the people to fight against the enemies of the Church.
  
  \seph{\Hapheron{}}
    He is the manifestation of the virtue of Unity, that is, unity as a people and a culture. He strives to turn the Iquinians\dash and all \humans\dash into a united front against their enemies. 
  
  \seph{\Izion{}}
    He Who Smiteth With Flame. Creates blasts of fire. Perhaps the \Sephirah{} most commonly used in combat. He represents the virtue of Justice, protecting Justice by destroying the wicked --- sinners or the enemies of the church. 
  
  \seph{\Teshiron{}}
    She embodies Faith and is charged with keeping people loyal to the Church and not asking unwanted questions. 

\end{sephlist}

\subsubsection{\Sephiroth{} of Earth}
\begin{sephlist}
  \seph{\Cushed{}}
    Used to shape, move and Sculpt earthen objects, but can also be used to paralyze a person. His virtue is Lawfulness, which helps the rulers keep people in check. 
  
  \seph{\Hoshied{}}
    He is Loyalty, which keeps people under control and discourages them from asking questions. 
  
  \seph{\Thimared{}}
    She embodies Obedience and strives to turn people into humble slaves. 
  
  \seph{\Yemared{}}
    She represents Tradition, which helps to prevent rebellion and keeps the social order from evolving. 

\end{sephlist}

\subsubsection{\Sephiroth{} of Water}
\begin{sephlist}
  \seph{\Gamishiel{}}
    She represents Sacrifice, and to reflect this, her month is only 20 days long where the other fifteen are 24 days each. Her duty is to make people work hard for their masters and not expect any rewards. 
  
  \seph{\Ishiel{}}
    A \Sephirah{} of Healing, the most commonly invoked. Her virtue is Patience, which makes people accept hardship and oppression. 
  
  \seph{\Omariel{}}
    She embodies Acceptance, which makes people accept hardship and oppression. 
  
  \seph{\Yeziel{}}
    He is Chastity. Sex is a dangerous thing, because it may open people's eyes to the Beyond. Also, it makes them harder to control. Also, \human{} sexuality is something the \banes{} very much want to harness and control, so they need \Yeziel{} to keep people's sexuality in check. 

\end{sephlist}



%\begin{comment}
\subsection{Virtues}
Each \Sephirah{} is associated with a `virtue'. These virtures are actually evil and used to control men. These sixteen evil virtues are: 

\begin{itemize}
	\item \Thimared{}: Obedience (prevents masses from rebelling and rulers from sympathizing). 
	\item \Hoshied{}: Tradition (prevents rebellion and change). 
	\item \Cushed{}: Lawfulness (control). 
	\item \Yemared{}: Loyalty (keeps people from questioning). 
	\item \Feazirah{}: Humility (keeps people from complaining). 
	\item \Hapheron{}: Unity (unity with one's own people only, leading to hate of outsiders). 
	\item \Izion{}: Justice (hate of everyone different). 
	\item \Razilah{}: Righteousness (keeps religion from evolving). 
	\item \Keshirah{}: Dilligence (keeps people working). 
	\item \Barion{}: Courage (in the face of outsiders, of course). 
	\item \Atzirah{}: Honour (keeps people inside the system out of fear of dishonour). 
	\item \Teshiron{}: Faith (religious faith, fanaticism). 
	\item \Yeziel{}: Chastity/sexual decency (because sex is dangerous and can lead people astray). 
	\item \Ishiel{}: Patience (makes people accept hardship and oppression). 
	\item \Omariel{}: Acceptance (makes people accept hardship and oppression). 
	\item \Gamishiel{}: Sacrifice (obvious). 
\end{itemize}
%\end{comment}



















\section{\Soulreapers}
\label{Soulreapers}
There exists a race of \soulreapers{} who prey on the \banes. They are inspired by the Soulreapers of Manticora's \emph{The Black Circus} albums. 

The \banes{} fight them, but they cannot destroy and exterminate them, for they races are tied to each other: The \soulreapers{} are born from \banes{} that fail and mutate.

See, the \banesz{} \matrixx{} is tied to \FatherErebos{} and his power. The \soulreapers, in turn, are an integral part of the \bane{} \matrixx. In a sense, the scourge of the \soulreapers{} is a curse upon the \banes{} from \FatherErebos{} as a punishment for the \banesz{} betrayal of their homeworld. 















\section{\TiphredSerah}
A \resphan{} faction loyal to the \banes. These guys should have some special traits. Maybe they are more sneaky, more ninja-like than the \KiriathSepher. Their rulers are unknown, but they are represented by a Speaker. Several of their leaders are \resphanesses. 

The \TiphredSerah{} often dress like vampires, with robes and high collars. Their traditional colours are blues, violets and purples.

The \TiphredSerah{} have a tradition for lacquering their fingernails in their clan colours. 















\section{Vaimons}
\label{Vaimon}
\label{Vaimons}

Each Vaimon clan has a traditional colour. There is no red clan, because red was the traditional colour of the Vaimons' enemies back at the time when the clans were founded. 









\subsection{Clan Redcor}
\label{Clan Redcor}
\label{Redcor}
\label{Redcor philosophy}
The Redcor believe that philosophy should be moral, clean, pure, structured, regulated, dogmatic and wholesome. Philosphy is a means of coming to a better understanding of the truths that are already revealed and known. 

Speculation into the dark, the unknown, the frightening is forbidden and considered heresy, a direct way to corruption and damnation. Evil thoughts are the way to evil deeds.









\subsection{Clan \TigerVaimon}
In order to make themselves less dependent on the Redcor, the Belkadian Empire formed their own \quo{clan} of Vaimons: Clan \TigerVaimon. 

The Redcor resent them and see them as rogues, not a true clan. 

The \TigerVaimons{} are not closely affiliated with the Belkadian church, because the High Kings didn't want to church to become too powerful. The High Kings wanted to keep their potential competitors scattered, so they could control them. 

In the Empire's days, all major imperial lords kept a small but very professional \cadre{} of \TigerVaimons. After the fall of the Empire, the clan fell apart and most were killed. (Why were they killed? And by whom?) 

Now there are only a few, scattered \TigerVaimons{} left, and few lords can muster a decent \cadre. Malcur only has a makeshift \cadre{} of one or two \TigerVaimons, some chaos sorcerers and hedge wizards, and Carzain \Shireyo. 

Do they use \nieur{}, too? They probably do.









\subsection{Archon Ward}
\label{Archon Ward}
A type of magical armour made by the Vaimons during the time of the \hs{Vaimon Empire}. An Archon Ward consists of a headband, a necklace and a number of bracelets. Together, when activated by a skilled Vaimon, these can form an energy shield surrounding the user. 

Archon Wards were extremely rare and expensive even in the empire. Several Vaimon Emperors are known to have worn them, but barely anyone else could afford them. 















\section{Vaimon Empire}
The Vaimon Empire existed from the year \yic{Founding of the Vaimon Empire}, where it was founded by Cordos Vaimon (see section \ref{Cordos Vaimon}), and until the \Darkfall{} (see section \ref{Darkfall}) where it fell, during the reign of \Belzir{} (see section \ref{Belzir}). 

\label{Vaimon Empire nativete}
Back then, people believed that the \Sephiroth{} and \angels{} (\resphain) were noble and good, having defeated the evil \pdaemons{} and being well on their way to vanquishing the last remnants of evil in the world. 

But the truth was slowly revealed to the learned during the reigns of the last few Emperors, including \VizicarFull{} (see section \ref{Vizicar}). But Vizicar still very much believed in the faith of \iquin{} and did not discover much, only a suspicion that something is lurking beneath the surface somewhere.

It all collapses under \Belzir, who learns far more than she should.







\subsection{The \Darkfall}
\label{Darkfall}
Perhaps the \Darkfall{} coincided with a terrible, bloody war between different \Resphan{} factions. Perhaps this was even the fall of \Kezerad. 

Inspired by the album \bandalbum{Symphony X}{Paradise Lost}.















\section{\Voyagers}
The \voyagers{} were the ones who created the \banes, and possibly also the \nephilim{} and other Mithian life. 

Compare them to \bandsong{Bal-Sagoth}{Voyagers Beneath the Mare Imbrium}.

See also: \bandsong{Bal-Sagoth}{As the Vortex Illumines the Crystalline Walls of Kor-Avul-Thaa}. 

\lyricsbalsagoth{The Scourge of the Fourth Celestial Host}{
They possess power unparalleled...\\
Ageless, remorseless. Without pity or conscience.\\
Manipulators of evolution on countless worlds.\\
Gods of the stars... the Celestial Host!
}

The \voyagers{} are quite alien, but they are still the most \human{} of the ancient races. 

\lyricstitle{\authorbook{\HPLovecraft}{At the Mountains of Madness}}{
\tho{Whoever these creatures had been, they were men!}}

Everything on Mith is descended from the \voyagers. This is \hs{why the \banes{} want Mith}!

\lyricsbalsagoth{Invocations Beyond the Outer-World Night}{
The legacy of the First Ones, spawn of the Mera!
}









\subsection{Appearance}
The \voyagers{} are alien, but somehow beautiful. At least to \humans. This is because \humans{} are created to serve the \banes{}, who in turn were created by the \voyagers{} and patterned themselves after them.

They are vaguely humanoid, but only vaguely.
%, shining white with great gossamer wings. 
They have a body and a head with some sensory organs on it. Then they have multiple limbs radiating in all directions, like a starfish. Four of these are bigger than the rest, and these sort of resemble humanoid arms and legs. Some of their limbs end in what look like gossamer wings.

Compare them to the Navigator from the \emph{Dune} TV miniseries.









\subsection{Origin}
Perhaps the \voyagers{} were in turn created by an even mightier, more ancient cosmic race.

\lyricsbalsagoth{The Fallen Kingdoms of the Abyssal Plain}{
Long ago, before the third of Earth's moons fell fiery from the star-seared sky, there were those whom we have come to call the First Ones. \\
These men-who-were-not-men were the creations of the Mera, beings from the far reaches of the limitless cosmos, whose essence still flickers latently within the minds of all their disparate progeny.\\
Praise the Mera, fathers of the First Ones, bondsmen of the K'laa, sworn foes of the Z'xulth!\\
Sired in the great spawning vats beyond the fathomless deeps of the Pre-Cambrian sea, the First Ones throve.\\
Those who were engineered to live on land duly constructed the grand Antarctic Megalopolis, ultimately becoming entangled in bitter conflicts with the hoary Serpent Kings before retreating into the subterrene depths of the vast inner world, whereas those First Ones that had chosen the embrace of the abyssal seas were the architects of vast and glorious submarine cities whose splendid spires and minarets towered proudly beneath the unfathomed waves.
}









\subsection{Servitor races}
The \voyagers{} had hordes of servitor races that dwelt with them and served them (different servant races in different aeons). Some envision this mythical time as a utopian age. 









\subsection{Technology}
\label{Voyager technology}
The \voyagers{} possessed ultra-badass technology.









\subsection{Craters and wrecks}
I should have some great craters and wrecks, at the sites where the \voyagersz{} great spaceships crashed down on Mith, tens of thousands of years ago. 

Mystic power radiates from these craters, creating vast maelstroms of energy. Monsters and undead spirits feed on this energy, and so throng around the craters.

This is inspired by the vulcanic vent that Laura Daughtery discovers in the first episode of the TV series \emph{Surface}.









\subsection{Voyagers today}
\label{Voyagers today}
A few \voyagers{} survive on Mith. They dwell in desolate-but-functional high-tech citadels in otherwise ruined cities. 

Compare to \authorbook{\HPLovecraft}{At the Mountains of Madness}.

Some of them function as remote, powerful \hs{cosmic gods}. Compare to the Elder Gods of the Cthulhu Mythos. 





\subsubsection{Ultima Thule}
Maybe I should have a mystic place far to the north, near Mith's North Pole, where old \voyager{} cities, millions of years old, are buried beneath the ice. 

The \vlekkeshsala{} know many of the place's secrets, but they are not telling. 
The \nagae{} know of the place and can be persuaded to lead people there, but they will not enter themselves. They know the danger. 

What is the danger? The \bladedpeople? Or shoggoth-like creatures, like in \authorbook{\HPLovecraft}{At the Mountains of Madness}.

\lyricsbalsagoth{In Search of the Lost Cities of Antarctica}{
Beneath the ice, the endless ice \\
of Pangaea's (now) axial (eternally frozen) frontier, \\
entombed for countless millions of years... \\
the lost cities of Antarctica!

Secrets locked within the ice, the endless ice of Antarctica,\\
'Neath the peak of Erebus the First Ones sleep, Lords of Pangaea,\\
Cities lost within the night, the frozen night of Antarctica,\\
Pre-Cambrian, the Voyagers, beyond the stars, Lords of Pangaea.
}

Some people see dreams visions of \voyager{} cities in their golden age. These cities surpass \emph{everything} that has ever existed on Mith. \emph{Except} the glorious and terrible dark citadels of the \hs\xss... and the halls of the \hs{cosmic gods}. Have \Ishnaruchaefir{} comment on this. 

\lyricsbalsagoth{In Search of the Lost Cities of Antarctica}{
Once, the coruscating spires of [the \voyagers] here offered their splendour to the heavens. \\
Now, those spires gleam no more, \\
save in dreams of verdant plains, \\
save in dreams of time-lost citadels. \\

Legacy of a utopia lost, \\
forever enshrined 'neath the ice...

Before the Nine Continents were formed from Pangaea's shattered surface... \\
Hewn from the Pre-Cambrian rock, \\
behold this primordial metropolis!
}

Also comparable to Bal-Sagoth's Ultima Thule. 





















\section{\Vreiiden}
\label{\Vreiiden}
\label{\vreiiden}
Flying, \dragon-like reptiles. They are vicious, savage beasts. They have great mouths filled with rows of dagger-like teeth. 

\label{Wyverns}
\label{Vreiid}
\label{Vraiid}
The \vreiids{} are a race of flying, reptillian monsters, like wyverns. They are savage and brutal creatures. They can be tamed to some extent, but it is difficult, and they'll never be very tame. They are used as mounts in \quo{dark} armies, like those of the Rissitics. 















\section{Wraiths}
\label{Wraiths}
\label{wraiths}
Have a race of black wraiths with blank faces, hair-like protuberances like the Protoss from \emph{Starcraft} and tails like Genie from Disney's \emph{Aladdin}. 

Maybe they are a type of \banes. 















\section{\Xzaishanns}
\label{Thzan-Tzai}
\label{Xzai-Shann}
\label{XS}
The \thzantzais{} are a race of \mdaemons{} native to \Machai. They are incorporeal creatures of immense power. They are served by horder upon hordes of minor \mdaemons{} and \mdaemons.

They are inspired by the alien invaders from \bandsong{Bal-Sagoth}{As the Vortex Illumines the Crystalline Walls of Kor-Avul-Thaa}, and the Great Old Ones from the Cthulhu Mythos by H.P. Lovecraft and others. 

The \xss{} have many \quo{lords} but not \quo{king}, nor even much of an organization. They are creatures of Chaos, remember. 

%\subsection{History}
%Later, the \thzantzai{} lords were reawakened by the fledgling \draecchonosh, who absorbed their power and being into themselves (see section \ref{Dragons}). The \draecchonosh{} then enslaved the lesser \thzantzai{}. They are now forced to unleash their wrath and hatred on the \dragonsz{} enemies. 

Are the \thzantzais{} \pdaemons{} or \mdaemons? I need to think more on this distinction.

Once, the \xzaishanns{} waged a \hyperref[XS war]{war against the \ophidians}. They were \hr{XS are defeated}{\quo{defeated}}. 









\subsection{Names}
I should have a pantheon of the \quo{Eldest Lords of the \Xzaishanns}. Their names might include: 

\begin{itemize}
  \item Abraloth. 
  \item Niil-shacht.
  \item Uruzgal. 
  \item Vol-croth. 
  \item Yoggranath. 
\end{itemize}










\subsection{Death and slumber}
\label{\xsic{} slumber}
\label{Dead \xss}
%The \xss{} are sleepy and tend to slumber dormant for thousands, if not millions of years. Possibly for \hs{astrological} reasons. 
The \xss{} are sleepy. They \quo{die} and lie dead and dreaming for thousands if not millions of years at a time. 
Possibly for \hs{astrological} reasons. 

Have some mythical references to the \hr{\Dragons{} worship dead gods}{\quo{dead gods} whom the \dragons worship}.

Compare to Cthulhu from \authorbook{H.P. Lovecraft}{The Call of Cthulhu}, who sleeps until \quo{the stars are right}. 

They were already dead when \hr{\Tiamat{} contacts \xss}{\Tiamat{} contacted them}, but they were dreaming, which allowed her to communicate with them.







\subsection{\ThzanTzaic{} \pdaemons}
There exist several species of \pdaemons{} that served the \thzantzais, and can now be forced to serve a Chaos sorcerer. 

One such species is mostly humanoid, but with a wide mouth filled with a row of dagger-like teeth, and great bat-like ears. Perhaps bat-wings, too. Like the monster in Clive Barker's \emph{Hell's Event} (\emph{Books of Blood II}).







