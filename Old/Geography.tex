\chapter{Geography}
\section{\Azmith{}}
\index{\Azmith}
\label{\Azmith}
A continent on the planet Mith consisting of Belkade, the Northern Kingdoms, Threll, the Imetrium, Uzur, Durcac, the Near Orient and the Serpentines. %Some scholars also consider Irokas a part of \Azmith{} while others do not. 

\quo{\Azmith} is Archaic Vaimon for \quo{all the world} (containing the word \quo{Mith}, meaning \quo{the world}). 









\section{The Belkadian Empire}
\index{Belkadian Empire}
The Belkadian Empire existed for a century or two and collapsed only around 30 years before our story begins (so 5-10 years before Carzain was born). 









\subsection{The fall of the Belkadian Empire}
It is still unclear and mysterious what actually happened at High King \LastHighKingz{} court. 

Perhaps \LastHighKing{} or his court Vaimons found out about the \Charade{} and the underground war, and acted to foil the factions' plans. 

Perhaps \LastHighKing{} or one of his advisors was planted by the \cuezcans{}. If so, he is still alive (or maybe undead) and at large, working underground to oppose the warring factions. 

Moro \Cornel, the Pelidorian archmage, might be in league with the fallen High King...

The \hs{\Kezeradi} might have had a hand in the fall of the Empire.

It might have had to do with \hs{\Semiza}. 







\subsection{Belkade today}
Belkade should be divided into a small number of large kingdoms, each of which is subdivided into multiple duchies, baronies and whatnot. Pelidor is one of the latter. 

The kingdoms need to be bigger so we can have bigger armies and wars on a more epic scale. 















\section{Far Orient}
\label{Far Orient}
\index{Far Orient}
\index{Orient!Far Orient}
The eastern part of the \hs{Orient}. 















\section{Geica}
\label{Geica}
\index{Geica}
The homeland of \hs{Clan Geican}.





\subsection{Fallen Emerald Palace}
Originally the Geican seat of power was the splendid Emerald Palace. 
But it was destroyed in the \darkfall. 

All Vaimon clans used to have such palaces of crystal. 
But today only the Redcor \hr{Topaz Chateau}{\TopazChateau} remains. 















\section{Heropond Forest}
\label{Heropond Forest}
\index{Heropond Forest}
A large forest in \PelidorContinent. Marks the border between Pelidor and \Scyrum. 















\section{The Imetrium}
\label{Imetrium}
\label{Imetric}
\index{Imetrium}









\subsection{Seafaring power}
The Imetrium is an kingdom of islands. 
As such, the Imetrians have become great seafarers. 
This is also because of their \hr{Imetrians and \nagae}{ties to the \nagae}. 









\subsection{The truth about the Imetrium}
The Imetrium is secretly backed by either the \nagae{} or the \cuezcans. 

Salacar is a saviour-figure of a kind. Similar to the \human{} \quo{Messiah} that is mentioned in \emph{The Bonehunters (Malazan Book of the Fallen 7)}, who might be Shadowthrone. 

Salacar is a \quo{promised child}, born to return his own people to glory and, in his creators' eyes, purge the world of evil. As such, he carries in him the souls of many of his forebears in him, and their many thousands of years worth of memories. He is an powerful \vertex, the \apex{} of a \cuezcan/\naga{} \matrixx. 

But all of it in secret. Although the Cabal and Sentinels might suspect.







\subsubsection{Ilcas suspects}
At some point in the story, Ilcas learns something disquieting and begins to suspect that the Imetric gods are more than they seem to be. 









\subsection{Politics}
\subsubsection{Ideology}
The Imetric system is a primitive communist plan economy. 
People are comparatively wealthy, well-fed, healthy and secure. 
But the system is also rather totalitarian, obsessed as they are with their \quo{justice}. 
There is much control and repression. 





\subsubsection{Ties to the \nagae}
\label{Imetrians and \nagae}
The Imetrium is tied to the \nagae. 
Perhaps Salacar is actually a \vlekkeshsal{} in disguise. 

Also, Dessali is a \naiad, a water-dwelling \quo{spirit}. 









\subsection{Arsenal}
\subsubsection{Icy armour}
Have Imetric warriors clad in armour made of what looks like transparent crystal, glass or ice. 
And matching weapons: Spears and swords of ice. 
Compare to the Stormriders in \authorbook{Ian Cameron Esslemont}{Night of Knives}. 















\section{\Imrath}
\label{\Imrath}
\index{\Imrath}
The kingdom where \hs{Silqua} and \hs{Cordos Vaimon} lived.

















\section{Moons}
\label{Moons}
\index{Moons}
Remember that the Moons have their own Realms. The \moonwolves{} live there, as do other things. 

The Moons play a large role in the \feud. There are \nexi{} there.







\subsection{\Dun} 
\label{Dun}
\index{\Dun}
\Dun, called the Gray Moon, is the larger of Mith's two moons. 

\Dun{} is a world of its own, full of life. 
Unlike Visha, which is desolate and dead. 





\subsubsection{Astrology}
In astrology, \Dun{} is considered mostly benevolent. 





\subsubsection{Astronomy}
\Dun{} is about the same size as Earth's Moon. It is closer to Mith than Earth's Moon is to Earth, so it appears larger in the sky. \Dun{} has a dark gray colour. 

\Dun{} is large enough to cause a solar eclipse. It is too large to cause the \squo{ring} effect known from solar eclipses on Earth, however. \Dun{} is larger than the Sun in the sky, so during an eclipse, the Sun is completely swallowed. 

\Dun{} itself is eclipsed when it passes behind the shadow of Mith. This is called a \Dun{} eclipse.\index{\Dun{}!\Dun{} eclipse} 

\Dun{} circles Mith once every 23-24 days, roughly corresponding to a month of the \hs{\VaimonCalendar}. 







\subsection{Visha}
\label{Visha}
\index{Visha}
Visha, called the Pale Moon, is the smaller of Mith's two moons. 

Where \Dun{} is teeming with life, Visha is dead and desolate, inhabited only by ghosts, cruel \hs{cosmic gods} and perhaps the corpses of some \hs{dead \xss}. 

The \hs{\moonwolves} originally came from Visha, but their realm was destroyed and they were driven out. Compare to \authorbook{\HPLovecraft}{The Doom That Came to Sarnath}.  

\lyricslimbonicart{Moon in the Scorpio}{
A mirror blank ocean above me decoy.\\
Superior forces that heal or destroy.\\
Take me astray into the moonlight above
through twilight eyes as a spectre shadow.

In an atmosphere supreme\\
forces dwell in domancy.\\
The essence of its spirit is evil,\\
as a curse upon thy name.

Midnight is the shepherd of mysterious powers\\
and moving shadows in the corner of the eye.\\
Moon's blazing intuition\\
contains what death requires.

Behold the sky above \\
when the moon is in the Scorpio.\\
A cold bleak light
}





\subsubsection{Astrology}
Astrologically, Visha is considered malevolent and a bringer of ill omens. 





\subsubsection{Astronomy}
Visha is only half the diameter of \Dun{}. 
It is also farther away. 

Visha is not large enough to eclipse the Sun. It is much smaller, so when Visha moves in front of the Sun, it is visible as a dark hole in the Sun. This phenomenon is known as a \squo{Sunhole}\index{Sunhole}. 

Visha also sometimes casts a shadow on \Dun{}. This is called a \squo{\Dun{} hole}.\index{\Dun{}!\Dun{} hole}\index{Visha!\Dun{} hole} 

Visha itself is eclipsed when it passes behind the shadow of \Dun{} or Mith iself. This is called a Visha eclipse\index{Visha!Visha eclipse}. 

Visha circles Mith once every 60 days or so.
















\section{Near Orient}
\label{Near Orient}
\index{Near Orient}
\index{Orient!Near Orient}
A collective term for the nearer part of the \hs{Orient}. 
Kingdoms in the Near Orient include Hazid. 















\section{Orient}
\label{Orient}
\index{Orient}
The area southeast of Belkade (east of Durcac and south of the \Serplands{}). 
Informally split into the \hs{Near Orient} and the \hs{Far Orient}. 
















\section{\PelidorContinent}
\label{\PelidorContinent}
\index{\PelidorContinent}
A continent in central \hs{\Azmith}. Contains \hs{Pelidor}. 















\section{Pelidor}
\label{Pelidor}
\index{Pelidor}
A duchy in the central \hs{\PelidorContinent}.







\subsection{House Pelidor}
\index{House Pelidor}
\index{Pelidor!House Pelidor}
House Pelidor, a \scathaese{} noble family of \Tepharin{} descent, are the current rulers of Malcur and the nation of Pelidor. Their head carries the title of \quo{duke}.

The Pelidors were once worshippers of a group of lost gods (the Dreaming Gods?), perhaps \Tepharin{} gods. They were keepers of the occult secrets of what lay hidden underneath Malcur. Charged to ensure that what lay beneath would not be awakened. 

But when the Belkadians invaded, most of the Pelidors were killed as dangerous heretics and heathens, and a child prince was made duke, now a Belkadian vassal. The country was forcibly converted to the Iquinian faith and the secrets of Malcur were lost. Many records were destroyed because the people should not know that kind of heresy. 

\Tiroco{} is also of House Pelidor. She is \Icorz{} half-cousin. She knows nothing of her family's legacy. 

But the secrets are not all forgotten. A select few keepers among the Pelidors have been handing down oral remembrance\dash always behind the duke's back\dash all the while doing research and trying to rediscover more. 

Liocai Pelidor, the sister of Duke \Icor, is one such keeper. She knows a shocking amount of things. Things that even the Sentinels don't know. 
 
This might be a good place to tie in the story of the mysterious fall of the Belkadian Empire. 









\subsubsection{Symbol: \Grulcan}
\label{Pelidor symbol}
The symbol of House Pelidor is a bronze-coloured \hs{\grulcan} bird on a blue background. They also \hr{Pelidorian war \grulcans}{use \grulcans{} as beasts of war}. 









\subsection{Besuld}
A port city in southeastern Pelidor. 









\subsection{Ghost Tower}
\label{Ghost Tower}
\index{Ghost Tower}
\index{Pelidor!Ghost Tower}
A spooky tower in the Wild near \hs{\Forklin} in \hs{Pelidor}

There are a bunch of ancient ruins near the Ghost Tower\dash parts of a larger complex, of which the tower is the only complete remnant. 

\label{Mystic lake near the Ghost Tower}
Maybe there is a mystical, magical lake near the Tower. 
There might be \hr{Mystic lake in Malcur}{a similar lake in Malcur}.

Compare to Lake Hali in \RWCTKIY. 





\subsubsection{\Haskelek{} myth}
\label{Haskelek myth}
\label{\Haskelek{} myth}
There are myths about how there allegedly lies a terrible \daemon{} imprisoned somewhere in the Ghost Tower\dash a \hs\Haskelek











\subsection{\Forklin}
\label{Forklin}
\label{\Forklin{}}
\index{Forklin}
\index{Pelidor!Forklin}
A city in northern \hs{Pelidor}, near the \hs{Ghost Tower}.









\subsection{Malcur}
\label{Malcur}
\index{Malcur}
\index{Pelidor!Malcur}
Malcur is the capital city of Pelidor, ruled by House Pelidor from Castle Pelidor in the heart of Malcur. 

Malcur is an ancient city. Castle Pelidor is Vaimon-built, but the catacombs beneath it are thousands of years older. They are disused and forgotten, and the Pelidors barely remember that they exist. %They were built by \ophidian-worshippers, or maybe Sentinel-allied people, before the Vaimon Empire.





\subsubsection{Ancient history}
\label{Ancient history of Malcur}
Malcur was originally a \thzantzaic{} stronghold, and the crypts date back to this time. Or maybe it was built on top of a \thzantzai{} tomb, containing powerful arcane glyphs and spells forming a part of the seal that separates \Machai{} (the \thzantzaic{} homeworld) from Mith\dash the very spells that formed the first foundation of the Shroud. The magic of these seals can be tapped for other purposes and used to weave the Shroud, making Malcur and its underground a potent \nexus{} point. 

Some terrible power lies entombed beneath Malcur. After the \Darkfall, someone attempted to release this power, but it failed. House Pelidor took over the castle and have been its stewards ever since. 

\Zeirathz{} terrible monsters (see section \ref{Zeirath's creatures}) are connected to Malcur's mystic past somehow.





\subsubsection{Description of the scenery}
The great buildings of Malcur, including the city walls, guard towers and the central \CastlePelidor, are built of shining white stone\dash think of a name for this fictional type of stone. 

It has plenty of monumental structures (see section \ref{Monuments})
In many places it is adorned and embossed with pink quartz and black opals. And there are enormous statues, obelisks, gargoyles or angels, friezes on the walls, ornate cathedrals and towers, and a behemoth central castle.

Describe it so it sounds exotic, epic and Bal-Sagoth-like. Almost like Kor-Avul-Thaa (but not quite).

Malcur is built utilizing a lot of \hyperref[Occult geometry]{occult geometry}. 





\subsubsection{Rich and poor quarters}
\label{Malcur rich and poor}
Malcur has a rich quarter, the High City, and a poor quarter, the Low City. 

The High City is made of very tall houses, palaces and towers. There are bridges between the towers so that the rich need never tread on the filthy ground where the poor live. 

The Low City spans everything from comfortable craftsmen to the worst slums. \hs{Rian} lives here.





\subsubsection{Recent history}
Malcur has never fallen in war and never been taken in a siege. At least, not since the Vaimons destroyed the city and rebuilt it. When the Belkadians invaded, the Duke Pelidor surrendered peacefully after lengthy negotiations. 





\subsubsection{\Nexus{} status}
Malcur is an important \nexus{}. This is why \Zeirath{} has brought \ghobaleth{} to Malcur (see section \ref{Zeirath's creatures}). 

Perhaps these \ghobaleth{} have been there for thousands of years, or have gradually dug their way to Malcur from Erebos over the course of thousands of years. 

But apart from the \ghobaleth, there is also \xzaishannic{} darkness lurking beneath Malcur. And perhaps degenerate creatures who live in caves underground and worship them. 

Maybe there dwell degenerate \troglodytes, digging tunnels under the crypts and worshipping their detestable gods, perhaps including \dragons.Or perhaps these are \nagae, living in undergound lakes and the like.





\subsubsection{Dead garden}
\label{Dead garden in Malcur}
\label{dead garden}
\index{Malcur!dead garden}
Somewhere in the centre of Malcur, near Castle Pelidor (but not too near), there lies a garden of crooked, dead trees. It is one of the mystic centres of the city, somehow connected with Malcur's occult past. 

The twisted trees are black and gray, leafless and withered-looking. They look dead, but they are alive. They radiate some kind of malice, and people steer clear of them. Only this one strange old woman lives near them. 

In or near the garden there lives a \hs{crazy old woman}. She knows quite a lot about the Beyond and the supernatural, but is also quite \hyperref[Madness]{mad}.

The garden is inspired by the Azath house and cemetery in Steven Erikson's \emph{Midnight Tides} (\emph{Malazan Book of the Fallen} 5). 

The garden is the core of the \nexus. It is here that one can most easily feel the \hyperref[Zeirath's creatures]{worms}. 

Despite their wicked look and feel, the trees are not necessarily evil, although they may tend to be inimical to humanoids. They possess some measure of natural power and wisdom. The crazy old woman can hear their thoughts (because she is mad and more free from the Shroud), and she relays their warnings to anyone who will listen (which is not many). The twisted trees may not love humanoids, but they fear the \ghobaleth{}, and they fear \Nithdornazsh.

\lyricsbalsagoth{In the Raven-Haunted Forests of Darkenhold, Where Shadows Reign and the Hues of Sunlight Never Dance}
{Can you not see the coils of the worm all about you?\\
Can you not hear the writhing of the worm beneath you?\\
Can you not scent the breath of the worm riding the wind?\\
Can you not touch the skin of the worm in all that surrounds you?\\
Can you not taste the ichors of the worm upon your tongue?\\
Do dreams of the worm not haunt your slumber?}





\subsubsection{Mystic lake}
\label{Mystic lake in Malcur}
\label{Lake in Malcur}
Maybe there is a mystical, magical lake near the garden. Maybe it lies between the garden and \CastlePelidor. 
There might be \hr{Mystic lake near the Ghost Tower}{a similar lake near the Ghost Tower}.

Compare to Lake Hali in \RWCTKIY. 









\subsection{\Redglen}
\label{Redglen}
\index{\Redglen}
\index{Pelidor!\Redglen}
A town in eastern Pelidor, near \hs{Heropond Forest} and west of \hs{Torgin}. It is the hometown of \hs{Carzain \Shireyo} and his family. 









\subsection{Torgin}
\label{Torgin}
\index{Torgin}
\index{Pelidor!Torgin}
A city in southern Pelidor, east of \Redglen. 















\section{\Redce}
\label{Redce}
\index{\Redce}
The homeland of \hs{Clan Redcor}. 

The capital city is that surrounding the Topaz \Chateau. This city is very much divided into rich and poor quarters. Like \hr{Malcur rich and poor}{Malcur}, but more extreme. The \Chateau{} is extremely tall, and the rich Redcor live infinitely far removed from the hardships, poverty and suffering of the common folk. 









\subsection{\TopazChateau}
\label{Topaz Chateau}
The central stronghold of the Redcor. 















\section{Rissitics}
\index{Rissitic}
Durcac, the Rissitic Dominion, is ruled by \HriistN, who is actually \HriistD{} in disguise. But Durcac is infiltrated. Chiefly by agents of other Sentinel factions, but also by Cabalists. 

Have some backstabbing like among Sha'ik's army in \emph{House of Chains} or the Malazans in \emph{The Bonehunters} (\emph{Malazan Book of the Fallen} 4 and 6, by Steven Erikson). 

Remember that the Cabal is more numerous and better organized than the chaotic Sentinels. 







\subsection{Durcac}
\index{Durcac}
Remember to have an explanation for why Durcac is full of desert. It was devastated and defiled in some terrible war in ages past. 







\subsection{Magic}
The Rissitic mages use the magic theory of the Three Worlds, which is developed by Rissit Nechsain from millennia of research. 

The power they wield is great, and many of them, being mortal humanoids, are not up to the task. Their minds and bodies succumb to the pressure and begin to \hyperref[The price of magic]{twist and warp under the influence of the Chaotic power} that they feebly try to master. They become scarred and misshapen, their bodies rotting inside. Compare to Hannan Mosag and his K'risnan in \SEMalazan. For this reason, mages wear full-body-covering robes and masks. At least the \Shessefkesad, and likely the \Dzeyrgvin{}, too. 

They work hard to preserve their sanity through prayer and mental exercises. Sometimes it fails in the long run. 

Some of the priests are offered the gift of undeath to preserve them, since a dead mummy does not mutate. 

The \Ashenoch{} are resistant to physical mutation, since their bodies are superhumanly empowered. They may still go mad, tho. 





\subsubsection{Name theory}
Rissitic magic uses, among other things, some \quo{true names} that, when known, can be used in spells (spoken or written) to give the caster power over the named subject. 

Each creature (and perhaps object?) has a Body-name, a Spirit-name and a Shadow-name. 

These names are most efficient against mortals. 
Mortals are generally \quo{simple} beings whose names are short and manageable. 
Immortals have true names, too, but these typically consist of very long and complex spells, or even rituals. 
Sometimes so complex that they are not worth the effort. 

No Rissitic mage has ever been able to use the true names of a \bane{} to any worthwhile effect. 
Attempting to use a \banez{} name (or even learn it) tends to badly damage the caster's sanity. 









\subsection{Order of the Jackal}
Perhaps have an Order of the Jackal, whose members wear jackal-like metal masks and armour. Compare to the movies \emph{The Scorpion King} and \emph{Stargate}. 

\lyricsbs{Monolith Deathcult}{Deus Ex Machina}{
For thy blasphemy thou shall be punished \\
The Systemlords bring thee thy avengers \\
Your soul will be possessed and brutally mentally slaughtered \\
Resistance shall be butchered by Jackal-headed iron warriors \\
The demons nestled themselves in your bodies \\
While you stare in the red gleaming eye of the serpent mask \\
Falcon claws will slit your throat to pieces \\
Thou shall be enslaved again
}









\subsection{Politics}
\subsubsection{Ideology}
The Rissitic system is somewhat chaotic and anarchistic on a small scale. 
At least, it is not nearly as centrally controlled as the Imetrium (which is a primitive communist plan economy). 
The Rissitic commoners are poorer, less healthy and less secure than Imetrians, but they are also more \quo{free}, in the sense that the laws and, more importantly, customs are less restrictive. 
For example, sexual morals are liberal. 









\subsection{Pyramids}
\label{Rissitic pyramids}
The Rissitics have some great pyramids. All Rissitics who die have their bodies interred in the pyramids, as dictated by their religion. 

In truth, these corpses are used in secret to amass vast legions of the undead. Like in \emph{Warhammer: Tomb Kings}. 







\subsection{Threat of destruction}
Perhaps the Rissitics, at least the upper echelons, are desperate to conquer for fear of being devoured from the inside by the Chaotic power that they utilize. 

\lyricsbs{Marduk}{Scorched Earth}{We must win to save us from the plague's grasping jaws...}















\section{Runger}
\label{Runger}
\index{Runger}
A kingdom in central Belkade. It borders Beirod to the south, Pelidor to the west (marked by the river Nerim), the Gwendor Sea to the north and the Lorn Sea to the east. 

%It borders the Hirum Gulf to the northwest, Pelidor to the southwest and Beirod to the south. 

It is ruled by the kings of House Runger. The current ruler is King Morgan I son of Uther I. Morgan has two daughters but no sons, so the heir to the throne is Prince Matthias, the husband of Morgan's eldest daughter, Estelle. Recently, King Morgan has allied himself with the Rissitics and plots to conquer Pelidor and perhaps other nearby kingdoms as well. 

The capital city is Dormina. 









\subsection{\EreshKal}
\label{\EreshKal}
\index{\EreshKal}
A \meccaran{} civilization that once existed in modern-day Runger. Allegedly, a lost \EreshKali{} temple lies hidden in Waythane Forest. 









\subsection{Waythane Forest}
\index{Waythane Forest}
\index{Runger!Waythane Forest}
A large forest in southern Runger. Allegedly hides the last temple of \hs{\EreshKal}. 















