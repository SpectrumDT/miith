\chapter{Glossary}



\begin{comment}
% \section{Creatures}
% \begin{gloss}



% \begin{comment}
\subsubsection{A-G}
% \end{comment}

\gitem{\Archon}
In Vaimon metaphysics, \Archons{} are supernatural beings and forces of nature. Different classes of \Archons{} include the \Sephiroth{} and \Kliffoth{}, who can be invoked to cast magic.  



\gitempl{Cortio}{Cortios}
A large theropod dinosaur, similar to a Tyrannosaurus or Allosaurus (having the size of the former but the strong forearms of the latter). In some cases Cortios can be tamed, and they are used as mounts especially by the Rissitic \Ashenoch.



\gitempl{\dax}{\daxes}
A male \scatha{}, usually an adult. 



\gitem{\dragon{}}
\Dragons{} are great winged reptillian creatures and some of the most powerful intelligent creatures on Mith. The are fearsome in combat and command terrible sorcery. Most \dragons{} dwell in the kingdom of Irokas. 



% \begin{comment}
\subsubsection{H-N}
% \end{comment}

\gitem{Ivory cobra}
A cobra snake that is white in colour and grows up to 150 cm in length. The ivory cobra is highly poisonous and its bite can easily kill a \human{} or \scatha{} (the teeth of an adult are strong enough to penetrate clothes and a \scathaz{} scales). 

The ivory cobra lives primarily in Durcac. The animal is sacred to the Rissitics and a symbol of their religion and empire. 



\gitempl{\Malach{}}{\Malachim{}}
%In Vaimon metaphysics, the \Malachim{} are a class of \Archons{} that may incarnate as \humans{}. An incarnation of a \Malach{} is called a Scion. 
The \Malachim{} are \resphan{} lords who has left their \resphan{} bodies to incarnate again and again as \humans{}. 

I don't know exactly what the original purpose of the \Malachim{} was, but somewhere, the process went wrong, and the \Malachim{} all had their memories of their previous lives as \resphain{} erased. They now recall only scattered fragments of their old lives, typically in fever-like dreams and \deajvus. 

%Like all the other \Malachim, Ramiel has lost his memory of his previous life as a \resphan{} lord, recalling only scattered fragments, typically in fever-like dreams and \deajvus. 

In Vaimon metaphysics, the \Malachim{} are considered a class of \Archons{}. A \human{} incarnation of a \Malach{} is called a Scion. Each Scion retains some memories of his previous incarnation. Typically these memories are locked away at birth and only awaken later, triggered by some massively emotional event. (For example, Carzain \Shireyo, a Scion of Ramiel, has his predecessor, \VizicarFull, awaken when Carzain fights his first battle to the death and kills a man.) 

From that point, the Scion has a split personality with two distinct persons inhabiting the same body. If the two personalities get along, they will absorb traits of each other and eventually merge into a single character. If the two do not get along, the Scion will degenerate into a mood-swinging maniac.

%There is a small number of \Malachim{} known. Ten or so. They include Ariel, Sachiel and Ramiel. 

The names of the \Malachim{} are Ariel, Nelchael, Ramiel, Sachiel, \Belzirmalach{} and more. A Scion does not necessarily remember his true name, but the Vaimons possess the knowledge to research a Scion's mind and ascertain his \Malach{} identity. 

Some of the \Malachim, including Ramiel, are of the \bloodresphain. 

%He was one of the original \bloodresphain{} who drank the blood of \Astorglax. 



\gitempl{\Maskim}{\Maskim}
A group of supernatural beings in Rissitic metaphysics, considered dangerous and evil and sometimes invoked in curses. \quo{\Maskim{} take you} is a strong curse in Rissitic.



\gitemplor{\meccaran}{\meccarans}{\meccara}
A race of amphibian humanoids. A \meccaran{} resembles an anthropomorphic frog with short arms but long and strong legs. An average adult female stands about 130 cm tall, the smaller male 120 cm. They stand and walk in a crouched position with the legs bent outwards. They are coloured in shades of green or brown. 

\Meccaran-dominated cultures tend to be more technologically primitive than \scathaese{} or \human{} cultures. 



\gitempl{mulgron}{mulgrons}
A huge dinosaur, similar to a Triceratops. Full grown they are six to ten metres long and weigh 5 to 13 tons. Their skin is tough, scaly and gray. 



\gitempl{\nephil}{\nephilim}
An ancient race of \human{}-like humanoids that inhabited the lands of Mith before the coming of the \dragons{}. They were all but exterminated by the \Dominators. Later, the \banes{} discovered the surviving \nephilim, who had fled into the Wild and declined into barbarism. They took the \nephilim{} and twisted them into a race of slaves for themselves. They became the \humans{}. 



% \begin{comment}
\gitem{\nycan{}}
An animal similar to a Deinonychus. \Nycans{} are fierce predators that hunt in packs, highly intelligent and with telepathic abilities. They are found mostly in the Imetrium and Irokas. In the Imetrium, \nycans{} are domesticated and used as beasts of war. 

\index{Secca (plural Seccae)}
\index{Crycos (plural Crycoi)}
\index{Destran}
Imetric \nycans{} are bred into a number of races. The Seccae (singular Secca) are the largest and the only \nycans{} strong enough to easily carry a rider. The Crycoi (singular Crycos) are the fastest, used as couriers (carrying letters) and as shock troopers. The Destrans (singular Destran) are bred for intelligence and sharp senses. 

The races are further divided into a number of \quo{breeds}, often named for the city, family or individual that breeds them or \quo{founded} the breed. 

\label{Mictzan}
\index{Mictzan}
Examples include the Mictzan breed of Destrans, named for the founder (an Imetric \scatha{} of Clictuan descent), and the Dorlinum breeds, of various races, bred in the Imetric city of Dorlinum. Especially renowned are the Dorlinum Secca, who are some of the largest and strongest \nycans{} known. 

See also section \ref{Nycan}.
\also{\nycaneer}
% \end{comment}



% \begin{comment}
\subsubsection{O-U}
% \end{comment}

\gitempl{\Ophidian}{\Ophidians}
%Before the \dragons{} there were the \ophidians: Intelligent snakes. They are snake-shaped and use telekinesis. They have powerful telepathy and a kind of racial memory. 

%Occasionally, the \ophidians{} ruled over the \nephilim{} as gods, but at some point the \nephilim{} began to hate them and waged a war of genocide against them.

%\subsubsection{\Ophidians{} as \dragons}
%Or perhaps the \ophidians{} are not a distinct species. Perhaps they \emph{are} the \dragons. In that case, the \dragons{} were originally more peaceful and uncaring, but \Tiamat{} allied them to \chaos{} and they become more violent, more cruel. 
The \ophidians{} are an intelligent race of snake-like creatures native to the Beast Realm. They are the ancestors of \dragons. 

The original \ophidians{} were intelligent snakes. They had no arms and legs but used telekinesis instead. They also had powerful telepathic abilities. Perhaps they have some kind of racial memory. Or perhaps they are simply immortal. 

Maybe they shed their skin to renew their youth, thus achieving immortality. In that case, the shed skin of an \ophidian{} may have mystic power. 

In ancient times, the \ophidians{} were one of the ruling races on Mith. Sometimes they ruled over the \nephilim. 

At some point, there was a war against the \thzantzais. I don't know where they came from. Maybe they were native to Mith, or maybe the \ophidians{} made the mistake of invading the \thzantzaic{} homeworld (\Machai?), opening a doorway to Mith for them. 

%\end{comment}
The \ophidians{} warred against the \thzantzais, possibly alongside other Mithian races. Eventually, some of the \thzantzai{} lords betrayed their kind and sided with the Mithians, giving them the edge they needed to banish the \thzantzais{} from Mith. \HesodNerga{} and his daughters \Tiamat{} and \KhothSell{} were heroes in this war. After the war, the Mithians turned on the turncoat \thzantzai{} lords and slew them. After all, who trusts a traitor?

Gradually, many \ophidians{} reshaped their bodies. They grew legs and became more lizard-like, more similar to the \dragons{} of today. Only few\dash the oldest\dash remained in snake form. The younger \ophidians{} forgot much of their telekinetic skill, using their hands and feet instead. 

Then some \ophidians{} craved more power. Perhaps their leader was \HesodNerga. But no, I think their leader was \Tiamat. She, together with her accomplices, raised up the fallen lords of the \thzantzais{} (perhaps the very ones who had initially betrayed their brethren). \Tiamat-tachi bound the weakened \thzantzais{} with spells, drained them of their power and absorbed their souls into themselves. This transformed them into the mighty \draecchonosh. 

%\begin{comment}
(The first generation of \draecchonosh{}, who came to be known as the \firstgendragons{}, directly absorbed \thzantzai{} essence. The second generation, the \secondgendragons{}, received it channeled from them, and later generations inherited it naturally.) 

When did this happen? Was it at the time of the \bane{} invasion, when the \ophidians{} needed more power? 

Anyway, the \draecchonosh{} became chaotic, \daemonic{} beings. Twisted by the \thzantzaic{} souls within them, they almost became \thzantzais{} in \ophidian{} form. Some of the other \ophidians{} were swayed by their power and glory and joined them, while others were repelled by their brutality and evil. 

Later, the term \quo\dragon, derived from \quo\draecchonosh, came to be used of all \ophidians, regardless of whether they possessed \thzantzaic{} blood or not. 

%\subsubsection{\Ophidian-\resphan{} connection}
Perhaps some of old \ophidians, repulsed by their \draconic{} brethren and their violent behaviour, have sided with the \resphain. Perhaps they helped found \Mystraacht. 



\gitempl{\Qliphah}{\Qliphoth}
In Vaimon metaphysics, the \Qliphoth{} are the \Archons{} of \nieur{}. They can be invoked to cast magic. There are many scores of known \Qliphoth; their total number is unknown and some believe it to be infinite. 



\gitempl{\scatha{}}{\scathae{}}
A race of reptillian humanoids. A \scatha{} resembles an anthropomorphic lizard with a long tail. They do not stand fully erect; their backs tend to slope at 30-45 degrees (closer to horizontal than vertical). An average adult is 160-170 cm tall and 190-200 cm long including the tail (males and females are of equal size). They have tough scaly skin and are herbivores. They are the dominant race in the Imetrium and Durcac. 

\Scathae{} are found in different colours. Those from the Imetrium usually have blue colours while those from Durcac are red. Green \scathae{} also exist in Irokas. 

\Scathae{} cannot blink. Instead, they lick their eyes to clean them. 

%An idea: A male \scatha{} is called a daecos and a female \scatha{} is a sphyle (these are not capitalized, just like \quo{man} and \quo{woman}). 



\gitem{Scion}
A \human{} incarnation of a \Malach{}. 
\also{Malach}



\gitempl{\Sephirah}{\Sephiroth}
In Vaimon metaphysics, the \Sephiroth{} are the \Archons{} of \iquin{}. They can be invoked to cast magic. There are sixteen of them. The \Sephiroth{} are associated with the four classical elements; four for each element. 

\index{\Atzirah}
\index{\Feazirah}
\index{\Keshirah}
\index{\Razilah}
The \Sephiroth{} of Air are are \Atzirah{}, \Feazirah{}, \Keshirah{} and \Razilah. 

\index{\Barion}
\index{\Hapheron}
\index{\Izion}
\index{\Teshiron}
The \Sephiroth{} of Fire are \Barion{}, \Hapheron{}, \Izion{} and \Teshiron{}. 

\index{\Cushed}
\index{\Hoshied}
\index{\Thimared}
\index{\Yemared}
The \Sephiroth{} of Earth are \Cushed{}, \Hoshied{}, \Thimared{} and \Yemared. 

\index{\Gamishiel}
\index{\Ishiel}
\index{\Omariel}
\index{\Yeziel}
The \Sephiroth{} of Water are \Gamishiel{}, \Ishiel{}, \Omariel{} and \Yeziel{}. 



\gitempl{\sphyle}{\sphyles}
A female \scatha{}, usually an adult. 



% \begin{comment}
\subsubsection{V-Z}
% \end{comment}

% \end{gloss}
\end{comment}











\begin{comment}
\section{Geography}
% \begin{gloss}

% \begin{comment}
\subsubsection{A-G}
% \end{comment}

\gitem{Andras}
A kingdom in the western Belkade. It borders the Imetrium, Threll, Ontephar, Scyrum and the \Risvaelsea. The Threll Mountains mark its northern border with Threll and the rivers Bron and Pylor mark its eastern border with \Scyrum{}. 

Andras is ruled by King Tiberius Andras. The kingdom is allied with the Imetrium, and the Imetric religion thrives alongside Iquinianism. Languages and ethnic groups include Belkadian, \Samurin, Imetric and the occasional pockets of \Tepharites{} and \Orticans. 



\gitem{\Azmith{}}
A continent on the planet Mith consisting of Belkade, the Northern Kingdoms, Threll, the Imetrium, Uzur, Durcac, the Near Orient and the Serpentines. %Some scholars also consider Irokas a part of \Azmith{} while others do not. 

\quo{\Azmith} is Archaic Vaimon for \quo{all the world} (containing the word \quo{Mith}, meaning \quo{the world}). 



\gitem{Beirod}
A kingdom in the southern Belkade. It borders Pelidor to the northwest, Runger to the northeast, \Scyrum{} to the west (marked by Heropond Forest), Gaznor to the south, the \Risvaelsea{} to the southeast and the Lorn Sea to the east.  

Languages and ethnic groups include Belkadian and \Ortic. 



\gitem{Belek}
A kingdom in northern Belkade. Belek is also the name of the royal house of the land. The capital city is Lendamere. The Belkadian Empire, during its time, was ruled from Lendamere by the High Kings of House Belek. House Belek still rule the (diminished) kingdom of Belek, but they no longer use the title High King. 
\also{High King, Belkadian Empire}



\gitem{Belkade}
A region located in the middle of \KnownWorld{} and comprising a large number of independent kingdoms (and great expanses of Wild). It is not well-defined or well-delimited, but is usually considered to cover (at least) all lands where the Belkadian tongue is spoken. 

%It includes, among other things, the nations of Belek, \Redce, Pelidor and Runger. 

Belkade includes the nations of Andras, Beirod, Belek, Ontephar, Pelidor, \Redce, Runger and \Scyrum. 



\gitem{Belkadian Empire}
A great, \human{}-dominated empire that existed from 3300 to 3700 \IC{}. At the height of its power, it covered all the lands that are now Belkade. The founder of the Empire was Uther the Tiger, who became its first High King. The rulers of the Empire were the High Kings of House Belek. The official language of the Empire was Belkadian, still spoken in most of Belkade. 

The Empire was allied with the Iquinian Churches of Redcor and \Yrgell{} and these were its official religions. In the beginning, the Empire had strong ties to the Redcor, but gradually, the Redcor dominance was phased out in favour of the \Yrgell. 

In 3700 \IC{}, the Empire collapsed for some reason. 
\also{Belkadian (language)}



\gitem{Belkadian (language)}
Once the official language of the Belkadian Empire and still spoken in much of Belkade. 

Belkadian names and words are meant to resemble English. 



\gitem{Bendaire}
A city in Belkade in a kingdom near \Redce. 



\gitem{Bron}
A river in western Belkade that, for much of its length, marks the border between Andras and \Scyrum{}. It runs from the Threll mountains into the \Risvaelsea{} and meets the river Pylor. 



\gitem{Clictua}
A \meccaran-dominated kingdom in the northeastern Uzur. One of the largest known kingdoms in Uzur. It borders the Imetrium to the north and Durcac to the east. 

The Clictua language is meant to resemble Nahuatl. 



\gitem{Dormina}
The capital city of Runger. 



\gitemthe{\Dragonridge}
Also called the \Dragonridge{} Mountains. A long and tall mountain ridge to the east of Belkade that marks the border between Irokas and the \Serpadj{} kingdoms. 



\gitem{Durcac}
A nation south of Belkade and east of the Imetrium but west of the Orient. Durcac is a theocratic state, ruled by the Rissitic religion, and the nation is also called the Rissitic Dominion or the Rissitic Empire. Durcac claims a huge area as its territory, but much of it is desert and mostly uninhabited. 

In addition to the cities and towns that form the Rissitic Empire, Durcac is home to a number of nomadic peoples, some of whom are Rissitic in name only, or not at all. 

The language of Durcac is Rissitic. 
\also{\HriistN, Rissitic (religion)}



\gitem{Falcus Aira}
The capital city of the Imetrium, built at the northern coast of the Naemor Strait. 



\gitem{Fendor}
An island in the \Risvaelsea. It is currently controlled by the Imetrium. It has two major towns: Fendacor in the east and \Cicora{} in the west.



\gitem{Gaznor}
A peninsula in southern Belkade, controlled by the Rissitic Dominion. It borders Beirod to the North and the \Risvaelsea{} on all other sides. 

Languages spoken include Rissitic, Belkadian and \Ortic. 



\gitem{Geica}
A kingdom in the Orient and the homeland of Clan Geican. The Geicans are the ruling class. Geica has a democracy and is ruled by a Senate. Only members of Clan Geican may vote or run for Senator. 
\also{Geican}



\gitem{Gwendor Sea}
A sea in the northen Belkade. It encloses Belek and borders Threll, Ontephar, Runger and \Redce{}, among others. To the west it flows into and becomes the Thenglain Sea. 



% \begin{comment}
\subsubsection{H-N}
% \end{comment}

\gitem{Hazid}
A nation in the Near Orient, near Geica. Ruled by a Sultan. 



\gitem{Heropond Forest}
A huge, Wild forest that marks the border from \Scyrum{} into Pelidor and Beirod. 
%makes up the eastern end of \Scyrum{} and the southwestern end of Pelidor (and perhaps more kingdoms as well). 
At its narrowest place it can be crossed via Leglan's Path. 
\also{Leglan's Pass, Wild}



\gitem{Hirum Gulf}
A gulf in central Belkade, a part of the Gwendor Sea. It borders the kingdoms of Ontephar, Pelidor and Runger. The rivers of Pylor and Nerim run into the Hirum Gulf. 



\gitem{Icconos}
A city in northern \Scyrum{}. 



\gitemi{The Imetrium (nation)}{Imetrium!nation}
A nation in western \Azmith{}. It borders Threll to the north (marked by the Threll Mountains), Andras to the northeast, the Risvael Sea to the east, Clitua and Kochu to the South (the Kochu border marked by the river Tsukachi) and the Thenglain Sea to the west. It is a theocratic state ruled by the religion also called the Imetrium. The population is predominantly \scathaese{}. 

%The Imetric religion is polytheistic. The five major gods make up \quo{the Tribunal}, which governs the Imetric pantheon and nation. The most important god is Salacar; the other members of the Tribunal are Dessali, Eoncos, \NishiS{} and \Hiothrex{}. 

%The symbol of the Imetrium is a white, four-spoked star, edged with blue within a circle of black against a white background. The colours of the Imetrium are white, blue and black. 

The official language is Imetric. Other languages, such as \Samurin, are repressed. 
\also{Imetrium (religion), Imetric (language)}



\gitemi{Irokas (kingdom)}{Irokas!nation}
A region located east of the \Dragonridge, inhabited and ruled by \dragons{}. Nominally a kingdom, ruled by the \DragonKing{} of Bloodline Irokas, but actually ruled by a number of warring \dragonlords{}. 
Often called \quo{\Dragonland} by the less learned. 
\also{Irokas Bloodline}



\gitem{Leglan's Pass}
A path through the narrowest part of Heropond forest, near Bryndwin (on the \Scyric{} side) and \Redglen{} (on the Pelidorian side. It was formerly used as a trade route, but has fallen out of use in recent decades and grown more Wild. 
\also{Heropond}



\gitem{Lorn Sea}
A narrow sea in eastern Belkade, marking the eastern borders of Runger and Beirod and the western borders of Thyrin, Sumian and Geica. It meets the Gwendor Sea to the north and the \Risvaelsea{} to the south. 



\gitem{Malcur}
The capital city of Pelidor. 
%\also{Pelidor}



\gitem{Martinum}
A city in the Imetrium, built on both sides of the narrow Martinum Strait which separates the \Samure{} Gulf (west) from the Risvael Sea (east). It is famous for the great Martinum Bridge, spanning the Strait and considered a marvel of architecture. 

Martinum was previously the capital of a kingdom by the same name. It is one of the largest, most prosperous and most strategically important cities of the Imetrium, militarily as well as economically. Martinum is ruled by a \Laccorin{}, the current ruler being Vian Martin, a descendant of the old royal line of Martinum. 



\gitem{Naemor Strait}
A strait in the Imetrium, separating the \Samure{} Gulf from the Thenglain Sea. The Imetric city of Falcus Aira is built at the Naemor Strait. 



\gitem{Near Orient}
A collective term for the nearer part of the area southeast of Belkade (east of Durcac and south of the \Serplands{} and Nom). Kingdoms in the Near Orient include Hazid. 



\gitem{Nerim} 
A fjord that runs south from the Hirum Gulf and marks the border between Runger and Pelidor. 



% \begin{comment}
\subsubsection{O-U}
% \end{comment}

\gitem{Ontephar}
A kingdom in southern Belkade. It borders Pelidor to the south, Runger to the east and \Scyrum{} to the south-west. It is ruled by an Archduke. Ontephar was previously the heartland of the \Tepharin{} Bacconate. 

Languages include Belkadian, \Tepharin{} and occasionally Imetric. 



\gitem{Ortaican Bacconate}
A great \scatha{}-dominated empire that flourished before the \darkfall{}. At the height of its power, it covered the lands that are now the Imetrium, most of Durcac and some of southern Belkade. \Scathaese{} cultures today, especially the Imetrium, owe much of their culture to the Ortaicans. (Among other things, the Imetric language is descended from Ortaican and uses the Ortaican alphabet.) 
\also{Bacconate}



\gitem{Pelidor}
%A nation in southern Belkade, not far from the Imetric border. It borders Runger to the northeast, Beirod to the east and \Scyrum{} to the southwest. The southwestern end of the country is part of the Heropond Forest. 

A kingdom in southern Belkade. It borders Runger to the northeast (marked by the river Nerim), Beirod to the southeast, \Scyrum{} to the west (marked by Heropond Forest) and the Hirum Gulf to the north. 

It is ruled by the Dukes of House Pelidor. The current ruler is Duchess \Iakis, widow of Duke \Icor. The capital city is Malcur. 

Languages include Belkadian, \Tepharin{} and \Ortic. 



\gitem{\Redce{}}
A nation in the northern Belkade, the homeland of Clan Redcor and ruled by the Redcor Conclave. 

The official language is Redcor Vaimon. Belkadian is also spoken. 



\gitem{\Redglen{}}
A town in eastern Pelidor, near Heropond Forest and west of Torgin. It is the hometown of Carzain \Shireyo{} and his family. 



\gitem{\Risvaelsea}
A large sea in the centre of \Azmith{} that marks the southern border of Belkade, the northern border of Durcac and the easter border of the Imetrium. In it lie the two large isles of Fendor and Tugan. 



\gitem{Runger}
A kingdom in central Belkade. 
It borders Beirod to the south, Pelidor to the west (marked by the river Nerim), the Gwendor Sea to the north and the Lorn Sea to the east. 

%It borders the Hirum Gulf to the northwest, Pelidor to the southwest and Beirod to the south. 

It is ruled by the kings of House Runger. The current ruler is King Morgan I son of Uther I. Morgan has two daughters but no sons, so the heir to the throne is Prince Matthias, the husband of Morgan's eldest daughter, Estelle. Recently, King Morgan has allied himself with the Rissitics and plots to conquer Pelidor and perhaps other nearby kingdoms as well. 

The banner of Runger is two brown wolverines chasing each other in a circle, against a tan background.

The capital city is Dormina. 



\gitem{\Samur}
\label{\Samurin}
\index{\Samur!\Samurin{} (language)}
A predominantly \scathaese{} people that originate from the area on the northern bank of the \Samure{} Gulf but later spread eastward into what is now the northeastern Imetrium, Andras and western \Scyrum. Their language, \Samurin{}, is still spoken in parts of Andras and \Scyrum, but repressed in the Imetrium. 



\gitem{\Samure{} Gulf}
A gulf in the Imetrium. It borders the Thenglain Sea to the west at the Naemor Strait near Falcus Aira and the Risvael Sea to the east at the Martinum Strait. 



\gitem{\ShiinMerodar}
Once the capital city of the Vaimon Empire, seat of the magnificent Rainbow Palace, \Merodar has now sunk beneath the sea. It lies somewhere between Belek and Ontephar. 



\gitem{\Shurco{} \Bacconate}
A \scathaese{} culture that existed during the time of the Vaimon Empire, controlling much of \DurcacContinent. Modern-day Durcac owes much of its cultural heritage to the \Shurco.



\gitem{\Scyrum{}}
A kingdom southeast of Belkade. 
Borders the Imetrium to the south, near Martinum, and borders Pelidor to the north-east. 
Much of the eastern part of the kingdom is made up by the large Heropond Forest. 

The capital city is Pylandos. Other cities include Icconos. 

Languages and ethnic groups include \Tepharin, \Samurin{} and \Ortic. 



\gitem{\Serpsea}
A sea that marks the eastern border of Belkade. The \Serp{} is long in the north-south direction but narrow in the east-west direction, and it twists and bends like a snake, hence the name. East of the \Serp{} lie a number of kingdoms called the \Serplands{}. 
\also{\Serplands}

%\gitem{\Serpriver}
%A great sea that runs along the eastern border of Belkade. In fact, the \Serpriver is often considered to be the easter border of Belkade by definition. East of the \Serpriver lie a number of kingdoms called the \Serplands. 
%\also{\Serplands}



\gitemthe{\Serplands}
Also called the \Serpadj{} kingdoms. Collective term for a number of kingdoms that lie between the \Serpsea{} to the west and the \Dragonridge{} to the east. The \Serpadj{} kingdoms are diverse in culture. Some are effectively vassal states of Irokas. 



\gitem{\Tepharin{} \Bacconate}
A nation, dominated by the \Tepharite{} people, that existed after the \darkfall{} but before the advent of the Belkadian Empire. At the height of its power, it covered much of modern-day Pelidor, Runger, Ontephar, \Scyrum{} and Beirod. The \Tepharin{} tongue is still spoken in parts of these countries. 
\also{\Tepharite}



\gitem{Thenglain Sea}
A great sea that marks the western border of \Azmith. 



\gitem{Threll}
A region east of Belkade, a devastated land haunted by monsters. The Threll Mountains mark the its southern border with the Imetrium and Andras. It borders Ontephar to the east, the Gwendor Sea to the north and the Thenglain Sea to the west. 



\gitem{Torgin}
A city in southern Pelidor, east of \Redglen. 



\gitemthe{Tribunal}
The Tribunal is the core of the Imetric pantheon, consisting of the five gods Salacar (the leader of the Tribunal), Dessali, Eoncos, \NishiS{} and \Hiothrex{}. 



\gitem{Tugan}
A small island in the \Risvaelsea. It is currently controlled by the Imetrium. It has one major town, Pandex. 



% \begin{comment}
\subsubsection{V-Z}
% \end{comment}

\gitem{Vaimon Empire}
An empire that once spanned most if not all of \Azmith, led by the Vaimon class of mages and ruled by a Vaimon Emperor. The capital city was \ShiinMerodar. The Empire was founded by Cordos Vaimon in the year \yic{Founding of the Vaimon Empire} and collapsed in the \darkfall{} in the year \yic{Darkfall}, making Vaimon Empress \Belzir{} its last ruler. 
%The \darkfall{} marked the collapse of the Vaimon Empire. 



\gitem{\Zarwec}
A \Serpentine{} kingdom. It is ruled by King Racul IV. The language spoken is \Zarweci. 



% \end{gloss}
\end{comment}











\section{Miscellaneous}
\begin{gloss}



\begin{comment}
\subsubsection{A-G}
\end{comment}

\gitempl{\Ashenoch}{\Ashenoch}
A Rissitic order of superhuman warrior-mages. 



\gitem{Baccon}
%The title of the kings and queens of certain, predominantly \scathaese{}, cultures, including the Ortaicans and the Turco. 
The title of the kings and queens of certain cultures, predominantly \scathaese{}. A kingdom ruled by a Baccon is called a Bacconate. In ancient times there were the great Ortaican and Turco Bakkonates. Today, there are only a few Bacconates left, some in southern Belkade and some in the Orient. 
%\also{Bacconate}



\gitem{Bacconate}
%A kingdom ruled by a Baccon. In ancient times there throve the great Ortaican and Turco Bacconates. Today, there are few Bacconates left, some in southern Belkade and some in the Orient. 
\seee{Baccon}



\gitempl{Becallios}{Becallioi}
Imetric term for \quo{devil}. Used as an exclamation, it is a strong curse and not to be used in finer company. 




\gitemthe{Cabal}
A shadowy organization that secretly serves the \banes. Cabalists are mostly \human{}. Many people indirectly serve the Cabal without ever knowing it. 



\gitem{Camaire}
In the \ImperialCalendar, Camaire is the festival that marks the end of the year. It falls on the last day of the month of \Gamishiel{}, midways between the summer solstice and the spring equinox.
\also{\ImperialCalendar}



\gitem{Caste system}
See section \ref{Caste system}. 



\gitem{Church of the Light}
%Another name for the Iquinian religion. 
\seee{Iquinian religion} 



\gitemaka{dagger sign}{sign of the dagger}
A Rissitic gesture of greeting, the dagger sign consists of a flat hand held up before one's face with the thumb-edge of the hand near the nose and the fingers pointing up. The hand is then moved down to form a fist touching the center of the chest, fingers inward. 

The sign is an imitation of Rissit's symbol of a snake coiled around a dagger. 
\also{\HriistN}



\gitem{\dai-}
In Imetric, \quo{\dai-} can be prefixed to a name, title or pronoun to form a polite vocative. (\quo{\Dai-} is capitalized if prefixed to an already capitalized word, such as a name, but otherwise not.)



\gitem{days (\ImperialCalendar)}
\index{\Corjin}
\index{\Setherab}
\index{\Rebecab}
\index{\Arcab}
\index{\Norquin}
\index{\Tirjin}
\index{\Kerzab}
\index{\Siljin}
In the \ImperialCalendar, a week is eight days long. Each day is named after of the Vaimon founders. %The days, in order, are \Corjin{} (after Cordos Vaimon, the first Emperor), \Setherab{} (after Sether Vaimon), \Rebecab{} (after Rebecca Redcor), \Arcab{} (after Arcan Delain), \Norquin{} (after Norcah Quaerin), \Tirjin{} (after Tiraad Geican), \Kerzab{} (after Kerzah \Irgel) and \Siljin{} (after Silqua Delain). 
%The days, in order, are \Corjin{}, \Setherab{}, \Rebecab{}, \Arcab{}, \Norquin{}, \Tirjin{}, \Kerzab{} and \Siljin{}. They are named, respectively, after Cordos Vaimon, the first Vaimon Emperor, Sether Vaimon, ), \Rebecab{} (after Rebecca Redcor), \Arcab{} (after Arcan Delain), \Norquin{} (after Norcah Quaerin), \Tirjin{} (after Tiraad Geican), \Kerzab{} (after Kerzah \Irgel) and \Siljin{} (after Silqua Delain). 
The days, in order, are:

\begin{enumerate}
	\item \Corjin{} (after Cordos Vaimon, the first Vaimon Emperor).
	\item \Setherab{} (after Sether Vaimon, son of Cordos and Silqua and the second Emperor).
	\item \Rebecab{} (after Rebecca Redcor, daughter of Cordos and Silqua).
	\item \Arcab{} (after Arcan Delain, Silqua's eldest brother).
	\item \Norquin{} (after Norcah Quaerin, son of Cordos and another wife).
	\item \Tirjin{} (after Tiraad Geican, son of Cordos and a third wife).
	\item \Kerzab{} (after Kerzah \Irgel, younger son of Cordos and Silqua).
	\item \Siljin{} (after Silqua Delain, Cordos' Empress). 
\end{enumerate}
\also{\ImperialCalendar, Vaimon}


%\gitem{Dorlinum breed}
%Dorlinum \nycans{} are those bred in the Imetric city of Dorlinum. Especially renowned are the Dorlinum Secca, who are some of the largest and strongest \nycans{} known. 



\gitem{\Draconic{} (language)}
The tongue spoken by \dragons{} and the official language of Irokas. 

The \quo{pure} version of the language is called Kingstongue. Kingstongue is a magical language, with almost every word and phrase being a spell of some sort. Kingstongue has remained mostly unchanged for many thousands of years because accurate pronunciation and phrasing are required for the spells to work. Chaos magic is very much based on Kingstongue. 

Kingstongue is an extremely complex language to learn, as the grammar is very cryptic. Even for \dragons{}, whose brains are superior to those of humanoids and pick up languages easier, it takes hundreds of years to learn correct Kingstongue. \Rachyth{} pick up Kingstongue reasonably well, but most humanoids never learn to construct more than the simplest sentences and spells. A good command of Kingstongue is a sign of status among \dragons{}. 

\quo{\Lowtongue{}} is the collective term for various simplified, degenerated dialects of the language. Most humanoid denizens of Irokas speak \Lowtongue, although most will understand Kingstongue. There are many variants of \Lowtongue, some so different from each other (and from Kingstongue) that they are barely recognizable. 
\also{Chaos magic}



\gitem{\Dun}
Dun is the larger of Mith's two moons, called the Gray Moon. In the sky it is slightly larger than Earth's moon and of a gray colour. In religion and astrology, \Dun{} is usually considered benevolent or neutral. 



\gitem{Geican}
A Vaimon clan. Their homeland is Geica, which is ruled by a democracy. 

The Geicans are known to use \nieur{} and \iquin{} alike. For this reason they are seen as evil diabolists by some, especially their ancient rivals the Redcor. 

As a culture, the Geicans are atheistic and anarchistic and hail the freedom of the individual as their highest ideal. 

The founder of Clan Geican was Tiraad Geican, son of Cordos Vaimon. 

The symbol of Geica and Clan Geican is a green eagle on a black background. The eagle symbolizes freedom, and the black background signifies that without freedom (i.e., outside the eagle) there is only darkness and evil. The traditional Geican colour is green. 
\also{Geica}



\gitem{glyph}
Glyphs are writing symbols used in Rissitic. Occult glyphs are used in much Rissitic magic. Glyphs often consist of pictures of animals, plants and objects as well as geometric shapes.



\gitem{glyph head}
A glyph head is the severed head of a creature, mummified and enscribed with occult glyphs. Glyph heads are used for storing spells or magical energy for later use. The creation of a glyph head requires the sacrifice of a living creature, and the soul of the head's former owner is bound in the process, used to seal the magical energy in the head. 

When the spells contained are released the soul is permanently destroyed and the head becomes useless (it cannot be enchanted to work as a glyph head again). 



\gitempl{gness}{gnesses}
Rissitic slang for the \scathaese{} hemipenis. A male \scatha{} does not have a \human{}-like penis but two small hemipenes. 



\gitempl{\Goyden}{\Goydens}
A savage people living in the Wild in southern Pelidor and northern Beirod. They are \humans, but some say they are half \human{} and half beast. They speak their own language and pray to their own gods. According to some rumours they can change between \human{} and animal form. 



\gitem{Gray Moon}
\seee{\Dun{}}



\begin{comment}
\subsubsection{H-N}
\end{comment}

\gitem{Hell}
In Iquinian mythology, a world of \daemons{} and supposedly the place where the souls of the wicked go after death. 



\gitem{High King}
The title taken by the kings of the Belkadian Empire. The first High King was Uther the Tiger. The 18th and last High King was Leopold II. The High Kings belonged to House Belek. %After the fall of the Empire, the rulers of Belek no longer use the title \quo{High King} (but simply \quo{king}). 
After the fall of the Empire, the rulers of Belek call themselves \quo{kings} rather than \quo{High Kings}. 

A High King's wife was called High Queen. The Empire never had a ruling High Queen.
\also{Belkadian Empire, Belek}



\gitemthe{\Imetriad}
The holy scripture of the Imetrium, containing the core of their beliefs and morals. 



\gitemi{Imetric (language)}{Imetrium!language}
The language spoken in the Imetrium. Imetric is descended from Ortaican and written using the Ortaican alphabet. 

Imetric is intended to look similar to Greek and Latin, but pronounced similar to English. 
\also{The Imetrium}



\gitemi{Imetric clerical ranks}{Imetrium!clerical ranks}
The titles used by the Imetric clergy are (in descending order): 
\Laccorin{} (similar to a cardinal),
\Ispan,
\Telphan,
\Amra{} (a regular priest).

A generic word for \quo{soldier} is \Stracos, plural \Stracoi. 
\also{The Imetrium}



\gitemi{Imetric military ranks}{Imetrium!military ranks}
%Pandeccor (super-general)
%Deccor (general or colonel)
%Retaxis (marshall)
%Salican (captain) 
%Vexstra (lieutenant)
%Corphin (sergeant)
%Incran (private)
The military ranks used by the Imetric army are (in descending order): 
\Deccor{} (equivalent to a general), 
\Retaxis{}, 
\Salican{}, 
\Vexstra{}, 
\Corphin{}, 
\Inclan{} (equivalent to a private). 

A generic word for \quo{soldier} is \Rengos, plural \Rengoi. 
\also{The Imetrium}



\gitemi{The Imetrium (religion)}{Imetrium!religion}
A religion that rules the theocratic nation also called the Imetrium, in western \Azmith, and has some adherents in eastern Belkade, especially Andras and \Scyrum{}. 

The Imetric religion is polytheistic. The five major gods make up \quo{the Tribunal}, which governs the Imetric pantheon and nation. The most important god is Salacar; the other members of the Tribunal are Dessali, Eoncos, \NishiS{} and \Hiothrex{}. 

The symbol of the Imetrium is a white, four-spoked star, edged with blue within a circle of black against a white background. The colours of the Imetrium are white, blue and black. 
\also{Imetrium (nation)}




\gitem{Irokas Bloodline}
The royal house of the kingdom of Irokas. The members of the Bloodline are all \dragons{} and include Tentocoth (the current \DragonKing{}), Noreocchyrias (his father, now dead), \Criocas{} and Thiencaste (Tentocoth's two daughters) among others. 



\gitem{\iquin{}}
The force of Light in Vaimon metaphyics. By the Iquinian Church viewed as the source of all good and worshipped as a divine force. Its manifestations are the \Sephiroth. 



\gitem{Iquinian Church}
Also called \emph{Iquinianism} or \emph{the Church of the Light}, a religion based on the worship of \iquin{}, the Light, seen as the source of all good, and the \Sephiroth{} as its manifestations. There are two major branches of the Iquinian church: The Redcor branch and the \Yrgell{} branch. 



\gitem{knight}
In Belkade, especially honoured warriors are given the status of knights. They are blessed by the Belkadian Church and carry the title \quo{sir} before their names. 

\label{hedge knight}
Most knights are noble-born, but commoners are sometimes knighted. The latter are called \quo{hedge knights}. 



\gitem{\Lowtongue}
Collective term for various simplified forms of the \Draconic{} language. \seee{\Draconic{} (language)} 



%\gitem{Mictzan breed}
%The Mictzan is a sub-breed of the Destran race of \nycans{}. It is named for a Clictuan \scatha{} who founded the breed. 



\gitem{months (\ImperialCalendar)}
\index{\Atzirah!month}
\index{\Feazirah!month}
\index{\Keshirah!month}
\index{\Razilah!month}
\index{\Barion!month}
\index{\Hapheron!month}
\index{\Izion!month}
\index{\Teshiron!month}
\index{\Cushed!month}
\index{\Hoshied!month}
\index{\Thimared!month}
\index{\Yemared!month}
\index{\Gamishiel!month}
\index{\Ishiel!month}
\index{\Omariel!month}
\index{\Yeziel!month}

The \ImperialCalendar{} has sixteen months, dedicated to the sixteen \Sephiroth{}.
The four months of spring are named for the \Sephiroth{} of Air are are \Atzirah{}, \Razilah, \Keshirah{} and \Feazirah{}. The summer months, named for the \Sephiroth{} of Fire, are \Barion{}, \Teshiron, \Izion{} and \Hapheron. The autumn months are associated with the element of Earth and are \Thimared, \Yemared, \Cushed{} and \Hoshied, and the winter months, dedicated to the \Sephiroth{} of Water, are \Omariel, \Yeziel, \Ishiel{} and \Gamishiel. 

Each month is 24 days long, split into three weeks of eight days each (beginning with \Corjin{} and ending with \Siljin). The exception is \Gamishiel{}, the last month, which is only 20 days long. %(\Gamishiel{} is the \Sephirah{} of Sacrifice.) 
\also{\ImperialCalendar{}, \Sephiroth}



\gitem{\MotherTiamat}
\Narkiza's flagship, named after the \Dragon{} Mother of myth. She is 40 metres long, carved in the likeness of a many-headed \dragon{} and armed with ten ballistae. 



\gitem{\nieur{}}
The force of Darkness in Vaimon metaphyics. By the Iquinian Church reviled as the source of all evil. Its manifestations are the \Qliphoth. 



\gitem{Nicca}
The hips on a \scatha{}. The Niccas are considered a sexual body part, important in determining attractiveness, and many cultures dictate that they must be covered in public. 



\gitem{North Star}
A star lying close to Mith's rotational axis in the northern direction. It is a white star of spectral class A and one of the brighter stars of the northern sky. Often considered by astrologers to have important occult qualities. The Northstar clan, in addition to their Imetric religion, venerate the North Star. Called \word{Telcastora} in the Ortaican language. (Note that since Mith is not Earth, the North Star is \emph{not} the same as Polaris, which is Earth's north star at the moment.)
\also{Northstar clan}



\gitemi{Northstar clan}{Northstar}
The Northstars are a \scathaese{} clan. The Northstar name is famous and prestigious (known throughout the Imetrium and parts of southern Belkade and northern Durcac), and the clan has produced several renowned heroes. The Northstars trace their history at least 1000 years back. Today they are Imetrians, but they retain certain pagan traditions, of which the most well-known is the veneration of the North Star. 

The Northstars have their own city, Telcarmium, in the western Imetrium somewhat east of Martinum, where many of them live. The vast majority of Northstars live in the Imetrium. They are a tightly knit family and maintain contact between all their members. There are a few people outside the Imetrium who bear the name Northstar but maintain no contact with the family. These are considered traitors or impostors by the clan. The official Northstar clan has about a thousand members. 

The name \quo{Northstar} is actually \word{Telcastora}, which is the name of the North Star in the Ortaican tongue. Throughout my writings, however, Telcastora has been translated into English \quo{Northstar}. 

Well-known Northstars include Ilcas and Cassili the Condor. 
\also North Star. 



\gitem{\nycaneer{}}
A \scatha{} who commands \nycans{}. \Nycaneers{} have a special, inborn talent for telepathy which lets them communicate with the \nycans{} and gives them a special affinity for the beasts. Only \scathae{} are born with this talent. See also section \ref{Nycaneer}. 
\also{\nycan{}}



\begin{comment}
\subsubsection{O-U}
\end{comment}

\gitempl{\Ortican}{\Orticans}
A \scathaese{} ethnic group and language somewhat widespread in Beirod, Gaznor and southern Pelidor and \Scyrum. The \Ortic{} language is descended from Ortaican.



\gitem{Pale Moon}
\seee{Visha}



%\gitem{Raeco Mannica}



\gitem{Real Life}
\quo{Real Life}, abbreviated RL, refers to the real world, as opposed to Mith. I use the term when comparing Mithian phenomena to those of the real world. 



\gitem{Redcor}
A Vaimon clan. They control the Redcor church, one of the two major divisions of the Iquinian religion (the other being the \Yrgell{}). Their homeland is \Redce, where their Conclave rules from the \TopazChateau. The Redcor are traditionally matriarchal and matrilinear. 

The Redcor are divided into Clerics and Templars. The Templars are warrior mages and the defenders and holy knights of the Redcor faith. They are split into two orders, the \ryzin{} (female) and \gandierre{} (male). 

The Clerics are scholars and priests and the spiritual and political leaders of the Redcor. The lowest ranked clerics are monks, bearing the title \emph{\frater} (plural \emph{\fratres}) if male or \emph{\soror} (plural \emph{\sorores}) if female. Male Clerics cannot rise above the rank of Frater. For female Clerics only, the higher ranks are \mater, \matron and \matriarch. The Redcor are ruled by the Conclave, which consists of all \matriarchs{} (who are all equal in status). 

Apprentices training to become Vaimons are called \neophytes.\index{\neophyte}

The founder of Clan Redcor was Rebecca Redcor, daughter of Cordos Vaimon and Silqua Delain. Their symbol is a yellow Sun on a blue background, and their traditional colour is yellow. 

Clan Redcor is divided into a number of fractions. The three largest factions are the Foxes, Swans and Tulips, whose symbols are, respectively, a red fox, a white swan and a yellow tulip. 



\gitemi{Rissitic (language)}{Rissitic!language}
The language of the Rissitic people of Durcac. 

Rissitic is harsh and guttural, meant to resemble Danish, Dutch and German. Some words are taken from Egyptian or Mesopotamian languages, however, since Rissitic mythology is inspired by those of Egypt and Mesopotamia. 



\gitemi{Rissitic (religion)}{Rissitic!religion}
The Rissitic religion, also called Rissitism, is based around the worship of the god \HriistN, also called Rissit. \quo{Rissitics} is also the term used by nonbelievers for adherents of the religion. Their homeland is Durcac. 

%The Rissitic religion uses a caste system. The castes are, in descending order of status: Tsalt (priests), Rekkan (knights), \Bedhin{} (craftsmen), \Kyth{} (soldiers), \Hok{} (commoners), Gzend (slaves). 

%Term for things relating to Rissit and his religion. The religion itself is called the Rissitic religion or Rissitism. 
\also{\HriistN, Durcac}



\gitemi{Rissitic units of measurement}{Rissitic!units of measurement}
\index{thumb (unit of measurement)}
\index{finger (unit of measurement)}
\index{stride (unit of measurement)}
A thumb is 3 cm. Three thumbs make a finger (9 cm), twelve fingers make a stride (108 cm).



\gitem{RL}
\seee{Real Life}



\gitemi{\Samurin{} (language)}{\Samurin{}!language}
\seee{\Samur}



\gitem{Sentinels of Mith}
A shadowy organization controlled by \dragons{}. Most Sentinels are \dragons, \rachyth{} or \scathae{}. 



\gitem{shade pearl}
A pearl harvested from a rare oyster in the Far Orient, dark gray in colour and around 2 cm in diameter. They are useful as an ingredient in a spell that creates a \quo{shroud} which conceals magic cast inside it from outside detection. 



\gitem{sorcerer/sorceress}
A practitioner of \quo{sorcery}. 
\also{sorcery}



\gitem{sorcery}
A word for magic, especially magic that is considered in some sense \quo{dark} or \quo{forbidden}. Used by the Iquinian church as a disparaging term for non-Iquinian magic. 



\gitem{tacupien}
A Pelidorian dance.



\gitem{\Telderain{}}
\Telderain{} is Ilcas' sword, an heirloom of the Northstar family. It is 400 years old and enchanted. It is a large bastard sword which can be wielded in one or two hands. It is of an archaic design, larger and heavier than most modern swords, and requires great strength to wield. It is made of a \dragonsteel-iron alloy. 



\gitem{Templar}
A Vaimon warrior-mage and knight. %The Redcor and \Yrgell{} still maintain orders of Templar Knights, but the Geicans do not. 
Today, only Clan Redcor maintains an order of Templars. 



%\gitem{Tepharin}



\gitempl{\Tepharite{}}{\Tepharites{}}
The \Tepharites{} are a people living in southern Belkade and including both \scathae{} and \humans{}. Prior to the advent of the Belkadian Empire they had their \Tepharin{} Bacconate. The \Tepharin{} language (related to Ortaican) is still spoken many places in Ontephar, Pelidor and \Scyrum. 
\also{\Tepharin{} Bacconate}



\gitem{Tiger}
The Tigers were the elite, professional soldiers of the Belkadian Empire, taking their name from Uther the Tiger. After the downfall of the empire the Tigers are no longer centrally organized. Some owe their allegiance to various lords, while others are mercenaries. 

The Tigers contain both cavalry and infantry, including archers. Some of them are knights, annointed by the Iquinian church. The Tigers have no mages. 



\begin{comment}
\subsubsection{V-Z}
\end{comment}

\gitem{Vaimon}
An order of \human{} mages, founded somewhere around the year 1 \IC{} by Silqua Delain and Cordos Vaimon (the first Vaimon Emperor), after whom it was named. Vaimon magic is based on the twin forces of \iquin{} and \nieur, and spells are cast by invoking the various \Archons{}. 

%In the \ImperialCalendar{}, a week is eight days long. Each day is named after of the Vaimon founders. The days, in order, are \Corjin{} (after Cordos Vaimon, the first Emperor), \Setherab{} (after Sether Vaimon), \Rebecab{} (after Rebecca Redcor), \Arcab{} (after Arcan Delain), \Norquin{} (after Norcah Quaerin), \Tirjin{} (after Tiraad Geican), \Kerzab{} (after Kerzah \Irgel) and \Siljin{} (after Silqua Delain). 

The Vaimons are divided into a number of autonomous \quo{clans}. Originally there were six clans: Sether (founded by Sether Vaimon, son of Cordos and Silqua and the second Vaimon Emperor), Redcor (founded by Rebecca Redcor, daughter of Cordos and Silqua), Delain (descended from Silqua's brothers, Arcan and Lestor), \Irgel{} (descended from Kerzah \Irgel, Cordos and Silqua's younger son), Geican (founded by Tiraad Geican, son of Cordos and one of his other wives) and Quaerin (founded by Norcah Quaerin, son of Cordos by a third wife). 

To this day, only the clans Redcor, \Yrgell{} and Geican survive, the clans Delain, Quaerin and Sether having died out. There exist also \quo{rogue} Vaimons, owing allegiance to no clan.

The Vaimons previously ruled a Vaimon Empire, but it fell in \yic{Darkfall} in what is called the \quo{\Darkfall}. 

%Originally there were six clans, but only the clans Redcor, \Yrgell{} and Geican survive to this day, the clans Delain, Quaerin and Sether having died out. 
%
%Originally all Vaimons claimed descent from Cordos and Silqua, but in later days the clans began to accept apprentices not of Vaimon lineage. 
\also{Cordos Vaimon, Silqua, Redcor, \Yrgell, Geican, Quaerin}



\gitemi{\VaimonCalendar{}}{Vaimon!Vaimon Calendar}
\label{Vaimon Calendar}
The calendar of the old Vaimon Empire, still used by Vaimons and in most of Belkade. \quo{$n$ \IC} denotes year number $n$ in the \ImperialCalendar, counting from the year when Cordos Vaimon was crowned Emperor (year 1 \IC{}). 

A year is 380 days long. 
%where year 1 \IC{} was the year when Cordos Vaimon was crowned Emperor. 

The \ImperialCalendar has sixteen months, dedicated to the sixteen \Sephiroth{}, and each week has eight days, named after the Vaimon founders. 

The end of the year is celebrated with the festival of Camaire on the last day of \Gamishiel{}, midways between the winter solstice and the spring equinox. 
\also{days (\ImperialCalendar), months (\ImperialCalendar)}



\gitemi{Vaimon (language)}{Vaimon!language}
\index{Archaic Vaimon (language)}
\index{Ancient Vaimon (language)}
\index{Modern Vaimon (language)}
\index{Redcor!language}
Once the official language of the Vaimon Empire, still spoken in \Redce{} and used by the Redcor. Previously used as an intercultural \emph{lingua franca}, it has been replaced by Belkadian in recent centuries. 

The term \quo{Archaic Vaimon} or \quo{Ancient Vaimon} is used to describe older forms of the language, as spoken in the Vaimon Empire. This is contrasted to \quo{Modern Vaimon}, as spoken in \Redce. Modern Vaimon is sometimes called \quo{Redcor Vaimon} or simply the Redcor language, but the Redcor frown upon this terminology, insisting that their tongue is the true Vaimon tongue. 

The Redcor dialect is meant to resemble French, but Archaic Vaimon is meant to sound like Hebrew. 



\gitem{\IC{}}
\seee{\ImperialCalendar}



\gitem{Visha}
Visha is the smaller of Mith's two moons, called the Pale Moon. It is bright bluish white in colour. Its size in the sky is somewhat smaller than Earth's Moon. In religion and astrology, Visha is typically portrayed as malevolent and a bringer of ill-omen. 



\gitemthe{Wild}
The uncharted wilderness between cities and villages, inhabited by dangerous beasts and monsters. %Devoid of civilized humanoids (by definition), but some barbarians and savages dwell in the Wild. 
Humanoids dwelling in the wild are labelled as savages and barbarians. 

\index{\wildfog}
Areas of Wild are sometimes shrouded in \wildfog, a fog-like substance that obscures vision. (\Wildfog{} is not regular fog. It is not made of water and may be found even in deserts.) 



\gitem{\Yrgell}
A Vaimon clan. They control the \Yrgell{} church, one of the two major divisions of the Iquinian religion (the other being the Redcor). The \Yrgell{} have no homeland nor central organization and are mostly peaceful (unlike the well-organized and ambitious Redcor). 



\end{gloss}











%\end{comment}
