\chapter{\Tuat{}}

\section{The Dimension of \Tuat{}}

'\Tuat{}' is a parallel dimension. Bla bla.

The planet \Tuat{} is tide-locked: It cannot rotate in its orbit, but has the same side turned towards the Black Sun at all times. This divides the planet into a bright, hot side and a dark, cold side. These are named Inferno and Stygia. 



\sectionn{Inferno}
Inferno is the hot side of \Tuat{}. Pits and chasms of fire scar the land, sulphur rains from the sky, and the terrible Black Sun gazes ever down upon the land. The atmosphere of \Tuat{} lacks the stuff which make the sky blue, so here the heavens are black. 

\subsectionthe{Pyre}
The Pyre is a great column of infernal fire. It marks the exact centre of Inferno, the one spot on \Tuat{} where the Black Sun stands at the zenith. 

The Pyre is kilometers wide and over 20 km tall.\footnote{This is more than twice the height of the tallest mountains on Earth in RL.} It is a matrix of tremendous Chaotic power. It draws souls in and gives birth to demons. Balrogs are created at the Pyre. 

Some claim that the Pyre is sentient. Some claim it is the body of Astaroth, the Supreme Devil.

\subsectionthe{Rivers of Inferno}
Inferno is crossed by a number of great rivers. 

\subsubsectionn{Phlegethon}
The river of fire. Flows out Eastwards from the Pyre, splitting the continent of Eastern Inferno into two. 

\subsubsectionn{Styx}
\subsubsectionn{Acheron}
\subsubsectionn{Lethe}

\subsectionn{Astaroth}
Legends and superstition speak of Astaroth an ancient creature of evil, immense in power, who is called the the Supreme Devil, the True Overlord of Inferno and all that is evil. 

It is unknown if Astaroth exists, but he has many cults scattered over \Tuat{} and other places, including Mith. More than once, a proclaimed Astaroth has proven fake, a lesser demon lord using the name for publicity. 

Some legends claim that Astaroth is sleeping (perhaps at the heart of the Pyre). Others say that he was cast down and imprisoned deepest down in Tartaros. Yet others believe that Astaroth left \Tuat{} to fight a great foe. All agree that he will some day return to claim what is rightfully his. 

Myths do not agree whether Astaroth ruled all of \Tuat{} or only Inferno. 

It is said that Astaroth created the Pyre and all demons. Some even say that he created all evil\footnote{This cannot possibly be true.}. 



\subsectionthe{Abyss}
A great chasm that runs between the North and South Poles of \Tuat{}\footnote{\Tuat{} does not rotate, so what are the poles? Are they the magnetic poles, or perpendicular to the orbit, or coincident with the rotational poles of Mith}, though the Pyre. Connects Inferno with Tartaros. There are kingdoms of devils in the Abyss. 



\subsectionn{Avernus}
The kingdom of Rauthor, bordered by the Abyss and the river Acheron. Engaged in a constand war with the kingdom of Candrazor. 

\subsection{The Kingdom of Candrazor}
The kingdom of Candrazor, bordered by the rivers Acheron and Styx. 

\subsection{The Kingdom of Tez'rik'rik}
The kingdom of Tez'rik'rik, bordered by the river Styx and the Abyss. 



\sectionn{Stygia}
The cold side of \Tuat{}. 

\subsectionn{Caina}
The coldest place in all of Stygia, centered around the Spire, and icy mountain directly opposite the Pyre.\footnote{Maybe 'Spire' sounds too much like 'Pyre'. Choose a different name?} 

Caina is ruled by the dark and terrible Triumvirate of Fell Angels.

\subsectionn{Cocytus}
Is this a part of Stygia or Tartaros? Maybe let Cocytus be a part of Stygia and let Nessus be part of Tartaros. 

\subsectionn{Cold Wraiths}
Great demons who live in Stygia. Comparable to Balrogs in power. 



\sectionthe{Twilight Zone}
The borderlands between Stygia and Inferno. Great clouds of ash block out the fell light of the Black Sun but provide sufficient greenhouse effect to drive away the ice of Stygia. 



\sectionn{Tartaros}
The underworld of \Tuat{}, a dark world of caverns and labyrinths. Some gigantic and horrible creatures dwell here, including the Dholes (enormous worms, as in HP Lovecraft - The Dream-Quest of Unknown Kadath). 



\sectionn{Pandaemonium}
The next lower dimension, one step more Chaotic than \Tuat{}. A twisted, nightmarish place of howling winds and few living creatures. On a rare occasion, \Tuat{} has been invaded by Pandaemoniac creatures. 

\subsection{The Lost Realm of Pluton}
Once a great kingdom in Inferno, Pluton was invaded by monsters from Pandaemonium a few centuries ago. Stories tell of tremendous, blasting winds laying waste to cities, and of titanic creatures that sucked hundreds of demons into their great mouths and devoured them whole. Even mighty devils such as Balrogs were no match for the terrible creatures. 

Today, Pluton is a deserted wasteland. It is unknown whether any of the Pandaemoniacs remain, but it is known that echoes of their Chaotic magic still lingers. Mysterious things happen, and demons (even armies of demons) have been known to vanish without a trace... perhaps swept away to nightmarish Pandaemonium. 

\section{Denizens of \Tuat{}}

\subsection{Balrog}
Singular Balrog, plural Balrogs. 

Some of the mightiest creatures in \Tuat{}. Great demons of fire and darkness. 

\subsubsection{Life cycle}
Balrogs are born from the Chaotic energies of the Pyre. They are asexual and cannot reproduce on their own. New Balrogs are born only in the Pyre. At higher TLs, it may become possible to channel the Chaotic energies and control the spawning of demons. 

Balrogs are immortal. They do not die of old age and can reincarnate themselves if their bodies are destroyed. Powerful magic is required to destroy them permanently. 

Balrogs do not automatically change as they age. They grow in power by experience or by devouring other demons.

\subsubsection{Appearance}
A great humanoid made of fire solid and darkness, with great wings\footnote{Regarless of whether real Balrogs have wings or not, my Balrogs do. And they can fly.} and horns. Similar to the Balrog in the Lord of the Rings movies by Peter Jackson. 

An average Balrog is 6 meters tall and massively built. They can change their size at will, shrinking to half their full size. Balrogs usually maintain their full size most of the time, for the frightening effect. They can also alter their body form at will, but only slightly (always fitting the description above). 

\subsubsection{Power}
Even the weakest newborn Balrog is a lesser god in power. The mightiest Balrog alive is Rauthor, who is one of the greatest gods on Mith. 



\subsection{Lemures}
Horrid creatures. Damned souls trapped in \Tuat{} most often take the form of Lemures. Semi-amorphous mounds of flesh, horrid caricatures of the forms they had in their previous lives. Each time a Lemure dies, its body mutates further into chaos. 

Lemures are mindless and know only two feelings: Suffering and \emph{extreme} suffering. The latter is their default state, and if they discover that a certain action lessens their agony, they will do that. Demonic spells exploit this to control the Lemures and form them into armies. 



\subsection{Imps}
Lesser demons. Fully intelligent and quite evil. Mortal, few supernatural powers. Exist in many types. 



\subsection{Phobian}
Singular Phobian, plural Phobians. Crab-like, intelligent creatures. Demons, but benevolent. Believe in the freedom of the individual. 

Not so powerful and often enslaved by greater demons. Their chaotic, anarchistic nature makes it hard for them to band together in kingdoms for protection, but there exist sanctuaries of peaceful demons, mostly under the protection of benevolent gods. 



\section{Gods and Characters of \Tuat{}}

\subsectionn{Rauthor, King of Balrogs}
\label{Rauthor}
Rauthor is the mightiest Balrog alive and monarch of the nation of Avernus. 
