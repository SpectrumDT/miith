\chapter{The History of Mith}



\section{Timeline}



\bepo{Pre-History}
\eepo



\bepo{The Age of \Dragons{}}

\event{First Dragons}{First \Dragons{} born.}
\event{Founding of Nom}{
  Florocthakhsur, Lord of House Tindarex, declares herself Queen of the \Dragon{} Kingdom of Nom, taking the royal name Nacheldryss, meaning `vengeful'.} 
\event{Fall of Nom}{The Fall of Nom.} 
\eventl{Founding of Irokas}{
  Hekhdraxsis, Lord of House Irokas, declares herself Queen of the \Dragon{} Kingdom of Irokas, taking the royal name Tentocoth, `unequaled'.} 
\eepo



\bepo{The Days of the Empire}

\event{Faegos birth}{Faegos Vaimon is born.} 
\event{Belandos birth}{Belandos Vaimon II is born, the first son of Faegos Vaimon.} 
\event{Cordos Vaimon birth}{Cordos Vaimon born, the first child of Belandos Vaimon II.} 
\event{Faegos death}{King Faegos dies. Belandos Vaimon II crowned king.} 
\event{Silqua birth}{Silqua Delain born.} 
\event{Silqua wed}{Silqua weds Cordos Vaimon.} 
\event{Belandos death}{King Belandos II dies. Cordos Vaimon crowned king. He begins to conquer a bunch of stuff.} 
\event{Silqua death}{Silqua Delain Vaimon dies.} 
\event{IC}{Cordos Vaimon declares himself Emperor of the Vaimon Empire.} 
\event{Belzir birth}{\Belzir{} daughter of Cormin born.} 
\eventl{Darkfall}{The \Darkfall{} and the end of the Empire.} 
\eepo



\bepo{The Modern Age}

\event{Noreocchyrias birth}
{Birth of Xoldm of House of Irokas (later to be King Noreocchyrias).} 
\event{Founding of Imetrium}
{Imetrium is formed.} 
\event{Noreocchyrias crowned}
{Xoldm is crowned king and takes the name Noreocchyrias, `feared by his enemies'.} 
\event{Tentocoth birth}
{Birth of Frycad of House Irokas (later to be King Tentocoth).} 
\event{Founding of Belkade}
{Belkadian Empire formed.} 
\event{Tentocoth Vaccashyth}
{Frycad comes of age and takes the name Vaccashyth.} 
\event{Tentocoth Rhoahathnex}
{Vaccashyth takes the name Rhoahathnex.} 
\event{Tentocoth Ibrethnavish}
{Vaccashyth takes the name Ibrethnavish.} 
\event{\EreshKal{} flourished}{The civilization of \EreshKal{} flourish in modern-day Runger.}
\event{\EreshKal{} destroyed}{\EreshKal{} is destroyed after several neighbouring lords band together to wage war against them.}
\event{Fall of Belkade}
{Belkadian Empire collapses.} 
%\event{Noreocchyrias abdicates}{King Noreocchyrias abdicates. His son Vaccashyth is crowned king and takes the name Tentocoth.} 
\event{Noreocchyrias death}{King Noreocchyrias dies. His son Vaccashyth is crowned king and takes the name Tentocoth.} 
\event{Carzain birth}{Carzain is born on the 12th day of \Izion.} 
\eventl{Runger war}{King Morgan of Runger invades Pelidor in the beginning of \Atzirah.}
\eepo


%\section{Timeline}



\newenvironment{epoch}[1]{\subsubsection{#1}\begin{tabular}{rp{10cm}}}{\end{tabular}}
\newcommand{\bepo}{\begin{epoch}}
\newcommand{\eepo}{\end{epoch}}
% \yearic is `year IC', as in, a year number in the Imperial Calendar 
% (reckoned from the founding of the Vaimon Empire)
\newcommand{\yic}[1]{\arabic{#1} IC}




%%% Here follows the definitions of a bunch of historical constants. 
% A `hiscb' is a `basic historic constant'. Its value is given as a constant. 
% A `hisc' is a `historic constant'. It's value is given as the value of a base hisc plus an offset. 
% NB: The base hisc is given as a counter name, without the `\value' tag. (It needs not be a hiscb, it can just be a hisc.) 
\newcommand{\hiscb}[2]{\newcounter{#1}\setcounter{#1}{#2}}
\newcommand{\hisc}[3]{\newcounter{#1}\setcounter{#1}{\value{#2}}\addtocounter{#1}{#3}}
% `birthandage' specifies a person's name (`foo') birth year and his age at death. 
% It also introduces two hisc's: `foo birth' and `foo death'. 
% Birth year is given as a base hisc and an offset. 
% Death year is computed as birth year + age. 
\newcommand{\birthandage}[4]{\hisc{#1 birth}{#2}{#3}\hisc{#1 death}{#2}{#3}}

% The 0th year of the Imperial Calendar, the founding of the Vaimon Empire
\hiscb{IC}{1000}
\newcommand{\ic}{\value{IC}}

% Cordos Vaimon
\birthandage{Cordos}{IC}{-50}{80}
% Belandos Vaimon II (Cordos' father)
\birthandage{Belandos}{Cordos birth}{-25}{65}
% Faegos Vaimon (Cordos' grandfather)
\birthandage{Faegos}{Belandos birth}{-20}{58}
% Silqua Delain
\birthandage{Silqua}{IC}{-35}{30}
%\hisc{Silqua birth}{\value{IC}}{-35}
%\hisc{Silqua death}{\value{Silqua birth}}{30}
% Belzir

% The Darkfall, the fall of the Vaimon Empire. 
\hisc{Darkfall}{IC}{1965} 
% The wedding of Silqua Delain and Cordos Vaimon. 
\hisc{Silqua wedding}{Silqua birth}{18}



\begin{comment}
\end{comment}



%\section{Pre-History}



%Years are given as \squo{BD}\index{BD (year numbers)} or \squo{AD}\index{AD (year numbers)}, signifying \squo{before the Darkfall} and \squo{after the Darkfall}, respectively. (See section \ref{Darkfall} about the Darkfall.) 

\paragraph{} Years are reckoned by the Imperial Calendar (IC), with year 0 IC being the Founding of the Vaimon Empire (see section \ref{Vaimon Empire}). 



\section{Primordial Time}
\subsection{The \voyagers{} and the \krakens{}}
\index{\krakens}
\label{Kraken}
\index{\voyagers}
\label{Voyagers}
Once there were the \voyagers, a race of mighty beings (gods, if one wills) who travelled across the Universe, seeding worlds they found with life. In primordial time, billions of years ago, they came to Mith, desiring to make the world theirs, a breeding ground for their creations. 

But Mith was not barren. Before the \voyagers{} came, Mith had her own children, her own indigenous life, and the greatest of them were the \krakens, the native overlords of Mith. Terrible creatures they were, and immortal. And they were jealous and cared not for invaders, but craved Mith as their birthright, and when the \voyagers{} sought to conquer their world, they fought. Few were the \krakens{}, numbering scarcely a dozen against the thousands of \voyagers{}, great and mighty in their own right. Yet in their wrath, the \krakens{} proved more than a match for the invaders, and in every conflict they would prevail. With valour and fury the \voyagers{} would fight, and with despair, for the \krakens{} were immortal and wielded the primal power of Mith herself. Their leader and mother, the great \Kraken{} Queen, was the mightiest of all the children of Mith, and none could stand before her when she rose in fury. 

But the \krakens{} were indolent, and they would often sleep, their entire race, for aeons at a time - millions, even hundreds of millions of years. And so, whenever the \krakens{} slept, the \voyagers{} would invade Mith, to populate the planet with myriad life forms of their devise. And every time, after an aeon, the \krakens{} would awake to vanquish the \voyagers{}, scour the world of their creatures and repopulate it with their own spawn. But inevitably, the \krakens{} would once more go dormant, and the \voyagers{} would return to begin their work anew. 

For the \voyagers{} seek ever to perfect life, and Mith is precious to them, for Mith is an anvil on which to forge and shape their creations, their works of art. And a most excellent anvil is Mother Mith, for it is her nature to take what is given to her and break it down and hammer it into a new shape, greater and mightier than before. The \voyagers{} have travelled the universe for billions of years, and they have known many planets. Many worlds are hostile, wastelands of death that will destroy life and cause it to decline and decay. Other worlds are indifferent, possessed of no powerful energy, positive or negative. They will allow life to exist, but they will not support it. On such worlds, life will persist, but it will stagnate and remain humble and never know greatness. Not so Mother Mith. She is possessed of a soul, and she is strong and fierce and indomitable. A loving mother is Mith, but also cruel. The weak among her children she will destroy, and they shall have no legacy and know only oblivion. But the strong among her children she will cherish and glorify. Forged in fire and tempered in blood they shall rise, growing ever stronger, and they shall know greatness. For Mith is a crucible of life, and her like is seldom found in the universe, and the \voyagers{} knew to cherish her. 

\subsection{\Moroch{} and the \nagae{}}
So the history of Mith is divided into such cycles of creation and destruction in the cosmic struggle between these factions, the native \krakens{} and the alien \voyagers{}. 

Our story begins in one such cycle. The \voyagers{} had filled Mith up with their creations, but they were fled, for the stars moved, and they knew that the \krakens{} would soon awaken. And one of the \krakens{} did awaken; in this day and age he is called \Moroch{}. What is his aim we cannot know, but when he awoke he began not to destroy, but to create. From the sea he took primitive beings, creatures of the \voyagers{}, and reshaped them in his own design. He gave them strength of body and mind, and bestowed upon them the power of thought, and he made them his servitors. 

The years passed, a million years and more, and still the \Kraken{} Queen and her brethren slumbered, and even \Moroch{} grew sleepy and fell dormant. But his spawn lived, and they grew and prospered, and they recalled their sire and offered him prayer and tribute. Hybrid children they were, born from the womb of the \voyagersz{} creation but fathered by a \kraken{} lord. Lowly they were, yet in this age they were the lords of Mith, and their built their cities in the deep oceans across the globe, and they were the \nagae{}. 

\subsection{Rise of the \Dragonlords}
%\section{Ancient Time}
\label{Origin of Dragons}
\index{\Dragon{}!Origin of the \dragons{}}
Over the course of hundreds of thousands of years, the \nagae{} developed their culture beneath the seas of Mith. Because they were unable to develop fire and forge metal under the sea, they remained technologically primitive for ages, but did develop some magic. The lords of the \nagae{} eventually learned to grow huge and immensely powerful using magic. They became the \leviathans{}. To serve them, the \leviathans{} bred other giant, mutated \nagae{}, who became the \linnorms. 

%But after uncounted millennia the \nagae{} ventured onto dry land, where they slowly evolved into Medusae. The \leviathans{} did not care to go on land and remained in the sea. Once on land, the Medusae rapidly advanced in technology, reaching some high TL... perhaps as high as TL9. The Medusa lords desired the physical power and splendour of the \leviathans{}, and so used biotechnology and magic to reshape their bodies into giant forms designed to signal power and inspire fear and awe. They became \dragons{}. 

After millennia uncounted, the \nagae{} eventually ventured onto dry land. The \leviathan{} lords cared not for the land and remained beneath the seas. In time the \naga{} lords of the land came to envy the \leviathans{} and desired to match them in splendour and might. Using bio-technology and sorcery they reshaped their bodies. Their bodies grew to great size and strength, and they took forms of beauty and majesty, to flaunt their power and inspire fear and awe. And their minds likewise grew strong and able, and they mastered science and magic. Terrible they were, titans of body and mind, and the overlords of the continents of Mith, and they were the \dragons{}. 

%They became \dragons{}. 

\label{Origin of Scathae}
\index{Scatha!Origin of the Scathae}
Eventually, the \dragons{} and \nagae{} came to desire to have a race of servitors. So, again using magic and biotechnology, they crossed \naga{} blood with that of land-living reptiles (including \nycans{}) and created a hybrid race, the \scathae{}. 

Beneath the oceans, the \krakens{} still slept, but \Moroch{} slept only lightly, and he would dream, and in his dreams he would touch the \nagae{}, and they would remember him and worship him. The \nagae{} were the children of the sea, and their blood was the water of the oceans, and their minds were the waves and the currents, and they remembered their origin and their creator. But the \dragons{} were creatures of the land and sky, and their blood was fire and ice, and their minds were wind and storm and thunder. The \dragons{} had come from the seas, but they had risen up to become something greater, and they were no longer of the sea, and they grew estranged from it, and they become removed from the \nagae{}, and they forgot \Moroch{}. They flew high above the world and gazed beyond the skies and toward the stars, and there they found new gods. 

%At some point it came to a great, cataclysmic war between the Medusa factions. It ended in a victory for the \dragon{} Lords and the near-total genocide of all Medusae and \nagae{} on the land. The \scathae{} were left alive because they were good and loyal servants. Thus begins the Age of Nom. 

\subsection{The Coming of the \Banes{}}
Deep in the vastness of space, many thousand light years from Mith, lies the sinister world of \Erebos{}. Thousands of years before the awakening of the \dragons{}, \Erebos{} had been visited by the \voyagers{}. 

(The \voyagers{} create \banes{}. \Banes{} rebel against \voyagers{} and destroy them. \Banes{} discover Mih and want to go there and conquer it, for some reason.) 

And so, the \baneking{} who later came to be called by the \dragons{} the \Voidbringer, succeeded in opening a great dimensional portal from \Erebos{} to Mith, and through it he sent forth millions of his \bane{} warriors to conquer and claim the new world in his name. 

The people of Mith were taken unaware when, on the Eve of the Black Gate, a massive rent opened in the sky and myriads of fiendish \bane{} warriors surged forth to conquer and destroy. Many millions perished in the initial assault and whole continents were laid waste before the \dragons{} were able to muster their forces in defense. The Eve of the Black Gate marked the beginning of the \Banewar, the most terrible and devastating war that Mith has seen since the \Kraken{} Queen last returned to her sleep. 

(The \dragons{} wage war. They need a champion to lead them. Tiamat steps forth. She persuades the \dragonlords{} to bestow upon her the ultimate power. With it, she leads the \dragons{} in combat and, after many great battles, vanquishes the \banes{} and slays the \banelord{} leading them. She and her associates manage to seal the Black Gate. Afterwards, Tiamat refuses to give up her power and, instead, kills and eats the \dragonlords{} and sets herself up as Queen.) 

\section{The Age of \Dragons{}}
\subsection{The Kingdom of Nom}
\label{Kingdom of Nom}
\index{Nom!Ancient Kingdom of Nom}
In time, the \nagae{} grew angry with their \draconic{} masters who scorned the worship of Father \Moroch{}. They rose up in rebellion, and it came to war. Fierce and terrible the battles raged, for decades and centuries, and the fields flowed red with blood, and the skies burned red with fire. Mother Mith was raising her children, and she raised them well. After a cataclysmic war, the \dragons{} stood victorious, and the \nagae{} were exterminated scoured from the surface of Mith, driven back to the sea and the embrace of Father \Moroch{}. 

First among the victors was Florocthakhsur, \dragonlord{} of House Tindarex, who went on to proclaim herself \DragonQueen{} of the new empire of Nom. She took the name Nacheldryss, \squo{vengeful} in the royal tongue of the \dragons{}, and thus founded the custom for a King or Queen to take a new throne-name upon coronation. Nacheldryss embraced the worship of the Astral powers, the Star-Gods, and they became the official religion of Nom. 

(What about the Chthonic powers? Who are they?) 

%Approximately 24,000 BD, Florocthakhsur, Lord of House Tindarex, declares her independence from the \leviathans{} and \nagae{} and crowns herself \DragonQueen{} of the new kingdom of Nom. She takes the name Nacheldryss and founds the custom for a King or Queen to take a royal name upon coronation. 

%Scorning the Chthonic powers worshipped by her ancestors, Nacheldryss embraces only the Astral powers. These become the official religion of Nom. 

%All Medusae and \nagae{} on land are exterminated. \nagae{} continue to exist underwater. 

%During this period, \humans{} rise to sentience but remain savages with almost no technology. 



\subsection{Creation of \humans{}}



\subsection{The \FirstBanewar}
\label{\FirstBanewar}
\index{\FirstBanewar, the}
A great civil war. Dark, forbidden magic was used. It resulted in a magical catastrophe. Terrible monsters were summoned from the \Baneworld{} of \Erebos{} and other places. The land of Nom was laid waste. 



\subsection{The Rise of the Kingdom of Irokas}
\label{Rise of Irokas}
\index{Irokas!Rise of Irokas}
In the ensuing chaos and strife, House Irokas seized power and become the new \DragonKing{} dynasty. The throne was moved to Mount Irokas and a new kingdom was named after them. But the kingdom of Irokas would not rise to the same power and glory as Nom, at least not for thousands of years. 

The new \Dragon{} Kingdom of Irokas is greatly weakened. The \dragons{} are decimated, with more than 75\% of their number slain in the war. The remaining \draconic{} \dragonhouses are splintered. Many are not loyal to the King. Irokas therefore has little power. The \dragons{} only rule the land that is now known as Irokas. On the rest of Mith, humanoid cultures rise to power. 

The Kings of Irokas issue a ban on item-based technology and some magic. 



\section{The Days of the Vaimon Empire}
\label{Days of the Vaimon Empire}
\index{Days of the Vaimon Empire, the}
Because of their infertility and long lifespans, it takes many millennia for the \dragons{} to replenish their numbers. Humanoids reproduce faster. Therefore, they increase in numbers and power. 

The \scathae{}, previously the dominant humanoids, have died by the billions in the \Banewar, and their kingdoms lie in ruins. While they rebuild, a new species rises out of savagery: \Humans{}. Previously, \humans{} had been considered a curiosity, a cunning ape that could be trained to perform tricks, but a primitive beast nonetheless. But with the \draconic{} and \scathaese{} kingdoms gone, the \humans{} are free to develop their own culture. 

There is plenty of conflict between \humans{} and \scathae{}, but also peaceful relations. Enough for the \humans{} to learn a lot of culture and science from the \scathae{}. The \scathae{} are primitive, bombed back to TL1 or so, but still better off than the \humans{}, with the bits of technology they've managed to salvage from before the \Banewar. 

In this age, the first Vaimons appear. They rise to build a great empire, and so, this age came to be known as the Days of the Empire. 

%Especially the Vaimons are powerful, because they discover \iquin{}-\nieur{} channeling magic. 

\subsection{Silqua the Prophet}
The history of the Vaimon Empire begins with \introp{Silqua Delain}{SIL-kwa de-LEJN}\index{Delain (Vaimon clan)!Silqua}, the daughter of \introp{Maegon Delain}{MEJ-gon de-LEJN}\index{Delain (Vaimon clan)!Maegon}, a noble in the \human{} kingdom of \intro{\Imrath{}}. She has two older brothers, Arcan and Lestor. 

At this point, \Imrath{} is at war with a neighbouring \scathaese{} kingdom. Relations between \humans{} and \scathae{} are very bad throughout the continent. Some of the \scathaese{} kingdoms are ruled by \dragons{} (overtly or covertly), and most \humans{} learn to hate the \dragons{} as evil monsters, and the \scathae{} (called \squo{Creeps}) as their evil monster-spawn. Among \humans{}, all magic is scorned as evil \squo{\intro{\dragoncraft}}. The overall technology is Bronze Age (about 500 BC in RL terms). 

Some nations (especially \scathaese{}) worship \draconic{} gods, or the \dragons{} themselves. Others worship other monstrous gods. In \Imrath{} (and several other \human{} kingdoms), all these \squo{demon gods} are scorned, and the official religion is a vague belief in \squo{the Light}, representing good and justice. This religion is based on fiction: The force of \squo{the Light} is not known to actually exist. 

As a young girl, Silqua is very intelligent and reads many books. She is religious and believes in the Light and the \Sephiroth{}\footnote{Are the \Sephiroth{} an accepted part of dogma at this time, or does Silqua invent them?}. 

At an age of 18 or so, Silqua is wed to Cordos Vaimon, the firstborn son of King Belandos Vaimon II and heir to the throne of \Imrath{}. (Cordos already has one wife, Delphias.) She eventually learns to love him. 

Somehow, Silqua discovers \iquin{} and \nieur{}. She believes that \iquin{} is the Light which her religion worships. She learns how to channel \iquin{} and thus cast magic. When this is discovered, Silqua is accused of practicing evil \dragoncraft. But she convinces Cordos that she is really good, that the magic she has discovered is truly the manifestation of the Light. She teaches a number of people to channel \iquin{}. These include Arcan, Lestor, Delphias and possibly Cordos Vaimon. She also teaches them to detect and recognize \nieur{}, but not to channel it. 

Silqua herself believes that she is the Prophet of the Light, born to bring knowledge of the Light to Mith. She also discovers \nieur{}, the force of Darkness. She believes that \nieur{} contains all that is evil and must be shunned. 

At some point, Belandos II dies and Cordos becomes king. He is quick to see the potential of Silqua's discovery and goes about organizing all believers of the Light into a single religious state. He manages to unite a large kingdom. Religious fervor is the key here, not the power of the magic itself. Silqua is a scientific genius and develops a lot of spells and techniques, but \iquin{}-\nieur{} magic is newly discovered, so all their spells are still primitive and somewhat weak. But the really important part is that Silqua's \iquin{} magic proves that the Light truly exists, and that she is its Prophet. 

Cordos desires to rule an empire, so he wages war on neighbouring kingdoms, \scathaese{} and \human{} alike. %Silqua initially approves of his war against the \scathae{}, since she has been raise to believe that the Creeps are evil monsters. 
Silqua initially approves, but along the way she begins to question whether war and conquest is the will of the Light. 

One time, Silqua is running around alone in the wild alone for some reason. She is captured by bandits, who keep her imprisoned and rape her repeatedly. She channels \nieur{} and kills a bandit, but is horrified by her actions. This leaves her traumatized, and she develops an irrational fear of both \nieur{} and sex. 

At some point, Silqua is captured by her enemies and killed. Cordos Vaimon continues his conquest and eventually comes to rule a great empire. This becomes the Vaimon Empire. 

\subsubsection{The Vaimon Founders}
The seven Vaimon clans are Redcor, Geican, \Yrgell, Quaerin, Delain, Ephrad and Sether. 

Clan Delain is descended from Silqua's brothers, Arcan and Lestor. 

Sether is descended from Sethor Vaimon, son of Cordos and Silqua. 

Redcor is descended from \Racel, daughter of Cordos and Silqua. 



\subsection{The Vaimon Empire}
\label{Vaimon Empire}
\index{Vaimon Empire}
\index{Vaimon!Origin of the Vaimons}
\label{Origin of Vaimons}
%Eventually, the descendants of Cordos and Silqua (and some others) change and mutate, becoming the Vaimon race. 
Cordos Vaimon founds the Vaimon order, and they become the ruling class of the empire that carries his name. 

The Vaimon Empire stands 2500 years and reaches TL5. 

%About 1500 years after the Fall of Nom, the Vaimon Empire is founded. It is ruled by a Vaimon Emperor from the Rainbow Throne. There are six Vaimon clans, each led from a magical throne of crystal: Diamond (Quaerin), Ruby, Sapphire, Emerald (Geican), Topaz (Redcor) and Onyx (Yrzhell). This Empire lasts for around 2000 years and reaches TL6. 

%The Emperor is selected astrologically: %When the Emperor dies, Imperial astrologers search for signs of the new Emperor's coming
%Imperial astrologers read the stars to ascertain the birth and location of an Imperial Scion. When a Scion is found, he or she (while a young infant) is taken to the Palace to be trained as the current Emperor's successor. 

%I don't know how Emperors are selected... 

\subsection{The \Darkfall}
\index{\Darkfall, the}
%The Darkfall, the fall of the Empire has to do with \Belzir{} (see section \ref{Belzir}), the Dark Queen, also called the Dark Prophet of \Nieur{}. She is a Vaimon woman of Clan Geican who becomes Empress (by perfectly normal means). She is evil and does not want to relinquish power. She uses dark magic to extend her life beyond its natural span. Perhaps she becomes a \nieur{} vampire? 

The Darkfall, the destruction of the Vaimon Empire, begins with \Belzir{} daughter of \Cormin, a young Vaimon girl of Clan Geican. As a girl, \Belzir{} is a nerd, an skinny, unattractive girl who fails to attract the attentions of boys and thus devotes her life to science and magic. She is a genius, with great talent for magic, especially \nieur{}, and already at a young age she discovers several new Kliffoth and new techniques to channel \iquin{} and \nieur{}, making her very powerful. 

At some point, \Belzir{} is chosen to become Empress. So she is schooled in all the things that an Empress must know, including all sorts of social relations. Despite the shortcomings she used to display in these areas, she learns quickly and becomes an adept communicator and a sly manipulator. But she still has problems attracting men, and the men she does attract seek her out for her position or her power. This is not enough for \Belzir; she wants them to desire her body. So she does a lot of research, and eventually manages to magically and permanently change her body, sculpting herself into the most beautiful and desirable woman she can imagine. 

What is \Belzir's relationship with the Cabal? Is she affiliated with it? Maybe the Cabal manipulates her into destroying the Empire because the Iquinian Church is a threat to them... 

Some people link her with the Prophecy of the Darkfall, which was foretold by Silqua. They believe that \Belzir{} is the Prophet of the Dark whose coming was foretold. (Note that several other people have born this accusation over the last 2500 years.) Eventually, \Belzir{} comes to fulfill certain parts of the Prophecy but clearly violates other parts of it. (She herself never believes the Prophecy, but jokingly accepts the title of Dark Prophet.) 

Anyway, she is ousted from power. But she manages to escape with her life. She plots to regain the throne. She allies herself with Clan Quaerin, in addition to her own Clan Geican. A great war ensues. 

\Belzir{} is slain, but her soul is not destroyed. Using powerful magic, her enemies successfully banish her soul from Mith (into the darkness of \nieur{}, as the Redcor tell the story), but they fail at destroying her. In time, she learns to communicate with people on Mith again and plots to return. 

Anyway, the defeat of the Queen does not end the war. Weapons of mass destruction are used. The Empire is destroyed and Mith bombed back several TLs. Thus end the Fourth Age. 



\section{The Modern Age}
At the beginning of the Modern Age, the technology is low. (Like the late Roman Empire, 300 AD-RL or so.) 

At about 450 AD, the Imetrium is formed. 

At some point, the Belkadians (a race of \humans{}) conquer a whole bunch of lands. They are secretly manipulated by the Redcor and the Iquinian Church. They come to rule a great empire, the Belkadian Empire, and the Belkadian language becomes dominant. After a few centuries, the empire collapses and splinters into a bunch of independent nations. The area is still collectively known as Belkade, and the Belkadian tongue is still spoken. 

Carzain is born in 1400 AD. 



\section{Epilogue}
Such is the history of Mith, a crucible of life, a fierce and bright flame amid the darkness of the Cosmos. Whatever is given to her, she will smelt down and reforge into something greater. 

The void gave her lowly chemicals, and she gave birth to the \krakens{}, immortal and terrible. 

The \voyagers{} gave her fish and reptiles and mammals, and she made them into \caderyns, \nycans{}, \cortios{}, whales and elephants, creatures wild and strong. 

\Moroch{} took the fish-reptiles from the sea and created \nagae{}. Mith took the \nagae{} and turned them into \dragons{}, majestic and proud. 

The \banes{} took the monkeys from the trees and created \humans{}. Mith took the \humans{} and turned them into Vaimons, noble and arcane. 

Only time will tell what further wonders and horrors will spring from the womb of Mother Mith, for time stands never still, and in her great forge, the fires burn ever hot... 



\section{Changes and new ideas}
The first \dragon{} was \HesodNerga{}. He set out to conquer the lands. At this time, they were inhabited by the \nephilim{}, who are the ancestors of \humans{}. Along with his daughter, Tiamat, and others of the first \dragons{} he wages a war of genocide against the \nephilim, and sometimes against the \Chthonians{}, the dwellers beneath the earth who resent the \draconic{} incursion. The \nephilim{} are easily slaughtered for the most part, but occasionally they offer genuine resistance. 

At some point, the \leviathans{} of the seas come to demand tribute of their vassals on the land. The \dragons{} refuse and the \nagae{} attack. So they find themselves waging a war on two fronts: A war of independence against their sires and a war of conquest against the \nephilim{} and \Chthonians{}. 

%Tiamat and her fellow \Dominators{} were some of the first \dragons{}. They rebelled against the \leviathans{} and found themselves waging a war on two fronts: A war of independence against their sires and a war of genocide against the dwellers of the land at the time, now called the \Chthonians{}. 

The \dragons{} were hard pressed, so Tiamat steps up to be their saviour and champion. The rulers of the \dragons{} at that time invest her with their power. Tiamat binds Chaos to her will and becomes one of the most powerful gods ever. At her side are her various lovers, of which the greatest are \ApepNesthra{} and \Iurzmacul{}. They become the \Dominators. 

With their legions of servitors, the \scathae{} and \cregorr{}, the \dragons{} were victorious: The \nagae{} were driven back into the sea and the \nephilim{} and \Chthonians{} were all but exterminated. 

Tiamat, however, is unwilling to relinquish power. She now wages a war of conquest against her own people, betraying many of her allies, butchering many and forcing the rest into submission. 

After this, Tiamat reigns for 3000-5000 years. 



\subsection{The \Bane{} Invasion}
\Semiza, the last surviving god-king of the \nephilim{}, still lives. Beneath the earth he dwells with the pitiful, degenerate descendants of the once-proud \nephilim. Consumed by hatred for the \dragons{}, \Semiza, a great mage, searches the Universe for powers to aid him in his revenge. With his mind he somehow reaches \Erebos{}. The \banelord{} Daggerrain becomes aware of him, so he contacts \Semiza{} and makes him an offer: If \Semiza{} helps him and his people to travel to Mith, they will destroy the \dragons{} and conquer the planet, granting \Semiza's people a place in the new world order. 

\Semiza{} agrees, and they begin working out the magic required to bring the \banes{} to Mith. Eventually, in a great ritual involving the sacrifice of countless \nephilim, \scathae{} and perhaps even \dragons{}, they succeed: The Vortex, the gate to \Erebos{}, stands open. The Vortex is actually not a mere portal but a deeper merging of the planes, a connection allowing a high degree of free movement from \Erebos{} to \Nyx{} and on to Mith. 

The \banes{} invade. The \baneking{} \Voidbringer{} wants to claim Mith as his own, so he sends Daggerrain along with millions and millions of \bane{} warriors and monsters. 
%Then all of a sudden, the \banes{} invade. The \baneking{} \Voidbringer{} wants to claim Mith as his own. 
The \dragons{} fight back and a cataclysmic war ensues, called the \Banewar{} (or maybe the First \Banewar). In the end, the \banes{} are defeated and the Vortex is closed. 
The connection between \Nyx{} and \Erebos{} is broken, leaving only a faint trickle of power. This leaves the \Dominators{} weakened and they all go dormant. 

%Also, unnoticed by the \dragonlords{}, the \banes{} leave behind the \humans{}, their created children who will later provide the conduit for the \banes{} to enter Mith once more. The \humans{} are created from the surviving \nephilim, twisted into \bane{} slaves. 

But some \banes{} remain, still led by Daggerrain. He approaches \Semiza{}, who had played no role in the First \Banewar{} after the summoning of the \banes{}. With the direct power almost all drained and destroyed, they will have to operate using stealth. As such, they will require non-\bane{} Mithian agents. The \nephilim{} are no good, but may still be of use. Daggerrain intends to create from \nephil{} stock a race of slaves through which they will work their schemes. \Semiza{} is dismayed at the prospect of having his people twisted into \bane{} slaves, but he sees this as the only way of getting his revenge on the \dragons{}, so he agrees. He himself also becomes the \banesz{} humble slave. And so he works to reduce his own people to wretched, feeble, creatures, savage and chaotic so that the \banes{} may utilize the power of Chaos through them, but still feeble of mind and easy to enslave. Thus the \nephilim{} become \humans{}. 

\subsection{The \Dragon{} Kingdom of Nom}
With the \Dominators{} absent, the \dragons{} fight amongst themselves. After a thousand years or two, some Bloodline gains dominance and founds the \Dragon{} Kingdom of Nom. The first \DragonKing{} (or Queen, maybe) is crowned. Nom conquers all of Mith and the \DragonKings{} reign for 15-30,000 years. 

At some point during the reign of Nom, a great civil war ensues. \Scathae{} use high-tech weapons against \dragons{}. The \dragonlords{} are dismayed by how \scathae{} with powerful weapons can easily kill \dragons{}, so after this war, technology is outlawed. This taboo would last for many thousands of years. 

Near the end of Nom's reign, the first \human{} empires arise, having learned civilization and technology from the \dragons{} and \scathae{}. I don't know if the \humans{} develop on their own for a while, or if the \banes{} are there to guide them all the way. At any rate, near the end, the \banes{} return to Mith and, with the backing of a \human{} empire, challenge the \dragons{}. Simultaneously, they have also infiltrated the \draconic{} courts and sown seeds of discord. 

So Nom is badly threatened by civil war already when the \banes{} attack. It comes to another cataclysmic war between \dragons{} and \banes{}. At a staggering cost, the \banes{} are finally defeated and believed to be driven off Mith again, but Nom is crumbling. Civil war breaks out and civilization collapses. 





%\subsection{Idea: The Three Worlds}
%All sorts of worlds exist in parallel and flow together. There are many such worlds, but the most important of them are Mith, \Nyx{} and the Chaos World. (The Chaos World might or might not be the same as \Makai.) So Rissitic metaphysics is actually correct. All worlds are coexistent, and in some sense all things exist in all worlds simultaneously. 

%I think the \dragons{} and \nagae{} live fully in the Three Worlds and can see freely into Chaos and \Nyx{}. Indeed, it is the nature of intelligent beings to know these Worlds. However, \scathae{} and \humans{} both were created as slave races, and as such, their creators saw fit to lock them up in the physical world. These creatures are by birth and by conditioning blind to the Worlds Beyond. 

%Or something like that. 

%Or maybe creatures can't see into the Worlds Beyond automatically, but only through special meditation or by taking drugs. In this case, the modern societies of \scathae{} and \humans{} are simply denied this opportunity. Knowledge of the Worlds is withheld from them, so they know only of the physical world. They are indoctrinated into blindness and their natural, latent vision is suppressed. Thus, when things from the Beyond occasionally manifest in the physical world where they might be seen, the minds of mortals block them out and conjure up mental illusions to hide them. Through magic and religious rituals, they glimpse the Beyond, but often they see only the small, controlled parts of the Cosmos that their shadowy overlords wish them to see, and those few who see further into the Beyond are branded as madmen or heretics. In this sense, the savage and primitive cultures are in a way more enlightened, as their religion might be free of Cabal/Sentinel control, and thus they might know more about the true universe. On the other hand, such cultures tend to be scientifically and morally savage as well. 

%At any rate, sometimes the barriers between the worlds break down and people can see or move into another world. This might happen in situations of stress/shock or through the use of meditation, magic or mind-affecting drugs. The wisest and mighties of mages know and can see all three worlds, perhaps more. 

%Animals have some limited, instinctive ability to see into Chaos, and the more intelligent animals can see into \Nyx{} as well. 

%\subsubsection{\Nyx{}}
%\Nyx{} is the World of Twilight. It is the home of the \Sephiroth. It might have been created by the \banes{}, or they may simply have occupied it. At any rate, \Nyx{} is the \banesz{} base of operation on Mith, and their powers come from there. The portals between \Nyx{} and \Erebos{} are now closed, but there remains much power in \Nyx{} for the \banes{} and their minions to channel. 

%Y'know, I think \Nyx{} might be a perverted, distorted version of an existing world, called the Dream. The Dream was originally the mental world of the \nycans{} and other animals. 

%I'm not sure if the entire Dream has been transformed into \Nyx{}, or if \Nyx{} is only a small part of the Dream. Probably the latter. 




\section{The Planes of Existence}
The universe consists of a number of \quo{planes}. 

%The various \quo{worlds} are now called \quo{planes}, to avoid confusion with Rissitic Three Worlds theory. 

%OK, so here's the deal: 
The deepest, original plane is Chaos. All other planes are subsets of Chaos. In a sense, they are different points of view from which a subset of Chaos may be seen, different ways of \quo{putting Chaos into order}. 

One such plane is the \quo{Mental Plane}. It is a plane of order and logic and the place where the thoughts of intelligent creatures come from. Within the Mental Plane is the Physical Plane, where Mith is. \Erebos{} is also a part of the Physical Plane, but geographically removed from Mith. The Physical Plane is more rigid in its natural laws than the Mental Plane is, and far more rigid than Chaos. 

\quo{The Beyond} is a collective term for everything besides Physical Plane, and also other parts of the Physical that are viewed through the Beyond. 

All creatures really extend beyond the Physical Plane. Their physical body is only a small part of their existence. The thing usually called the \quo{soul} is the part that dwells in the Dream - or something. But the true body (or true soul, if you prefer) is much larger, encompassing the physical body and the mental body/soul, but also extending into Chaos.\footnote{Perhaps a creature even extends into \emph{all} planes of existence. But probably not. That sounds too happy; it smacks of \quo{cosmic love and harmony}. No, I think a creature only occupies a very small part of Chaos. Nevertheless, it's a larger part of the universe than most creatures think.} Most creatures are unaware of their true self and therefore cannot consciously use it to affect other planes beyond the Physical.\footnote{And, to a limited degree, the Mental. Empathy is actually the act of reaching out through the Mental Plane and sensing others' souls, and \quo{frame control} (the act of affecting other people's thoughts simply by acting a certain way --- used in leadership and seduction) is reaching out through the Mental Plane and poking people.} The art of learning to control your true self and use it to perceive and affect the world is Magic. 

\subsection{Seeing into the Beyond}
Most animals have some of their consciousness in Chaos and the \quo{lower} regions of the Mental Plane, and a bit of it in the \quo{higher} Mental Plane. 
%Most animals are animated by Chaos and, to a lesser amount, the Dream. 
But they are still creatures of the Physical Plane. They have a very limited, instinctive ability to see into Chaos. Some of the \quo{higher} animals, such as dogs, cats, whales, \cortios{} and especially \nycans{}, have a limited ability to see into the Dream and \Nyx{}. 

Humanoids have some natural ability to see into the Beyond, but they cannot easily control it. However, the mind does not like these multiple planes, for the humanoid mind is weak and not apt at understanding the true nature of the universe. So the midn has a tendency to hide the Beyond and create a simplified view of the world, which, as the person grows older, becomes the only way he is able to see the world. Occasionally, we see evidence of the Beyond, but our brain explains it away and we make up lies and illusions to cover it up. 
Those few who do see into the Beyond and acknowledge what they find are branded as madmen and heretics. 
Young children have not yet fully developed such illusions, so they see the real world somewhat clearer (although they don't always understand what they see). That is one of the reasons children often claim to see monsters and faeries --- they really do. 

But this is cultural. A culture who believe certain things about the Beyond will see things that way. They will see certain snippets of the Beyond that fit their beliefs. Some cultures deliberately encourage a knowingly false view of the world to further their own ends. Through magic and religious rituals, they glimpse the Beyond, but often they see only the small, controlled parts of the Cosmos that their shadowy overlords wish them to see. 

People from primitive cultures sometimes see into Chaos. Especially Meccara. They have a greater talent for seeing Chaos than \humans{} and \scathae{}, but they do not see \Nyx{} and the Dream well. 

%At any rate, most of the time creatures can't see into the Beyond (a collective term for everything besides Physical Plane - and also other parts of the Physical that are viewed through the Beyond). 

At any rate, sometimes the barriers between the worlds break down and people can see or move into another world. This might happen in situations of stress/shock or through the use of meditation, magic or mind-affecting drugs. The wisest and mighties of mages know of the planes and can see into them. 

All magic works by binding spirits and forcing them to do your bidding. A \Sephirah{} or \Kliffah{} is actually a whole race of such beings... maybe. 

\subsection{Chaos}
The Plane of Chaos is the oldest and the \quo{deepest} of all the planes worlds. 
The \krakens{}, who may have created Mith, are creatures of Chaos. Tiamat and the \Dominators{} draw their power from Chaos. In the realm of Chaos there dwell \daemons{}. Maybe the \nagae{} and \leviathans{} live in some sort of harmony with these \daemons{}...? At any rate, Tiamat's power stems from her having enslaved a bunch of \daemons{} and forced them to do her bidding. Through these \daemons{} she works her magic. 

\dragons{} are able to see into Chaos, and so are the most learned Sentinels. 

%\Makai{} might \emph{be} the World of Chaos, but I think rather that \Makai{} is a world separate from Mith but accessible through Chaos. It is closer to Chaos than Mith. 
Some believe that \Makai{} \emph{is} Chaos, but this is not true. Rather, \Makai{} is another world in the Physical Plane, but it is accessible from Mith \emph{through} Chaos, and \Makai{} is more connected to Chaos than Mith and thus more permeated with Chaotic power. 

\subsection{The Dream}
The Dream is a part of the Mental Plane. It is a place of \quo{higher} thoughts and emotions, including the more social emotions of compassion and altruism. It was founded by the \nycans{}, and possibly also the \nephilim. 

\subsection{\Nyx{}}
\Nyx{}, the Twilight World, is a part of the Physical Plane, or at least very close to it. 
%\Nyx{} is a part of the Physical Plane, or at least very close to it. 
It is a \quo{pocket dimension}, a shallow mirror image of Mith... perhaps. Or perhaps it is a deeper, truer world than Mith. 

I am not sure whether the \banes{} created \Nyx{} (by twisting it out of the existing planes) or whether they simply occupied it. Perhaps \Nyx{} is a twisted mirror of the Dream, i.e., the \banes{} took part of the Dream and corrupted it, turning it into their own domain. As such, it is very much true and in a sense more true than Mith, but it is unnatural and artificial. Or perhaps \Nyx{}, with all its death, decay and stagnation, is perfectly natural. 

The purpose of \Nyx{} is to serve as a conduit between \Erebos{} and Mith. \Nyx{} was initially connected to \Erebos{}, but the \dragons{} successfully sealed it off from \Erebos{}. Today the \banes{} use \Nyx{} as a base of operations and a place to hide away from the \dragons{}. 

\Nyx{} is coexistent with Mith, and all things on Mith extend into \Nyx{}. %\Nyx{} is somehow less \quo{substantial} than Mith, so ...
It is possible, albeit difficult, to see from Mith into \Nyx{} and the other way around. \Banes{}, some of their creatures and the more learned Cabalists can do this easily, and can also easily move between the worlds. 
The \dragons{} and Sentinels know that \Nyx{} exists, but they do not fully understand it, and it is difficult for them to reliably see and move into \Nyx{}. \dragons{} can do it quite easily, and the best Sentinels also, but they don't understand it well enough to teach it to the lower-ranking Sentinels. So they have a hard time finding out what the \banes{} are up to. 

It is possible to stray close to and far from the border between \Nyx{} and Mith. To \quo{surface} is to move closer to Mith, and to \quo{submerge} is to move closer to \Nyx{}. In order to see into the other world you have to move close to the border, and that makes you easier to detect. 

The term \quo{\Nyx{}} is also used in a broader sense to refer not only to the Physical world of \Nyx{} but also to the part of the Metal Plane that surrounds it. This extended \Nyx{} is the source of \iquin{} and the \Sephiroth. This \nyxian{} power originated from \Erebos{}, but even with the gates to \Erebos{} closed, the \banes{} can still draw on this power. 

\subsubsection{\Iquin{} and \nieur{}}
\Iquin{} and \nieur{} are actually things designed by the \banes{}. 
%The force of \iquin{} originates from (the extended) \Nyx{}. It is a force created by the \banes{} in order to better control the \humans{} and to help them reopen the gate to \Erebos{}. \Iquin{} is divided into \Sephiroth, who each represent a \quo{virtue}. These virtues are actually means of mind-controlling people and tying them to the \banes{}' cause. \Iquin{} was created shortly before Silqua's discovery of it. 

%\Nieur{} is something closer to the \banes{}' original magic. 
%It is a combination of \draconic{} Chaos Magic theory (which works well on Mith) and the \banes{}' original magic theory (which works well on \Erebos{} but less well on Mith, now that it is sealed off from \Erebos{}). 
Originally, the \banes{} relied on magic that drew its power from \Erebos{}. But when Mith and \Nyx{} was sealed off from \Erebos{} at the end of the First \Banewar, the \banes{} lost the source of the magical power and were left crippled, and the \dragons{} and their minions were easily able to slaughter the millions of \banes{} still on Mith. Daggerrain and some other \banelords{} survived and were able to hide. Deprived of their magic, they had to develop a new magic theory. So they came up with \nieur{} theory, which is some principles from old \bane{} magic theory combined with some Chaos theory. So \nieur{} is a chaotic force, and the \Kliffoth{} are more or less the same as the \daemons{} which Chaos magicians invoke. 

But the \banes{} wanted a better tool to control their \humans{}. The problem was that if you taught \nieur{} magic to \humans{}, it tended to reinforce their chaotic nature and make them difficult to control. But without magic at all, the \humans{} were wimps. And so they created \iquin{}. It is a source of magical power, originating from (extended) \Nyx{}, but simultaneously a means of controlling people. \Iquin{} is divided into \Sephiroth, who each represent a \quo{virtue}. These virtues are designed to mind-control people and tie them to the \banesz{} cause, so that they may be manipulated to help the \banes{} achieve their end-goal of reopening the gate to \Erebos{}. %See, the more people invoke the \Sephiroth{}, the more they tie themselves 

When you invoke a \Sephirah, it touches your mind and subtly influences you. If you invoke the \Sephiroth{} often and with great force, you will gradually be more and more brainwashed by them. The \Kliffoth{} work in a similar way, but they twist people in a more chaotic way. People who channel \nieur{} a lot tend to become somewhat mad. The \Kliffoth{} and \Sephiroth{} tend to pull in opposite directions, though, so if you channel both regularly you may be able to retain your sanity and free spirit somewhat longer. 

So \iquin{} was created, and shortly after the \banes{} inspired Silqua to discover it, then helped Cordos Vaimon build his empire. Unfortunately, the Vaimons also learned of \nieur{}. This was not part of the \banesz{} plan, but probably sabotage from the \dragons{}. At any rate, at this point the \banes{} discourage the use of \nieur{} by \humans{}. High-ranking Cabalists still get to use it, and the \banes{} themselves use it, but these operate in the hidden anyway. So they try to forbid the use of \nieur{}. Therefore, the Redcor and other Vaimon clans use \iquin{} exclusively and scorn \nieur{} as a force of evil. Only a few clans, including the Geicans, are rebels and use \nieur{}. 

%\Iquin{} was created shortly before Silqua's discovery of it. 



%\subsection{Idea: \banes{}, \Sephiroth{} and \Kliffoth{}}
%\Nieur{} is actually the original thing. It is a chaotic force, similar to the Chaos that the \dragons{} use. 

%So at some point, the \banes{} wanted a new, better thing to control the \humans{}. So they took some of the \Kliffoth{} and twisted them for their own purposes. They became the \Sephiroth, whose purpose is to seduce and control the \humans{} 



\subsection{The Realms}
The universe can be divided into a number of Realms. 
The Realms are coterminous and overlapping, and it is possible to walk from one Realm into the \quo{corresponding} location on an adjacent Realm. 

The Realms include: 

\begin{description}
	\item[The Realm of Beasts:] The world of the land-living beasts of Mith, and some of the aquatic ones. The \nephilim{} and \meccara{} are native creatures of the Beast Realm (some tribes of them are still readily able to move into the deeper layers of the Realm). 
	\item[The Realm of the Deep:] The world of the deep seas. There is a gradual transition between the Deep Realm and the Beast Realm as one dives through the sea. The \krakens, \nagae{} and \vlekkeshsala{} are natives of the Deep Realm.
	\item[\Machai, the Realm of \Daemons:] A separate world, home to \daemons. Called also the Realm of Chaos, because it is perceived by some to be closer to Chaos than the other realms. 
	\item[\Erebos, the Realm of Darkness:] A separate world, the homeworld of the \banes. 
	\item[\Nyx, the Realm of Twilight:] A shallow, artificial Realm engineered by the \banes, intended as a conduit between \Erebos{} and the Beast Realm. The \resphain{} are dwellers in the Twilight Realm.
	\item[Mith, the Realm of the Veil:] Mith is not an actual Realm. It is a thin sliver of the Beast Realm. 
\end{description}

Originally the Realms were closely connected and it was fairly easy to travel betwen them. But in the great \Resphan{} War, both sides unleashed terrible spells in their attempts to lock the other side out of the world and imprison them in their own Realm. These spells reached out into the universe and tore at the fabric of the Realms themselves. 


The result of this cosmic catastrophe was the Veil\index{Veil, the}. The Veil is a phenomenon that not only makes the Realms less penetrable and more difficult to traverse, but also a mental illness of a sort, a collective delusion that has crept into the minds of all creatures of the known Realms, hiding the true nature of the Realms. Within the Veil, creatures can only see a narrow sliver of their own Realm; they cannot see nor move into the deeper layers of their Realm nor into other Realms, except in special places, following \emph{threads}\index{threads in the Veil} in the Veil, or at special occasions where there occurs a temporary \emph{tear}\index{tears in the Veil} in the Veil\footnote{The latter is what happens to Catrian in the chapter \quo{Silenced}.}. 



\subsection{The truth about the Rissitics}
\HriistN{} is actually the son of the \Dominator{} \ApepNesthra{} and a high-ranking Sentinel. This is not advertised because Rissit wants his Sentinel allegiance kept secret. 

The Rissitic metaphysics and the Three Worlds theory is a further development on Chaos magic invented by Rissit and \ApepNesthra{}. It is quite revolutionary: Spirit magic uses the energy of \Nyx{}, previously dominated by the Cabal. This is quite a breakthrough for the Sentinels. And Body magic manipulates the forces of the Physical Plane, ie., the normal, physical world of Mith. This is also quite new. 



\subsection{The truth about the Imetrium}
The gods of the Imetrium know more about the terrible nature of the world than they let on. They know that Mith is a horrible place. They know of Chaos and a little bit about \Nyx{}, and they know of the many terrible creatures that lurk in the world. They have a good idea of the history of \dragons{} and \scathae{} and suspicions regarding the origin of \humans{}. They do not know, however, that the \banes{} are still on Mith. They know that the Cabal and Sentinels exist, but they don't know what they are up to or who they really are, although they suspect. 

Salacar deals with it through denial and fanaticism. They hide the truth from their people. 


