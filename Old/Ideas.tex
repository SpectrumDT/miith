\chapter{Miscellaneous stuff}
Notes, fragments and ideas. 

\section{Creatures}
\subsection{Kraken}

Ocean-dwelling alien creatures of divine power. They are native to Mith and their race has existed for many millions of years. They look like enormous squid. Cthulhu elements. Remain dormany for millions of years due to astrological issues. 

The greatest of them, Moloch, is one of the mightiest gods on Mith. He may be the oldest Kraken alive... their father? Salacar has seen Moloch in a divining vision and fears him, but knows little about him. (Does Salacar know that Moloch is a Kraken?)

Kraken are asexual and reproduce by parthenogenesis. They are immortal and very rarely reproduce. The youngest Kraken alive is many million years old. 

Some marine creatures (y'know, deep ones) worship them. Other sea-dwelling creatures (and gods?) fear them. They also have cults on dry land. 

The Kraken have formidable powers of hiding/obscuring (compare to the ink cloud of a squid). 

Kraken may have waged war against Invaders and/or Tyrant Worms. 

\subsection{Marine civilizations}
\begin{itemize}
\item Crustacean men. Terraxuil?
\item Shark men and/or ray men (ixitxachitl). 
\item Some kind of deep ones who worship the Kraken. 
\item Ancient, fallen civilization - Lemuria?
\item Ancient, fallen cities - R'lyeh?
\end{itemize}

\subsection{Caederyn}
Enormous predatory fish. Heavily armored and with extreme regenerative powers. Caederyn are some of the most dangerous creatures of the seas. It is unknown if they are native to Mith or alien, but no related species are known. It is also unknown if they reproduce. No very small specimens nor eggs have been seen, and if two Caederyn meet, they usually fight (sometimes to the death). Fiercely territorial. Caederyn have a voracious hunger and are extremely aggressive. They are very stupid, but immune to all forms of mind control, even that of gods. They are virtually fearless and usually fight to the death. Global population is a few thousand. Full-grown Caederyn are around 30 meters long and very massive. Behemoths of 40 meters have been seen. 

\subsection{Dragons}
Dragons are great, reptillian creatures. All dragons are intelligent and most have an affinity for magic. Many species have a breath weapon. Some have other supernatural powers. Many species can fly. Life spans vary between species and races, typically 1000-2000 years. A few species live up to 3000 years. 

Dragons belong to the group Archosauria and are related to reptiles, dinosaurs and birds. They are quite genetically unstable and tend to mutate a lot, which explains why there are so many species and races. There are maybe around 25 Dragon races on Mith, each with an average of a few thousand individuals. The most common are Fire Dragons, Ice Dragons and Sea Dragons. 

\begin{itemize}
\item Fire Dragon. 
\item Ice Dragon. 
\item Several species of sea dragons (some of them sea serpents?). 
\item Hydrae (heads, regeneration, very mutable). 
\item Maybe the Suchrevian yellow dragon should be a Hydra? 
\item Leviathan Dragon. Ocean dwelling, largest species on Mith, most long-lived. One individual female is 2500 years old and the oldest and mightiest dragon known on Mith. Usually benevolent. They are some of the few creatures willing to take on Caederyn. 
\item Eagle Dragon. Beatiful and mighty feathered dragon. Leading figures in the war against the Invaders, now extinct on Mith. 
\end{itemize}

\subsection{Drakes}
Drakes are creatures related to and resembling dragons, but non-intelligent. (Not all dragon-kin are drakes, tho.) There exist many species and races. Some are quite small. One species is the Solar Drake, connected to Salacar and used as beasts of war in the armies of the Imetrium. 

\subsection{Jinn}
Desert-dwelling wind demons, native to Sulchrev desert. Also called Djinn. Used to rule the desert and its inhabitants, but overthrown by Rissit. Now mortal enemies of the Nechsaites. Jinn are mid-powered demons. They can assume incorporeal, invisible form (where they can affect only the minds of others, and only have their own minds affected). 

They can possess other creatures. Possessed creatures may spawn Half-Jinn. How do the Jinn reproduce among themselves? 

Some of the Jinn follow the goddess Es'phet. 

\subsection{Vaim�n}
Ancient civilization, now fallen. 

\subsection{Eta}
Rat-men. Small, but common. The name is Scathaese (singular Eta, plural Etae). Very widespead and numerous, but often subjugated. 

\subsection{Nur}
Doglike humanoids. Native to the land of Col. The name is Scathaese (singular Nur, plural Nuri). Also live many other places, but many of them are Imetrics. 

\subsection{Tchacolda}
Centaur-like creatures. Body of an antelope, humanoid torso and antelope head with horns. Do they have hooves or feet? 

Native to the cold tundras of the Northern Kingdoms to the North of Col. Aggressive warrior people. 

\subsection{Meccara}

\subsection{Fittera}

\section{Gods}

\subsection{Sherioch}
Sea-dwelling war god. The brother of Eoncos and son of Daxian and Isxae. Sherioch respects his brother and will sometimes come to his aid. 

He is a chaotic creature and believes that conflict and combat is the natural order of the Universe. He has a certain code of honor, however. 

Sheriochites are warriors, mercenaries, pirates and such. He is worshipped especially in the Northern Kingdoms. 

His avatar swims the seas of Mith in the form of a red shark. He hunts Caederyn for the challenge. Legends say that Sherioch will reward any heroes who slay his avatar in fair combat. (This has supposedly been done a few times in history.)

\subsection{Rekhtet}
The Queen of Swarming Insects. An insect goddess who serves Nechsain. Commands the insects of the desert to attack the foes of Nechsain. Also guards the Pyramids... or what?

\subsection{Es'phet and Tchesef}
Tchesef is an earth god and Es'phet is a wind goddess. They used to be lovers. 

Long ago, they served the Imetrium, but they betrayed them and joined Rissit. They served him for many years (decades? centuries?). Es'phet took a liking to the Jinn and again betrayed her master to side with them. 

After Es'phet's betrayal, Tchesef and other gods were compelled to take powerful oaths of loyalty to Nehsain.

What are Tchesef's own feelings?

Tchesef has power over the earth and the sand. Es'phet has power over winds and storms (but not rain or thunder). 

\subsection{Daxian}
Weather god, the Lord of Wind, Rain and Thunder. Worshipped in the Northern Kingdoms. 

\subsection{Isxae}
Goddess of law and civilization. Worshipped in the Northern Kingdoms. She is the wife of Daxian and the two are the primary gods thoughout the Northern Kingdoms. 

Isxae, with Daxian, is the mother of Sherioch and Eoncos. 




