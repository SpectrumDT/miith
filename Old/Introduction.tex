\chapter{Introduction}

\section{What is this all about?}
The Mith Universe is my attempt at creating a complete, fictional world that is internally coherent but independent of existing fantasy worlds. The setting is basically medieval fantasy. 

%Here I will explain some of the primary guiding principles and axioms that I have attempted to follow while designing the Mith Universe. 

One of my primary guiding principles is logical consistency. I have striven to create a world that is coherent and makes sense, one that is `realistic' in the sense that there are no contradictions or `plot holes'. 

%Another principle is completeness. I have tried to explain, in as much detail as possible, how my world works, and to create a comprehensive system of metaphysics to explain the supernatural elements of my world. 

Following these principles, I have attemped to create a fantasy world the way I think it ought to be. The result is something that borrows (or `steals', if you will) elements from many other worlds and systems (literature, RPGs, movies, mythology) to create something (hopefully) new and original. 



\section{What is the point?}
So, what is this good for? Well, the Mith world is intended for two purposes: Fiction and role-playing. 

One side of it is that I would like to write stories about the characters in my world. I have not written anything big yet, but I am working on it. 
%significant yet, but I might try to do so in the future. 

The other side is that the Mith world is intended for role-playing use. Even though the amount of actual role-playing I have done is very limited, I consider myself as having roots in RPGs. As such, many of the descriptions here are formulated as an RPG book might be, with advice to game masters (GMs) and players. 



\section{Role-playing in the Mith Universe}
As stated, the Mith world is intended partly for role-playing use. However, everything is presented in an informal manner. I have not formalized the statistics following any existing RPG system. So if you, as a role-player or group of players, want to try and play a campaign in the Mith world, you will need to come up with stats for all sorts of creatures, spells and things. 

If you actually do create formalized stats for my creatures for any system, or otherwise employ and expand upon my creations, I would love to know. Please do contact me with feedback. 



\section{Who are you, anyway?}
I am Claus. Hello. How are you doing? 

Anyway, my real name is Claus Appel. I often go by the nickname `Spectrum'. I am from Denmark, born on the 2nd of October, 1984. At the time of writing this, I study Computer Science at the University of Copenhagen. 

I have been a fan of fantasy and role-playing since I was, like, 6 years old or so, and have been creating worlds of my own since I was a little kid playing with my friends. (Especially noteworthy in this regard are my sister Laura and my two cousins Kirsten and Tue. It was when playing with these guys that I made my first alleys into creating fictional worlds.) 

My main interest in fiction has always been worlds and settings. When reading books or watching movies, I tend to regard the stories and characters as secondary; what really interests me is the background, the `Big Picture'. Some people prefer to write fiction by making up stories and characters, and then gradually fleshing out the background as needed. I do it the other way around: I start out by designing a great universe, complete with inhabitants, civilizations, religions and even an underlying system of complex metaphysics. Then I put in some characters and try to figure out something I could have them do, eventually creating a full-fledged story... maybe. 



\section{What if I think it sucks?}
If you have any sort of feedback regarding the Mith Universe - suggestions, complaints, requests for further information - you can contact me by e-mail. My address is: \href{mailto:spectrumdt@gmail.com}{spectrumdt@gmail.com}

Don't hesitate to contact me if you have relevant feedback. 



\section{Terminology}
The terminology I use in the following is mostly informal, with little reference to RPG statistics. There are, however, a few terms that I use which may need explanation. 

\subsubsection{TL: Tech Level}
One game term I use is tech levels, abbreviated TL. This is taken directy from the GURPS RPG in unaltered form. For those readers who don't know GURPS, the TL system gives a rough indication of how technologically advanced a given society (or item, or other) is. TL0 is Stone Age, TL3 is medieval, and TL7 is modern day (about 1950 to early 21st century). On the planet Mith (at the time I describe as `current'), the most advanced civilizations are late TL3. 

\subsubsection{RL: Real Life}
RL is an abbreviation for `real life'. It means just that, the real world, as opposed to the Mith Universe or any other fictional world. (The term is also used to distinguish traditional social interaction from `virtual' communities such as various Internet-based communities.) 

I typically use the term when comparing my world to the real world. Example: `On Mith, unlike in RL, the gods are real, existent beings who often take a very active interest in the world situation.'\footnote{If you are religious and find this statement offensive, you may want to immediately go and read the `Disclaimer' section.} 



\section{Disclaimer}
\begin{verbatim}
#include "disclaimer.h"
\end{verbatim}

\subsection{Mith is fiction}
The Mith Universe and pretty much everything in it is a work of fiction. It does not really exist.\footnote{Unless you believe in the theory that everything one can imagine already exists in some parallel Universe...} It is also not, and should not be viewed as, an allegory of any RL-issue. 

The Mith Universe is a non-political project. The views and morals of the various characters and cultures in my world should not be taken as representing my actual political views. Similarly, when I make comparisons between my world and RL, such statements are not necessarily political in nature.\footnote{Sometimes they are, but not by default.} 

\subsection{Political correctness}
I will not force myself to be politically correct. This means, among other things, that I use the male pronoun (`he') in a neutral sense. Because writing `he or she' once every three lines of the entire book is just gay\footnote{There, I did it again. This time I used the word `gay' in a derogatory sense. I am not homophobic, but I sometimes use the word `gay' as an insult. Live with it.}. 

\subsection{Beware: I am depraved}
Readers and parents, beware. I, the writer, am a dangerously demented person, and my blasphemous writings may pervert and corrupt your innocent soul, or that of your child. 

%My evil, wicked traits include the following: 

Here follows a listing of some of my evil, wicked personality traits. It is not necessarily complete; I may be more depraved than I dare admit, even to myself. But I refuse to submit to self-censorship, so my views sometimes shine through in my writings. Don't say you weren't warned. 

\subsubsection{Atheism}
I am an atheist. I do not believe in any kind of gods or supernatural beings. This sometimes manifests itself in some aggressive, contemptuous statements that attack or belittle religions and religious people. If you are deeply offended by such statements, you should proceed with caution, because you might encounter some. %(I am not saying that you \emph{will}, just that you might.) 

\subsubsection{Communism}
I am a communist sympathizer. I have some very `red' views here and there, and they sometimes shine through. The Imetrium, the archetypal `good' civilization on Mith, has some communist elements. This may be offensive to some readers.%\footnote{Although if this really offends you, then you have some serious issues. I suggest you learn to be less anal.} 

\subsubsection{Sex! Lots of sex!}
OK, claiming that the document contains `lots of sex' is an overstatement. But my point is that in the following, I will mention and discuss such despicable sexual perversions as masturbation, homosexuality, polygamy and more, as if they were perfectly normal and morally acceptable phenomena (because in my twisted, sinful mind, they are). This does not mean that my writing is openly erotic or pornographic (although it might be).%\footnote{It might have been, except that the inhabitants of Mith are not human. It's hard to make situations really erotic when the participants are all scaly lizard men...} 
It just means there there are references to sexuality here and there, including `deviant' sexuality. 

So if you live in a primitive, barbaric culture and suffer from the superstition that reading about sex is sinful or otherwise bad, then you might want to read no further. Or, if you are a parent, you might want to force your bigoted opinions down the throat of your child, and forbid him to read any further. 

Or, if you are a child in said culture, you might want to either show this disclaimer to your parents or guardians and await their judgement, or keep the book well hidden and make sure no one knows what filth you are reading.\footnote{If it was me, I'd go for the latter option, but you might be more pious than I.} 



\section{Influences}
So, you're wondering, where did I steal all this stuff from? Well, among my influences are (in no particular order): 

\subsubsection{Literature}
\begin{itemize}
  \item Robert Jordan
    \begin{itemize}
      \item \emph{Wheel of Time} series
    \end{itemize}
  \item Michael Moorcock 
    \begin{itemize}
      \item \emph{Elric of Melnibon�} series 
    \end{itemize}
  \item H.P. Lovecraft 
    \begin{itemize}
      \item Many stories 
    \end{itemize}
  \item George R.R. Martin
    \bi
      \item \emph{A Song of Ice and Fire} series
    \ei
  \item Robert E. Howard 
    \begin{itemize}
      \item \emph{The Gods of Bal-Sagoth}
      \item \emph{The Isle of the Eons}
      \item \emph{Conan} stories
    \end{itemize}
  \item J.R.R. Tolkien 
    \begin{itemize}
      \item \emph{The Lord of the Rings}
      \item \emph{The Silmarillion}
    \end{itemize}
\end{itemize}

\subsubsection{RPGs}
\begin{itemize}
  \item \emph{Dungeons and Dragons}. 
  \item \emph{Rifts} and other Palladium games. 
  \item \emph{GURPS} - various books. 
  \item \emph{Drakar och Demoner} (a Swedish fantasy RPG). 
  \item \emph{Call of Cthulhu}. 
  \item \emph{Stormbringer} aka \emph{Elric!}. 
\end{itemize}

\subsubsection{Mythology}
\begin{itemize}
  \item Christian
    \begin{itemize}
      \item Dante Alighieri - \emph{Inferno}
      \item John Milton - \emph{Paradise Lost}
    \end{itemize}
  \item Greek/Roman
    \begin{itemize}
      \item Homer - \emph{The Iliad}, \emph{The Odyssey}
      \item Aeschylus - \emph{Oresteia}
    \end{itemize}
  \item Egyptian
    \begin{itemize}
      \item Nile (death metal band with lyrics based on Egyptian mythology and history) 
    \end{itemize}
  \item Mesopotamian (Babylonian and Sumerian)
    \begin{itemize}
      \item Simon's \emph{Necronomicon} (fiction, loosely based on Mesopotamian mythology as well as other things) 
    \end{itemize}
\end{itemize}


