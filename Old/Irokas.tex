%\chapterrr{Irokas, the Dragon Kingdom}{Irokas}{Irokas}
\chapter{Irokas, the \Dragon{} Kingdom}
\labdex{Irokas}

%Irokas is located to the Northeast. It is a huge country, similar in size to Russia. Nominally it is ruled by a \DragonKing{} from the throne on Mount Irokas, but in practice, it is ruled by a multitude of Noble Bloodlines, many of whom are unsupporting of, or directly hostile toward the \DragonKing{}. 

East of Belkade lies the \draconic{} kingdom of \pronun{Irokas}{IH-ro-kaas}. It is a huge country, larger than all of Belkade, almost as large as all the rest of \KnownWorld{} put together. Irokas is home to the majority of all \dragons{} on Mith and ruled by a \DragonKing{}. 

The adjective associated with Irokas is \pronune{Irokassian}{ih-ro-K�S-see-an}. Irokas is sometimes called \quo{\intro{\Dragonland}} by outlanders. 

\section{Irokas society}
\subsection{Demography}
\textbf{Total population:} 15 million. 

\bi
  \item By race: 
    \bi
      \item \Dragons{}: 1000. 
      \item \Scathae: 9 million. 
      \item \Humans: 4 million. 
      \item \Meccara: 2 million. 
      \item \Rachyth: 50,000. 
      %\item Vaimons: 20,000. 
      \item \Dragonbrothers{} (various races): 200. 
    \ei
  \item Nobles (\dragons{} and \dragonbrothers{}), by occupation:
    \bi
      %\item \DragonKing{}: One. 
      %\item \Dragonlords{}: 70-100. (Note that each of these also belongs to one of the occupations listed below.) 
      \item Warriors and knights: 40\%. 
      \item Scientists and mages\footnote{This may be misleading. All \dragons{} learn magic, whatever their occupation. A \dragon{} `mage' is a \dragon{} whose primary occupation is the study of magic.}: 15\%. 
      \item Priests: 15\%. 
      \item Aristocrat\footnote{`Aristocrat' \dragons{} are those who don't really do anything and simply enjoy the life of an privileged noble. They may be active politicians who pull strings and seek power, or they may be lazy hedonists (a respected occupation in Irokas).}: 30\%. 
    \ei
  \item Commoners, by occupation and status: 
    \bi
      \item Vassals. 
        \bi
          \item Knights and officers. 
          \item Scientists, scholars and priests. 
          \item Head servants. 
        \ei
      \item Freemen. 
        \bi
          \item Soldiers and guards. 
          \item Scholars, minor priests. 
          \item Craftsmen. 
          \item Labourers. 
        \ei
      \item Serfs. 
      \item Slaves. 
      \item Savages. 
    \ei
\ei

\subsection{The class system}
The \dragons{} make up the ruling class of Irokas. The \dragons{} are organized into a number of noble \introi{Bloodlines}{Bloodlines!in Irokas}\index{nobility!in Irokas}. There are 25-30 Bloodlines. The government system is a feudal monarchy: The common folk serve a \introi{Liege}{Liege!in Irokas}, who in turn owes his allegiance to the King. 

The nobility are those who belong to one of the Irokas Bloodlines. Virtually all \dragons{} in Irokas belong to a Bloodline (those few who do not either live as outlaws or flee Irokas). Some \dragons{} who do not live in Irokas are still Irokas citizens and belong to a Bloodline, but most outlander \dragons{} are not a part of Irokassian society. In addition, there are a few non-\dragons{} who have been accepted into Irokassian nobility (see \dragonbrothers{}, section \ref{Dragonbrother}, below). No citizen can belong to more than one Bloodline. In rare cases, a \dragon{} can be expelled from his Bloodline. 

Every Bloodline has a leader, called the \intro{\dragonlord{}}. The \dragonlord{} has supreme authority over his Bloodline. He can be addressed by his Bloodline name with the prefix `Lord' or `\dragonlord{}' (so \Ishnaruchyfir{} (see section \ref{Ishnaruchyfir}) is Lord Brannocthur). Different Bloodlines have different traditions for selecting their Lord. In some Bloodlines, the \dragonlord{} is always the oldest living member of the Bloodline; in others, the Lord is the oldest child of the previous Lord who carries the Bloodline name (a \dragon{} does not necessarily carry the Bloodline name of his parent). The royal family is Bloodline Irokas, and the \DragonKing{} is always Lord Irokas. 

Among \dragons{} and \dragonbrothers{}, men and women are equal in status, and the \DragonKing{} may just as well be a Queen. (Note that a female monarch is called a \DragonQueen, but a female Bloodline head is still call a \dragonlord{}.)  

Non-\dragon{} inhabitants of Irokas (predominantly \scathaese) make up the common folk. They serve the \dragon{} Lieges as peasants, craftsmen, soldiers and menial servants. 

%There are many social classes in the country: 

Here is an overview of the social classes in Irokassian society, with assigned GURPS status. 

\bd
  \item[\DragonKing{}] (status 7), nominally the monarch of Irokas. 
  \item[\introi{High Priests}{High Priests!in Irokas}] (status 6). Each of the major gods of Irokas has one High Priest. 
  \item[\Dragonlords{}] (status 4 to 6, depending on Bloodline strength), the leaders of the Bloodlines. 
  \item[Adult \dragons{}] (status 2 to 5) fill roles as warriors, priests and scientists (especially mages). 
  \item[\Draconic{} children] (status 1 to 5, depending on the Bloodline) are apprentices to adult \dragons{}. 
  \item[\Dragonbrothers{}] (status 2 to 5) fill all sorts of roles. 
  \item[Outlander \dragons{}] (status 1 to 2) are respected for being \dragons{}, but are still outsiders in Irokas society. 
  \item[\introi{Vassals}{Vassals!in Irokas}] (status 1 to 3) are high-status non-\dragons{}. They are knights, mages and priests. 
  \item[\introi{Freemen}{Freemen!in Irokas}] (status -1 to 1) are city folk. They typically fill roles as craftsmen, soldiers/guards, minor scholars and head servants. 
  \item[\introi{Serfs}{Serfs!in Irokas}] (status -3 to 1) are bound to serve a \draconic{} Liege for life. Most are peasants. They are not property and have some rights. Note that you can have a high status in your own community (like a village mayor or healer) and still be a serf. 
  \item[\introi{Slaves}{Slaves!in Irokas}] (status -4) can be owned by any non-slave citizen (even a serf). Slaves are property, have no rights and may not own property. 
  %\item[Savages] are free people who swear their allegiance to no \dragon{} Liege. They may have any status
\ed

A special group are the `savages': Free people who swear their allegiance to no \dragon{} Liege. They may have any status in their own society (from a status -4 slave to a status 6 chieftain), but are considered outlaws and barbarians by the \dragons{} and their servants. 



\subsectionnn{\Dragonbrothers{}}{Dragonbrother}{\dragonbrother{}}
Normally, Irokassian nobility consists of \dragons{} alone, but once in a rare while, a non-\dragon{} may be granted an `honorary citizenship', noble status and Bloodline membership. These are called \dragonbrothers{}. 

To become a \dragonbrother{}, a person must be made Ward (see section \ref{Ward}) to a \dragon{} (or another \dragonbrother{}) of an existing Bloodline. Then he must earn a name (see section \ref{Draconic Names}). His name must be blessed by a High Priest, and his admission into the Bloodline must be blessed by the \dragonlord{} of that Bloodline. When these conditions are fulfilled, the applicant becomes a \dragonbrother{} and gains two new names: A manhood name and the name of the Bloodline he has joined. (His previous name is now considered an egg-name.) 

A \dragon{} who does not belong to an Irokassian Bloodline (typically because he is born outside Irokas) may also apply for Bloodline membership. Such \dragons{} must follow the same process as \dragonbrothers{} do, but they have an easier time gaining acceptance because they are, after all, \dragons{}. 

The tradition of creating \dragonbrothers{} is a recent one, less than a thousand years old. Many Bloodlines disapprove of it (\dragons{} are snobs and racists of the worst order), and there are still very few \dragonbrothers{} (about 200, less than one for every ten \dragons{}). Most \dragonbrothers{} (75\%) are \rachyth, taken as Wards by their \draconic{} parent. 

A \dragonbrother{} is considered equal in status to a \dragon{}. He may acquire names, take Wards and do anything a \dragon{} can. In principle, a \dragonbrother{} could become \dragonlord{}, High Priest or even King, but none of these have happened (and are not likely to happen for many centuries, if ever). 



\subsection{Language}
The official language of Irokas is called \introe{Kingstongue}. Kingstongue is extremely difficult for non-\dragons{} to learn. This is partly because of the pronunciation; Kingstongue has many harsh sounds and strange consonant clusters which are hard for other creatures to pronounce (\scathae{} and \nagae{} can do it fairly well, it is more difficult for \meccara{} and \humans). But the chief reason is the grammar. Kingstongue has a myriad of cryptic rules, exceptions and irregularities in grammar and syntax. \Dragons{} understand and follow these rules intuitively, without thinking, but for most non-\dragons{}, they are not remotely obvious, and memorizing them all is a monumental task. The reason for this is that the \draconic{} brain has a language center that is much more well-developed than those of most creatures, so the \dragons{} have a much stronger intuition for language than other creatures. This also means that \dragons{} have a great language talent and pick up foreign languages easily, even late in their lives. 

Kingstongue is a magical language, and many words and sentences spoken in the language have magical power. All magical words used in \draconic{} spells are actual words in Kingstongue. Because of its power, many non-\dragon{} mages (attempt to) learn Kingstongue for use in magic. Outside Irokas, Kingstongue is often simply called `\draconic{}' or the \intro{\draconic{} tongue}. Precise grammar and pronunciation is necessary for the magic to work. This, combined with \draconic{} snobbery, ensures that Kingstongue is very uniform across Irokas, with almost no dialectal variation. (Non-\dragons{} often speak Kingstongue with a strong accent, with the result that their spells don't work.) 

Most commoners in Irokas never learn proper Kingstongue, so they speak a vulgarized, simplified form of it, called Lowtongue. (The name `Kingstongue' originates from this distinction between the `proper' speech of \dragons{} and the `low' speech of the common folk.) Lowtongue is not a single language but exists in countless variants all over Irokas. Northern Lowtongue is unintelligible to a commoner from southern Irokas, but most commoners understand Kingstongue. (There are, however, isolated communities who merely pay their taxes to their Liege and almost never see a \dragon{}. These are unlikely to understand Kingstongue.) 

Irokassian `savages' (who serve no \draconic{} Liege) speak even more twisted forms of Lowtongue (sometimes completely unrecognizable) and rarely understand a word of Kingstongue. 

A phrase in Kingstongue: \emph{Geshdahaggloth bshanlachun sulgaa ukkho-Rakkonoz}. `May your \daemon{} be ever strong' (ie., the Guardian \Daemon{} of your Bloodline). A benediction; effectively: `May good things happen to your Bloodline.' 



\subsection{Politics}
In principle, by Irokassian law the \DragonKing{} rules supreme, not bound by any constitution or rules. In practice, the King is dependent upon the support of the \Dragonlords{}, something that is not freely given. In fact, since the \banewar{} and the formation of the kingdom of Irokas, no \DragonKing{} has truly ruled more than half of what is formally his kingdom. 

Bloodline Irokas, the royal family, is the mighties of the







\section{Religion}
%\subsection{Gods}
The \dragons{} and people of Irokas worship a number of gods. These can be divided into three groups: The \Dominators{}, the Progenitors and the Other Gods and \daemons{}. Of these, the \Dominators{} are the most important. They are the mythical creators of all \Dragonkind{} and are venerated by all \dragons{}. The \Dominators{} are Tiamat, \Iurzmacul, \Typhon{}, Khoth-Sell, \ApepN{} and Shenimuss. 

The Progenitors are a younger generation of gods. They are the founders of the Bloodlines that rule Irokas, and Nom before it. Each Bloodline has one Progenitor, and some have more than one. Each Progenitor is worshipped only by those of his own Bloodline. 

\Daemons{} are creatures from other worlds that can be communed with using magic. Much \draconic{} magic involves summoning forth \daemons{} and coercing them to do your bidding. The mightier \daemons{} cannot be subjugated and must be bargained with. Some of the greatest of these (sometimes called \Daemon{} Lords) are worshipped as gods. `Other Gods' is a generic term for powerful, non-\draconic{} and non-humanoid gods. \Daemons{} and Other Gods are not well-defined terms, and there is overlap between the two. 

\subsectionthe{\Dominators{}}
\subsubsectionn{Tiamat}
Tiamat, called the Chaos Queen, is the most powerful and most important of the gods of Irokas. She is considered the Great Mother of \Dragonkind. She is worshipped as Chaos and Power incarnate and as a god of life and fertility. 

Tiamat's symbol is an image of herself; a Hydra, customarily depicted with seven heads. Her colour is all the colours of the rainbow. 

\subsubsectionn{\ApepN{}}
\pronun{\ApepN{}}{�-pep-NES-thra} is the god of immortality and undeath. He is the rival of Khoth-Sell. 

\ApepN{} is represented by a salamander, and his colour is white. 

\subsubsectionn{Khoth-Sell}
\pronun{Khoth-Sell}{KHOTH-sel} is the goddess of death. She is the daughter of Tiamat and \Iurzmacul{}. Cold and cruel, but not quite as monstrously alien as \NerrhanKoss{}. 

Khoth-Sell's sacred animal is a worm. Her colour is black. 

\subsubsectionn{Shenimuss}
\pronun{Shenimuss}{SHE-ni-mus} is the \draconic{} goddess of knowledge and science. She is the daughter of Tiamat by an unknown father. 

Shenimuss' symbol animal is a bat, and her colour is gray. 

\subsubsectionn{\Typhon{}}
The most brutal and cruel of the \Dominators{}, \pronun{\Typhon{}}{TAJ-fon} is the god of war. He is the son of Tiamat. His father is not known, but some believe that Tiamat mated with a cruel \Daemon{} from another world to conceive \Typhon{}. 

\subsubsectionn{\Iurzmacul{}}
\Iurzmacul{} is the \draconic{} god of wisdom and the sea. He was one of Tiamat's many mates and the father of Khoth-Sell. He is the gentlest and most benevolent of the \Dominators{}. 

\Iurzmaculz{} animal is the crocodile and his colour is blue. 



\subsection{Others}
\subsubsection{Moloch}
Moloch is a Kraken, considered an Other God by the \dragons{}. He is known to predate Tiamat and is believed by some the true creator of \Dragonkind. 

\subsubsectionn{\NerrhanKoss{}}
\label{Nerrhan-Koss}
%Dark, evil star-god. A la Gol-Goroth from `The Gods of Bal-Sagoth'. 
\pronun{\NerrhanKoss{}}{NE-rrhan-kos} is one of the most well-known of the Other Gods. He is worshipped as a god of knowledge, who holds and teaches countless sinister and occult secrets of the universe. His church is a rival to that of Shenimuss. 

\NerrhanKoss{} is actually a neutron star, a type of dead star. He was originally a large bluish-white star of spectral class A weighing around 7 solar masses. Somehow, the star became sentient and began exploring the universe using telepathy. After about a billion years\footnote{This number is pulled out of thin air. How long does a 7-solar-masses star live?}, \NerrhanKoss{} went supernova and collapsed into a neutron star, but he managed to preserve his consciousness using magic. He is thus, in a certain sense, an undead star. 

\NerrhanKoss{} emits beams of light from his poles, and is thus a pulsar, but these beams never strike Mith, so to the Mithians, \NerrhanKoss{} appears completely black. Consequentially, he cannot be seen in the sky with the naked eye, and his location can only be discovered using magic. The sky is well-mapped, however, and any \dragon{} with some knowledge of astronomy will know where he is, even if he cannot be seen. He lies in the constellation known to the Vaimons as the Salamander, about 15,000 light years from Mith. 

The priests of \NerrhanKoss{} fervently spread the belief that the god looks after his followers and will strike down those who harm them. This is a lie: \NerrhanKoss{} cares nothing for his followers and will never aid them directly. Furthermore, he is completely amoral, does not believe in revenge and punishment and does not have feelings of love, hate or anger, so he would never see a reason to do such a thing. But \NerrhanKossz{} clergy benefit from this widespread superstition and the fear it generates, so they maintain it. On more than one occasion, the priests have successfully faked such `divine retribution' using their own magic. 

\NerrhanKoss{} is sometimes described as a god of darkness. His priests will explain that this is purely metaphorical, partly because of his appearance as a dark, invisible speck in the sky, but more importantly because of the `dark', hidden nature of his teachings. 



\subsection{Irokastic Afterlife}
Those who follow the religion of Irokas are reborn on distant planets. They retain their memories, locked away in their subconscious. In their dreams, this knowledge can be unlocked. \Dragons{} often pray to their ancestors for this knowledge. Sometimes, a dreaming ancestor responds with advice. 







%\sectionn{\Draconic{} history and mythology}
%It came to be that the \nagae{} who dwelt upon the earth came to desire power and greatness as embodied by the Leviathans of the sea. Using



\sectionn{\Draconic{} Magic}\label{Draconic magic}\label{Irokas magic}
The \dragons{} of Irokas have their own magical tradition, which is the oldest magic theory known on Mith (excepting alien civilizations). 

\subsectionnn{\Draconic{} Names}{Draconic Names}{Names!\draconic{} names}
An important part of \draconic{} magic are names. The \dragons{} of Irokas carry mystic names with great occult power. As a \dragon{} ages, he will take additional names, so an old \dragon{} will carry several names. 

\subsubsection{Irokas naming tradition}
A hatchling \dragon{} has a single name, called the \emph{egg-name}\index{Names!egg-name}\index{Egg-name}. This name, traditionally chosen by the mother, has no magical significance. It is typically meaningless and chosen for its sound. Among some Bloodlines, the egg-name is given to the egg before it hatches (hence the word), while in other parts the name is not chosen before the hatchling is born. 

Shortly after hatching (a matter of days or weeks), a young \dragon{} goes through a ritual of naming, where he is acknowledged as a legitimate child and given the name of his Bloodline. 
%Traditionally, a child inherits the Bloodline name of the oldest of his two parents. 
A child can receive the Bloodline-name of his father or mother, but not both. 
This is the first magical name a \dragon{} receives. 

A young \dragon{} must learn something, so he is given as an apprentice to a master. The teacher is called the \introe{Warden}, and the apprentice is the \introe{Ward}. The Warden is sometimes a parent or close family member. The youngster must choose to specialize in some profession, typically as a warrior, mage or priest. All \dragons{} are taught some magic, combat and Irokas culture. 

In order to be considered an adult, a \dragon{} must pass a test of manhood (or womanhood). This is a `graduation' from apprenticeship. The young \dragon{} chooses a manhood-name. [THINK OF A BETTER TERM!] 

\subsubsection{Things names can do}
\begin{itemize}
	\item Make you stronger, faster or tougher. 
	\item Make you smarter, allow you to think better in a certain aspect. 
	\item Give you commanding presence, making it easier to lead or bully others. 
	\item Give you charming presense, making it easier to persuader people, making them like you. 
	\item Give you frightening presence, making it easier to intimidate others. 
	\item Increase your ability in a certain area of magic. 
	\item Increase your ability in a certain area of physical or mental activity. 
	\item Sharpen your senses. 
	\item Make you resistant to certain forms of attacks and/or magic. 
\end{itemize}

\subsection{List of \draconic{} spells}
\spelldra{Agony}
Causes victim to writhe in pain. 

\spelldra{Alter Breath}
Personal only. This is not a single spell but a group of spells that lets a \dragon{} alter the effect of his own breath weapon. Must be maintained. As long as the spell lasts, the \dragon{} may choose to use his normal breath or the altered one. 

\spelldra{Blight}
Infests an area with dark energy. Destroys vegetation and damages creatures. 

\spelldra{Decrepify} 
Permanently lames a limb or destroys sight, hearing or an equivalent sense. 

\spelldra{Destroy Undead}
Like the Slay spell, but works on undead creatures. 

\spelldra{Drain Lifeblood}
Personal only. For the duration of the spell, your physical attacks drain life energy and transfers it to you. (This can be physical damage or mental energy - two variants of the spell.) Only works with unarmed attacks, and only those that draw blood. A bite attack is most effective, but a claw or tail strike works as well. 

\spelldra{Earthquake}
Causes a big earthquake. 

\spelldra{Enslave}
Makes a person serve and obey you. Will only work on creatures of significantly weaker will. Difficult to cast. Once successfully cast, the subject will believe that he serves willingly. Maybe the spell requires a Rune to be inscribed and must be maintained through the Rune? 

Do not confuse with the Rissitic Spirit spell of the same name (p. \pageref{Rissitic spell Enslave}). 

\spelldra{Healing}
Heals yourself or another. May regenerate lost limbs. 

Do not confuse with the Rissitic spell of the same name (p. \pageref{Rissitic spell Healing}). 

\spelldra{Invulnerability}
Personal only. Makes the caster extremely tough, almost invulnerable. Expensive in energy to cast and maintain, can only be maintained for a short while. 

\spelldra{Iron Skin}
Makes the subject resistant to physical damage. 

Do not confuse with the Rissitic spell of the same name (p. \pageref{Rissitic spell Iron Skin}). 

\spelldra{Mystic Eye}
Summons and invisible \daemon{} that can move around. The mage can see through its eyes. 

\spelldra{Rune of Contingency}
A Rune is inscribed on a surface and enchanted with one or more spells. It will release the spell(s) when some simple condition is met. 

\spelldra{Shatter}
Damages and attempts to destroy a single non-living item. 

\spelldra{Slay}
Slays or damages one or more creatures. The spellword is \introep{khestni}{KHEST-nee} (litterally `die'), possibly with additional words added for specific effect. 

Do not confuse with the Rissitic spell of the same name (p. \pageref{Rissitic spell Slay}). 

\spelldra{Slaying Touch}
Personal only. For the duration of the spell, your touch inflicts magical damage by draining life force. Life force is \emph{not} gained by the caster. 

\spelldra{Terrify}
Causes terror in all people who already have a cause to fear the \dragon{}. 







\section{Characters}



\subsectionn{Tentocoth}
\DragonKing{}. Age: 500? 

A competent King, but weighed down by his father's immense expectations. 

\paragraph{Name:}
Frycad Vaccashyth Rhoahathnex Ibrethnavish \pronun{Tentocoth}{TEN-to-koth}. 

\paragraph{Appearance:}
Colour: Deep blue, with his belly a brighter, teal colour. Black markings on his head and brighter blue patterns on his wings. 



\subsectionn{Noreocchyrias}
Previous \DragonKing{}, father of Tentocoth. Died at the old age of 1600. Noreocchyrias was a very capable King, but ruled with great brutality. Earned many enemies. 

Noreocchyrias wanted Tentocoth to reunite the Kingdom of Irokas and bring glory to their dynasty. 

\paragraph{Name:}
\pronun{Noreocchyrias}{no-re-ok-CHAJ-ree-�s}

\paragraph{Appearance:}
Colour: Dark ultramarine blue. 



\begin{comment}
\subsectionn{\Ishnaruchyfir}
Ishnaruchyfir is an old, venerable \dragon{} warrior, \dragonlord{} of the Bloodline Brannocthur and one of the most powerful nobles in Irokas. Not only is he the father of Queen Drinzethel, he is also of royal blood, being the cousin of King Noreocchyrias (his mother was the daughter of a reigning Queen). He is King Tentocoth's most trusted and loyal supporter. 

Above all else, Ishnaruchyfir is loyal to the traditions of Irokas. He is highly conservative and resists change. Among other things, he greatly resents skekrathuin\footnote{Skekrathuin are a \draconic{} weapon, the equivalent of swords. They are large metal blades worn on the arms and tail. Usually, skekrathuin are strapped onto the arm, but sometimes they are held in the hand. Most skekrathuin are single-edged and meant for swiping outward.} and refuses to use them. He also refuses to wear any kind of armor or clothes and walks fiercely naked at all times (except, sometimes, for a flag with the blazon of his Bloodline, flying from his horn). 

\paragraph{Name:}
\pronun{\Ishnaruchyfir}{EESH-naa-roo-'chaj-feer}, \dragonlord{} of the \pronun{Brannocthur}{BR�N-nok-thoor} Bloodline. 

\paragraph{Appearance:}
Not very long, but very massive and heavy. He is coloured in black and red. 

He has a single backward-pointing horn on his forehead. 
\end{comment}



\subsectionn{Thiencaste}
Crown Princess, daughter of Tentocoth. Objects to her father's brutality. May be seduced by the Imetrium... or \NerrhanKoss. 

\paragraph{Name:}
\pronun{Thiencaste}{thee-en-K�S-te}.

\paragraph{Appearance:}
Colour: Very beatiful, like her mother. Shining yellow, like the Sun, with green markings. 



\subsectionn{\Cryocas{}}
Daughter of Tentocoth. Was originally Crown Princess (by decree of the augurists/prophet-priests), but when Thiencaste hatched, it was decreed that she instead should inherit the crown. \Cryocas{} now hates her usurping sister and covets nothing more than the crown of Irokas. She is willing to to any length to become Queen, including assassinating her sister, or conspiring with evil forces such as Xarocchetsel. 

\Cryocas{} is extremely intelligent and talented in the art of intrigue. She is so devious that she regularly makes blatant, embarrasing mistakes, to keep her enemies off guard and hide her true intelligence. She fears \NerrhanKoss{} and Xarocchetsel, because she knows that he will try to manipulate her. She may, in time, conspire with her grandfather and grandmother. 

\Cryocas{} is married to \Nisgzarchief{}, who shares her dreams of the crown. She was married at a very young age, as she tricked her parents and his into believing that they arranged the marriage, and to their own advantage. 

She is much more clever than her father and can easily manipulate him. The King cannot fathom that she might betray him or Thiencaste, just like he would never, under any circumstances, betray his own father and legacy. 

\paragraph{Name:}
\pronun{\Cryocas{}}{KREE-o-k�s}.

\paragraph{Appearance:}
Colour: Deep violet. 

She is very long (as long as King Noreocchyrias at her age), but thin and slender. 



\subsectionn{\Nisgzarchief{}}
The husband of Princess \Cryocas{}. A very talented young mage, and hungry for power. 

\paragraph{Name:}
\pronun{\Nisgzarchief{}}{nees-GZAAR-chee-ef}.

\paragraph{Appearance:}
Short and fat. His colour is black. 



\subsection{The priest of \NehranKoss-in-disguise}
Apparently nice guy, who represents \NehranKoss-in-disguise. Secretly evil. Serves the Xarocchetsel. Tries to convert Thiencaste to his religion, so he can manipulate her. 



\subsectionn{Xarocchetsel}
Evil \dragon{}. A la Gothan from `The Gods of Bal-Sagoth'. Wants to usurp power and rule behind the throne. Everyone distrusts him, but fear him. Tentocoth dares not take overt action against him, for fear of \NerrhanKossz{} retribution. 

\paragraph{Name:}
\pronunlong{Xarocchetsel}{ksa-rok-CHET-sel}{Notice that the initial X is pronounced as an actual X, ie., a [KS], not an [S] or [Z].}.

%Xaroc is an archaic word for `student, seeker of knowledge'. Chetsel means `adament, unyielding'. 

\paragraph{Appearance:}
Colour: Gray. 



\subsectionn{Drinzethel}
\DragonQueen, wife of Tentocoth and mother of Thiencaste. (Does the \DragonKing{} have more than one wife?) She is good and wants her family to be happy. Not much into politics. 

\paragraph{Name:}
\pronun{Drinzethel}{DRIN-ze-thel}.

\paragraph{Appearance:}
Drinzethel is considered a very beautiful \dragon. Her scales are sparkling emerald green with yellow and black markings. 



\subsectionn{Dielkethiryn}
Cousin of Tentocoth, daughter of Noreocchyrias' sister. She claims that Noreocchyrias unlawfully stole the throne that belonged to her mother, and that Tentocoth's rule is unjustified. Has some public support among the Bloodlines, and much more in secret. 

She is an extremely competent leader, perhaps more so than Tentocoth. A la Santil-k�-Erketlis. Also more idealistic and less cruel than Noreocchyrias and Tentocoth. May ally herself with the Imetrium in order to win the throne. 

\paragraph{Name:}
\pronun{Dielkethiryn}{dee-el-KE-thi-rajn}.

\paragraph{Appearance:}



\subsectionn{Rochydanoss}
A Dark Lord of some kind. A plotter and schemer who wants to rule a whole lot of stuff. 

\paragraph{Name:}
\pronun{Rochydanoss}{ro-CHAJ-da-nos}.

\paragraph{Appearance:}



