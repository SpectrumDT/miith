\chapter{The Kinsari Civilization}

The Kinsari species is native to the isles of Eruil, where they maintain a thriving civilization.

\section{Dylur and the Shunuil}

Dylur is a planet orbiting the star Mith, about 150 light years from Myrrnoth. Dylur used to be a sentient creature and a nexus of great magical power. The magical forces created the occasional dimensional portal to open, and from time to time, the planet was visited by strange lifeforms. Dylur was the only sentient lifeform in the Mith system, but over the millennia, it gradually learned to communicate with the creatures that travelled to it.

Dylurian climate was not unlike that of Myrrnoth, and eventually, life evolved on the planet. Dylur desired to have intelligent creatures living on it, so it bent its efforts on evolving intelligent life. After many millions of years, it finally succeeded in breeding such a people: The bird species that became known as Shunuil. 

Dylur was a thoroughly gentle and benevolent creature, and so with the guidance of their parent \footnote{Dylur is a neuter creature, so I will use "`parent"' rather than "`father"' or "`mother"'} world and god, the Shunuil created a peaceful and utopian society. Though their civilization lived for millions of years, they never rose above stone age technology (TL0), for they had plenty of food and no natural enemies. The only science in which the Shunuil became advanced was magic. 

Eventually, however, the immense magical energies attracted enemies...

Magical catastrophe makes all eggs hatch into males. 
They create birthing flowers to reproduce. 
They discover Banes heading for them. 
Dylur commits suicide, transferring all the magical power he can into the ten Holders. 
Holders escape. 
Banes arrive and slaughter the remaining Shunuil. 
Today, dead Dylur is a Bane fortress. 
Holders travel the Universe for about 100 years.
They come to Myrrnoth, discover Kinsari. 
Spend 300 years caring for Kinsari, then flee. 
This was 500 years ago. 

\subsection{Dylur}

The magical energies permeating it gave Dylur divine powers equal to a divine rank of rank 10. 

Alignment: Neutral good. 
Divine rank: 10. 
Home: The mortal plane, in orbit around Mith. 
Portfolio: Magic, life. 

\subsection{The Shunuil species}

The Shunuil are birds, evolved from wading birds similar to storks or flamingos. Full grown, they stand about 150 cm tall but are very slender and delicate of build. They have large wings and are very capable (but not very fast) flyers. They have beautiful plumages that shine with all the colors of the rainbow. There is great variation in their colors, and no two individuals are alike, but red colors are the most common. They are carnivorous and eat fish. 

The wings are used only for flying, but the feet are dextrous and can be used as hands. Each foot has four digits. They cannot easily "`sit"' on their rump, and thus can use both hands only in flight. They have a good sense of balance, however, and can easily stand on one foot and use the other as a hand. 

Physically, the Shunuil are weak, having no natural weapons to speak of. They can bite with their beaks and claw with their talons, but neither are very effective weapons. Generally, a Shunuil will be no match for most creatures of similar size. On Dylur, however, every Shunuil was a mage and could, if necessary, wield the immense magical power of their parent world in defense. They may wield weapons with their feet, but on Dylur, all they had were sticks and stones.

Shunuil have high-pitched but melodious voices and may learn to speak the languages of most creatures on Myrrnoth. On Dylur, they all spoke a common language, now known as Dylurian. At TL0, their average lifespan was 40 years. The Dylurian population never exceeded five million. 

Now, the Shunuil are all but extinct. Apart from the ten holders, there may be small pockets of them scattered across the Universe, but none are known to exist. 
