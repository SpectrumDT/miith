\documentclass[a4paper,10pt]{book}
\usepackage[english]{babel}
\usepackage[latin1]{inputenc}
\usepackage{verbatim}
\usepackage{graphicx}

\title{The Myrrnoth Universe}
\author{By Claus Appel (spectrumdt@gmail.com)}

\begin{document}

\maketitle

\tableofcontents

\chapter{Introduction}

Elements of the Myrrnoth world date way back. Milestones:

\begin{itemize}
\item{1997:} Scathae created. (Or was it 1996?)
\item{1999:} Cayderin fish created. 
\item{2000:} Rauthor and Balrogs created.\footnote{Of course, I didn't create 
the name 'Balrogor the overall concept, but I made my own interpretation of 
them.} 
\item{2004:} Imetrium created. Nechsaitic society sketched out. I start writing 
this down. 
\end{itemize}

\chapter{Myrrnoth}

Myrrnoth is a planet. It is inhabited by many intelligent species and 
civilizations, who worhip various pantheons of gods. 

The planet Myrrnoth (or at least the part of it which I have yet made up) can 
be divided into three areas:

\begin{enumerate}
\item The Old Continent, containing the homelands of most of the dominant 
species. 
\item The New Continent, containing a number of multicultural societies, most 
notable of which is the Yet Unnamed Empire (ruled by the Yet Unnamed God, who 
maintains the neutrality of the Empire, in respect to the various religions and 
powers of the Old Continent). Most native species are primitive or few in 
number. 
\item The seas, inhabited by many civilizations and gods.
\end{enumerate}



\section{The Old Continent}

There are a number of civilizations in the Old Continent. The ones I have spent 
the most time describing so far are:

\begin{enumerate}
\item The Scathae and the Imetrium. 
\item The Rachyth and the Nechsaitic Dominion.
\item The Kinsari culture.
\item The Tchacolda kingdoms.
\item The Meccara tribes.
\item Irokas, the land of Dragons. 
\item The subterranean Mlisshur. 
\end{enumerate}



\subsection{The Geography of the Old Continent}

\subsubsection{RL analogy}
I like to think of the geography of the Old Continent as analogous to Europe 
and Asia in RL. Note that the following comparison is purely geographical, not 
political in any way. 

Imetric Caeoll is the equivalent of Spain and Portugal, while Sulchrev 
corresponds to Turkey and the Middle East. The Tchacoldan kingdoms are 
Scandinavia and the Meccara live in Central Europe (Hungary, Czech Republic and 
so on). Distant and mystic Irokas then corresponds to Russia, while the Orient 
(India, China) is inhabited by strange and exotic civilizations. To comlete 
this analogy, the New Continent is the equivalent of America.

\subsubsection{Caeoll - Heartland of the Imetrium}
To the west and south of Caeoll lies the ocean. Most of the western coast is 
obstructed by a great mountain range. The land has some accessible coastline to 
the south. 

To the north lie the Northern Kingdoms. The Imetrium would like to control more 
coastline, so they sometimes make war on the Northern Kingdoms to expand their 
borders. Maybe there is a great river that runs through the Imetrium but ends 
in non-Imetric territory? If so, they would like to conquer it. 

\subsubsection{Northern Kingdoms}
The Northern 'Kingdoms' are not actually kingdoms, but a collection of clans, 
tribes and city-states that sometimes band together under a Khan or something. 
Many Tchacolda live here. What more?

The kingdoms have plenty of sea coast and rivers (fjords?). Some people in the 
kingdoms occasionally go on viking raids. 


\chapter{The History of Mith - one version}

Billions of years ago:

Nether beings are native to Mith, called Dark Elders. 
Dark Elders dwell in the oceans and dominate Mith. 
Dark Elders go dormant. 
A number of dynasties of gods rule Mith. 



- 1 million:

Mith inhabited by different cultures and gods, called Old Gods. 
Mighty Nether powers come to Mith, called Invaders. 
There are 10 Invaders. They are led by Hyrakht. 
Each Invader represents a force of evil. 
Hyrakht = oppression, Ezzeyrath = envy. 
Others: Fear, anger, ignorance, apathy, arrogance. Last three unknown. 
Invaders conquer Mith, Old Gods flee or perish. 
Invaders create Tyrant Worms (Unn-Hathr) to enslave population. 
Worms are powerful psionics. Use mind control to rule. 
Resistance is formed, led by Eagle Dragon King Rei-Canix. 
Resistance endures for hundred thousands of years under the unbroken dynasty of the House of Rei-Canix. 
Worms fear water. Worm mind control does not work through water. 
Life goes on beneath the seas. 

Rebellion is in hiding. Seems weakened. 
Tyrant Worms rebel against Invaders. 
Invader Ezzeyrath betrays his fellows and sides with Worms. 
Great Cataclysm ensues. 
Ezzeyrath slain by Hyrakht. 
Rebellion emerges from hiding. 
Five other Invaders perish. 
Hyrakht curses Worms, so that Sunlight kills them. 
Hyrakht and two other Invaders imprisoned on Mith. 
One Invader flees. 
99\% of Worms die. 
Rebellion triumphs, Worms forced underground. 
Eagle Dragon species exterminated in the war, only survivors are a few royal eggs.
Only one egg survives and hatches, becoming Princess Iza-Lyranis. 

Golden Age ensues, lasting for approximately 2000 years. 
No gods live on Mith, only distant alien gods are worshipped. 
Mith in peace, with the Princess as a global symbol of peace. 
Mith people develop high technology. 
Reach TL 8/9, develop primitive space travel. 
Princess Iza-Lyranis, last Eagle Dragon on Mith, dies at age 2066. 

With the Princess' death, religious conflicts escalate.
War of Oblivion begins. Lasts for about 50 years.
Most sentient life wiped out, Mith bombed back to before the Stone Age.

-4,000:
New gods arrive or appear on Mith.

-3,000:
Hriist and Salacar come into being on Mith. 
They begin breeding the Rachyth and Scatha species.

-2,000:
Nishi joins Salacar.

-1,000:
Scathae and Rachyth begin using tools.
Eoncos and Thaemas join Salacar.

-500:
Rachyth develop hieroglyphic writing. 
Rachyth begin to enslave the D�nac. 

-200:
Scathae develop alphabetic writing. 
Scathae begin to use bronze.

0: 
Salacar unites Imetric empire. Scathaese language spreads. 
Dessali and Hiothrex join the Imetrium. 

1:
Salacar declares year 1 of the United Imetrium. 

100:
First great war between Imetrium and Hriistites. 

400:
Rachyth develop green bronze. 

600:
Scathae discover iron. 



\chapter{The History of Mith - yet another version}
The history of the planet Mith has been shaped by several great wars, known as Cataclysms. There have been four such Cataclysms in recorded history, and there may have been more before that. Mithian history is divided into four Ages, with each Cataclysm marking the end of an Age.\footnote{Some people may feel that cataclysms are cheesy, because everyone and their mother has them in their fictional world. I don't care. My goal is not to be original and innovative at all costs, but simply to create a cool world.} 



\section{The Five Ages}
Here follows a description of the five Ages and the four Cataclysms. 

\subsection{The First Age: The Age of Beauty}
\label{Age of Beauty}
\index{Age of Beauty}
The first known Age is the Age of Beauty. 'Known' is a strong word, however; the Age ended many thousands of years ago, and no records of it survive. The Age is known only from ancient myth, and it is likely that most such myths, if not all, are pure fiction. 

The Age of Beauty is commonly believed to have spanned many thousands of years. Some believe that the Age was infinitely long, dating back to the mythical 'beginning of time', infinitely far back, and up to the Invasion. The latter is completely false, since the planet Mith has a finite age of around five billion\footnote{I will use a 'billion' to mean $10^9$, a thousand millions. As far as I understand, this is the default practice in English.} years, but the Mithians do not necessarily know that. 

According to myth, the Age of Beauty was a time of bliss and happiness. Everything on Mith was perfect and beautiful, all creatures lived in harmony, and war and evil were unknown. 

All this changed with the First Cataclysm, known as the Invasion. 

\subsection{The Second Age}
\subsubsection{The Invasion}
\label{Invasion}
\index{Invasion, the}
At some point, several thousand years ago, a race of alien creatures from another planet arrived at Mith and decided to colonize the planet. These creatures, now known as the Invaders, possessed vastly superior knowledge of technology and magic (probably TL10) and were able to subjugate or exterminate all native Mithians with ease.\footnote{At least, this is the story most often heard. Other myths claim that the native Mithians were advanced and able to defend themselves, and that Mith only succumbed after a long, devastating war.} After this, the Invaders reigned supreme and built great cities all over the planet. 

\subsubsection{The Age of Darkness}
\label{Age of Darkness}
\index{Age of Darkness}
The reign of the Invaders is now known as the Age of Darkness. Its length is not known. It is believed to have been at least 1000 years, but it may have been tens or even hundreds of thousands of years. All stories portray the Invaders as extremely monstrous and evil, keeping the native Mithians cruelly enslaved and oppressed. 

The leader of the Invaders was named Hyrakht\footnote{[HAJ-raakht]}\index{Hyrakht}. Another of the great Invader Lords was Ezzeyrath\footnote{[EZ-zej-raath]}\index{Ezzeyrath}. Some myths claim that each of the Invader Lords was the personification of some universal force of evil. Hyrakht is said to represent Oppression while Ezzeyrath embodied Envy. This interpretation is likely to be pure fiction, though. (Hyrakht and Ezzeyrath are the names that appear in the most common myths. They are also known by many other names.) 

The Invaders had several races of monstrous creatures for their servants, of which the best known are the Glekkyat\footnote{[GLEK-kj�t]} Worms\index{Glekkyat}. The origin of the Glekkyat is disputed. They may have been native Mithians that existed before the Invasion, they may have been bred and modified from native creatures, or they may have been alien creatures brought here by the Invaders. 

%Something about Hyrakht and Ezzeyrath
%Something about Tyrant Worms

\subsubsection{The Insurrection}
\label{Insurrection}
\index{Insurrection, the}
A Mithian resistance existed. They fought the Invaders and their minions using stolen Invader technology and magic and with the aid of mighty alien gods known as the Star-Gods\index{Star-Gods}. Still, the Mithians could do little against the Invaders. The tables did not turn before the Glekkyat started to rebel against their masters. Even so, the Insurrection might have failed, if not the Invader Lord Ezzeyrath had decided to betray his fellows and side with the rebels. 

The restult was a great and terrible war, lasting several decades if not centuries. During this war, the surface of Mith was devastated and entire continents were destroyed as others arose. Myths tell that the conflict culminated in a mighty duel between Hyrakht and Ezzeyrath, in which both slew each other. This duel is believed to be fictitous, but it is known that the Invaders were ultimately defeated. At the end, the Invaders managed to activate a mighty weapon that wrought immense destruction among their foes, killing millions if not billions. Especially the Glekkyat were hit hard by this final strike of the Invaders and were all but wiped out. Some kind of curse was cast upon the suriving Glekkyat, with the effect that could no longer endure the light of the Sun but were forced to flee underground. The vast majority of the Glekkyat were destroyed, but a few survived and may still exist in the Modern Age. The Invaders themselves are believed destroyed, though some tales claim that they (or at least some of them) were merely banished or imprisoned. 

Dragons are known to have been among the native Mithians that fought in the Insurrection. Fighting alongside them were several other races that are now extinc, wiped out in the war or its aftermath. Some tales also feature Vamons, Scathae and Humans in the Insurrection, whereas other myths claim that these races did not rise until millennia later. 



%But their reign did not last. Stories disagree whether the Resistance\index{Resistance, the} had survived unbroken since the Invasion or whether it arose at some later point (the latter is generally considered much more likely), but it is known that at some point, a native Mithian Resistance had gathered a great following. Armed with stolen Invader technology and magic and with the aid of mighty gods from the stars\footnote{Gods from the Stars}. The result was a great and terrible war, lasting several decades, now known as the Liberation. During the Liberation, the surface of Mith was devastated and entire continents were destroyed as others arose. 


\subsection{The Third Age}
\subsubsection{The Draconic Age}
\label{Draconic Age}
\index{Draconic Age}
The Insurrection left Mith bombed back to the Stone Age. New civilizations slowly rose from the ashes. Among the survivors of the war, the most powerful were the Dragons. Over the course of millennia, they slowly developed their own culture. The great leader of the Dragons during the Insurrection was Queen Tyndarea\footnote{[tin-DAA-ree-a]}\index{Tyndarea}, so in her honour, the entire race named themselves the Tyndarean\footnote{[tin-DAA-ree-an]}\index{Tyndarean Dragons} Dragons. 

Under the leadership of House Endarex\footnote{[en-DAA-reks]}\index{Endarex}, the descendants of Queen Tyndarea, they built a great empire, centered at the land of Nom, to the east of what is now Irokas. House Endarex became the first dynasty of Dragon Kings. The Dragons of Nom retained their belief in the Star-Gods who had aided them during the Insurrection. These are the gods still worshipped in Irokas today. 

Technology progressed only slowly. The Draconic Age lasted thousands of years, perhaps 20,000 years. There were several civil wars and several dynasties of Dragon Kings. 

\subsubsection{The Fall of Nom}
\label{Fall of Nom}
\index{Fall of Nom}
A great civil war. Dark, forbidden magic was used. It resulted in a magical catastrophe. Terrible monsters were summoned from the Baneworld of Erebos and other places. The land of Nom was laid waste. In the ensuing chaos and strife, House Irokas seized power and become the new Dragon King dynasty. The throne was moved to Mount Irokas and a new kingdom was named after them. But the kingdom of Irokas would not rise to the same power and glory as Nom, at least not for thousands of years. 



\section{The Age of Dragons}



\section{The Days of the Empire}
\label{Days of the Empire}
\index{Days of the Empire}
The new Dragon Kingdom of Irokas is greatly weakened. The Dragons are decimated, with more than 75\% of their number slain in the war. The remaining Draconic Houses are splintered. Many are not loyal to the King. Irokas therefore has little power. The Dragons only rule the land that is now known as the Kingdom of Irokas. On the rest of Mith, humanoid cultures rise to power. 

Because of their infertility and long lifespans, it takes many millennia for the Dragons to replenish their numbers. Humanoids, even Vaimons, reproduce faster. Therefore, they increase in numbers and power. Especially the Vaimons are powerful, because they discover Iquin-Nieur channeling magic. 

About 1000 years after the Fall of Nom, the Vaimon Empire is founded. It is ruled by a Vaimon Emperor from the Rainbow Throne. There are six Vaimon clans, each led from a magical throne of crystal: Diamond (Quaerin), Ruby, Sapphire, Emerald (Geican), Topaz (Redcor) and Onyx (Yrzhell). This Empire lasts for around 2000 years and reaches TL6. 

The Emperor is selected astrologically: %When the Emperor dies, Imperial astrologers search for signs of the new Emperor's coming
Imperial astrologers read the stars to ascertain the birth and location of an Imperial Scion. When a Scion is found, he or she (while a young infant) is taken to the Palace to be trained as the current Emperor's successor. 

\subsubsection{The Fall of the Empire}
\label{Fall of the Empire}
\index{Fall of the Empire}
The Fall of the Empire has to do with Bel'zhir\index{Bel'zhir}, the Dark Queen\index{Dark Queen, the}. She is a Vaimon woman of Clan Geican who becomes Empress, by perfectly normal means. She is evil and does not want to relinquish power. She uses dark magic to extend her life beyond its natural span. Perhaps she becomes a Nieur vampire? 

Anyway, she is ousted from power. But she managed to escape with her life. She plots to regain the throne. She allies herself with Clan Quaerin, in addition to her own Clan Geican. A great war ensues. 

Bel'zhir is slain, but her soul is not destroyed. Using powerful magic, her enemies successfully banish her soul from Mith (into the darkness of Nieur, as the Redcor tell the story), but they fail at destroying her. In time, she learns to communicate with people on Mith again and plots to return. 

Anyway, the defeat of the Queen does not end the war. Weapons of mass destruction are used, including TL6 atomic bombs and various magical doomsday devices. The Empire is destroyed and Mith bombed back several TLs. Thus end the Fourth Age. 

\subsection{The Fifth Age: The Modern Age}
\label{Modern Age}
\index{Modern Age}



\include{Imetrium}

\include{Nechsain}

\chapter{The Kinsari Civilization}

The Kinsari species is native to the isles of Eruil, where they maintain a thriving civilization.

\section{Dylur and the Shunuil}

Dylur is a planet orbiting the star Mith, about 150 light years from Myrrnoth. Dylur used to be a sentient creature and a nexus of great magical power. The magical forces created the occasional dimensional portal to open, and from time to time, the planet was visited by strange lifeforms. Dylur was the only sentient lifeform in the Mith system, but over the millennia, it gradually learned to communicate with the creatures that travelled to it.

Dylurian climate was not unlike that of Myrrnoth, and eventually, life evolved on the planet. Dylur desired to have intelligent creatures living on it, so it bent its efforts on evolving intelligent life. After many millions of years, it finally succeeded in breeding such a people: The bird species that became known as Shunuil. 

Dylur was a thoroughly gentle and benevolent creature, and so with the guidance of their parent \footnote{Dylur is a neuter creature, so I will use "`parent"' rather than "`father"' or "`mother"'} world and god, the Shunuil created a peaceful and utopian society. Though their civilization lived for millions of years, they never rose above stone age technology (TL0), for they had plenty of food and no natural enemies. The only science in which the Shunuil became advanced was magic. 

Eventually, however, the immense magical energies attracted enemies...

Magical catastrophe makes all eggs hatch into males. 
They create birthing flowers to reproduce. 
They discover Banes heading for them. 
Dylur commits suicide, transferring all the magical power he can into the ten Holders. 
Holders escape. 
Banes arrive and slaughter the remaining Shunuil. 
Today, dead Dylur is a Bane fortress. 
Holders travel the Universe for about 100 years.
They come to Myrrnoth, discover Kinsari. 
Spend 300 years caring for Kinsari, then flee. 
This was 500 years ago. 

\subsection{Dylur}

The magical energies permeating it gave Dylur divine powers equal to a divine rank of rank 10. 

Alignment: Neutral good. 
Divine rank: 10. 
Home: The mortal plane, in orbit around Mith. 
Portfolio: Magic, life. 

\subsection{The Shunuil species}

The Shunuil are birds, evolved from wading birds similar to storks or flamingos. Full grown, they stand about 150 cm tall but are very slender and delicate of build. They have large wings and are very capable (but not very fast) flyers. They have beautiful plumages that shine with all the colors of the rainbow. There is great variation in their colors, and no two individuals are alike, but red colors are the most common. They are carnivorous and eat fish. 

The wings are used only for flying, but the feet are dextrous and can be used as hands. Each foot has four digits. They cannot easily "`sit"' on their rump, and thus can use both hands only in flight. They have a good sense of balance, however, and can easily stand on one foot and use the other as a hand. 

Physically, the Shunuil are weak, having no natural weapons to speak of. They can bite with their beaks and claw with their talons, but neither are very effective weapons. Generally, a Shunuil will be no match for most creatures of similar size. On Dylur, however, every Shunuil was a mage and could, if necessary, wield the immense magical power of their parent world in defense. They may wield weapons with their feet, but on Dylur, all they had were sticks and stones.

Shunuil have high-pitched but melodious voices and may learn to speak the languages of most creatures on Myrrnoth. On Dylur, they all spoke a common language, now known as Dylurian. At TL0, their average lifespan was 40 years. The Dylurian population never exceeded five million. 

Now, the Shunuil are all but extinct. Apart from the ten holders, there may be small pockets of them scattered across the Universe, but none are known to exist. 




\chapter{Cosmology}

\section{Chaos and Cosmos}

\begin{quote}
A fundamental axiom of classic Dessalic metaphysics is the idea that the Universe is driven by two fundamental forces, Chaos and Cosmos. Chaos is the force of disorder, conflict and destruction, whereas Cosmos is the force of order, harmony and creation. 

According to ancient, apocryphal legends of unknown origin, in the beginning there was only Chaos. From the random twisting of Chaos, Cosmos was born. According to these legends, the Cosmos, and the Universe we know, is a small bubble floating in an infinite ocean of Chaos. 

The source of these stories cannot be ascertained, but they cannot be litterally true, for our research indicates that the very concepts of time and space are Cosmic in nature and have no meaning or existence in a Universe of pure Chaos. Thus, any talk of a 'time before the Cosmos' or a 'place beyond the Cosmos' is nonsensical. 
\end{quote}

%The stories also tell of a titanic war, as the gods of Cosmos clashed with the monstrous beings of Chaos for control of the Universe. 

\section{Dimensions}
The Universe is divided into a number of parallel universes, or \emph{dimensions}. The dimensions are coexistent, and every geographical location in one dimension corresponds in some sense to an equivalent location in all neighbouring dimensions. Also, many natural and supernatural forces reach into and affect objects in other dimensions. 

\subsection{Gravity}
As an example, the force of gravity reaches into neighbouring dimensions in almost unchanged form. For this reason, large bodies such as stars and planets will almost always correspond between dimensions. So where the planet Myrrnoth lies in the Myrrnoth dimension, there lies a planet of identical mass in the dimension of Hell, following the same orbit around the Black Sun, which is the Hellish equivalent of the Myrrnoth Sun. 

\subsection{Proximity}
Some dimensions are easily reached, while others can only be reached using very powerful magic or under special circumstances. This is usually approximately reciprocal, so if it is easy to get from A to B, it is just as easy to get from B to A. 

If two dimensions are easily reachable from each other, they are considered 'close' or 'neighbours'. A dimension that cannot be easily reached is considered 'distant'. 

Of course, 'easy' is relative. Unless you are near a portal, it is not possible to travel between dimensions without the use of powerful magic (or, at higher TLs, science). 

\subsection{Material and Ethereal Planes}
There are many different dimensions in the universe. The most well-known and well-traveled dimension can be divided into material and ethereal planes. 

The Myrrnoth dimension is a material plane, as are Hell and Caelum. 
Do material planes have any identifying characteristics? 

As far as we know, it seems that every material plane has a corresponding ethereal plane. The ethereal plane corresponding to the Myrrnoth plane is called Limbo. 

\subsection{Cosmic and Chaotic Dimensions}
Each dimension maintains a certain balance between Cosmic and Chaotic forces. Some dimensions are more Chaotic, however, while others are more Cosmic. The dimensions seem to be arranged in a "chain", each dimension bordering onto a Chaotic neighbour and a Cosmic neighbour. 


\chapter{\Tuat{}}

\section{The Dimension of \Tuat{}}

'\Tuat{}' is a parallel dimension. Bla bla.

The planet \Tuat{} is tide-locked: It cannot rotate in its orbit, but has the same side turned towards the Black Sun at all times. This divides the planet into a bright, hot side and a dark, cold side. These are named Inferno and Stygia. 



\sectionn{Inferno}
Inferno is the hot side of \Tuat{}. Pits and chasms of fire scar the land, sulphur rains from the sky, and the terrible Black Sun gazes ever down upon the land. The atmosphere of \Tuat{} lacks the stuff which make the sky blue, so here the heavens are black. 

\subsectionthe{Pyre}
The Pyre is a great column of infernal fire. It marks the exact centre of Inferno, the one spot on \Tuat{} where the Black Sun stands at the zenith. 

The Pyre is kilometers wide and over 20 km tall.\footnote{This is more than twice the height of the tallest mountains on Earth in RL.} It is a matrix of tremendous Chaotic power. It draws souls in and gives birth to demons. Balrogs are created at the Pyre. 

Some claim that the Pyre is sentient. Some claim it is the body of Astaroth, the Supreme Devil.

\subsectionthe{Rivers of Inferno}
Inferno is crossed by a number of great rivers. 

\subsubsectionn{Phlegethon}
The river of fire. Flows out Eastwards from the Pyre, splitting the continent of Eastern Inferno into two. 

\subsubsectionn{Styx}
\subsubsectionn{Acheron}
\subsubsectionn{Lethe}

\subsectionn{Astaroth}
Legends and superstition speak of Astaroth an ancient creature of evil, immense in power, who is called the the Supreme Devil, the True Overlord of Inferno and all that is evil. 

It is unknown if Astaroth exists, but he has many cults scattered over \Tuat{} and other places, including Mith. More than once, a proclaimed Astaroth has proven fake, a lesser demon lord using the name for publicity. 

Some legends claim that Astaroth is sleeping (perhaps at the heart of the Pyre). Others say that he was cast down and imprisoned deepest down in Tartaros. Yet others believe that Astaroth left \Tuat{} to fight a great foe. All agree that he will some day return to claim what is rightfully his. 

Myths do not agree whether Astaroth ruled all of \Tuat{} or only Inferno. 

It is said that Astaroth created the Pyre and all demons. Some even say that he created all evil\footnote{This cannot possibly be true.}. 



\subsectionthe{Abyss}
A great chasm that runs between the North and South Poles of \Tuat{}\footnote{\Tuat{} does not rotate, so what are the poles? Are they the magnetic poles, or perpendicular to the orbit, or coincident with the rotational poles of Mith}, though the Pyre. Connects Inferno with Tartaros. There are kingdoms of devils in the Abyss. 



\subsectionn{Avernus}
The kingdom of Rauthor, bordered by the Abyss and the river Acheron. Engaged in a constand war with the kingdom of Candrazor. 

\subsection{The Kingdom of Candrazor}
The kingdom of Candrazor, bordered by the rivers Acheron and Styx. 

\subsection{The Kingdom of Tez'rik'rik}
The kingdom of Tez'rik'rik, bordered by the river Styx and the Abyss. 



\sectionn{Stygia}
The cold side of \Tuat{}. 

\subsectionn{Caina}
The coldest place in all of Stygia, centered around the Spire, and icy mountain directly opposite the Pyre.\footnote{Maybe 'Spire' sounds too much like 'Pyre'. Choose a different name?} 

Caina is ruled by the dark and terrible Triumvirate of Fell Angels.

\subsectionn{Cocytus}
Is this a part of Stygia or Tartaros? Maybe let Cocytus be a part of Stygia and let Nessus be part of Tartaros. 

\subsectionn{Cold Wraiths}
Great demons who live in Stygia. Comparable to Balrogs in power. 



\sectionthe{Twilight Zone}
The borderlands between Stygia and Inferno. Great clouds of ash block out the fell light of the Black Sun but provide sufficient greenhouse effect to drive away the ice of Stygia. 



\sectionn{Tartaros}
The underworld of \Tuat{}, a dark world of caverns and labyrinths. Some gigantic and horrible creatures dwell here, including the Dholes (enormous worms, as in HP Lovecraft - The Dream-Quest of Unknown Kadath). 



\sectionn{Pandaemonium}
The next lower dimension, one step more Chaotic than \Tuat{}. A twisted, nightmarish place of howling winds and few living creatures. On a rare occasion, \Tuat{} has been invaded by Pandaemoniac creatures. 

\subsection{The Lost Realm of Pluton}
Once a great kingdom in Inferno, Pluton was invaded by monsters from Pandaemonium a few centuries ago. Stories tell of tremendous, blasting winds laying waste to cities, and of titanic creatures that sucked hundreds of demons into their great mouths and devoured them whole. Even mighty devils such as Balrogs were no match for the terrible creatures. 

Today, Pluton is a deserted wasteland. It is unknown whether any of the Pandaemoniacs remain, but it is known that echoes of their Chaotic magic still lingers. Mysterious things happen, and demons (even armies of demons) have been known to vanish without a trace... perhaps swept away to nightmarish Pandaemonium. 

\section{Denizens of \Tuat{}}

\subsection{Balrog}
Singular Balrog, plural Balrogs. 

Some of the mightiest creatures in \Tuat{}. Great demons of fire and darkness. 

\subsubsection{Life cycle}
Balrogs are born from the Chaotic energies of the Pyre. They are asexual and cannot reproduce on their own. New Balrogs are born only in the Pyre. At higher TLs, it may become possible to channel the Chaotic energies and control the spawning of demons. 

Balrogs are immortal. They do not die of old age and can reincarnate themselves if their bodies are destroyed. Powerful magic is required to destroy them permanently. 

Balrogs do not automatically change as they age. They grow in power by experience or by devouring other demons.

\subsubsection{Appearance}
A great humanoid made of fire solid and darkness, with great wings\footnote{Regarless of whether real Balrogs have wings or not, my Balrogs do. And they can fly.} and horns. Similar to the Balrog in the Lord of the Rings movies by Peter Jackson. 

An average Balrog is 6 meters tall and massively built. They can change their size at will, shrinking to half their full size. Balrogs usually maintain their full size most of the time, for the frightening effect. They can also alter their body form at will, but only slightly (always fitting the description above). 

\subsubsection{Power}
Even the weakest newborn Balrog is a lesser god in power. The mightiest Balrog alive is Rauthor, who is one of the greatest gods on Mith. 



\subsection{Lemures}
Horrid creatures. Damned souls trapped in \Tuat{} most often take the form of Lemures. Semi-amorphous mounds of flesh, horrid caricatures of the forms they had in their previous lives. Each time a Lemure dies, its body mutates further into chaos. 

Lemures are mindless and know only two feelings: Suffering and \emph{extreme} suffering. The latter is their default state, and if they discover that a certain action lessens their agony, they will do that. Demonic spells exploit this to control the Lemures and form them into armies. 



\subsection{Imps}
Lesser demons. Fully intelligent and quite evil. Mortal, few supernatural powers. Exist in many types. 



\subsection{Phobian}
Singular Phobian, plural Phobians. Crab-like, intelligent creatures. Demons, but benevolent. Believe in the freedom of the individual. 

Not so powerful and often enslaved by greater demons. Their chaotic, anarchistic nature makes it hard for them to band together in kingdoms for protection, but there exist sanctuaries of peaceful demons, mostly under the protection of benevolent gods. 



\section{Gods and Characters of \Tuat{}}

\subsectionn{Rauthor, King of Balrogs}
\label{Rauthor}
Rauthor is the mightiest Balrog alive and monarch of the nation of Avernus. 




\chapter{Miscellaneous stuff}
Notes, fragments and ideas. 

\section{Creatures}
\subsection{Kraken}

Ocean-dwelling alien creatures of divine power. They are native to Myrrnoth and 
their race has existed for many millions of years. They look like enormous 
squid. Cthulhu elements. Remain dormany for millions of years due to 
astrological issues. 

The greatest of them, Moloch, is one of the mightiest gods on Myrrnoth. He may 
be the oldest Kraken alive... their father? Salacar has seen Moloch in a 
divining vision and fears him, but knows little about him. (Does Salacar know 
that Moloch is a Kraken?)

Kraken are asexual and reproduce by parthenogenesis. They are immortal and very 
rarely reproduce. The youngest Kraken alive is many million years old. 

Some marine creatures (y'know, deep ones) worship them. Other sea-dwelling 
creatures (and gods?) fear them. They also have cults on dry land. 

The Kraken have formidable powers of hiding/obscuring (compare to the ink cloud 
of a squid). 

Kraken may have waged war against Invaders and/or Tyrant Worms. 

\subsection{Marine civilizations}
\begin{itemize}
\item Crustacean men. Terraxuil?
\item Shark men and/or ray men (ixitxachitl/ixtlilxochitl). 
\item Some kind of deep ones who worship the Kraken. 
\item Ancient, fallen civilization - Lemuria?
\item Ancient, fallen cities - R'lyeh?
\end{itemize}

\subsection{Caederyn}
Enormous predatory fish. Heavily armored and with extreme regenerative powers. 
Caederyn are some of the most dangerous creatures of the seas. It is unknown if 
they are native to Myrrnoth or alien, but no related species are known. It is 
also unknown if they reproduce. No very small specimens nor eggs have been 
seen, and if two Caederyn meet, they usually fight (sometimes to the death). 
Fiercely territorial. Caederyn have a voracious hunger and are extremely 
aggressive. They are very stupid, but immune to all forms of mind control, even 
that of gods. They are virtually fearless and usually fight to the death. 
Global population is a few thousand. Full-grown Caederyn are around 30 meters 
long and very massive. Behemoths of 40 meters have been seen. 

\subsection{Dragons}
Dragons are great, reptillian creatures. All dragons are intelligent and most 
have an affinity for magic. Many species have a breath weapon. Some have other 
supernatural powers. Many species can fly. Life spans vary between species and 
races, typically 1000-2000 years. A few species live up to 3000 years. 

Dragons belong to the group Archosauria and are related to reptiles, dinosaurs 
and birds. They are quite genetically unstable and tend to mutate a lot, which 
explains why there are so many species and races. There are maybe around 25 
Dragon races on Myrrnoth, each with an average of a few thousand individuals. 
The most common are Fire Dragons, Ice Dragons and Sea Dragons. 

\begin{itemize}
\item Fire Dragon. 
\item Ice Dragon. 
\item Several species of sea dragons (some of them sea serpents?). 
\item Hydrae (heads, regeneration, very mutable). 
\item Maybe the Suchrevian yellow dragon should be a Hydra? 
\item Leviathan Dragon. Ocean dwelling, largest species on Myrrnoth, most 
long-lived. One individual female is 2500 years old and the oldest and 
mightiest dragon known on Myrrnoth. Usually benevolent. They are some of the 
few creatures willing to take on Caederyn. 
\item Eagle Dragon. Beatiful and mighty feathered dragon. Leading figures in 
the war against the Invaders, now extinct on Myrrnoth. 
\end{itemize}

\subsection{Drakes}
Drakes are creatures related to and resembling dragons, but non-intelligent. 
(Not all dragon-kin are drakes, tho.) There exist many species and races. Some 
are quite small. One species is the Solar Drake, connected to Salacar and used 
as beasts of war in the armies of the Imetrium. 

\subsection{Jinn}
Desert-dwelling wind demons, native to Sulchrev desert. Also called Djinn. Used 
to rule the desert and its inhabitants, but overthrown by Rissit. Now mortal 
enemies of the Nechsaites. Jinn are mid-powered demons. They can assume 
incorporeal, invisible form (where they can affect only the minds of others, 
and only have their own minds affected). 

They can possess other creatures. Possessed creatures may spawn Half-Jinn. How 
do the Jinn reproduce among themselves? 

Some of the Jinn follow the goddess Es'phet. 

\subsection{Vaim�n}
Ancient civilization, now fallen. 

\subsection{Eta}
Rat-men. Small, but common. The name is Scathaese (singular Eta, plural Etae). 
Very widespead and numerous, but often subjugated. 

\subsection{Nur}
Doglike humanoids. Native to the land of Caeoll. The name is Scathaese 
(singular Nur, plural Nuri). Also live many other places, but many of them are 
Imetrics. 

\subsection{Tchacolda}
Centaur-like creatures. Body of an antelope, humanoid torso and antelope head 
with horns. Do they have hooves or feet? 

Native to the cold tundras of the Northern Kingdoms to the North of Caeoll. 
Aggressive warrior people. 

\subsection{Meccara}

\subsection{Fittera}

\section{Gods}

\subsection{Sherioch}
Sea-dwelling war god. The brother of Eoncos. Sherioch respects his brother and 
will sometimes come to his aid. 

He is a chaotic creature and believes that conflict and combat is the natural 
order of the Universe. He has a certain code of honor, however. 

Sheriochites are warriors, mercenaries, pirates and such. 

His avatar swims the seas of Myrrnoth in the form of a red shark. He hunts 
Caederyn for the challenge. Legends say that Sherioch will reward any heroes 
who slay his avatar in fair combat. (This has supposedly been done a few times 
in history.)

\subsection{Rekhtet}
The Queen of Swarming Insects. An insect goddess who serves Nechsain. Commands 
the insects of the desert to attack the foes of Nechsain. Also guards the 
Pyramids... or what?

\subsection{Es'phet and Tchesef}
Tchesef is an earth god and Es'phet is a wind goddess. They used to be lovers. 

Long ago, they served the Imetrium, but they betrayed them and joined Rissit. 
They served him for many years (decades? centuries?). Es'phet took a liking to 
the Jinn and again betrayed her master to side with them. 

After Es'phet's betrayal, Tchesef and other gods were compelled to take 
powerful oaths of loyalty to Nehsain.

What are Tchesef's own feelings?

Tchesef has power over the earth and the sand. Es'phet has power over winds and 
storms (but not rain or thunder). 

\subsection{Daxian}
Weather god, the Lord of Wind, Rain and Thunder. Worshipped in the Northern 
Kingdoms. 

\subsection{Kurur}
Goddess of something. Worshipped in the Northern Kingdoms. 




\end{document}
