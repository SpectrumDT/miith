\chapter{Magic on Mith}

\sectionn{Schools of magic}
Magic exists in the Mith universe and is an important part of it. Every self-respecting society has mages of some kind. But magic is not just magic. There are many different approaches to magic. Pretty much every civilization has its own magic theory, its own tradition of magic. 

But why does this diversity persist? Why do mages from all over the world not come together, exchange knowledge, do some research and find out which theory of magic is correct? Well, for several reasons: 

\begin{enumerate}
	\item Science is difficult. Each theory has its pros and cons, and it is not always possible to `combine' them. 
	\item Pride. Often, scientists are very proud of their own theory, the one they were brought up with or may have helped develop. People want their own view to be correct. This makes them inflexible, irrational and unwilling to compromise and acknowledge possible flaws in their theory. 
	\item Distrust. Mith is not a friendly place where everyone gets along. Wars are fought between nations every day, and even in peace time there is the threat of war, old hatreds, racism, feelings of cultural superiority, and the fear of the unknown, all of which inhibit communication and the free exchange of scientific knowledge. 
	\item Secrecy. Apart from the irrational `distrust' above, many cultures have dreams of world domination. These are unwilling to share their knowledge, because knowledge is power, and they do not want to share power. 
	\item Religion. Some magic is directly connected to some religion and weaves ties between the spellcaster and the gods of that pantheon. Typically, no one wants to be dependent upon the gods of another religion. And even in the cases where the magic is directly connected to specific gods, the theory is often intertwined with the dogmas and world view of some religion, which nonbelievers will be hesitant to accept. 
	\item Complexity. The magical Universe is vast, dark and mysterious. At TL3, no civilization is close to understanding more than the tiniest fraction of it. The truth of how magic works in the Universe is perhaps unknowable, and at any rate far too complex to have been even glimpsed. But magic is a strange thing and manifests itself in many forms. 
	\item Fear. This is a consequence of the fear of the unknown. The Universe of magic is vast and dark, and much magic will be horrible and frightening to the uninitiated. There is, of course, the very real risk that a spell or ritual might go horribly wrong and cause an explosion, transform the caster into an undead monster or accidentally conjure some demon from an alien world, but even apart from that, the supernatural tends to instill people with irrational fear and loathing unless they are very strong-willed or have been brought up to accept it. Therefore, most people will accept the magic of their own culture, but view most foreign magic as something foul and evil. 
\end{enumerate}

The result of all these factors is that every culture has its own magic, its own methods for casting spells. Schools of magic on Mith include: 

\bi
  \item Chaos magic, also called \draconic{} magic (section \ref{Draconic magic}). 
  \item The Vaimon theory of Iquin and Nieur (section \ref{Vaimon magic}). 
  \item The Rissitic theory of the Body, Spirit and Shadow Worlds (section \ref{Rissitic magic}). 
  \item The Imetric theory of Mana (section \ref{Imetric magic}). 
  \item Shamanistic magic (section \ref{Shamanistic magic}).
  \item True \bane{} magic (not yet described).
\ei



\section{The source of magic}
All magic works by reaching out with the mind and touching the world in ways not otherwise possible. 

In most cases, the mage does not perform all the effects of magic himself. Rather, with his spell he calls upon some external force or creature, binds it to his will and uses its power to affect the world. The Vaimons use the \Sephiroth{} and \Kliffoth{} for this purpose while Chaos mages call upon various \daemons{}. Some of these forces are mindless or mostly mindless, while others are living creatures with more or less inteligence.

These external forces are normally unlimited (although they might be destroyed). But when casting magic, the mage must use some energy of his own to control the forces of magic, and a mortal mage's supply of energy is finite. As the mage expends energy, he will grow tired and exhausted and need to rest. 

Furthermore, in order to call upon the forces of the supernatural, the mage must spend some time attuning himself to these forces. This takes the form of meditation or prayer, where the mage contacts the occult forces and builds a connection to them. As he casts magic, his connection to the forces will become worn and decay, so every mage must periodically meditate to re-attune himself to the sources of his magic. 



\section{Learning magic}
It is sometimes said that to become a mage one must possess some unique, inborn talent or gift, but this is actually not true. The only character traits needed to learn magic are the intelligence to understand the theory of magic and the strength of will to bind the forces of the occult to your bidding. 

In addition, one must have access to teaching materials and a teacher. It is possible to learn magic by self-study from books and scrolls alone, but this is a slow, difficult and dangerous process --- magic is dangerous work, and it is easy to hurt yourself if you don't know what you're doing. Similarly, it is possible to learn magic directly from a teacher with no written materials available --- there are mages among pre-literate cultures, and their magic can be potent, if primitive. But learning from a teacher with books and scrolls available is the best and most common method. (Note that in order to use books one must of course be able to read, and very few people on Mith are literate.) 

Learning magic is a long and difficult process. It usually takes years to learn to cast but the simplest spells and a decade or more to become a competent mage. Like all skills, magic is most easily learned at an early age. Many mages --- and most of the skilled ones --- were apprenticed as children, before the age of ten, and studied the art for at least fifteen years. 

\subsection{Losing one's magical ability}
There are only a few things that can permanently weaken or destroy a mage's ability to cast magic. These include brain damage, amnesia, insanity/mental illness and extreme shock/mental trauma. 

Most mages also use physical gestures to cast magic, so a mage who is suddenly maimed will find it harder to cast his spells. This is not an absolute hindrance, however, and in most cases you can learn to cast your spells anyway without relying on physical gestures. (There are exceptions, though. Some spells require a physical ritual, such as the drawing of occult symbols, and such spells cannot be cast if the ritual cannot be completed. Still, sometimes it is possible to have an assistant perform these parts of the ritual instead.)



\section{Casting spells}
The exact process of casting a spell varies between the different scools of magic. 

Vaimon magic uses mostly fast, simple `point-and-click' spells: You simply invoke the names of one or more \Archons{}, mentally visualize the effect you want, and perform a simple gesture such as pointing your finger. %Stricly speaking, neither gesture nor speech are actually needed. The name is a way of contacting the \Archon{}, which is not necessary if you are experienced and finely attuned to Iquin or Nieur. (Or is it? Maybe the invocation is always necessary.) 
The gesture is not really needed. It is merely a psychological aid to help shape the spell, which is unneeded if you have fine control over the \Archons{}. 
The invocation of the name may or may not be needed, I'm not sure. If the spell is difficult or requires much power, you may have to keep chanting the name(s) over and over. 
Vaimons can also combine into a circle to perform ritual magic. Such a circle is much less flexible and slower to react than a single caster, but can cast more powerful and complex magic. 

Chaos magic, as used by the \dragons{} and others, relies much more on words. A spell is a sentence in Kingstongue which invokes the \daemon{} or \daemons{} you want and (superficially) describes the effect. Chaos magic also uses occult symbols. For simple spells, the symbol is usually already encarved on some item, such as a staff or amulet, or as a tattoo, and the item is used in the spellcasting. In more complex (and longer) spells, symbols are drawn while casting the spell, on the ground, on paper or on one's own body. Chaos magic spells are usually cast by a single caster, although they may be long, requiring a complex ritual. 

Rissitic magic (which is a further development on Chaos magic) likewise uses long incantations, and also uses symbols more extensively. Most spells require an occult Glyph to be drawn. Glyphs can be drawn in the air with your fingers, or (better) with a special stylus, or they can be drawn on a surface. Rissitic magic focuses more on elaborate but powerful arcane rituals rather than quick spells. Quick spells exist, but they are difficult and their casting consumes much energy from the caster's body. Quick spells are used especially by the \Ashenoch, who have plenty of physical strength to draw upon. But most spells are cast as rituals taking rather a long time and requiring several mages, usually a lead mage and some assistants. 

\subsection{Laboratory magic}
Some magic is slow, taking hours or days, and may even require all sorts of material components. Such magic is best performed in a laboratory of some sort, and will here be called `laboratory magic'. Laboratory magic typically includes all spells that are to have a permanent effect, including the creation of enchanted items. Major summonings are also often laboratory spells, as are major shape-changing spells (from \dragon{} to humanoid form). Major healings (for deep injuries or serious diseases) are also laboratory work. 

Laboratory work sometimes requires ingredients: Exotic minerals, plants or parts of creatures, or even living creatures. 

\subsection{Spellcasting aids}
Most magic, except laboratory magic, requires to items to be cast. But you can have items that help you cast spells. Typical spellcasting aids are staves, wands, rings and amulets. Such items should be made from special materials (exotic sorts of wood, occult metals or body parts of creatures) and enchanted with arcane sigils. Then they will help the mage channel his magical energy, so that he can cast stronger spells and expend less energy. It might also make it easier to contact the occult forces, so spells can be cast faster and with less risk of failure. 

There are also casting aids that are consumed and destroyed when the spell is cast. The most common example is the various herbs used in healing and in minor blessings and curses. (It should be noted, however, that sometimes that which is called `herbal magic' is not magic at all, but simply natural medicine.) In most cases, these components are not strictly needed but serve to boost the spell's effect --- or, in some cases, the spell boosts the herbs' natural effect. 



\section{The power of a mage}
Some mages are more powerful than others. There are a number of factors determining the magnitude of effects that a mage can accomplish with magic. These include: 

\begin{itemize}
	\item Knowledge of spells. Working magic is complex work, and knowing a clever spell will let you accomplish your work much faster, more effectively and with less expenditure of energy than a simple, naive spell. 
	\item Willpower. Strength of will alone is important, as it lets you bind more power to your will. This one is open-ended and constrained solely by the will of the mage, but even so, even with infinite willpower there are limits to what you can do.
	\item Innate power of the soul. This is a fuzzy concept as of yet, but the idea is that some creatures, even if their willpower is the same, have more mental `muscle' and can handle more magic power than others. 
\end{itemize}

As a mage casts spells, he will expend energy and become tired. This is similar to physical fatigue, but not quite the same. 

Perhaps to regain your magical power you need to meditate and attune yourself to the mystic forces that you channel. In a religiously oriented magic theory, this meditation will take the form of prayer - to the \Sephiroth{} or to Rissit Nechsain or whoever. 



\section{Side effects of magic}
One problem with magic is that it is addictive. Once you begin casting magic, you will have an urge to keep doing it, casting more spells and drawing more magical power. This addiction in itself is perhaps not so serious, but magic has other effects which are amplified by this compulsion to keep using it. 

Chaos tends to twist the mind into something more chaotic and bestial. It brings forth primal urges of lust, greed and aggression and encourages the mage to act on these instincts. As a result, a Chaos mage tends to become warped and mad with time. Most \dragons{} exhibit these traits due to practicing Chaos magic; indeed, these traits are so pervasive among their race that it is considered their natural state (although a \dragon{} might turn out differently if it refrained from using Chaos magic all the time), and people whose minds are twisted through Chaos magic are said to acquire \draconic{} minds. 

Nieur has an effect similar to that of Chaos. 

Iquin, on the other hand, brainwashes the channeller with the \quo{virtue} that the \Sephirah{} in question represents. The more the mage channels the \Sephirah{}, the more its virtue will be ingrained in his mind. So people who channel Iquin a lot tend to become zealously devoted to the Iquinian ideology (if not to the Church itself). 

Imetric magic has an effect similar to that of Iquin, except that it indoctrinates you into Imetric virtues, and not quite as strongly. 

All these effects fall are inflicted most strongly upon those mages of weaker mind. The more strong-willed mages are able to resist it better and tend to keep their personality more intact in the long run. 

If you practice two or more different kinds of magic, each with different side effects, these side effects may pull in different directions. This is the case with a Vaimon who uses both Iquin and Nieur: Iquin works to make him principled and virtuous, whereas Nieur works to make him savage and uninhibited. In such cases, these opposing forces may sometimes balance each other out, so that the mage can remain stable. On the other hand, they may also cause the mage to go schizophrenic or otherwise mad (perhaps even developing a genuine split personality) from the strain of having his mind pulled in two opposite directions. 

Iquin has the special side effect of extending the lifespan of the Vaimons who channel it. Vaimons who channel Iquin a lot tend to live up to twice as long as normal for their race and standard of living. No other school of magic has this as an intrinsic effect, but most schools have some spells that may be purposely used in an effort to extend one's life. These spells may then it turn have side effects of their own. (An example of this is life drain, which gradually turns the caster into an undead Reaver.) 



\section{Divine magic}
Divine magic is the art of channelling the power of a god or godlike being in order to cast magic. I am not sure how this works. 

Perhaps it is just like other magic. Meaning: Perhaps all magic is really like this, reaching out to draw upon other creatures' power and compel them to do stuff. 

Perhaps it is different from other magic somehow. 



%\section{Psionics}



\section{Effects of magic}
\subsection{Effects of different schools of magic}
Iquin magic is elemental-based, so its effects are things that use the four elements. Iquin mages can hurl fire, ice and lightning at their enemies and move earth, air and water around. 

Imetric magic, in effect, might resemble psionics more than traditional magic. It has a lot of telekinesis, telepathy and clairsentience. It is meant to look less `occult' and more, I dunno, more `clean' and `nice' than most schools of magic. 

Rissitic Body magic directly affects bodies. It may grant strength or cause weakness or paralysis; it may heal or harm. `Harm' comes in the form of broken bones, wounds springing open, internal haemorrhaging or even the whole body bursting apart. Spirit magic affects the mind. This includes mind control, mind reading and divination spells. 

Chaos magic, Vaimon Nieur magic and Rissitic Shadow magic are the most diverse and versatile forms of magic. They can have all sorts of effects. Often they have to do with opening portals to the Beyond and possibly summoning creatures to your aid. 



\subsection{Magical healing}
\subsubsection{Naturalist healing}
Magical healing might work by touching the recipient's Chaos body, providing it with some energy and helping it to heal itself. In such cases, a portion of the energy comes from the healer and a portion comes from the recipient's own body. Such healing is rather easy, but crude, and cannot heal difficult injuries or badass diseases --- things the body couldn't heal on its own. The healing might be assisted by herbs.

The Vaimons and most primitive cultures use naturalist healing. Most Chaos mages also use naturalist healing. This is because the theory of Chaos magic was developed by \dragons{}, and \dragons{} have formidable natural healing capabilities, so this kind of healing is very effective on them. For the \draconic{} Chaos mages, healing their servitors (\scathae{} or others) was never a high priority...

\subsubsection{Surgical healing}
Another approach is to use magic like surgery. Instead of relying on the body's natural ability to heal itself, the mage might rely on his own skill and knowledge to directly control the healing process. This requires the mage to know a lot about anatomy and surgery, so it is difficult to learn. The process itself is also difficult and consumes time and energy from the caster. The upside is that this healing is more sophisticated and can heal more grievous wounds and diseases. 

This kind of healing might also involve herbs or drugs, and the mage/doctor might use mundane surgical tools in conjunction with his spells. 

The Imetrians and Rissitics use surgical healing. Recently, some Vaimons in Geica have begun experimenting with it. 

\subsection{Magic in war}
Magic has a number of uses on the battlefield. 

A Vaimon can call down the elements to kill his opponents. He can also fly over walls or use Earth magic to break the walls. 

A Rissitic or Chaos mage can summon \daemons{} or other creatures to his aid. 

\subsection{Magical transportation}
You can use magic for travelling purposes in a number of ways. 

\subsubsection{Super-speed}
There are spells that can be cast on a creature to temporarily (or even permanently) increase its speed or endurance. Such spells may come with a side effect: After the effect wears off, the subject might experience extreme fatigue and/or weakness or even permanent damage, aging or death. 

\subsubsection{Flight}
Vaimon magic can cause the mage to fly, by invoking the \Sephirah{} \Atzirah{}. 

The Imetrians have a similar spell. It gives better control and maneuverability but less speed, and it is more difficult to learn. 

Rissitic and Chaos magic have no flying spells. (The \dragons{} can fly on their own, so they have no developed flying spells.) 

\subsubsection{Summoning}
A Rissitic mage or Chaos mage might be able to conjure a creature and coerce it to carry him as a rider. Such a creature might be able to run, swim, fly or perhaps even teleport. 

\subsubsection{Teleportation}
Chaos magic spells exist that teleport the mage from one teleporter to another. There are several of these teleporters scattered across the world, but the secrets of their making have been lost. 

\subsection{Telepathy and empathy}
Telepathy is a kind of magic. It allows you to communicate silently, mind-to-mind at a distance. 

The distance is still limited. The best telepaths can manage a range of a few kilometres. 

If the people communicating have no shared language, only pictures, sensations and vague emotions can be transmitted. This is slow and difficult to make sense of. 

Sending thoughts is easy. Reading them is much harder. Even a skilled telepath can only read surface thoughts and feelings, and only vaguely. Smooth communication requires that both parts be telepaths. 

The Imetrians have the most advanced telepathy. The \banes{} also have it, but they don't teach it to any but their trusted servitors. 



\section{Suppressing magic}
It is possible to suppress magic. Suppression works on a person or an area and prevents the person, or everyone in the area, from casting magic of one or more types. 

\subsection{Suppression spells}
One way is by casting a suppression spell on one or more persons, thus suppressing any spellcasting on their part. There are two problem with such spells. One problem with such spells is that they must be continuously maintained by one or more mages. Another problem is that the suppression can be broken if the subject is a stronger or more crafty mage than the one holding the spell. 

\subsection{Suppression drugs}
Another way is using drugs. The most well-known of such are the toadstools called `witchbane'. When eaten, drunk or injected into the veins, after being properly prepared and distilled (into a soup-like fluid), witchbane affects the victim's mind, making it difficult to think clearly and making spellcasting all but impossible. (Even if the afflicted victim manages to access his magical power, he is unlikely to be able to cast a coherent spell, but may be able to unleash some random magical mayhem.) 

The advantages of witchbane is that the toadstools are rather widespread and the concoction is easy to make. The drawback (if the victim must be kept alive and mostly unharmed) is that ingesting it in large quantities will cause permanent brain damage, potentially resulting in insanity, loss of intelligence or loss of motor skills (stuttering, uncontrollable shaking or paralysis). (At this time, no magical or mundane cures for any of these effects are known.) A victim can be pacified for at most 4-6 hours without risking serious permanent damage. (Of course, all these effects will affect anyone, whether mage or not.)

There are several species of witchbane toadstools, which can be more or less potent and more or less dangerous. 

\subsection{Inhibitors}
%A more potent method is using special items, called inhibitors. 
An inhibitor is a specially crafted and enchanted item made from exotic crystals and precious metals that suppresses spellcasting in an area. Some inhibitors cover only a small area, such as one person. These may take the form of collars or shackles. Others are strong enough to cover an entire room. These may have any form. 

The main advantage of inhibitors is that they are stable. An inhibitor functions constantly, with no operator necessary, until it is worn out. Typically they last many years. 

The disadvantage of inhibitors is that they are prohibitively expensive. Creating them is very difficult and time-consuming laboratory work, and the ingredients are rare and exotic. Another problem is that the inhibitor can never be turned off, which is inconvenient to the user. 

Inhibitors can be destroyed with physical force, but they are always enchanted to make them extremely durable, to prevent a prisoner from simply smashing them.



\section{Natural magic}
`Natural magic' is a kind of magic that is not cast as spells as such, but is more of a natural ability. Such magic cannot be suppressed. \dragon{} flight is an example of it. 



\section{Magical items}
Items can be enchanted in many ways. These include:

\begin{itemize}
  \item 
    Any kind of item can be made stronger, more durable. (This is quite simple.)
  \item 
    Weapons can be made to hit harder and more accurately. (This is quite complicated. I don't know how it works yet.)
  \item 
    Items can be enchanted to act as a focus for a spell, mystically as well as physically. An example might be a cannon made to fire fireballs or lightning bolts. Creating a cannon that can cast these spells on its own is extremely difficult, but you can make a cannon that can be operated by a mage, allowing him to cast spells through the cannon more powerful than he could otherwise cast. The physical shape of the cannot could also aid in shaping the spell, making it easier and safer for the mage. 
  \item 
    It is \emph{not} possible to make enchanted items with arbitrary superpowers. For instance, no `Frying Pan of Ultimate Gourmet Cooking' or `Comb of No Bad Hair Days Ever'.
  \item 
    You can enchant an item to carry a limited (small) number of `charges' of a certain spell (or more than one). Examples are potions that are ingested to activate the spell. These may utilize the ordinary chemical properties of the ingredients in addiction to or in conjunction with the magic, to create more powerful effects (healing or whatever) than otherwise possible. 
\end{itemize}

\subsection{Making magical items}
Making a magical item is generally difficult and time-consuming laboratory work and often requires exotic ingredients. If a magical items is to be really effective, you cannot just enchant an existing item. Rather, it must be created from scratch, with the magic being addded from the beginning, weaving the magic in at every stage while it's being forged or assembled. 

Even creating a trivial magical item is tremendous work, so mages tend to make a few powerful ones rather than many little ones. The exception is one-off items like potions, which are relatively easy to make. 



\begin{comment}
\section{What can be done with magic}
\begin{itemize}
	\item You cannot create things out of nothing. The law of conservation of matter is still in effect. 
	\item Neither can you destroy matter entirely. 
%	\item At TL3, there are no spells for creating `real' illusions, ie., holograms which everyone can see. But you can create a mental illusion which only certain people can see. See section \ref{Illusions}  
  \item Probably more... 
\end{itemize}

\section{Shape-changing}
Both \dragons{} and \banes{} use magic to polymorph into different shapes. 



%\sectionn{Illusions}


\section{Who can cast magic?}
In some worlds, magic is a special gift, a talent that comes naturally to a select few people. In such worlds, usually only those specially gifted individuals can use magic at all. 

In my world, every intelligent creature can learn to use magic. Magical skill and power is very much a matter of training and experience, but 

\section{Clerical magic}
Priests can also cast magic. 

\section{Dying energy}
When creatures die, they release extra energy. Is this Ki or Mana? 

Draining the dying energy may destroy the soul in the process...? 

\section{Curses}
\subsection{Dying curses}
Upon death, some creatures are able to cast a `dying curse' on their enemies, typically the creature's killer(s). A dying curse is magical in nature, and works much like any other curse. Dying curses are usually cast by mages, using their dying energy for greater effect. 

However, in rare cases, even a non-mage may, through the sheer power of hate, be able to channel their dying energy into a curse. 

\end{comment}



