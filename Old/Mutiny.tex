\documentclass[a4paper,12pt,oneside]{article}
\usepackage[english]{babel}
\usepackage[latin1]{inputenc}
\usepackage{verbatim}
\usepackage{graphicx}
\usepackage{makeidx}
\usepackage{tipa}
%\usepackage{showidx}
\usepackage[pdftex,colorlinks]{hyperref}

\title{Sentinels of Mith\\Fragments}
\author{Claus Appel \\ 
E-mail: \href{mailto:spectrumdt@gmail.com}{spectrumdt@gmail.com}}

\makeindex
\frenchspacing
\pagestyle{headings}



\begin{document}



\input{CommandsMath}
\input{CommandsBeginEnd}
\input{CommandsLabelIndex}
\input{CommandsMith}
\input{CommandsTimeline}



\maketitle
\tableofcontents

%\newpage



%\section{Timeline}



\bepo{Pre-History}
\eepo



\bepo{The Age of \Dragons{}}

\event{First Dragons}{First \Dragons{} born.}
\event{Founding of Nom}{
  Florocthakhsur, Lord of House Tindarex, declares herself Queen of the \Dragon{} Kingdom of Nom, taking the royal name Nacheldryss, meaning `vengeful'.} 
\event{Fall of Nom}{The Fall of Nom.} 
\eventl{Founding of Irokas}{
  Hekhdraxsis, Lord of House Irokas, declares herself Queen of the \Dragon{} Kingdom of Irokas, taking the royal name Tentocoth, `unequaled'.} 
\eepo



\bepo{The Days of the Empire}

\event{Faegos birth}{Faegos Vaimon is born.} 
\event{Belandos birth}{Belandos Vaimon II is born, the first son of Faegos Vaimon.} 
\event{Cordos Vaimon birth}{Cordos Vaimon born, the first child of Belandos Vaimon II.} 
\event{Faegos death}{King Faegos dies. Belandos Vaimon II crowned king.} 
\event{Silqua birth}{Silqua Delain born.} 
\event{Silqua wed}{Silqua weds Cordos Vaimon.} 
\event{Belandos death}{King Belandos II dies. Cordos Vaimon crowned king. He begins to conquer a bunch of stuff.} 
\event{Silqua death}{Silqua Delain Vaimon dies.} 
\event{IC}{Cordos Vaimon declares himself Emperor of the Vaimon Empire.} 
\event{Belzir birth}{\Belzir{} daughter of Cormin born.} 
\eventl{Darkfall}{The \Darkfall{} and the end of the Empire.} 
\eepo



\bepo{The Modern Age}

\event{Noreocchyrias birth}
{Birth of Xoldm of House of Irokas (later to be King Noreocchyrias).} 
\event{Founding of Imetrium}
{Imetrium is formed.} 
\event{Noreocchyrias crowned}
{Xoldm is crowned king and takes the name Noreocchyrias, `feared by his enemies'.} 
\event{Tentocoth birth}
{Birth of Frycad of House Irokas (later to be King Tentocoth).} 
\event{Founding of Belkade}
{Belkadian Empire formed.} 
\event{Tentocoth Vaccashyth}
{Frycad comes of age and takes the name Vaccashyth.} 
\event{Tentocoth Rhoahathnex}
{Vaccashyth takes the name Rhoahathnex.} 
\event{Tentocoth Ibrethnavish}
{Vaccashyth takes the name Ibrethnavish.} 
\event{\EreshKal{} flourished}{The civilization of \EreshKal{} flourish in modern-day Runger.}
\event{\EreshKal{} destroyed}{\EreshKal{} is destroyed after several neighbouring lords band together to wage war against them.}
\event{Fall of Belkade}
{Belkadian Empire collapses.} 
%\event{Noreocchyrias abdicates}{King Noreocchyrias abdicates. His son Vaccashyth is crowned king and takes the name Tentocoth.} 
\event{Noreocchyrias death}{King Noreocchyrias dies. His son Vaccashyth is crowned king and takes the name Tentocoth.} 
\event{Carzain birth}{Carzain is born on the 12th day of \Izion.} 
\eventl{Runger war}{King Morgan of Runger invades Pelidor in the beginning of \Atzirah.}
\eepo





\section{The Mutiny}
\timee{Carz}{11th}{\Yeziel}

\new{}
\talk{Father,} said Carzain. \talk{I do not trust them.} 

\talk{What do you mean, son?} asked Nishain.

\talk{The mercenaries. They are up to something.}

\talk{Do you think they will betray us?} Nishain glanced at the mercenary captain. \talk{No, I cannot think that. I have dealt with Captain Arturo before. He is an honourable man.} 

\talk{The captain may be,} said Carzain, \ta{but has less savoury characters among his men. The one with the nose - do not look!} he hissed as his father started to turn his head. \ta{That one - Fisher, I think - he is the worst. He is always casting strange looks at us, at the captain and at the cargo.}

\talk{Even so, I cannot believe that they will mutiny. Captain Arturo would not let that happen.}

\talk{They are thugs, scum.} said Carzain. \ta{But I hope you are right, Father.}  

Carzain was not reassured. They were five days' travel out of Martinum and still ten days from \Redglen{}, so if the mercenaries did indeed plan to betray and rob them, they would be far from civilization when it happened. 

He looked at Fisher. He was a thin man with a long, narrow nose, medium-length, stringy black hair, a short stubble of black beard and a hunched posture. He wore brigandine armour and carried both a sword and a spiked mace in his belt. \tho{He looks the typical thug,} thought Carzain, \tho{ready to cut his neighbour's throat and rob him at the first opportunity. Truly the scum of Mith.} Carzain scowled at the man, then caught himself and suppressed it. \tho{If they notice my suspicion, they'll kill me first.} 

And so, through the entire day, Carzain went and eyed the soldiers suspiciously while doing his best to hide it. 

There was the captain, Arturo, a middle-aged man of perhaps forty, with rather well-kept graying hair and a short gray beard. He wore chain mail and carried a long sword and a round, wooden shield. He walked straight and commanded his men with quite admirable dignity and authority. \tho{Yes, Father is probably right. Arturo is a good man. Quite the opposite of Fisher.} Again he had to fight the urge to scowl at the ugly man. \tho{You can tell their character by the looks in their eyes and the way they walk.} Carzain was confident of his ability to read people and not inclined to doubt his own conclusions. 

There was Roger, a balding man with a goatee wearing studded leather and carrying a broad sword. \tho{His mouth has a vicious sneer to it. He is one of the bad ones.} 

And there was the boy, Gasper. He was the youngest of the soldiers, younger than Carzain by at least a few years. He had cut his brown hair short, and judging by the looks of his cheeks, he was working very hard to grow a beard and look rugged. \tho{An insecure kid eager to play with the big boys, eager to earn a their respect and be a big, tough man. Classic gang lackey, just like those worms Niclas and Owen back home.} 

He surveyed the caravan. To this left he found Hopper. \tho{I don't know about her. I can't read her, she is a Meccaran.} Hopper was the only non-Human of the band and the only woman. She wore no armour but carried two javelins and several daggers. He studied her characteristic hopping gait. \tho{Hopper is probably not even her true name. I suppose she has some Clictua name or whatnot, something the soldiers can't pronounce.} 

He looked around again. To his right he saw old Harold. He was the oldest of the guards, of Nishain's age at least, half-bald and with a long, gray beard. He wore studded leather and wielded a mace. He looked up at Carzain and smiled. \tho{Harold is perhaps the only one of them who smiles. He is an honest man.} 

There were nine mercenaries in all, and Carzain did not know the names of the last three, nor their moral character - but he suspected the worst. 





\new{}
Even so, the day passed without incident. \tho{Of course,} thought Carzain. \tho{They want to get us further into the wilderness before they strike.} Having left the Pylandos Road to head west into Pelidor, they had to cross the wilder regions of Scyrum, into the great Heropond Forest. The paths had been quite civilized in the beginning, when they were still close to the Pylandos Road and the towns near it, but they had gotten more rudimentary and ill-kept by the hour, and by now, close to halfway through the forest, it was almost pure wilderness. 

Heropond was not meant to be crossed. It did not \emph{want} to be crossed. Stories abounded of monsters in the forest, and while they were certainly exaggerated - such things always were - Carzain did not doubt that there was some truth in them. No one crossed Heropond. The forest was undoubtedly one of the main reasons why there was seldom war between Pelidor and Scyrum: Most of the border between the kingdoms was nigh-impassable, so they had little border terrain they could possibly fight over. Merchants and other travellers took the much longer way north through Pylandos or the even longer way south by sea, but Nishain and Carzain had estimated that the time saved would be worth the trek through the forest. Perhaps that had been poor judgement. 

He eyed Fisher with distaste. Having found himself unable to keep from studying the mercenaries, Carzain had taken to looking down his nose at them, using half-feigned arrogance to mask his suspicion and scorn. Fisher noticed his gaze and scowled back at him. Carzain answered with an impetuous pout. Fisher snorted, then turned his back. \tho{Yes, thug, that's right. I am just an upper middle class boy, a pampered scholar who thinks he is better than you. Blind to the world beyond my books. Nothing to worry about.} He hoped the soldier bought it. 

%Another day passed and still no incident. Every morning and evening, Carzain sat and studied their maps.

%Another day passed and still no incident. 

That evening, as the guards sat around the fire, playing dice and laughing raucously, Carzain pulled his father aside again. \ta{Father, I still think something up. Fisher is looking meaner by the minute. We cannot trust them.} 

\ta{Carzain, you judge people too quickly. Certainly, Fisher is an unkempt sort, but mercenaries \emph{are} a rough bunch. They all are, but that does not make them marauders and mutineers.}

Carzain grimaced but said nothing but a low \talk{Hrumph}. 

Nishain smiled and put a hand on his shoulder. \ta{Don't worry, son. I am sure we will be fine. Now, let us get some rest, don't you think?}

%At some point, Carzain's suspicions are proven correct as the mercenaries rebel, killing their own captain and attempting to kill Carzain and Nishain and rob them. Fortunately, Carzain, suspecting foul play, is only feigning sleep. He leaps up and challenges them, preparing to defend himself with magic. He realizes that he is being overconfident and will most likely be killed, but he enters combat anyway. 



\new{}
\talk{What's this? No! Help! Gaa...} Carzain's eyes popped open even before the man's shout was cut short, in time to notice the mercenary bent over him - Roger, the one with the goatee. Swiftly he lashed out with the dagger he had concealed in his hand. His attack was clumsy, but he managed to give the thug a gash across his face. Roger yelled out in pain and suprise and staggered enough for Carzain to roll out of his grasp and climb to his feet. 

\talk{The kid's up!} yelled a nearby thug, a fat, bald man wearing brigandine armour and wielding a wicked-looking spiked mace. \talk{Get him!} He ran toward Carzain, who had to think quickly. 

Carzain had never before faced an actual fight to the death, but many were the books and stories he had studied about famous Vaimon warriors in the past and their glorious battles, and many were the imaginary battles he had fought in his head, not least these past few days. Reaching out through the Cosmos with his mind, he touched the flow of power that was Iquin. Amid the dazzling river of the Light he found the \Sephirah{} he sought: \Izion, the Bringer of Fire. 
%He had never before faced an actual fight to the death, but many were the books and stories he had studied about famous Vaimon warriors in the past and their battles, and many were the imaginary battles he had fought in his head, not least these past few days. 
But even so, the shock of seeing a flesh-and-blood, mace-swinging attacker rush at him had caused him to hesitate precious moments, and the mercenary was almost upon him. Carzain flinched away, eyes averted, half-expecting to feel his skull crack open as he shouted: \talk{\Izion!} 

There was a flash of light and flames leapt from his outstretched hands. The brigand cried out in pain and startlement and staggered back a few steps. 

%\thought{Back-stabbing scum} was the thought that went through Carzain's head. \tho{I will kill you!} 

Heartened by success, Carzain pressed the attack. 
In his many mind-battles, he had fought expertly with a sabre and called upon a host of Sephiroth for terrible effect, but in there here and the now, in the heat of battle, it was all he could manage to maintain his hold on \Izion{}. He called upon the \Sephirah{} again, drawing deeper of its power this time, and a bolt of fire sprang forth to strike the large brigand square in the chest, and the man was cast to the ground, screaming and burning. 

\thought{Betraying scum,} thought Carzain. 
\tho{I will kill you!} \talk{\Izion!} he called, hurling another fireball at the fallen bandit, who shrieked once and then became still. 

Carzain stood and watched the smoldering corpse for what seemed like many moments. Thoughts flew through his head. \tho{I have killed a man. I have fought my first battle, and I won. I killed him. I swore to myself that I would kill him, and I did.} Then his eyes widened with sudden realization. \talk{Father!} He turned and was relieved to see Nishain alive and stirring, but with a bandit - whose name Carzain did not know - chrouching near him wielding a \Durcaci{} scimitar. \talk{No!} The thug stared at him. Evidently he had been about to kill Nishain but had recoiled when Carzain began casting spells.

Carzain looked about him to see five thugs standing, all with weapons drawn, all glaring at him, their eyes showing both fear and anger. \tho{I have won one battle, but more battles await. All these men... I must kill them... or they will kill me.}
As he thought this, a strange trance came over him. \tele{Very well,} he heard himself think, \tele{I will kill them. For my life and my father's life, I will kill them!} \tho{Were those my thoughts?} It felt almost like a strange voice in his head. His body was moving now, and he saw himself advance on the brigand standing near his father. 

Nishain was awake now. \talk{What is this?} he cried. \talk{What is going on?} He looked around. \talk{Captain Arturo, where are you? Carzain? What are you doing?} 

This caused the bandit near him - his name might be John, Carzain thought - to snap out of his shock, and he moved towards Nishain, scimitar raised. 

Seeing the blade swoop towards his father's head, Carzain found himself detachedly considering that perhaps he should be feeling fear. But in the grip of this strange ecstasy he found himself strangely calm, feeling first and foremost a cold anger and a rush of excitement in anticipation of the impending battle. 
%Carzain's body was strangely calm. 
He felt himself lift a hand and invoke a name. \talk{\Sezyron{}!} He felt himself reach out into Iquin and call upon the \Sephirah{} with astounding ease. Power surged through him, power such as he had never channeled before. Power flowed from the \Sephirah{} and into his body, into his arm, into his hand. There was a crack like thunder, and lightning lanced from his outstreched hand in a blinding flash. It struck the scimitar-wielding thug - John? - with tremendous force, and he flew many yards to collapse as a broken, charred, unmoving heap. 

The mercenaries gasped. \talk{Black magic!} he heard one croak. \talk{This is not what we bargained for, Fisher!} The speaker was Hopper. 

\tele{No. Not what you bargained for,} Carzain heard the voice say. Cold, sardonic - his own voice? Or someone else's? 

While he pondered this, Carzain felt himself turn towards the Meccaran. \ta{\Sezyron!} he called, and again lightning sprang from his hand, and the woman was incinerated where she stood. 

Two of the surviving renegades now turned their tails and ran. One was Roger, whom Carzain had cut before with his dagger; the other was the boy, Gasper. Carzain considered that he might spare the boy, but the entity possessing him was having none of that. 
Reaching now into Nieur, the force of Darkness, he heard himself call out
\ta{\Horvaleth!} He felt a cold, malevolent presence sweep over him as he channeled the power of the \Kliffah{} of the Cruel Winter, and he watched as shards of ice flew from both of his hands to impale both of the running men through their backs. They fell to their faces, each in a puddle of blood and snow. 

Carzain looked around and saw only one thug still standing: Fisher, crouching some yards away with his boardsword clutched in both hands. \talk{Fisher. This was your work.} Carzain, half-accustomed to being a passenger in his own body, was surprised to realize that he could no longer feel the possessing entity, and that the words he had spoken were his own. 

Fisher looked around him, saw his comrades who had run and been slaughtered. He turned to Carzain. \ta{Black magic fiend! I'll cut you to pieces!} He charged, sword raised high. 

Carzain, still in the habit of relying on his body reacting on its own, was cast off guard. He jumped aside, but Fisher was not easily surprised and twisted his sword around so that it tore into the flesh of Carzain's right side and lower arm. 

Carzain ran a few steps before turning, the pain in his wound having rekindled his will to fight. Fisher attacked again, but Carzain swifty raised his good hand and invoked his \Sephirah. \ta{\Izion! \Izion! \Izion!} Again and again he called the name, and again and again projectiles of fire sprang from his hand. 

Carzain laughed at his victory, then screamed as pain lanced up his left leg. Fisher, after being struck once or twice, had rolled aside, then, while Carzain stood half-blinded by the glare of his own fire, had seen his chance to creep forth and slash at the mage's leg. Carzain cursed his own overconfidence as he turned his hand to Fisher. \ta{\Izion!} he cried, launching a firebolt at point-blank range straight at the brigand's face. Fisher did not cry out, merely crumpled. \ta{Die, backstabbing lowlife! Die!} Carzain yelled, shaking all over with fury as he channeled \Izion's flames to incinerate the body of the fallen renegade. 



\new{}
He was still burning away at the corpse when he heard the voice behind him. \ta{Carzain!}

Startled, he spun around, but it was only Nishain. \ta{Father! Are you hurt?} 

\ta{No, I am well,} said Nishain. \ta{But you... you are wounded! We must have you treated.}

Even with the withdrawal of the strange ecstasy, Carzain was still in the grip of the fury of battle. Finally snapping out of it, he began to truly feel his injuries. He grunted in pain, attempting to cover the cuts with his hands and shifting his weight away from the bad leg. 

\ta{Come over here and sit,} said his father, lending him a shoulder and leading him to the wagon so he could sit down. 

\ta{Hold still, let me...} said his father. He touched Carzain's side, near the wound and invoked: \ta{\Bihirai.} Carzain felt how the \Sephirah{} reached out through his father, seeming to caress him with its fingers. He opened himself up to the \Sephirah{} and felt a surge of fatigue as it consumed his bodily energy in its efforts to heal his wound. Gradually, the wound closed and the bleeding stopped, the pain subsiding somewhat, but the cut was still there. 

Nishain let out a sigh. \ta{This is as good as I can manage. Give me that leg.} He took Carzain's wounded leg in his hands and invoked the \Sephirah{} again, and once more Carzain felt \Bihirai{} consume his energy. Again, the wound closed but could still be felt. 

\ta{I can do little more,} said Nishain. \ta{I am not as great a healer as your mother. But we have some herbs that will help. I will get them.} 

For a while neither spoke. Nishain searched through the wagon for the herbs and tools while Carzain sat nursing his wounds and pondering what had happened. 

Nishain came back. \ta{I found the \jiliba{} berries and the \dvingen{} leaves. I will prepare them.} He sighed. \ta{And some food, I suppose.} He looked at his son with concern and some measure of unease. \ta{You should get some rest, son.} 

And so Carzain rested while his father made a fire and prepared breakfast and medicine. 

The silence stretched, became embarrassing. Carzain was not comfortable discussing what had happened and his father was not comfortable asking. Nishain was purposefully ignoring the charred bodies scattered around the camp, evidently focusing intently on the simple task at hand as a means of coping with the shock of the attack. 

Finally, Nishain took the pot of cooked herbs from the fire. \ta{It should be ready now.} He poured some in a cup and handed it to Carzain. \ta{Here, \jiliba{} tea.} 

\ta{Thank you.} Carzain took the cup and drank. \Jiliba{} tea was known to help you recover from exhaustion. 

Nishain took forth some straps of cloth. \ta{Here, you should also get a poultice for the cuts.} He took the pot with the dissolved \dvingen{} leaves. \ta{This will sting a bit at first.} 

Carzain grunted as he let his father foment his wounds with the \dvingen{} mash and wrap them up. \ta{There, this should help it heal.} 

\ta{Yes. Thank you, Father.}

Nishain returned to the fire and started to prepare food, as an excuse for prolonging the otherwise awkward silence. After a while, he returned with with some roasted pork, lettuce and wine. They ate for a while in silence. 

%At length, Nishain came back with some roasted pork, lettuce and wine. 

%(Some more stuff. They eat for a while in silence.) 

In the meantime, Carzain looked around. Apart from the six mutineers he had killed, there were three more bodies. Those were Arturo, the old Harold and the last soldier, a young, beardless man. Carzain did not know his name, but he believed it was this man's cry that had awakened him. His throat had been slit, and as far as he could see, the same had been done to the two others. 

At length, Nishain sighed. \ta{You were right, son. I should not have disbelieved you earlier.} 

Carzain gave a non-committal \ta{Hmm}. \tho{I guess it was guilt, as much as shock, that kept Father silent for so long,} he mused. 

Again, for a while no one spoke. 

\tho{I suppose I should help assuage it,} thought Carzain. \ta{But you were right about Arturo, Father,} he said. \ta{He is over there.} He pointed. \ta{They mutinied against him, slit his throat. It seems I was right about Fisher being the leader.}

Nishain nodded. 

There was another pause, until finally Nishain dared to ask: \ta{Carzain... how did you... how did you...?} 

\tho{How did I kill those men, you mean.} Carzain hesitated before answering. He \emph{had} killed them. He had quickly come to terms with that fact, and had found that he felt no guilt or grief over it, but even so, he found that he was not comfortable discussing it with his father. \ta{I don't know,} he said, buying himself more time to think. \tho{Fool,} he thought to himself. \tho{Why do I suddenly feel abashed?} He snorted inwardly at himself. \tho{Like a damned Iquinian.} Carzain used to pride himself of being free from such sentiments of irrational shame. He and his father were close, and he had always been able to talk to his father about anything. Even sex, which many people balked at. In this, among other things, he favoured his Geican heritage. 

%But here he had found something that his mind feared to talk about. 
But here he had found a subject matter that gave pause to him. 
\tho{Perhaps I am no better than all those fools who feel shame at everything.} This caused him to frown. \tho{Pull yourself together, Carzain.} \ta{Yes,} he said at last, \ta{I killed those men. I...} - again he stalled - \ta{I don't quite know how.} 

\ta{You... you fought with such... such \emph{power},} his Father began. \ta{And the \Archons{} you called... \Sezyron... and \Horvaleth... I had no idea you could do that.} 

\tho{Neither did I,} Carzain wanted to say. But what he said was: \ta{No... I didn't realize I had such power. I think... I think when I realized they had betrayed us and wanted to kill us, I was... I was seize by an immense anger, an... an immense \emph{hate}.} Carzain was somewhat relieved at this turn of the conversation. This was something he could talk about, something he knew. As something of an outsider in his home town, he had always had enemies. Right from when he was a child, there had been other children, stronger and bigger than he, who had picked on the `strange' Vaimon boy. Even after he had grown up and learned to defend himself, there were those who would attempt to molest him when given the chance. Yes, Carzain knew of anger and hate against an enemy, and he knew how to talk about it. 

\ta{I think the anger gave me strength. You know I have been in fights often enough before where I wanted to kill my opponent.} They had discussed this before, and Nishain had shown more sympathy for his son's violent urges than an Iquinian would. \ta{This was the first time I was able to kill an enemy. I don't know... this passion might have enabled me to channel more power than otherwise. It has happened before.} It was true; according to the literature, strong emotion could often enable a mage to perform feats otherwise beyond his ability. It might be the explanation. Still, Carzain had his doubts, as this did not seem to explain his feeling of being possessed. 

His father nodded. \ta{Hmm... yes... that might be. But even so, those \Archons{}... \Sezyron{} is a difficult \Sephirah. I have tried to invoke him, but I have never gotten more than a feeble spark out of it. Did you learn that by yourself?}

\ta{Something like that...} Carzain answered evasively. He reached out for the wine, and Nishain handed it to him. \tho{No, I haven't learned it. I had never invoked him before now, and I have no idea how I did it.} But, for some reason, he did not want to tell his father that. 

\ta{And \Horvaleth! A terrible \Kliffah, cruel and dangerous. Or so I have read. I have never tried to channel her power. Do we even have any books that describe how to invoke her? Or how do you know of her?}

Carzain saw that his father was intent on pressing the issue. \tho{Pull yourself together, Carzain! I just killed six armed soldiers. I need not fear to talk to my own father.} He took a deep breath. \ta{I don't.} 

\ta{What?}

\ta{I don't know how to invoke her. Yes, I did, but I don't know how. It just came to me. It was the same with \Sezyron.} 

Nishain gave him a wondering, incredulous look.

\ta{I fought the first man on my own, calling upon \Izion. I killed him, and then I felt...} - again he hesitated and had to force himself to go on - \ta{I felt this weird... \emph{ecstasy}. As if some force had seized control of my body. As if I was possessed by some entity. Perhaps an \Archon{} or a Daemon. Some entity that used by body to fight.} He paused to drink deep of the wine. %\ta{I could feel the Archons, feel their power being channelled through my body. But I did not invoke them. My mouth spoke their names...}
\ta{That was when I channelled those \Archons. I cannot tell you how.}

Nishain gaped. \ta{Possessed by an \Archon? Great \Sephiroth! I can't b...} He stopped himself, then hesitated before saying: \ta{Very well, I believe you. \Sephiroth{}, but this is a lot to take in...} 

\ta{I don't know if that is what happened, but that is how it felt. As if something was in control of me.} He smiled. \ta{Except at the end. The ecstasy vanished at the end, when there was only Fisher left. I fought him myself.} He gestured at his wounds with a sarcastic grin. \ta{And see what that got me.} He gritted his teeth. \ta{But I killed him.}







\section{Pronunciation Guide}
\label{PronunIPA}
This is the correct way to pronounce Mithian names and words. Pronunciation is given in IPA (International Phonetic Alphabet) as well as in my own makeshift phonetics. 

\begin{pronunciationenvironment}
%\pitem{\Ashenoch}{['�SEnox]}{[�-she-noch]}
%\pitem{\Bedhin{}}{[bE'Di:n]}{Bedhin}{be-DHEEN}
%\pitem{Belkade}{['bElkeId]}{BEL-kejd}
%\pitem{Belkadian}{[bEl'keIdI@n]}{bel-KEJ-dee-an}
%\pitem{Bryndwin}{['b\rr IndwIn]}{BRIND-win}
\pitem{Carzain}{[ka\gr'zaIn]}{kaarh-ZAJN}
\pitem{Clictua}{['klIktua]}{KLIK-too-ah}
%\pitem{\Deracille}{[de\gr a'sil]}{d�-rha-SIL}
%\pitem{\Faledh{}}{['f�lED]}{F�-ledh}
\pitem{Heropond}{['hE\rr opO:nd]}{HE-ro-pond}
%\pitem{Ilcas}{['Ilkas]}{IL-kas}
%\pitem{Imetrian}{[I'mEt\rr i'n]}{ih-MET-ree-an}
%\pitem{Imetrium}{[I'mEt\rr i@m] or [I'met\rr ium]}{ih-MET-ree-um}
\pitem{Iquin}{['i:kwin] or ['i:kwIn}{EE-kween}
\pitem{Iquinian}{[i:'kwIni�n] or ['i:kwIni@n]}{ee-KWIN-ee-an}
\piteml{\Izion}{['i:Zon]}{EE-zhon}{Izion}
%\pitem{Martinum}{['ma:tIn@m] or ['ma:tInum]}{MAR-ti-num}
\pitem{Meccaran}{[mE'ka:\rr an] or [mE'ka:\rr @n]}{me-KAA-ran}
\pitem{Meccara}{[mE'ka:\rr a]}{me-KAA-ra}
\pitem{Mith}{['mi:T]}{MEETH}
%\pitem{Mulgron}{['m2lgron]}{MUL-gron}
\pitem{Nieur}{['nj3:\gr]}{NJ��RH}
\pitem{Nishain}{[ni'SaIn]}{ni-SHAJN}
%\pitem{Nycan}{['naIk�n]}{NAJ-kan}
%\pitem{Nycaneer}{[,naIk�ni:\rr] or [,naIk�ni:@]}{naj-ka-NEER}
\pitem{Pelidor}{['pElIdO:] or ['pelIdO:\rr]}{PE-li-dor}
%\pitem{Pelidorian}{[pElI'dO:\rr i@n]}{PE-li-dor}
\pitem{Pylandos}{[paI'l�ndos]}{paj-LAN-dos}
\pitem{\Qliphah}{['klIfa]}{KLI-fa}
\pitem{\Qliphoth}{['klIfoT]}{KLI-foth}
%\pitem{Scatha}{['ska:D a]}{SKAA-dha}
%\pitem{Scathae}{['ska:D eI]}{SKAA-dhej}
%\pitem{Scathaese}{[ska:D a'i:z]}{skaa-dha-EEZ}
\pitem{Scyrum}{['saI\rr @m]}{SAJ-rum}
\pitem{Sephirah}{['sEfIra]}{SE-fi-ra}
\pitem{Sephiroth}{['sEfIroT]}{SE-fi-roth}
\piteml{\Sezyron}{[sEZy'\gr on]}{se-zhy-RHON}{Sezyron}
\pitem{\Shireyo}{[Si'\rr eIo] or [Si'\rr eIoU]}{shi-REJ-o}
%\pitem{Uzur}{['u:zu:\rr]}{OO-zoor}
%
\end{pronunciationenvironment}




\end{document}

