\section{Rejected creatures}



\begin{comment}
\section{Species and Races}

\subsection{Generally on species}
Needless to say, species is a very important aspect of a character. Mith is inhabited by multiple intelligent species, many of which should perfectly well as player characters. 
\end{comment}

\begin{comment}
\section{No humans}
\label{Humans}
\index{Humans}

There are no humans on Mith. 

Most fictional universes have humans in them. But my problem with this is that the systems invariably become human-centered. Humans are always the dominant race, and many other creatures are based on humans. This problem appears in all kinds of fantasy as well as RL mythology. Examples: An elf is a human with pointy ears. A dwarf, gnome, hobbit or halfling is a small human. A centaur is a human torso on a horse body. A medusa is a human with snakes for hair. A harpy is a birdlike human. A manticore is a lion with a human face. And so on. 

The problem is that humans have a special status above other creatures. They are central to the world in a way that orcs and lizardmen can never be. Damn it, where is the lizardman/crocodile centaur? Where is the serpent monster with an orc head? No, only human hybrids are allowed. Even the undead are always humans, with few exceptions. (In D\&D (2nd edition), there is \emph{one} undead creature called an `Undead Dwarf'. Go figure...) 

When designing Mith, I realized that the temptation to do this is very real. As soon as humans exist, it is just so easy to make other creatures that look like humans, or monsters that have human parts. I did not want to fall in this trap, so I decided that the best course would be to outlaw humans alltogether. 

So no humans. 

There are also no demihumans - no elves, dwarves or the like. And while I was at it, I figure it would work best if I went all the way and used only my own humanoid races. This means there are also no orcs, no trolls, and so on. Instead, Mith is populated by Scathae, Meccara and other creatures of my own creation. 

To say that I `went all the way' is a bit of an exaggeration. There are still classical mythological/fantasy monsters on Mith. There are Dragons, and there are Manticores and Medusae. But those monsters that have human parts in the original context have been altered and these aspects taken out. 

%\subsection{But I use them anyway...}
Despite the fact that humans do not exist on Mith, I may still refer to them. I do so only for comparative purposes or as hypothetical examples. I will say things such as `Scathae have eyesight as good as that of humans', `Meccaran have shorter arms than humans', etc. 
\end{comment}



\begin{comment}
\subsectionn{Fittera}
The Fittera are a race of amphibian humanoids, closely related to the Meccara. Fittera are adapted to cooler, drier climates, while the Meccara prefer warmer, more humid climates. Fittera are larger and stronger than Meccara, but less intelligent and less civilized. 

\subsubsection{Name}
Singular \emph{Fitteran}\footnote{[FI-te-ran] (Imetric word). Note that the stress is different from that in `Meccara'}, plural \emph{Fittera}\footnote{[FI-te-ra]} (Imetric declination) or \emph{Fitterans} (English declination). %The distintion here is the same as with \emph{Meccara}/\emph{Meccarans} (see under Meccara, p. \pageref{Meccara}). This grammar is Imetric. 
The associated adjective is \emph{Fitteran}. 

\subsubsection{Physique}
A Fitteran looks like a larger Meccaran. An adult female stands 160-170 cm tall and weighs around 80 kg. A male is 150-160 cm tall and weighs 70 kg. They are colored in shades of brown. 

Fitteran senses are like those of Meccara. 

They have the same regenerative abilities as Meccarans do. 

\subsubsection{Biology}
Fittera belong to the same genus as the Meccara but a different species. Their biological properties are mostly the same as those of the Meccara, but Fittera are less dependent on water and do not need to bathe nearly as often. Their lifespans are the same as the Meccara's. 

\subsubsection{Psychology}
Fittera are less intelligent than Meccara and most other intelligent creatures. They rarely understand science or intellectual art. In multiracial societies, Fittera usually live as peasants, labourers or soldiers. Fitteran mages are very rare. There are communities where Meccarans and Fitterans live together. In these tribes, the leaders will usually be Meccaran.\footnote{This is for Darwinian reasons: A tribe led by a stupid Fitteran chieftain is more likely to be wiped out by enemy tribes with more competent leaders.}

The Fitterans' communities are primitive barbarians, typically living as TL0-TL1 hunters. Tribes may be led by chieftains or shamans. Some of these tribes are peaceful `noble savages', living in harmony with nature. Other tribes are aggressive and may conduct raids on other communities.\footnote{Warlike Fitteran tribes may be used as a kind of `Orcs', if needed.} 

\subsubsection{Habitat}
Fittera are most common in the Middle Lands and not uncommon in the Rissitic Dominion, Geica and Uzur. 
\end{comment}



\begin{comment}
\subsection{Kinsari}
\label{Kinsari}
\index{Kinsari}
The Kinsari are simian humanoids. They originate from the isle of Erul, where they are common and have their own civilization. They are widespread across many other lands, especially in the Imetrium, but are uncommon all other places than Erul. They are especially noted as practitioners of psionics. 

\subsubsection{Name}
\emph{Kinsari} is both singular and plural. This grammar is the Kinsari language (as spoken in Erul). The associated adjective is \emph{Kinsari}. 

\subsubsection{Physique}
Kinsari are large monkeys. They stand around 160 cm tall and weigh around 55 kg, with the female being slightly smaller than the male. They stand erect on two feet and are covered in short fur. Their color is dark brown to black, but the belly and head are white. 

Their feet are very dextrous and may double as hands. They have a long, prehensile tail that can be used as a primitive `arm'. Kinsari are able climbers and are at home in the treetops. 

Kinsari senses are like those of humans. 

\subsubsection{Biology}
\subsubsection{Psychology}
\subsubsection{Habitat}
\end{comment}



\begin{comment}
\subsection{Lupin}
\label{Lupin}
\index{Lupin}
\subsubsection{Name}
\subsubsection{Physique}
\subsubsection{Biology}
\subsubsection{Psychology}
\end{comment}



\begin{comment}
\subsection{Gargoyles}
\label{Gargoyle}
\index{Gargoyle}
The Gargoyles are monstrous humanoids

\subsubsection{Name}
\subsubsection{Physique}
\subsubsection{Biology}
\subsubsection{Psychology}
\subsubsection{Habitat}
\subsubsection{Attributes}
\begin{description}
	\item[Horror effect:] Minor. 
\end{description}
\end{comment}



\begin{comment}
\subsection{Tchacolda}
\label{Tchacolda}
\index{Tchacolda}
\subsubsection{Name}
\subsubsection{Physique}
\subsubsection{Biology}
\subsubsection{Psychology}
\subsubsection{Habitat}
\end{comment}




