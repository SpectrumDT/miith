\chapter{Geography}

The planet Mith (or at least the part of it which I have yet made up) can be divided into three areas:

\begin{enumerate}
\item The Old Continent, containing the homelands of most of the dominant species. 
\item The New Continent, containing a number of multicultural societies, most notable of which is the Yet Unnamed Empire (ruled by the Yet Unnamed God, who maintains the neutrality of the Empire, in respect to the various religions and powers of the Old Continent). Most native species are primitive or few in number. 
\item The seas, inhabited by many civilizations and gods.
\end{enumerate}



\subsection{The Old Continent}

There are a number of civilizations in the Old Continent. The ones I have spent the most time describing so far are:

\begin{enumerate}
\item The Scathae and the Imetrium. 
\item The Rachyth and the Nechsaitic Dominion.
\item The Kinsari culture.
\item The Tchacolda kingdoms.
\item The Meccara tribes.
\item Irokas, the land of Dragons. 
\item The subterranean Mlisshur. 
\end{enumerate}



\subsubsection{RL analogy}
I like to think of the geography of the Old Continent as analogous to Europe and Asia in RL. Note that the following comparison is purely geographical, not political in any way. 

Imetric Col is the equivalent of Spain and Portugal, while Sulchrev corresponds to Turkey and the Middle East. The Tchacoldan kingdoms are Scandinavia and the Meccara live in Central Europe (Hungary, Czech Republic and so on). Distant and mystic Irokas then corresponds to Russia, while the Orient (India, China) is inhabited by strange and exotic civilizations. To comlete this analogy, the New Continent is the equivalent of America.

\subsubsection{Col - Heartland of the Imetrium}
To the west and south of Col lies the ocean. Most of the western coast is obstructed by a great mountain range. The land has some accessible coastline to the south. 

To the north lie the Northern Kingdoms. The Imetrium would like to control more coastline, so they sometimes make war on the Northern Kingdoms to expand their borders. Maybe there is a great river that runs through the Imetrium but ends in non-Imetric territory? If so, they would like to conquer it. 

\subsubsection{Northern Kingdoms}
The Northern 'Kingdoms' are not actually kingdoms, but a collection of clans, tribes and city-states that sometimes band together under a Khan or something. Many Tchacolda live here. What more?

The kingdoms have plenty of sea coast and rivers (fjords?). Some people in the kingdoms occasionally go on viking raids. 




