\chapter{Vaimon Cultures}
%\label{Vaimon}

\sectionn{Clan Redcor} 
The \pronun{Redcor}{rh�d-KORH} are matriarchal, led by a council of women, the Conclave. They rule the kingdom of \introlp{Redc�}{Redce}{rh�d-K�}, which lies in the Northern Belkade. The Conclave gather around the Topaz Throne. The traditional colour of Clan Redcor is yellow. 

The Redc� are leaders of the Church of the Light\index{Church of the Light}. This is a religion based on Iquin-Nieur theory (see section \ref{Vaimon magic}), the idea that the universe is based on two primal forces - Iquin and Nieur, Light and Shadow. The Redcor and the Church of the Light believe that Iquin is the primary, the first and most important of the two, representing all that is just and good. %The Redcor insist that Nieur should be translated `Shadow', because, according to their religion, Iquin existed before Nieur, and Nieur is a corruption of Iquin, something secondary that exists only at the mercy of the Light.\footnote{This is comparable to the role of the Devil in certain Christian interpretations: God and Satan are not equally matched adversaries. Rather, God is seen as all-powerful, and Satan, with all his evil, exists only because God, in all his good, allows him to. Supposedly, this somehow makes sense.} 

The Redcor see themselves and their church as the champions of good. They seem to combat and destroy evil wherever they encounter it. The Church of Iquin is widespread throughout most of Belkade. The Redcor do not directly rule outside Redc�, but they are skilled manipulators, pulling strings and directing events from behind the thrones. 

\subsection{Some stuff about Redcor religion} 
The Redcor worship Iquin, the Light, which is seen as personified. The Light, when interpreted as a person/god, is sometimes referred to as the \intro{Spirit of the Light}. The Spirit is viewed as a supreme, perfect being, all-good and very powerful, perhaps even all-powerful. The Spirit of the Light is seen as something special because it is `not of this world' but something transcendent. People of the Iquin Church tend to look down on religions that worship `earthly gods' (such as the Imetrium), which are seen as inferior, false gods. 

\sectionn{Clan Geican}
The homeland of Clan \pronun{Geican}{GEJ-k�n} is called \introp{Geica}{GEJ-k�} and lies to the east of Belkade. 

Geicans are recognizable by their narrow beaks and high-set eyes. The traditional colour of Clan Geican is green, and many Geicans wear green robes of some sort. The throne of Geica is of Emerald. 

They are scientists and study all sciences, including magic. They are known to use Nieur and actively pursue the study of it. They are believed to dabble in all sorts of black and evil magic and to consort with evil powers. 

The Geican culture is atheistic. They reject the personification of Iquin and Nieur that the Redcor believe in. According to the Geicans, Redcor religion is a lie: The Spirit of the Light does not exist, Iquin (like Nieur) is just an impersonal, amoral force. All Redcor `miracles' are regular magic and the Redcor metaphysics is based on a magic theory that is flawed and contradictory. The Redcor world view is a supremely naive fairy tale that blatantly ignores and denies many elementary facts known to any serious scholar. The Geicans see the Redcor as cowards who close their eyes to the real world because they are afraid to face the dark and terrible truth. In turn, they view themselves as a superior and wiser people, true scientists who pursue the truth without fear. 

Natually, this belief is blasphemous to the Redcor. To make matters worse, Clan Geican is known to have utilized black magic and consorted with the Dark Queen Bel'zhir in the past. In fact, according to Redcor historians, the Geicans' alliance with the Dark Queen was the cause of the great war that destroyed the Vaimon Empire and led to the Fourth Cataclysm. As a result, the Redcor believe that the Geicans are all malicious diablolists and alienists bent on conquering or destroying the world. (The Geicans themselves claim that the Redcor are intolerant bigots and prejudiced against them because of a select few genuine villains in Geican history. They are also quick to quote a record of Redcor atrocities in return.) 

%Geicans insist that Nieur should be translated `Darkness', rather than `Shadow', as the Redcor do. To them, Nieur is not a feeble reflection of Iquin, but a primal force, possibly even more primal than Iquin. 
%Some Geican scholars use the parable of the night sky: The natural state of the sky, they claim, is total darkness. The stars, try as they might, cannot eliminate the darkness. If you remove all the stars, the blackness will remain, but it is, of course, not possible to remove the darkness and have only the stars remain. 
% Geican scholars believe that Nieur is, at the very least, equal in status to Iquin. Some say that Nieur is 

Clan Geican is not centrally organized. Geica is a small nation and many Geicans live abroad, if they can avoid Redcor persecution. Durcac and the Imetrium has many Geican inhabitants. If asked, any Geican will tell you that their clan is unorganized and that they owe no allegiance to the Emerald Throne in Geica. The Redcor suspect that this is a lie and that  Geicans living abroad are secret agents and spies of some evil conspiracy. 

\sectionn{Clan Yrzhell}
The \pronun{Yrzhell}{yrh-ZHEL} have lost their original homeland and their Onyx Throne. Both were destroyed in the Fourth Cataclysm. Now they are a disorganized and wandering people. 

The traditional colour of Clan Yrzhell is black. 

The Yrzhell are a nature-loving people who practice a druid-like Animist religion. They are generally peaceful, unlike the militaristic Redcor. They live in forests in small communities or stuff like that. 

\sectionn{Clan Quaerin}
Clan \pronun{Quaerin}{kwe-REEN} live in the cold, mystic land of Zoitan far to the north. Almost no Quaerin are known to dwell in any other lands than Zoitan, and their dealings with the rest of the world are very limited, so little is known about them and their culture. They are ruled by an Immortal Czar who sits on the Diamond Throne. Their traditional colour is white. 



\sectionnn{Vaimon Magic: Iquin and Nieur}{Vaimon magic}{Vaimon magic}
Traditional Vaimon metaphysics tells that the Universe is governed by two basic forces, named \introeplong{Iquin}{EE-kween}{Note that in the Redcor dialect, this is one of the few words with stress not on the last syllable.} and \introeplong{Nieur}{NJ��RH}{Rhymes with French \emph{seigneur}.}. Iquin is translated 'Light', and is considered gentle and preserving, whereas Nieur, translated 'Darkness' or 'Shadow' (more on this below), is seen as aggressive and destructive. Iquin and Nieur combine to create the four elements of Air, Earth, Fire and Water, which in turn make up the world. All physical matter is composed of the four elements, but the mind and everything psychic/spiritual is made of the pure essence of Iquin and Nieur. 



Each Vaimon clan has their own interpretation of Iquin-Nieur magic theory. 

The Redcor insist that Nieur should be translated `Shadow', because, according to their religion, Iquin existed before Nieur, and Nieur is a corruption of Iquin, something secondary that exists only at the mercy of the Light.\footnote{This is comparable to the role of the Devil in certain Christian interpretations: God and Satan are not equally matched adversaries. Rather, God is seen as all-powerful, and Satan, with all his evil, exists only because God, in all his good, allows him to. Supposedly, this somehow makes sense.} The Geicans, on the other hand insist that Nieur should be translated `Darkness'. To them, Nieur is not a feeble reflection of Iquin, but a primal force, possibly even more primal than Iquin. 




\subsection{What can be done with Iquin and Nieur}
Here follows some guidelines regarding what is possible to do with the four elements under Vaimon magic theory. 

With all elements, there are certain things that can only be done with Nieur and some things that can only be done using Iquin. By default, most of the things mentioned below can be done using either Iquin or Nieur. The main exception is Fire magic, which is clearly split in two aspects, `positive' (Iquin) and `negative' (Nieur). 

With most Iquin-Nieur magic, the effects are temporary and must be consciously maintained. In order to make a spell permanent, you need to Sculpt an item with the spell. 

\subsubsection{Air}
Air magic is used to move air, create wind and influence the weather. Air can also detect things through the air and create lightning. 

Air can create wind to push things around. This is effectively a form of crude telekinesis. The easiest task is to make a blast of wind to blow something (or someone) away from you.\footnote{This is similar to the `force push' used by the Jedi in the Star Wars movies.} Throwing objects around (in arbitrary directions) is the next step. With practice, you can develop enough fine control to pick up items or move them around. Holding items completely still in the air is extremely difficult, possible only for a rare expert. If there is wind already, you can also use Air to calm it. 

An average mage will have an effective strength comparable to a regular human's arm strength when pushing or carrying things with Air. A powerful mage may have Air strength many times stronger than a man. Effective dexterity is very low at first (only crude shoving) but can be trained up to be almost equal to an average person's manual dexterity. A group of powerful mages can generate enough wind to push a small sailing boat, but only slowly - boats are heavy. 

If there is sand or debris available, you can use Air to create a scathing wind blast. 

Air, sometimes combined with Water, can also be used to influence the weather. A simple skill (using Air alone) is weather forecasting\index{Weather forecasting with Air magic}, allowing you to feel the air and predict the weather up to a few days in advance. This skill is very rough and unreliable, however - meteorology is a complex science. With much practice and experience, you can build up skill almost as good as modern day (TL7) meteorology\index{Meteorology}. Effective range will never be higher than two to three days, and errors are always possible. Note that raw power means little in this regard, giving only a small bonus. What really matters is skill and experience. 

If there are clouds, Air and Water can force it to rain. Water cannot be created from nothing, so such a rain will likely be light and short. If much water is present in the atmosphere, it was probably going to rain anyway, but magic can speed it up. If it is already raining (or going to in the near future), Water and Air can stop it, or Air can be used to create wind and blow the clouds away. If there are no clouds, you can use the weather forecasting skill to locate some and create winds to summon them, but this is difficult and slow. Affecting the weather requires a strong mage or several average mages. 

If there are heavy clouds, Air can create lightning. A lightning bolt will strike from the sky, not from the caster's hand. Lightning is harder to create than fire, but it is more precise. A lightning attack does more damage than an equivalent fire attack, but strikes only a small target, whereas fire is easy to spread over a large area. If you need to kill a single enemy mage or general, lightning will probably be best, whereas fire is more effective against massed foes. 

Using Nieur with Air, you can create so-called `dark lightning'\index{Dark Lightning}. This looks similar to lightning, but it strikes from the mage's hands and requires no clouds. Regular lightning strikes in an instant, but dark lightning can be channelled in a continuous stream. At full force it inflicts as much damage as fire, but it can also be restrained to cause pain but little damage. The damage is electrical, and resistance to electricity applies as normal. Dark lightning can only be used in an atmosphere, and water blocks it completely.\footnote{Dark lightning is similar to, and inspired by, the `force lightning' used by the Sith in the Star Wars movies.} 

Using Nieur, it is also possible to cut off a person's breath, strangling him. Killing someone in this way takes more than a minute and requires concentration - if the mage is attacked and takes damage, the spell is broken and the victim can breathe again. 

On a theoretical note, Air influences gas of any kind, not only regular atmospheric air. Water vapour, for instance, counts as air. (When manipulating rain, you are dealing with transitions from vapour to liquid water, therefore both Air and Water are needed.) If an Air mage is transported to an alien planet with a different atmosphere, Air telekinesis will still work (albeit perhaps slightly different, if the new air is more or less dense than Mithian air). Weather magic will work differently, because the physical properties of alien air will be different. With training, you can learn to influence the weather of an alien planet. The Air magic remains fundamentally the same. Dark lightning will work unchanged in most atmospheres. Most Air magic is useless underwater, and all Air magic is useless in a vacuum. 



\subsubsection{Earth}
Earth magic is used to move and shape earth. It affects all solid materials, including earth, stone, metal and ice. Note that with Iquin, only dead material can be affected. Living matter, including formerly living matter such as wood and bone, is immune. Using Nieur, such matter \emph{can} be influenced, however. In the following, `earth' (non-capitalized) will be used to denote any non-living (or formerly living) solid matter. 

%It does \emph{not}, however, affect living matter, including formerly living matter such as wood and bone. 

The traditional use of Earth is to reshape earth. You can dig holes and build mounds, or bend stone and metal. With more skill (and time), you can shape earth into more elaborate shapes. It helps if you can touch the earth you are trying to affect with your hands or another part of your body, but it can be done at a distance. The harder and thicker the material, the more difficult it is to shape. For instance, if a mage is thrown in prison, he could bend the thin metal bars quite easily, but only a very strong Earth mage could carve a hole in a stone wall and walk through. 

Earth can also be used to change the properties of earth, making it harder or softer. This is not permanent and must be maintained, but it can be used (for instance) to soften earth to make it easier to dig through, or strengthen a wall or door (unless made of wood) to prevent an enemy from breaking through. Earth hardened in this way becomes more brittle; if hit with sufficient force, it will not bend but shatter. Using Earth together with Water, you can even make earth as soft as liquid.\index{Liquid earth} This is hard and can only be done with earth that is very soft to begin with, or very small amounts of it. (This is not the same as melting with Fire; the matter becomes soft but not hot.) Note that when the spell ends, softened earth will return to its former consistency, but not its former \emph{shape}.

Earth magic is not telekinesis. You cannot make earth fly, only bend and reshape it. So the imprisoned mage could not simply pluck the keys from the wall, but he could bend the peg where they are hanging, so they would fall to the ground. He could then, with time and effort, have the floor move and push the keys over to him (creating a little mound that moves across the floor, pushing the keys). (This assumes that the floor is not made of wood...) 

Earth magic can \emph{never} create earth out of nothing. 

An entire science withing Earth magic is Sculpture, the art of creating items from earth. Proper Sculpting takes time and requires proper tools, but it can create impressive items. Sculpture often uses other elements as well, and frequently makes use of occult geometry. Many Vaimon weapons, armor, buildings and statues are Sculptured. 

With Nieur, Earth can cause matter to shatter or even disintegrate\index{Disintegration} (crumble to dust). This can be used on walls, shields and weapons, and even living flesh. Being Nieur-based, disintegration \emph{can} affect wood and other living matter. Disintegration requires touch; only the mightiest mages can disintegrate at a distance. 



\subsubsection{Fire}
Fire magic creates, controls or destroys heat, fire and light. Fire has a much sharper distinction between Iquin- and Nieur-based effects than the other three elements. Fire magic can be divided into two categories, `positive' and `negative'. `Positive' Fire magic is that which \emph{creates} fire, heat and light. It is Iquin-based. `Negative' Fire magic \emph{destroys} fire to create cold and darkness. It is Nieur-based. 

With Iquin and Fire it is possible to create a flame from nowhere, but it is not possible to sustain it. Fire can only be created in a flash; if it touches something flammable, it will keep burning, otherwise it will die. This means that you can hurl a fireball against an enemy, but not keep a little ball of fire in your hand to warm yourself. 

It is possible to create \emph{light} out of nowhere and maintain it, however, since light is pure energy and not (necessarily) a chemical reaction. Light still requires energy, however, so energy must be spent to maintain it. A minor mage will be able to maintain light equal to a torch with only little effort, whereas a very powerful mage can maintain the equivalent of a large bonfire. A strong mage can create light like sunlight, but only in a brief flash. By expending extra energy and channeling more Iquin, it is possible to create light that has the magical properties of sunlight\footnote{What magical properties \emph{does} sunlight have?}. 

Fire can also be used simply to create heat. No fuel source is needed, only energy. Heat disperses quickly, however, and is expensive to maintain over longer periods, but it can be used to dry your clothes, for example. Creating cold is similar. 

In combat, Fire can be used to hurl a ball or spray of fire. This is the Vaimon mage's archetypal attack. Having flammable materials at hand helps. It is effective, for instance, to douse your enemies with flammable oil and then breathe fire on them. 

Channeling Nieur, it is possible to snuff out fires and create cold. Directly chilling a person is difficult and generally ineffective, but you can combine it with Air to create a blast of freezing wind; this is an effective weapon. If water is available (like during rain or in a wet bog), you can channel Water to add ice shards to the attack for extra damage. 

Nieur can also create darkness. You can shroud a small area in a magical darkness that actively absorbs light. Such darkness is opaque; creatures within it cannot see anything, and creatures outside cannot see through. Darkness needs not be total blackness, however. There are many degrees of darkness. The brighter light there is, the harder it is to darken it (obviously). In daylight, an average mage can create total darkness in a small area (a cubic meter or two) or relative darkness in a larger area. In torchlight or moonlight, it is easier to create more darkness. Darkness will work against magical light. If a mage tries to darken another mage's light, the result will be a magical tug-of-war, with strength, skill and effort determining the victor. 



\subsubsection{Water}
Water magic is used to move water. It works on any kind of liquid matter, not only regular water. Water is also the main component of healing magic. 

Water magic can create currents in water. If swimming or sailing in water, you can have the water push you along, letting you move faster. You can also use it to make waves or calm them. Holding water in an unnatural position (like a column) is possible, but difficult, and requires constant maintenance and concentration. 

Using Water and Earth, you can make water solid.\index{Solid water} Solidified water looks and feels like ice, but it is not cold. This can be used to walk on water. As with liquefied earth (see under Earth magic), this is not permanent and must be maintained. 

Water magic can also be used to detect water. This is useful in a dry area. If water is found underground, you can use Water together with Earth to summon it to the surface (digging a crack, pulling the water up and closing the crack again below it). 

Some things that can \emph{not} be done with Water magic, although you might think they could: 

\begin{itemize}
  \item Water magic can \emph{not} create water out of nothing. 

  \item Water magic can \emph{not} be used to control or communicate with water-dwelling creatures. 

  \item Water magic can \emph{not} create ice or cold. To create cold, you need Fire magic and Nieur. 
\end{itemize}

\index{Healing!with Iquin-Nieur magic}
Magical healing, under the Iquin-Nieur theory, is strongly Iquin-based. It uses primarily Water, but also Air and Earth and a small amount of Fire. The explanation is that life energy is tied to blood and circulates in the body through the blood. So to influence the flow of life energy, you must influence the blood. 

Iquin healing can heal wounds and disease, mend broken bones and torn flesh. If treatment begins quickly, a severed limb can be reattached, but only if the original limb is available. If the limb is lost, it cannot be regrown. 

Using Nieur, Water magic can be used to `turn water to dust'. This transforms a quantity of water into a solid but powdery gray substance resembling dust or snow. The `dust' is not cold and will not melt under heat, but Iquin-Water can change it back to normal water. The dust is denser than water, so a body of water will collapse to a smaller volume (about one third of the original volume). Dust can be walked on if it is not too thick - 50 cm of dust will support a 60 kg Vaimon. Dust will not stick together and cannot be used to build anything. The transformation is temporary and must be maintained. The dust is solid matter, so if you need to manipulate it (other than to change it back), you need Earth magic. 

An especially nasty use of this is transforming blood to dust in another person's veins. This inflicts damage and can kill, but leaves no external wounds. Touch is required - only a master mage can turn blood to dust at a distance. 

Perhaps Nieur-Water can also turn water poisonous or acidic... 



\subsubsection{Other things}
Detecting other magic \\
Countering and dispelling other magic \\
Teleportation \\
Resurrection - requires Nieur \\
Immortality - believed to be possible with Nieur \\
Force field - requires item to focus? \\
Blade - create a sword-like thing out of the elements \\



\subsubsection{Sculpture}
\index{Sculpture}
Sculpture is the art of making magical items and infusing them with flows of Iquin and/or Nieur. 



\subsubsection{What cannot be done}
Among the things that are \emph{not} possible (or not believed to be possible) using Iquin-Nieur magic are:

\begin{enumerate}
	\item Creating illusions. 
	\item Mind control. 
	\item Paralysis. 
	\item Draining life energy. 
\end{enumerate}



