\chapter{Pronunciation Rules}
\label{pronunciation}

\section{Phonetics}
To explain pronunciation of names and words in my fictional languages, I present here a system of phonetic transcription. This system does not necessarily conform to any phonetic standard, but it's the one I am going to use. Whenever I use phonetics, I will enclose a word in sharp brackets, [like this]. 

\subsection{Consonants}
\begin{itemize}
	\item{} [C] will \emph{not} be used in phonetic writing, except in the combination [CH]. The sound of the C in English \emph{cool} is represented by [K]. The sound of the C in English \emph{race} is represented by [S]. 
	\item{} [CH] is like the CH in German \emph{ach} or \emph{buch}, or Scottish \emph{loch}. 
	\item{} [DH] is a 'soft D', like in Danish or Spanish. Also called 'voiced TH', like the TH in English \emph{these clothes}. 
	\item{} [G] is like the G in English \emph{go}. It is never 'softened' to a J (like in English \emph{gin}). This sound is represented by [DZH]. 
	\item{} [J] is like a J in Danish or German, or like the Y in English \emph{yes}. The sound of the J in English \emph{James} is represented by [DZH]. 
	\item{} [KH] is like the CH in German \emph{ich} or \emph{nicht}. It is similar to, but distinct from, [CH]. 
	\item{} [R] is a retroflex R, like in English \emph{race}. A phonetic [R] is never silent. 
	\item{} [RH] is a guttural R, like in Danish or German. 
	\item{} [RRH] is like [RH], but longer. This sound is rare and difficult to say, but the RR in German \emph{zerreissen} is sometimes pronounced like [RRH]. If that makes sense... 
	\item{} [RR] is a 'rolling R', like in Spanish \emph{se�ora}. Do not confuse this with [RRR]. 
	\item{} [RRR] is like [RR], but longer, like the RR in Spanish \emph{carro} or \emph{perro}.\footnote{I know [RRR] looks stupid, but I am running out of R's, OK?} 
	\item{} [S] is like the S in English \emph{sing}. It is \emph{never} voiced (for voiced S, see [Z]). 
	\item{} [SH] is like the SH in English \emph{shine} or \emph{fish}. 
	\item{} [TH] is like the TH in English \emph{thing}. It is never voiced (for voiced TH, see [DH]). 
	\item{} [TSH] is [T] followed by [SH]. The end result is the same as the CH in English \emph{charm} or the TCH in German \emph{quatsch}. 
	\item{} [Y], when used in phonetic writing, is always a vowel. See below. The sound of a consonantal Y is represented by [J]. 
	\item{} [Z] is a voiced S, like the Z in English \emph{zebra}. 
	\item{} [ZH] is a 'voiced SH', like the SI in English \emph{vision}. 
\end{itemize}

All other consonants are pronounced like they usually are in English. All consonant clusters, other than those given here, are pronounced like the consonants they are made of, following each other. So, for instance, [TCH] is [T] followed by [CH], and is not to be confused with the TCH in English \emph{witch} (this sound is represented by [TSH]). 

\subsection{Vowels and diphthongs}
\begin{itemize}
	\item{} [A] is a 'normal' A. It's like [AA], but shorter. Like a standard A in Spanish or German. 
	\item{} [�] is a 'flat' A, like the A in English \emph{hat} or {man}. Do not confuse this with the 'A umlaut' in German or Swedish. 
	\item{} [AA] is a 'dark' A, like the A in English \emph{father}, or in \emph{castle} if using British pronunciation. 
	\item{} [AI] can be considered a falling diphthong of [A] and [U], or [A] followed by the consonant [J]. The result is like the Y in English \emph{cry}. Do not confuse with [EI]. 
	\item{} [AU] is a long O, like the AW in English \emph{lawn} or the O in \emph{boring}. 
	\item{} [AW] can be considered a falling diphthong of [A] and [U], or [A] followed by the consonant [W]. The result is like the OW in English \emph{cow}. 
	\item{} [E] is like the E in English \emph{pet}. 
	\item{} [EE] is a long I, like the EE in English \emph{sleeve}. 
	\item{} [EI] can be considered a falling diphthong of [E] and [I], or [E] followed by the consonant [J]. The result is like the EY in English \emph{hey} or the AY in \emph{slay}. Do not confuse with [AI]. 
	\item{} [I] is a short I, like the I in English \emph{Nick}. 
	\item{} [IA] is a falling diphthong, like the EA in English \emph{ear} or \emph{fear}. Note that the rising diphthong of I and A is represented instead by [JA] (or [JAA], or whatever). 
	\item{} [O] is a short O, like the O in English \emph{hot}. 
	\item{} [�] is like the OH in German \emph{wohnen} or the O in Danish \emph{pose}. Similar to the O in English \emph{woman}. 
	\item{} [OO] is a long U, like the OO in English \emph{cool}. 
	\item{} [U] is a short U, like in German \emph{hund}. 
	\item{} [Y] is a vowel. It is like Y in German or Danish, or like � in German. 
	\item{} [YY] is like [Y], but longer, like the � in German \emph{gem�se}. 
\end{itemize}

\subsection{Syllables and stress}
Syllables are separated by a dash (-). 

The primary stressed syllable is capitalized. A syllable with secondary stress is preceded by a comma (,). 




