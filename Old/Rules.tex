\section{Rules}

\subsection{Alien creatures}
%Some creatures on Mith are \quo{alien}. 

\quo{Alien} is a generic term for creatures (and things) that are strange and different in a way that seems inherently monstrous and unnatural to \quo{normal} creatures. 

It is no secret that this concept is a blatant attempt to bring Cthulhu elements into the world of Mith. In the Cthulhu Mythos (the stories by H.P. Lovecraft and others), there are creatures and things so terrible that a man can be driven mad by fear and loathing at the sight of them, or merely from knowing (or suspecting) that they exist. Lovecraft's point was that these things are frightening, not because they came from some Hell or were otherwise evil, but because they are just so alien and incomprehensible to Man that their very existence shatters Man's illusions about a safe, benevolent Universe that he may hope to understand and accept. 

But Lovecraft's task was easier. The setting of his stories is 20th century RL, so he could throw all the fictional elements in the \quo{strange and horrible} category. Everything that does not exist in RL is alien and monstrous. My task is harder, for the setting of my world is wholly fictional. And having every creature scream and go babbling mad at the sight of any other fictional creature is not exactly what I am aiming for. 

Therefore, I have tried to categorize the denizens of Mith. Some of them are known to be natural and normal inhabitants of the world, and are not considered horrible. These include most of the humanoid races, most animals and certain monsters. Such creatures (such as \dragons) may be fierce, frightening, even terrifying, but even so, what they inspire is a rational fear for one's life and limbs.\footnote{In certain cases, the fear of something \quo{natural} can be just as irrational and just as scarring as the fear of something \quo{unnatural}. For instance, if you take a person with a severe phobia of spiders and lock him in a room with dozens of harmless spiders for a couple of hours, he may end up just as mad as the deep-sea swimmer who gazed into the eye of a \kraken. Fear, insanity and suffering remain psychological phenomena, not metaphysical.} 

\subsubsection{Horror effect}
\quo{Horror effect} is an informal estimate of how horrible a creature or thing is to encounter. 

\begin{description}
	\item[Slight:] Like an average inhabitant of Innsmouth: Repulsive and unwholesome, but not bad enough to be cause for serious alarm (or serious suspicion that something unnatural is going on). 
	\item[Minor:] 
	\item[Medium:] 
	\item[Major:] 
	\item[Extreme:] Like being cast ashore on R'lyeh and meeting Cthulhu. 
\end{description}




