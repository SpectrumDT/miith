\chapter{Older Story Ideas}
\section{Iashere, Ccuco and Glet}
Kochu Iashere is a female \meccaran, a warrior and adventurer of the Kochu tribe. Kochu is the most \squo{civilized} of the \meccaran{} tribes, having had many dealings with the Imetrium and absorbed some of Imetric culture. (The Kochu language is meant to resemble Chinese and Japanese.) 

Clictua is another \meccaran{} tribe, and along with Kochu one of the most powerful of the tribes. Clictua is relatively \squo{civilized} but more aggressive and warlike than the Kochu. What more? (The Clictua language is meant to resemble Nahuatl. Nahuatl is a Central American tongue spoken by the Aztecs and others.) 

...

Iashere had heard of the Clictua chieftain, Ccuco, but had not met her in person. She had assumed that the stories were exaggerated, but she could see now that they were not. Ccuco was indeed a behemoth woman, as tall as a \scatha{} and looking as strong as a [insert huge monster here]. As was evident from her markings, her totem animal was the Manticore. And indeed she looked as ferocious as one, Iashere thought. 

She bore a great bladed polearm polearm weapon of metal, of an exotic design that Iashere did not recognize. She doubted that it was of \meccaran{} origin. Ccuco also carried several daggers, some of bone, some of metal. One of the daggers had a serrated blade and looked particularly wicked. Iashere wondered how effective this knife would be in actual combat, or whether it was a ceremonial weapon... 

And then she noticed him, Ccuco's companion. It was a small male \meccaran, looking very insignificant next to the chieftain, who towered nearly twice his height. Yet at the same time there was something unsettling, something menacing about the man. Surely this must be Glet, Ccuco's mate. Iashere found it disturbing that she had not seen him at all before now; it was as if he had appeared out of nowhere. 



\section{The Grace of a \Bane}
The \bane{} moved in a soft, fluid manner. In another creature, Carzain considered, this might be considered \squo{graceful}. But there was nothing graceful, nothing beautiful, about the \Bane. Fluid it might be, but it was like the slime of a foetid bog. 



\section{Xarocchetsel's Sacrifice}
Rrhun'Xaroc was an acolyte in the church of Nerrhan-Koss. One day, he was summoned by High Priestess Nochur-Thettec-Kyrhess. She said to him: `Acolyte Xaroc, do you covet knowledge? Do you covet power?'

`Yes, your Excellency,' said Xaroc. `Above all else.' 

`Nerrhan-Koss can give it to you. This night I received a vision from our Master. He sees great promise in you, Xaroc. He is willing to bestow a great blessing upon you, and you alone. But first, you must make a sacrifice.' 

`Yes, your Excellency. What does our Master crave of me?' said Xaroc. 

Nochur-Thettec-Kyrhess answered, but in a voice that was clearly not her own. `Acolyte Rrhun'Xaroc, to be worthy of the blessing of Nerrhan-Koss, you must be cleansed. You must sever all early fetters and embrace only the essence of Nerrhan-Koss.' Though spoken by the High Priestess's mouth, the seem to come out of the endless, formless abysses beyond the Cosmos, and the chilling voice struck Xaroc with a primal dread. In the following instant, Xaroc desired only to flee far away from this terrible avatar and never again glance at the vast, bleak mountain spire which formed the Observatory Temple of Nerrhan-Koss. 

But Xaroc did not flee, for he knew that this must be the voice of his Master. And if Nerrhan-Koss deigned to speak directly to him, then it must mean that the occult, arcane wisdom he desired could finally be within his reach. So, with his limbs still cold with horror, he gathered all his courage and spoke: `I desire nothing more than to embrace you, Master. What do you command?' 

The terrible voice of Nerrhan-Koss spoke again: `To be cleansed, you must perform the deed that is utmost anathema to you. You must sacrifice all that you hold dear in this world. Only then will you be free to serve me. Only then will you be able to learn the dark secrets of the Cosmos. Are you ready, Rrhun'Xaroc?' 

With every word that the possessed Priestess spoke, Xaroc wished more and more to run far away... nay, more than that. He wished to kill himself here and now, that he may forget the echoes of that terrible voice, which he knew would haunt him till the end of his days. But great was his lust for power and knowledge, and in the end it overcame his dread, and he answered: `Yes, Master. I am ready.' 

Th possessed Priestess gazed intently upon him, then her face contorted into a strange, twisted smile. In the next instant, the mystic possession seemed to leave the High Priestess. She was stricken dumb and stood baffled for many moments, and when at last she spoke again, now in her own voice, it was slowly, as if she had not spoken a word in centuries: `This you must do: You have a wife, Leod'Fiher, whom you love, and who has borne you precious eggs. First, you must take your eggs, all three of them, to the Observatory Temple, and present them as a sacrifice to our Master. Place them upon the Altar and smash them, with your own bare claws, and let the worms devour their juices.' 

Seeing the expression of horror in the face of young Xaroc, Nochur-Thettec-Kyrhess smiled a cruel smile before continuing. `But that is not all. You must take your one true love, your wife, Leod'Fiher, and bring her here. You shall cast her into the great caverns below the Temple and imprison her there. There she shall remain, and in time, she will transform into an abomination. Your true love will become a hideous monster that will strike horror and disgust into the hearts of mortals, \dragon{} and humanoid alike. And she shall be damned to suffer, in grief and loathing over what she has become, while her mind slowly slips away and she becomes lost to pain and dementia.' 

`This is the sacrifice that our Master demands. What say you, Acolyte Xaroc?' 

Xaroc answered slowly: `But, your Excellency... you say I should bring my \emph{three} eggs. But my wife has lain four eggs? What of the last egg?' 

The High Priestess let out a cruel laughter. `I know that very well. A test of your devotion, Acylote. Had you said nothing, brought three eggs to the altar tomorrow and hidden the fourth, you would have been punished terribly for attempting to betray our Master.' 

After a long pause, she elaborated, as were it a trivial detail: `All four eggs are to be destroyed. So, Rrhun'Xaroc, will you do this?' 



\bc
\section{Daggerrain's Monologue}
\emph{Daggerrain} is a \banelord. The name is given to him by Rissit, because the \banez{} telepathic signature brought to Rissit's mind the image of thousand daggers and knives falling from the sky like rain.} 

The Voyagers were a race of mighty beings - gods, you would call them - who travelled across the Universe, seeding worlds they found with life. Many ages ago, they came Erebos and began to shape life there. Eventually, they created us, the \banes, the pinnacle of life on Erebos. 

But the Voyagers grew dissatisfied with us, their creations, and one day they abandoned us and left Erebos. (Or did the \banes{} rebel against them and drive them from Erebos? Yes... that's probably it...) 

And so, we ruled supreme on Erebos. But soon, our race began to decline. Without the Voyagers to maintain us, we failed and crumbled. And so the \banes{} grew less powerful, over thousands if not millions of years. 

Somehow we discovered some stuff... probably some lost Voyager vault of knowledge. At any rate, we discovered that the Voyagers had gone to Mith. Not once, but several times. Mith was important to them. But they had not been alone. Erebos had been barren when they came here, and even today, the world holds no life save the Voyagers' creations and their descendants. 

Not so Mith. Mith had her own children, her own indigenous life, and the greatest of them were the Kraken, the native overlords of Mith. Terrible creatures they were, and immortal. Few they were, numbering scarcely a dozen against the thousands of Voyagers, great and mighty in their own right. Yet the Kraken in their wrath proved more than a match for the invaders, and in every conflict the Kraken would conquer. With valour and fury the Voyagers would fight, and with despair, for the Kraken were immortal and wielded the primal power of Mith herself. Great Tiamat was their queen, the mightiest of all the children of Mith, and none could stand before her when she rose in fury. 

But the Kraken were indolent creatures, and they would often sleep, their entire race, for aeons at a time - millions, even hundreds of millions of years. And so, whenever the Kraken slept, the Voyagers would invade Mith, to populate the planet with myriad life forms of their devise. And every time, after an aeon, the Kraken would awake to vanquish the Voyagers, scour the world of their creatures and repopulate it with their own spawn. But inevitably, the Kraken would once more go dormant, and the Voyagers would return to begin their work anew. 

For the Voyagers seek ever to perfect life, and Mith is precious to them, for Mith is an anvil on which to forge and shape their creations, their works of art. And a most excellent anvil is Mith, for it... (does stuff). 

This and more we learned from the vaults of knowledge. We also learned the location of Mith amid the threads and flows of the Cosmos, and we learned the spells to travel here. The wise among us, our great \banekings, knew that here must lie secrets of the Voyagers' art, and the knowledge and power to save our people and return us to greatness. So we came. 

At the time we arrived on Mith, the Voyagers had fled the planet, for they knew that the Kraken would soon awaken. Indeed, one of the Kraken was already awake; in this day and age he is called Moloch. Great and powerful is Moloch, yet among the Kraken he is of the lowest order, and next to mighty Tiamat he is naught. What is his aim we cannot know, but when he awoke - a million years hence or more - he began not to destroy, but to create. From the sea he took primitive beings, creatures of the Voyagers, and reshaped them in his own design. He gave them strength of body and mind, and bestowed upon them the power of thought, and he made them his servitors. 

Millennia passed, and still Tiamat and her brethren slumbered, and even Moloch grew sleepy and fell dormant. But his spawn lived, and they grew, and they prospered, and they recalled their sire and offered him prayer and tribute. Hybrid children they were, born from the womb of the Voyagers' creation but fathered by a Kraken lord. Lowly they were, yet in this age they were the lords of Mith, and their built their cities in the deep oceans across the globe, and they were the Nagae. 
\end{comment}



\section{Carzain's story}
There is this guy, Carzain (see section \ref{Carzain}). It is discovered that the blood of the ancient Vaimon Emperors runs strong in his veins. Indeed, Carzain is the \emph{Scion of the Emperors}, whose coming is foretold in prophecies. He is a very talented Iquin-Nieur mage. 
%He is a very talented sorceror, with great power in all four elements. 

He grows up in the town of \Redglen{} in the country of Pelidor in southern Belkade. 

%He grows up in a town in \Belkade, north of Martinum. His mother is Redcor and runs a book store in their town. %His father is Geican and originally fled from Geican persecution in Redcor-controlled lands. His father is Geican and used to work and study at a Geican academy of magic which was destroyed in a war. Both of them are mages and scholars, and young Carzain learns much magic from them. 

Carzain befriends Ilcas Northstar, a \scathaese{} \nycaneer{} and a soldier/hero in the Imetric army. He learns some combat skills from him. He also befriends the scholar Curiet Serpentin, likewise a \scatha{} from the Imetrium. 

%What is Carzain's early history? Is he a kid that doesn't know much about the real world and fighting and stuff, or is he already a tough adventurer? 

\subsection{More}

\subsubsection{Stuff about him}
%Eventually Carzain becomes Emperor of a large realm. 
Carzain is a villain, but an honourable villain and an all-round awesome character. 

One of the themes is that good does not exist. Carzain meets all sorts of civilizations and orginizations who claim that they fight for what is good, but all of them end up being brutal and totalitarian, oppressing people with differing views. So Carzain concludes that any attempt to be \squo{good} is futile. In his view, evil is better than good, because evil admits to be evil, while good is hypocritical. 

Carzain gets to read the Book of Nom. Being the awesome character that he is, he survives. But it does strengthen his belief that good is futile. 

Carzain travels all over the world and learns all sorts of magic and stuff. The story follows him, and also other things. In the end he returns to Belkade to crown himself Emperor. 

The symbol that Carzain chooses for himself is the Peryton. He has a Peryton, a hind (female) named Venom, sent by \Belzir{} to serve as his mount. 



\subsubsection{Other characters and things}
A main plot is that Rissit is attempting to expand his borders to the north and conquer Belkade. To this end, he sends three Ashenoch: \Narkiza{}, Geldashad and \Dzerezdin{}. 

Rissit also sends \banes. 

%In one of the Belkadian kingdoms the king is secretly an evil Shadow \dragon{}. Is he allied with Rissit? He might be... 

I also have to use the Quaerin for something intelligent. 

And I need to have this all connected with the story of the \dragonkings{}. 



\newpage
\section{\Dragonkings{}}
Tentocoth is the \dragonking{} of Irokas. His daughter Thiencaste is crown princess. His other daughter, Criocas, covets the throne. Thiencaste is dissatisfied with the evil society of Irokas, so she travels the world. Maybe she meets Carzain? 

Anyway, Criocas and Nisgzarchief (her husband) also travel the world. Maybe it is customary for young \dragons{}, especially royalty, to travel and learn stuff. Maybe it is not. But these do. 

Thiencaste willingly relinquishes her throne. She agrees to take part in a black magic ritual which destroys her name and House. She is thus cast out of House Irokas and cannot inherit the throne, which passes to Criocas. She also cannot use the name Thiencaste and must take a new one. She moves out into Belkade and attempts to do something good. Perhaps she falls in love with a humanoid. Perhaps she joins the Imetrium. 

What is her relationship with Carzain? She must have one... 

There is also Ishnaruchyfir. He is an awesome character, so I need to put him to some use. Here is an idea: 

\paragraph{The Ancient \Dragon{} Horn}
Irokas is threatened by some menace. Perhaps it has to do with the rebellion. Yes, I guess it does. A rebellion is launched against House Irokas, led by the rebel Queen. 



%\section{Characters on Mith}

\subsection{Morals and Alignment}

\subsubsection{The concepts of good and evil}
In my world, moral concepts such as good and evil are \emph{not} objective. These terms cannot be defined objectively; they remain cultural phenomena and personal opinions. Persons and civilizations can only be properly described by their values, ideals and opinions, not by some archetypal `aligment'. This represents my actual view of moral philoshopy. 

% Unlike in some RPGs, such as Dungeons and Dragons and the Palladium RPGs, people are not defined by an `alignment'. This represents my own world view. Concepts of good and evil cannot be defined objectively; they remain cultural phenomena and personal opinions. Therefore, persons and civilizations can only be properly described by their values, ideals and opinions, not by some archetypal `aligment'. 

This being said, I do use the concepts of good and evil. I just need to stress that when I use these terms, they represent my own subjective moral judgement. For instance, I tend to describe the Imetrium as `good', but others may find that such a centralistic theocracy with communist leanings is by no means `good'. 

At any rate, good and evil are not physical or metaphysical concepts in any sense. Therefore, it is not possible, for example, to cast a spell that damages evil creatures only. It is also not possible to cast a spell that harms only creatures that adhere to a particular ideology or religion, since `adhering' to a particular view is not well-defined and detectable.\footnote{Even if it can be considered objective in some sense, to design a spell that would detect it is completely unrealistic at TL3, since such a spell would have to measure brainwaves, which is very poorly understood even at TL7. Perhaps at high TLs this will be possible, but in a medieval setting, forget it.}
It is, however, possible to have a spell affect, for instance, only those who are under the blessing of Rissit Nechsain, because such a blessing is an objective and detectable metaphysical phenomenon.\footnote{A simple spell of this type would only be able to detect a certain set of `standard blessings' that Rissit might use. A more advanced spell might be able to detect any blessing bestowed by Rissit by detecting his magical `signature'. It might be possible, however, for Rissit (or his priest) to `mask' his magical signature in order to fool such a spell.}

\subsubsection{Determining morals and personality}
So, what determines moral outlook and `alignment'? What determines whether a given person is good or evil? 

Well, I tend to support the theory of free will. This means that any living creature is able to make (and responsible for) its own choices. The mind is affected by external and internal forces, but only to a limited degree. The person always has the final say. Only in extreme situations (and depending on the person's willpower) may very strong emotions (such as fear) be able to overcome the rational mind and take control of the body. 



\begin{comment}
This section will be heavily opinionated and influenced by my own view of moral philosophy. If you, as a GM, have a different world view, you may discount my rantings here. But this is the world view on which I have based my world. 

A basic concept is the axiom of free will: Any living creature has a free will. The mind is non-deterministic. The choices that a creature makes are not (entirely) controlled by outside forces and cannot be predicted (with certainty). 

The mind is affected by external and internal forces, however. The mind is subject to causality, but not entirely. Such causal forces affect the statistical likelihood that a certain person, in a certain situation, will do a certain thing, but only statistically. The individual person always retains his free will to choose between alternatives. 

So, what does all this have to do with anything? Well, it has to do with how a personality is formed, and how a person becomes good or evil. 

In my world, the forming of a person's personality is determined by three factors: Genetic, social and personal. 
\end{comment}




But a personality is not shaped by the person alone. In my system, there are three factors to consider regarding the development of a person's mind. 



\begin{itemize}
  \item \textbf{The genetic factor} is determined by the brain in which the mind resides. Different species are evolutionally conditioned to think in a particular way, to have certain insticts and intuitions. This affects a person's choices, including moral choices. For example, Scathae are by nature social creatures, so a Scatha would be genetically inclined to embrace `social' ideologies, such as utilitarianism or elitism\footnote{`Elitism' is to be understood as the view that a certain group is more important, more worthy, than others, and has a natural right to rule. The Rissitic religion is elitist. An example of an elitist ideology in RL is nazism. I generally consider elistism to be evil.}. Dragons, on the other hand, are more solitary by nature, and as such are more likely to embrace individualistic world views such as anarchism or nihilism. 
  \item \textbf{The social factor} is how a person is raised and educated. It is determined by the surrounding society and the people in charge of raising the person (parents, school, church). For instance, a person born and raised in the Imetrium will be indoctrinated to think in terms of Imetric values, such as the individual being prepared to make sacrifices for the greater good of the community. Such a person is less inclined to develop an anarchistic, independent attitude. Social indoctrination sometimes backfires, however, creating rebels who purposely reject the values of their society. 
  \item \textbf{The personal factor} is determined by the individual. The personal factor can be further divided into two: 
  \begin{itemize}
    \item \textbf{Free will:} This constitutes the conscious choices that the person makes in his life. 
    \item \textbf{Residual personality:} Keep in mind that in my world, an infant is not a blank slate, since it is an incarnation of a soul that has lived before. The soul will carry over some degree of personality from his previous life. This `residual personality', in turn, is determined by genetic, social and personal factors from the person's previous life. 
  \end{itemize} 
\end{itemize} 



I will not go into details about the relative importance of the three factors. I will generally assume that the personal factor is stronger than the genetic one, ie., that any intelligent creature has the ability to understand and adopt any moral code, even one far removed from the typical behaviour of the species.\footnote{This may not apply to creatures that are very `monstrous' and `alien'.} This means, for instance, that even a Balrog (normally very evil) may choose to be sincerely gentle and good. This just happens very rarely. 





%\input{DarkQueensResurrection}
