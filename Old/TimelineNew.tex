\section{Timeline}



\newenvironment{epoch}[1]{\subsubsection{#1}\begin{tabular}{rp{10cm}}}{\end{tabular}}
\newcommand{\bepo}{\begin{epoch}}
\newcommand{\eepo}{\end{epoch}}
% \yearic is `year IC', as in, a year number in the Imperial Calendar 
% (reckoned from the founding of the Vaimon Empire)
\newcommand{\yic}[1]{\arabic{#1} IC}




%%% Here follows the definitions of a bunch of historical constants. 
% A `hiscb' is a `basic historic constant'. Its value is given as a constant. 
% A `hisc' is a `historic constant'. It's value is given as the value of a base hisc plus an offset. 
% NB: The base hisc is given as a counter name, without the `\value' tag. (It needs not be a hiscb, it can just be a hisc.) 
\newcommand{\hiscb}[2]{\newcounter{#1}\setcounter{#1}{#2}}
\newcommand{\hisc}[3]{\newcounter{#1}\setcounter{#1}{\value{#2}}\addtocounter{#1}{#3}}
% `birthandage' specifies a person's name (`foo') birth year and his age at death. 
% It also introduces two hisc's: `foo birth' and `foo death'. 
% Birth year is given as a base hisc and an offset. 
% Death year is computed as birth year + age. 
\newcommand{\birthandage}[4]{\hisc{#1 birth}{#2}{#3}\hisc{#1 death}{#2}{#3}}

% The 0th year of the Imperial Calendar, the founding of the Vaimon Empire
\hiscb{IC}{1000}
\newcommand{\ic}{\value{IC}}

% Cordos Vaimon
\birthandage{Cordos}{IC}{-50}{80}
% Belandos Vaimon II (Cordos' father)
\birthandage{Belandos}{Cordos birth}{-25}{65}
% Faegos Vaimon (Cordos' grandfather)
\birthandage{Faegos}{Belandos birth}{-20}{58}
% Silqua Delain
\birthandage{Silqua}{IC}{-35}{30}
%\hisc{Silqua birth}{\value{IC}}{-35}
%\hisc{Silqua death}{\value{Silqua birth}}{30}
% Belzir

% The Darkfall, the fall of the Vaimon Empire. 
\hisc{Darkfall}{IC}{1965} 
% The wedding of Silqua Delain and Cordos Vaimon. 
\hisc{Silqua wedding}{Silqua birth}{18}



\begin{comment}
\end{comment}