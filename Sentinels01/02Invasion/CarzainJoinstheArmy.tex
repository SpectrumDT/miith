\bookchapter{War is Coming}

\begin{comment}
\subsection{Declaration of War}
\end{comment}

\stamp
  {\dateCarzainJoinsTheArmy}
  {Town square, \Redglen{}, Pelidor}

%\dateplace{Runger war}{2nd}{\Razilah}{Town square, \Redglen{}, Pelidor}
%
\ta{War is coming!} the messenger shouted as he rode through the town. \ta{War! The war is coming!}

In his wake, frightened husbands hugged their wives and mothers held close their children, as if afraid that the war would be coming this instant. 

The messenger had now reached the town square and was reining in. 

\ta{What do you mean \quo{war is coming}?} asked a listener, a \sphyle{}\dash{}Mesna, the retired old tailor. 

\ta{King Morgan of Runger has declared war on Pelidor! Even as we speak he marches on the northeastern border with an army of thousands of men.} 

\ta{The northeastern border? Hah!} shouted another, a young \dax. \ta{That is as far from \Redglen{} as one can get. What do we care? Why should this affect us?}

\ta{Fool!} Mesna snapped. \ta{Do you know nothing? Morgan Runger is a cruel, evil man, and his hate of Pelidor is well known. He is not here to seize a few acres of the Nerim's south bank. He comes to conquer Pelidor. All of Pelidor!}

\ta{There is no way he could do that!} a young man retorted. \ta{The Duchess will send an army and stop him.} 

\ta{Yes, she will,} the old \scatha{} agreed. \ta{And that army will be you, and you,}\dash{}she pointed out the two young men who had spoken\dash{}\ta{and other young men and women like you, drafted from all parts of the country.} At this, the \human{} youth gulped audibly, and the \scatha{} took an intense interest in the cobblestones. \tho{Yes,} thought the old woman, \tho{I fear this war is another of the big ones. One that will leave the kingdom depopulated, sobbing to see the young soldiers depart, and wailing to see only half of them return.} Mesna herself had fought in the last great war in Pelidor, twenty years ago, and had survived only through luck. 

Many people clustered together around the messenger, asking questions, but Mesna had heard enough. No doubt her daughter would be conscripted and sent to fight, and Mesna wanted to spend this day with her, this day that might be their last together. 



\begin{comment}
\subsection{Upper Citizenry}
\end{comment}

\placestamp{Town hall, \Redglen{}}
%
\ta{... so being one of the few true scholars in \Redglen{}\dash{}indeed, one of the few literates\dash{}it follows that I am clearly indespensible to the town. And it goes without saying that the same applies to my wife, her being \Redglen{}'s leading healer and all\dash{}and Redcor-born, I remind you.} As Nishain rambled on the clerk was left nodding, no longer even striving to get a word in. \ta{Needless to say, it falls to our son to carry on this legacy and serve the community in the decades to come. So, you see, the \Shireyo{} household, in its entirety, obviously represents an invaluable resource to \Redglen{}, and as such must be counted among the Upper Citizenry.} 

\ta{Um... yes, I... see your point, \Mr{} \Shireyo{},} the clerk stammered. 

\ta{%
  Jolly good. Now, the fact that we do not appear listed as such on the official records is obviously but an unfortunate error, the result of the town officials' apparently failing to taking sufficient note of our arrival and our settling here. Of course, we cannot have such errors in the town records. Having done my civil duty by bringing the issue to your attention, I trust you will see to it that the matter is swiftly rectified.} 
\tho{What was his name? Oh, right. \Symeon.}
\ta{Am I right, \Mr{} \Symeon{}?} 
\tho{Calling people by name usually helps.} 

\ta{Yes... uhh... yes, I will.}

\ta{%
  Splendid! After all, we must all do ours to keep society up and running, must we not?} 
He peered over the desk at \Symeon{}'s papers. 
\ta{%
  Did you get the names down correctly? My wife's name is \quo{\Deracille}, spelled with a... yes, that is correct.} 

Nishain got up. 
\ta{%
  Well, you seem to have it all well in hand, so I will leave you to your work. Good to do business with you, and good day, \Mr{} \Symeon{}.}

\ta{Um... yes... likewise, \MrShireyo,} \Symeon{} managed. 
\ta{Yes, I will get it, uh, fixed.}

Nishain left the town hall with a satisfied smile. 
\tho{That should take care of matters.} 
%When he had heard the news of war, he had immediately recognized the danger and known that he must take steps to fend it off. 
The Upper Citizenry comprised the upper tiers of the middle class: merchants, scholars and craftsmen considered particularly important to the community. 
He and his wife had never particularly cared about whether or not they officially qualified for this status, but when he had heard the news of war, he had quickly recognized that said membership carried one very tangible privilege: 
The Upper Citizenry were exempt from military service. 
Realizing this, he had immediately known that he must take steps to get his family \quo{promoted}. 
Nishain had no wish to be drafted and forced to fight in the army, and he \emph{certainly} did not want to see his son sent off to war. 
No matter how well Carzain might have handled himself a year ago in Heropond. 

Right from his arrival, Nishain had deftly fast-talked the town hall clerk, plowing through with serious arguments and banalities alike and rarely letting the other man say a word. His stance had been clear\dash{}that he was right as a matter of course and no one would dream of contradicting him\dash{}and in situations such as this, an air of confidence often counted for more than did arguments. Nishain was convinced that he had left \Mr{} \Symeon{} remembering that his proposal had been completely reasonable. 

\tho{%
  Nicely done, Nishain. 
  Problem solved. 
  
  One would think that if the dukes were clever, they would draft all mages they could find into their \ishroth. 
  Huh. 
  I wonder why they don't. 
  I suppose there are political reasons. 
  Maybe the dukes are afraid to anger the mages.
  Or something. 
  Anyway, it's lucky for us. 
  No army service for the \Shireyo s.} 

Thinking back to that day in Heropond, Nishain could not help but shudder a little. 
He had watched as Carzain... transformed. 
His eyes, his stance, his voice... all had seemed to belong to another man. 
And he had proceeded to kill four people, each in a more brutal manner than the next. 

None of his Vaimon learning had enabled him to explain the eerie possession. 
It had not happened again, and there had been no lasting effects, Carzain insisted. 
Nishain had come to the conclusion that it must have been some \qliphah{} or other evil spirit dwelling in the haunted Heropond Forest that had crept into his son's body in a moment of distress. 
\tho{%
  Yes, that must be it.
  It's the forest that was evil. 
  Carzain was just a random victim.} 

But that was not all. 
After Carzain had regained his senses, his behaviour had been disturbing. 
He had felt no remorse over having killed the bandits with horrible magic. 
Rather, he felt pride, even joy. 
This development scared Nishain. 
Granted, those people had been mutineering scum and one might argue that they deserved death, but even righteous death should not be dealt out with a smile, Nishain felt. 

He told himself that it was probably the after-effects of the \Izion, the manifestation of Justice, %whom Carzain had employed a lot. 
of whose power Carzain had drawn deep that day. 

Nishain did not trust the \sephiroth. 
The Iquinians claimed that the \sephiroth{} were good, whereas the \qliphoth{} were evil fiends that twisted the mind. 
But the \sephiroth{} twisted the mind as well, Nishain thought. 
They forcibly imprinted their so-called virtues in people's brains. 
And were they truly virtues? 
The brand of Justice that made a man kill with a smile... was that a virtue? 
Nishain had his doubts. 

Any sensible Vaimon\dash at least, any sensible Geican\dash knew that magic was a delicate balance between the rigid \sephiroth{} and the chaotic \qliphoth. 
His clan did not believe that the \sephiroth{} were good, nor the \qliphoth{} evil. 
Both sides were inherently neutral, and both sides had their dangers. 
Draw too much on either side and you invited insanity. 
The trick was to balance the opposing forces. 
Everyone knew, though, that the \qliphoth{} of the Midnight Circle were the darkest and most dangerous of all, and most knew well enough to steer clear of them. 
That way lay madness. 

\tho{%
  That must be it. 
  It was the \sephirah. 
  And the evil forest. 
  They conspired to drive Carzain temporarily out of his mind. 
  That was it. 
  That is all there is to it.}

He did not wish to contemplate the alternative: 
That this viciousness was a part of his son, 
a facet of Carzain's personality that he kept hidden.
A secret darkness he carried with him. 
Because if that were true, what would happen then, if that viciousness was drawn forth again and allowed to fester and grow?
What would happen if he were put on a battlefield? 
Would he go mad?

\tho{%
  No. 
  That's nonsense. 
  There's nothing wrong with Carzain. 
  He's as sane as I, or \Roanne{}, or any other. 
  It was the forest. 
  And the \sephirah. 
  
  Besides, nothing bad is going to happen to him. 
  He is certainly not going to be a soldier. 
  I have made sure of that.
  
  No one in their right mind would want to fight...}






\begin{comment}
\subsection{I want to fight}
\end{comment}

\placestamp{Army barracks, \Redglen{}}
%
A hand was slapped down onto the table. 
\talk{I want to fight.} 

The officer on duty, a middle-aged \dax, looked up. 
The speaker was a young man, a bit tall tall but light of build. 
Skin a bit darker than most. 
Clean-shaven, but with long, black hair falling to his shoulders. 
A look of determination was upon his face. 

\talk{I wish to join the army}, the \human{} repeated. 

The officer lazily licked one eye, wondering. 
%Given the sentiments of the populace regarding Duchess \Iakis, 
%Volunteers for the war were the last things he would expect. 
With the draft in process, he had not expected additional commoners to volunteer for the war. 
\talk{Um... yes,} he mumbled. 
\talk{Your name?}

\talk{\CarzainDeracilleShireyo{}.} 

%The officer gestured to the scribe, a \sphyle{} of his age. She recorded the name on her scroll. 
\talk{Have you any previous soldiering experience?} asked the officer, now composed. 

\talk{No. But I am a mage.}

This caused the officer's tongue to involuntarily dart out and lick first one eye, then the other. %The scribe did the same. 
\talk{A what?} said the officer. 

\talk{%
  My father and mother are Vaimons. 
  And I am trained in the use of Vaimon magic.}

\talk{Are you... are you of the church? Are you Redcor? Or \Telcra?}

Carzain gave a wicked smile. 
\talk{Neither. I practice \emph{black} magic.}

The officer fidgeted. 
\ta{%
  Um... I see. Come, uh... come with me. 
  You will want to talk to the Captain.}

Carzain followed the officer out of the door and into the training yard. Dozens of recruits were there, marching, running or practicing with spears, guns and other weapons. He was led to near the centre of the yard, where there stood a large \dax, long and bent with age, Tassian, with scales of ultramarine and red epaulets of rank on his shoulders. 

\ta{Captain Gemadon,} said the officer. 
\ta{I bring here a new recuit. He claims to be a mage.} 

Gemadon turned. \ta{A mage?} He licked his left eye, studying Carzain. \ta{That would be you?}

\ta{Yes. I am Carzain \Shireyo, son of Nishain \Shireyo, Vaimon mage.} 

\ta{Hm. Very good. Thank you, Sergeant.} 
The officer saluted and departed, apparently relieved to be rid of the company of the \quo{black sorcerer}. 
\ta{Vaimon, you say? Has the church thrown in their support?}

\ta{%
  I am not of the church. 
  My father is Nishain \Shireyo{} of Clan Geican, and my mother is \Roanne{} \Deracille{}, previously a \Soror{} of Clan Redcor. 
  I have learned the art from them, but I am unaffilliated with any clan or church.}

\ta{So a rogue Vaimon.}

Carzain smiled. \ta{A rogue, yes.}

\ta{And you wish to join the army? Why?}

%Carzain smiled. \ta{To see the world. Meet interesting people. And kill them.}
Carzain gave the captain a dubious frown. 
\tho{%
  Why does he ask me this? Surely he, an experienced soldier and officer, must know what draws people to war.} 
\ta{%
  For the adventure. For the glory. For the spoils.} 
He almost said \quo{to be a man}, but somehow that sounded comical to his ears. 

%The captain gave a forced smile. \ta{Seriously.}
Gemadon's mouth twitched, as if about to stick out his tongue and lick his eye. The look he gave Carzain was doubting, almost sad. \ta{Are you sure? Are those things really enough to make you long for war?}

Carzain eyed at him for a moment. \ta{Yes. Yes, I am. Why this interrogation? Why do I have to explain my reasons to you? I was under the impression that you were badly in need of soldiers.} 

\ta{%
  You should be careful with that tongue of yours, \Shireyo, when addressing a superior officer.}

\ta{You are not my superior,} said Carzain. Gemadon's frown deepened at this. 
\ta{%
  Not yet. You will be that only when I am formally enlisted. Tell me, if you please, what privileges will I receive as an army mage?} 

Gemadon barked a short, gruff laugh. \ta{You have a high opinion of yourself, \Shireyo, coming here and making demands.}

\ta{%
  Of course. My father taught me to always, in a bargaining situation, be aware of who needs whom more. Being of the Upper Citizenry and thus impervious to the draft, my impression is that the Duchess needs me more than I need her.}

Gemadon's scowl deepened again, but he said nothing. 
\ta{%
  Negotiating these terms is not my place, as you will not be serving under me\dash{}you will be sent to the \ishrah{} in Malcur. But to my knowledge, as a mage you will be provided with at least a suit of armour, a sword and a \relc.} 

\ta{And comfortable housing and food, I assume.}

\ta{%
  Yes, yes. Is there anything else you want? Perhaps a girl sent to your bed every night?}

\ta{Yes, thank you, that would be very nice.} 
The captain was clearly tiring of him, but Carzain's daring had been bolstered after he learned that he would not be placed under Gemadon, and thus was unlikely to reap any punishment for this audacity.

Gemadon gave him a chilly stare, not appreciating the joke. After a pause he said: 
\ta{%
  If you enlist, you will leave for Malcur at daybreak tomorrow. Now make up your mind. Do you sign up?}

%Carzain paused, not wanting to appear to eager\dash{}another piece of his father's advice: \tho{} 
Carzain paused for a short while before replying: 
\ta{Yes, I will sign up.} 

\ta{Good. Report to Lieutenant Nimloc, the scribe.} 
He turned away, indicating that the conversation was over. 

Entering the building that Gamelon's thumb had indicated, Carzain found an old \sphyle{} seated at a desk and surrounded by scrolls, books and writing aids. \ta{Are you Lieutenant Nimloc?}

\ta{What do you want?} she asked, not lifting her eyes. 

\ta{I want to enlist.}

\ta{What company?} 

\ta{The \ishrah{} in Malcur.}

This caused Nimloc to raise her eyes and study him. \ta{\Ishrah? Oh. A wizard. I assume you are literate, then. Saves me the trouble.} She handed him a scroll and a feather pen. \ta{Sign your name here.} He did as instructed, noting without surprise that the vast majority of names on the list had been entered with the same hand\dash{}Nimloc's, evidently\dash{}with the soldiers themselves merely signing with a cross. 

She took back the scroll and read. 
\ta{%
  Hm. \CarzainDeracilleShireyo. 
  You will be sent to Malcur to join up with the rest of the \ishrah{}. 
  Say your farewells in \Redglen, then return at dawn tomorrow with a minimum of personal belongings\dash{}no more than you can easily carry. 
  A steed will be provided to take you to the capital. 
  You leave at sunrise, so do not be late.} 
She met his eyes again. 
\ta{Welcome to the Pelidorian army, \MrShireyo.}



\begin{comment}
\subsection{Cordos Vaimon reborn}
\end{comment}
\begin{comment}
\placestamp{The streets of \Redglen}
%
%
%(Carzain is walking in \Redglen{} with the Goldsmith boys, Leopold and Frederick. Leopold is his friend from \Redglen{})
%
\ta{You did \emph{what}}? Leopold Goldsmith regarded Carzain with wide eyes. 

Frederick Goldsmith laughed. \ta{You joined the army?} He slapped Carzain's back. \ta{Delightful! That will make us soldier buddies.}

\ta{What on Mith possessed you do to that, Carzain?} asked Leopold, still incredulous. 

\ta{I thought you of all people should know, Leopold,} said Carzain. \ta{Because \Redglen{} is a dump!}

Out of the corner of his ear he heard a low voice: \ta{Look, there's Ladyboy and his lady-friends.} He threw a glance in the voice's direction and was annoyed, if not surprised, to see Gregory Blacksmith and three of his lackeys, Owen, Niclas and Bernd. It was Owen who had spoken. 

\ta{Oh, look,} said Carzain, facing Gregory and the others and loud enough for everyone to hear, \ta{it's the All Boys' Club.}

The Goldsmiths both turned their heads to look. \ta{Carzain, is it wise to challenge them?} said Leopold softly. \ta{They are bigger than us, and they outnumber us.}

\ta{You should know them better than this, Leopold. They are worthless; they won't try to attack us unless they outnumber us at least two to one. Besides, it's always better to call them out than to cower.}

The two groups were standing still now, facing each other. Leopold looked skittish, but Frederick cut an imposing figure, Tiger insignia on his shoulders and a sword at his hip. Niclas sneered, Bernd figdeted, and Owen gave them his usual glare, his entire face radiating stupidity. Gregory, the unchallenged leader of the pack, drew himself up to his impressive height and moved his bulk up to stand little more than a yard from Carzain. The Vaimon was no short man, but even at this distance the blacksmith's huge son loomed over him, almost like the Imetric mulgron.

\ta{Well, well,} said the big smith, \ta{if it isn't Ladyboy and his lady-friends.} 

\ta{My, my, such eloquence. Such inventiveness. You must be proud, Owen, to have your husband imitate your words.} 

\ta{Husband?} Owen began. \ta{He's not my...} A glance and a grunt from Gregory cut him short. Heat was rising in the smith's face.

This scenery caused Leopold to forget his fear and burst out a laugh. Gregory fixed red eyes on him and growled. The goldsmith backed away a step, but Frederick was quick to move to his brother's side. He did not touch his sword, but positioned himself with his left side turned slightly forward, so that his body language emphasized the weapon. \tho{A subtle but poignant message.} Carzain regarded the soldier with admiration. It had been a long while since he had last met him; Frederick had last visited \Redglen{} four years ago to announce to his proud family his admission into the Tigers, the Belkadian warrior elite. At that time he had been in the army a year and seen battle a few times, but he was still green. \tho{Now look at him. A true veteran.}

\ta{Look at you, Ladyboy,} snarled Gregory. 
\ta{You still look like a girl! That hair and your... your...}

\ta{... and my androgynous charisma, yes, yes, I know,} Carzain interjected, emboldened by the Tiger's support. 
\ta{I know that you are attracted to me, but unlike you and the All Boys' Club,}\dash{}he gestured at Gregory's lackeys\dash{}\ta{I don't fuck men. 
Now, if you will excuse us. 
You know, unlike you, I have actual lady-friends to visit.} 
He turned to walk away. 

\ta{Don't run from me, Ladyboy!} yelled Gregory.

\ta{Or you will do what?} said Frederick, his tone imperious. 

The smith had no answer to this and could only glare with bared teeth as Carzain and the Goldsmiths walked past. 

\ta{You took quite a chance there, Carzain,} said Frederick once they were well away. 

\ta{%
  No, I didn't. 
  I have had Gregory as an enemy all my life. 
  I know when he is a threat and when he is not. 
  He only attacks if his gang greatly outnumbers yours.} 
He paused, his masculine pride making him hesitant to reveal too much of what he was thinking. 
But at last he said: 
\ta{%
  You helped. 
  You did a good job of intimidating them. 
  Had it been just me and Leopold and a random third guy, he might well have jumped at me at that last remark of mine.} 

A year and a half his senior, Gregory had been Carzain's childhood enemy as long as he could remember, bullying and picking on him until he grew old enough to outmaneouvre the brute. The name \quo{Ladyboy} had appeared when he started wearing his hair long, but encouraged by his father, he had kept the hair and merely laughed at the name. By his late teens, Carzain\dash{}unlike Gregory and his gang\dash{}was becoming highly successful with women, which had served to strengthen his confidence, and since the brutes displayed no such skill with the ladies, he had mercilessly used this fact to mock them back. The term \quo{All Boys' Club}, for example, was a subtle accusation that Gregory and his friends had turned to homosexuality from lack of female company. Not true, of course, but claims of homosexuality were so delightfully effective taunts, and Carzain aimed to hit his enemies where it hurt, and hard. 

These thoughts of hate were filling Carzain's head\dash he was prone to such things\dash when along came a distraction that made him smile. 

%Then along came a distraction from these thoughts of hate\dash Carzain was prone to such things\dash.

\ta{Haha, look, there's my boy,} he said. Frederick looked in the direction he indicated to see a pretty, young woman with a boy in tow, a toddler of two or three years with black hair. 

\ta{Who is that?} asked Frederick. 

%Leopold laughed knowingly as Carzain responded: 
\ta{That's Charles, Edgar the Wood Carver's boy,} said Carzain. Leopold laughed knowingly beside him. 

There was a pause. Frederick studied his friends, trying to read their smiles. \ta{And?} he said at length. \ta{Why do you call him \quo{my boy}?}

\ta{%
  The wood carver is a fat and dreary man, but his wife is a cute, little thing. 
  I was shagging her for a while, a conspicuous nine months before little Charles was born. 
  Add his voluminous black hair and handsome features and put two and two together.} 
His smile was self-satisfied, smug, and when Leopold laughed again, Frederick could not help but join in. 

\ta{But we were interrupted,} remarked Leopold. 
\ta{Carzain, you never explained why you wanted to go to war.}

\ta{%
  Is it not obvious? As I said before, \Redglen{} is a dump. There is nothing here, nothing interesting. Sure, it is enough for ordinary people\dash no offense intended, Leopold. I know your craft and art is important to you, but you know me. I crave more. I've always craved more. I am a Vaimon, by the \Qliphoth! I mean, look at my parents! Town scholars and healers!} 

\ta{That is a respected occupation, is it not?} asked Frederick. 

Carzain's smile was agonized. 
\ta{%
  Respected, yes. 
  But I do not want to be respected. 
  I want greatness. 
  I want to be a hero! 
  I do not want to become like my parents. 
  I want glory, like the Vaimons of the old Empire.} 

\ta{Hahaha,} Leopold burst out. 
\ta{I've heard this talk of heroism before, but I didn't think you were serious.}

\ta{So no chance of you joining us on the field?}

\ta{%
  \Sephiroth, no. 
  No chance of me running off to get myself killed. 
  Besides, with Frederick already out fighting, someone has to carry on the goldsmithing. 
  You wouldn't want all the pretty ladies to go in want of jewelry, now, would you?}

\ta{Nyeh, I don't know,} said Carzain. 
\ta{If I have my way they'll quickly be taking them off anyway.} 
All three laughed at this. 

\ta{Ah, there you go again with your megalomania, Carzain,} said Leopold. \ta{%
  In the bedroom, and now on the battlefield, too. 
  You're crazy, you know that? 
  Do you think you are, what, Cordos Vaimon reborn?}

\ta{Yes!} 
He paused for dramatic effect. 
\ta{That, or close enough.} 
He eyed his friends, as if daring them to challenge his claim. 
\ta{%
  I \emph{will} be a hero. I am born for greatness. And I will achieve it or die trying. Wait and see.} 


Carzain still mused as they resumed their walk. 
He thought back to his recurring nightmare of being cast adrift, helpless in an endless void. 
Of being nothing. 
\tho{%
  I'll show them. 
  I am not nothing. 
  I am something. 
  I am born for greatness. 
  I am larger than life. 
  Something greater than ordinary people.
}
% \tho{I am born for greatness. Somehow I have always known this. Somehow I have always known that I am... larger than life. Something greater than ordinary people.} 

He surveyed the town around him.  
The common streets and houses. 
The common people going about their common tasks. 

\tho{%
  \Redglen. I am greater than you. 
  And I will rise above you. 
  Wait and see.
  
  I will be a hero. 
  
  I am not nothing.
  
  I am something.} 
\end{comment}




\begin{comment}
\subsection{Frederick imagines}
\end{comment}
\begin{comment}
\new
Leopold still chuckled, but Frederick, hearing the conviction in Carzain's voice, felt a pang of... something. 
\tho{Could the crazy chap be right?}
He studied Carzain and tried to picture him as a king or a great hero. 
% , and was surprised to have the image come to him easily, seeming not at all far-fetched. \tho{\Sephiroth,} he thought. \tho{Who knows, maybe the crazy chap is right.}
% The image came easily enough:
He imagined Carzain clad in shining armour and the diadem of a Vaimon Emperor, riding a proud \relc{} or other beast, at the forefront of a vast and majestic army. 
Crying out royal commands, casting spells to smite his foes. 
For an instant, the idea seemed almost probable. 
Then Frederick shrugged and giggled inside. 
\tho{%
  Yeah. Right. 
  \quo{Cordos Vaimon reborn}. 
  Haha.
  What a weirdo.}
\end{comment}



\begin{comment}
\subsection{Sir Guy}
\end{comment}
\begin{comment}
\new
%They walked to the Goldsmith house, where they found the boys' father, Gavin, and his wife, whose name Carzain could never remember. 
Carzain walked with the Goldsmith brothers to their home, where he stopped by to talk to their granduncle, Sir Guy. 
He was a Tiger knight, a veteran of many battles in his time, but now old and retired. 
%Being a hedge knight of no wealthy family\dash having earned his title by deed rather than blood, as he would say\dash he had been forced, as he grew too old to fight, to move into the household of his nephew, Gavin Goldsmith, Frederick and Leopold's father. 
Many in \Redglen{} called him \quo{Sir Guy Goldsmith}, even though he was not and had never been a goldsmith. 

The old man lightened up as he entered. 
\ta{Ah, Carzain, my boy. I haven't seen you in a while.}

Carzain returned the smile. 
\ta{How are you, Sir Guy?}

\ta{Still alive,} he chuckled, \ta{still alive. 
I would get up to whoop you one over the head, but you know how my arms and legs are these days.} 

\ta{Haha. I am certain you would. 
At any rate, I came to tell you that I am off to put your teachings to use.}

\ta{You are what?}

\ta{I joined the army.}

\ta{You did? Hahahaha! Splendid! Come over here so I can slap you!} 
Carzain did, and the old man managed a hearty if feeble slap on his back. 

He had met Sir Guy in his childhood when he befriended Leopold and Frederick, and the old soldier had welcomed the willful young Carzain as a third grandnephew. 
Carzain, who had only his parents and envied his peers their large extended families, had in turn accepted Sir Guy as a grandfather of sorts. 
Often the Tiger would play with the boys and instruct them in the use of weapons and the art of war. 
Urged on by Sir Guy, the boys had constructed play swords and dilligently practiced with them. 
Of course, this had been in the days when the Tiger was still young enough to hop about waving swords, before the stiffness had seized his joints. 
Carzain remembered those days fondly, and he did not doubt that the skills taught him by the veteran knight had had a great impact on his life and adolescent struggles, against Gregory Smith and other enemies\dash in terms of not only physical combat (where he had great need of his skills to counter Gregory's size), but also confidence, mental confrontation. 

\ta{So?} Sir Guy inquired. \ta{When? Where? How?} 

And so Carzain spilled the story of his visit to the barracks and how he was to depart the following morning. 

\ta{It doesn't surprise me,} the old man said. 
\ta{Ah, I remember the good old days with the Tigers. They were hard days, but we cherished them. There's nothing to bring you closer to your comrades than a company of enemy soldiers out to kill you.} 
He rambled on for a while about his soldiering days. 
Being a hedge knight of no wealthy family\dash having won his title by deed rather than blood, as he was wont to proclaim\dash he had been forced, as he grew too old to fight, to move into the household of his nephew, Gavin Goldsmith. 
Frederick and Leopold's father was not a poor man, and Sir Guy was kept alive and well, but it was no secret that he longed for his youth, the excitement and the glory. 

\ta{It is harsh business, war,} he was telling, \ta{but the sense of triumph makes it all worth it. The glory, the accolades. And the girls, hahahaha. The girls will love you when you return home in victory\dash but of course, in your case they already do.} 
He laughed. Then his face became serious. 
\ta{You will do well Carzain. \Itzach, it probably won't be long until they make you a knight. If I can do it, so can you. You will accomplish great things, I think.}

\ta{Yes, I think so, too, but the Goldsmith boys doubt my heroism. 
I can't imagine why.} 
They both laughed again. 

Carzain now made to take his leave, but Sir Guy stopped him. 
\ta{One thing, before you go.} 
With some difficult\dash but refusing to take a helping hand\dash the veteran moved to the big chest by the wall, rummaged through it, and produced an oblong object wrapped in cloth. 

\ta{When Frederick went off to fight, I gave him my sword. Now, I want \emph{you} to have this.} 
He unwrapped the cloth to reveal a long dagger, more than a foot in length, double-bladed and finely, if simply, carved. Carzain accepted it and made as if to test the blade. \
ta{Careful!} Sir Guy admonished. \ta{I've taken good care of it. It's still quite sharp.}

\ta{It looks good.} Carzain swung the weapon around. \ta{It feels good.}

\ta{It is good. It's a Tiger dagger. Fine quality. There's \dragonsteel{} in the blade.} 

Carzain studied it again. 
\tho{\Dragonsteel.} 
It was a powerful metal, rare and arcane. 
Some heroes of legend had wielded mighty weapons made entirely of \dragonsteel. 
While such artifacts were very rare in Belkade, weapons forged of even thin \dragonsteel{} alloys were still cherished and valuable. 
\ta{Thank you, Sir Guy. This is a great gift.}

\ta{It does me no good. 
A fine weapon like this deserves to see action again. 
Just promise me you'll stab a few of those Rungeran bastards with it, all right?}

\ta{I will, Sir Guy. I will.} 
\end{comment}



%\new
%Carzain tells his parents that he is going off to war. They are not happy. 
%\ta{You did \emph{what}?} 



\begin{comment}
\new
Carzain talks to his parents.

And to Sir Guy Goldsmith.

And to some fellow of the Galisetti family.

\new
Carzain takes off. He travels to the capital city. 

There he meets Archibald Curwen. 
\end{comment}






