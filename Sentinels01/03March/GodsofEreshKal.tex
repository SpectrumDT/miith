\bookchapter{The Gods of \EreshKal}
\stamp
  {\dateCharcoalReadsMoreDiary}
  {Pelidorian army camp}

Charcoal lit his pipe and dropped into his chair. 

\ta{\Qliphoth, I almost forgot how tiring a war can be. 
And we're not even fighting yet.} 
He produced Tantor's diary and leafed through, looking for the place where he'd last left off. 

\ta{Now\prikker last thing I remember you were drooling over the \Takestsha{} woman.
Let's see what's next\prikker} 
He skimmed the first page. 
\ta{Ooo. 
This looks juicy\prikker}



% They enter the temple. 
% They spring the occasional trap, killing a few soldiers. 
% All the while, Tantor hears ominous sounds\dash footsteps, whispers, drooling, distant howls, the rattle of chains.  
%
% The temple is huge, so they are forced to camp and sleep inside. 
%
% They are awakened when a group of savage \troglodytes{} attack. 
% The savages worship at the mouth of the temple and sometimes venture inside. 
% They are accompanied by horrid creatures, mutants from the \Wylde{}, resembling toads grown huge in size and with the body shape of apes. 
% After a prolonged fight they fight off the attackers, but many fall, among them Tantor's apprentice, Micah. 
% He's upset about that. 
%
% They push on.
%
% At last they reach the inner sanctum. 
% They find a monster. 
% They fight it. 
% \Orla{} dies. 
% Takestsha has them grab several stone tablets and scrolls. 
% Then they go. 
% 
% Rosen is sad at her master's death. 
% Takestsha tells her to shape up, but Rosen stands up to her, calling her ruthless and cruel (which she is). 
% Takestsha slaps her around. 
%
% Along the way, some hints are dropped about how Takestsha knows about the temple. 
% Supposedly she was a highly skilled student at some mage guild, but her masters were after her because she always wanted to delve too deeply into things. 
% On two occasions she narrowly avoided accidents that would have killed her, and she began to suspect that the masters were trying to kill her. 
% So one night she broke into the library, broke through some magical seals, stole several highly guarded, secret volumes and fled the mage guild. 
% There are dropped even subtler hints that her masters were Sentinels. 
% Fearing that her masters were after her, she hid. 
% And she studied the writings she had stolen. 
% 
% She realized that this knowledge was a doorway to immense power, but she couldn't accomplish it all by herself. 
% 
% So after an unspecified period of time, she approached Morgan Runger. 
% She offers to share her knowledge with him so that together they may conquer and rule. 
% She knows how to acquire great power, but she needs some \quo{starting capital} to do it. 
% So she and Morgan can help each other. 
% Of course, this entire story is pure fiction. 
% Takestsha is really a Dragon and her knowledge is Sentinel knowledge. 
% The story is just there to give a plausible excuse for her role and her knowledge, to convince a Cabalist reader that she is not a Sentinel. 
%
% How does Tantor get these hints? 
% Perhaps Takestsha has sex with him. 
% And he is ashamed, because he has a wife, so he doesn't write it directly. 
% But Charcoal is savvy enough to figure it out. 
%
% They return to the king's castle. 
% Takestsha talks to him in private. 
%

\begin{diary}%
\diarystamp{\dateEreshKalDayTwo}

\new
Alas! 
Truly the darkness of the \qliphoth{} shadows this cruel day. 
Truly the Light has abandoned this cursed temple, aye, this entire cursed forest. 

Where to begin? 

Many things happened today. 
Awful things. 
I almost cannot bear to relate it all, but I must. 
I will begin at the beginning. 

We awoke to find ourselves under attack. 
Those filthy, savage \meccara{} evidently still dwell in the temple\dash degenerate bastard descendants of the \EreshKali, perhaps. 
They crept up on us in the cover of the gloom like the cowards they are. 
Our sentries only spotted them at the last moment, quickly raising an alarm to awaken the rest of us. 

The mongrels swarmed around us with their bows, slings and blowdarts. 
There were probably a few dozen of the wretches, but at the time it felt like hundreds. 
It was a bloody battle. 
I myself would have been felled thrice over by their arrows and darts if not for the brave soldiers protecting us of the \ishrah. 

Alas, not all of us were not so lucky. 

It was Micah's first battle, yet he fought well. 
Never did he succumb to panic. 
Never did he lose his head. 
Always he kept with the \ishrah, and even helped to kill several of those vile things. 
Never have I been more proud than when I saw my son tear through three pieces of \meccaran{} filth with the power of \Izion. 

But as the cruel fates would have it, the soldiers could not protect us all, and my son was struck twice by arrows. 
Craven mongrels, the savages had coated their weapons with vile poisons. 
I and the \ishrah{} fought to the last breath against the cowardly venom, but in the end we failed. 
I failed. 

My only son is dead. 

Dear \iquin, my son is dead. 
Killed by the foul treachery of dirt-eating barbarians. 
Killed by my own incompetence. 
Killed by my own lack of judgement. 

I should never have brought him on this insane trip. 
What was I thinking? 
Micah was no \ishrah{} mage. 
He was just an apprentice. 
He was just a child. 
Only sixteen years old. 
Much too young to be dragged into this filthy den of heathen witchery. 
What was I thinking? 

Surely I am damned. 
Surely the \sephiroth{} will damn me for throwing away the most precious thing in the world: 
the life of my beloved son. 
I, wretched, can only pray that they will take good care of his soul, safeguard it into the Light. 

In the end we vanquished the scum. 
We killed at least fifteen and forced the rest to flee. 
I regretted at the time that we were unable to exterminate all of those beasts. 

The vermin had two witch doctors with them, working their foul sorcery, but \Miss{} \Takestsha{} slew them both. 
Her \Rethyactic{} sorcery was grim to behold, but lily-white next to the blasphemous bestiality of the \EreshKali{} shamans. 
They also had beasts of war: hideous things, like dogs but bigger, uglier, more rabid. 
Three of them, perhaps; I cannot say for sure. 
%I saw none of the hounds among the slain, so they must have escaped with their depraved masters. 

In addition to poor Micah, seven soldiers lay dead, and four \relcs. 
%We cannot cart the bodies around in here, so we resolved to leave some of our party behind to guard the
\Takestsha{} believed that we were close to our destination, so we wrapped up the bodies and loaded them onto the \relcs{} and \belwans. 
% We have enough horses to carry them
Most of the living will have to walk from now on. 

And \Takestsha{} was right. 
After what cannot have been much more than two hours, we found a pair of tall but now broken doors leading into a large chamber. 

By the feet of those high doors were encamped the rest of the \meccaran{} tribe. 
There was maybe a score of them, maybe thirty, including another spellcaster, and with several of the vile hounds. 

The rabble fought to defend what we realized must be their inner sanctum, but without the element of surprise the barbarians proved no match for our well-trained warriors. 
We slaughtered the vermin and left none alive, unless some cowards slunk away unseen. 
Only three of our men were lost, plus a \relc{} and a \belwan{}. 
I myself dispatched their fell shaman, causing his brain to sizzle and boil through the power of \Razilah. 

Not that any of this will bring Micah back\prikker 

With the degenerates and their beasts wiped out, we entered through the fallen doors to explore our prize. 
We stepped into a huge chamber; a hundred yards from wall to wall, perhaps, 
%Fifty yards from wall to wall at least, if not twice that, 
and the ceiling just as high. 

Frightful statues ringed the hall. 
Some were of \meccaran{} warriors and shamans. 
Of these most were about life-sized, but a few were two or three times bigger\dash kings and chieftains, I assumed. 

But ugly and bestial though \meccara{} may be, their likenesses were the least disconcerting. 
Much worse were those effigies of hideous monsters and \daemons{} that we also found. 
%The rest of the effigies were those of hideous monsters and \daemons. 
Some were warped beasts, like those monster-hounds that the degenerates breed, and other things that might lurk in this Light-forsaken \Wylde{}. 
Yet others depicted abominations gigantic and horrid\dash twenty yards high, perhaps\dash that must be the morbid, inhuman gods of this depraved people. 
False gods they must be, for the entities carved here were such freakish and loathsome aberrations as 
%Yet others depicted such freaks and abominations as 
cannot possibly exist on \Miith{}, nor any sane world.
%, only the deepest pits of \itzach{}. 
At least, I pray to \Iquin{} that such grotesqueries not exist outside the diseased dreams of insane barbarian witch-doctors, deep in some toxic drug-induced stupour. 
Their aspects were horrid, with their bloated bellies, their multitudes of paws and tentacles\prikker nay, I will say no more. 
Some things are best left unsaid. 
%Their aspects were so horrid that I cannot bring myself to write about them here; indeed, I averted my eyes from their forms as best I could, and this was not easy, for the idols were huge, and encircling the entire chamber. 

In the centre of the great hall there rose from the dome-like floor a small ziggurat, ten yards high and terraced with steps. 
Atop it stood a stone altar, encarved with the claws, heads and mouths of repellent things. 
On this altar we found several stone tablets full of writing. 
I could not read the hieroglyphs, but from my limited knowledge, I would guess them to be a degenerate variant of \Draconic{} runes. 
It seems, though, that \Miss{} \Takestsha{} could read them, for she skimmed through them with great interest. 

\ta{These are what we seek,} she proclaimed after some study. 
\ta{These tablets contain the occult knowledge of ancient \EreshKal.} 
She also instructed us to pry loose several glyph-carved plaques found at the feet of the giant statues. 

\ta{We have found our goal,} \Takestsha{} told us. 
\ta{With these, the deaths of our brave companions will not have been in vain. 
Come, let us leave this place quickly. 
There may be more of the savages and their tamed monsters.} 

And so we made our way in forced march out of the cursed temple, retracing our steps and not stopping until we reached the tunnel mouth where we had left the remainder of our party. 
When we emerged back into the forest it turned out to be the middle of the night, so we soon settled down to rest. 

I see \Takestsha{} stalking around the camp like some vulture. 
\quo{Not in vain}, she said. 
Bah. 
My son was killed, as were over a dozen valiant soldiers, all so that a power-crazed sorceress could get her hands on the loathsome \daemon{} worshippers' foul grimoire. 

Damn you, \Takestsha. 
Damn you. 

Oh, \Iquin, how will I tell my wife this? 
\end{diary}









\noindent
\ta{So, \Takestsha{} and company have found this \quo{\daemon{} worshippers' grimoire}. 
Does that mean it's in the Rungerans' hands now? 
Are they marching against us wielding the sorcery of \EreshKal? 
Scary shit\prikker}









\begin{diary}%
\diarystamp{\dateEreshKalLeftBehindDayOne}

\new
I should not have been so quick to judge \Miss \Takestsha, nor blame her for Micah's fate. 
I talked to her last night after I put my diary away, and she told me her story. 
Now that I have gotten to know her better, I can see her actions in a new light. 
% 
% \Takestsha{} is actually an intriguing and amazing woman. 
% And beautiful. 
% 

\Miss \Takestsha{} was originally apprenticed in some \ishrah{}, the location of which she was not yet willing to reveal. 
Her teachers were power-mongering misers who kept her locked down in ignorance despite her obvious talents. 
She would devour any material they taught her, and they would withhold further knowledge from her, afraid that she would surpass them in skill and power. 
So she found herself forced to go behind their backs. 
She would sneak into the libraries at night and study the forbidden texts. 

In those covert studies she would uncover \quo{terrible secrets}, as she calls them. 
Sinister, forbidden truths purposely hidden from the world, she claims. 
She has not yet shared these \quo{secrets} with me, but I must surmise that they include, among other things, the detailed story of \EreshKal, and perhaps their language, too. 

But her clandestine research could not go unnoticed forever, and eventually \Takestsha{} learned that her mentors, fearful of her growing power, were plotting to have her killed before she became too dangerous. 
And so she grabbed what books and materials she could and hastily fled. 
She then spent months trekking around \Velcad, ever in hiding from her former masters and their agents, who would kill her for her forbidden knowledge. 

So it was that she sought out King Morgan. 
Using her well-honed skills of magic and stealth she managed to get close to the king and strike a bargain: 
He would take her under his protection, shielding her from the mages and their assassins, and in turn she would provide him with arcane weapons and knowledge. 
It was her end of the agreement that led us out on this whole journey. 

And now we have what we seek. 

% I still grieve for poor Micah. 
% But it was wrong of me to vent my anguish upon \Takestsha. 
% She is a magnificent woman. 
% And so beautiful. 
% I picture her in my mind: 
% Rounded face, long radiant hair, hardened eyes marked with a hint of tears. 
% Her body, so strong, and yet so vulnerable. 
% Thinking of her plight and ordeal makes me want to embrace her tight\prikker 









\end{diary}









\noindent
% \ta{Oh, \emph{now} I get it!} Charcoal laughed. 
% \ta{You fucked her, didn't you? 
%   Hah! 
%   You boned her! 
%   Tantor, you sleazy bastard.
%   Or, perhaps I should say, \Takestsha, you sleazy bastard. 
%   Taking advantage of an old man's grief like that. 
%   You should be ashamed of yourself.
%   Ha ha ha ha ha!} 
% 
%   \tho{This whole chapter makes so much more sense now\prikker
% 
%   \quo{I talked to her last night\prikker} 
%   Yeah, sure you did, and I think I know which body parts did the talking. 
%   \quo{An intriguing and amazing woman}, yeah, I bet. 
%   What happened to that wife you mentioned in the last entry, eh, Tantor? 
%   Forgotten in the heat of battle? 
% 
%   And then those wicked, wicked mentors of hers who were oh-so-mean to poor little \Takestsha. 
%   I was wondering the first time around why this tale was so massively biased. 
%   Turns out it was all good old-fashioned pussy bias. 
%   Ha ha ha.}

Curwen scratched his beard and puffed on his pipe. 
\tho{Hmm. 
  Interesting, this \Takestsha character.
  On the run from some bad, bad people trying to keep a lid on the truth, is she? 
  Hm. 
  By the looks of it, she's \Rethyax-schooled rather than Vaimon. 
  Sounds like those mentors are Sentinel-aligned, then. 
  Our side doesn't teach Chaos.} 
He sighed. 
\tho{%
  So the Sentinels created her and let her escape, and now it's up to me to clean up their mess? 
  A  mess armed with some old and potentially potent sorcery, no less. 
  That's just typical. 
  Fucking Sentinels.}

Charcoal browsed forward, skimming several pages. 
% The next several entries were quite eventless, consisting mostly of Tantor mourning his son while drooling over \ps{\Takestsha}{} womanly virtues. 
% Every now and then he dropped a reference to his wife, with some guilt over his adultery shining through. 
% 
After Tantor's party returned to Dormina, the sorceress apparently gave up Tantor as a lover. 
His fawning over her became more pining, more desperate. 
Charcoal, not interested in some old man's self pity, breezed forward. 
The upcoming war against Pelidor was not referenced directly, but there were passages here and there that seemed to have to do with the preparations for war. 
Charcoal eagerly searched for any tidbits of strategic interest, but found nothing that was new to him. 

%But then, near the end of the account, things started getting interesting. 
The account ended by the beginning of \Gamishiel. 
Stuck to the end of it was a scrawled piece that did not look like a regular diary entry\prikker 









\begin{diary}%
  % The last part is not a diary fragment, but a letter to Ambrose Onatol. 
  % Tantor has been studying and practicing the Eresh-Kali magic under Takestsha's guidance. 
  % Takestsha has warned him not to describe it in his diary for fear of too much knowledge falling into the wrong hands.
  % (Maybe have Charcoal call Takestsha out on her hypocrisy here.) 
  % But the magic is so nasty and mentally scarring that Tantor cannot bear to keep it secret any longer. And moreover, he fears the terrible evil that his Ishrah, now led by Takestsha, is about to unleash. 
  % And so he sends his diary to Onatol. He says some stuff about: "I don't know what I am expecting, but\prikker please, put this account to good use."
  Dear \Ambrose{} \Onatol, 

  I have not described what has happened to the \EreshKali{} fragments since we retrieved them because we have all been forbidden to divulge any further details. 
  I choose now to disobey my king's orders because I fear we are dabbling in powers much more dangerous and evil than we realize. 

  Under Miss \ps{\Takestsha}{} guidance, the \ishrah{} has been at work translating and studying the tablets. 
  It is difficult work, and most of us have managed to develop only a rudimentary understanding of the \EreshKali{} magic theory. 
  Only \Takestsha{} is close to mastering it, presumably due to having studied similar materials before her eviction from her previous \ishrah. 

  To my Vaimon-trained eyes, the \EreshKali{} sorcery looks like a form of Chaos magic, but darker, more twisted than any \Rethyactic{} magic I have encountered. 
  Even just subvocalizing the spellwords or visualizing the glyphs, I swear I can feel the presence of the wicked gods of \EreshKal, breathing down my neck. 
  I sense them as clearly as I sense the \sephiroth{} or \qliphoth{} when invoking my own magic; but this feeling is thoroughly different, much more hideous than even the most sinister \qliphah. 
  When I try to cast the spells, I can feel those dark gods flowing through me, like a poison distilled from pure black evil running through my veins. 

  Lately I have begun to feel the \daemons{} of \EreshKal{} even when not consciously calling upon them. 
  They are possessing me, seeping into my mind and body with their foul essence. 
  They haunt me night and day, their monstrous faces seeming to peer back at me from the shadows at the edges of my vision. 
  The strain is affecting my health and sanity. 
  I am succumbing to paranoia, cramps, dizziness and fits of pain, and I can tell that I am not the only one of the \ishrah{} thus affected. 
  But the last time one of us dared speak out against this blasphemy we were all savagely berated by \Takestsha{} for disloyalty towards our king. 
  Ever since we returned from Waythane she has led the \ishrah{} with an iron fist. 
  I do not know whether to blame her for bringing this evil upon us or to pity her as a fellow sufferer\dash I wonder if these fits of anger are her way of coping with the strain of this sorcerous scourge. 

  Now, this last part should I under no circumstances be telling anyone, but I feel I must do so anyway: 
  The king intends to launch a full-scale invasion against Pelidor, and he intends to make full use of the \ps{\ishrah} new weapons. 
  Within months we will be marching against your country. 
  I shudder to think of the consequences if the gods of \EreshKal{} are unleashed on the battlefield. 

  Therefore, my friend \Ambrose, I am sending you this warning, along with my diary. 
  I hope it reaches you. 
  I imagine I feel the \daemons{} looking over my shoulder even now, poised to strike me down for this betrayal. 
  If the king or anyone else intercepts my message, I am a dead man. 

  I don't know what I expect you to do with it. 
  I don't know what I am hoping for. 
  I don't know what to do myself. 
  I can only hope that somehow you will be able to put my account to good use. 

  Please help ward off this evil. 
  Please help save us all. 

  \begin{flushright}
  \Jirad{} Tantor
  \end{flushright}
\end{diary}









\noindent
\tho{Finally an explanation of how it ended in \ps{\Onatol} hands. 
  I was getting impatient about that,} thought Charcoal. 

\tho{It doesn't sound good. 
  The Rungerans and their rogue sorceress now possess this lost magic of \EreshKal. 
  Tantor makes it sound very evil. 
  Granted, Tantor is a wuss, but still, I know a warning flag when I see one. 

  I can't immediately gauge the import of this development. 
  I will have to report it to the \resphain.}

\tho{I'm curious to see how it's really going to turn out, though.}
Charcoal felt a tinge of anticipation. 
Some perverse part of his mind was actually looking forward to meeting the Rungerans on the battlefield, now. 

\tho{Who would have thought that this trivial war might turn out so interesting?}









