\bookchapter{The Mystery of \EreshKal}
%By virtue of his skills and his contacts, Charcoal was privy to many things, having witnessed both Constance Kintair's interrogation and the attempted arrest of \Ambrose{} \Anatoli. When the apprehension party had begun collecting clues from \Anatoliz{} quarters, Charcoal had been quick and stealthy in grabbing a journal and a stack of letters that caught his eye, stealthily tucking them away before the investigators could get their hands on them. 
%
%\tho{Interesting stuff, by the looks of them,} he mused as he strolled down the stairs from \Anatoliz{} tower room. From his many years in the organization, Charcoal had developed an intuition for such things. \tho{I will need to study these when I have the time.} 
%
%\tho{But first things first. The duchess wants little Constance killed. She's no use to me dead, so I'll need to visit her again and see if I can't extract some more \quo{information} from her, before they cut off her pretty head.} 
%
%And so he descended into the dungeon. 
%
%The dark stone walls were thick down here. Stained with blood in places and adorned with chains, hooks, iron maidens and all sorts of imaginative torture instruments. Long poles with barbs, scissors-like objects, evil-looking masks and other things. 
%
%Some of the instruments were not at all useful, Charcoal knew, their only purpose to evoke an intimidating atmosphere. Some were even encarved with the forms of evil \daemons{} or gargoyles to inspire greater dread and apprehension.
%
%Charcoal made his way through narrow, oppressive, vermin-infested corridors to Constance's cell, where the guard quickly opened the door upon his approach. 
%
%Entering the cell, he found Constance strapped naked to a table. Another guard, Nobb, a big, stupid man, was having his way with her. Judging from Constance's lack of movement, he had been at it for a while. 
%
%\ta{Move over, Nobb,} said Charcoal, reaching down to undo his belt. \ta{My turn.}
%
%
%
\stamp
  {\dateCharcoalReadsDiary}
  {Pelidorian army camp}
%
%
%
\tho{Finally,} thought Charcoal as he lowered himself into the chair in his tent. \tho{Some time to myself.}

%Charcoal sat down in his tent.

With the coming war and the army on the move, his last several days had been frantic with work and preparation. \hs{Cabal} business as well ash is \quo{everyday} duties in the ducal army had kept him busy all hours of the day. 

\tho{I've barely had the energy to eat and fuck,} he complained to himself. Which was an overstatement, of course. Always time to eat and fuck.

But now he had finally issued enough orders and delegated enough responsibility that he could divert his attention to other important matters. 

He walked over to his private chest and softly invoked several \hr{Qliphah}{\Qliphoth}. His trained eyes could see the \hr{Archon}{\psp{\Archons}} spindly forms as they came into the world, hearkening to his call. With his mind he directed them in the delicate dance, slowly unravelling the seals placed on the chest. At length, the work was done and the mystic lock slid apart with mathematical precision. Charcoal paused a quarter of a heartbeat to admire the intellectual beauty of his work, then released his \Qliphoth. He bent down to open the chest and dug up the letters he had snatched from \ps{\Onatol}{} room, them slammed the chest shut. He was not \naive{} enough to leave the chest open while reading; you never knew when you might be interrupted, or who might sneak in. Caution paid off in his line of work. Again he called his \Qliphoth, but the invocation to reactivate the lock was much simpler than the one to unlock it. That was the genius of the spell: Easy to close, but near-impossible to open. 
%He invoked the \Qliphoth{} to release the arcane seals on his private chest, then opened it and dug up the letters he had snatched from \ps{\Onatol}{} room. 

Before sitting down he fetched his pipe and a clump of weed. Not the strong kind of weed, of course. Not now. He needed to think. Sitting down, he lit the pipe and began to read the letters. 

\tho{All right, now let's see. This should be interesting\prikker}

The first page showed the title: 







\begin{diary}%
  \bf From the diary of \Jirad{} Tantor\\
  \hr{Ishrah}{\Ishrah} of \hs{Dormina}, \hs{Runger}
\end{diary}







\noindent
Charcoal scratched his beard. 
\tho{Runger? \Onatol{}'s been corresponding with a Rungeran mage? Juicy\prikker}







% This is \Jirad{} Tantor's diary. Tantor is a high-ranking mage at Morgan Runger's court. 
% We start off in Morgan's throne room, where Tantor describes the King's new advisor, Takestsha. She is a Human woman of Rissitic origin (actually the Dragon \Nzessuacrith, daughter of \Ishnaruchyfir, in disguise, but Tantor doesn't know that). 
% He doesn't know her history, but one day she just shows up and the king introduces her as his new advisor. 
% At the beginning, she just sits at the king's councils and only occasionally speaks. 
\begin{diary}%
  \diarystamp{\dateTantorMeetsTakestsha}
  
  \new
  %A newcomer has joined the king's council today. 
  A strange thing happened today. 
  At the royal council meeting, \maybehr{Morgan Runger}{King Morgan} arrived in the company of a strange woman. 
  She was young\dash in her mid-twenties, I would estimate\dash and very well-shaped. 
  Dark brown hair hanging loose in curls. Her ears each adorned with an exotic black jewel, and a similar larger one on her bosom. 
  A green-and-black dress that clung tightly to her curves, cut too low for propriety. 
  Some among the assembly snorted or grumbled at her indecency, while others merely drooled. 
\end{diary}







\noindent
Charcoal laughed to himself. 
\tho{%
  Haha. 
  And by not commenting further you place yourself squarely in the \quo{drool} camp. 
  Eh, Tantor?
  
  Interesting reading, though. 
  Not just a Rungeran mage, but a member of King Morgan's council, no less. 
  
  But back to business\prikker}







\begin{diary}
Taking his high seat, the king introduced her: 

\ta{Council, this is \Takestsha. As of today she is a member of this council.}

\Takestsha{} acknowledged the assembly only with a curt nod before seating herself on the king's left and proceeding to speak not a word for the duration of the meeting. She merely sat there and studied us all with those piercing green eyes. I once met her gaze and tried to hold it, but the cold, indiferent look she gave me sent shivers down my spine, and I had to look away. 

After a while of the woman's watching and listening\dash like a cat watching mice\dash \hr{Pater}{\Pater} Andros felt compelled to stand and object: 

\ta{My king, who is this woman? Why is she here? These are delicate matters we are discussing, unfit for the ears of a random\prikker \emph{concubine}!}

King Morgan's call to order was brusque: \ta{I need not explain my reasons to you, \pater. But \Miss{} \Takestsha{} is more than a mere concubine, and you all will treat her accordingly!}

\Takestsha{} herself showed no reaction at all to this exchange. She observed intently, but with the same cold detachment that she maintained throughout the meeting. The intriguing question of whether her role was that of \quo{concubine and more} or \quo{not concubine at all} remained unanswered, however, and judging by the looks in the others' eyes as they covertly studied her, I was not alone in wondering. 

At the conclusion of the meeting, none of us was any the wiser: Who is this \Takestsha, what is her relation to king, and why is she here? Perhaps tomorrow will tell.
\end{diary}



\noindent
\tho{So, the king introduces a mystery woman\dash concubine or not\dash who turns the old men's heads and stares down a mage.} 
Intrigued but impatient, Charcoal browsed ahead, skimming several pages.







\begin{diary}%
\diarystamp{\dateTakestshaIntroducesEreshKal}

\new
We had a disturbing discussion in the council today. \Miss{} \Takestsha{} told us the tale of \hr{Eresh-Kal}{\EreshKal}, a \hr{Meccaran}{\meccaran} tribe who allegedly dwelt here in Runger hundreds of years ago. According to \Takestsha, the \EreshKali{} worshipped strange gods and wielded great arcane power, until they were finally conquered by the \hr{Iquinian Church}{Iquinian} \humans. But even though their culture was destroyed, one of their great temples remains, hidden deep in \hs{Waythane Forest}.

If \Takestsha{} and her sources are to be believed, this lost temple contains a great wealth of occult knowledge. She believes that this represents a gold mine of power and has, as it seems, convinced the king to send an expedition to search for it. 
%As it appears, \Miss{} \Takestsha{} has convinced the king to send an 

Unsurprisingly, this caused an uproar, especially from the \pater, who was less than thrilled at the prospect of harvesting and employing these rumoured artifacts of heathen black magic. 

\Takestsha{}, cutting through the commotion,  turned and addressed me. \ta{\Mr{} Tantor. You are renowned as a man of science, so surely this must be of interest to you. I request that you join our expedition.} 

That \Takestsha{} herself would partake and lead the mission was evidently a foregone decision. I cast a glance at the king, who remained silent, but whose gaze implied that this was more than a mere request. I felt compelled to acquiesce. 

There were more protests, but the king silenced them. The council was soon adjourned. 

I know not what to expect of this \EreshKal, if it exists at all, but I confess that it makes me uneasy.
% 
% Afterwards, I went to the library to seek out 
% 
% 
% 
% One day Takestsha persuades the king to send his mages and troops out to a region in the \Wylde{} to recover some mystic artifacts. 
%
\end{diary}








\begin{diary}%
\diarystamp{\dateTantorResearchesEreshKal}

\new
After much searching in my own books and those in the library, I managed to track down a mention of \EreshKal. \hr{Soror}{\Soror} \Sylvie{} \Dereine{} of the \hs{Redcor} relates that the \EreshKali{} civilization flourished around \yic{Eresh-Kal flourished}, ruling much of what is now Runger. She describes the \EreshKali{} as brutal savages, worshipping horrid \hr{Daemon}{\daemons} and ruling by the force of their hideous sorcery, powered by the blood and souls of the innocent. 
%\Dereine{} relates how, 

After suffering a century or more of their depredations, several neighbouring princedoms, prodded along by Redcor advisors, ultimately banded together to eradicate wicked \EreshKal. The war is glossed over with little detail, but but it is clear that, while the \EreshKali{} sorcerers wrought great destruction, they were eventually vanquished by the united armies and the grace of the \hs{Light}. 

Granted, it is not unheard of for a Redcor historian to be biased against outsiders, but \ps{\Dereine} tale is chilling nonetheless, and hardly makes me any more thrilled at the prospect of the upcoming journey.
\end{diary}








\begin{diary}%
\diarystamp{\dateTakestshaSetsOut}

\new
%
% They go there. 
% They are Takestsha, Tantor and two other mages: \Orla{} of Fanshire and Ryco nel-Ossa. Also a bunch of soldiers and some apprentices. Tantor's apprentice is Micah of Penster and \Orlaz{} apprentice is Rosa Sandhome (the only other woman on the expedition). 
%
Today we set out on the journey to find this alleged \quo{lost temple} of \EreshKal. 

I am bringing Micah along. 
A field expedition will do him good. 
He has been locked up in the laboratory far too much. 
Many might say that especially the young should be kept far away from anything that smacks of black magic. 
I say the boy needs to grow up and see the real world. 
I would not have my son remain an apprentice forever. 

\Miss{} \Takestsha{} is leading the mission, of course. 
Apart from the three of us, the team has two more mages: 
\Orla{} of Fanshire and his apprentice.

We are accompanied by two score soldiers, commanded by Captain Garog son of Otonn. 
To provide religious support we have a \hr{Frater}{\Frater} Enthon with us. 
The entire company is mounted, with several pack \belwans{} and spare \relcs. 

%The journey to Gedrock Forest is estimated to take six to eight days
We estimate that it will take us six or seven days to reach the edge of Waythane Forest. 
\end{diary}







\begin{diary}%
\diarystamp{\dateTakestshaReachesGedrock}

\new
%
Today we arrived at our last supply station, the village \hs{Gedrock}, and are spending the night here. 
%station, the village Gedrock, behind us. 
Before settling in I spoke to some of the locals, most notably a \Mrs{} Murein, a \quo{wise woman} of sorts. 

\ta{Take care if you venture into the forest, good sirs,} Murein told me. \ta{%
  It is evil, I tell you, evil. \quo{\hr{Wild}{\Wylde}} does not describe it. I have seen many a forest in my days, and \Wylde{} though they may be, none of them are like Waythane. The air is thick with evil in there. Wicked spirits and wraiths haunt the forest\dash you can hear them wailing in the night. And the beasts are fell, terrible things. They are all half \daemon, I swear. 
  
  The savages that dwell in the forest are a vile sort. They are \meccara, but of a degenerate, bestial kind. They breed monsters as dogs for use in battle. And when they kill you they will drain you dry and drink all your blood, then throw your dried corpse to their hounds. 
  
  And that's if those ghastly people don't capture you as a sacrifice to their blasphemous gods. Horrid and ancient they are, their gods, and they feed on the blood, flesh and souls of men. 
  You can feel their malicious presence in the air, growing thicker, more palpable as you go deeper into the forest. The Light holds no sway in there, sir. Waythane is a dark place, ruled by the nameless gods of the \Wylde{}.}

The part about the Light holding no sway is, of course, nonsense. As is much of the rest, no doubt. 
Yet I must confess that I was disturbed the old woman's story, and the conviction in her voice. 
I excused myself quickly after this, not feeling entirely up to the task of berating \Mrs{} Murein for her impiety and superstition. 

We will be heading into the forest first thing tomorrow. \Takestsha{} estimates that it will take us two or three days to find the temple. 

Before going to sleep I walked to the edge of town to get a look at the woods. 
It doesn't look all that different from any other forest. 
There \emph{were} shapes and shadows between the trees that might look like ghosts, but what piece of \Wylde{} doesn't have shadows and things that, to the vivid imagination, looks like ghosts? Nothing to worry about. 

No, nothing at all to worry about. 
\end{diary}







\begin{diary}%
\diarystamp{\dateWaythaneDayOne}

\new
%
We ventured into the forest this morning and have been travelling all day. 

I have to admit that Waythane is a scary place. Even compared to other \Wylde{}s it seems \Wylder{} than most. The vegetation is thick, which not only makes our progress slow; it also covers the forest floor in perpetual darkness\dash you can barely make out the Sun at all during the day, and now, in the evening, I don't see much of a difference. 

The old woman was right about the forest being full of wild beasts. While they might not be \quo{half \daemon} as she claimed, they may still pose a threat. Several times we have heard the howling of wolves in the distance, and we are pestered by snakes, spiders and even some very aggressive birds\dash crows or something similar. One soldier was bitten by a snake this evening as we made camp, but \Orla{} healed him. The critter was not deadly poisonous. 

Settling in for the night, I am very glad that we have a fire and many soldiers with us. Not because of ghosts or dark gods, of course, but because of the dangerous beasts, and perhaps the savages. We haven't seen any of them yet, although one boy said he thought he saw something that looked like a misshapen \meccaran{} on one occasion. It could be anything, though. 

No sign of wraiths or evil gods. Nothing strange. Just a dense forest with wolves and snakes. It has an eerie atmosphere, true, but that is just the darkness and the cramped feeling caused by the tall, looming trees. 

Nothing to fear. There are soldiers everywhere, and we are several mages. Nothing to fear. 

%Goodnight, diary. 
\end{diary}







\begin{diary}%
\diarystamp{\dateWaythaneDayTwo}

\new
%
The soldier who was bitten by a snake yesterday died during the night. 
Evidently the snake was more poisonous than we thought. 
We buried him in a shallow grave in the forest. 
\Frater{} Enthon performed his last rites. 

Above us, in the crowns of the hoary and towering trees, the fog lies thick. 
By day, the Sun is a gray blur, as if it had died and become a ghost. 
By night, the moons peer through the veil of mists like a pair of pale and ghastly eyes. 

The eerie forest sounds are intensifying. 
We are hearing more of the wolf howls, and more clearly. 
I think they are closing on us. 
But surely our numbers will be enough to deter them. 

Interestingly, I often see \Miss{} \Takestsha{} standing around or striding about at night, staying awake together with the soldiers on night watch duty. Perhaps she feels that a mage should be awake in case of an attack. Admirable altruism. I hope she does not end up requesting the same of me, though. You know, diary, how much I value my night's sleep.
\end{diary}







\begin{diary}%
\diarystamp{\dateWaythaneDayThree}

\new
%
We lost a \hr{Belwan}{\belwan} today. 

Some time in the morning came suddenly loud, chilling screams coming from somewhere far off in the forest. Birds, I suppose, though many of the men believe them to be \daemons. No doubt the men, unsettled by the gloomy atmosphere and the death of a comrade, have become more vulnerable to superstition. 

But moreover, the sound frightened one of the pack \belwans{}. 
It bolted off in panic through the woods and we never saw it again. 

This is bad, because that \belwan{} carried a lot of needed supplies. 
I suggested we turn back for more, but \Miss{} \Takestsha{} insists we move on. 
We will have to ration the food now. The men, of course, see it as ill omen, and morale suffers. 

The forest is growing even denser and darker, and I have to admit that I am slowly succumbing to superstitious fancies myself. The light and the atmosphere is playing tricks on my mind. I keep imagining shapes out of the corner of my eye\dash humanoids, monsters, ghosts\dash that disappear when I gaze directly at them. 

If this temple exists, I hope we find it quickly so we can get out of this damned \Wylde{}.

On another note, I seem to notice that \Takestsha{} is always awake when I tuck myself in, and when I stir in the morning she is always up. 
She seems to sleep little. 
\end{diary}







\begin{diary}%
\diarystamp{\dateWaythaneDayFourTempleFound}

\new
%
We have found the temple! 

But first things first. 

We were attacked this night. Local savages hid in the foliage and pelted us with arrows and blowdarts, only to slink away before we could pursue them. 
I did not get to see them myself, but some of the men describe them as \meccara{} wearing strange fetishes and body paint. Even the watchful \Takestsha{} seemed taken aback by the ambush. 

Their savages' darts were poisoned. We had six men dead plus a \hr{Relc}{\relc}, and several wounded. 

We pushed on. Morale was low. 

At a point which I estimate to be mid-afternoon (with the sky shrouded in dense \wildfog{} it is hard to tell), we found the temple entrance. 

I had expected more. 

It was a rather small edifice, eight or nine feet tall and wide. A simple stone doorway. Little more than a glorified tunnel mouth. I had expected eldritch runes and huge monstrous statues, but the entrance was plain and bare of adornments. \Takestsha{} assured us that this is a mere side entrance. The main temple itself, she related, is likely collapsed and overgrown by now, but she had hoped that one or more of the side entrances might still be accessible. 

\quo{Hoped} was the word she used, which worries me. What if we had found none? Would this entire trek and the deaths of several brave men have been for nothing, then? 

At any rate, the entrance was blocked by a solid if crude stone door. It took a fair amount of magical and manual labour to pry it away. 
%\dash dirty and tedious brute-force work, but not otherwise noteworthy.
% which there is no need to describe here. 
%
We left one third of the soldiers outside to guard the entrance and took the other two thirds with us, including most of the \relcs. 

At first, the calm of the tunnel was a relief from the unsettling noises of the forest. 
It was quite narrow, so that we could only walk two men abreast, and mostly bare earth and stone, broken by the occasional vines, mosses, dead leaves, dirt and mud. 

The uneven corridor sloped gently downward for at least two hundred yards before the first fork. At the crossroads\dash and every subsequent such crossing, it would turn out\dash\Takestsha{} seemed to know the way, hesitating only a moment before picking a direction and leading us all onward. I tried to pry out of her how she knew this, but she remained secretive, at times deflecting my questions with merely an endearing smile, a crooked twisting of her beautiful eyes. 

After a few twists and turns and a total distance of what I estimate as maybe four hundred yards we emerged from the narrow tunnel into a much larger one, probably six yards wide and tall. 
%\Takestsha{} explained that this was the temple itself
Dank and foreboding though the place might be, we were all, men and beasts alike, grateful to leave the claustrophobic rathole behind and poured out into the larger tunnel with renewed vigour. 

Here in the temple proper were more signs of civilized habitation. Torches (now long extinguished) hung on the walls or lay on the floor when their cressets had crumbled. The mostly smooth limestone walls, encrusted with lichen and hardy subterrene vines, were occasionally decorated with carvings that might be writing in some unrecognizable language. At times I noticed \Takestsha{} studying the carvings, as if for directions. She was unwilling when questioned to admit to any understanding of the cryptic writings, and indeed, willing to give me little more than her enigmatic but charming smiles. I have my doubts. She would, after all, hardly be the first mage to deny her knowledge of occult matters. 

We wandered for what must have been several hours, and despite \Miss{} \ps{\Takestsha}{} seeming confidence I began to suspect that we were going in circles. Ultimately we were forced to make camp and rest for the night\dash or, what we assumed to be the night, far from the Sunlight as we are.

Without the activity of riding to occupy my mind, and freed from the clamour of marching feet and hooves, I came to notice sounds that had hitherto been inaudible, and gradually, a feeling crept over me of dread even more sinister than up in the forest. 
As I write this, I imagine I hear the shuffling of feet in far-off corridors, an unnatural piping of the wind, and hissing noises louder than they have any right to be. 
Looking into Micah's eyes, and those of the soldiers still awake, I get the impression that we are all hearing these sounds, but no one, myself included, dared to bring up the subject. 
I think we are all trying to block them out, to tell ourselves that they are not real. 

Perhaps they are not real.

Perhaps it is nothing. 

\Takestsha{} is still awake. I met her gaze just before, and found her captivating green eyes both reassuring and disturbing. Reassuring to have a knowledgeable mage to guide and guard us. Disturbing because of the mystery: From where does she have her mysterious knowledge? What more does she know that she is not telling? And, perhaps worst of all, what does she \emph{not} know and is covering up? 

She strikes a remarkable figure, though, standing there. On guard as a bird of prey\dash her green-and-black colours make me think of the eagle banner of Geica, which is a beautiful peace of heraldry. And shapely, with her leathern travelling dress accentuating her more-than-ample feminine curvature. 

But now I really must sleep.
\end{diary}



\noindent
\ta{Hah!} Charcoal laughed aloud. 
\tho{%
  \quo{Sleep}. Yeah, right, Tantor. After a little handiwork under your blanket, you mean. Fawning over the \Takestsha{} girl like that. I can almost smell your heated breath and sweat on the paper, you pig. Haha.}

But before he could resume his reading, he was interrupted by a knock on his telt pole. \ta{Sir?} came the call of a soldier from outside. 

% \ta{What!?} barked an annoyed Charcoal. 
% 
% \ta{The Marshal requests your presence, sir.}
% 
% \ta{All right. Tell 'im I'm coming!}
% 
% \ta{Sir!} Booted feet receded.

\tho{%
  Crap. 
  I suppose I'd better answer that. 
  It was just getting interesting, with the sex and the dark magic creeping in. And now duty calls. Oh, well. Better seal it back in the chest again.
  
  I'll get back to you soon enough, Tantor. 
  
  Hopefully with more black magic. 
  
  And more sex.}

% After an exhausting, many days long trek through the \Wylde{}\dash where they have to fight beasts and savages\dash they find the entrance to an ancient temple. 
% They enter the temple. They spring the occasional trap, killing a few soldiers. All the while, Tantor hears ominous sounds\dash footsteps, whispers, drooling, distant howls, the rattle of chains.  
% The temple is huge, so they are forced to camp and sleep inside. 
% They are awakened when a group of savage \troglodytes{} attack. The savages worship at the mouth of the temple and sometimes venture inside. They are accompanied by horrid creatures, mutants from the \Wylde{}, resembling toads grown huge in size and with the body shape of apes. 
% After a prolonged fight they fight off the attackers, but many fall, among them Tantor's apprentice, Micah. He's upset about that. 
% They push on.
% At last they reach the inner sanctum. They find a monster. They fight it. \Orla{} dies. Takestsha has them grab several stone tablets and scrolls. Then they go. 
% Rosen is sad at her master's death. Takestsha tells her to shape up, but Rosen stands up to her, calling her ruthless and cruel (which she is). Takestsha slaps her around. 
%
% Along the way, some hints are dropped about how Takestsha knows about the temple. Supposedly she was a highly skilled student at some mage guild, but her masters were after her because she always wanted to delve too deeply into things. 
% On two occasions she narrowly avoided accidents that would have killed her, and she began to suspect that the masters were trying to kill her. So one night she broke into the library, broke through some magical seals, stole several highly guarded, secret volumes and fled the mage guild. 
% There are dropped even subtler hints that her masters were Sentinels. 
% Fearing that her masters were after her, she hid. And she studied the writings she had stolen. She realized that this knowledge was a doorway to immense power, but she couldn't accomplish it all by herself. So after an unspecified period of time, she approached Morgan Runger. She offers to share her knowledge with him so that together they may conquer and rule. She knows how to acquire great power, but she needs some \quo{starting capital} to do it. So she and Morgan can help each other. 
% Of course, this entire story is pure fiction. Takestsha is really a Dragon and her knowledge is Sentinel knowledge. The story is just there to give a plausible excuse for her role and her knowledge, to convince a Cabalist reader that she is not a Sentinel. 
%
% How does Tantor get these hints? Perhaps Takestsha has sex with him. And he is ashamed, because he has a wife, so he doesn't write it directly. But Charcoal is savvy enough to figure it out. 
%
% They return to the king's castle. Takestsha talks to him in private. 
%

