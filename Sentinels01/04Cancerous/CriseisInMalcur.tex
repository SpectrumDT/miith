
\bookchapter{\Criseis in \Malcur}
\begin{comment}
  \section{\Criseis smells the city}
\end{comment}
\Criseis should be surrounded in more dark magic.

\Criseis{} goes to \Malcur to see how things are going. 

She uses a flying spell.
She alights on the roof of a high tower and gazes out over the city. 

She reaches out with her senses. 
Her senses are very keen. 
She can clearly feel that something nasty is afoot. 

First of all, there is a smell of disease.
A dusty smell of death.  
It is a smell she knows well. 
In recent decades she gets it in almost every city in \Velcad, and many other places as well. 
There has been a steady increase in disease these last many years. 
There is a particular sickness, or family of sicknesses, that is spreading all over \Azmith. 
A leprosy-like wasting sickness. 
The \quo{ghost-sickness} it was called in some places. 
In others, the \quo{living rot}. 
Here in Pelidor the poor unimaginatively called it simply \quo{the Disease}. 

It could just be some natural, earthly disease. 
Or it could be something more sinister. 
\Criseis has her suspicions, but so far they are nothing but hunches. 

There are also other things afoot in \Malcur, but they are subtler. 
There is an eerie taint that is seeping in. 
Like moisture seeping up from underground, causing the city's foundations to rot and moulder away. 
But this, too, is just a vague sensation. 
Nothing she can pinpoint. 

\tho{Well, I must get to work.
  I have much work to do. 
  Both the Sentinels and the Cabal are up to something here in \Malcur.
  I need to find out what.
  About both of them.
  
  It seems I am always out on reconnaisance like this.
  I often wonder how useful all my reconnaisance is.
  Master \Quessanth has other channels.
  He knows many things that he does not tell me.
  Often I come to him bringing what I think to be brand-new information, only to find out that he has known long in advance. 
  I wish he would tell me more so I would not waste my time.
  I know he has his reasons, but he is not easy to work for.
  
  Hah. 
  What am I talking about?
  I serve the one who is called \quo{Exile} and \quo{Destroyer}, hated and feared by Cabal and Sentinels alike.
  Most people, even immortals, would consider themselves lucky to merely survive an encounter with him, and I complain about his being \quo{not easy}?
  Hah. 
  \Criseis, you are getting old and complacent.
  
  At any rate. 
  Master \Quessanth may or may not know what is going on in \Malcur.
  He definitely knows more than I do.
  But I will still do what I came to do.}

(Read about \Criseis and how she \maybehr{Criseis fears Dragons}{fears the \dragons}.)





\begin{comment}
  \section{Questions a thug}
\end{comment}
So, wrapped in Shroud to make her unnoticeable, she sweeps through the slums of \Malcur. 
After some searching, she finds what she is looking for. 
A bandit with the smell of \rethyactic sorcery about him.
\tho{Probably one of these Black Plague thugs. 
  In league with the Sentinels.}

She swoops down before the bandit. 
It is a \human man, rather young. 

He sneers with hate as he sees her. 
He is about to threaten or attack her. 

\ta{Halt,} she commands him in \TrueDraconic. 
(Make up a spellword.)
He immediately stops up. 

\tho{Good. He is weak of will. This should be easy enough.}

He looks stupidly at her for a moment. 
Then he backs away.
Fumbles for his dagger. 

\ta{Stand still.}
The spellword forces him. 
He stands still. 
\ta{Whom do you serve?}

Thug: \ta{I serve (some Sentinel boss).}

\ta{Whom else?}

Thug: \ta{(Some more names.)}

\Criseis recognizes one name. 
Good. 
She is on the right track. 
\ta{What is your masters' plan? Tell me everything you know.}

The thug talks. 
She can tell he doesn't like it. 
He is afraid. 
At first he wanted to lunge and knife her.
Now he just wants to run away. 
But she uses her mind control to hold him and force him to tell what he knows.
(Her mind control only works on weak minds.)
With her \uber-sensitive senses, she can tell that he is not lying.  

\tho{I am afraid cannot let you run. 
  I do not want it known that I am interrogating people.}

She then uses her mind control to make the man stab himself. 
Once, twice, three times. 
(Her mind control only works on weak minds.)
Portray how the bandit struggles against the spell.
Describe his dread and horror as he is forced to kill himself.

She did not like having to do that.
She is no \dragon. 
She does not like killing. 
But this was a bad man, she reminds herself. 
She saw the stupid aggression on his face and in his body language. 
She is good at reading people's personalities, and this man was scum. 
A coward and a bully, preying on the weak. 
He had probably made many lives miserable in his time. 
\Miith was better off without him.
Yes. 
It was right to kill him. 
Yes. 

She bends down. 
Ruffles through his pockets. 
She does not need anything he owns, of course, but she wants it to look like the man was mugged and killed by another robber. 
She finds a few copper coins, which she takes. 
The man's knife is probably also worth a bit, so she takes that, too. 

Then she walks away. 
She dumps the money at the feet of some random beggar on her way. 





\begin{comment}
  \section{Visits \Uswa}
\end{comment}
\Criseis{} goes to the \maybehs{dead garden} and talks to \maybehr{Uswa}{\Uswa}. 
She recognizes that, though mad, \Uswa{} knows more than most people credit her for. 

First we see \Uswa{} alone. 
She is mumbling. 
She communes with \quo{things in the ground} and \quo{things in the air}. 
She knows that there is more than one \quo{tribe} of \quo{things} vying for dominance. 
She can see the chains leading down into the deep, and the spiritual chains that tie people to \Nyx{} and the Cabal \Matrix. 

Then \Criseis{} comes. 
\Uswa{} tells her stuff. 

\Uswa{} sees through \ps{\Criseis} mortal guise and sees that she is really immortal. 

\Uswa is in bad shape due to malnourishment and hunger.
\Criseis gives \Uswa some coins in sympathy.





\begin{comment}
  \section{Meets \Teshrial}
\end{comment}
\Criseis is about to leave \Malcur. 
She has snooped around a lot. 
Then she feels a presence approaching. 
An immortal surfacing through the Shroud. 

She closes her eyes and feels.
It is a \resphan. 
A \ketheran. 

She hears a voice.
\ta{Snooping around in my city, are you?}

\ta{Lord \Teshrial,} she says. 
She replies in the same language, the \resphan tongue.
She turns around to face him. 
She catches a flicker of a grimace on his face; perhaps he is disappointed at how she detected him when he though he was sneaking up on her. 

She looks up at him. 
He is huge. 
She is taller than many \scathae, but he stands almost twice her height. 
He is impressive. 
His vast wings make him look even huger. 
His white feathers, hair and robes make him shine like a sun of white light. 

She has only met him once before, and back then she was safe, protected by her master. 
In comparison, \Teshrial looked much less impressive. 
Now, when she is alone and unprotected, she feels the full force of \ps{\Teshrial} presence. 
He is intimidating. 
His black face looks down at her from very far above, with a predatory smirk. 

Before she had not respected \Teshrial so much. 
She has researched him a bit. 
He is young by \resphan standards. 
Much younger than she.
So she had not taken him seriously as a threat. 
But now, looking at him, she realizes he is actually very dangerous. 
He is an expert fighter and mage. 
She can tell just by his body language, the way he moves.
And the way he came at her through the Beyond. 
He is formidable, perhaps more so than many \resphain twice his age. 

\tho{%
  We have a dangerous tendency to think that \quo{elder} equals \quo{powerful} and \quo{younger} equals \quo{weak}.
  \Teshrial is a reminder that this is not always true.}

He is armed with a \ruthil sword and two \ghijed pistols. 
But even without them, she knows he could tear her apart with magic alone. 

\Teshrial:
\ta{So. What are you up to?}

\Criseis:
\ta{I fear I cannot divulge that, Lord \Teshrial.}

He comes closer. 
\ta{You have some nerve to come here into our territory and spy on us, \scatha.
  And then being cheeky to a \ps{\resphan} face.}
He looks at his fingers, as if distracted and uninterested. 
\ta{Your name is \Criseis.}

\ta{Yes.}

He looms over her.
\ta{Do you fear me, \Criseis?}

She forces herself to look him in the eye.
\ta{Yes, Lord \Teshrial, I fear you.
  But you do not want to threaten me. 
  I am under \ps{\Ishnaruchaefir} protection.
  If I am harmed, he will avenge me.}

\ta{I do not fear your master, \Criseis.}

\Criseis was afraid he would say that. 
From her Sentinel agents she has heard stories about how \Teshrial is allegedly planning to face \Ishnaruchaefir again and defeat him. 
So if she wants to keep \Teshrial from killing her, she has to play harder. 

\Criseis:
\ta{%
  \emph{You} may not fear my master's wrath, Lord \Teshrial. 
  You are a formidable warrior, I know. 
  But not everyone is so fortunate. 
  If you harm me, my master may decide to respond in kind. 
  I hear you breed \humans.
  Do you care their lives, my lord?
  How would you like to have a village of them suddenly consumed in a fireball?
  That is the least of what my master might do to you if I were killed.}

\Teshrial face becomes outraged.
\ta{Are you threatening me, \scatha?}

\Criseis:
\ta{Not I, my lord.
  I would never condone the slaughter of defenseless \humans.
  But my master is more cruel than I, and I cannot answer for his anger.}

\Teshrial looks at her with hate.
He does not like how she threatens his \humans.
And \Criseis does not like to do it, either.

\Criseis: 
\ta{So far Master \Quessanth{} has maintained a certain diplomatic code of conduct. 
  A certain mutual respect. 
  He will abandon all such restraints if you harm me.
  I do not want that, and I think neither do you, my lord.}

\Teshrial:
\ta{Hm. 
  Away with you.
  I am no coward, to attack a defenseless mortal.
  Even a spy.}
\Criseis notices how \Teshrial twists it so it looks like he acts out of his own chivalry, not in response to her threat.

\Teshrial:
\ta{But know this:
  I will face \Ishnaruchaefir again.
  And next time I will not be unprepared.
  Next time he will not be so lucky.}

\Criseis looks at his face. 
There is a new self-assurance there.
This is not just smug overconfidence. 
This is determination. 
He knows something. 
He believes he has the means to do what he says.
And \Teshrial is no fool. 
This is no empty threat.
\Criseis does not like it.
He is up to something, and she wants to find out what. 

She resolves to do more research. 
Snoop some more. 
Squeeze more information out of her contacts in \Nyx. 
She needs to know what \Teshrial is up to. 

She submerges into the Beyond and flees. 





\begin{comment}
  \section{\Teshrial thinks}
\end{comment}
\Teshrial stands alone and watches her go. 
He had never intended to kill her. 
That would be low. 

But now he is shaken. 
At the \ps{\scatha} threat. 

He should not be surprised. 
He has read about the many atrocities the Destroyer has committed in his fits of rage. 
But until now it had been academic for him. 
To have the lives of his own \humans threatened so directly\prikker it brought the whole thing much closer. 
Down to a personal level. 
And it struck harder that way. 

To think that the Destroyer would stoop so low as to hold innocent \humans hostage. 
That he would torment and kill the precious things just out of spite. 
Such cruelty!
Such evil! 

Thinking of it fills \Teshrial with renewed hate and renewed determination. 
This wicked monster must die. 







