
\bookchapter{\Teshrial Meets \Urizeth}

Remember to have plenty of \maybehr{Resphan wing body language}{wing body language} from the \resphain. 

Make clear that \Teshrial is obsessed with \Firaxel and falling badly in love with her. 
    
Mention that the \resphain \maybehr{Resphain resent Mirage Asylum}{resent how \Ishnaruchaefir hides in his Mirage Asylum}. 

Mention that \Teshrial does not understand much occultism.
He is first and foremost a martial artist.
He is a skilled mage, but he knows magic only as a tool. 
He does not understand the deeper, darker mysteries of how magic works, or of the sources from which the \resphain draw their magical power. 
(This is a taboo for the \resphain. 
 They talk of how their magic comes from the Rose, and they do not want to dig any deeper than that.
 They fear to think about the \SitraAchras. 
 Make this clear from \Teshrial's internal monologue, but without stating it outright.
 Move this to a better section. 
 And insert pointers to it from all relevant chapters.)

See also the sections on \maybehr{Nyx}{\Nyx} and \maybehs{dark ancient cities}. 
    
Have references to the fact that \maybehr{Heart weakened}{the Heart is weakening} (and read the section). 
Perhaps in conjuction with magic or birth or the Shroud.  
Or the Nadir. 

Have throwaway references to \maybehr{Mystic names}{mystic names and places}, like Shung. 
Read about mystic names. 



\begin{comment}
  \section{Teshrial flies}
\end{comment}

\Teshrial flies through \Nyx. 

Since his last talk with \Menessiaraid he has looked around for a scholar to help him. 
There are not many who understand \WanderersInDarknessEmph. 
Also, \Teshrial is none to keen on admitting to everyone that he is interested in \WanderersInDarknessEmph. 
It is, if not forbidden, at least ill-seen. 
It is a piece of evil that a wholesome \resphan with a reputation to maintain does not want himself to be seen anywhere near. 
So he has been covert. 

He has talked to a few people in \CiriathSepher, but he has found none that he felt he could trust. 
Moreover, he is beginning to think he should look outside \CiriathSepher. 
A \CiriathSepher would quickly leak the rumour, and it could damage his reputation. 
A \TiphredSerah, on the other hand, might be more discreet. 

So he thought of this \Urizeth of \TiphredSerah, whom \Lothagiel mentions with favour. 
She has a rumour of being a weirdo, but then, what would you expect of a \TiphredSerah astrologer and poetry analyst? 
\Teshrial decides it is worth a shot, so he contacts \Urizeth. 
He sweet-talks her (via messenger), telling her he is very interested in her research and wants her help with an occult task he is working on. 

Now he has set up a meeting with \Urizeth and is going into \TiphredSerah to visit her. 

\begin{comment}
  \section{Teshrial meets Urizeth}
\end{comment}

He lands and is shown in. 

There is \Urizeth. 

(Read about \Urizeth! Make her really eccentric!)

She is a skinny \resvil of medium height with an angular face. 
Her hair and feathers are a dark gray \colour, apparently her natural \colour (not dyed). 
She wears a dress of sky blue, white and black. 

\Urizeth: 
\ta{Lord \Teshrial.}

\Teshrial: 
\ta{Lady \Urizeth. I thank you for your hospitality.}

She offers a drink. 
They smalltalk a bit. 

Bored, he let his eyes wander, gazing out over the walls. 
They were perhaps a hundred metres high and covered in shining mirrors. 
Wherever he gazed, his and his companions' reflections looked back at him. 
% made of solid dark iron but covered in shining white enamel. 
% Deep grooves ran from floor to ceiling, their edges looking, from a distance, as sharp as swords. 
The hall was brightly lit by huge golden chandeliers, each one holding more than a hundred candles. 
They hung from the ceiling, but no higher than ten metres above the floor. 
Above them there was no illumination. 
The reflective mirror walls did their best to light up the entire hall, but in the end the light faltered and failed, and the walls simply vanished in the gloom overhead. 

\tho{%
  A reminder, perhaps, that all light is ephemeral and ultimately does not hold a candle to the primordial darkness. 
  Or something. 
  
  Haha. 
  Hold a candle. 
  Good one.}

At intervals, distant balconies, ledges and catwalks were dimly visible, but the ceiling was shrouded in pitch darkness. 
Only with his arcane senses could \Teshrial{} perceive the strange mosaic carved into that unseen dome above. 
He could not quite interpret the motif, but he recognized those weird, twisted shapes of uneathly geometry as a veiled metaphor expressing the cosmic \maybehr{Matrix}{\matrices} and their \maybehs{constellations}. 
The \matrices, those mystic star-maps that depicted the forces of the universe and their balance of power. 
\tho{%
  So, a dimly lit mosaic showing a weird picture of an obscure metaphor for an occult representation of a metaphysical abstraction of political reality. 
  Aaargh. 
  Now I remember why I chose \emph{not} to pursue the visual arts.
  Or \matrix{} theory, for that matter.}
% veiled metaphors for fearful, cosmic themes pertaining to his people's dark ancestry, and to the world of \Erebos{} and its terrible overlords. 

\index{\resphan}%
Disturbed by these sinister hints, he retracted his aethereal senses from the mystic mosaic and returned his attention to the\prikker

\Urizeth: \ta{So. What do you want?}

\Teshrial: 
\ta{Well, Lady \Urizeth, I am interested in your research on the poem \WanderersInDarknessEmph.}

\Urizeth: 
\ta{Really? You did not strike me as a poetry or astrology enthusiast, Lord \Teshrial.}

\Teshrial: 
\ta{Exactly. And therein lies my problem. Tell me, Lady \Urizeth, do you know the name \Lothagiel?}

\Urizeth (thinks for a while): 
\ta{No.}

\Teshrial: 
\ta{\Lothagiel was a \resphan of \Mystraacht who, some centuries ago, tried to combat the Destroyer, \Ishnaruchaefir.}

\Urizeth: 
\ta{Hm. From your wording I gather he failed and was destroyed?}

\Teshrial: 
\ta{Yes. But despite that, he did a good job of preparing himself, and there is much we can learn from him.}

\Urizeth:
\ta{Learn? To what end?}

\Teshrial:
\ta{To complete the task at which he failed.}

\Urizeth (laughs): 
\ta{You mean to slay \Ishnaruchaefir?}

\Teshrial (leans forward, reveals his emotional involvement):
\ta{Yes. Lady \Urizeth, I mean to challenge and destroy \Ishnaruchaefir.And I want your help. This will be the greatest \resphan triumph in the \feud since the Unspoken Covenant began, and I want you to be a part of it, Lady \Urizeth.}

When \Urizeth first talks about \Ishnaruchaefir, she \maybehr{Ishnaruchaefir's names}{lists his full name and explains all the parts she knows}.
\quo{Quessanth} is an egg-name and has no meaning.
The meaning of \quo{\Ishnaruchaefir} is unknown.
The \dragons have never taught the \resphain the secrets of their language.

\QuessanthIshnaruchaefir was perhaps the greatest and mightiest of his people, second only to the fallen \Nexagglachel. 
It was he who led his race in the slaughter of thousands of \resphain who would never rise again. 
But in his murderous zeal hye made terrible pacts with those unnameable, loathsome elder cosmic gods, the \xss. 
Pacts so hideous that it horrified even his brethren.
So after the catastrophic Shrouding which ended the war, \Ishnaruchaefir was named \quo{Exile} by his own brother. 

The \dragons are terrible beings, and none more so than \Ishnaruchaefir.
\Urizeth confesses that until now she had never met a \dragon in the flesh.
She had read much about them and done research, but only from a theoretical point of view. 
She \quo{knew} how terrible and mighty they were, but not until now did she really understand it.
Now it is suddenly real to her in a new way. 

\ta{%
  The \dragons are keepers of a wisdom far more ancient than what we \resphain know.
  \QuessanthIshnaruchaefir himself was an old and mighty patriarch among his people long before even \Thanatzil, father of the \resphan race, was conceived.
  Their race is perhaps as old and terrible as even the \SitraAchras.
  Indeed, when the \SitraAchras first set foot on \Miith, long before the \resphain\prikker then the spawn of \Sethicus were there to challenge them.}

\Teshrial knows that \Sethicus is the mythical founder of the \draconic race.
Their equivalent of \Thanatzil.

\Teshrial shudders (gyser) when \Urizeth mentions the \SitraAchras, the \resphain's sinister gods/creators/progenitors/ancestors.

\Urizeth is wise in the occult matters, but even she does not know much about the \SitraAchras.
It is forbidden knowledge. 
But it is rumoured that the first primordial war of the \SitraAchras against the brood of \Sethicus was fought a million years ago, and that \Ishnaruchaefir himself fought in it. 
\Urizeth cannot vouch for those legends. 
She does not know how true they are. 

She also knows little about the Durance. 
She knows only legends. 
The \resphain never learned much about what happened during the Durance, since no one would tell them. 

\Teshrial is disturbed.
He knew even less about the matter than \Urizeth did, for he has always shunned all literature having to do with the \SitraAchras.
He is afraid of them. 
Only recently, after \Urizeth's death, did he begin studying the occult. 


\Urizeth (retains her composure):  
\ta{Why me, Lord \Teshrial?}

\Teshrial: 
\ta{\Lothagiel believed \WanderersInDarknessEmph contained the key to defeating \Ishnaruchaefir. He quoted and referred to your work on the subject. You are one of the pre-eminent experts on the poem. I believe that you are the right person to help me.} 

(More sweet-talking.)

(\Urizeth is shrewd, but she is something of a nerd. \Teshrial is a superior negotiator. He convinces her that she wants to \cooperate with him.) 

Mention that \Urizeth is working from \Essenai's translation of \WanderersInDarknessEmph. 
She rants:
\talk{%
  \Essenai was a wise scholar, before \Kezerad was destroyed and she was imprisoned in her \beacon and turned into her current form.
  Now I suppose she is done thinking deep thoughts.}
The last is said with a grimace, and \Teshrial is not sure if \Urizeth is being sad or morbidly amused. 



\begin{comment}
  \section{Lothagiel's notes}
\end{comment}

\Urizeth:
\ta{You have my attention, Lord \Teshrial. Show me \ps{\Lothagiel} notes, please.}

He shows them to her.

\Urizeth: 
\ta{Interesting. Indeed, these are references to my writings.}

\Teshrial:
\ta{You knew nothing of this? I am surprised he did not contact you himself, given his regard for your work.}

\Urizeth: \ta{When did this take place?}

\Teshrial: \ta{(Some year.)}

\Urizeth: \ta{Ah. At that time, relations between \TiphredSerah and \Mystraacht were strained to say the least. It might have been diplomatically infeasible. Moreover, \Lothagiel was \Mystraacht, and I am sure I need not tell you how headstrong they come.}

She reads through the notes. 
He lets her take her time. 
She gets more excited as she reads through them. 
As far as \Teshrial can see, she is able to understand it all much quicker than he was when he read them. 
She lights up with an expression of dawning understanding. 

\Urizeth:
\ta{This is\prikker fascinating. 
  A recurring Nadir\prikker yes, that makes perfect sense.
  The Exile is a major \vertex, after all.
  Strange, though, that no one has mapped it before, if it is true\prikker}

\Teshrial:
\ta{\Lothagiel has a reference to your name here: 
  \quo{\Urizeth: waters = \bane \matrix}.
  Can you tell me what he means?}

\Urizeth:
\ta{Yes. 
  The \quo{waters} in question are a metaphor that occurs many times in the text.
  I am convinced it refers to the \matrix of the \banelords.}

\Teshrial:
\ta{Then what does the rest of the passage mean? Do you know?}

\Urizeth:
\ta{I have some ideas, yes\prikker}
Then she composes herself.
She realizes she should not be giving away information for free. 
\ta{So. Lord \Teshrial.}

\ta{Lady \Urizeth.}

\ta{What is your proposal?}

Negotiations begin.
He wants her to help him. 
He will not pay her anything, but the message he sends is this:
\tho{%
  If I fail in my task, then everything you will have told me must have been useless, and thus you deserve nothing. If, on the other hand, I succeed, then I will see to it that you are acknowledged as a vital ally.
  You will be known world-wide as the scholar who cracked the mystery that allowed the \resphan people to win their greatest victory in the entire Time of the Covenant.}

At the end, \Urizeth agrees:
\ta{Very well, Lord \Teshrial. Let me keep a copy of the notes.
  I will study them and tell you everything I learn.}

\ta{Splendid. I look forward to our collaboration, Lady \Urizeth.}









