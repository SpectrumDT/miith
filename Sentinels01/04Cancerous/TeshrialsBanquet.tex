
\bookchapter{\ps{\Teshrial} Banquet}





Make sure to make the atmosphere in this chapter as dark as possible. 
Teshrial-tachi drink human blood from chalices, and eat Human and Scatha flesh. 
They have huge, deformed ogre servants and monsters. 
And there is dark sorcery all over the place. 

See also the sections on \maybehr{Nyx}{\Nyx} and \maybehs{dark ancient cities}. 

Remember to have plenty of \maybehr{Resphan wing body language}{wing body language} from the \resphain. 

Make clear that \maybehr{Resphain love conflict}{\resphain love conflict}.
\Teshrial sort of likes having been killed. 
He enjoys hating \Ishnaruchaefir. 
It is just as strong a feeling as being in love.
    
Mention that the \resphain \maybehr{Resphain resent Mirage Asylum}{resent how \Ishnaruchaefir hides in his Mirage Asylum}. 

Remember to read about the {\Malcur venture}. 

\Teshrial and \Menessiaraid are gay lovers. 
They kiss and have sex.
    
Have references to the fact that \maybehr{Heart weakened}{the Heart is weakening} (and read the section). 
Perhaps in conjuction with magic or birth or the Shroud.  









\begin{comment}
\subsection{Flying}
\end{comment}

\Teshrial{} fights a friendly duel with his friend Tyrsed.
Tyrsed is a Thelyad, but still a good friend. 
Many of Teshrial's friends are Thelyadeth. 
After all, there are not many Ketherain. 
Anyway, they fight. 

Tyrsed is good. 
Teshrial has learned much from him, especially in his youth. 
Tyrsed is somewhat older (though not as much older as Menessiaraid), and for much of their youth he was better than Teshrial. 
One of the many people who Teshrial learned from and strove to emulate. 
But these days he is surpassing Tyrsed and winning most of their jousts. 

Tyrsed gets a nasty blow in, penetrating Teshrial's wing. 
Then Tyrsed taunts him.
That is his mistake. 
He gets overconfident and wastes time taunting. 
He fails to see how Teshrial recovers, closes in and strikes. 
Hard. 
A bit too hard. 
Teshrial had expected Tyrsed to see it coming, but he got a really lucky strike in. 
He knows he has made a mistake as soon as he sees his Senaan impale Tyrsed's chest. 

Tyrsed admits as much with his last breath:
"Fuck. I should have seen that coming. You got me, Teshrial."
Then Tyrsed dies. 

Teshrial stands back thinking. 
"Hm. Maybe I should not have done that."
Killing people in a duel is somewhat frowned upon, but also sort of cool, since it proves how obviously superior you are. 

But enough of that. 
It is well enough that the duel ended.
The fighting was just a stress-down thing. 
He has only five hours or so before his guests arrive. 
He has preparations to do. 

"Good thing it was him who died and not me.
That would have been awkward."
He imagines his servants going: 
'We are very sorry, but Lord Teshrial is dead for the time being\prikker'
"By the Rose, that would have been bad. 
It would make me look ridiculous. 
Childish, reckless, incompetent, irresponsible. 
And in front of Firaxel, of all people.
Come to think of it, this stress-off duel was actually a singularly bad idea.
Lucky nothing bad happened.
Well, less lucky for Tyrsed.
But I will send my people to tend to him.
He should recover in time to be able to attend some of the party, at least."

Today was a big day for Teshrial. 
Several of his important friends and relatives were coming to his banquet. 
Ketherain, heroes and contacts from the other dynasties. 

And most important of all, SHE would be there. 

SHE was Firaxel. 
Descendant of a Sathariah of \CiriathSepher, and that other Sathariah of \TiphredSerah. 
Brilliant scientist and artisan. 
Also a weapons designer. 
And perhaps the most beautiful Resvil Teshrial had ever met. 

He has been wooing her for a while. 
Finally, she has accepted his invitation. 
He must look his best. 
Firaxel is a highly desired Resvil, so her standards are high.
But he \emph{will} have her. 
He will make sure of that. 









\begin{comment}
\subsection{Party begins}
\end{comment}
\new
The party is beginning. 
Menessiaraid is there. 
Ganethed is there. 
Several others. 
Even Achsah is there. 
She is, after all, fourth circle, and one of his closer associates in the Cabal. 
It would be rude to exclude her, even though she is a Beuzed. 

Teshrial has given much thought to his own attire. 
White has always been his \colour, so he has chosen a white coat with purple edges. 

His eyes are pink, which presents a challenge. 
If handled clumsy they can give him a too effeminate look. 
He has chosen not to hide them but to embrace them. 
Hiding them would be lame. 
That would display weakness and insecurity.
No, that would not do.
So he has dyed his long white hair with a single magenta stripe, running down from the top of his head to his left shoulder. 
The \colour accentuates his eyes, and in a good way. 
He also wears a magenta sash, from right shoulder to left hip, to match. 

In his belt he wears two ghijed holsters but no ghijedeth. 
He must show that he is a warrior, but with moderation. 
He must not seem to eager. 
Carrying actual guns would look paranoid or barbaric. 
No, the \CiriathSepher are subtler than that. 
Holsters are enough.

Then a trumpet sounds from above to mark a new arrival. 
Teshrial looks up, hopeful. 
In from a high window comes a shape. 
A Resvil. 
Feathers white and tipped with violet. 
Black hair. 

It is her. 
Firaxel. 

Dress violet and azure. 
Instead of a skirt, her dress splits from the waist down into several tails flowing out in all directions. 

She circles down and lands. 
Her dress-tails, made of very thin, light fabric, gently drift down and settle around her legs. 

He approaches her. 

"My Lady Ketheran Firaxel. 
 How delighted I am to see you."
He sweeps up her wing with his and brings it closer to him, then gently runs his fingers down her feather-tips.
"Once again I find myself stunned by your beauty, Lady Firaxel." 

"Thank you, my Lord Teshrial."
She withdraws her wing, slowly but confidently. 
No chance of letting him get too intimate just let. 
Of course. 
He would have it no other way. 
He would have been disappointed in her had she done otherwise. 
No one wants a Resvil who does not know her own worth. 

They chat and exchange a few more pleasantries. 

Teshrial: 
"I believe you all know each other? 
 Splendid. 
 It seems we are all gathered now.
 All save my good friend Lord Tyrsed, who is, regrettably, unable to join us until later this evening. 
 I must take the full blame for that. 
 We jousted earlier today, and a blunder of mine unfortunately left the good Lord Tyrsed indisposed."
They all laugh at that.
They know what "indisposed" means. 
They have all seen it happen before. 
"But rest assured Lord Tyrsed will join us as soon as he is able.
 I hope to see him back to health before the night is done.
 Now, without further ado, let us proceed to the feast."









\begin{comment}
\subsection{Teshrial's battle}
\end{comment}
\new
The meal is in full gong. 
Teshrial has been flirting with Firaxel all evening.
She teases him. 
Resviel are expert teases. 
He knows it is a good sign. 

Teshrial himself makes an effort to display a moderate amount of attraction, but not too much. 
Must not appear needy. 
No Resvil wants a Resphan who throws himself at her feet. 
He must impress her while not seeming like he is trying to impress her. 

And she is hard to impress, he knows. 
Firaxel is a Ketheran of the finest blood. 
She is an especially beautiful Resvil and a formidable scientist and sorceress. 
She is a bridge between \CiriathSepher and \TiphredSerah and a goldmine of political connections. 
All those are important virtues, but Firaxel has another virtue that towers over all those: 
She is fertile. 
She is not old - no more than 1500 years - but she has already borne all of three children. 
And that attracts Teshrial. 

Teshrial wants children. 
Every Resphan's goal and duty is to replenish his race. 
That is what they live for. 
It is part of their great mission: 
The perpetual improvement of the Resphan race towards perfection. 
He is a desirable Resphan himself, being a rising star in the ranks of \CiriathSepher heroes. 
He has been able to attract his share of Resviel over the centuries. 
Ketherain and Thelyadeth alike. 
But however pleasant that has been, it has not really paid off. 
None of them have been fertile. 
Not good enough. 
Teshrial craves more. 
Fucking a barren Resvil is only one step up from masturbation. 

Firaxel, now, is the real deal. 

One of the guests now speaks. 
"So, Lord Teshrial. 
 We have heard of how you encountered the legendary Quessanth Ishnaruchaefir.
 Will you not tell us about it?"

Teshrial smiles. 
\tho{I thought you would never ask.}
He could not bring up the subject of his own initiative. 
That would feel like bragging and would lower his value. 
Nor could he have one of his close friends ask him about it. 
That would be conspicuous, a feeble attempt at subtlety.
That would make him look not only a glory-hound, but also crude and without finesse. 
So he had to wait for someone else to bring up the subject. 

"Well, since you ask.
 My esteemed colleague, Achsah, picked up signs of the Destroyer in \Malcur. 
 I feared he would endanger our operations there, so I felt it my duty to intervene."
And Teshrial tells the story. 
He is careful to balance the presentation of himself. 
He wants to look heroic, but plausibly so. 
In \CiriathSepher one always walks the tightrope of etiquette.

And it works. 
Firaxel's eyes widen. 
He has her undivided attention. 

He focuses how he was outclassed by the dreaded Ishnaruchaefir and had no hope of winning against him. 
Thereby implicitly emphasizing how brave he was to stand against him. 

Firaxel smiles. 
Her lips part, ever so slighty.
\tho{Did she just lick her lips.}

Achsah supports him:
"I had not the courage to face the Destroyer.
 I am not afraid to confess that.
 When my Lord Ketheran Teshrial fell and the Destroyer addressed me, I was just about to flee."

Teshrial emphasises how he used skill, speed and cunning, but was ultimately brought down by brute force. 
This gains him extra respect. 
It is well-known that the average Resphan is no match for the average Dragon in raw power. 
It is also widely accepted in \CiriathSepher that this advantage is "cheating" on the Dragons' part and earns them no respect. 
In their eyes, Teshrial was the winner "in spirit", and Ishnaruchaefir managed to kill him simply by a fluke of nature. 
Dumb luck from Nature's side, that the Dragon was born with a bigger and tougher body. 

Teshrial tells of how he was ultimately overcome and his body slashed apart. 
He describes the pain as the \XzaiShann-enchanted glaive tore through his flesh with its multiple serrated blades.
How the unnatural "tears" dripping from the blade seethed and ate his flesh like acid. 
The blackness and desolation as his senses fled. 
The dull, eerie pain as he lay in limbo, barely conscious, and dimly felt his body dissolve, severing his last connection to the physical world. 
Against his own expectations, Teshrial shuddered with genuine revulsion/horror/unpleasantness as he recalled it. 
It had been scarring. 
He had been killed before, but never had he had his body completely destroyed. 
The glaive Rystessakhin was a gruesome weapon. 
As was to be expected, giving its history. 
The weapon had been instrumental in Ishnaruchaefir's great crime against the world, his destructive strike against both the Resphain and his own people. 
The story of Rystessakhin was intertwined with the dark tale of the Destroyer's atrocities. 
The atrocities that made him the Exile, reviled by all. 
He is so evil he was even cast out of the Sentinels (or so \Teshrial believes). 

Teshrial glances at Firaxel. 
He notices that Firaxel gets aroused when she hears of his courage. 
She is a bit impressed and aroused when she hears the story of how Teshrial stood his ground alone against the dreaded \Ishnaruchaefir{} and had the courage to challenge him. 
And even to fight to the death! 
Ishnaruchaefir has a reputation for destroying people's souls, so it is impressibe enough that Teshrial survived. 
Teshrial thinks he was in no real danger, with the Shroud and all, but he does not mention that. 

Firaxel is turned on. 
Her breathing quickens. 
Her breasts heave, ever so slightly.
She is turned on. 
She tries to hide it a little bit, but not so much. 
It is embarrassing to let show that you are so emotionally affected by what other people say and do. 
But on the other hand, there is no shame in being attracted to feats of heroism. 
Quite the contrary. 

He is happy. 
\tho{What do you know.
  It looks like my lapse into emotion worked in my favour.
  Hm. 
  I suppose there is some kind of lesson to this.
  "Genuine emotion is sometimes better than a carefully maintained facade."
  Or something like that.
  Heh.}

He knows he has won a small victory and is one step closer to his goal. 

It is now time to bring in Evith. 

"Oh, look. 
 The main dish is coming in now."









\begin{comment}
\subsection{Evith}
\end{comment}
\new
Remember to read about {\resphan} culture and {\naor}{\naorim{} (food slaves)} before writing this!

We follow a {\naor} girl, {Evith}. 
She is to be eaten, and she {Voluntary drain}{offers herself gladly}. 
Her religion tells her that she will become one with her gods and achieve eternal bliss when she is eaten. 
They will eat more than one \human{} that day (four in all or so), but she is the delicacy of the evening, bred specifically by \Teshrial{} and some of the best he has to offer. 
And she knows it. 
She is top quality and very proud of being able to serve in this manner. 
This is the culmination of her life, the moment she has looked forward to as long as she can remember. 

She is giddy and excited, but also scared. 
Her handlers calm her down and reassure her. 
\begin{prose}
  \ta{It will be fine, Evith. 
    After all, you do not have to \emph{do} anything. 
    Just submit. 
    Fear not.
    You are in good hands. 
    The executioner is skilled and will take the best care of you.
    Now go.}

  She thanks them. 
  Hugs them. 
  Says goodbye. 
\end{prose}

When the moment of the {Communion} comes, Evith approaches the \resphain{}, prostrates herself and thanks them for blessing her with the Communion. 
\ta{I live to serve. My flesh and soul is yours.} 

Evith then strips naked (by taking off the special ceremonial robe), kneels down on a special block. 
Then another servant kills her with a {\gelveir}. 

\Firaxel{} is offered some of the most important parts of Evith. 
This is a chivalrous custom. 
A \resvil{} must always be kept as healthy as possible, because it makes her more fertile. 

Evith thinks back to the night before (her last night alive). 
She had sex with \Teshrial. 
It is customary for the one chosen for Communion to sometimes be \honoured with this special boon on her or his last night. 
It was Evith's first and only sex ever. 
She was nervous and afraid. 

\begin{prose}
  \Teshrial: 
  \ta{What ails you, child? Confide in me. I bid you.}
  
  Evith: 
  \ta{I am sorry, my divine lord \ketheran. 
    But\prikker I don't know what to do.
    I do not have the expertise of a [professional courtesan].}
  
  \Teshrial: 
  \ta{Sweet Evith, you are no [professional courtesan], 
    nor would anyone ask you to be.
    The courtesans are to be \honoured, but you are something special.
    You are a flower that will bloom only once, and there is a precious magic in that that no professional can duplicate.}
\end{prose}

She remembers the gentle touch of his silken soft feathers. 









\begin{comment}
\subsection{Ishnaruchaefir is a menace}
\end{comment}
\new
Clarify that \Ishnaruchaefir is a genuine and current threat to the \resphain, not just a past threat.
He keeps fucking up their schemes, and he is endangering a long-term scheme that is vital to \CiriathSepher if they want to rise to supremacy and realize their worth.
He preys on the \resphain and kills them and their servants. 
He has been passive for a long time, but now he is becoming a serious menace, and the Cabal fear him.

And he is casting spells (storm beacons and the like) that prevent the Cabalists from doing their thing in \Malcur.
They have a constructive goal in \Malcur. 
Clarify that.
They think \Ishnaruchaefir is the biggest threat against that goal, but it turns out Secherdamon is a worse threat. 

\Ishnaruchaefir is not so big a threat that all \resphain in the world are after him, though.
So far he is just endandering the \Malcur gambit, which only a small part of \CiriathSepher really care about. 
(Although the ones that do care have very high expectations of this gambit and hope it will determine the future fate of \CiriathSepher, if not all \resphain. 
 The ones not part of the gambit are more \skeptical. 
 Azraid has hopes for the venture, but remains \skeptical and aloof.)

Anyway, there are several \resphain who want to do \Ishnaruchaefir in, but he is notoriously elusive.
But he has promised \Teshrial to give him a rematch, and told \Teshrial to contact him when he is ready.
In some very clear terms.
So the \resphain know that if they want to do \Ishnaruchaefir in, \Teshrial is their best bet.

\Ishnaruchaefir is a tremendously powerful sorcerer and can call up hordes of monsters/demons to fight for him. 
He achieved much of his terrorism this way.
He functions as a one-man army of Chaos. 
This is one reason why the \resphain dread him so. 

Or perhaps \Ishnaruchaefir needs not attack and destroy stuff in order to be a threat. 
I just need to clarify that if he is not stopped soon (chased away or preferably killed), he will wreck everything they have worked for in \Malcur.
When he is at his full strength, he could attack in force and drive the Cabal out of \Malcur entirely.
It is known that he takes an interest in \Malcur, so he likely has long-term evil plans there.
He gave hints of that in WSB. (Make him give hints!)
That must not be allowed to happen.









\begin{comment}
\subsection{Teshrial's quest}
\end{comment}
\new
The dead Evith gets eaten. 

Teshrial tells of how he is not done with Ishnaruchaefir. 
He intends to prepare himself better and face him again. 

Some of the guests are horrified to hear this.
After all the horrible things he went through the first time, he wants to do it again?

Teshrial knows he has to play this exactly right.
He must not look the braggart, not seem a reckless glory-hound willing to do crazy things - or just claim willingness to do crazy things - to hog the limelight. 
He must appear cool, level-headed.
A man with a plan. 
Not a boy with everything to prove. 

He does well. 
He makes sure not to boast too much. 

He thinks about it to himself. 

He thinks about his mother and her great deeds. 
And his father, too. 

His prime goal is to score himself some Resvil poontang. 
\tho{I deserve \resvil{} pussy. 
  I have earned it. But I will earn it even more.}
But in a nicer phrasing. 

\Teshrial{} tells the others about the quest he has undertaken to bring down the mighty \Ishnaruchaefir. 
He knows it is risky and dangerous, but he wants to take risks and be bold. 
The \resviel{} love a hero. 
And he notices how \ps{\Firaxel} eyes light up when he tells about it, subtly implying his own daring. 

\tho{It is working.
  You can see it in her eyes.
  She is taking the bait.
  I must do this.
  I will not fail at this task.
  I will defeat the Destroyer.
  And then Firaxel will be mine.}

He cannot wait to have children. 
He and Firaxel will produce great children. 
They have great blood running in their veins. 
He is sure he would be a great father. 
He would teach by example and lead his sons and daughters up to be great warriors and researchers and artisans - heroes of the Resphain. 
It is going to be great. 
As soon as he can make Firaxel realize how worthy a mate he is. 

He imagines all the great things his children will accomplish. 
Not just for themselves nor for him, but for \CiriathSepher.
For the Resphan race. 
For their great project. 
Their great purpose. 

It is going to be great.

Teshrial is very satisfied with how this party is turning out. 







\begin{comment}
\subsection{Dezruth}
\end{comment}
\new
\Dezruth notices that \Teshrial moons over \Firaxel.

\Dezruth:
\ta{You want her. Am I right? You want her soft, juicy body.}
Make \Dezruth sound very chauvinist and sex-fixated.

\Teshrial mumbles.
He is not happy about the \Mystraacht's crude tone. 
It is disrespectful to him and above all to \Firaxel.
He does not want to think of \Firaxel as a piece of flesh.

\Dezruth:
\ta{If I were you, I would show her that I wanted here.
  By force.
  Right here, right now.
  I would grab her and kiss her.
  Then spank her bare ass here on the table.
  Show her who is boss.}

\Teshrial is not happy about the \Mystraacht lack of sexual norms.
He is a chivalrous man. 

\Teshrial halfway regrets having invited \Dezruth.
He likes \Dezruth. 
\Dezruth is not a bad \resphan, but he is awfully \Mystraacht.
He is too crude for this company, too barbaric.
\Teshrial should be more careful when sorting his friends into groups, he muses.







\begin{comment}
\subsection{Menessiaraid}
\end{comment}
\new
At first, Teshrial simply talks big about vanquishing \Ishnaruchaefir, but has yet to formulate any real plan for how to do so. 

Menessiaraid comes up to him. 
"So, Teshrial. 
 Do you have a plan for how you will overcome Ishnaruchaefir?"

Teshrial looks around. 
"Mmm. Still working on that."
He does not want everyone to know that he has no plan. 
He knows he cannot just waltz up to Ishnaruchaefir and challenge him to single combat. 
He would get wasted.
He has to plan ahead, investigate, find out Ishnaruchaefir's weaknesses and set a trap. 
But he has not planned much out yet. 

"It will be dangerous, you know."

"I know."

Menessiaraid: 
"I am not sure you do. 
 Ishnaruchaefir is infamous for destroying the souls of his opponents.
 I doubt there is any place you will be safe.
 I think you were in danger even in your first fight with him."

Teshrial: 
"Really?"

Menessiaraid:
"I believe there are reports of his having destroyed Resphain even quite deep inside the Shroud.
 He is powerful enough to overcome it if he really puts his mind to it.
 You should be careful."

Teshrial quickly suppresses a worried frown.
He is not happy to hear this. 
"You are right. 
 I should be careful.
 And I know I must do a lot more research.
 I will go study the records of sightings of him.
 See how he has fought in the past.
 When he has won and lost, and why."

Menessiaraid:
"It is an exciting thing, for sure, this quest of yours.
 If you dig in the archives, I am sure you will find that some research has been done before.
 The Destroyer has been a foe of our people for ages, after all.
 But such research has always been somewhat moot. 
 Ishnaruchaefir is elusive surfaces rarely.
 Seldom has anyone been given the opportunity to confront him on anything but the Destroyer's own terms. 
 If at all.
 
 But you have been given an extraordinary chance, Teshrial.
 From what you tell, Ishnaruchaefir offered to accept your challenge to a rematch, should you issue it.
 I do not know if this has ever happened before.
 It is an exciting opportunity.
 Not just for you, but for all of us.
 A chance to be rid of the Destroyer.
 
 He probably knows this, of course.
 So it will be a battle of wits.
 I think you will need all the help you can get in order to set up a proper trap and get at Ishnaruchaefir's weaknesses.
 You have to \quo{solve his Aenigma}, so to speak."

A few have believed to have found the answer and then gone off to fight \Ishnaruchaefir{}. 
None returned to tell the tale. 
\Menessiaraid{} knows of one such would-be hero who challenged \Ishnaruchaefir{} and fell: 
{\Lothagiel}. 
\Menessiaraid{} recommends that \Teshrial{} seek out \Nemuragh, who was a close friend or family member of \Lothagiel. 
\Teshrial{} later {\Teshrial{} seeks out widow}{does}. 

Or wait. 
Maybe it is not \Menessiaraid but \Dezruth who comes with this suggestion. 

\Lothagiel and \Nemuragh are \TiphredSerah. 








\begin{comment}
\subsection{Party ends}
\end{comment}
\new
The party is ending and people are leaving. 

Teshrial and \Firaxel{} softly caress each other with the tips of their feathers. 

\Teshrial{} mentally comments to himself how un-beautiful \Achsah{} is compared to \Firaxel. 
(Contrast this to how beautiful \Achsah{} appears to Needle.) 

At the end of the party, she kisses him on the lips. 
That is another victory for \Teshrial, but no guarantee yet. 
She gives him one of her feathers as a gift. 
This is arrogant of her, and he worries that he has lost value by supplicating and treasuring it. 

\Teshrial thinks to himself: 
\ta{You will be mine. Oh, yes. You will be mine.}





