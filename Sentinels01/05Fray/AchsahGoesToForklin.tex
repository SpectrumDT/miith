
\bookchapter{\Achsah Goes to \Forklin}

Read about \Forclin. 



\begin{comment}
  \section{Achsah suspects}
\end{comment}

\Achsah{} is attempting to unravel what \Ishnaruchaefir{} and \Secherdamon{} are up to. 
Insert some musing about the terrible dark lord \Secherdamon, and the mystic immortal \Ishnaruchaefir, whose motives no-one knows, but whose badass-ness is universally feared. 

She talks to Needle, after Needle has led a raid on some Sentinels and killed one of their mages. 
Needle has also gained much information from her spies about what the Sentinels are on about. 
She has plans to smack down on some more of their lairs soon. 
\Achsah compliments Needle on work well done. 

After the meeting with Needle, \Achsah goes out into \Malcur. 

She thinks about Needle. 
Needle is very handy. 
The Sentinels know about Charcoal, but they don't know about her. 
She is all close to \Tiroco{} and can spy on her and have her shadowed, and the Sentinels don't know it. 

\Achsah{} kind of likes Needle. 
She loves her for her imperfections. 
\Achsah{} herself is imperfect, too, being \bezed-born. 

But \Achsah is worried. 
Needle is being too effective. 
It is too easy. 
Too good to be true. 

She flies above the city and looks down. 
She can feel faint streams of magical energy, tell-tale signs of the Sentinels' spells. 
It has gotten more obvious lately. 
She needs but close her eyes and reach out with her senses into the Beyond. 
Then she can feel they ley lines pulsing like veins. 
She can feel the chaotic energy the Sentinels are channelling, pumping into the city. 

Granted, \Achsah is a High Telepath. 
Not every \resphan would be able to feel all this. 
But even so, it is odd that the Sentinels have not realized she could feel it. 
The Sentinels should be smarter than this. 
Something does not add up. 

She becomes convinced that there is something wrong with the whole \Malcur deal. 
It is too obvious. 
The Sentinels are leaving too many clues out in the open.
She has been a Sentinel and opposed \Secherdamon{} for thousands of years. 
She knows that being this overt is not his style. 
He is smarter and more stealthy than this. 

She becomes convinced that there is something deeper going on. 
She suspects that \Malcur is a decoy. 

At this point \Achsah has received word from Charcoal. 
He suspects the Ghost Tower has something to do with the Sentinels' plan. 
He has an idea that they are trying to capture \Forklin and the Tower. 
At first \Achsah dismissed it, but now that she thinks about it, it would make a lot of sense. 
This whole Rungeran invasion is awfully convenient. 
It is possible the invasion was masterminded by the Sentinels, and that they hope to seize \Forklin by mundane means (thus obeying the Unspoken Covenant). 

She submerges out of \Azmith and makes her way back to \Nyx. 









\begin{comment}
  \section{Teshrial stressed}
\end{comment}
\new 
\Teshrial remembers when he was told that \Urizeth had died. 

\Urizeth's next-of-kin: \ta{\Urizeth is dead.}

\Teshrial: \ta{She is \shout{what}?}

\ta{She was killed yesterday. 
  Her soul is still unharmed, but her body was completely destroyed, so it will take her some time to recover.}

\Teshrial: \ta{By \Erebos! 
  This is terrible news. 
  Have you been able to communicate with her?
  Who or what killed her?}

The other \resphan gave \Teshrial an accusatory stare. 
\ta{She was assassinated by the Destroyer. \Ishnaruchaefir.}

\Urizeth's death is a great setback for the \Malcur project.
But there are deputy occultists who can take over and manage the \noggyaleth until she gets better.
When she returns, she can probably still manage them.
And if she chooses to back out entirely, they can find another \noggyal-handler.
The astrology problem is worse.
\Teshrial has no backup astrologer.
And he is convinced she will back out of their project to kill \Ishnaruchaefir.

\Teshrial is poring over his notes. 
He is distressed. 

He has a picture of \Firaxel on his table. 
He looks at it all the time. 
It helps keep him sane.
It reminds him of why he keeps doing this, in spite of the setbacks and the bleak prospects and the horrible, incomprehensible poetry. 

\Urizeth is dead. 
Evidently \Ishnaruchaefir must have somehow found out that they were researching him and decided to put a stop to it. 

How did he find this out? 
Does he have spies all the way into \Nyx? 
Or does he have access to powerful divination magic?

A more immediate concern: \Teshrial has lost his astrologer. 
\Urizeth will revive, of course, but \Teshrial fears she is lost to him for good. 
He does not think she will have the courage to still oppose the Destroyer after he killed her so easily. 
If she stays, she might get destroyed, and \Teshrial knows \Urizeth is no warrior.
She has no obligation to stay, and her self-preservation might very well override her motivation to help him.
So far she has been helping \Teshrial mostly out of scientific curiosity, or so it seems to \Teshrial.
So he has good reason to fear \Urizeth is gone for good from his POV.

He is especially distressed because it had been a long time since his last meeting with \Urizeth, and he is sure she had discovered many new things in the meantime, which he now cannot learn from her. 

Was \Ishnaruchaefir trying to destroy \Urizeth, and she escaped through luck?
Or did he intend to let her revive, perhaps to carry back the message?
Either way the threat was clear. 
It was a warning. 
It must have been. 
It was \Ishnaruchaefir's way of saying:
\ta{Do not mess with me.
  I will kill anyone who tries to \quo{research} me.}

\Teshrial had lost \Urizeth. 
Should he search for another astrologer? 
Would that even be worth it?
If \Teshrial found another astrologer, would \Ishnaruchaefir not simply kill him as well?
Besides, once the rumour got out regarding \Urizeth, who would want the job? 

\Teshrial fears he is on his own. 
If he wants to solve the Destroyer's riddles, he will have to do it himself. 

But there are also good news: 
\Ishnaruchaefir took action. 
That means he feared what \Teshrial and \Urizeth were doing and wanted to put a stop to it. 
That means they were doing the right thing. 
They were onto something. 

\Teshrial is now convinced that there are truths waiting to be discovered within the pages of \WanderersInDarknessEmph. 
Powerful truths which the Destroyer fears. 
Truths that could destroy him. 

\Teshrial will continue. 
He must continue. 
With no astrologer to assist him, he must continue the research on his own. 
So he has taken to reading \WanderersInDarknessEmph himself, despite having little occult experience. 
(He is first and foremost a martial artist. Stress this!)

(Refer to the \maybehr{Resphain and forbidden books}{forbidden books of the \resphain}.)

He is reading cryptic WID passages. 

\begin{poetry}
  He that smote apart the veils and fabrics of the worlds\\
  and conjured down the pall of the black stars' howling empyrean.
\end{poetry}

He does not understand them, and he does not want to, because they are horrible and evil, but he must. 
They are wracking his sanity. 

He begins to see hints at connections he does not want to see. 
He does not understand any of it yet, and he tends to forget it whenever he is not reading. 
But whenever he reads the poem, he gets these feelings that the universe is much vaster and darker and more horrible than he is willing to admit. 
Images intrude on his mind. 
Images of horrible creatures: 
The \umbrae that feed on \resphan souls. 
The \banes, those sinister alien overlords with whom his people are allied. 
The \xzaishanns, those brutal and incomprehensible gods whom the \dragons invoke. 
He does not like it. 
He does not like it one bit. 

Have lots of \quo{unplumbed gulfs}. 
And look at some \emph{Warhammer 40,000} quotes. 

He thinks about \dragons.
They are horrible Elder creatures, far older than the \resphain. 
He senses their immense age and power and their burning hate for his kind. 
He begins to know fear. 
He imagines he hears the beat of their black leathern wings from moonless gulfs. 

\citeauthorbook[p.213]{RobertEHoward:TheKingandtheOak}{Robert E. Howard}{%
  The King and the Oak%
}{
  And through the tossing, monstrous trees there sing a dim refrain\\
  Fraught deep with twice a million years of evil, hate and pain:\\
  \ta{We were the lords ere man had come, and shall be lords again.}
  
  Kull sensed an empire strange and old that bowed to man's advance\\
  As kingdoms of the grassblades bow before the marching ants,\\
  And horror gripped him; in the dawn like someone in a trance
}

\lyricsbs{Bal-Sagoth}{%
  Into the Silent Chambers of the Sapphirean Throne 
  (Sagas from the Antediluvian Scrolls)%
}{
  Grim-eyed legions wait brooding\\
  'neath the banner of the Serpent King.
  
  The Atlantean sword beckons me,\\
  and I descend from Moon-shrouded skies \\
  into the Tower of the Black Serpent.
  
  And lo, I hear the beat of black leathern wings from Moonless gulfs. \\
  Dark spirits wander the silent halls of the Sapphirean Throne.\\
  And in dreams I see the ocean rise to devour the gleaming spires\\
  as the shades of immortals guide me to the Valley of Silent Paths\prikker
  
  The Topaz Throne of Kings is crack'ed,\\
  aeon-veiled, enrob'd in black.
  
  Black wings above the land of dreams\prikker
  
  The ivory worm now sleeps entombed.
}

Teshrial is getting obsessive. 
The dark, occult, scary Aenigmata of the poem are getting on his mind. 
He can't think of anything else. 
It is affecting his sanity. 

\WanderersInDarknessEmph is like \emph{The King in Yellow}. 
It is hideous, but also fascinating. 
\Teshrial fears it, but he cannot look away. 

He hears voices. 
He sees visions of \Ishnaruchaefir and the Mirage Asylum and \Zaz and \Urzaz and the \chimaera and the \malgryph.

And he sees the cruel \draconian and \ophidian empires that existed before the \resphain came and vanquished them. 
The \resphain have {their own myths of vanquished monsters}.
Read about them. 

\citeauthorbook[p.252]{HPLovecraft:TheBlackTomeofAlsophocus}{H. P. Lovecraft}{%
  The Black Tome of Alsophocus%
}{%
  I remember the first time I heard the voices; weird unhuman sibilant voices, issuing forth from the outer reaches of blackest space, where amorphous beings cavort and caper before a great black fetor-belching idol worn by the passing of uncountable centuries.
  With the commencement of these voices came visions of horrifying intensity, dread chimeras of duel black and green suns, shining on towering monoliths and citadels of evil, which rose, tier upon tier, as if seeking to escape their earthly attachments.
  But these dreams and illusions were as nothing compared to the dread colossus that was later to encroach upon my consciouness; even now I cannot recall the horror in its entirety, but when I thinkg on it I have an impression of vastness, of size beyond measure, and groping tentacles, pulsating, as if with an intelligence of their own, alive with malignant depravity.
  Around this base enormity pranced cadaverous monstrosities, their voices rising in a cacophonous chant\prikker 
  
  \quo{%
    Nyarlathotep [\Ishnaruchaefir] rules in Sharnoth, beyond space and timeM in his gignatic ebony palace he awaits his second coming, served by his minions he broods and festers in blackest night.
    Let none meddle with spells and enchantements concerning him, for he is quick to trap the unwary.
    Let the ignorant beware, heed the \emph{Black Tome}, for terrible indeed is the wrath of Nyarlathotep.}
}









\begin{comment}
  \section{Achsah and Teshrial}
\end{comment}
\new 
\Achsah seeks out \Teshrial's citadel. 
She is his close co-worker, so she is immediately allowed in by the servants. 

She finds \Teshrial in his study. 
Studying. 
Poring over piles of paper and graph-glass. 

It has been several days since she has last seen \Teshrial, and she is dismayed by what she sees. 
\Teshrial is usually impeccably dressed and bathed. 
Today he looks like a wreck of himself. 

True, he is still better dressed and more presentable than many \resphain, but compared to his own standards, he is positively haggard. 
Clearly he has not slept in some time. 
His clothes show signs of use, which is otherwise a taboo for him. 
His hair has tangles in it. 
He is muttering to himself, lost in thought. 

\Achsah walks up to him. 
Kneels down.
\ta{My Lord \Ketheran.}

\ta{What?} he snaps.
\ta{Oh. It's you.}

\Teshrial composes himself, or tries to. 
He sits up straight and looks at her. 
Tries to put on his usual smug, unfazed \facade, but fails. 

He fidgets. 
Usually he is self-assured enough to sit still, but now his fingers are twitching. 
His eyes flicker. 
Several times his mouth starts to move and she can see his has to force it shut. 

\tho{Damn. He is hit hard.}
She knows the story of how \Ishnaruchaefir himself came to assassinate \Teshrial's accomplice. 
It was an unmistakable threat. 
\tho{Kill first, negotiate later. A very \draconic move.}

It must be hard for \Teshrial. 
Not only has he lost a valuable ally (for \Urizeth is unlikely to come back and help him after this attack from the Destroyer) but he must also be shaken by the brutality and efficiency of the attack. 
It is one thing to know that you risk your life on the battlefield. 
It is another thing entirely to know that you might be assassinated in your own home.

\Achsah does not like \Teshrial, but she still feels compassion to see him in this disturbed and disturbing state.

\tho{Where did it happen? 
  Did \Ishnaruchaefir really sneak into \Nyx?
  \Erebos, I hope not.}

\Teshrial (impatient): \ta{What is it?}

\Achsah explains her misgivings as above. 
She suspects \Malcur is a decoy and that \Forklin is the Sentinels' real goal. 
She cites past experience with \Secherdamon. 
He is known to conduct elaborate decoy schemes like this. 

Besides, \Ishnaruchaefir is active in \Malcur. 
It is unthinkable that \Ishnaruchaefir and \Secherdamon are working together. 
They would not set up shop together like that. 
No, \Achsah suspects that \Secherdamon knows \Ishnaruchaefir has some doings in \Malcur, and so uses the whole \Malcur deal as a decoy to keep attention away from his real project. 

So she would like to go to \Forklin. 

\Teshrial:
\ta{%
  Yes. 
  Very well, \Achsah.
  Do as you will.
  I charge you with leadership of this whole venture.}

\Achsah (bows, still on her knees):
\ta{As you wish, my Lord \Ketheran.}

She goes. 
She flies off to \Forklin, where she hopes to discover something. 









