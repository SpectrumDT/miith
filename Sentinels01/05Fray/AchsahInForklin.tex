% 
\bookchapter{\Achsah in \Forklin}
\Achsah is now in \Forklin. 
She alights on the roof of a high tower and gazes out over the city. 
She can clearly feel that something metaphysical is afoot. 
She believes she is right. 
The Sentinels are up to something here. 
Probably \Secherdamon. 
She will find out what. 

Se looks at the \maybehs{Ghost Tower} and comments that it is \maybehr{Ghost Tower history}{definitely \resphan-built}. 
Despite what the common folk may think. 
Her people once had a city here.
It was destroyed in the \resphanwars. 
Or something. 
(Read about the \maybehs{Ghost Tower}. 
 If no information is there, make it up and add it.)

She thinks about the war.
She looks at the Rungeran and Pelidorian armies and \ishroth. 
If the fighting has already begun, she looks at the fighting or its aftermath.

The \ishroth will probably be the deciding factor, she thinks.
The Imetrians might also factor in. 
She has heard from Charcoal of this \quo{\EreshKali} magic which the Rungerans allegedly possess. 
It has yet to be unveiled. 
Charcoal thinks it will play a large role in the coming war. 
He fears it.
And maybe he is right to fear it.

\Achsah is suspicious about the \EreshKali magic. 
\tho{%
  Can it be true that there was this great magic hidden right there in the middle of Runger?
  Hm.
  Yes.
  I suppose it can.}

\Achsah tries to focus on her task at hand, but her mind wanders.
She keeps thinking about \Teshrial and \Ishnaruchaefir. 
She is worried about it. 
Partially because the recent encounter and battle with \Ishnaruchaefir was a scarring experience. 
Especially seeing how easily the Destroyed killed \Teshrial was scary.
And now \Teshrial wants to do it again.
She is not comfortable with \Teshrial's idea of confronting \Ishnaruchaefir in battle. 
She thinks \Teshrial will get himself killed. 
\Teshrial is a vain and condescending snob, but he is not a bad \resphan.
He is no worse than most of \CiriathSepher. 
She cannot expect better treatment. 
She is \ashenblooded, after all. 
Her mother was a hairy, brutish \nephil. 
(\Achsah has lived with discrimination for millennia, so she has internalized the idea that she is inferior to purebloods.)

\Teshrial does not deserve to be destroyed by \Ishnaruchaefir. 

She thinks about \Teshrial.
She has heard about the great risk he taking. 
\Achsah{} understands \ps{\Teshrial} motivation. 
He wants children. 
\Achsah{} can never have children. 
She wishes she could at least have the vain hope of once being a mother. 
So she tries to wish the best of success for him, at least. 

She remembers \Ishnaruchaefir when they met him in the dead garden in \Malcur. 
When \Ishnaruchaefir{} approached the city, \Achsah{} \maybehr{Detecting Vertices}{could feel him from miles away}. 
He was a behemoth \vertex. 
It felt like a humongous 200 ton sauropod stomping through the city. 
He had not only his own \vertex{} power, but also that of the glaive sending out tremours through the Shroud. 

\Achsah fears \Ishnaruchaefir.
She lived during the Incursion, after all. 
Or the \secondbanewar, as the Sentinels call it. 
She remembers his atrocities. 

\Achsah once \maybehr{Achsah met Ishnaruchaefir}{encountered \Ishnaruchaefir on the battlefield}. 
She saw his visage, ablaze with a hatred toward her kind that burned even fiercer than the \xs-spawned sorcery that enveloped him. 
She felt his presence, radiating a promise of vengeance and horrible death. 
And she had not stood her ground. 
She had not even let him come near her. 
She had fled from his path in panic. 
The rest of the battlefield had seemed like a sanctuary, then. 

\begin{prose}
  \tho{I had no courage to face him. 
    But \Teshrial\prikker he stood his ground. 
    He met the Destroyer in single combat, and he fought to the death.
    \Teshrial{} is brave, I must give him that. 
    Much braver than I.} 
\end{prose}



She loathes, fears and despises \Ishnaruchaefir. 
But at the same time, it is hard not to be captivated by the force of his personality. 
Such will. 
Such power. 
Such glory. 
Such ferocity. 

There is raging bloodlust and savagery in him, she knows. 
But at other times, such as this, there is mournful feeling. 

He is an outcast, despised and condemned by his own people, feared and reviled by all. 
Yet he stands proud like a king. 
She can feel great pride and confidence radiating from him.
\Ishnaruchaefir is every bit the \dragonking{} that his brother \Nexagglachel{} was. 
She has seen the legendary \Nexagglachel once, when he was a prisoner of the \resphain, and even then his pride and dignity towered above all. 
\Nexagglachel was now dead forever, and \Ishnaruchaefir had taken his place as the \resphain's greatest enemy. 

She feels some kind of kinship with \Ishnaruchaefir in that moment. 
She is also an Exile of sorts herself, being \ashenblooded. 

(%
  Not too much kinship, mind you. 
  I do not want to dilute the Cosmic Horror.
  Perhaps she sees him as an ideal or possible mentor-figure instead of a kindred spirit.
  An example of how awesome you can be as an Exile.
  Over the years, \Achsah has secretly admired \Ishnaruchaefir and taken inspiration from him.
  She admires how he bears his exile with pride and disdain and still holds firm and does what he believes must be done.%
)

\begin{prose}
  \tho{If only I had his strength.
    His disdain.
    His pride.} 
\end{prose}









