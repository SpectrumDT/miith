
\bookchapter{The Ghost Tower}





\begin{comment}
  \section{The Fall of Dendrum}
\end{comment}

\stamp
  {\dateBilaComesToForklin}
  {A council room, \Forclin}

Dendrum had fallen. 

To Sethgal this news, recently delivered by a courier, had been expected, but still disconcerting. 
The Rungerans had besieged the city with their vast army and powerful siege cannons and given the defenders a frightening ultimatum: 
\hypota{Surrender or we will kill and rape and enslave you all.} 
They then began bombarding the walls with their cannons to show that their threat was not empty.
After few days, Dendrum caved in. 

Now another messenger had arrived, a civilian survivor of a more recent Rungeran attack. 
Sethgal had told the guards to show her in immediately. 
And here she came. 

She was a young \scathaese girl, less than fifteen years old, wearing the garb a common villager.
Evidently too weak to walk alone, she was being supported by a \scathaese soldier. 
She was dragging a leg, her breathing was \labour{}ed, and her cobalt-scaled body was covered in dark sores and blisters that did not look like normal cuts or bruises. 
She lifted her eyes to take in the large and imposing room, its dark stone walls and pillars. 
Then she noticed the two \human mages who were present, Archibald Curwen and \CarzainShireyo. 
Immediately she flinched and tried to draw back, almost collapsing onto the floor. 
But Sethgal beckoned her closer with a commanding gesture, and, with her guide's help, she obeyed. 

\talk{You need not fear us, child.
  You are safe here.
  I am Sethgal Pelidor. 
  Who are you?}

\talk{My name is Bila, my lord,} said the girl.
\talk{From\prikker from Lunum.}

Lunum, he knew, was a village not so far east of \Forclin. 
\talk{What happened to you, Bila, and to Lunum?}

Young Bila then told the story of how an army had fallen on her village. 
The soldiers had killed all \scathae they found, as well as all \human men. 
The \human women they had raped and carried away\dash for as Bila told the story, the Rungeran army was mostly \human. 
Worse, they killed not only with swords and guns, but with sorcery\dash what Bila described as \quo{a cloud of pure, screaming evil} that blackened and withered the flesh \quo{like fire and ice and leprosy all at once}.
Bila had hidden in a cellar and listened to the howling of \daemons, the horrible cries of the slain and violated and the wild, vicious shouts of the conquerors. 
Finally, when the soldiers set fire to the house and smoke and flames began to fill the cellar, Bila had to flee. 
She snuck past the Rungerans, stole a \relc and rode west along the railroad. 
She rode almost without pause for what must have been many hours, until at last she was found and rescued by a patrol of Pelidorian soldiers who were kind enough to bring her back to \Forclin. 

\talk{You poor, brave girl,} said Sethgal.
\talk{You need healing right away.}
He looked at the Vaimons. 

\talk{I'll do it,} said \Shachar and approached the girl.

Bila yelped and shied away.
She looked up at \Shachar with big eyes full of horror. 
\Shachar backed off, his otherwise haughty face now sad with compassion.
\talk{On second thought,} he said, 
\talk{we should get her a healer of her own race.}

\talk{You are right,} said Sethgal. 
He understood. 
Bila had developed a fear of \humans now, and especially \human sorcerers. 
\thought{And with good reason.}
\talk{You!} he said to a servant.
\talk{Find a suitable healer. 
  You!
  Take Bila to a bed.
  You have done well, Bila, by bringing this news to us. 
  You will be rewarded. 
  We cannot bring your family back, but we can avenge them. 
  I hope you can take some consolation in that.}

Bila did not look consoled as the servants led her away, only sick with pain and sorrow. 
\thought{I should not have said that,} thought Sethgal.
\thought{I should have said something reassuring:
  \quo{Everything will be all right.}
  Something like that.}

\talk{She looked awful,} said \Shachar when Bila was gone.
\talk{A healer may be able to ease her pain, but I doubt she will live long.}
\Shachar's face had now reverted to a professional mask, but his voice still betrayed a hint of sadness. 

\talk{What were those awful sores on her?} asked Sethgal. 

\talk{Effects of the Rungeran \ishrah's sorcery.
  I saw something similar on the bodies in Gilwaed.}

\thought{\Sephiroth preserve us from evil.} 

\talk{It would appear your intelligence was correct, sirs,} said Sethgal after a while.
\talk{%
  But something puzzles me:
  Why have we heard nothing about this in the reports from Dendrum?
  In fact, according to the couriers the Rungerans had used no magic at all. 
  They spoke of cannons and threats and great numbers of warriors, but no sorcery. 
  Why is that?}

\talk{Good question, Marshal,} said Curwen. 
\talk{If my information is correct, the Rungeran \ishrah have only recently learned these new spells.
  I would surmise they use them against these villages for practice, in preparation for a more serious battle.}

\talk{But then why not use it against Dendrum?} asked Sethgal. 

\talk{Secrecy,} Curwen suggested.
\talk{They do not want us to know about it.
  Consider, \rah[Sethgal], that were it not for me and \Shachar this would be the first time you heard of it.
  If you had only that girl's word that the Rungerans wielded some never-seen-before, terrible sorcery, would you not dismiss her as a shocked and ignorant commoner?}

\talk{Point taken, Lord Curwen.}

\talk{Moreover,} said \Shachar, 
\talk{I suspect morale reasons.
  They could destroy the villages, but Dendrum they wanted to capture, as neatly as they could.}

\talk{Then would they not want to intimidate the defenders?}

\talk{To intimidate them, yes, but not drive them out of their senses. 
  You have not seen the site of such an attack, \rah[Sethgal].
  I have.
  Now, as you may have guessed I have a rather liberal view of dealing with dark powers. 
  And even I was struck by a feeling of vast cosmic malice.
  The people of Dendrum, being faithful \iquinians, would see the blackest horror and evil.
  There is a risk that it would drive them not to surrender but to a suicidal panic.
  They would fight to the last man, instinctively fearing that anything, \emph{anything}, would be better than allying oneself with such evil.
  And who knows?
  Perhaps they would be right.}





\begin{comment}
  \section{Carzain looks for the Ghost Tower}
\end{comment}
\begin{garbage}
\new
Carzain looks for the Ghost Tower and wonders why he can't find it. 
He begins to suspect there is something mystic about it. 
He asks around in the city about it.
He gives money to some beggars and asks them questions. 
(Make sure Carzain is sympathetic so readers like him.)
But people shy away and refuse to answer, or even make warding gestures. 

\tho{Oooookay. There is definitely something mystic about that tower.}

He then rides outside the city again to look at it.
(Now, being affiliated with the army, he can freely leave and enter the city.)

He notices that he can see the tower when he is some hundred paces down the road, or out near the \wylde border. 
(Make it clear that the \wylde is a scary place. Carzain only dares go near it because he is so badass.)
But when he comes closer to the city proper, out of the \wylde, the tower becomes blurry until it is entirely invisible. 

\tho{This is a very interesting phenomenon.
  I will have to remember to ask Curwen about it.}

Remember to have both Carzain and Vizicar speaking. 
 
\end{garbage}







\begin{comment}
  \section{Carzain asks Curwen about Ghost Tower}
\end{comment}
\placestamp{Outside the council room}

With the day's planning done, Archibald Curwen was about to return to his chambers when Carzain \Shachar called to him.

\talk{Oh, \Mister Curwen!
  I actually came here for a reason.
  Tell me something.}

Curwen halted.
\talk{What?}

\talk{When I first came to \Forclin, I noticed this tower.
  A high, spindly, gray thing, shrouded in clouds. 
  Looked like a mage's tower.
  But when I looked for it in the city, I could not find it, and no one could direct me to it.
  I asked and got some bad directions, which had me wandering around for hours until I realized that no one knows where the damn tower is.
  It should be one of the tallest buildings in the city, but it is like they do not even know it exists.
  I have also tried to look out over the city from high places, but I cannot see the tower anywhere. 
  Needless to say, this made me feel a certain professional curiosity.
  Do you know what is going on?}

\talk{Ah,} said Curwen.
\talk{Yes, I know something about it.
  It is called the Ghost Tower.
  It's uninhabited\dash by humanoids, at least.
  There is sorcery in it, but I do not know all the details.
  The locals feared it, so for generations they pretended the tower was not there.
  You know, of course, how the supernatural and frightening can affect weaks minds, and nowadays the denial is so strong that most literally don't know the tower is there.}

\talk{Think it's dangerous?} said \Shachar.

\talk{Possibly. 
  It's supposed to be haunted by monsters.
  But I'll tell you where to find it\prikker}

\new
After \Shachar left\dash in search of the Tower, presumably\dash Curwen found a high window and looked out. 
He reached out into the Beyond with his senses, penetrated the Shroud, and there it was.
The Ghost Tower was well-hidden in the Shroud, but to the wise mage who knew where to look, it was visible as a blurred outline in the mist. 

He had not told \Shachar everything he knew. 
The Tower was ancient. 
It predated even the oldest Vaimon structures in \Forclin by a thousand years or more, for it was not built by mortal hands. 
It was a \resphan tower, reared in time immemorial when the immortals still dwelt in this Realm, before they retreated to night-shrouded \Nyx.
It still stood there, hoary and indestructible, suffused as it was with the \resphain's elder sorcery. 
% But it was not empty and abandoned, for it was constructed to serve as a gateway between the Realms. 
It was uninhabited by mortal \humanoids, but the mighty \resphain still visited it from time to time, for it served as a gateway to \Nyx, allowing them to come into this Realm and battle their ancestral foes, the so-called Sentinels of \Miith. 
% And it is haunted by monsters, because of its special status as a gateway through the Shroud. 

\thought{The Sentinels. 
  That's right. 
  I must not waste time on idle thoughts. 
  
  The Sentinels. 
  What are they planning?
  They must be involved in this war.
  They always are.
  What is their role?
  Is it true, as Tantor writes, that this \Takestsha is fleeing from her Sentinel masters?
  If so, what are they up to? 
  And what is she up to?
  
  And what if it is not true?
  Could \Takestsha be a Sentinel?
  
  Damn it.
  I don't know enough.
  Not yet.}




\begin{comment}
  \section{Carzain finds Ghost Tower}
\end{comment}
\placestamp{Near the Ghost Tower, \Forclin}
The Ghost Tower was a massive thing. 
It continued upwards for many yards until it was lost in the thickening fog.
This weird fog, Carzain thought, must be a variant of \wildfog, and must be what made the tower invisible from elsewhere in the city. 

There was a feeling of the mystical and arcane about the tower, and of great age. 
But not decrepitude. 
Rather an ancient sorcerous power, woven into the stones, that had never diminished and merely lay dormant. 

The edifice was built of a kind of dark gray rock. 
Up close Carzain could see that the rock was very smooth and quite glossy, which must have been what made the tower look bright in \colour from a distance. 
The rock could not be native to this reigon, for it looked very different from the other kinds of stone used in \Forclin. 
It, and other things, made the tower look like an intrusive foreign element, a bloody spear violently thrust into the ground. 

The architecture was unlike any they had seen in \Forclin.
Indeed, rarely had Carzain or Vizicar seen anything like it. 
The wallsides were uneven, curved, wavy. 
At a fleeting glance one might mistake it for shoddy masonry, but upon closer inspection it gave an impression of deliberate design, of superhuman skill and accuracy. 

\vizicar{Clearly not shaped for an earthly purpose, but for an occult one,} said the voice of \VizicarDurasRespina.

The smooth stones were carved all over with subtle lines and dots that formed weird patterns, which, when seen on a larger scale, combined to form larger patterns, and even larger ones.
The lines were fine and hard to make out at a distance, but Carzain guessed this system of patterns within patterns continued all the way up to the very largest scale. 
They reminded Carzain of things he had seen before in spellbooks, but being no scholar of occult geometry he could not interpret them, only form vague guesses. 
The lines of power resembled constellations\dash stars, planets and nebulae, suggesting immense desolate spaces of black emptiness\dash and when he followed them with his eyes they seemed to lead into darkened regions of the Beyond where he dared not look. 

The sight of these lines, these hideously suggestive patterns, summoned into his mind the shades of unwelcome memories; from this life, from Vizicar's, and from his even more ancient life as the sorcerer \TydesmosGendarInCaphet, centuries before Vizicar. 
From passages in forbidden spellbooks he dimly recalled evil hints of vastly powerful elder forces. 
Of elder races that came to \Miith out of benighted Realms of horror in the far past, bent on enslaving the world. 
Some said the elder races had failed and been exiled from the world, but other, eerier sources claimed that they had \emph{succeeded} and even now ruled the world from the shadows. 
Carzain did not know if any part of those morbid myths was true, but he knew there were things in the universe immensely greater and darker than any \human or \scathaese agency, and experience led him to suspect the worst. 

As he idly ran his eyes across the occult lines while his mind wandered, it was as if his eyes glimpsed undefinable but gruesome images, and he felt the tinge of a subtle, instinctive dread creep over him. 
He dimly realized that what he glimpsed here was a small corner of a vastly larger and more horrifying pattern, a whole he did not want to know. 

And Carzain \Shachar, otherwise known as a brave man, who had so often before stood undaunted in the face of the horrors of the occult, succumbed to the grip of a vague fear, discarded his plan to find a way into the Ghost Tower and instead turned and walked away. 









