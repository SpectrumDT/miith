
\bookchapter{Ilcas and \Tiroco}
\begin{comment}
  \section{The Imetrians make an offer}
\end{comment}

There are Imetrians in \Malcur. 
They have brought \nycans. 
\Tiroco remembers how she saw the \nycans out of a window. 
She and compares them to \grulcans.
A \nycan is slightly lighter than a \grulcan, but probably as strong, and faster.
They mostly do not carry riders.
She does not understand how the handlers control them.
Telepathy, as far as she understands.
She has also heard that \nycans are as intelligent as humanoids, but she does not know if that is true.
She also compares them with the Allosaurus-like dinosaurs that the Rissitics, among others, are known to use in combat.
Then she starts worrying about the Rissitic invasion.
Is it wise to send almost all military in Pelidor east to fight the Rungerans when the Rissitics might come knocking on their southern border any day?
The military leaders seemed to think it was a sound plan, so they did it, but now \Tiroco becomes afraid. 
She fears the Rissitics.
They are evil and use black magic and monsters.
She hopes they are stopped by the Vaimons so they never reach Pelidor.


\Tiroco:
\ta{Very well. Show the Imetrian delegation in.}

Mention that Pelidor has good diplomatic relations with the Imetrium and has allowed the Imetrians to build some churches in Pelidor. 
Not everyone likes the Imetrians. 
Many accuse them of being \quo{slimy and cold, like fish}. 
And for being heathens and idolaters and practitioners of black sorcery. 

That may be true, but they can help. 
And the situation is bad. 
Runger is a larger country than Pelidor and stronger in military matters. 
\Tiroco needs all the help she can get. 

The gates open to admit two Imetrians: 
\Ispan Ulphon Nestor and \Retaxis Telcastora Ilcas. 
They are unarmed. 
Remember to describe their clothes and appearance. 

The approach the throne. 

They both bow.
\ta{\Rinyuth[\Tiroco] Pelidor.}

She greets them both with name and title. 

Ulphon: 
\ta{Pelidor has a history of prosperous friendship with the Imetrium. Bla bla bla.}

Telcastora: 
\ta{We come to you with an offer.}
Ulphon looks at him annoyedly for coming straight to the point. 

\Tiroco:
\ta{What kind of offer?}

Telcastora: 
\ta{To aid you in this war.}

Ulphon:
\ta{Yes, \rinyuth[\Tiroco]. 
  We offer military assistance, in return for suitable\prikker diplomatic assurances.}
Now it is Telcastora's turn to look annoyed. 

\Tiroco likes Telcastora. 
He looks like he really means it. 
He wants to help. 
The \Ispan, OTOH, seems more interested in what he stands to gain from her. 

\Tiroco:
\ta{Very well, sirs. 
  You have my attention.}





\begin{comment}
  \section{Negotiations}
\end{comment}

And so the negotiations begin.
She learns that they can contribute with up to 500 men. 
This surprises her. 
She knew the Imetrians had some soldiers in Pelidor. 
They had to protect their churches and pilgrims. 
But 500 men sounds like a lot. 
And that is just how many men they can \emph{spare}. 
In total they must have even more. 
That is a lot. 
They could conquer parts of the country with such a force. 
The entire Pelidorian army is not much more than 10,000 men. 

In addition to the \scatha and \humans, they have many \relcs, a few \murocs and several \nycans. 
\Nycans are terrible beasts. 
She saw some of them as the Imetrian group entered the city. 
They are feather-clad theropod saurians. 
As large as \relcs, and with fierce claws and teeth. 
The claws on their hind-feet are as large as swords. 
She has heard a \nycan can run faster than a \nycan and disembowel an \armoured man with a casual kick. 
She has also heard some say that they are not actually beasts, but are secretly as intelligent as humanoids. 
She does not know if she believes that, but even if it is not true, the \nycans are no doubt still a fearsome force. 

In exchange for their aid, they want more diplomatic benefits. 
They want to be allowed to build more churches, and to proselytize, and they want freedom from Iquinian prosecution. 

\Tiroco is hesitant. 
She worries about letting heathenism run rampant in Pelidor. 
The Vaimons will not like it. 
They will argue that letting heathens and sorcerers into the land is not worth it. 
But then, if pressed, the Vaimons would probably say it would be better for Pelidor to be conquered by Iquinian Runger than to be infested with Imetrians. 
\Tiroco cannot use that attitude. 
She must think first and foremost about preserving the nation of Pelidor. 

She thinks the Imetrian offer is good. 

\Tiroco:
\ta{Pray tell, \Retaxis Telcastora.
  What do you intend to do?}

Telcastora:
\ta{If we commit our men, then I will lead them myself.}

\Tiroco likes the sound of that. 
Telcastora looks dependable, but fearsome. 
And from what she has heard, Telcastora is a \nycaneer. 
It is he who commands the dreaded \nycans. 
She would not like to stand against them. 

Still. 
With the campaign of faith \Tiroco has announced, she worries it will make her look bad and dishonest if she is too eager to grant concessions to the Imetrians. 

\Vincerre:
\ta{Do not listen to them, \rinyuth[\Tiroco]. 
  We are better off without their mercenaries.}
\Tiroco notices Telcastora frowns at being labelled a mercenary.
\Vincerre continues: 
\ta{The \sephiroth will not look kindly upon us if we make deals with them. 
  It is written: 
  \quo{Ally not thyself with the heathen.
    For a friend of heathens is like unto a heathen himself.
    He that bargaineth with the heathen shall be a heathen in the eyes of \iquin, and the favour of \iquin shall flee from him, and shall abandon him, and \itzach shall be upon him.}
  So the matter is clear.
  We will be cursed if we buy the aid of foreign mercenaries.
  No, we must keep ourselves pure and trust in \iquin to deliver us.}

Moro \Cornel:
\ta{\Rinyuth[\Tiroco], I would hesitate to seek an answer in theology.
  I believe this matter is too important for that.}

\Vincerre:
\ta{What? That is nonsense. 
  Is it not written: 
  \quo{Know that no battlefield is forgotten by the \sephiroth.
    No battle is ever won save by the grace of \iquin, and no battle is ever lost save for the corruption of \itzach.
    All that cometh to pass upon the field of war passeth as the \sephiroth have decreed.}}

And so they went on and on, until at last \Tiroco was too tired to hear more of it. 

\Tiroco:
\ta{Enough. 
  \Ispan, \Retaxis.
  Pelidor accepts your offer.}

The Vaimons are not happy. 
\Cornel seems relieved. 
The Imetrians are happy; Telcastora looks happy to be able to get out and do something productive, whereas Ulphon looks more proud of himself. 

They negotiate a bit more to find the exact terms of the agreement. 
\Tiroco leaves this to her advisors.
\Tiroco rises and leaves, telling them to come back and ask for her signature when they are done negotiating. 





\begin{comment}
  \section{Razor sees strange stuff}
\end{comment}
\new
Razor is in \Malcur. 
He and his fellow \nycans have been shoved into a stables of sorts and Ilcas has made them promise to stay there until he gets back. 
Razor would like to explore the city, but he knows the people in the city would not look kindly on a \nycan walking around alone. 
People did not trust \nycans. 
Outside the Imetrium, everyone assumes a \nycan is some frothing \wylde monster that will kill and eat anyone it meets. 

So he is confined to a stables, as if he some mindless beast. 

He is not happy. 
He has been confined before, but this is different. 
Razor does not like this city. 
There is something wrong with it. 
He can feel it. 
And smell it. 
There is something evil here. 
Something wrong. 

The other \nycans cannot feel it; he has asked them. 
But he has always had sharper senses than any of them. 

This city is evil. 
It is faint, but it is all over the place. 
It feels like he is in the belly of some enormous beast. 
Like everything is slowly being digested. 
Like its disgusting belly juices are slowly seeping into everything. 

Razor snoops around in the building. 
He might not be able to learn anything, but it's worth a try.

He sniffs with his nose and with his telepathic senses. 
Tries to find traces of something suspicious.
Some place where the taint is stronger than elsewhere. 

He snoops.

He catches a faint scent. 
Idly he follows it into a corner. 
He scrapes at a pile of junk with his foot. 

Then he strikes something icky. 
Wet and soft, in a disturbing way. 
He scapes some more junk away and sticks his snout down there to get a closer look. 

There is a section where the outer layer of the stone wall is crumbled away to reveal a layer of\prikker something.
What is that? 
Flesh?
Ohmygod, that's flesh!
Is it a corpse buried down there?
No. 
It pulses. 
It bleeds where his claw scraped it. 
It's alive. 

An amorphous, slimy mass of flesh. 
It seems to somehow meld into the stone. 
It does not smell like the flesh any creature Razor knows. 
Maybe a little bit like\prikker \scatha? 
That can't be right. 
He pokes it some more. 

Then he reaches out with his mind and tries to touch it telepathically. 
He pokes with his mind\prikker 

\begin{dream}
  He receives an image of some immeasurably vast creature. 
  Shapeless, with a huge number of limbs or tentacles. 
  An ancient abomination, older than everything he knows. 
  Older than anything he can imagine. 
  Blind and mindless, but hungry. 
  Its central body, lurking far beneath the earth, is withered and dead.
  But the tentacles are alive and writhing, pumping hideous blood up from the deep. 
  
  Those tentacles. 
  They are everywhere around him. 
  They are everywhere in the city. 
  Every stone bleeds its foul-smelling slime. 
  Every wall hides ghastly, pulsing flesh. 
  Pulsing with unnatural life-in-death. 
  
  Everything bleeds. 
  Everything bleeds. 
\end{dream}

The scene ends with Razor being horrified, as if waking from a nightmare. 
He tears himself away from the gruesome vision.







\begin{garbage}
\begin{comment}
  \section{Ilcas talks to \Tiroco}
\end{comment}
\new
\Tiroco leaves the chamber with her servants. 
Halfway down the corridor she hears a voice call to her from behind. 

\ta{\Rinyuth[\Tiroco]!}

She turns around. 
\ta{\Retaxis Telcastora.}

He talks a little fluff. 
He tells her he wants to talk to an \Ishrah{} mage and asks her if there is one she trusts. 

\Tiroco{} hesitates due to her own falling-out with Moro. 
But she collects himself and reminds herself that Moro is good. 
So she directs Ilcas to Moro. 
(She noticed that Moro had also left the council chamber.)

\ta{Thank you, \rinyuth[\Tiroco].
  I will seek her out.}
\Tiroco turns away.

Telcastora:
\ta{One more thing: 
  I promise you, \rinyuth[\Tiroco], we will do our best on the field.
  My men, my \nycans and I.
  We will \honour our alliance.}

\ta{Thank you, \Retaxis Telcastora.}
She looks around to check that no one is listening.
\ta{Whatever the Vaimons may say, I am relieved to have warriors like you on our side.}

Telcastora nods a farewell.
Then turns away. 





\begin{comment}
  \section{Ilcas talks to Moro}
\end{comment}
\new
Moro is also wandering back to her chambers.
Ilcas finds her. 

\ta{Archmage Moro \Cornel.}

He tells her about the visions Razor has been having. 
He warns her about the evil in the city. 
He and Razor give all the details they can. 
Moro thanks him. 
But she is suspicious: Why do this? 
Ilcas is not required to by their agreement. 

But Ilcas is a hero. 
Since Razor is sure this alien force is \quo{evil}, Ilcas-tachi feel it is their duty as good Imetrians to inform the people who are allies of their people. 

Moro learns some useful stuff. 
\end{garbage}





\begin{comment}
  \section{Ilcas talks to Nestor}
\end{comment}
\placestamp{On the road east of \Malcur}

When it is settled, the Imetrians ride from \Malcur towards \Forklin. 
They ride \relcs. 
(You can't ride \nycans{} over long distances (they \maybehr{Nycan endurance}{lack the endurance}), and \mulgrons{} are too slow.) 

Telcastora says to Ulphon:
\ta{It looks to me like the Pelidorians got a favourable deal.
  I thought you would have driven a harder bargain, \Ispan.}

Ulphon: 
\ta{This deal was important for us, too.
  You know what the stakes are, \Retaxis.
  We want Pelidor to survive, for more than one reason.
  For one thing, we have never had good relations with Runger, so if Pelidor falls to Runger, we will lose influence in the region.}

Telcastora:
\ta{I still believe we should have told the \rinyuth about our more important reason.}

Ulphon:
\ta{Our suspicion that Runger is in league with Durcac?
  No, we should not.
  First of all, we have nothing in the way of proof.
  It would make us sound like doomsayers and rumour-mongers.
  Second, our intelligence is our own. 
  Dessali teaches that knowledge is precious.
  We cannot just give it out for free.}

Telcastora:
\ta{I still would have told them if it had been my responsibility.
  Everyone fears the Rissitics.
  We should be standing united against them.}

Ulphon:
\ta{We will, when it becomes necessary.}





\begin{comment}
  \section{Razor talks to Ilcas}
\end{comment}
\new
Razor is still disturbed by the things he has seen. 

After tearing himself away from the vision, he called some of the other \nycans over to see what he had found.
When they got there, the patch of flesh was gone and only stone could be seen.

Some of them believed him when he told them, others did not. 
Countess, the pack leader, tried to comfort him and get him to relax. 
Assured him that it was nothing, that everything would turn out all right. 
But then, she would. 
She always does that. 
Countess means well\dash she always does\dash but she does not know what she is talking about. 
Razor he has always had sharper senses than any of them. 
His visions are accurate, and he knows what he saw. 

He has turned this thing over in his head for quite a while now. 
Now he has to share it with Ilcas. 
He is a wise \scatha. 
Ilcas will believe him. 
Ilcas will understand. 









