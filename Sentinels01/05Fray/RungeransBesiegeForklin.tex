
\bookchapter{A Machine of Flesh and Iron}
\begin{comment}
  \section{The Rungerans come into view}
\end{comment}

\stamp{\dateRungeransBesiegeForclin}{\Forclin city wall}
The Rungerans were here. 

They had come crawling from the railroad out of the \wylde like a hive of ants; thousands upon thousands of men, and great rail cars pulled by giant \saurians. 
It seemed that for hours they just kept pouring out onto the fields.
Now, outside the range of the cannons on the walls, the Rungeran army was assembling and setting up their camps.
By evening they covered the fields; an ocean of fires, torches, glinting metal and billowing banners bearing the two wolverines that were the standard of House Runger.
Under the blackening sky, the world had been painted in the tan and brown \colours of Runger. 

Sethgal stood on the walls looking out over the Rungeran camp. 
They had a great infantry, consisting of \scathae and \humans alike. 
Their heavy infantry, as could be expected, was mostly \scathaese, while their cavalry was mostly \human, made up largely of the \human-dominated Rungeran nobility. 
Conversely, the Pelidorian nobility and cavalry was \scatha-dominated, whereas most of their \humans were found in the light and medium infantry and among the gunners and archers. 

The main body of the cavalry were \relcers, bolstered with a strong core of the mighty \murocs. 
In addition, the Rungerans also had a number of \mezolisks; lizard-like, carnivorous saurians, as long as \relcs but far heavier. 
Where \relcs were swift runners on two legs, the \mezolisks were slow but ferocious quadrupeds. 
Sethgal wondered how his \grulcans would fare against these monsters. 

His own army was smaller, not much more than half the Rungerans' number. 
They would have to make the most of their fortified position. 
Sethgal was nervous about that. 
Some centuries ago, fortifications like this had been nigh-impregnable, with sieges lasting months or even years. 
But today, with the advent of more powerful cannons that could punch through stone walls, castles and cities had become far more vulnerable. 
\Forclin's walls were strong, forged with \Ortaican sorcery, and in all its history \Forclin had never fallen to anything less than a years-long siege. 
But the city had not seen a major siege in two hundred years' time, so Sethgal was not sure if its ensorcelled walls would still hold up to today's weapons. 

\tho{I need more men. I need reinforcements.}

Sethgal was still hoping for a positive answer from the theocrats of the Imetrium.
In addition, he had sent out officers to draft some irregulars from the tribes in this area. 
The tribes were mostly barbarians, only loosely tied to Pelidor, but they had once sworn oaths of fealty to the \rayuth, and in a dire situation like this they owed it to their kingdom to do their part. 
At least that was how Sethgal saw it.
He only hoped his envoys would be able to convince the savages to fight.





\begin{comment}
  \section{Sethgal and \Esmerel}
\end{comment}
\new
Sethgal left the wall in order to walk and clear his mind. 
Looking at the enemy's superior numbers all day was exhausting and depressing him. 

Inside the walls he was met by a \human woman in a fiery yellow gown\dash unmistakably a Vaimon \cleric of \ClanRedcor. 
She was accompanied by two men in breastplates and yellow capes\dash evidently \gandierres, the \templar warrior-mages of the Redcor. 
She approached him. 

\talk{You are \rah[Sethgal] Pelidor, are you not?} said the woman in a strong alto voice. 
The voice of one accustomed to being shown great respect. 

\talk{I am he.}

The Redcor stared down at him from a great height, for she was taller than most women, and even a large \scatha like Sethgal was shorter than the average \human in his natural, almost-horizontal posture. 

\talk{I am \Matron \ChyrieEsmerel, \cleric of \ClanRedcor.
  I have come to oversee this war, and to provide whatever assistance will be appropriate.}

\ta{Assistance?} said Sethgal, surprised.
\ta{I did not think \ClanRedcor supported this war?
  Especially not with the Rissitic invasion.}
He knew that the Redcor were bent on opposing the invading Rissitics in the south and had sent many Vaimons and warriors to assist the beleaguered nations. 
He had not thought they would have the resources to support Pelidor against Runger, nor the interest to do so. 

\ta{That is true,} said \Esmerel.
\talk{% 
  We are not here to fight your war for you, \rah[Sethgal].
  We will give assistance where it is fitting, but we will not fight on the battlefield.}

\ta{Then why are you here, \matron?}

\ta{%
  We are merely here to oversee the warring parties. 
  To ensure that, even in war, the One Light is held sacred.
  Beyond that, Redcor business is for Redcor to know.}



\new
When \Esmerel had left, Sethgal still brooded. 
\Esmerel was a Vaimon of \iquin, initiated into the higher mysteries of the One Light, and as such deserved a certain respect. 
But Sethgal did not much like her. 

\tho{Vaimons. 
  You impose on us and meddle in our affairs, but you will not help us when we need it.
  Even against a foe obviously allied with the darkest evil.}

Then again, that might be why she was here. 
\hypota{To oversee the warring parties,} she had said.
\hypota{Ensure that the One Light is held sacred.} 
If Lord Curwen's story could be believed, Morgan Runger and his \ishrah were engaged in dire blasphemy. 
It was likely that the Redcor's intelligence had likewise caught wind of this.
Perhaps this was what \Esmerel was here to monitor. 

It made sense.
If King Morgan had any Redcor emissaries in his own court, he must surely have sent them away by now, for they would never tolerate his conduct. 
So the Redcor had to work through Pelidor. 

\thought{So, now we are Redcor tools.} 

Sethgal was as faithful an Iquinian as anyone, but he did not much like the Redcor's variant of the religion. 
They seem rabid to him, always preaching about asceticism and the impending Eschaton, the end of the world. 

Then he recalled Carzain \Shachar's words:
\hypota{Anything might be better than this evil with which the Rungerans have allied themselves.}
For a moment, he felt a pang of fear in his soul and mouthed a prayer to the One Light. 
But no. 
In his rational mind he did not believe that. 

\thought{This may be a terrible trial, and it may see us all dead, but it is not the Eschaton.} 





\begin{comment}
  \section{Carzain and Vizicar}
\end{comment}
\stamp{\dateRungeransBringUpCannons}{\Forclin city wall}
It was the next morning, and Carzain \Shachar stood on the city wall, alongside Archibald Curwen, \rah[Sethgal] Pelidor and other officers. 
Now, out there on the plain, the Rungeran siege cannons could be clearly seen. 
In the gloom of the last evening they had been hidden, but now, in the sunlight of morning, they shone in all their terrible glory. 
% They stood in clusters on the highest pieces of ground the Rungerans had been able to find. 
% Fortunately this was not very high\dash generations of \Forcliners had flattened all hills within hundreds of yards of the city walls. 
And there were a lot of them. 

\talk{Those cannons are my greatest fear,} said Sethgal.
\talk{After the stories we have heard from the bombardment of Dendrum, I am not optimistic.}

Curwen grunted. 
\talk{Hm. 
  The cannons will be a problem, true.}
He paused to cough, then sucked on his pipe again.
\talk{But if you ask me, Marshal, I am more afraid of their \ishrah.}

Sethgal scanned the Rungeran camp. 
\talk{Can you point them out?} 

Curwen pointed. 
\talk{I think that's them, near that orange tent and the big cluster of cannons.
  \MrShireyo, you have younger eyes than either of us. 
  Can you see them?}

Carzain looked and described to them what he saw. 
Of especial interest was a woman in a brown dress. 
\talk{I think she must be the one I saw when I spied on them,} he said. 

\talk{Yes,} said Curwen, \talk{that confirms my own sources.
  I think she's their leader, and the one who introduced these new spells.}

\talk{At least they cannot reach us yet,} said Sethgal.
\talk{They will have to come closer before they have a chance to hit anything. 
  And then we will be able to retaliate with our own artillery.
  They will find that \Forclin is not so easy to bombard.}

Curwen blew out a stream of smoke from his pipe. 
\talk{I don't like the fact that the mages have set up camp next to the cannons,} he mumbled.
\talk{That doesn't bode well.}

Carzain took in the sight of the Rungeran army camp again. 
He could not but admire the sight. 
The men, the blades, the monsters.
A machine of war.
A machine made of flesh and iron.

But in his heard he heard the scoffing voice of Vizicar:
\vizicar{I suppose it is impressive by the standards of this primitive day and age.
  But in my time, in the \caliphate, this was nothing.
  I have seen armies twenty times the size of this.
  I have commanded such armies. 
  I have vanquished such armies\prikker}

He allowed himself to dream away, carried on a stream of images from Vizicar's life.
Of vast silver-shining legions; 
 of tens of thousands of men arrayed in arrow-straight lines; 
 of hundreds of Vaimons unleashing the wrath of \iquin and \itzach in devastating concert. 
So vivid was the vision that he could hear the thunderous crash of the terrible spell\prikker

No. 

Wait.

That was not a part of the vision. 

That boom had been real. 

Carzain opened his eyes. 
A puff of smoke was rising from the hill where the Rungeran \ishrah stood. 

\tho{Are they firing? 
  Why?
  They are way out of range. 
  Practice?}

Then there came another crash, this one from below. 

A slight tremour shook him. 

A cannonball had struck the wall.

The Rungeran cannonade had begun. 









