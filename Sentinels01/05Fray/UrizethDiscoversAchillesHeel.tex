
\bookchapter{\Urizeth Discovers Achilles Heel}
\Teshrial is back at \Urizeth's place.
\Urizeth has healed a little more since last time, but she is still not looking good.

She is telling him about her new discoveries.  
She has been studying \WanderersInDarknessEmph as well as historical records, and she believes she has found a weakness that can be exploited to defeat \Ishnaruchaefir. 
He has an Achilles heel. 

\Urizeth has found something.
She shows him where in the poem she found her clues.
She points out some lines and explains that they are pivotally important. 
They break the usual heptametre of the poem, which she suspects means there must be something special about them.
At first she thought the deviation from rhythm was a translation error, but she has looked into an original version written in \TrueDraconic, and as far as she can make out (she does not really understand \TrueDraconic) this strange rhythm is also present in the original.
So, you see\prikker

\Teshrial tries to understand it, but it goes over his head.
Instead he gets sucked into the text and zones out. 
He begins to imagine images of \dragons and hear the beat of black leathern wings from moonless gulfs.

\Urizeth:
\ta{\Teshrial? Are you listening?}

\Teshrial:
\ta{Um\prikker yes.}




\begin{comment}
  \section{\Zaz and \Urzaz}
\end{comment}
\WanderersInDarknessEmph spoke of a mysterious pair of entities named \maybehr{Zaz}{\Zaz and \Urzaz}. 
It was unclear whether these were \dragons, \xss, cosmic gods or even purely metaphorical entities, personifications of something abstract.
Compare to Gog and Magog from the \emph{Bible}.

\Urizeth discovers that \Ishnaruchaefir apparently fears them and takes damage from them.  
Some \WanderersInDarknessEmph passages describe how the \Zaz and \Urzaz are anathema to him.

The Exile feared the \quo{body of \Zaz} and \quo{that which issueth forth from \Urzaz}.
Some clues said that he feared the \malgryph (the \quo{body of \Zaz}), that he quailed before it, that it held the power to cast him down and destroy him.

\Urizeth had long had trouble interpreting \quo{that which issueth forth from \Urzaz}.
It could be the very \quo{being} or \quo{aura} of \Urzaz, or it could be his breath or something else that emanates from him.
But now that \Teshrial specifically asks her to look into the problem, she remembers some old research she has done.
It did not lead to much back then, but now she digs it up and looks at it with renewed motivation.

\target{Urizeth researches Chimaera}
She had been trying to nature of the mysterious \Zaz and \Urzaz, and had an idea they were connected with the \quo{\Chimaera}.
The \Chimaera is a creature, probably a metaphoric one. 
It is related to and perhaps identical to \Zaz and \Urzaz.
Perhaps it is the union of these two (possibly contrasting entities) that form the \Chimaera (a \chimaera is a crossbreed or mashup or combination). 

\target{Urizeth thinks Zaz and Urzaz are the Chimaera}
\Urizeth suspected a link from \Zaz and \Urzaz to the \Chimaera. 
But \WanderersInDarknessEmph is a huge poem, and she had not been looking in the right places.
Now her attention is turned towards the parts that deal with the Exile, and here there are some very clear indications that (once you thoroughly interpret them) strongly suggest that \Zaz and \Urzaz more or less \emph{are} the \Chimaera.

\target{Urizeth researches Malgryph constellation}
In connection to all this, there was \maybehr{Malgryph constellation}{a constellation called the \Malgryph}.
It had a very obscure meaning in ancient \draconian occultism.
It was never used in the \rethyactic tradition except in the vaguest of references, so \Urizeth-tachi had great difficulty researching it. 
They would have to consult a \dragon or \quiljaar sorcerer to learn what the \Malgryph meant.
(Maybe they tried contacting a \quiljaar, but \Ishnaruchaefir got to him first and coerced him into silence.)
But \Urizeth knows that the stars representing \Zaz and \Urzaz are part of the \Malgryph constellation.
It is possible that the \Malgryph \emph{is} the \quo{\Chimaera}.
A \malgryph is, after all, a mix of different beasts and hence a \chimaera of sorts.





\begin{comment}
  \section{The \chimaera and the \malgryph}
\end{comment}
A clever reading of the poem suggests that the body of \Zaz is the same as the \quo{\Chimaera's ichor}. 

The \Chimaera's ichor is a physical substance.
\Urizeth remembers that there exists some practical research regarding this.
It was suspected that the \Chimaera's ichor might have interesting arcane uses, so alchemists did a lot of research on its nature and composition and how to reproduce it.

\Urizeth searches in the archives and finds some material about it.
It turns out that the alchemists did indeed succeed in brewing some \Chimaera's ichor.
It seemed to satisfy the properties described in the poem, so the alchemists were confident they had the right mixture.
Sadly, they failed to find any use for it, so the research project fizzled and was forgotten.
But now \Urizeth and \Teshrial have rediscovered it, and \Urizeth believes the \Chimaera's ichor is vital to defeating \Ishnaruchaefir.

After some more reading and interpretation, \Urizeth believes they need not merely the ichor.
They need the \malgryph itself.
They have to research and discover the \maybehr{Malgryph summoning}{spell that lets them summon a \malgryph}, and then unleash it upon \Ishnaruchaefir.
\Teshrial is \skeptical, since he believes to know that \malgryphs do not truly exist. 
\Urizeth tells him how that works. 

\Urizeth believes that if they can brew some \Chimaera's ichor, they can use it to summon the \malgryph and have it fight \Ishnaruchaefir.
But it must be done stealthily. 
\Ishnaruchaefir already suspects what they are up to.
He knows they are researching him and reading \WanderersInDarknessEmph, so he may suspect they have uncovered this secret. 
That might in fact be the very reason why he killed \Urizeth.
So they need to summon the \malgryph in some stealthy, sneaky manner. 

\Teshrial thanks and commends \Urizeth for her great work. 
He jokes that she is stealing all the glory from him.
At this rate, \emph{she} will go down in history as the one who defeated \Ishnaruchaefir, and he will just be remembered as her pawn. 

\Urizeth: 
\ta{Well, we have both been killed by him once. 
  On that count we are even.
  But I intend to stay holed up in my citadel from now on\dash at all times if possible\dash so any further heroism will have to come from you.}

\Teshrial:
\ta{Heroism.
  Yes.}
\Teshrial remembers that one day in the near future he will have to face \Ishnaruchaefir in combat. 
He shudders to think of it.
He still does not think of himself as ready. 
He expresses this doubt to \Urizeth.

\Urizeth:
\ta{You are right.
  \Ishnaruchaefir is a \shaeeroth.
  One of the greatest \dragons ever to have lived. 
  He is legendary for a reason.
  The Nadir and the \malgryph may not be enough against him.
  He is resourceful. 
  We can never have too many weapons.}

\Teshrial:
\ta{This business of ours has become quite the talk of the dynasty, you know.}

\Urizeth:
\ta{Really? No, I did not know.}

\Teshrial reminds himself that she is a huge nerd.
\ta{You should get out more.}

\Urizeth:
\ta{I would rather not. Last time I left my citadel I got killed.}

They both LOL. 

\Teshrial:
\ta{My point is that this quest has already made us famous.
  Well, me, at any rate.}
He rarely talks about her because he does not want everyone to know he associates with such a weird nerd. 
Now that he thinks about it and looks at her, he feels kind of shallow for denying her in front of his friends. 
But he represses that thought and continues. 
\ta{I think it has made me famous enough that I can call on some aid.
  I am of \Azraid's bloodline, you know.}

\Urizeth:
\ta{No, I did not know.}

\Teshrial:
\tho{What? 
  She does not even know my breeding?
  I thought everyone knew that!
  But I guess, since everyone knows, I never remind anyone.
  After all, I cannot go and namedrop \quo{my grandfather, the High Lord} all the time.
  That would be extremely bad form.
  But really! 
  How could she not know I am of \Azraid's bloodline?
  Everyone knows that, by the Rose!}

\Urizeth notices his expression of disbelief. 
She shrugs.
\ta{I knew you were a \ketheran, but I have never looked into your background.}

\Teshrial does not know what to say. 
\ta{Um. Right. Well. Anyway.
  I will seek an audience with \Azraid.
  I am sure the High Lord will not turn away a famous, heroic descendant. 
  Especially since I am on a mission that could be one of the greatest turning points in the \resphan race's history.
  No, he will not turn me away.
  I will go to him.
  I am sure he can give me advice on how to proceed.}

The chapter ends with \Teshrial musing to himself about how much help he can get and how awesome it will be.















