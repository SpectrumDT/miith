\bookchapter{Nadir Begins}
\placestamp{Mirage Asylum}
\maybehr{Ishnaruchaefir's Nadir}{\ps{\Ishnaruchaefir} Nadir} is coming, and he feels its onset. 
He prepares for it. 
Sets \Rystessakhin{} in a special place where she can \quo{recharge}. 

\Ishnaruchaefir \maybehr{Ishnaruchaefir bleeds in Nadir}{bleeds and looks terrible} when he is in the Nadir. 

Make clear that this is of vital importance for the Shroud and to protect \Miith{} from alien menaces (the \banes). 
This makes \Ishnaruchaefir{} dreadfully vulnerable, and \Criseis{} and the grandchildren know that he runs a terrible risk by going into battle at such a time. 

But to \Ishnaruchaefir, it is worth it. 
The resurrection of \Nithdornazsh{} would be a monumental victory for the Sentinels. 
He does not tell the people around him about his plan, of course. 

He leaves instructions with \Criseis{} and \Thiencaste-tachi about what to do if he dies. 
How to take care of \Rystessakhin{} and all that. 

\Ishnaruchaefir: 
\ta{If I fall, you know what to do.}

\Criseis{} is very worried when she sees him put down \Rystessakhin. 
\maybehr{Nadirs get worse}{His Nadirs are getting worse} every time. 
Something must be profoundly wrong with the Shroud. 
She has tried to pry out of him what is wrong, but he is not talking much. 
When he puts down the glaive he stands tall and arrogant and does not let anything show. 
But \Criseis{} has served him for ten thousand years and knows him better than any other. 
She can detect all the little telltale signs: 
the way he hesitates for the briefest moment; the way he stands up a few millimetres taller afterwards when he no longer carries the glaive's burden. 
It all hints of a terrible, excuciating pain that would be unendurable to any lesser person\ldots{} even a lesser \dragon. 

She feels his pain. 
It is unfair, she thinks, that he has to carry this terrible burden and danger, and get nothing but scorn in return from the world. 
And she fears for the future. 
\maybehr{Nadirs get worse}{The Nadirs are getting steadily worse}. 
What if some day the burden gets so heavy that even he can no longer shoulder it? 

Before his great duel, \Ishnaruchaefir{} draws some energy from \Rystessakhin \dash as much as he dares\dash and uses it to empower his \maybehs{ward runes}. 
(Remember to read about \maybehs{ward runes}.)

Maybe \Criseis does not see \Ishnaruchaefir up close. 
She fears to get close to him.
He is radiating powerful and dangerous energy.
A lash with one of those whirling energy threads might kill her.
So she stays far away.
She contacts him with telepathy. 
(This way, the reader also only sees \Ishnaruchaefir as a blur.)

\Criseis warns him:

\begin{prose}
  \Criseis:
  \ta{\Teshrial{} is well-prepared this time. 
    He has studied your strengths and weaknesses.
    I believe he has studied \WanderersInDarknessEmph.}
  
  \Ishnaruchaefir:
  \ta{Has he now?}
  (Smug, mysterious smile.
  He knows \Teshrial{} has taken the bait and fallen for the story of his alleged Achilles Heel.)
  
  \Criseis:
  \ta{Do not do this, master!
    I beg you.
    It is obviously a trap.
    And you are at your weakest.
    You may fall!}
  
  \Ishnaruchaefir{} (looking wistful, contemplating the possibility that he might die):
  \ta{I might.
    But know this, \Criseis:
    This battle will be of pivotal importance.
    This I predict.
    A mighty storm is brewing on the Pelidorian horizon.
    This storm will herald a \thirdbanewar.
    And that is a war I intend for my race to win.
    It is a risk, but I am willing to take it.
    For \Nexagglachel.
    For our people.
    Even\ldots{} even for \Secherdamon, perhaps.}
\end{prose}





