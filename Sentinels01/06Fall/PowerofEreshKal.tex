\bookchapter{The Power of \EreshKal}

\begin{comment}
  \section{Takestsha casts the witch-storm}
\end{comment}

\stamp{\dateTakestshaStormsForclin}{War room, \Forclin}

\begin{comment}
  \subsection{Pelidorian war room}
\end{comment}

Back in \Forclin. 
The remnants of the sally are back inside the city walls. 
They have suffered heavy losses. 
\Dornaer is confirmed dead. 
\Rah[\Theal] \Kintaer is missing and presumed dead or captured. 
And the Rungeran \ishrah still stands, with not one mage fallen.
And yet the sally was not a complete failure, as Carzain \Shachar is explaining. 
Sanyor managed to sneak in and cast a spell on the cannon. 
Carzain cannot explain the details, since he only got a glimpse of it, but Sanyor invoked the water-related \sephiroth and somehow made the insides of the big cannon moist. 
And not something that can be easily cleaned out, it would seem, for the big cannon had not fired since then. 

The Pelidorians have thus gained a respite.
It lasts a day. 
Then, the morning after, stuff happens. 
For now a messenger runs into the war room and tells them the Rungerans are on the move. 
They go out to look. 



\begin{comment}
  \subsection{\Takestsha moves}
\end{comment}
\new
The Pelidorians awaken to find that some big trenches and ramparts have been dug in the night. 
Sethgal is awakened by a messenger who tells him that something has happened. 
It is now early morning. 
He goes to the wall.
He looks out over the battlefield. 
There is are big trenches halfway between the Rungeran camp and the wall. 
They must have been dug in the night. 

Down in the trenches can be seen glimpses of soldiers. 
His cannoneers cannot hit them now. 

How did this happen? he asks.
How could the Rungerans have dug all these large trenches without anyone noticing?

\Esmerel suspects sorcery. 
She has spoken to several people who have dreamt strange dreams in the night, of sinister forces slithering beneath the earth.
You may not believe me.
You tend not to heed Redcor warnings until it is too late. 
But I believe these dreams are visions. 
Not exact representations of reality, perhaps, but indications.

\ta{I have to agree,} says Curwen.
There is a slight hint that he does not like to agree with the Redcor woman.
\ta{There must be sorcery involved.
  We already know that their \ishrah wields magic of a kind unknown to us.}

\Esmerel:
\ta{Of a kind unknown to us?
  I beg to differ.
  I have been to the south recently, and I recognize a a distinctly Rissitic taint to their sorcery.}

This surprises Curwen, although he tries to hide it. 
\ta{That is possible,} he acknowledges.
\ta{At any rate, they wield spells unknown to us.
  They may have at their disposal forces capable of doing this.}
He gestures out over the trenches. 
\ta{And now their \ishrah is down there.
  Behind the tallest, strongest rampart.}

Sethgal looks and sees that they are well protected.
The cannons cannot reach them down there, for the wall is tall and thick and solid.
He also notices that the ground around the trench, and the other trenches, has been dug up and is now full of holes. 
It is as if the plain has been undermined by giant moles from beneath the earth. 

Curwen:
\ta{They are casting a spell.}

This surprises and worries Sethgal. 

Curwen:
\ta{They have been at it since I got here, but so far we have seen no results.
  It must be a long ritual spell.}

Sethgal: 
\ta{Like what they used against us in the last battle?}

\ta{Likely.
  The energies feel similar.
  But the spell last time did not take nearly this long.}

\ta{Are we within their range?}

\ta{Normally I would say no, but under these circumstances I will not promise anything.
  Lord Marshal, we should sally out and destroy them now while we have the chance.}

Sethgal:
\ta{Hmm.
  We have their \ishrah right under our walls.
  That leaves them in a vulnerable position.
  They are no longer in the middle of their camp with the entire Rungeran army protecting them.
  If we sally out against them, we should logically have a much better chance of destroying them than last time.}
He looks out. 
There are soldiers hiding in all the ramparts, but there cannot be very many.
Most of the Rungerans are far away.
If he sallies out against the \ishrah, Morgan might send reinforcements to protect them, even if that means sending his men within range of the Pelidorian artillery. 
But such reinforcements would take some time to arrive, so if Sethgal is lucky he should be able to destroy the \ishrah before his men get swamped by Rungerans. 
\ta{But Morgan knows this.
  So either he is desperate and wants to take the city \emph{now}, or this is a trap to lure us out.
  Why would Morgan be in a hurry?
  As the besieger, time ought to be on his side.
  I suspect some sort of trap.}
He considers it for a while. 
\ta{No. 
  I will not do anything rash.
  I must consider this first.
  Keep an eye on their spellwork, Lord Curwen, and let me know if anything happens.}




\begin{comment}
  \subsection{Witch-storms}
\end{comment}
\new
\Takestsha pushes her \ishrah to begin a big-ass spell. 
They begin casting it by morning. 
A big-ass ritual spell. 
It takes a long time to cast. 
Hours. 

As the day passed, things begin to happen. 
The guards report seeing hideous apparitions flying in the air, like last time but more indistinct, more ghostly. 
Soldiers and civilians claim to hear fiends shrieking in the wind, and the wind itself seems to bite with a more vicious cold than before. 

But at first they do no real damage, other than to morale.

From the walls you can see the \ishrah calling down ghastly elder things from the outer spaces of the aeons.

As afternoon slowly begins to darken into evening, the witch-storm intensifies. 
Everyone has to admit that the fiends were not just fevered visions.
The shapes can now be clearly seen whirling around the mages and flying above the city. 
And you can hear them howl, louder and louder. 

A few people get stricken with unnatural sores and pains.
A few weak people die. 
These are sick people, beggars and wounded soldiers, people who might have died anyway. 
But they did have the strange sores, so people believe the dark magic killed them.
Which, in a way, it did.
It is much easier for the spell to strike people who are already weak, since their saving throws will be lower.

Fear and panic begins to spread. 
Morale deteriorates. 

Something has to be done. 

The effects of the magic have to be subtle, though!
No direct attacks!
Only things that might conceivably be coincidence. 





\begin{comment}
  \subsection{War room}
\end{comment}
\new
Sethgal and Curwen realize they have to do something. 
They sit in a war room. 

Curwen explains. 
It is clear that the Rungerans' spell is intensifying. 
\Takestsha and her companions are feeding more power into those \daemons, and they will grow more powerful by the moment. 
If they are allowed to go on all day the spell will be unstoppable in the end. 
No one knows exactly what will happen then, but there is reason to believe it will be the same thing as last time. 

The Pelidorian cannons have tried to bombard the \ishrah's position, but it is a very well-dug rampart, and the Rungerans can repair it. 
There is not much the cannons can do. 
According to Lord Curwen the rampart is also protected by powerful wards and difficult to penetrate with magic. 

The Pelidorians have to sally out and stop them. 

It will not be easy, but not all is lost. 
The \ishrah is fairly close to the wall, since evidently their spells have a limited range. 
And the Rungerans do not have their entire army stationed around the \ishrah to protect them, since they would be sitting ducks inside the range of the Pelidorian cannons and gunners. 
They only have a fairly small (and very well-protected) army guarding the \ishrah. 

They face three challenges.

First, they will have to deal with whatever troops the Rungerans have hidden in their trenches. 
So far so good. 
That is the easy part. 

Second, they have to reach the \ishrah before reinforcements arrive from the Rungeran lines.
That, in itself, is also doable, but it will not be easy. 
The ground is dug up and full of gopher holes, so they cannot easily charge out. 
This was no doubt an intentional part of the Rungerans' sorcery. 
Also, there are a lot of Rungerans.
The Pelidorians were outnumbered last time and the situation has only gotten worse. 
The Pelidorian artillery can offer some support by shooting at the Rungeran reserves that come in to help, but in the course of a battle it is limited how much damage the cannons can do, since cannons fire slowly and are not very accurate against personnel to begin with. 

Third, they have to deal with the \ishrah mages themselves and whatever sorcerous protection they may have. 
This includes the howling witch-storm.
And the Rungerans probably have other secret weapons prepared. 

Sethgal is not too confident. 
There are too many unknowns. 
And morale is low.
No one is eager to face the witch-storm. 
The One Light may be on their side, but the Outer Darkness stands against them, dreadfully tangible in all its destructive horror. 

Sethgal sends out a sally with his strongest remaining fast cavalry. 
And with slow cavalry as a second wave to reinforce them and help smash through the Rungeran ranks. 
Sethgal leads the Pelidorian attack himself. 
He feels that he needs to, for morale reasons.
Two great leaders have fallen.
They do not have many left. 
The people must not see that their last leaders huddle behind the walls and send their underlings out to die.

\ta{You should take \Shachar with you, Lord Marshal,} says Curwen. 

\Shachar nods. 
He did not look so good when he returned with his tail between his legs from his last battle, but he has mostly recovered. 

Curwen: 
\ta{I will remain behind.
  I will be of more aid from the inside than from the outside.
  I have a plan that may yet turn the tide of sorcery in our favour.}





\begin{comment}
  \section{Sethgal rides out}
\end{comment}

\begin{comment}
  \subsection{Sethgal faces Rungerans in trenches}
\end{comment}
\new
By late afternoon the Pelidorians ride out. 
As soon as the Pelidorians open the gate and begin to move out, the Rungerans send cavalry reinforcements up towards the city. 
The Pelidorians do not have much time until the Rungeran reinforcements arrive.
They must get in there and get near the \ishrah. 

(Make it clear in this chapter that Sethgal-tachi are desperate. 
 When they ride out they expect to die.
 But at the last moment they are saved by the Imetrians.)

The Pelidorians have to ride some way from the gate to reach the \ishrah, and there are some other trenches in the way. 
Rungeran troops swarm out from the trenches. 
There are not so many of them. 
Enough that Sethgal's cavalry can easily defeat them, but enough to slow the Pelidorians down. 

Sethgal grits his teeth and calls the attack. 




\begin{comment}
  \subsection{Curwen goes to the Ghost Tower}
\end{comment}
\new
\target{Charcoal at the Ghost Tower}
Throughout much of the story, Archibald Curwen (Charcoal), supposedly the sneaky master Cabalist, is duped, manipulated and played for a fool by his enemies. Sentinels and other agents seem to run circles around him. 
He's been played for a \trope{XanatosSucker}{Xanatos Sucker} the whole book. 
But at the end he finally realizes what's going on around him and strikes back. 

Near the end\dash perhaps after having discovered one or more of the people who have been cheating him, such as \Sanyor{}\dash Charcoal shows what a formidable agent he truly is. 

Curwen realizes that \Takestsha and her \ishrah are too powerful and dangerous.
He cannot just use conventional tactics against them.
Instead, he gets an idea. 
He devises a master plan to dispel the Rungerans' \EreshKali magic and strike a hard blow to their forces. 
But knows that will be hard. 
He cannot do it without help. 

So he gets another idea. 
He can use the Ghost Tower.

Curwen has been in \Forclin before. 
He knows the Ghost Tower and has even been inside it. 
He knows it is a conduit to the Realm where the \resphain live (though he has not been there). 
He formulates a plan to use the Tower in his counterspell against \Takestsha by channelling energy through the Tower's \nexus, or something like that. 

So he leaves Carzain in charge of the \ishrah and departs the battlefield, heading for the Ghost Tower. 





\begin{comment}
  \subsection{Sethgal gets zapped by the \ishrah}
\end{comment}
\new
Sethgal nears the \ishrah's position. 
They are getting close.
But it is slow moving over this broken ground. 
The \ishrah strikes back. 

The witch-storm has intensified.
It towers above them now.
A churning vortex of half-ghostly, half-fleshly forms writhing and whirling. 
A vast hideous pillar of evil that darkens the sky and turns afternoon into twilight. 
It strikes at them.
It disgorges its screaming horrors. 
Shapeless things fly at them.
The things rend and tear the flesh, and the soldiers are defenseless against them, for normal weapons seem powerless against the bodiless wraiths.

(The Pelidorian \ishrah mages rain down their own magic. They try to counter the witch-storm.)

Many Pelidorians die.
Their morale falters. 
They break formation and want to flee.
Sethgal has to use all his leadership talent just to get them to continue the attack.
They lose precious momentum.
Sethgal curses.
Ahead he can see Morgan's hordes coming closer every moment. 

Another strike from the \ishrah.
This time directly at Sethgal's position. 
Fiends whirl at them, lashing out with their claws and mouths. 
By virtue of his quickness, luck, \armour and the wards his own \ishrah have woven around him, Sethgal himself survives mostly unscathed, but his \relc founders. 
He drags himself out of the poor downed \saurian's saddle and shouts to his men.
One gives him a \relc. 
He resumes command, but the blow has stolen even more momentum from his men. 
And here come the Rungeran reinforcements.
They will now have to fight their way through them to get to the \ishrah. 

But they cannot retreat and try again. 
Time and attrition are working against them, not for them. 
They must succeed now or they will fail. 

Sethgal gives another speech and calls the attack. 





\begin{comment}
  \subsection{Curwen contacts \Achsah}
\end{comment}
\new
On the way to the Tower, Curwen recites an orison to contact \Achsah and ask for her advice. 

\begin{prose}
  \Achsah:
  \ta{What? Do you have urgent news about the Sentinels?}
  
  Curwen:
  \ta{Well\prikker no, my Lady \Resvil. But\prikker}
  
  \Achsah:
  \ta{Then do not pester me.
    Figure it out yourself.
    I am busy.}
\end{prose}

\Achsah is unwilling to help. 
She has her hands full. 
She is stressed. 
She is sure the Sentinels are up to something really nasty here.
She tries to figure out what it is. 
She has no time to advise Charcoal. 
He is on his own.
She is sure he can solve his own problems. 
He is a skilled mage and a high-circle Cabalist. 





\begin{comment}
  \subsection{Nephil amputee}
\end{comment}
\new
Sethgal hears an animalistic roar. 
He looks up just in time to see a giant monster come charging him.
It is a thing like a giant \human, but broader, hairier and more bestial. 
Clad in impossibly thick and heavy \armour. 
An ogre. 
A \nephil. 

The ogre charges. 
With the great blades on its arms it slashes through a \scatha and her \relc in one blow. 
Other riders bravely attack the thing with their lances and swords, but it tears them apart. 

Sethgal:
\ta{Gunners! Where are our gunners?}

\tho{And our \murocs. 
  These \nephilim are too power.
  We will need our heavy beasts to deal with them.}

But no help is forthcoming.
Sethgal and his men have to fight off the \nephil as best they can. 
Some of them get their sidearm pistols and shoot at it, but it is wearing thick \armour, so they only manage to score a few hits. 
One heroic knight charges it on his \grulcan. 
The great \grulcan-bird leaps up on the \nephil's back and rips at it with its sword-like claws and beak.
The rider stabs into the cracks of its \armour with his lance. 
The ogre roars in pain and ties to shake off the bird. 
Meanwhile Sethgal and his other knights move in and attack. 
One rider gets close and fires his pistol right into the ogre's face. 
It roars and rears up and finally manges to shake off the guys on its back. 

Sethgal himself rides by it and slashes its unprotected foot, causing a deep bleeding gash. 
The monster swings at him.
He ducks.
He is whacked on the back of his helm, but he narrowly avoids getting his head chopped off. 
His helm is bent out of shape.
He cannot see nor breathe.
He has to struggle hard to get the helm off, and when he finally tears it off it rips away many scales with it and leaves what must be an ugly scar on the top of his head and some minor scrapes on his snout. 
He turns back to the \nephil.
It is still fighting. 
He strikes with his sword. 
He aims for a crack in its armour but misses.
His blow glances off harmlessly. 
But he gets the monster's attention.
Again it swings at him, and again he narrowly evades. 
But he has done his part.
For now another knight rams it at full speed and impales it with a spear through its head. 
The slayer quickly flees on his \relc, for the monster spasms with its bladed arms and almost chops him apart. 
But now it lies bleeding, and soon it lies still. 

After they kill it they realize something: 
It has no hands.
Its hands have been amputated and replaced with blades, leaving it a cripple that can only destroy and cannot survive on its own. 

Sethgal thinks about how \human-like and yet bestial the thing is. 
Read about \maybehr{Nephil beast-men}{\nephil beast-men}. 
Sethgal can use this as an example to prove how close to beasts \humans really are. 
\Nephilim are just one step above animals and one step below \humans.
Sethgal, at least, would like to tell himself that his race stood more than a few steps above the beasts of the \wylde. 

Sethgal breathes a sigh of relief. 
It took a hard fight and the sacrifice of brave warriors' lives, but the monster is dead. 
There is hope for Pelidor yet. 

Then he looks up, and his heart sinks. 
For just ahead, barrelling their way through the Pelidorian ranks straight, come three more \nephilim.
The brave Pelidorians close ranks to stop them, but the monsters chop through them relentlessly and trample their bloodied remains into the dirt. 

Sethgal grits his teeth and raises his sword and prepares to fight the \nephilim.

\tho{Perhaps this is the day I die.
  If that is so, then I will die a knight, defending Pelidor and \iquin itself from evil.}

The ogres shamble nearer. 
Sethgal expects them to come and chop him down any moment. 
Then they slow down. 
The first one drags its feet as if suddenly stuck in muck. 
Sethgal looks at the ground at its feet and takes a step back in horror. 

There is some amorphous slithering thing clinging to the ogres' legs. 
It looks like an animated mass of blood and severed limbs. 
The ogres are howling terrified now and trying frantically to tear themselves loose, but to no avail. 
The thing of slime grabs and sucks at them as if trying to suck them apart and digest them and absorb their flesh into itself. 
One \nephil falls to its knees, and the amorphous thing creeps up its body and oozes through the cracks in its \armour.
Blotches of living and crawling blood can be seen on the terrified ogre's face. 

\ta{\Rah[Sethgal]!} comes the loud shout of a \human's voice behind him.
Sethgal looks back and sees Carzain \Shachar. 

\ta{What are you waiting for? 
  Finish them off! 
  I cannot keep this up forever!}

\tho{Sweet merciful Silqua preserve me.
  It is of \Shachar's doing.}

Sethgal pulls himself together, though.
With a short inspiring speech he makes his men advance. 
The ogres are writhing blindly on the ground, but dangerous enough with their bladed arms. 
None of his men are willing to go close, so Sethgal has to go first.
It takes all of his will and riding skill, but he makes his mount go up to one.
He takes a swing at its head with his sword. 
The ogre is too weak to truly defend itself. 
He chops into it with his sword, again and again and again, until what was once its head now hangs in bloody tatters. 
Inspired by his example, his men are now brave enough to go in and put the other two \nephilim out of their misery. 

Sethgal notices that the slithering mass is gone. 
There are now only three dead \nephilim smeared in blood and gore. 

\Shachar rides up to him.

\ta{You did this?} asks Sethgal with loathing in his voice.
\ta{How?}

\Shachar gives him a serious look. 
\ta{Do you truly wish to know, \rah[Sethgal]?}

\ta{No. No, I do not.}

\tho{Sweet merciful Silqua.
  Is this how desperate our need is, that our own \ishrah must resort to such awful necromancy?}

After a while he adds: 
\ta{You came just in time, \MrShachar.}
He cannot bring himself to meet the mercenary sorcerer's gaze. 

\ta{But we are not out of this yet.}

Then there is a loud howling from above. 
They have to run for their lives as a shrieking shadow of pure death dives at them from the sky. 
Sethgal dodges and escapes with his life, but he can feel its unnatural cold wash through him as it flies by, numbing him and draining his vitality. 
The strangled cries from behind tell him that some of his warriors did not make it. 
He does not want to turn around and see the looks of agony on their dead, withered faces. 

Above them the witch-storm still swirls and howls and lashes out with its tendrils of black horror. 

\tho{We have to do something. 
  We have to reach the \ishrah.
  We have to stop this.}






\begin{comment}
  \subsection{Esmerel}
\end{comment}
\new
\Esmerel is in \Forclin.
She spends part of her time healing the people wounded in the bombardment.
When she has some free time, she goes and watches Carzain from a murder hole somewhere in the wall. 
She is interested in Carzain. 
She suspects he is a Scion. 
Maybe.

Carzain unleashes all his dark, sinister magic.
This scares Sethgal and the other Pelidorians around him.
Carzain's magic looks almost as evil as the Pelidorians', just on a smaller scale. 

Remember to read about \Esmerel. 
She is a nerd. 





\begin{comment}
  \subsection{Imetrians come to the rescue}
\end{comment}
Sethgal looks around. 
There are still Rungerans everywhere. 
There are many more ogres and heavy Rungeran cavalry between them and the \ishrah.
They cannot just charge through. 
They would be destroyed if they tried. 
They must find another way. 

Sethgal is usually a highly skilled tactician.
Sethgal wishes he could rely on tactics to win through. 
But here in the middle of a battlefield full of death and unnatural terror there can be no tactics. 
He cannot get a hold of his army. 
They barely have any morale left. 
Coming out here in person was the right decision at the time, but what can Sethgal do out here?
Even the men right next to him barely have the strength of will left to obey his commands and fight. 
Many are fleeing, many are dead. 
How can he hope to rally his forces and make any kind of focused attack?
How can his forces do anything but be swept away? 

He prays to the \sephiroth for guidance. 
He prays to be delivered from doubt and fear and despair.
Please deliver me from the corruption and madness which emanate from \Isphet. 
He prays for the strength to shoulder the responsibility that is his. 
He is the centrepiece of this army.
He must not succumb, for if he succumbs, so will the army.
The battle will be lost. 
\Forclin will be lost.
Pelidor will be lost.
The evil of the Outer Darkness will triumph.
He cannot let that happen.
So please, \sephiroth, give me a sign.
Tell me what to do.
Give me the clarity to see the way through this. 

While he prays men die all around him. 
The witch-storm strikes. 
\Shachar has to use his spells just to keep himself and Sethgal alive. 
Interject this in between Sethgal's prayings above. 

No answer from the \sephiroth. 

And now a new force of Rungeran cavalry appears.
They come at them and are just about to charge and ride Sethgal's men down.
He has few men left.
They are desperately outnumbered.
\Shachar's sorcery might be able to confuse the Rungerans and slow them down, but Sethgal does not think it can stop this many. 
And even if it can, what then? 
There will be more. 
The Rungeran reinforcements are pouring in like a tidal wave. 
They will overpower the Pelidorian defenders no matter what. 

Sethgal steels himself.
He will not die cowering. 
He will meet their charge and die with defiance. 
He raises his sword and silently calls out his men to advance and attack.
He lifts his feet in order to kick his \relc into motion. 

Then something happens! 
Loud horns sound in the north! 
Confusion spreads through the Rungeran ranks.
They were just about to set into a charge, but now they hesitate. 
Their \relcs prance uncertainly. 

A rumble of feet from the north.
Renewed sounds of battle. 
And over it all the calling of loud horns. 

Sethgal realizes what it is. 

It is the Imetrians!
The Imetrians come!
The allies that Sethgal had so long hoped for!
They are here! 

Sethgal had long hoped for Imetrian reinforcements. 
He was about to abandon hope.
But now they come. 

\ta{It is the Imetrians!
  The Imetrians!
  Our allies are here!
  Come, brethren!
  Strike north!
  We must join our allies!}

His forces follow him, vitalized with new energy. 
The Rungerans pull back in confusion, and the few stragglers that get in Sethgal-tachi's way are cut down. 

He shouts to his warriors as he rides. 
He hopes to attract and encourage and embolden as many Pelidorians as he possibly can. 
\ta{Rally, warriors of Pelidor!
  To me!
  To me!
  The Imetrium has come!
  Aid has come!
  Rally!
  We strike north!
  North!
  Follow me!}
His efforts succeed, and gradually more and more Pelidorian soldiers in orange and blue rally around him. 
They are now again a formidable force. 
They cut northwards through the Rungeran forces. 

Ahead now they can see great banners bearing a four-pointed star. 
Through the fray they glimpse warriors clad in black and white, and many fierce \saurians. 
All around Rungerans are dropping, pierced with arrows. 

Sethgal's men cut through another rank of enemies. 
Sethgal himself cuts down a Rungeran soldier, severing head from body with his sword. 
And there they are.
The warriors of the Imetrium. 

There are hundreds of them. 
Not a large army, but enough to make a difference. 
The Imetrians are all cavalry. 
The Imetrians have some elite archers. 
They are much more powerful than gunners, but much fewer in number, because they have to be experts.

And there are \nycans. 
They are terrible and awesome to behold. 

Sethgal rides up and hails them. 
As a noble who has done much dealing with the Imetrium, he speaks the Imetric tongue. 
He shouts a greeting in Imetric at the top of his lungs. 

Telcastora Ilcas rides forth to greet him.
They know each other.

Ilcas and Sethgal clasp hands and greet each other.
Sethgal is happy to have the Imetrians on his side in the coming siege.

\begin{prose}
  Sethgal: 
  \ta{I knew we could rely on you.}
  
  \tho{That's actually not true. 
    I thought we couldn't. 
    I \maybehr{Sethgal curses Imetrians}{cursed the Imetrians earlier} for being faithless allies.
    But they proved me wrong.} 
\end{prose}

Carzain rides up. 
Sethgal introduces Carzain. 
Ilcas clasps Carzain's hand. 
Ilcas introduces himself as \Retaxis Telcastora Ilcas. 

\begin{prose}
  Carzain: 
  \ta{\quo{\Retaxis}? 
    Is that title new? I seem to remember you had another one last time.}
  
  Ilcas: 
  \ta{%
    Yes. 
    I have been promoted since last we met.} 
  He smiles. 
  \ta{%
    Salacar be praised, I now hold the same rank as my wife. 
    It is just an \honorary rank, mind you. 
    I do not command troops as a \Retaxis{} normally would. 
    I requested this reward because I was sick of being outranked by my wife.}

  Sethgal:
  \ta{Such precision. How did you know where to strike?}

  Ilcas:
  \ta{Using telepathy our mage was able to contact one of your mages on the wall.
    He briefed us about the situation.}

  Sethgal:
  \ta{You have a mage?}
  
  Ilcas:
  \ta{Had. She was killed in the attack.}
\end{prose}

Sethgal and Ilcas talk. 
Ilcas has heard the basics of their idea.
Sethgal fills him in.
They have to attack the \ishrah. 
At this point they do not have the tactical surplus to create diversionary \manoeuvres or pincer attacks or anything. 
They should strike quickly while the Rungerans are still somewhat disarrayed. 
They should strike immediately with all the strength they can muster. '

Ilcas:
\ta{We can make it to the centre.
  These Rungerans are many, but we are strong, and we are fast.
  But I am worried about this \ishrah.
  Can we stand against them?}

Sethgal:
\ta{We will have to.}

Ilcas nods with determination.
\ta{Can we rely on magical support ourselves?}

Sethgal:
\ta{Out here with us we have only \MrShachar.
  The rest of our \ishrah is supporting us from the walls.
  We will have to trust them to know when and how to act.}

Ilcas nods.
\ta{Let us go, then.}

They move out. 





\begin{comment}
  \subsection{\Achsah suspects \Takestsha}
\end{comment}
\new
\Achsah, who is in \Forclin, looks at the \quo{\EreshKali} spells cast by \Takestsha-tachi. 
\Achsah{} wonders when she first observes the \EreshKali{} magic. 
It does not feel like anything she would expect them to have. 
It also does not feel like what she would expect ancient \meccara{} to have. 
It is new to her, and it makes her suspicious. 
But what it \emph{does} smell like is Rissitic magic. 
That makes her even more suspicious. 
(She has heard Charcoal's account of Tantor's diary, but Charcoal has never seen Rissitic magic, so he cannot draw the connection.) 

Then she realizes that those spells are actually meant to tear the Shroud and reach into the Beyond, where it can summon\prikker stuff. 
And the Ghost Tower, which is in close proximity now that the Rungerans have breached \Forclin, acts as a catalyst. 
Those spells are tearing at the very fabric of the Shroud. 
Something fucking nasty is breaking through, or so \Achsah thinks. 
In reality the summoning spell at \Forclin is a smokescreen. 
It is meant to warp the Shroud and look big and impressive, but it doesn't actually \emph{do} anything. 
It's just meant to attract attention and convince everyone that the real stuff is happening in \Forclin, near the Ghost Tower. 

\Achsah now strongly suspects that \Malcur is a decoy and that the Sentinels' real goal is \Forclin. 

But still she keeps watching.
She does not intervene. 
Partially because she must remain ready and keep a bird's eye view of the action and cannot afford to commit herself to any narrow battlefield action.
Partially because of the Unspoken Covenant.
She will not be the first to break it. 





\begin{comment}
  \subsection{Imetrians fight}
\end{comment}
\new
Ilcas shouts from his saddle in \Tepharin: 
\ta{The Rungerans are setting up a line of gunners.
  Behind those infantry lines there.}

Sethgal wonders for a moment how Ilcas knows that.
Then he remembers that Ilcas is a \nycaneer and allegedly wields supernatural powers of perception.
Perhaps through the blessing of the mysterious gods of the Imetrium, Ilcas is in communion with strange forces. 

\ta{My forces are not sufficiently \armoured to face a barrage of gunfire with impunity.
  \Shachar, can you do something about them?}

\ta{What do you have in mind, Telcastora?}

\ta{As soon as we break through this infantry, the gunners will want to fire. 
  I need you to distract them long enough for for my \nycans to close and dispatch them.}

Carzain looks at the scary lizards, and Sethgal can tell that even the mercenary sorcerer looks disconcerted by them.
\ta{I'll see what I can do.}

Ilcas:
\ta{\Rah[Sethgal], what about their \ishrah?}

Sethgal glances at the towering vortex of power. 
\ta{If we move fast they will not be able to hit us properly.
  When we break their lines they will not be able to strike at us lest they hit their own troops.}

\tho{I hope.}

No more time to talk.
They have almost reached the Rungeran infantry. 
Arrows fly over Sethgal's head from the mounted Imetric archers behind. 
They could not fire with any great accuracy, but it was enough to fell a few Rungerans and disturb their formation. 

They charge into the Rungeran lines, easily breaking them. 
At the same moment, Sethgal hears shocked screams from the Rungerans further ahead. 
\tho{Great \sephiroth.
  What is \Shachar doing to them?}

They kill the Rungeran footmen.
Sethgal looks ahead. 
The gunners ahead are in disarray.
Many of them are struggling against some slithering things on the ground. 
Sethgal can tell \Shachar's sorcery is at work again. 
He does not want to look more closely. 
(Maybe move the whole spell scene down here.)

Some few gunners are ready to shoot, though. 
But now the \nycans attack. 
They rush in from the sides, too quick to shoot.
The gunners try to fire, but Sethgal sees no \nycans going down.
The \saurians sprint, and in the blink of an eye they close the gap. 
They start tearing the Rungeran gunners apart. 
Sethgal is amazed at how effective the monsters are.
He has never seen them fight before. 
They are quick as lightning and can disembowel a \human or \scatha with one blow of their bladed feet. 

Then the Imetrian soldiers close. 
Describe how skilled, disciplined and fearless the Imetrians are, with their Imetric gods giving them courage and strength. 
Sethgal almost feels the power of their heathen gods course through them and giving them strength. 
They are supernaturally strong and effective.
They fight with tremendous coordination, making them super-effective.

\tho{Is this telepathy?

  Whatever it is, we are pushing forward.
  We might reach their centre yet.}

Remember to read about \maybehs{Telcastora Ilcas} and the \nycans{} before writing this. 





\begin{comment}
  \subsection{The magic of Eresh-Kal}
\end{comment}
\new
The \Rungertemple{} magic might be defiler magic (as in \emph{Dungeons and Dragons: Dark Sun}), sucking life out of the world and leaving it gray, dusty and dead. Alternately, it might be bestial, destructive and chaotic magic, appalling in its sheer hate, ferocity and inhumanity. 

The magic involves the conjuration and binding of terrible \daemons{} from \Chaos. These should be as horrible, inhuman and Cthulhu-like as possible. The \daemons{} are the source of their power; they're the ones wreaking the destruction. 

It is hinted that the \daemons{} are not really bound; they are just playing along for their own unfathomable reasons, and may decide to turn on the mages any time. Have at least one scene where the sorcery suddenly backlashes on one of the mages and he dies a horrible death, his flesh boiled and burnt and his soul consumed by the \daemon{} he unleashed. 

\citeauthorbook[p.115--116]{MichaelMcBride:LaArmadaInvencible}{Michael McBride}{%
  La Armada Invencible%
}{
  There had been three cloaked bodies atop each of the two highest crests of land, joined at the hands and spinning in circles.
  \prikker 
  A cone of light the consistency of smoke had arisen from between them and reached up into the sky, expanding into massive thunderheads from where the pinnacle met with the azure atmosphere.
  These clouds grew in height and width until they appeared to be mountains forming above, their black hearts darkening until day turned to night, spreading across the entire horizon, which came to life with strobes of electricity slashing through the fabric of reality.
}

Make it clear that the foolish \humans{} are playing with powers far beyond their understanding. 

It is very hard, taxing and traumatic work for the mages. The sorcerers, being ill-informed and ill-educated in the use of this great power, are twisted by it. Their bodies become warped and misshapen, and they go more and more mad. Compare to Hannan Mosag and his K'risnan in \cite{StevenEriksonIanCameronEsslemont:MalazanBookoftheFallen}.

\lyricsbs{Monolith Deathcult}{%
  1917 - Spring Offensive (Dulce Et Decorum Est)
}{
  Creeping like a snake from a can, \\
  the slithering stench of yellow death.\\
  Chemical flame of decay\\
  burning skin and intestine.\\
  Regurgitating the bloody guts.\\
  Spewing last life from a wretched soul.
}



\begin{comment}
  \subsection{Takestsha}
\end{comment}
\new
\Takestsha is disappointed. 
The Pelidorian sallies have been quite effective. 
She is not winning, and she should be.
She is in a hurry.
She should take \Forclin today. 
It is time for her to bring out the really big guns. 

\Takestsha and her mages have to start their big attack spell. 
\Takestsha knows it is risky. 
Her mages are not holding up as well as they should. 
They are weaker than she had hoped. 

If it were up to her, \Takestsha would be patient and lay a prolonged siege. 
But \maybehr{Psyrex tells Nzessuacrith to capture Forklin quickly}{\Secherdamon and \Psyrex have asked her to make haste}. 

\target{Takestsha will not become Nzessuacrith too soon}
\Takestsha \emph{knows} what \Secherdamon's plan is.
She knows her own attack is a decoy.
Her mission is to attract the \resphain's attention and fool them into coming to fight her. 
Conquering \Forclin is just a means to that end. 
She can, of course, assume \draconian form right away.
But she will not break the Unspoken Covenant for no reason.
She will first go as far as she can in her \human guise.
Only when she absolutely has to will she break her disguise.
If she were to take \draconian form too early, it would be suspicious, and the \resphain might not be fooled. 
It must look like she was forced to unveil herself. 

So she has to act in haste. 
She decides she must take some risks she would otherwise not have taken. 

So she and her \ishrah begin their great spell that will bring down \Forclin. 
It may be foiled, but \Takestsha hopes it will work. 

\Takestsha must push the Rungeran mages really hard. 
Push them to their limit and beyond it. 
Her master spell is a colossus on feet of clay, and she knows it. 

Remember to use occult spellwords.
Read about spellwords. 

\lyrics{%
  Neras gulja! 
  Neras gulja! \\
  Zegaled \KhothSell!\\
  Zurra! 
  Zurra! 
  \KhothSell!\\
  Neras! 
  Zurra!
}





\begin{comment}
  \section{Attack against the Ishrah}
\end{comment}
\new
The Pelidorians want to attack the Rungeran \ishrah. 

Maybe there have been several quicker sallies. 
Previously the Pelidorians just tried to do as much damage as they could. 
Now Curwen believes the Rungeran \ishrah is becoming a real threat. 
He asks Sethgal to send a sally out to destroy the \ishrah.
Sethgal is convinced and agrees. 

The Imetrians support them.
The Imetrians brought one mage with them. 
Unfortunately, he was killed in their initial charge. 

Sethgal will lead a diversion. 
Ilcas will lead the real strike force that will strike at the \ishrah. 
Carzain will support the Imetrians with his magic. 
Curwen will stay in \Forclin. 
The rest of the mages will support Sethgal. 





\begin{comment}
  \subsection{Curwen begins his counterspell}
\end{comment}
\new
Archibald Curwen is at the Ghost Tower. 

When Curwen enters the Ghost Tower, he finds that it is much larger on the inside than on the outside.
Curwen knows this is a Shroud phenomenon.
In the city, the repressive Shroud twists the mind and the eye and makes the tower look small, and takes a man along paths that make the tower look small.
But in here, the Shroud is weaker, so the true extent of the Tower reveals itself to him.
Or something like that.

Have a \quo{moon-shrouded crystal} or the like inside the Tower. 

\lyricsbs{Bal-Sagoth}{%
  Enthroned in the Temple of the Serpent Kings%
}{
  Deep within the glacial, ice-veiled temple,\\
  ancient enchantments summon the shades of the dreaming Serpent Kings,\\
  and the Ophidian Throne once again draws power from the Moon-shrouded crystal.
}

He begins his counterspell. 

\lyricslimbonicart{Solace of the Shadows}{
  I set the stones for invoking ceremonies.\\
  In the twilight zones arise abstract galaxies.\\
  The magic eye unveils the blackened skies.\\
  A new horizon begins to each one that dies.
}

He is frightened by the magnitude of the powers he unleashes. 
But also thrilled, exhilarated.

\lyricslimbonicart{Solace of the Shadows}{
  The desolation makes me feel\\
  so dark, so cold, the silence.\\
  So dark, so cold, the emptiness.\\
  Solace of the shadows.
  
  Night surrounds and embraces me.\\
  Darkness holds the secrets of man's fears.\\
  It captures my heart as the purgatory sears.\\
  I cast now the spell, as I cross through raging flames,\\
  into darkness, cursing names.
}





\begin{comment}
  \subsection{\Ishrah strikes back}
\end{comment}
\new
\Takestsha and her \ishrah mages realize they risk getting swarmed by these soldiers. 
So they turn their magic on them. 
It is horribly destructive and evil.







\lyricsauthorbookpage{Ben Counter}{Galaxy in Flames}{211}{
  Images tumbles through his brain, dark and monstrous, and he fought to hold onto his sanity as visions of pure evil assailed him. 
  
  Death, like a black seething mantle, hung over everything.
  
  Tendrils of darkness wound through the air, destroying whatever they touched. 
  He screamed as he saw the flesh sloughing from Mersadie's bones, looking down at his hands to see them rotting away before his eyes. 
  His skin peeled back, the bones maggot-white. 
}







\lyricsauthorbookpage{James Swallow}{The Flight of the Eisenstein}{262-265}{
  She screamed and flailed at him, beating her hands on his chest plate. 
  He could see now where her hands were bloody with self-inflicted scratch marks. 
  \ta{Eyes and blood,} she wailed. 
  \ta{But inside the pestilence!}
  
  \prikker
  
  \ta{Unclean, unclean! \prikker 
  I have seen it! Inside the eyes!} 
  She tore wildly at her face, ripping the skin and drawing blood. 
  \ta{You see it too!}
  
  \prikker
  
  The adjutant\prikker pressed into him and raked bloody fingers over his torso. 
  \ta{You see!} she gasped. 
  \ta{Soon the end comes! All will wither!}
  
  \prikker The woman's upturned face became paper, aged and crumbling. 
  She slid away from him\prikker turning into rags of meat and dead flesh, ash and then nothing. 
  
  \prikker
  
  The resplendent marble-white of their \armour bled away to become dis\coloured by a feeble, sickly green, the shade of new death. 
  The ceramite warped and became rippled, merging with their flesh until it stained and throbbed. 
  Parasites and bloated organs pulsed within, and in some places wounds opened like new mouths, red-lipped with tongues of distended bowel and duct. 
  
  \prikker The malformed shapes of his warriors crowded in, words falling from their cracked, lisping maws. 
  \prikker 
  Beyond the men he saw a ghostly form towering above them, too tall to fit into the cramped corridor yet there before him, beckoning with skeletal claws. 
  
  \ta{Mortarion?} he asked. 
  
  The twisted image of his primarch nodded, the figure's blackened hood dipping in sluggish acknowledgement. 
  What Garro could see of his primarch's \armour was no longer shining with steel and brass, but dis\coloured and corroded like old copper, wound with soiled bandages and soiled with rust. 
  The Death Lord was no more and in his place stood a creature of pure corruption. 
  
  \ta{Come, Nathaniel.} 
  The voice was a whisper of wind through dead trees, a breath from a sepulchre. 
  \ta{Soon we will all know the embrace of the Lord of Decay.} 
  
  \tho{The end comes.} 
  The words tolled in his mind like a bell, and Garro looked down at his hands. 
  His gauntlets were powder, flesh was sloughing off his fingers, bones emerging and turning into blackened twigs. 
  \ta{No!} he forced the denial from his throat. 
  \ta{This will not be!} 
}





\begin{comment}
  \subsection{Ilcas-tachi attack the \ishrah}
\end{comment}
\new
\target{Ilcas-tachi attack the Rungeran Ishrah}
Sethgal and his forces are strained to the utmost.
He has to hold back the mundane Rungeran army, which is enough of a problem already.
It is twice as large as his own, and the walls of \Forclin are crumbling under the Rungeran cannonade. 

Meanwhile, Telcastora Ilcas leads his Imetrians in a brave attack against the Rungeran \ishrah.
This is something Sethgal has asked them to do. 
The Imetrian mage, Ulphon, \maybehr{Ulphon Nestor dies}{has been killed}, so Ilcas asks Carzain to cover them. 
He does. 

Sethgal overhears someone (Ilcas or an Imetric priest) giving the soldiers a pep-talk. 

\begin{prose}
  Imetric priest: 
  \ta{If we die on this day, we shall live again!}
  
  Sethgal: 
  \tho{I know the Imetrians believe that they reincarnate when they die.
    I wonder if that is true.
    It is certainly not true for us Iquinians. 
    We die and go into the Light.
    But for them\prikker who can say?}
\end{prose}

The Imetrians scare Sethgal. 
They fight with great zeal, \maybehr{Imetrian coldness}{but their fervour is\prikker cold}. 
Calculating. 
Reptilian. 
He is quite disturbed.

Make sure to describe Ilcas as a super-badass fighter. 
Compare to Tisamon in \cite{AdrianTchaikovsky:ShadowsoftheApt} and Joscelin Verreuil in \cite{JacquelineCarey:KushielsLegacy}. 

\target{Ilcas injures Takestsha}
The \ishrah have massive ranks of soldiers protecting them.
But the Imetrians are fearsome fighters, and Carzain is a badass mage.
They break through and kill several mages. 
Ilcas and his \nycans even manage to seriously injure \Takestsha. 
She would have died if she were an ordinary mortal. 

They notice that \Takestsha is wearing trousers today, not a dress. 
She must be expecting to see close combat. 

This attack is won primarily by the Imetrians. 
Emphasize the courage and superhuman skill of Ilcas and his \nycans. 
They are awesome forces of destruction. 
Carzain also fights well, but he has a secondary role. 
Carzain gives them some artillery support and protects them against enemy magic, but he does not fight in \melee himself.
Carzain does not kill much.
He is mostly a distraction. 
This is the Imetrians' hour of triumph. 
Carzain's moment of glory comes later when he \maybehr{Carzain fights Takestsha alone}{fights \Takestsha alone}. 

\Takestsha knows this is bad.
Her spell is strained as it is.
If these interlopers kill too many of her mages, her spell will certainly fail. 

So she diverts her attention from the spell and unleashes some nasty spells against the attackers.
As nasty as she can make them in her current state. 
(She is in a weakened humanoid form, deep in the Shroud, and she is a bit exhausted, and she is stressed because she has so many things on her mind and must maintain so big and complex a spell.)

She kills several Imetrians and forces the rest to retreat.
Then her soldiers are able to close their ranks.
Carzain-tachi are overwhelmed.
They have no chance but to flee to save their own hides. 





\begin{comment}
  \subsection{The \EreshKali magic backfires}
\end{comment}
\new
\target{Eresh-Kali magic backfires}
Curwen is working on his counterspell.
When Carzain-tachi attack \Takestsha, he sees the opening he needs. 
He strikes with the full force of his counterspell. 

The spell catches \Takestsha-tachi at the worst possible time. 
The Rungeran \ishrah is reduced. 
Several mages have been killed in the \maybehr{Ilcas-tachi attack the Rungeran Ishrah}{Imetric attack}. 
The remaining ones cannot endure all the stress and strain.
The \EreshKali spells backfire on them. 
This kills the entire Rungeran \ishrah{}.
Only \Takestsha survives, and she is badly wounded. 

This buys the Cabalists time to send in reinforcements, including \banes{} and perhaps \resphain, which forces \Nzessuacrith{} to breach the \charade{} and assume her \draconic{} form to fight them off. 

By now \Forclin{} is pretty doomed, but now \Nzessuacrith{} is weakened enough for \Achsah{}, her fellow \resphain{} and their \maybehr{Umbra}{\umbrae} to have a fighting chance against her. 





\begin{comment}
  \subsection{Curwen dies}
\end{comment}
\new
\target{Curwen dies}
\Takestsha is not pleased to be thus thwarted. 
In the midst of the destruction, she reaches out and strikes back through the counterspell.
She grabs hold of Curwen's spells and twist them against him.
She kills Curwen.

He fought well and bravely.
He made a difference.
But he was just a mortal against an immortal, so he paid for that difference with his life. 











