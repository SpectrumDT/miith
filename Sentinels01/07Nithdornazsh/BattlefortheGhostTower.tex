\chapter{Battle for the Ghost Tower}
The Sentinels and Cabal battle for control of the Ghost Tower. 

Note that \maybehr{Immortals inside the Shroud}{the Shroud suppresses the immortals' powers}. 
I need to deal with that somehow. 

Read about: 
  \maybehr{Resphan}{\resphain},
  \maybehr{Dragon}{\dragons},
  \maybehr{Umbra}{\umbrae},
  \maybehr{Resphan equipment}{\resphan equipment},
  \maybehr{Weapons}{weapons},
  \maybehs{technology}. 
Read about \maybehs{Chaos magic}, and remember to invoke \Sethicus and \Tiamat. 

\Achsah retains command of her group of \resphain. 
The \resphain who come to help \Achsah \maybehr{Achsah's rank}{stand below \Achsah{} in rank} and must obey her commands. 
This galls them, for some of them are purebloods. 
But \Achsah is highly talented and experienced, more powerful than many purebloods.
(Some of the others are \thelyadeth or \gessurim, others \bezedeth.)

\maybehr{Ashenblood lesser immortality}{\Bezedeth do not possess True Immortality}, so if they die, they are gone for good. 
This means \Achsah must be very brave and convince her fellow \bezedeth to be likewise when they have to go up against \Nzessuacrith.
    
Maybe the \resphain wear \maybehr{Glass armour}{\armour made of glass or crystal}.

\Nzessuacrith attacks her foes with dark curses invoking the \xss. 
See the section on \maybehr{Magic visuals}{magic visuals}, especially \maybehr{Curses of destruction visuals}{curses of destruction}.





\section{Carzain attacks \Takestsha}
\target{Carzain fights Takestsha alone}
\Takestsha \maybehr{Takestsha retreats to heal}{has retreated from the battle to heal}. 
But Carzain is not done with her.
He fights his way through the Rungeran ranks using might and stealth.
He has left he Imetrians behind now. 
He is alone. 
This is his moment of glory. 

He tracks \Takestsha and attacks her. 
With the great sorcerous power he and she unleash (and the way \maybehr{Takestsha bleeds power}{\Takestsha bleeds arcane power}), no Rungeran soldier dares come anywhere near. 
They fight. 

\Takestsha uses energy claws. 

Fortunately, Carzain is \uber-powerful for a mortal. 
Read about \maybehs{Carzain's strength}. 
\Takestsha keeps underestimating him and slapping him with too little power, and he keeps climbing to his feet again. 

Carzain is overpowered, but he fights bravely. 
He was wounded already when he approached her (having taken wounds in the charge against the \ishrah) and sustains many more wounds in the battle, but by heroic willpower he keeps himself alive and fights on. 
He pushes himself to the utmost, and in his hour of need, on the brink of death, he manages to unlock some of the dark power that lies sleeping within him. 
He gets closer to his true self. 

Ramiel's dark, \draconic{} blood awakens, and the death-ravening black fury fills him. 
He fights with reckless abandon and a savage laugh on his lips, with the ferocity of a \dragon{}. 

Carzain does not understand why \Takestsha is so powerful.
He is willing to swear that her magic is \rethyactic in nature.
But she is too fast. 
Conventional wisdom has it that in close combat, a Vaimon will always defeat a \rethyax. 
The \rethyax's magic may be more powerful, but the Vaimon's magic is faster. 
But not her. 
She is fast as fuck. 

Finally she pushes him away and transforms. 
She breaks her \human form and begins to mutate and grow. 
She begins to return to her \draconian form.

This gives pause to Carzain. 
He starts to suspect he has gaped over more than he can handle. 
But he does not back away.
He keeps up his attack. 

She fights on while she is mutating into her true form. 
She badly wounds him. 
Then his \malach self begins to awaken. 
Somehow the savage battle triggers something in him.
It is the first time in all his Scion lives that Ramiel has faced a \dragon this close. 
He has many strong memories of fighting against \dragons, including \Nzessuacrith herself. 
Now they come back to him. 
The fact that his body is badly wounded helps. 
It is more traumatic, and it forces his desperate mind to dig deeper for reserves of power. 
He has to fight for his life, harder and more desperately than he has ever fought before. 
Besides, he is surrounded by immense power that rends the Shroud in tatters. 
This makes it easier for him to see through his own inner Shroud and access powers and memories that he did not know he had. 

And Carzain indeed finds new reserves of power. 
He manages to unleash sorcery beyond any \human. 
A trace of his \sathariah power that is awakening. 

This is his \quo{\maybehr{Carzain's Sephiroth epiphany}{Sephiroth epiphany}}. 
As part of his epiphany, Carzain sees visions of \Mystraacht.
Also read about \malachim, \carcers, Carzain, Vizicar and Ramiel.

\citeauthorbook[p.243]{RobertEHoward:KingsoftheNight}{Robert E. Howard}{%
  Kings of the Night%
}{
  The sun was sinking into the western sea; all the heather swam read like an ocean of blood.
  Etched in the dying sun, as he had first appeared, Kull stood, and then, like a mist lifting, a mighty vista opened behind the reeling king.
  Cormac's astounded eyes caught a fleeting gigantic glimpse of other climes and spheres\dash as if mirrored in summer clouds he saw, instead of the heather hills stretching away to the sea, a dim and mighty land of blue mountains and gleaming quiet lakes\dash the golden, purple and sapphirean spires and towering walls of a mighty city such as the earth has not known for many a drifting age.
}

\citebandsong{DarkEmpire:DistantTides}{Dark Empire}{A Soul Divided}{
  Since my birth, haunting visions have occurred to me. \\
  A strange power controlling my destiny. \\
  Burned in the fire, but my blackened soul has remained.\\
  My dark desire will infect this mortal plane.
  
  Lying here, left for dead. Bandaged and alone.\\
  A separate half I'll create to undermine their throne. \\
  Feel your aggression. The taste of my hate inside of you\\
  feeds my obsession, until the time you set me free. \\
  This was meant to be
  
  Remember me, your true self. Its who you must become. \\
  You and I, are one and the same, the Chosen One. \\
  Rock turn to rust. My minions rise and humanity\\
  all turns to dust. None will ever stop my insanity.
  
  Banish it to the void. Seal your fate at once. \\
  Let the light take you in. See it through. \\
  I am immortal. Even if I'm wiped away,\\
  I can't be stopped, no. My influence has spread.
  Your kind is dead
}

\citebandsong{DarkEmpire:HumanityDethroned}{Dark Empire}{Eyes of Defiance}{
  Shadows of the past are consuming all I see.\\
  Breaking through the darkness there's another chance for me.\\
  Once again inside me breathes the energy of life. \\
  Wash away my sins, look into my eyes.\\
  I will defy!
}

\lyricstitle{Draft excerpt from the chapter \quo{The Terror of \EreshKal}}{
  It was hideous, but also invigorating. 
  It was terrifying, but also intoxicating. 
  It was a myriad feelings at once. 
  Death and life, horror and joy. 
  It spoke to something deep within him. 
  It made him feel\prikker alive. 
  It awakened a hidden side of him. 
  He felt it as a revelation, bringing him one step closer to who he really was. 

  It had happened again. 
  One minute Carzain had stood there taking in the feeling of the sorcerous residue that hung over Gilwaed. 
  The next minute all sorts of images had flashed before his eyes: 
  Dark voids where blind, shapeless horrors roamed shrieking. 
  Lying dreaming in a sea of misty nothingness. 
  Rainbow-\coloured crystal castles. 
  Armies tens of thousands strong, bearing standards that had been familiar but which he could now no longer recall. 
  White, jewelled halls hundreds of yards across and pearly towers thousands of yards high. 
  A cave of twisted vines and bulbs. 
  Then the vast towers again, now blackened with ash and gore. 

  And a voice had spoken. 
  {\vizicar{Who am I?}} it had said. 
  {\vizicar{Who am I really?}}

  There had been an unwelcome feeling of exhilaration. 
  A rush, as if a strong, cold wind was blowing through him. 

  But he forces his attention back to the matter at hand. 
} 

He wounds \Takestsha a bit. 
She curses and draws back. 
\Takestsha blasts him again. 

This time he cannot get back up.
He is struck down to the ground, writhing in pain and helpless. 

Remember to have great mind-shattering revelations of cosmic magic when \Nzessuacrith assumes her true form. 
Compare to \cite{StephenMarkRainey:Signals}. 

\Takestsha is stunned a bit and draws back to think, and to complete her transformation. 
She feels the \sathariah power radiating from him. 
(Do not mention the word \quo{\sathariah}.) 
This surprises and startles her.
She realizes that he is more than a mere \human. 
He is a Scion. 
But in a way, this also reassures her.
It would be embarassing for her to live with knowing that a mere \human could cause her such trouble. 
Now that she knows he is a Scion. 
She understands better. 

While Carzain is down, \Takestsha completes her transformation.
She now stands before him in her full, terrible glory as \Nzessuacrith. 

She is just about to finish off this insolent mortal.
Then needles of shimmering energy lance into her.
She looks up.
It is \Achsah and her \resphan kin.
They have finally arrived at \Forclin to repel her.
\Nzessuacrith welcomes them.
She forgets about the high-powered \human and leaps into the air, eager to confront her hated foes. 

Carzain \maybehr{Ilcas rescues Carzain after fight with Takestsha}{falls unconscious and is rescued by Ilcas}. 





\section{\Nzessuacrith in \draconian form}
\Nzessuacrith{} has finally been forced to leave the body of \Takestsha{} and assume her true, \draconic{} form and enter the fray. 

\Nzessuacrith{} wears \maybehs{ward runes}. 
(Remember to read about \maybehs{ward runes}.)

Before entering combat, \Nzessuacrith{} casts a spell that makes her natural weapons poisonous. 

Read about \maybehr{Nzessuacrith}{\Nzessuacrith}, \maybehr{Achsah}{\Achsah}, \maybehr{Dragons}{\dragons}, \maybehr{Resphan}{\resphain} and \maybehr{Umbra}{\umbrae} before writing this section. 
And read some RPG books for inspiration on spells and weapons and magical items they might use. 

\Nzessuacrith{} is surprised that she is able to take \draconic{} form so deep in the Shroud. 
She realizes that her own \EreshKali spells have significantly weakened the Shroud (locally and temporarily, that is).
As she spells backfired, this effect could potentially have become even stronger and more out-of-control. 
She speculates that the Ghost Tower might be exterting a further destabilizing influence on the Shroud. 
But is she not quite convinced by this explanation. 
She fears the Shroud is \maybehs{unravelling}. 

\begin{prose}
  \tho{The barriers between the Realms are breakdown down.
    The great cosmic Seals are leaking.
    Like water seeping through holes in a rotted dam.
    What happens if\dash when\dash the river breaks through?
    A \thirdbanewar? 
    Can \Miith{} survive a \thirdbanewar?
    
    And what is causing it?
    Is it the weakening of the Heart?
    The conflict of the \matrices?
    What?}
\end{prose}

Have some mortals who are stricken with terror and awe at the sight of the \dragon{}. 
Perhaps Carzain and his party. 
They have heard \maybehs{myths} of \dragons, but their true power and majesty is downplayed in the Iquinian myths. 
The sight of an actual \dragon{} blows away all the myths and faerie tales. 

Delph dies in this final, climactic battle. 
But his rat lives.

Carzain sees the \maybehr{Umbra}{\umbrae} that \Achsah-tachi summon. 
He likens them to bats, but \maybehr{Umbra like bat}{different}. 
They also \maybehr{Umbra sounds}{hear the \umbrae{} howl}. 






\section{\Achsah meets the challenge}
The sight of \ps{\Nzessuacrith}{} unmasked has drawn \Achsah{} $100\%$ from \Malcur to the Ghost Tower. 
Charcoal's plan\dash aided by Carzain\dash has bought the Sentinels enough time to call in reinforcements and ultimately fight off the Sentinels. 

We see \Achsah{} at the Tower. 
Perhaps she is riding a terrible but splendid monstrous steed.

\lyricsbalsagoth{When Rides the Scion of the Storms}{
  I see him\prikker \\
  grim and noble astride his great winged steed, \\
  gleaming spear crackling in his grasp, \\
  beckoning me onwards to the next life\prikker \\
  to ever more slaughter and carnage\prikker \\
  Yes, adour and brooding spirit he is, \\
  and in his burning eyes I see \\
  a great secret which I must discover,\\ 
  a powerful mystery I alone must solve.
}

Does she actually wield a spear?

\begin{prose}
  \Achsah{} mocks \Nzessuacrith{}: 
  \ta{I had expected more. I had feared I would be facing \QuessanthIshnaruchaefir.} 
  
  \Nzessuacrith{} mocks \Achsah{} in turn: 
  \ta{I had expected a \ketheran. 
    Not some pathetic half-\human{} scum. 
    Without the stolen \draconic{} blood you \resphain{} are worthless. 
    You, \Achsah, are nothing but a swarthy \human{} with delusions of grandeur.} 
  
  \Achsah: \ta{A \human, am I? We will see about that.}
  
  They fight. 
  Then, a bit later: 
  
  \Achsah: 
  \ta{If you did your research, \Nzessuacrith, you would know that I am of \Merkyrah, and as such not not half \human{} but half \nephil.} 
  
  \Nzessuacrith: 
  \ta{Hah! 
  Feebly trying to defend the last shreds of your dignity? 
  Very well, I take back my last insult. 
  You, \Achsah{}, are nothing but a pathetic, swarthy \nephil{} with delusions of grandeur.} 
\end{prose}

Notice that \Nzessuacrith{} is proven right. 
\Achsah{} is soundly beaten and only prevails when Ramiel, a \sathariah, shows up to help her. 





\section{They battle}
Enter a great battle between \Nzessuacrith{} and \Achsah. 
This battle is long, hard, bloody, brutal and dirty. 

\Achsah{} draws up her full power, which is considerable. 
What she lacks of inborn gifts she makes up for in age and experience. 

\lyricslimbonicart{Beyond the Candles Burning}{
  I am a dark star rising on the raveous bleaky sky,\\
  a black diamond slunning so deep within the night.\\
  Maliciously I dwell in a bluish shaded beam\\
  with a stonecold heart into the core of my being.
}

Perhaps they fight in humanoid form first before going into their monstrous forms. 

\Achsah{} summons a bat- or Balrog-like monster to ride. 
This might or might not be an \maybehr{Umbra}{\umbra}. 
Compare to the monster that Durza summons in the movie \cite{Movie:Eragon} (not present in the book \cite{ChristopherPaolini:Eragon}). 

Describe the awesome forces of magic unleashed.

See the section on \maybehr{Magic visuals}{magic visuals}.

\Nzessuacrith{} is weakened, but still a badass motherfucker. 
She kills more than one \resphan{} (but not permanently). 

\Nzessuacrith{} does \emph{not} use the spell \word{\maybehs{khestni}}. 
She has nastier weapons at her disposal. 

\Nzessuacrith{} uses magic to enhance all of her moves. 
She growls words to power to punctuate every attack or parade. 
Unlike \Ishnaruchaefir. 

\Nzessuacrith summons monsters to do her bidding.

\citeauthorbook[p.344]{ClarkAshtonSmith:TheDarkEidolon}{Clark Ashton Smith}{%
  The Dark Eidolon%
}{
  Yea, the undying worms of fire and darkness have come forth like an army at thy summons, and the wings of nether genii have risen to occlude the sun when you called them.
}





\section{An \umbra escapes}
An \umbra{} escapes from the battlefield after its handler is killed. 
\Achsah{} reflects that it will probably go into the \Wylde{} and live off whatever mortals it can catch. 

There it will live out the rest of its days. 

The rest of its days\prikker how much is that? 
Are \umbrae{} immortal? 
\Achsah{} does not know. 

And do they reproduce in the \Wylde{}?
How do they reproduce? 
Do they reproduce at all? 
\Achsah{} does not know. 

The \banelords{} probably know, she reflects. 
But she has never been in a position to ask a \banelord{} a question. 
And she would not dare even if she could. 
She is not too proud to admit that the \banelords{} make her shit herself with fear. 
She remembered feeling their evil presence during the Incursion. 
And the presence of the dreaded \Voidbringer. 
The fear has not decreased in her memory, even after all these millennia. 
She still remembers it vividly. 

That is another thing the young \resphain{} do not understand. 
They do not fear the \banes{} as they should. 
The mortals have the right idea here, she thinks. 
Many mortals see their gods as terribly frightening, alien powers around which one must tread very carefully.

\begin{prose}
  \tho{We \resphain{} could learn from that.
    We like to think of ourselves as being on top of that ladder of power.
    But we are not.
    There are things out there far older, far darker and far more powerful than we.}
\end{prose}





\section{\Nzessuacrith flees}
\target{Ramiel scares Nzessuacrith}
\Nzessuacrith was badly wounded even before she assumed \draconian form.
She is fighting several powerful \resphain mounted on their terrible \umbrae. 
She is holding up, but it is a hard fight for her. 

\Achsah is skilled. 
Eventually her determination, bravery and good combat tactics win the day. 
\Nzessuacrith is overpowered forced to flee. 

\Nzessuacrith rationalizes it, telling herself that \Nithdornazsh must be about to rise. 
The \resphain cannot return to \Malcur in time to stop the ritual now. 
So she permits herself to flee from the field of battle. 

The \resphain are themselves badly wounded.
She has killed some of them (non-fatally).
They are in no condition to pursue. 





% \Nzessuacrith{} could defeat the \resphain, but \maybehr{Charcoal at the Ghost Tower}{Charcoal's spell antagonizes her}, and ultimately she loses and must retreat.
% 
% Now, Charcoal's magic alone would not be sufficient to make any difference, since \Nzessuacrith{} is a powerful \dragon. But Carzain/Vizicar is there. The \nieur{} ritual that Charcoal unleashes somehow causes the \sathariah{} within Carzain to stir and awaken a little bit. Acting instinctively, Ramiel adds his power to the ritual. 
% 
% Perhaps Ramiel senses that \Nzessuacrith{} is family. 
% She is, after all, kin to \Nexagglachel, the sire of the \satharioth. 
% 
% Now, Ramiel doesn't have much power available. 
% He is still in deep sleep. 
% But he makes his \emph{presence} known. 
% \Nzessuacrith{} feels the smell of a \sathariah, and it distracts and alarms her. 
% She had not counted on the innocuous Scion being a \sathariah. 
% It throws her off balance for a moment, and this is enough for \Achsah-tachi to wound her and drive her off. 





\section{Conclusion}
At the end of the day, the Tower is in Cabalist hands. 

The battle has wrought destruction of Godzilla-like proportions to the surrounding countryside. The Tower itself is untouched, because it is built with the technology of the ancients. \Forclin{} is laid in ruins, except for the castle and certain walls and towers, who are superhuman. 

\Nzessuacrith{} (\maybehr{Nzessuacrith likes beauty}{who likes beautiful things}) mourns the devastation wrought on the beautiful \Forclin. 

Now \Achsah{} and \Nzessuacrith{} know who Carzain is\dash to an extent, at least.
\Achsah{} is $100\%$ sure the \vertex{} is a \sathariah{} aligned with the \maybehs{Midnight Bat}. 
Which means it must be a Scion. 
Which means there are only two possibilities. 
She reports this to \Azraid, who \maybehr{Azraid learns of spike}{muses about it}. 

\Nzessuacrith{} and \Ishnaruchaefir{} are less certain about this. 
They have not studied the \vertex{} (or the theory of \malachim) as much as \Achsah{} has. 

Some of the Cabalists muse over how they underestimated Charcoal. 
They believed to have him figured out, but he still managed to thwart their plans. 

Meanwhile, in \Malcur, \maybehr{Ishnaruchaefir kills Teshrial}{\Ishnaruchaefir is about to kill \Teshrial}, and \maybehr{Nith'dornazsh rises}{\Nithdornazsh is rising}.





\section{Consequences for mortals}
Sethgal still lives. 
Much of \Forclin has been devastated, but all is not lost. 

It has gone worse for the Rungerans. 
Carzain had chased \Takestsha deep behind the Rungeran lines before she transformed.
So the battle of the immortals took place (initially) in the middle of the Rungeran army. 
In the process, the army took bad casualties, and its morale is broken. 
The remnants have scattered. 

After the immortals' battle, the Pelidorian irregulars finally came. 
Not all the tribes had answered the summons, but there were enough.
They arrive to the battlefield late, but they are quite effective when they get there. 
They fall upon the Rungerans from the rear, like the sneaky gits they are. 
They attacked the bewildered Rungerans' flank and were able to chase them away, for the moment at least.

The war with Runger is probably over. 

Carzain has vanished.
Maybe so have the Imetrians. 
Sethgal now has to try to rebuild. 





\section{Morgan Runger}
Morgan Runger sees his army panic and begin to scatter while the immortals battle. 
He realize the invasion is lost. 
He orders his remaining forces to retreat. 

He rides away in his howdah.
Awestruck he watches the destruction behind him while absently fondling the breasts of one of his naked concubines. 

Maybe Morgan's party is hit by a stray fireball and killed. 
Maybe he escapes back to Runger. 










