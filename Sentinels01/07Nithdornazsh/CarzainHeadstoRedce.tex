\chapter{To \Redce}





\section{Ilcas rescues Carzain}
\target{Ilcas rescues Carzain after fight with Takestsha}
After his fight with \Takestsha, Carzain falls unconscious and lies bleeding to death. 
But Ilcas Northstar has seen his heroic battle against the \dragon-sorceress. 
Ilcas fights his way through the panic and the destruction caused by the warring immortals. 
He reaches Carzain's unconscious body and hauls it to safety. 

Ilcas cannot heal Carzain himself. 
But he finds (or is found by) \Esmerel, who offers her help.
She heals Carzain's grievous wounds and broken limbs. 
She saves his life. 
It takes an awful toll on her.
She looks ten or twenty years older afterwards. 
But she does what she feels she has to do.
She suspects this man is something really special, and she will sacrifice much to secure him for \ClanRedcor.
Besides, now he owes her. 

\Esmerel convinces Ilcas to help her transport Carzain to \Redce. 
They set out while Carzain is still unconscious. 

Carzain later awakens.
He learns that \Esmerel has saved his life, at great cost to herself. 
This means he owes her.
In exchange she coerces him into promising to go with her to \Redce and help \ClanRedcor against a dark enemy. 
He does not want to serve the Redcor, but he lets her coerce him.
He lets her think he is cowed by guilt and obligation and will now serve them faithfully. 
But in truth, he has his own plans. 




\section{To \Redce}
\Esmerel{} wants to flee back to \Redce{} and tries to persuade Carzain to come with her, tempting him with romises of glory and greatness if he allies himself with the Redcor. 
Also, \Esmerel{} promises him that they can help him master his \maybehr{Kenosis}{\Kenosis}. 
He wants this, since in his current state he is an unstable madman and a danger to everyone. 
Also, he hopes if he allies with the Redcor, he can get his hands on \maybehr{Iolivine's notes}{\ps{\Iolivine} notes on Scions}, which \maybehr{Redcor bogarted Iolivine's notes}{the Redcor are bogarting}. 

Ilcas agrees to accompany them, because the Imetrians and Redcor want to temporarily join forces to drive back the Rissitics. Carzain agrees, partly influenced by Ilcas' decision. 

I need to stress the fact that Carzain does not agree to serve the Redcor. He is somehow tricked and coerced into coming with them, and thus can claim to be a prisoner of sorts in the Topaz \Chateau. 

How exactly does this work?

They slip out of \Forclin{} amid the chaos and make their way north to \Redce{}. 
But their way goes past the Ghost Tower, so Carzain becomes involved in the events there. 

Remember that the Redcor should refer a lot to their scripture and their historical heroes, such as Silqua and Rebecca Redcor. 

Remember to have a \quo{creepy lizard}/\quo{creepy \human} scene, where Carzain and Razor suspect one another.





\section{Carzain is like Sephiroth}
Remember that \maybehr{Carzain is Sephiroth}{Carzain is supposed to be like Sephiroth} from \cite{VideoGame:FinalFantasyVII}. 

At the end of \TwilightAngelRememberEmph, he has a \maybehr{Carzain's Sephiroth epiphany}{Sephiroth-style epiphany} and turns evil. 

The epiphany has to do with Curwen (\maybehr{Charcoal at the Ghost Tower}{his plan to use the Ghost Tower}) and \Takestsha.
Carzain comes into combat with \Takestsha and realizes that she is \Nzessuacrith. 
She uses dark magic against him. 
This is traumatic. 

This is the first battle of this scale he has fought in this life.
He has fought smaller battles as Carzain, and bigger ones as Vizicar.
After a battle, he is usually loath and tired of all the slaughter and bloodshed and glad to see it end. 

After this battle, Carzain realizes that he is not loath of the slaughter as usual.
Rather, he finds that he has relished it as never before. 
He realizes that his true nature is a dark angel of battle.
He feels he has taken a great step forward to finding his soul. 





\section{Carzain espies Morgan Runger}
On the battlefield, Carzain espies \maybehs{Morgan Runger} in the distance. 

\vizicar{%
  Feh. Kings who are not mages themselves are mere upstarts. Such a coward. He commands his sorcerers to work their dire magics, but dares not do it himself.}

Vizicar is quite the pro-mage nazi. 







\section{Carzain sees reapers on the battlefield}
\target{Carzain sees reapers near Forklin}
After the battle of \Forclin, Carzain sees some \quo{reapers} on the battlefield. They are \maybehs{Worm Cult reapers}, as well as \maybehr{Crows and ravens}{crow- or raven-like men}. 

They appear as unclear ghosts. They might be an apparition created in his mind from the fog, smoke, animal and corpses. He is in doubt: \tho{Am I seeing things?}

They pick up some dead. Carzain seems to see corpses\dash or their souls\dash rise to wander, limp or crawl away into the Beyond. 

Perhaps the Worm Cult reapers fight the Ravens. 





\section{Carzain and Vizicar talk}
Carzain and Vizicar talk. 
Vizicar theorizes that it was he who \maybehr{Vizicar drives Carzain to war}{subconsciously imparted to Carzain a craving to fight and seek glory}, thus causing him to go off to war. 





\section{Carzain-tachi encounter a \bane}
Carzain-tachi encounter a \bane. 
Ilcas Northstar holds up an Imetric holy symbol and recites a prayer/\maybehs{orison} to ward off evil and keep it at bay. 
It seems to work for a moment, but then the \bane{} advances again. 





\section{Ilcas kills prisoners}
There is a scene where Telcastora Ilcas, together with some Redcor, have taken some enemies prisoner. 
These are Pelidorian soldiers who have deserted and turned into bandits. 

Ilcas is about to kill them. 

\begin{prose}
  \Racel/\Esmerel: 
  \ta{No! Stop. If you kill them, you are no better a man that they.}
  
  Ilcas: 
  \ta{What are you talking about? Fuck that!} 
  He kills the men. 
  \ta{Of course I am better than they. 
    They were not only brigand scum, they were also traitors. 
    They were paid and armed by their kingdom and entrusted to protect their people from enemies. 
    They betrayed that trust and turned against their own people. 
    They were the worst kind of filth. 
    I am doing the world a favour.}
\end{prose}

Ilcas truly hates these men's guts. 
Traitors and malfeasants are pieces of shit. 

\begin{prose}
  \Esmerel: More whining. 
  
  Ilcas: 
  \ta{We are not under the jurisdiction of Redcor law!} 
  
  \Esmerel: 
  \ta{Nor are we under Imetric law!}
  
  Ilcas: 
  \ta{Look around you. 
    Pelidor has fallen. 
    This place is \Wylde{}. 
    We are under no law. 
    When protected by no law it falls upon each of us to act on our morality as best we can. 
    I did exactly that.}
\end{prose}

Carzain stands next to Ilcas and admires the manliness. 

Maybe Ilcas concludes with something like this: 

\begin{prose}
  Ilcas: 
  \ta{%
    Maybe we \scathae{} have a more rational outlook on death and killing than you \humans{} do. 
    I don't know.}
\end{prose}

Perhaps the last part is said in private to Carzain out of earshot of the Redcor. 

Ilcas notices that during the killing, Carzain stood by and looked on with a bloodthirsty grin. 
This disturbs Ilcas. 
He had good reasons for killing them, but he did not \emph{enjoy} it. 
And now that he thinks about it, Carzain has been overfond of killing for a long time. 
He was also too proud for his own good when he told of the mercenaries he had killed in Heropond last year. 
Ilcas fears Carzain is turning into a bloodthirsty maniac. 





\section{Ilcas wants to feed his sword}
In truth, Ilcas killed the prisoners not just for the sake of punishment, but also to feed his sword. 
\Telderain{} hungers for blood and soul energy (although it does not eat entire souls; it just leaves them somewhat drained), otherwise it goes crazy and tries to drive Ilcas crazy, too.  

But he doesn't tell this to the Redcor. 
He doesn't want them to know that he wields a black magic sword filled with blood-drinking \daemons. 





\section{Later Ilcas lets a hostage die}
Later, Ilcas lets a hostage die. 
Perhaps a child, or an egg. 
He knows that sacrificing the hostage to kill the villain is preferable to letting the villain get away. 
But some of the dumbass Redcor, such as \Racel, are unwilling to see that. 





\section{Ilcas cares for his \nycans}
At some point, Ilcas Northstar endangers the rest of his companions for the sake of one of the \nycans. 

Some of the Redcor are butthurt about this.

\begin{prose}
  \Esmerel: 
  \ta{%
    You would endanger all of our lives for the sake of an \emph{animal}?}
  
  Ilcas: 
  \ta{\Matron, maybe I owe it to you to make our position clear. 
    Let me give you a run-down of my loyalties, in descending order of priority. 
    One: the Imetrium. 
    Two: the Telcastora clan.
    Three: my wife and children.
    Four: the \nycans.
    Five: you people.}
  He looks at Curiet and jokingly adds.
  \ta{Six: Serpentin.}
  
  Curiet:
  \ta{What? Why I am at the bottom?
    All of them are Vaimons! They can defend themselves. 
    I am a defenseless civilian!}
  
  Ilcas looks at him.
  \ta{Point taken. 
    Correction. 
    Five: Serpentin. 
    Six: you Vaimons.}
\end{prose}









