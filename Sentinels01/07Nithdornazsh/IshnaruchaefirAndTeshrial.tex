\bookchapter{I Am the Shape of Things to Come}
\target{Ishnaruchaefir kills Teshrial}
\Ishnaruchaefir{} has taken \ps{\Teshrial} \quo{bait}. 
He arrives in the dead garden in answer to the challenge. 

\Teshrial{} has several trump cards:
\begin{itemize}
  \item Astrology. 
  \item The Achilles Heel. 
  \item The Shroud.
  \item \Noggyaleth.
  \item \NeoResphan metamorphosis. 
\end{itemize}
    
\Ishnaruchaefir{} uses the spell \word{\hs{khestni}} once during the fight. 
It hurts \Teshrial, but does not kill him. 

\Ishnaruchaefir{} wears \hs{ward runes}. 
(Remember to read about \hs{ward runes}.)

Maybe the \resphain wear \hr{Glass armour}{\armour made of glass or crystal}.

Read about: 
  \hr{Resphan}{\resphain},
  \hr{Dragon}{\dragons},
  \hr{Teshrial}{\Teshrial},
  \hr{Ishnaruchaefir}{\Ishnaruchaefir},
  \hr{Resphan martial arts}{\resphan martial arts} (\Teshrial{} walks the \hs{Path of Ice}),
  \hr{Resphan equipment}{\resphan equipment},
  \hr{Weapons}{weapons},
  \hs{technology}. 
Read about \hs{Chaos magic}, and remember to invoke \Sethicus and \Tiamat. 

\Ishnaruchaefir \hr{Ishnaruchaefir bleeds in Nadir}{bleeds and looks terrible} when he is in the Nadir. 
Every spell he casts causes more wounds to spring open on him\dash{}he pays for his magic with blood and pain.

\Ishnaruchaefir uses some magic, but not so much. 
His magical power is depleted during his Nadir; his physical strength is much more intact (although not quite intact). 
So he mostly fights using his physical strength. 

The Cabal plan is nearly complete. 
The Cabalist \Malcur venture, like \Secherdamon's plan, also coincides with \Ishnaruchaefir's Nadir. 
The \noggyaleth have grown numerous and large and powerful.

Throughout the section, have throwaway references to \hr{Mystic names}{mystic names and places}, like Shung. 




\subsubsection{\Menessiaraid looks on}
\Teshrial{} brings one spectator to the fight: 
His good friend \Menessiaraid. 

He only brings one spectator. 
Otherwise he fears \Ishnaruchaefir, suspecting an ambush, would refuse to fight. 
On the other hand, they both knew they could not fight in private. 
Rumours of this fight have been going all over the place among the dynasties. 
They would need to have \emph{someone} there to witness it. 

So \Menessiaraid{} is there alone, and he does not fight. 
He is a skilled telepath, and his head is crowded full of powerful \resphain{} who want to see the fight through his eyes. 

This means there are fewer \resphain{} left to keep an eye on \Forclin{} and \Malcur. 
Which is exactly what \Ishnaruchaefir{} and \Secherdamon{} want. 





\subsubsection{\ps{\Teshrial} perspective}
The battle should be seen from \ps{\Teshrial} perspective. 
He tries every means at his disposal to defeat \Ishnaruchaefir, and more than once he thinks he has succeeded, but \Ishnaruchaefir{} keeps getting back up. 
Describe \ps{\Teshrial} anguish when he dies. 





\subsubsection{They fight in humanoid form}
\Ishnaruchaefir{} arrives. 
He does not carry his glaive, \Rystessakhin, but only a pair of \skekrathuins. 

They meet in a fairly tightly Shrouded layer of the Realm. 
They are forced to fight wearing \quo{\hs{Masks}}. 
This means that \Teshrial{} has something of an upper hand. 
He is \hr{Immortals inside the Shroud}{not as weakened as \Ishnaruchaefir{} is by the Shroud}. 

First they fight in humanoid form. 
\Teshrial{} challenges \Ishnaruchaefir{} to come down and face him in humanoid form. 
\Ishnaruchaefir{} accepts. 

\Teshrial{} is comparatively young and inexperienced. 
He has never seen \Ishnaruchaefir{} in combat before and underestimates him. 
He mocks \Ishnaruchaefir, calling him a decayed, outdated relic of the far past. 
He believes he can out\manoeuvre the old, bitter, set-in-his-ways \shaeeroth.
He underestimates him. 
Otherwise, \Teshrial{} would have realized how fucked he was, and would have fled. 

Even so, the battle is non-trivial, even for \Ishnaruchaefir. 

They duke it out. 
\Teshrial{} loses. 

At the beginning of the fight \Ishnaruchaefir{} is cool and calm. 
He keeps up this \facade{} as long as he is fighting in humanoid form. 
But when he reverts to \draconian{} form to battle the \noggyaleth{}, and later Mutant-\Teshrial, he lets loose all his \draconian{} fury and lets his hatred against the \resphain{} guide him. 

\Ishnaruchaefir{} does not use so much magic. 
His physical prowess in humanoid form is enough to defeat \Teshrial{}. 
His body is wicked-psycho-tough and durable on its own. 
It is part of his tactic to rely on physical strength as long as possible, thus conserving his magical reserves for when he really needs them. 
He doesn't go all-out magic until mutant-\Teshrial{} shows up. 





\subsubsection{\Noggyaleth}
\Teshrial{} is losing. 
So he pulls out one of his trump cards: 
He summons his \noggyaleth. 

The \noggyaleth have been burrowing through the ground beneath \Malcur, corrupting it.
Read about how the \noggyaleth \hr{Noggyal corruption}{corrupt the planet}. 

Now the \noggyaleth, having long hidden beneath the earth, burst forth.

When \Criseis sees a \noggyal:
\citeauthorbook[p.147]{JohnGlasby:TheOldOne}{John Glasby}{The Old One}{
  Even in retrospect it is not possible to convey in words the nature of that monstrosity which squeezed its vast bulk through the gaping abyss.
  It held a hint of noxious plasticity, of writhing tentacles which changed their number and shape.
  But more than anything, I had the impression of gigantic size, that huge as that part of it looked where it almost completely blocked the opening, there was an infinitely greater bulk mercifully hidden from us.
  
  [\prikker]
  
  But I know there was nothing imagniary of halucinatory about the black, coiling tentacle that seized Dorman around the waist and bore him, kicking and screaming frantically, into the gaping, beaked maw which appeared as if from nowhere beneath that single glaring red eye!
}

\Ishnaruchaefir{} fights them. 
\Teshrial{} stands back and supports them with spells. 

Have a description of the monsters that swarm out to attack \Ishnaruchaefir. 

The \noggyaleth cause the very earth to rise and attack \Ishnaruchaefir.

\citeauthorbook[p.290]{DavidDrake:ThanCursetheDarkness}{David Drake}{%
  Than Curse the Darkness%
}{%
  Pulsing, rising, higher already than the giants of the forest ringing it, the fifty-foot-thick column of what had been earth dominated the night.
  A spear of false lightning jabbed and glanced off, freezing the chaos below for the eyes of any watchers. 
  From the base of the main neck had sprouted a ring of tendrils, ruddy and golden and glittering overall with inclusions of quartz. 
  They snaked among the combatants as flexible as silk; when they closed, they ground together like millstones and spattered blood a dozen yards up the sides of the central columns.
}

But \Ishnaruchaefir is prepared. 
He has prepared spells that breaks their hold over the earth. 
He forces them to come out in the flesh, without the protection of the earth, and fight him naked. 
They do. 

See the section on \hr{Magic visuals}{magic visuals}, especially \hr{Summoning magic visuals}{summoning}.

\Teshrial{} has been fooled into believing that his \noggyaleth{} are a secret trap which \Ishnaruchaefir{} doesn't suspect. 
He keeps them hidden and only summons them in the last minute when he really needs them. 
But unbeknownst to him, \Ishnaruchaefir{} has predicted this move. 
And the fight has been planned so it coincides with \Psyrex-tachi's summoning of \Nithdornazsh. 
This means that the \noggyaleth{} are disoriented and weakened and slow to answer \ps{\Teshrial} call. 
This gives \Ishnaruchaefir{} plenty of time to prepare for them and pick them off one by one when they arrive. 
They cannot lie in wait and ambush him as \Teshrial{} wanted them to. 

\Ishnaruchaefir{} assumes his true, \draconian{} form to fight them. 
Insert an epic description of the mighty \dragon{} here, a la \bandsong{Bal-Sagoth}{Black Dragons Soar Above the Mountain of Shadows}. 

\Teshrial{}, desperately intent on vanquishing \Ishnaruchaefir{}, summons reinforcements, bleeding the city dry of Cabal monsters. 
This is what \Ishnaruchaefir{} wants, because it leaves \Malcur vulnerable to \ps{\Psyrex}{} spell. 

When \Ishnaruchaefir{} fights, he has bound \daemons{} that whirl around his body and act as \armour and weapons. 
They are black, gray, white, purple and blood red. 
Compare to the battle between Fulgrim and Ferrus Manus in \authorbook{Graham McNeill}{Fulgrim}. 

The \noggyaleth grab on to \Ishnaruchaefir with their sucking mouths and grasping limbs/pseudopods.
They drag him down and engulf and swallow him.
Then they try to drown and crush and devour and digest him.

\lyricstitle{Draft excerpt from the chapter \quo{What Slithers Beneath}}{
  %The crushing, drowning sensation had passed, and Rian felt like he was thinking clearly again. Yet the scene before his eyes still seemed more like a hazy dream than reality. 
  Rian was not sure if he was awake or dreaming. 
  \tho{I hope I am dreaming.} 
  Some veil like dark smoke obscured the garden, hiding the black one from view. %, giving him only vague glimpses of the black one. 
  
  Then there came a monstrous sound. 
  
  Roaring. 
  
  Shrieking. 
  
  Groaning. 
  
  Rattling. 
  
  From not one throat, but many. If, indeed, things capable of making such sounds had anything as familiar as throats. 
  
  It came from deep beneath the earth. It came from all around him. It came from inside his head. But above all else, it came from the darkened garden. From the centre of the opaque haze. 
  
  He heard a voice, then. Deep, growling, inhuman. But definitely a voice, speaking powerful words in an alien tongue. 
  %It spoke 
  \tho{The voice of the sorcerer.} Rian could not remember how he knew this, but he though that un-\scathaese{} voice was somehow that of the dark-scaled sorcerer. 
  
  And a faint, distant howling, barely audible above the monstrous roaring. \tho{Or did I imagine it?}
  
  There! Within the cloud, a flash. The glinting of metal. 
  
  \tho{The scythe\prikker} 
  
  The groaning noises intensified. 
  
  The battle had begun. 
  
  %And there! A glipse of what he thought was the black one
  Terrible sounds of combat could be heard. 
  The swish of steel through the air. 
  The grinding of huge jaws. 
  The clash of massive bodies. 
  Grunts of pain. 
  And terrible words in no \human{} tongue. 
  
  And through the fog, through the blur that veiled everything, he saw glimpses. Flashes of light\prikker \tho{Lightning? Fire?} 
  The writhing of towering things\prikker \tho{A worm? Worms?}
  And black scales. Immense dark wings. Claws. Teeth. Horns. Blades. 
  The splattering of alien blood. 
  
  Rian did not know how long the battle went on. It could have been heartbeats, or a whole afternoon. \tho{Or a whole night, if I am dreaming.} 
  
  Then suddenly, a high shriek ripped through his ears like a knife. 
  
  The shriek became a rattle. 
  
  The thrashing of a huge body, or bodies.
  
  Seething, like boiling water. 
  
  Thumping. 
  
  Then silence. 
  
  A long moment passed. 
  
  \tho{%
    Is the battle over? 
    
    Who won?}
  
  The silence stretched. Again he had to struggle to stay awake. The world swam before his eyes. \tho{Must not doze off. Must not.}
  
  Then, movement. 
  
  Rian tried to focus his blurred eyes. At the edge of the garden, something. Something was emerging. 
  
  Black. With a great bladed weapon. 
  
  \tho{The sorcerer. He prevailed.}
}





\subsubsection{\Teshrial summons \malgryph}
When \Teshrial starts to conjure the \malgryph, \Ishnaruchaefir acts afraid and tries to stop the summoning.
\Teshrial realizes it will be more difficult than he thought to summon the \malgryph, so he uses his secret weapon and transforms into his monstrous form.
This gives pause to \Ishnaruchaefir, for it is an unexpected move. 
\Teshrial is able to push back \Ishnaruchaefir and overpower him for a while. 
Long enough to buy time for himself to summon his \malgryph.
(Make sure \Teshrial looks heroic and self-sacrificing here. He does not like turning into a monster, but he is noble and selfless and does it anyway. He thinks of his beloved and hopes she will forgive him for the way he has defiled his own body.)
But \Ishnaruchaefir laughs and casts his own spells.
He takes control of the \malgryph and turns it against \Teshrial and his worms.





\subsubsection{\Teshrial mutates}
The \noggyaleth{} are vanquished. 

\Teshrial{} now uses his last resort.
He uses \Azraid's spell and transforms into a \neoresphan. 

\Teshrial should not be seen so clearly in his new monstrous form.
We do not see most of it form his POV.
We only follow \Teshrial's POV when we need to hear about the spell and how strange and cosmic he feels of a sudden.
Mostly we follow \Menessiaraid's POV.
He can see \Teshrial only dimly, for the Shroud obscures him.
It is all very obscure and occult and confusing and mystical. 
Do not tell the reader everything \Menessiaraid sees.

\Teshrial absorbs the remains of the \noggyaleth into himself.
Such is the gluttonous vampiric nature of a \neoresphan. 
He merges with them, thus mutating into a colossal monster. The spell is dangerous and quite likely irreversible\dash it might drive him mad, prevent him from changing back and dooming him to spend the rest of his life as an insane abomination. 

We see \Teshrial{} from the inside. He knows the danger and fears it, but he is consumed by his eagerness to slay the mythical \Ishnaruchaefir{} and prove his worth, so he can advance in the hierarchy of the \ketherain. 

The mutation thing is the only one of \ps{\Teshrial} traps which \Ishnaruchaefir{} failed to anticipate. 
It startles him and throws him off balance for a moment. 

\Teshrial{} suggets that they drop their \quo{\hr{Masks}{masquerade}}. 
This surprises \Ishnaruchaefir{}. 
It is rare for a \resphan{} to be the first to suggest to \quo{drop the masquerade}. 

\Teshrial begins his mutation spell. 
It is a tremendous and powerful spell. 
It is mentally traumatic for him. 

\citeauthorbook[p.254--255]{HPLovecraft:TheBlackTomeofAlsophocus}{H. P. Lovecraft}{%
  The Black Tome of Alsophocus%
}{%
  Again I made the five concentric circles of fire on the floor, and standing in the innermost one, invoked powers beyond all imagining with an incantation so inconceivably terrible that my hands trembled as I made the mystic passes and symbols.
  The walls dissolved and the great black wind swept me away through dark gulfs of space and grey regions of matter.
  I \travelled faster than thought, past unlit planets and vistas of unknown realms which swirled and shifted across immensurable distances; the stars flashed by so rapidly that they appeared as gossamer-fine threads of brightness interlaced across the universe, minute shooting stars of brilliance shining against black aether that was darker than the fabled depths of Shung.
}

The mutation takes time. 
But \Ishnaruchaefir{} does not take advantage of this. 
He politely stands back and gives \Teshrial{} time and room to transform. 
He is curious and wants to see this new weapon in action and test its mettle against his own. 
Knowledge is power, and he is willing to take even foolish chances to gain knowledge. 
He knows that if he has underestimated \Teshrial, he may well perish. 
But he chances it. 

\Teshrial{} mutates. 
He becomes a \neoresphan and comes to \hr{Neo-Resphan appearance}{look like one}. 
Then, using his newly \hs{increased vampire powers}, he absorbs the bodies of the living and dead \noggyaleth in order to grow to huge size.

\Menessiaraid gazes upon \ps{\Teshrial} mutated form. 
\Menessiaraid is horrified by the sight, but also impressed.
He sees him as an awesome, magnificent angel, both terrible and beautiful.
A vision of the greatness, potential and future of the \resphan race.

\Teshrial says: 
\talk{I am the shape of things to come.}
This should be the chapter title.
It is also a phrase that \Menessiaraid will come to ponder greatly after the battle is over. 
What could \Teshrial have meant? 
Is this wondrous horror really the shape of things to come?

\lyricslimbonicart{Beyond the Candles Burning}{
  I am a dark star rising on the raveous bleaky sky,\\
  a black diamond slunning so deep within the night.
  Maliciously I dwell in a bluish shaded beam\\
  with a stonecold heart into the core of my being.
  
  Beyond the candles burning, beyond all minds eye.\\
  A vast emperic enigma awaits me as I die.\\
  In a graceful dance obscene, in a ring of fire,\\
  I obtain my majesty as flames caressing higher.
  
  Release my spirit, unleash my soul.\\
  From the darkest dungeon, oblivion calls.\\
  In the phallic halls of ancient forlorn\\
  a cold sanctuary in doom is born.
}

\lyricslimbonicart{Solace of the Shadows}{
  I require the solace of the shadows,\\
  so the night can be redeemed.\\
  As the winds of darkness whispers my name,\\
  a kiss of death I receive.
  Nocturnal enchanter, to thine art I yield.\\
  Within the candlelight a rapture is now revealed.
}

In his \neoresphan{} form, \Teshrial{} gains \hs{increased vampire powers}. 
He can now drain \ps{\Ishnaruchaefir} life force even from a distance. 
\Ishnaruchaefir{} realizes this and becomes more careful. 
He figures out that \Teshrial{} now also stands a better chance of absorbing his soul if he should win. 

\Ishnaruchaefir{} realizes that \Teshrial{} \quo{knows} about his Achilles Heel. 
So he pretends to fear this and makes certain to guard the Achilles Heel, to goad \Teshrial{} on. 

\Teshrial{} fights bravely and savagely. 
A battle of Godzilla-like proportions ensues. 
Perhaps \ps{\Teshrial} mutated form resembles the monster Orga from the movie \cite{Movie:Godzilla2000}. 

\target{Ishnaruchaefir impaled by spines}
\Ishnaruchaefir{} has his body impaled by two or three huge spear-like spines. 
One of his hearts is impaled, but \hr{Dragons have three hearts}{\dragons{} have three hearts}, so he can bear it. 

He can fight on fine despite the pain. 
But he doesn't play stoic. 
He growls and pants with pain of his many wounds. 
\Teshrial{} thinks he is finished, so he gets overconfident. 
He steps closer to finish off \Ishnaruchaefir{} and consume his soul to gain his power (he knows he can, with his increased vampire powers). 

\tho{%
  This will make me as great as a \sathariah. 
  Nay, greater!
  With this power I will be greater than even \Azraid!
  All glory will be mine!
  All \resviel{} will be mine!}

But this was what \Ishnaruchaefir{} wanted him to do. 
He is not dead, and now he leaps up and kills \Teshrial. 





\subsubsection{\Teshrial dies}
When \Ishnaruchaefir is just about to kill \Teshrial, he tells him:

\begin{prose}
  \ta{So you sought to use the \malgryph against me, did you? 
    Too late. 
    That would have worked five thousand years ago. 
    But I have grown stronger since then. 
    I have overcome many of my old vulnerabilities.}
\end{prose}

\Menessiaraid hears the above. 
That is deliberate from \ps{\Ishnaruchaefir} side.
He does not want to arouse suspicion. 
He does not want anyone to suspect that the relevant \WanderersInDarknessEmph passages are fakes that he planted. 
So he uses the above as a cover story. 

Have a sad scene from \ps{\Teshrial} POV as he dies. 
He thinks of his beloved, of \hr{Teshrial's date}{the amazing sex she promised him} and of the life they could have had together. 





\subsubsection{\Ishnaruchaefir and \Criseis after the battle}
After the battle, \Ishnaruchaefir heads off to \Malcur.

\Ishnaruchaefir admits to \Criseis that he is displeased with the fact that the \resphain have now learned of his Nadir and how to map it.
\Urizeth still lives, after all.
And she is not fool.
She has probably taken measures to ensure that her discoveries will live on even if she is destroyed. 

But it it of no matter. 
The \resphain were bound to find out sooner or later. 
Now that the \thirdbanewar is looming, \Ishnaruchaefir will have to take a much more active role than he has done previously. 
He could not hope to keep his Nadir cycle secret forever. 
He has taken a great chance and gone into combat during his Nadir this once.
It will not happen again. 

\Ishnaruchaefir confesses in private to \Criseis that the part about \Ishnaruchaefir's \quo{overcoming his old vulnerabilities} was a lie. 
It is \Criseis who pieces together the story and realizes that \Ishnaruchaefir has planted the fake \WanderersInDarknessEmph verses and thus masterminded \Teshrial's quest against him. 
It is she who tells the reader this.
She asks her master if what she suspects is true. 
He refuses to comment, just smiles to himself.





\subsubsection{\Menessiaraid after the battle}
\Menessiaraid goes away. 
He is very sad that his friend has died, and horrified to witness the power and cunning of the Destroyer. 
But he is also sort of hopeful. 
If \Teshrial can turn into such a magnificent monster, then it bodes well for the potential of the \resphan race. 
He goes away awestruck at his friend's courage and greatness, and with high hopes for the future and the Quest for Perfection.









