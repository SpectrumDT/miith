
\part{\Ophidians}























\chapter{\Dragons: Ancients}















\section{\ApepNesthra}
\target{Apep-Nesthra}
\index{\ApepNesthra}
A \dragon, the consort of \Tiamat{} and one of the \firstgendragons. 















\section[Tyarith Xzai-Rasshana]{\TyarithXserasshana}
\target{Tiamat}
\index{\TyarithXserasshana}
\TyarithXserasshana{} was a lover of \Sethicus and one the very first \dragons. 

Her father was \hr{Hesod-Nerga}{\HesodNerga}. 







\subsection{Name}
Her taken name \quo{\Kserasshana} means \quo{\xs-like}. 
She took it after she became the founder of the \dzraicchenosses. 

She founded the tradition of \hr{Draconic names}{\quo{taken names}}, which the \dragons{} kept using for many thousands of years. 





\subsection{Physique}
\target{Tiamat's true form}
Among the \dragons{} she is depicted as a \dragon{} with multiple heads. 





\subsection{Arsenal}
\target{Tiamat's power}
Physically and metaphysically \Kserasshana{} was no mightier than \Secherdamon{} or \Ishnaruchaefir{} became (at Carzain's time). 
Her magic was strong, but also primitive and \naive. 
The idea that \Kserasshana{} was super-duper-\uber-powerful is a romanticized myth. 






\subsection{History}
\subsubsection{\Xserasshana{} and the \firstgendragons}
%\Tiamat{} may have left \Miith{} behind permanently and forgotten about it, preferring to explore and conquer the infinite Cosmos and Chaos.

%Or maybe she sleeps for millennia, 
\Tiamat{} now sleeps for millennia, 
because that is her nature as a Dreaming Queen of the \ophidians{}. She intends to awaken again some time, thousands of years in the future. But some \dragons{} fear that the \banes{} will reach the Heart of \Miith{} in her absence and gain the power to prevail, destroying even \Tiamat{} and her \firstgendragons. 

They attempt to awaken her.

Eventually, \Tiamat{} half-stirs and agrees to provide her spawn with some power with which to oppose the enemy. 





\subsubsection{Mythical status}
\Tiamat{} is a mythical, divine figure. Like the Shadow-King from Bal-Sagoth. She has a number of artifacts that are now highly sought-after relics. 

\lyricsbalsagoth{The Obsidian Crown Unbound}{
  [The Wizards of Vyrgothia:]\\
  Darkly bejeweled circlet of night, \\
  Crown of the Elder King,\\
  Unfettered at last the Trinity of Might, \\
  the Sceptre, the Sword and the Ring!
}









\subsection{Personality}
\target{Tiamat's personality}
\Kserasshana{} \hr{Tiamat's power}{was not as \uber-powerful as some think}. 
But she had one great asset: 
\Kserasshana{} was a great leader. 
She managed to unite the \dzraicchenoss{} people under her banner, a feat which no other leader before or after her ever accomplished. 
At the end of the \firstbanewar, she even led her \firstgendragons{} in a suicide attack to create the \hr{Crystal Sphere}{\CrystalSphere}. 
\Kserasshana{} was cruel, but also noble. 
She sacrificed herself for her people. 















\section{\VardredSethicus}
\target{Sethicus}
\index{\Sethicus}
An \ophidian. 









\subsection{History}





\subsubsection{Youth}
\Sethicus underwent some horribly disturbing revelations in his youth.
They led him on a path of mysticism.
He began to romance with the \xss more and more. 

Compare him to Sephiroth from the game \cite{VideoGame:FinalFantasyVII}.

\lyricstitle{
  \href
    {http://finalfantasy.wikia.com/wiki/Final_Fantasy_VII}
    {Final Fantasy Wiki: Final Fantasy VII}
}{
  Five years before the beginning of the game, Cloud and Sephiroth were sent to Cloud's hometown of Nibelheim to investigate the Mako Reactor there. Inside, Sephiroth found Jenova, a creature Shinra mistook as an Ancient and whom had been called Sephiroth's mother. Sephiroth begins to look deeper into his past and the Jenova Project from which he was born. It was led by Professor Gast and the deranged Professor Hojo. What he finds drives him insane. Believing himself to be the last Ancient, Sephiroth begins to take revenge on humanity by burning Nibelheim to the ground. Lost in the fires is Cloud's mother and Tifa's father. Cloud runs up to confront Sephiroth, but his recollection fails him before he can reach the end of the story.
}





\subsubsection{Rebellion}
He became an \ophidian{} leader with controversial ideas. 
He practiced black \xsic{} magic and waged wars. 

Compare him to Urizen from \authorbook{William Blake}{The Book of Urizen} and other works. 

\Sethicus criticized the \ophidian{} society. 

\lyricsbs{Emperor}{In the Wordless Chamber}{
  In the wordless chamber\\
  they feared death desperately.\\
  Thus they clustered to the fruits of the earth,\\
  craving diversion,\\
  as if to avoid knowing why.
  
  In the wordless chamber\\
  they feared life desperately.\\
  Thus they proclaimed any given truth\\
  and swallowed,\\
  as if to justify their fear.
}





\subsubsection{Durance}
His enemies united against him and conquered him. 
But \hr{Draconic immortality}{he was immortal as a \dragon}. 
So they \hr{Sethicus imprisoned}{imprisoned him}. 
He spent millennia in \hs{Durance}. 





\subsubsection{Followers}
Even in his \hs{Durance}, \Sethicus had followers.
Some \ophidians \hr{Ophidians follow Sethicus under Durance}{carried on his religion}. 





\subsubsection{Death and transcendence}
\target{Sethicus became a god}
\target{Sethicus becomes a god}
\Sethicus \hr{Sethicus dies}{died in the \firstbanewar}. 

But unbeknownst to most, a remnant of \Sethicus's powerful soul that lingered on as an undead, quasi-conscious presence.
When \Sethicus died, his occult mastery was so great and his will so powerful that his soul did not perish, but transcended from \Miith and became a disembodied god.
His spirit lived on, albeit in a shattered, weakened state.
According to legend, he went on to ride with the \xss in eternity.
He could still be contacted to some extent via his mummified body.
His wraith maintained some fetters to the body.

A bit like the Emperor of Mankind in \cite{RPG:Warhammer40000}.

After his permanent death, \ps{\Sethicus} soul lived on. 
He transcended to a higher spiritual world and became a higher being, perhaps one like the \xss. 
He was the wisest of all \dragons and knew much Gnosis that no other \dragon had attained, so even when he had no more power in \Miith, he was still worshipped as a spiritual guide. 

This was because of his great spiritual mastery, in accordance with \hr{Sethican philosophy}{\Sethican philosophy}. 

See also the section on \hr{Draconic immortality}{\draconic immortality}. 





\subsubsection{Mythical archetype}
\target{Sethicus as archetype}
In \draconian metaphysics, \Sethicus had a status as a mythical primogenitor. 

He is featured heavily in \WanderersInDarknessEmph as a supreme founder god, from whom the three \quo{wanderers} derive their natures and powers. 

Compare to Adam Kadmon in \Cabbalist theory (whose body contains the ten \sephiroth) and Albion in William Blake's mythology (the primal man of whom the Four Zoas are mere fragments). 

\lyricswikipedia{Adam_Kadmon}{Adam Kadmon}{
  The conception of Adam Kadmon becomes an important factor in the later Kabbalah of Luria. 
  Adam Kadmon is with him no longer the concentrated manifestation of the Sefirot, but a mediator between the \emph{En-Soph} ("Infinite") and the \emph{Sephiroth}. 
  The \emph{En-Soph}, according to Luria, is so utterly incomprehensible that the older Kabbalistic doctrine of the manifestation of the \emph{En-Soph} in the \emph{Sephiroth} must be abandoned. 
  Hence he teaches that only the Adam Kadmon, who arose in the way of self-limitation by the \emph{En-Soph}, can be said to manifest himself in the \emph{Sephiroth}. 
}









\subsection{Name}
\quo{\Vardred} was his egg-name. 
\quo{\Sethicus} was his manhood name. 
He also had a \shaeeroth name and probably some more.

Later, \hr{Vizsherioch}{\Vizsherioch} \hr{Vizsherioch takes the name Sethicus}{took the name \quo{\Sethicus}} after him. 

To the \Ortaicans, \Sethicus was known as the \taortha \hr{Settras}{\Settras}.









\subsection{Personality}





\subsubsection{Innovator}
\Sethicus himself was a brilliant scientist and \hr{Sethicus brought innovation}{made many exciting new discoveries} in various fields. 
He sped up the otherwise slow \ophidian society. 









\subsection{Philosophy}
\target{Sethican philosophy}
Where most \ophidians were \hr{Ophidian philosophy}{atheistic and nihilistic}, \Sethicus was a mystic and founded a religion. 





\subsubsection{Gods}
\Sethicus believed:
\begin{itemize}
  \item 
    That there existed a number of great gods who were immanent parts of the universe and thus worthy of idolization.
  \item 
    That there existed a mystic spiritual world connected to those gods.
  \item
    That great insight, happiness and power could be found by exploring said mystical world.
\end{itemize}

Some of \ps{\Sethicus} great \hs{cosmic gods} were entities well-known to the \ophidians.
Other \ophidians accepted them as existing and powerful beings but did not believe in their immanent, spiritual aspect. 

These cosmic gods were not \xss. 
\Sethicus knew that they were far older and mightier and \quo{higher} than the \xss. 
But he believed the \xss were a wiser and more spiritually advanced race than the \ophidians and that much could be learned from them. 
\Sethicus approached the \xss as mentors and patrons on his quest towards a higher goal. 
The \xss were not themselves the end goal. 





\subsubsection{Planes}
\target{Sethican planes}
\Sethicus believed that the world was composed of a number of parallel \quo{planes}, each plane deeper than the next. 
The planes (arranged with the deepest plane first) were:

\begin{description}
  \item[\DaathKurZulNathla] was the plane of primal chaos. 
    It was blind, mindless, formless force of change, creation and destruction.
    It was the deepest plane which all others were built upon and emanated from. 
    
    Compare it to Azathoth from the Cthulhu Mythos. 
    
    \DaathKurZulNathla was also a plane where conflicting forces collided.
    \Sethicus believed that such conflict between opposing forces was the source of all life and motion.
    He saw sexual reproduction (where opposite genders collide to create new life) as one of the many manifestations of this phenomenon. 
    
    \Sethicus considered the \noggyal \hs{mother-mass} to be an entity of \DaathKurZulNathla, insubstantial and creative/destructive as it was. 
  
  \item[\Osserylloch] was the plane of the basic emotions or motivations, including fear, hunger, anger and lust. 
    
  \item[\Barbeloth] was the plane of materia. 
  
  \item[\YothUnXachtyon] was the plane of consciousness, intelligence. 
\end{description}

Note that in \Sethicus's world view, emotion (and thus spirit) preceded matter. 
This conflicted with traditional \hr{Ophidian philosophy}{\ophidian philosophy}. 
The \ophidians believed that life was random, mechanical, diverse, unrelated to each other, without any built-in soul. 
\Sethicus believed that all life sprang from a single uniform spritual source, namely the primal chaos. 

\Sethicus's theory inspired the \hr{Ortaican planes of existence}{\Ortaican theory of planes of existence}.






\subsubsection{\Ophidians}
\Sethicus suspected that \hr{Ophidians related to XS}{the \ophidians were related the \xss}. 





\subsubsection{Personal development}
\Sethicus believed in personal spiritual development. 
His goal was not simply magical and worldly power for their own sake, but as stepping stones toward reaching a higher state of existence. 

The \dragons took a great leap in their development when, with the help of \KhothSell, they \hr{Draconic immortality}{gained their True Immortality}. 
\Sethicus himself had achieved so great a spiritual mastery that his soul was able to survive the permanent destruction of his body and \Sethicus \hr{Sethicus becomes a god}{become a disembodied god}. 






\subsubsection{Sound}
\target{Sethican sound mysticism}
\Sethicus had much mysticism regarding sound and words. 
He believed that words (spoken with will and intelligence behind them) possessed power that reached down into the \hr{Sethican planes}{deep planes}. 

Therefore \hs{Chaos magic} had such an emphasis on spoken incantations.

\Sethicus was one of those who developed/discovered the \hr{True Draconic}{\TrueDraconic} tongue. 






\subsubsection{\Voyagers}
\Sethicus had a low opinion of the \voyagers.
He saw them as lower creatures, usurpers who foolishly challenged the superior \xss (his adored masters and mentors) and tried to take their place.
The \voyagers believed themselves great enough to create new life, but it led to disaster and their downfall (in the form of the \hr{Noggyal}{\noggyaleth} and \hr{Bane}{\banes} which rebelled against the \voyagers). 

It is debatable how true \Sethicus's view was.
The \voyagers were a very powerful and highly developed civilization, and they did successfully \hr{Voyagers challenge XS}{challenge the \xss}. 























\chapter{\Dragons: \Shaeeroth}















\section{\IrocasSecherdamon}
\target{HriistD}
\target{Secherdamon}
\index{\Secherdamon}
\index{\IrocasSecherdamon}
\IrocasSecherdamon{} was one of the \shaeeroths, the youngest of the three sons of {\Tiamat} and the brother of \Ishnaruchaefir and \Nexagglachel. 
His father was \hr{Apep-Nesthra}{\ApepNesthra}.

He always looked up to \Nexagglachel, and after his brother's death \Secherdamon{} assumed his role as \dragonking. He is one of the leaders of the Sentinels and one of the most influential \dragons{} in the world.

One of his secret identities was \HriistN. 
In this guise he led the Rissitic Empire of Durcac. 

In a sense, \Secherdamon{} was a planner, schemer and creator. 









\subsection{Arsenal}





\subsubsection{Cannot tread on \Miith}
To gain all his power, \Secherdamon{} has absorbed too much \xzaishannic{} power. 
He has been warped into \xzaishann-like being. 
Now he is affected by the \hr{XS slumber}{sleepiness of the \xss} and cannot tread the world of the Shroud. Unlike \Ishnaruchaefir, who can still travel freely.

A vital part of his master plan is to open the path so that he himself may enter the world of the Shroud. \hr{Secherdamon wants Nithdornazsh}{\Nithdornazsh{} is a stepping-stone} on this path.

Soon, he will enter \Miith{} in all his glory.

\lyricsbalsagoth{
  Behold, the Armies of War Descend Screaming From the Heavens
}{
  [THE DISCIPLES OF ZAKUMAKURA:]\\
  Since before mankind hurled himself squamously from the sea we have awaited the awakening of great Zakumakura!\\
  Now... the Dragon-King shall at last rise to claim his earthly throne!\\
  Cast your gaze to the firmament and know fear, for His forces fill the sky!\\
  Behold, the armies of war descend screaming from the heavens!
}





\subsubsection{Happiness}
\Secherdamon{} knew some magic that could make him forever happy by flooding his brain with super-charged pleasure. 
And it would \emph{work}, with no nasty side effects. 

But he knew that if he started doing it, there was a chance he would keep doing it and neglect his quest and leave \Miith{} to its fate, which is something he would not do. 
So he sacrifices happiness for the sake of his war. 

Perhaps he regained this happiness \hr{Secherdamon dies}{when he died}. 





\subsubsection{Power}
At the time of the \hs{Unravelling}, \Secherdamon{} was one of the mightiest \dragons{} alive. 
In many respects he was more powerful than his brother, \Ishnaruchaefir. 
\Ishnaruchaefir{} might be able to best him in a duel, but on a larger scale \Secherdamon{} had nastier weapons at his disposal and far more political influence (although \hr{Ishnaruchaefir and the Sentinels}{\Ishnaruchaefir{} had more contacts than he let on}). 

In fact, \Secherdamon{} eventually gathered much more political power than \Nexagglachel{} ever did. 
See, in \ps{\Nexagglachel} time the \hr{Dragons disorganized}{\dragons{} were disorganized} because they had nothing to unite \emph{for}. 
But after the \secondbanewar{} they found a common, external enemy in the \resphain.
This helped \Secherdamon{} unite them. 
They had something to unite \emph{against}. 









\subsection{History}





\subsubsection{The first born \dragon}
\Secherdamon was born the son of \Tiamat and \ApepNesthra. 
He was the first \dragon ever to be born as a \dragon.
All older \dragons had been born as \ophidians and turned into \dragons. 
As such, \Secherdamon considered himself a special \quo{pureblood}. 





\subsubsection{Seeking Gnosis}
\target{Secherdamon seeks Gnosis}
Already in his youth, before the \secondbanewar, \Secherdamon{} was a scientist who sought for Gnosis. 
He sought to understand the Aenigmata of the \xss. 
Then \hr{Ishnaruchaefir steals Secherdamon's research}{\Ishnaruchaefir{} stole his research}.

\target{Secherdamon becomes better when denied Gnosis}
This made him bitter. 
He felt betrayed. 
He had \emph{earned} that Gnosis much more than \Ishnaruchaefir{} had.
\Secherdamon{} had worked hard on it his entire life. 
\Ishnaruchaefir{} had emotionally blackmailed him into handing over all his centuries of research, and then stabbed him in the back, refusing to share the fruits of their joint labour. 

That was one of the reasons why he hated \Ishnaruchaefir{} so much. 

Later he finally \hr{Secherdamon gains Gnosis}{gained the Gnosis he sought}. 





\subsubsection{Early history}
Originally, \HriistD{} was only a weak \vertex, living in the shadow of his elder and more capable brothers. 

After the \hr{Second Banewar}{\secondbanewar} he was too weak to participate in the \hr{Shrouding}{\SecondShrouding}, and it pained him. 





\subsubsection{Becomes a \shaeeroth}
After the \Shrouding, \Secherdamon \hr{Secherdamon becomes Shaeeroth}{became a \shaeeroth} and \hr{Secherdamon takes the name Veldraxx}{took \quo{\Veldraxx}} as his \hr{Shaeeroth name}{\shaeeroth name}.





\subsubsection{Takes the name \Nexagglachel}
\target{Secherdamon takes the name Nexagglachel}
In the process of developing his new \hr{Rissitic magic}{Rissitic magic theory}, \Secherdamon made many great discoveries and achieved much Gnosis.
Eventually, he had so much power that he could take another name. 

He took the name {\quo{\Nexagglachel}} (after his brother \hr{Nexagglachel}{\RaemythNexagglachel}, whom he admired). 

It was at the same time that he took up the mantle of \quo{\RissitNechsain}, \hr{Rissit is a saviour}{the saviour of the \Ortaican people}. 





\subsubsection{Gained his Gnosis}
\target{Secherdamon gains Gnosis}
\Secherdamon{} spent his entire life \hr{Secherdamon seeks Gnosis}{seeking the Gnosis of the \xss}. 
For thousands of years he laboured without result. 
He blamed \Ishnaruchaefir{} and his Shrouding for that. 
The Shroud made it much harder to do research. 
\Secherdamon{} could feel himself growing stupid as the Shroud closed in, like a noose tightening around his neck. 
He \hr{Secherdamon hates the Shroud}{hated the Shroud}. 

It was not until after \hr{Vizsherioch}{\Vizsherioch} was born that they, working together, finally found that Gnosis. 
\Secherdamon{} knew, though, that \Vizsherioch{} saw deeper and realized more far-reaching Gnoses, insights that he could not communicate to his father. 
\Secherdamon{} was saddened at the thought that he might never learn this Gnosis. 





\subsubsection{Breaking the Shroud}
\Secherdamon{} had his own plan to break out of the Shroud. 

It was the reason why \hr{Rissitic creativity}{the Rissitics could be such free-thinkers}. 

When at last he \hr{Secherdamon gains Gnosis}{found his Gnosis}, he was (almost) ready to invoke \hr{Naath-Kur-Ramalech}{\NaathKurRamalech} and destroy the Shroud. 





\subsubsection{The Dagger}
\target{Dagger}
\index{Dagger, the}%
\ps{\Secherdamon} goal in this whole Shroud-drilling business was not so much to allow his Rissitics to be creative. 
That was just a nice bonus.
His real objective was to shape a powerful \vertex{} that would be able to cut through all obstacles, pierce the Shroud and slash open the way for \xsic{} domination. 

In the beginning, \Secherdamon{} thought he himself was going to become the Dagger. 
Later he decided this was a blind alley, and began creating \Vizsherioch, intending for him to become the Dagger. 

Early on, \Secherdamon{} made sure to prepare a place in his \matrix{} for the Dagger. 
But for a long time this slot stood empty, because no one was powerful enough to claim the title. 

The resurrection of \Nithdornazsh{} was \hr{Vizsherioch and Nithdornazsh}{a vital step in the forging of the Dagger}. 

Finally, \hr{Vizsherioch becomes the Dagger}{\Vizsherioch{} became the Dagger}. 









\subsection{Names}
\Secherdamon's full name was \Irocas \Veldraxx \Nexagglachel \Secherdamon. 

\begin{itemize}
  \item 
    When he \hr{Secherdamon becomes Shaeeroth}{became a \shaeeroth}, he \hr{Secherdamon takes the name Veldraxx}{took \quo{\Veldraxx}} as his \hr{Shaeeroth name}{\shaeeroth name}. 
    
  \item 
    Later he \hr{Secherdamon takes the name Nexagglachel}{took the name \quo{\Nexagglachel}} (after his brother \hr{Nexagglachel}{\RaemythNexagglachel}, whom he admired). 
\end{itemize}








\subsection{Personality}





\subsubsection{Behaviour and titles}
\Secherdamon{} might use the royal \quo{we} (\emph{pluralis majestatis}). 
Were he to speak Japanese he might use \emph{ore-sama} or a similar pronoun.

His servitors, such as \LocarPsyrex, call him \quo{Exalted Lord}. 





\subsubsection{Generosity}
\target{Secherdamon's reputation}
\Secherdamon{} is known as a generous benefactor to those who are loyal to him. 
He does care about his servants and allies. 

\Ishnaruchaefir{} has \hr{Ishnaruchaefir's reputation}{a different reputation}. 





\subsubsection{Leadership}
\Secherdamon{} tries to be a heroic and self-sacrificing leader, like \TyarithXserasshana{} and \Nexagglachel. 
A pillar of strength and a guiding beacon for his people. 
He does not always succeed, but he tries his best. 

Compare to Anomander Rake in \cite{StevenErikson:TolltheHounds} especially. 

\Secherdamon{} has fought hard and long to reach the position where he is. 

\citebandsong{Ihsahn:TheAdversary}{Ihsahn}{%
  And He Shall Walk In Empty Places
}{
  \quo{
    Remember this, you others.\\
    The fire and the fury,\\
    the strength and defiance,\\
    this you admire, this you desire.\\
    I had to win them for myself.}
}

\TyarithXserasshana{} was his great idol. 
He admires her passion, willpower and ruthlessness. 

\citebandsong{Ihsahn:TheAdversary}{Ihsahn}{%
  And He Shall Walk In Empty Places
}{
  In remembrance of the adversary\\
  I conjure up the lion will:\\
  Hungered Violent Solitary Godless.
}

He will reach his goal, no matter the cost.

\citebandsong{Ihsahn:TheAdversary}{Ihsahn}{%
  And He Shall Walk In Empty Places
}{
  And he shall walk in empty places,\\
  with a claim on destiny and self at hand.\\
  An endless journey towards the rising sun.
}





\subsubsection{Megalomania}
\target{Secherdamon's megalomania}
\index{stewardship}
\Secherdamon{} sees himself as \Miith{}'s saviour and protector against the wicked \banes. He stood up and took upon himself the mantle of the planet's steward when no one else was willing and worthy of it. 

In his view, \resphain{} and \humans{} are a disease, an infestation which must be destroyed. 

\lyricsbalsagoth{%
  In the Raven-Haunted Forests of Darkenhold, Where Shadows Reign and the Hues of Sunlight Never Dance
}{
  I am the immortal King of the Deep Woods,\\
  servitor of the Old [\Firstgendragons].
}

\index{stewardship}
He has inherited and gained much of his power from the \firstgendragons. He sees it as a divine gift and a confirmation of his status as the \Miith{}'s steward, caretaker and ruler\dash the one responsible for the planet's future. 

To his eyes, \Tiamat{} and the \firstgendragons{} are the ultimate apotheosis: The combination of natural (\ophidian) and supernatural (\xzaishannic) power, the perfect harmony of Order and Chaos, intelligence and brute force. And he is their representative, their high priest, their heir... perhaps even their peer. 

\lyricsbalsagoth{%
  In the Raven-Haunted Forests of Darkenhold, Where Shadows Reign and the Hues of Sunlight Never Dance
}{
  I hear the whispered words of the [\daemons].\\
  Such ancient secrets they sing...
}

\target{Nzessuacrith fears for Secherdamon's sanity}
He is arguably going insane. 
He has been too close to the \xss, and his world view, goals and ethics have become gradually twisted. 
\Nzessuacrith{} fears for him. 





\subsubsection{Science}
\target{Secherdamon's science}
\target{Secherdamon's research}
\ps{\Secherdamon} greatest obsession is actually not power, but \emph{knowledge}. He thirsts to know everything, unravel all secrets of the universe. This obsession has made him perhaps the greatest scientist on \Miith{}, and his research has created enormous amounts of new knowledge of magic and other sciences. 

A notable example is his development of the Rissitic magic theory. 

Another is the creation of his son \hr{Vizsherioch}{\Vizsherioch}.

His research is not just for discovering new things, but very much also for rediscovering \hr{Dragons have forgotten}{the knowledge they once possessed}. 

\target{Secherdamon hates the Shroud}
Because of his scientific disposition, \Secherdamon{} \emph{hated} the Shroud and hated \Ishnaruchaefir{} for masterminding it. 
To him, the stupefying Shroud was an atrocity against \Miith, nature, the universe, and indeed the very concepts of Knowledge and Truth. 
Aenigma and Gnosis. 
The foundations of the greatness of the \dzraicchenosses. 





\subsubsection{Ideal: The old \dragonland}
\ps{\Secherdamon} ideal is the old fallen \dragonland, the ancient empire of his people in their glory days. 

\lyricsbs{Arcane Wisdom}{%
  Abyssic Wrath of the Death-Philosophy (Once a Proud and Ancient Civilization)
}{
  Bathing in ice-cold rivers by the starlit night. \\
  Galloping on the fronts by the shivering light. \\
  Brave Europa, with our kin shinning bright, \\
  united we stood when need appeared to fight.
  
  Abyssic Wrath of the Death-Philosophy. \\
  Abyssic Wrath of the Death-Philosophy.
  
  Tears blurring our vision. \\
  Lethe in our swords. \\
  To the Strife!
  
  A myriad centuries later, \\
  the need is now stronger. \\
  Clearly what we had is ours no longer. \\
  Time hast come to avenge their mourning souls. \\
  Take back what is ours, \\
  put to the sword the weak ghouls.
}









\subsection{Physique}
\Secherdamon{} was golden in \colour, with stripes of black and red. 
He conveyed an image of blazing fire with some black ash/coal/smoke interspersed. 
His eyes were pearly white. 
He had a crown-like formation of many horns. 

\Secherdamon was longer than \Ishnaruchaefir but lighter. 
Compared to the \hr{Dragon size}{standard \draconian proportions}, he was quite long and slim. 

He had multiple heads. 
After he became a \shaeeroth, he changed his body into a multi-headed hydra to emulate \Tiamat, his great ideal. 

\Secherdamon was terrible to look upon. 
Even more so than \Ishnaruchaefir.
\Secherdamon had gathered power and worked hard to make himself imposing and regal and dominant and godlike.

\citebandsong{Nile:Ithyphallic}{Nile}{
  As He Creates So He Destroys
}{
  No living creature can look upon his face\\
  And endure its terrible heat and black radiance\\
  That is like the reverberating unseen rays of molten iron\\
  Which strike and burn the skin of those who would dare\\
  Gaze into the countenance of the idiot god
}









\subsection{Politics}
Among \ps{\HriistD}{} allies is \Xarocchetsel, who has his own mystic agenda. And his son, \Vizsherioch, and \Nzessuacrith. 

He has all sorts of creatures, \Miithian{} and alien, serving him.

\lyricsbalsagoth{
  In the Raven-Haunted Forests of Darkenhold, Where Shadows Reign and the Hues of Sunlight Never Dance
}{
  Swaying serpents ring my oak-hewn throne,\\
  Night and Shadow are my hunting dogs.\\
  Ravenous, they howl to be unshackled,\\
  that their maws may be glutted \\
  (with the blood of my foes).
}

He is the closest thing the Sentinels of \Miith{} have to a supreme leader. 

\lyricsbs{Emperor}{I Am the Black Wizards}{
  Mightiest am I, \\
  but I am not alone in this cosmos of mine. \\
  For the black hills consists of black souls, \\
  souls that already dies one thousand deaths. \\
  Behind the stone walls \\
  of centuries they breed their black art. \\
  Boiling their spells in cauldrons of black gold. 
  
  Far up in the mountains, \\
  where the rain fall not far, \\
  yet the Sun cannot reach. \\
  The wizards, my servants, \\
  summon the souls of macrocosm.
}





\subsubsection{Plagued with corruption}
\target{Secherdamon plagued with corruption}
\Secherdamon's Sentinels were plagued by much corruption, infighting and inefficient bureaucracy. 
\Secherdamon had much trouble getting anything done. 
He could not be everywhere at once. 
And it was hard for him to determine which subordinates were faithful and which were twisting his will and running their own businesses behind his back. 
And he was dependent on them, so he could not just crack down on them. 

This is an aversion of the trope \quo{\trope{TheTrainsWillRunOnTime}{The Trains Will Run On Time}}.

This was why the Rissitic Dominion \hr{Rissitic tribes divided}{fell apart in \Narkiza's time}.









\subsection{\HriistN} 
\target{Rissit Nechsain}
\target{Rissit}
\index{\HriistN}
\index{Rissit}
The primary god of the Rissitic religion. 
\quo{\Hriist} is his true \emph{spirit name}, \quo{\Nechsain} is his title. 
Also called \quo{Rissit} by nonbelievers. 
%(\quo{\Hriist} is difficult to pronounce: The initial HR is pronounced as a [H] followed by an [RH] (guttural R). Both T's should be audible. Also note that in \quo{\Nechsain}, the CH is indeed a [CH], not a [KH] (unlike similar words like German \emph{nächste}, [NEKH-ste]]).)

Rissit's symbol is a dagger thrust into the ground with a cobra snake coiled around it, as if crawling down from the heavens. 

He was originally an \hr{Ortaican gods}{\Ortaican{} god}. 
His following \hr{Rissitics were Ortaican}{was an \Ortaican{} sect}. 
But he later split with the rest and formed his own religion. 

Unbeknownst to his Rissitic worshippers, he is actually the \dragon{} \hr{Secherdamon}{\IrocasSecherdamon}. 



\subsubsection{The Seven Scorpions}
There are seven powerful \daemonic{} demigods that serve \Nechsain. They take the form of scorpions or scorpion-like humanoids or monsters. Sometimes the seven merge together in one body, resembling a scourge/whip with seven tails, each shaped like the tail of a scorpion. 

When \Nechsain{} fights, the seven scorpions swarm around him and fight for him.















\section{\QuessanthIshnaruchaefir}
%\sectioncharunspec{\Ishnaruchaefir}{\dragon}{\male}
\target{Ishnaruchyfir}
\target{Ishnaruchaefir}
\index{\Ishnaruchaefir}
\index{\QuessanthIshnaruchaefir}
A \shaeeroth, the second of the three sons of \Tiamat and the brother of \Nexagglachel{} and \Secherdamon.
His father was \Iurzmacul. 

\Ishnaruchaefir{} was an explorer, unraveller and destroyer. 
He was \hr{Ishnaruchaefir's power}{immensely powerful}. 









\subsection{Glaive} 
\target{Ishnaruchaefir's glaive}
\target{glaive}
\ps{\Ishnaruchaefir} weapon of choice is a \quo{glaive}, named \Triestessakhin.

Within it, he has \hr{Ishna slays his beloved}{bound the soul of his beloved}, \Triestessakhin{}. 
It is also his \hr{Weaving artifacts}{weaving artifact}. 

\Triestessakhin{} has the form of a massive, scythe-like polearm with multiple blades sticking out in both directions. 
Heavily inspired by the one wielded by Crusnik in the anime \cite{Anime:TrinityBlood}.

The weapon holds tremendous emotions\dash all of the bruised emotions from \ps{\Ishnaruchaefir} and his love's relationship and mutual betrayal. 

It is clearly alive. 
It moans, howls, screams, cries and whispers to him. 
Its words are mostly nonsense. 
\Triestessakhin{} is quite insane after thousands of years of imprisonment, after all. 

She can also communicate with others, mostly in dreams. 
Perhaps she seeks out people (Ramiel?) and haunts them like a ghost. 

She loathes her own role as an instrument of the Shrouding. 
She is a good, loving mother-type character. 
\Ishna{} sadly comments that she was \quo{too good for this world}.

At times the glaive can be seen to weep tears or blood.

Perhaps at the end of the story, \hr{Ishnaruchaefir's glaive is destroyed}{the glaive will be destroyed}. 

He does not carry \Triestessakhin{} with him at all times. 
Most of the time it follows him around in the form of a maddened, grieving ghost which only he\dash and the especially clear-sighted\dash can see and hear. 
When he needs it as a weapon, he must shed some of his own blood in order to give it form. 
See an example of this \hr{Ishnaruchaefir summons his glaive}{when he fights the \ghobal{} in \Malcur}.

The glaive's blade is constructed using lots of \hs{occult geometry}. 
Thus the strange shape. 
It utilizes occult knowledge that he learned from some \hs{cosmic gods}. 





\subsubsection{Name}
Do not call the glaive \quo{\Rystessakhin}. 
Just \quo{the glaive}.
At least, not in the beginning. 









\subsection{History}





\subsubsection{Early history}
In the olden days of the \draconian{} empire, \Nexagglachel was the leader of many \dragons. 

In the beginning, \Ishnaruchaefir{} was top motivated and strove after knowledge, power, skill and mastery. 
This is one of the reasons why he became so \trope{Badass}{badass}. 

Later his priorities changed. 







\subsubsection{Death and rebirth as a \shaeeroth}
\target{Ishnaruchaefir's death and rebirth}
Early on, \Ishna{} went through a ritual of death and rebirth, in order to unlock his true power and turn him into a \hr{Shaeeroth}{\shaeeroth}.

Compare to how \hr{Rissit's death and rebirth}{\Secherdamon later did the same}.











\subsubsection{Responsible for \ps{\Nexagglachel}{} fall}
\Ishna{}, despite his best efforts, was responsible for his brother's death and ignominy, and for gifting the \resphain{} with more power than they could otherwise have hoped for. 
\HriistD{} hated him for it.
He never quite forgave himself for it, either. 
This shaped \Ishna{} into a reluctant, desperate saviour-figure.





\subsubsection{Inherits stewardship of \Miith}
\target{Ishnaruchaefir's stewardship}
\index{stewardship}
In their last conversation together, \Nexagglachel{} \hr{Ishnaruchaefir gains stewardship}{entrusted to \Ishnaruchaefir{} the \quo{stewardship} of \Miith}. 
It would now be up to \Ishnaruchaefir{} to lead the \dragons{} and defend the planet. 

This heavy burden of guardianship shaped \ps{\Ishnaruchaefir} life. 
He refused to let himself fail his promise to \Nexagglachel, the brother he adored and admired. 
He promised to \quo{do what must be done}, no matter how hard, how painful, how much strength it would take. 
This oath he intended to keep, with fanatical devotion. 
He would betray \Rystessakhin, \Secherdamon, \Nzessuacrith{} and the entire planet before he betrayed \ps{\Nexagglachel} trust. 
\Nexagglachel{} made \emph{him} responsible, and it was \emph{his} burden to bear, no one else's. 
He would not let \Rystessakhin{} or \Secherdamon{} or anyone else shoulder the burden in his stead. 
That would be shirking his responsibility, which would be betrayal. 
His responsibility, his inheritance from \Nexagglachel, becomes his life. 







\subsubsection{Leading the \dragons{} to war}
\Ishna{} fought against the \resphan{} menace, \hr{Ishnaruchaefir leads the Dragons to war}{leading his \draconic{} brethren and their massive armies into battle}.







\subsubsection{The \SecondShrouding}
\Ishnaruchaefir{} \hr{Ishnaruchaefir and the Shrouding}{was one of the masterminds behind the \SecondShrouding}. 

He saw \hr{Ishnaruchaefir chooses eternal war}{three possible futures}: 
Death, Chaos or eternal war. 
He chose eternal war, hoping to contain it and limit its damage. 







\subsubsection{Slaying his love}
\Triestessakhin, \ps{\Ishnaruchaefir} beloved, was against the \SecondShrouding{}. 
They argue and fight over it. 
Eventually \hr{Ishnaruchaefir slays his beloved}{\Ishnaruchaefir{} kills \Triestessakhin}. 

Mourning what he had done, \Ishnaruchaefir{} captured the soul of \Triestessakhin{} and bound it inside \hr{Ishnaruchaefir's glaive}{his glaive}, which he now always carries on him. 
He went on to use the glaive as his \hs{weaving artifact} as he and his fellows wove the \hr{Shrouding}{\SecondShrouding}. 

He remains to this day one of the pivotal \vertices{} keeping the Shroud in place.

But slaying his beloved was traumatic and became a turning point in his life. It distanced him further from Chaos, the \firstgendragons{} and his fellow \dragons.





\subsubsection{Angsting}
He angsts over having killed his love. 

\lyricsbs{Dreamsfear}{As Darkness Falls}{
  I am the master of my destiny.\\
  But this man in the mirror I see \\
  cannot be me.\\
  Seems just like yesterday:\\
  My life lay before me.\\
  But now I am reaching the end.\\
  What have I done?
  
  Once upon a time I had a dream, long ago,\\
  that everything I touched belonged to me and turned to gold.\\
  Now I know that dream can never be, it's far too late.\\
  Now I know that dream can never be, it's far too late.
  
  I just want to get the chance to be the best I can be. \\
  Looking to the future I can't see what's in store for me.\\
  Looking to the future I am blind, can't see the light.\\
  Looking to the darkness I am blind, can't see the light.
  
  Can't see the light. \\
  As darkness falls.
}

He thinks back about \Triestessakhin{} and begins to doubt. 
Was she right after all? 

\lyricsbs{Dreamsfear}{Burning Bridges}{
  Some things I took for granted, \\
  within which beauty lies.\\
  But I admit I never \\
  ever saw it in your eyes.\\
  Now, maybe I don't have the gift\\
  to see the things that you see,\\
  like when you said that you could see\\
  something beautiful in me.
}




\subsubsection{Hated as the Destroyer and Betrayer}
\target{Destroyer}
Since the \SecondShrouding, \Ishnaruchaefir{} was reviled by mortals and \resphain{} alike as the Destroyer: 
The wicked one who masterminded the \SecondShrouding{} which tore apart the world and laid waste to the Realms, killing millions, including many of his own race. 
He is the single greatest killer in history since the \firstbanewar. 
Everyone fears him for that. 

The title \hr{Ishnaruchaefir gains title of Destroyer}{actually began during the \secondbanewar}.

There are even \hr{Destroyer myth}{myths about him}. 

His fellow \dragons{} do not blame him for killing mortals, for few \dragons{} place much stock in mortal lives. 
But they blame him for killing and destroying \Rystessakhin{} and innumerable others of their kind. 
Destroying other \dragons{} is a serious crime among their race. 
(Temporarily killing them is OK.)








\subsubsection{Disappearance into the void}
\target{Ishnaruchaefir goes into the void}
After slaying \Triestessakhin, \Ishnaruchaefir{} is wracked with grief. 
He withdraws into himself. 

Seeking new answers, perhaps as an escape from himself or as a help in moving on from his grief, he begins to explore the cosmos beyond the known Realms. 

\lyricsbs{Emperor}{The Eruption}{
  ... and after years in dark tunnels\\
  he came to silence\\
  there was nothing...
  
  he realised that the cheering cries of worship\\
  were but echoes of his harsh outspoken word\\
  reflecting back at him from cold and naked walls\\
  in hollow circles fled illusions of wisdom he had heard
  
  \quo{From nothing came all I ever knew}
  
  and he beheld the ruins\\
  of an empire torn apart\\
  yet, no grief nor rage did bind him\\
  just silent and bewildered\\
  by the emptiness\\
  he stumbled off his throne
  
  suddenly, the walls around him cracked wide open\\
  and an endless void appeared in flickering, grey light\\
  \quo{What force, but silence, has deprived me of my coil?\\
  No trail to guide me. No point of reference in sight.}
  
  \quo{By nothing, resurrection will be pure.}
  
  and he beheld the ruins\\
  of an empire torn apart\\
  wiping dust off his shoulders\\
  just silent now in this emptiness\\
  leaving all behind
  
  step by step, past all past\\
  slowly he approached the surface\\
  nothing left to sacrifice\\
  the mirrors mocked him on the way
}

He flees into the outer universe, unwilling to face his own Aenigma. 

\citebandsong{Emperor:Prometheus}{Emperor}{He Who Sought the Fire}{
  And again he came to cherish,\\
  the comfort of mysteries.\\
  Inspiring and far away,\\
  timeless in the moment,\\
  they painted immortality.
}

He has a trauma which he must overcome before he can come to Gnosis of himself. 

\citebandsong{Emperor:Prometheus}{Emperor}{He Who Sought the Fire}{
  Forever drawn towards the centre\\
  of this ensorcelling flame.\\
  Tet, still in fear of the sacrifice\\
  it would take to know its name.
}

\target{Glaive must be destroyed}
The trauma has to do with \Rystessakhin. 
He must first free himself of his guilt and unhealthy attachment to her. 
He must make peace with her. 
In order to do that, \hr{Glaive is destroyed}{the glaive must be destroyed}. 

\citebandsong{Emperor:Prometheus}{Emperor}{He Who Sought the Fire}{
  \quo{Do not despair}, said the mystery.\\
  \quo{You will always have a friend in me.\\
    until the day you break my code.\\
    then I will be gone and you are free...\\
    ...to manifest another.}
}

\lyricsbs{Arcane Wisdom}{
  Seeker of All Wisdom and Knowledge (A Journey Beyond the Star-Realm)
}{
  Haunting memories of past deeds \\
  ablaze the mind, bind the senses. \\
  Yet the wisdom seeker flows on. \\
  High above mankind he soars.
}

\lyricsbs{Arcane Wisdom}{Of Skyfire Burning Deep}{
  My dark awakenings, abandoning my own self \\
  in search of other worlds. \\
  Not of sorrow and demise, yet of a star wintry cold; \\
  leaving me as dead to this outer parallel.
}

\lyricslimbonicart{Interstellar Overdrive}{
  Dark cosmic void, a neverending universe.\\
  The final frontier, so dark and mysterious.\\
  I saw the dying sun as I waited for the night.\\
  Now cold silence reigns, magic darkness domains.
  
  Unfold thy secrecy.
}

At some point, \Ishnaruchaefir{} disappears. 
In his exploration of the cosmos, he is sucked away and not heard of for a thousand years or more. 

\lyricslimbonicart{Dynasty of Death}{
  Through dark tunnels in levitation, \\
  black cosmic space in manifestation.\\
  I escape the earthly pandemonium\\
  into a vast nocturnal sanctum.\\
  A nemesis for all evil I confess.\\
  My soul bleeds by all the forces I possess\\
  A benediction of unholy wrath and sorrow.\\
  My heart is buried in dark catacombs of horror.
  
  Sucked into that hole, that deep black hole.\\
  Not for a thousand years will I manage to crawl\\
  out of this darkness,\\
  this supernatural darkness.\\
  All the vibes are insane\\
  in the wilderness of pain.
}

He feels like he is dead.  
Maybe he \emph{is} dead in a sense. 
(\KhothSell, remember.) 

\lyricsbs{Hate Eternal}{Path to the Eternal Gods}{
  On this journey into death, \\
  I am beside myself in tremendous bliss,\\
  For this allegiance has been made. \\
  Shall my sins be absolved, \\
  washed away by the blood of the sacred lambs? \\ 
  Yet I am not amongst the flock.
  
  The remnants of my life now become my vast illusions,\\
  whilst exiled from your grace. 
  
  With fortitude, with courage, 
  I face all my fears, \\
  on my path to the eternal gods. \\
  With my wrath, with my disdain, 
  I face all my fears, \\
  on my path to the eternal gods.\\
  As I cross over into my final place, 
  I face all my fears, \\
  on my path to the eternal gods.\\
  As I bear trials in my final resting place, 
  I face all my fears, 
  on my path to the eternal gods.\\
}

But at last he returns: 
Changed. 
Mightier. 
Wiser. 
Grimmer. 

He has communed with dark cosmic gods and found great power. 

\lyricsdimmuborgir{Dreamside Dominions}{
  Losing control in seductive madness.\\
  Spiritual revelations, apocalyptic hypnosis.\\
  Dead \colours appear within unshallow graves.\\
  Alone in awe I face abhorrence below.
}





\subsubsection{Takes the name \Tzeorossh}
\target{Ishnaruchaefir takes the name Tzeorossh}
In his self-imposed exile, \Ishnaruchaefir used the Gnosis he had gained from his sojour into the void and took a new name:
\quo{\Tzeorossh}, meaning \quo{Exile}.

The title of \quo{\hs{Exile}} was originally given to him by \Secherdamon in hatred.
Eventually, as \Ishnaruchaefir secluded himself in the Mirage Asylum, he came to accept his status as an exile, so he accepted the name and even formally and metaphysically adopted it.





\subsubsection{Why do the cosmic gods live?}
Once, \Ishnaruchaefir asked a great cosmic god:
\ta{Why do you live? What is your purpose? What is your goal, your motivation?}

The god answered: \ta{We \emph{know}.}

\Ishnaruchaefir pondered that ever since. 
He was sure there was some great insight hidden in that answer. 
Perhaps because the cosmic gods \emph{knew}, they were content and happy and needed never strive for anything ever again. 
Perhaps the gods \emph{knew} the future and thus had no need of motivations, since they knew that everything they would ever do was already determined. 
Or maybe the truth was something else entirely.

\Ishnaruchaefir pondered that question till the end of his days. 
He never solved it.
Not at the time of the \thirdbanewar and not after it. 





\subsubsection{Since then}
In the millennia since then, \Ishnaruchaefir{} has opposed the \banes{} and \resphain{}, and has sometimes worked with his fellow \dragons, but he has always been a loner who went his own way and answered to no one, showing up whenever he chose to do battle and then disappear again. 

He has worked as a wandering warrior for an aeon, showing up when everyone least expects it to kick the butts of the Cabal or some other organization opposing him, his Bloodline or his moral principles. 

One of his reasons for waging war against his fellow \dragons{} is his belief that it actually strengthens them: A strong enemy breeds unity. If he can keep the \dragons{} organized in two warring factions, then they are much better off than if they were all just squabbling amongst themselves. Or so \Ishna{} reasons.

He has not always been fighting on the Sentinel side. At times he has directly opposed them, saving \resphain{} from \draconic{} attacks and waged war against \dragons. For this reason, \HriistD{} does not trust him, and the Cabal don't know what to think. 

He actually liked the idea of the \resphain{} being imbued with \draconic{} blood, because he belied they would inherit \ps{\Nexagglachel}{} hatred of the \banes{}, and they would plot against their creators. This was one reason why he let it happen. But it still somewhat pains him. He was correct, tho: Some of the \satharioth{} went on to form \Kezerad, \Mystraacht{} and \Baelzerach. 

%His fortress of \Nithdornazsh{} has fallen into disrepair. In fact, it has withered and died. But its soul lingers and can be reborn. It just needs to feed. (Resurrecting a living fortress has very rarely been done, so no one suspects that's what \Ishna{} is up to.)





\subsubsection{Maintaining the Shroud}
\target{Ishnaruchaefir maintains the Shroud}
\Ishnaruchaefir would like to be free of \Rystessakhin and the glaive, but he needed them in order to maintain some power over the Shroud.
Without him the Shroud might grow unstable, and he would certainly lose influence.

More importantly, the \hs{Mirage Asylum} would collapse, for it was maintained by his magic and could not survive without him.
And he could not live in peace or get any work done (scientifically and politically) without a sanctuary.

So he must keep \Rystessakhin enslaved as a bound wraith, however much it pained him and her.
After all, he still had his stewardship.
He had a responsibility for the future of the world, so he could not afford to lose his personal power nor his political and metaphysical influence on the world.

\Ishnaruchaefir must continuously work to maintain the Shroud. 
This forced him to \hr{Ishnaruchaefir's Nadir}{go into a periodic Nadir}. 










\subsection{Names, titles and reputation}
\target{Ishnaruchaefir's names}
\Ishnaruchaefir's full name was \Quessanth \Melechet \Nierzshah \Tzeorossh \Ishnaruchaefir.

\begin{itemize}
  \item 
    \quo{\Melechet} was his \hr{Shaeeroth name}{\shaeeroth name}. 
  \item 
    \quo{\Nierzshah} means \quo{\hs{Destroyer}}.
    He \hr{Ishnaruchaefir takes the name Nierzshah}{took this name during the \secondbanewar}. 
  \item 
    \quo{\Tzeorossh} means \quo{\hs{Exile}}.
    
    He \hr{Ishnaruchaefir takes the name Tzeorossh}{took this name after the \Shrouding}. 
    The title of \quo{\hs{Exile}} was originally given to him by \Secherdamon in hatred.
    
    When he, once in a while, introduced himself with his full name, he would make a sarcastic grimace when he said \quo{\Tzeorossh}.
\end{itemize}

\target{Ishnaruchaefir's titles}
\Ishnaruchaefir{} had a number of titles and kennings by which he was sometimes known in various stories and myths, most notably the epic poem \emph{\hr{Wanderers in Darkness}{\WanderersInDarkness}}. 
These include:

\begin{itemize}
  \item The \hs{Destroyer}.
  \item The \hs{Exile}.
  \item Wanderer in Darkness. 
\end{itemize}






\subsubsection{Mythical status}
\target{Myths about Ishnaruchaefir}
\Ishna{} was infamous among the Vaimons and other learned people. 
He was referred to as an \quo{Immortal} and an evil \chaos{} sorcerer. 
But most people don't know that he is a \dragon. 

Remember to have many references to him as a mythical, mystical figure. 

Among other things, he was known in a twisted version as the evil god \hr{Isphet}{\Isphet} in \hs{Iquinian mythology}.

\Ishnaruchaefir was \hr{Ishnaruchaefir in Ortaican mythology}{not mentioned at all} in \hr{Ortaican mythology}{\Ortaican mythology}.





\subsubsection{Reputation}
\target{Ishnaruchaefir's reputation}
\Ishnaruchaefir{} has a reputation for being dangerous, unreliable and a liar. 

He has a worse reputation than \hr{Secherdamon's reputation}{\Secherdamon{} does}. 










\subsection{Personality}





\subsubsection{Burdens}
\target{Ishnaruchaefir's compassion}
\Ishnaruchaefir carried a number of burdens:

\begin{description}
  \item[\ps{\Nexagglachel} stewardship:]
    \Ishnaruchaefir inherited from \Nexagglachel a \hr{Ishnaruchaefir's stewardship}{stewardship}, a responsibility to defend \Miith against invaders. 
  
  \item[\ps{\Rystessakhin} compassion:]
    \Rystessakhin was \hr{Rystessakhin's personality}{very kind and compassionate}
    After \hr{Ishnaruchaefir kills Rystessakhin}{killing her}, he felt that he had robbed the world of a great benefactor. 
    He felt some obligation to carry on her will now that she could not.
    So occasionally he would be prone to fits of compassion for lesser beings when he remembered how his beloved would have felt towards them.
    This was what happened when he \hr{Ishnaruchaefir saves Criseis}{saved \Criseis}, and again when he \hr{Ishnaruchaefir saves Rian and Neina}{saved Rian and Neina}.
  
  \item[Destroyer:]
    Though he tried to repress and hide it, \Ishnaruchaefir had some measure of guilt over \quo{\hr{Ishnaruchaefir destroys the world}{destroying the world}}. 
    Especially because he knew \Rystessakhin would not have wanted it.
  
  \item[Maintainer of the Shroud:]
    \Ishnaruchaefir was \hr{Ishnaruchaefir and the Shroud}{partially responsible for maintaining the Shroud}. 
    He possessed a \hs{weaving artifact} in his \hs{glaive}. 

\end{description}






\subsubsection{Code of \honour}
\target{Ishnaruchaefir's code of honour}
\Ishnaruchaefir{} has a certain code of \honour and is often willing to \honour truces and treat his enemies with a certain respect. 
But if prodded, he is prone to flying into a genocidal rage and commit horrible atrocities. 

The best example of this is the story of the deaths of \hr{Criseis's siblings}{\ps{\Criseis} siblings}. 

One should note that \resphan{} lives mean \shout{nothing} to him. 
\Human{} lives neither. 





\subsubsection{Contrast to Ramiel}
In a sense, \Ishna{} stands in contrast to Ramiel: 
\Ishna{} was born of Chaos but seeks to keep the Chaos within him in check. 
Ramiel, on the other hand, is born of the cold power of \Erebos, but actively embraces Chaos.

At times, they meet halfways and come to a sort of understanding as respecting enemies.





\subsubsection{Fake weaknesses}
\target{Ishnaruchaefir's fake weakness}
Ever since the \secondbanewar, \Ishnaruchaefir{} pretended to be at once stronger and weaker than he really was. 
He was universally feared, which was good. 
But he knew it would also come in handy if people would underestimate him. 

So he cultivated an image as a \trope{XanatosGambit}{Xanatos Gambiteer} supreme, a schemer with the cunning and ruthlessness to utterly destroy anyone who tried to match wits with him. 

But at the same time, he downplayed his physical prowess. 
He was one of the mightiest \dragons{} ever to live, but he faked being weaker. 
Among other things, he planted fake \quo{Achilles' heels} in \emph{\hr{Wanderers in Darkness}{\WanderersInDarkness}}. 
And when he, once in a rare while, entered into physical combat, he would act weak. 
He fled from several engagements he could have won. 

Once or twice, he even let himself be killed. 
Especially if the circumstances of the battle in question could be connected to one of his fake Achilles' heels. 

Under circumstances connected to one of his \emph{genuine} weaknesses, however, he tried to the very best of his ability to hide it.
This was made easier by the fact that he, by default, was playing weak, and therefore had reserves of strength he could call upon to offset any imposed genuine weakness. 





\subsubsection{Fear of the darkness within himself}
\Ishnaruchaefir{} has some \quo{deep caves and tunnels} which he fears. 
These exist in the world around his Mirage Asylum, but they also reflect, and are shaped by, the darkness within himself. 
(Remember, the Shroud shapes our perception of the world around us, so a voyage through space can simultaneously be a voyage through one's own mind.) 

At times he needs to venture into them. 

Compare to Chia Black Dragon and her caves in \cite[p.120-150]{StephenMarley:ShadowSisters}. 

\Ishnaruchaefir{} fears the \xsic{} blood within him, and the madness and evil it brings. 
This is one of the reasons why he has done such research into the far cosmos: 
He is trying to escape from that which is inside him.

\lyricslimbonicart{Behind the Darkened Walls of Sleep}{
  Behind the darkened walls of sleep, \\
  as body rest and mind goes deep. \\
  A door opens in my heart. \\
  A dark euphoria.
  
  As twilight falls, the awakening 
  of the creature within me. \\
  We are bound, we are blessed 
  in supernatural darkness. 
  
  Behind the darkened walls of sleep.
}





\subsubsection{Impulsiveness}
\target{Ishnaruchaefir's impulsiveness}
\Ishnaruchaefir{} often acted impulsive, volatile and unpredictable. 
But this was actually a ruse to throw his enemies off his trail, hide his true long-term plans and make his actions harder to predict. 

In truth, \Ishnaruchaefir{} expended all of his impulsiveness back in the \secondbanewar, where, during the \SecondShrouding, he devastated half of \Miith{} in a moment of irrational anger. 
He learned from that and never let his anger control him again.





\subsubsection{Loner}
\target{Ishnaruchaefir is a loner}
\Ishnaruchaefir{} was something of a loner. 
This was always true. 
Since he was young he always went his own ways and sought his own counsel. 
That is why he became a great sorcerer, philosopher and fighter, but not nearly the ruler, politician or diplomat that his brothers were. 

His solitary demeanour was one of the reasons why he decided to turn on \Triestessakhin{} and kill her. 
He was certain his own plan was best. 
He trusted himself more than he did her. 

Since then, he isolated himself more and became even more of a loner. 

\target{Ishnaruchaefir never apologizes}
\Ishnaruchaefir{} \emph{never} apologizes to anyone (except perhaps \Criseis, and then only indirectly). 
He will not ask anyone for forgiveness or love.
He gave that up \emph{long} ago. 

\target{Ishnaruchaefir nerd}
\Ishnaruchaefir{} also does not have great social skills. 
He is cunning and can be a great manipulator, but mostly from afar. 
He is a bit of a nerd. 
So he often leaves it to \Criseis{} to handle things involving social interaction. 





\subsubsection{Shroud policy}
\target{Ishnaruchaefir and the Shroud}
Throughout the story, \Ishnaruchaefir{} helps several people break free of the Shroud. 
Some of the good guys get the impression that \Ishnaruchaefir{} opposes the Shroud and fights against it. 
But this is not true. 
\Ishnaruchaefir{} is pro-Shroud and pro-\hr{Charade}{\charade} and always has been (\hr{Ishnaruchaefir and the Shrouding}{he helped forge it}, after all). 
But he wants to help create some heroes that can fight for whatever causes he deems just. 
And often, a hero is better if he sees through the Shroud. 





\subsubsection{Sex life}
\Ishnaruchaefir{} ventures out of his \hs{Mirage Asylum} every now and then to get himself some tail. 
He usually seduces some \sphyle. 
He has few scruples about this and will gladly fuck very young girls or married \sphyles. 
He doesn't commit rape, but he can be \emph{very} persuasive. 

Occasionally he seeks out other \dragons{} for sex. 
He is hated and feared by many, but his legendary status, immense power and personal charisma means that many female \dragons{} go for him anyway. 

Sometimes he changes shape and has sex with a \resvil, \human{} or \nephil{} just for the Hell of it. 

He does not look for serious relationships. 
That did not work out last time, and he doesn't \emph{really} need one (unlike a \human{} or \scatha). 
He is, after all, \hr{Ishnaruchaefir is a loner}{a loner}. 

He does \emph{not} have sex with \hr{Criseis}{\Criseis}. 
She is more like a daughter to him. 





\subsubsection{Spouts inanities}
\target{Ishnaruchaefir's inanities}
To test people, \Ishnaruchaefir{} sometimes spouts lines that look profound at a glance but are really just inane platitudes. It's a test to see if the other guy accepts them as profound wisdom or recognizes them as inane and calls him out on it. 

As he says: 
\ta{Who needs profundity when you have a reputation?}





\subsubsection{Torpor}
\Ishnaruchaefir{} spent much of his time lying around in \hs{torpor} in the Mirage Asylum, pondering some of his many Aenigmata. 





\subsubsection{Understands \ps{Nexagglachel} curse}
\target{Ishnaruchaefir understands the curse}
\Ishnaruchaefir{} \hr{Ishnaruchaefir eats Nathrach}{slew the \sathariah{} \Nathrach{} and ate his soul}, including his \hr{Fragments of Nexagglachel}{\Nexagglachel{} fragment}. 
From this, \Ishnaruchaefir{} {inherited a certain intuitive understanding} of \hr{Curse}{\NexagglachelsCurse} and the psychology of the \satharioth. 

This gave him a certain edge against the \satharioth, and against his fellow Sentinels. 
\Secherdamon{} knew much of the \resphain{} and had studied them much, but he did not have the same first hand experience and understanding of their psychological condition (how the Curse actually \emph{feels}) as \Ishnaruchaefir{} had. 









\subsection{Physique}
\Ishnaruchaefir{} was obsidian black with fiery red eyes. 
He had four long, slim, curved horns on his head. 

\Ishnaruchaefir was 25 metres long, with \hr{Dragon size}{standard proportions and weight} for a \dragon of his length. 





\subsubsection{\Human{} form}
If \Ishnaruchaefir{} were a human (or if he were to take \human{} form), he would look much like the Count of Monte Cristo as portrayed in the anime \cite{Anime:Gankutsuou}. 





\subsubsection{\Scathaese{} form}
In his \scathaese{} form his scales are onyx black. 
He wears black \armour, lined with edges of silver and blood red, and a black cloak. 









\subsection{Politics}





\subsubsection{\Baelzerach}
\target{Ishnaruchaefir and Baelzerach}
\Ishnaruchaefir{} has a tribe of \Baelzerach{} \resphain{} allied with him who will help him if he needs it. 
Their chieftain is \hr{Najarod}{\Najarod}. 





\subsubsection{Cosmic gods}
\target{Ishnaruchaefir and cosmic gods}
\ps{\Ishnaruchaefir} power does not all stem from the \xss{}. He has also had dealings with the \hs{cosmic gods}. This is one of the reasons why he is so badass. But his dealings with these alien, incomprehensible powers have also driven him somewhat mad. He has seen glimpses of the truth beyond Chaos and Darkness. 

\target{Ishnaruchaefir and Zaz}
Among other cosmic gods, \Ishnaruchaefir had some dealings with the enigmatic pair \hr{Zaz}{\Zaz and \Urzaz}. 

There was a time when he \hr{Zaz denies Ishnaruchaefir}{appealed to them for aid and was violently denied}. 

Since then, he repaired his relations with them and learned how to better interact with them. 
They became \quo{allies} of his, and he could command much of their power.

\lyricsbalsagoth{Unfettering the Hoary Sentinels of Karnak}{
  What sublime power awaits the aspirant, the querent who dares seek answers in those shadowed places where men of lesser fortitude fear to gaze?
  
  The path to elucidation is seldom devoid of thorns, the road to knowledge rarely free of perils!
}

\lyricslimbonicart{The Dark Paranormal Calling}{
  I cross dimensions unseen \\
  to ride on the axis of dreams.\\
  As I drift on through the dark corridors of post mortem,\\
  the only light in the darkness\\
  is the flame that burns in my soul.
  
  I intend to follow \\
  the eternal flame of my secrecy.\\
  Emancipate the mortal world \\
  as minds redeem from the mortuary.\\
  The only life in the darkness\\
  is the force that yearns in my soul.
}





\subsubsection{\Criseis}
\target{Ishnaruchaefir and Criseis}
\Ishnaruchaefir{} holds \Criseis{} tight and has feelings for her almost like a daughter. 
She serves as a replacement for \Nzessuacrith, with whom he fell out. 

He frequently brings her along on his journeys, for a number of reasons: 
\begin{enumerate}
  \item 
    \hr{Ishnaruchaefir's senses}{Unlike him}, she \hr{Criseis' senses}{has sharp senses}. She can scout for him. 
  \item 
    She asks to come along. 
    She \hr{Criseis contains Ishnaruchaefir}{hopes to contain his destructive behaviour}. 
\end{enumerate}






\subsubsection{Family}
\target{Ishnaruchaefir's family}
\Ishnaruchaefir was the second son of \TyarithXserasshana.
His father was \hr{Iurzmacul}{\Iurzmacul}. 

\Ishnaruchaefir{} loved \Triestessakhin. 
They had one child together, \Nzessuacrith. 

He also had three sons (with other mothers). 
They were \hr{Ishnaruchaefir's sons die}{all killed in the \secondbanewar}. 
After this, he poured all of his love on \Nzessuacrith, and they were closely knit. 
This made it extra hard for both of them when they split (after \Ishnaruchaefir{} killed \Triestessakhin). 

\Ishnaruchaefir{} also had three grandchildren, the sons and daughters of his fallen sons. 
They were all black \dragons, and young enough to have only one name each. 
They were: 
\begin{itemize}
  \item \hr{Rathyon}{\Rathyon}. 
  \item \hr{Tentocoth}{\Tentocoth}. 
  \item \hr{Thiencaste}{\Thiencaste}. 
\end{itemize}





\subsubsection{\NerrhanKoss}
\Ishnaruchaefir{} owes some measure of allegiance towards the \xs{} \hr{Nerrhan-Koss}{\NerrhanKoss}, who is sort of his mentor and patron. 
It was \NerrhanKoss{} who \hr{Glaive origin}{gave him his glaive, sort of}. 





\subsubsection{Sentinels}
\target{Ishnaruchaefir and the Sentinels}
After the \SecondShrouding, \Ishnaruchaefir{} was reclusive and seemed to have no contact with the world, hidden away in his Asylum as he was. 

There were persistent rumours that \Ishnaruchaefir{} had been cast out of the Sentinels of \Miith{} for his atrocities, being too evil even for them. 
The stories came from the fact that \Secherdamon, one of the most influential Sentinels, publicly denounced \Ishnaruchaefir. 

But this was not true. 
\Secherdamon{} had no authority to \quo{exclude} \Ishnaruchaefir{} from the Sentinels. 
In secret, \Ishnaruchaefir{} actually maintained more Sentinel contacts and covert influence than most people believed. 





\subsubsection{\Zaz and \Urzaz}










\subsection{Skills and powers}





\subsubsection{Dull senses}
\target{Ishnaruchaefir's senses}
\Ishnaruchaefir{} has a weakness. 
Because of the heavy burden he bears (\Rystessakhin), the chaos he carries with him and the pain he endures, his senses are dulled. 
They are still sharper than a mortal's, but by \draconic{} standards they are dull. 
This applies whether he physically carries the glaive or not. 
It applies to both his physical and metaphysical senses. 

This is one of the reasons why he so often drags \Criseis{} along with him: 
She \hr{Criseis' senses}{has sharp senses} and can scout for him. 





\subsubsection{Howling}
When \Ishnaruchaefir{} draws deep of his dark, cosmic magic, you can hear a faint howling, coming from infinitely far off, from eternally dark halls beyond the firmament. Like grim monstrosities mindlessly howling\dash blind, cruel, uncaring like the universe itself. Like the piping, dancing Outer Gods at Azathoth's court in \authorbook{\HPLovecraft}{The Dream-Quest of Unknown Kadath}. 

This has to do with the dark \hs{cosmic gods} whom \Ishnaruchaefir{} has studied to master his magic. 





\subsubsection{Languages}
\target{Ishnaruchaefir's languages}
\Ishnaruchaefir{} spoke several languages. 

He spoke an archaic form of Imetric. 
He had had dealings with the Imetrians, since Salacar was his kinsman. 

But he never learned \Velcadian. 
Before the \thirdbanewar{} he had not been to \Azmith{} for centuries, so he had never learned it. 
Fortunately, \hr{Criseis languages}{\Criseis{} did speak \Velcadian}, so he could use her as a translator. 





\subsubsection{Nadir}
\target{Ishnaruchaefir's Nadir}
There are periods where \ps{\Ishnaruchaefir} \vertex{} goes into a natural Nadir. 
This has to do with his status as one of the chief architects and \hr{Ishnaruchaefir maintains the Shroud}{maintainers of the Shroud} and the wielder of an important \hs{weaving artifact}. 

\Rystessakhin{} is a central keystone in the Shroud. 
Occasionally the glaive needs to be \quo{recharged}. 
This leaves \Ishnaruchaefir{} weary and weak. 
Also, he cannot wield \Rystessakhin{} in combat.
The glaive is a very powerful weapon, so not having it makes a significant difference. 

The Nadir happens semi-regularly and can be predicted astrologically. 
(Although the cycle varies, so you have to be close in time before you can predict it with any accuracy.)
It happens once every 30 years or so. 
(But do not mention any exact number in the books. I don't want to paint myself into a corner.)
The Nadir lasts about a week. 
He is weakest by the middle of that week. 

According to \WanderersInDarknessEmph, the Nadir occurs \quo{when the \hs{Exile} is engulfed by the briny waters}. 
This is an astrological sign. 
\WanderersInDarknessEmph also reveals that he is at his weakest in the middle of the period, and there are further astrological signs to mark when this happens. 

The Nadir is very hard and traumatic for \Ishnaruchaefir, not only metaphysically but also emotionally. 

Compare to two scenes in \cite{StevenErikson:TolltheHounds}:
The scene where Anomander Rake puts the sword Dragnipur away for a short while, demonstrating what an immense burden it is, and the scene where Rake is weak and vulnerable after having slain Hood and absorbed his wicked-powerful soul into Dragnipur. 

Also compare to Chia Black Dragon's \quo{dying time} in \cite{StephenMarley:SpiritMirror}. 

\target{Nadirs get worse}
The Nadirs are getting worse each time, because the Shroud is \hs{unravelling} and it takes more effort to hold it together. 

\target{Ishnaruchaefir bleeds in Nadir}
When \Ishnaruchaefir is in his Nadir, he gets bleeding wounds all over his body. 
And one can see a myriad long, aethereal tendrils radiating out from him, through which power drains out of him to sustain the Shroud and the glaive. 

They spring open on their own because of the power he has to expend in his weakened state.
He is paying \hr{The cost of magic}{the cost of magic}.
 
Compare him to Anomander Rake in Darujhistan in \cite{StevenErikson:TolltheHounds}. 

\Ishnaruchaefir's Nadir happened at times when the \quo{tides} of the Shroud were low, meaning that the Shroud was weak and permeable. 
At these times, \Ishnaruchaefir had to work hard to keep the Shroud stable. 
This hard spellword was what made him weak. 
He had to open himself up to the world in order to pull its strings, and this openness made him vulnerable. 

If he failed, he risked a nasty backlash against himself and his Mirage Asylum. 
The Asylum was unstable by nature and prone to collapsing or flying off into space if not maintained. 

It was unclear whether the Shroud itself might collapse without \Ishnaruchaefir's support. 
\Ishnaruchaefir himself tended to believe that his work was necessary to keep the Shroud alive.
His critics tended to think his work was of purely local significance. 







\subsubsection{Power}
\target{Ishnaruchaefir's power}
\target{Ishnaruchaefir's rage}
\target{Ishnaruchaefir's inner strength}
\Ishnaruchaefir was one of the mightiest \dragons in the history of \Miith. 
He was a tremendously powerful sorcerer and could call up hordes of monsters/demons to fight for him. 
He was also immensely strong physically and \hr{Ishnaruchaefir's inner strength}{possessed vast reserves of mental strength}. 

The sources of his power were manifold:

\begin{itemize}
  \item 
    Some of his metaphysical stemmed from the \xss. 
    He was, after all, a \shaeeroth.
  \item 
    Some of his power \hr{Ishnaruchaefir and cosmic gods}{stemmed from the cosmic gods}, such as \hr{Zaz}{\Zaz and \Urzaz}, with whom he had \hr{Ishnaruchaefir and Zaz}{dealings}. 
  \item 
    Some of his power stemmed from pure rage.
    In his normal cold, badass state, \Ishna{} is a terrific enough opponent. 
    But when his rage is triggered, his unholy \xzaishannic{} blood boils, and he flies into a blind fury, unconsciously tapping deep into the \chaotic{} power which he otherwise keeps in check. 
    In this state, he can cause tremendous destruction\dash and often has.
    
    In his fury, he relives the moment where he slew his beloved, and the events that led up to that moment\dash the conflicts, the lies and betrayals.
  \item 
    Another source was his pure inner strength. 
    This was a combination of his passion, his motivation and his uncaringness. 
    He possesses immense determination when he needs something done, but at the same time he has a certain detachedness that makes him fearless and unwavering. This is important, because fearlessness is one of your most important weapons when trying to master the energy of Chaos. 
    
    \Ishna{} masters Chaos because he \emph{must}, and because he does not fear it.
    
    \hr{Secherdamon}{\Secherdamon}, for all his millennia of research and all his dark pacts, has never achieved the same level of raw power (although he has garnered a greal deal more political power). At his core, \Secherdamon{} still has too much neediness, greed and fear. He lacks the cool, calm inner strength of his brother. 
\end{itemize}









\subsubsection{\Vertex{} status}
\target{Exile}
\ps{\Ishnaruchaefir} \vertex{} is called \quo{the Exile}. 
It is visible as a dim cloud of darkness in the night sky (to the occult astrologer who is using her spiritual sight and knows what to look for). 

His function and status as a \vertex{} is connected to his dark past and his betrayals. 
They have shaped him into the \vertex{} that he is today. 

He is often called a \quo{rogue \vertex}, aligned with no \matrix. 
This is strictly not correct. 
He has his own \matrix, with which his grandchildren and \Criseis{} are also aligned. 
And he is loosely affiliated with the \hs{Pyre} and some of the other Sentinel \matrices. 
But the links are tenuous and hazy and difficult to interpret. 















\section{\RaemythKhivaashNexagglachel}
\index{\RaemythKhivaashNexagglachel}
\index{\Nexagglachel}
\target{Nexagglachel}
A \shaeeroth, the eldest of the three sons of \Tiamat and the brother of \Ishnaruchaefir{} and \Secherdamon.

\Nexagglachel was the son of \Tiamat and \Sethicus, and thus the greatest of the three brothers.

\Nexagglachel{} was a ruler, leader and preserver of order. 









\subsection{History}





\subsubsection{Durance}
After \ps{\Sethicus} rebellion failed, \Nexagglachel and the other \dragons were entombed. 
\Nexagglachel was \hr{Nithdornazsh was Nexagglachel's tomb}{bound in \Nithdornazsh}. 

When he awoke, he used \Nithdornazsh as his citadel. 





\subsubsection{Almost rebuilt the \draconian{} civilization}
\Secherdamon{} and \Ishnaruchaefir{} believe that the \draconian{} people were well on their way to rebuilding all the glory they had lost in the \firstbanewar. 
\hr{Nexagglachel could not rebuild Ophidian civilization}{It was difficult, but they were making progress}. 
The \resphain{} destroyed all that when they assassinated him. 
His two brothers would hate the \resphan race forever for that crime. 





\subsubsection{Fall}
After the \resphan{} rebels had \hr{Rebels conquer Merkyrah}{conquered \Merkyrah}, they \hr{Fall of Nexagglachel}{captured \Nexagglachel}, killed him and \hr{Origin of Satharioth}{created the \satharioth} from his blood.

\target{Nexagglachel sacrifices himself}
It is hinted that \Nexagglachel{} had predicted what was coming, and that he willingly sacrificed himself, for two reasons:

\target{Nexagglachel makes Satharioth hate Banes}
\begin{enumerate}
 \item To save and spare his younger brother, \Ishnaruchaefir.
 \item To live on in the \resphain{} and instill in them \hr{Satharioth hate Banes}{a deep hatred of the \banes} that would eventually cause them to betray their masters. 
\end{enumerate}

It is also hinted that he predicted that his two brothers would, in time, grow more powerful than he. 





\subsubsection{Captivity}
\Nexagglachel{} was \hr{Fall of Nexagglachel}{captured by the \banes}. 
\hr{Nexagglachel in captivity}{He was tortured, but remained defiant}. 





\subsubsection{Sacrifice and victory}
In his \hr{Nexagglachel wants to be Tiamat}{striving to be as great as \TyarithXserasshana}, \Nexagglachel{} eventually accepted the cards that chance had dealt him. 
And he played those cards. 
If he were to die, then he would not die in vain. 
He would become a ghost, a plague, a \hr{Curse}{Curse} upon the \resphain{} and their \bane{} masters. 

\Nexagglachel \hr{Nexagglachel lives on in Satharioth}{did not perish but lived on inside the \satharioth}. 





\subsubsection{Victory}
\Nexagglachel set out to bring about the downfall of the \resphain, and he succeeded. 
When at last \hr{Daggerrain falls}{\Daggerrain{} was defeated}, it was very much due to \NexagglachelsCurse. 
He had done it. 
He had sacrificed himself for his people and brought ruin to the hated \banes, just like his mother had done. 
He was now her equal, and second to no \dragon. 





\subsubsection{Legacy}
After his death, \Nexagglachel{} was remembered fondly.

\citebandsong{Ihsahn:angL}{Ihsahn}{Threnody}{
  He lies quiet now, \\
  in the nothing.\\
  And there is no epitaph, \\
  no stone.
  
  Walker of barren paths. \\
  Seer of night.\\
  Friend of shadows.\\
  A carrier of light.
}

\Ishnaruchaefir{} and \Secherdamon{} both felt lost and alone without him to guide them. 

\citebandsong{Ihsahn:angL}{Ihsahn}{Threnody}{
  There are no promises\\
  in his solitary grave.\\
  There is no salvation.\\
  Only words.
}

They sought strength in his memory. 

\citebandsong{Ihsahn:angL}{Ihsahn}{Threnody}{
  But what then are these precious streams\\
  of coldness from the heights?\\
  They will never reach the fields below.
}

But he still lived on in the form of a hidden ally: The Curse. 

\citebandsong{Ihsahn:angL}{Ihsahn}{Threnody}{
  And his legacy flows\\
  like a river from ice.\\
  The hungry heart opens\\
  and drinks from this fountain.\\
  So cold.
}







\subsection{Names and reputation}
\target{Nexagglachel is a powerful name}
\quo{\Nexagglachel} was a very powerful name. 
The Sentinels who founded \Ortaican myth wanted to \hr{Ortaicans use Nexagglachel's name}{milk it for everything it was worth}. 

Later, \Secherdamon even \hr{Secherdamon takes the name Nexagglachel}{took the name \quo{\Nexagglachel} as one of his own}.

\quo{\Khivaash} was \Nexagglachel's \hr{Shaeeroth name}{\shaeeroth name}. 




\subsubsection{\Mezzagrael}
In \hr{Ortaican mythology}{\Ortaican mythology}, \Nexagglachel was known under the name \quo{\hr{Mezzagrael}{\Mezzagrael}}. 









\subsection{Personality}
\Nexagglachel{} was one of the noblest of \dragons. Much better than his two brothers. \Ishnaruchaefir{} acknowledges this, and even angsts about it a bit, while \Secherdamon{} rejects the notion. 

\Nexagglachel{} was a great and brave hero in some of the ancient wars. 

%\target{Ishnaruchaefir wars against Nexagglachel}
%Perhaps he fought against \Nexagglachel, and attempted to rally \dragons{} to this conflict, in an attempt to create cohesion rather than division among the \draconian{} people. He hoped to split them into two warring factions instead of a dozen warring factions.
He waged wars against other \draconian{} kingdoms and attempted to rally \dragons{} to this conflict, in an attempt to create cohesion rather than division among the \draconian{} people. He hoped to split them into two warring factions instead of a dozen warring factions. \Ishnaruchaefir{} and \Secherdamon{} fought by his side. 

\target{Nexagglachel wants to be Tiamat}
\Nexagglachel{} hoped and strove to become \hr{Tiamat's personality}{as great a leader as \TyarithXserasshana}. 
He wanted to unite the \dzraicchenosses{} as she had done, but never succeeded. 
The political situation was against such a unification. 
\Nexagglachel{} was every bit as skilled and strong-willed as \Kserasshana{} and could have done it, but events conspired against him. 









\subsection{Physique}
\target{Nexagglachel's appearance}
\Nexagglachel was pearly white with black eyes. 















\section{\Vizsherioch}
\target{Vizsherioch}
\index{\Vizsherioch}
\Vizsherioch{} is a \dragon, the son of \Secherdamon. 
In him, his father invested not only \draconic{} power, but also stolen \bane{} and \resphan{} power, intending to create a \draconic{} counterpart to the \satharioth. 

Vizsherioch is the Son of Chaos, the messiah of the \dragons and the harbinger of the \xs and the hordes of Chaos. 

He is evil and badass. 
Compare him to Blackheart from the movie \cite{Movie:GhostRider}.









\subsection{Name and titles}
\Vizsherioch and probably some other names. 

One of his titles, which he used himself, was \quo{Chaos Incarnate}. 
Another title was \quo{Son of Chaos}.

Later, \hr{Vizsherioch}{\Vizsherioch} \hr{Vizsherioch takes the name Sethicus}{took the name \quo{\Sethicus}}. 









\subsection{Physique}
\target{humanoid Vizsherioch}
He often takes on a humanoid form: 
A \scatha{} with ivory white scales. 
This makes him seem at once innocuous and freakish. 

\Vizsherioch appears as a \dax in his prime, with pearly white scales, wearing a loose robe of white, silver and gold. 

He is fearful to look upon, even to mighty ones such as \LocarPsyrex. 
Where \Secherdamon is fiery bright, his son \Vizsherioch is dark and sinister. 
Not in \colour, but in feel. 
A vast darkness follows behind him and around him. 

His eyes are frightening, even for \Psyrex. 
\ps{\Secherdamon} eyes are terrible enough, but \Psyrex is used to them. 
There is passion, fervour and desire in the eyes of \Secherdamon, and anger and hate, too. 
But in \ps{\Vizsherioch} eyes there are hints of otherworldly madness. 










\subsection{History}





\subsubsection{Birth}
The newborn \Vizsherioch{} fancied himself a reborn \xs. 

\lyricslimbonicart{From the Shades of Hatred}{
  A thousand years time dimension \\
  in subconscious incarceration. \\
  My hatred to man, has transformed me \\
  into a habitation for demons. \\
  A devil incarnation. \\
  In the forgotten past, ages ago, \\
  beast became my alter ego.
  
  The god in me, infernal black divinity.\\
  After years of agony and pain \\
  hatred is all that remains.
}

It had taken \Secherdamon{} thousands of years of planning and research to finally create \Vizsherioch. 
Compare him to Set Abominae from \cite{IcedEarth:SomethingWicked}. 





\subsubsection{Prophecies}
With his \bane{} blood he was a super-\dragon. 
Some wise people foresaw that he was destined to conquer and rule. 

\lyricsbalsagoth{Naked Steel (The Warrior's Saga)}{
  Born beneath the thrice-cursed cromlech \\ 
  (destined for deeds of greatness),\\
  Three stars aligned to assauge thine (newborn) cries,\\
  Foretold, the hilt of Red-Tooth awaits thine hand \\
  (kingdoms shall fall before thee!),\\
  And in the Nine Scrolls thine death prophesized.
}

Some foresee that he will fail and die a tragic death at a young age.

\lyricsbalsagoth{Naked Steel (The Warrior's Saga)}{
  This heart that pounds like a hammer,\\
  this heart that pounds so strong,\\
  this heart that pumps a great warrior's blood,\\
  this heart will pound for half as long.
}

It would be cool to have him break this prophecy. 





\subsubsection{\Malcur}
The summoning of \Nithdornazsh in \Malcur is part of \ps{\Secherdamon} plan to bring \maybehr{Vizsherioch}{\Vizsherioch} into Ascendancy. 
\Nithdornazsh{} is to become \ps{\Vizsherioch} citadel, a \nexus{} from which he can grow strong and spread his tendrils (politically and metaphysically) into the Realm of the Shroud. 
This is a vital step in the forging of the \maybehs{Dagger}. 
When the \Nithdornazsh{} project is complete, \Vizsherioch{} is more Dagger-y than ever. 

Previously, \Secherdamon{} had kept \Vizsherioch{} sequestered and hidden. 
He is his only son and the fruit of thousands of years of hard work, so \Secherdamon{} is very protective and does not want to lose him. 





\subsubsection{Becomes \shaeeroth and takes new names}
\target{Vizsherioch takes the name Sethicus}
\Vizsherioch \hr{Vizsherioch becomes Shaeeroth}{became a \shaeeroth}. 

In the process, he took two new names. 
He was the first \dragon ever to achieve manhood and turn \shaeeroth at the same time.
He took \quo{\hr{Sethicus}{\Sethicus}} as one of his names.
Maybe he took \quo{\Secherdamon} as the other. 





\subsubsection{Role}
He will become a pivotal player in the later books of \SentinelsofMith.

%Perhaps he will die at the end. Or perhaps \Secherdamon{} will die and \Vizsherioch{} will inherit his empire. I haven't quite decided who, but one of them should die. 
Eventually, \hr{Secherdamon dies}{\Secherdamon{} dies} and \Vizsherioch{} absorbs his power and soul into himself. 





\subsubsection{Extreme plan}
\ps{\Vizsherioch} plan was more extremist than \ps{\Secherdamon}. 
\Secherdamon{}, despite his dubious sanity, cared first and foremost about the \draconian{} race and wanted to save them. 
\ps{\Vizsherioch} loyalties, on the other hand, lay on the side of the \xss. 
He wanted to destroy the Shroud and let the \xss{} into the world, and dreamt of turning \Miith{} into an eternal Realm of Chaos. 

\Vizsherioch{} realized that the \banelords{} also wanted to unravel the Shroud. 
And so a doublecrossing race began, with both sides of the Feud trying to pull the right threads and so unravel the Shroud so it crumbled into a pattern that was in their favour and would let their own variant of \trope{CosmicHorror}{Cosmic Horrors} into the world. 





\subsubsection{Builds cult}

\Vizsherioch revelled in his megalomania and built a cult around himself.
This picked up speed after \hr{Secherdamon dies}{\Secherdamon died}, because \Vizsherioch now knew the responsibility lay with him alone.

He took the \hs{Dark Crescent} and reshaped it into a focused group that served him. 

\citebandsong{Nile:InTheirDarkenesShrines}{Nile}{
  Churning the Maelstrom
}{
  I Am the Uncreated God\\
  Before Me The Dwellers in Chaos are Dogs\\
  Their Masters Merely Wolves\\
  I Gather The Power\\
  From Every Place\\
  From Every Person\\
  Faster Than Light Itself
}









\subsection{Personality}
Have scenes with \Vizsherioch{} that show his personality and interests. What are his interests? Art? Science? Sex?




\subsubsection{He revels in evil}
\Vizsherioch{} revels in evil, death and destruction. 
He hopes it will bring him closer to his \xsic{} nature and fulfill his dream of becoming a \xs{} incarnation.

\lyricslimbonicart{The Ultimate Death Worship}{
  O' Darkness my master and mentor, \\
  witness the blood I shed. \\
  Victorious dreamlike death I enter, \\
  floating the streams so red. \\
  Destruction is the jewel of the black heart. \\
  To treat life as nothing holy. \\
  Hatred is the diamond in blasphemous art, \\
  as death you kiss infernally.
}





\subsubsection{Split personality}
\Vizsherioch{} has a split personality and changes back and forward between \quo{young mode} (a \dragon{} only few thousands or even hundreds of years old) and \quo{old mode} (when he feels like he's a \xs, millions of years old). 

When he speaks in his deep-pitched, magical voice, it is even deeper, darker and more foreboding than that of \Ishnaruchaefir{} or \Secherdamon. 





\subsubsection{Older than he looks}
In a sense, \Vizsherioch{} is older than he looks, having inherited some memories from the \resphan{} and \xzaishann{} blood in his veins.

He sees himself as the ultimate combination of the best aspects of the old and new generations of creatuers. 

\ta{%
  You think me young. 
  Fools. 
  The blood of the \xzaishanns{} runs stronger in me than in anyone else. 
  I AM [insert names of ancient \xzaishannic{} lords]. 
  
  This body is but temporary. 
  I am, I will be, and I have always been. 
  I am eternal. 
  \shout{I am Chaos}!}

\lyricsdimmuborgir{The Chosen Legacy}{
  I am the first creature of this Kingdom.\\
  I will be the One\\
  to out live His time\\
  with the triumph of free will.
}

He has memories going back millions of years. 

\lyricsbs{Monolith Deathcult}{Deus Ex Machina}{
  Atlantis was built when you amoebas crawled through filth. \\
  I am Holiness Divine, your Lord and Master, \\
  the Supreme God and Creator.
}







\subsection{Politics}

\Vizsherioch{} is very loyal and loves his father. Not only has he inherited genes of loyalty from his \bane{} blood, but his father has also been careful to treat him with love and respect and groom him to be a close ally and heir. 

He has been around \hr{Nzessuacrith}{\Nzessuacrith} for much of his life and sees her as a sister-figure of sorts. 
He calls her \quo{cousin} (which she is). 
This unnerves her. 
She secretly resents and envies him for having such a close and loving relationship with his father, whereas she herself has not gotten along with her father, \Ishnaruchaefir, in thousands of years. He knows this, and secretly holds her in contempt. 









\subsection{Skills and powers}





\subsubsection{Immortality}
\target{Immortal Vizsherioch}
\Vizsherioch{} is immortal. 
He can be killed, but he will come back to life. 

This is a state-of-the-art version of immortality, devised by \Secherdamon{} as an improvement on the \hr{Ophidian immortality}{skin-shedding}, \hr{Draconic immortality}{\KhothSell-based} and \hr{Malach immortality}{reincarnation-based} immortality techniques. 

Actually, he is more immortal than other immortals. 
He has a soul in him that is almost a \xsic{} soul, making it indestructible by any means known to the \Miithian s. 
If \Miith{} were destroyed by the \Voidbringer, \Vizsherioch{} would still live on as a \xs{} fledgling. 





\subsubsection{Weaving artifacts embedded in him}
\Secherdamon{} had \hr{Secherdamon steals weaving artifacts}{stolen several weaving artifacts}. 
When he created his son, he sealed and embedded the artifacts within \ps{\Vizsherioch} body, along with all of their power. 

This was impossible at the time of the \SecondShrouding{}, but \Secherdamon, being \hr{Secherdamon's science}{one of the greatest scientists and sorcerers of all time}, has researched and learned much since then, and his knowledge of magic is far deeper and more detailed now.





\subsubsection{\XzaiShannic{} weaknesses}
\target{Vizsherioch's XS blood}
\Vizsherioch{} has even more \xsic{} blood than his father, and suffers even more from their inherent weaknesses. 
He is prone to the \hr{XS slumber}{\xsic{} slumber}. 

But he also has \bane{} blood and can draw on \Erebean{} power to counter these ill effects. But when \hr{Daggerrain falls}{\Daggerrain{} falls}, his connection to \Erebos{} fails and \hr{Vizsherioch falls asleep}{he falls prey to the \xsic{} slumber}. 

On the other hand, his \xsic{} nature might make him truly immortal. 























\chapter{\Dragons: Others}
















\section{\AeocrithRystessakhin}
\target{Triestessakhin}
\target{Rystessakhin}
\index{\Rystessakhin}
\index{\AeocrithRystessakhin}
A \dragon, \ps{\Ishnaruchaefir} beloved and the mother of \Nzessuacrith. 

She opposed the scheme of the \hr{Shrouding}{\SecondShrouding}, which \Ishnaruchaefir{} advocated. 
They fought over this. 
Eventually \hr{Ishnaruchaefir slays his beloved}{he killed her}. 

\Ishnaruchaefir{} bound her soul in \hr{Ishnaruchaefir's glaive}{his glaive}, which became his \hs{weaving artifact}. 
\ps{\Triestessakhin}{} became condemned to an eternity as an unwilling \trope{BarrierMaiden}{Barrier Maiden}.

\lyricsbalsagoth{The Hammer of the Emperor}{
  A garland of newborn stars to adorn thee...\\ 
  the Permian Extinction, a parting gift.\\
  May your maleficent soul walk only in dark places.
}

\Triestessakhin{} is now condemned to an eternity as an unwilling \trope{BarrierMaiden}{Barrier Maiden}. 
She suffers. 

\lyricslimbonicart{Funeral of Death}{
  Blood is dripping as mind's tripping.\\
  In twilight sleep death is reaping.\\
  Blood stained, feels cold\\
  in solitude as night grows old. \\
  Death salvation, life capitulation, \\
  blood stained, feels cold, in the frozen soul.
  
  Life is a mandatory sacrifice \\
  for the eternal dream of Paradise. \\
  Make the time stand still, \\
  as silence is the last will.
}









\subsection{Consciousness or lack thereof}
\target{Rystessakhin's consciousness}
\Rystessakhin{} retained very little consciousness in her state. 
Her soul had been crippled and warped from the imprisonment and the immense strain that she carried in her efforts to uphold the Shroud. 
She had lost most of her personality and had no hope of regaining true consciousness ever again. 

\Ishnaruchaefir{} told himself that she lived and retained some measure of consciousness, and could even be contacted. 
But he deceived himself. 

Once in a while, he meditated on the glaive and sought out her soul on the astral plane (or whatever), where they would have \quo{sex}... of a sort. 
They clashed in a violent, sexual storm of love, hate, lust, pain, sorrow and conflict. 
When he returned from these seances, he is at once rejuvenated and aged. 
Haggard, but vitalized. 
To him, she remained within his reach (German: \emph{zum Greifen nah}), and yet forever lost to him. 

\lyricsduana{beached}{Beached}{
  \textbf{the storm} 
  
  white veins of light \\
  streak thru' clouds blackened by lust \\
  betray Earth \& Sea \\
  amidst their lovers' spat 
  
  waves crash dangerously \\
  'gainst jagged rock \\
  foaming \\
  a red passion 
  
  Midnite embraces them\dash{}still they rage \\
  the violent spray whips \\
  exposed leg of beach \\
  with icy fingers \& howl in climax
  
  \textbf{the calm} 
  
  scales\dash{}iridescent \& smooth \\
  (so like human fingernails) \\
  polished abalone \\
  cold grey flesh slicked with mucus 
  
  locked \\
  into a salty stare \\
  unblinking\dash{}that lovers' gaze \\
  tears frozen \\
  like crystals of ice \\
  distant \\
  and hidden \\
  beneath the promise \\
  of a kiss\dash{}blue lips \\
  the treasure of pearls lie \\
  cold \& mute 
  
  porpoise song \\
  (so erotically mournful) \\
  flows a liquid harmony \\
  \dash{}unheard\dash{} \\
  by their shell-like ears \\
  \& washed upon a sandy tomb 
  
  whisper \\
  timeless Ocean 
  
  silence\dash{}so final
}





\subsubsection{\Criseis{} disbelieves}
\Criseis{} is \skeptical. 
She belives that, at best, \Triestessakhin{} is locked away in permanent sleep and with no consciousness. 
More likely, she is forever dead and destroyed and exists only as an echo of her soul imprinted in the glaive, and as a memory in \ps{\Ishnaruchaefir} mind. 
That is why \Criseis{} is so worried whenever her master talks to the glaive: 
She feels he is indulging his mad delusion that \Triestessakhin{} is still alive, and fears that dwelling on it will only cause him to spiral further into madness and denial. 









\subsection{Personality}
\target{Rystessakhin's personality}
\Rystessakhin was something of a \quo{\trope{HighQueen}{High Queen}}.
She was noble and loving and good and loved all creatures.
When \Ishnaruchaefir preferred to look outward into the universe and had little interest in the affairs of mortals, \Rystessakhin looked inward and down. 
She cared deeply for the lesser creatures and was always advocating that the \dragons be more humane towards them.

Later, \Ishnaruchaefir, after killing her, would \hr{Ishnaruchaefir's compassion}{feel some obligation towards carrying on her will}. 









\subsection{Skills and powers}





\subsubsection{Not a \shaeeroth}
\Rystessakhin never became a \shaeeroth. 
But at the time of the \secondbanewar, she was probably the most highly respected and admired of all non-\shaeeroth \dragons. 















\section{Candrazor}
\target{Candrazor}
\index{Candrazor}
%\sectioncharunspec{Candrazor}{\dragon}{\male}
An ancient \dragon{} and a Sentinel Lord Questor. 
Another of his true names is Rochydanoss. 
He has a number of secret identities, including Vlad \Ceskav{} in \Zarwec{} and Lord Cassander. 










\subsection{Cassander}
%\sectioncharunspec{Cassander}{\human}{\male}
A \human{} lord and sorcerer living somewhere in \Velcad{}. Actually the \dragon{} Candrazor in disguise. 

He looks much like the Count of Monte Cristo as portrayed in the anime \cite{Anime:Gankutsuou}. 









\subsection[Vlad Ceskav]{Vlad \Ceskav}
%\sectioncharunspec{Vlad \Ceskav}{\scatha}{\male}
A \scatha, a \Zarweci{} lord and archmage. Actually the \dragon{} Candrazor in disguise.















\section[Cryocas Nzessuacrith]{\CryocasNzessuacrith}
\target{Nzessuacrith}
\index{\CryocasNzessuacrith}
\index{\Nzessuacrith}
A \dragon, the daughter of \QuessanthIshnaruchaefir and \AeocrithRystessakhin.









\subsection{History}





\subsubsection{\Takestsha}
\target{Takestsha}
\index{\Takestsha}
\Takestsha, ostensibly a \human{} woman, is \Nzessuacrith{} in disguise. Allegedely, she is a mage on the run from her masters because she discovered something she was not supposed to learn. She has manipulated \hs{Morgan Runger} into \hs{Pelidor-Runger war}{waging war against Pelidor}. 

As part of \hr{Secherdamon wants Nithdornazsh}{\ps{\Secherdamon} \Nithdornazsh gambit}, \Takestsha led a \hr{Tantor's journal}{Rungeran expedition} to \hr{Eresh-Kal}{\EreshKal}.





\subsubsection{Sex with \Jirad Tantor}
\target{Takestsha's reasons for having sex with Tantor}
On the \EreshKal expedition, \Takestsha \hr{Tantor has sex with Takestsha}{had sex} with \hr{Jirad Tantor}{\Jirad Tantor} after Tantor's son had been killed. 
\Nzessuacrith{} had a number of reasons for doing this: 

\begin{enumerate}
  \item 
    Boredom. 
    He's not a good lover, but it's better than twiddling her thumbs. 
    Remember, \dragons{} have no sexual taboos. 
  \item 
    It ties him closer to her, and she wants to have the Rungeran \ishrah{} tied to her. 
  \item
    She felt sorry for him after his son died. 
    Given her own bad relationship with her father (\Ishnaruchaefir), she has a weakness for such parent-child relationships, so seeing his grief toucher her heart, and she wanted to make it up to him. 
  \item
    The \quo{leaked diary} plot was actually not formulated at this point. 
    It was only later (or perhaps during the act) that \Nzessuacrith{} came up with the idea to kill two flies with one swat and use his diary to leak the information they wanted leaked to the Pelidorians. 
\end{enumerate}









\subsection{Names}
\Cryocas was her egg-name.
\Nzessuacrith was her adult name. 

To the \Ortaicans she was known as the \taortha \hr{Usherain}{\Usherain}. 









\subsection{Personality}





\subsubsection{Art}
\Nzessuacrith{} enjoys the arts and actively pursues them. 
She has written a great wealth of poetry and literature, including some long epic poems. 
Some suspect her of being the true identity of \hr{Melcryth}{\Melcryth}, the author of \emph{\hr{Wanderers in Darkness}{\WanderersInDarkness}}, but she denies this. 





\subsubsection{Likes beautiful things}
\target{Nzessuacrith likes beauty}
\Nzessuacrith{} kind of likes beautiful things, such as buildings. 
She has no compunctions against killing any number of people, but she hates destroying beautiful buildings and will go out of her way to avoid it.





\subsubsection{Sex}
She has worked undercover a lot and has had plenty of sex. 
She dislikes sex with \human{} men, because they tend to want a submissive woman, and submissiveness, needless to say, goes against her \draconic{} nature. 
She resents \hr{Morgan has sex with Takestsha}{being forced to have sex} with \hs{Morgan Runger}, who is bad in bed. 

Even so, it's not hard for her to do it. 
\Dragons{} have no sexual taboos, so having sex is no more controversial than engaging in conversation. 










\subsection{Physique}
In her true, \draconian{} form, she has scales shining the \colour of steel or silver. 
She has a vertical \quo{comb} of horns on her head and a sail that runs down the length of her spine. 
Maybe she has fins on her head. 









\subsection{Politics}





\subsubsection{\Vizsherioch}
\Nzessuacrith{} does not quite trust \hr{Vizsherioch}{\Vizsherioch}. 
She fears him. 
She is unnerved whenever he calls her \quo{cousin} (even though they \emph{are} cousins). 
He is soft-spoken and polite (by \draconic{} standards), and this just makes him even more creepy and sardonic. 
There is something very sinister within him. 

There is too much alien \xs{} nature in him. 
She fears that \Secherdamon{} has gone too far in breeding and moulding \Vizsherioch, and \hr{Nzessuacrith fears for Secherdamon's sanity}{she fears for \ps{\Secherdamon} sanity}. 

\Vizsherioch{} \hr{humanoid Vizsherioch}{often appears in humanoid form}. 
\Nzessuacrith{} believes this is to masquerade as a harmless \scatha{} and hide his true, horrible power. 









\subsection{Skills and powers}
\target{Nzessuacrith's stealth}
\Nzessuacrith was a master of stealth.
She had studied it idly even before the \secondbanewar, and after that she had spent thousands of years perfecting it. 
She could hide herself better than any other immortal, or so she claimed.

Because she was so stealthy, she was also very skilled at detecting others.









\subsection{\Usherain}
\target{Usherain}
\Nzessuacrith was a goddess of the \hr{Taortha}{\taortha} pantheon of \hr{Ortaica}{\Ortaica}. 
Here she was known under the name \Usherain. 
She held a portfolio overlapping with that of \hr{Nasshikerr}{\Nasshikerr}. 















\section{\Laccashyth}
A female \dragon. 
A \shaeeroth. 














\section{\Rathyon}
\target{Rathyon}
\index{\Rathyon}
A \dragon. 
The youngest of \ps{\QuessanthIshnaruchaefir} three grandchildren. 









\subsection{Arsenal}
\subsubsection{Fast flyer}
\Rathyon{} was a fast flyer. 
The fastest \dragon{} alive, it was claimed. 









\subsection{History}
He was an egg during the (end of the) \Secondbanewar{} and the \SecondShrouding{}, where his parents both perished. 
\Ishnaruchaefir{} took him in. 

He was already named before he hatched. 
His mother had given him an egg-name. 

\Criseis{} was almost like a mother to him. 

















\section{\Tentocoth}
\target{Tentocoth}
\index{\Tentocoth}
A \dragon. 
The middle of \ps{\QuessanthIshnaruchaefir} three grandchildren. 
Younger brother of \Thiencaste, cousin of \Rathyon. 









\subsection{Personality}
\subsubsection{Music}
\Tentocoth{} was a musician and spent much of his free time composing music, \hr{Draconic music}{in the \draconic{} style}. 
















\section{\Thiencaste}
\target{Thiencaste}
\index{\Thiencaste}
A \dragon. 
The oldest of \ps{\QuessanthIshnaruchaefir} three grandchildren. 
Older sister of \Tentocoth, cousin of \Rathyon. 















\section{\Vexstrasshin}
\target{Vexstrasshin}
\index{\Vexstrasshin}
A \dragon{}. 
She was active during \ps{\Semiza} lifetime and tried to stop him and \Thanatzil. 
She succeeded... sort of. 

She was eventually \hr{Vexstrasshin dies}{killed by \ps{\Daggerrain} trickery}. 















\section{\Zessuruch}
\target{Zessuruch}
\index{\Zessuruch}
A young \dragon. 
At the time of the \thirdbanewar{} she was one of the youngest \dragons alive. 









\subsection{History}
She was \hr{Teshrial kills Zessuruch}{killed at one point} by \hr{Teshrial}{\Teshrial} and some other \resphain. 
But not permanently. 









\subsection{Personality}
In her youth, \Zessuruch was quite overconfident. 
After \hr{Teshrial kills Zessuruch}{she was killed in combat}, she became more careful. 























\chapter{Others}















\section{\HesodNerga}
\target{Hesod-Nerga}
\index{\HesodNerga}
An \ophidian{}, the father of \hr{Tiamat}{\Tiamat}.
















\section{\Iurzmacul}
\target{Iurzmacul}
\index{\Iurzmacul}
A \nagalord. 
Once an ally of \TyarithXserasshana. 

He might be the father of \hr{Ishnaruchaefir}{\Ishnaruchaefir}.



















\section{Maegon}
\target{Maegon}
\index{Maegon}
A \nagalord. 
One of the better known \nagalords, alongside \hr{Iurzmacul}{\Iurzmacul}. 

A sea god, known and feared by many coastal dwellers. 
Even the powerful sea god \hs{Shellagh} fears Maegon. 

He is worshipped in the Imetrium. 

Compare him to Dagon from \HPLovecraft's Cthulhu Mythos. 















\section{Salacar}
\target{Salacar}
\index{Salacar}
Salacar is the chief god of the Imetric Tribunal and the ruler of the Imetrium. 
\also{Imetrium, Tribunal}









\subsection{Physique}
In his true form, Salacar is a giant \nagalord. 
He often takes the form of a \scatha. 
In all forms, his scales are glittering emerald green. 































\part{\Resphain}























\chapter{\CiriathSepher}















\section{\Azraid}
\target{Azraid}
\target{Gevural}
\target{Gepheral}
\index{\Azraid}
\index{\Gevural}
\Azraid was a \sathariah \resphan and one of the original \hr{Resphan rebellion}{\resphan{} rebels}. 
He was the \hr{High Lord of Kiriath-Sepher}{High Lord} of \hr{CS}{\CiriathSepher}, and a Cabalist of the \hr{Cabalist circles}{first circle}. 









\subsection{Equipment}





\subsubsection{Spear}
Maybe he wields a spear. 

\lyricsbalsagoth{When Rides the Scion of the Storms}{
  I see him... \\
  grim and noble astride his great winged steed, \\
  gleaming spear crackling in his grasp, \\
  beckoning me onwards to the next life... \\
  to ever more slaughter and carnage... \\
  Yes, a dour and brooding spirit he is, \\
  and in his burning eyes I see \\
  a great secret which I must discover,\\ 
  a powerful mystery I alone must solve.
}










\subsection{History}





\subsubsection{Name}
\Azraid was originally named \Gevural.
He forsook that name and took instead the name \Azraid, which means... something... in the High \Resphan{} tongue. 
(His two names are inspired, respectively, by the \Sephirah{} Geburah/Gevurah and the angel Azrael.)






\subsubsection{Priest}
Before the \hs{Delving}, young \Gevural{} was a cleric-in-training. 
He was very much into philosophy and theology. 

He was extremely intrigued when they discovered \Semiza{} and heard his story. 
\Gevural{} became just as obsessed with the darkness and the \banes{} as \Zachirah. 
But unlike \Zachirah, \Gevural{} was always very rational and critical. 

\citebandsong{DeathspellOmega:Kenose}{%
  DeathspellOmega
}{
  \Kenose
}{
  The pursuit of perversity, is it not but a mask\\
  on the search for meaning and knowledge?
}





\subsubsection{Remained loyal}
After the \hr{Shrouding}{\Shrouding}, the \resphain{} \hr{Dynasties split}{splintered into dynasties}, and many fled from the alliance with the \banelords. 
\Azraid{} was one of few \resphan{} lords who remained \quo{loyal}. 





\subsubsection{\NeoResphain}
\Azraid{} was experimenting to create the new \hr{Neo-Resphan}{\neoresphain}: 
A stronger, more pure breed of \resphan; a superior merger of Chaotic and Entropic power. 

\target{Azraid never became Neo-Resphan}
\Azraid led and oversaw the \neoresphan project, but he never underwent the treatment himself. 
He took no chances. 
His life was too valuable to risk.
He needed to be there to stop the \banes.
\Azraid was very unlike \Sethicus, who was a brave pioneer who always went further than any other. 









\subsection{Personality}





\subsubsection{Anti-\bane{} monologue}
\begin{prose}
  \ta{%
    The \banes{} call us their heirs, but they have never let us inherit anything. 
    They still use us as their servitors. 
    This is not betrayal. 
    I simply take what I was promised, what I am rightfully owed. 
    
    Despite how far we have come, we are still bound by the \banes{} and the Entropy and inevitable decline that they embody. 
    We must free ourselves from the \banelords{} in order to fullfill our potential. 
    In order to rise above their fate... and their damnation. 
    
    As for the \Voidbringer... I do not claim to understand its motives, but I very much doubt that the well-being of our people is a high priority in its mind.
    Ergo we must free ourselves from it.}
  
  \new
  \ta{%
    Of all \resphain, bar \Thanatzil, I stand closest to the \banes. 
    But I do not understand them.
    And that scares me.}
\end{prose}

\citebandsong{DeathspellOmega:FasIteMaledictiinIgnemAeternum}{%
  DeathspellOmega
}{
  A Chore for the Lost
}{
  God of terror, \\
  very low dost thou bring us, \\
  very low hast thou brought us...
}





\subsubsection{He is \ps{\Daggerrain}{} blind spot}
\Azraid{} is \hr{Daggerrain's blind spot}{\ps{\Daggerrain}{} blind spot}:
\Daggerrain{} saw weakness and transparency of the \nephilim, and later that of \humans, and so he underestimated the magnitude of raging inner conflicts that a \resphan{} can endure and still manage to conceal. 
(\Dragons{} do not hide their conflicts quite so well, because of their \chaotic{} nature.) 

The point is that \Daggerrain{} knew that Ramiel could not be trusted, and therefore took precautions against any betrayal by him. 
But \Daggerrain{} had never doubted \Azraid, the \psp{\banelords}{} most favoured and beloved son. 
\Azraid{} seemed almost a \banelord{} with a \human-like face, but he turned out to be much more treacherous, \chaotic{} and unpredictable than expected. 







\subsubsection{Obsession and pragmatism}
\Azraid{} is driven by a craving to understand and master Darkness... his own inner Darkness. 

He learns to love darkness, death, decay, pain and horror. He is a great artist, poet, philosopher and explorer.

\lyricsbalsagoth{The Power Cosmic: Epilogue}{
  Such power as was wielded by Zurra corrupted his heart, master.\\
  His quest for the Lexicon was not a desire born of the eternal search for cosmic enlightenment, but rather of a vain hope that such elucidation would allow him to understand the horrors which blighted his own immortal soul...
}

Among other things, he is obsessed with death. 
He is a necrophiliac and has built strange museums (musea?) filled with the dead, dying and undead. 
His palace is full of bones, corpses and pools of blood. 

He is fascinated by \Erebos{} and its decay and death... \hr{Erebos undead}{perhaps undeath}?

He seeks insight by challenging all standards of ethics and aesthetics and exploring all the forbidden alternatives.

As he likes to say: 
\ta{Beauty... and hideousness. In their exploration lies great wisdom.}

He admires \hr{Khoth-Sell}{\KhothSell} and her sons, because they, in some sense, embody death. 

\target{Azraid's pragmatism}
On the other hand, \Azraid{} can very pragmatic when it comes to other things. He does not really care about ideology, nor the \bane{} legacy, nor the war between \banes{} and \dragons. He only really cares about his own philosophy, and everything else is a chore to serve that end. This is why he \hr{Azraid adopts Merkyran imagery}{allowed \CiriathSepher{} to degenerate into something resembling \Merkyrah}, to which other \resphan{} lords objected. 

\lyricslimbonicart{The Black Hearts Nirvana}{
  Hallowed be the darkness that coronates my soul.\\
  Deep within its shelter I seek my highest goals.\\
  I shall release what is conquered \\
  from which that I now possess.\\
  All lifeforce is abandoned \\
  into the arms of death.
  
  Beyond the great dark adventures, \\
  in streams from the vast mysteries,\\
  limbonic spheres enclose me. \\
  My star is the death of memories.
}

\lyricslimbonicart{In Abhorrence Dementia}{
  I admire the spiritual force of evil:\\
  A pure supreme instinction in survival.\\
  Never underestimate the powers of hatred\\
  when the blackness overwhelming.
  
  With a hostile image against all living,\\
  the splendid visions of malignant breeding.
}

It is actually not just perversions, but a part of his millennia-spanning plan to overthrow the \banelords.

\lyricslimbonicart{The Black Hearts Nirvana}{
  Through lifetime I've reached for the candle\\
  in search for the legends of time.\\
  Cause how many moments isn't a century\\
  when everything dies behind the eyes?\\
  Cause how many moments isn't a century\\
  when everything inside just dies?
}

\lyricslimbonicart{Unleashed From Hell}{
  My art is a reflection, \\
  a mirror of tormented images.\\
  A labyrinth of morbid minds. \\
  Cathedral halls of stone.\\
  Abysmal ruins.
  
  Walking a path of putrefied flesh\\
  in the garden of rotting sculptures.\\
  Everything dwells in an aura of death\\
  and the presence of nocturnal vultures.
}







\subsubsection{Politeness}
\Azraid{} was mostly very calm and polite. 
Unlike many \resphain, he rarely resorted to insults and direct threats. 
He would be very nice and friendly to other \resphain{} and calmly \quo{request} things of them instead of rudely ordering them to.

Of course, everyone knew how powerful he was and how much he could fuck them over if they refused him. 
But even so, his good manners made him somewhat more likable, and he gained fewer enemies than he might have had he been more brutal. 





\subsubsection{Sex}
\Azraid{} does not have as much sex as most \resphain{}. 
Sex is life, after all, and \Azraid{} is more preoccupied with death. 

When he does have sex, necrophilia is among his perversions. 
Also sado-masochism and vore, but those are pretty common among the \resphain.  

He still has sex regularly, though. 
As the High Lord and the greatest of all \resphain, it is his duty to do his best to breed. 
But he has little success. 
He is infertile. 
His transformation into a freak with an \hr{dead hand}{evil hand} has harmed his fertility. 

He did manage to sire \ps{\Zereth} (\emph{after} he had become a freak), but that was it. 
He has never been able to produce another child. 

But he keeps trying. 
Not because he enjoys it. 
Indeed, he is quite bored with sex and has more important things to consider. 
But because it is his duty. 
He does want more children, after all. 

Once every two or three nights, he has a \resvil{} sent to his bed. 
There is no end to the number of \resviel{} willing to stand in line and wait to be allowed to have sight with the great \Azraid. 
He has servants charged with selecting the best and most fertile of them. 









\subsection{Physique}
\target{Azraid's appearance}
He has long white hair that hangs down in wisps. 
Almost aethereal-looking. 
Or like a spider's web. 
His feathers are a very pale gray. 
His skin is a bit pale, too.

He bears a white cloak that hides and covers his \hr{dead hand}{dead left hand}. 

He seems to shine with a light that somehow Shrouds everything else in darkness, making \Azraid{} the only thing that is bright and visible. 

He is not tall, only 200 cm, but his aura and charisma make him look taller. 

He looks weird. 
His skin has wrinkles, like the ones \nephilim{} and \humans{} are prone to get when they grow old. 
\Resphain{} don't develop wrinkles under natural circumstances. 
It is a side effect of the transformation that also gave him his \hr{dead hand}{evil hand}. 
The wrinkles are freakish, but somehow they don't make him seem like a weak mortal. 
Rather, they convey a sort of otherworldly dignity and authority. 
It is a proof of his macrocosmic wisdom and experience. 





\subsubsection{Dead hand}
\target{dead hand}
%He has a rotten, dead
His left hand is rotten and dead, and oversized like a bird's talon. 
Within it is contained much of his arcane power, which is based on death and decay. 
Compare with \hr{Entropy}{the \ps{\banes}{} Entropy}.
It is sickly brown in \colour. 

\Azraid keeps the hand hidden at most times so he can unveil it for greater effect at dramatically opportune moments. 

Remember to have a cool scene somewhere in the background story. 
\Azraid{} is empowered by an occult ritual and gains his dead hand. 
He also gains power over \hs{wraiths}. 









\subsection{Politics}





\subsubsection{\Banes}
\target{Azraid hates Banes}
\target{Azraid plots against Banes}
Ostensibly, \Azraid{} is very loyal to the \banes. 
He has resisted falling prey to \hr{Curse}{\NexagglachelsCurse}. 

But unbeknownst to everyone, \Azraid{} has secretly hated the \banelords{} and plotted against them for thousands of years, ever since they manipulated him into \hr{Azraid kills Damiarch}{killing his brother, \Damiarch}. 

He has spent his entire career planning and preparing so that he might eventually betray the \banes. 
His life and all his obsessions are built up around this. 

He has worked to forge a new power source for their people, independent of \Erebos. 
He intends to usurp and conquer \Nyx{} and sever it from \Erebos{} forever. 

He fears that the \banes{} will send another invasion force. 
But he hopes that if he free \Miith{} of \Iquin{} and \Daggerrain, then his people will have couple hundred or thousand years to evolve and grow stronger, so they will be able to stand against a new invasion. 

Near the end of \SentinelsofMiithEmph, it turns out that \Azraid{} has been \hr{Azraid stockpiles Erebean power}{stockpiling \Erebean{} power} for himself.





\subsubsection{\CiriathSepher}
\target{Azraid and Ciriath-Sepher}
The \CiriathSepher did not always love \Azraid.
But he was an extremely powerful and skilled \sathariah. 
Even the other dynasties acknowledged that and respected and even feared \Azraid for it.
So the \CiriathSepher were glad to have \Azraid.
He gave them prestige in front of the other dynasties.

\target{Azraid obscure}
\ps{\Azraid} role in the dynasty was rather obscure and unclear. 
This was intentional on his part. 
He liked to keep to the shadows and only exercise his power subtly. 
He played the part of the absent philosopher and artist rather than the active ruler. 

This had several advantages:

\begin{enumerate}
  \item 
    That way, he did not appear as a tyrant or a threat to anyone, so he made fewer enemies. 
  \item 
    By staying out of public sight he ensured that his enemies had little hold on him. 
    Little was known of his comings and goings, nor of his strengths and weaknesses, so those who wanted to attack him did not know how to get at him. 
\end{enumerate}

This \quo{keeping to the shadows} would never have worked in \Mystraacht. 
They would quickly lose respect for him and depose him. 
But it worked in \CiriathSepher{} because of his great social skills, the way he was able to pull strings and exercise subtle influence, and constantly keep up an image of mysterious power and allure. 
The sophisticated and artistic \CiriathSepher{} respected him for it. 





\subsubsection{Family}
\Azraid{} had one child: \hr{Zereth}{\Zereth}. 
He did not like to share power and was hesitant to spread his lifeforce around. 







\subsubsection{\Nexagglachel}
\Azraid{} was always the one of the \satharioth{} who understood \hr{Nexagglachel}{\Nexagglachel} the best. 
To this day he hears the voice of \Nexagglachel{} in his head. 
And to this day he needs to remind himself that it is not real. 
\Nexagglachel{} is dead. 
\Azraid{} is talking to himself, to a fragment of \ps{\Nexagglachel}{} memory and will that lingers on as a ghost in his mind ever since he \hr{Fragments of Nexagglachel}{consumed his brain}. 









\subsection{Skills and powers}





\subsubsection{High Telepath}
\Azraid{} is a highly skilled \hs{High Telepath}. 















\section[Firaxel]{\Firaxel}
\target{Firaxel}
\index{\Firaxel}
A \ketheran{} \resvil. 
\Teshrial{} was infatuated with her and wanted to have sex with her. 









\subsection{Physique}
She painted her face: 
Wavy white lines, like light or fire radiating out from her lovely, beautiful face. 

Sometimes she would treat her hair with some stuff so it would rise up in the air and then fall down, like a fountain. 

Her skin would be polished to blank onyx. 

Her hair and feathers were reddish brown.
She would often dye it purple. 
She disliked her natural \colour. 

She dressed in blue and violet in a manner reminiscent of and incorporating influences of \TiphredSerah. 
She was, after all, partially of \TiphredSerah{} descent. 









\subsection{History}





\subsubsection{Scientific career}
\target{Firaxel is a scientist}
\Firaxel{} is a scientist and researcher. 









\subsection{Politics}
\subsubsection{Family}
\Firaxel{} was descended from \hr{Shehizol}{\Shehizol} and \hr{Quelthah}{\Quelthah}, both high-ass \ketherain.
She was high status. 















\section{\Ganethed}
\target{Ganethed}
\index{\Ganethed}
A \thelyad{} \resphan{} of \CiriathSepher{} and a Cabalist of the \teshrialcircle circle. 
A colleague and rival of \Teshrial. 









\subsection{Physique}
\index{beard!\Ganethed}
\Ganethed{} was quite broad and stocky and powerfully built\dash a throwback to a more \nephil-like body form. 
Strong and muscular, but slightly fat. 
He wore a beard, unusual for \resphain. 

He wore his hair short. 

His hair and feathers were brown. 
He dyed them golden.









\subsection{Politics}





\subsubsection{\Urizeth}
\Ganethed was a kinsman of \hr{Urizeth}{\Urizeth}.















\section{\Harbeth}
\target{Harbeth}
\index{\Harbeth}
\Harbeth, called \quo{the Raven of the Battlefields}, was a \resphan{} lord, a \sathariah. 
He was charged with binding and \quo{safeguarding} the spirits of the fallen dead, protecting them from foreign powers who would steal them (such as the \hs{Worm Cult reapers}). 

\meta{%
  His name, title and image is inspired by the \Cabbalistic \Qliphah{} Hareb-Serap, and especially the portrayal of him in the RPG \emph{Kult}.} 









\subsection{Politics}
\target{Harbeth is Azraid's heir}
In the \hr{CS order of succession}{\CiriathSepher{} order of succession}, \Harbeth{} was \ps{\Azraid} heir apparent. 
But \Harbeth{} was a brutal and scary hawk. 
Unpopular but feared. 
No one wanted him to be High Lord. 
So they preferred \Azraid. 
Even those who hated \Azraid{} tended to figure \quo{better the devil you know...}. 

This was a deliberate ploy from \ps{\Azraid} side. 
\Harbeth{} was a close and trusted ally of his. 









\subsection{Physique}
\Harbeth{} was extremely tall\dash 260 cm, one of the tallest of all \resphain.
He was also broad-shouldered.
But very gaunt, almost skeletal. 
He looked kind of like a living corpse. 

His hair and feathers were a drab gray.
He never dyed them.















\section{\Jeshred}
\target{Jeshred}
\index{\Jeshred} 
A \thelyad{} \resvil{} of \CiriathSepher.
Mother of \Zereth. 















\section{\Mehaloch}
\target{Mehaloch}
\index{\Mehaloch}
A \sathariah of \CiriathSepher. 
\Mehaloch was a cruel \quo{devourer}.
He killed and destroyed a lot in his hunger.
He was killed early on, during the \secondbanewar.
Compare to Darth Nihilus from \cite{VideoGame:KOTORII}.















\section{\Menessiaraid}
\target{Menessiaraid}
\index{\Menessiaraid}
A \ketheran{} \resphan{} of \CiriathSepher. 
He was a friend of \hr{Teshrial}{\Teshrial}; an older, more experienced \emph{sempai}-type. 
He had once been \ps{\Teshrial} teacher. 









\subsection{Physique}
\Menessiaraid{} was bald. 
He was as tall as \Teshrial, but more gaunt and less attractive. 

His feathers were pure black.
He dyed them with patterns of white amid the black. 









\subsection{Personality}





\subsubsection{Religions}
\target{Menessiaraid creates religions}
\Menessiaraid worked on fabricating and maintaining religions, dogma and mythology. 
But not on \Azmith. 
Have references to this. 
Make the reader be critical of religions. 















\section{\Shehizol}
\target{Shehizol}
\index{\Shehizol}
A \resvil{} of \CiriathSepher, a \sathariah. 
Ancestor of \hr{Firaxel}{\Firaxel}. 















\section{\Teshrial}
\target{Teshrial}
\index{\Teshrial}
A \resphan{} lord of the \CiriathSepher. 

He is the superior of \Achsah{} and Charcoal and oversees many affairs in \Malcur, \Scyrum{} and their general area. He is a Cabalist of the \hr{Cabalist circles}{third circle}. 








\subsection{Physique}





\subsubsection{Appearance}
\target{Teshrial's appearance}
\Teshrial{} had black skin, but he dressed in white.
His hair and feathers were a rather light gray.
He dyed them pure white. 

Maybe he wore all sorts of golden jewelry. 

%He carries a sceptre-like mace or morning star.
He carries a short rod that resembles a sceptre or mace. It can be used as such, but it is actually a \hr{Resphan technology}{high-tech weapon}, a gun of sorts. 

He wears a beautiful, crown-like circlet and a white robe adorned with gold, silver, obsidian, jade, onyx and so forth. 

\target{Teshrial is androgynous}
He has a somewhat androgynous \quo{\bishounen{} look}

His eyes are bright purple, close to pink. 
They are pretty, but he finds them a bit too effeminate. 
Maybe he wears some makeup to try and make the \colour look better. 









\subsection{Equipment}




\subsubsection{Crystal \armour}
\Teshrial{} wears \armour and wields weapons \hr{Resphan crystals}{made of crystal}. 





\subsubsection{\Ruishagh: His demesne and manor}
\target{Ruishagh}
\index{\Ruishagh}
\Teshrial{} owns the estate of \Ruishagh. 
It has a central house, \Ruishagh{} Manor. 





\subsubsection{Weapons}
\target{Turishah}
\index{\Turishah}
At first, \Teshrial{} wielded the \hr{Senaan}{\senaan} \Turishah, forged specifically for him. 
It had a silvery blade. 
But this weapon was destroyed by \QuessanthIshnaruchaefir{} in his first battle with \Teshrial. 

\target{Ossiraith}
\index{\Ossiraith}
Later \Teshrial{} was entrusted with \Ossiraith, previously wielded by \Menessiaraid. 
\Ossiraith{} was green. 









\subsection{History}
\target{Teshrial's history}





\subsubsection{Origin}
\Teshrial{} was a \ketheran. 
\hr{Teshrial's family}{His family} were some very high-ranked \resphain. 

He is much younger than \hr{Achsah}{\Achsah}, but higher ranked in the Cabal than she. 
She \hr{Achsah hates Teshrial}{hates him for it}.





\subsubsection{Kills \Zessuruch}
\target{Teshrial kills Zessuruch}
At some point before the beginning of \emph{\TwilightAngelRemember{}}, \Teshrial and some of his brethren battled the \dragon \hr{Zessuruch}{\Zessuruch} and slew her. 
Not permanently, though. 

This was a great accomplishment for \Teshrial, who proudly styled himself \quo{\Dragon-slayer} after this. 






\subsubsection{Unfinished business}
\target{Teshrial's unfinished business}
In the years before the \thirdbanewar, \Teshrial wooed \hr{Firaxel}{\Firaxel}. 
He wanted to score her and have children with her. 
He gradually fell more in love with her and became more obsessed. 

\Firaxel showed signs of interest. 
They even kissed and cuddled. 
But \Firaxel was unwilling to make any promises.
She was not sure she wanted children with him. 

There was a rival whom \Firaxel was having sex with.
\Teshrial was worried about him.
He hoped that if he defeated \Ishnaruchaefir and advanced science in the process, \Firaxel would leave the rival and have children with \Teshrial.
He tried to coax some kind of reassurance out of \Firaxel, but she remained \blase and would promise nothing.
At the end, \Teshrial was hopeful but not certain. 

This is a dramatic reason for this:
If I want the reader to miss \Teshrial when he dies, he must have some unfinished business.
That way, there is tension because the reader expects that \Teshrial's life story will continue beyond his duel with \Ishnaruchaefir, because he has other conflicts he still needs to resolve. 
So it will be more of a surprise when he dies. 

Also, \Teshrial should have a personal development that is incomplete. 
He is slowly becoming less of a snob.
He learns to appreciate \Urizeth's friendship and even respects \Achsah. 





\subsubsection{Death}
Near the end of \emph{\TwilightAngelRemember{}}, \Teshrial{} was \hr{Ishnaruchaefir kills Teshrial}{killed and destroyed by \Ishnaruchaefir}.









\subsection{Personality}





\subsubsection{Culture}
\Teshrial{} knows about \hr{Bloodwine}{bloodwine} because he is supposed to as a noble. 
In secret, he prefers drinking raw blood from the vein. 
He knows this is uncultured, so he keeps it secret. 





\subsubsection{Motivation}
\hr{Teshrial's family}{His family} are some very high-ranked \resphain. 
As such, he has great pressure upon him to do great deeds. 
He wants to prove to the world that he is great, just as great as his family. 
This makes him zealous and impatient and overconfident. 

But he has also trained all his life and is psycho-skilled. 
The battle with \Ishnaruchaefir{} means everything to him. 
He cannot and must not lose. 

Make it clear that \ps{\Teshrial} motivation for fighting \Ishnaruchaefir{} and wooing \Firaxel{} is not purely selfish. 
He has many reasons for doing what he does:

\begin{enumerate}
  \item 
    He wants to gain sexiness in the eyes of the \resviel. 
    \Firaxel{} first of all.
  \item 
    He wants to win glory for himself and gain status and rank in the dynasty and the admiration of his fellow \resphain. 
  \item 
    He wants revenge for his humiliating defeat. 
  \item 
    But he also wants to do heroic things because he genuinely thinks they are right. 
    He wants to destroy \Ishnaruchaefir{} not just out of a personal vendetta, but because he is an evil menace\dash so evil and insane that even his own people, even his brother and daughter, revile him as anathema. 
    He wants to rid the world of the evil Destroyer, who has caused the Resphain and everyone else so much grief and harm. 
    And \Teshrial{} wants to avenge the many victims (warriors and civilians, mortal and immortal alike) whom the wicked Exile has slain over the millennia. 
  \item 
    He wants to have children, not just for his own vanity and legacy's sake, but because his race needs children. 
    He genuinely thinks he has good genes, and it is his duty to carry them on. 
\end{enumerate}

His hatred of \Ishnaruchaefir is as exhilarating and pleasurable as is his love for \Firaxel. 
Remember, \hr{Races love war}{violence is pleasure}.





\subsubsection{Sexuality and procreation}
\target{Teshrial's virginity}
\Teshrial{} has had sex with many people; male and female, \ketherain and \thelyadeth{}, \humans{} and \nephilim. 
Even an \ashenblood{} \resvil{} once. 
(He is ashamed of that last one and hides it, hoping no one will find out.)

But never a truly high-quality \ketheran{} \resvil. 
And \quo{high-quality} means \emph{fertile}. 
All the \resviel{} he has fucked have been infertile. 

\target{Teshrial wants children}
That pains him. 
He wants children. 
He hasn't been able to sire any yet, but through no fault of his own. 
He is certain that his seed is powerful and can sire many \resphan{} children. 
Strong children. 
\Ketheran{} children. 
He is sure he would be a great father, too. 
He would love his children and bring them up to achieve great and wonderful things. 

He is certain it is not his fault. 
He is fertile. 
He has impregnated several \human{} women (and eaten them) in his short life. 

But all the \resviel{} he has fucked have been barren. 
He knows because he has researched them. 
Of all those he has fucked, none has ever given birth, or even conceived. 
It is their fault. 
He is sure it is. 
\begin{prose}
\tho{Bah. 
  Fucking an infertile \resvil{} is just one step up from masturbating.} 
\end{prose}

\ps{\Teshrial} seed deserves better. 
He deserves someone great. 
A \ketheran, and one as talented and beautiful as he. 
He deserves \hr{Firaxel}{\Firaxel}. 
She is a miracle. 
Fertile. 
Full of life. 
She has given birth twice already. 
That is what makes her so sexy and irresistible. 
It also means she has suitors everywhere. 
\Teshrial{} knows this. 
He must do something great to be worthy of her. 







\subsection{Politics}
\subsubsection{Family}
\target{Teshrial's family}
\ps{\Teshrial} were some very high-ranked \resphain. 
His father was \hr{Tuerdal}{\Tuerdal}, a hero of the \hr{Kezeradi War}{\Kezeradi{} War}. 
His mother was \Zereth, the daughter of mighty \hr{Azraid}{\Azraid}. 





\subsubsection{\Ghobaleth}
\Teshrial{} had an ace up his sleeve: 
\hr{Teshrial's creatures}{A group of terrible \ghobaleth{} hiding beneath \Malcur.}

\target{Teshrial fears Noggyaleth}
\Teshrial was afraid of the \noggyaleth. 
Every time \Teshrial saw the \noggyaleth (or even thought of them), he was filled with fear and loathing. 
He was not ashamed to admit he fears them. 
Admit it to himself, that is.
He was still too ashamed to admit it in front of others.

In the \Malcur venture, it was \Urizeth who usually performed the spells to direct the \noggyaleth.
\Teshrial steered clear of learning those spells. 
Ostensibly because he thought it was nerd business and unsuitable for a \resphan like him; a gentleman of the finest breeding and the finest culture, and a master of martial arts, that noblest and most beautiful and artistic of pursuits. 
He used to joke that dealing with slimy, icky things like \noggyaleth was not his style. 
But the truth was that \Teshrial had always been afraid of \noggyaleth and unwilling to admit it. 

Later he was forced to learn the \noggyal spells. 




\subsubsection{Status}
\Teshrial{} is of the finest, most prestigious bloodline (that of \Azraid).
He is young and not highly experienced, but he is a rising star and already held in high regard. 
The dynasty has high hopes for him. 
He is expected to go on to do great heroic deeds in his life. 







\subsection{Skills and powers}





\subsubsection{Fighting skill}
\target{Weapon master Teshrial}
\Teshrial{} was an expert weapons master with both melee weapons, firearms and magical devices. 
He walked the \hs{Path of Ice}. 

\Teshrial was a rare prodigy.
Before his death he was young, but a better warrior than many \resphain much older than he.

\hr{Menessiaraid}{\Menessiaraid} once told him: 
\begin{prose}
  \Menessiaraid:
  \ta{%
    You are better than I, Teshrial. True, I win about half of our matches, but only because we have fought thousands of times and I know you so well. I have some chance to predict your style. No outsider could do that.}
\end{prose}

Compare him to Lucius from the \emph{Horus Heresy} books. 

\ps{\Teshrial} specialty is duelling, ie., single combat. 
Few \resphain{} can match him one-on-one. 

On the other hand, \Teshrial{} is not so great a commander or researcher or philosopher, or an enchanter of items or anything. 
He is also not a great \human{} breeder. 
He is proud of the \humans{} he breeds, but the breeding business is more a hobby than a profession. 

He wields a \hr{Senaan}{\senaan}. 





\subsubsection{Foreign languages}
\target{Teshrial speaks poor Draconic}
\hr{Resphain speak poor Draconic}{Like most \resphain}, \Teshrial spoke \Draconic only poorly. 















\section{\Tuerdal}
\target{Tuerdal}
\index{\Tuerdal}
A \ketheran{} \resphan{} of \CiriathSepher. 
\ps{\Teshrial} father. 
A hero of the \hr{Kezeradi War}{\Kezeradi{} War}. 















\section{\Urizeth}
\target{Urizeth}
\index{\Urizeth}
A \thelyad{} \resvil{} of \CiriathSepher. 
A historian. 

She compiled a big annotated \emph{\hr{Wanderers in Darkness}{\WanderersInDarkness}} that gathered several versions in one and compared them in detail, with analysis and interpretation. 









\subsection{History}





\subsubsection{Against \Ishnaruchaefir}
In \TwilightAngelRememberEmph, \Urizeth agreed to help \Teshrial in his quest to defeat \QuessanthIshnaruchaefir

\Urizeth was \hr{Ishnaruchaefir kills Urizeth}{killed by \Ishnaruchaefir} after he found out. 

\Urizeth's body was not destroyed.
It was retrieved by her family or friends, but it was blasted and burnt and in a bad shape.
When she revived she was hairless and feathless and horribly scarred. 
Her limbs were destroyed. 
She could neither walk nor fly. 

After a while, she regained the use of her hands to some extent.
But she still had to have servants assist her if she wanted to read or anything. 

She gradually recovered. 
Near the end of \TwilightAngelRememberEmph, \Urizeth could walk again, albeit unsteadily.
She had not many feathers yet and could not fly, but that would come.









\subsection{Personality}





\subsubsection{Eccentricity}
\target{Urizeth is eccentric}
Make \Urizeth really eccentric, even disturbingly so.
She is quite deranged after having studied WID and its terrible Aenigmata for so long.
\Teshrial is uncomfortable in her company, but he also knew that she was very knowledgeable and that her dark arts and dark insights would be of invaluable help to him.
After her death and imperfect rebirth she looked even more like a madman. 
Complete with mad cackling.
Maybe.

\Urizeth must have a nicely quirky and eccentric personality.
She is a nerd and does not have \Teshrial's social status or social skills.
She is somewhat mad, but she does have a strong will. 
She is older, wiser and more experienced than he and will not let him push her around.
She tries to be a wise mentor to \Teshrial, but does not succeed.
\Teshrial, on the other hand, knows she is a strange nerd.
He fears to have his reputation tainted by associating with her.
She is not only a \thelyad of TiphredSerah, but also very uncool.
\Teshrial is a huge snob, and so is his circle of friends.
He has to remind himself that working with this freak will pay off in the end.
But later in the story, he develops a genuine respect for \Urizeth, and later still even fondness.
She mourns him when he dies.









\subsection{Politics}





\subsubsection{Cabal}
\target{Urizeth is not a Cabalist}
\Urizeth was not a Cabalist at all. 
She was hired for the \hr{Malcur venture}{\Malcur venture} as an external consultant because of her great occult expertise. 





\subsubsection{\Ganethed}
\Urizeth was a kinswoman of \hr{Ganethed}{\Ganethed}.















\section{\Vesrai}
\target{Vesrai}
\index{\Vesrai}
A \ketheran{} \resvil{} of \CiriathSepher. 
Daughter of \hr{Zereth}{\Zereth}. 
% Mother of \hr{Teshrial}{\Teshrial}. 















\section{\Zereth}
\target{Zereth}
\index{\Zereth}
A \ketheran{} \resvil{} of \CiriathSepher. 
The only child of \hr{Azraid}{\Azraid}.
Her mother was \hr{Jeshred}{\Jeshred}. 

Had a number of children, including \hr{Teshrial}{\Teshrial} (fathered by \hr{Tuerdal}{\Tuerdal}). 

She was one of the runners-up in the \hr{CS order of succession}{\CiriathSepher{} order of succession} (but not the first after \Azraid). 

She was one of her father's closest allies. 
She was not so much a warrior, but an expert in all sorts of social relation management. 
She was very popular in \CiriathSepher{}: 
The princess, loved by all. 























\chapter{\Kezerad}


















\section{\Essenai}
\target{Essenai}
\index{\Essenai}
\Essenai was a \sathariah \resvil of \Kezerad.
She was very wise and full of insight.
As a \sathariah she had much occult, cosmic insight. 

She wrote a translation of \WanderersInDarknessEmph into the \Resphan tongue.

She died in the \hr{Fall of Kezerad}{fall of \Kezerad} and became a \sephirah.

\hr{Sithiyacaan loves Essenai}{\Sithiyacaan loved her} and mourned her greatly.


















\section{\Eryal}
\target{Eryal}
\index{\Eryal}
A \thelyad{} \resvil{} of \Kezerad. 
She became a \malach{} and incarnated as \hr{Silqua}{Silqua \Delaen}. 















\section[Sithiyacan]{\Sithiyacaan}
\target{Sithiyacaan}
\index{\Sithiyacaan}
\target{Last Kezeradi prince}
\Sithiyacaan{} is a \resphan{} lord, a \sathariah{} and the last surviving \hr{Kezerad}{\Kezeradi} prince. 

He is a terribly grim warrior, fighting for good but with brutal means. He is cruel, driven by revenge. 

Compare to Silchas Ruin from \cite{StevenErikson:ReapersGale}.









\subsection{History}
When \hr{Azraid coup}{\Azraid{} does his coup}, \Sithiyacaan{} realizes that the rebels are evil. He speak out, but is put back in his place. He stays and tries to pull the \hr{Kiriath-Sepher founded}{newly-formed \CiriathSepher} in a better direction

He takes part in the slaying of \Nexagglachel{} and becomes a \sathariah. But soon after that, he breaks with the other \resphain{} and goes off to found \Kezerad.

After the fall of \Kezerad, he is extremely bitter and disillusioned. In order to sever his \hs{Kezeradi telepathy}, he has to brutally suppress his emotions and his memories of home. This has forced him to perfect the technique of suppressing his true self and disguising himself as a harmless \human, but it has also driven him somewhat mad. 

He was somehow and to some extent responsible for the fall of \Kezerad, and he feels endless guilt over this. This is part of what drives him mad.

\lyricsbalsagoth{The Scourge of the Fourth Celestial Host}{
  [NORRIN-RADD:]\\
  I am the last scion of Zenn-La,\\
  Never more to embrace Shalla-Bal.\\
  I was born to soar beyond the stars...
  
  The edge of oblivion beckons...\\
  The blood of countless billions stains these silvern hands...\\
  but I must... I will endure!
}

\lyricsbs{Aeternus}{Denial of Salvation}{
  For all the thousands I've killed,\\
  for all the children I've tortured,\\
  for all the souls I've burned,\\
  hear a demon's cry.
  
  The darkened lusts\\
  you carved into me\\
  suffocates the remains\\
  of my human sanity. \\
  Why me? \\
  This I never desired.
  
  See me.\\
  Free me.\\
  Unchain me.
  
  This life eternal\\
  is not what you spoke off.\\
  Why this horned, grim face,\\
  this cold, black body?\\
  I want to be set free.
  
  Save me.\\
  Free me.\\
  Unchain me.
  
  I am a demon.\\
  I serve the wicked.\\
  I'm trapped eternally.\\
  Now I must return\\
  to dimensions of\\
  rage, pain, war, hate.\\
  Where is my god?
}





\subsubsection{Family}
He is blood-kin to \hr{Eryal}{\Eryal}. 
Perhaps even her father. 





\subsubsection{Back in the war}
At some point (during \SentinelsofMithEmph) he resolves to get involved in the \secretwar{} again. 
But he is still mad and ineffectual much of the time. 
In his \human{} guise, he has occasional glimpses of insight and \quo{wise}, \quo{prophetic} outbursts.

As the story progresses, he learns to better control his mind and feelings and sheds his feeble, maddened persona.

He once faces Ramiel in combat and proves his equal.

Compare him to Silchas Ruin from \cite{StevenEriksonIanCameronEsslemont:MalazanBookoftheFallen}. 
But with his madness and split personality, he might be more of a Silchas Ruin, Udinaas, Kettle and Rhulad all wrapped into one.

With his switching between ineffectual and badass forms, he resembles Abel Nightroad (Crusnik 02) in the anime \cite{Anime:TrinityBlood}.









\subsection{Physique}
\Sithiyacaan{} was one of the physically largest \satharioth. 
He was almost as tall as \Harbeth{} and \Zachirah. 

His feathers and hair were blood red.









\subsection{Split personality}
He has developed a split personality of sorts, so that in his guise\dash typically that of a mad old beggar\dash he does not remember more than fragments of his true past. He also tends to babble incoherently and be plagued by nightmares, but very vague ones. 

\target{Sithiyacaan's appearance}
But he still retains his inherent \resphan{} power. Perhaps he is even a \hr{Sathariah}{\sathariah}. In a dire emergency, his memory will return, and he casts off his \human{} shell and transforms into his true form.





\subsubsection{Fights to regain sanity}
\Sithiyacaan{} fights to regain his sanity. 
But he fears to regain his powers, in case his madness should return. 

\citebandsong{Ihsahn:angL}{Ihsahn}{Elevator}{
  All lights disperse\\
  and the devil takes me down.
  
  There is panic in my fascination.\\
  Like soothing wine is my despair.\\
  Gracefully I fall to pieces.
  
  Then lights disperse\\
  and the devil takes me down.
  
  The gears keep turning\\
  and the ropes stretch far\\
  in this world of hopelessness.
}

But he is making progress.

\citebandsong{Ihsahn:angL}{Ihsahn}{Elevator}{
  I have come a long way now.\\
  The fatal riddles beckon me.\\
  I have come a long way now.\\
  A leap from faith and gravity.\\
  I have come a long way now.\\
  To find the nest where treasures sleep.\\
  I have come a long way now.\\
  The fairest lies are hidden in the deep.
}

He still angsts.

\citebandsong{Ihsahn:angL}{Ihsahn}{Elevator}{
  There is vanity in my destruction.\\
  There is mockery in my ordeal.\\
  Indefinite is the course of my decent.
}





\subsubsection{His true form}
His true form is that of a terrible fallen angel, radiating fell power and hatred. His wings are tattered and broken, his features harrowed, his figure emaciated, his face furrowed by lines of shed tears. 

His eyes are dry and dead, all tears of compassion shed and dried millennia ago. Or... maybe he sheds tears of blood\dash in real-time, running down from his eyes. Especially when he's in combat and forced to wield the \nieur-based power that is hateful to him.

Contrast him with \hr{Ramiel's appearance}{Ramiel}, who revels in his \resphan{} glory.

In this form, he is prone to be carried away by rage\dash the rage he otherwise so brutally suppresses\dash and wreak great destruction. When he returns to his senses, he is disgusted by what he has done, and it drives him to repress his true self even further. He hates the wretched horror he has become. 





\subsubsection{Imprisoned within himself}
The true \Sithiyacaan{} is incarcerated in a prison in a repressed section of his own mind. A personal Hell, a mini-Realm within himself. There he lies in torture and hates himself. 

Compare to the Seraph Inarius from the manual to the game \emph{Diablo}. He is imprisoned in a hall of mirrors and forced to gaze upon his own misshapen form for eternity. 

When from time to time he awakens, he is wrapped in chains. His weapons are blades on chains that he swings around. Compare to the {Ghost Rider} from comics and a movie (\cite{Movie:GhostRider}) of that title. 

When he awakens, he tears his \human{} body to pieces, blood and guts spraying everywhere. When he falls asleep again, the Shroud rebuilds his \human{} form, but he's not feeling well. 

He mentally flees from all the suffering, trying to block it out. 
But it still speaks to him. 

\lyricslimbonicart{When Mind and Flesh Depart}{
  A cryptic message comes from the heart\\
  as I see and experience the bleeding art.\\
  The pain is only to avoid\\
  'cause the spirit enters a greater void.
}

Once in a while he lets rip, releasing all the suffering contained within him. 

\lyricslimbonicart{When Mind and Flesh Depart}{
  I cleanse within ascending steams.\\
  Arising shadow, the dark soul releases\\
  all its power.
  
  When mind and flesh depart.
}

\lyricslimbonicart{Infernal Phantom Kingdom}{
  A grim darkened spirit\\
  in a world of woe.\\
  Imprisoned evil beauty\\
  from the cold depths below.\\
  Linger in perpetual dreamstate,\\
  in the grip of a powerful rage.
  
  Summon the oblivion.\\
  Hear demons call from the dungeon.\\
  Light has forever abandoned this land.\\
  Life has forsaken this souls.\\
  Reaching out from the cold.\\
  A dark and hellish void,\\
  beyond the entrance of imagination.
}









\subsection{Skills and powers}





\subsubsection{Power}
\Sithiyacaan{} is one of the mightiest \resphain{} in the world. 
Stronger even than most \satharioth. 
Up there with \hr{Azraid}{\Azraid} and \hs{Ramiel} (even after \hr{Ramiel is overpowered}{Ramiel becomes \uber{} after eating \Belzir}). 

This is because of \ps{\Sithiyacaan}{} \hr{Madness}{madness}, which grants him dark insight and thus dark powers. 
But he is very unstable and only rarely able to utilize his full powers. 
\trope{SanityHasAdvantages}{Sanity Has Advantages}. 





\subsubsection{Vampirism}
Having no feelings of his own (because he represses them so badly), he has learned to steal the emotions of others and feed on them. Recall that \hr{Resphan vampirism}{\resphain{} are vampiric by nature}. This leaves the victim with an uncanny feeling of emptiness, loss and pointlessness. 

In extreme cases, \Sithiyacaan{} over-feeds, draining all emotion from his victims, leaving them mad, crippled or dead. He is prone to doing this in his \quo{fallen angel} form. Perhaps he even drains their soul. 

When he does it in his true form, he tends to smile or laugh and lick his lips. Compare him to the Crusnik from \emph{Trinity Blood}, who also does this.

In rare cases he will use his emotion-draining skill as a weapon.









\subsection{Politics}





\subsubsection{\Essenai}
\target{Sithiyacaan loves Essenai}
\Sithiyacaan loved \Essenai. 
He mourned greatly when she was turned into a loathsome \sephirah. 
Freeing his beloved from her vile prison was a very important motivation for him. 























\chapter[Mystracht]{\Mystraacht}















\section{\Cishiel}
\target{Cishiel}
\index{\Cishiel}
A \ketheran{} \resvil{} of \Mystraacht. 
Daughter of \hs{Ramiel}. 









\subsection{History}





\subsubsection{Youth}
\Cishiel{} was born \emph{after} Ramiel became a \sathariah. 
(This meant that she was a proper \ketheran. Which she would not have been had she been born before he actually became a \sathariah.) 





\subsubsection{Ramiel disappeared}
She was very young when her father went missing as a \malach. 
She herself \hr{Cishiel wanted to be a Malach}{wanted to be a \malach}, but Ramiel forbade it. 

Now that her father was gone, there were rivals who wanted to kill her, since as Ramiel's sole heir she was a contender to the throne of \Mystraacht. 
But she was lucky to have some strong and resourceful family members who protected her while she grew up. 

She grew into a very competent and deadly \resvil. 
When \hr{Ramiel returns to Mystraacht}{Ramiel returned}, she supported him. 
But she did not forgive him for abandoning her when she was a little girl. 





\subsubsection{Cabal rank}
At the time of the \thirdbanewar, \Cishiel was a Cabalist of the \cishielcircle circle. 









\subsection{Physique}




\subsubsection{Appearance}
\Cishiel usually dyed her hair and feathers fiery orange. 









\subsection{Politics}





\subsubsection{Children}
\Cishiel had a son or even two.
They were young and had accomplished little.
The younger (if there were two) was just a child, 100 years old at the time of the \firstbanewar.





\subsubsection{\Eryal}
\target{Cishiel hates Eryal}
\Cishiel{} hated \Eryal. 
She blamed her for driving Ramiel and \Shiaraid{} apart, which lead to their downward spiral and drove them to their fall from grace \malach{} fiasco. 
If it were not for \Eryal, \Cishiel{} would be a highly respected princess of \Mystraacht. 
Instead she had to fought for everything she gained, including her life.





\subsubsection{Mother}
\Cishiel's mother was not \Shiaraid but some other \resvil.
It was in one of those many and long periods where Ramiel and \Shiaraid were not on speaking terms.
\Cishiel's mother was killed in the \resphanwars, after the \malach fiasco but before the inception of the Cabal.





\subsubsection{Ramiel}
Ramiel was \Cishiel's father. 

\ta{Ramiel and Cishiel have sex}{Ramiel and \Cishiel may have had an incestuous relationship}.








\subsection{Skills}





\subsubsection{Martial arts}
\Cishiel walked the \hs{Path of Darkness}.















\section{\Dasteron}
\target{Dasteron}
\index{\Dasteron}
A \ketheran{} \resphan{} of \Mystraacht{} who coveted the throne of the \hs{Overlord}. 
He was \hr{Dasteron dies}{ultimately slain} and \hr{soul-eating}{devoured} by Ramiel. 









\subsection{Physique}
\target{Dasteron's appearance}
\index{beard!\Dasteron}
\Dasteron{} is slightly taller than Ramiel, and somewhat broader, too. 
But Ramiel has a broader wingspan (\hr{Resphan wingspan is important}{which is important}). 

He is bald and bearded, looking a bit like Anton Szandor LaVey. 









\subsection{History}





\subsubsection{Cabal rank}
At the time of the \thirdbanewar, \Dasteron was a Cabalist of the \dasteroncircle circle. 









\subsection{Equipment}





\subsubsection{\Scaleron}
\target{Scaleron}
\index{technology!weaponsmithing}
\Scaleron{} was a \hr{Belthrad}{\belthrad} sword forged and wielded by \Dasteron{} and later taken from him by Ramiel. 
It was red in \colour. 

\ps{\Scaleron} design was inspired by \hr{Ascaril}{\Ascaril}. 
\Dasteron{} had never seen \Ascaril, as \hr{Ascaril destroyed}{it was destroyed} before he was born, but he studied descriptions and depictions of it, and even some of \ps{\hr{Lyorith}{\Lyorith}} old schematics and design notes, which she used to forge \Ascaril. 

But \Dasteron{} believes that \Scaleron{} is superior to \Ascaril. 
It is, after all, made with superior technology. 









\subsection{Politics}





\subsubsection{Family}
\Dasteron{} was the son of \hr{Ozariel}{\Ozariel}. 

\target{Dasteron's cousins}
He had two cousins: 
\Sargamel and \Themirod. 
Both \thelyadeth, kin of \ps{\Dasteron} mother. 
Both were powerful and influential \Mystraacht lords. 
They were his close allies, and he could never have \hr{Dasteron becomes Overlord}{become Overlord} without their loyal support. 
After Ramiel's takeover \hr{Dasteron's cousins oppose Ramiel}{they became his staunch rivals}. 









\subsection{Skills}





\subsubsection{Combat skill}
\target{Dasteron's skill}
\Dasteron{} was an extremely skilled warrior and mage. 
He had to become an \uber-skilled fighter in order to rise in the \Mystraacht{} ranks and \emph{almost} make himself Overlord. 
He did not have the raw strength of a \sathariah, so he trained to compensate. 

\target{Dasteron's upbringing}
One reason why \Dasteron{} was \hr{Dasteron stronger than Ramiel}{stronger than Ramiel} was that \Dasteron{} had lived his entire life as a \Mystraacht{} warriors. 
He has been raised \trope{TheSpartanWay}{The Spartan Way}. 

All his skill, all his power and all his fame were things he had \emph{earned} by hard work, not just gotten handed to him by virtue of birth or \sathariah{} power. 

\target{Dasteron's paths}
He was a rare expert who mastered all three \hs{Paths}: 
\hr{Path of Ice}{Ice}, \hr{Path of Light}{Light} and \hr{Path of Darkness}{Darkness}. 
He used mostly Light because he wanted to foster a very \Mystraacht{} macho-image. 
He kept his mastery of Ice and Darkness secret so he could use it as a surprise in combat. 





\subsubsection{Smithing}
\target{Dasteron's smithing}
Apart from powermongering, \ps{\Dasteron} greatest passion was weaponsmithing. 
He forged the sword \hr{Scaleron}{\Scaleron} and many other useful magical items, which he would bear and use in combat. 
He was \trope{CrazyPrepared}{Crazy Prepared}. 















\section{\Dezruth}
\target{Dezruth}
\index{\Dezruth}
A \thelyad{} \resphan{} of \Mystraacht. 
An acquaintance and ally of \hr{Teshrial}{\Teshrial}. 









\subsection{Appearance}
\target{Dezruth's appearance}
He had chestnut hair and a matching moustache. 















\section{\Gilchad}
\target{Gilchad}
\index{\Gilchad}
A \thelyad{} \resphan{} of \Mystraacht. 
An accomplice of \Cishiel. 

















\section{\Lyorith}
\target{Lyorith}
\index{\Lyorith}
A \resvil{} of \Merkyrah{} and later \Mystraacht. 
Lover of \Nathrach. 
Mother of Ramiel. 









\subsection{Arsenal}
\subsubsection{\Ascaril}
\target{Ascaril}
\index{\Ascaril}
\index{technology!weaponsmithing}
A \hr{Belthrad}{\belthrad} sword, forged by \Lyorith. 
It was a marvel, creating utilizing bits of \hr{Bane technology}{\bane{} (ie, \voyager) technology}. 

It was first wielded by \Lyorith. 
When she died \Nathrach{} inherited it. 
When \hr{Nathrach dies}{\Nathrach{} died} Ramiel inherited it. 
Ultimately \hr{Ascaril destroyed}{the sword was destroyed} when Ramiel fell. 





\subsubsection{Weaponsmithing}
\index{technology!weaponsmithing}
\Lyorith{} was a great artisan, sculptor and smith. 
When she joined the rebels against \Merkyrah, she developed a taste for violence and bent her passion towards weaponsmithing. 
\Semiza{} and the \banelords{} taught the \resphain{} some secrets of technology. 
Using these new techniques, \Lyorith{} forged many powerful weapons. 
The greatest of them was the sword \hr{Ascaril}{\Ascaril}, which she wielded. 

She also made \hs{Ramiel's guns}. 









\subsection{History}
\Lyorith{} was persuaded to join \Mystraacht{} with \Nathrach{} and Ramiel. 

She perished during the \Merkyran{} war. 















\section[Nathrach]{\Nathrach}
\target{Nathrach}
\index{\Nathrach}
A \resphan{} of \Mystraacht, a \sathariah. 
Killed near the end of the \secondbanewar. 

He was a close friend of \hr{Zachirah}{\Zachirah} and the father of \hs{Ramiel}. 









\subsection{Personality}
\subsubsection{Ambition for Ramiel}
He places great pressure on his son, Ramiel. 
He was a good and loving father, but also a stern one. 
He applauded Ramiel's achievements, but also kept pushing him on to new, greater achievements. 
He kept telling Ramiel: 
\ta{\hr{Ramiel can do better}{You can do better.}}















\section{\Ozariel}
\target{Ozariel}
\index{\Ozariel}
A \ketheran{} \resphan{} of \Mystraacht. 
Son of \hr{Zachirah}{\Zachirah}. 
Father of \hr{Dasteron}{\Dasteron}. 















\section{Ramiel}
\target{Ramiel}
\index{Ramiel}
%A \Malach{} whose incarnations include \TydesmosFull{} and \VizicarFull. 
Ramiel is a \Malach, a \resphan{} lord who has left his \resphan{} body to incarnate again and again as a \human{} Scion. 
He was one of the original \satharioth{} who drank the blood of \Nexagglachel. 

%Each Scion of Ramiel retains some memories of his previous incarnation. Typically these memories are locked away at birth and only awaken later, triggered by some massively emotional event. 

In the middle days of the \VaimonCaliphate, Ramiel was incarnated as \hr{Tydesmos}{\TydesmosGendarInCaphet}, a Vaimon archmage obsessed with arcane knowledge and power. 

In the late days of the \caliphate, Ramiel incarnated into \hr{Vizicar}{\VizicarDurasRespina}, one of the last \VaimonCaliphs and a great conqueror and statesman. 

In the year \yic{Carzain birth}, Ramiel incarnated as \CarzainDeracilleShireyo, a rogue Vaimon in Pelidor. In the year \yic{Mutiny}, the dormant personality of Vizicar awakened in young Carzain's mind after Carzain fought his first battle to the death and killed his first man. 









\subsection{Physique}
\target{Ramiel's appearance}
In his \resphan{} form, Ramiel was a dark angel. 
His skin, hair and feathered were ebon black, as was the metal \armour he summoned to encase him. 
He wielded a sword, like a scimitar but as long as a claymore. 

He was dark and terrible, but beautiful and awe-inspiring. 
A true \sathariah{} lord in his best shape. 
Contrast him with \hr{Sithiyacaan's appearance}{\Sithiyacaan}, who had the appearance of a tragic fallen angel.

Ramiel was not extremely tall. 210 cm or so. 









\subsection{Arsenal}





\subsubsection{Guns}
\target{Ramiel's guns}
\index{\Currah}
\index{\Strith}
Ramiel originally had two powerful guns, \Strith{} and \Currah. 
They were made for him by \Lyorith. 
He lost them when he fell. 
Later \hr{Ramiel regains guns}{they were returned to him by \Cishiel}. 









\subsection{History}
\subsubsection{He can do better}
\target{Ramiel was a prodigy}
The young Ramiel was very skilled and was praised as a prodigy. 
He was also very happy and jolly, telling jokes and stuff. 

He also cared a lot about art and culture. 
He painted and wrote music. 

\target{Ramiel can do better}
His father, \Nathrach{}, would applaud his successes, but also drive him onward to new achievements. 
\Nathrach{} would always tell him: 
\ta{You can do better.} 
This line becomes one of Ramiel's driving forces: 
He wants to prove to everyone, especially himself, that he can always \emph{do better}. 





\subsubsection{\Semiza{}: You are nothing}
\target{Ramiel is nothing}
Ramiel was one of \hr{Explorers meet Semiza}{the explorers who found \Semiza}. 
%At the time when \hr{Explorers meet Semiza}{the explorers met \Semiza}, Ramiel was young. 
At this time, Ramiel was young. 
\Semiza{} showed them \hr{Semiza shows tailored visions}{terrible visions, tailored to each of them}. 
Ramiel was confronted with the true vastness, cruelty and indifference of the universe. 
He saw how small, weak, humnle, wretched, insignificant and conceited his people were, and it filled him with angst. 

Since \hr{Ramiel was a prodigy}{young Ramiel was praised as a prodigy}, this revelation struck him a doubly hard blow. 
All his youth he had thought he was something great and had something to be proud of; that \hr{Ramiel can do better}{he could do better}. 
So the realization of his worthlessness was so much greater shock and trauma. 




\subsubsection{Nightmare: He is nothing}
\target{Ramiel dreams of being nothing}
Ramiel's oldest, most recurring and most terrible nightmare is one where he is micro-small, powerlessly and aimlessly adrift in an infinitely huge, black, cold void, where colossal and humongous gods fly past without sparing him a glance or even noticing him. 
He fears being crushed like a fly by sheer chance, with nothing he can do to prevent this meaningless, ignominious end. 

He can do nothing. 

He is nothing. 




\subsubsection{Nightmare variant: A blade of grass}
In a variant of the \quo{nothing} nightmare, Ramiel dreams that he is a blade of grass. 
Lame, blind, shackled to the ground with roots that he will never be able to break because they are part of him. 
It is his very being that keeps him chained. 

Humongous dinosaurs and mammals trample around him. 
They eat the other straws around him. 
He trembles, hoping that he is not next. 

Insects gnaw at him. 
There is pain. 
And powerlessness. 

\lyrics{I have no mouth, and I must scream.}





\subsubsection{Ambition: To prove the converse}
\target{Ramiel's ambition}
Ramiel is never the same again (at least, not for thousands of years). 
A dark cloud hangs over him. 
He is no longer naturally jolly and happy\dash and when he tries it feels fake, forced. 
His humour is rougher, meaner. 

Some of the older \resphain{} worry about this change in young Ramiel. 
This includes \Sithiyacaan{} and some of the priests. 

From this point, his whole life becomes a struggle to prove the converse: 
To challenge the enormous, all-consuming cosmos. 
So since then Ramiel's driving force has been \emph{ambition}: 
The craving for greatness, to prove his worth. 
Not so much for the sake of power or luxury, but in order to feel that he is something great. 

He gives up his interest in the arts. 
They seem pointless: 
If humanoids are inherently worthless, then by extension their creations are worthless, too, so why bother? 





\subsubsection{Growing machismo}
\target{Ramiel grows macho}
Ramiel grew much more macho after he joined the rebellion. 
He abandoned \hr{Ramiel's manners}{his manners} and grew proud and ambitious. 
Almost fanatically so. 
He began to radiate masculinity and power, but trained eyes could detect an undercurrent of desperation, neediness, uncertainty. 





\subsubsection{Sadism}
\target{Ramiel develops sadism}
His obsession with being \quo{nothing} also caused him to develop a sexual sadism. 
He took pleasure in the submission of others, in exercising his power over them. 
He did not allow sadism to take over his life in other aspects, but in the bedroom he fully embraced it and practiced it. 
This worked great with \Shiaraid, who was a masochist.






\subsubsection{Life as a \sathariah}
Maybe Ramiel's past is intimately linked to \Cuezca. 
Which would explain his \hr{Dreaming of Cuezcan Spires}{dreaming of \Cuezcan{} spires}. 
Perhaps the founding of the \Mystraacht{} faction, or the turning it took, was linked to \Cuezca. 

Ramiel has always coveted the throne of the \Mystraacht{} \apex{}, and much of his life has been shaped by his striving for the throne. 





\subsubsection{Lovers}
\target{Ramiel scored as Sathariah}
As a \sathariah, Ramiel attracted more \resvil lovers than ever before.
This was partially because of \hr{Sathariah social status}{his social status as a \sathariah}, but also because of his inner demons and darkness and insanity.
The \resviel could feel the violent emotions swirling inside him, \hr{Races love war}{and it turned them on}.






\subsubsection{Died many times}
Ramiel was killed many times in the rebellion (the War of Awakening), the \secondbanewar and the \resphanwars. 
He was brave and reckless, as a true \resphan man.
\Shiaraid died much less often, since she was sneaky and kept to the background, as a true \resvil.





\subsubsection{Fall from grace}
\target{Ramiel's fall from grace}
Somehow Ramiel is disgraced. 
\hr{Mystraacht summon Daemons}{As the \Mystraacht{} were wont}, he conjured \daemons{} and commanded them in war, like the Warlord of Blood from \emph{Diablo}. 
But he had enemies, inside \Mystraacht{} and outside, who were willing to sell him out. 
During a battle, his \daemons{} deserted and turned on him and his allies. 
It was depicted as his fault for being too weak and overconfident, but in truth he had complete control of the situation until he was backstabbed by his own. 
He knew this, but was unable to prove it, and so he lost his \honour among the \resphain. 





\subsubsection{Becoming a \malach}
\target{Ramiel becomes a Malach}
%Perhaps it was because of betrayal, defeat and disgrace that Ramiel elected to become a \Malach, 
After \hr{Ramiel's fall from grace}{his fall from grace}, Ramiel volunteered for the dangerous but glorious \Malach{} experiment. 
To regain his \honour. 

Or perhaps his plan from the beginning was to become a \malach{} in order to be reborn, live many lives and thus acquire wisdom and skills and emerge reshaped as a wiser, mightier \resphan{} and \vertex. 





\subsubsection{Incarnations}
Ramiel had a number of incarnations as a \malach: 

\begin{enumerate}
  \item 
    A few early ones, including at least one who lived in the \VaimonCaliphate. 
  \item 
    \hr{Tydesmos}{\TydesmosGendarInCaphet}.
  \item 
    \hr{Vizicar}{\VizicarDurasRespina}.
  \item 
    Two short-lived ones. 
  \item 
    \hr{Carzain}{\CarzainDeracilleShireyo}
\end{enumerate}






\subsubsection{Haunted by dreams}
\target{Ramiel haunted}
\target{Ramiel dreams}
In his incarnations as Carzain and Vizicar, Ramiel dreams. 

Vizicar appears to Carzain as a benevolent ally, a wiser, more experienced mentor character. But all is not fine and dandy with him. Vizicar has his own traumata, and Carzain inherits them. 

Occasionally Carzain has dreams and hallucinations. 
He sees \hr{Ramiel dreams of being nothing}{the devouring emptiness of eternity}, and despairs at his own insignificance. 
He \hr{Ramiel is nothing}{fears being nothing}.

He also dreams of being haunted by people. 
These are people from Vizicar's life, his friends, associates and allies whom he betrayed. Somehow they are tied to him and continue to haunt him. This may be caused by his own fear and paranoia, perhaps even guilt.
On occasion, these ghosts are even able to take physical form and come into the physical world to attack him. 

The ghosts appear undead and rotting. 
Some of them are \resphain; now fallen carrion angelds. 
Like in \authorbook{Alan Campbell}{Scar Night}. 

These apparations may be created entirely by his mind, but I am more leaning toward the idea that they are the actual souls of the original people, kept chained by the \ps{\Malach}{} emotions and forced into a state of undeath. 

You see, while Vizicar maintains his composure and sounds perfectly sane and balanced, he suffers from control-mania and paranoia. When people come close to him, he starts to fear their influence, and he has killed many of his lovers and friends out of this fear. 

Or something. 

There is also a theme of Ramiel questing for the truth of his existence, his true origin and purpose, which might be purposefully kept from him, perhaps by the \banes. 
(This is inspired by the trailer I saw for the movie \cite{Movie:BourneUltimatum}.) 
This must be tied together with Vizicar's derangement somehow. 

Something Ramiel fears is his forgetfulness. (This needs to have a cool name: The Decay, the Fading, the Oblivion, something like that.) 
See, \Malachim{} are immortal, but their memories are not. 
They fade with each new incarnation. Even now, with Carzain, Tydesmos is little more than a faint voice in Vizicar's head, a memory of a memory, a ghost of a ghost. And Vizicar can feel the memory of his own life fading. He fears that when Carzain dies, he will fade away entirely. But he also believes that the fading of memory can be arrested, perhaps even reversed, if he can solve the riddle of his past and origin. 





\subsubsection{Scenes from a memory}
Ramiel's dreams are comparable to \cite{DreamTheater:ScenesfromaMemory}: Scattered fragments of past lives. 

And visions of the Metropolis: The endless, massive, decaying, haunted eternal city that is \Nyx. Compare to the \emph{Kult} RPG. 





\subsubsection{Grows to hate \dragons{} and \banes}
After he \hr{Shiaraid dies}{betrayed and slew \Shiaraid}, whom he loved, he became increasingly bitter. 
He \hr{Ramiel blames both sides for the tragedy}{blamed both \dragons{} and \banes{} for this tragedy} and grew to hate both sides. 
This was part of his motivation for betraying the \banelords. 





\subsubsection{Apotheosis and wisdom}
\target{Ramiel is wiser from walking the earth}
Ramiel \hr{Ramiel's final awakening}{finally achieved} the \hs{Apotheosis} he had sought for. 

When he awakens, Ramiel is stronger than ever before. 
Partially because he has spent thousands of years \trope{WalkingTheEarth}{Walking the Earth} as a Scion, which has given him a broad selection of useful experience. 
In total, this proves to have been a more enriching learning experience than it would have been had he lived these millennia as a \resphan. 
(And also less dangerous.
As a \malach he was very hard to kill.)

\target{Ramiel is traumatized from awakening}
He is traumatized and shaken. 
His sanity has suffered harsh blows.
But he is back with a vengeance and revels in his regained and newfound power.

He is wiser now. 
This helps him deal with and come to terms with his many mental problems and gain some semblance of sanity. 

\hr{Ramiel is mad after awakening}{Later, it will turn out} that Ramiel's sanity really has suffered badly from all his experiences and all his revelations.

\target{Ramiel is critical of Mystraacht ideology}
After his Apotheosis, he is more thoughtful and philosophical than his old self. 
He is more critical of \Mystraacht ideology. 
He still uses the traditional \Mystraacht macho ideals in his rhetorics (\hr{Ramiel uses macho rhetoric as Overlord}{as Overlord} and \hr{Ramiel uses macho rhetoric against Dasteron}{when competing for the throne}), but internally he questions it a lot, and in his actual policy he is often far more pragmatic than would be expected of a \Mystraacht (especially suprising to those who knew his old self). 
The \Mystraacht way is to charge in with all your macho bravado, and to worry about bravery and cowardice and reputation. 
Ramiel ends up being much more rational and careful.
He is also critical of \Zachirah's \quo{Religion of Evil}. 
Back in the day he followed \Zachirah's ideology unquestioningly and with great fervour, but the wiser Ramiel of today is smarter than that and recognizes the insanity and danger of the ideology. 





\subsubsection{Grows super-powerful}
\target{Ramiel is overpowered}
Now that he has regained his full \sathariah{} power, Ramiel is pretty \uber. 
See, he has eaten \Belzir, who was also a \sathariah. 
So now he has double \sathariah{} power. 
He has two \hr{Fragments of Nexagglachel}{fragments of \Nexagglachel}. 

\target{Ramiel is underestimated after Apotheosis}
This has never happened before, so people he encounters do not know what it entails or how to account for it. 
So they underestimate him. 





\subsubsection{Grows super-powerful}
\target{Ramiel is mad after awakening}
Ramiel \hr{Ramiel is overpowered}{may be powerful after his Apotheosis}.
But having eaten \Shiaraid is not purely a blessing. 
In addition to her power, Ramiel has also inherited \hr{Shiaraid's curse}{\ps{\Shiaraid} curse} of self-destructive madness. 
In him it does not take the form of sexual masochism. 
That is too far from his true personality. 
In him it takes instead the form of recklessness and a willingness to take insane chances. 
This is merely an extrapolation of a trait he already had. 
It also makes him more sadistic and disdainful, thus inviting others to bring about his downfall. 

It takes a while for him to realize this. 
He fears it. 
He knows he is turning evil and mad. 
He fights it. 
But he never manages to entirely rid himself of it. 
\hr{Curse lives on}{\NexagglachelsCurse lives on}.

Moreover, though Ramiel tries to hide it, his sanity really has suffered badly from all his experiences and all the revelations during \hr{Ramiel's final awakening}{the process of Apotheosis}.
It is what turns him against the \banes and makes him take such mad chances.
And he is wracked with much anguish and guilt over slaying \hr{Shiaraid dies}{\Shiaraid} (and, to a later extent, \hr{Dasteron dies}{\Dasteron} and \hr{Ramiel kills Gilchad}{maybe \Gilchad}).

But he also feels an alienation from his fellow \resphain, as if all mortal and immortal life is really a bunch of worthless, insignificant specks of dust.

And he still feels a desperate need to \hr{Ramiel is nothing}{prove that he is more than nothing}.

\citebandsong{Nile:AnnihilationoftheWicked}{Nile}{User-Maat-Re}{
  User-Maat-Re, thou hast done nothing.
}





\subsubsection{Final closure}
At the end, he finally wins a victory of sorts over \Daggerrain, who is a representative of this dark, overpowering cosmos, \hr{Ramiel is nothing}{in comparison to which he is nothing}. 
This makes Ramiel realize all the above, giving him a birds-eye-view of his life. 
He finally understands what he has been struggling to grasp his whole life. 

Finally he gets some closure, finds some peace. 
He has moved the universe. 
He has made a difference. 
He has proven his worth. 

This makes him more good. 
He resolves to oppose \Azraid{} and try to lead his people in a better direction. 
It also leads him to understand \Ishnaruchaefir{} better. 

And he resumes his interest in the arts. 
He has now \quo{proven} that humanoids are not entirely worthless. 
Therefore, by extension, art might have some merit after all. 

Ramiel: 
\ta{20,000 years ago I started composing a symphony. 
  I still remember much of it. 
  Perhaps it is time I finished it...}










\subsection{Personality}
Like all the other \Malachim, Ramiel has lost his memory of his previous life as a \resphan{} lord, recalling only scattered fragments, typically in fever-like dreams and \deajvus. 

Ramiel years for his lost memory and makes it his great quest to rediscover his true identity.





\subsubsection{Afraid of water}
\target{Ramiel fears the sea}
Ramiel is somewhat afraid of the sea. 
He doesn't get seasick, but he gets uncomfortable and prefers to sequester himself below decks, certainly not going anywhere near the railing. 
He is scared he will fall in. 

This has to do with \hr{Ramiel defeated at sea}{a traumatic experience where he lost a naval battle}. 





\subsubsection{Fear of being nothing}
His greatest fear was that of \hr{Ramiel is nothing}{being nothing}. 
He could not stomach the thought of submitting to anyone. 
This was one of the reasons why he ultimately betrayed the \banelords. 

Under the influence of \hr{Curse}{\NexagglachelsCurse}, this fear occasionally surged to really hysterical levels. 





\subsubsection{Manners}
\target{Ramiel's manners}
In his youth, Ramiel was more polite. 
He would say \quo{Lord \Zachirah}, \quo{Lord \Damiarch} and so on. 

This stopped when he and the other rebels became convinced that the \Merkyran{} religion was an oppressive, destructive slave morality. 

Everyone \hr{Ramiel grows macho}{noticed his growing machismo}.





\subsubsection{Megalomania}
\target{Ramiel's megalomania}
Ramiel sees himself \hr{Secherdamon's megalomania}{in a similar light as \Secherdamon}: 
As an heir to divine power and dominion. 

But he is not content to pray and hope for gifts of power. 
He wants to usurp his creators and take it by force. 
And in \SentinelsFinalBook, he tries to do exactly that, by \hr{Ramiel betrays Voidbringer}{betraying the \Voidbringer}. 
He means to supplant the old and stagnant \banes{}, wipe away the failures of the past and bring \quo{new blood} to power. 

He fights for the survival and future of the \resphan{} and \human{} races, and for the proud legacy of the \banes{} and their progenitors, the \voyagers. 
He means to carry on that legacy, by any means possible.





\subsubsection{Willing to destroy and rebuild}
Unlike \Nzessuacrith, who has \hr{Nzessuacrith likes beauty}{a weakness for beautiful things}, Ramiel has no problem with destroying things around him. 
His prefers to fight with no reservations, and then, after he has won, rebuild things, stronger and more beautiful than ever. 









\subsection{Politics}
\subsubsection{Allies and enemies}
\target{Ramiel's enemies}
He has enemies among the \resphain, who do not wish to see him awaken. Among these are rival \Mystraacht{} princes. Yet other \resphain{} love him and want him back. Some of them actively support him\dash overtly or covertly. Some of these eventually backstab him. 

One of his primary enemies is a manipulative and beautiful \resvil{} who desires to be the \Mystraacht{} Overlord herself. Compare her to Azshara from the \emph{Warcraft: War of the Ancients} books by Richard Knaak. 

Have a Scabandari Bloodeye-like character who betrayed Ramiel. Today he is a mighty \resphan{} lord. 
He might even be \Teshrial, or \Azraid. 





\subsubsection{\Azraid{} and \Damiarch}
Ramiel respects \Azraid{} as a competent ruler and a great \resphan. But he is also somewhat repulsed by \Azraid, with all his strange perversions and obsessions. He is too bizarre, too mad.

In the beginning (during the \hr{Merkyran rebellion}{\Merkyran{} rebellion}), Ramiel was young, rash and impulsive. 
He greatly admired \Damiarch{} and looked up to him as a hero-figure to emulate. 
He has since accepted \ps{\Azraid}{} betrayal of \Damiarch, but never forgiven him. 




\subsubsection{\Eryal}
\target{Ramiel resented Eryal}
Ramiel never liked \Eryal. 
He looked down on her, seeing her as a soft, weak, \Merkyrah-sympathizing coward. 
Not \Mystraacht{} material by a long shot. 

Also, Ramiel was jealous because \Shiaraid{} loved \Eryal{} more than she did him. 
He could not fathom what \Shiaraid{} saw in the girl, but he resented \Eryal{} for it. 
\hr{Resphain are possessive}{\Resphain{} are possessive} of their women. 





\subsubsection{Family}
Ramiel was the son of \hr{Nathrach}{\Nathrach}. 
He had a daughter, \hr{Cishiel}{\Cishiel}, born shortly before he went missing as a \malach. 
She was \hr{Ramiel meets Cishiel}{there to greet him} when he \hr{Ramiel returns to Mystraacht}{returned to \Mystraacht}. 

\target{Ramiel and Cishiel have sex}
It should be hinted that Ramiel and \Cishiel may have an incestuous relationship. 
Once in a while he spanks her and says: \ta{Do not disobey me, Cishiel.}
She gets aroused and whispers: \ta{Yes, My Overlord Sathariah.}





\subsubsection{\Yurideth}
Ramiel claimed he had never had sex with a \hr{Yurid}{\yurid}.
He had on some occasions taken prisoners, made them into \yurideth and given them to his men, but he had never raped a \yurid himself, he claimed.
He found it distasteful.





\subsubsection{\Vizsherioch}
Ramiel sees \Vizsherioch{} as his great rival. 
He longs to challenge this half-blood \xs{} in order to \hr{Ramiel's ambition}{test his mettle against a representative of the dark, endless universe}. 
But he never manages to do this satisfactorily. 

After \hr{Daggerrain falls}{\ps{\Daggerrain}{} fall} he makes this into a \quo{New Year's resolution}: To one day confront and defeat \Vizsherioch. 
(At this point, \Vizsherioch{} goes dormant, but everyone knows that he is bound to return sooner or later.) 





\subsubsection{\Vorcanth}
\target{Ramiel's wolves}
Ramiel has a group of monsters that serve him, his faithful hounds. 
They are wolf- or hyaena-like monsters. 
They are the \hs{Moon-Wolves}. 

\lyricsbs{Bal-Sagoth}{
  Starfire Burning Upon the Ice-Veiled Throne of Ultima Thule
}{
  The mystic wolves of the frost-moon (slowly, silently) encircle me,\\
  Their eyes are blazing azure, \\
  and their fur is whiter than the sublime snows.
}

The wolves are connected with astrology, remember. 

\lyricsbs{Emperor}{Cosmic Keys to my Creations and Times}{
  They are the planetary keys to unlimited wisdom \\
  and power for the Emperor to obtain. \\
  (They) being the gods of the wolves \\
  whom upon they bark at night, \\
  requesting their next victim in thirst of blood. \\
  I enjoy those moments I may haunt with these beasts of the night.
}







\subsection{\Carcer}
\target{Ramiel binding souls}
\target{Ramiel's bound souls}
\index{\carcer!Ramiel}
\hr{Malachim binding souls}{As a \Malach{} (and a \sathariah, to boot)}, Ramiel possessed a \sephirah-like power to bind the souls of the slain to himself. 
He had three Scion incarnations, all of which were great warrior-mages who fought and killed many people, so he ended up with a big-ass \carcer{} full of souls. 

In his Carzain incarnation, his full \Malach{} powers were latent at first, he did not have conscious control over this ability. 
The result was that the souls of all the people he had killed or whose deaths he caused, as well as the souls of many of his allies (who were not killed by him, but were emotionally bound to him), were bound to him as ghosts. 
They haunted him at night, and sometimes even manifested as wraiths to attack him physically\dash perhaps dragging him into \Nyx{} with them. 
This was a major trauma of his. 

Compare this to Karsa Orlong from \cite{StevenEriksonIanCameronEsslemont:MalazanBookoftheFallen}, who has a horde of fallen souls bound to him by chains.

Gradually, during the story, as he comes to understand his power and his nature better, he learns how to manage these ghosts\dash how to dismiss and suppress them, how to call them forth and commune with them (they don't have much sanity left, but sometimes you can talk to them and get them to make sense). 
He even learns how to control them and channel their power as a magical weapon. 

\citebandsong{DeathspellOmega:FasIteMaledictiinIgnemAeternum}{%
  DeathspellOmega
}{
  The Shrine of Mad Laughter
}{
  Yet it is a world below \\
  and above and in all eternity, \\
  a gift of fever, \\
  the wind of death that sustains the life in me. \\
  Yes, the lightness of hovering in permanent anguish.\\
  I dared to borrow those words, \\
  to articulate them and to savour their turpitude, \\
  as I beheld the shrine of mad laughter.
}

Ramiel begins to realize the parallel between himself and the \Sephiroth, who also chain dead souls. He begins to entertain ideas of divinity. 

But ultimately, he needs to truly use this power as he was meant to. 
His personalities as Carzain, Vizicar, Tydesmos and others must be broken down and destroyed. 

He still harbours feelings of guilt and fear with regard to the ghosts. 
He is violently forced to confront these feelings when his tormentor sets the ghosts loose upon him. 
He fights them, they fight them, and they seem to destroy each other. 
It is pretty traumatic. 

In the end, his Scion personalities are destroyed and unravelled, and he devours and absorbs the ghosts into himself. 
He is now a mighty devourer of souls. 
He is now a \Malach. 
He is now the true Ramiel. 

\lyricsbs{Emperor}{Wrath of the Tyrant}{
  Carnage consumes the emptiness. \\
  Wait till my spirits come forth. \\
  Violate all his chosen ones. \\
  Drink the fires of death. 
  
  Carrying the deaths of his fallen warriors \\
  deep inside of him, in his eyes. \\
  Walk upon this Earth tonight, \\
  carrying the staff of cold souls.
}





\subsubsection{Wielding the bound souls as a weapon}
\target{Bound souls as a weapon}
It is possible to channel the power of the bound souls, momentarily release them into the world and thus wield them as a weapon. The souls are driven mad in their prison and filled with hate, bitterness and rage, a longing to share their terrible pain and sorrow with others. 

Thus unleashed as a torrent of necromantic powers, they slash and tear at the souls of any victims they can catch. They attack by infecting others with their own pain, despair and madness.

If their victim is mentally strong, he can resist the attack and suffer only some mental anguish. 

If he is weak-willed, the ghosts will tear through his mental defenses and be able to harm his physical body\dash slashing it as if with knives, or perhaps causing it to decay and decompose with \quo{instant leprosy}, in extreme cases causing it to collapse in a heap of misshapen, cancerous flesh. 

If a victim survives such an attack, he will likely be permanently scarred and maddened. 

\lyricsbs{Exmortem}{Bitter Discipline}{
  I enter the stage as an evil demagogue, \\
  demanding obedience and devotion \\
  from the legions of dead.
  
  Thousands of men \\
  wandering in darkness. \\
  Searching for their God. \\
  (Here I am.) \\
  Welcome to Hell.
  
  Worship pain, worship Death. \\
  Onto the dark side, let the damned rise. 
  
  Bitter discipline.
}





\subsubsection{Wearing the bound souls as \armour}
Ramiel also learns to shape the bound souls into a magical \armour encasing him. This is as potent as an \hs{Archon Ward}.

Compare to Wismerhill from \FLuneNoire, who can command the winds to encircle him like \armour. 















\section[Shiaraid]{\Shiaraid}
\target{Shiaraid}
\index{\Shiaraid}
A \sathariah{} \resvil{} of \Mystraacht, and later a\malach{}. 
Known for her incarnations as \hr{Delphine}{\Delphine} and \hr{Belzir}{\Belzir}.








\subsection{Arsenal}





\subsubsection{Binding souls}
\target{Shiaraid binding souls}
\index{\carcer!\Shiaraid}
\hr{Malachim binding souls}{As a \malach}, \Shiaraid{} had a \sephirah-like power to bind souls to her. 

\Delphine{}, her first incarnation, \hr{Delphine binding souls}{had bound almost zero souls} (virtually only that of \Eryal). 

\Belzir{} built up \hr{Belzir binding souls}{a more impressive \carcer}. 





\subsubsection{Stealth}
\target{Shiaraid's stealth}
\Shiaraid{} was one of the stealthier \Malachim. 
No one ever suspected that \Delphine{} was a Scion, \hr{Delphine never knew she was a Scion}{not even \Delphine{} herself}. 
And in her incarnation as \Belzir{}, it \hr{No one knew Belzir was a Scion}{did not come out until late in her life} that she was a Scion. 
She herself learned it far earlier than anyone else did. 








\subsection{History}





\subsubsection{\Semiza}
\target{Shiaraid develops sadomasochism}
It was her meeting with \Semiza{} that made \Shiaraid{} turn to sadomasochism. 
The urges had always been lying in her, but latent.
She repressed the urges as \quo{unnatural}, because of her repressive religious conditioning in \Merkyrah. 

\Semiza{} made her realize what she truly wanted. 
Once awakened, the urges would not go away and continued to haunt her. 
And the rebellion gave her the courage to accept and embrace her sexuality. 
Dually, her unfulfilled sexuality (and the realization that it would forever remain unfulfilled if she had to follow \Merkyran{} law) was a major part of her motivation for joining the rebellion. 

\Shiaraid{} learned much about herself when she gave herself away fully in sexual submission. 

\citebandsong{BeyondTwilight:FortheLoveofArtandtheMaking%
}{%
  Beyond Twilight%
}{%
  For the Love of Art and the Making%
}{
  On your knees\dash
  You start to gain a new perspective\\
  On your knees\dash
  Your eyes gazing upwards\\
  On your knees\dash
  Where you choose to be
}





\subsubsection{Unpopular}
\target{Shiaraid unpopular}
Before her \kenosis, \Shiaraid{} had lost a lot of her popularity among her fellow \resphain. 
\hr{Curse}{\NexagglachelsCurse} was affecting her really badly. 
It hit her worse than it did the other \satharioth. 
She was more susceptible to it somehow. 
So she was descending into madness, prone to fits of 
This ostracized and isolated her from her allies and friends. 

Near the end, nearly the only ones who stood on her side were Ramiel and \Eryal, and possibly \Cishiel. 
And even their affections were faltering. 
\Shiaraid{} was growing pretty desparate. 
Of course, that just made her even more mentally unstable, and she fell victim to the Curse even more. 
It was a vicious circle. 

After her \kenosis, most \resphain{} were glad to be rid of her and did not want her back. 
This meant that her \hs{Royalist Faction} had no allies among the \resphain. 









\subsection{Physique}
\Shiaraid's hair and feathers had an orange tint to them. 









\subsection{Personality}





\subsubsection{Dogs}
\target{Shiaraid's dogs}
\Shiaraid{} always liked dogs. 
She kept some of them in all her incarnations. 





\subsubsection{Sexuality and curse}
\target{Shiaraid's sexuality}
\target{Self-destructive Shiaraid}
\Shiaraid{} is a sado-masochist. 

She is a type like Melisande Shahrizai from \authorseries{Jacqueline Carey}{Kushiel's Legacy}: 
A dominatrix and manipulator supreme. 
But she also has a submissive side. 
She enjoys both dominating others and being dominated herself. 

This is partly because she inherited the \hr{Fragments of Nexagglachel}{fragment of \Nexagglachel} that let him endure pain and even take pleasure in it. 

\target{Shiaraid's curse}
It is also due to \hr{Curse}{\NexagglachelsCurse}. 
He is inside her and twists her mind, making her insane and self-destructive. 
It compels her to destroy herself and those she holds dear, even taking pleasure in it. 
She \hr{Satharioth despair at the curse}{fears this}, but cannot stop herself. 
She knows her lusts are dangerous to her. 

\begin{prose}
  \tho{It is madness. But it feels so sweet...} 
\end{prose}

\target{Shiaraid's tragedy}
And so, \ps{\Shiaraid} tragedy is the Curse, which forces her to destroy herself and those she loves: 
\Aryal, \Zachirah{} and Ramiel. 
Whenever she has once again caused ruin for herself and those she loves, she imagines she hears \Nexagglachel{} laughing inside her head, mocking him as she once mocked him in his captivity. 
She hates him, but still some part of her recognizes the justice in it. 










\subsection{Politics}





\subsubsection{Family}
\Shiaraid{} was the daughter of \hr{Zachirah}{\Zachirah}. 
She had two sons, born before or during the \secondbanewar. 
But after she became a \malach{} and went missing, her sons were both killed by contenders for the throne of \Mystraacht{} who did not want to have more of \ps{\Zachirah} heirs alive than necessary. 





\subsubsection{Lover of \Aryal}
\target{Shiaraid and Eryal lovers}
\Shiaraid{} and \Aryal{} used to be lovers. 
\Shiaraid{} was the dominant one and \Aryal{} the submissive. 

Compare to Melisande Shahrizai and \Phedre{} in \authorseries{Jacqueline Carey}{Kushiel's Legacy}. 





\subsubsection{Lover and friend of Ramiel}
\Shiaraid{} was a close friend and lover of Ramiel. 
The two were about the same age. 















\section{\Zachirah}
\target{Zachirah}
\index{\Zachirah}
A \resphan{} of \hr{Merkyrah}{\Merkyrah}.
The father of \hr{Shiaraid}{\Shiaraid} and the mentor of \hs{Ramiel}. 
He was one of the \hr{Delving}{\Delvers} and a \hr{Sathariah}{\sathariah}. 
He became the founder and first Overlord of \hr{Mystraacht}{\Mystraacht}. 








\subsection{Physique}
\target{Zachirah's appearance}
\index{beard!\Zachirah}
\Zachirah{} was huge: 
250 cm tall or more, and very muscular. 
Almost as massive as a \nephil. 

He wore a beard. 

His hair and feathers were almost black, but with a hint of red or brown. 









\subsection{Arsenal}
\subsubsection{Slave \resviel}
\target{Zachirah's slave Resviel}
After he became a \sathariah{} and the Overlord of \Mystraacht, \Zachirah{} became massively sexy in the eyes of many. 
He embodied, in a sense, all the masculine power of the perfect \resphan. 

He was such a pick-up artist that even proud \resviel{} begged and fought for the chance to become his personal sex slaves. 
He took four such slaves. 
They were permanently chained half-naked to his throne and from that moment on lived as his property and playthings, dedicating their lives entirely to his pleasure. 

\hr{Resphain are possessive}{\Resphain{} are possessive} of their women, but no \resphan{} in recorded history had been able to enslave \resviel{} like \Zachirah{} did. 
This was part of the reason why he was so admired and respected. 
It showed off his immensely superior manliness. 
This \uber-manly frame was how he was able to rule \Mystraacht{} for so long. 
People dared not go up against so macho a \resphan. 

\Zachirah{} had absolutely no sexual taboos and would often have one of his slaves suck his dick while he was talking. 
He would also often beat his slaves when he felt like it. 
Most of all he enjoyed the status and the envy of the other \resphain{} (no one else was manly enough to make pureblood \resviel{} serve them as willing slaves). 

Once he became displeased with a slave. 
He ordered her fellow slaves to torture her to death, after which he ate her soul. 
The vacancy was filled in \emph{no time}, and with several applicants vying violently against each other for the position. 

Sometimes he took his throne and slaves into battle, where they would fight for him with magic. 

They \hr{Zachirah dies}{died with him}. 









\subsection{History}





\subsubsection{\ps{\Semiza}{} revelations made him evil}
\hr{Semiza shows tailored visions}{\Semiza{} shows tailored visions} to each of the \hr{Delving}{\Delvers}. 
\Zachirah{} is shown things that make him evil and hateful. 

He immediately adores and worships the \banes. 

\lyricsbs{Hate Eternal}{Behold Judas}{
  As I stand before thee, Master of the Arcane.\\
  Lest we forget your burden, father.\\
  From beneath the binding \\
  of the irreverent one who is shroud in darkness.\\
  Revealed! Heathen to all that is sacred!
  
  I serve myself unto thee, Master of the Labyrinth.\\
  Succumb we must to this promised design.\\
  From within you suffer, \\
  your knowledge of your transgressions.
  
  I bow down before you, master of the kingdom,\\
  under the guise of a holy existence.\\
  From the depths of all time \\
  is the man who claims to be our savior.\\
  Emerged! Traitor to all that is holy!
}

He was originally a critic of the \Merkyran{} system. 
\Semiza{} recognizes his \skepticism and disgruntlement and uses it as a lever to twist \Zachirah{} and make him hate every aspect of \Merkyrah{} and everyone in it. 
He turns to the opposite of \Merkyran{} values (pacifism and stuff) out of pure spite and hate. 

Furthermore, \ps{\Zachirah}{} ambition for glory, \honour and achievement is twisted into a cruel lust for power. 

\lyricsdimmuborgir{D\o{}dsferd}{
  I d\o{}dsdalens \o{}de skj\o{}nnhet\\
  har min sjel vandret vill.\\
  Mens m\o{}rket og sorgen r\aa{}det\\
  tentes hatets flammende ild.
}





\subsubsection{Going mad}
\target{Zachirah goes mad}
He slowly goes mad. 

\lyricsbs{Hate Eternal}{Sacrilege of Hate}{
  Struggling with conflict \\
  over my deep-seeded hate. \\
  Striving with constant angst. \\
  Breed all of my pain. \\
  Breed all of my pain. 
  
  Fighting with anger.\\
  Overcoming all that is sane.\\
  Writhing to be ordained. 
  
  Feeding on grief. \\
  Needing pain ever so deep. \\
  Reasoning through rational. \\
  Breed all of my pain. \\
  Breed all of my pain. 
  
  Seeking my vengeance.\\
  Feeding on pity and empathy.\\
  Searching for infamy.\\
  Blessed with blasphemy.\\
  
  I live off all of the meek.\\
  Your feebleness so weak.\\
  With greed I solidify your fate.\\
  I'll bleed you of your hate. 
}

He is gradually corrupted. 

\lyricsbs{Vital Remains}{Sanctity In Blasphemous Ruin}{
  Buried beneath the centuries, \\
  memories of horrific prophecies, \\
  the word of God forced upon the weak.
  
  I watch the blind in disgust in their cathedrals of God, \\
  as they pray for his return. \\
  We, Legion, shall reap and crush their hope. \\
  Our flag overshadows the worthless one, \\
  for all the world to see.
  
  My sanctum, this majesty of sin, \\
  these structures of malevolence inherit my darkened spirit. \\
  Storm the gates, spiral portal descends. \\
  I quicken with the burning of barren relics.
  
  Your beloved fixture of flesh and oak \\
  ingest the silhouettes from below. \\
  Growing blacker, blacker, blacker, blacker, \\
  ebony is the \colour of our salvation.
}





\subsubsection{Founds \Mystraacht}
He turns to evil and \hr{Zachirah founds Mystraacht}{founds the cruel \Mystraacht}. 

\lyricsbs{Hate Eternal}{Darkness by Oath}{
  I swore by oath of darkness, of legion, of one. \\
  I swore by the truth. \\
  I swore my allegiance to darkness, of legion, of many. \\
  I swore by my faith. 
}

He lives now to serve the \banes{} and to wreak evil. 
  
\lyricsbs{Hate Eternal}{Darkness by Oath}{
  By my strength I await the \\
  ever present overboding entities. \\
  Initiate my path of deceit. \\
  Initiate my reign of disdain.
}

He wants to free the \banes{}, whom he sees as his people's true gods and the world's rightful fathers and masters. 
  
\lyricsbs{Hate Eternal}{Darkness by Oath}{
  Bring forth centuries of misery. \\
  Return thyself to the throne. \\
  Emancipate thee from binding restraint. \\
  Bring forth centuries of pain. \\
  Return thyself to the eternal reign. \\
  Emancipate thee from binding restraint. 
}





\subsubsection{Death}
He was \hr{Zachirah dies}{killed during the \resphanwars}, victim of betrayal. 
After this, Ramiel and \Shiaraid{} (both \Mystraacht) were dis\honoured and turned to the \Malach{} experiment. 









\subsection{Personality}
\target{Zachirah's ambition}
\Zachirah{} was, at first, \honour{}able enough, but hard, stern and very ambitious. 
He desired power and status, for himself and for his bloodline. 





\subsubsection{Ambition for Ramiel}
He places great pressure on his pupil, Ramiel. 
He was a good and loving father, but also a stern one. 
He applauded Ramiel's achievements, but also kept pushing him on to new, greater achievements. 
He kept telling Ramiel: 
\ta{\hr{Ramiel can do better}{You can do better.}}





\subsubsection{Dark Lord}
\Zachirah{} became an evil dark lord. 

\lyricsdimmuborgir{Hunnerkongens Sorgsvarte Ferd Over Steppene}{
  Du levde i m\o{}rke.\\
  Du vandret i sorg.\\
  Du plyndret med st\aa{}l til hest.\\
  En stolt og stridig konge,\\
  som erobret hver en borg,\\
  for s\aa{} \aa{} heise fanen til fest.
  
  Attila, hunnernes konge.\\
  Krigenes herre, v\aa{}r far.\\
  Du hentet din styrke fra m\o{}rke,\\
  og p\aa{} tokt med deg jeg n\aa{} drar.\\
}























\chapter{\TiphredSerah}















\section{\Dorzand}
\target{Dorzand}
\index{\Dorzand}
\Dorzand was a \sathariah of \TiphredSerah.
He was very mysterious but charming.
A great manipulator.
A \quo{Count Dracula} type character. 
(The romanticized Dracula, not the actualy Vlad III Draculea.)

He was wise and saw deep.
He was closer to the banelords than many.
\Azraid dealt with \Dorzand, but did not trust him.

His name is based on Dorozhand, a god from 
\cite[\quo{Of Dorozhand}]{LordDunsany:TheGodsofPegana}. 















\section{\Ishicah}
\target{Ishicah}
\index{\Ishicah}
A \resvil{} of \TiphredSerah, a \sathariah{}. 









\subsection{History}
\Ishicah{} tried to stage a coup and make herself queen of \TiphredSerah. 
She figured that \CiriathSepher{} was better off with a strong ruler, and \Mystraacht{} had been likewise when \Zachirah{} lived, so she would be doing \TiphredSerah{} a favour. 

But she was thwarted and disgraced. 
So she joined the \hr{Malach project}{\Malach{} project}. 

She became a \Malach.

Later she was \hr{Ishicah enslaved}{captured and enslaved by the \Ortaicans} and died (permanently) in captivity. 















\section{\Lothagiel}
\target{Lothagiel}
\index{\Lothagiel}
A \ketheran{} \resphan{} of \TiphredSerah. 









\subsection{History}
\Lothagiel{} once tried to slay \Ishnaruchaefir{}. 
He failed. 
He did a great deal of research on \ps{\Ishnaruchaefir} Aenigma before facing him, and left behind a lot of useful research notes, which would later be \hr{Teshrial gets notes}{discovered by \Teshrial}. 

\Lothagiel{} studied both history and myth. 

The myths, including the poem \emph{\hr{Wanderers in Darkness}{\WanderersInDarkness}}, read like a prophecy of when \Ishnaruchaefir{} will fall. 
\Lothagiel{} did not believe in prophecies, but he believed the myths contained a core of truth. 
So he studied all the sources he could. 

It turned out there was good evidence that \Ishnaruchaefir{} had certain \hr{astrology}{astrological} weaknesses related to the \hr{Matrix}{\matrices}. 
(Remember, the \matrices{} are related to astrology.) 
Under certain patterns of stars, he was weakened, and to certain weapons he was vulnerable. 
\Lothagiel{} procured some of these weapons and plotted how to get to face \Ishnaruchaefir{} at the right time in the right place. 

\Lothagiel{} knew about \ps{\Ishnaruchaefir} Nadir, but he did not know about the \quo{Achilles' heel}. 
He did have suspicions about the heel, but he never managed to crack the mystery of it. 

In the end, \Ishnaruchaefir{} picked up word that \Lothagiel{} was researching him. 
He did not like to be investigated and decided to put a stop to it. 
So he sought out \Lothagiel, tricked him, lured him out, waylaid him and destroyed him. 
At this time, \Lothagiel{} was still in the preparation phase and not ready to fight him. 
And besides, \Lothagiel{} had never intended to face \Ishnaruchaefir{} in single combat. 

This served as a warning to all: 
\ta{Do not fuck with the Destroyer. 
  Do not plot against the Destroyer.
  Do not even \emph{think} of opposing the Destroyer.
  If you do, the Destroyer will fucking kill you (Steve Ballmer style).}















\section{\Nemuragh}
\target{Nemuragh}
\index{\Nemuragh}
A \thelyad{} \resphan{} of \TiphredSerah. 
Gay lover of \Lothagiel. 















\section{\Quelthah}
\target{Quelthah}
\index{\Quelthah}
A \sathariah \resphan{} of \TiphredSerah. 
Ancestor of \hr{Firaxel}{\Firaxel}. 






















\chapter{Others}
\section{\Achsah}
\target{Achsah}
%\sectioncharunspec{\Achsah}{\resvil}{\female}
\Achsah{}, a \resvil{}, is \ps{\Teshrial} subordinate within the Cabal. 
She operates in \Malcur and the surrounding area. 

She is a Cabalist of the \hr{Cabalist circles}{sixth circle}. 







\subsection{Physique}
\Achsah{} is quite stocky of build. 
Heavier than \Teshrial. 
This is due to her \nephilic{} heritage. 







\subsection{History}





\subsubsection{\Merkyrah}
\Achsah{} was born in the city of Ishiin in \Merkyrah. 
She was one of the original \quo{\hr{Early Resphan fallen ones}{fallen ones}} who joined the rebellion and served the rebels as eager groupies. 

She is an \quo{\ashenblood}, born of a \nephilic{} mother. 
In \Merkyrah{} she was an outcast, despised by the noble \resphain. 
The rebels came and talked to her of how they would overthrow the oppressive order and create a new paradise where the \resphain\dash \emph{all} \resphain\dash would get all the things they were owed. 
They talked of awakening the true nature of their people, and how the outcasts were not bad but simply misunderstood by the narrow-minded and evil church. 
She loved it. 
Also, there were some sexy guys among them. 
Ramiel not least. 
So she joined them and willingly served. 
She was no great hero in those wars, but she served to the best of her ability. 





\subsubsection{Encountered \Ishnaruchaefir}
\target{Achsah met Ishnaruchaefir}
\Achsah once encountered \Ishnaruchaefir on the battlefield. 
She saw his visage, ablaze with fierce chaotic sorcery and hatred toward her kind. 
She felt his presence, radiating a silent promise of death and vengeance. 
And she had not stood her ground. 
She had not even let him come near her. 
She had fled from his path in panic. 
The rest of the battlefield had seemed like a sanctuary, then. 










\subsection{Arsenal}





\subsubsection{High Telepath}
\Achsah{} is a moderately skilled \hs{High Telepath}, but not an expert. 





\subsubsection{Rank}
\target{Achsah's rank}
\Achsah{} is a \bezed{}. 
Because of her lowly origin, she now belongs to the lower class, despised among the \resphain, despite her great skill and power. 

But she is very old and experienced and skilled, so she has climbed up the ladder and now holds a very high Cabal rank. 
She outranks many purebloods. 
This galls them. 
They must obey her, but they are not required to call her by any title. 
(But she is allowed to \emph{not} call them \quo{my lord \thelyad} and the like. Which she would otherwise have to with someone of equal or higher Cabal rank.)

As a commoner, she belongs to none of the \resphan{} dynasties. 









\subsection{Personality}





\subsubsection{Goals}
\Achsah{} dreams of being recognized and loved by her fellow \resphain. 
She has no wings, but she is still a capable, faithful and beautiful \resvil, and worthy of love and respect. 
She hates many \resphain{} because they look down on her and regard her as nothing due to her low birth. 

She doesn't suck up to \Teshrial{} or his ilk. 
She doesn't need their respect. 
She already hates them, and they hate her, so she doesn't try to change it. 
But it hurts her. 
Every reminder that she will never be a true \resvil{} in their eyes is painful and cuts like a knife. 
Even from \Teshrial. 

So when Ramiel comes along and treats her with some more respect, she pledges herself to him. 





\subsubsection{Interests}
\Achsah{} likes children and likes playing with them. 
She is envious of families because, as an \ashenblood, she is sterile and can never have children of her own. 
This makes her resentful of parents, but she likes the children. 
She has, at times, sought out work as a teacher or governess for other \ps{\resphain}{} children. 
It doesn't always work out so well, because they (parents and children alike) scorn her for her impure blood. 

She also likes mortal children and keeps several of them as slaves. 
She treats them well. 
Some think she has sex with them, but this is not true (mostly), and she is hurt by the accusation. 





\subsubsection{Philosophy}
\Achsah{} is a philosophical person. 
There are times when she questions and doubts her evil ways.

\lyricslimbonicart{Under Burdens of Life's Holocaust}{
  A veil of darkness rest upon my shoulders.\\
  I reign and serve and obedient beholder,\\
  'cause the shadows are my hearts domain\\
  and where I wander.
}






\subsubsection{Sexuality}
\target{Achsah's sexuality}
\Achsah{} is bisexual and plays her cruel, demented sex-games on men and women alike.

She likes rugged, manly men. 
She is one of the reasons why \hr{Achsah rewards Charcoal}{she has sex with Charcoal}. 
This is because of her half-\nephil{} heritage, and because she resents the pureblood \resphain{} who scorn her. 






\subsubsection{Wishes she had wings}
\Achsah{} has no wings, and it pains her. 
She often looks up at the sky and wishes she could soar free like a pureblood. 
She can fly using magic, of course, but it's not the same. 
She's not as fast nor as \manoeuvrable, and it just doesn't feel right. 
She always feels like a fake encroaching on the turf of the purebloods. 







\subsection{Politics}





\subsubsection{Attracted to Ramiel}
She was once a groupie of \hs{Ramiel}. She is attracted to him. 

When she learns that Ramiel is back, she helps him in hope of currying favour. Covertly at first, then overtly.





\subsubsection{She hates \Teshrial}
\target{Achsah hates Teshrial}
She hates \Teshrial, because he is younger and less experienced than she but placed above her.















\section{\Damiarch}
\target{Damiarch}
\index{\Damiarch}
\Damiarch{} was a \resphan, the elder brother of \Gevural{} (later known as \Azraid). 
He was the leader of the \hr{Explorers meet Semiza}{expedition that found \Semiza} entombed beneath \Merkyrah. 

\Damiarch{} was noble and good and wanted nothing more than to enlighten his people with the truth, and help them evolve, break the cycle of decay and become greater than they were. 
He \hr{Merkyran rebellion is necessary}{knew that he might have to use brutal methods to do so}, and he lamented this, but the ends justified the means. 
From the start it was \Gevural{} who was the extremist one, willing to go to any lengths for their cause, and with little remorse. 









\subsection{Arsenal}





\subsubsection{Binding souls}
\target{Eryal binding souls}
\hr{Malachim binding souls}{As a \malach}, \Eryal{} had a \sephirah-like power to bind souls to her. 
This ability was especially strong in her, despite her being a lowly \thelyad. 
This was because of her charming and attractive personality, which attracted people to her. 
The souls stayed bound to her almost because they \emph{wanted to}, because the \emph{loved} her. 
Almost. 

The \hr{Banelords wanted to use Eryal's Carcer}{\banelords{} wanted to use \ps{\Eryal} \carcer} as a blueprint for their giant \carcer, \iquin. 









\subsection{History}





\subsubsection{Family}
She is blood-kin to \hr{Sithiyacaan}{\Sithiyacaan}. 
Perhaps even his daughter. 





\subsubsection{Lover of \Shiaraid}
\Shiaraid{} and \Aryal{} \hr{Shiaraid and Eryal lovers}{used to be lovers}. 





\subsubsection{Hesitant to rebel}
In the \Merkyran{} rebellion, \Eryal{} did not want to rebel.
She was optimistic and assumed the best about the \Merkyran{} system. 
She was unhappy and afraid when she heard that \Shiaraid{} wanted to rebel. 
She tried to talk \Shiaraid{} out of it. 
Some of the other rebels caught wind of this and recommended that \Eryal{} be killed so she couldn't inform on them. 
But \Shiaraid{} would not hear of it. 
She knew \Eryal{} would never betray her. 

After much work, \Shiaraid{} finally convinced/bullied \Eryal{} into joining her. 
\Eryal{} felt guilty about it. 
But she wanted to be with her lover, and this was the only way. 





\subsubsection{Role in \SentinelsofMiithEmph}
What is she doing now? 
Probably something to atone for \hr{Silqua blames herself}{the evil she blames herself for having done}.

Perhaps she has something to do with \Sithiyacaan. 
Perhaps she is a voice in his head, pleading and begging him to return and help the world, instead of just trying to forget everything. 

She is not incarnated again. 
She sleeps deeper than most \malachim. 
This is because she is more good and therefore has more traumata to repress. 









\subsection{Politics}





\subsubsection{Ramiel}
\hr{Ramiel resented Eryal}{Ramiel resented \Eryal}. 















\section{\Kezrabal}
\target{Kezrabal}
\target{propaganda minister}
\index{\Kezrabal} 
\index{propaganda minister}
A \resphan{} of \Merkyrah.
Known as the \quo{propaganda minister} of the \Merkyran{} rebellion. 









\subsection{History}
\subsubsection{Early history}
\Kezrabal{} was not on \ps{\Damiarch} expedition, but he joined the rebels soon after. 





\subsubsection{Rebellion}
He was one of the most fanatic, extremist and downright evil of the rebels. 
He hated the church. 
He was obsessed with rebellion, in any form. 
He quickly embraced all sorts of perversity and wickedness, reasoning that \quo{anything else is an improvement}. 
Preferably as \quo{else} as at all possible.  

\citebandsong{DeathspellOmega:SiMonumentumRequiresCircumspice}{%
  DeathspellOmega
}{
  Sola Fide
}{
  O Satan, I acknowledge you as the Great Destroyer of the Universe.\\
  All that has been created you will corrupt and destroy.\\
  Exercise upon me all your rights.\\
  I spit on Christ's redemption and to it I shall renounce.\\
  My life is yours Lord, let me be your herald and executioner.\\
  My actions shall lead other hearts away from salvation.\\
  All shall acknowledge Your sacred royalty and crawl in terrified devotion.
}

He regarded \Merkyrah{} and its religion with contempt. 

\citebandsong{DeathspellOmega:Kenose}{%
  DeathspellOmega
}{
  \Kenose
}{
  Observe \Merkyrah, the chariot of the glory of God\\
  Adrift and exiled, the Pilgrim of Light, grandiose and weeping\\
  Thine aura, compared, is but pale and frail, \\
  alike to the one of an ailing child...
}

He made himself a prophet of evil. 

\citebandsong{DeathspellOmega:SiMonumentumRequiresCircumspice}{%
  DeathspellOmega
}{
  Second Prayer
}{
  Oh Satan, you're the God before whom I stand\\
  Live your life in me,\\
  See how I erase my name from the lamb's book of life\\
  And reject the benefit of the holy wounds\\
  I will walk before thee, lord, in the land of the living\\
  For you teacheth my hands to war and my fingers to fight\\
  And sow seeds that do not proceed of the natural order\\
  They shall grow to columns of the holy lair\\
  That which harbours the dragon with seven heads.
}

\citebandsong{DeathspellOmega:SiMonumentumRequiresCircumspice}{%
  DeathspellOmega
}{
  \Hetoimasia
}{
  Hearken, thou, until I relate things\\
  that shall come to pass in latter ages of the world,\\
  for we are the seeds of the triumph yet to come...\\
  Only a few, in the multitudes upon earth,\\
  shall be aware of what they do, but all will\\
  court the assassination of Christ's redemption, again and again...
}

\index{propaganda minister}%
He \hr{Kezrabal becomes propaganda minister}{became the \quo{propaganda minister} of the rebellion}.
His passion for rebellion made him a great demagogue. 
He had a way of making people follow him. 
They were seduced by the sheer faith and fervour that radiated from him.





\subsubsection{\Sathariah}
He became one of the leaders of \CiriathSepher{} after the rebellion. 
He tried to become a \sathariah, but \Nexagglachel{} destroyed him. 















\section{\Najarod}
\target{Najarod}
\index{\Najarod}
A \resphan{} of \Baelzerach. 
Chieftain of a tribe \hr{Ishnaruchaefir and Baelzerach}{allied with} \QuessanthIshnaruchaefir. 

\Najarod{} is one of the few people who can enter and leave the \hs{Mirage Asylum}















\section[Thanatzil]{\Thanatzil}
\target{Thanatzil}
\index{\Thanatzil}
\Thanatzil{} was the \banemessiah, the first \resphan{} ever and the founder of the \resphan{} race. 

His name is inspired by \href{http://en.wikipedia.org/wiki/Thanatos}{Thanatos}, the personification of Death in Greek Mythology. Interestingly, the Greek Thanatos was the son of \Erebos{} and \Nyx. 

He is a rather tragic \trope{AntiVillain}{Anti-Villain}. 

\lyricsdimmuborgir{Allegiance}{
  Cuddled through a cold womb he was, \\
  pitch black and without sunshine rays.\\
  Hell patiently awaiting him on blood spilled soil. \\
  A noble grief stirred heart, always ready to die.
  
  In sinister systematisation, submission is golden. \\
  As an apprentice to violence, slaughter and bloodshed. \\
  He was like an object (that is) being processed, \\
  a force-fed destructor ready for abomination.
  
  The vast solitude in him witnessed it all, \\
  those self afflicting eyes and their fear painted faces. \\
  Made out of utter discipline, failure unacceptable. \\
  Hosts to oblivion, exploring the darkest of places, \\
  Stench of rotten flesh breathing down his neck.
  
  Every day seemed like an endless night. \\
  When would he ever wake from this void? \\
  No other voice than his own will ever tell\\
  what was real and where he had been, \\
  what he had done.
  
  Life forever lost its innocence, \\
  Never to see the light of day again. \\
  He pondered his last few steps\\
  into the realms of Death \\
  with his hands bloodstained.
  
  Courage and consistency. \\
  Bravery and valor. \\
  Honor and pride. \\
  For what was it all worth?
}









\subsection{Personality}





\subsubsection{Serious}
\Thanatzil had to grow up very quickly. 
He had to be responsible and serious and not childlike.
In Japanese he would be using the pronoun \emph{\quo{watashi}} rather than \emph{\quo{boku}} even as a young boy. 































\part{Other Immortals}























\chapter{\Banes}
\section{\Daggerrain}
%\sectioncharunspec{Daggerrain}{\bane}{\neuter}
\target{Daggerrain}
Daggerrain is the \baneoverlord, the lord of all \banes{} on \Miith{} and surpreme leader of the Cabal. He serves the \baneking{} \Voidbringer, who still reigns on \Erebos. Daggerrain was the one who first contacted \Semiza{} after the \dragons{} had invaded the kingdoms of the \nephilim{} and is at least ten thousand years old. 

In all his thousands of years on \Miith{}, Daggerrain has never seen combat, for as the \baneoverlord{} he considers his person far too valuable to risk by entering battle. He often contacts high-ranking cabalists using \hr{Telepathy}{telepathy}, but other than \banelords{} and the lords of the \resphain, almost no one has physically met him. His residence is unknown.

Daggerrain's goal is not only his people's survival, but also a striving for evolution and perfection. He wants his people to become perfect, the greatest of all creatures of the universe. In this regard, compare him to the Zerg Overmind from the game \cite{VideoGame:Starcraft}.









\subsection{Name}
The name \Daggerrain{} is a rendering of his true name in the \hr{Bane telekinesis}{\ps{\banes}{} tactile language}. 
It was formulated by \Semiza, who felt \ps{\Daggerrain}{} terrible presence and likened it to being stranded amid an endless plain where sharp knives rain from the sky.  









\subsection{Master plan}
\target{Daggerrain's master plan}
\ps{\Daggerrain}{} long-term goal is to re-open the way to \Erebos, the so-called \quo{Gates of Apocalypse}. Compare to the song \bandsong{Mistigo Varggoth Darkestra}{The Key to the Gates of Apocalypse}. 

His master plan is brilliant and fool-proof. Remember to have scenes emphasizing how immensely skilled \Daggerrain{} is at running his \trope{XanatosGambit}{Xanatos Gambits}. 

Even with \ps{\HriistD}{} plotting and \ps{\Ishnaruchaefir} determined effort, the \dragons{} cannot match \ps{\Daggerrain}{} twenty thousand years of planning. 

He is the mastermind behind the wicked \hr{Sephirah plan}{\sephirah{} plan}. 





\subsubsection{The \firstbanewar}
Perhaps \Daggerrain{} never intended to win the \firstbanewar. Perhaps he just needed some fighting to distract everyone from the fact that he was building \Nyx{} as a gateway. He expected to lose the war, but retain the gateway. 





\subsubsection{Blind spot}
\target{Daggerrain's blind spot}
\Daggerrain{} has a blind spot: 
One parameter that he never quite understood and was never able to fully factor into his calculations. 

\begin{itemize}
  \item 
    Perhaps it is \hr{Ramiel betrays Banes}{Ramiel and his betrayal}. 
  \item 
    Perhaps this factor is \hr{Nexagglachel is Daggerrain's blind spot}{\Nexagglachel{} and his curse}.
  \item 
    Perhaps it is \hr{Azraid}{\Azraid} and \hr{Azraid hates Banes}{the hatred he secretly harbours} against the \banes.
\end{itemize}















\section{\Voidbringer}
\target{Voidbringer}
The \baneking{} whom \Daggerrain{} and all his \banes{} serve. 

\Voidbringer{} is a terrible \trope{CosmicHorror}{Cosmic Horror}, a \trope{SealedEvilInACan}{Sealed Evil in a Can} who waits to be unleashed upon the world. 

Compare him to Cthulhu from \authorbook{\HPLovecraft}{The Call of Cthulhu}.

The \Voidbringer{} is the eternal hunger, the endless devouring emptiness, a manifestation of \Bane{} \hs{Entropy}. If it ever comes to \Miith{} it will absorb and consume the \hr{Heart}{Heart of \Miith} and leave the planet a dead husk. Already when people on \Miith{} start summoning it, creatures all over \Miith{} can feel that something is horribly wrong, that some evil force is leeching life force from the Heart.

The \Voidbringer{} is mightier than even the \xss, powerful enough to almost be a \hs{cosmic god}. 

\citeauthorbook[p.122--124]{RobertEHoward:TheScreamingSkullofSilence}{Robert E. Howard}{%
  The Screaming Skull of Silence%
}{
  Silence!
  Utter and absolute!
  Throbbing, billowing waves of still horror!
  Men opened their mouths and shrieked but there was no sound!
  
  The Silence entered Kull's soul; it clawed at his heart; it sent tentacles of steel into his brain. 
  He clutched at his forehead in torment; his skull was bursting, shattering.
  In the wave of horror which engulfed him, Kull saw red and colossal visions\dash the Silence spreading out over the earth, over the Universe!
  Men died in gibbering stillness; the roar of rivers, the crash of seas, the noise of winds faltered and ceased to be.
  All Sound was drowned by the Silence.
  Silence, soul destroying, brain shattering\dash blotting out all life on earth and reaching monstrously up into the skies, crushin teh very singing of the stars!
  
  And then Kull knew fear, horror, terror\dash overwhelming, grisly, soul-killing.
  Faced by the ghastliness of his vision, he swayed and staggered drunkely, gone wild with fear.
  Oh gods, for a sound, the very slightest, faintest noise!
  
  \ldots 
  
  The silence surged wrathfully about him.
  
  Mortal, who are you to oppose me, who am oldetr than the gods?
  Before Life was I was, and shall be when Life dies.
  Before the invader sound was born, the Universe was silent and shall be again.
  For I shall spread out through all the cosmos and kill Sound\dash kill Sound\dash kill Sound\dash kill Sound! 
  
  The roar of Silence reverberated through the caverns of Kull's crumbling brain in abysmal chanting monotones as he struck on the gong\dash again\dash and again\dash and again!
}























\chapter{Cosmic Gods}
\section{Achernar}
An evil star. 
The name is taken from \authorbook{Clark Ashton Smith}{The Isle of the Torturers}. 















\section{Targoros}
A mythical evil god. There is a celestial body, a large red nebula near the North Star, called the Eye of Targoros. 















\section{\XzaiShanns}
\target{Individual XS}














\subsection{\KhothSell}
\target{Khoth-Sell}
\index{\KhothSell}
%A \firstgendragon, the \draconic{} goddess of Death. 
A \firstgendragon, a \xs. 
Often depicted as female.

She was worshipped by the \dragons, \Ortaicans/\rethyaxes and Rissitics.
She was seen as a goddess of life, death, rebirth and immortality. 

\lyricsbs{Hate Eternal}{Catacombs}{
  Lord of Mictlan, land of the dead, \\
  deity of death and darkness.\\
  That which lies upon the graves. 
}

She was monstrous, alien and remote. 
But primal. 
She \emph{was} Death in a very real sense... somehow.
See also the section on \hr{Draconic immortality}{\draconic immortality}.

She was also a goddess of life and fertility. 
Compare her to Shub-Niggurath, the \quo{Black Goat of the Woods with a Thousand Young} from the Cthulhu Mythos. 

\lyricsbs{Emperor}{Wrath of the Tyrant}{
  He is the wind, He is the storm. \\
  He is the woods, He is the roots. \\
  Nobody will escape the wrath of the Tyrant. \\
  Forever the Beast shall wander the Earth.
}

In a sense (according to \hr{Sethican philosophy}{\Sethican philosophy}), \KhothSell was present inside all life.

\citeauthorbook[\quo{First Thought in Three Forms}, p.86--100]{%
  BentleyLayton:TheGnosticScriptures%
}{%
  Bentley Layton%
}{%
  The Gnostic Scriptures%
}{
  It is I who am the sound that was shown forth by my thinking. \\
  For it is I who ma the conjoined.\\
  I am called the thinking of the invisible.\\
  I am called the unchangeable voice.\\
  I am called she-who-is-conjoined. \\
  I am unique, incorruptible.\\
  It is I who am the mother of the sound:\\
  I speak in many ways; I complete the entirety; acquaintance (\emph{gnosis}) exists within me\dash acquaintance with the endless.\\
  It is I who speak in every creature; and I have been recognized by the entirety.\\
  It is I who impart the voice of the sound into the ears of those who have become acquainted with me, who are children of the light. \\
  And I came, for a second time, in the manner of a woman; and I spoke with them.\\
  And I shall instruct them about the coming end of the realmsAnd I shall instructed them about the beginningo fht e coming realm, which does not expereince change, and in which our appearance will change.\\
  They shall become purified within the aeons, in which I showed myself forth in the thinking of the image of my masculinity.\\
  I have put myself witin those who are worthy in the thinking of my unchangeable eternal realm.\\
  For I shall tell you a mystery of this realm,\\
  And I shall instruct you about the agencies that are within it.\\
  Birth is the production of an echo. For hour engenders hour, day engenders day, months produce months [...]. \\
  In such terms, this realm has become complete. \\
  And it has been reckoned, and is slight.\\
  For finger has loosened finger, \\
  and bond has been bound by bond. 
}

















\subsection{\KyaethemChreiAz}
\target{Kyaethem Chrei Az}
A \xzaishann.

\target{Kyaethem Chrei Az associated with air or water}
\KyaethemChreiAz was associated with the sea. 

He was a Cthulhu type, a Kraken, a Leviathan. 
Complete with a huge staring eye. 


















\subsection{\NaathKurRamalech}
\target{Naath-Kur-Ramalech}
\index{\NaathKurRamalech}
\NaathKurRamalech{}, is a \hr{XS}{\xs} who \emph{is} the dimensional barriers around \Miith{}. The Shroud is a part of him and his body, which he has allowed the \dragons{} and \banes{} to shape and twist. 

The gate to \Erebos{} is also a part of him.

Compare to the Outer God Yog-Sothoth from the Cthulhu Mythos by H.P. Lovecraft and others. 

When \NaathKurRamalech{} is invoked people nearby can feel the cosmic darkness stirring and vibrating deep in their bodies and souls. 

\lyricsbalsagoth{The Obsidian Crown Unbound}{
  The Ogre-Mage and the Swordmaster began to utter fearsome words in a tongue which was ancient ere the gleaming stars shifted upon the fathomless countenance of the distant heavens, words which in truth were not words, but rather a resonant key which would aspire to unlock a dire power which had reposed shackled since the fall of the legendary Shadow King himself, whose ebon circlet's power they even now sought to thwart.\\
  The incantation they gave voice to in the midst of that sanguineous turmoil which engulfed them was not so much heard by those within earshot as perceived, sensed as a vague disturbance in the fabric of reality, as fuliginous ripples on the surface of a hitherto still and placid pool, growing ever larger and more far reaching; an unnerving and unnamable sense of change which insinuated itself into the mind of the listener and suggested with a cold and disturbing quasi-certainty that something of preternaturally ineffable magnitude was transpiring, as surely as a festering and gangrenous corpse would split to spill its noisome gore.\\
  And as that maddeningly implacable incantation reached its resounding climax, a momentary silence enshrouded the battlefield, swathing the vista of chaos in an aura of noiselessness more pure and untainted than the tranquility of the boundless and stygian void.\\
  It was as if time itself had halted for one immemorial moment.\\
  And it was in that oddly immeasurable instant that the dark and peerless power unfettered by those grim pseudo-words finally, ultimately, made itself known before the sundered gates of ancient Gul-Kothoth...
}





\subsubsection{Philosophical role}
In \hr{Sethican philosophy}{\ps{\Sethicus} original philosophy}, before the \hr{Crystal Sphere}{\CrystalSphere}, \NaathKurRamalech was given little importance. 
He had only a small role in theology and philosophy.
This changed with the forging of the \CrystalSphere.
The \dragons and \ophidians realized how powerful \NaathKurRamalech was and how greatly they needed him.
He grew in importance in their later philosophy. 

Prayers to \NaathKurRamalech:

\citeauthorbook[p.92]{BentleyLayton:TheGnosticScriptures}{Bentley Layton}{
  The Gnostic Scriptures
}{
  Next, the perfect cild showed itself unto its eternal realms (aeons), which had come into existence for its sake.\\
  It showed them forth and bestowerd glory upon them, and gave them thrones.\\
  It stood at rest within the glory by which it had glorified itself.\\
  They praised the perfect child, the anointed (Christ), the deity, the noly-begotten, and they glorified it, saying:\\
  
  It exists! It exists!\\
  O child of god! O child of god! \\
  It is this that exists!\\
  O eternal realm (aeon) of the eternal realms,\\
  You who gaze at the eternal realms that you have engendered!\\
  For, you have engendered by your will alone.\\
  Therefore [we] glorify you.\\
  Ma! M\=o!\\
  You are omega, omega, omega! You are alpha! You are being!\\
  O eternal realm of the eternal realms! O eternal realm that gave itself!
}





\subsubsection{Status}
\target{Naath is the greatest}
According to some traditions, it was not \hr{Satha is the greatest}{\RuinSatha who was the greatest of the \xss}, but \NaathKurRamalech. 
He was certainly the \pps{\dragons} most important ally when in came to protecting \Miith from the \banes. 

Some believed that \NaathKurRamalech was younger and weaker than \RuinSatha, but simply seemed more powerful because he was closer tied to \Miith and took slightly more interest in the doings of his \Miithian worshippers. 














\subsection{\NerrhanKoss}
\index{\NerrhanKoss}
\target{Nerrhan-Koss}
\target{Nerran-Koss}
A \xs. 
One of the more mysterious of them, \NerrhanKoss{} is an alienist and occultist among the \xss{} themselves and consorts with \hs{cosmic gods} that are as gods even compared to the \xss. 

\NerrhanKoss{} had some dealings with \QuessanthIshnaruchaefir{} and \hr{Glaive origin}{sort of gave him his glaive}. 















\subsection{\RuinSatha}
\index{\RuinSatha}
\target{Ruin Satha}
\target{Satha}
\RuinSatha was a \xs. 
He was a chaotic dominator, said to reign from a basaltic throne at the seething and fiery centre of Chaos.





\subsubsection{Physique}
\target{Ruin Satha and fire}
\RuinSatha was considered the \xs god of fire. 
The \dragons and \rethyaxes derived much of their \hr{Ruin Satha fire magic}{destructive fire magic} from him. 

He was represented as an inferno of fire. 
He was depicted in varying \colours.
Sometimes yellow like regular flames. 
Sometimes green, black, gray or purple. 
Sometimes many \colours at once. 

Make sure he is a sickly and horrid flame. 
Not just a regular flame. 

\lyricsbalsagoth{The Scourge of the Fourth Celestial Host}{
  [UATU:]\\
  And lo, the Exterminator,\\
  the Destroyer of Worlds,\\
  the Purifier of Galaxies...
}





\subsubsection{Philosophical role}
\target{Ruin Satha philosophy}
\RuinSatha was the perfect, all-consuming, purifying flame. 
As such, \hr{Sethican philosophy}{\Sethican philosophy} considered him the ur-example of all life: 
A primal force that devoured and destroyed in order to live. 
A cosmic predator. 
\RuinSatha demonstrated that destruction could bring creation and innovation and new life, which became an important element in \ps{\Sethicus} philosophy. 

In a symbolic sense, \RuinSatha existed inside all living creatures.
He was the hunger, the aggression, the driving force. 

\target{Sethicus believes Ruin Satha created Ophidians}
\Sethicus believed the \ophidians were \hr{Ophidians related to XS}{were related to the \xss}.
He imagined they had been imparted hunger/will/motivation by \RuinSatha, and then physical life by \KhothSell. 
In that order. 
\RuinSatha represented, to \Sethicus, a deeper kind of life, a spiritual life. 
\KhothSell represented physical, bodily life, which was also important, but secondary to spiritual life. 





\subsubsection{Status}
\target{Satha is the greatest}
\RuinSatha was the first \xs with whom \Sethicus made a pact. 
\Sethicus would go on to make many important pacts with him. 
\Sethicus saw \RuinSatha as the greatest and most important of the \xs, and so he was represented as such in \draconian mysticism. 

\Sethicus believed that \RuinSatha dwelt in, and extended out through, the plane of \DaathKurZulNathla, the deepest plane of primal chaos. 
\RuinSatha's fire was a thing of \DaathKurZulNathla's chaos.
As such, \Sethicus believed \RuinSatha had achieved greater spiritual perfection than any other \xs. 
Therefore \Sethicus considered \RuinSatha the greatest of the \xss.
This need not imply that \RuinSatha was the most powerful or ruled over the other \xss, merely that he was the most enlightened of them. 

Not all later cultists (including \dragons and the \hr{Ophidians follow Sethicus under Durance}{\ophidians that idolized \Sethicus under the Durance}) understood this subtlety of \Sethican mysticism. 
In their watered-down theology, \RuinSatha became the \quo{king} of the \xss.

Yet others claimed that it was not \RuinSatha but \hr{Naath is the greatest}{\NaathKurRamalech who was the greatest of them}. 













\subsection{Lesser \XzaiShanns}





\subsubsection{\HothNrul}
\target{Hoth-Nrul}
\index{\HothNrul}
A minor \xs. 

At one point it was \hr{Hoth-Nrul summoned}{summoned to \Miith}. 





\subsubsection{\Ubloth}
\target{Ubloth}
\index{\Ubloth}
\Ubloth was a minor \xs godling that dwelt in a cave in \hr{Kai-Leng}{\KaiLeng} beneath Mount \hr{Shrun}{\Shrun} near \hr{Yormis}{\Yormis}. 

\target{Ubloth cult}
\Ubloth was worshipped by some of the \hr{Cults in Yormis}{\rethyax cultists in \Yormis} and gave them magical power in return. 
The \Ubloth cultists believed that \Ubloth was kin to the \Ortaican \hr{Primordial}{\Primordials} and therefore worthy of worship.
\hr{Moro and the Ubloth cult}{Moro \Cornel believed otherwise}. 

\Ubloth was a fairly simpleminded and unambitious creature.
It took the form of a huge, amorphous thing. 
Compare to Abhoth from \cite{ClarkAshtonSmith:TheSevenGeases}.


% 


\subsubsection{\Yolbaoth}
\target{Yolbaoth}
\index{\Yolbaoth}
\Yolbaoth was a minor \xs. 
He was quite willing to communicate with mortals and appeared rather eager to receive worship and give magical power in turn. 
Many of the darker \hr{Rethyax}{\rethyaxes} worshipped \Yolbaoth. 

Because of \Yolbaoth's great willingness to make pacts, he was invoked in many \draconic spells. 

\Yolbaoth dwelt in \hr{Kai-Leng}{\KaiLeng} underneath \Azmith. 















\section{\Zaz and \Urzaz}
\target{Zaz}
\index{\Zaz}
\index{\Urzaz}
\WanderersInDarknessEmph spoke of a mysterious pair of entities named \Zaz and \Urzaz. 
It was unclear whether these were \dragons, \xss, cosmic gods or even purely metaphorical entities, personifications of something abstract.
Compare to Gog and Magog from the \emph{Bible}.

\hr{Urizeth thinks Zaz and Urzaz are the Chimaera}{\Urizeth thought that \Zaz and \Urzaz} were the same as the \quo{\hr{Chimaera}{\Chimaera}}. 
But their true nature was more complicated and ambiguous than that. 

\Zaz and \Urzaz were real cosmic gods, albeit highly obscure ones. 
Compare them to Kur'oc and Gul'kor.
There was a time when \hr{Zaz denies Ishnaruchaefir}{\Ishnaruchaefir appealed to them for aid}. 

The stars in the sky that represented \Zaz and \Urzaz lay in the \hr{Malgryph constellation}{\Malgryph constellation}.






































\chapter{Imetric Gods}















\section{Dessali}
\index{Dessali}
\target{Dessali}
An \hs{Imetric} goddess who represents reason and knowledge. 

She was originally a Naiad, a water-dwelling spirit or demigod. 
She was a very inquisitive mind and discovered the truth of the Realms and the \feud{} through her research. 
She maintained her sanity by holding on to \quo{reason} as her guiding principle: If her axioms failed her, then she must discard them and seek new axioms. 
This has since then defined her existence and turned her into the badass thinker that she is today. 















\section{Eoncos}
\index{Eoncos}
\target{Eoncos}
Imetric god of war, strength and bodily health. 
His symbol is a \nycan, and he sometimes takes the shape of one. 

He might be related to the \hr{Ortaican gods}{\Ortaican{} gods}. 

He was originally a \nycan{} god worshipped by the \nycaneer{} tribes that lived side-by-side with the \Ortaicans. 
He succeeded in tying all (or most of) the tribes together and \hr{Imetric-Nycaneer alliance}{allying them with the nascent Imetrium}. 















\section{Hiothrex}
\index{Hiothrex}
\target{Hiothrex}
Imetric god of vengeance. 
He is really a \hs{Thorn Angel}. 















\section{\NishiS}
\index{\NishiS}
\target{Nishi-Settias}
Imetric goddess of life and death. 
Might originally be some sort of dryad-thingy. 






































\chapter{\Taorthae}















\section{Daxian}
\index{Daxian}
\target{Daxian}
An \hr{Ortaican gods}{\Ortaican{} god}, associated with weather and the \Wylde{}. 
His partner and wife is \hs{Isxae}. 















\section{Isxae}
\index{Isxae}
\target{Isxae}
An \hr{Ortaican gods}{\Ortaican{} goddess}, associated with law and rulership. 
Her partner and husband is \hs{Daxian}. 















\section{\Nasshikerr}
\index{\Nasshikerr}
\target{Nasshikerr}
An \hr{Ortaican gods}{\Ortaican{} god} of shadows, stealth and the hidden. 
Ofttimes a patron of thieves, spies and outcasts. 
Perhaps also a god of the underworld.

He was really a \quiljaar. 









\subsection{Physique}
\Nasshikerr looked grotesque, \hr{Appearance of Ortaican gods}{as \Ortaican gods tended to do}.
He often took a form resembling a chameleon, with a \scatha-like face and recognizable facial expressions. 

Describe \Nasshikerr as slithering, writhing, loathsomely serpentine.















\section{\NerrhanKoss}
\index{\NerrhanKoss}
Maybe \hr{Nerrhan-Koss}{\NerrhanKoss} the \xs{} is also considered a \hr{Taortha}{\Taortha}. 
An \hr{Ortaican gods}{\Ortaican{} god} of the afterlife and the occult. 
One of the darkest and most frightening of the \Ortaican{} gods. 
He is actually a \hr{front-end}{front} for a \xs. 















\section{Shellagh}
\target{Shellagh}
\index{Shellagh}
An \hr{Ortaican gods}{\Ortaican{} god}, associated with the sea. 
He helps people pass the sea safely. 
His symbol is a shark. 

He might be the son of \hs{Isxae} and/or \hs{Daxian}.
He might be the brother of \hs{Eoncos}. 

He fears \hs{Maegon}, a more powerful sea god who is actually a \nagalord. 
For this reason (and others), the Imetrians look down on Shellagh and see him as a weak and cowardly god. 





























\chapter{Others}















\section{\Haskelek}
\target{Haskelek}
\target{Haskelek story}
The \Haskelek{} is an old ally of the Sentinels and a terribly evil \daemon. 
It was brought to \Miith{} once about 1000 years ago (after Vizicar's reign but before \ps{\Belzir}) where it conquered and controlled a kingdom in central \Velcad{}, current day Pelidor and surroundings. 
But after a prolonged war, the \Haskelek{} was defeated by the Vaimons. 

%Unable to destroy it, the Vaimons imprisoned the \Haskelek{} in his temple, located in the middle of a huge, wild forest south or southeast of Pelidor. There it lay dormant, but its evil seeped out and corrupted the beasts and people in the surrounding area. The forests around its temple became dark, twisted and hostile (even more so than the regular \Miithian{} nature, which can be harsh enough) and the humanoid tribes to become degenerate savages who now worship the \Haskelek, or the memory of it. 

They were unable to destroy it, but they somehow managed to split its essence into several parts. Each part was now weak enough that they could imprison and entomb each of them in separate places. 
% and thus were able to imprison and entomb each of the weakened parts. 

In \emph{\TwilightAngelRemember{}}, the Rissitic and Rungeran armies \hr{Rissitics start Haskelek plan}{capture some strategic spots} that are vital in order to effect the resurrection of the \Haskelek, but they don't \hr{Raising the Haskelek}{begin the actual resurrection ritual} before \emph{\CarzainWithRedcorBook}. 







\subsection{One part is in \Redce}
One of the parts of the \Haskelek{} is hidden in \Redce, in a sacred place which the Redcor have been guarding for over a thousand years. 

This is a nice way to have Carzain fight the \Haskelek, since \hr{Carzain goes to Redce}{he is in \Redce}.









\subsection{Hatred}
When the \Haskelek{} awakens, it is hateful and filled with scorn against mortals. It longs to lash out at them.

\lyricslimbonicart{Behind the Mask Obscure}{
  In hellfire and damnation \\
  my undead soul walks the land.\\
  Throught the endless mist of time.\\
  I`m born to darkened adventures, \\
  retaliated in life obscure.
  
  To seek vengance for my pains,\\
  to serve the hatred in my veins.
  
  In centuries I have wandered,\\
  with deaths shadows drifting faithfully.\\
  In the darkest forests in man's domain\\
  I received my strenght and sorcery.\\
  My demon search for a doorway to be free,\\
  for once again to desecrate the hearts serenity.\\
  Transcend mortality, live throught eternally\\
  and feast upon all misery\\
  that is gathered here in life.
}









\subsection{The \Haskelek{} race}
The \Haskelek{} race is a people of great scientists and sorcerers. 

\lyricslimbonicart{Behind the Mask Obscure}{
  I`m born to darkened adventures, \\
  retaliated in life obscure.
}

The imprisoned one was mightier than most of its kind. 
So mighty that it almost rivaled a \dragonlord. 
Compare to Raest, the Jaghut Tyrant from \cite{StevenErikson:GardensoftheMoon}. 



















\section[Melcryth]{\Melcryth}
\target{Melcryth}
The alleged author to whom the poem \emph{\hr{Wanderers in Darkness}{\WanderersInDarkness}} is attributed. 
\quo{\Melcryth} is believed to be a pseudonym. 
His real identity, or even his race and gender, is unknown. 

Some suspect \hr{Nzessuacrith}{\CryocasNzessuacrith} of being \Melcryth, but she denies this. 































