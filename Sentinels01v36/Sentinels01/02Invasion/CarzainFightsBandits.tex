
\bookchapter{Carzain Fights Bandits}
Read about \maybehs{Carzain}/\maybehs{Ramiel} and about \maybehs{technology}. 

Read some Robert E. Howard before I write the scenes with Carzain and Tantor in the \wylde. 
    
Replace \quo{forest} with \quo{\maybehr{Jungle}{jungle}}. 
Read about jungles. 

Remember to see the section about the \maybehr{Wylde}{\wylde}, and especially the one on \maybehr{Travelling through the Wylde}{travelling through the \wylde}. 
    
Make clear in the first Carzain chapters that Carzain and Vizicar do not literally talk to each other. 
In reality, they have much more direct access to each other's thoughts and memories. 
Their thoughts and memories interact in a more fluid manner. 
It is just presented in the text as an inner dialogue for the reader's sake.

Carzain is a mercenary. 
He is hired by a village or somesuch to fight a gang of bandits. 

\begin{comment}
  \section{The village}
\end{comment}

Carzain has just arrived to the village.
The people of the village are recovering from a recent attack.
Many are dead or violated.
They mourn and curse the \humans.
Make it sad and horrible, so the reader really hates the raiders.
The \scathae complain that this should not happen.
This is not the \Human Age anymore. 
\Humans do not rule \Miith.
Honest \scathae should not live in fear of their depredations.
It is unjust.
Besides, under normal circumstances \scathae are perfectly capable of defending themselves.
It is just this village that is small and peaceful and defenseless, while the \humans are armed and vicious.

Make clear that the village is in the middle of nowhere, an island surrounded on all sides by an ocean of \wylde.
They are part of Pelidor, but cut off from the outside.
A decade ago, the last Vaimon in the village died, and they were unable to find a replacement.
The border totems began to fail, and eventually the \wylde reclaimed the road to the outside.
Now they are alone, and neither the church nor the \rayuth does anything to help them.

The village has only few guns.
The raiders have more guns.
The brave villagers managed to chase off the bandits, but with heavy losses.
They fear the raiders will come back and try to conquer the entire village.
The raiders also have \relcs.
\Relcs are very much military animals.
Armies have them, but few civilians. 
They have other and slower animals.
This means the villagers are hopelessly outmanoeuvred.

There is one female raider, Faeni.
She is the leader's personal lover.
She fires guns too.

The villagers know Carzain's reputation.
When he tells them his name, they are scared. 
He has a reputation as a monster who fights monsters. 
Read about Carzain's reputation! 

They mistrust and hate Carzain for being \human.
But he is a strong alpha male, so he still gets into the village and coerces them into telling him their story.
He is moved (but still sort of stoic).
He is low on supplies and needs the cooperation of the villagers, so he has to do something for them.
He promises that he will get them revenge and retrieve the boy the bandits have kidnapped.
They ask him why he would help them.
He tells them he has no love for his own race.
It is certainly no better than the \scathae.



\begin{comment}
  \section{The bandits}
\end{comment}

There are maybe 10 bandits or so. 
Maybe one or a few leader bandits are big-ass motherfuckers with \nephil blood.
Half-ogres.
These men have the strength of a \nephil and the petty cruelty of a \human.
The most dangerous traits of both races.
A nasty combination.

They are encamped in some spooky, possibly haunted place, like a tower ruin or a cemetery.

The bandits hide in some ruins that still have \wylde \eidola.
It was once a part of the village, but the \wylde reclaimed it and the road to it.
But the totems still have some power.
They keep the worst things of the \wylde at bay, allowing only minor things through.
Minor things that the armed bandits feel confident they can defend themselves from.
(Have some such things, or at least mention them and have the bandits expect to see them. Maybe nycans. 
The bandits are overconfident if they think they can take on nycans.)

Carzain plays on their fears and perhaps even utilizes the magic of the place directly. 
First he hides in the woods around them and picks them off one by one.
Then he takes on the entire rest of the group singlehandedly.

Make Carzain something of a Batman type character who strikes from the shadows and uses stealth, cunning, technology and brute force.
He has to do some really impressive, flashy things that scare the shit out of the villains and impress the reader. 
Such as necromancy.
Don't make him toss fireballs.
Make his magic more sinister, like ripping people's hearts out of their bodies from a distance, or summoning spirits or ghosts to fight for him.

Carzain uses the \qliphah \Shurreem. 
It is a bluff, but do not tell the reader this until after the battle. 



\begin{comment}
  \section{Killing the last bandits}
\end{comment}

When he is fighting the last bandit, the bandit asks him why he, a \human, would side with the creeps.
Carzain says he has no particular love for \humans, and certainly not bandit scum like him.

When Carzain kills the bandits, one of them escapes together with Faeni.
He has long lusted after her, but she has always treated him like shit.
Now he rapes her.
He finishes raping her, and she is lying crying and sobbing and shuddering on the ground in pain.
Then Carzain kills them.
It is hinted that he stayed and watched while she was raped.
Carzain tells her that is was her just punishment for being a bandit girl.

After Carzain has killed all the bandits, he is sad to have had to kill the girl.
She was pretty.
Well, semi-pretty.
But he could not forgive her crimes.
And he would never stoop to rape.
Not even when she deserved it.
Besides, watching that ugly rape sort of turned him off.
He is sad that it was a \scatha village and not a \human one he rescued.

Maybe drop the kidnapping story.
Maybe Carzain just comes back to the village with the heads of the bandits.

Have some POV with Faeni before her rape.
Make the reader really, really hate her.
She tortured those \scathae and loved it.
And she hates the guy who later rapes her.
Make her treat him like shit.

Alternately, maybe Carzain kills the guy when he has only groped and kissed and bound and stripped and beaten and humiliated her, but not stuck his dick in her.
Carzain: \ta{I considered letting him have you, as just punishment for your crimes. But it turns out I am a less cruel man that I thought. Lucky for you.}
Then he kills her.



\begin{comment}
  \section{Killing the last bandits}
\end{comment}

Carzain returns to the village. 
He remarks to himself that the \Shurreem bluff worked. 

Carzain stays a night, maybe. 
Then he moves out.
He has heard a rumour that a nearby town is plagued by a \quo{\dragon}. 
The \quo{\dragon} is probably a \vreiid.
\Vreiid were rare and dangerous giant flying reptiles that resembled the \dragons of mythology. 
They were often referred to as \dragons by the people whom they occasionally preyed upon. 

A \vreiid is a very dangerous monster.
It is hideous and gives off a feeling of great age and evil. 
The fact that the giant creatures fly makes them even harder to defend against. 
But even so, a \vreiid has no magic and does not appear to be highly intelligent. 
Fearsome things, but less fearsome than the \dragons described in mythology.

Carzain is not sure if true \dragons exist. 
But he has some reason to suspect that there do exist something greater than the \vreiiden, something closer to the \dragons of myth. 

If I choose to use the above, then read about \maybehr{Vreiid}{\vreiiden}. 









