
\bookchapter{Carzain Back in \Forklin}
Read about \maybehs{Carzain}/\maybehs{Ramiel}. 

Read about \Forclin. 
    
\begin{comment}
  \subsection{Carzain and Curwen}
\end{comment}
Carzain rides towards \Forklin.

Vizicar is with him in his head and comments on the look of the city. 
He knew the early \Ortaicans{}, so he recognizes the design a little bit. 
But the \Ortaicans{} evolved a \emph{lot} in the centuries after Vizicar's time. 
Have Carzain comment on how the \Ortaican-built city of \Forklin looks very different from Vaimon-built ones such as \Malcur. 
\maybehr{Vaimon Middle-East}{Vaimon things look Middle-Eastern}. 
Vizicar is very interested in Forclin, its design and fortifications. 
It was built after his time, in \ps{\Ortaica} glory days.

He also sees the Ghost Tower and wonders about it. 
It does not look Vaimon, nor \Ortaican.

Also let Carzain see some dinosaurs and pterosaurs. 
Cannibalize the old \quo{Carzain rides to \Forklin} chapter. 

He notices the \maybehr{Ortaican gargoyles}{\Ortaican gargoyles} in \Forklin.

Vizicar notices the statues in \Forklin (of \sephiroth and mortals). 
He looks at a statue of \Feazirah. 
She is depicted kneeling with her head bowed, since she is the \sephirah of Humility.
He concludes that they must be newer than the \caliphate's time. 
Vizicar did not know every piece of his empire (it was huge), so for all he knew, this place could very well have been the site of a major Vaimon city back then. 
The \Ortaican parts could be newer. 
And he does not have the architectural expertise to tell which is older, the \Ortaican or the Vaimon parts of the architecture. 
But one thing convinces him: 
The statues are clothed. 

See, \maybehr{Vaimon modesty}{the Vaimon clans had different ideals of modesty}. 
In his time, this area was dominated by Clan Sether, who were among the least taboo-afflicted clans.
Their statues were naked. 
Nowadays, the area is dominated by Redcor and \Telcra.
Redcor have always been afraid of sex. 
\Telcra have inherited taboos from the Redcor. 
So their statues are clothed. 

\begin{prose}
  Carzain: 
  \ta{So... you look at a kneeling woman and get disappointed because she's not naked.
    And then you spin a long story of history and architecture to hide that.}
  
  Vizicar:
  \ta{Exactly.}
\end{prose}

Carzain sees traces of \Ortaican art and depictions of \dragons.
The \Ortaicans liked \dragons and even worshipped them.
The \Iquinians see \dragons as evil incarnate.
Carzain himself does not know what to think.
But he does think.

The villagers know Carzain's reputation.
When he tells them his name, they are scared. 
He has a reputation as a monster who fights monsters. 









\begin{comment}
  \subsection{Carzain meets Curwen}
\end{comment}
\new
Carzain has dark skin.
He is deviant enough that \maybehr{Carzain is a Demihuman}{some might call him \demihuman}.
Curwen \maybehr{Curwen is a True Human}{is a purebred \truehuman}, and also a racist. 
He respects Carzain in spite of his strange breeding. 


Curwen sits in his tent. 
Someone taps on the tent pole.

Soldier: 
\ta{Sir?}

Curwen:
\ta{What?}

Soldier:
\ta{There is a \human outside the city gates demanding to see you, sir.
  Says his name is \CarzainShireyo.}

Curwen:
\tho{\Shireyo? Here? Interesting.}
\ta{Very well. Take me to him.}

Curwen comes to the city gates and goes out. 
There he sees a \human man in his thirties sitting on a grayish-green \relc. 
As he comes closer, he recognizes the man. 
Curwen has not seen him for more than five years, but this is definitely Carzain \Shireyo. 
\Shireyo fought in the army under Curwen many years ago. 
He was a green but competent Vaimon back then.
After a year's time, \Shireyo had left the army to travel the world as a freelance mercenary.
Curwen has always known \Shireyo was fairly talented. 
It will be interesting to see what he has made of himself. 

Carzain:
\ta{\Mr Curwen. How good of you to come and see me. Did you miss me?}
He hops down from his \relc. 

Curwen comes over and shakes his hand. 
\ta{\MrShireyo. It's been a while.}

\ta{Eight years. You look like yourself, \Mr Curwen.}

\ta{Thank you. You... look less like yourself.}

\ta{Hah. I should hope so.}

Curwen studies him. 
They are the same height. 
\Shireyo is still nowhere near Curwen's girth, but neither is he the slim reed of a youth he was when Curwen met him. 
Back then, Curwen (even though much older) could have bent him in half. 
The current \Shireyo has grown more heavily muscled, but without Curwen's fat. 
And Curwen has not gotten any younger. 
Nor has he lost any weight. 
\Shireyo is probably physically stronger than he is now. 

Carzain was still mostly clean-shaven, and also retained his long, thick mane of black hair. 
Curwen envies him that. 
He never minded when his own hair and beard started to go gray. 
Gray suits him.
But these last years his hair has been thinning, and Curwen is not happy about that. 

Curwen:
\ta{So, what brings you here?}

Carzain:
\ta{I heard Morgan Runger was on the move.}
He grins.
\ta{I had a lengthy debate with myself over which side to pick, but eventually I decided to go with my homeland.
  So I am now willing to lend my aid to the Pelidorian \ishrah.}

\ta{So you're back under my command?}

\ta{I did not say that.}
He grins again.
\ta{We can negotiate the exact terms later. 
  But more importantly, I have some news that may interest you.
  I took the liberty of doing some reconnaisance.}

This piques Curwen's interest. 
The army has sent out scouts, of course, but only commoners.
No mages. 
The \ishrah is small.
They have not had any to spare.
So Curwen is very interesting to hear a mage's perspective on it. 

\ta{Splendid, \MrShireyo.}
He grabs his shoulder.
\ta{Why don't you come inside and tell me what you have learned?}





\begin{comment}
  \subsection{Carzain tells Curwen news}
\end{comment}
\new
Carzain and Curwen sit down to eat. 
Curwen is a gourmet and tells about the exotic foods and wines.

Curwen offers Carzain a smoke.
Carzain declines.

Carzain tells about his reconnaisance. 

When Curwen hears Carzain's story, he is very interested. 
It fits what he has been reading in Tantor's diary. 

Carzain tells Curwen how many guns he saw in the army camp. 
(Don't mention the number on screen.)
Curwen knows that this is a lot of cannons. 
He is less than happy. 
He tells himself he will have to downplay the importance of the cannons when he has to convince Sethgal-tachi that they want to stay in \Forklin. 
    
Remember that there are \maybehr{TBW railroads}{railroads}. 
The Rungerans travel by railroad towards \Forklin. 

Armed with Carzain's information, Curwen convinces Sethgal that it is best to stay holed up in \Forklin{} and meet the Rungerans there. 
See, Carzain's findings further confirms Curwen's suspicion that the Rungerans want \Forklin{} for mystic reasons. 

In the scene where Curwen meets Carzain, have a scene from Curwen's point-of-view where he envies Carzain his thick, blair hair. 
Curwen has mixed feelings about \maybehr{Curwen's appearance}{his own gray hair}. 

Carzain leaves. 
\begin{prose}
  Carzain: 
  \tho{Vizicar, remind me again why we did not accept that smoke.} 
  
  Vizicar: 
  \vizicar{In my days, smoking was a plebeian thing to do. 
    We royals never touched it. And I am not about to start now.} 
\end{prose}


Carzain later walks around in the city. He sees the \maybehr{Morbus}{\Morbus} at work, and how it transforms people into half-dead and later undead abominations. 

He talks to people and hears \maybehr{Haskelek myth}{myths about the Ghost Tower and the \Haskelek}. 
(Or perhaps he already heard that myth when \maybehr{Carzain passes near the Ghost Tower}{he passed near the Tower earlier}.)



\begin{comment}
  \subsection{War room}
\end{comment}

\placestamp
  {War room, Pelidorian army camp}



Curwen-tachi already know the Rungeran army is fairly close. 
They are debating whether they should march out and meet it or whether they should stay in \Forklin. 

Curwen has now learned from Carzain that the Rungerans have some new kind of badass magic. 
Something \Shireyo described as \quo{deeper, darker, more destructive} than anything a regular \ishrah ought to possess. 
\Shireyo also described a brunette woman who seemed to command the \ishrah. 

This fits in nicely with what Curwen has read in Tantor's diary. 
It is very nice and interesting from a scientific standpoint. 
But from a pragmatic, military standpoint it is bad. 

This magic of \EreshKal must be real. 
Curwen can no longer doubt the authenticity of the diary. 
Whatever this \EreshKali magic is, it is real. 
The Rungerans have it, and they are coming. 
And \Takestsha is real. 
Curwen is afraid of her.
He is sure she is formidable.

Bottom line is, Curwen does not want to sally out and meet the Rungerans. 
Their army alone is bad enough.
It is close to twice as big as the Pelidorian army, so the odds are against them to begin with.
If he has to face the Rungerans \emph{and} their new, unknown magic, he wants to do it from a position of strength. 
He wants to hold the higher ground. 
He wants \ps{\Forklin} walls between him and \Takestsha. 

They need to stay in the city.
He must convince Sethgal about that. 



\ta{The Rungerans are advancing faster than we expected,} Curwen was explaining. 
\ta{At this rate they will surely reach Dendrum before we do.
  If they have not already.}

\ta{Then we will have to move faster,} said Sethgal Pelidor. 

\ta{We can't,} said Curwen. 
\ta{Look here, \maybehs{Marshal}.}
He pointed to the map. 
\ta{They were \emph{here} when I left them. 
  That's five days ago.
  We are \emph{here}.
  There is no way we can overtake them.}

\ta{I must agree with Captain Curwen,} said \Dornaer. 
\ta{%
  Given this information it must be clear that we cannot reach Dendrum before the Rungerans.
  We will have to rethink our strategy.}

\tho{Shit,} thought Sethgal. 
\tho{It is as the saying goes. 
  No plan survives the encounter with the enemy.}
\ta{Very well. Then what do we do instead?
  Captain Curwen, what is your impression regarding the strength of the Rungeran army?
  How do you estimate our chances against them on an open battlefield?}

\ta{Not good, sir.}

\ta{Hm,} said Sethgal. 
\ta{I am reluctanct to yield eastern Pelidor.}

\ta{The Rungerans have not been reluctant to claim it,} said \Dornaer. 
\ta{I fear that choice has been taken out of our hands.
  We cannot simply march on them and rout them.}

\ta{True. We will need a city in which to make our stand.} 
Sethgal studied the map. 

\ta{Our best choice is...} Curwen began. 

\ta{\Forklin,} Sethgal interrupted him. 
\tho{I know. 
  Don't try to take over control of the discussion, Curwen.} 

\ta{Do you suggest turning back to \Forklin, Marshal?} asked \Dornaer. 

Sethgal remained silent, thinking. 

\ta{It is our best bet,} said Curwen. 
\ta{\Forklin{} is strong. It can be fortified.}

\ta{But now we have marched all this way!} \Dornaer{} objected. 
\ta{We cannot let that go to waste and turn back!} 

\tho{There. Good. Contradict each other.
  Bring up the arguments for me. 
  I need to consider all alternatives, but if I am to affirm my leadership, I must not be seen arguing a losing position.} 

\ta{We must not be stupid and stubbornly cling to a dead-end strategy,} Curwen argued. 

\ta{But think of the consequences to morale if we turn back!} said \Dornaer.  
\ta{The soldiers will see us as indecisive fools!}

\ta{%
  Think of the consequences to morale if we aim for Dendrum and arrive to find it captured. 
  Or worse yet, if we engage the Rungerans and lose.
  We need a city. 
  And that city is \Forklin. 
  Marshal, surely you see I'm right.}

Sethgal still hesitated. 
\tho{I do not want to cede any part of the country. 
  I need to win this war, \Itzach{} damn it.
  My \rayuthship depends on it. 
  I need to prove my worth so the rest of the House has no choice but to make me \rayuth.
  Slinking back to \Forklin{} and letting Morgan Runger occupy all eastern Pelidor is hardly heroic.}

And there was another reason for his reluctance, one he was less happy to admit to himself. 
\tho{I don't like Curwen.
  I don't want to dance to his tune.
  That damn mage is always trying to worm his way into court. 
  Trying to run things behind the back of the House.
  I have noticed the way he uses the soldiers to do his own covert work when he thinks I'm not looking. 
  
  You may be the son of some obscure nobleman in Belek, but you are not part of Pelidorian nobility. 
  You should not be sticking your nose in politics that don't concern you. 
  You should stick to your magic and leave the government of Pelidor to the Pelidors.
  
  I don't know what you are after. 
  I suppose you are just an ambitious man who wants to gobble up political power and influence.
  Or perhaps you are bitter and hungry because you tasted power growing up as a noble, but was then denied a throne.}
Curwen was, after all, a younger son, and Sethgal knew it was tradition in Belek for the eldest son to inherit the family title and estate. 
This practice seemed foolish to him. 
\tho{%
  Why not choose the most able person, like we do here in Pelidor? 
  
  Or, at least, like we ought to do. 
  If House Pelidor had truly selected the best person for the job, then I would be \rayuth now.
  Electing \Icor{} instead of me was a mistake. 
  I have to make the House see that.%
}

\tho{Are you like me, Curwen? 
  Longing for the recognition you feel you deserve?}
For a brief moment he felt a sort of kinship with the mage. 

\ta{Marshal Sethgal?} asked \Dornaer. 
\ta{What do we do?}

Sethgal stood up. 
\ta{I must agree with Captain Curwen's proposition.
  With the intelligence we have, the best course of action is to retreat to \Forklin.
  Sirs, issue the orders. 
  We turn around immediately.}

Thinking about Curwen's situation had opened Sethgal's eyes a bit. 
\tho{I must not let myself be governed by stupid indignation. 
  I must be rational and do the smart thing. 
  Do what is right for our country.
  Otherwise I am not worthy to be Marshal, much less \rayuth.
  
  I will win this war. 
  I will be \rayuth. 
  I must.
  Pelidor needs a strong ruler.
  
  Besides, someone must relieve poor \Tiroco...}








\begin{comment}
  \subsection{Carzain sees Morbus but not Tower}
\end{comment}
\new
Carzain later walks around in the city. He sees the \maybehr{Morbus}{\Morbus} at work, and how it transforms people into half-dead and later undead abominations. 

He looks for the Ghost Tower and wonders why he can't find it. 
He begins to suspect there is something mystic about it. 
He asks around in the city about it.
He gives money to some beggars and asks them questions. 
(Make sure Carzain is sympathetic so readers like him.)
But people shy away and refuse to answer, or even make warding gestures. 

\tho{Oooookay. There is definitely something mystic about that tower.}

He then rides outside the city again to look at it.
(Now, being affiliated with the army, he can freely leave and enter the city.)

He notices that he can see the tower when he is some hundred paces down the road, or out near the \wylde border. 
(Make it clear that the \wylde is a scary place. Carzain only dares go near it because he is so badass.)
But when he comes closer to the city proper, out of the \wylde, the tower becomes blurry until it is entirely invisible. 

\tho{This is a very interesting phenomenon.
  I will have to remember to ask Curwen about it.}

Remember to have both Carzain and Vizicar speaking. 







\begin{comment}
  \subsection{Carzain reads about Scions}
\end{comment}
\new
Carzain is always researching what he is, always looking for more knowledge of Scions. 
He hopes it will lead him to \apotheosis. 

So he goes into a library in \Forclin{} and looks around. 
There isn't a big public library, but there are churches, and they have books. 
He goes into a church, introduces himself as an \ishrah{} mage and pulls a \trope{BavarianFireDrill}{Bavarian Fire Drill}.
He knows he has to use the same body language, frame control and social engineering skills that he masters when picking up women.
It works.
He gets his hands on the writings.

He also finds references to \maybehr{Vizicar's notes}{Vizicar's notes about \Malachim}, but can find no fragments of the notes themselves. 

He also finds references to \maybehr{Iolivine's notes}{\ps{\Iolivine} notes}. 
He has heard before that these notes are of exceptionally high quality, but he had been unable to find any copies of them. 
\maybehr{Redcor bogarted Iolivine's notes}{The Redcor are bogarting the notes}. 
This is one of the reasons he wants to go to \Redce. 

Among other things, he finds references to \Belzir. 
He already knows she was a Scion, allegedly an incarnation of an \quo{evil} \malach, and that she caused the \HundredScourges with her wicked \malach powers. 









