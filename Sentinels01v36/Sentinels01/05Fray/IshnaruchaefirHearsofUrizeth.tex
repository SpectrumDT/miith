
\bookchapter{\Ishnaruchaefir Hears of \Urizeth}
\placestamp{\maybehs{Mirage Asylum}}

Make sure the Mirage Asylum is dark, sinister and horrible. 
\Criseis should also be surrounded in more dark magic.
Read about the \maybehs{Mirage Asylum}. 

Read about \dragons and technology. 

Have \Criseis use {living machines}. 
    
Remember that the \maybehr{Horrors of the Void}{voids between the Realms are fucking dangerous}.
There lurk all sorts of horrors just outside the gates of the Asylum.
\Criseis (with her super-senses) can see them when she looks out of the big, yawning holes in the Asylum. 

Have throwaway references to \hr{Mystic names}{mystic names and places}, like Shung. 

Do not describe \Ishnaruchaefir clearly.
Cut out after his first line and let his conversation with \Criseis take place offscreen.
Compare to the descriptions of Shai'itan in \emph{Wheel of Time}. 

Alternately, \Criseis never moves close to him.
She only sees him from afar.
Lightnings strike near him as he lies and thinks.
He contacts her with telepathy soon when she enters.
He can tell she is worried and wants to hear her out and thus take her burden of knowledge off her shoulders and upon his own.

\Criseis has been out of the Asylum doing reconnaisance and talking to Sentinel contacts.
Now she is back into the Asylum, having recently surfaced. 
She appears in a special \quo{warp gate} place where it is possible to surface without suffering damage. 

Servants are waiting for her there. 
Some of them come up to her with a mount. 
A \relc. 

\Criseis: 
\ta{Thanks, Kiisuit, but I am too tired to ride. Get the carriage instead.}

The \scatha servant, Kiisuit, brings up a carriage. 
\Criseis gets in. 

Kiisuit: \ta{Where to, \Mistress \Criseis?}

\Criseis: \ta{I need to report to the \dragonlord.}

Kiisuit: \ta{To the \dragonlord. Yes, \Mistress.}

They ride away. 

They ride across causeways and streets. 
The Mirage Asylum looks like a gigantic castle floating free in space. 
A castle with no walls, only floors, stairs, ramps and columns. 
Above and below them there are stars. 

The \quo{stairs} and \quo{ramps} can be dozens of metres wide and hundreds of metres long. 
And winding. 

Of course, all this is completely natural to natives such as Kiisuit who have lived their entire lives in the Asylum. 
\Criseis is one of the few people in the Mirage Asylum to have seen the outside world, so she knows how strange it is. 

In the distance they can see the Observatory, where \Ishnaruchaefir lies resting, \maybehr{Ishnaruchaefir lies thinking in Mirage Asylum}{as he is wont to do}. 
The Observatory is a pedestal where \Ishnaruchaefir often sits/lies when resting and thinking. 

There he lies, in his true, \draconic form. 
His eyes staring into space. 
His body barely moving. 
He is in the \ophidian trance. 
His body hibernates; only his brain is working at full capacity. 
As she comes closer she can feel the difference (with her extraordinaryly sharp senses): 
His body is cold. 
It does not radiate the chaotic fury and warmth it would if he were in his active state. 
No, he is cold like a reptile. 

He does not move at all as she comes close. 
But she can feel his mental presence. 
He is still far away in thought, but he knows she is there. 

When she is few metres from his head she speaks:
\ta{Master \Quessanth.}

His right eye moves to gaze on her.
It is the only movement he has made so far. 
She can feel his mind turning a little bit towards her. 

\Criseis: 
\ta{I have spoken to some of our Sentinel agents. I have news I feel you should hear.}

He does not move, but she can feel his attention focusing on her a little bit more. 

\Criseis:
\ta{It is about this \resphan whom you fought recently. \Teshrial.
  I have learned new things that makes me believe he is more dangerous than I had first thought.}

No reaction. 
But she still has his attention. 

\Criseis:
\ta{I believe you should be careful with this \Teshrial, Master \Quessanth.
  He harbours a grudge against you and is planning his revenge. 
  And he is a clever and resourceful one. 
  
  Do you remember \Lothagiel? 
  The \resphan who was researching how to kill you, some centuries ago?
  Well, it turns out \Lothagiel left behind some notes describing his research. 
  Our spies tell me that \Teshrial has gotten his hands on those notes.}

This made \Ishnaruchaefir stir slightly. 
\Criseis was happy. 
She always felt some satisfaction when her lord deigned to react when she talked to him.

\ta{%
  \Teshrial evidently hopes to continue where \Lothagiel left off. 
  He has found out that there are clues in the poem \WanderersInDarknessEmph and is studying it.}

\Ishnaruchaefir spoke: 
\ta{Is he now?}

\tho{Whoa}, thinks \Criseis. 
If he speaks, he must think this is important. 
I am glad. 
I think it is important. 

\ta{Yes, Master.
  Apparently he is consulting one who is supposed to be an expert on the poem. 
  A \thelyad \resvil, a certain \Urizeth{} of \TiphredSerah.}

\ta{\WanderersInDarknessEmph, you say? Hm...}

\Criseis:
\ta{Master.
  When you killed \Teshrial, you offered to face him again if he were to wish it.
  I believe he \emph{will} challenge you.
  Do you truly intend to indulge him?}

\ta{Perhaps.}

\ta{Please, Master \Quessanth, do not underestimate him.
  He is onto you.
  He is working to find your weaknesses, and I fear he will succeed.
  I do not know how much knowledge is hidden in that poem, but I feel we cannot afford to be overconfident.
  If \Teshrial learns...}

He interrupts her.
\ta{Tell me of this \Urizeth.}
He smiles diabolically. 









