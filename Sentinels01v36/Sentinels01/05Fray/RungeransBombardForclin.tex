\bookchapter{Rungerans Bombard \Forclin}
\begin{comment}
  \section{View from Forclin}
\end{comment}
Remember to describe both armies as \hr{Glorious armies}{magnificent and glorious}.

Remember to read about \hr{TBW technology}{technology at the time of the \thirdbanewar}. 

Sethgal stands on the walls and looks at the Rungerans through his spyglass.
Remember to have Sethgal tell us an estimate of how big the Rungeran army is, and how big their own army is. 

\lyricsbalsagoth{The Obsidian Crown Unbound}{
  At length, the dawn approached tentatively, and with the first signs of the newborn sun etching its promise upon the skies, the martial preparations commenced in earnest.\\
  A brief perfunctory exchange between the Imperial Herald and the fortification's Watch Commander held no surprises, and the Emperor's banner was duly driven into the seared earth before Gul-Kothoth with a chilling finality.\\
  Vast siege engines and powerful ballistae were hauled inexorably into position, alongside a battery of katapelte and petrobolos.\\
  The one hundred thousand strong Imperial Frontier Army, having planted their regimented blazons into the arid soil, waited with a disciplined patience born of never having met defeat in pitched battle or siege, the dreaded Imperial War-Leopards straining noisily against their iron-link leashes to the rear of the cohorts of conscripts and auxiliaries.\\
  The pitiless Iron Phalanx and their Lord Militant Commander had assumed position at the head of the army's Alpha Wing, polished swords, spears and poll-axes reflecting the glow from the myriad torches and braziers which still burned about the Imperial Host.\\
  And behind them were drawn of the legendary Legion of the Ebon Tiger, Pride of the Emperor, the infantry and cavalry famed throughout the Great Northern Continent, personal regiment of the feared general Baalthus Vane.\\
  True to their martial reputation, the six thousand strong Legion were inscrutable in their jet black \armour, their sable banner billowing in the chill breeze which skittered over the plain.\\
  And finally, astride his azure-shaffroned warhorse and surrounded by his elite guard, the silvern-\armoured Emperor Koord himself studied the precipitous gates with a disdainful scruntiny.\\
  At the Emperor's right hand was the renowned Swordmaster of Kyrman'ku, an eastern bladesman of preternatural skill and the most revered and expensive mercenary in the Imperium.\\
  At his left, the infamous Ogre-Mage of the Black Lake brooded silently, swathed in a stygian cloak and fuliginous cowl and exuding an aura of implacable malevolence, which unnerved even the bravest of the Imperial troops.\\
  The Emperor had deemed the services of these two nefarious renegades pivotal to the execution of the Final Campaign, for they alone had knowledge of the mysterious arcane rite known as The Words Which Unfetter.\\
  And, behind their titanic time-worn palisades, the defenders of Gul-Kothoth beheld this awesome force ranged against them and shuddered, not with fear, but with an awful and night-cold anticipation.
}



\begin{comment}
  \section{Cannonade}
\end{comment}
The Rungerans start bringing in their cannons. 
The cannons come in by railroad. 

But the Pelidorians have disabled the railroads. 
They have taken down the rails and dug up the ground so the rails cannot be easily repaired. 
They have also taken down some \eidola along the Ucarn road. 
As a result, the roads have become less safe.
The \wylde is slowly creeping in. 

Destroying the roads is controversial.
The religious people are not happy about it. 
The Iquinians see the maintenance of roads as a religious duty.
By destroying roads, Sethgal is allowing the evil of the \wylde to seep into the world and disrupt civilization.
It is bad etiquette.
But Sethgal does it anyway.
He is pragmatic. 
This can slow the Rungerans down, and that is a good thing. 
It gives his people more time to prepare. 

The Rungerans are forced to haul in their cannons over the rough land. 
They use \nephil ogres to do some of their heavy work.

The Rungerans have to fight their way through land that is gradually becoming \wylde. 
This is not very dangerous, for the Rungerans have a large army, but it does slow them down.
Besides, they have to devote resources to actively maintaining the road behind them so they have supply lines. 
This weakens their combat forces. 
Sethgal is sneaky. 

But the Rungerans will not be deterred. 
They send up forces first to encircle and besiege the city.
Then they stand there. 
They are encamped outside cannon range from the walls. 
Gradually they fill it up with more men. 
At last they bring in their cannons. 
They have ogres that serve as heavy muscle for the cannons, to haul them back into place after each shot. 

The Rungerans begin their cannonade.
This was not the original plan.
They would have preferred a longer siege.
But \Takestsha has received word that she must attack. 
So she presses Morgan Runger into sounding the attack.
They begin bombaring the walls of \Forclin with their cannons. 

The Pelidorian cannons are mounted on the walls and towers. 
This means they should have longer range than the Rungeran cannons.
But the Rungerans have tricks up their sleeves.
The Rungerans have a few enchanted super-cannons. 
They have super-cannons \emph{and} the mages use their magic to further empower the cannon, so it gets even more evil.
Now they have huge range and power. 
And they fire big explosive shots that can punch through walls and even damage the ramparts. 

They also send some shots over the walls to wreak havoc in the city. 
Perhaps some of those shots have disease in them. 
Pieces of animal carrion deliberately defiled with necromantic magic to make them fester with nasty diseases. 
The priests and the Redcor come and try their spells and prayers to kill the infestation before it infests everyone. 

The Pelidorian \ishrah try to counter the Rungeran cannons.
But they are outnumbered and outgunned. 
The Rungerans have more powerful cannons. 
They must have procured some of the most powerful cannons in Runger. 
Besides, they have \uber-long range. 

The Pelidorian \ishrah have no spells that are really effective at that long range. 
Cannons usually have longer range than mages to begin with. 
And the Rungerans are combining magic with cannons, making their range even longer. 
The Pelidorians don't know the spells the Rungerans are using, so they cannot duplicate the feat. 

The Pelidorian \ishrah mages try to combine their powers and strike against the Rungeran \ishrah. 
But the attack fails. 
The Rungerans have set up a magical protection field. 
Such protection fields are static and difficult to move.
But if an \ishrah has to stand still for a longer period, a protection field is very effective. 
It can be powered with big symbols drawn on the ground, and talismans laid out in geometric patterns. 
It offers fairly good protection. 
Besides, when the Pelidorian mages begin their attack spell, the Rungerans immediately detect their efforts and are able to prepare and defend themselves. 

The Pelidorian cannons have an effective range of about 250 metres from the walls (would be 200 metres from the ground).
The Rungerans are firing at a range of 300 metres. 
Archers and crossbowmen cannot fire effectively at much more than 200 metres. 
Muskets have much shorter range than that. 

Move these data about cannons to a more suitable section. 

The Pelidorians have some elite archers. 
They are much more powerful than gunners, but much fewer in number, because they have to be experts.
They hail from a special part of the country where bowmanship is 

The Pelidorians try to return fire with their own cannons in hope of hitting the mages and/or their cannons.
But their accuracy is abysmal at this range. 
They manage to get one shot in that hits fairly close to the Rungeran \ishrah and wounds one mage and some assistants. 
But that is all. 
They never succeed at hitting again. 




\begin{comment}
  \section{Sethgal and Curwen plan sally}
\end{comment}
\new
Curwen is frustrated. 
He goes to Sethgal.

Sethgal is preparing a sally.
He has a lot of cavalry (including heavy monsters) that is just sitting on its ass and eating supplies.
They need to do some good. 
Besides, he is not happy about this bombardment. 
It is taking a toll on the city.
It might send disease running rampant in the city.
And it hurts morale.
And it might even break the walls, allowing the Rungerans to break into the city.
Something must be done. 
(Also, in a scene from Sethgal's POV, mention that there is a naval battle going on on the river, which might be very important. Just mention it a few times, do not actually show the battle.)

Curwen talks to Sethgal.
Curwen is afraid of the Rungeran \ishrah.
He tells Sethgal he has received intelligence that the Rungeran \ishrah has some new doomsday weapon.
At first he thought it was something minor (this is a lie), but then he saw them use the cannons.
The Rungeran mages used spells that Curwen had never seen before, and it was disquieting.
He fears they have some even worse things up their sleeve. 
He wants the sally to do everything in their power to take out the \ishrah. 
Sethgal agrees. 

Sethgal:
\ta{Will you lead the \ishrah into battle?}

Curwen:
\ta{No. 
  But do not mistake this for cowardice. 
  Someone must remain behind and maintain the big overview of the battle.
  Something might happen that I need to react to swiftly.
  And I am the one who understands best what they might be up to.
  I will send \Shireyo to lead some of the mages into battle.}

Sethgal then looks out over the battlefield.
He is grimly determined to rout these Rungerans. 

\lyricsbs{Bal-Sagoth}{
  And Lo, When the Imperium Marches Against Gul-Kothoth, Then Dark Sorceries Shall Enshroud the Citadel of the Obsidian Crown
}{
  A seething forest of blackened blades.\\
  A churning sea of ebon war-chariots.\\
  A searing storm of flaming shafts.\\
  All this havoc and more shall I unleash against my foe...\\
  Into battle! The Legion shall march... the fall of Gul-Kothoth is nigh!
}

See also \cite{RobertEHoward:KingsoftheNight}. 




\begin{comment}
  \section{Sally}
\end{comment}
\new
The Pelidorians sally out. 
The \ishrah{} mages are equipped with plate \armour. 
They are valuable and must not die. 

Remember, the Pelidorians mustn't be portrayed as too good. 
The \hr{Rungeran temple magic}{Rungeran magic} is monstrous, but the magic wielded by the Pelidorian \ishrah{} is also bloody and grim.
The mages hide behind their troops and rain down death on the Rungeran soldiers in the distance using their vicious magic. 
The mages ride on animals such as \relcs. 
(The Pelidorians have \grulcans, but they are not suitable for mages. They have no \lothae or \mezolisks.)

(Do not name any Pelidorian mages other than Curwen and Carzain. That is all I need. Then I can also dodge the problem of who is a Vaimon and who is a \rethyax. Maybe most of them are Vaimons.)

The mages are protected by a vanguard of \grulcan-riders.
They are supported by a heavy cavalry riding \murocs. 
They are led into battle by knights.
A lead knight leads them in some prayers to the One Light before they go out.
The knight might be \Dornaer.
Or even Sethgal.
Yes, I think Sethgal decides to lead the sortie himself. 
He lets \Dornaer remain behind and manage the city.
She is to inherit command if Sethgal falls. 

Remember to have plenty of religion on the battlefield. 
The Iquinians pray for deliverance from \hr{Isphet}{\Isphet} and his evil.
Remember the \hs{Iquinian clerical hierarchy}. 

Remember that both sides have Iquinian knights. 
These have superpowers.

The Pelidorians need to have a strategy.
When they try to attack the Rungeran \ishrah and artillery, the Rungerans will of course suspect this and send their army to try and stop them. 
The Rungeran army will march to meet the Pelidorian sally. 
This will bring the Rungerans within range of the Pelidorian artillery and archers and crossbowmen on the walls, as well as the remaining \ishrah mages. 
The remaining mages on the walls have higher effective range than the ones on the battlefield because they have a better view and better concentration. 
The Pelidorian sally is outnumbered, but it is performing a precision strike. 
They will need all their firepower, magic, artillery and cunning to pull off this attack. 

The Rungerans strike back with their \mezolisks and their ogres. 
(They have no \lothae.)

The Rungerans have a number of \nephil \quo{ogres}. 
They look like \humans, but bigger, uglier and more monstrous. 
They have bestial faces with big-ass nasty teeth. 
Compare to the Ghouls in the game \cite{VideoGame:WarcraftIII} and the more monstrous Persians in the movie \cite{Movie:300}. 
Perhaps they are like the Blunderbores of the game \cite{VideoGame:DiabloII}: 
Giant-sized, wretched abominations, mentally warped and underdeveloped, and kept as slaves, beasts or worse. 
Some of them have had their arms amputated at the elbows, like the giants in a deleted scene in \cite{Movie:300}. This makes them unable to feed themselves in the wild should they flee, so they are forced to remain and live as slaves. 
When they march into battle, some of them have blades attached to their arms, like the monstrous executioner in \cite{Movie:300}.

Move this text about \quo{Blunderbores} to a more suitable section. 




\begin{comment}
  \subsection{Carzain as Batman}
\end{comment}
Have combat scenes at \Forclin with Carzain as a Batman-type dark hero. 
He comes out of the shadows and strikes down his prey. 

\lyricsbs{Hammerfall}{Renegade}{
  He stalks in shadow lands, soundless, with gun in hand\\
  Striking like a reptile, so fierce\\
  No chance to get away, no time for your last prayer\\
  When the prowler sneaks up from behind
  
  An outlaw chasing outlaws, the hunter takes his pray\\
  The law of the jungle he obeys\\
  Craving for the danger to even out the scores\\
  Face to face, once and for all
  
  Renegade, renegade\\
  Committed the ultimate sin\\
  Renegade, renegade\\
  This time the prowler will win
}



\begin{comment}
  \section{Doomed Rungeran Scatha}
\end{comment}
\new
In the siege on \Forclin, have a Rungeran \scatha with some POV scenes.
He is a great and brave warrior.
He is loyal to his king and fights for his country.
He wants to secure a good future for his children.
Runger is \human-dominated, so getting up in the world is non-trivial for a \scatha.
Fighting is his greatest talent, so that is what he does.
All for his children's future.
When the Pelidorians sally out, he leads a counter-charge against them. 

Then the Imetrians arrive.
The doomed guy fights bravely, but he is brutally and swiftly killed by \nycans.

Maybe he manages to {kill Ulphon Nestor the mage}.




\begin{comment}
  \section{Imetrians to the rescue}
\end{comment}
\new
Somehow the attack goes badly for the Pelidorians.
But then the Imetrians show up.
They are only few, but they are enough to shake up the Rungerans and hurt their discipline and morale and destabilize their ranks.
This gives the Pelidorians time and space to rally.

Sethgal had long hoped for Imetrian reinforcements. 
He was about to abandon hope.
But now they come.

A small army of about 500 Imetrians fight at \Forclin. 
Not many. 
But these are not conscripts.
They are all cavalry.
And they are elite fighters.
And they have great \saurians with them. 
And a mage.
And a seasoned hero.
All in all, they are a formidable fighting force, much more so than their numbers might suggest. 

Describe how skilled, disciplined and fearless they are, with their Imetric gods giving them courage and strength. 
They are supernaturally strong and effective.

An Imetric priest says: 

\begin{prose}
  Priest: 
  \ta{We fight an important battle, brethren.
    There are few of us, but the power of our gods will run through us all the stronger for it.}
\end{prose}

And it is true. 
You can see their heathen gods are with them. 

We don't see this from the Imetrians' POV. 
After all, the Imetrians are just a minor player in the story so far. 
They should not get a big, dramatic role until I have had the time to develop them into something cooler, more badass, more background-rich, more well-rounded. 

We see it from Sethgal's POV. 
Or maybe \Dornaer. 
He is one of the only people present who speaks some Imetric. 
Sethgal is happy to have the Imetrians on his side in the coming siege.

\begin{prose}
  Sethgal: 
  \ta{I knew we could rely on you.}
  
  \tho{That's actually not true. 
    I thought we couldn't. 
    I \maybehr{Sethgal curses Imetrians}{cursed the Imetrians earlier} for being faithless allies.
    But they proved me wrong.} 
\end{prose}

Sethgal overhears someone (Ilcas or an Imetric priest) giving the soldiers a peptalk. 

\begin{prose}
  Imetric priest: 
  \ta{If we die on this day, we shall live again!}
  
  Sethgal: 
  \tho{I know the Imetrians believe that they reincarnate when they die.
    I wonder if that is true.
    It is certainly not true for us Iquinians. 
    We die and go into the Light.
    But for them\ldots{} who can say?}
\end{prose}

The Imetrians scare Sethgal. 
They fight with great zeal, \hr{Imetrian coldness}{but their fervour is\ldots{} cold}. 
Calculating. 
Reptillian. 
He is quite disturbed.

(Maybe make a footnote about how the warm-blooded \scathae{} do not see themselves as \quo{reptiles}. At the very least, mention this in the glossary.)

Curwen and Carzain are together.
They hear about the arrival of the Imetrians.
They go to meet them. 

They see the Imetrians. 
Carzain looks at their leader.
\tho{Hey, isn't that... Telcastora Ilcas?}

Curwen goes over to Ilcas and introduces himself.

Ilcas introduces himself as \Retaxis Telcastora Ilcas. 
Ulphon Nestor is also there. 

\begin{prose}
  Carzain: 
  \ta{\quo{\Retaxis}? 
    Is that title new? I seem to remember you had another one last time.}
  
  Ilcas: 
  \ta{%
    Yes. 
    I have been promoted since last we met.} 
  He smiles. 
  \ta{%
    Salacar be praised, I now hold the same rank as my wife. 
    It is just an \honorary rank, mind you. 
    I do not command troops as a \Retaxis{} normally would. 
    I requested this reward because I was sick of being outranked by my wife.}
\end{prose}

Ilcas also meets Curwen. 
Ilcas frowns when he sees \maybehs{Curwen's pistol}. 
It is \maybehr{Rissitic economy}{Rissitic-made}, and Ilcas does not like Rissitics. 
His \maybehr{Ilcas becomes Retaxis}{last major mission}, which gained him the rank of \Retaxis, had to do with opposing the Rissitics. 

Compare them to Haldir's Elves at Helm's Deep in the movie \cite{Movie:LordoftheRings:II} (not present in the book \cite{JRRTolkien:LordoftheRings:II}). 

Remember to read about \hs{Telcastora Ilcas} and the \nycans{} before writing this. 





\begin{comment}
  \section{Razor is more wary of Carzain}
\end{comment}
\new
Have a scene from Razor's POV. 

Since their last meeting, Carzain has grown darker. 
Vizicar has awakened again and is now closer to the surface. 
Therefore, more of his dark, wicked \sathariah{} nature shines through. 

Razor notices this immediately, and he is now more wary of Carzain than before. 
Does Razor hide this, or does he openly display his unease? 
He probably hides it. 
Razor is a sneaky bastard. 





\begin{comment}
  \section{Esmerel}
\end{comment}
\new
\Esmerel is in \Forclin.
She spends part of her time healing the people wounded in the bombardment.
When she has some free time, she goes and watches Carzain from the walls. 
She is interested in Carzain. 
She suspects he is a Scion. 
Maybe.





\begin{comment}
  \section{Sethgal and Ilcas}
\end{comment}
\new
After the battle, have a scene where Sethgal and Ilcas talk about leadership and strategy and experience. 

Ilcas is a good, inspiring leader because of his faith. 
He believes in his cause with great fervour, and this inspires people to follow him.
But he is no great strategist or leader. 
So he can make people follow him, but would not know where to lead them. 

Sethgal is the other way around. 
He is highly skilled as a general, but he does not have quite the same charisma, the same passion. 
His men admire and respect him, but they do not \emph{love} him. 
This is one of his weaknesses. 

Perhaps, Sethgal reflects, this is one of the reasons why he was not elected \rayuth{}. 

Also, remember to display \hs{Ilcas' racism}. 
He accidentally mentions that he and his men are of superior breeding than the Pelidorians. 
Then, when Sethgal is offended, he tries to back down and mitigate what he said. 





