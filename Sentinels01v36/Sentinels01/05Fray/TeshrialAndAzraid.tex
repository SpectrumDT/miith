
\bookchapter{\Teshrial and \Azraid}
Remember to read about \maybehr{Azraid}{\Azraid} and \Teshrial before writing this. 

\begin{comment}
  \section{Azraid's throne chamber}
\end{comment}

\begin{comment}
  \subsection{Teshrial meets Azraid}
\end{comment}

\Teshrial is in \Azraid's palace, waiting. 
He looks around. 
The placy is awfully creepy.
Full of morbid statues and murals.
He imagines he sees scary things in the corners and shadows. 
(Read about \maybehr{Azraid}{\Azraid} for more.)

At last, a servant comes out:
\ta{Lord \Ketheran \Teshrial? 
  The High Lord \Sathariah will see you now.}

\Teshrial is shown into an inner chamber of \Azraid's.
It is even more grotesque. 
The statues and paintings are perverse, blasphemous. 
Things no decent, right-thinking \resphan should want to be near. 

The hall is colossal and dark. 

\citeauthorbook[p.257]{HPLovecraft:TheBlackTomeofAlsophocus}{H. P. Lovecraft}{%
  The Black Tome of Alsophocus%a
}{%
  I was in a long hall with a hight vaulted ceiling that was supported by pillars of purest ebony; along both sides of this chamber were lined creatures of various nightmare shapes.
  Khnu, the ram-headed was there, as was jackal-headed Anubis and Taueret the Mother, terrible in her obesity.
  Leprous beings gibbered and leered, and cancerous things eyed me with malignity; through these ranks of amorphous and hellish creatures my body dragged itself against my will.
  Talons clawed at my arms and legs, and my stomach twisted with revulsion at the touch of their diseased flesh. 
  The air was filled with the sound of their titterings and screamings as they danced obscenely and capered around me in a blasphemous ritual of depravity: and at the far end of the hall was the most terrifying sight of all, that dread black colossus of my visions, the inhabitant of the palace, Nyarlathotep.
}

He looks at a faceless thing and shudders. 
It reminds him of things he would much rather forget. 
\tho{The \banes. Those soul-devouring things that we cannot escape.}

And others give him associations of the horrid \xss that the evil \dragons worship. 
Why does a \resphan of \CiriathSepher have all this ghastliness?

But of course, \Azraid is no regular \resphan. 
He is the High Lord. 
Perhaps even more telling, he is a \sathariah. 
In his veins flows the stolen blood of \dragons. 
It is no wonder that \Azraid is \emph{different}.
No wonder that he flaunts the customs of \CiriathSepher. 
But still, \Teshrial does not like what he sees. 

\Teshrial makes his way through.
He is very careful not to touch any of the grotesque things.
Some of them are definitely alive, and more might be alive. 

\Teshrial is so caught up by the scenery that he does not even detect \Azraid's \vertex as he approaches. 
He is woken from his thoughts by a voice:

\Azraid: \ta{\Teshrial.}

\Teshrial turns to see the High Lord approach.
He bows. 
\ta{My High Lord \Sathariah.}

\Azraid: 
\ta{You may rise.}

\Teshrial rises. 
\Azraid is a short \resphan, actually. 
Little over two metres. 
\Teshrial, with his two-and-a-half metres, towers over him.
But it is only up close that this becomes obvious. 
\Azraid has a great, sinister, intimidating presence around him that makes him seem larger than he is. 

\Azraid's hair and feathers are very pale white. 
Not a beautiful white, like \Teshrial's dyed \colour. 
There is something morbid, bone-like about \Azraid's \colour. 
His natural \colour, \Teshrial believes. 
His hair is thin, wispy and somewhat (tjavset, traevlet, uglet). 
Reminding of a spider's web. 

\Azraid keeps his wings hanging down at his sides. 
The wings are a bit tattered, as if he has lost many feathers. 
His right hand is free. 
The left wing covers his left hand. 
No wonder why. 
(%
  \Teshrial knows \Azraid's left hand is the source of some of his power, and that it is not a \resphan hand. 
  But do not say that so explicitly. 
  \Azraid keeps the hand hidden at most times so he can unveil it for greater effect at dramatically opportune moments.%
)

For all his decrepitude, \Azraid still looks awesome.
His oddities give him an air of otherworldly wisdom. 
(Show this, don't just tell it.)

\Azraid:
\ta{\Zereth's son \Teshrial. 
  I have heard much of your exploits lately.
  You were in battle with \QuessanthIshnaruchaefir, and from what I hear you plan to face him again.
  Not many would make that choice.
  I could name \satharioth who dread to face \Ishnaruchaefir in combat.}
He holds a pause and looks thoughful, almost rueful, as if to indicate that he himself is one of those \satharioth who fear the Destroyer. 
His left hand twitches slightly under his wing.
(\Teshrial does not see the hand, he just notices twitching of that general area.)

\Azraid: 
\ta{One can only admire your conviction/determination/dedication.}

\ta{Thank you, High Lord.}

\ta{So. What can I do for you?}



\begin{comment}
  \subsection{Teshrial asks for help}
\end{comment}

\Teshrial{} talks about his plan to slay \Ishnaruchaefir. 
He has two trump cards:
\begin{itemize}
  \item The Nadir. 
  \item The Achilles Heel. 
    He is researching it together with \Urizeth and expects to have fully cracked it soon.
\end{itemize}

But \Teshrial is worried. 
He knows how dangerous it is.
The Destroyer is a terrible opponent. 
\Teshrial is not confident that he is ready. 
If he is to have a chance to win, he will need more weapons than this. 

\Teshrial also knows that if he wins, it will be a tremendous victory not just for him, nor for his \Malcur venture, nor for \CiriathSepher, but for the Cabal and the entire \resphan race. 
So \Azraid must have an interest in seeing to it that \Teshrial wins. 
So \Teshrial comes to him seeking advice and aid. 

\Azraid can only agree that \Teshrial's current weapons are far from enough. 
\Azraid has encountered and fought \Ishnaruchaefir. 
\Ishnaruchaefir is a force to give pause even to the Lords of the \SitraAchra. 
He has earned his title \quo{Destroyer} many times over. 
Last time they fought, it took \Azraid, some other \satharioth and many allies just to repel the Destroyer. 

\Azraid begins to walk around and talk, halfway to \Teshrial, halfway to himself.

But those times, \Ishnaruchaefir was in top shape. 
\Azraid did not know about this Nadir.
He is very interested to hear about that. 
If it is true, then it is a momentous discovery, and \Teshrial and \Urizeth deserve great credit for that feat alone. 

Another factor plays in:
\Azraid has not faced \Ishnaruchaefir in combat since the Incursion, which is thousands of years ago. 
Since then he has only rarely been seen, and even more rarely been in combat. 
It is possible that \Ishnaruchaefir has grown weaker since then.
He is responsible for many tremendously powerful spells which were draining to cast and even more draining to maintain. 
However, it is also possible that \Ishnaruchaefir has grown even stronger. 
No one knows what he has been up to in his millennia-long exile. 
But \Azraid does not think he has been idle.
\Ishnaruchaefir is too smart for that. 
He has probably been honing his skills. 
Pondering the Aenigmata of the universe.
Gaining new knowledge and spells. 

\Ishnaruchaefir has been hiding in his Mirage Asylum, a hidden Realm which no one has been able to penetrate despite many attempts. 
The Asylum has been one of the Destroyer's greatest assets. 
It is a secret base where he is nigh-invulnerable. 
If only there were a way to breach the Asylum. 

But now there is a way, if not to breach the Asylum, then at least to lure out \Ishnaruchaefir. 
You have done well so far, \Teshrial.
You have found several openings against the Destroyer. 
Merely making him willing to fight you is not the least of them. 

Tell me again about this Achilles Heel. 

\Teshrial describes the myths from \WanderersInDarknessEmph, as well as he has been able to understand them from \Urizeth's explanation.
He tells \Azraid of the \malgryph and of \Zaz and \Urzaz. 

\Teshrial: 
\ta{I will utilize his Achilles Heel, as described in the prophecies.}

\Azraid: 
\ta{Prophecies. 
  You do not believe in such superstition, do you?}

\Teshrial: 
\ta{No, my High Lord \Sathariah, not literally, of course.
  But myths are often truth shrouded in poetry.}

\Azraid: 
\ta{True.
  \WanderersInDarknessEmph has proven itself full of remarkable insight in the past, hidden in symbolism. 
  So you may be right. 
  But I hope you are interpreting this correctly, \Teshrial.} 

\Azraid{} is not sure he believes in the Achilles Heel. 
(This is foreshadowing of the fact that the heel is fake, so that does not seem like an \trope{AssPull}{Ass Pull}.) 

\Azraid: 
\ta{%
  Can it really be true, that \ps{\Ishnaruchaefir} greatest weakness was just under our noses all this time?
  It seems to good to be true.}

\Teshrial{} comments (inside his head) on the fact that \maybehr{Azraid's appearance}{\Azraid{} has wrinkles}. 
And his monstrous hand, which he always keeps hidden\dash with good reason, as \Teshrial{} knows. 
The hand twitches as \Azraid wanders around and muses. 

\ta{But \Ishnaruchaefir{} is known to take reckless chances.
  You just might fool him. 
  Very well. 
  You have my blessing. 
  But you are right. 
  You will need more weapons.}

\Azraid comments that just defeating and killing \Ishnaruchaefir would be a momentous feat.
Only a very few times has that ever been accomplished. 
But if they just kill the Destroyer, he will return and they will have accomplished nothing. 
They must permanently destroy him.
This is a much harder task, but they must do everything in their power. 
The \resphain have been given a rare chance.
If \Ishnaruchaefir is not destroyed and revives, who knows when they will get another chance to destroy him?

\Teshrial: 
\ta{Yes.
  And if anyone can help me, it must be you, my High Lord \Sathariah.}



\begin{comment}
  \subsection{Azraid gives suggestions}
\end{comment}

(%
  Wait. 
  Why is \Azraid so interested in destroying \Ishnaruchaefir? 
  I though the two were laying plans together.
  Maybe the below part where \Azraid suggests methods to kill \Ishnaruchaefir should be moved to the chapter where \Teshrial inspects his \noggyaleth.
  Maybe \Azraid only suggests the \neoresphan treatment.
  Maybe it is \Urizeth who suggests the Shroud and the \noggyal ambush.)

\Azraid: 
\ta{What are the terms of his challenge?
  Do you get to choose the time and the battle ground?}

\Teshrial:
\ta{He did not say, but I intend to try.
  If I tell him the time and place I choose, he would have to be a coward to refuse.}

\Azraid:
\ta{I am not sure how much faith you should put in \Ishnaruchaefir's trustworthiness.
  He has not always been known to fight \honour{}ably.
  There are reasons why his own brother calls him a betrayer, after all.
  But still... 
  \Ishnaruchaefir does seem to have a certain code of \honour.
  Yes, it is worth the try.
  I assume you intend to meet him at the depth of his Nadir.
  But where?}

\Teshrial:
\ta{I... I do not know, High Lord.}

\Azraid:
\ta{You will want to set the duel deep in the Shroud.
  There he will be weakened.
  As a colossal \vertex, he will be constricted by the Shroud and unable to bring his full power to bear.
  You, as a smaller \vertex, will be less constrained.}

\Teshrial:
\ta{But, High Lord, will that not breach the Unspoken Covenant?}

\Azraid makes a dismissive gesture and almost snorts.
\ta{To the \SitraAchra with the Unspoken Covenant. 
  There are more important things at stake here than secrecy.}

\Teshrial is taken aback by this.
He had not expected the High Lord to flaunt one of the basic laws of the Cabal. 

\Azraid:
\ta{You will want to set a trap.
  I hear you and your fellows have done much progress with setting up a project in \Malcur.}

\Teshrial:
\ta{\Malcur? Uh, yes, High Lord. 
  We are still wrestling for complete control of the \nexus, but our \noggyaleth are by now firmly entrenched and have dug deep into the Shroud there.}

\Azraid:
\ta{Ah, yes. 
  \Noggyaleth.
  Very good.
  And the Sentinels do not suspect?}

\Teshrial:
\ta{They suspect, certainly.
  They know about our activity there, and about the \nexus.
  The \noggyaleth, on the other hand, are a well-concealed secret.
  They are not aligned with our \matrix there and have not even touched the \nexus.
  I am convinced the Sentinels do not suspect their work.}

\Azraid:
\ta{But the Destroyer has been to \Malcur in person?}

\Teshrial:
\ta{Yes, but we worked quickly to delay him and pull back the \noggyaleth.
  When he came close to the surface, there is no way he could have detected them.
  I died seeing to that, as you know.}

\Azraid:
\ta{Good. 
  Well done, \Teshrial.
  So we have ourselves a trap.}

\Teshrial:
\ta{You mean... tell the Destroyer to meet me \emph{in} \Malcur?
  And have our \noggyaleth lie in wait... yes.
  Excellent.}
\Teshrial thinks.
The \noggyaleth are powerful.
They are terrible monsters.
(He is afraid of them and shudders to think of them. Read about \Teshrial and \noggyaleth.)
If anything can help him overcome the Destroyer, it is those things. 

This means the fight must take place in \Malcur. 
\Teshrial is \skeptical about that. 
He fears it will hurt the \Malcur venture.
\Azraid convinces him it is a good idea.
Much must be risked in war. 

\Teshrial:
\ta{But do you not think he will suspect a trap, High Lord?}

\Azraid:
\ta{Of course. 
  \Ishnaruchaefir is as cunning as any of us. 
  He will more than suspect.
  But he has a tendency to want to keep his word.
  Besides, he is still a \shaeeroth against a single \resphan.
  He will be confident of his superiority.
  He may allow you much freedom, perhaps even for the sake of sport.
  \Ishnaruchaefir has been known to take great risks in the past.
  With skill, we can exploit that.}

Again \Azraid's hand twitches. 
His wings slip, and for a brief second \Teshrial can see a misshapen finger stick out between the feathers.
It has a hideous blotched brown \colour. 
It is long and stretched and gnarled. 

\Teshrial asks if \Ishnaruchaefir will take the bait and accept such terms for the duel.
\Azraid suspects that he will. 
\Azraid knows that \Ishnaruchaefir{} is inquisitive and willing to take silly chances or be chivalrous for the sake of his curiosity. 
\Teshrial can take advantage of this. 
It can buy him time to pull off some of his gambits and traps. 

\Azraid:
\ta{It is something of a coincidence, though, is it not?
  Your venture in \Malcur is nearing fruition. 
  \Secherdamon's Sentinels begin taking increasingly drastic measures to thwart you.
  And now, not long before the conjunction, \Ishnaruchaefir appears.
  Granted, coincidences do happen, but it is suspicious.}

\Teshrial:
\ta{What do you suspect, my High Lord \Sathariah?}

\Azraid:
\ta{I do not know.
  \Ishnaruchaefir must be up to something, but I cannot figure out what.
  \Secherdamon, at least, I can try to predict.
  I have fought him continuously since the Incursion.
  I have come to know him well.
  But \Ishnaruchaefir... he remains a mystery to me.
  He has hidden himself well.}

\Azraid walks around again.
\ta{\Ishnaruchaefir is not in league with \Secherdamon, is he?
  No, that cannot be.
  They have hated each other for ages.
  And I have read the \matrices. 
  The Exile is nowhere close to intersecting with the Pyre.
  No, that is crazy talk.
  
  Does \Ishnaruchaefir have some plan that crosses both the Cabal and \Secherdamon?
  Or does he have a goal independent of his brother's?}

\Azraid spreads his hands and wings (keeping the strange hand concealed).
\ta{Well, I doubt we will solve this riddle today.
  Let us focus on the matter at hand.}

\Teshrial:
\ta{Yes, High Lord.
  So far we have the Nadir, the \malgryph, the Shroud and the \noggyal trap on our side.}
\Teshrial smiles. 
At last he is beginning to feel good and confident.
\ta{With this, I believe I have a chance to prevail.}
\tho{Yes. The Destroyer will fall.
  And \Firaxel will love me.}
He beings thinking about \Firaxel.
How beautiful she is, and how wonderful sex with her will be. 

\Azraid:
\ta{Yes. Perhaps.
  But we can never be too sure.
  I have one more idea that may help you, \Teshrial.}

\Teshrial:
\ta{You do? What is it?}

\Azraid:
\ta{A new weapon that my researchers and I are developing. 
  Something that will make you part of the \emph{future}.
  Come, let me show you.}






\new
\begin{comment}
  \section{Laboratory}
\end{comment}

\begin{comment}
  \subsection{Coming to the laboratory}
\end{comment}
\Azraid flies with \Teshrial out of the throne room.
\Azraid keeps his evil hand hidden in his robe when flying.
They fly down into the deep.
Into the lower levels of the tower.
Here there are laboratories where \Azraid's scientists work on their sinister projects. 

\Teshrial comes into the laboratory halls. 
They are scary.
He can see strange swirling lights.
Hear strange noises.
Smell strange scents. 
He looks around. 
In one dark corner, far above the floor, he thinks he dimly discerns the hideous shapes of \SitraAchra \screamers. 



\begin{comment}
  \subsection{Azraid tells about the weapon}
\end{comment}

\Azraid explains about some of the experiments that his scientists are working on. 
First he swears \Teshrial to secrecy.
These are special experiments that can be vital to the Quest for Perfection and the future of the \resphan race.
But they are still in early development stages, and \Azraid does not want them known to the general public. 
\Teshrial must swear not to reveal what he sees. 
He agrees. 

\Azraid shows him some \neoresphain. 
Or at least prospective \neoresphain. 
\Teshrial sees them (offscreen) and shudders. 
He is horrified, but also fascinated. 

Do not describe to the viewer what \Teshrial sees. 
The \neoresphain should be kept mysterious.
Describe only what \Teshrial feels. 
Horror and amazement.
Describe the overall impression the \neoresphain give, and perhaps describe a few body parts. 

\Azraid:
\ta{This is the future of our race you see.}

\Teshrial:
\ta{Yes...}
\Teshrial can imagine that. 
He sees potential for greatness here.
Terrible greatness. 
He feels the wingbeats of history.
He feels something immensely vaster and mightier and eviler than any \resphan.
He sees an image of the past, present and future legacy of the \resphan race.
He sees vistas of their cruel greatness.
A race that will conquer the stars and galaxies. 
It is beautiful but terrifying.

\Azraid explains that it normally takes years or decades to create a \neoresphan.
But they are also working on some other experiments.
They have some treatments that are supposed to metamorph a regular \resphan into a \neoresphan, and much faster than the traditional treatments. 
So far the treatments have been problematic, but recently the scientists have made what they believe to be a breakthrough.
The new version of the metamorphosis spell is ready, but it is not field tested, and it has only been tested at a part of its potential strength, not at full power. 
They fear that the full-powered version will have unhealthy side-effects. 

The scientists believe it will work. 
It it (almost) certain that it will bring great power, but it may come at a cost. 

\Teshrial is not sure he catches \Azraid's drift. 

\Azraid explains.
The metamorphosis gives a \resphan a power boost. 
It will be very useful for \Teshrial if he is in trouble and about to lose his duel.
Then he can use the metamorphosis. 
If he is in trouble, it will save him from permanent destruction. 
Any side effects are preferable to destruction. 

\Azraid{} is good at marketing the thing. 
\Teshrial{} will bring extra glory to himself if he does his people this further service and field-tests their secret weapon. 



\begin{comment}
  \subsection{Teshrial considers and agrees}
\end{comment}

\Teshrial is uncertain.
He hesitates.
Looks at the \neoresphain some more.

He is afraid. 
But then he thinks of his quest.
His goal.
\Firaxel.
He remembers that \Firaxel is a scientist and very interested in the Quest for Perfection.
She will be happy and impressed with him if he takes this great risk for the sake of the Quest and the advancement of science.
\Teshrial resolves to go through with it. 
For his quest's sake. 
For \Firaxel. 

He tells \Azraid he agrees. 

\Azraid-tachi are happy.
He also undergoes some alchemical and magical treatment that is meant to transform and strengthen his body. 
They give \Teshrial some medicine which he must take regularly.
They will set in motion the preliminary stages of metamorphosis.
This stage is well-tested and should work. 
But the main thing is the metamorphosis spell, which they will teach to \Teshrial (or give to him in the form of another elixir or item). 
It will activate all the \neoresphan matter inside him (which he will by then have developed with the help of the medicines he is to ingest). 
This will metamorph him into a true \neoresphan. 
He will gain power like never before. 

It works. 
When the day of the duel arrives his body has been strengthened, so he is more powerful than normal even in his \resphan{} form. 
(Mention this in the later \Teshrial chapters, and show him taking the medicine.)

\Teshrial agrees. 
He takes the weapon, but makes no guarantees. 
He does not try to hide that he is still anxious. 
He intends to hold it back as a last-resort back-up weapon, 
He stresses that he will use the metamorphosis only if he is about to lose. 
He fears it and will not use it unless he has to. 
\Azraid accepts this. 
He wishes \Teshrial the best of luck and success. 





