
\bookchapter{\Teshrial and \Criseis and \Ishnaruchaefir}
\begin{comment}
  \section{Resphan citadel}
\end{comment}

\Teshrial is sitting in his citadel near \Malcur with \Ganethed and some \bezedeth. 
They are discussing their plans for \Malcur.
Their plan is progressing nicely.
The \noggyaleth are burrowing.
The Silver-Shining Rose \matrix is getting well entrenched in the \Malcur \nexus. 

Have some references to the fact that stuff is easier now than it was some centuries ago. 
(This is because the Shroud is unravelling.)
\Teshrial is young, so he does not believe it.
\Ganethed is older.
He clearly remembers how the Shroud has become more and more permeable lately. 

Then suddenly the klaxons go off. 
Their measuring apparatuses have detected a powerful \vertex in \Malcur. 
\Ganethed goes over to one of the machines and looks at the readings. 
It bodes ill.
It is \Ishnaruchaefir. 

\Teshrial: 
\ta{Are you certain?}

\Ganethed: 
\ta{No, but I think it is him.}

\Teshrial is distressed. 
\tho{Crap. 
  What is he up to now?
  I am not fully prepared.
  The Nadir has not begun yet.
  I am not ready to fight him at full strength.
  But we must do something.}
  
\ta{We have to go down there and confront him.}

\Ganethed:
\ta{Wait. 
  Think this through.
  Look at the apparatus.}
The thing is reacting madly.
All lights are flashing. 

\Ganethed:
\ta{Look how obvious it all is.
  Last time when the Destroyer came, he was circumspect about it.
  Only the High Telepath, \Achsah, was able to detect his \vertex signature.
  Now the machine is reacting madly.
  And this is not even a very sensitive machine.}

\Teshrial:
\ta{Is that not just because he has gotten closer to the surface this time, with no one to stop him?}

\Ganethed:
\ta{I do not think so.
  I suspect that whatever the Destroyer hopes to accomplish in \Malcur, he could have done it in a more stealthy manner than this if he wanted to.
  No, I think this is a deliberate provocation.
  He is being obvious in order to bait us.
  He wants us to go down and meet him.}

\Teshrial:
\ta{You may be right.
  But we cannot do nothing, can we?}
\tho{What damage can he do if we let him run unchecked?
  Can he detect the \noggyaleth and take action against them?
  Possible.
  Will he be able to sever our \matrix ties to the \nexus?
  Possible.
  I do not know what he is up to, but I cannot take the risk.}

\Teshrial:
\ta{I am going down there.}

\Ganethed:
Sigh. 
\ta{Very well.
  I will come with you, \Teshrial.}
He addresses the two \bezedeth.
\ta{And so will you two.}

The \bezedeth are not happy about that.
One of them casts a glance at the measuring apparatus. 
They look nervously at each other, clearly thinking:
\hypota{Us? Confront the Destroyer?
  Confront a \vertex of that magnitude?
  Are you crazy?}
But they shoulder up and do what they must. 

Then one of them says:
\ta{Ought someone not to stay behind?
  I mean, if the Destroyer means to lure us out, it might be because he wants to divert our attention from something else.}

\Ganethed:
\ta{Hm.
  Good point.
  Very well.
  You stay behind.
  You come with us.}



\begin{comment}
  \section{Travelling to \Malcur}
\end{comment}
\new
The three \resphain set out for \Malcur.

Read about the Shroud.
Read about the Beyond and the horrors of the void. 

Remember that the \resphain cannot travel freely through the Realms.
They must travel along special pathways so as to remain safe from the horrors of the void. 
They follow the safe \quo{caravan} routes laid out and empowered and kept safe and maintained by the Silver-Shining Rose \matrix. 
They are themselves aligned with the Rose, so they can travel its caravan paths. 
And they can easily see the paths.
They light up for them like shining lines or tunnels through the chaotic, twisting void.

\Teshrial casts some spells to light the way and keep the pathway stable and closed.
In a sense they are drilling a tunnel through the void.
A sealed tunnel where horrors cannot attack them. 
There are things out there that even the \resphain fear. 
\Teshrial sees glimpses of fearful shapes; the elder monstrosities of the void. 
But he averts his eyes and stares straight ahead. 
That is the best way, he knows. 
Keep focus on the path ahead.
That way, you coerce the Shroud to work with you and shape itself around you, keeping attackers out. 

On the way down, \Teshrial thinks about the situation.
\Ganethed is probably right. 
\Ishnaruchaefir is a devious snake. 
He wants to lure the \resphain out. 
They are doing exactly what \Ishnaruchaefir wants them to do. 
But they still have to respond.
Who knows what he would do if they do not come?

Who knows what he will do if they do come?
Will he try to kill them? 
Or is the \bezed right?
Is it a distraction from something else entirely?
It could be.

\Ishnaruchaefir promised \Teshrial a rematch.
Has he come to claim that rematch now?
\Teshrial feels fear and anger. 
But also pleasure.
It is exciting.
Maybe he will fight \Ishnaruchaefir again.
Their first fight was dreadfully traumatic, but also exhilarating.
Some part of him wants to feel that excitement again.
From a certain point of view that is insane, he reflects.
But that is his \resphan nature, and he would not trade it away.
Remember, \resphain love conflict. 



\begin{comment}
  \section{Arrival in \Malcur}
\end{comment}
\new
The \resphain arrive in \Malcur.
Well, not quite \Malcur itself.
Not the surface of the Shroud where the mortals can see them. 
But close. 
Close enough to be well-protected by the Shroud. 
Close enough to be safe from the predators of the Beyond. 
Close enough that \Ishnaruchaefir will be weak. 
Now that they are here, they can clearly feel the tremours themselves. 
They follow them to their source. 
It is the dead garden.
There they find \Criseis, but no \Ishnaruchaefir. 
She carries his presence with him.
She acts as an anchor, connecting him to the Shrouded Realm. 
Just like she did last time. 
He is drifting somewhere in the Beyond, beneath the surface of \Azmith. 
He must be close to the surface, but not close enough to be visible to the naked eye. 
Not through the Shroud.

He is close, but not so close that he is starting to break through.
In fact, as far as \Teshrial can determine (measuring by eye, by rule of thumb), \Ishnaruchaefir is no closer to the surface now than he was when \Teshrial-tachi set out from \Nyx.
Strange.
Surely \Ishnaruchaefir could have broken through into \Malcur if he really wanted to.
This supports the theory that the Destroyer is baiting them and wants them to come and confront him. 



\begin{comment}
  \section{Criseis is afraid}
\end{comment}
\new
\Criseis sees the \resphain.
She stiffens, intimidated by the menacing \resphain. 
Actually she already detected them from far away.
But she fakes surprise.
She wants to keep her super-senses a secret. 
She makes a peace sign.
She is afraid of \Teshrial.
He is more bitter and hateful towards her than last time. 
But he does not harm her. 

\Criseis is afraid.
Not just for herself but for her master as well.
She knows \Ishnaruchaefir fears \WanderersInDarknessEmph.
He has expressed this fear many times. 
But she does not know that he has planted his own myths in it.
But she knows he has a plan.
She saw him smile diabolically when she told him about \Teshrial's plan.
She just hopes her master's plan is enough.

But maybe I do not want to show all the above. 
I do not want to paint \Ishnaruchaefir as a good guy.
Maybe the reader should know only little of \Criseis's thoughts. 



\begin{comment}
  \section{The Resphain land}
\end{comment}
\new
The \resphain land. 

\Teshrial takes no chances.
He draws his \senaan. 
This is not \Turishah, but some lower quality \senaan he has picked up temporarily.
He misses \Turishah.
It was a great weapon.
But he will have to make do.
No time to worry about that now. 

\Teshrial:
\ta{You again.
  What are you and your master up to?}

\Criseis bows and replies in the \resphan tongue:
\ta{Greetings, Lord \Teshrial.}
She makes the peace sign again.
\ta{I am not here to fight.
  Please do not harm me.}
  
\Teshrial is not happy when she says that.
It feels like a nasty threat in his ears. 
He remembers the stories of \Ishnaruchaefir's atrocities. 
He remembers how much he hates the Destroyer. 

\Criseis:
\ta{My master simply wishes to send you a message.}

\Teshrial:
\ta{Message?}



\begin{comment}
  \section{Ishnaruchaefir possesses Criseis}
\end{comment}
\new
\Criseis suddenly spasms and convulses. 
\Ishnaruchaefir comes to possess her.

The \resphain recognize that \Ishnaruchaefir's presence is surfacing. 
\Ganethed moves forward as if to attack.

\Teshrial:
\ta{No! Wait! Do not attack!}

\Ganethed:
\ta{But if we kill her, we may be able to sever his anchor.}

\Teshrial:
\ta{%
  Maybe she speaks the truth.
  Maybe he is just here to send a message.}
\Ganethed still looks unconvinced. 
\ta{%
  Trust me, we do not want to kill her.
  If we do, the Destroyer will exact a gruesome revenge.}

\Ganethed acquiesces.
So they watch events unfold.

\Ishnaruchaefir then comes up to possess \Criseis's body and speak through her.
\Criseis goes into a trance. 
\Teshrial can recognize the feeling of \Ishnaruchaefir.
He can see the suggestion of a vast, black, \draconian form around her. 
He feels much Cosmic Horror. 
\Criseis's physical body remains unchanged, but \Teshrial can discern \Ishnaruchaefir's soul and presence permeating it. 
Her scales are still her usual bronze-like reddish \colour, but when he looks at her with his aethereal senses she seems almost to morph into the Destroyer's black \draconian shape. 
And he can feel \Ishnaruchaefir's massive, fearsome power radiating from her. 
\Teshrial is relieved that he is deep in the Shroud where \Ishnaruchaefir cannot easily enter and cannot easily bring his full power to bear. 
We are not in the Nadir yet, but we are close. 

\Criseis opens her eyes.
They shine yellow. 
Through them, \Teshrial imagines he sees the mystic fire of the \xss, from which the \dragons derive their magical power. 

She smiles diabolically.
Though her mouth has only flat \scathaese teeth, \Teshrial imagines he sees a long row of sharp \draconian teeth and a snaking tongue. 

She opens her mouth and speaks in a deep, growling voice.
In that voice echoes the howling of the blind, furious \daemons of the lightless voids. 

\Ishnaruchaefir:
\secherdamon{We meet again, \Teshrial.
  So, you had the courage to meet me.}
He speaks \CommonDraconic, which \Teshrial understands to some degree. 
\secherdamon{I see your health has improved since I last saw you.}

\Teshrial{} acts overconfident and arrogant. 
He fakes bravado. 
Like a whelp who thinks he is something big. 
This is what he intends. 
He hopes to provoke \Ishnaruchaefir{} and goad him into coming to fight him, to \quo{teach him a lesson}. 

\Teshrial:
\ta{Hm.
  You refuse to show yourself. 
  It does not take much courage to face your \scatha lackey. 
  What do you want?}

\Ishnaruchaefir:
\secherdamon{I promised to fight you again.
  I grow tired of waiting, \resphan.
  Are you ready to face me?}

\Teshrial:
\ta{I am sorry, Exile, but I do not fight at your command.}

\Ishnaruchaefir:
\ta{Indeed?}
His lips quiver as if to laugh. 
His expression says: 
\tho{As opposed to today, and the last time I forced you to come out and fight me?}

\Teshrial ignores the taunt. 
He keeps his head up high.
\tho{Remember, \Teshrial. 
  Be proud.
  Be arrogant.
  Be stupid.}
\ta{But I will tell you when.
  Come back here to \Malcur, in the Shroud, on [some date].
  Then I will fight you.}

\Teshrial notices the \scatha's eyes widen when he tells him the date.
It is at the center of \ps{\Ishnaruchaefir} Nadir. 

\Ishnaruchaefir lays back his head\dash \Criseis's head\dash and laughs.
\ta{%
  Hahaha.
  I see you have done your research well. 
  You and that scribbler of yours, \Urizeth.}
\Teshrial involuntarily feels his face harden when \Ishnaruchaefir mocks \Urizeth.
\Teshrial still hates him for that dishourable assassination.
In fact, \Teshrial surprises himself a bit.
He did not realize he cared for \Urizeth so much.

\Ishnaruchaefir:
\ta{%
  I am touched that you would go through all that effort for me.
  It must have been hard and tedious work.}

\tho{%
  Is that fear I detect?
  Is he afraid?
  Is he faking bravado to hide how much my words affect him?
  Just like I am doing?}

\Teshrial:
\ta{Do you refuse?
  Your legend attributes many flaws to you. 
  Cowardice is one of them.
  Besides, you have been hiding for millennia in your Mirage Asylum.
  Feel free to refuse if the thought scares you.}
\tho{I am not overacting, am I?
  No.
  This is not uncommon \CiriathSepher rhetoric.
  I am not breaking character.}

\Ishnaruchaefir:
\ta{%
  Hah.
  Nay, I will not renege on my word. 
  Not today, at least. 
  You are smaller than I, so it is only fair to grant you this concession.}
He switches to \TrueDraconic to seal the agreement.
\ta{%
  So be it, \Teshrial.
  Thou shalt have thy duel.
  I will return here on [some date] and face thee.
  But know this:
  I will face thee and thee alone.
  No other \resphan nor \resvil shall fight alongside thee.
  Dost thou accept?}

\Teshrial struggles to understand the \TrueDraconic.
It is hard. 
The language has never been fully deciphered.
\ta{%
  Of course, Exile.
  What do you take me for?
  You will fight no other \resphan nor \resvil but me.
  I would have it no other way.}

\tho{\quo{No other \resphan}. 
  Hah.
  That is easy enough to promise.
  But just wait and see what I have in store for you, Destroyer.}

\Ishnaruchaefir replies in \CommonDraconic again.
\ta{Good.}
He addresses the other \resphain.
\ta{Remember this.
  If your Cabal ever wants to kill me, \Teshrial here is your best bet.
  You do not want to waste this chance.}
Back to \Teshrial:
\ta{%
  Be sure to keep your word, \Teshrial. 
  If you break it, I will refuse to fight, and return later to exact a terrible vengeance.
  I will abide by a certain code of \honour, but you know what I can do if you force my hand.}

\Teshrial remembers the story of how \Ishnaruchaefir terrorized the \resphain after \Criseis' siblings were murdered, so he takes \Ishnaruchaefir's warning very seriously.

\Teshrial reminds himself how quickly and easily \Ishnaruchaefir found out about \Urizeth and how easily he murdered her.
This shows how deep his Cabal spies go and how good his intelligence is.

\Teshrial remembers how \Ishnaruchaefir murdered \Urizeth.
Now \Ishnaruchaefir claims to have a code of \honour.
This seems hypocritical and offensive to \Teshrial.
He calls \Ishnaruchaefir out on the fact that he ambushes defenseless \thelyad \resviel and murders them. 

\Teshrial:
\ta{%
  I give little for your so-called code of \honour, Exile.
  Ambushing and murdering a defenseless \thelyad \resvil.
  Is that what passes for honourable among your race?}

\Ishnaruchaefir: 
\ta{Hah. I let her revive, did I not?
  In retrospect, maybe I should not have.
  But a little astrology will not be enough to defeat me.}

\tho{Ha! He does fear the poem and the truths it holds.
  And he does not know half of what I know.
  Just you wait, Destroyer.}

\Ishnaruchaefir whispers in a low voice.
He confides this to \Teshrial alone, with a sardonic gleam in his eye: 
\ta{%
  Let me tell you something.
  Tracking and killing her was easy. 
  I could do it again any time I wish. 
  For instance, I have heard that there is a certain other \resvil in your life, \Teshrial.
  Let us hope she comes to no harm.}

(This is a bluff. It was through luck that \Ishnaruchaefir was able to get at \Urizeth so quickly and efficiently. He doubts he would be able to do the same with \Firaxel.)

\Teshrial:
\tho{Firaxel! He knows about \Firaxel!
  Don't you dare lay a hand or claw on her, you vile fiend.}
\Teshrial wants to shout and curse, but he controls himself.
Even though he is dealing with a dishonourable opponent, he is still \CiriathSepher. 
He will not let his temper get the better of him. 
Instead he forces himself to hold his head up high.
He meets \Ishnaruchaefir's eyes.
He sees the immense evil burning in those eyes.
The dark, primal evil of the \xss. 
The malice of the dark stars burning in the farthest cosmos. 
It takes a test of will to not look away.
But \Teshrial manages to hold that terrible gaze. 
He says: 
\ta{Spare me your threats, \Ishnaruchaefir.
  I am no betrayer like you.
  I will keep my word.}

\Ishnaruchaefir:
\ta{Good.
  Till we meet again.}
He turns around and walks away.
\Teshrial sees a brief glimpse of the dark, twisting chaos of the void as the Destroyer tears the Shroud aside. 
\Criseis's body submerges and vanishes from sight. 
The rip quickly seals itself, and the Shrouded world is back to normal. 

(End the scene here.
 \Teshrial-tachi stick around until the Destroyer's \vertex presence can no longer be felt. 
 Then they retreat.) 









