
\bookchapter{\Urizeth Revives}

Remember to have plenty of \maybehr{Resphan wing body language}{wing body language} from the \resphain. 

At first, \Teshrial did not like \Urizeth.
Later he warmed to her, after her display of loyalty and bravery and determination, enough to rival his own. 
(He is, after all, much physically stronger than she is, and so in less danger of being destroyed. He is a ketheran, male and a great martial artist. She is \thelyad, female and a nerd.)
She also warms to him.
At first she sees him as an arrogant, self-important, condescending snob.
But later she realizes he is a fairly decent \resphan with noble motives.
And he learns to lose some of his snobbery.

Make clear that \maybehr{Resphain love conflict}{\resphain love conflict}.
\Urizeth sort of likes having been killed. 
She enjoys hating \Ishnaruchaefir. 
She has never been killed in combat before, so this intense passion is new to her. 
She now understands a bit of how a warrior feels. 
This brings her and \Teshrial closer together. 

\begin{comment}
  \section{Teshrial meets Urizeth}
\end{comment}

\Teshrial has received word from \Urizeth. 
She has asked him to come see her in her tower. 
She has evidently recovered enough to be seen, but not enough to go abroad. 

\Urizeth{} is thoroughly killed, so it takes several days for her to come back. 
Six days or so after her death, she sends word to \Teshrial{} and asks her to come see her. 
He does. 

(Read about \Urizeth! Make her really eccentric! She is worse now!)

When he sees her, she is in terrible shape. 
She looks like a mummy; a shrivelled, mutilated husk of a \resvil. 
She can talk and move her arms, but her legs and wings are not yet fully regrown, so she is confined to a wheelchair for the time being. 
\Teshrial{} is grossed out, but he fully sympathizes. 
He remember how badly shape he himself was in after \Ishnaruchaefir{} had killed him. 

It is a disturbing experience for \Teshrial.
The \CiriathSepher have a taboo against seeing other newly-revived \resphain (other than close family members and friends).
\Teshrial had never seen a newly revived until recently when he himself died.

Compare to Efrel in \cite{KarlEdwardWagner:DarknessWeaves}. 

\Teshrial is impressed at the level-headed manner in which she bears her injuries. 
It must be a \TiphredSerah thing, he concludes. 
A \CiriathSepher would be devastated and lock himself up and let no one but his most trusted confidants see him until he was fully healed. 
\Teshrial knows; that was what he did.
A \Mystraacht, on the contrary, would probably display his wounds with pride, as a testament to his bravery or somesuch. 
The \TiphredSerah, as far as he understands, have a philosophy that \quo{looks can be deceptive} and that appearances should not be given much weight. 
\Teshrial supposes this mentality is the reason why \Urizeth is able to bear her wounds with so little emotion. 

\begin{comment}
  \section{Urizeth tells her story}
\end{comment}

\Urizeth{} tells him the story of how she was killed, and the warnings and threats \Ishnaruchaefir{} gave her. 

\Ishnaruchaefir{} seeks out \Urizeth{} while she is out in a Shrouded Realm, visiting a demesne. 
He sneaks up on her as close as he can while avoiding detection (which is not very close). 
Then he charges. 
She immediately senses a huge-ass \vertex{} coming straight at her, so she tries to flee. 
But she is unprepared. 
Besides, she is no athlete and hence no fast flyer. 

\Urizeth does not see \Ishnaruchaefir.
She only feels him approach as a tremendous \vertex, a behemoth tremour in the Web of Realms. 
She tries feebly to defend herself, but he blasts her out of the sky and kills her before she even sees him.
    
Remember that the \maybehr{Horrors of the Void}{voids between the Realms are fucking dangerous}.
Maybe \Ishnaruchaefir came from out of the void, where no sane being would otherwise dare venture. 
He was almost a \quo{horror of the void} himself. 

\Ishnaruchaefir was like a \xs. 
Have \Urizeth describe him as if he were almost a \xs.

He catches her. 
He blasts her with an attack spell. 
It does not kill her, but it shreds her wings, grounding her. 
Now she cannot escape. 

\begin{prose}
  \Urizeth: 
  \ta{You... \Ishnaruchaefir!}

  \Ishnaruchaefir: 
  \ta{\Urizeth.
    It has come to my attention that you are... researching me.
    I must interpret this as a challenge. 
    And I am not pleased.
    I fear I must make an example.
    To you and all your kind.}
\end{prose}

Then he kills her. 
She defends herself, but she knows she has no chance. 

When she dies, he makes a mock attempt at destroying her soul. 
She fights back, using all the \TiphredSerah{} stealth and cleverness at her disposal. 
She succeeds, and her soul survives and eludes his grasp.
He lets her think she outsmarted him, but in reality he wanted her to do it. 
Maybe he might have been able to destroy her, but he let her go. 

\Urizeth's body was badly damaged, but it was not destroyed. 
After the assassination her body was retrieved by some of her family or friends.
Since then she has been slowly healing. 

\Urizeth has been killed, and easily.
But \Urizeth{} refuses to give in to \Ishnaruchaefir's threats. 
She also refuses to wait for her body to recover. 
She wants revenge on the evil \dragon. 
She wants to get back to work as soon as possible, so she can help \Teshrial{} devise a way to rid the world of this cruel monster for good. 



\begin{comment}
  \section{Urizeth has mapped the Nadir}
\end{comment}

But it will not be easy. 
The \dragons are terrible beings, and none more so than \Ishnaruchaefir.
\Urizeth confesses that until now she had never met a \dragon in the flesh.
She had read much about them and done research, but only from a theoretical point of view. 
She \quo{knew} how terrible and mighty they were, but not until now did she really understand it.
Now it is suddenly real to her in a new way. 

But there is hope.
\Urizeth has mapped \Ishnaruchaefir's Nadir cycle. 
She has noticed an interesting thing in \ps{\Lothagiel} notes: 
When \Ishnaruchaefir does appear during his Nadir, it is without his glaive. 
This is an important clue. 
The glaive is perhaps his most powerful weapon, so it makes no sense for him to not wield it. 
He must be unable to, for some reason. 
This is useful to know. 
It is well-known that \Ishnaruchaefir's \vertex is closely tied to the glaive, and it to him. 
After all, the glaive lay at the heart of the events that made \Ishnaruchaefir the Exile, back during the Incursion. 
Yes, the glaive is central to the Destroyer's Aenigma. 

Do not call the glaive \quo{\Rystessakhin}. 
Just \quo{the glaive}. 

\Teshrial: 
So, when is the next Nadir. 

\Urizeth: 
Very soon. 

She remarks that it is no coincidence that the Nadir falls now.
It falls at a time when the Shroud is in \quo{ebb}. 
Their own plan is also scheduled around the ebb and flow of the Shroud.
But, as far as she can calculate, the deep point of \Ishnaruchaefir's Nadir comes slightly before the climax of their plan. 
Very likely \Ishnaruchaefir plans to weather his Nadir and then quickly come back in time to fuck up their plan.
But if they are lucky, \Teshrial can arrange to fight \Ishnaruchaefir at the bottom of the Nadir, before the climax.
Thus saving their asses.

Perhaps \Ishnaruchaefir needs not attack and destroy stuff in order to be a threat. 
I just need to clarify that if he is not stopped soon (chased away or preferably killed), he will wreck everything they have worked for in \Malcur.
When he is at his full strength, he could attack in force and drive the Cabal out of \Malcur entirely.
It is known that he takes an interest in \Malcur, so he likely has long-term evil plans there.
He gave hints of that in WSB. (Make him give hints!)
That must not be allowed to happen.
Furthermore, even now that he is weak, he might be up to something.
If \Urizeth's conclusions are correct, then these Nadirs happen to him regularly, and if so, \Ishnaruchaefir must have learned long ago to live with them and still get stuff done.
One must not assume that he is harmless in his Nadir.
(Maybe it is \Azraid who speculates the above to \Teshrial.)

\Teshrial professes how glad he is to have \Urizeth back. 
She is also glad to be back. 
Have a slightly heartwarming scene that shows how the two have become better friends. 

\Urizeth promises that she will not get killed again. 
She believes she is on the verge of some more breakthroughs that will help them against \Ishnaruchaefir. 
It is difficult, but there is light at the end of the tunnel. 
She is hopeful.
So he becomes hopeful as well. 

The chapter ends on this note of hope and the promise of more discoveries to come. 









