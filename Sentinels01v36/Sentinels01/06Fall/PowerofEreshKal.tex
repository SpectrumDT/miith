\bookchapter{The Power of \EreshKal}
\begin{comment}
  \section{Takestsha and Morgan Runger}
\end{comment}
\new
\begin{comment}
  \subsection{Morgan Runger}
\end{comment}
We follow Morgan Runger. 
He has made an alliance with the Rissitics. 
Originally, he was thrilled at the prospect of rising to power as an ally of Durcac, but of late he has come to doubt, for two reasons. 
One, he fears that \Nechsain{} will subjugate him as a vassal rather than a true ally. 
Two, and perhaps more importantly, he beginst to ponder the ethical consequences of his actions. 

He oversees \Takestsha{} and her fellow sorcerers as they invoke the terrible spells gleaned from the tablets recovered from \Rungertemple, and he witnesses firsthand the horrible destruction they cause. 

He imagines that the \sephiroth{} are weeping, mourning his fall from grace.

\lyricslimbonicart{Beneath the Burial Surface}{
  The sky is darkening, soon the night befall.\\
  Righteously angels are weeping for my soul.\\
  All childhood dreams are soon to be lost,\\
  all innocence to be shattered.
  
  I am the fallen from grace.
  
  My face is a river.\\
  See my eyes as they drown in black.\\
  My sacred doom and nemesis\\
  beneath the burial surface\\
  To the final act of the immortal sin\\
  I am lead by funeral winds.
}

He sees the terrible wickedness of the \EreshKali{} magic.

\lyricslimbonicart{Darkzone Martyrium}{
  Black energies in the twilight space\\
  come shivering through the shallow haze.\\
  Into darkness so impure divine.\\
  A bloodshed emotion to evil wine.
}

Cannibalize the scenes with \Takestsha from \quo{The \Caliph Inviolate}! 
(Look in the \quo{Carzain Prequel} folder.)

\begin{comment}
  \subsection{Sex with Takestsha}
\end{comment}
\begin{prose}
  Morgan Runger had approached her from behind and was few steps away now. 
  She had smelled and heard him approach from a mile off, of course. 
  \ta{They were Pelidorian scouts,} she told him, not turning. 
  \ta{We have chased them off. I say we let them run.}
  
  \ta{Mhm. Someone will deal with it,} said the king. 
  Untroubled. 
  Morgan had seen the fire, as had everyone, but evidently it did not scare him. 
  He was confident that his bolstered \ishrah{} could counter any sorcerous attack. 
  Just as she wanted him to be. 
  
  He came up close behind and wrapped his arms around her. 
  Began to grope her body. 
  \tho{%
    Heh. Delegating responsibility so you can have your pleasure? How very kingly.}
  Morgan had grown quite shameless about their affair, despite the fact that, as a nominal \Iquinian, he was theoretically obliged to be faithful to his wife. 
  But he was a king and could do whatever he wanted. 
  Do \emph{whomever} he wanted. 
  \tho{%
    And, as any man might, he wants to remind everyone how beautiful a woman he is fucking.}
  
  % She doesn't mind, of course. 
  % As a \dragon{} she has no sexual shame. 
  \Takestsha{} reflected that perhaps she ought to fake some shame and modesty to make her guise as a \human{} woman more believable. 
  But then again, she was playing the role of the mysterious and erotic sorceress, and part of her allure was her rejection of conventional morals. 
  
  \tho{%
    Oh, yes. 
    I suppose I had better have sex with Morgan. 
    Have to keep my pet king compliant. 
    
    Tee-hee.
    The discovery of the Scion has made me in a good mood. 
    Who knows? 
    I might even enjoy it tonight.}
\end{prose}

\begin{comment}
  \subsection{The magic of Eresh-Kal}
\end{comment}
The \Rungertemple{} magic might be defiler magic (as in \emph{Dungeons and Dragons: Dark Sun}), sucking life out of the world and leaving it gray, dusty and dead. Alternately, it might be bestial, destructive and chaotic magic, apalling in its sheer hate, ferocity and inhumanity. 

The magic involves the conjuration and binding of terrible \daemons{} from \Chaos. These should be as horrible, inhuman and Cthulhu-like as possible. The \daemons{} are the source of their power; they're the ones wreaking the destruction. 

It is hinted that the \daemons{} are not really bound; they are just playing along for their own unfathomable reasons, and may decide to turn on the mages any time. Have at least one scene where the sorcery suddenly backlashes on one of the mages and he dies a horrible death, his flesh boiled and burnt and his soul consumed by the \daemon{} he unleashed. 

Make it clear that the foolish \humans{} are playing with powers far beyond their understanding. 

It is very hard, taxing and traumatic work for the mages. The sorcerers, being ill-informed and ill-educated in the use of this great power, are twisted by it. Their bodies become warped and misshapen, and they go more and more mad. Compare to Hannan Mosag and his K'risnan in \cite{StevenEriksonIanCameronEsslemont:MalazanBookoftheFallen}.

\lyricsbs{Monolith Deathcult}{%
    1917 - Spring Offensive (Dulce Et Decorum Est)
}{
  Creeping like a snake from a can, \\
  the slithering stench of yellow death.\\
  Chemical flame of decay\\
  burning skin and intestine.\\
  Regurgitating the bloody guts.\\
  Spewing last life from a wretched soul.
}



\begin{comment}
  \subsection{Takestsha}
\end{comment}
\Takestsha is disappointed. 
The Pelidorian sallies have been quite effective. 
She is not winning, and she should be.
She is in a hurry.
She should take \Forclin today. 
It is time for her to bring out the really big guns. 

\Takestsha and her mages have to start their big attack spell. 
\Takestsha knows it is risky. 
Her mages are not holding up as well as they should. 
They are weaker than she had hoped. 

If it were up to her, \Takestsha would be patient and lay a prolonged siege. 
But \hr{Psyrex tells Nzessuacrith to capture Forklin quickly}{\Secherdamon and \Psyrex have asked her to make haste}. 

\target{Takestsha will not become Nzessuacrith too soon}
\Takestsha \emph{knows} what \Secherdamon's plan is.
She knows her own attack is a decoy.
Her mission is to attract the \resphain's attention and fool them into coming to fight her. 
Conquering \Forclin is just a means to that end. 
She can, of course, assume \draconian form right away.
But she will not break the Unspoken Covenant for no reason.
She will first go as far as she can in her \human guise.
Only when she absolutely has to will she break her disguise.
If she were to take \draconian form too early, it would be suspicious, and the \resphain might not be fooled. 
It must look like she was forced to unveil herself. 

So she has to act in haste. 
She decides she must take some risks she would otherwise not have taken. 

So she and her \ishrah begin their great spell that will bring down \Forclin. 
It may be foiled, but \Takestsha hopes it will work. 

\Takestsha must push the Rungeran mages really hard. 
Push them to their limit and beyond it. 
Her master spell is a colossus on feet of clay, and she knows it. 





\begin{comment}
  \section{Attack against the Ishrah}
\end{comment}
\new
The Pelidorians want to attack the Rungeran \ishrah. 

Maybe there have been several quicker sallies. 
Previously the Pelidorians just tried to do as much damage as they could. 
Now Curwen believes the Rungeran \ishrah is becoming a real threat. 
He asks Sethgal to send a sally out to destroy the \ishrah.
Sethgal is convinced and agrees. 

The Imetrians support them.
The Imetrians brought one mage with them. 
Unfortunately, he was killed in their initial charge. 

Sethgal will lead a diversion. 
Ilcas will lead the real strike force that will strike at the \ishrah. 
Carzain will support the Imetrians with his magic. 
Curwen will stay in \Forclin. 
The rest of the mages will support Sethgal. 





\subsubsection{Curwen goes to the Ghost Tower}
\target{Charcoal at the Ghost Tower}
Throughout much of the story, Archibald Curwen (Charcoal), supposedly the sneaky master Cabalist, is duped, manipulated and played for a fool by his enemies. Sentinels and other agents seem to run circles around him. 
He's been played for a \trope{XanatosSucker}{Xanatos Sucker} the whole book. 
But at the end he finally realizes what's going on around him and strikes back. 

Near the end\dash perhaps after having discovered one or more of the people who have been cheating him, such as \Sanyor{}\dash Charcoal shows what a formidable agent he truly is. 

Curwen realizes that \Takestsha and her \ishrah are too powerful and dangerous.
He cannot just use conventional tactics against them.
Instead, he gets an idea. 
He devises a master plan to dispel the Rungerans' \EreshKali magic and strike a hard blow to their forces. 
But knows that will be hard. 
He cannot do it without help. 

So he gets another idea. 
He can use the Ghost Tower.

Curwen has been in \Forclin before. 
He knows the Ghost Tower and has even been inside it. 
He knows it is a conduit to the Realm where the \resphain live (though he has not been there). 
He formulates a plan to use the Tower in his counterspell against \Takestsha by channelling energy through the Tower's \nexus, or something like that. 

So he leaves Carzain in charge of the \ishrah and departs the battlefield, heading for the Ghost Tower. 





\subsubsection{Curwen contacts \Achsah}
On the way to the Tower, Curwen recites an orison to contact \Achsah and ask for her advice. 

\begin{prose}
  \Achsah:
  \ta{What? Do you have urgent news about the Sentinels?}
  
  Curwen:
  \ta{Well\ldots{} no, my Lady \Resvil. But\ldots{}}
  
  \Achsah:
  \ta{Then do not pester me.
    Figure it out yourself.
    I am busy.}
\end{prose}

\Achsah is unwilling to help. 
She has her hands full. 
She is stressed. 
She is sure the Sentinels are up to something really nasty here.
She tries to figure out what it is. 
She has no time to advise Charcoal. 
He is on his own.
She is sure he can solve his own problems. 
He is a skilled mage and a high-circle Cabalist. 





\subsubsection{\Achsah suspects \Takestsha}
\Achsah, who is in \Forclin, looks at the \quo{\EreshKali} spells cast by \Takestsha-tachi. 
\Achsah{} wonders when she first observes the \EreshKali{} magic. 
It does not feel like anything she would expect them to have. 
It also does not feel like what she would expect ancient \meccara{} to have. 
It is new to her, and it makes her suspicious. 
But what it \emph{does} smell like is Rissitic magic. 
That makes her even more suspicious. 
(She has heard Charcoal's account of Tantor's diary, but Charcoal has never seen Rissitic magic, so he cannot draw the connection.) 

Then she realizes that those spells are actually meant to tear the Shroud and reach into the Beyond, where it can summon\ldots{} stuff. 
And the Ghost Tower, which is in close promixity now that the Rungerans have breached \Forclin, acts as a catalyst. 
Those spells are tearing at the very fabric of the Shroud. 
Something fucking nasty is breaking through, or so \Achsah thinks. 
In reality the summoning spell at \Forclin is a smokescreen. 
It is meant to warp the Shroud and look big and impressive, but it doesn't actually \emph{do} anything. 
It's just meant to attract attention and convince everyone that the real stuff is happening in \Forclin, near the Ghost Tower. 

\Achsah now strongly suspects that \Malcur is a decoy and that the Sentinels' real goal is \Forclin. 

But still she keeps watching.
She does not intervene. 
Partially because she must remain ready and keep a bird's eye view of the action and cannot afford to commit herself to any narrow battlefield action.
Partially because of the Unspoken Covenant.
She will not be the first to break it. 





\subsubsection{Curwen begins his counterspell}
Archibald Curwen is at the Ghost Tower. 

When Curwen enters the Ghost Tower, he finds that it is much larger on the inside than on the outside.
Curwen knows this is a Shroud phenomenon.
In the city, the repressive Shroud twists the mind and the eye and makes the tower look small, and takes a man along paths that make the tower look small.
But in here, the Shroud is weaker, so the true extent of the Tower reveals itself to him.
Or something like that.

Have a \quo{moon-shrouded crystal} or the like inside the Tower. 

\lyricsbs{Bal-Sagoth}{%
  Enthroned in the Temple of the Serpent Kings%
}{
  Deep within the glacial, ice-veiled temple,\\
  ancient enchantments summon the shades of the dreaming Serpent Kings,\\
  and the Ophidian Throne once again draws power from the Moon-shrouded crystal.
}

He begins his counterspell. 

\lyricslimbonicart{Solace of the Shadows}{
  I set the stones for invoking ceremonies.\\
  In the twilight zones arise abstract galaxies.\\
  The magic eye unveils the blackened skies.\\
  A new horizon begins to each one that dies.
}

He is frightened by the magnitude of the powers he unleashes. 
But also thrilled, exhilarated.

\lyricslimbonicart{Solace of the Shadows}{
  The desolation makes me feel\\
  so dark, so cold, the silence.\\
  So dark, so cold, the emptiness.\\
  Solace of the shadows.
  
  Night surrounds and embraces me.\\
  Darkness holds the secrets of man's fears.\\
  It captures my heart as the purgatory sears.\\
  I cast now the spell, as I cross through raging flames,\\
  into darkness, cursing names.
}





\subsubsection{Ilcas-tachi attack the \ishrah}
\target{Ilcas-tachi attack the Rungeran Ishrah}
Sethgal and his forces are strained to the utmost.
He has to hold back the mundane Rungeran army, which is enough of a problem already.
It is twice as large as his own, and the walls of \Forclin are crumbling under the Rungeran cannonade. 

Meanwhile, Telcastora Ilcas leads his Imetrians in a brave attack against the Rungeran \ishrah.
This is something Sethgal has asked them to do. 
The Imetrian mage, Ulphon, \hr{Ulphon Nestor dies}{has been killed}, so Ilcas asks Carzain to cover them. 
He does. 

\target{Ilcas injures Takestsha}
The \ishrah have massive ranks of soldiers protecting them.
But the Imetrians are fearsome fighters, and Carzain is a badass mage.
They break through and kill several mages. 
Ilcas and his \nycans even manage to seriously injure \Takestsha. 
She would have died if she were an ordinary mortal. 

This attack is won primarily by the Imetrians. 
Emphasize the courage and superhuman skill of Ilcas and his \nycans. 
They are awesome forces of destruction. 
Carzain also fights well, but he has a secondary role. 
Carzain gives them some artillery support and protects them against enemy magic, but he does not fight in \melee himself.
Carzain does not kill much.
He is mostly a distraction. 
This is the Imetrians' hour of triumph. 
Carzain's moment of glory comes later when he \hr{Carzain fights Takestsha alone}{fights \Takestsha alone}. 

\Takestsha knows this is bad.
Her spell is strained as it is.
If these interlopers kill too many of her mages, her spell will certainly fail. 

So she diverts her attention from the spell and unleashes some nasty spells against the attackers.
As nasty as she can make them in her current state. 
(She is in a weakened humanoid form, deep in the Shroud, and she is a bit exhausted, and she is stressed because she has so many things on her mind and must maintain so big and complex a spell.)

She kills several Imetrians and forces the rest to retreat.
Then her soldiers are able to close their ranks.
Carzain-tachi are overwhelmed.
They have no chance but to flee to save their own hides. 





\subsubsection{The \EreshKali magic backfires}
\target{Eresh-Kali magic backfires}
Curwen is working on his counterspell.
When Carzain-tachi attack \Takestsha, he sees the opening he needs. 
He strikes with the full force of his counterspell. 

The spell catches \Takestsha-tachi at the worst possible time. 
The Rungeran \ishrah is reduced. 
Several mages have been killed in the \hr{Ilcas-tachi attack the Rungeran Ishrah}{Imetric attack}. 
The remaining ones cannot endure all the stress and strain.
The \EreshKali spells backfire on them. 
This kills the entire Rungeran \ishrah{}.
Only \Takestsha survives, and she is badly wounded. 

This buys the Cabalists time to send in reinforcements, including \banes{} and perhaps \resphain, which forces \Nzessuacrith{} to breach the \charade{} and assume her \draconic{} form to fight them off. 

By now \Forclin{} is pretty doomed, but now \Nzessuacrith{} is weakened enough for \Achsah{}, her fellow \resphain{} and their \hr{Umbra}{\umbrae} to have a fighting chance against her. 





\subsubsection{Curwen dies}
\target{Curwen dies}
\Takestsha is not pleased to be thus thwarted. 
In the midst of the destruction, she reaches out and strikes back through the counterspell.
She grabs hold of Curwen's spells and twist them against him.
She kills Curwen.

He fought well and bravely.
He made a difference.
But he was just a mortal against an immortal, so he paid for that difference with his life. 











