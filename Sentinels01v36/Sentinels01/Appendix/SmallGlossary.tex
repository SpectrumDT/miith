
\bookchapter{Glossary}

\paragraph{Note:} 
The \quo{Meta} tags mark information written from a out-of-universe, non-\Miithian{} perspective. 
These may be comparisons with the real world or trivia about the origins of concepts. 















\section{Characters}
\subsection{Immortals}
\begin{gloss}







\begin{comment}
\paragraph{\Achsah}
\end{comment}
\gitemcharacter{\Achsah}{\resphan}{\female}
\target{Achsah}
A \hr{Cabal}{Cabalist} of the \achsahcircle{} circle. 
She is of common blood and serves \hr{Teshrial}{\Teshrial}. 

\meta{%
  \quo{\Achsah} is originally the name of a minor character that appears in the Old Testament (see Judges 1:12).}
 





\begin{comment}
\paragraph{\Azraid}
\end{comment}
\gitemcharacter{\Azraid}{\resphan}{\male}
\target{Azraid}
A \hr{Sathariah}{\sathariah} and the High Lord of \hr{CS}{\CiriathSepher}. 







\begin{comment}
\paragraph{\Criseis}
\end{comment}
\gitemcharacter{\Criseis}{\scatha}{\female}
\target{Criseis}
Servant and companion to \hr{Ishnaruchaefir}{Quessanth \Ishnaruchaefir}. 







\begin{comment}
\paragraph{\IrocasSecherdamon}
\end{comment}
\gitemcharacter{\IrocasSecherdamon}{\dragon}{\male}
\index{\Secherdamon|see{\IrocasSecherdamon}}
\target{Secherdamon}
A \hr{Shae'eroth}{\shaeeroth}, the third-born son of \hr{Tiamat}{\Tiamat}. 
% His scales are pure black. 







\begin{comment}
\paragraph{\LocarPsyrex}
\end{comment}
\gitemcharacter{\LocarPsyrex}{\scatha}{\male}
\index{\Psyrex|see{\LocarPsyrex}}
\target{Psyrex}
A \hr{Sentinels of Miith}{Sentinel} sorcerer, leader of the \hs{Dark Crescent}. 
Said to have \hr{Daemon}{\daemonic} blood. 







\begin{comment}
\paragraph{\QuessanthIshnaruchaefir}
\end{comment}
\gitemcharacter{\QuessanthIshnaruchaefir}{\dragon}{\male}
\index{\Ishnaruchaefir|see{\QuessanthIshnaruchaefir}}
\target{Ishnaruchaefir}
A \hr{Shae'eroth}{\shaeeroth}, the second-born son of \hr{Tiamat}{\Tiamat}. 
Wielder of the \hs{glaive} \Triestessakhin. 
% His scales are pure black. 







\begin{comment}
\paragraph{\RaemythKhivaashNexagglachel}
\end{comment}
\gitemcharacter{\RaemythKhivaashNexagglachel}{\dragon}{\male}
\index{\Nexagglachel|see{\RaemythKhivaashNexagglachel}}
\target{Nexagglachel}
A \hr{Shae'eroth}{\shaeeroth}, the first-born son of \hr{Tiamat}{\Tiamat}. 
Perished before the \secondbanewar. 
% His scales are pure black. 







\begin{comment}
\paragraph{\TyarithXserasshana}
\end{comment}
\gitemcharacter{\TyarithXserasshana}{\dragon}{\female}
\index{\Kserasshana|see{\TyarithXserasshana}}
\target{Tiamat}
Ancient queen of the \hr{Dzraic'chenoss}{\draecchonosh}. 
Perished before the \secondbanewar. 
She had three sons: 
\hr{Nexagglachel}{\Nexagglachel}, \hr{Ishnaruchaefir}{\Ishnaruchaefir} and \hr{Secherdamon}{\Secherdamon}. 







\begin{comment}
\paragraph{\Teshrial}
\end{comment}
\gitemcharacter{\Teshrial}{\resphan}{\male}
\target{Teshrial}
A \hr{Ketheran}{\ketheran} \hr{Resphan}{\resphan} of \hr{CS}{\CiriathSepher}, a \hr{Cabal}{Cabalist} of the \teshrialcircle{} circle. 
His mother is \maybehr{Zereth}{\Zereth} and his father is \maybehr{Tuerdal}{\Tuerdal}. 

\appearance{%
  250 cm tall. 
  Black skin. 
  Hair and feathers dyed white. 
  Pink eyes. 
}






\end{gloss}









\subsection{Mortals}
\begin{gloss}







\begin{comment}
\paragraph{Arcan \Delaen}
\end{comment}
\gitemcharacter[dead][Arcan Delain]{Arcan \Delaen}{\human}{\male}
\target{Arcan Delain}
\index{\Delaen!Arcan \Delaen}
The elder brother of \hr{Lestor Delain}{Lestor \Delaen} and \hs{Silqua Vaimon}. 
One of the first Vaimons. 







\gitemcharacter[dead][Belzir]{\Belzir}{\human}{\female}
\target{Belzir}
The last \hs{Vaimon} \Calipha.
\also{\hr{Vaimon Caliphate}{\VaimonCaliphate}}






% \begin{comment}







\begin{comment}
\paragraph{\CarzainDeracilleShireyo}
\end{comment}
\gitemcharacter[live][Carzain]{\CarzainDeracilleShireyo}{\human}{\male}
\index{\Shireyo!Carzain \Deracille{} \Shireyo}
\target{Carzain Shireyo}
A rogue \hs{Vaimon} living in \hr{Redglen}{\Redglen}; the only child of \hr{Nishain Shireyo}{Nishain \Shireyo} and \hr{Roanne Deracille}{\Roanne{} \Deracille}. 

\appearance{%
  180 cm tall, light of build. 
  Shoulder-length black hair, slightly wavy. 
  Clean-shaven.
  Brown eyes.}
\also{\hs{carzain} (animal)}







\begin{comment}
\paragraph{Cordos Vaimon}
\end{comment}
\gitemcharacter[dead]{Cordos Vaimon}{\human}{\male}
\index{Vaimon!Cordos Vaimon}
\target{Cordos Vaimon}
Originally the king of {\Imrath}, who went on to found the \hs{Vaimon} order and the \hr{Vaimon Caliphate}{\VaimonCaliphate}, both named after him. 
In the year 1 \VC{} he was crowned as the first \VaimonCaliph. One of his wives was \hs{Silqua}. 
\also{\hs{Silqua}, \hs{Vaimon}, \hr{Vaimon Caliphate}{\VaimonCaliphate}}







\begin{comment}
\paragraph{Cuthran the Victorious}
\end{comment}
\gitemcharacter[dead]{Cuthran the Victorious}{\scatha}{\male}
\target{Cuthran the Victorious}
The founder and first High King of \hr{Great Velcad}{\GreatVelcad}. 







\begin{comment}
\paragraph{Grith Ecallivan}
\end{comment}
\gitemcharacter[dead]{Grith Ecallivan}{\human}{\male}
\target{Grith Ecallivan}
A hero of the ancient \hr{Vaimon Caliphate}{\VaimonCaliphate}. 
He was a scoundrel, a rogue who went his own ways, but he was loyal to the Vaimon cause and served the \caliphate in his own ways. 
He was condemned as an outlaw in his lifetime, but was posthumously recognized and honoured as the hero he was. 
The tales relay how he studied his enemies from the inside and always found out how to strike at their weaknesses and steal their secrets. 







\begin{comment}
\paragraph{\Icor{} Pelidor}
\end{comment}
\gitemcharacter{\Icor{} Pelidor}{\scatha}{\male}
\index{Pelidor!\Icor{} Pelidor}
\target{Icor}
\target{Icor Pelidor}
The \rayuth of \hs{Pelidor}. 
A descendant of the \Malcuric{} branch of \hs{House Pelidor}. 
Married \hr{Tiroco}{\Tiroco{} Pelidor} and was elected \rayuth in \yic{Icor becomes duke}. 
As of spring \yic{Runger war} they have three children: Roric, Frico and a yet unnamed egg. 







\begin{comment}
\paragraph{Lestor \Delaen}
\end{comment}
\gitemcharacter[dead][Lestor Delain]{Lestor \Delaen}{\human}{\male}
\target{Lestor Delain}
\index{\Delaen!Lestor \Delaen}
Older brother of \hs{Silqua Vaimon}. 
One of the first Vaimons. 
He was a controversial figure because he used the power of \itzach, even the feared \hr{Midnight Circle}{Midnight \Qliphoth}. 







\begin{comment}
\paragraph{Nishain \Shireyo}
\end{comment}
\gitemcharacter[live][Nishain Shireyo]{Nishain \Shireyo}{\human}{\male}
\target{Nishain Shireyo}
A \hs{Vaimon} of \hr{Geican}{\ClanGeican}, living in \hr{Redglen}{\Redglen} as an apothecary, scholar and healer. 
Married to \hr{Roanne Deracille}{\Roanne{} \Deracille}. 
Father of \hr{Carzain Shireyo}{Carzain \Shireyo}. 

\appearance{%
  Being Geican, Nishain has rather dark skin and a large nose. 
  He wear short, curly black hair and a bushy, black moustache but no beard. 
  His eyes are brown. }







\begin{comment}
\paragraph{\Roanne{} \Deracille}
\end{comment}
\gitemcharacter[live][Roanne Deracille]{\Roanne{} \Deracille}{\human}{\female}
\target{Roanne Deracille}
A \hs{Vaimon} living in \hr{Redglen}{\Redglen} as a healer. 
Married to \hr{Nishain Shireyo}{Nishain \Shireyo}. 
Mother of \hr{Carzain Shireyo}{Carzain \Shireyo}. 
Previously a \soror{} of \theTulipFaction{} of \hr{Redcor}{\ClanRedcor}. 







\begin{comment}
\paragraph{Silqua Vaimon}
\end{comment}
\gitemcharacter[dead]{Silqua Vaimon}{\human}{\female}
\target{Silqua}
\target{Silqua Vaimon}
\index{\Delaen!Silqua \Delaen}
\index{Vaimon!Silqua Vaimon}
%Female \human{} \birthtodeath{Silqua}. 
The first \hs{Vaimon}, the one who discoved \hr{Iquin}{\iquin} and \hr{Itzach}{\nieur}. 
Called \quo{the Prophet} by the \hs{Iquinian Church}. 
She was born Silqua \Delaen, the daughter of Lord \maybehr{Maegon Delain}{Maegon \Delaen}. 
In \yic{Silqua married} she married \hs{Cordos Vaimon}, prince and later king of {\Imrath}, and became Silqua Vaimon. 

She had two older brothers: 
\hr{Arcan Delain}{Arcan} and \hr{Lestor Delain}{Lestor}. 







\begin{comment}
\paragraph{Telcastora Ilcas}
\end{comment}
\gitemcharacter{Telcastora Ilcas}{\scatha}{\male}
\index{Ilcas!see{Telcastora Ilcas}}
\target{Ilcas}
An \hr{Imetrium}{Imetrian} soldier with the rank of {\IlcasStartRank}. 
Has two \hr{Nycan}{\nycan} companions: Countess and Razor. 







\begin{comment}
\paragraph{\Tiroco{} Pelidor}
\end{comment}
\gitemcharacter{\Tiroco{} Pelidor}{\scatha}{\female}
\index{Pelidor!\Tiroco{} Pelidor}
\target{Tiroco}
\target{Tiroco Pelidor}
The \rinyuth of \hs{Pelidor}. 
A descendant of the \Forkliner{} branch of \hs{House Pelidor}. 
Married \hr{Icor}{\Icor{} Pelidor} in \yic{Icor becomes duke}. 







\begin{comment}
\paragraph{Uther the Tiger}
\end{comment}
\gitemcharacter[dead]{Uther the Tiger}{\human}{\male}
\target{Uther the Tiger}
Full name: High King Uther I, son of Patrick of House Belek. 
Once king of Belek. 
An ally of \hs{Cuthran the Victorious}. 
Founder of the \hs{Tiger} order. 



% \end{comment}
\end{gloss}



















\section{Cosmos}
\begin{gloss}







\begin{comment}
\subsection{\Erebos}
\end{comment}
\gitem{\Erebos}
\target{Erebos}
A \hs{Realm}. %Called by some the Twilight Realm.
Called by some the Realm of Darkness. 
% Homeworld of the \hs{\banes}. 

\meta{%
  In Greek mythology, Erebos is the personification of primordial darkness.}
\also{Realms}








\begin{comment}
\subsection{\iquin}
\end{comment}
\gitem{\iquin}
\index{Light}
\target{Iquin}
\target{Light}
The force of Light in \hs{Vaimon} metaphyics. 
By the \hs{Iquinian Church} viewed as the source of all good and worshipped as a divine force. 
Its manifestations are the \hr{Sephirah}{\Sephiroth}. 

\Iquin{} is also supposedly a place of bliss where the souls of the vituous go after death. 







\begin{comment}
\subsection{\itzach}
\end{comment}
\gitem{\Itzach}
\target{Itzach}
The force of Darkness in \hs{Vaimon} metaphyics. 
By the \hs{Iquinian Church} reviled as the source of all evil. 
Its manifestations are the \hr{Qliphah}{\Qliphoth}. 

\Itzach{} is also supposedly a place of torment where the souls of the wicked go after death. 







\begin{comment}
\subsection{\Nyx}
\end{comment}
\gitem{\Nyx}
\target{Nyx}
A \hs{Realm}. 
Called by some the Twilight Realm.

\meta{In Greek mythology, Nyx is the personification of the night.}
\also{Realms}







\begin{comment}
\subsection{Realms}
\end{comment}
\gitem{Realms}
\target{Realm}
Parallel worlds.







\begin{comment}
\subsection{Realm of Chaos}
\end{comment}
\gitem{Realm of Chaos}
\target{Realm of Chaos}
The mythical world where the \hr{Daemon}{\daemons} allegedly come from. 

\end{gloss}



















\section{Creatures}
\subsection{Supernatural}
\begin{gloss}







\begin{comment}
\paragraph{angel}
\end{comment}
\gitem{angel}
\target{angel}
A creature in \hs{Vaimon} mythology, associated with \iquin. 
Angels are considered a class of \hr{Archon}{\Archons} and lesser cousins of the \hr{Sephirah}{\Sephiroth}. 
They are often depicted as \hs{humanoids} with feathered wings. 







\begin{comment}
\paragraph{\Archon}
\end{comment}
\gitem{\Archon}
\target{Archon}
A supernatural being in \hs{Vaimon} metaphysics. % that can be invoked to cast magic. 
Classes of \Archons{} include the \hr{Sephirah}{\Sephiroth} and \hr{Qliphah}{\Qliphoth}. 







\begin{comment}
\paragraph{\bane}
\end{comment}
\gitem[\banes]{\bane}
\target{Bane}
An alien race, native to \hr{Erebos}{\Erebos}. 
% Some \hs{\resphain} are known to consort with them. 








\begin{comment}
\paragraph{\daemon}
\end{comment}
\gitem[\daemons]{\daemon}
\target{Daemon}
\target{Daemons}
Mythical, incorporeal entities, said to inhabit the mythical \hs{Realm of Chaos}. 
Chaos magic involves dealing with and summoning \daemons. 







\begin{comment}
\paragraph{\dragon}
\end{comment}
\gitem{\dragon}
\target{Dragon}
A race of \hr{immortal}{immortals}. 
They resemble giant reptiles with four legs, two wings and a long neck and tail. 





\begin{subgloss}
  \begin{comment}
  \subparagraph{\draecchonosh}
  \end{comment}
  \gitem[\draecchonosh]{\draecchonosh}
  \target{Dzraic'chenoss}
  A formal word for \quo{\dragon} in \maybehr{True Draconic}{\TrueDraconic}. 
  
  
  
  
  \begin{comment}
  \subparagraph{\shaeeroth}
  \end{comment}
  \gitem[\shaeeroths]{\shaeeroth}
  \target{Shae'eroth}
  A group of powerful \dragons. 
  Their number includes, among others, the three sons of \hr{Tiamat}{\Tiamat}: 
  \hr{Nexagglachel}{\Nexagglachel}, 
  \hr{Ishnaruchaefir}{\Ishnaruchaefir} and 
  \hr{Secherdamon}{\Secherdamon}. 
\end{subgloss}









\begin{comment}
\paragraph{\ghobal}
\end{comment}
\gitem[\ghobaleth]{\ghobal}
\target{Ghobal}
A giant monster, perhaps wormlike in shape. 









\begin{comment}
\paragraph{god}
\end{comment}
\gitem{god}
\target{god}
\target{gods}
Generic term for \hs{immortal} beings of great power. 
Examples include the \hs{Imetric} and \hs{Rissitic} gods. 









\begin{comment}
\paragraph{\grimrat}
\end{comment}
\gitem{\grimrat}
\target{Grimrat}
A species of monsters from \hr{Nyx}{\Nyx}. 









\begin{comment}
\paragraph{immortal}
\end{comment}
\gitem{immortal}
\index{immortal}
\target{immortal}
Generic term for any creatures who do not die of old age and wield power beyond that of regular \hs{humanoids}. 
Immortal races include, among other things, the \dragons{} and \resphain.  







\begin{comment}
\paragraph{\malach}
\end{comment}
\gitem[\malachim]{\malach}
A class of \hr{Archon}{\Archons}. 







\begin{comment}
\paragraph{\Maskim}
\end{comment}
\gitem[\Maskim]{\Maskim}
A race of supernatural beings in \hs{Rissitic} mythology. 
Considered dangerous and evil and sometimes invoked in curses. 
\quo{\Maskim{} take you} is a strong curse in Rissitic.

\meta{%
  The name Maskim, also spelled Masqim, is taken from the \emph{Simon Necronomicon}, which is in turn allegedly based on Mesopotamian mythology.
  I haven't been able to find out whether the Maskim are genuine mythical creatures or something \quo{Simon} made up.} 







\begin{comment}
\paragraph{Psychic Eye}
\end{comment}
\gitem{Psychic Eye}
\target{Psychic Eye}
An invisible telepathic \daemon. 
A sorcerer can summon the Psychic Eye and then telepathically connect with it to share its senses. 









\begin{comment}
\paragraph{\qliphah}
\end{comment}
\gitem[\qliphoth]{\qliphah}
\target{Qliphah}
\target{Qliphoth}
The \hr{Archon}{\Archons} of \hr{Itzach}{\nieur}. 

\target{Circle of Noon}
\target{Noon Circle}
\index{Noon Circle}
\index{Circle (in Vaimon mysticism)}
At least 100 \qliphoth{} are known and named. 
They are categorized into a number of \quo{Circles}. 
(Following this system, the \hr{Sephirah}{\Sephiroth} are sometimes considered the Circle of Noon.) 





\begin{subgloss}
  \begin{comment}
  \subparagraph{Circle of Dusk}
  \end{comment}
  \gitem{Circle of Dusk}
  \target{Circle of Dusk}
  \target{Dusk Circle}
  \index{Dusk Circle}
  The most harmless ones. 
  \target{Gavron}
  \index{\Gavron}
  Includes \Gavron. 




  \begin{comment}
  \subparagraph{Circle of Twilight}
  \end{comment}
  \gitem{Circle of Twilight}
  \target{Circle of Twilight}
  \target{Twilight Circle}
  \index{Twilight Circle}
  The more dangerous ones. 
  \target{Djerzad}
  \index{\Djerzad}
  \target{Iphicoss}
  \index{\Iphicoss}
  \target{Kithvaz}
  \index{\Kithvaz}
  Includes \Djerzad, \Iphicoss{} and \Kithvaz. 




  \begin{comment}
  \subparagraph{Circle of Midnight}
  \end{comment}
  \gitem{Circle of Midnight}
  \target{Circle of Midnight}
  \target{Midnight Circle}
  \index{Midnight Circle}
  The most powerful and dangerous of all. 
  Feared among the Geicans as bringers of madness. 
  \index{\Bozchul}
  \target{Bozchul}
  \target{Nyxachel}
  \target{Horvaleth}
  \index{\Nyxachel}
  \index{\Horvaleth}
  Includes \Bozchul, \Horvaleth{} and \Nyxachel{}. 
\end{subgloss}
\meta{%
  In Cabbalah (Jewish mysticism), the ten \qliphoth{} are evil counterparts to the \sephiroth.}
\also{\hr{Sephirah}{\Sephirah}}









\begin{comment}
\paragraph{\resphan}
\end{comment}
\gitem[\resphain]{\resphan}
\target{Resphan}
\target{Resvil}
An \hs{immortal} race. 
Sometimes classified as angels or gods. 

They resemble \hr{Human}{\humans}, but with gray or black skin, and taller. 
An average male stands 210-250 cm tall, a female only about 180 cm. 
Some \resphain{} have feathered wings, with a wingspan of around twice their height. 

%\Resphain{} are often seen by \humans{} as very beautiful and imposing. 

\quo{\Resphan} is also used to refer specifically to the male of the race. 
The female is called a \resvil{} (plural \resviel). 

\index{dynasty}
\Resphan{} nobility is organized in number of \quo{dynasties}. 
These are: 

\index{\KiriathSepher}
\index{\TiphredSerah}
\index{\Mystraacht}
\index{\Kezerad}
\index{\Baelzerach}
\target{CS}
\target{TS}
\target{Mystraacht}
\target{Kezerad}
\target{BZ}
\begin{itemize}
  \item \KiriathSepher. 
  \item \TiphredSerah. 
  \item \Mystraacht.
  \item \Kezerad. 
  \item \Baelzerach. 
\end{itemize}

\meta{%
  \quo{\KiriathSepher} is originally the name of a city that appears in the Old Testament (see Judges 1:12).}





\begin{subgloss}
  \begin{comment}
  \subparagraph{\bezed}
  \end{comment}
  \gitem[\bezedeth]{\bezed}
  \target{Bezed}
  \target{Beuzed}
  \index{\ashenblood|see{\bezed}}
  The lower classes among the \resphain. 
  Disparagingly called \quo{\ashenblooded}.
  
  
  
  
  
  \begin{comment}
  \subparagraph{\ketheran}
  \end{comment}
  \gitem[\ketherain]{\ketheran}
  \target{Ketheran}
  A noble class among the \resphain. 





  \begin{comment}
  \subparagraph{\Merkyrah}
  \end{comment}
  \gitem{\Merkyrah}
  The first great empire of the \resphain. 
  
  
  
  
  
  
  \begin{comment}
  \subparagraph{\thelyad}
  \end{comment}
  \gitem[\thelyadeth]{\thelyad}
  \target{plainblood}
  \target{Thelyad}
  \index{plainblood|see{\thelyad}}
  The lesser nobility among the \resphain. 
  Disparagingly called \quo{plainbloods}.
  
  
  
  
  
  \begin{comment}
  \subparagraph{\sathariah{} (plural \satharioth)}
  \end{comment}
  \gitem[\satharioth]{\sathariah}
  \target{Sathariah}
  %A group of powerful \resphan{} lords. 
  The highest of \resphan{} nobility. 
\end{subgloss}







\begin{comment}
\paragraph{\sephirah}
\end{comment}
\gitem[\sephiroth]{\sephirah}
\target{Sephirah}
\target{Sephiroth}
In \hs{Vaimon} metaphysics, the \sephiroth{} are \hr{Archon}{\Archons} of \hr{Iquin}{\iquin}. 
There are sixteen \sephiroth, each said to embody a virtue. 
They are depicted as anthropomorphic and each with a fixed gender. 

They are split into four \quo{elements}: 
The Eye, the Passion, the Tear and the Voice. 





% Each Sephirah has: A name, a classical element, a power, a virtue and an explanation. 
\newcommand{\sephitem}[6]{\item #1 (#3): \target{#2} #5}

\newenvironment{sephirahlist}[1]{%    
  \item[\Sephiroth{} of the #1:] \ 
  
  \begin{itemize}
}{%
  \end{itemize}
}





\begin{description}
\begin{comment}
\subparagraph{Eye}
\end{comment}
\begin{sephirahlist}{Eye}
\sephitem
  {\Cushed}
  {Cushed}
  {\male}
  {Used to shape, move and Sculpt earthen objects.}
  {Lawfulness}
  {Helps the rulers keep people in check. }
\sephitem
  {\Omariel}
  {Omariel}
  {\female}
  {}
  {Acceptance}
  {Makes people accept hardship and oppression.}
\sephitem
  {\Yemared}
  {Yemared}
  {\female}
  {Cause paralysis.}
  {Tradition}
  {Prevent rebellion and keeps the social order stable and static.}
\sephitem
  {\Yeziel}
  {Yeziel}
  {\male}
  {}
  {Chastity}
  {Sex is a dangerous thing, because it may open people's eyes to the Beyond. Also, it makes them harder to control. Also, \human{} sexuality is something the \banes{} very much want to harness and control, so they need \Yeziel{} to keep people's sexuality in check. }
\end{sephirahlist}




\begin{comment}
\subparagraph{Passion}
\end{comment}
\begin{sephirahlist}{Passion}
\sephitem
  {\Barion}
  {Barion}
  {\male}
  {}
  {Courage}
  {Encourages the people to fight against the enemies of the Church.}
\sephitem
  {\Hoshied}
  {Hoshied}
  {\male}
  {}
  {Loyalty}
  {Keeps people under control and discourages them from asking questions.}
\sephitem
  {\Razilah}
  {Razilah}
  {\male}
  {Create lightning.}
  {Righteousness}
  {Opposes any kind of heresy, blasphemy and unorthodoxy; crusades against the heathens.}
\sephitem
  {\Teshiron}
  {Teshiron}
  {\female}
  {}
  {Faith}
  {Keeps people loyal to the Church and not asking unwanted questions.}
\end{sephirahlist}





\begin{comment}
\subparagraph{Tear}
\end{comment}
\begin{sephirahlist}{Tear}
\sephitem
  {\Feazirah}
  {Feazirah}
  {\female}
  {The gentle wind.}
  {Humility}
  {Keeps people pacified and keep them from complaining.}
\sephitem
  {\Gamishiel}
  {Gamishiel}
  {\female}
  {}
  {Sacrifice}
  {Make people work hard for their masters and not expect any rewards. (Her month is only 20 days long.)}
\sephitem
  {\Hapheron}
  {Hapheron}
  {\male}
  {}
  {Solidarity}
  {Hate of outsiders. Turns the Iquinians, and all \humans, into a united front against their enemies.}
\sephitem
  {\Ishiel}
  {Ishiel}
  {\female}
  {Healing}
  {Patience}
  {Makes people accept hardship and oppression.}
\end{sephirahlist}





\begin{comment}
\subparagraph{Voice}
\end{comment}
\begin{sephirahlist}{Voice}
\sephitem
  {\Atzirah}
  {Atzirah}
  {\male}
  {The carrying wind. Used for lifting objects or flying.}
  {\Honour}
  {Keeps people inside the system for fear of dishonour while encouraging them to strive to please their masters and the Church.}
\sephitem
  {\Izion}
  {Izion}
  {\male}
  {Creates blasts of fire. Perhaps the \Sephirah{} most commonly used in combat. }
  {Justice}
  {Destroys the wicked: Sinners, and the enemies of the church.}
\sephitem
  {\Keshirah}
  {Keshirah}
  {\female}
  {The powerful wind. Used to create controlled gusts of wind. }
  {Dilligence}
  {Makes people work hard for their masters.}
\sephitem
  {\Thimared}
  {Thimared}
  {\female}
  {Limited mind control.}
  {Obedience}
  {Turn people into humble slaves. Keeps the masses from rebelling and rulers from sympathizing.}
\end{sephirahlist}
\end{description}





\meta{%
  In Cabbalah (Jewish mysticism), the ten \sephiroth{} are different manifestations of God... or something like that. 
  Contrary to popular belief, the name does not originate in \emph{Final Fantasy VII}. }







\begin{comment}
\paragraph{\Taortha}
\end{comment}
\gitem[\Taorthae]{\Taortha}
\target{Taortha}
A race of gods. 
Once worshipped in \hr{Ortaica}{\Ortaica} and still revered by the \hr{Rethyax}{\rethyaxes}. 
  
\begin{gloss}
  \begin{comment}
  \subparagraph{Daxian}
  \end{comment}
  \gitem{Daxian} 
  \target{Daxian}
    God of weather and the \Wylde{}. 
    
  \begin{comment}
  \subparagraph{Isxae}
  \end{comment}
  \gitem{Isxae}
  \target{Isxae}
    Goddess of law and rulership. 
  
  \begin{comment}
  \subparagraph{\Nasshikerr}
  \end{comment}
  \gitem{\Nasshikerr}
  \target{Nasshikerr}
    God of shadows and the hidden. 
  
  \begin{comment}
  \subparagraph{\NerrhanKoss}
  \end{comment}
  \gitem{\NerrhanKoss}
  \target{Nerrhan-Koss}
    God of the afterlife and the occult. 

  \begin{comment}
  \subparagraph{Shellagh}
  \end{comment}
  \gitem{Shellagh}
  \target{Shellagh}
    God of the sea. 
\end{gloss}








\begin{comment}
\paragraph{\xzaishann}
\end{comment}
\gitem[\xzaishanns]{\xzaishann}
\target{XS}
A race of alien \hs{gods}. 
Their names include:

\begin{gloss}
  \begin{comment}
  \subparagraph{\KhothSell}
  \end{comment}
  \gitem{\KhothSell} 
  \target{Khoth-Sell}
  
  \begin{comment}
  \subparagraph{\KyaethemChreiAz}
  \end{comment}
  \gitem{\KyaethemChreiAz} 
  \target{Kyaethem Chrei Az}
  
  \begin{comment}
  \subparagraph{\NaathKurRamalech}
  \end{comment}
  \gitem{\NaathKurRamalech} 
  \target{Naath-Kur-Ramalech}
\end{gloss}









 
\end{gloss}









\subsection{Intelligent mortals}
\begin{gloss}








\begin{comment}
\paragraph{\cregorr}
\end{comment}
\gitem[\cregorrs]{\cregorr}
\target{Cregorr}
A \hs{humanoid} race.









\begin{comment}
\paragraph{\human}
\end{comment}
\gitem{\human}
\target{Human}
A \hs{humanoid} race. 
One of the most widespread humanoid races on \Miith{}, alongside the \scathae. 
They are the dominant race in northern and eastern \hr{Velcad}{\Velcad}. 









\begin{comment}
\paragraph{humanoid}
\end{comment}
\gitem{humanoid}
\target{humanoid}
\target{humanoids}
Generic term for intelligent, bipedal, mortal races. 
Includes \hr{Scatha}{\scathae}, \hr{Human}{\humans} and \hr{Meccaran}{\meccara}. 







\begin{comment}
\paragraph{\meccaran}
\end{comment}
\gitem[\meccara][\meccarans]{\meccaran}
\target{Meccaran}
A race of amphibian \hs{humanoids}. 
A \meccaran{} resembles an anthropomorphic frog with short arms but long and strong legs. 
An average adult female stands about 130 cm tall, the smaller male 120 cm. 
They stand and walk in a crouched position with the legs bent outwards. 
They are \coloured in shades of green or brown. 

\Meccara{} are notable for their regenerative capabilities, able to regrow entire limbs in a matter of weeks or months. 

Most \meccara{} are left-handed. 

\Meccara{} are the third most widespread humanoid race on \Miith{}, after \hr{Scatha}{\scathae} and \hr{Human}{\humans}. 








\begin{comment}
\paragraph{\nephil}
\end{comment}
\gitem[\nephilim]{\nephil}
A \hs{humanoid} race.








\begin{comment}
\paragraph{\nycan}
\end{comment}
\gitem{\nycan}
\target{Nycan}
A slim, carnivorous bipedal \hr{Saurian}{\saurian}. 
They are anywhere from 2 to 7 metres long, depending on race and breed (half of that length being tail).
They are covered in feathers and \coloured in yellow, orange or red with dark stripes. 

Highly intelligent pack hunters, \nycans{} can be tamed and used as trackers, beasts of war, and even mounts (in case of large breeds).

In combat, their primary weapons are the two oversized claws on each foot. 
They also use their front claws and teeth. 

\meta{%
  Compare to dromaeosaurid dinosaurs like \latinname{Deinonychus} or \latinname{Velociraptor}. 
  The name \quo{\nycan} is based on Greek \emph{onychos} ($o \nu \upsilon \chi o \varsigma$), meaning \quo{claw}. }







\begin{comment}
\paragraph{\scatha}
\end{comment}
\gitem[\scathae]{\scatha}
\target{Scatha}
A race of reptillian \hs{humanoids}. 
A \scatha{} resembles an anthropomorphic lizard with a long, rigid tail. 
%They do not stand fully erect; their backs tend to slope at 30-45 degrees (closer to horizontal than vertical). 
An average adult is 170-180 cm tall or long. 
%An average adult is 160-170 cm tall and 190-200 cm long including the tail. 
Males and females are of equal size. 
They have tough scaly skin and are herbivorous. 

The \scathae{} are one of the most widespread humanoid races on \Miith{}, alongside \hr{Human}{\humans}. 
They are the dominant race in the \hs{Imetrium} and \hs{Durcac}. 

\Scathae{} cannot blink. 
Instead, they lick their eyes to clean them. 

\index{\dax}
\index{\sphyle}
\target{Dax}
\target{Sphyle}
A male \scatha{} is called a \dax{} (plural \daxes). 
A female is called a \sphyle{} (plural \sphyles). 

The adjective associated with \quo{\scatha} is \quo{\scathaese}. 

The \scathaese{} race can be divided into three major subraces: 

\begin{subgloss}
  \begin{comment}
  \subparagraph{\Tassian}
  \end{comment}
  \gitem[\Tassians]{\Tassian}
    \target{Tassian}
    The most widespread subrace. 
    Blue scales. 
    Common in the \hs{Imetrium} and southern \hr{Velcad}{\Velcad}. 
    Peoples include the \hr{Mastheno}{\Masthenon} and \hr{Ortaica}{\Ortaicans}. 
    
  \begin{comment}
  \subparagraph{\Mekrii}
  \end{comment}
  \gitem[\Mekriis]{\Mekrii}
    \target{Mekrii}
    Red or reddish brown scales.
    Most common in \hs{Durcac} and the \hs{Orient}. 
    Peoples include the \hr{Shurco}{\Shurco} and \hs{Rissitics}. 
    
  \begin{comment}
  \subparagraph{\Loi}
  \end{comment}
  \gitem[\Lois]{\Loi}
    \target{Loi}
    Green or greenish brown (very rarely black).
    Most common in the north and the {\Serplands}. 
\end{subgloss}








 
\end{gloss}









\subsection{Beasts}
\begin{gloss}







\begin{comment}
\paragraph{\belwan}
\end{comment}
\gitem{\belwan}
\target{Belwan}
%A race of monsters native to \Erebos. 
A herbivorous, quadrupedal mammal. 
It is around 2 metres long with brown or gray fur. 
They look somewhat like bears with oversized forearms and long, flat snouts, but they are actually ground sloths. 
Can be tamed and used as pack animals. 

\meta{%
  Compare to prehistoric ground sloths such as \latinname{Megatherium}.}







\begin{comment}
\paragraph{carzain (animal)}
\end{comment}
\gitem{carzain (animal)}
\target{carzain}
\maybehr{Vaimon language}{Vaimon name} for a mustelid (weasel-like animal) common in \hr{Redce}{\Redce} and northern \hr{Velcad}{\Velcad}. 
It grows up to 45 cm long. 
It is normally black, but turns white in winter. 
Carzains are sometimes hunted or bred for their fur. 







\begin{comment}
\paragraph{\corgorah}
\end{comment}
\gitem[\corgoroth]{\corgorah}
\target{Corgorah}
A large carnivorous \hr{Saurian}{\saurian}. 
An adult is 10-15 metres long and weighs 3-7 tonnes. 
They are covered in tough, scaly skin, brown and gray in \colour. 

%In some cases Cortios can be tamed, and they are used as mounts especially by the Rissitic \Ashenoch.

\meta{Compare to theropod dinosaurs like \latinname{Tyrannosaurus} or \latinname{Allosaurus}, with the size of the former but the strong forearms of the latter.}









\begin{comment}
\paragraph{\grulcan}
\end{comment}
\gitem{\grulcan}
\target{Grulcan}
A large flightless predatory bird. 
They are normally 2.5 to 3 metres tall, but may grow as large as 3.5 metres. 
Their plumages are orange and brown reminiscent of bronze, and also white. 

\Grulcans{} can be domesticated and used as beasts of war. 
They fight with their beaks and talons. 

A \grulcan{} appears on the banner of Pelidor.

\meta{%
  Compare to prehistoric birds like \latinname{Brontornis} and \latinname{Gastornis} (aka \latinname{Diatryma}).}









\begin{comment}
\paragraph{\miksha}
\end{comment}
\gitem{\miksha}
\target{Miksha}
A fast-running bipedal \hr{Saurian}{\saurian}. 

\meta{%
  Compare to Ornithomimid dinosaurs like \latinname{Gallimimus} or \latinname{Struthiomimus}.}








\begin{comment}
\paragraph{\mulgron}
\end{comment}
\gitem{\mulgron}
\target{Mulgron}
A stocky quadrupedal \hr{Saurian}{\saurian} with thick, elephantine legs, three forward-pointing horns on its head and a wide frill covering its short neck. 
Can grow ten metres long. 

\meta{%
  Compare to ceratopsian dinosaurs like \latinname{Triceratops}.}







\begin{comment}
\paragraph{\quilrai}
\end{comment}
\gitem[\quilrais]{\quilrai}
The largest species of \hr{pteran}{pterans} known on \Miith{}.
It is white and azure in \colour and has a wingspan of up to 15 metres. 









\begin{comment}
\paragraph{pteran}
\end{comment}
\gitem{pteran}
\target{pteran}
Flying lizard-like creatures related to \hr{Saurian}{\saurians}. 
Species include the \ravcor{} and the \quilrai. 

\meta{%
  Compare to pterosaurs like \latinname{Pteranodon}.}









\begin{comment}
\paragraph{\ravcor}
\end{comment}
\gitem{\ravcor}
A species of \hr{pteran}{pterans}. 
It is black and white in \colour and has a wingspan of up to four metres. 







\begin{comment}
\paragraph{\relc}
\end{comment}
\gitem[\relcs]{\relc}
\target{Relc}
A herbivorous, quadrupedal \hr{Saurian}{\saurian}. 
It around 5-6 metres long and has an elaborate crest on its head. 
\Relcs{} vary widely in \colour. 
Can be tamed and used as mounts or pack animals. 

A person who rides a \relc{} is called a \relcer{}. 

\meta{%
  Similar to dinosaurs like \latinname{Saurolophus} or \latinname{Corythosaurus}, but smaller. }







\begin{comment}
\paragraph{\saurian}
\end{comment}
\gitem{\saurian}
\target{Saurian}
Any of a diverse group of reptile-like animals. 
Many species are large. 
Species include the \hr{Nycan}{\nycan}, \hr{Relc}{\relc}, \hr{Corgorah}{\corgorah}, \hr{Mulgron}{\mulgron} and \hr{Tondra}{\tondra}. 
Related to \hr{pteran}{pterans}. 

\meta{%
  Dinosaurs. }







\begin{comment}
\paragraph{\tondra}
\end{comment}
\gitem{\tondra}
\target{Tondra}
A very large, quadrupedal, herbivorous \hr{Saurian}{\saurian} with a massive body and a long neck and tail. 
It is heavily built and \coloured in shades of white, beige and tan. 
One of the largest known \saurians, it can reach 35 metres in length and weigh 100 tonnes. 

\Tondras{} were domesticated by the \hr{Vaimon Caliphate}{\VaimonCaliphate} and other cultures of their time, but no longer. 

\meta{%
  Compare to sauropod dinosaurs like \latinname{Brachiosaurus} or \latinname{Apatosaurus} (aka \latinname{Brontosaurus}). 
}



\end{gloss}



















\section{Geography and Cultures}
\begin{gloss}



\begin{comment}
\paragraph{\Azmith}
\end{comment}
\gitem{\Azmith}
\target{Azmith}
A continent on the planet \Miith{} consisting of \hr{Velcad}{\Velcad}, the {Northern Kingdoms}, {Threll}, the \hs{Imetrium}, \hs{Uzur}, {\Durcaccontinent}, the Near \hs{Orient} and the {\Serplands}. %Some scholars also consider Irokas a part of \Azmith{} while others do not. 

\quo{\Azmith} is \maybehr{Archaic Vaimon language}{Archaic Vaimon} for \quo{all the world} (containing the word \quo{\Miith}, meaning \quo{the world}). 







\begin{comment}
\paragraph{Beirod}
\end{comment}
\gitem{Beirod}
\target{Beirod}
A kingdom in the southern \hr{Velcad}{\Velcad}. 
It borders \hs{Pelidor} to the northwest, \hs{Runger} to the northeast, \hr{Scyrum}{\Scyrum} to the west (marked by \hs{Heropond Forest}), {Gaznor} to the south, the \Risvaelsea{} to the southeast and the Lorn Sea to the east.  

Languages and ethnic groups include \maybehr{Velcadian language}{\Velcadian} and \Ortic. 









\begin{comment}
\paragraph{Belek}
\end{comment}
\gitem{Belek}
\target{Belek}
A kingdom in eastern \hr{Velcad}{\Velcad}. 
Ruled by House Belek.







\begin{comment}
\paragraph{Clictua}
\end{comment}
\gitem{Clictua}
\target{Clictua}
A \meccaran{} tribe. 







\begin{comment}
\paragraph{Durcac}
\end{comment}
\gitem{Durcac}
\index{Durkhak}
\target{Durcac}
The homeland of the \hs{Rissitics}, located in southern \hr{Azmith}{\Azmith}. 
%Pronounced Durkhak in the Rissitic language. 
Governed by a theocracy. 

\quo{Durkhak} is the Rissitic pronunciation. 
\quo{Durcac} is the \Velcadian{} pronunciation. 
\also{Rissitics}









\begin{comment}
\paragraph{\Galessan}
\end{comment}
\gitem{\Galessan}
\target{Galessan}
\target{Pelidor Continent}
A continent in central {\Azmith}. 
Contains \hr{Scyrum}{\Scyrum}, \hs{Pelidor}, \hs{Beirod}, \hs{Runger} and {Gaznor}. 









\begin{comment}
\paragraph{Geica}
\end{comment}
\gitem{Geica}
\target{Geica}
A kingdom in the Orient and the homeland of \ClanGeican. 
Governed by a democracy.

The symbol of Geica and of \ClanGeican is a green eagle on a black background.
\also{Geican}







\begin{subgloss}
  \begin{comment}
  \subparagraph{Geican}
  \end{comment}
  \gitem{Geican}
  \target{Geican}
  A \VaimonClan. 
  The Geicans hail the freedom of the individual as their highest ideal. 
  
  Their homeland is Geica.
  
  Unlike the Redcor and \Telcra, which are fundamentally religious organizations, Geican philosophy is atheistic. 
  They believe that it is the \Archons{} who serve the Vaimons, not the other way around. 
  
  The founder of \ClanGeican was Tiraad Geican, son of Cordos Vaimon. 
  Their symbol is the silhouette of a green eagle on a black background, and their traditional \colour is green. 
  \also{Geica, Vaimon}
\end{subgloss}







\begin{comment}
\paragraph{\Goyden}
\end{comment}
\gitem[\Goydens]{\Goyden}
A savage people living in the \Wylde{} in southern Pelidor and northern Beirod. They are \humans, but some say they are half \human{} and half beast. They speak their own language and pray to their own gods. According to some rumours they can change between \human{} and animal form. 









\begin{comment}
\paragraph{\GreatVelcad}
\end{comment}
\gitem{\GreatVelcad}
\target{Great Velcad}
An empire that once covered all of the area now simply called \hr{Velcad}{\Velcad}. 
Ruled by the \hr{High King}{High Kings} of House \Velcad. 







\begin{subgloss}
  \begin{comment}
  \subparagraph{High King}
  \end{comment}
  \gitem{High King}
  \target{High King}
  Title used by the rulers of \GreatVelcad{}. 
  The High King's wife was called High Queen. 
  (There was never a ruling High Queen.)
\end{subgloss}














\begin{comment}
\paragraph{Heropond Forest}
\end{comment}
\gitem{Heropond Forest}
\target{Heropond Forest}
\target{Heropond}
A large \Wylde{} forest in \hr{Pelidor Continent}{\PelidorContinent}. 
Marks the border between \hs{Pelidor} and \hr{Scyrum}{\Scyrum}. 





\begin{subgloss}
  \begin{comment}
  \subparagraph{Leglan's Pass}
  \end{comment}
  \gitem{Leglan's Pass}
  \target{Leglan's Pass}
  A path through the narrowest part of Heropond forest, near \hr{Bryndwin}{\Bryndwin} (on the \hr{Scyrum}{\Scyric} side) and \hr{Redglen}{\Redglen} (on the Pelidorian side. 
  It was formerly used as a trade route, but has fallen out of use in recent decades and grown more \hr{Wild}{\Wylde}. 
\end{subgloss}














\begin{comment}
\paragraph{The Imetrium}
\end{comment}
\gitemthe{Imetrium}
\target{Imetrium}
\target{Imetric}
%A nation south-east of \Velcad{}. It is a theocratic state, and \quo{the Imetrium} is also the name of the religion. Its people are called Imetrians. They speak the Imetric language. 
The Imetrium is a theocratic nation in western \hr{Azmith}{\Azmith}. 
It is also the name of a religion. 
The religion has some adherents in eastern \hr{Velcad}{\Velcad}, especially {Andras} and \hr{Scyrum}{\Scyrum}. 

The symbol of the Imetrium is a white, four-spoked star, edged with blue within a circle of black against a white background. 
The \colours of the Imetrium are white, blue and black. 

Imetric names are traditionally given with the family name first and the personal name last. 





\begin{subgloss}
  \begin{comment}
  \subparagraph{Clerical ranks}
  \end{comment}
  \gitem{clerical ranks}
  \index{Imetrium!clerical ranks}
  The titles used by the Imetric clergy are (in descending order): 
  
  \target{Laccorin}
  \target{Ispan}
  \target{Telphan}
  \target{Amra}
  \begin{itemize}
    \item \Laccorin{} (similar to a cardinal).
    \item \Ispan.
    \item \Telphan. 
    \item \Amra{} (a regular priest).
  \end{itemize}
  
  A generic word for \quo{priest} is \Stracos, plural \Stracoi. 
  % \also{The Imetrium}



  \begin{comment}
  \subparagraph{Fendor}
  \end{comment}
  \gitem{Fendor}
  \target{Fendor}
  An island in the \Risvaelsea, controlled by the Imetrium. 
  The two major towns on Fendor are \Fendacor{} (east) and \Cicora{} (west). 



  \begin{comment}
  \subparagraph{Imetric gods}
  \end{comment}
  \gitem{Imetric gods}
  \target{Imetric gods}
  The Imetric religion is polytheistic. 
  The most important gods are: 
  
  \begin{subgloss}
    \gitem{Salacar} 
    \index{Salacar}
    \target{Salacar}
      God of justice, head of the pantheon.
    \gitem{Dessali} 
    \index{Dessali}
    \target{Dessali}
      Goddess of reason, science and knowledge. 
    \gitem{\NishiS}
    \index{\NishiS}
    \target{Nishi}
      Goddess of life and death. 
    \gitem{Eoncos}
    \index{Eoncos}
    \target{Eoncos}
      God of strength, bravery and war. 
    \gitem{\Hiothrex}
    \index{\Hiothrex}
    \target{Hiothrex}
      God of vengeance. 
  \end{subgloss}



  \begin{comment}
  \subparagraph{Martinum}
  \end{comment}
  \gitem{Martinum}
  \index{Martinum}
  \target{Martinum}
  A port city. 
  One of the greatest cities of the Imetrium. 
  
  
  
  \begin{comment}
  \subparagraph{Military ranks}
  \end{comment}
  \gitem{Military ranks}
  \index{Imetrium!military ranks}
  The military ranks used by the Imetric army are (in descending order): 
  
  \target{Deccor}
  \target{Retaxis}
  \target{Salican}
  \target{Vexstra}
  \target{Corphin}
  \target{Inclan}
  \begin{itemize}
    \item \Deccor{} (equivalent to a general). 
    \item \Retaxis{}. 
    \item \Salican{}. 
    \item \Vexstra{}. 
    \item \Corphin{}. 
    \item \Inclan{} (equivalent to a private). 
  \end{itemize}

  A generic word for \quo{soldier} is \Rengos, plural \Rengoi. 
  % \also{The Imetrium}
\end{subgloss}






\begin{comment}
\paragraph{\Iquinian Church}
\end{comment}
\gitem{\Iquinian Church}
\target{Iquinian Church}
\target{Iquinian}
A Vaimon religion based around the worship of \Iquin{} and the \sephiroth. 
The religion is split into two major branches: 
The \hs{Redcor} Church and the \hr{Telcra}{\Telcra}. 

The core message of Iquinianism is to follow the sixteen virtues of the \sephiroth.






\begin{subgloss}
  \begin{comment}
  \subparagraph{Clerical ranks}
  \end{comment}
  \gitem{Clerical ranks}
  \index{\Iquinian{} Church!clerical ranks}
  The titles used by Iquinian clergy are (male/female, in descending order): 
  
  \target{Patriarch}
  \target{Patron}
  \target{Pater}
  \target{Frater}
  \target{Matriarch}
  \target{Matron}
  \target{Mater}
  \target{Soror}
  \target{Neophyte}
  \index{\Patriarch}
  \index{\Patron}
  \index{\Pater}
  \index{\Frater}
  \index{\Matriarch}
  \index{\Matron}
  \index{\Mater}
  \index{\Soror}
  \index{\Neophyte}
  \begin{itemize}
    \item \Patriarch/\Matriarch. 
    \item \Patron/\Matron.
    \item \Pater/\Mater.
    \item \Frater/\Soror. 
    \item \Neophyte. 
  \end{itemize}
\end{subgloss}







\begin{comment}
\paragraph{\Masthenon}
\end{comment}
\gitem{\Masthenon}
\target{Mastheno}
\target{Masthenon}
\index{\Masthenon}
A \hs{Tassian} \scathaese{} people that flourished during the time of the \hr{Vaimon Caliphate}{\VaimonCaliphate}. 

A single person is a \quo{\Masthen}, plural \quo{\Masthenon}, adjective \quo{\Mastheno}. 
\quo{The \Masthenon} is also the name of their civilization as a whole. 

The \hr{Ortaica}{\Ortaicans} descend from the \Masthenon. 
The \hr{Shurco}{\Shurco} have some \Masthenon{} heritage in their blood and culture, but also other things. 
















\begin{comment}
\paragraph{\Marcil}
\end{comment}
\gitem{\Marcil}
\index{\Marcil}
\target{Marcil}
A nation in northeastern \Velcad{}. 
It borders \hr{Redce}{\Redce} to the west, {Rosval} to the north, {Gonarod} to the south and the {\Serpsea} to the east. 





\begin{subgloss}
  \begin{comment}
  \subparagraph{Bendaire}
  \end{comment}
  \gitem{Bendaire}
  \target{Bendaire}
  \index{Bendaire}
  A city in \Marcil. 
\end{subgloss}







\begin{comment}
\paragraph{Nerim}
\end{comment}
\gitem{Nerim}
\index{Nerim}
\target{Nerim}
A fjord that runs south from the Hirum Gulf and marks the border between \hs{Runger} and \hs{Pelidor}. 







\begin{comment}
\paragraph{Ontephar}
\end{comment}
\gitem{Ontephar}
\index{Ontephar}
\target{Ontephar}
A kingdom in \PelidorContinent. 
It borders Pelidor to the south, Runger to the east and \hr{Scyrum}{\Scyrum} to the south-west. 
It is ruled by an Archduke. 
%Ontephar was previously the heartland of the \Tepharin{} Bacconate. 

Languages spoken include \Velcadian{}, \Tepharin{} and occasionally Imetric. 







\begin{comment}
\paragraph{Orient}
\end{comment}
\gitem{Orient}
\target{Orient}
\index{Orient}
The area southeast of \Velcad{} (east of Durcac and south of the {\Serplands}). 
Informally split into the {Near Orient} and the {Far Orient}.  







\begin{comment}
\paragraph{\Ortaica}
\end{comment}
\gitem{\Ortaica}
\target{Ortaica}
A great \hr{Scatha}{\scatha}-dominated \hr{Bacconate}{\bacconate} that flourished after the \hr{Hundred Scourges}{\HundredScourges}. 
At the height of its power, it covered all of southern \hr{Azmith}{\Azmith}, including the lands that are now the \hs{Imetrium}, \hs{Durcac}, southern \hr{Velcad}{\Velcad} and even \hs{Uzur} and \hs{Geica}. 
\Scathaese{} cultures today, especially the Imetrium, owe much of their culture to the Ortaicans. 

The \Ortaicans{} worshipped the pantheon of the \hr{Taortha}{\Taorthae}. 








\begin{comment}
\paragraph{Pelidor}
\end{comment}
\gitem{Pelidor}
\index{Pelidor}
\target{Pelidor}

A kingdom in southern \hr{Velcad}{\Velcad}. 
It borders \hs{Runger} to the northeast (marked by the river Nerim), Beirod to the southeast, \hr{Scyrum}{\Scyrum} to the west (marked by \hs{Heropond Forest}) and the Hirum Gulf to the north. 

Governed by an elective monarchy. 
The ruling family is House Pelidor. 
The current ruler is \hr{Icor}{\rayuth[\Icor] Pelidor}. 

The primary languages are \Tepharin{} and \Velcadian{}. 





\begin{subgloss}
  \begin{comment}
  \subparagraph{Besuld}
  \end{comment}
  \gitem{Besuld}
  \index{Besuld}
  A port city in southeastern Pelidor. 
  
  
  
  
  
  \begin{comment}
  \subparagraph{Black Plague}
  \end{comment}
  \gitem{Black Plague}
  \index{Black Plague}
  \index{plaguer}
  \target{Black Plague}
  \target{plaguers}
  A thieves' guild in \hr{Malcur}{\Malcur}. 
  Rumoured to have dealings with sorcerers. 
  Their members are sometimes nicknamed \quo{plaguers}. 





  \begin{comment}
  \subparagraph{Dendrum}
  \end{comment}
  \gitem{Dendrum}
  \index{Dendrum}
  \target{Dendrum}
  A city in eastern Pelidor. 
  Lies near the \hs{Ucarn}, approximately halfway between \hr{Forklin}{\Forklin} and {Ucarnum}. 






  \begin{comment}
  \subparagraph{\Forklin}
  \end{comment}
  \gitem{\Forklin}
  \index{\Forklin}
  \target{Forklin}
  A city in eastern Pelidor, one of Pelidor's largest and most fortified cities. 
  Was once an independent city-state, before being subsumed into the duchy of Pelidor. 
  
  
  
  
  
  \begin{comment}
  \subparagraph{Gilwaed}
  \end{comment}
  \gitem{Gilwaed}
  \index{Gilwaed}
  A village in eastern Pelidor, north of the \hs{Ucarn}, somewhere between \hr{Forklin}{\Forklin} and {Ucarnum}. 
  The name is \Tepharin. 
  
  Population: Approximately 100 \scathae. 
  
  
  
  
  
  \begin{comment}
  \subparagraph{House Pelidor}
  \end{comment}
  \gitem{House Pelidor}
  \index{Pelidor!House Pelidor}
  \index{House Pelidor}
  \target{House Pelidor}
  The ruling noble family of Pelidor. 
  Split into two branches, one based in \hr{Malcur}{\Malcur} and one based in \hr{Forklin}{\Forklin}. 
  Infighting between the two branches is not unheard of. 
  
  
  
  
  
  \begin{comment}
  \subparagraph{Kenshaer}
  \end{comment}
  \gitem{Kenshaer}
  \index{Kenshaer}
  \target{Kenshaer}
  \target{Kenshaer Forest}
  A \hr{Wild}{\Wylde} forest in eastern Pelidor, north of the \hs{Ucarn} between \Forklin{} and \hs{Dendrum}. 
  
  
  
  
  
  \begin{comment}
  \subparagraph{\Malcur}
  \end{comment}
  \gitem{\Malcur}
  \index{\Malcur}
  \target{Malcur}
  The capital city of Pelidor. 
  %\also{Pelidor}
  Was once an independent city-state, before being subsumed into the duchy of Pelidor. 
  
  
  
  
  
  \begin{comment}
  \subparagraph{\Redglen}
  \end{comment}
  \gitem{\Redglen}
  \index{\Redglen}
  \target{Redglen}
  A town in eastern Pelidor, near \hs{Heropond Forest} and west of Torgin. 
  It is the hometown of \hr{Carzain Shireyo}{Carzain \Shireyo} and his family. 
  
  
  
  
  
  \begin{comment}
  \subparagraph{Torgin}
  \end{comment}
  \gitem{Torgin}
  \index{Torgin}
  A city in southern Pelidor, east of \Redglen. 





  \begin{comment}
  \subparagraph{Ucarn}
  \end{comment}
  \gitem{Ucarn}
  \target{Ucarn}
  \target{Ucarn Road}
  \index{Ucarn}
  A river in eastern \hs{Pelidor}. 
  Runs from \hr{Forklin}{\Forklin} in the west out to the port city of {Ucarnum} at its mouth in the east, where it meets the (much wider) \hs{Nerim}. 
  
  A large and well-kept road, called simply the Ucarn Road, is built along the banks of the river. 
\end{subgloss}









\begin{comment}
\paragraph{\Redce}
\end{comment}
\gitem{\Redce}
\index{\Redce}
\target{Redce}
A nation northeast of \Velcad{}. 
The homeland of \ClanRedcor. 
A theocracy ruled by the Redcor. 







\begin{subgloss}
  \begin{comment}
  \subparagraph{Redcor}
  \end{comment}
  \gitem{Redcor}
  \index{Redcor}
  \target{Redcor}
  A \VaimonClan and a religious organization, a branch of the Iquinian Church. 
  The Redcor homeland is \Redce, where their Conclave rules from the \TopazChateau. 
  
  The founder of \ClanRedcor was Rebecca Redcor, daughter of Cordos Vaimon and Silqua \Delaen. 
  Their symbol is a yellow Sun on a blue background, and their traditional \colour is yellow. 
  
  The \vclan is split into three \RedcorHouses: 
  
  \begin{subgloss}
    \gitem{\TheFoxFaction} 
    \index{\FoxFaction} 
      The most warlike faction, and perhaps the most free-thinking. 
      Their sign is a red fox. 
    \gitem{\TheSwanFaction} 
    \index{\SwanFaction} 
      The most peaceful faction. 
      Their sign is a white swan. 
    \gitem{\TheTulipFaction} 
    \index{\TulipFaction} 
      The most orthodox faction. 
      Their sign is a yellow tulip.
  \end{subgloss}
  
  
  
  
  
  \begin{comment}
  \subparagraph{\gandierre}
  \end{comment}
  \gitem[\gandierres]{\gandierre}
  \target{Gandierre}
  An all-male order of Vaimon warrior-mages under \ClanRedcor. 
  
  
  
  
  
%   \begin{comment}
%   \subparagraph{\ryzin}
%   \end{comment}
%   \gitem[\ryzins]{\ryzin}
%   \target{Ryzin}
%   An all-female order of Vaimon warrior-mages under \ClanRedcor. 
  
  
  
  
  
  \begin{comment}
  \subparagraph{\TopazChateau}
  \end{comment}
  \gitem{\TopazChateau}
  \target{Topaz Chateau}
  The central stronghold of the Redcor in \Redce. 
\end{subgloss}








\begin{comment}
\paragraph{Rissitics}
\end{comment}
\gitem{Rissitics}
\index{Rissitics}
\target{Rissitic}
\target{Rissitics}
Rissitics are the people who worship \HriistN{} (also called Rissit) and his pantheon. 
The Rissitics rule the theocratic nation of \hs{Durcac}. 

\index{Caste system}
\index{Tsalt}
\index{\Rekkan}
\index{\Bedhin}
\index{\Kyth}
\index{\Hok}
\index{Gzend}
The Rissitic people are organized in a caste system. 
The castes, in descending order of rank, are the Tsalt (priests and mages), \Rekkan{} (knights), \Bedhin{} (craftsmen), \Kyth{} (warriors), \Hok{} (labourers) and Gzend (slaves). 






\begin{subgloss}
  \begin{comment}
  \subparagraph{\Ashenoch}
  \end{comment}
  \gitem{\Ashenoch (plural \Ashenoch)}
  A Rissitic order of superhuman warrior-mages. 




  \begin{comment}
  \subparagraph{dagger sign}
  \end{comment}
  \gitem{dagger sign}
  \index{sign of the dagger}
  The \quo{dagger sign} or the \quo{sign of the dagger} is a Rissitic gesture of greeting. 
  It consists of a flat hand held up before one's face with the thumb-edge of the hand near the nose and the fingers pointing up. 
  The hand is then moved down to form a fist touching the center of the chest, fingers inward. 
  
  The sign is an imitation of \hs{Rissit}'s symbol of a snake coiled around a dagger. 
  
  
  
  
  \begin{comment}
  \subparagraph{\HriistN}
  \end{comment}
  \gitemcharacter{\HriistN}{immortal}{\male}
  \index{\Nechsain}
  \target{Rissit}
  The foremost god of the {Rissitics}. \Hriist{} is his true name and \Nechsain{} his title, both of which are used only by believers. Nonbelievers call him Rissit. 
  
  Rissit's symbol is a dagger thrust into the ground with a cobra coiled around it, as if crawling down from the heavens. 
\end{subgloss}
\also{\hs{Durcac}}







\begin{comment}
\paragraph{Runger}
\end{comment}
\gitem{Runger}
\index{Runger}
\target{Runger}
A kingdom in central \hr{Velcad}{\Velcad}. 
It borders \hs{Beirod} to the south, \hs{Pelidor} to the west (marked by the river \hs{Nerim}), the Gwendor Sea to the north and the Lorn Sea to the east. 

%It borders the Hirum Gulf to the northwest, Pelidor to the southwest and Beirod to the south. 

Governed by an hereditary monarchy. 
The ruling family is House Runger. 
The current ruler is King {Morgan Runger} I, son of Uther I. 

The banner of Runger is two brown wolverines chasing each other in a circle, against a tan background.





\begin{subgloss}
  \begin{comment}
  \subparagraph{Dormina}
  \end{comment}
  \gitem{Dormina}
  \index{Dormina}
  \target{Dormina}
  The capital city of Runger.
  
  
  
  
  
  
  \begin{comment}
  \subparagraph{\EreshKal}
  \end{comment}
  \gitem{\EreshKal}
  \index{\EreshKal}
  \target{Eresh-Kal}
  A \hr{Meccaran}{\meccaran} civilization that once existed in modern-day Runger.
  Allegedly, a lost \EreshKali{} temple lies hidden in \hs{Waythane Forest}.  

  
  
  
  
  
  \begin{comment}
  \subparagraph{Gedrock}
  \end{comment}
  \gitem{Gedrock}
  \index{Gedrock}
  \target{Gedrock}
  A village in Runger, near \hs{Waythane Forest}. 
  
  
  
  
  
  
  \begin{comment}
  \subparagraph{Waythane Forest}
  \end{comment}
  \gitem{Waythane Forest}
  \index{Waythane Forest}
  \target{Waythane Forest}
  \index{Runger!Waythane Forest}
  A large forest in southern Runger. 
  Allegedly hides the last temple of \hr{Eresh-Kal}{\EreshKal}. 

  
  
  
  
  
\end{subgloss}







\begin{comment}
\paragraph{\Scyrum}
\end{comment}
\gitem{\Scyrum}
\target{Scyrum}
A nation southwest of \Velcad{}. 
Borders Pelidor to the east, with the border marked by \hs{Heropond Forest}. 

The capital city is Pylandos. 
Other cities include Icconos. 

Languages and ethnic groups include \Tepharin, \Samurin{} and \Ortic.  





\begin{subgloss}
  \begin{comment}
  \subparagraph{\Bryndwin}
  \end{comment}
  \gitem{\Bryndwin}
  \target{Bryndwin}
  A village in eastern \Scyrum, near \hs{Heropond Forest}. 

  
  
  
  
  
  \begin{comment}
  \subparagraph{\Pylandos}
  \end{comment}
  \gitem{\Pylandos}
  \target{Pylandos}
  The capital city of \Scyrum. 

  
  
  
  
  
  \begin{comment}
  \subparagraph{\Pylandos{} Road}
  \end{comment}
  \gitem{\Pylandos{} Road}
  \target{Pylandos Road}
  A great road that runs from \Pylandos{} in the north and south down through \Scyrum. 
\end{subgloss}









\begin{comment}
\paragraph{\Shurco}
\end{comment}
\gitem{\Shurco}
\target{Shurco}
\index{\Shurcarie}
The \Shurcarie{} was a \scathaese{} \hr{Mekrii}{\Mekrii} \hr{Bacconate}{\bacconate} that existed during the time of the \hr{Vaimon Caliphate}{\VaimonCaliphate}, controlling much of \DurcacContinent. 
It was defeated in a series of wars and did not last after the \hr{Hundred Scourges}{\HundredScourges}. 

Modern-day \hs{Durcac} owes much of its cultural heritage to the \Shurco.









\begin{comment}
\paragraph{\Tepharite}
\end{comment}
\gitem[\Tepharites]{\Tepharite}
\target{Tepharae}
\target{Tepharin}
The \Tepharites{} are a people living in southern \Velcad{} and including both \scathae{} and \humans{}. 
They once ruled a \hr{Bacconate}{\bacconate}, \Tepharae. 
The \Tepharin{} language is still spoken many places in \hs{Ontephar}, \hs{Pelidor} and \hr{Scyrum}{\Scyrum}. 
%\also{\Tepharin{} Bacconate}









\begin{comment}
\paragraph{\Thbatswa}
\end{comment}
\gitem{\Thbatswa}
\target{Thbatswa}
A \meccaran{} tribe living in Pelidor and surrounding kingdoms. 





\begin{subgloss}
  \begin{comment}
  \subparagraph{Bled}
  \end{comment}
  \gitem{Bled}
  \index{Bled}
  \target{Bled}
  A \hr{Thbatswa}{\Thbatswa} god. 
  Lweddish
  
  
  \begin{comment}
  \subparagraph{Lweddish}
  \end{comment}
  \gitem{Lweddish}
  \index{Lweddish}
  \target{Lweddish}
  A \hr{Thbatswa}{\Thbatswa} god. 
  
  
  \begin{comment}
  \subparagraph{Thudun}
  \end{comment}
  \gitem{Thudun}
  \index{Thudun}
  \target{Thudun}
  A \hr{Thbatswa}{\Thbatswa} god. 
\end{subgloss}







\begin{comment}
\paragraph{Uzur}
\end{comment}
\gitem{Uzur}
\target{Uzur}
\index{Uzur}
A region south of the \hs{Imetrium} and southwest of \hr{Velcad}{\Velcad}. 
Much of it is covered in \hr{Wild}{\Wylde} forests and swamps, but there are nations that thrive here. 
\hr{Meccaran}{\meccara} are the dominant humanoids there. 









\begin{comment}
\paragraph{\VaimonCaliphate}
\end{comment}
\gitem{\VaimonCaliphate}
\target{Vaimon Caliphate}
\index{\VaimonCaliphate}
A \human-dominated empire ruled by the \hs{Vaimons}. 
It existed from the year \yic{Founding of the Vaimon Caliphate}, where it was founded by \hs{Cordos Vaimon}, and until the \hr{Hundred Scourges}{\HundredScourges} where it fell, during the reign of \hr{Belzir}{\Belzir}. 





\begin{subgloss}
  \begin{comment}
  \subparagraph{\HundredScourges}
  \end{comment}
  \gitem{\HundredScourges}
  \target{Darkfall}
  \target{Hundred Scourges}
  The series of calamities, worldly and metaphysical, that destroyed the \VaimonCaliphate. 
  Began in the year \yic{Darkfall}. 
  
  The first Scourge was Immortality, blamed on the Vaimon \Calipha \hr{Belzir}{\Belzir}. 
  Some of the other Scourges include War, Crime, Disease, Stillbirth and Famine. 




  \begin{comment}
  \subparagraph{Rainbow Palace}
  \end{comment}
  \gitem{Rainbow Palace}
  \target{Rainbow Palace}
  \index{Rainbow Palace}
  The palace of the \VaimonCaliph in \hr{Merodar}{\ShiinMerodar}. 




  \begin{comment}
  \subparagraph{\ShiinMerodar}
  \end{comment}
  \gitem{\ShiinMerodar}
  \target{Merodar}
  The ancient capital city of the \VaimonCaliphate.
  Destroyed in the \hr{Hundred Scourges}{\HundredScourges}. 
\end{subgloss}









\begin{comment}
\paragraph{\Velcad}
\end{comment}
\gitem{\Velcad}
\index{\Velcadian!language)}
\target{Velcad}
\target{Velcadian}
A region encompassing a large part of central \Azmith. 
This was previously the empire of \hr{Great Velcad}{\GreatVelcad}, but is now split into many independent kingdoms and much \Wylde{}. 

The \Velcadian{} tongue is spoken in much of \Velcad{}, especially among merchants and the upper classes. 









\begin{comment}
\paragraph{\Vidra}
\end{comment}
\gitem{\Vidra}
\target{Vidra}
An island kingdom in the northern \hr{Velcad}{\Velcad}. 









\end{gloss}



















\section{Miscellaneous}
\begin{gloss}



\begin{comment}
\paragraph[Baccon]{\baccon}
\end{comment}
\gitem[\baccons]{\baccon}
\index{\bacconate}
\target{Bacconate}
A ruling council in certain (predominantly \scathaese{}) cultures. 
Typically aristocratic in nature. 
A kingdom ruled by a \baccon{} is called a \bacconate{}. 
In ancient times there were the great \hr{Ortaica}{\Ortaican} and \hr{Shurco}{\Shurco} \Bacconates. 
Today, there are some \bacconates{} left in southern \Velcad{} and the \hs{Orient}. 







\begin{comment}
\paragraph{Black Dawn calendar}
\end{comment}
\gitem{Black Dawn calendar}
\index{Black Dawn calendar}
\target{Black Dawn calendar}
A \resphan{} calendar. 
Year \ybd{Murder of the Dawn} is the year of the Murder of the Dawn. 

Year \ybd{Murder of the Dawn} corresponds to \yds{Murder of the Dawn}. 
Year \ybd{IC} corresponds to \yic{IC}. 







\begin{comment}
\paragraph{The Cabal}
\end{comment}
\gitemthe{Cabal}
\index{Cabalist}
\target{Cabal}
\target{Cabalist}
\target{Cabalists}
A secret organization. 
Its members are called Cabalists.







\begin{comment}
\paragraph{cavalry}
\end{comment}
\gitem{cavalry}\index{cavalry}
General term for mounted soldiers. 
Possible mounts include \relcs, \mikshas, \grulcans, \mulgrons, elephants and other beasts. 







\begin{comment}
\paragraph{\dai}
\end{comment}
\gitem{\dai}
\target{Dai}
In \maybehr{Imetric language}{Imetric}, \quo{\dai} can be prefixed to a name, title or pronoun to form a polite vocative. 
(\quo{\Dai-} is capitalized if prefixed to an already capitalized word, such as a name, but otherwise not.)







\begin{comment}
\paragraph{Dark Crescent}
\end{comment}
\gitem{Dark Crescent}
\index{Dark Crescent}
\target{Dark Crescent}
A cult affiliated with the \hr{Sentinels}{Sentinels of \Miith}. 
Its leader is \hr{Psyrex}{\LocarPsyrex}. 







\begin{comment}
\paragraph{\Draconian{} Supremacy calendar}
\end{comment}
\gitem{\Draconian{} Supremacy calendar}
\target{Draconian Supremacy calendar}
A \draconic{} calendar. 
Year \yds{XS} is the year where \TyarithXserasshana{} first successfully invoked the \xss. 

Year \yds{Murder of the Dawn} corresponds to \ybd{Murder of the Dawn}. 
Year \yds{IC} corresponds to \yic{IC}. 









\begin{comment}
\paragraph{ducata}
\end{comment}
\gitem[ducatae]{ducata}
\index{ducata}
\target{ducata}
\target{ducatae}
An Imetric coin. 
Slighly smaller than a \hr{Velcadian sun}{\Velcadian{} sun}. 









\begin{comment}
\paragraph{\dvingen}
\end{comment}
\gitem{\dvingen}
\index{\dvingen}
\target{Dvingen}
A plant. 
Its leaves are used in healing. 
Its root, when chewed, is a strong narcotic. 









\begin{comment}
\subparagraph{\empyrean}
\end{comment}
\gitem{\empyrean}
\target{Empyrean}
\index{\empyrean}
In Vaimon metaphysics, the \quo{plane} that the \hr{Archon}{\Archons} are said to inhabit. 









\begin{comment}
\paragraph{glaive}
\end{comment}
\gitem{glaive}
\index{glaive}
\index{\Triestessakhin}
\target{glaive}
Term used to refer to \Triestessakhin, the scythe-like weapon wielded by Quessanth \Ishnaruchaefir. 








\begin{comment}
\paragraph{heathen}
\end{comment}
\gitem{heathen}
\index{heathen}
Iquinian term for nonbelievers. 









\begin{comment}
\paragraph{The \Imetriad}
\end{comment}
\gitemthe{\Imetriad}
The holy scripture of the Imetrium.







\begin{comment}
\paragraph{invocation}
\end{comment}
\gitem{invocation}
\index{invocation}
\target{invocation}
\target{invocations}
A spell that calls upon one or more entities and attempts to compel said entities to do the mage's bidding. 
Invocations are more difficult to learn than \hs{orisons}, but more powerful. 








\begin{comment}
\paragraph{\ishrah{} (plural \ishroth)}
\end{comment}
\gitem[\ishroth]{\ishrah}
\target{Ishrah}
A group of \hs{mages} working together. 
The term is \maybehr{Vaimon language}{Vaimon}, but today also applied to non-Vaimon mages. 







\begin{comment}
\paragraph{\jiliba}
\end{comment}
\gitem{\jiliba}
\target{Jiliba}
\index{\jiliba}
A bush. 
Its berries can be used to brew an energizing tea. 







\begin{comment}
\paragraph{knights}
\end{comment}
\gitem{knights}
\target{knight}
\target{knights}
\index{knights}
In \Velcad{}, \hs{Tiger} warriors may be given the status of knights. 
Knighting is performed by the Iquinian Church. 
Most knights are of noble birth.
\also{\hs{Tiger}}







\begin{comment}
\paragraph{The Light}
\end{comment}
\gitemthe{Light}
\seee{\iquin}







\begin{comment}
\paragraph{mage}
\end{comment}
\gitem{mage}
\index{mage}
\target{mage}
\target{mages}
A person trained to cast magic. 
Most mortal mages belong to either the \hs{Vaimon} order or the \hr{Rethyax}{\rethyax} order. 

In Vaimon tradition, a group of mages is called an \hr{Ishrah}{\ishrah}. 
\also{\hr{Ishrah}{\ishrah}, \hr{Rethyax}{\rethyax}, \hs{Vaimon}}







\begin{comment}
\paragraph{Marshal}
\end{comment}
\gitem{Marshal}
\index{Marshal}
\target{Marshal}
The supreme commander of the armies in certain \Velcadian{} countries. 
Often only temporarily appointed. 







\begin{comment}
\paragraph{Murder of the Dawn}
\end{comment}
\gitem{Murder of the Dawn}
\index{Murder of the Dawn}
A pivotal metaphysical event in \resphan{} history that heralded the downfall of \Merkyrah. 







\begin{comment}
\paragraph{\nycaneer}
\end{comment}
\gitem{\nycaneer}
\index{\nycaneer}
\target{Nycaneer}
A \hr{Scatha}{\scatha} with an inborn talent for telepathy and a special affinity with \hr{Nycan}{\nycans}. 







\begin{comment}
\paragraph{orison}
\end{comment}
\gitem{orison}
\index{orison}
\target{orison}
\target{orisons}
A simple spell. 
Consists of calling upon one or more entities and making a petition. 
Orisons are far easier to learn and cast than \hs{invocations}, but not nearly as powerful or flexible. 







\begin{comment}
\paragraph{\ranger}
\end{comment}
\gitem{\ranger}
\index{\ranger}
\target{Ranger}
A person skilled at surviving in and navigating through the \Wylde{}. 
They sometimes work as pathfinders or hunters. 
Sometimes considered outsiders and freaks by others. 







\begin{comment}
\paragraph{\rethyax}
\end{comment}
\gitem[\rethyaxes]{\rethyax}
\target{Rethyax}
A \hr{Scatha}{\scatha}-dominated order of mages, founded by the \hr{Ortaica}{\Ortaicans}. 
They practice a variant of Chaos magic, which involves the summoning of \hr{Daemon}{\daemons}. 

The associated adjective is \quo{\rethyactic}. 
\also{\Ortaica}







\begin{comment}
\paragraph{\Secondbanewar}
\end{comment}
\gitem{\Secondbanewar}
\target{Second Banewar}
\target{Incursion}
\index{Incursion}
An ancient war.
Called \quo{the Incursion} by the \resphain. 
Ended in the year \yic{Second Banewar ends}. 







\begin{comment}
\paragraph{\senaan}
\end{comment}
\gitem[\senaan]{\senaan}
\target{Senaan}
A class of swords used by the \resphain, especially \CiriathSepher. 
A \senaan{} is extremely long, 300 cm or more. 
The handle is rather long, comprising up to one fourth of the total length. 
The blade is pointed, curved and usually single-edged, designed for both stabbing and cutting. 







\begin{comment}
\paragraph{Sentinels of \Miith}
\end{comment}
\gitem{Sentinels of \Miith}
\index{Sentinels of \Miith}
\target{Sentinels}
\target{Sentinels of Miith}
A secret organization. 







\begin{comment}
\paragraph{sun}
\end{comment}
\gitem{sun}
\index{sun}
\target{Velcadian sun}
A standardized gold coin with a Sun printed on it. 
Used in \Velcad{}. 
Slightly larger than an Imetric \hs{ducata}.







\begin{comment}
\paragraph{tacupien}
\end{comment}
\gitem{tacupien}
\index{tacupien}
A Pelidorian dance.







\begin{comment}
\paragraph{\Telcra}
\end{comment}
\gitem{\Telcra}
\index{\Telcra}
\target{Telcra}
\ClanTelcra{} is the youngest \hr{Vaimon}{\VaimonClan}, the only one founded after the fall of the \hr{Vaimon Caliphate}{\VaimonCaliphate}. 
It is one of the two major branches of the \hr{Iquinian Church}{Iquinian religion}. 
It was founded in \hr{Tepharae}{\Tepharae}. 
Since the collapse of \Tepharae, \ClanTelcra{} has no central organization. 
Many members serve in the \hr{Ishrah}{\ishroth} of various kings or lords. 







\begin{comment}
\paragraph{Tigers}
\end{comment}
\gitem{Tigers}
\index{Tigers}
\target{Tiger}
An \hr{Velcadian}{\Velcadian} order of elite soldiers, founded by and named after \hs{Uther the Tiger}. 
Originally they were controlled by \hr{Great Velcad}{\GreatVelcad} and formed the backbone of the Imperial army. 
After the fall of that empire, the Tiger order has become splintered and decentralized. 
Today, most \Velcadian{} rulers maintain a small army of professional Tigers and mercenaries, and draft commoners to bolster them when necessary. 

Some Tigers are \hs{knights}. 
\also{\hr{Great Velcad}{\GreatVelcad}, \hs{knights}, \hs{Uther the Tiger}}







\begin{comment}
\paragraph{units of measurement}
\end{comment}
\gitem{units of measurement}
\index{units of measurement}
\index{foot (unit of measurement)}
\index{metre (unit of measurement)}
\index{yard (unit of measurement)}
\index{pace (unit of measurement)}
\index{man-height (unit of measurement)}
\meta{%
  This book uses various systems of measurements. 
  The metric system is meant to represent the scientific measurements that an advanced society (such as the \resphain{} or the Imetrium). 
  Feet, inches and yards represent the simpler measurements used by more primitive societies. 
  (A foot is 30.5 cm, not necessarily corresponding to a \Miithian{} \human{} or \ps{\scatha} foot.) 
  \quo{Paces}, \quo{man-heights} and the like indicate rough estimates, typically made by an uneducated person. }







\begin{comment}
\paragraph{Vaimon}
\end{comment}
\gitem{Vaimon}
\index{Vaimon}
\target{Vaimon}
\target{Vaimons}
A \hr{Human}{\human}-dominated order of \hs{mages}. 
The Vaimons are divided into a number of clans.
%, including Redcor and Geican. 
Today only the clans \hs{Redcor}, \hs{Geican} and \hr{Telcra}{\Telcra} survive. 
Extinct clans include \Zether and \Delaen. 

Vaimons affiliated with no \vclan are called rogues. 






\begin{subgloss}
  \begin{comment}
  \subparagraph{\cleric}
  \end{comment}
  \gitem{\cleric}
  \target{Cleric}
  Among the Redcor or \Telcra, a Vaimon working as a priest. 
  
  
  
  
  
  \begin{comment}
  \subparagraph{resonance}
  \end{comment}
  \gitem{resonance}
  \target{resonance}
  \quo{Resonance} is a state of being connected and \quo{attuned} to the \Archons. 
  Vaimons achieve resonance through meditation or prayer, usually practiced daily. 
  
  
  
  
  
  \begin{comment}
  \subparagraph{\templar}
  \end{comment}
  \gitem[\templars]{\templar}
  \target{Templar}
  A Vaimon who is not a \cleric. 
  May function as a warrior mage. 
\end{subgloss}








\begin{comment}
\paragraph{\VaimonCalendar}
\end{comment}
\gitem{\VaimonCalendar}
\index{Vaimon!Vaimon Calendar}
\target{Vaimon Calendar}
The calendar of the old \hr{Vaimon Caliphate}{\VaimonCaliphate}, still used by \hs{Vaimons} and in most of \hr{Velcad}{\Velcad}. 
\quo{$n$ \IC} denotes year number $n$ in the \ImperialCalendar{}, counting from the year when \hs{Cordos Vaimon} was crowned \caliph (year 1 \IC{}). 

A year is 380 days long. 

The \ImperialCalendar{} has sixteen months, dedicated to the sixteen \hr{Sephirah}{\Sephiroth}. 
Each week has eight days, named after the Vaimon founders. 

The end of the year is celebrated with the festival of \Camaire{} on the last day of \Gamishiel{}, midways between the winter solstice and the spring equinox. 






\begin{subgloss}
  \begin{comment}
  \subparagraph{days}
  \end{comment}
  \gitem{days}
  \index{days}
  \target{days}
  \index{\Corjin}
  \index{\Zetherab}
  \index{\Rebecab}
  \index{\Arcab}
  \index{\Norquin}
  \index{\Tirjin}
  \index{\Kerzab}
  \index{\Siljin}
  In the \hr{Vaimon Calendar}{\VaimonCalendar}, a week is eight days long. 
  Each day is named after of the \hs{Vaimon} founders.
  The days, in order, are:
  
  \begin{enumerate}
    \item 
      \Corjin{} (after \hs{Cordos Vaimon}, the first \VaimonCaliph).
    \item 
      \Zetherab{} (after \Zether Vaimon, son of Cordos and Silqua and the second \caliph).
    \item 
      \Rebecab{} (after Rebecca \hs{Redcor}, daughter of Cordos and Silqua).
    \item 
      \Arcab{} (after Arcan \Delaen, Silqua's eldest brother).
    \item 
      \Norquin{} (after Norcah Quaerin, son of Cordos and another wife).
    \item 
      \Tirjin{} (after Tiraad \hs{Geican}, son of Cordos and a third wife).
    \item 
      \Kerzab{} (after Kerzah Irgel, younger son of Cordos and Silqua).
    \item 
      \Siljin{} (after \hs{Silqua}, Cordos' wife). 
  \end{enumerate}
  
  
  
  
  
  
  
  \begin{comment}
  \subparagraph{months}
  \end{comment}
  \gitem{months}
  \index{months}
  \target{months}
  \index{\Atzirah!month}
  \index{\Feazirah!month}
  \index{\Keshirah!month}
  \index{\Razilah!month}
  \index{\Barion!month}
  \index{\Hapheron!month}
  \index{\Izion!month}
  \index{\Teshiron!month}
  \index{\Cushed!month}
  \index{\Hoshied!month}
  \index{\Thimared!month}
  \index{\Yemared!month}
  \index{\Gamishiel!month}
  \index{\Ishiel!month}
  \index{\Omariel!month}
  \index{\Yeziel!month}
  
  The \ImperialCalendar{} has sixteen months, dedicated to the sixteen \Sephiroth{}.
  The four months of spring are \Atzirah{}, \Razilah, \Keshirah{} and \Feazirah{}. 
  The summer months are \Barion{}, \Teshiron, \Izion{} and \Hapheron. 
  The autumn months are \Thimared, \Yemared, \Cushed{} and \Hoshied. 
  The winter months are \Omariel, \Yeziel, \Ishiel{} and \Gamishiel. 
  
  Each month is 24 days long, split into three weeks of eight days each (beginning with \Corjin{} and ending with \Siljin). 
  The exception is \Gamishiel{}, the last month, which is only 20 days long. %(\Gamishiel{} is the \Sephirah{} of Sacrifice.) 
  \also{\Sephiroth}
\end{subgloss}








\begin{comment}
\paragraph{\Wylde}
\end{comment}
\gitemthe{\Wylde}
\target{Wylde}
\target{Wild}
The uncharted wilderness between cities and villages, inhabited by dangerous beasts and monsters. 
Humanoids dwelling in the wild are labelled as savages and barbarians. 

\index{\wildfog}
Areas of \Wylde{} are sometimes shrouded in \wildfog, a fog-like substance that obscures vision. 
(\Wildfog{} is not regular fog. 
It is not made of water and may be found even in deserts.) 



\end{gloss}



