
\part{Denizens of \Miith}
\chapter{The \Ophidian Peoples}
The \ophidian{} races include all the creatures descended from the old \ophidian{} people. 
The most powerful of these races are the \dragons, but the most widespread are the \scathae. 

This section describes the \ophidian{} races as well as political groups among them. 















\section{\Firstgendragons}
\target{Elder Dragons}
The \firstgendragons{} are the founders of the \draconian{} people. 
They include \hr{Tiamat}{\Tiamat} and \hr{Apep-Nesthra}{\ApepNesthra}. 









\subsection{History}
\subsubsection{Worshipped as dead gods}
\target{Elder Dragons worshipped}
The \firstgendragons{} \hr{Elder Dragons die}{died at the end of the \firstbanewar}. 
But the heroic role they played in the war overshadowed their cruel tyranny. 
Today their corpses are \hr{Elder Dragons worshipped}{worshipped as dead gods}, close to the \xss{} in status. 

Kind of like how RL religions treat their \quo{prophets} with a veneration second only to that afforded the gods. 

\lyricsauthorbookpage{Graham McNeill}{False Gods}{313}{
  His father had become a carrion god who neither felt his subjects' pain nor cared for their fate. 
}

\lyricsauthorbookpage{Graham McNeill}{False Gods}{273}{
  I am Horus, forged of the Oldest Gods,\\
  I am he who gave way to Khaos.\\
  I am that great destroyer of all.\\
  I am he who did what seemed good to him,\\
  and set doom in the palace of my will.\\
  Mine is the fate of those who move along\\
  the serpentine path.
}

\Tiamat{} is invoked in Chaos magic. 
She was the one who made the original pacts with the \xss, so she must be invoked when one seeks aid from the \xss. 

\lyricsbs{Monolith Deathcult}{Den Ensomme Nordens Dronning}{
  Sleep, oh Majesty, surrounded by massive darkness. \\
  Her skin torn apart by sub-zero claws. \\
  Buried deep in thy ice dungeon. \\
  Sleep, oh Majesty, Lonely Queen of the North.
}















\section{\Cregorr}
\target{Cregorr}
\index{\cregorr}
A \saurian{} humanoid race. 









\subsection{Physique}
A \cregorr{} looks like a medium-sized theropod (\latinname{Ceratosaurus} or \latinname{Megaraptor} or the like). 

In a sense, the \cregorrs{} are more sophisticated than \scathae, \nephilim{} and \humans{} because they were designed by the \ophidians{}  with all the superior science they possessed when they were at the top of their power. 
Among other things, \cregorrs{} can regenerate lost limbs. 








\subsection{Biology}
How long does a \cregorr{} live? 

\begin{itemize}
  \item 
    They might live long because of their regenerative abilities and the way they were \quo{\hr{Origin of Cregorr}{intelligently designed}}. 
  
  \item 
    But they might also be short-lived because they live fast and \quo{burn out} quick. 
    After all, when you have a race of warrior servants, it is not important for them to have long lives. 
    It is more important that they mature and breed quickly. 
\end{itemize}









\subsection{History}
The \cregorrs{} were \hr{Origin of Cregorr}{created by the \ophidians{} during the \firstbanewar}. 

After the \firstbanewar{} they escaped to \hr{Cregorr did not dominate}{live in the \wylde{} as barbarians}. 

Later they were \hr{Cregorr came to serve Dragons}{\quo{tamed} by the \dzraicchenosses{} and made to serve them}. 









\subsection{Politics}










\subsection{Psychology}





\subsubsection{Never Iquinian}
The \cregorrs{} are never \hs{Iquinian}. 
Their race is too old and bestial, their minds too simple and guile-less. 
Iquin, based on lies and asceticism and self-denial as it is, does not work on them. 
They do not understand it and cannot be made to believe in it. 

So in the Iquinian world, \cregorrs{} are mistrusted and hated. 
So they only live in the \wylde{} as barbarians. 





\subsubsection{Noble savages}
The \cregorrs{} have \xsic{} blood, so they have a primal fury inside them, \hr{Scatha fury}{just like the \scathae{} do}. 
But the \cregorrs{} are more liable to openly display their ferocity than are the \scathae. 

They are savage and barbaric, but with a certain \naivete{} and honesty. 
Their minds are not made to understand lies and intrigue, so they are honest and gullible. 
They are noble savages. 





\subsubsection{Religion}
\target{Cregorr worship XS}
Many \cregorrs{} worshipped the \xss, or the memory of them. 
They did not understand magic, but they understood their instincts, and somewhere inside them they had a craving towards some mighty beings which they felt must exist. 
They would seem to glimpse these gods in their \hs{dreams} or daydreams, or see hints of them in their own bodies and all around them. 
These beings were the \xss. 

The \cregorrs{} had simple minds, so they accepted their instincts at face value rather than try to cover them up in lies. 
So they worshipped these gods whom they saw in their dreams. 

They also believed that they themselves were descendants of these gods, in some distant past. 
They sort of felt the presence of their gods in their bodies. 
And this belief was true, in a sense. 

They understood little to nothing of sorcery and summoning, so they were rarely if ever able to actually commune with the \xss{} or cast any actual magic. 
But the veneration of the memory of the \xss{} triggered something instinctive inside them, which helped them access the well of Chaotic \xsic{} fury and power inside them. 
So their religious rituals (dances and orgies and whatnot) made them stronger. 

Their religion \emph{worked} for them, so they knew it must be real. 
So they kept on practicing it. 

One other factor worked in their favour: 
The simple \cregorr{} minds were almost unaffected by the \hr{Ophidians lose telepathy}{mind-rotting radiation that afflicted their \ophidian{} masters}. 















\section{\Dragon}
\target{Dragon}
\target{Dragons}
\index{\dragon}









\subsection{Name}
In the \Miith{} world, the word \quo{\dragon} is etymologically descended from \emph{\Draecchonosh{}} (plural \emph{\Draecchonosh{}}, which is a \draconic{} word of power with connotations of strength and ferocity. 









\subsection{Biology}
\index{technology!bio-technology}
\Dragons{} are descended from \naga{} lords who transformed themselves using bio-technology and magic, as described in section \ref{Origin of Dragons}. 

\Dragons{} reproduce by internal fertilization and lay eggs. They are very infertile. A mating only rarely causes impregnation. A female lays one to three eggs at a time (rarely four or five), but even under good conditions, only half of these eggs survive to hatch, and many of the young die early on. An average female can lay 10-15 eggs in her lifetime if she is sexually active thoughout her fertile life (of these, 2-4 young can be expected to survive). The female carries the eggs for slightly less than a year, and after she lays them they take about one and a half year to hatch. If a female is impregnated, she will know it a few (3-5) days after the mating (but even at this stage, miscarriages do occur).  
Psychologically \dragons{} have few or no sexual taboos, mating with whomever they choose. 

A \dragon{} is sexually mature at an age of about 200 years. 
\Dragons{} are immortal and can live forever until slain. 

\target{Dragons have three hearts}
\Dragons{} have three hearts. 





\subsubsection{Demographics}
In the entire history of \Miith, there has existed no more than 100 \dragons put together.

At the beginning of the \thirdbanewar, there were around 20 \dragons alive, plus up to 10 more \hr{Aloof Dragons}{aloof or \quo{dead but dreaming} ones}. 





\subsubsection{Diet}
\target{Draconic diet}
\Dragons{} are natural predators, used to eat anything that moves. But their eating habits differ from those of the \resphain{} (see section \ref{Resphan diet}). 

\Resphain{} are cannibals, eating \resphan{} meat alongside that of all other intelligent creatures they can catch. In contrast, \dragons, despite their chaotic nature and their tendency to war amongst themselves, have a universal taboo against eating the flesh of other \dragons{}, \ophidians{} or \rachyth. A cannibal \dragon{} would be seen as an abomination by his fellows, and would be hunted down and slain. 

Where \resphain{} love to eat enslaved creatures who willingly submit and let themselves eat, \dragons{} live for the thrill of the hunt. They want their prey to flee and fear for its life. When they catch their prey, they want to be able to taste its fear, pain and anger, for such feelings are manifestations of life itself, and \dragons{} eat life, whereas \resphain, in a sense, eat death. 





\subsubsection{Living machines}
\target{Dragons are living machines}
In a sense, the \dragons were \hs{living machines}, designed by \Sethicus and other \ophidian scientist-sorcerers to be stronger and more resilient than any machine of metal. 





\subsubsection{Reproduction}
\target{No new Dragons were born}
\target{No new Dragons are born}
\Dragons were a species unto themselves. 
They could interbreed only with other \dragons, not \quiljaaran or \ophidians.
(But what about \Iurzmacul, father of \Ishnaruchaefir?)

\Dragons did not just give birth naturally like other creatures (even \resphain) did. 
For a pair of \dragons, to create an offspring was a long, very arduous and very arcane process. 
They must invoke the \xss and draw much power and life-energy from \KhothSell and from the Heart of \Miith, and deliberately and permanently invest much of their personal power in the child. 
Then the female would get pregnant and, later, lay an egg. 

In the age of the Shroud, this process was fraught with danger. 
The Heart was difficult to reach, so it took a lot of work and wasted energy to reach it and coax any new life out of it. 
It took more energy than usual, and the \draconic parents-to-be risked permanent bodily harm.
On top of that, it was uncertain. 
The egg might easily die without hatching, or the hatchling might be sickly and die young.
Most \dragons decided that the permanent investiture of personal power was not worth the risk, so they simply stopped having children.
After all, they were immortal, so they did not \emph{need} to replenish their race. 

The \dragons{} suffered greatly under the \hr{Heart weakened}{weakening of the Heart}. 
This is because \draconian{} souls were so powerful that it took a fucking lot of Heart energy to create one. 

Since the inception of the Shroud and the weakening of the Heart, almost no new \dragons were born.
Most \dragons alive in the \thirdbanewar were exceedingly old, being adult already when the first \resphain came to \Miith. 
In fact, \dragons were even longer-lived and slower-breeding than the \ophidians from which they were descended. 

But the \hr{Aloof Dragons}{aloof \dragons in \Machai} had \hr{Aloof Dragons do not care that no new Dragons are born}{stopped caring}.

It was sometimes said that the \dragons were a dying race.
But this was actually never true. 
Their number remained almost constant from the end of the \secondbanewar to the \thirdbanewar.
Remember, it took an assload of \resphain to take down just a single \dragon.






\subsubsection{Smelling the air}
\Dragons smell the air with their tongues just like snakes do. 
Most \ophidian-descended races do (but not \scathae or \cregorrs). 









\subsection{Physique}
A \ps{\dragon}{} natural form was that of a large reptillian quadruped with a long neck and tail. 
A \dragon{} walked on straight legs like a mammal, not with the legs spread outward like a lizard or crocodile. 
The tail was prehensile. 
A ridge of small spikes ran down the spine, from the head to the end of the tail. 

The forelegs and hindlegs were the same length. 
The hindclaws were larger and more bestial, whereas the foreclaws were more fine and dextrous, almost as fine as \human or \scatha hands. 

\Dragons{} have long, sharp teeth for biting and cutting, but also some flat teeth (at the back of the mouth) for chewing. 
They usually have a number of horns on the head, usually pointing backward. 
The shape and number of these horns varies a lot between individuals. 

\Dragons{} come in all \colours. 

\Dragons{} have good regenerative abilities. They heal wounds quickly even in combat and can heal wounds most creatures can't. They can't regrow limbs naturally, but with the aid of spells, a \dragon{} can reattach severed limbs or even regrow new ones. 





\subsubsection{Size}
\target{Dragon size}
A \dragon's length was approximately one quarter body, one quarter neck and one half tail.
Wingspan was about equal to total length.
Most \dragons were 12-20 metres long. 
The greatest \dragons could reach as much as 30 metres. 

A 25 metre \dragon weighed about as much as an \latinname{Allosaurus}. 





\subsubsection{Appearance}
\target{Draconic appearance}
With their immortality and Chaos-born power, \dragons{} radiate an impression of ancient, alien might. Their \ophidian{} eyes are cold, baleful and unfathomable. 

\lyricsbs{Steven Erikson}{Reaper's Gale}{
  \ldots{} we looked to the east, and there saw, rising vast and innumerable on the cloud-bound horizon, \dragons. Too large to comprehend, their heads above the Sun, their folded wings reaching down to cast a shadow that could swallow all of Drene. This was too much, too frightening [\ldots{}] for their dark eyes were upon us, an alien regard that drained the blood from our veins, the very iron from our swords and spears.
}

\citebandsong{KarlSanders:SaurianMeditation}{Karl Sanders}{
  Dreaming Through the Eyes of Serpents
}{
  After many trips with my son to the zoo, I noticed that crocodiles, monitors, snakes\dash{}and pretty much all the Reptiles\dash{}have this way of sometimes staying completely motionless when they want to with this deep, cold, timelessly penetrating glare. 
  It struck me as something like a trance state. It caused me to wonder what in the world would a snake meditate upon\ldots{}
}

\citeauthorbook[p.214]{HenryJVesterIII:TheResurrectionofKzadoolRa}{Henry J. Vester III}{
  The Resurrection of Kzadool-Ra
}{
  The being prokected an aura of incalculable age and wisdom, and of powers gained on worlds long lost in space and time.
}

\lyricsbible{Job 41:15--26}{
  [His] scales [are his] pride, shut up together [as with] a close seal. \\
  One is so near to another, that no air can come between them. \\
  They are joined one to another, they stick together, that they cannot be sundered. \\
  By his neesings a light doth shine, and his eyes [are] like the eyelids of the morning. \\
  Out of his mouth go burning lamps, [and] sparks of fire leap out. \\
  Out of his nostrils goeth smoke, as [out] of a seething pot or caldron. \\
  His breath kindleth coals, and a flame goeth out of his mouth. \\
  In his neck remaineth strength, and sorrow is turned into joy before him. \\
  The flakes of his flesh are joined together: they are firm in themselves; they cannot be moved. \\
  His heart is as firm as a stone; yea, as hard as a piece of the nether [millstone]. \\
  When he raiseth up himself, the mighty are afraid: by reason of breakings they purify themselves. \\
  The sword of him that layeth at him cannot hold: the spear, the dart, nor the habergeon. 
}






\subsubsection{Metaphysical nature}
\target{Dragons radiate life}
The \dragons{} are a force of Chaos: Destruction and creation alike. They are associated with light, fire, storm and lightning. They radiate life, energy and passion, but also fury and destructiveness. Like \hs{Nature}. 

This is in contrast to the \banes{} and \resphain, who are \hr{Bane parasitism}{dark and parasitic}. 





\subsubsection{Ward runes}
\index{ward rune}
When fighting in their true forms, \dragons{} do not wear \armour. 
That would not be economically feasible or practical for creatures of their size. 
But they do often wear \hs{ward runes}. 





\subsubsection{Weapons}
\target{Skekrathuin}
\index{\skekrathuin}
In battle, \dragons{} sometimes wield \skekrathuins{}\dash sword-like blades strapped to one's forearm. 
(Their hands are not made for holding weapons.)

\Dragons{} use their long, strong, prehensile tail a lot in combat. 
The tail might not be the \ps{\dragon} most powerful weapon, but it is certainly the swiftest and most dextrous. 

\target{Zrekklakh}
\index{\zrekklakh}
They often wear blades on their tails, called \zrekklakh{} (akin to \hr{Skekrathuin}{\skekrathuin}). 

They rarely wield firearms or other ranged weapons. 
They prefer to engage in \melee{} or use magic. 









\subsection{Psychology}
\Dragons{} are creatures of strong passions. They have the same emotions as \humans{} (love, hate, lust, pride\ldots{}), but on a larger scale. They are very individualistic creatures and, as a rule, consider their own needs and desires first and their fellows second.
\Dragons{} are often ruthless and arrogant, seeing humanoids as inferior, unworthy savages (though, of course, some \dragons{} are gentle and kind). 

\Dragons{} are proud and aggressive. They instinctively seek to dominate other creatures and do not welcome being given orders. This is the main reason why the \draconic{} kingdoms of Nom and Irokas have always been plagued by bloody wars (even more so than humanoid kingdoms): \Dragons{} are not genetically disposed to follow a King, so they tend to splinter and rebel. 

The \dragons{} have something of the \xs-born chaotic savagery in them, but they also have the cold, calculating \ophidian{} patience. 
Where \resphain{} and many mortals have a tendency to act cool and calm on the surface to hide the rage inside, a \dragon{} will often wear anger, aggression and physical violence as an outward persona, with the cold rational mind lurking beneath, observing and analyzing. 

\Dragons{} are not necessarily extremely evil just because they worship \xss. 
They just have a morality that is inhuman, superhuman and at times\ldots{} evil. 
One reason for this alien morality is their immortality and supreme power. 
It tends to erode your respect and regard for the lives of smaller, weaker, shorter-lived beings.




\subsubsection{Forgetfulness}
\target{Dragons have forgotten}
In the \hr{Ophidian golden age}{Golden Age of the \ophidian{} civilization},
the \dragons{} were even greater than they are today. 
But the \ophidian{} civilization \hr{Fall of the Ophidians}{fell}, and the surviving \ophidians{} fell into a long dark age. 
They forgot much of what they once knew, remembering only fragments of their tremendous science and magic.  

Many of the \daemons{} that once served as the \psp{\dragons}{} slaves have turned on them, and the \dragons{} have forgotten then spells to command them. 
Compare to the Azghouls of the RPG \emph{Kult}. 

All of \hr{Secherdamon's research}{\Secherdamon's research} is not just for discovering new things, but very much also for rediscovering the knowledge they once possessed. 

They wanted their golden age back.

\citeauthorbook[p.141]{RobertEHoward:TheCurseoftheGoldenSkull}{Robert E. Howard}{%
  The Curse of the Golden Skull%
}{
  Rotath's weird inhuman eyes smoldered with a terrible cold fire.
  A pageant of glory and splendor passed before his mind's eye.
  The acclaim of worshippers, the roar of silver trumpets, the whispering shadows of mighty and mystic temples where great wings swept unseen\dash then the intrigues, the onslaught of the invaders\dash death!
}





\subsubsection{Madness}
Even \Sethicus and \Tiamat{} feared \hr{Sethicus maps the way to Machai}{the cosmic truths that they discovered}. 
And even in the age of the \thirdbanewar, the \dragons{} knew that they balanced on the edge of an abyss of \hr{Madness}{madness}. 

\lyricsbs{Limbonic Art}{The Yawning Abyss of Madness}{
  The yawning abyss of madness.\\
  A cryptic slaughter by hate.\\
  Darkness is the only survivor\\
  as evil dominion terminates.\\
  The yawning abyss of madness.
}

\lyricsbalsagoth{The Ghosts of Angkor Wat}{
  I have concluded that these
  perceived parallels and their possible significance carry me ever closer to
  the centre of this great global web, the strands of which I have been
  traversing in my long quest for enlightenment, and yet I now fear that the
  owner of this web has surely felt the tremblings I have caused along its
  delicate filaments, and may well feel compelled to investigate the
  disturbance\ldots{}
}

Some of them, such as \Ishnaruchaefir, met this fear with a \quo{Devil-may-care} daredevil attitude. 

The \psp{\dragons}{} use of \hr{XS}{\xsic} magic warps their minds and makes them more volatile, chaotic and emotional. 
They are more resistant to this than mortals because their minds are stronger, but on the other hand, the \dragons{} channel far more such magic than mortals do. 
Over the millennia this seeps deep into their minds and shapes their personalities. 
They live and breathe the \xs{} essence day and night. 





\subsubsection{Social intelligence}
\target{Dragons are not social}
\Dragons{} are fundamentally \emph{not} social creatures. 
They are fiercely independent and self-sufficient. 

This is probably the \psp{\dragons}{} greatest weakness against the \resphain, who because of their \hr{Resphain are social}{superior social skills} are better organized, whereas the \dragons{} tend to war and squabble amongst themselves. 
True, the \resphain{} war amongst themselves, too, but that is a direct consequence of the \hr{Curse}{\draconic{} blood in their veins}. 

It can also play to the \psp{\dragons}{} advantage, though. 
Unlike \resphain, \dragons{} are not vulnerable to social pressure and cannot be easily manipulated through taunts. 
A sneaky \dragon{} will understand the effectiveness of taunts against \resphain{} and use it against them. 

Because of their non-social nature, \dragonlords{} often keep \scathaese{} or \hr{QJ}{\quiljaaran} advisors and assistants to help them in social matters. 
\Secherdamon{} has \LocarPsyrex, and \Ishnaruchaefir{} has \Criseis. 

\target{Draconic sincerity}
\Dragons{} are liable to swing back and forth between brutally evil and genuinely compassionate and caring. 
They have strong emotions and tend to display them sincerely. 
Unlike the \resphain{}, who, being highly social creatures, are \hr{Resphan hypocrisy}{scheming and hypocritical}. 










\subsection{Skills and powers}





\subsubsection{Blood grants immortality}
\target{Dragon blood gives immortality}
It was known (in myth at least) that \draconian blood conveyed immortality.
Everyone knew that tidbit.

It was not necessarily the nice kind of immortality, though. 
It could be \hr{Psyrex's undeath}{undeath like \Psyrex's}.

The person who drank \draconian blood became addicted to it and must keep drinking it if he wanted to stay immortal. 

\Dragon blood worked best for \scathae, because \scathae were \hr{Scathae have potential for greatness}{designed with a potential for greatness in mind} (as \hr{Ortaican potential for greatness}{the \Ortaicans believed}). 

For \humans and others, drinking \draconian blood was very dangerous.
It had to be prepared with lots of special spells in order to be just drinkable.
Deriving any power from it required lots of more spells. 
An ordinary \human who drank \draconian blood would either die in horrible agony or mutate into a monster (and become mindless or raving mad in the process). 






\subsubsection{Dark knowledge}
\target{Dragons have dark knowledge}
\hr{Resphain and forbidden books}{Dark though the \pps{\resphain} books might be}, {the \dragons wielded even darker knowledge}. 

They were a truly ancient race.
The blood of the \xss was in their veins and they saw deep into the nature of the Cosmos.

Even \Criseis \hr{Criseis fears Dragons}{shuddered to think of it}. 







\subsubsection{Immortality}
\target{Draconic immortality}
\Dragons{} were \hs{True Immortals}. 
They were the \emph{first} True Immortals. 
They learned immortality though a pact with \hr{Khoth-Sell}{\KhothSell}, whom they worshipped as their goddess of death and immortality.

\hr{Sethican philosophy}{\Sethican philosophy} saw the \Draconic immortality as a great step forward for the spiritual development of the \ophidian race. 

\Draconian immortality was powerful. 
A \dragon killed by normal means would retain quite some consciousness and even a minimum of sorcerous power. 
And even after its body was permanently destroyed and its magical power broken or eaten, the \dragon's consciousness might possibly still live on. 
It might have little power (at least in \Miith and in familiar Realms), but it could retain its memories and wisdom and would be free to journey to other worlds and build for itself a new life. 
This was no guarantee, however. 
It depended on the spiritual mastery of the individual \dragon (in accordance with \hr{Sethican philosophy}{\Sethican philosophy}). 

For example, \Sethicus \hr{Sethicus becomes a god}{lived on and became a disembodied god} after his death. 
\Nexagglachel \hr{Nexagglachel lives on in Satharioth}{lived on inside the \satharioth}. 

\Draconic immortality was different from (and superior to) \hr{Ophidian immortality}{that of the \ophidians}, which was based on the shedding of skin. 

See also the section on \hr{Kinds of immortality}{different kinds of immortality}. 





\subsubsection{Power compared to \resphain}
\target{Dragons vs Resphain in power}
In combat a \dragon is worth more than 10 \resphain. 
The average \dragon can take on at least 20 purebloods. 
But if the \resphain are \hr{Umbra power}{mounted on \umbrae} then it only takes an average of 6 of them.










\subsection{Culture}





\subsubsection{Aloof Elder \Dragons}
\target{Aloof Dragons}
Some \dragons have fled \Miith and dwell in \Machai. 
This is their \quo{second homeland}, because they draw so much of their power and their being from there.
So some are content to dwell in \Machai and concern themselves with \Machaic affairs and remain aloof from \Miithian affairs. 
Some had remained in their tombs, dead but dreaming. 

But the aloof, retired \dragons still watched \Miith from the shadows.
They were still able to pay attention to what happened in the world around them to a degree, through their \daemons and \homunculi.


\citebandsong{Nile:AnnihilationoftheWicked}{Nile}{
  Von Unausspechlichen Kulten
}{
  I Hath Dreamed Black and Grim, Desolate Visions\\
  of the Pre-Human Serpent Folk \\
  and Communed with Long-dead Reptiles.\\
  Silently Watching Through the Ages in Cold, Curious Apathy.\\
  The Unending Sorrows and Suffering of an Abysmal \Human{}kind.
}

\target{Aloof Dragons do not care that no new Dragons are born}
Some of them knew that \hr{No new Dragons were born}{no new \dragons were born}.
But they had stopped caring. 
They were powerful and immortal and had the luxury to wait a few thousands or even tens of thousands of years between procreating. 
Many of them were over 20,000 years old and saw the war with the \resphain as a temporary nuisance. 

\Ishnaruchaefir and \Secherdamon were similarly old and would likely have thought the same, but the war came horribly close to them when some of their close family members (Nexagglachel and \hr{Ishnaruchaefir's sons die}{\Ishnaruchaefir's three sons}) were destroyed by the \resphain.

Later, in the \thirdbanewar, \Secherdamon or \Ishnaruchaefir had to convince the aloof ones that the war was more than a nuisance and that its conclusion would have great repercussions for all \dragons, even the ones who had fled to \Machai.

Late in the \thirdbanewar, Secherdamon or \Ishnaruchaefir finally \hr{Ishnaruchaefir leads Dragons to war in TBW}{manages to rouse} some of these aloof, retired \dragons.
Not all, but enough.

\target{Encountering Dragons in dreams}
It was possible to encounter the sleeping \dragons when \hs{dreaming}.
Like Cthulhu, they reached out to touch the minds of mortals.
And they could use mind control.





\subsubsection{Ancient homeland}
There exists \hr{Fallen Dragonland}{a place that once used to be the capitol and homeland of the proud \draconic{} empire, but now lies in ruins}.





\subsubsection{Architecture}
\target{Draconic architecture}
\index{architecture!\draconic}
The \dragons{} built huge, cyclopean edifices. 
Their buildings were often \quo{organic} and \quo{bestial} in shape, reminiscent of the carcasses of gargantuan animals and monsters. 
The bodies of the buildings were invariably vast, even bloated. 

They would decorate buildings with thin towers resembling spines, horns and antlers. 
These towers often served no useful purpose; they built them just because they could. 

An example is \hr{Nith'dornazsh}{\Nithdornazsh}. 

Contrast with \hr{QJ architecture}{\quiljaaran{} architecture} and \hr{Resphan architecture}{\resphan{} architecture}. 

\Draconian citadels were enormous. 
They sprouted huge, bulbous, misshapen spires.
They catered to the \xss and to the principles of Chaos. 
They were built in accordance with Chaotic occult geometry in order to more fully utilize the power of the \xss. 

The ancient \draconic{} buildings are built by \daemons{} under the \psp{\dragons} command. 

That is, if the buildings are not alive. The living buildings \emph{are} the \daemons.

See also the sections on \hr{Nyx}{\Nyx} and \hs{dark ancient cities}. 





\subsubsection{Factions}
Remember to have fractions among the \dragons{}, and Sentinels in general. 
Their history must be as diverse and as bloody as that of the \resphain!

Speaking of which\ldots{} how much of the Sentinel organization is controlled by \dragons? There are also \rachyth, remember. And the \Baelzerach, they might be Sentinel-allied, too. And there are groups within the Sentinels who do not work for \dragons{} at all, but worship their own gods, and are just in because they want to get rid of the \banes{} and \resphain{} (this is a strong argument for any \scatha{} or \rachyth). 

And there are cults who seek to resurrect the \xzaishanns. Perhaps there are people with \xzaishannic{} blood. People, or descendants of people, who partook in \Tiamat-tachi's original ritual of summoning the \xzaishannic{} power. These people were meant as sacrifices, but a few of them survived and were imbued with \xzaishannic{} power. They fled and hid, but their descendats live on, as does their hatred for the \draecchonosh. 

And then there are the \ophidians{} and the \nagae{}, who may sometimes work with the Sentinels. 









\subsubsection{Etiquette}
\Dragons{} respect power. 
And they respect the confidence and assertiveness that stems from power. 
To get a \ps{\dragon}{} respect it is perfectly fine to brag and show off. 
You just have to have the inner game to back it up. 

Humility is not a virtue. 
It is seen as a sign of cowardice and weakness, and people who display it are looked down upon as insignificant prey and pawns. 

\hr{Dragon violence}{Physical violence} is a common part of \Draconic{} behaviour. 





\subsubsection{Language}
\Dragons speak \hr{Draconic language}{\Draconic}. 

\target{True Draconic signifies emotion}
When \dragons felt the need to express great emotion, they would often switch to \TrueDraconic. 





\subsubsection{Music}
\target{Draconic music}
\index{music!\draconic}
\Draconic{} music is extremely complex and strange (both rhythmically, melodically and harmonically). 
You practically need to have a big, complex, \draconic{} brain in order to understand it. 
To mortal ears it often sounds like cacaphonous, chaotic noise. 





\subsubsection{Name}
\target{Draconic names}
All \dragons{} have an \quo{egg-name}. 
Many also have a taken name. 





\subsubsection{Non-cannibalism}
\target{Dragons do not eat Dragons}
\index{cannibalism!\Dragons}
The \dragons{} have one rock-solid and enforced tradition: 
You do not eat the flesh nor souls of other \dragons. 
You can fight and hurt them all you want, and killing the body is also OK. 
Even destroying souls is forgivable if for a good reason. 
But eating a fellow \dragon{} is an unforgivable atrocity. 

It is, however, acceptable to sacrifice a \ps{\dragon} soul to the \xss. 
\TyarithXserasshana{} \hr{Tiamat kills Hesod-Nerga}{did it to \HesodNerga}. 
\IrocasSecherdamon{} \hr{Secherdamon sacrifices Dragons}{did it to some of his rivals} during his rise to power. 









\subsubsection{Religion}
\target{Dragons worship dead gods}
The \dragons{} worship the corpses of dead gods. 
These are both the \hr{Dead XS}{dead \xss} and the \firstgendragons{} (\Tiamat-tachi). 
According to myths, these corpses will one day awaken to new life. 

When in deep emotion, \dragons would swear by the names of the \xss and the dead gods, including \Sethicus and \Tiamat. 

\ps{\Xserasshana} corpse now lies entombed in \Dathka{} in the ancient \hr{Dragonland}{\dragonland}. 

The dead \dragons lived on in myth\dash{}and still did in the Age of the Shroud:

\citebandsong{Nile:Ithyphallic}{Nile}{
  What Can Be Safely Written
}{
  On the walls of lost cities\\
  And in the carvings of madmen\\
  Who have glimpsed him in their dreams\\
  Is his image delineated\\
  Within a tomb protected by great seals he lies in death\\
  Under the weight of the dark waters of the deep\\
  Yet he dreams still, and in his dreams continues to rule this world\\
  For his thoughts master the wills of lesser creatures
}
They also worship the \xss. 

\citebandsong{Nile:AnnihilationoftheWicked}{Nile}{
  Chapter of Obeisance Before Giving Breath to the Inert One in the Presence of the Crescent-Shaped Horns
}{
  Khensu Neter Hetef,\\
  who Possesseth Absolute Dominion \\
  over the Evil Spirits that Infest the Earth and Sky.\\
  He of the Silence of the Moon.\\
  Giver of Oracles. He that Must Forever Wax and Wane.\\
  Thou Art in Union with Thoth,\\
  the Excellent Tehuti of Truth and Time.\\
  Keeper of the Lunar Cycle,\\
  whose Hands are Able,\\
  whose Tongue is Mighty in Speech.\\
  Author of the Works of Knowledge.\\
  Writer of the Ancient Wisdom.\\
  Master of the Words of Power. 

  I am He Who Calleth Down Curses \\
  and Commandeth the Elements unto Darkness.\\
  I Hath Uttered the Hidden Words.\\
  Even unto the Divine Words which Art Written in the Book of Thoth.
}





\subsubsection{Rulership}
The \dragons{} are a chaotic people. 
Any central organization, such as monarchy, is unnatural and unstable for them. 
\hr{Tiamat}{\Tiamat} stayed in power as long as she did by brute force and effort, and her rule was never as absolute as she would have liked or later generations have made it sound. 





\subsubsection{Technology}
\target{Dragon living technology}
\Draconic technology focused much on \hr{Living machines}{living machines} and bio-technology. 
They used \hr{Symbiotes}{symbiotes} in combat. 
Read those sections. 

The \dragons are supposed to have a \quo{living} theme.
As opposed to the \resphain, who, \hr{Resphan dead technology}{with their metal/glass/crystal technology}, have a more \quo{dead}, \quo{artificial} theme. 





\subsubsection{Violence}
\target{Dragon violence}
Physical violence is a common part of \Draconic{} behaviour and social interaction. 
\Dragons{} will snap and lash out at one another. 
This is a sign of friendship, not animosity. 
Just staring coldly at each other is a sign of disrespect or hate. 

In the old days, before the \secondbanewar, one would often see \dragons{} fighting fiercely, even to the death. 

Non-lethal but bloody fights are a social custom among friends.
It is even expected. 
(Compare to how, among \humans, men tend to call each other rough names and mock-fight as a sign of friendship.)

\Dragons{} are quick to kill enemies and un-friends, and ferociously. 
There are many personal feuds, where enemies will try to \quo{gank} one another at sight. 
(Compare to multiplayer games like \cite{VideoGame:WorldofWarcraft}.)















\section[Locul]{\Locul}
\target{Locul}
\index{\locul}
The \loculs{} were a race of \saurian{} humanoids.









\subsection{Physique}
\Loculs{} resembled \scathae{}, but smaller, smoother and less \armoured. 





\subsubsection{Weak}
\target{Locul weak}
\Loculs{} were not warriors.
They were weak and fearful. 
This made them \hr{Nephilim kill Locul}{easy prey for the marauding \nephilim}. 









\subsection{History}
The \loculs{} were created by the \ophidians{} as a servitor race.

After the \firstbanewar{} they continued to serve the \dragons{} and \quiljaaran. 

In the \hr{Aryothim kill QJ}{\aryoth-\quiljaaran{} wars} many \loculs{} were killed. 

Most of their \quiljaaran{} masters were killed, and the \loculs{} could not survive well on their own \hr{Locul slave mentality}{given their slave mentality}. 
Many more were killed by the \aryothim{} and \nephilim, who had an irrational hate of all reptillian races. 

Later the \dzraicchenosses{} took the \loculs{} and used them \hr{Origin of Scathae}{to breed the \scathae}. 
The surviving \loculs{} were displaced by the more aggressive \scathae{} who were now spreading all over \Miith. 

Eventually they died out, somewhere after the rise of the \dzraicchenosses{} and before the \secondbanewar. 





\subsubsection{\QuilJaaran{} miss them}
\target{QJ miss Locul}
After the \loculs{} had become extinct, many older \quiljaaran{} would mourn the loss. 
To the \nephilim{} who killed them, the \loculs{} were loathsome reptillian creeps.
But to the \quiljaaran{} they were gentle and lovable creatures, faithful and devoted. 
Much more lovable than the hard and fierce \scathae{} that replaced them. 

For millennia, some \quiljaaran{} elders wished to have the \loculs{} back. 
They were benevolent creatures that harmed no one. 
They were the children of a more enlightened age, according to the nostalgics. 









\subsection{Psychology}





\subsubsection{Friendly}
\target{Locul friendly}
The \loculs{} were friendly and trusting. 
Guile and aggression had been bred out of them. 
This made them \hr{Nephilim kill Locul}{easy prey for the marauding \nephilim}. 





\subsubsection{Slave mentality}
\target{Locul slave mentality}
The \loculs{} were bred as servitors. 
They were docile and had a slave mentality that made it difficult for them to act alone without a master. 















\section{\Nagae}
\target{Naga}
\target{Nagae}
\index{\naga{} (plural \nagae)}
\index{\nagalord{} (plural \nagalords{})}
\index{\naga{} (plural \nagae)!\nagalord{} (plural \nagalords{})}
\target{Vlekkesh'sala}
The \nagae{} are a race of sea-dwelling reptillian/ichthyic humanoids. 
They are long-lived (possibly immortal) and maintain their cities and kingdoms beneath the sea. They are the oldest humanoid species on \Miith{}. 

The more powerful of their race grow to huge size.
These \nagalords{} are called \nagalords{} in the language of Nag.









\subsection{Biology}
\Nagae{} live very long. A typical \naga{} has a lifespan of over 1000 years, and exceptional individuals may reach 2000 years or more. 





\subsubsection{Considered \demiscathae}
\target{Nagae considered Demiscathae}
Some thought of the \nagae as \hr{Demiscatha}{\demiscathae}.
Those who had actually seen and dealt with the \nagae knew better.
The \nagae were a far more ancient race than the \scathae. 
As old as the \dragons and older. 





\subsubsection{Habitat}
\Nagae preferred to live in the salt water of the oceans, but they sometimes came into rivers. 





\subsubsection{Immortality and shedding skin}
\target{Naga immortality}
\target{Nagae shed skin}
\Nagae were \hr{Lesser Immortality}{Lesser Immortals}
\Naga immortality worked similar to \hr{Ophidian immortality}{\ophidian immortality}

The \nagae shed their skin.
This was a part of their immortality. 

The Imetrians \hr{Imetric Naga skin clothes}{made clothes from shed \naga skin}. 





\subsubsection[Scatha/Naga hybrids]{\Scatha/\naga hybrids}
\target{Naga-Scatha hybrids}
\target{Scatha-Naga hybrids}
In some coastal/island communities there dwell \scathae{} that interbreed with \nagae{}. The \nagae{} are more horrible, more primal kin to the \scathae, and the \scathae{} view them with fear, horror, loathing and awe. 

At times, children are born/hatched who are atavistic and look like monstrous \nagae. 

Half-\nagae walked more hunched-over than normal \scathae.
They had short legs and long tails. 
Their heads were strangely narrow and flat (in the vertical direction). 
Their bodies were flexible, and when they walked they seemed to writhe and wiggle in a repulsively fluid manner.
A particularly loathsome trait is their prehensile, snaking tails (alien to the \scathae, whose tails are rigid).
They were horrible to look at for a normal \scatha.

Compare to the Shake people in \cite[pp.360--361]{StevenErikson:ReapersGale}, or the people of Innsmouth in \authorbook{H.P. Lovecraft}{The Shadow Over Innsmouth}, or the people of Imboca in the movie \movie{Dagon}. 









\subsection{Culture}






\subsubsection{Nag}
All \nagae{} in the waters near \Azmith{} belong to the kingdom of Nag and speak the language known as Nag. 

\Dragons{} can pronounce Nag flawlessly, and \scathae{} can learn it to some extent, but it is nearly unpronounceable to \humans{}. 
A few \nagae{} learn other tongues. 
A \naga{} encountered on the land is $20\%$ likely to know \CommonDraconic and $5\%$ likely to know Rissitic, Imetric or some ancient \scathaese{} tongue. 
A \naga{} is also $5\%$ likely to understand a bit of \Velcadian{} or Vaimon, but they can pronounce \human{} and Vaimon tongues only with great difficulty. 
(Those \nagae{} who do understand land-dweller tongues are usually the leaders and mages.) 






\subsubsection{Technology}
The \nagae relied mostly on biotechnology and sorcery (including biomancy/bio-magic) and not so much on metal and the like. 

\citeauthorbook[p.92--93 of 138]{KarlEdwardWagner:DarknessWeaves}{%
  Karl Edward Wagner%
}{%
  Darkness Weaves%
}{
  \ta{%
    In the eons before man walked the earth--when the sea was a vast, teeming wilderness of primitive life,
    its oceans far more immense than those of today--the race of creatures known to mankind as the
    Scylredi arose and flourished. Most of the continents we know today had not yet risen from the primeval
    sea, and only a few jungle-choked land masses stood out from the boundless seas of Elder Earth. The
    Scylredi lived beneath this ancient sea and created for themselves a civilization beyond man's wildest
    conception. Here in this very region they built their cities, for at that time all these islands lay upon the
    ocean floor.
    "They were a strange race, these creatures of awesome antiquity. Nothing on earth truly resembled
    them, even then. Were they some freak of evolution, a race from another world--or perhaps, like man,
    the result of some insane god's whimsy? Who can say at this distant age? The most ancient writings that I
    have studied are uncertain on so many points. But then, this earth has held many strange races about
    which mankind can only speculate, and all but a fragment of the secrets of prehuman history has been lost
    forever.
    
    "Whatever their origin, the Scylredi were as gods themselves. They had control of powers both natural
    and supernatural. They used the great beasts of the primordial sea for their own purposes, controlling
    fantastic monsters known to mankind only through legend. With their knowledge of the physical sciences,
    they built great submarine seacraft--unearthly engines in which they traveled the oceans and waged war
    with the other inhuman races of Elder Earth. That age was a far more violent world than the earth of our
    day, and there were many powerful forces the prehuman races must constantly contend against in the
    battle to survive. They were versed in the elder sorceries, as well--the secrets of the gulfs beyond our
    stars--and legend only hints at some of the hideous deeds that were committed by the Scylredi in their
    wars.
    
    "Magnificent fortresses they raised--huge basalt structures that surpassed human imagination. The ruins
    of these great castles can be seen today--on hillsides where they have crumbled for millennia, ever since
    the waters receded from these islands. This very fortress, Dan-Legeh, is their creation. For the Scylredi,
    it is only a minor citadel, and built after their race had declined. It was an age of giants, and the Scylredi
    commanded both sorcery and science in their constant battle for supremacy in that prehistoric age of
    chaos.
    
    But as the centuries passed, their power slipped from them. Perhaps it was the shrinking of the great
    seas, or the cooling of the earth that caused their decline. It is recorded that there was a long period of
    horrific warfare between the Scylredi and some other race of elder beings. The conflict was waged with
    weapons of unimaginable power. Many of their colossal basalt castles were blasted into fused rubble,
    their gigantic seacraft destroyed, their fearsome servants annihilated, and the greater part of the Scylredi
    were killed. Both races lay near to extinction upon the termination of that war, and the scattered survivors
    were left to mourn amidst the ruins of their vanished civilizations.}
}








\subsection{History}
The \nagae{} are related to the \hr{Ophidians}{\ophidians}. 
They were once \hr{Ophidians and Nagae were one species}{one species}. 








\subsection{Name}
Singular \emph{\naga{}}, plural \emph{\nagae{}}.%
\footnote{%
  The form \quo{\nagae} is not etymological justified, since \quo{\naga} is not Latin or Greek but from some Indian language, I believe. But this is the declination I use, because Latin grammar is cool.} 

The associated adjective is \emph{\naga{}}. 

\emph{Nag} one of a number of \naga{} nations. 
Its language is also \emph{Nag} and the adjective is \emph{Nagan}. 

Interesting tidbit: 
According to \DIBiggestSecret, the word \quo{Naga} means, in some Indian language, \quo{those who do not walk, but creep}. 

The \nagae were also called \quo{\ichthyans}. 





\subsubsection{Those who do not walk, but creep}
Interesting tidbit: 
According to \DIBiggestSecret, the word \quo{Naga} means, in some Indian language, \quo{those who do not walk, but creep}. 









\subsection{Physique and metaphysique}
A \naga{} is only vaguely humanoid, with an elongated snake- or eel-like body with a long tail and four limbs. 
They have what resembles a mix of fish and reptile characteristics. 

\Nagae{} are found in all kinds of \colours and patterns, but in Nag, various shades of green are most common. 

A typical \naga{} is 1.5 to 2 meters long and weighs 40-80 kg, but some are much larger, growing as long as 5 meters and weighing up to a ton. 
Their arms and legs are short. 
They can walk on land, but they are not very agile. 
In the water, on the other hand, they are very fast and \manoeuvrable, swimming with their legs and tail (rather like a seal). 

\Nagae{} can breathe both water and air, but they need to immerse themselves in water regularly. 
If they are kept out of the water for much more than 24 hours, they will dehydrate, weaken and die. 

\Nagae{} can regenerate lost limbs, but slowly. 
More slowly than \meccaran{} regeneration. 
An arm or leg takes about a year to regenerate. 

\index{technology!\naga}
\Nagae{} fight with weapons. 
Most of their weapons are daggers, spears and javelins. 
\Naga{} technology is low, so many of their weapons are primitive, made of stone or bone. 
The \nagae{} cannot forge iron or bronze, but some of their weapons are made from exotic metals unknown on the land. 
Occasionally, a \naga{} may be encountered wielding a weapon made by land-dwellers. 
(Of course, iron weapons will rust underwater, but \dragonsteel, \truesilver{} and certain enchanted weapons are immune to rust.) 

About \naga{} out of every twenty is a mage. 
\Naga{} magic is alien and very different from most land-dweller magic, but bears similary to \draconic{} magic. 





\subsubsection{Power over water and ice}
The \nagae{} wield great power over water and ice. They live near the poles in the summer and migrate to the equator in the winter. 

Some of the greatest \nagalords{} have frozen themselves into thrones of ice, where they lie asleep for thousands of years at a time. 





\subsubsection{A spear of ice}
Have a \naga{} who wields a spear of ice. 








\subsection{Psychology}







\subsection{Politics}
The \nagae{} work against the Cabal and Sentinels. They want to prevent either faction from controlling the Heart of \Miith{}. 

What is their end goal? Is it good or evil? 

But on the whole, they don't do much, \hr{Ophidians today}{like the \ophidians}. 

The \nagae{} use a \dweomer{} slightly different from that of the \dragons. 
Some fear that if the \banes{} conquer the \dragons, they will come after the \nagae{} and their \dweomer{} next. 
Others don't believe it, or believe that the \banes{} will never succeed. 
So they won't help in the war against them. 





\subsubsection{Coral reefs}
The \nagae{} know of the \hs{coral reefs} and their intelligence and power, and they fear them. 
Perhaps they worship them as gods. 





\subsubsection{Myths about them}
\target{Myths about Nagae}
There were myths about \quo{merfolk}: 
\Scatha-like creatures that dwelt in the sea and rivers. 
These were based on stories about sightings of \nagae. 

















\section{\Ophidians}
\target{Ophidian}
\target{Ophidians}
\index{\ophidian}
The \ophidians{} are an intelligent race of snake-like creatures native to the Beast Realm. 
They are the ancestors of \dragons, \quiljaaran{} and \nagae. 



The original \ophidians{} were intelligent snakes. They had no arms and legs but used telekinesis instead. They also had powerful \hr{Telepathy}{telepathic} abilities. Perhaps they have some kind of racial memory. Or perhaps they are simply immortal. 

The \ophidians{} are an ancient race, having existed for many millions of years. 
They have seen the rise and fall of many civilizations of lesser beings (most of which destroyed themselves out of folly). 

They fulfilled a role as guardians of \Miith{}, and sometimes rulers. Occasionally, \ophidians{} would enthrone themselves as lords of the \nephilim{} or other lesser creatures\dash at times with evil intent. They may be something like the Jaghut in the \emph{Malazan Book of the Fallen} books. 

\target{Ophidian power source}
The \ophidians{} wielded\dash and wield\dash powerful magic. 
This magic is not of Chaos, but born of the native, natural power of \Miith{} and her Heart. 
It is similar to the \Wylde{} power that the \hr{Druids}{druids} use. 
The \hr{Kezeradi Iquin}{old incarnation of \iquin} used by the \hr{Kezerad}{\Kezeradi} was a modified, idealized version of this \dweomer{}.

In times of great need, such as when facing the \xzaishann, the \ophidians{} could also ask for help from a pantheon of aloof, mysterious \hs{cosmic gods}. 

The \dragons, \quiljaaran{} and \nagae{} are all descended from the \ophidians.
Those who have retained the old-style \ophidian{} are now called \quo{\trueophidians}. 

\lyricsbalsagoth{
  Into the Silent Chambers of the Sapphirean Throne (Sagas from the Antediluvian Scrolls)
}{
  Winged dragon coiled in thrice,\\
  bane of flame in shadowed ice.
  Flooded by the bloated Moon,\\
  the ivory worm now sleeps entombed.
}

They are intrinsically bound to \Miith{}, its life and future. Their blood is Life itself. And their venom is Death. 

Have some philosophy about how one bodily fluid from the \ophidians{} gives life while another takes life away. 









\subsection{Biology}
The \ophidians{} developed psionics and telekinesis before they developed hands. 
They had relatively advanced magic and psionics while they were still at the Stone Age stage with respect to tools. 
They learned immortality when they were at the Iron Age stage. 





\subsubsection{Immortality}
\target{Ophidian immortality}
\Ophidians{} are immortal. They shed their skin periodically to renew their youth and rebirth themselves into new life. 
The shed skin of an \ophidian{} may have mystic power. 

\hr{Draconic immortality}{\Draconic{} immortality} is different. 

See also the section on \hr{Kinds of immortality}{different kinds of immortality}. 

Perhaps only the nobility were immortal. 

Perhaps their immortality is psionic and not \hr{sorcery}{sorcerous} in nature. 





\subsubsection{Metabolism control}
\target{Ophidians evolved}
\target{Ophidians half warm-blooded}
\index{metabolism control}
\index{warm-blooded}
\index{cold-blooded}
The \ophidians{} descended from reptiles who were making the transition from \quo{cold-blooded} to \quo{warm-blooded}. 
Crocodiles, lizards and the like remained cold-blooded, while synapsids, dinosaurs and pterosaurs evolved and become warm-blooded. 

Another group of animals were stranded midways. 
They developed the curious ability to switch back and forth, controlling their own metabolism to \quo{cool down} and become cold-blooded or warm-blooded as they wished. 
In a sense, this gave them the best of both worlds: 
The dynamic speed of warm-blooded creatures coupled with the patience of cold-blooded creatures. 

\citeauthorbook[p.80]{RobertTBakker:TheDinosaurHeresies}{%
  Robert T. Bakker%
}{%
  The Dinosaur Heresies%
}{
  The great serpents succeed by being something a warm-blooded mammal could never be\dash a hunter of infinite patience\ldots{}
}

The early \ophidians{} were crocodile-like in form. 

Despite their advantages, they were still not quite as effective as the more specialized true cold-bloods and true warm-bloods, so they remained a niche group. 
For many millions of years they evolved to fill special niches. 
Instead of huge size like some \saurians{} (or tiny size, like mammals), the proto-\ophidians{} were forced to take a different evolutionary route: 
They evolved intelligence and telepathic abilities. 

\target{torpor}
\index{torpor}
The \ophidians{} learned that if they made sure to be as inactive as possible, they could conserve their bodily energy, keep it in reserve and use it to power their brains and their psionics and sorcery. 
They could go into a state of torpor where their bodies would hibernate and operate at a very low metabolism, but their brains and minds could still work at full capacity. 
By utilizing their torpor strategically, the upper classes could hoard vast amounts of energy and redirect it for magical purposes. 

They developed relatively small bodies (little bigger than those of \humans). 
Since they moved slowly and relied much on stealth and their psionics, their legs atrophied and they ended up crawling on a snake tail. 
To compensate, they began breeding and taming various beasts for use as mounts, so they could move around without having to move too much. 





\subsubsection{Parentage}
The \ophidians{} descend from both the \voyagers{} and the \xss, and as such represent a powerful combination of the two forces. 
Perhaps this is connected to \hs{sexual mysticism}, with \xsic{} \Chaos{} as their \quo{mother} and the \voyagers{} as their \quo{father}. 
Anyway, this combination of powers is what makes the \ophidians{} the greatest and mightiest race on \Miith{}, able to dominate the planet for millions of years. 









\subsection{Name}
\target{Names for Ophidians}
The \ophidians{} are sometimes called Serpentines, Serpentine People, Serpent Folk or Serpent Men. 
These names \hr{Names for QJ}{are also used for the \quiljaaran}. 









\subsection{Culture}





\subsubsection{Architecture}
\target{Ophidian architecture}
The \ophidians{} were fond of monumental architecture and often built enormous Cyclopean buildings and cities.
They engaged in great luxury, just because they could. 









\subsection{History}
\subsubsection{Origin}
The \ophidians{} \hr{Ophidians evolved}{evolved from reptiles} who were making the transition from \quo{cold-blooded} to \quo{warm-blooded}. 

They \hr{Early Ophidian culture}{quickly developed intelligence and culture}. 

They \hr{Ophidian civilization}{developed a great and mighty civilization}. 





\subsubsection{Possibly descended from \xzaishanns}
\target{Ophidians related to XS}
Some, notably \hr{Sethicus}{\Sethicus}, theorized that the \ophidians were related to the \xss.
Perhaps even created by them, or at least evolved from creatures created by them.

\Sethicus believed they had been \hr{Sethicus believes Ruin Satha created Ophidians}{imparted hunger/will/motivation by \RuinSatha, and then physical life by \KhothSell}. 





\subsubsection{Rise of the \draecchonosh}
Some \ophidians{} craved more power. Their leader was \Tiamat. They \hr{Origin of Draecchonosh}{transformed themselves into \draecchonosh}. 

It came to a \hr{Draecchonosh war}{conflict between the \draecchonosh{} and the \trueophidians}.





\subsubsection{Relationship with \XzaiShanns}
\target{Ophidians and XS}
The \ophidians{} are descended from \voyager-spawned beings, but they also had some amount of \xsic{} heritage in their blood. 
That was part of the reason why they became intelligent and grew to dominate \Miith{}: 
They had not only the \voyager-bred docility but also the \xsic{} aggression and evolutionary volatility. 

The \ophidians{} knew about the \xss{} and feared and shunned them. 
They also knew about their own \xsic{} blood, and they were not happy. 
To them, the \xss{} were not only outer \daemons, but also inner \daemons{} which they struggled against and fled from. 
They fought their inner Chaos using logic, self-control, meditation, philosophy and knowledge. 

Compare to the conflict between Chaos and Darkness in \cite{StevenEriksonIanCameronEsslemont:MalazanBookoftheFallen}. 





\subsubsection{Relationship with other humanoids}
Maybe they kept \nephilim{} and \meccara{} as slaves or pets. 
They regarded them as clever animals that could learn to talk, but still just animals, nothing near their own level. 





\subsubsection{Various descendants}
\target{Ophidian descendants}
The \ophidians{} were the ancient civilization from which many others derived. 
The \dragons, \quiljaaran{} and \nagae{} are all descended from them.
There are also a few \quo{true} \ophidians{} left, who may be tens or even hundreds of thousands of years old. 





\subsubsection{Tried to rebuild}
\target{Noggyaleth plague Ophidians}
Throughout the ages, the \ophidians tried to rebuild their civilization, but always the \hr{Noggyal}{\noggyaleth} would ruin it.
Over the millennia the \ophidians lost many of their records to \noggyal attacks and other conflicts, and it was hard for them to replenish or replace them.
Their numbers were dwindling and there were few left who remembered the old \ophidian civilization. 

Mostly, the \ophidians remained underground in small groups.
There they worked on long-term plans and conducted research into science and sorcery. 
They still did at the time of the \thirdbanewar. 

Most \ophidians were affiliated with the Sentinels, but many had goals of their own as well.

The \ophidians learned the art of shapeshifting even before the Shroud came. 
They used it to infiltrate many things, including the \aryothim and possibly even the \resphain. 
This was a great asset for the Sentinels. 
An \ophidian was a much smaller \vertex than a \dragon, so it could more easily hide.
This would be impossible for a \dragon in the long run (even a master of disguise such as \Nzessuacrith). 

An \ophidian could masquerade as a \bezed. 
This was not widely advertised among the \resphain.
Few such spies existed, and much fewer had ever been caught. 
The \resphan leaders were afraid of the spies, but they were not willing to admit that they existed.
They did not want to cause a widespread panic (like what happened in Battlestar Galactica).





\subsubsection{\Saphyrae}
\hr{Saphyrae}{\Saphyrae} was the closest the \ophidians came to a new empire. 





\subsubsection{Shapeshifting and infiltration}
The \ophidians learned the art of shapeshifting even before the Shroud came. 
They used it to infiltrate many things, including the \aryothim and possibly even the \resphain. 
This was a great asset for the Sentinels. 
An \ophidian was a much smaller \vertex than a \dragon, so it could more easily hide.
This would be impossible for a \dragon in the long run (even a master of disguise such as \Nzessuacrith). 

An \ophidian could masquerade as a \bezed. 
This was not widely advertised among the \resphain.
Few such spies existed, and much fewer had ever been caught. 
The \resphan leaders were afraid of the spies, but they were not willing to admit that they existed.
They did not want to cause a widespread panic (like what happened in Battlestar Galactica).









\subsection{Physique}
\Ophidians{} are snake-like humanoids. 
They have two arms but no legs, only a snake tail. 

They are taller and bigger than \humans{} and \scathae, but they are not so strong or fast. 

They can only creep slowly on their snake tails. 
They very often ride some kind of mount to increase their mobility (\hr{QJ often ride}{like \quiljaaran}. 

But they have \hs{psionic} powers! 

\Ophidians looked not just serpent-like, but also \dragon-like. 
Their heads were much like those of \dragons (but without the horns). 
They had bony ridges on the top of the head and down along the spine and tail. 








\subsection{Politics}





\subsubsection{\Ophidians{} today}
\target{Ophidians today}
Perhaps the old, wise \ophidian{} lords and gods were drained and depleted after the harrowing war against the terrible \xzaishanns{} and went into dormancy. 
They are less violent than the wicked \draecchonosh, but cold, inhuman and alien. 

Perhaps many of them were slain by the nascent \draecchonosh. 

Perhaps the survivors sleep and dream in dark places beneath the earth. 

Perhaps they are weakened by the \dragon/\bane{} war, or the Shrouding. 

Maybe they just sleep and figure: 
\ta{Those \draecchonosh{} think they're so tough. 
  Let them deal with the \banes.} 
If the \banes{} were to conquer the \dragons, the \ophidians{} and \nagae{} \emph{might} be able to defeat them. 
They themselves believe they could. 
Others are less sure. 

Perhaps the \ophidians{} believe, \hr{Ishnaruchaefir chooses eternal war}{like \Ishnaruchaefir}, that eternal war is preferable to what would happen if one of the races won. 
They've seen what happened when the \draecchonosh{} won, after all, and it wasn't pretty. 





\subsubsection{Remnants of the \ophidian{} civilization}
The \scathae{} \hr{Origin of Scathae}{were created} with bits of the original serpent people, salvaged from mummies. 

There exist ruins from the civilization of the serpent people. 

Compare to the serpent people from H.P. Lovecraft's stories such as \emph{The Nameless City}.

Perhaps they live on as a \quo{forgotten race}. Perhaps they live hidden among the \scathae{} and interbreeding with them, like the reptillian humanoids from \DIBiggestSecret. 







\subsubsection{Cults}
There are serpent cults that worship the old \ophidians. They are mostly aloof and merely seek wisdom. Only occasionally do they involve themselves in the \feud. 







\subsubsection{\Ophidian-\resphan{} connection}
Perhaps some of old \ophidians, repulsed by their \draconic{} brethren and their violent behaviour, have sided with the \resphain. Perhaps they helped found \Mystraacht. 

Perhaps \Ishna, being less evil and chaotic than many \dragons, has dealings with the traditional \ophidians{}, and with \Mystraacht. 









\subsection{Psychology}
\subsubsection{Evil}
Note that while the \ophidians{} are less chaotic and violent than \dragons, this does not mean that they are not evil. 
They are cold, calculating, emotionless and aloof. 
Being immortal, they possess inhuman patience and perspective, and tend to see the short lives of lesser creatures as expendable. 
\Dragons{} are likewise, but more violent and passionate. 

\lyricsbalsagoth{A Tale From the Deep Woods}{
  The orm-garth awaits me, darkly astir with ophidian malice\ldots{}
}

Some \trueophidians{} even hate all humanoids and \dragons{} and want to exterminate them, to make \Miith{} clean again.





\subsubsection{Magic}
Some \ophidian{} cultures rejected \hs{sorcery} (summoning-based magic) as evil. 
They used only \hs{psionics}. 
Their psionics were advanced and sophisticated, but lacked the brute force available in sorcery. 
That is why \hr{Sethicus}{\Sethicus} was able to defeat them. 





\subsubsection{Scientific disposition}
\target{Ophidian philosophy}
The old \ophidians{} were cool and dispassionate. 
They were scientists. 
They had no religion, only rational philosophy. 
They did, however, know of the existence of a number of powerful cosmic entities and \quo{\hs{gods}} and would sometimes make deals with them. 
But \hr{Ophidians and XS}{not with the \xss}, except for certain rebels and mavericks. 

The \ophidians were atheistic. 
They believed the universe was fundamentally devoid of meaning, purpose, reason and soul.
Life and thought existed in the universe, but only as \quo{passengers}, not as something inherent or immanent. 

As such, life was meaningless. 
The \ophidian philosophy was quite nihilistic. 

The \hr{Sethican philosophy}{philosophy of \Sethicus} renounced this world-view in favour of a more mystical one.





\subsubsection{Ouroboros}
They had a philosophical principle which they symbolized as the Worm Ouroboros. 
But this was not a god, just a metaphor. 

\index{cannibalism!\Ophidians}
Later, some \quo{fallen} \ophidians or \quiljaaran twisted the memory of this Ouroboros and turned it into a cannibal god that devours its own children (perhaps like the Hindu goddess Kali). 





\subsubsection{Slow lifestyle}
\target{Ophidians are slow}
The \ophidians were generally long-lived, slow-living and slothful, so they developed new ideas only slowly. 
Hence \hr{Early Ophidian culture}{their culture lasted for millions of years}.

\Sethicus \hr{Sethicus brought innovation}{sped things up}. 





\subsubsection{Telepathy}
Originally \hr{Ophidian telepathy}{the \ophidians{} were telepathic}. 
But \hr{Ophidians lose telepathy}{they lost it}. 









\subsection{Story ideas}
\subsubsection{Sleeping \ophidians}
Someone, on a journey through the world (probably the \Wylde{}), encounters one or more sleeping \ophidians{}. 
Probably in a cave beneath the earth, or in an ancient, abandoned temple or palace. 

They communicate with the \ophidians{} in dreams. 
Or perhaps the humanoids merely feel the \psp{\ophidians} dreams, almost being sucked into them because of their mental great power. 

The humanoids might be \Shilred-tachi, or perhaps Carzain-tachi in some later story. 















\section{\QuilJaar}
\target{QJ}
\index{\quiljaar}
The \quiljaaran{} are an ancient race, as old as the \dragons{} and \vorcanths. 
They are one of \hr{Ophidian descendants}{several races descended from the \ophidians}. 

Maybe merge the \quiljaaran{} with the \bladedpeople. 

Compare them to the Jaghut and the Forkrul Assail from \cite{StevenEriksonIanCameronEsslemont:MalazanBookoftheFallen}. 









\subsection{Name}
Singular \emph{\quiljaar{}}, plural \emph{\quiljaaran{}}, adjective \emph{\quiljaaran{}}. 

\target{Names for QJ}
They are sometimes called Serpentines, Serpentine People, Serpent Folk or Serpent Men. 
These names \hr{Names for Ophidians}{are also used for the \ophidians}. 










\subsection{Biology}





\subsubsection{Demographics}
At the time of the \thirdbanewar there were 1000-3000 \quiljaaran worldwide. 
\hr{Dark Crescent QJ}{Some of them worked for \Secherdamon}.










\subsection{Culture}





\subsubsection{Architecture}
\target{QJ architecture}
\index{architecture!\quiljaaran}
The \quiljaaran{} built squat, functional and geometric houses and holds. 
Round, bulbous things; squares and rectangles, standing cylindical towers and semicylindrical longhouses. 
Always nice geometric forms when possible (sometimes in accordance with \hs{occult geometry}). 

Contrast with \hr{Draconic architecture}{\draconic{} architecture} and \hr{Resphan architecture}{\resphan{} architecture}. 





\subsubsection{Languages}
The \quiljaaran{} language is meant to resemble Chinese. 
A bit. 





\subsubsection{Nations}
The \quiljaaran{} race is split into a number of \quo{nations}. 
Each nation has its own characteristic language, culture, facial/bodily traits and \colours. 

The nations include:
\begin{gloss}
  \gitem{\KyanHweDin} 
  The most populous nation (today, at least). 
  Their language sounds kind of like Chinese. 
  
  \gitem{\Okiriru:}
  They are dark gray in \colour and have glasses-like markings on the skin folds on their necks (like certain cobras). 
  Their language sounds like of like Japanese. 
\end{gloss}





\subsubsection{Riding mounts}
\target{QJ often ride}
\QuilJaaran often rode on mounts. 
They could only move fairly slowly on their own. 





\subsubsection{Science and philosophy}
\target{QJ philosophers}
\index{technology!\quiljaaran}
The \quiljaaran{} cared about philosophy and science for its own sake. 
Unlike the \aryothim, who \hr{Aryoth inventors}{cared more about technology}. 





\subsubsection{Seamanship}
The \quiljaaran{} \hr{QJ seamanship}{were generally poor sailors}. 





\subsubsection{Technology}
\target{QJ weapons}
Weapons and tools designed by the \quiljaaran{} were often based on magic from the ground up: 
Pistols that fired lightning bolts, etc. 
Unlike \hr{Aryoth weapons}{\aryoth{} weapons}, which were physical and enhanced with magic. 










\subsection{History}
The \quiljaaran{} fought alongside the \dragons{} in the \firstbanewar. 
They \hr{QJ supremacy}{rose to prominence after that war}, but lost it again. 









\subsection{Physique}
\QuilJaaran{} are snake-like humanoids. 
They have only two arms, no legs.
Instead they slither on a long snaky tail. 

Unlike \dragons{} and \nagae, their tongues are long, strong and prehensile and can be used as grasping tools, like a tentacle or an elephant's trunk. 

They can extend the skin on their necks, like cobras. 









\subsection{Politics}





\subsubsection{Sentinels}
\target{Dark Crescent QJ}
In the \hs{Age of the Shroud}, many \quiljaaran were active members of the \hr{Sentinels}{Sentinels of \Miith}. 
Some of them worked for \hr{Secherdamon}{\IrocasSecherdamon}, possibly as part of his \hs{Dark Crescent}. 

At the time of the \thirdbanewar there were 1000-3000 \quiljaaran worldwide. 
100-200 of them worked for \Secherdamon.
10-15 of these were active on \Azmith.
5-6 of them were among the Rissitics. 

\target{QJ in Yormis}
Several \quiljaaran dwelt in or near \Yormis in disguise. 
Some of them were part of the Dark Crescent.
Others were independent.
They practiced their dark science and philosophy. 
Some worshipped the \xs (even \Ubloth) 









\subsection{Psychology}





\subsubsection{Apathy}
\target{QJ apathy}
The \quiljaaran{} always had some tendency towards apathy. 
They were philosophical and non-aggressive and generally just wanted to be left in peace. 
This made it harder for them to withstand the \hr{Aryothim kill QJ}{attacks of the aggressive \aryothim}. 





\subsubsection{Not inventive}
\target{QJ not inventive}
Owing to their apathy, the \quiljaaran were not inventive. 
They had a lust for philosophy and knowledge, but they had a hard time putting it to practical use. 

\hr{Inventiveness}{Inventiveness is not something all creatures have}. 









\subsection{Skills and powers}





\subsubsection{Magic}
The {\quiljaaran} had \hr{QJ magic}{their own style of magic}. 
















\section[Scatha]{\Scatha}
\target{Scatha}
\target{Scathae}
\index{\scatha{} (plural \scathae)}
The \scathae{} are reptillian humanoids. They are one of the dominant humanoid species on \Miith{}, especially common in the Imetrium and in the Rissitic Dominion. 

\Scathae{} are very common and widespread. %In a sense, they are the 'humans' of \Miith{}. 









\subsection{Name}
Singular \emph{\scatha{}}, plural \emph{\scathae{}}. 
This grammar is Imetric. 
The associated adjective is \emph{\scathaese{}}. 

A male \scatha{} is called a \dax. 
A female is called a \sphyle. 









\subsection{Physique}
\Scathae{} are large, bipedal reptiles. An average adult \scatha{} is 2 meters long, 160 cm tall and weighs 80 kg. Males and females are the same size. They have a tail and do not stand fully erect. The tail accounts for about $40\%$ of the length and is not prehensile. They have five fingers on each hand and four toes on each foot. 

As they age, \scathae{} continue to grow in height and length, but not much in weight. Very old \scathae{} tend to be long and gaunt. 

A \scatha{} is significantly stronger than a \human. 
Comparable to how much a \human{} man is stronger than a woman. 

\Scathae{} are strong for their size, but not very dextrous nor fast. They have thick scales, especially on the back, shoulders and overarms, which provide protection in combat. Their teeth and nails are small, like \humans{}', and not useful in combat. 

%\Scathaese{} eyes are covered by a fully transparent but solid membrane that protects from sandstorms and the like. The outermost layer of this membrane is shed once every two months or so (similar to how a snake sheds its skin). They also have normal eyelids over this. As a special feature, 
\Scathae{} blink upwards with their lower eyelids, unlike \humans, who blink downwards with their upper eyelids. A \scatha{} has a long tongue and can lick his own eyes. 

\Scathae{} are diurnal and do not see well in the dark. 

Males are generally dark in \colour whereas females have brighter scales. Males are recognizable by two bony ridges on the top of the head (one over each ear). Females have a single ridge in the middle of the head, generally larger than that of the males. 

In combat, \scathae{} use weapons, such as swords or spears. They have good eyesight and make good archers or gunners. 





\subsubsection{Potential for greatness}
\target{Scathae have potential for greatness}
\hr{Origin of Scathae}{The \scathae were created by the \dragons} using bio-magic learned from the \xss.
The process of creating the \scathae was very much inspired by the way the \dragons had created themselves.
The \scathae were designed as \dragons in miniature.

As such, \scathae genuinely did have more individual potential for greatness than \humans did.
\hr{Humans suck}{\Humans were worthwhile only in a big mass}.
Each \scatha could reach great spiritual advancement and become a demigod.
Look at \hr{Criseis}{\Criseis} and \hr{Psyrex}{\LocarPsyrex}. 

This was also why \hr{Dragon blood gives immortality}{drinking \draconian blood} worked better for \scathae than for \humans. 





\subsubsection{Senses}
\Scathaese{} senses are like those of \humans, except the sense of touch, which is rather dull (due to their thick scales). 

\target{Scathaese colour vision}
\Scathae{} have \colour vision different from that of \humans. They can distinguish between some nuances that \humans{} cannot tell apart, and vice versa. \hr{Curiet's colour vision}{This is brought up} by \hs{Curiet Serpentin}.









\subsection{Biology}
%\Scathae{} are warm-blooded reptilians. They are Archosaurs, related to dinosaurs and birds. The species is a result of natural evolution and has a few closely related (but unintelligent) species. They prefer temperate to subtropical climates with low humidity. They are omnivores with feeding habits similar to those of \humans{}. (Today, most \scathae{} are civilized and live off farming.) The \scathae{} evolved in the sandy deserts of the East, and the violent sandstorms are one of the reasons why they developed their thick, protective scales and eye-membranes. 

\Scathae{} are warm-blooded reptillians. They did not evolve naturally, but were the creations of the early \dragonlords{} who mixed \dragon{} and \naga{} genes with those of land-dwelling reptiles (including \nycans{}). They are herbivores. 

\Scathae{} have two genders and reproduce by internal fertilization. The female lays a small cluster of eggs (typically one or two). Males and females are born in equal numbers. 

\Scathae{} have a life expenctancy of around 75 years at TL3. Ancient \scathae{} of up to 100 years are exceptional but not unheard of. They reach sexual maturity in their early teens. They are usually considered adult around the age of 20 (varying depending on culture). 

Among \scathae{}, males and females are very similar, physically as well as psychologically, and the genders usually equal in society. Because of this, homo- and bisexuality is rather common. \Scathae{} are usually monogamous, but they may or may not mate for life.  





\subsubsection{\Demiscathae}
\target{Demiscatha}
There existed a number of \quo{\demiscatha} races. 

\begin{gloss}
  \gitem{\hr{Scatha-Naga hybrids}{\scatha/\naga hybrids}}
    These hybrids, such as the ones in the Imetrium, were considered \demiscathae by outsiders. 
  
  \gitem{\hr{Naga}{\naga}}
    Some \hr{Nagae considered Demiscathae}{thought of the \nagae as \demiscathae}.
    Those who had actually seen and dealt with the \nagae knew better.
\end{gloss}

There were also \quo{\hr{Demihuman}{\demihumans}}.

The \scathae were more tolerant of their different kin than the \humans were, but after millennia of interbreeding there had emerged a race of \quo{standard} \scathae, and other variants came to be called \quo{\demiscathae}. 

Rethink the idea of the different \scatha ethnicities!





\subsubsection{\Scatha/\naga hybrids}
\Scathae could interbreed with \nagae, \hr{Scatha-Naga hybrids}{giving rise to hybrid children}. 









\subsection{Psychology}
The \scathae{} are naturally evolved from animals that were both predators and prey. Where many other creatures of the same areas developed great size and horns or spines to defend themselves, the proto-\scathae{} instead developed pack tactics, intelligence and hands. 

As the descendants of pack-living prey animals, \scathae{} have a strong sense of commitment to their fellows. This makes the typical \scatha{} loyal and patriotic, but also xenophobic. \Scathae{} are fiercely devoted to their community (family, clan, nation), slow to trust outsiders and zealous about punishing betrayers and criminals. 

\Scathae{} value security and a well-ordered life. They prefer to know their place in the pack/community and function well in a class system. \Scathae{} make good craftsmen and soldiers. They have little craving for excitement and rarely go out to \quo{adventure}. \Scathaese{} adventurers are typically on some kind of quest for their community, rather than simply seeking fame and wealth. 





\subsubsection{Ferocity}
\target{Scatha fury}
The \scathae{} actually have a deep-seated, bestial, primal fury and ferocity. 
It is tempered and held in abeyance by cold, rational \ophidian{} genes and millennia of breeding and conditioning, but it still lurks underneath the surface and can erupt in full flame. 
After all, the \scathae{} do have \xsic{} blood. 
They were created by the \dzraicchenosses{} in (a shadow of) their own image. 

Most of the time, this ferocity is sublimated into some other lust. 
Often religious or ideological fervour. 

Sometimes, \hr{Imetrian coldness}{as with the Imetrians}, the ferocity manifests as a cold, merciless fanaticism. 





\subsubsection{Societies}
\Scathaese{} societies tend to be more collective and less monarchical than \human{} ones. 
\Scathae{} do not have the same hunger for personal power, but tend to be more social and pro-community, so it is easier for them to form stable systems of many-man-rule. 
Such as \hr{Bacconate}{\bacconates}. 





\subsubsection{\XzaiShanns and aspects of the mind}
Different aspects of the \scathaese mind \hr{Primordials and the Scathaese mind}{were associated with different \xss}.
There was some truth in this. 
(See the section.)









\subsection{Demographics}
\subsubsection{Habitat}
\Scathae{} are among the most widespread humanoids on \Miith{}. 
They are common in all lands except the Northern Kingdoms, but especially dominant in the Imetrium, Durcac and Irokas. 





\subsubsection{Subraces}
The \scathaese{} race can be divided into three major subraces: 

\begin{gloss}
  \gitem[\Tassians]{\Tassian}
    \target{Tassian}
    The most widespread subrace. 
    Blue scales. 
    Common in the Imetrium and southern \Velcad{}
    Peoples include the \hr{Masthenon}{\Masthenon} and \hr{Ortaica}{\Ortaicans}. 
  \gitem[\Mekriis]{\Mekrii}
    \target{Mekrii}
    Red or reddish brown scales.
    Most common in Durcac and the Orient. 
    Peoples include the \hr{Shurco}{\Shurco} and \hs{Rissitics}. 
  \gitem[\Lois]{\Loi}
    \target{Loi}
    Green or greenish brown (very rarely black).
    Most common in the north and the \Serplands. 
\end{gloss}










\subsection{History}
The \scathae{} were \hr{Origin of Scathae}{created to serve the \dragons}. 

Despite this, they were not slaves to the same degree as \humans. 
They were bred from \naga{} stock and as such, they were free citizens to a higher degree than \humans{} ever were. 
They had more capability to think for themselves, and they understood more of the world around them, and of their masters. 
After the \hr{Shrouding}{\SecondShrouding}, though, they became more enslaved, as their minds succumbed to the Shroud.

Perhaps they were created to replace the old \hr{Ophidian humanoids}{\ophidian{} humanoids} when they became extinct. 





\subsubsection{The first \scathae}
The first generation of \scathae{} to be created were more powerful and had more \draconian{} and \xsic{} blood than modern \scathae. A few of these prototypes survive, including \hr{Criseis}{\Criseis} and \hr{Psyrex}{\LocarPsyrex}. 









\subsection{\Troglodytes}
\target{Troglodyte}
\index{\troglodyte}
\Troglodytes{} are a degenerate subrace of \scathae{} (or several superficially similar subraces) that dwell in caves underground. They are often found worshipping the \daemonic{} monsters of \hs{Kai Leng}. 

To the eyes of other \scathae, the \troglodytes{} are frightening and loathsome abominations. They remind the \scathae{} of some primal, primitive aspect of their own nature and origin, something which the \scathae{} would rather forget. 









\subsection{Politics}





\subsubsection{\Humans}
\Scathae and \humans \hr{Scathae and Humans hate each other at first}{hated each other when they first met}.
After some centuries they slowly learned to accept one another. 















\section{\Shaeeroth}
\target{Shae'eroth}
\target{Shaeeroth}
\index{\shaeeroth}
The \shaeeroths{} were powerful \draecchonosh, with \xsic{} blood coursing through their veins. 
\Draconic{} nobility, in a sense. 

They were an attempt to create \xs{} hybrids: 
Superior \dragons, \hr{Dragons want to usurp the XS}{reincarnations of the \xss}. 









\subsection{Becoming a \shaeeroth}
\target{Shaeeroth ritual}
The \shaeeroths{} were not born \shaeeroths. 
They became \shaeeroths{} by imbibing \xsic{} \quo{blood} in a magical ritual. 
They would then die and be reborn as \shaeeroths. 

Only the most powerful \dragons{} could become \shaeeroths. 
Even \Secherdamon, for example, did not become one until \hr{Secherdamon becomes Shaeeroth}{late in his life}. 

\citebandsong{DeathspellOmega:SiMonumentumRequiresCircumspice}{%
  Deathspell Omega
}{
  Blessed are the Dead Whiche Dye in the Lorde
}{
  Stare wide-eyed at this dense pitch boiling by the art divine\\
  Amniotic liquid of another kind\\
  That flesh and blood can not inherit the kingdom of God\\
  Behold the transformation, servant
  
  Like a malignant tumour and sudden growth of cancer divine\\
  A rebirth in putrefaction irreversible, \\
  corruption does not inherit uncorruption
}

\citebandsong{Nile:BlackSeedsofVengeance}{Nile}{
  The Black Flame
}{
  Open, For Me the Gates Shall Open\\
  Over the Fire of the Spirit, The Breath Drawn by the Gods. \\
  Arise Apophis Return, That I Might Return, \\
  Borne by the Flame Drawn by the Gods Who Clear the Way that I Might Pass.\\
  The Gods Which Sprang from the Drops of Blood \\
  which Dripped From the Phallus of Set\\
  That I might be Reborn\\
  For I am Khetti Satha Shemsu, Seneh Nekai\\
  And Will Become Set of a Million Years
  
  Akhu Amenti Hekau\\
  I shed My Burnt Skin and am Renewed
}

The process of becoming a \shaeeroth involves lowering the walls of denial in one's mind and surrendering to the vast, cruel cosmos.
You have to realize that you are nothing next to the true forces of the universe.
Only then can you begin to acquire true power. 
Your ego and fears and wishes have to be broken down in order to be built up into something new and stronger.

\citebandsong{Nile:Ithyphallic}{Nile}{
  Language of the Shadows
}{
  Abandon hope\\
  And I shall become free\\
  And with freedom acquire emptiness

  With the mind cleansed and empty\\
  There is the void known as despair\\
  A gateway upon an emptiness endless and vast

  In despair the language of the shadows is intelligible\\
  In madness all sounds become articulate

  Terror and despair they guide me\\
  Into nightmares that follow one upon the other\\
  Like windblown grains of sand

  [solo: Dallas]

  I have become as the wastelands\\
  Of unending nothingness\\
  Now shall the night things\\
  Fill me with their whisperings\\
  And the shadows reveal their wisdom
}





\subsubsection{\Shaeeroth names}
\target{Shaeeroth name}
As a part of the process of becoming a \shaeeroth, the \dragon had to earn and take a new name.
\hr{Draconic names}{\Draconian names were magical}.









\subsection{History}
\target{No more Shaeeroth created}
After \Secherdamon, for millennia no \dragon{} succeeded in becoming a \shaeeroth. 
Several \dragons{} attempted it, but all perished and were destroyed. 

Scholars speculated that this was because it was too hard to get in contact with the \xss, for a number of reasons:

\begin{enumerate}
  \item The Shroud made it hard to communicate with anything.
  \item The \xss{} were in deeper sleep than usual.
  \item 
    \Tiamat{} was the true link to the \xss.
    Only she had the skill to rouse them. 
    When she perished, it was hard for the remaining \dragons{} to replicate her feat and her special affinity with the monstrous alien \xss. 
\end{enumerate}

The next \shaeeroth{} after \Secherdamon{} became \Vizsherioch. 
He \hr{Vizsherioch becomes Shaeeroth}{became a \shaeeroth} after the resurrection of \Nithdornazsh. 










\subsection{Demographics}
Their numbers included \hr{Nexagglachel}{\Nexagglachel}, \hr{Ishnaruchaefir}{\Ishnaruchaefir} and, much later, \hr{Secherdamon}{\Secherdamon} and \hr{Vizsherioch}{\Vizsherioch}. 

























\chapter{The \Erebean Races}
The \Erebean{} races are the \banes{} and other monsters native to \Erebos{} or \Nyx. 















\section{\Bane}
\target{Bane}
\target{Banes}
\index{\banes}
The \banes{} are a terrible race of alien creatures. They are not native \Miith{} but come from the world of \Erebos, sometimes called the \Baneworld. 

The \banes{} were originally one of several races that warred for control of \Erebos. They won in the end by being the coldest, most ruthless and most efficient, almost machine-like. 

They have cruelly exploited their homeworld until \Erebos{} is now a dying husk, almost all its life energy drained away. The \banes{} know that they cannot evolve further while still tied to \Erebos\dash indeed, \Erebos{} and the \banes{} it are doomed to decay and extinction. So the \banes{} are searching for a new planet to leech from, so that they may survive and grow in power. (See also section \ref{Fighting for survival}.)

\citeauthorbook[p.137]{RobertEHoward:TheAltarandtheScorpion}{Robert E. Howard}{%
  The Altar and the Scorpion%
}{
  \ta{The real gods are dark and bloody!
    Remember my words when soon you lie on an ebon altar behind which broods a black shadow forever!
    Before you die you shall know the real gods, the powerful, the terrible gods, who came from forgotten worlds and lost realms of blackness.
    Who had their birth on frozen stars, and black suns brooding beyond the light of any stars!
    You shall know the brain shattering truth of that Unnamable One, to whose reality no earthly likeness may be given, but whose symbol is\dash the Black Shadow!}
    
    The girl ceased to cry, frozen, like the youth, into dazed silence.
    They sensed, behind these threats, a hideous and inhuman gulf of monstrous shadows.
}









\subsection{History}
\subsubsection{Origin}
The \banes{} were \hr{Banes are created}{created by the \voyagers}. 





\subsubsection{Interstellar empire}
\target{Interstellar Bane empire}
At the time of the \firstbanewar, the \banes already had an interstellar empire (just like the \ophidians did). 
They had been expanding for thousands of years, conquered dozens of planets, absorbed the local life and used it to evolve themselves, and then having moved on, leaving behind a barren, dead ruin, infested with wretched, degenerate survivors. 

\target{Why the Banes want Miith}
But \Miith was the important planet that they had been searching for all along. 
It had been settled and populated by the \voyagers, their own creators. 
It contained a Heart full of \voyager-tampered energy, and the \hr{Noggyal}{\noggyal} \hs{mother-mass}, which was the \quo{other half} that the \banes wanted. 

Fortunately for everyone else, the \banes had only primitive space travel, and they were un-creative.
So their interstellar empire could spread only slowly, and not far.
If the \bane swarm spread too far apart, the farthest parts withered and died.
They needed the \noggyal \hs{mother-mass} in order to truly expand their dominion. 





\subsubsection{\Firstbanewar}
The \banes{} \hr{First Banewar}{invaded \Miith}. 
They were defeated. 





\subsubsection{\Secondbanewar}
\hr{Resphan rebellion}{Some \resphain{} rebelled against \Merkyrah}. 
They won.
They summoned the \banes{} again. 
The \hr{Second Banewar}{\secondbanewar} began. 

Eventually \hr{End of the Second Banewar}{everyone lost}. 
The \banelords{} were gone again. 





\subsubsection{\Banelords{} imprisoned}
\index{\CrystalSphere}%
\Tiamat-tachi successfully banished the \banelords{} from \Miith{}. 
They were frozen in the icy \hr{Crystal Sphere}{\CrystalSphere}. 

\Semiza{} and \Thanatzil{} tried to free them but failed. 

The \resphain{} (just before the \Secondbanewar) managed to free some of them, but not all. 

The \banelords{} that were buried deepest (including \Daggerrain) have been frozen for ten thousand years. 

Compare to the devils in the manga \cite{NagaiGo:Devilman}. 

The \lesserbanes{} were also blocked out by the \CrystalSphere. 
They could not come to \Miith{} from \Erebos; however, many \lesserbanes{} remained frozen in the \CrystalSphere{} and could be brought to \Miith{} (such as by \hr{Banes possess Humans}{possessing a \human{} body}). 






\subsubsection{The return of the \banelords}
\target{Return of the Banelords}
\index{\CrystalSphere}%
As the Shroud \hr{unravelling}{unravelled}, the \CrystalSphere{} \hr{thawing}{began to thaw}. 
Some of the \banelords{} were able to break free of their icy prison and could now walk on Mith again. 
In the decades before the \hr{TBW}{\thirdbanewar}, the \banelords{} returned and resumed command of the Cabal, demanding that all \resphain{} bow to them. 
Not all \resphan{} lords were happy about this. 

The \dragons{} had long feared that the \banelords{} might return. 

\lyricsbalsagoth{Invocations Beyond the Outer-World Night}{
  But, it is here written that one day, when even the War of the Lexicon and the cataclysmic Great Chaos War have faded to naught but distant memory, a great conflict shall be waged between the forces of Order and the dread avatars of the Z'xulth. \\
  Vile fiends of the Outer Darkness, They-Who-Lurk-And-Breed-In-Limbo, the Dwellers in Eternal Shadow unleashed through The Gate to That Which Lies Beyond! \\
  The Black Galaxy disgorges its malignant horrors! \\
  Mankind shall suffer inestimably at the hands of these sinistrous black titans of maleficent Chaos!
}









\subsection{Name}
Singular \emph{\bane{}}, plural \emph{\banes{}}. 

The \bane{} race as a whole was also called \quo{the \Bane}. 

The name \quo{\bane} was \hr{Semiza names Banes}{invented by \Semiza}. 

The \dragons originally had \hr{Draconic names for Banes}{other names for the \banes}. 

\target{Sitra Achra}
\emph{\SitraAchras} was an alternate \resphan name for the \banes. 
It means approximately \quo{those from the ouside} in the \Resphan tongue 

\emph{\SitraAchra} is Hebrew for \quo{external forces}, \quo{other side}, \quo{side of evil}. 









\subsection{Physique}





\subsubsection{Types of \banes}
The \SitraAchra race is split into multiple \quo{tiers} or \quo{castes}. 
These include: 
\target{Banelord}

\begin{description}
  \item[\Banekings] are the supreme lords of the \bane{} people. 
    Only one \baneking{} is known, namely the dread \hr{Voidbringer}{\Voidbringer}. 
  \item[\Banelords] are as powerful as gods or \dragons. 
    Only a few dozen \banelords{} have ever come to \Miith{}. 
  \item[\Greaterbanes] are tall, stately and fearsome. 
    Compare them to the Nazg\^ul from J.R.R. Tolkien's \emph{Lord of the Rings} or the Myrddraal from Robert Jordan's \emph{Wheel of Time}. 
    Also called \quo{\baneknights}. 
  \item[\Screamers] are flying monsters with bodies specially designed to combat \dragons.
    They are as intelligent as \lesserbanes. 
  \item[\Lesserbanes aka \stalkers] are short and wicked. 
    A \lesserbane{} is approximately as physically strong as a \resphan, but most \lesserbanes{} know no magic. 
  \item[\Banespawns] are crawling, worm-like vermin. 
\end{description}





\subsubsection{Arms, legs and posture}
Maybe \banes{} have six limbs. 
The middle pair sits midway between the arms and legs. 
This pair can be used as extra arms or extra legs. 

The legs (both pairs) are sprawled out a bit, kind of like on a \meccaran. 
But a \bane{} has a small, slim, compact body where a \meccaran{} has a bigger belly. 

Maybe they have small, round, hooflike feet. 

In combat they draw their heads down between their shoulders to protect their sensory organs. 
In peacetime they hold their heads high for a better view.





\subsubsection{Brain}
\Banes{} have their brain in the torso rather than the head. 
They have some gill-like openings in the body that ventilate the brain. 

When they carry \armour, it is split-mail-like stuff made of overlapping plates, so as to let in some air so their brains can breathe. 





\subsubsection{Cold aura}
\target{Banes are cold}
\Banes{} are cold as death. Colder, in fact. Their body temperature is very low, below room temperature, near the freezing point. They spread deathly cold around them that seems to sap the life of creatures around them. 

This cold is one of the tell-tale signs that a \bane{} is nearby. When the \bane{} is submerged in \Nyx{} the cold cannot be felt, but as it surfaces, the cold spreads out like a draught. 





\subsubsection{Coming to \Miith{} through a \human}
\target{Banes possess Humans}
\Banes{} cannot come to \Miith{} directly. 
Instead, a \bane{} has to possess a \human{} body. 
(It must be a \human{} or \resphan. No other creatures will work.) 
The \bane{} takes the body and twists it into a \bane{} body. 
The \human{} is killed and \hr{Life drain}{his soul consumed} in the process. 

The victim needs not be a cool \human. 
It can equally well be a total loser. 
The \bane{} just needs the body as a shell. 
It then creates its own \bane{} body from it. 

Compare to the Agents from \emph{The Matrix} movies, or the parasitic plant from \authorbook{Clark Ashton Smith}{The Seed from the Sepulcher}. 





\subsubsection{Eat souls easily}
\target{Banes eat souls easily}
The perhaps most terrifying power of the \banes was their ability to \hr{soul-eating}{eat souls}. 
They needed no elaborate spells to do it. 
It was an inborn, natural ability of theirs. 
They could eat souls just by killing people\dash even powerful immortals such as \dragons and \resphain. 

Their \resphan allies knew this, and \hr{Resphain fear Banes}{it scared them shitless}. 

The \banes{} were terribly dangerous \trope{CosmicHorror}{Cosmic Horrors}. 





\subsubsection{Face}
A \bane{} has no face.
The front of its head is completely blank and smooth. 
This is very scary.

\Banelords{} have \armoured heads with frills, kind of like the Xenomorph Queens in the \movie{Alien} movies. 
Also compare to the Yautja Predators in the \movie{Predator} movies. 




\subsubsection{Fear rain}
Banes dislike rain and fear hail.
It is painful to their telekinetic sense.
They will retreat out of hail and not go out in it unless they absolutely must.






\subsubsection{Flying}
Greater \banes can fly using a number of long limbs resembling insect antennae with feeling-hairs on them.





\subsubsection{Giant form}
\Banelords{} are able to transform into a gigantic, monstrous \quo{combat form}\dash useful if you need to fight \dragons{} or other large monsters. 

The monstrous form looks bat- or ray-like. Like the \quo{Balrog} in the movie \emph{Eragon}. 







\subsubsection{Horror}
The \banes{} are a \trope{CosmicHorror}{Cosmic Horror}. 
Even the \resphain{} fear them. 
Even \lesserbanes{} are frightening and loathsome, and \resphain{} shy away from them.

When \humans{} see a \bane{} and especially its blank face, they see themselves reflected back at them in that empty visage. 
It is cruel and horrible, because it hints at the fact that \humans{} are descended from \banes. 
It conjures up vague feelings of hideous truths about yourself that you do \emph{not} want to face. 

The fear that the \banes{} invoke is meant to parallel, among other things, the fear of decay, boredom, hopelessness, decrepitude, illness, old age.
The fear of being alone and abandoned and left to wither unseen. 

\lyricsbs{Cradle of Filth}{Bathory Aria}{
  The Spirits have all but fled judgement.
  \\
  I rot, alone, insane,
  Where the forest whispers puce laments for me
  from amidst the pine and wreathed wolfsbane
  beyond these walls, wherein condemned
  to the gloom of an austere tomb
  I pace with feral madness sent
  through the pale beams of a guiltless moon
  who, bereft of necrologies, thus
  Commands creation over the earth. 
  \\
  Whilst I resign my lips to death,
  a slow cold kiss that chides rebirth. \\
  Though one last wish is bequathed by fate. \\
  My beauty shalt wilt, unseen save for twin black eyes that shalt come to take my soul to peace or Hell for company.
  
  [Quoted words above are from Hammer Film's \quo{Countess Dracula} (1970). The singer is Imgrid Pitt, the actress who played the role of Elizabeth in that film.]
}

The \banelords were indescribably dreadful to behold. 

\citeauthorbook[p.257]{HPLovecraft:TheBlackTomeofAlsophocus}{H. P. Lovecraft}{%
  The Black Tome of Alsophocus%a
}{%
  The air was filled with the sound of their titterings and screamings as they danced obscenely and capered around me in a blasphemous ritual of depravity: and at the far end of the hall was the most terrifying sight of all, that dread black colossus of my visions, the inhabitant of the palace, Nyarlathotep.
  
  The Old One looked upon me intently, his gaze tearing at my soul and filling me with a horror so terrible that I screwed my eyes shut so as not to see that terrible visage of unnameable evil.
  Under that gaze my being began to melt away, as if it was being absorbed by some irresistible force.
  I was losing what little identity was left to me; my necromantic powers, which I now realized were as nothing compared to the powers of the inhabitant of this dark world, were stripped from me and scattered across the universe, never to be recovered. 
  
  Under that gaze my mind and soul were attacked from all sides by fear and loathing; I staggered as he tore at my being, peeling away my life layer by layer. 
  Sheer desperation took hold of me, but I was powerless to fight, unable to hold back the irrestible force that overwhelmed me.
}





\subsubsection{Mouths in their hands}
\Banes{} have mouths in the palms of their hands. 
Their four fully opposable fingers are very rough and also act as a sort of mandibles or tongues. 

Maybe their telekinetic feelers are in the hands. 





\subsubsection{No bones}
\Banes had no bones.
Their bodies were flexible and could squeeze through very small openings (although they would have to leave any solid items behind). 





\subsubsection{Shrouded in darkness}
Because of their connection to \Nyx{} and \Erebos, when they appear in \Miith{} \banes{} are always \hs{Shrouded} in an unnatural aura of darkness. 

\lyricsbalsagoth{Six Keys to the Onyx Pyramid}{
The fiends seemed inexplicably to be an extension of the night, as if their misshapen bodies were actually somehow composed of the darkness itself. 
Even as I gazed directly at them, I found I could not truly focus on their stygian forms\ldots{} 
their bodies appearing to shimmer and shift like the ripples of a heat-haze upon an arid plain.}





\subsubsection{Telekinesis}
\target{Bane telekinesis}
\Banes{} have no sense of vision and cannot see. 
In its place they have a long-distance tactile sense. 
It is similar to bats' sonar sight, but based on telekinesis rather than sound. 

This sense lets them see perfectly in darkness, but water blurs it, and glass blocks it completely. 
Dense smoke or debris also blurs it. 
So do nets and grids and the like\dash their \quo{feelers} have a certain granularity. 

Their high sensitivity to touch is the reason why \banes{} shun the rain. 

The \banes{} native languages are kinetic, not sound-based. 
That is why they adopt somatic names like \quo{\Daggerrain}. 

Remember that even feeble \lesserbanes{} should have telekinetic powers. 





\subsubsection{Weaknesses}
\Banes{} have a certain fear of rough weather. 
Rain, snow and strong wind scares them, and they avoid it if possible. 
Hail and sandstorms are dreadfully painful and will scare them away. 

This is due to their tactile senses, which render them very sensitive to touch-based things. 









\subsection{Biology}





\subsubsection{A force of Entropy}
\index{Entropy}
\target{Entropy}
\target{Bane parasitism}
The \banes{} are a force of Entropy. This is a major theme. The \banes{} are inherently destructive and parasitic. They can sustain themselves and grow stronger only by feeding upon their own kind. 

This means that the \banes{} are doomed to stagnation and decline. 

The \banes{} are creatures of the cold, darkness and emptiness. They steal, absorb and swallow everything. This is unlike the \dragons, who \hr{Dragons radiate life}{radiate life and light alike}. 

\hr{Daggerrain}{\Daggerrain} and his master, the \hr{Voidbringer}{\Voidbringer}, understand this. They seek to improve their race and let the \bane{} people evolve, to achieve perfection. To do this, they must look outwards and find new sources of life, new cosmic \dweomers{} to drain, new power with which to infuse their race. 

This is why they have set their eyes on \Miith{}. 
\Miith{} has the \hr{Heart}{Heart of \Miith}, a powerful \dweomer{}, ultimately based on Chaos. 
Drawing on the life-giving power of the Heart, and working together with \hr{Semiza}{\ps{\Semiza}} sorcerers, \hr{Semiza designs Resphain}{they were able to create} the \hr{Resphan}{\resphain}, who were intended as the new generation of \banes, empowered with the creative force of the Chaotic Heart. 
The \hr{Satharioth}{\satharioth} took this a step forward and stole the Chaotic \xzaishannic{} power from the \dragons, making themselves even greater. They see themselves as the future of the \bane{} race, the heirs of the \bane{} legacy.





\subsubsection{Life cycle}
\index{cannibalism!\banes}
\target{Bane cannibalism}
The \banes are cannibalistic. 
There are several tiers of them, from the lowly \banespawn{} through \lesserbanes{} to \banelords. 
\Banes{} can rise in the ranks only by devouring the souls of other creatures, including (mandatorily!) other \banes. 
New \banespawn{} are created by sacrificing a \lesserbane{} or a \banelord. 







\subsubsection{Sex}
\hr{Daggerrain}{\Daggerrain}{} believes that one way to escape the Entropy that afflicts the \bane{} race is by adopting and mastering the art of sex. See, sexual reproduction is a method that utilizes Chaos (random selection of genes, random mutations) to improve life, taking the best of what exists and adding potential improvements. As such, it is superior to the \ps{\banes}{} method of asexual reproduction, where they can create only clones that need to feed on their own kind in order to grow. 

The \hr{Heart}{Heart of \Miith} is inherently tied to life, sex and reproduction. Sex is an inherently chaotic force/phenomenon. The \banes{} intend to harness sex and use it to evolve themselves towards perfection. 

\Daggerrain{} sees the \hr{Resphan}{\resphain} as the way to go, but \humans{} are useful guinea pigs. 

The \banes{} and \resphain{} conduct all sorts of sexual experiments. Those overseen by the \banelords{} are well-thought-through and for the benefit of their science. Those conducted by the \resphain{} and \resviel{} are often just depraved pleasure disguised as science. 

Remember to have evil, wicked, sadistic sex-experiments! 









\subsection{Personality}





\subsubsection{Hivemind}
The \banes had a hivemind. 
The overmind was the \baneking \Voidbringer.
There was only one \baneking. 





\subsubsection{We}
\Banelords, when talking, will always say \quo{we} instead of \quo{I}. 
This reflects their communal nature. 
\Banelords{} have little individualism and think about the collective instead. 

The only times they use \quo{I} is when referring specifically to the individual \banelord, usually for some physical purpose 
(as in \quo{I am not strong enough to counter this \dragon\ldots{}}). 









\subsection{Equipment}





\subsubsection{Bane technology}
\target{Bane technology}
\index{technology!\bane}
The \banes{} command large reserves of spaceships and other technological artifacts salvaged from the \voyagers{} on \Erebos, and perhaps also other cultures whom the \banes{} have conquered. These were remnants of the \hr{High-tech civilization}{interstellar civilization that once existed}.

During the \hr{FBW}{\firstbanewar}, an important edge that the \banes{} had over the \Miithians was their superior technology. 
%They commanded reserves of technological artifacts salvaged from the \voyagers{} on \Erebos, and perhaps also other cultures whom the \banes{} had conquered. 

%, since the \banes{} themselves were afflicted with stagnation and had great difficulty creating truly new things. 


The \banes{} do not understand the technology. They are unable to create new things, or even replicate existing ones, and are hard pressed to just repair and maintain what they have. 
This is partly because the \voyagers{} were infinitely more advanced than their \bane{} spawn, and partly due to their being \hr{Entropy}{a force of Entropy, afflicted with stagnation and decay}. 
After all, 
\hr{Semiza}{\Semiza} had to 
\hr{Semiza designs Resphain}{help them design and create} their 
\hr{Resphan}{\resphain}{}.

The \hr{Resphan}{\resphain}{} are \hr{Resphan technology}{much more creative}.





\subsubsection{Space travel}
The \banes{} have spaceships. 
These are stolen from the \voyagers{} and tens of thousands of years old, if not millions. 

\Bane{} space travel is slow, partially because the \banes{} do not wholly understand the technology (including the navigational magic) and partially because the ships are old and damaged and the \banes{} can't repair them. 















% \begin{comment}
% \section{\Baneknight}
% The \banes{} most commonly summoned are the \baneknights{}. For this reason, they are typically referred to simply as \quo{\banes}. \Banespawn{} are rarely summoned, because they are difficult to communicate with and control, but a summoned \baneknight{} may sometimes bring along \banespawn{} as its own servants. 
% 
% In the following, \quo{\bane} will refer to the \baneknights{}. 
% 
% 
% 
% 
% 
% 
% 
% 
% 
% \subsection{Name}
% Singular \pronune{\bane}{BEJN}, plural \emph{\banes}. The word and grammar is English. 
% 
% 
% 
% 
% 
% 
% 
% 
% 
% \subsection{Physique}
% Seen from a distance, a \bane{} looks like a humanoid in a long, flowing robe, frayed and tattered at the edges. The robe is actually the creature's body, and the tattered egdes are small pseudopods used for walking. It has a head, torso and a number of arms (see below), but no legs, only a number of pseudopods or tentacles (each 20-50 cm long).  
% 
% Typically, a \bane{} will have two arms, but some have more, and they seem able to sprout new arms from their body as needed and retract them again when not needed. It is believed that growing additional arms and maintaining them is a psychically taxing, so that the \bane{} does not use it more than it has to. \Banes{} with only one arm have been sighted, but they typically maintain at least two. A newly sprouted arm will be weaker and less dextrous than normal\footnote{Perhaps a $50\%$ penalty to strength and a $30\%$ penalty to dexterity} and will grow to full strength gradually over a few minutes, perhaps 2-5 minutes. A \ps{\bane}{} arms do not have elbows; they are fully flexible and should perhaps be called tentacles. At the end an arm splits into a number (variable, but usually around four) of small \quo{fingers}. \Banes{} can also sprout new fingers and other appendages at will, and create bladed claws from their fingers. 
% 
% \Banes{} crawl only slowly on their tentacles. Maximum running speed is like a fast walk for a \human, and typical walking speed is like a slow walk. They can use magic to leap, fly or teleport, but this costs them valuable mana and casting time, so they only use it if necessary. 
% 
% \Banes{}' height varies from 170 to 250 cm. Their body is fully corporeal and solid and slightly denser than the flesh of most \Miithian{} creatures. Weight varies from 120 to 250 kg. 
% 
% The head has no recognizable features, including eyes. Sometimes you can feel the \quo{gaze} of the \bane{} upon you, but this is actually a psychic sensation of the \bane{} telepethically peering into your mind - a very unsettling experience, if not outright terrifying. The \ps{\bane}{} head does seem to contain vital organs, however, as a powerful blow to the head will sometimes kill the creature. 
% 
% The \ps{\bane}{} skin is tough and leathery on the body, head and major appendages. On the smaller or more recently generated limbs, the skin is more soft and supple. The creatures sometimes wear visible \armour, made of an alien metal and adorned with grotesque and hideous patterns and symbols. Some people have reported markings on a \ps{\bane}{} \armour resembling stylized drawings of known alien monsters. 
% 
% \Banes{} are \coloured in pure black, but wounds sometimes reveal blue-gray \quo{flesh} inside. A cut also causes a thick, pale white vapour to pour forth. This \quo{\baneblood} is heavier than air and will spill onto the ground. Large amounts of \baneblood{} spilt will leave a pale gray stain and kill vegetation in a small area. \baneblood{} smells foul but is not known to have any harmful effect on animal life. The blood may have magical properties if collected, but this is only speculation. 
% 
% In combat, \banes{} will typically strike in close combat with weapons. The weapons they typically wield are wicked swords and daggers, maces studded with spikes and blades, and (less often) strange weapons resembling a whip or flail. They seem to prefer close combat; if possible, a \bane{} will always strike its killing blow in melee. It is speculated that they \hr{Life drain}{drain life energy} from their opponents this way. \Banes{} are sometimes seen pulling weapons seemingly out of their bodies, including large weapons that it ought not be possible to conceal. If disarmed, a \bane{} will pull out a new weapon if it has one, or strike with its bare tentacles (forming claws at the end). A \bane{} will never pick up and use a \Miithian{} weapon, with the rare exception of some magical artifact. 
% 
% \Banes{} very much prefer to fight in close combat. All \banes{} are mages and can attack at a distance using magic, but they are reluctant to do so. They may do it in order to kill or subdue a fleeing victim if vitally necessary, but otherwise, \banes{} will only strike at range in order to defend themselves. They never wield ranged weapons such as bows. 
% 
% What senses do \banes{} have? They don't have smell and taste. Do they have vision and hearing? Do they have telekinetic and \hr{Telepathy}{telepathic} sense? Perhaps even more exotic senses? 
% 
% \Banes{} are somewhat resistant to heat/fire and cold attacks, and with their alien biology, they are immune to \Miithian{} poisons and diseases. Electricity and acid do full damage, as do attacks with physical weapons. They do not breathe, and as such, cannot be strangled or suffocate. 
% 
% 
% 
% 
% 
% 
% 
% 
% 
% \subsection{Biology}
% \Banes{} are from an alien world called \Erebos. Only a very few people from \Miith{} have ever visited \Erebos, so little is known of it. The Chronicler of Nom did not visit the \Baneworld{} (as far as is known), but he did communicate with \banes{} and learn something of it. He speaks of tremendous cities, both hideous and beautiful at the same time, of colossal monolithic towers and castles, and of massive armies of \banes{} and monsters fighting terrible wars, apparently against other \quo{nations} of \banes{} (nations presumably ruled by \banekings) as well as other monstrous races. He even speaks of dark temples where the \banes{} worship alien gods in grotesque rituals. 
% 
% \Banes{} do not eat, drink or breathe. It is unknown what they live on, if anything. It is believed that they \hr{Life drain}{drain life energy} from other creatures, for a summoned \bane{} will sometimes attack and kill living creatures, preferably intelligent creatures, for no apparent reason. 
% 
% \Banes{} do not seem to have genders. It is unknown how they reproduce, but the Book of Nom states that all \banes{} are born as \banespawn{} and slowly advance through the ranks. The Book seems to hint that the lower ranks of \banes{} (possibly only the Spawn) have finite life spans while the higher ranks are immortal. 
% 
% When slain, a \bane{} will dissipate into a pool of putrid \baneblood, leaving a dark and tainted stain of many square meters. On such a \bane{} \quo{tomb}, no wholesome vegetation will grow for decades (although it is sometimes infested by hideous, abnormal fungi and mosses), and there is a distinct atmosphere of unnatural evil. Seeing a \bane{} tomb has a Minor Horror Effect. 
% 
% 
% 
% 
% 
% 
% 
% 
% 
% \subsection{Psychology}
% Little is known about the \banes{}' mindset. They are alien creatures not like anything on \Miith{}. 
% 
% \Banes{} have no voices and cannot speak, but they can hear and understand speech.\footnote{Or can they? Maybe they can't hear, but use \hs{telepathy} to scan people's minds.} If a \bane{} needs to communicate, it will do so telepathically. \Banes{} encountered on \Miith{} are always able to understand one or more \Miithian{} languages, typically Kingstongue and/or Ancient Vaimon. It is believed that the majority of \banes{} know little to nothing of \Miith{}, and that those successfully summoned are always \quo{learned} \banes{} who have studied \Miith{} and its creatures. 
% 
% When summoned, a \bane{} must be bargained with. No magic is known for enslaving \banes{} and controlling them against their will, and attempts to do this typically result in the summoner dying a gruesome death. Typically, the \bane{} agrees to perform some service in exchange for a gift. Suitable gifts are live sacrifices (large numbers of intelligent creatures), arcane knowledge and magical items. Especially prized are items made to combat \banes{}. It is believed that the \banes{} covet these items partly to prevent others from using them against them, and partly to use them in their wars against fellow \banes{} on \Erebos. A \bane{} may also demand that the summoner perform some obscure magical ritual for it (typically refusing to reveal the purpose of it). Sometimes, a \bane{} will crave strange things in return, such as a large quantity of a particular mineral, or a specific person to be captured alive. 
% 
% Communicating with a \bane{} is a horrifying experience. People have reported that hearing the telepathic voice of a \bane{} inside your head feels like a terrible violation of the mind. One person described it as the equivalent of \quo{being pinned down, stripped naked and having someone carve their message in runes on your chest with a knife}. Such an experience will have a Moderate Horror Effect. That being said, \banes{} do not communicate much. They will not, for instance, comment, taunt nor threaten their opponents in combat. The only ones likely to ever hear the \ps{\bane}{} voice are the summoner and his allies. 
% 
% If asked questions, \banes{} will often refuse to reply. They have no recognizable concepts of politeness, so if asked a question it will or can not answer, the \bane{} will typically not explain but simple ignore it. Even if it deigns to answer, asking questions about the \banes{} themselves and their homeworld is risky business, as the \bane{} will tend to show rather than tell, telepathically showing images of the \baneworld{} and the \banes{}' life. Such an experience will have a Major to Extreme Horror Effect. (It should be noted that the \bane{} cannot use this as an attack. For the \bane{} to \quo{explain} something to you, you must be willing to accept the \quo{transfer}.) 
% 
% \Banes{} have no recognizable morals and principles. A \bane{} can be stealthy if ordered to, or has a reason to fear being discovered, but they will kill and destroy anyone and anything if they must, and sometimes will for no discernible reason. 
% 
% It is unknown how the \banes{} communicate among themselves. It might be telepathically, or using some strange senses that \Miithians do not have. Their own names cannot be translated into spoken languages. If it needs to communicate with \Miithian{} creatures, a \bane{} will sometimes adopt a spoken name for others to use, or allow its summoner to give it a name. Rissit Nechsain typically gives his \banes{} such names as Direfrost, Illwinter or Coldscar. 
% 
% \Banes{} wield great magical power, but it seems that they do not use it to its full effect. For example, they can attack at a distance using magic, but will not do so unless absolutely necessary. Some scholars believe that \bane{} magic is simply mana-expensive, so that they are not able to use it very often. Others believe that the \banes{} are bound by some alien code of \honour or religion (perhaps superstition) that restricts their actions. Yet others speculate that perhaps the \banes{} fear to use their magic because it is unnatural and frightening to them, like the way many \Miithian{} mages feel about their magic. 
% 
% 
% 
% 
% 
% 
% 
% 
% 
% \subsection{Habitat}
% \Banes{} may be found in any terrain. 
% 
% 
% 
% 
% 
% 
% 
% 
% 
% \subsection{Attributes}
% \begin{description}
%   \item[Horror effect:] 
%     Moderate to see the \bane{} close up, Minor if seen from a distance.
%     
%     Minor to see a \bane{} \quo{tomb} (see under Biology). 
%     
%     Moderate to hear the \ps{\bane}{} telepathic voice. 
%     
%     Major to Extreme to listen to a \bane{} explaining about the \baneworld. 
% \end{description} 
% \end{comment}















\section{\Flyingpolyps}
\target{Flying polyps}
\index{\flyingpolyp}
There dwelt flying polyps beneath \Nyx and \Erebos, where they burrowed and slithered.

They were some of the \hr{Horrors of the Void}{horrors which even the \resphain feared}. 
Even the \banes feared these monsters.

Compare them to:
\begin{itemize}
  \item The flying polyps from \cite{HPLovecraft:TheShadowOutofTime}. 
  \item Beholders and illithids from \cite{RPG:DungeonsandDragons}.
  \item The Tyrant Worms of my older ideas.
\end{itemize}











\subsection{Biology}

An idea is that the \noggyaleth{} feed on magical power, so \ps{\Teshrial} \noggyaleth{} are only truly formidable if the Sentinels show up with some great power. Or maybe they feed on \vertices. That might be evil. 

I need to think more on how exactly this works. 

They were asexual and reproduced by budding.





\subsubsection{Habitat}
The \flyingpolyps dwell in the depths below \Nyx. 
The \quo{bottom floor}. 
They slither and burrow in dimensions that exist parallel with the spires of \Nyx and gradually lead to the surface of \Erebos. 
(\hr{Nyx is above Erebos}{\Nyx exists in the skies high above \Erebos}.)
These dimensions are inaccessible and non-permeable to the \resphain and \banes. 
Only the \flyingpolyps can cross them due to their exceptional burrowing abilities. 

It is their kind that have carved out \Erebos, leaving only a mass of twisted spires and no ground. 
The also dug out \Nyx. And now they are in the process of undermining \Miith{}, turning it into a dead husk, an empty shell. 

In the mystic gloom of the deep abyss underneath \Nyx, you can sometimes hear the writhing of the horrible \hr{Ghobal}{\noggyaleth}, or feel the tremours of their passing and their burrowing.
Once in a rare while you can feel a tower faintly trembling. 
This happens when a \noggyal{} violently collides with the \CrystalSphere{}\dash the monster scrapes the edge of the Sphere but fails to penetrate. 















\section{\Noggyal}
\target{Noggyal}
\target{Ghobal}
\target{Ghobaleth}
\index{\noggyal}
The \noggyaleth{} were a race of amorphous giant monsters. 
They were native to \Miith, but related to the \banes. 









\subsection{Name}
Singular \emph{\noggyal{}}, plural \emph{\noggyaleth{}}. 

The name is inspired by the \Qliphah{} Golab, in \Cabbalah mysticism.









\subsection{Physique}
\Noggyaleth looked shoggoth-like:
An amorphous, ever-twisting mass of slime that only vaguely resembled flesh.
A \noggyal was flexible and chaotic and could reshape its body at will.
Often they would be covered with eyes and other sensory organs.
They had little individuality but could join together to increase their intelligence.

They could can create pseudopods/tentacles when needed, or even halfway dissolve their own bodies. 
This sometimes made them appear like a writhing mass of dozens of worms. 
Especially when seen through the Shroud. 

They stank hideously.
Their smell was alien and unnatural, but yet familiar in a way that was so deeply disturbing that no mortal could bear to consider it.

The \noggyaleth were great burrowers.
With their acidic secretions they could burrow holes and tunnels through the ground as well as through the barriers between the Realms.

\citeauthorbook{LinCarter:TheNecronomiconTheDeeTranslation}{Lin Carter}{
  The Necronomicon: The Dee Translation (part I.VII.III)
}{
  And then, at length, there flashed upon my vision one glimpse of a depth and of an abomination more horrible than any that I had glimped before. I looked upon a foul black pit, with a carven rim of beslimed rock about it, all drowned in Plutonian gloom, litten only by the vile phosphorescence of the primal white jelly of the proto-Shoggoths\ldots{} and amidst the hideous slime and the obscene stench I saw the bubbling, quivering plastic horrors, those shuddering towers of gelatinous, liquescent filth, studed with naked and protruding and staring eyeballs\ldots{} and I shrieked, and fled, back down the pathways of space and time and dimension, knowing in that last, soul-blighting glimpse the nodding flowers that blossomed in the scummed shallows of that lake of bubbling filth\dash{}\emph{and shrieked, and fled, knowing at last where the Black Lotus bloomed, and upon what unspeakable slime it feeds.}
}

Compare them to:
\begin{itemize}
  \item The shoggoths in \cite{HPLovecraft:AttheMountainsofMadness}.
  \item The Dholes/Bholes in \cite{HPLovecraft:TheDreamQuestofUnknownKadath}. 
  \item To a lesser extent, the sandworms of \cite{FrankHerbert:Dune}.
  \item The beholders from \cite{RPG:DungeonsandDragons}.
\end{itemize}
 









\subsection{Biology}
An idea is that the \noggyaleth{} feed on magical power, so \ps{\Teshrial} \noggyaleth{} are only truly formidable if the Sentinels show up with some great power. Or maybe they feed on \vertices. That might be evil. 

I need to think more on how exactly this works. 

They were asexual and reproduced by budding.





\subsubsection{Corrupting the planet}
\target{Noggyal corruption}
The \noggyaleth burrowed through the planet and attempted to gain control of it.
But it was hard for them on their own.
They were devious and cunning but \hr{Noggyaleth do not plan}{not cut out for long-term planning}.
In the time of the \ophidian empire, the \ophidians kept the \noggyaleth in check, so they accomplished nothing noteworthy. 
But even after the \ophidians had fallen, the \noggyaleth still achieved nothing noteworthy in their many thousands of years. 

When the \resphain came and allied with the \noggyaleth, things began to speed up. 

\citeauthorbook[p.287]{DavidDrake:ThanCursetheDarkness}{David Drake}{%
  Than Curse the Darkness%
}{%
  \ta{Maybe they aren't gods at all, him and the others\ldots it and the others Alhazred wrote of.
  Call them cancers, spewed down on earth ages ago.
  Not life, surely, not even \emph{things}\dash but able to shape, to misshape things into a semblance of life and to grow and to grow and to grow.} 
  
  \ldots
  
  \ta{Into this earth, this very planet, if unchecked.}
  
  \ldots
  
  \ta{Not \quo{rule} the world,} she corrected.
  \ta{Rather \emph{become} the world.
  This thing, this seed awakened in the jungle by the actions of men more depraved and foolish than I can easily believe\ldots this existsence, unchecked, would permeate out world like mould through a loaf of bread, until the very planet became a ball of viscid slime hurtling around the sun and stretching tentacles towards Mars.}
}





\subsubsection{Mother-mass}
\target{mother-mass}
There existed a great \quo{mother-mass}. 
It was the very first proto-\noggyal from which all others had budded. 
The mother-mass was the primordial slime from which all \Miithian{} life descended. 

Compare it to Ubbo-Sathla, a Great Old One from the Cthulhu Mythos. 
Ubbo-Sathla is the shoggoth mother-mass.
Also compare it to Abhoth from \cite{ClarkAshtonSmith:TheSevenGeases}. 

The mother-mass \hr{Noggyaleth were the first life}{was the first life on \Miith}. 
It was \hr{Voyagers create mother-mass}{tampered with by the \voyagers}, but it predated them. 

The mother-mass was not a fully physical thing. 
It was an insubstantial, formless, twisting mass. 
It was completely mindless. 
Physical \noggyaleth would sometimes bud from it. 

\hr{Sethicus}{\Sethicus} knew of the mother-mass.
In \hr{Sethican philosophy}{his mysticism}, he considered it to be a thing of \DaathKurZulNathla, the deepest plane of primal chaos. 









\subsection{History}





\subsubsection{Origin}
\target{Noggyaleth were the first life}
The \noggyaleth were native to \Miith.
The \noggyal \hs{mother-mass} was the very first life on \Miith. 
The \noggyaleth were truly ancient creatures, older than the \ophidians.

The mother-mass was the primordial slime from which all \Miithian{} life descended. 

The \noggyaleth were deeply and tightly connected to the Heart of \Miith, being the first life.
All other life, even \ophidians, somehow descended from the horrid \noggyaleth.

They embodied creative chaos, but not intelligence and planning.

The \noggyaleth might have been kin to the \xss.
Some \draconic myths described them as the spawn of \RuinSatha or \KyaethemChreiAz, but this is very vague and uncertain.





\subsubsection{\Voyager Age}
\target{Voyagers train Noggyaleth}
When \hr{Voyagers come to Miith}{the \voyagers came to \Miith}, they took the \noggyaleth and altered them to make a race of intelligent, powerful, versatile slaves.
They were originally completely mindless, but the \voyagers gave them intelligence.

The \voyagers trained the \noggyaleth and taught them many skills of the mind and body. 
In a sense, the \voyagers taught the \noggyaleth to be \hs{living machines}:
Stronger, more resilient and more versatile than any machine of metal. 

Perhaps the \noggyaleth rose up against the \voyagers and drove them out.
Or perhaps the \voyagers were driven out by the \xss and the \noggyaleth just remained.





\subsubsection{\Ophidian Age}
The \ophidians knew that the \noggyaleth existed. 
They did not know the \noggyaleth's true nature, although a few occultists suspected.
The \ophidians feared the \noggyaleth just as they did the \xss, but the \ophidian magic was mostly powerful enough to keep the chaotic, bestial \noggyaleth at bay or even destroy them.
In earlier days, \hr{Ophidian-Noggyal wars}{the \ophidians waged great wars against the \noggyaleth} and drove them underground.





\subsubsection{Role}
The \noggyaleth{} dwell in the depths below \Nyx. 
The \quo{bottom floor}. 
They slither and burrow in dimensions that exist parallel with the spires of \Nyx{} and gradually lead to the surface of \Erebos. 
(\hr{Nyx is above Erebos}{\Nyx{} exists in the skies high above \Erebos}.)
These dimensions are inaccessible and non-permeable to the \resphain{} and \banes. 
Only the \noggyaleth{} can cross them due to their exceptional burrowing abilities. 

It is their kind that have carved out \Erebos, leaving only a mass of twisted spires and no ground. 
The also dug out \Nyx. And now they are in the process of undermining \Miith{}, turning it into a dead husk, an empty shell. 

In the mystic gloom of the deep abyss underneath \Nyx, you can sometimes hear the writhing of the horrible \hr{Ghobal}{\noggyaleth}, or feel the tremours of their passing and their burrowing.
Once in a rare while you can feel a tower faintly trembling. 
This happens when a \noggyal{} violently collides with the \CrystalSphere{}\dash the monster scrapes the edge of the Sphere but fails to penetrate. 





\subsubsection{Burrowing through the Shroud}
The Cabal used \noggyaleth.
They were useful because they could drill through the Shroud. 

By burrowing through the crust of \Miith the \noggyaleth made the planet's barriers unstable. 
They worked to slowly undermine the \CrystalSphere and open a back door from \Miith to \Erebos.





\subsubsection{In \Malcur}
\ps{\Teshrial} monsters in \Malcur (see section \ref{Teshrial's creatures}) are \noggyaleth. 









\subsection{Politics}





\subsubsection{\Banes}
\target{Banes and Noggyaleth}
\Banes were related to \noggyaleth.
When the \voyagers created the \banes, they used some \noggyal matter to make them.
The \banekings learned of the \noggyaleth's existence.
They realized that there lay the path to their future.
If the two races could assimilate each other, they would be greater than their creators the \voyagers in all things.
They would possess the best of both worlds: 
Cold intelligence and wild creative chaos.
When the \banes destroyed the \voyagers, they quickly made plans to go to \Miith and unite with the \noggyaleth.
But the \ophidians, who were the heirs to the \voyagers on \Miith, would have none of that.

Individual \banes could not just merge with individual \noggyaleth.
The \baneking himself must come to \Miith and absorb into himself the great \hs{mother-mass} of the \noggyaleth, which lay seething and bubbling deep beneath the planet's surface, feeding directly on the Heart.
In a sense, the \noggyal mother-mass was an extension of the heart.





\subsubsection{\Resphain}
\target{Resphain and Noggyaleth}
The \resphain employed \noggyaleth, but they did not understand them.
They thought of the \noggyaleth as bestial, semi-intelligent things, tamed and subdued with the power of spells.
They did not suspect the true extent of the \noggyaleth's intelligence and power, nor their ties to the \banelords' long-term plan.
They assumed they had achieved power over the \noggyaleth, but in truth the \noggyaleth had more freedom and plans of their own than the \resphain realized.









\subsection{Psychology}

The \resphain believed that \noggyaleth were mindless things that existed only to crawl, burrow and feed.

In reality, the \noggyaleth were more intelligent than that. 

\target{Noggyaleth do not plan}
They were devious and cunning but not cut out for long-term planning.
The \voyagers who designed them had seen to that. 

\citebandsong{Nile:Ithyphallic}{Nile}{
  Eat of the Dead
}{
  The highest fulfillment of man\\
  Is to become food for the crawling things\\
  That burrow and slither in human flesh\\
  Unceasing in mindless hunger\\
  Remorseless undefiled by reason\\
  The worms of the tomb they are pure

  Their purity elevates them\\
  Above the putrefying pride of our race

  The destiny of man is\\
  Merely to be\\
  The nourishment of the worm\\
  Yet their excrement bestows higher wisdom

  From decay arises new life\\
  Fill myself with that which rots\\
  And I shall be reborn

  By writhing upon my belly like a mindless worm\\
  I shall rise up in awareness of truth\\
  I gnaw upon my own decaying flesh\\
  And my mind is forever purged\\
  Of the corruption of faith
}


















\section{\Ophan}
\target{Ophan}
\index{\ophan}
\Ophanim were creatures that lived in \Nyx.
They were actually creations of the \banes that had migrated to \Nyx.
Here they came to be enslaved by the \resphain.
They were living chariots with wheels and lots of eyes.



















\section{\Screamer}
\target{Screamer}
\index{\Screamer}
\Screamers were a kind of \banes. 
Their forms were designed to better combat \dragons. 









\subsection{Biology}
\Screamers were regular \banes reshaped into special forms. 









\subsection{Physique}
A \screamer looked kind of like the Xenomorphs from \emph{Alien} and \emph{Alien versus Predator}. 
They had wings and could fly, like Zerg Mutalisks from \emph{Starcraft}. 
And great blades, like Zerg Hydralisks from \emph{Starcraft}. 
They were extremely fast and agile, built to evade a \dragon's melee and spell attacks.









\subsection{Psychology} 
\Screamers possessed humanoid intelligence, but they did not learn spells. 
They were supported by \lesserbane, \baneknight or \banelord sorcerers. 



















\section{Thorn Angel}
\target{Thorn Angel}
\index{Thorn Angel}
A race of creatures, probably not native to \Miith{}. 
There exist only a score or so of them. 
They form a single band. 
Their leader is \Hiothrex{}, and they all serve the Imetrium. 















\section{\Umbra}
\target{Umbra}
\index{\umbra}
A monster native to \Erebos. 









\subsection{Biology}






\subsubsection{Reproduction}
\target{Umbra reproduction}
No one knows how, or even if, \umbrae{} reproduce. 
It is unknown if they reproduce \hr{Wild Umbrae}{in the \Wylde{} on \Miith}, or if all \umbrae{} hail directly from \Erebos. 





\subsubsection{Soul-eating}
\target{Umbrae eat souls}
\Umbrae{} feed on lifeforce. 
They can \hr{soul-eating}{eat the souls} of other beings\ldots{} and they are frighteningly good at it. 
They have an easier way of overpowering a soul and devouring it than other beings know of. 

It was from the \umbrae{} that the \Merkyran{} rebels \hr{Rebels learn soul-eating from Umbrae}{reverse-engineered the technique} of eating souls. 









\subsection{Habitat}





\subsubsection{In the \Wylde}
\target{Wild Umbrae}
There live \umbrae{} in the \Wylde{} on \Miith. 
These are all beasts that were brought to \Miith{} by the \resphain{} or \banes{} but broke control and escaped. 
Or perhaps the descendants of such runaways, \hr{Umbra reproduction}{if that is possible}. 

In the \wylde in \Azmith one can occasionally see giant \umbrae soaring high above. 
Amorphous fearful shapes that cast a vast, ominous shadows.
Hence the name \quo{\umbra}. 









\subsection{History}





\subsubsection{Origin and relationship with \banes}
\target{Umbra origin}
The \umbrae{} originate from \Erebos{} where they prey on \banes. 
The \banes{} fight them, but they cannot destroy and exterminate them, for they races are tied to each other: 
The \umbrae{} are born from \banes{} that fail and mutate.

See, the \ps{\banes}{} \matrixx{} is tied to \FatherErebos{} and his power. 
The \umbrae, in turn, are an integral part of the \bane{} \matrixx. 
In a sense, the scourge of the \umbrae{} is a curse upon the \banes{} from \FatherErebos{} as a punishment for the \ps{\banes}{} betrayal of their homeworld. 
They are born enemies of the \banes{}, created by \quo{natural} processes to fight the \bane{} overpopulation (\hr{Umbra menace growing}{and later the \Merkyran{} overpopulation}). 





\subsubsection{Preying on \resphain}
Then, when \Nyx{} was created, the \umbrae{} came there. 
They soon began feeding on the \resphain, who had powerful souls and tasted like \banes. 
The \hr{Merkyrans fear Umbrae}{\Merkyrans{} feared them}. 









\subsection{Name}
Singular \emph{\umbra{}}, plural \emph{\umbrae{}}. 









\subsection{Physique}
\Umbrae are very much inspired by the picture \cite{Picture:GunnerRomantic:BermudaTaowls}. 

An \umbra{} looks somewhat like a huge, black, flying manta ray or bat. 
Its surface is hazy and unclear, as if it exudes a a cloud of smoky darkness. 

\target{Umbra like bat}
They are capable of hanging completely still in the air.
They flap their wings very slowly and casually when at all. 
Their wide, flat body is thicker in the middle. 
They have a mouth on the front end with four jaws. 
These jaws can unfold to grasp and latch onto a victim, like the vampires in the movie \cite{Movie:BladeII}. 

They are not entirely black. 
Blotches of moldy yellow can be dimly seen through the cloud of black and gray smoke. 

\Umbrae can grow to humongous size.
Larger than \dragons (albeit less powerful). 
The main body can be up to 10 metres long. 
Total length and wingspan can be as much as five or six times the body length. 
Only rare \umbrae grow to such huge size, though. 

An \umbra{} has a long, strong, whiplike tail. 
The manta-like taper to whip-thin appendages. 
The tail and whip-ends are razor-sharp and can be used as weapons in combat. 

They also have various other appendages on their back and belly. 
These are short and of obscure purpose. 

They have no separate \quo{back} and \quo{belly}. 
They can flip over at will. 

Like \banes, \umbrae{} have no eyes or other readily identifiable sensory organs. 

They have more than one vast mouth near the front. 
Sometimes when they open their mouths, pale gray ghostly light shines out.
Perhaps they have a breath weapon, like Godzilla's energy beam breath from the Millennium \emph{Godzilla} movies. 

\Umbrae{} fight by slashing with their razor-sharp wings, their mouths and their tails. 





\subsubsection{Dislike light}
\target{Umbrae dislike light}
\Umbrae disliked light. 
They would be less inclined to attack a brightly lit place. 
In Realms with day and night, \umbrae were sluggish in daylight and preferred to attack at night. 
An \umbra could still fight at full power in daylight if it had to, though. 





\subsubsection{Sounds}
\target{Umbra sounds}
\Umbrae{} emitted some very deep-pitched, mournful-sounding, droning howls. 
But loud. 
These sounds were felt (as vibrations) rather than heard. 









\subsection{Skills and powers}





\subsubsection{Cause agoraphobia}
An idea might be to have some character who develops agoraphobia (fear of open spaces) after being attacked by \hr{Umbra}{\umbrae} that swoop down from the sky. 
Possibly a \resphan{} in \Merkyrah. 

\lyricstitle{\emph{Call of Cthulhu RPG} p.50}{
  Agoraphobia: 
  \ta{%
    The sky is so wide, so heavy, so massive. It spreads into infinity with stars and clouds held up by who knows what. Monsters come from sky and space.}
}





\subsubsection{Power and danger}
\target{Umbra power}
The \resphain{} tame the \umbrae{} and use them as beasts of war to great effect. 
\Umbrae{} are very powerful. 
Three \resphain{} on \umbrae{} are a serious threat to even a \dragon{} (\hr{Dragons vs Resphain in power}{normally it takes ten \resphain{} to bring down a \dragon}).

\Umbrae{} are alien monsters from \Erebos{} and must be controlled with occult, incomprehensible \bane{} magic. 
The \resphain{} utilize these spells, but they do not understand them. 
They only know that the spells work (most of the time), but not how or why. 
But this lack of understanding makes the control crude and brittle, liable to break. 
And \umbrae{} are vicious and aggressive. 
Often an \umbra{} has broken control and killed the \resphan{} who tried to ride it (\hr{Umbrae eat souls}{sometimes even permanently})\dash even a moment of broken control is enough for the deadly \umbra{} to kill its handler. 

For this reason, the \resphain{} fear them and do not use them in combat as much as they otherwise could. 
The \umbrae{} are a terrific weapon, but they are too dangerous. 

If an \umbra{} breaks control completely, it \emph{cannot} be brought back into the fold. 
It will escape \hr{Wild Umbrae}{into the \Wylde}. 















\section{Weaver}
\target{Weaver}
\index{Weaver}
%An enormous monster, the shape of a dark cloud with a number of legs or tentacles. Looks vaguely like a giant spider. 

A Weaver is an alien monster from the \baneworld. They are enormous and extremely dangerous monsters, feared even by \dragons{}. 

%It looks like a great cloud of dark fog sprouting a number of legs and tentacles. Some find that a Weaver looks somewhat like a bloated giant spider. 

%\subsection{Name}
%\emph{Weaver} is English. There is no associated adjective. 
%As in English. 









\subsection{Physique}
A Weaver looks like a great cloud of dark fog sprouting a number of legs and tentacles. Some find that a Weaver looks somewhat like a bloated giant spider. 

They are huge in size, easily reaching 10 meters in diameter. Small Weavers down to 4 meters have been encountered. The average Weaver in Threll will be 7-8 meters in diameter, up to 10 at most. Larger one have been sighted in Nom. There are reliable reports of Weavers up to 30-40 meters in diameter, and a few explorers tell of behemoths growing as large as 100 meters. They are quite massive; a 10 meter Weaver weighs an estimated 20 tons, and the huge ones may weigh hundreds of tons. 

Weavers cannot fly or jump; they can only crawl, but they do this rather fast (slightly faster than a running \human{}). They are roughly symmetrical in all directions (having no front and back) and can move in any direction. 

A Weaver's central body is not visible because the creature excretes a cloud of opaque gray gas. This gas is somewhat lighter than air and constantly rises up from the creature like smoke; presumably exhaled as the Weaver breathes. The gas is poisonous if inhaled. 

A Weaver gives off a horrible stench, the reek of an unnatural and loathsome abomination. It can be smelled over a hundred meters away (depending on its size). The monster constantly emits hissing and groaning noises (believed to be incidental). 

Only the legs and tentacles are visible beyond the cloud of gas. The legs are segmented and somewhat spider- or insect-like, rhe tentacles soft and flexible and up to 20 meters long. Leg length up to is about $\fracs{1}{3}$ of the diameter, tentacle length is up to about $\fracs{3}{2}$ times the diameter. A Weaver has about 10-15 legs and 20-30 tentacles. The tentacles frequently sprout smaller sub-tentacles along the way. Some of the tentacles end in a sharp sword-like claw (these can be long, up to $\fracs{1}{10}$ diameter). The creature is covered in very tough, leathery skin with small spikes (centimeters in length) irregularly jutting out. The legs are more heavily \armoured than the tentacles. 

A strong wind (natural or artificial) may blow away the dark fog to reveal the body inside. It will become apparent that there is no central hub, only a mass of tentacles splitting off from and rejoining each other. The tentacles are much thicker near the centre of the creature (up to $\fracs{1}{5}$ of diameter). 

Scattered about on the creature's body are dozens of mouth-like orifices. The size of these holes varies from few centimeters to over a meter near the centre of the monster. The dark gas rises out of these holes, and it will shove captured victims into them. Once inside the Weaver, a victim will be slowly digested. This is a slow process, however; a swallowed victim is more likely to die from breathing the poisonous gas, or from suffocation. A swallowed victim can be rescued if (a part of) the monster is sliced open. 

In combat, a Weaver's primary weapon is to lash out with its tentacles. It will attempt to grab victims, pull them closer to its body, grab them with more tentacles, then pull them apart while slashing them with their claws. If the victims (or the parts of dismembered victims) are small enough, it will attempt to swallow them. The Weaver can also use its legs as bludgeoning weapons to kick or stomp. 

If the Weaver cannot easily grab its opponents with its tentacles, it will shoot its \quo{web} at them. This \quo{web}, from which the Weavers derive their name, is a mass of long strands of tough, sticky, pale white material. Each strand is 1-3 cm thick, and the Weaver can fire a score of them at a time towards the same target. The strands are fired from small holes in the tentacles (different from the mouth openings described above), and they have been known to fire them at ranges of 50 meters or more. The web is nearly impossible to tear, but it can be cut or burnt, and great cold causes it to freeze and shatter. A Weaver has a large supply of the web, and will generally never run out of it during a single battle. 

Weavers have some resistance to magic. Spells cast directly upon the creature tend to be absorbed and dispelled. Spells cast indirectly upon the monster (like a fireball) work normally. 

Slaying a Weaver is a monumental task. Once slain, the creature's body will thrash and convulse for a minute or so, then lay still. It will also exhaling the dark gas, but it still smells. After some hours, its flesh will dry out and crumble away. If the creature is cut up, a lot of web can be gathered (up to $\frac{1}{200}$ of the creature's weight). 









\subsection{Biology}
Almost nothing is known of the Weavers' biology. Presumably they eat the flesh of creatures they swallow. It is unknown how (or whether) they reproduce. It is assumed that the larger Weavers are older, but this is conjecture. 









\subsection{Psychology}
It is unknown how intelligent Weavers are. No one has ever successfully communicated with one, and they rarely show signs of intelligent behaviour, but some believe that they are extremely intelligent. Attempts to communicate with them using \hs{telepathy} usually result in serious damage to the telepath's mental health. 

Weavers are always aggressive when encountered and will attack pretty much every creature they detect. If its prey flees, the Weaver will pursue for a short time, but it will not move far from its original position. The Weaver itself will never flee, however; if pressed, it will fight to the death. 

Weavers are solitary; no one has encountered more than one at a time. It is unknown how and why they avoid each other and what would happen if two were to meet, but it is believed that each Weaver maintains a territory (this also explains why they rarely move far from the place they are encountered). 

\Banes{} greatly fear Weavers and will flee before them. 

Rarely, \banes{} may be encountered together with a Weaver, sometimes even hunting together. Such \banes{} can often be seen casting spells and performing strange rituals involving the Weaver. It is believed that these are religious rituals and that the renegade \banes{} serve and worship the Weaver. %These may be compulsion spells to keep the Weaver under their control, but it is also possible that the rites are of a religious nature and that the renegade \banes{} serve and worship the Weaver. 









\subsection{Habitat}
Not native to \Miith{}, the Weavers originate from \Erebos. Most likely, the Weavers on \Miith{} were summoned during the \banewar. On \Miith{}, they are known to exist only in Nom and Threll. All weavers encountered in Threll have been fairly small, no larger than 10 meters. In Nom, enormous specimens have been encountered. 

Where do they dwell? In caves or out in the open? In valleys or in mountains? 

The total number of Weavers is unknown, but Threll is estimated to home between 100 and 1000 of them. Nom is huge and unmapped, so no one can reliably estimate the number of Weavers there. 







\subsection{Weaver web}
\target{Weaver web}
\index{Weaver!Weaver web}
Weaver web may be gathered, during a fight or when the Weaver is dead. It dries in a couple of minutes, after which it is no longer (very) sticky. 

The chief use of Weaver web is that \banes{} fear it. \Banes{} seem to have a superstitious fear of Weavers and will flee from anything that smells like a Weaver, including web. One kg of web can be smelled by the \banes{} up to 30 meters away, and they will be hesitant to approach any closer than that. (Only \banespawn{} and \lesserbanes{} fall for this. \Banelords{} are not fooled. If led by a \banelord{}, \lesserbanes{} will attack, but will be at a disadvantage because they still fear the web.) 

The web dries up completely in a matter of days if left unprotected. Wrapped up and sealed, it can last up to a month, and preservation spells are known that will make it last several months. The older it is, the less it smells, and the less effective it is. When it dries up, it stops smelling and crumbles. 

The chief disadvantage of the Weaver web is that while it repels \banes{}, it also repels everyone else. The stench of it is loathsome and sickening, and animals and humanoids instinctively hate it. The easiest way to destroy the web is to burn it (it will produce some awful brown smoke, but then it will be gone).









\subsection{Weaver maggots}
\target{Weaver maggots}
\index{Weaver!Weaver maggots}
Weaver maggots are small creatures that crawl around on a Weaver and inside its mouths. They are gray, wormlike, 20-50 cm long (with thickness at the middle equal to about $\fracs{1}{3}$ of the length) and weigh 1-4 kg. They have dozens of small lumps scattered over their bodies which they use for slowly crawling about. A maggot has no top and bottom, it can crawl on any side. It has a front end with several (3-7) toothless mouths. It has tough, leathery skin covered by a thin layer of sticky slime, like a slug. They also exhale the dark gas, but in much smaller amounts, too little to cause more than coughing and irritation. 

The maggots are harmless, although they will try to eat organic matter that doesn't move. They look and smell revolting, however, so many people will feel a strong dislike of them and an urge to kill them. Killing them is easy; they can be crushed or cut in half. They can also be easily captured. There is no known use for them, and they rarely live more than a week in captivity, but nevertheless, researchers are sometimes willing to buy them to study. 

\index{cannibalism!Weaver maggots}
Every Weaver has plenty of maggots crawling about on it, 6-10 of them per ton of weight. When the Weaver is killed, the maggots will immediately begin to eat the flesh of their host. They are cannibalistic and will eat other dead maggots. Presumably, the maggots are a symbiotic or parasitic species, or possibly immature Weavers. 















\section{Wingworm}
Wingworms are loathsome monsters from \Erebos resembling flying worms (hence the name). They are unintelligent beasts and the \banes{} sometimes use them as mounts. 

%\subsection{Name}
%Singular \emph{}








\subsection{Physique}
A Wingworm looks like a huge worm. 























\chapter{The \Resphain}

















\section{\Baelzerach}
\target{BZ}
\target{Baelzerach}
\target{Bael'zerach}
\target{Bael'Zerach}
The \daemonic{} \resphain{} who dwell in \Machai{}. 
They have severed all ties to the \banes. 
\Baelzerach{} is called a dynasty, but this is a bit of a misnomer, since they have no central organization and are split into a number of independent tribes. 

Sometimes they wage war against the \dragons, sometimes they work together. 

The \Baelzerach{} have reshaped their bodies into \daemonic{} forms, with fiery red skin, horns, wings, tails and/or goat-like legs. 
The \resviel{} are more \human-looking than their males, tho. 

\Ishna{} played a part in the founding of \Baelzerach{} and has a place in their mythology as a legendary figure, half hero and half Devil. 

The \Baelzerach{} enjoy to hunt, kill and eat in the \draconic{} manner (see section \ref{Draconic diet}), rather than \hr{Resphan diet}{eating submissive slaves like other \resphain{} do}. 
This is one of the things that causes other \resphain{} to label them as barbarians. 

\Baelzerach{} alone of the great dynasties has no \satharioth{} among its numbers, since they branched off from \KiriathSepher{} before \hr{Origin of Satharioth}{the \satharioth{} were created}. 
They do have \ketherain, since there has been interbreeding among the dynasties, but they are usually not referred to as such among the \Baelzerach, who reject the notion of a \KiriathSepher{}-based aristocracy. 









\subsection{History}





\subsubsection{Absorbing tribes}
\target{Bael'Zerach absorbs tribes}
After the \hs{Murder of the Dawn}, some \Baelzerach{} rebels left the other dynasties and went out to join the \hr{Early Resphan tribes}{early \resphan{} tribes}. 
They mixed with the tribes and merged with them. 
Later all these tribes came to be considered \Baelzerach{} by the dynasties. 









\subsection{Politics}





\subsubsection{Dark gods}
\target{Bael'Zerach diabolism}
Some \Baelzerach{} worshipped the \hr{Gods in Nyx}{\xss{} and other dark gods that dwelt in \Nyx}, just like \hr{Early Resphan diabolism}{other \resphan{} tribes had done before them}. 















\section{\CiriathSepher}
\target{Ciriath-Sepher}
\target{CS}
\index{\CiriathSepher}
\KiriathSepher{} was the oldest and most traditional of the \resphan{} factions. 
They were loyal to the \banes. 

\target{High Lord of Kiriath-Sepher}
%They are ruled by a High Lord (like a king). 
The first (and only known) High Lord of \KiriathSepher{} was \hr{Azraid}{\Azraid}. 









\subsection{Aesthetics}
The \CiriathSepher{} traditionally prefer to dress in bright \colours: 
White, yellow, orange, silver, gold, bronze. 

They are the \quo{lightest} of the dynasties. 
They were the ones who devised the idea of marketing \Iquin{} as \quo{the Light}. 
(The \Kezeradi{}, who created \Iquin, were aesthetically darker, and they never called their \dweomer{} \quo{the Light}.) 

%They typically dress in white, silver and golden \colours. 
The other factions criticize them for having adopted too much of the \Merkyran{} imagery and aesthetics, and perhaps even their philosophy. This is a result of \hr{Azraid adopts Merkyran imagery}{\ps{\Azraid}{} policy}.





\subsubsection{The Silver Starlight Rose}
\target{CS symbolism}
One of the symbols of \CiriathSepher{} is the Silver Starlight Rose. 
It is a stylized rose-like flower shining silver like a star. 
Its thorns are daggers and swords. 

The Rose has symbolic value (for \hr{Mystraacht symbolism}{unlike \Mystraacht}, the \CiriathSepher{} \emph{do} believe in symbolism). 
It symbolizes beauty and strength; sophistication and art, but also martial power. 









\subsection{Culture}
The \CiriathSepher{} were the most social, diplomatic and extroverted of the dynasties (albeit not necessarily the most free-thinking). 





\subsubsection{The Dance}
\target{Dance}
\index{Dance, the}
\quo{The Dance} was the \CiriathSepher{} name for their delicate and intricate system of etiquette. 
It permeated (and very nearly governed) the life of every \resphan of \CiriathSepher{}. 





\subsubsection{Education}
\target{Ciriath-Sepher education}
\CiriathSepher{} had one central school where all children were educated. 
After all, there were never many children at a time. 

\Mystraacht, on the other hand, \hr{Mystraacht education}{raised each child separately}. 






\subsubsection{Government}
\CiriathSepher{} was sort of a constitutional monarchy. 
The High Lord had a lot of power, but the system ran fine without him, and he could not just decide everything. 
There was a \quo{council} or some such that could override him in many cases. 
They could not depose him, though, nor override his choice of successor. 

\target{CS order of succession}
\hr{Azraid}{\Azraid} made sure that \CiriathSepher{} had a very clear order of succession in case he were to die. 
That way, outsiders knew that \CiriathSepher{} would not be plunged into chaos if someone were to assassinate \Azraid, and insiders knew that \emph{they} would not be able to grab the throne by killing \Azraid. 

People also knew that \Azraid{} was not indispensable. 
After all, it was \hr{Daggerrain}{\Daggerrain} and not he who was the supreme leader of the Cabal. 

Among other things, this had the advantage that \Azraid{} could relatively easily meet with outsiders (even Sentinels) to negotiate. 
They knew that trying to assassinate him was perhaps not worth the effort. 

Furthermore, {\Azraid} purposefully \hr{Azraid obscure}{kept his role obscure}. 

The first in line to inherit \ps{\Azraid} throne was \hr{Harbeth}{\Harbeth}\ldots{} and \hr{Harbeth is Azraid's heir}{no one wanted that to happen}. 
\hr{Zereth}{\Zereth} was one of the next in line. 

\target{CS heirs die}
Not everything turned out according to \ps{\Azraid} \trope{XanatosRoulette}{Xanatos Roulette}. 
The various heirs got killed and destroyed during \SentinelsofMithEmph. 
Among other things, \hr{Harbeth dies}{\Harbeth{} died}. 

\Azraid{} was not happy. 
This endangered the future of \CiriathSepher. 
But \Azraid{} continued with his plan anyway. 
He was perfectly willing to sacrifice \CiriathSepher{} to save \resphan-kind. 

Besides, \hr{Azraid protects Ramiel}{he had Ramiel}. 





\subsubsection{Philosophy}
The \KiriathSepher{} are very lawful and concerned with \honour, etiquette (the \quo{\hs{Dance}}), tradition and principles. 
They see themselves as a bastion of righteousness and civilization. 
This puts them in stark contrast to the \Mystraacht. 





\subsubsection{Taboos around the newly-revived}
In \CiriathSepher, it was considered taboo to look on a recently dead and newly-revived \resphan.
Only close friends and family members were allowed to see the revived until he had healed. 

This was because the \CiriathSepher placed so great emphasis on appearances, propriety and dignity. 
It is comparable to the nudity and sex taboos in many RL cultures. 









\subsection{Politics}





\subsubsection{\Azraid}
See the section about \hr{Azraid and Ciriath-Sepher}{\Azraid and \CiriathSepher}.















\section{\Kezerad}
\target{Kezerad}
\target{Kezeradi}
A \resphan{} faction who, out of moral scruples, defied the \banes{} and created their own kingdom, supposedly one based on justice and good. 









\subsection{Aesthetics}
Their traditional \colours are gold and bronze. 
Or are they? 
\Kezerad{} shouldn't be too bright. 
Remember, I am trying to subvert the \quo{light is good, dark is evil} trope. 







\subsection{Arsenal}
\subsubsection{Telepathy}
\target{Kezeradi telepathy}
Back in the day, the \Kezeradi{} shared a \hr{Telepathy}{telepathic} bond and were all intimately bound to each other. They even shared this bond, to some extent, with their \human{} and \nephilic{} subjects. This empathy is one of the reasons why \Kezerad{} was such an enlightened realm. 

This could also be a weakness in war, since the bond meant that they shared pain, so the massacre of one \Kezeradi{} village or city would create a mental backlash that could demoralize the rest. (Compare with the destruction of the planet Alderaan in \emph{Star Wars IV: A New Hope} and Obi-Wan Kenobi's remark that: \ta{It is as if a million voices suddenly cried out in pain, and then fell silent.})

And, even worse, when the \Kezeradi{} lords were captured and transformed into the horrid \Sephiroth, the bond persisted, giving the \Sephiroth{} great mental power over the remaining people of \Kezerad. The invaders had planned this well indeed. 

After the fall of the \Kezeradi{} civilization, the conquerors utilized the psychic bond to hunt down the remaining \Kezeradi{} and eradicate them. In order to survive, those who escaped had to deaden their telepathy and block out all empathic feelings from their minds. 
(Compare this with the Protoss Dark Templar from the game \cite{VideoGame:Starcraft}, who allegedly severed their nerve endings to forever off themselves off from the telepathy of the Protoss race.) 
This gradually turned the survivors into bitter husks, desperately longing for intimacy but knowing that to give in to feelings is to be destroyed. 

%The \Kezeradi{} still have a deep connection to the \Sephiroth, since the \Sephiroth{} are each forged around a core of a \Kezeradi{} soul. See, back in the day, the \Kezeradi{} shared a telepathic bond and were all intimately bound to each other. 







\subsection{History}
\subsubsection{Inception}
\target{Founding of Kezerad}
\Kezerad{} was formed shortly after the \Merkyran{} rebellion by \resphain{} disgruntled with \ps{\Azraid}{} methods, and \ps{\Zachirah}{} even more. 

They acknowledged that \Merkyrah{} was bad, but felt that this new thing they had become was worse still. 
They wanted to learn from both ways and build a better society. 
They wanted to be good again, not evil. 

\citebandsong{DeathspellOmega:FasIteMaledictiinIgnemAeternum}{%
  Deathspell Omega
}{
  The Repellent Scars of Abandon and Election
}{
  Nothing of what man can know, to this end, \\
  could be evaded without degradation, without sin. \\
  Is it no burden to bear \\
  the repellent scars of abandon, of election?\\
  It leaves but a state of supplication and deserted expanses, \\
  an absorption into despair.
}

Even if this new way may be \quo{true}, even if it is the \quo{purpose} for which the \resphain{} were created, it is still morally wrong. 

\citebandsong{DeathspellOmega:FasIteMaledictiinIgnemAeternum}{%
  Deathspell Omega
}{
  The Repellent Scars of Abandon and Election
}{
  The existence of things cannot enclose \\
  the death which it brings to me.
}

\Sithiyacaan, gripped by \hr{Curse}{\NexagglachelsCurse}, began to feel self-destructive. 

\citebandsong{DeathspellOmega:FasIteMaledictiinIgnemAeternum}{%
  Deathspell Omega
}{
  The Repellent Scars of Abandon and Election
}{
  The existence is itself projected into my death, \\
  and it is my death which encloses it. \\
  Am I deranged?
}

They split off from the Cabalist dynasties and built a beautiful, happy, almost utopian kingdom of \resphain, \humans, \nephilim{} and possibly other creatures. 
They created their own pure and good \dweomer{}, \iquin, which was less dependent on \Erebos{} and instead drew upon the natural energy of \Miith{} and its Heart. 
Perhaps they had help from elder \ophidians{} and/or other wise creatures in establishing this. 

The foundation of \Kezerad{} must have been \emph{after} the birth of the \satharioth, but \emph{before} the \hr{Malach project}{\Malach{} project}. 
See, \hr{Eryal}{\Eryal} was one of the founders of \Kezerad{} after \hr{Shiaraid and Eryal driven apart}{her falling-out with \Shiaraid} but before they both became \malachim. 





\subsubsection{The fall of \Kezerad}
\Kezerad{} \hr{Fall of Kezerad}{was destroyed}. 





\subsubsection{\Kezeradi{} today}
There are surviving \Kezeradi. They are in an uneasy alliance with the \hr{Cuezcan}{\cuezcans}, since both want to see the \bane{} faction destroyed. \Sanyor, the \scathaese{} chaos sorcerer who is Curwen's second-in-command but will betray him, is a \cuezcan{} in disguise and working to free the \Sephiroth. 

The \Kezeradi{} have an image of \quo{fallen angels} about them. Not \quo{fallen} in the usual sense of having turned to evil, but in the sense that they were once an idealistic people, believing in good and beauty, but have become hardened, bitter and disillusioned. They look angelic, but harrowed: Bright-\coloured skin and great feathered wings, but the wings are tattered and torn, their skin deathly pale or blotched and dis\coloured, and their once beatific visages are grim, contorted from millennia of grief, pain and rage. 





\subsubsection{\Sithiyacaan: The last \Kezeradi{} prince}
\hr{Sithiyacaan}{\Sithiyacaan} is a great hero who is the last surviving \Kezeradi{} lord. 





\subsubsection{New \Kezerad}
Some surviving \Kezeradi{} have built a new \Kezerad, a new beautiful, harmonious kingdom. It is much smaller than the original \Kezerad, but good. Once in a while they are able to rescue someone and bring them there. It's almost a Tanelorn-like place (as in Michael Moorcock's stories). This is one of the silver linings of the story. 

\hr{Sithiyacaan}{\Sithiyacaan} knows about the place, but he does not know its location, and he can never go there. It would make him remember too much, and the \Sephiroth{} would gain access to his mind and learn the place's location and come to destroy it. 

This causes him great distress. 









\subsection{Politics}





\subsubsection{\TiphredSerah}
During the \hr{Resphan Wars}{\resphanwars} the \TiphredSerah{}, being \hr{Tiphred-Serah free-thinking}{the most free-thinking of the dynasties}, were the ones \hr{Tiphred-Serah and Kezerad ally}{most liable to ally with \Kezerad}. 















\section{\Malachim}
\target{Malach}
%In Vaimon metaphysics, the \Malachim{} are a class of \Archons{} that may incarnate as \humans{}. An incarnation of a \Malach{} is called a Scion. 
The \Malachim{} are \resphan{} lords who has left their \resphan{} bodies to incarnate again and again as \humans{}. 

The names of the \Malachim{} include: 
\begin{itemize}
  \item \hr{Eryal}{\Eryal}.
  \item \hr{Ishicah}{\Ishicah}.
  \item Nelchael.
  \item \hs{Ramiel}.
  \item Sachiel.
  \item \hr{Shiaraid}{\Shiaraid}.
  \item Two more. 
\end{itemize}










\subsection{Biology}






\subsubsection{Immortality}
\target{Malach immortality}
The \Malach{} model of immortality through rebirth is intended as an improved form of the \hr{Ophidian immortality}{\ophidian{} shedding of skin}. It's an attempt to achieve immortality without stagnation. It's also inspired by \hr{Draconic immortality}{\ps{\KhothSell}{} project of \draconic{} immortality}. 

They derive sexual power from this, too, because whenever a \malach{} is born anew, he gains some new, improved lifeforce from his parents. Compare this with the Xenomorphs of the \emph{Alien} movies, who improve themselves by stealing the genes of their hosts. 

Many women die in giving birth to a Scion, because the \malach{} \hr{Life drain}{drains} too much lifeforce from its host. 

See also the section on \hr{Kinds of immortality}{different kinds of immortality}. 





\subsubsection{\Kenosis{} and \Apotheosis}
\target{Kenosis}
\target{Apotheosis}
\index{\kenosis}
\index{\apotheosis}
Vaimon metaphysics holds that a Scion is created when a divine \ps{\Malach}{} descends to \Miith{} and incarnates in a \human{} body. 
The \Malach{} willingly gives up its divinity in order to walk on \Miith{}. 

\quo{\Kenosis} is the term for the reconciliation and between the mortal \human{} and the divine \malach. 
For a Scion, to achieve \kenosis{} is to become at peace with your nature and thus gain full access to the memories of your recent incarnations (to the extent that these memories carry over). 

\quo{\Apotheosis} is that which all Scions must strive for: 
The re-realization of the divine potential that lies dormant within the Scion. 

Only by \Kenosis{} and \Apotheosis{} in combination can the divine \Malach{} and the mortal Scion be fully reconciled and in harmony/balance/union. 










\subsection{History}





\subsubsection{Origin}
Some of the \Malachim, including Ramiel, are of the \satharioth. 

The \malachim{} were created as part of a \hr{Malach project}{secret project}.
But somewhere the process went wrong, and the \Malachim{} all had their memories of their previous lives as \resphain{} erased. 





\subsubsection{Amnesia}
They now recall only scattered fragments of their old lives, typically in fever-like dreams and \deajvus. 





\subsubsection{Captive \malach}
\target{Captive Malach}
Unbeknownst to anyone, \Azraid{} and one of his inner circles of researchers had, at some point, managed to capture a \malach{} or two. 
These they kept hidden and used as study objects. 
\Azraid{} wanted to unlock the secrets of the \ps{\malachim} creation and utilize that knowledge in his own \hr{Neo-Resphan}{\neoresphan} project. 

Compare to the anime \cite{Anime:NeonGenesisEvangelion}, where NERV and SEELE create the Evangelia as clones of a captive Angel. 









\subsection{Name}
\emph{Neshamah} is a Hebrew word that, in \Cabbalah, refers to one of the \quo{highest} parts of the \human soul.









\subsection{Scions}
\target{Scion}
\target{Scions}
\index{Scion}
A \human{} incarnation of a \Malach{} is called a Scion. 
Each Scion typically retains some memories of his previous incarnations, but how this works differs from \Malach{} to \Malach. 

For example, \hs{Ramiel} tends to be born with no memories at all, and then have his \emph{previous} incarnation \quo{awaken} later, triggered by some massively emotional event. (\hr{Carzain}{Carzain \Shireyo}, a Scion of Ramiel, has his predecessor, \hr{Vizicar}{\VizicarDurasRespina}, awaken when Carzain fights his first battle to the death and kills a man.) 
From that point, the Scion has a split personality with two distinct persons inhabiting the same body. If the two personalities get along, they will absorb traits of each other and eventually merge into a single character. If the two do not get along, the Scion will degenerate into a mood-swinging maniac.

\Shiaraid{} remembers a lot of sensations and emotions, but no concrete thoughts and words. 

Other \Malachim{} work differently. 

A Scion does not necessarily remember his true name, but the Vaimons possess the knowledge to research a Scion's mind and ascertain his \Malach{} identity. 





\subsubsection{Conjuctions}
Perhaps there are astrological reasons that determine when and where the Scions incarnate. 
Perhaps there are big Conjuctions that cause many Scions to incarnate shortly after each other, so their lives overlap. 
These times tend to have lots of \vertex/\matrix{} activity, too. 





\subsubsection{Scions are often only children}
\target{Scions are often only children}
Scions often have no siblings. 
They may have older siblings, but rarely younger siblings. 
It takes a great toll on the mother to give birth to so powerful a soul, so if it does not kill her outright it tends to render her barren, spent. 










\subsection{Skills and powers}





\subsubsection{Binding souls}
\target{Malachim binding souls}
\target{Malachim bind souls}
All \malachim{} had the power to capture the souls of others (including, but not limited to, people they slew in combat) to themselves. 

This was a more powerful version of the \hr{Resphan vampirism}{vampiric powers of all \resphain}.

This ability was extra powerful among \sathariah{} \malachim.







\subsection{Vaimon view}
In Vaimon metaphysics, the \Malachim{} are considered a class of \Archons{}. 
















\section[Merkyrah]{\Merkyrah}
\target{Good Resphan Empire}
\target{Merkyrah}
The \resphain{} were successfully born. 
But most people around them were killed. 
They ended up raised by some \nephilic{} commoners who really had little conception of what was going on. 
So the young \resphain{} were never told of their legacy. 

\Daggerrain{} had calculated with the possibility that \Thanatzil{} might fail, so he had installed in the \resphain{} an instinctive knowledge of their true nature. 
But he had underestimated the weakness of the \nephilic/\human/\resphan{} mind, and how abhorrent the truth would seem to the abandoned, hapless \resphain{} alone in the world. 

They repressed the truth. 
Their collective denial drew upon the cosmic barriers already woven in the \firstbanewar{} and used threads of these to create a sort of Shroud that prevented the \banelords{} from communicating with them. 

In this haze of delusion and denial they developed a culture of their own. 
At first they lived as wretched barbarian scavengers amid the \hr{City of Nyx}{endless decaying ruins of \Nyx}, its spires towering enormous and dark and frightening around them. 
They feared and fled from the many monsters that dwelt in \Nyx. 
But they had some magic, and that gave them a bit of an edge. 

It took them hundreds of years, but eventually they built a great empire and called it \Merkyrah.

Meanwhile, \ps{\Daggerrain}{} plans were set back a thousand years or more while he waited for the \resphain{} to rediscover the truth\dash as he knew that they must, sooner or later.

The \Merkyrans{} had their own \dweomer: 
The \hr{Old Good Iquin}{first incarnation of \iquin}, \quo{the Light}. 
It was based on \Nyxian{} energy and drew its power directly from \Nyx{} and the \banelords, but the \resphain{} had managed to reshape it into something that felt good and pure. 

\target{Good Resphain and God}
They worshipped this Light, personifying it as a god\dash perhaps even as \quo{God}. 

They saw it as their duty to do good and spread the cause of good on \Miith{}. With the \hr{Dragon war when Resphain appear}{\dragons{} weakened by internal war}, there was plenty of room for the nascent \resphan{} culture to expand and rise to power. 
The \resphain{} were particularly effective because they were so moral and well-organized, unlike the chaotic \dragons. 
They built a great empire where they ruled over \humans, \nephilim{} and perhaps other things. 
Their rule was mostly benevolent. 









\subsection{History}




\subsubsection{\Resphain were lost and afraid}
After \ps{\Thanatzil} fall, the \resphain{} were lost and afraid, without even their memories. 

\lyricsbs{Emperor}{Grey}{
  when all is dark\\
  there are no points of reference\\
  and we no longer navigate by the stars\\
  we just end up somewhere\\
  \ldots{}nowhere\ldots{}
}

They longed for a belief to cling to, something to give meaning to their frightening lives. 





\subsubsection{Created as a Shroud anomaly}
\Merkyrah{} was founded due to a Shroud anomaly. 
The \banes{} tried to open the way from \Erebos{} through \Nyx{} into \Tembrae. 
It went awry. 
There was a great implosion, and all the \resphain{} (as well as a lot of \nephilim, who lived, and some \humans{}, who all died) were sucked into \Nyx. 

Here they now had to build their new lives. 





\subsubsection{\Semiza helped them}
In the beginning, \hr{Semiza helps fugitives establish themselves in Nyx}{the fugitives in \Nyx were aided by \Semiza}. 





\subsubsection{Repressive religion won}
After the exile into \Nyx, the early \resphain{} quickly splintered into many warring tribes. 
They had an instinctive bloodthirst and a craving for fratricide and cannibalism. 
But they did not have the technology to \hr{soul-eating}{eat souls}, so they could not draw much advantage from all this cannibalism. 

Ergo, \resphain{} who followed their warlike nature had no particular advantage over those who lived in denial. 
So it came to be that some fractions rose to prominence with a very ascetic and \resphan-nature-denying religion. 

These formed \Merkyrah, which became the predominant \resphan{} kingdom. 





\subsubsection{Monster-hunting}
\target{Merkyran hunters}
In the beginning, \Nyx{}\dash being a bastard child of \Erebos{} and \Miith\dash was full of nightmarish horrors and monsters. 
But over the course of many centuries, the \Merkyrans{} and other \resphain{} pushed back these monsters, and even exterminated some of them. 

Compare to the later \human{} \hr{Myths of vanquished monsters}{myths of Iquinian heroes vanquishing inhuman Elder Races and monsters}. 

The \resphain{} ate these monsters.
They cooked them using religious magic. 
They did not understand the principles back then, but their arcane cooking released some soul-power from the monsters, which the \resphain{} devoured. 
This was very healthy for them and helped them sate their \hr{Resphan vampirism}{vampiric thirst} and thus grow and stay alive. 

This was actually part of \ps{\Daggerrain} master plan. 
He knew from the start that the \resphain, having \bane{} blood in their veins, would be vampiric/cannibalistic parasites, so he prepared a Realm full of nice, nutritious monsters into which he could plunk his \resphain. 
Thus they could live there for some centuries or millennia and grow big, strong and plentiful. 

In \Merkyrah, \quo{hunter} was a common, respected and very valuable profession in society. 
\Nyx{} had little possibility to support agriculture, so they had to live off hunting. 

Near the end of the time of \Merkyrah, the dependence on monster-hunting became a problem.
The \resphain{} had hunted several species into extinction, and severely thinned the populations of many of the rest.
At least, the more manageable ones. 
The most dangerous and/or less edible monsters were still around and posed as great a threat as ever. 
This meant that near the end, hunting had become sparse. 
To find good edible game the \resphan{} hunters had to venture far away from their safe home towers (either out or \emph{\hr{safe zone}{down}}), and they had to go through large stretches of land that was almost devoid of game suitable for \resphan{} consumption but full of dangerous monsters for whom \emph{\resphain{}} were attractive prey (such as the dreaded \hr{Umbra}{\umbrae}). 

This scarcity of prey meant that the \resphain{} were beginning to suffer from malnutrition. 
Their health was failing and their lifespans dropping. 
They were becoming frail and weak, which only made hunting more difficult and compounded the problem. 

The monster populations \hr{Nyx monsters}{since recovered}. 





\subsubsection{The Heart is strained}
\target{Merkyrah strained the Heart}
At the end of \ps{\Merkyrah} time, the \hr{Heart}{Heart of \Miith} was being strained. 
Right after the \firstbanewar, \hr{Heart after First Banewar}{the Heart had been going strong}, but now, thousands of years later, there were hundreds of thousands of \resphain{} alive, which was more than the Heart could safely support. 

As a result, everyone's health was deteriorating\dash especially the \resphain{} themselves, but also other creatures, including those in \Tembrae{} that knew nothing of the \resphain. 
The \dragons{} \hr{Dragons wonder why the Heart is weak}{wondered about this}, but could not figure it out. 

\target{Umbra menace growing}
This Heart-weakening had another side-effect, which was nastier from a \resphan{} point-of-view:
It brought more \umbrae. 
This was also \hr{Umbra origin}{how the \umbrae{} had originated on \Erebos}: 
When the life-giving \dweomer{} was being wrung dry by \bane{} wickedness, it would retaliate by pumping out \bane-eating monsters. 
Now \umbrae{} were multiplying like crazy, and it seemed there was nothing the hapless \Merkyrans{} (and other \resphan{} tribes) could do about it. 





\subsubsection{\Daggerrain knew it was time}
At the time shortly before the \quo{\hs{Delving}}, \Daggerrain{} knew that \Merkyrah{} had played its role, and that he must set things in motion to bring his \resphain{} back into the fold. 
The reasons for this included (but were not limited to):

\begin{itemize}
  \item 
    The \resphan{} population was nearing its natural ceiling and \hr{Merkyrah strained the Heart}{straining the Heart of \Miith}. 
  \item 
    The \resphain{} were having trouble sustaining themselves by \hr{Merkyran hunters}{hunting monsters}. 
  \item 
    The \hr{Umbra menace}{\umbra{} menace} was growing. 
\end{itemize}










\subsection{Infinite city}
\target{Merkyrah is one huge city}
\Merkyrah{} is one huge, infinite city. 
It is based on \Erebos{} and beautified by the good \resphain. 

Beyond the borders of the civilized \Merkyrah{} lie more builings, but there are empty, haunted ruins, shunned by all. 
By venturing beyond these ruins one can gradually transit into \Erebos\dash if you're really lucky. 
\Nyx{} is a shadow of \Erebos, remember. 





\subsubsection{Cathedon}
The capital citadel of \Merkyrah{} is Cathedon. 
(The name is taken from \authorbook{William Blake}{The Four Zoas}.) 









\subsection{Mythology}
\target{Merkyran myths}
Ignorant of the truth, the \Merkyrans{} fabricated creation myths. One myth claimed that they were created by the Light as angels of good, and that it was their very nature to be good and noble. Another, more truthful myth hinted that dark powers had had a hand in their creation, or that their people had committed some terrible crime early on in their existence. 

Whatever the explanation, there prevailed the notion that the \resphan{} race was somehow afflicted by a grievous original sin for which they must atone. This made them zealous, fanatic in their crusade to spread what they considered the ways of good. With this (vaguely defined) \quo{evil} hanging as a shadow over their history, they must strictly discipline themselves to suppress and overcome this inner evil. This made them repressive and intolerant, and they came to expect the same perfection from others as they allegedly strove to achieve for themselves. 

As a matter of course, they saw themselves as destined to rule as enlightened rulers over \humans{} and \nephilim{} at the very least. 

For the first long while they did not make war against the \dragons. They might be religious, but they were not stupid, and they knew that they did not yet have the strength to oppose the mighty \dragons. 

They were obsessed with religion, morality, sin, shame, guilt, atonement and duty. Officially they preached humility and shame over their dark past (of which almost nothing was known, and false myths proliferated), but underneath the \facade{} they grew arrogant and domineering, addicted to control and power, the power to control other people's emotions and thoughts. Almost like \authorbook{George Orwell}{1984}. 

The \Merkyrans{} disliked the body all kinds of sex and pleasure. They preach asceticism and flagellation.

They dyed their black skin white or otherwise bright as a signal that they had repented their original sin and were striving to change, reform.

\Merkyrah{} was just, in a sense. 
But it was also repressive and conservative and conformist. 
Free-thinkers were frowned upon. 





\subsubsection{\Dragons{} symbolize evil}
\target{Merkyran myths of Dragons}
The \Merkyrans{} use \dragons{} and other reptillian creatures to symbolize all sorts of evil. 
According to their myths, they were driven from their ancestral lands by wicked \dragons. 





\subsubsection{Black stars are feared}
\target{Merkyran myths of black stars}
The \Merkyrans{} had myths about black stars. 
They were said to be terrible wounds in the sky, through which the hideous outer void bled all its evil and corruption. 

These were based on a memory of the \hs{black stars} of \Erebos. 





\subsubsection{\Thanatzil}
The \Merkyrans{} had a twisted myth about \Thanatzil. 
It was pretty far from the truth. 
The rebels who found \Semiza{} learned the true story of what happened to \Thanatzil. 











\subsection{Religion}
\subsubsection{Decay}
Perhaps their religion causes decay. 

\lyricsbs{Vital Remains}{Descent Into Hell}{
  Trapped behind the false moral confines of Christianity.\\
  Your self-destructive belief system is a festering cancer.\\
  It should have become extinct a long time ago.
}

Perhaps the \Merkyran{} priests are poisoned by their own lying religion, degenerating into crippled, sickly ancients. 
Compare to the priests in the movie \emph{300}. 





\subsubsection{The voices of God}
\target{Voices of God}
%They believed to hear the voice of God, but it was really just their own imagination, given shape by the Shoud.
There was a group of fallen \resphain{} who remembered the truth. 
They died, but stuck around as ghosts. 
They were quite weak, but from time to time they were able to communicate with the living \resphain{} in dreams. 

The \Merkyrans{} believed such visitations to be the \quo{voice of God}, but in reality it was just the echoes of a bunch of powerless ghosts. 

Perhaps they were imprisoned by the same Shroud that encased \iquin{} and gave it a sheen of benevolence. 

The \quo{voices} kept the truth hidden because they hated it and wanted their people to remain pure, innocent and good. 

Perhaps the oldest \resphain, who today hold positions as high priests, knew the truth deep down. 
They knew that their religion was a lie and that their god did not exist. 
They created the religion as a means to suppress the truth and keep the \ps{\resphain}{} minds occupied and pacified. 

Perhaps they are the ones faking the \quo{voice of God}. 
Or perhaps they are in league with the ghosts. 

Perhaps the high priests did not initially know the truth, having blocked it out. 
But then they \hr{Search for God}{searched for God}. 
They discovered that there was no God, only a vast, consuming darkness (a vision of \Erebos). 
They realized the truth and decided that they must hide it. 







\subsubsection{Search for God}
\target{Search for God}
Once, the \Merkyran{} church experimented with journeying to the innermost reaches of their mind to search for \quo{God}. 
But they found a terrifying, frightening emptiness and recoiled. 

They rationalized this, interpreting that God does not want to be searched for. 
He wants to be blindly believed in. 









\subsection{Immortality?}
Were the \Merkyrans{} immortal? 
There are two possibilities: 

\begin{enumerate}
  \item 
    No, they were not. 
    They lived only a few centuries, 1000 years at most. 
    Violently denying their true nature and refusing to consume souls to sustain themselves, they eventually died of starvation and mental stuntedness. 
  \item
    Yes, they were immortal. 
    Moreover, they lacked the technology to destroy souls, so whenever one died he would always return. 
    Even the very oldest \Merkyrans{} remained alive at the time of the rebellion. 
    
    Later the rebels would discover the means to permanently destroy a \resphan{} soul. 
    This was considered horrifyingly evil black magic. 
\end{enumerate}









\subsection{Culture}





\subsubsection{\Bezedeth}
\target{Bezed status in Merkyrah}
In \Merkyrah, \bezedeth had an understandable reputation for being cowards, compared to the True Immortal purebloods. 
They were scorned by the \Merkyran church. 
Being mortal they were condemned to oblivion, while the purebloods would receive the mercy of God and live forever. 

This made them very bitter, so when the rebels came and preached to them the \bezedeth were \hr{Bezedeth rebel against Merkyrah}{all too eager to rebel against \Merkyrah}.

\citebandsong{Nile:InTheirDarkenesShrines}{Nile}{
  The Blessed Dead
}{
  Looked Down Upong With Scorn\\
  We Work the Fields of the Masters\\
  And Share Not the Bounty of the Black Earth

  Destitute Servile Cast Out\\
  Affording No Tomb\\
  We Shall Be Buried\\
  Unprepared in the Sand

  We Shall Never Be The Blessed Dead

  Scorned By Asar\\
  Condemned at the Weighing of the Heart\\
  We are Exiled from the Netherworld\\
  Serpents fall Upon us Dragging us Away\\
  Ammitt Who Teareth the Wicked to Pieces

  Pale Shades of the UnBlessed Dead\\
  None Shall Enter Without the Knowledge\\
  Of the Magickal Formulas\\
  Which is Given to Few to Possess

  Not for Us to Sekhet Aaru\\
  Our Souls Will be Cut to Pieces with Sharp Knives\\
  Tortured Devoured\\
  Consumed in Everlasting Flames

  We Shall Never Be The Blessed Dead
}





\subsubsection{Demographics}
\target{Merkyran demography}
There were relatively many \resphain{} in \Merkyrah. 
More than one per 100 mortals. 
This was before the devastating \secondbanewar{} and \resphanwars, and before the \hr{Heart}{Heart of \Miith} became \hr{Heart weakened}{so badly weakened}. 





\subsubsection{Fallen ones}
\target{Early fallen Resphain}
\target{Early Resphan fallen ones}
\target{Merkyran fallen ones}
But there were some \quo{fallen} \resphain. These were unable to suppress the darkness within them and fully embrace their religion. These live either in secret, pretending piety while feeling blasphemy inside, or as overt outsiders and criminals, or even mafia. 

Many of the fallen ones, perhaps most, were \quo{\hr{Ashenblood}{\ashenbloods}}, born of \human{} mothers. In \Merkyrah, they were outsiders because they killed their mothers and were born with blood on their hands. This was a visible reminder of the \resphan{} \quo{original sin}, which the \Merkyrans{} feared. Therefore the Church cannot abide the sight of them, and they are expelled and shunned. 

Some of them learned to channel \hr{Itzach}{\itzach}, which was forbidden. They liked \nieur, because it was more suited to their nature and felt more true, more \resphan-like than the \hr{Old Good Iquin}{false Light} that the church forced down their throat.

Some of these are full of \trope{Wangst}{Wangst}, feeling guilt and shame over their wickedness and weakness, their inability to rise above their base, bestial nature. Others embrace the darkness and chaos and becomes violent, cruel miscreants. These often form criminal gangs or cults, where they have sexual orgies and stuff. 

A few fallen ones are powerful individuals who choose this path for themselves, or who gain strength through adversity. But most of them are weaklings who are simply to feeble of mind to suppress their inner darkness and conform to their religion. They are a pretty sorry bunch, and most are very unhappy. They use sex (and violence and other decadence) in an attempt to fill the devouring void within them, but they don't succeed. They suffer, like gothics or emos. 





\subsubsection{Breeding creatures}
The \Merkyrans{} breed some creatures to serve them\dash or perhaps they \emph{created} them, using the remains of \hr{Bane technology}{\bane/\voyager{} technology}, in the very beginning, before they lost all their memory. 

These creatures are beautiful and benevolent, shaped as such by the good, \hr{Old Good Iquin}{\iquin-based magic} of the \resphain{}. 

Maybe they look like Faerie Dragons or Wisps from \emph{Warcraft III}, or maybe something with gossamer wings.

But as the rebels' evil seeps out and permeates their society and their world, these creatures become twisted and evil monstrosities.

\lyricsduana{Chrysalis}{chrysallis}{
  and she weeps at the fragile beauty \\
  that warps the darkness \\
  and captures her \\
  once again in the web of life}





\subsubsection{Sex and marriage}
\target{Merkyran monogamy}
\Merkyrah{} practiced monogamous marriage.
Sex outside marriage was a sin and a crime. 
(They were pretty crazy.) 





\subsubsection{Technology}
\target{Merkyran technology}
\index{technology!\Merkyrah}
The \Merkyrans{} have low technology, but they have some artifacts of \hr{Bane technology}{\bane{} technology}. 
They do not understand these, so they tend to worship them as religious relics\dash or shun them as devices of evil. 

Only later do the \resphain{} \hr{Resphan technology}{learn about the lost science}. 









\subsection{Politics}





\subsubsection{\Umbra{} menace}
\target{Merkyrans fear Umbrae}
\target{Umbra menace}
The \Merkyrans{} were preyed on by \umbrae. 
They greatly feared the \umbrae. 
In fact, a major pillar of their religion was that it was supposed to protect them from the horrid \umbrae{}. 

When \umbrae{} attacked, the \resphain{} would seek refuge in their churches, which were somewhat protected by magic (but still not entirely safe). 

\citebandsong{Ihsahn:TheAdversary}{Ihsahn}{Will You Love Me Now?}{
  And they gathered\\
  in their halls of justice,\\
  halls of mirrors,\\
  halls of echoes.
  
  And they gathered\\
  in their houses of worship,\\
  within the walls of the unspoken,\\
  sheltered from the rain.
}

These churches should be a bit scary and life-denying. 
Like the book \emph{The House With No Windows}. 

Interestingly, the \umbrae{} by far preferred to attack \resphain. 
They would very rarely attack \nephilim{} and \humans, so the mortals had little to fear from the monsters. 
They still feared them, though. 
Partly because their religion told them to, and partly because it was horrifying to see the gods you worshipped get eaten by monsters. 

Remember also that \resphain{} were weaker in the days of \Merkyrah{} than they later became.

Near the end of the time of \Merkyrah, the \hr{Umbra menace growing}{\umbra{} menace was growing} because \hr{Merkyrah strained the Heart}{\Merkyrah{} strained the Heart of \Miith}. 

Later the rebels would \hr{Resphain learn to command Umbrae}{learn to command the \umbrae}. 

Pureblood \resphain were, if anything, more terrified of \umbrae than \bezedeth. 
The \bezedeth \hr{Ashenblood lesser immortality}{were only Lesser Immortals}, so they knew plenty of things that could kill them. 
Purebloods were True Immortals and usually feared nothing, so the thought of a monster that could permanently destroy them filled them with dread. 





\subsubsection{\Umbra{} siege}
The \hr{Umbra menace}{\umbra{} menace} had another consequence: 
It shaped the dwellings and cities of the \resphain. 
\Nyx{} was a vast Realm, and the \resphain{} would like nothing more than to spread out and conquer it all. 

But the \umbrae{} were a problem. 
See, whenever a small group of \resphain{} were out in the open, it would attract \umbrae.
This made it hard for them to start new settlements. 
They had to build large, compact cities so they could easily summon large numbers of \resphan{} warriors to defend themselves against \umbra{} attacks. 
This kept all \resphain\dash \Merkyrans{} and \hr{Resphan tribes}{tribesmen} alike\dash isolated in small clumps of \Nyx, spreading only slowly and with great difficulty. 

These claustrophobic living conditions were one of the reasons why the different \resphan{} tribes fought and hated each other so much. 
If they disagreed on something, they could not simply go different ways. 
They had to live near each other. 





\subsubsection{Other \resphan{} tribes}
\target{Resphan tribes}
\target{Early Resphan tribes}
\Merkyrah{} did not rule all \resphain{} in \Nyx. 
It lay at war with some other tribes. 
Compare to Deepgate and the Heshette in \authorbook{Alan Campbell}{Scar Night}. 

But \resphain{} were \hr{immortality}{immortal}. 
When they were killed in war they just came back to life a bit later (in the same body, if possible). 

\target{Early diabolist Resphain}
\target{Early Resphan diabolism}
These tribes worshipped dark gods: 
\XzaiShanns{} or cosmic gods that had taken up residence in \Nyx{}, or just had some tenuous connection there. 
Some of those gods \hr{Gods in Nyx}{still lived in \Nyx{} thousands of years later}. 

Some of these tribes were later \hr{Bael'Zerach absorbs tribes}{absorbed into \Baelzerach}. 
Other tribes \hr{Awakening}{Awakened} on their own after hearing the word from the dynasties. 

The rest of the tribes who did not \hr{Awakening}{Awaken} were no match for the Awakened \resphain{}, so they were hunted to extinction and eaten. 















\section[Mystraacht]{\Mystraacht}
\target{Mystraacht}
The fiercest, most chaotic, most belligerent of the \resphain{} (barring the \hr{BZ}{\Baelzerach}). Ramiel belongs to them. 









\subsection{Aesthetics}
The \Mystraacht{} traditionally prefer dark \colours with a \quo{violent} feel to them: 
Black, red (ranging from blood red to fiery orange), sinister purple, deathly gray. 
They glorify their martial prowess and often dress in the trappings of war: 
Metal \armour, or robes designed to remind of \armour. 

\index{beard!\Mystraacht}
\Mystraacht{} men wear beards more often than other \resphain. 
Beards give a masculine, \quo{bestial} image which the \Mystraacht{} like. 
Besides, \Zachirah{} did it and everyone else jumped on the bandwagon. 





\subsubsection{Black Bat}
\target{Mystraacht symbolism}
One of the dynasty's symbols is a black bat with blood-red fangs, eyes, claws and ears. Some of their warriors wear bat-like masks. 

Contrary to what some believe, the black bat has no deep metaphorical significance. 
\hr{CS symbolism}{Unlike the \CiriathSepher}, the \Mystraacht{} do not believe in metaphors and deep meaning. 
They believe in power and fear. 
The bat was simply chosen because it looked physically impressive and could be used to inspire fear.

\lyricsbalsagoth{
  The Splendour of a Thousand Swords Gleaming Beneath the Blazon of the Hyperborean Empire 
  - Part III: 
  Cry Havoc for Glory, and the Annihilation of the Titans of Chaos
}{
  \ldots{} none who gaze in awe beyond the mists and are blessed to behold it shall ever forget the splendour of a thousand swords gleaming beneath the blazon of [the Black Bat of \Mystraacht].
}





\subsubsection{Halls of iron}
\Mystraacht{} citadels tend to be dark, made of iron or the like. 
They are stark and martial, adorned with blades, spikes and trophies of war, but not frivolous luxury (unlike the more foppish \KiriathSepher). 








\subsection{Culture}
\target{Mystraacht philosophy}
The \Mystraacht{} are the most overtly evil of the \resphan{} dynasties. 
They openly embrace both of their heritages: 
The \SitraAchra legacy, with all its parasitism, betrayal and patricide, and the \chaotic{} legacy, with all its violence and savagery. 

In contrast to the \KiriathSepher, who are \hr{Dance}{concerned with etiquette}, fashion and principles, the \Mystraacht{} are much more chaotic and take a pragmatic approach: 
If it works, then it is good. 
If it fails, then it is ill. 

\target{Mystraacht not diplomatic}
But they were also less diplomatic than the other dynasties because of their prideful, violent disposition, so they had a hard time forming alliances. 

\hr{Zachirah}{\Zachirah} formulated the \Mystraacht{} ideology along these lines: 

\lyricsdimmuborgir{Det Nye Riket}{
  V\aa rt hat skal vinne.\\
  V\aa r ondskap skal gro.\\
  A feste seg i unge sjeler.
  
  Den siste krig skal vi vinne,\\
  og de godes blod skal falle som regn.\\
  Deres korte sjeler skal samles.
  
  Vi skal r\aa de over kaos og evig natt.\\
  Vi skal glemme de kvinnelig vikante m\oe dre\\
  og utslette alt.
  
  Et rike skal reise seg\\
  i asken av brennte hjem.\\
  Det er kun en herre hersker.\\
  Vi heller deg, Satan, de sterkes konge.
  
  Din tid er kommet.
}

They see themselves as more \quo{pure}, more true to their legacy of Darkness and Chaos than the other dynasties. 

\lyricsdimmuborgir{Progenies of the Great Apocalypse}{
  We, who not deny the animal of our nature. \\
  We, who yearn to preserve our liberation. \\
  We, who face darkness in our hearts with a solemn fire. \\
  We, who aspire to the truth and pursue it's strength.
  
  Are we not the undisputed prodigy of warfare, \\
  fearing all the mediocrity that they possess? \\
  Should we not hunt the bastards down with our might, \\
  reinforce and claim the throne that is rightfully ours?
  
  Consider the god we could be without the grace. \\
  Once and for all. \\
  Diminish the sub principle and leave it's toxic trace. \\
  Once and for all.
}





\subsubsection{\Daemon{} summoning}
\target{Mystraacht summon Daemons}
They embrace and endorse Chaos much more overtly than the other dynasties. 
They conjure and bind \daemons{}.
The other dynasties do not approve of this, for several reasons: 
Principles, envy, fear of \ps{\Mystraacht}{} growing power. 

Compare to the story from the game \cite{VideoGame:Diablo} about the Warlord of Blood, a Vizjerei mage who summoned demons and became too powerful and too mad and evil, so the other mages banded together to take him down. 

Maybe this was the reason for \hs{Ramiel's fall from grace}.



\lyricstitle{\emph{Call of Cthulhu} RPG p.118}{
They [the Serpent People] built black basalt cities and fought wars, all in the Permian aera or before. 
They were great sorcerers and scientists, and devoted much energy to calling forth dreadful \daemons{} and brewing potent poisons.}





\subsubsection{Dark knights}
\target{Dark knights of Mystraacht}
Have some sinister dark knights in \Mystraacht. Like \hr{Dark Crescent Knights}{the Dark Crescent knights}. 

Compare to the Lords of Negation in \FLuneNoire. Maybe give them a similar title. 

Maybe they resemble Revenants from the game \emph{Warcraft III}. 





\subsubsection{Education}
\target{Mystraacht education}
\Mystraacht{} had every new child raised locally where its parents lived. 
They did not feel it worth the effort to gather the children in a school, since there were so few of them anyway. 

\CiriathSepher, on the other hand, \hr{Ciriath-Sepher education}{had one central school where all children were educated}. 





\subsubsection{Government}
\Mystraacht{} was \emph{de facto} a sort of elective monarchy. 
\Zachirah{} assumed Overlordship not just by being the founder, but also by being the biggest, baddest and sexiest of the lot. 

When \hr{Zachirah dies}{he died}, there was no mechanism in place for choosing a successor, so the belligerent \Mystraacht{} squabbled amongst themselves for thousands of years. 

Eventually \hr{Dasteron becomes Overlord}{\Dasteron{} made himself Overlord} by gathering enough support to beat down those who disagreed. 
This had taken more than a thousand years of concentrated work. 





\subsubsection{Less snobbish}
The \Mystraacht{} are less snobbish than the \KiriathSepher. 
They don't hate the \ashenblooded{} so much. 
They respect people who are strong and capable, not just people of high birth. 





\subsubsection{Sex and gender roles}
\target{Mystraacht sexuality}
\Mystraacht{} \resviel{} are more promiscuous than those of \CiriathSepher{} and \TiphredSerah. 
The \Mystraacht{} believe in the freedom to act on one's desires. 
Sexuality should be embraced wildly and without inhibition. 

Sometimes they have sex in public. 
Often to show off. 

\target{Mystraacht fertility}
The \Mystraacht were somewhat more fertile than the other dynasties.
It was believed that this was because they had fewer sexual inhibitions.
They revelled in sex without restraint, and their bodies (and the Heart) rewarded them for it.

\target{Mystraacht amazons}
Many \resphan{} societies had a \hr{Resviel do not fight}{principle that \resviel{} should not fight in physical combat}. 
\Mystraacht{} had no such rule. 
Many \Mystraacht{} \resviel{} fought as \quo{amazons}. 





\subsubsection{Warrior ethos}
They idolize and glorify their martial strength. 
They see physical and magical prowess as an ideal. 
It signifies being in touch with your true nature as a child of Chaos and Darkness. 

\target{Mystraacht trial by combat}
This also means that \quo{trial by combat} is an accepted means of settling disputes among the \Mystraacht. 
Such fights are to the death, but not \hr{Immortality}{annihilation}. 

Unlike \KiriathSepher, \Mystraacht{} has no law against killing another \resphan, as long as it happens in open, honourable combat. 

\lyricsbs{Exmortem}{Terror Mundi}{
  War emperors.\\
  Apocalyptic death knights.\\
  Cries of battle.\\
  Death before dishonour.
  
  Stressed with wicked intolerance,\\
  blessed with winds of chaos.\\
  Unconquerable demons.\\
  Lick the bones of war.\\
  Release the fury.
  
  Terror Mundi.
}

\lyricsbs{Exmortem}{Gruesome Icons}{
  Ride the storm of mayhem.\\
  Fly with the wings of the damned.\\
  Blackwinged Deathknight.\\
  Your will is supreme.
  
  Praise the fall of [\Merkyrah]. \\
  I'll show you the way to obscurity.\\
  Images of defunct future.\\
  A mirror of the underworld.
}

\lyricsbs{Hate Eternal}{To Know Our Enemies}{
  Prepared for death by this code of \honour, our weapon is our soul.\\
  Chivalrous in our stand with valor.\\
  Herald by many, but feared by all, revered in our essence.\\
  Gallant in our fight we will conquer.
  
  We shall command! \\
  We shall command!
  
  With the mind we control our bodies, even if we rule no longer.\\
  In unification we shall learn. \\
  You can not dare abolish our strengths. With wisdom we shall lead. \\
  It is the way of the warriors.
  
  Propagate the masses, with the embodiment of who we are.\\
  Iconoclastic new age. We shall transform.
}

\lyricsbs{Hate Eternal}{It Is Our Will}{
  It is our will to conquer formidable foes. \\
  we will not concede. We shall not concede. \\
  To call upon all of our strengths. \\
  You will not halt us. You will not halt us.\\
  It is our destiny, it is ever so powerful,\\
  This mass that haunts us.\\
  We shall overcome all.\\
  Rise above we shall, rise above we shall.\\
  Rise above on our road of descent.
  
  To embrace the whole of our power. \\
  We will not concede. We shall not concede.\\
  To overcome hordes of the blind.\\
  You will not halt us. You will not halt us.\\
  To thrust ourselves in destiny. \\
  We will not concede. We shall not concede.\\
  To avenge the fallen spirits.\\ 
  You will not halt us. You will not halt us. 
}

\lyricsbs{Vital Remains}{The Night has a Thousand Eyes}{
  We rose from the Earth and fell from the Heavens, \\
  exalted saints of flesh and will. \\
  Far into the opaque silk that is the night. \\
  We are the provenance of fear 
  and the heralds of the profane. \\
  
  Call us fiends (oh, the apostasy), \\
  call us demons (oh, the apostasy), \\
  but we are just wolves in our right, \\
  hunting and feasting on the human breed, \\
  so infantile and yet so ripe. 
  
  Let us prey.
}









\subsection{History}





\subsubsection{Dominance}
For a long time during the \resphanwars, \hr{Mystraacht dominance during Resphan Wars}{\Mystraacht{} was the most powerful of the dynasties}. 





\subsubsection{Fall}
But \hr{Mystraacht betrayed}{then they fell}. 









\subsection{Politics}





\subsubsection{\Banes}
Ostensibly they are loyal to the \SitraAchra, but in secret they have their own \matrixx{} and are plotting against their creators. 





\subsubsection{Leadership}
\target{Overlord}
They once had a Overlord, but now they have only Princes.

Currently, the \Mystraacht{} are somewhat without direction, and have been stagnating for some thousand years. They need an \apex{} to their \matrixx, but all their worthy candidates fight amongst themselves and antagonize each other. 







\subsubsection{Ramiel's return}
\target{Mystraacht Matrix}
But one day, Ramiel returns to claim the throne of \Mystraacht. 

\lyricsbalsagoth{Dreaming of Atlantean Spires}{
  The Topaz Throne is beckoning,\\
  the jewelled sword awaits my graps.\\
  The Dreaming Gods now grimly brood\\
  in the silence of Atlantean Spires.
}

Ramiel has always desired the \Mystraacht{} throne. 
Before he became a \malach, however, there was another \Mystraacht{} lord who outranked him. 
He has been slain and has no obvious heir. 

Ramiel has sinister dreams about the \Mystraacht{} \matrix{} and the throne. 
Insert a scene with Vizicar/Carzain dreaming about it. 
Really Bal-Sagoth-esque. 

\lyricsbalsagoth{
  Into the Silent Chambers of the Sapphirean Throne 
  (Sagas from the Antediluvian Scrolls)
}{
  Torches glow in silver cressets\\
  in the Temple of the Serpent.
}

Do the \Mystraacht{} have \ophidian{} connections?















\section{\Resphan}
\target{Resphan}
\index{\resphan{} (plural \resphain)}
The \resphain{} are the progenitors of \humans{}, created to be the heirs of the \SitraAchra. 
They were bred from \nephilic{} stock, endowed with \erebean{} minds and infused with the life-giving Chaotic energy from the Heart of \Miith. 









\subsection{Name}
Singular \emph{\resphan{}}, plural \emph{\resphain{}}. The adjective is \emph{\resphan{}}. 

The word \quo{\resphan} is also used to refer specifically to the male of the species. The female is called a \resvil, plural \resviel. 









\subsection{Biology}





\subsubsection{Blood}
Drinking \resphan blood is very healthy for \humans.
It heals wounds and rejuvenates and extends lifespan.

But it is also addictive.
Psychologically at first.
Repeated consumption over short time leads to physical addiction.
Once badly addicted, a \human will need it regularly or she will die.

Drinking \resphan blood also has a psychological effect of making the drinker instinctively devoted to the \resphan in question.
She will effectively fall in love and will feel an urge to serve him.





\subsubsection{Demographics}
The \resviel were much less common than their male brethren. 
Only about $20\%$ of all \resphain were female. 





\subsubsection{Failed \resphain{} and \humans}
Perhaps some of the early attempts at creating \resphain{} and \humans{} resulted in horrid abominations. 
Some of them still exist, used by the \resphain{} as cattle or slaves. 
Others have survived in the \hr{Wild}{\Wylde}. 





\subsubsection{Immortality and healing}
\target{Resphan immortality}
\target{Ashenblood lesser immortality}
\Resphain were \hr{immortal}{immortal}.

Only pureblood \resphain possessed \hs{True Immortality}.
\Ashenbloods died permanently when they were killed.

See also the section on \hr{Kinds of immortality}{different kinds of immortality}. 

A \resphan could heal almost any injury given time. 
A pureblood could even regrow his entire body from scratch if it had been destroyed. 

\target{Ketherain heal faster}
The higher \quo{tiers} of \resphain (\satharioth and \ketherain) healed faster than the lower tiers (\thelyadeth and \bezedeth). 






\subsubsection{Procreation}
\Resphain{} can be born of \resvil{}, \human{} or \nephilic{} mothers, but the father is always a \resphan. 

The children born of a union between a \resphan{} and a mortal (\human{} or \nephil) woman will always be a \resphan{} or \resvil, and as full-blooded as the child of a \resphan{} and a \resvil. 

A mortal man cannot impregnate a \resvil; her body will accept nothing less than \resphan{} seed. 
(This is a deliberate design feature: A \ps{\resvil}{} womb is precious and cannot be wasted on a bastard half-breed; a \resphan{} has plenty of sperm to spread around.)

A \resvil was fertile all the time.
They had safe periods and fertile periods. 

A \ps{\resphan} seed is always fertile and may impregnate any woman he fucks, mortal or immortal. 

Unlike mortal women, a \resvil{} remains fertile throughout her entire life. 
Her body grows new eggs periodically, so she can bear a theoretically unbounded number of children. 

\Resphain{} are born more well-developed than mortals. 
This is because they have \hr{Life drain}{drained} so much life-force from their mother. 
For the same reason, they need not drink milk. 
They start feeding on flesh, blood and souls immediately. 

A \resvil mother always died in childbirth.
Every new \resphan life was bought with death.
Usually it took only days for the mother to revive, though.
But in time she would recover fully. 

A mortal woman who gives birth to a \resphan{} child will invariably die in childbirth as the parasitic \resphan{} sucks all life-force out of her. 
She will also suffer while carrying the wicked child. 

When it is discovered that a mortal woman bears a \resphan{} child, she will usually be eaten. 
This is a great \honour for her. 
But some women escape or are released. 
They give birth to \hr{Beuzed}{\bezed} children. 

The \Mystraacht were \hr{Mystraacht fertility}{slightly more fertile than the other dynasties}.





\subsubsection{Purebloods and \ashenbloods}
\target{Pureblood Resphain}
\target{Ashenblood}
The \resphain{} are divided into four social classes:
\begin{gloss}
  \gitem[\satharioth]{\sathariah}
  The highest of \resphan{} nobility, with the stolen blood of \Nexagglachel{} in their veins. 
  See the \hr{Sathariah}{main section about the \satharioth}. 
  
  \gitem[\ketherain]{\ketheran}
  \target{Ketheran}
  Higher nobility; the descendants of \satharioth. 
  They, too, carry the \draconian{} blood, but diluted.   
  
  \gitem[\ruistheleth]{\ruisthel}
  \target{Ruisthel}
  Lower nobility. 
  They are descendants of the \satharioth, but born before they became \satharioth. 
  As such, they carry no \draconian{} blood, but still stand above the \thelyadeth. 
  
  The word comes from a root \quo{ruis} meaning \quo{old, original, venerable}\dash also found in \quo{\hr{Ruishagh}{\Ruishagh}}. 
  
  \gitem[\thelyadeth]{\thelyad}
  \target{Thelyad}
  Full-blooded \resphain, but with no \sathariah{} blood. 
  The \ketherain{} sometimes disparagingly call them \quo{plainbloods}. 
  
  \gitem{pureblood}
  Collective term for all full-blooded \resphain{}, born of a \resvil{} mother. 
  Only purebloods have wings. 
  
  \gitem[\gessurim]{\gessur}
  \target{Gessur}
  Full-blooded \resphain, but born of a \hr{Yurid}{\yurid} mother. 
  They were considered below \thelyadeth{} in status, but still above \bezedeth. 
  As purebloods they counted as members of their father's dynasty. 
  
  \gitem[\bezedeth]{\bezed}
  \target{Bezed}
  \target{Beuzed}
  Half-breeds, born of a \resphan{} father and a \human{} or \nephilic{} mother. 
  They are considered inferior by their full-blooded brethren. 
  They lack wings, often have gray skin rather than pure black, and are generally smaller and weaker. 
  They are disparagingly called \quo{\ashenbloods}\dash a reference to how they kill their mothers during birth. 
  
  \Human-born \resphain{} are short and skinny. 
  
  \index{beard!\bezed}
  \Nephil-born \resphain{} are broader, heavier and more hairy, but not necessarily taller. 
  If male, they will have plenty of beard. 
  
  The \bezedeth{} are particularly reviled because they take up space in the \hr{Matrix}{\matrix} and consume precious \hs{Heart} energy but will never be able to give anything back to the \resphan{} race, because they are sterile. 
  
  A \bezed{} \resvil{} can never become pregnant. 
  \Bezed{} males can sometimes impregnate a mortal woman (never a \resvil), but the child is always stillborn and often deformed. 
\end{gloss}





\subsubsection{Relation to \aryothim}
\hr{Resphan-Aryoth relationship}{The \resphain were descended from \aryothim to some extent}.





\subsubsection{Sexual maturity}
\target{Resphan sexual maturity}
Immediately when they are born, they resemble mortal children of about five years, if perhaps somewhat smaller. 
They grow to physical adulthood quickly, and are sexually mature when they are less than ten years old. 

Or perhaps the above was only true for \Thanatzil, who was a special case. 
Perhaps regular \resphain{} mature more slowly. 









\subsection{Culture}





\subsubsection{Architecture}
\target{Resphan architecture}
\index{architecture!\resphan}
The \resphain{} liked tall, spindly towers with bridges and causeways. 

Very unlike the bulky, bloated edifices of \hr{Draconic architecture}{\draconic{} architecture} and \hr{QJ architecture}{\quiljaaran{} architecture}. 

See also the sections on \hr{Nyx}{\Nyx} and \hs{dark ancient cities}. 





\subsubsection{Bloodwine}
\target{Bloodwine}
The \resphain{} drink a class of beverages called blood wines. 
They are sort of wine-like, but with the blood of intelligent creatures distilled into them, and with life force bound in it. 
It is brewed with some special stuff that keeps the blood from coagulating. 

One of the more popular types of bloodwine is \ethylshe. 

All \resphan{} purebloods are expected to know stuff about wine. 
It is a culture thing. 
It shows you know culture and fine arts and stuff. 
It's a way of signalling status. 

Bloodwine with \resphan{} blood is the best kind, surpassed only by bloodwine with \draconic{} blood. 





\subsubsection{Citadels, demesnes and manors}
The \resphain{} have most of their citadels in \hr{Nyx}{\Nyx}, where they feel most at home are are in contact with their roots and their \hr{Dweomer}{\dweomers}. 

They also have \emph{demesnes} (estates with farmland). 
These are not in \Nyx{} but in \hr{Azmith}{\Azmith} and other \hr{Tembrae}{\Tembraean} Realms, since \hr{Nyx is a parasite Realm}{\Nyx{} has no fertile farming land} and must steal from other Realms. 
Each demesne has a central manor from which the \resphain{} rule. 






\subsubsection{High Telepaths}
\target{High Telepath}
A High Telepath is a \resphan{} who specializes in telepathy. 
They often work as messengers. 
Many higher \resphan{} lords have a High Telepath employed (if the lord isn't trained as a High Telepath himself). 





\subsubsection{Language}
The \hr{Resphan language}{\Resphan{} language} (spoken by most \resphain, but not all) is based on Hebrew. 

\Mystraacht, \CiriathSepher, \TiphredSerah{} and \Kezerad{} all speak the same language, but in different dialects. 
So do some \Baelzerach, but some of them speak completely different languages. 

It is worth noting that this language uses a Spanish-style \quo{rolling} R. 





\subsubsection{Morality}
Some \resphain{} believe that they truly are the noble and good angels that they style themselves to be. They see the \dragons{} and their spawn as dangerous, vicious savages, and themselves (and, perhaps, their \SitraAchra sires) as a superior civilization with an inborn right to rule. So they truly believe that what they are doing is morally right. Remember, \trope{UtopiaJustifiesTheMeans}{Utopia Justifies the Means}.

Others are more amoral in their outlook and simply see the \secretwar{} as a struggle for survival between two peoples who cannot and will not coexist. (See section \ref{Fighting for survival}.)





\subsubsection{Myths of vanquished monsters}
\target{Resphain vanquish monsters}
The \resphain had their own legendary tales of vanquished monsters. 
Before the coming of the \resphain, \Miith was dominated by monstrous, cruel and incredibly ancient empires of the \dragons and \ophidians. 
The \resphain came to \Miith and began pushing back these Elder monsters to claim the world.
\Miith was their birthright because they were the superior race; more beautiful and perfect, unlike these bestial and hideous monsters that inhabited \Miith before them. 

See also the \hr{Myths of vanquished monsters}{Iquinian myths of vanquished monsters}. 





\subsubsection{The Purpose}
\target{Resphan purpose}
The \resphan{} race had a big purpose: 
To improve themselves, to evolve, and finally to attain perfection. 

\target{Resphain grow stronger}
And they did. 
Their numbers might continue to dwindle, but the ones who survived were the strongest, and they kept growing stronger. 

It helped that \iquin{} gave them a lot of power. 

\target{Resphan experiments on Humans}
The \resphain practice a lot of eugenics.
They want to improve and purify their race. 
To this end, they perform a lot of experiments on \humans:
Breeding and sorcery and medicine. 
They hope to find results from \human research that will carry over to the \resphain themselves. 
Compare to the Nazis and their race theory. 
Especially \Mystraacht had some strict ideals of physical purity, strength and health, tying in with their macho warrior ideology.





\subsubsection{Religion}
The \resphain had almost no religion.
They did not form any formal religious beliefs because they feared to think about it.
If they began to ponder the nature of the universe and the meaning of life, they would have to factor in their own nature, and thus their own origin.
They knew their origin: They were engineered by the \banes.
Try as they might, they could not escape that fact.

They revered the Lord of the \SitraAchra, but not much.
Some few took to worshipping the \SitraAchras as gods, but most could not bring themselves to do that.
They \hr{Resphain fear Banes}{feared the \banes} and did not like to talk or even think about them, for the \banes were horrible monstrosities.
They served the \banelords in name, but in their everyday lives the \resphain would prefer to pretend the \banelords did not exist.
They certainly had no desire to worship them.

So the \resphain shied away from the subject of the meaning of life.
Therefore they never developed much in the way of religion and mythology and philosophy.

They had some kind of psychological need for religion, like mortals did, but they filled that need with superstition and informal magibabble such as invoking their \matrices as \quo{gods}, as if the \matrices could be prayed to.
They saw their \matrices and the Heart as something great and powerful to swear and curse by. 
They knew the \matrices did not answer prayers, but like some atheists do, they sometimes succumbed to the psychological urge to pray.
Besides, they \emph{knew} how the \matrices worked (sort of), so it could not form the basis of a real religion. 
(Religions are based on ignorance and fiction and mumbo-jumbo.)







\subsubsection{Sex and gender roles}
The \resviel{} are highly prized for their scarcity and because they are the key to propagating their race. 
There is a lot of chivalry in the \resphan{} cultures (although some dynasties more than others). 

A \resvil{} must always be kept as healthy as possible, because it makes her more fertile. 
Therefore, it is customary at meals that a \resvil{} always be offered the best part of the body that is eaten. 

\Merkyrah{} \hr{Merkyran monogamy}{practiced monogamous marriage}, but that was crazy, so pretty much all later \resphan{} dynasties and tribes abolished that practice in favour of polyandry or promiscuity.

In most \resphan{} societies, a \resvil{} had a duty to do her utmost to get pregnant. 
There was great status in getting pregnant and bearing children. 
During her fertile periods, a \resvil{} was expected to have as much sex as is reasonably possible, but also required to be selective, so as not to risk letting an inferior \resphan{} breed. 
So she should have sex with as high-quality \resphain{} as possible. 
She held the key to her race's future in her womb, and it was her duty to do her best to become pregnant. 
(Outside her fertile periods she was more free to have sex as she chose.)

\target{Resviel do not fight}
For this reason, \resviel{} were also usually not expected to fight as warriors. 
They should support the \resphain{} in war, but not fight on the frontline. 
They were too valuable for that. 
\Mystraacht{} abolished this principle, though, and \hr{Mystraacht amazons}{had plenty of \quo{amazon} \resviel}. 





\subsubsection{Vampire lords}
\target{Resphan vampire lords}
Some \ashenblood \resphain lived among mortals as vampires.
\quo{\Reavers} they were called. 





\subsubsection[Yurideth]{\Yurideth}
\target{Yurid}
\index{\yurid}
The \resphan{} dynasties kept some captive \resviel{} as sex/breeding slaves. 
They were called \yurideth{} (singular \yurid{}) and were universally looked down on with contempt. They were slave whores who had lost ownership of their cunts and wombs, otherwise a \resvil{}'s most prized possession.
They were kept imprisoned and subdued with spells so they could escape or defend themselves. They were raped continuously by \resphain{} deemed fertile. The purpose was to get the slaves to breed. Every dynasty wanted to breed more children. 

The word \emph{shiphchah} or \emph{shifchah} is Hebrew for \quo{maidservant} or \quo{slave girl}. 

When it was detected that a \yurid{} was pregnant, she was quickly sedated and sent into a sorcerous sleep, to prevent her from harming herself and the baby. 

The \yurideth{} hated their fate and would often try hard to spite their captors by ensuring that they do not bear any children. Some of them got their spirits broken, however, and ended up as wretched, servile things that willingly tried to please their captors. 
(These were considered repellent abominations by their fellow \resphain{}, including their captors. Such submission was unworthy of a \resvil{}. A \resvil{} was supposed to guard and protect and take care of her womb with dignity.)

\Resphain{} born as the children of \yurideth{} were considered to have lower status than \thelyadeth. They were called \hr{Gessur}{\gessurim} (singular \hr{Gessur}{\gessur}). 

The Cabalist dynasties only openly admit to having \yurideth{} captured from \Kezerad{} and \Baelzerach. 
But in reality, they also have \yurideth{} captured from each other, kept secret. 
Many of these have lived as \yurideth{} ever since before the days of the Cabal, but some have been kidnapped in the time since then, when the dynasties were officially allies. This is extremely secret. 
(Although everyone suspects that it is going on.)

As part of the \hs{Consolidation} (the big process of forming the Cabal and allying the three Cabalist factions), every dynasty released several \yurideth{} and let them return to their families. They told the other dynasties that they had now released all of their Cabalist \yurideth{}, but they all lied and kept some in their dungeons/harems.
At the time of the \thirdbanewar, there were some \resviel{} alive who carried deep scars from having spend hundreds of even thousands of years in slavery as \yurideth{}.

\Yurideth{} are kept in seraglios. A seraglio is often also equipped as a debauched pleasure chamber. Or torture chamber, for the unfortunate \yurid{}. The slave whores have no rights, remember. \Resphain{} with sadistic tendencies to go them to inflict all the torture on them that they cannot inflict on free \resviel{}. Some \resphain{} have a nasty habit of taking it out on a \yurid{} whenever they have some grudge against a \resvil{}.

Each dynasty has about a dozen \yurideth{}. It is hard to keep the slaves healthy, for many reasons (among other things, the spells used to sedate them are unhealthy), so they tend to die after some few centuries. 

Some \yurideth{} are maimed, with spells cast on them to suppress their regeneration. 
Cutting off the wings is particularly common, since it is highly demeaning and de-\resphan{}izing. It marks them as less than pureblood \resphain{}, little better than \bezedeth{} or even mortals.

But a \yurid{} still had hope. 
If she escaped, she could heal all her physical wounds (including regrow her wings) and perhaps come to terms with her psychological wounds. 
\Resviel{} were strong. 









\subsection{Equipment}
\target{Resphan equipment}
See also the general section on \hs{technology}. 
(Really. Do it.)





\subsubsection{\Bane sorcery and forbidden books}
\target{Resphain and forbidden books}
The \resphain possessed a number of \quo{forbidden books}, pertaining to the \SitraAchras and their magic or other dark cosmic phenomena. 
These were dark and mysterious works of a blasphemous nature, horrible and repulsive because they went against the world-view that the \resphain liked to maintain. 

Especially, everything that had to do with the \SitraAchras was considered \quo{blasphemous} and \quo{banned}. 
\hr{Resphain fear Banes}{The \resphain feared the \banes} and hated to talk about them or remember the fact that the \banes existed.

The \resphain had a number of texts on \SitraAchra sorcery.
They could cause madness even in \resphain.

The books were not literally forbidden. 
It was not illegal to read them, but it was considered in very bad taste. 
No respectable \resphan would sully himself with such dark sorcery. 

The poem \WanderersInDarknessEmph was one such \quo{forbidden} text. 

But dark though the \pps{\resphain} books might be, \hr{Dragons have dark knowledge}{the \dragons possessed even darker knowledge}. 





\subsubsection{Crystal technology}
\index{technology!crystal technology|see{crystal}}
\index{crystal}
\target{Resphan crystal technology}
\target{Resphan crystals}
\target{Glass armour}
The \resphain{} had technology based on crystal. 
They wore \armour made of glass onyx and other crystals. 





\subsubsection{Monsters}
The \resphain{} tame monsters and use them as beasts of war. 
These include the \hr{Umbra}{\umbrae}. 

And also a species of crawling, somewhat lobster-like things (Cthulhu horror!). 
These things crawl on the sides of the tall buildings of \Nyx. e
The \resphain{} outfit them with cannons, like in the movie \cite{Movie:D-War}. 





\subsubsection{Technology}
\target{Resphan technology}
\index{technology!\resphan}
The \resphain{} inherited much of the \hr{Bane technology}{\ps{\banes}{} technological artifacts and knowledge}. 

They had them \hr{Merkyran technology}{already in \Merkyrah}. 
Back then they didn't know what to do with it. 
In later ages they learned much more. 

The \resphain{} were inherently much more creative than their \bane{} sires and were able to create new inventions. 
But at the time of the \thirdbanewar they were still far from the level of the \voyagers, and there was still much of their stolen \voyager{} technology that they did not understand. 

Some pieces of technology the \resphain used included:
\begin{itemize}
  \item \hr{Glow-moss}{\Glowmoss}.
  \item \hs{Graph-glass}.
\end{itemize}

\target{Resphan dead technology}
\Resphan technology focused much on metal, glass and crystal.
They are supposed to have a \quo{dead}, \quo{artificial} theme.
As opposed to the \dragons, who, \hr{Dragon living technology}{with their living technology}, have a more \quo{living} theme. 






\subsubsection{Weapons}
\target{Resphan weapons}
A lot of \resphain{} dual-wield weapons. 
But not huge-ass weapons like \senain{} or \belthradeth. 
They are too long and unwieldy. 

See also the section on \hr{Resphan equipment}{\resphan equipment}. 

\Resphan swords and other melee weapons were covered with fields of shimmering energy.
Like forceblades from \emph{GURPS} and vibro-blades from \emph{Rifts}.
In all sorts of fantastic \colours. 

Remember that different weapons are connected to the different Paths of \hr{Resphan martial arts}{\resphan martial arts}. 



\begin{gloss}
  \gitem[\belthradeth]{\belthrad}
  \target{Belthrad}
  \index{\belthrad}
  The \belthrad{} (plural \belthradeth) is a class of sword, favoured by \Mystraacht. 
  A \belthrad{} is shorter and broader than a \hr{Senaan}{\senaan}. 
  It has a shorter range, but is faster and packs a meaner bite. 
  It is considered a less sophisticated and more brutal weapon. 
  
  The blades \hr{Ascaril}{\Ascaril} and \hr{Scaleron}{\Scaleron} were \belthradeth. 
  
  
  
  \gitem{guns}
  \target{Ghijed}
  \index{\ghijed}%
  They often wield pistols. 
  A typical \resphan{} pistol is the \ghijed{} (plural \ghijedeth). 
  
  
  
  \gitem[\kilghain]{\kilghan}
  \target{Kilghan}
  \index{\kilghan}%
  \Resphain{} almost never use conventional shields. 
  It is not effective. 
  Instead, they wear vambraces on their wings, called \kilghain{} (singular \kilghan). 
  These are made of metal or crystal and used to parry attacks. 
  They come in many shapes and sizes.
  Some are narrow ridges on the very edge of the wing, others are big things covering the entire wing. 
  (Like the metal wings seen on the art for \cite{SymphonyX:ParadiseLost}.) 
  
  
  
  \gitem[\ruthiel]{\ruthil}
  A \ruthil{} (plural \ruthiel) was a \resphan{} sword, designed by \TiphredSerah. 
  They were shorter than \senain{} and ligher and slimmer than \belthradeth. 
  They were especially favoured by \resviel, who usually relied on speed and skill rather than brute force. 
  
  \Ruthiel{} were often dual-wielded or used with a sidearm such as a pistol. 
  
  
  
  \gitem[\senain]{\senaan}
  \target{Senaan}
  \index{\senaan}
  A \senaan{} (plural \senain) is a \resphan{} sword, designed by \CiriathSepher. 
  It is extremely long and slim, because it must reach further than the \resphan{} wielder's wings, and it must be usable against large opponents such as \dragons. 
  It has a long haft. 
  It resembles a Japanese \zanbatou, or almost spear or naginata. 
  
  The \senaan{} is probably the biggest \resphan{} weapon that can still be called a sword. 
  There exist even longer weapons that resemble lances or halberds more than swords. 
\end{gloss}









\subsection{\NeoResphain}
\target{Neo-Resphan}
\index{\neoresphan}
The \neoresphain were designed to be the next, improved generation of the \resphan race. 
They were an experiment by \hr{Azraid}{\Azraid} and other visionary scientists. 
They were meant to be the successors of \resphain. 
They are an attempt to perfect the \resphan{} race, \hr{Resphan purpose}{as is their purpose}. 

%Who are they? Are they the very first batch of experimental \resphain, or are they a new, stronger variant? 
Perhaps they even rival the \satharioth{} in power? 

This project was, in a sense, the Cabal parallel to \ps{\Secherdamon} \Vizsherioch-project. 





\subsubsection{Appearance}
\target{Neo-Resphan appearance}

\begin{itemize}
  \item 
    Maybe they are \bane-like, with semi-blank \bane-like faces. 
    
    Compare to the woman on the cover of \bandalbum{Hour of Penance}{The Vile Conception}.
    
  \item 
    Maybe they can change back and forth between \human-like and \bane-like forms. 
    In their \bane-like form they are hideous and loathsome to behold even for \resphain. 
    
  \item 
    Maybe they resembles Spawn from \cite{ToddMcFarlane:Spawn}, 
    or a Xenomorph from \cite{Movie:Alien}, 
    or Venom from \cite{StanLeeSteveDitko:SpiderMan}. 
    
  \item 
    Alternately, they might look like the angels in the Bible: 
    With six wings, and four faces (the three of which are animal faces).
    And iron teeth and brass nails (or a metal that resembles brass in \colour). 
    And covered with eyes. 
    And ten horns (that may have eyes and even mouths on them). 
    And four arms (each arm under a wing), where each arm has its own face and wings. 
    See especially Daniel 7, Revelation 4, Ezekiel 1 and more. 
    Look up \quo{Hierarchy of angels} on Wikipedia. 
  
  \item 
    Maybe they shine and burn with white or black fire. 
    Beautiful and terrible. 
  
  \item 
    \target{Neo-Resphain resemble Noggyaleth}
    Maybe they are covered with eyes like \noggyaleth.
    Maybe they have other \noggyal-like traits. 
    Maybe \hr{Neo-Resphain and Noggyaleth}{\noggyal matter was used to create them}. 
  
  \item 
    Maybe they are \dragon-like. 
\end{itemize}



\citeauthorbook[\quo{Death and Punishment}, p.243--245]{%
  AlanUnterman:TheKabbalisticTradition%
}{%
  Alan Unterman%
}{%
  The Kabbalistic Tradition%
}{
  He immediately opens his eyes and sees an angel whose width is from one end of the world to the other.
  From the soles of his feet to the crown of his head, he is full of eyes, his garments are fire, his clothing is fire, he is entirely fire and he has a knife in his hand on the end of which a bitter drop is suspended.
  From this drop the man's body will become putrid, and from this his face will turn green. 
}





\subsubsection{History}
The \neoresphain{} were developed during the course of {\SentinelsofMithEmph}. 
\hr{Teshrial}{\Teshrial} partook in \hr{Teshrial's experimental weapon}{some \neoresphan-related experiments} in \hr{Teshrial's quest}{his quest} to slay \QuessanthIshnaruchaefir. 

\hr{Azraid}{\Azraid} led and oversaw the \neoresphan project, but \hr{Azraid never became Neo-Resphan}{he never underwent the treatment himself}. 

At the end of {\SentinelsofMithEmph}, after \hr{Daggerrain falls}{the fall of \Daggerrain}, they were almost ready. 
\Azraid{} predicted that they would be the future of their people, potentially more powerful than \banes, \resphain{} or \satharioth. 






\subsubsection{Mature state of \resphain}
The tranformation from \resphan to \neoresphan was a metamorphosis, like what insects and amphibians do. 
In a sense, the \resphain were neotenic, like axolotls: 
They had the natural potential to metamorph into a more powerful mature form (the \neoresphan form), but they were inhibited and unable to leave their immature, undeveloped \human-like forms. 
The \neoresphan project was intended to bring the \resphain out of their childhood, allow them to break their weak shells.





\subsubsection{\Noggyal connection}
\target{Neo-Resphain and Noggyaleth}
Maybe \noggyal matter was used to create the \neoresphain. 
The \banes \hr{Banes and Noggyaleth}{wanted to merge with the \noggyaleth}, remember.
The \neoresphain with \noggyal genes might have been a step in this direction. 

Maybe the \neoresphain \hr{Neo-Resphain resemble Noggyaleth}{physically resembled \noggyaleth} to reflect this. 




\subsubsection{Skills and powers}
\target{Neo-Resphan powers}
\target{increased vampire powers}
The \neoresphain{} have increased vampire powers and can drain life energy from living targets at a distance. 
(\hr{Resphan vampirism}{\Resphain{} are vampiric}, remember.) 









\subsection{Physique}





\subsubsection{Appearance}
\Resphain{} look like \humans, but taller and more beautiful, more perfect. 
Most markedly, they sprout a pair of great feathered wings on their backs. 
On some \resphain{} these wings are tattered and torn\dash why? 
Obvious candidates for this are the \Kezeradi{} survivors, and maybe Ramiel, if he is able to assume \resphan{} form at times, before he has regained his full memory (and, with it, his full perfection). 

By nature, \resphain{} have glossy onyx black skin. 
(This was a side effect of the mingling of \SitraAchra and \nephilic{} genes.)
The skin \colour varies a bit, though, even among purebloods. 
It can be slightly gray, or bluish, or chocolatey brown, or very dark red.

\Resphan hair and feathers can be black, but they can also be other \colours: Grays, browns, reds, or even (rarely) white, yellow or orange.
All a \resphan's body hair and feathers will usually be of one single \colour.

Many dye their hair (white or silvery \colours being most common), and some even dye their skin. 





\subsubsection{Beards}
\index{beard!\resphain}
Male \resphain{} rarely wear beards. 
They can grow them, but most shave them off. 
The reason for this is that \nephilim{}, and \nephil-born \hr{Ashenblood}{\ashenbloods}, are very hairy and bearded, and purebloods want to distance themselves from these lower, more bestial people. 





\subsubsection{Beauty}
To \human{} eyes, the \resphain{} seem superhumanly perfect and beautiful\dash which they are, for they are the superbeings of whom \humans{} are but a shallow copy. 
Moreover, \humans{} were designed as slaves of the \resphain, so they are genetically disposed to love, admire and worship the \resphain. 
A \resphan{} visage inspires instinctive feelings of love and subservience in a \human. 

To \nephilim, \resphain{} appear mighty, godlike, larger-than-life, but not necessary glorious and lovable. 
Rather, their beauty is a cold, alien, frightening kind. 
Still, they have a certain alluring aura of power and mystery, 
\Nephilim{} (and other creatures) may serve \resphain{} out of fear or awe, but not out of the same instinctive love and admiration that \humans{} do. 





\subsubsection{Ethnicity}
In terms of facial traits the \resphain{} most closely resembled East Asians such as the Japanese. 





\subsubsection{Power}
It was known that \satharioth were more powerful than other purebloods, and that purebloods were more powerful than \bezedeth. 

It was commonly believed that \ketherain were more powerful than \thelyadeth, but this was a dubious assertion.


See also the section on \hr{Dragons vs Resphain in power}{\dragons versus \resphain} and about \hr{Umbra power}{\umbra power}.






\subsubsection{Size}
\target{Resphan size}
An adult male full-blooded \resphan{} is typically 300 cm tall. 

The \resviel{} are shorter. 
200-250 cm on average. 
They are physically not as strong as the \resphain, but faster and more agile. 

\Bezedeth{} are smaller than purebloods.

\target{Resphan vs Aryoth size}
\hr{Aryoth size}{Compared to an \aryoth}, a \resphan is as tall or taller, but much less massive. 








\subsection{Politics}





\subsubsection{\Banes}
The Cabalist dynasties of \KiriathSepher, \TiphredSerah{} and \Mystraacht{} served the Lords of the \SitraAchras, but they preferred not to admit it. 
They occasionally swear by the \banelords, but rarely. 
They were hesitant to admit that there existed those even mightier and darker than they. 
The vain \KiriathSepher{} most of all.

\target{Resphain fear Banes}
Moreover, the \resphain feared the \banes. 
\Resphain feared \banes because \hr{Banes eat souls easily}{\banes could eat souls very easily}. 
That was what they \emph{did}. 
A few \lesserbanes were, on average, no match for a \resphan. 
But on the off chance that the \banes were to overpower the \resphan, they would be able to eat him with no trouble at all. 
The \resphain knew this, and it made them uneasy and fearful around \banes 
Even a single \lesserbane was enough to badly unsettle a \resphan. 
The \resphain recognized the Banes as the Cosmic Horrors they were.

This also allows me to have more \trope{CosmicHorror}{Cosmic Horror} feeling in the story, even from an immortal POV.

A consequence of this was that any books and the like pertaining to the \SitraAchras and their magic, and other dark cosmic phenomena, \hr{Resphain and forbidden books}{were considered \quo{forbidden}}. 

Note: 
The correct term was \quo{Lords of the \SitraAchras}. 
\quo{\Banelord} was considered a crude, rude term. 
By the \resphain, that is. 
The \banes themselves did not care. 





\subsubsection{Dynasties}
The \resphan{} nobility is divided into five great dynasties: 
\KiriathSepher, \TiphredSerah, \Mystraacht, \Kezerad{} and \Baelzerach. 

The dynasties of \KiriathSepher{} and \TiphredSerah{} are loyal to the \SitraAchras. 
\Mystraacht{} claim loyalty but secretly plot against the \SitraAchras. 
\Kezerad{} and \Baelzerach{} have forsaken their creators entirely. 

Some of the rebels are the \satharioth{} who drank the blood of \Nexagglachel{} and inherited his greed, his ambition, and \hr{Curse}{above all his hatred of the \banes}. 
These \resphain{} see themselves as true \Miithians, the ultimate heirs to the legacy of \Miith{} and \Erebos{} alike, possessing both \nephilic, \draconic{} and \bane blood. 

The dynasties count as members only pureblood \resphain{}. 
The \ashenblooded{} commoners are not members of the dynasties, although most of them are in service to one dynasty. 





\subsubsection{Economy}
\target{Resphain's parasitic economy}
The \resphan{} dynasties base their economies mostly on the \quo{stock} they hold in all sorts of mortal kingdoms. 
Through the Cabal they manipulate mortals and make sure that some food and other goods are channeled to them. 
That's what they live on. 

But there are some places where the \resphan{} have farms and the like directly controlled by them. 

\hr{Nyx is a parasite Realm}{\Nyx{} is a parasite Realm}.





\subsubsection{\Noggyaleth}
See the section on \hr{Resphain and Noggyaleth}{\resphain and \noggyaleth}. 









\subsection{Psychology}





\subsubsection{Delight in conflict}
\target{Resphain love conflict}
\Resphan psychology differed from \human psychology.
In real life it is often asserted that all true happiness derives from \quo{love}, and that all other desires are hollow and unfulfilling in the end.
This may or may not be true in RL, but it was not true for the \resphain.

\Resphain derived genuine pleasure from conflict, from living out \quo{negative} emotions such as anger, hate and pride.
To some extent, they \emph{enjoyed} anger and pain.
It was an affirmation that they were alive and active.
Pain reminded them of their bravery and daring.
Anger reminded them of their will, their goals and motivation.

See also the section on how \hr{Races love war}{the immortals loved war}.





\subsubsection{Driven temporarily mad}
The \resphain were driven to a crazed fury during the War of Awakening (the rebellion against \Merkyrah).
They were mad with bloodlust and with the desire for change.
It is what made them so insane and violent.
It is what made even sensible and compassionate \resphain commit terrible, bloody atrocities.
It is what caused them to embrace such a twisted \trope{ReligionOfEvil}{Religion of Evil} and wage a war of genocide against their own people and their hapless servitors.
Their true, chaotic power brought a terrible rush.
Only later as they got more used to their powers did they learn to get over this rush and still maintain a level head.
Many came to repent their evil deeds during the rebellion.
Many of these repenting ones ended up joining \Kezerad.





\subsubsection{Possessiveness}
\target{Resphain are possessive}
Male \resphain{} are always possessive of their women. 
Every \resphan{} wants to \emph{own} his \resviel{} and thus control their pussies and wombs. 
But it never happens. 
The \resviel{} are too strong to let themselves dominate. 

Except in the case of \Zachirah, who was actually able to keep \hr{Zachirah's slave Resviel}{slave \resviel}. 
That was one of the reasons he was so admired. 





\subsubsection{Racial memory}
The \resphain{} have inherited some degree of racial memory from their \SitraAchra forebears. At least, the oldest, greatest, most pureblooded of the \resphain{} have. They remember vague impressions of the  \hr{Voyager}{\voyagers}. 





\subsubsection{Social intelligence}
\target{Resphain are social}
\Resphain{} are social creatures, very much conscious of social things such as etiquette and fashion. 
(One particularly strong manifestation of this was the \CiriathSepher{} \quo{\hs{Dance}}.)
They are highly empathic and good at understanding emotions. 
But they are also vulnerable to social pressure, and their pride and fear of social rejection can be used against them by a clever manipulator. 

Their social intelligence allows the \resphain{} to form a highly organized and efficient society. 
This is probably their most important edge in their war against the \dragons: 
They are better organized than \hr{Dragons are not social}{less social \dragons}. 

\target{Resphan hypocrisy}
Their social nature also makes the \resphain{} very scheming and hypocritical, as opposed to the \dragons, who are \hr{Draconic sincerity}{more sincere}. 





\subsubsection{Taboos}
The \resphain, \hr{Resphain are social}{being highly social beings}, are obsessed with etiquette and taboos. 
Some taboos that many \resphan{} cultures share include: 
\begin{itemize}
  \item 
    You do not talk about birth and bloodline directly. 
    At least, not your own, and not that of those of lower status. 
    You subcommunicate \quo{\ashenblood}, but you don't say it. 
    (It's OK to talk about those of higher blood, though.)
  \item 
    Likewise with age. 
    Age gives some kind of status, but it's the unspoken kind, not the kind you bring up in conversation. 
  \item 
    You do not praise yourself or call yourself by titles (such as \quo{\hr{Ketheran}{\ketheran}}). 
    But it is OK to praise others, especially those of higher status than yourself. 
    So a high-ranking \resphan{} will often keep a \quo{herald} around to introduce him with all his due pomp and ceremony. 
    \subitem
      The \Mystraacht{} do not have this rule. 
      They flaunt their greatness. 
      They revel in pride and violence. 
      This is actually one of the points of conflict between \Mystraacht{} (who \quo{embrace their nature}) and \KiriathSepher, who are sophisticated and foppish. 
\end{itemize}





\subsubsection{Wings and their role}
\target{Resphan wing body language}
Pureblood \resphain{} have wings. 
These play a major role in their body language. 

\begin{itemize}
  \item 
    Wings folded across the back (so the tips cross) is modest and polite. 
  \item 
    Wings hanging loose from the shoulders is casual. 
  \item 
    Wings spread wide signifies pride, or very-casualness. 
    It is rude except with inferiors or close friends. 
  \item 
    Wings half-spread and raised is threatening. 
\end{itemize}

They use special chairs with holes for wings, and special wide sofas with wing-rests so you can spread them wide (casual use only). 

\Resphan{} wings are stronger and more durable than they look. 
In combat they may be swung as weapons. 
In war a \resphan{} might wear bladed shins on the wings, and maybe even full metal plating over them. 
Compare to the illustrations for the album \bandalbum{Symphony X}{Paradise Lost}. 

\target{Resphain enjoy flying}
The \resphain{} enjoy flying. 
Where mortals, especially those of high status, often disdain walking and resort to riding on beasts or in carriages, \resphain{} love using their wings. 
The wings symbolize their superhuman status, and they are proud of their ability to fly. 
Especially because only purebloods have them. 

\target{Resphan wingspan is important}
Wingspan is an important aspect when measuring how attractive or impressive a \resphan{} looks. 
Just as important as height. 









\subsection{Servants and slaves}
The mortal underlings of a \resphan dynasty were called \hedrim, singular \hedor. 
The \hedrim[\Mystraacht], for example, were the \hedrim that served \Mystraacht. 

There are some sickly, degenerate \humans{} who work as the \ps{\resphain}{} slaves. 
They do menial labour in foundries, mines, the bowels of ships and the like, far out of the sight of their masters, who do not want to see these ugly wretches. 

There are also some \quo{elite} slaves, bred to be beautiful and healthy, and groomed and nurtured to preserve them. They serve the \ps{\resphain}{} immediate needs as manservants, butlers and sex slaves. 
These upper slaves sometimes command slaves of their own. 





\subsubsection{Livery}
\target{Resphan slave livery}
\CiriathSepher{} slaves were dressed nicely in clothes that identify their owner through heraldry and symbolism. 

\Mystraacht{} slaves, on the other hand, were often naked, wearing only a metal collar to mark their owner. 
The \Mystraacht{} were fond of very direct displays of dominance. 





\subsubsection{\Naorim}
\target{Resphan food slaves}
\target{Naor}
\index{\naor}
The \naorim{} (singular: \naor) are a high caste of slaves. 
They exist to be eaten by the \resphain. 

The food slaves are considered holy among other slaves and have very high status. 
They are religiously dedicated to what they see as a sacred duty to their living gods, and are happy to die and serve the \resphain{}, to be devoured with body and soul by their beloved masters. 

\target{Communion}
\index{Communion}
The \naorim{} are taught that being eaten is something sacred. 
It is ritualized and called the Communion. 
They are told that when they are eaten, they will \quo{become one} with their \resphan{} gods and live on forever inside the \resphain, inseparable from them in body and soul. 
This is sort of true\ldots{}

\target{carver}
\target{Gelveir}
\index{\gelveir}
\index{carver}
The Communion subject is killed by a specially trained \emph{carver} (a \human) with a special ritual dagger called a \gelveir. 

The Communion rite is symbolically based on \hr{Thanatzil must die}{\ps{\Thanatzil} sacrifice}: 
\quo{%
  The legendary Messiah of the \resphain{}, who gave up his life that his people might live.}

It is customary for the \naor{} chosen for Communion to sometimes be honoured with a special boon on her or his last night: 
To have sex with a \resphan{} or \resvil. 
This will be the slave's first and only sex ever, and therefore something special and precious.  

\target{epitaph}
\index{epitaph}
At the Communion, the \naor{} is expected to recite an epitaph, praising her masters and asking them to accept her sacrifice. 
By tradition, the \naor{} should compose her own epitaph, so it is personal and sincere and shows her personal dedication and love for her masters. 









\subsection{Skills and powers}





\subsubsection{Foreign languages}
\target{Resphain speak poor Draconic}
The \resphain \hr{Resphain learn Miithian languages}{tried to learn \Miithian languages when they came to \Tembrae}. 

Only very few \resphain ever mastered the \Draconic tongue. 
Partially because it was very hard, and partially because it was hard to find anyone willing to teach them. 
No \dragon ever taught a \resphan language. 
But the \resphain were able to find a few \quiljaaran who spoke \Draconic and could be persuaded or coerced into teaching. 
These \quiljaaran did not speak perfect \Draconic, though. 
And no \resphan ever learned \TrueDraconic. 






\subsubsection{Martial arts}
\target{Resphan martial arts}
\target{Paths}
\index{Paths}
The \resphain{} practiced a number of martial arts. 
The most common one was a system called the \quo{Paths}. 
It was a supernatural martial art, combining physical body training, weapon skill, psychometabolism and sorcery. 
As the plural name suggests, the \quo{Paths} was split into a number of specializations.

There were three Paths: Darkness, Ice and Light. 

Note that despite the similar terminology, the Darkness/Light distinction has nothing to do with the distiction of \iquin/Light and \itzach/Darkness. 

See also the section on \hr{Weapons}{weapons} for more martial arts. 


\begin{description}
  \item[The Path of Darkness:]
    \target{Path of Darkness}
    \index{Paths!Path of Darkness}
    Focused on stealth, misdirection, subterfuge, mental attacks, illusion and necromancy. 
    
    Favoured by \TiphredSerah. 
    
    
    
  \item[The Path of Ice:]
    \target{Path of Ice}
    \index{Paths!Path of Ice}
    Focused on control, perfection, accuracy and a cool overview. 
    
    Favoured by \CiriathSepher. 
    
    
    
  \item[The Path of Light:]
    \target{Path of Light}
    \index{Paths!Path of Light}
    Focused on fire, lightning, ferocity and all-out-attack. 
    
    Favoured by \Mystraacht{} (despite how \Mystraacht{} aesthetics otherwise had a very \quo{dark} theme).
\end{description}










\subsection{Vampirism and cannibalism}
\index{cannibalism!\resphain}
\target{Resphan vampirism}
\target{Resphan parasitism}
\target{Resphan cannibalism}
\target{Resphan diet}
\Resphain{} are vampiric creatures. They must consume the blood, flesh, \hr{Life drain}{life-force} and \hr{soul-eating}{souls} of other creatures in order to sustain themselves and hold at bay the devouring \hr{Entropy}{Entropy} within them, a curse which they inherited from their \SitraAchra sires.

The \hr{Merkyrah}{\Merkyrans} refused this, and that made them weak, easily overcome by the \hr{Resphan rebellion}{rebels}. 

Most \Resphain{} happily eat \human, \scatha{} or even \resphan{} flesh\dash or \dragon{} flesh, whenever they can get their hands on it. They gain life energy this way, especially if it's powerful creatures like other \resphain{} or \dragons. 

The \resphain{} are naturally cannibals who crave the flesh of their own kind. 
\Draconian{} blood may be more potent, but the most delicious treat a \resphan{} knows is the blood and flesh of another \resphan{}. 
The mightier the better. 
Ideally, a \hr{Sathariah}{\sathariah}.

Like their \SitraAchra sires, the \resphain{} can grow in power especially by consuming the flesh and souls other \resphain. 
That is \hr{Cannibal Malachim}{what the \malachim{} did}.

They can live on mortal souls for a while, but in order to survive in the long run they must consume other immortals. 
This means that the \resphain{} are a parasitic race who must always expand, lest they be forced to turn to feeding upon themselves. 
They are a destructive, invasive species. 
That is why the \feud{} \hr{Fighting for survival}{can never be resolved}. 
In the millennia after the \hs{Incursion}, the \resphain{} were unable to expand as much as they needed to, and their population dwindled as a result. 
The Cabalist dynasties waged wars against \Kezerad{} and \Baelzerach{} in order to sate their hunger for immortal souls. 




\subsubsection{\Bezedeth need less sustenance}
\Bezedeth were smaller and weaker than purebloods.
They had less lifeforce. 
This had the benefit that they did not need quite as much lifeforce to survive as purebloods. 
They could survive for a long time on mortal souls alone and only needed to consume immortal power once in a while or they would weaken.

Even if deprived of immortal sustenance for long, they would not die quickly but weaken, fall into torpor and slowly wither away over a course of decades.
Some would set themselves up as rulers of mortals\dash \hr{Resphan vampire lords}{immortal vampire lords}.





\subsubsection{Power and hunger}
\target{Resphan power and hunger}
The mightier a \resphan{} is, the hungrier he is, and the more he must feed. Otherwise he risks losing power, permanently. 

The \Malachim{} do not lose power permanently if they fail to feed. That is one of their superpowers. 




\subsubsection{\Resphan{} captives}
The evil \resphain{} keep enemy \resphain{} captive to eat. 
They are bound with spells so they cannot escape, and then killed, eaten and allowed to reincarnate over and over. 

It's more nourishing to eat the soul, of course, but \resphan{} meat alone is pretty tasty, too. 





\subsubsection{Sexual connotations}
Some of them develop a \quo{vore} fetish, take a sexual delight in eating flesh. Some take this to masochistic extremes, letting their own flesh be eaten and then using magic to regrow it. (The magic to regrow limbs is extremely expensive luxury. A \resphan{} needs to sacrifice something like three \humans{} just to regenerate his little finger.) 

Occasionally, a \resphan{} will bloodlet himself and drink his own blood as a kind of masturbation.

















\section{\Satharioth}
\target{Satharioth}
\target{Sathariah}
The \satharioth{} are a group of exalted \resphan{} lords, empowered with \draconic{} blood \hr{Fall of Nexagglachel}{stolen from \Nexagglachel{} of the \shaeeroths}. 

Ramiel and \Shiaraid{} belong to this group.









\subsection{Physique}
The \satharioth{} are terrible to behold. 

\lyricstitle{\SETolltheHounds{} p.108}{
  Through the gate\ldots{} he saw \emph{him} [Anomander Rake] approaching. 
  From the below city. 
  His forearms sheathed in black glistening scales, his bared chest made a thing of natural \armour. 
  The blood of Tiam ran riot through, fired to life by the conflation of chaotic sorcery, and his eyes flowed with ferocious will. 
}









\subsection{Culture}





\subsubsection{Social status}
\target{Sathariah social status}
The \satharioth had very high social status. 
They were the top of the \resphan aristocracy in all dynasties (except \Baelzerach, which had no \satharioth). 

This has several reasons:
\begin{itemize}
  \item 
    The \resphain who became \satharioth were high-ranking, influential ones to begin with.
  \item 
    \Satharioth were physically, mentally and metaphysically powerful, which made it easy for them to gain political power as well.
  \item 
    The \resphain had an \hr{Races love war}{inborn psychological tendency to idolize the fierce and violent}. 
    The \satharioth \emph{embodied} the fierce and violent, so the \resphain loved them. 
\end{itemize}












\subsection{Demographics}
Originally there were thirteen (\hr{Satharioth die}{surviving}) \satharioth{}, of which three were \resviel. 
Since then, four or so have been permanently killed. 
Three \satharioth{} have become \malachim: \hs{Ramiel}, \hr{Shiaraid}{\Shiaraid} and \hr{Ishicah}{\Ishicah}. 
One, \hr{Sithiyacaan}{\Sithiyacaan} has vanished. 
This leaves five \satharioth{} alive, active and kicking. 

The original thirteen are divided as follows: 
\begin{gloss}
  \gitem{\KiriathSepher} (4)
  \begin{enumerate}
    \item \hr{Azraid}{\Azraid} (alive). 
    \item \hr{Harbeth}{\Harbeth} (alive). 
    \item \hr{Mehaloch}{\Mehaloch} (slain in the Incursion). 
    \item \hr{Shehizol}{\Shehizol} (?). 
  \end{enumerate}
  \gitem{\Kezerad} (2)
  \begin{enumerate}
    \item \hr{Sithiyacaan}{\Sithiyacaan} (missing but alive). 
    \item One more female (\emph{not} \Aryal). 
  \end{enumerate}
  \gitem{\Mystraacht} (4)
  \begin{enumerate}
    \item \hr{Nathrach}{\Nathrach}.
    \item \hs{Ramiel} (\Malach). 
    \item \hr{Shiaraid}{\Shiaraid} (\Malach). 
    \item \hr{Zachirah}{\Zachirah} (slain by treachery). 
  \end{enumerate}
  \gitem{\TiphredSerah} (3)
  \begin{enumerate}
    \item \hr{Dorzand}{\Dorzand} (?). 
    \item \hr{Ishicah}{\Ishicah} (\Malach, \hr{Ishicah enslaved}{destroyed in \Ortaican{} captivity}). 
    \item \hr{Quelthah}{\Quelthah} (?). 
  \end{enumerate}
  \gitem{\Baelzerach} 
    The only great dynasty to have no \satharioth{} among its numbers. 
\end{gloss}










\subsection{Biology}
\subsubsection{Attempts to create more \satharioth}
Since the inception of the original \satharioth, there have been attempts at creating more of them. 
All have failed, for a number of reasons:

\begin{itemize}
  \item 
    The \satharioth{} and \banelords{} keep the technology a secret because they don't want competition. 
  \item
    To create new \satharioth{} you would need to consume an \uber-powerful \vertex{}. 
    \Nexagglachel{} was one of the greatest, most formidable \dragons{} that ever existed. 
    Few of his like exist today, and none of those are stupid enough to let themselves capture. 
\end{itemize}





\subsubsection{Fragments of \Nexagglachel}
\target{Fragments of Nexagglachel}
When he was killed, \hr{Nexagglachel}{\Nexagglachel} was symbolically dismembered, and each \sathariah{} consumed a part of his body: 

\begin{itemize}
  \item \Azraid: Brain.
  \item Ramiel: A claw. 
  \item \Shiaraid: \hr{Shiaraid's sexuality}{The part that let him endure pain}. 
  \item \Zachirah: Some sexual organ. 
\end{itemize}











\subsection{Psychology}





\subsubsection{\NexagglachelsCurse}
%\subsubsection{Hatred of the \banes}
\target{Curse}
\target{Nexagglachel's curse}
\target{Satharioth hate Banes}

\Nexagglachel \hr{Nexagglachel lives on in Satharioth}{did not perish but lived on inside the \satharioth}. 

The \satharioth{} and \ketherain{} harboured a deep-seated, irrational hatred of their \bane{} masters. 
The reason for this was twofold:

\begin{enumerate}
  \item 
    It was a a part of their heritage from the \banes, who were \hr{Bane cannibalism}{cannibalistic and patricidal by nature}. 
  \item
    It was amplified by the soul of \Nexagglachel. 
    He was \hr{Nexagglachel lives on in Satharioth}{dead, but he lived on in some form} in the blood of the \satharioth. 
    It is hinted that he \hr{Nexagglachel sacrifices himself}{willingly sacrificed himself} in order to \hr{Nexagglachel makes Satharioth hate Banes}{sow hatred} between the \resphain{} and their \bane{} progenitors.
\end{enumerate}

This hatred made the \satharioth{} \hr{Satharioth betray Banes}{betray the \banes} and other \resphain{} several times. 

The \resphain{} still fear \Nexagglachel{} and his corrupting influence. 
Perhaps they perform religious rituals to keep him at bay, so they can freely use his power. 
Compare to the Egyptian rituals used to pacify the \dragon{} Apep.





\subsubsection{Perhaps \Nexagglachel{} is \ps{\Daggerrain}{} blind spot}
\target{Nexagglachel is Daggerrain's blind spot}
But \ps{\Nexagglachel}{} ghost is clever, and he eludes and manipulates them. 
Perhaps he is \hr{Daggerrain's blind spot}{\ps{\Daggerrain}{} blind spot}, the factor that \Daggerrain{} never managed to understand and enter into his calculations. 
Or perhaps the entire \feud{} is actually very much a contest of sneakiness between \Nexagglachel{} and \Daggerrain, with the \dragonlord{} hiding from \Daggerrain, using the knowledge of the \banes{} gained from merging with the \resphan{} soul to hide from the \banelord{}, gain insight into his plans and subtly subvert them. 
This may also be what eventually allows \hr{Ramiel betrays Banes}{Ramiel to betray \Daggerrain{} and overthrow him}.





\subsubsection{The curse causes them to descend into madness}
\target{Madness of the Curse}
The curse causes the \satharioth{} to slowly descend into madness and dementia. 

\lyricslimbonicart{Dynasty of Death}{
  In the dark caves of oblivion\\
  bad blood rises from the Mega Therion [\Nexagglachel].\\
  From vast stalactite halls so undivine,\\
  through the misty corridors of time.\\
  As one lives one shall die.\\
  Ad noctum.
}

\lyricslimbonicart{The Yawning Abyss of Madness}{
  Again I drift the halls of wondering. \\
  The black castle of solitude. \\
  On the very edge of sanity \\
  in mental cryogenic interludes. \\
  I have slipped into the seventh, \\
  the seventh circle of Hell, \\
  in realms where deadly shadows \\
  infest every cell. 
  
  Internal ceremonies in ritual death. \\
  External bleedings for the demon of madness. \\
  Hide from the torture of the dazzling light. \\
  The demolition voice shall speak tonight. 
  
  While I'm staring down into the darkest pit, \\
  an ocean black as the night, \\
  so infinite deep and consuming, \\
  it swallows all life force with might. 
  
  Again I drift the halls of wondering, \\
  as I focus for the darkness to come. \\
  In anguish minds uplift the conquering \\
  to cross the line of death beyond.
  
  An abstract reality and bottomless insanity. \\
  To search for the powers to please\\
  the subconscious spirit of disease. \\
  Time found no remedy, \\
  cause winds of darkness was stealing me. 
  
  The yawning abyss of madness. \\
  A cryptic slaughter by hate. \\
  Darkness is the only survivor \\
  as evil dominion terminates. \\
  The yawning abyss of madness.
}

Some of them despair at the curse. 

\target{Satharioth despair at the curse}
\lyricsbs{Emperor}{Grey}{
  where lights are dim and shades of black are grey\\
  from the moment of arrival we are led astray\\
  with nothing but a distant cry from deep within a soul\\
  a wordless voice to guide us on the way
  
  desperately we name the voice and make the cries our own\\
  as if to deny the fact that we are all alone\\
  in solitude we mingle, disillusioned we fall prey\\
  where lights are dim and shades of black are grey\\
  I can always live another day\ldots{}
}





\subsubsection{Dreaming of \ophidian{} eyes}
\target{Satharioth dream of Ophidian eyes}
The \satharioth dreamt of \ophidian eyes. 
Frozen eyes. 
Eyes of black ice. 
Full of cold hatred. 

They were the eyes of \Nexagglachel, staring cold at them. 
Whenever they closed their eyes they found him staring back at them, and they remembered \hr{Nexagglachel's hatred}{his promise to destroy them}. 




\subsubsection{The curse lives on}
\target{Curse lives on}
Even after \ps{\Nexagglachel} short-term goal is accomplished and \hr{Daggerrain falls}{\Daggerrain{} falls}, the curse lives on. 
\Nexagglachel{} still hates the \satharioth{} and all the \resphan{} people, and still wants to see them suffer and destroy themselves. 















\section{\TiphredSerah}
A \resphan{} faction loyal to the \banes. 









\subsection{Aesthetics}

The \TiphredSerah{} often dressed like vampires, with robes and high collars. 
Their traditional \colours were blues, violets and purples.

The \TiphredSerah{} had a tradition for lacquering their fingernails in their dynasty \colours. 









\subsection{Culture}
The \TiphredSerah{} were more introverted and secretive than the other dynasties. 
They were schemers, in a colder, more rational, more long-term-thinking way than the \CiriathSepher{} with their petty intrigues or the \Mystraacht{} with their open and brutal animosity. 

They were sneaky, ninja-like guys. 

\target{Tiphred-Serah free-thinking}
They were also the most free-thinking of the dynasties. 

One could argue that they were nerds. 





\subsubsection{Government}
Their rulers were unknown, but they were represented by a Speaker. 
Several of their leaders were \resviel. 









\subsection{History}





\subsubsection{\ResphanWars}
\TiphredSerah \hr{TiphredSerah in the Resphan Wars}{got their butts kicked} in the \resphanwars. 





\subsubsection{Inception of the Cabal}
It was people from \TiphredSerah{} who \hr{Tiphred-Serah form Cabal}{founded the Cabal} and \hr{Tiphred-Serah formulate Unspoken Covenant}{helped formulate} the \hs{Unspoken Covenant}. 









\subsection{Politics}





\subsubsection{\Kezerad}
During the \hr{Resphan Wars}{\resphanwars} the \TiphredSerah{}, being \hr{Tiphred-Serah free-thinking}{the most free-thinking of the dynasties}, were the ones \hr{Tiphred-Serah and Kezerad ally}{most liable to ally with \Kezerad}. 

























\chapter{Gods and Aliens}















\section{\Archon}
\target{Archon}
\index{\Archon{} (plural \Archons)}
In Vaimon metaphysics, \Archons{} are supernatural beings and forces of nature. 
Different classes of \Archons{} include the \hr{Sephiroth}{\Sephiroth} and \hr{Qliphah}{\qliphoth}, who can be invoked to cast magic. 
The \hr{Malach}{\Malachim} are also considered \Archons. 















\section{Cosmic gods}
\target{Cosmic gods}
\target{Cosmic god}
\target{cosmic gods}
\target{cosmic god}
\target{Klatrymadon}
\index{cosmic gods}
I need to have some immensely powerful, mystic forces. Enigmatic gods whose motives are unknowable, but whose names are invoked in spells, and who will sometimes answer. Kind of super-\Qliphoth. 

They should be invoked early on in the story, side-by-side with \ps{\Ishnaruchaefir} name, to give him mythical status. 

Compare them to Klatrymadon and Zuranthus, Kur'oc and Gul-kor from the Bal-Sagoth mythology.

\lyricsbs{Bal-Sagoth}{Summoning the Guardians of the Astral Gate}{
Ka-kur-ra, I summon thee.\\
Zul'tekh Azor Vol-thoth.\\
Mighty Xuk'ul, arise.\\
Kur'oc, Gul-Kor, come forth.

The threshold looms, \\
(the star-way between dimensions stretches before me\ldots{}) \\
The Gate To That Which Lies Beyond yawns wide\ldots{} \\
Unspeakable forces gibber and pulsate in the Outer Darkness\ldots{} \\
Elder horrors dwell here, things which were ancient and revelled in sublime galactic malevolence when even Xuk'ul was naught but a bloated cosmic maggot, writhing and suckling at the breast of its amorphous mother\ldots{} \\
They-Who-Lurk-And-Breed-In-Limbo\ldots{} \\
the squamous sovereigns of the elder void!}

Perhaps the cosmic gods have a Nyarlathotep-like figure who is their representative. 









\subsection{A class above the \xss{} and \voyagers}
The \xss{}, \banelords{} and \voyagers{} are well above \dragons{} and \resphain{} in power, but they remain within the bounds of imagination\dash for the master races, not necessarily for mortals. 

But the cosmic gods are a class above them. They are as far above the \xss{} as the \xss{} are above \dragons{} and the \dragons{} are above \scathae. 

Comparing with the Cthulhu Mythos by H.P. Lovecraft and others:

\begin{itemize}
  \item Cosmic gods correspond to Outer Gods.
  \item Voyagers correspond to Elder Gods. 
  \item \XzaiShanns{} correspond to Great Old Ones. 
\end{itemize}








\subsection{Creating the \banes}
Possibly, the \banes{} were twisted by the design of a cruel cosmic god. See section \ref{Cosmic god creating the Banes}


















\section{\Daemons}
\index{\daemon}
\Pdaemons{} are physical creatures from \Machai{} or other \chaotic{} realms. 





\subsection{Mighty \daemonic{} races}
Have some mighty races of \daemons{} that can only be contacted and bargained with, rarely bound. Compare to the \quo{star-spawn of Cthulhu} in the RPG \emph{Call of Cthulhu}. 





\subsection{Bat-like \daemon}
Have some minor \pdaemons{} with a mouth filled with a broad row of dagger-like teeth\dash a truly wicked smile. Inspired by Clive Barker's \emph{Hell's Event} (from \emph{Books of Blood II}). 

Also, it has batlike ears. 















\section{Gods}
\index{gods}
\subsection{\Human{} gods}
Have a race of \human{} gods. They might be called the \quo{Titans} or something like that. 

They are actually descendants of the \Kezeradi. They may or may not know of their own origins, and they may or may not still have ties to \Kezerad. 

Daxian and Isxae are the foremost among their number. The Imetric god Eoncos is also one of them. 







\subsection[Scathaese gods]{\Scathaese gods}
And remember to have some \Ortaican{} and \Shurco{} gods. (Turco? Thurco? Sturco? Durco?)















\section{\Krakens}
\target{Kraken}
\index{\kraken{} (plural \krakens)}
The \krakens{} are native \Miithian{} gods. 
Billions of years ago they ruled the Realm. 
Then they became tired and went dormant. 















\section{\Maskim}
\target{Maskim}
\index{\Maskim{} (plural \Maskim)}
A group of supernatural beings in Rissitic metaphysics, considered dangerous and evil and sometimes invoked in curses. 
\quo{\Maskim{} take you} is a strong curse in Rissitic.















\section{\Voyagers}
\target{Voyager}
\target{Voyagers}
\index{\voyager}
The \voyagers{} were the ones who created the \banes, and possibly also the \nephilim{} and other \Miithian{} life. 

Compare them to \bandsong{Bal-Sagoth}{Voyagers Beneath the Mare Imbrium}.

See also: \bandsong{Bal-Sagoth}{As the Vortex Illumines the Crystalline Walls of Kor-Avul-Thaa}. 

\lyricsbalsagoth{The Scourge of the Fourth Celestial Host}{
  They possess power unparalleled\ldots{}\\
  Ageless, remorseless. Without pity or conscience.\\
  Manipulators of evolution on countless worlds.\\
  Gods of the stars\ldots{} the Celestial Host!
}

The \voyagers{} are quite alien, but they are still the most \human{} of the ancient races. 

\lyricstitle{\authorbook{\HPLovecraft}{At the Mountains of Madness}}{
  \tho{Whoever these creatures had been, they were men!}
}

Everything on \Miith{} is descended from the \voyagers. This is \hr{Why the Banes want Miith}{why the \banes{} want \Miith}!

\lyricsbalsagoth{Invocations Beyond the Outer-World Night}{
  The legacy of the First Ones, spawn of the Mera!
}









\subsection{Physique}
The \voyagers{} are alien, but somehow beautiful. At least to \humans. This is because \humans{} are created to serve the \banes{}, who in turn were created by the \voyagers{} and patterned themselves after them.

They are vaguely humanoid, but only vaguely.
%, shining white with great gossamer wings. 
They have a body and a head with some sensory organs on it. Then they have multiple limbs radiating in all directions, like a starfish. Four of these are bigger than the rest, and these sort of resemble humanoid arms and legs. Some of their limbs end in what look like gossamer wings.

Compare them to the Navigator from the \emph{Dune} TV miniseries.

Their wings extend into mystic dimensions Beyond and are only half visible\dash they shimmer in and out of the dimensions that mortals can see. 
The rest of their bodies can also \quo{disappear} into these obscure dimensions. 

Compare to the Mi-Go, as described in \emph{Delta Green} (\emph{Call of Cthulhu} RPG). 









\subsection{Arsenal}





\subsubsection{Servitor races}
The \voyagers{} had hordes of servitor races that dwelt with them and served them (different servant races in different aeons). Some envision this mythical time as a utopian age. 





\subsubsection{Technology}
\target{Voyager technology}
\index{technology!\voyager}
The \voyagers{} possessed ultra-badass technology.





\subsubsection{Ultima Thule}
Maybe I should have a mystic place far to the north, near \Miith{}'s North Pole, where old \voyager{} cities, millions of years old, are buried beneath the ice. 

The \nagalords{} know many of the place's secrets, but they are not telling. 
The \nagae{} know of the place and can be persuaded to lead people there, but they will not enter themselves. They know the danger. 

What is the danger? The \bladedpeople? Or shoggoth-like creatures, like in \authorbook{\HPLovecraft}{At the Mountains of Madness}.

\lyricsbalsagoth{In Search of the Lost Cities of Antarctica}{
  Beneath the ice, the endless ice \\
  of Pangaea's (now) axial (eternally frozen) frontier, \\
  entombed for countless millions of years\ldots{} \\
  the lost cities of Antarctica!
  
  Secrets locked within the ice, the endless ice of Antarctica,\\
  'Neath the peak of Erebus the First Ones sleep, Lords of Pangaea,\\
  Cities lost within the night, the frozen night of Antarctica,\\
  Pre-Cambrian, the Voyagers, beyond the stars, Lords of Pangaea.
}

Some people see dreams visions of \voyager{} cities in their golden age. These cities surpass \emph{everything} that has ever existed on \Miith{}. \emph{Except} the glorious and terrible dark citadels of the \hr{XS}{\xss}\ldots{} and the halls of the \hs{cosmic gods}. Have \Ishnaruchaefir{} comment on this. 

\lyricsbalsagoth{In Search of the Lost Cities of Antarctica}{
  Once, the coruscating spires of [the \voyagers] here offered their splendour to the heavens. \\
  Now, those spires gleam no more, \\
  save in dreams of verdant plains, \\
  save in dreams of time-lost citadels. \\
  
  Legacy of a utopia lost, \\
  forever enshrined 'neath the ice\ldots{}
  
  Before the Nine Continents were formed from Pangaea's shattered surface\ldots{} \\
  Hewn from the Pre-Cambrian rock, \\
  behold this primordial metropolis!
}

Also comparable to Bal-Sagoth's Ultima Thule. 









\subsection{History}
\subsubsection{Origin}
Perhaps the \voyagers{} were in turn created by an even mightier, more ancient cosmic race.

\lyricsbalsagoth{The Fallen Kingdoms of the Abyssal Plain}{
  Long ago, before the third of Earth's moons fell fiery from the star-seared sky, there were those whom we have come to call the First Ones. \\
  These men-who-were-not-men were the creations of the Mera, beings from the far reaches of the limitless cosmos, whose essence still flickers latently within the minds of all their disparate progeny.\\
  Praise the Mera, fathers of the First Ones, bondsmen of the K'laa, sworn foes of the Z'xulth!\\
  Sired in the great spawning vats beyond the fathomless deeps of the Pre-Cambrian sea, the First Ones throve.\\
  Those who were engineered to live on land duly constructed the grand Antarctic Megalopolis, ultimately becoming entangled in bitter conflicts with the hoary Serpent Kings before retreating into the subterrene depths of the vast inner world, whereas those First Ones that had chosen the embrace of the abyssal seas were the architects of vast and glorious submarine cities whose splendid spires and minarets towered proudly beneath the unfathomed waves.
}





\subsubsection{Craters and wrecks}
I should have some great craters and wrecks, at the sites where the \psp{\voyagers} great spaceships crashed down on \Miith{}, tens of thousands of years ago. 

Mystic power radiates from these craters, creating vast maelstroms of energy. Monsters and undead spirits feed on this energy, and so throng around the craters.

This is inspired by the vulcanic vent that Laura Daughtery discovers in the first episode of the TV series \emph{Surface}.





\subsubsection{Voyagers today}
\target{Voyagers today}
A few \voyagers{} survive on \Miith{}. They dwell in desolate-but-functional high-tech citadels in otherwise ruined cities. 

Compare to \authorbook{\HPLovecraft}{At the Mountains of Madness}.

Some of them function as remote, powerful \hs{cosmic gods}. Compare to the Elder Gods of the Cthulhu Mythos. 















\section{\XzaiShanns}
\target{Thzan-Tzai}
\target{Xzai-Shann}
\target{XS}
\index{\xzaishann}
The \xzaishanns{} were a race of alien monsters or gods. 
Some of the dwelt on \Miith, especially \hr{Machai}{\Machai}. 
They were creatures of immense power, possibly incorporeal. 
They were served by horder upon hordes of minor \mdaemons{} and \mdaemons.

They are inspired by the alien invaders from \bandsong{Bal-Sagoth}{As the Vortex Illumines the Crystalline Walls of Kor-Avul-Thaa}, and the Great Old Ones from the Cthulhu Mythos by H.P. Lovecraft and others. 

The \xss{} had many \quo{lords} but no \quo{king}, nor even much of an organization. 
They were creatures of Chaos, remember. 

Each \xs was seen as the god of some particular portfolio. 

Some \draconian philosophers speculated that when the different individual \xss seemed to have different powers and specialization (and came to be seen as gods of some \quo{portfolio}), it was perhaps not a result of the gods' actual powers, but rather their interests.
Perhaps \NerranKoss was just a philosophical god with an interest in history and such matters. 
So when people asked him such questions, he was more likely to yield a useful answer than most other \xss.
And thus he became seen as a god of occult knowledge. 
It was unknown if this theory was true, but it was accepted by several. 








\subsection{Names}
I should have a pantheon of the \quo{Eldest Lords of the \XzaiShanns}. Their names might include: 

\begin{itemize}
  \item Abraloth. 
  \item \hr{Khoth-Sell}{\KhothSell}. 
  \item \hr{Kyaethem Chrei Az}{\KyaethemChreiAz}. 
  \item \hr{Naath-Kur-Ramalech}{\NaathKurRamalech}. 
  \item Niil-shacht.
  \item Uruzgal. 
  \item Vol-croth. 
  \item Yoggranath. 
\end{itemize}

Other \xss{} include:

\begin{itemize}
  \item \hr{Hoth-Nrul}{\HothNrul}. 
  \item \hr{Ubloth}{\Ubloth}.
  \item \hr{Yolbaoth}{\Yolbaoth}. 
\end{itemize}

See the section on \hr{Individual XS}{individual \xss}. 










\subsection{Biology}
\subsubsection{Collective beings}
Perhaps each \xs{} is actually a collective being, a hiveminded colony of many smaller \daemons. 
Compare to \hs{coral reefs}. 









\subsection{Psychology}





\subsubsection{Speech}
\target{XS speech}
The \xss{} speak weird. 
Maybe in poems or koans. 

This is the closest thing a mortal (or immortal) mind can come to a translation of the \ps{\xss} alien thoughts into familiar concepts. 
Even the minds of great, wise immortals like \QuessanthIshnaruchaefir{} are horribly inadequate. 

The \xss{} speak in raw Aenigmata, raw Gnosis. 










\subsection{History}





\subsubsection{Empire}
The \xss ruled a vast interstellar empire spanning thousands of planets, perhaps millions. 
They knew secrets of dimensional travel that few races had achieved.
The \banes did not have such a dimensional travelling skill. 
This technology was one of the reasons why the \xss had grown to be such a successful and powerful race. 




\subsubsection{Dwelling in \Miith and Beyond}
Some \xss dwelt on \Miith itself, such as \hr{Ubloth}{\Ubloth}, who dwelt near \Yormis.
The Shroud made them even more sluggish and slothful. 





\subsubsection{Death and slumber}
\target{XS slumber}
\target{Dead XS}
%The \xss{} are sleepy and tend to slumber dormant for thousands, if not millions of years. Possibly for \hs{astrological} reasons. 
The \xss{} are sleepy. They \quo{die} and lie dead and dreaming for thousands if not millions of years at a time. 
Possibly for \hr{Astrology}{astrological} reasons. 

Have some mythical references to the \hr{Dragons worship dead gods}{\quo{dead gods} whom the \dragons worship}.

Compare to Cthulhu from \authorbook{H.P. Lovecraft}{The Call of Cthulhu}, who sleeps until \quo{the stars are right}. 

They were already dead when \hr{Sethicus contacts XS}{\Sethicus contacted them}, but they were dreaming, which allowed her to communicate with them.





\subsubsection{Worshipped}
The \xss{} are worshipped by the \dragons{} and their allies. 

\lyricsbs{Hate Eternal}{Praise of the Almighty}{
  Praise this massive force of hate. \\
  Praise the strength of thee. \\
  Worship all they have become. \\
  I am one with thee. 
  
  I await the powerful entities \\
  to enact my destiny. \\
  I anticipate to obliviate. \\
  I must summon thee.
  
  Praise the almighty ones. \\
  Praise the strength of thee. \\
  Worship all they have become. \\
  Become one with thee. \\
  Praise the old darkest ones, \\
  thy mighty force of thee. \\
  Worship all they have become. \\
  Succumb all to thee.
  
  In awe I gaze at absolution. \\
  Staring through the haze of the blackened skies. \\
  In awe I gaze at pure perfection. \\
  Wandering through the haze of the blackened skies. 
}









\subsection{\xzaishannic{} \pdaemons}
There exist several species of \pdaemons{} that served the \xzaishanns, and can now be forced to serve a Chaos sorcerer. 

One such species is mostly humanoid, but with a wide mouth filled with a row of dagger-like teeth, and great bat-like ears. Perhaps bat-wings, too. Like the monster in Clive Barker's \emph{Hell's Event} (\emph{Books of Blood II}).























\chapter{Other Intelligent Races}















\section{\Aryoth}
\target{Aryoth}
\target{Aryothim}
\index{\aryoth}
An elder race of humanoid giants. 
Related to the \nephilim, but immortal and far more powerful. 


Compare to the Thelomen Toblakai from \cite{StevenEriksonIanCameronEsslemont:MalazanBookoftheFallen}. 









\subsection{Biology}





\subsubsection{Diet}
\target{Aryoth diet}
The \aryothim did not eat souls per se, but they did require large amounts of freshly killed animal flesh to \hr{Immortal voracity}{sustain their immortal lives}. 
Some religious rituals performed during the killing (akin to halal slaughter) helped and made the flesh more nutritious.

The \aryothim \hr{Aryoth religion}{built a lot of religion on this}. 









\subsection{Culture}





\target{Aryoth inventors}
\index{technology!\aryoth}
The \quiljaaran{} often looked down on the \aryoth{} and considered them uncivilized barbarians due to their warlike tendencies. 
But they were actually great inventors. 
Where \hr{QJ philosophers}{the \quiljaaran{} cared more about philosophy and science} for its own sake, the \aryothim{} were more interested in practically applied technology. 

Also, the \aryothim \hr{Aryothim hate sorcery}{hated sorcery}, so they spent much time researching the natural sciences instead. 





\subsubsection{Magic}
The {\aryothim} had \hr{Aryoth magic}{their own style of magic}. 





\subsubsection{Religion}
\target{Aryoth religion}
The \aryothim developed many religions. 
All the major ones had \hr{Aryothim hate sorcery}{the taboo against sorcery} as a central tenet. 
(Not that it was not broken from time to time.)

\Aryoth religions were based on a number of things. 

They \hr{Aryoth diet}{had to eat a lot}. 
This became a religious thing. 






\subsubsection{Seamanship}
The \aryothim{} \hr{Aryoth seamanship}{were often great sailors}. 





\subsubsection{Technology}
The \aryothim \hr{Aryoth technology}{developed quite a lot of technology}. 

\target{Aryoth weapons}
They were fond of guns. 
They invented many kinds of guns and would often use them. 

\Aryoth{} weapons were physical, enhanced with magic, unlike \hr{QJ weapons}{\quiljaaran{} weapons}, which were based on magic from the start. 









\subsection{History}





\subsubsection{Origin}
See the section on \hr{Origin of Aryothim}{the origin of the \aryothim}. 



\subsubsection{Relation to \resphain}
\target{Resphan-Aryoth relationship}
The \resphain were descended from \aryothim to some extent. 
\hr{Semiza has Aryoth blood}{\Semiza had \aryoth blood}.









\subsection{Physique}
They looked like \nephilim, but bigger and stronger. 

A \aryoth{} was bigger and physically stronger than a \quiljaar. 
But the \quiljaaran{} typically had more powerful magic, which evened the fight. 

\index{beard!\aryoth}
A \aryoth{} face was \human-like, but bestial-looking to \human{} eyes. 
They had tusks like boars, and lots of hair and beard (for men). 
Their faces sometimes gave associations of a male lion with its mane. 

\Aryoth{} women had \hr{Nephil breasts}{as many breasts as \nephil women}. 





\subsubsection{Size}
\target{Aryoth size}
An \aryoth male was up to 300 cm tall and very heavily built (\hr{Resphan vs Aryoth size}{compared to a \resphan} or a \human). 











\subsection{Politics}
The \aryothim{} were often allied with the \vorcanths. 
They often fought against the \quiljaaran. 









\subsection{Psychology}
They were tough and belligerent warriors. 















\section{\Cuezcans}
\target{Cuezcan}
\target{Cuezcans}
\index{\Cuezca}
\index{\Cuezca!\Cuezcan{} race}
The \cuezcans{} saw their civilization destroyed by the war between the \dragons{} and \banes{}, and they hold a grudge. 
They work to play the Cabal and Sentinels against each other, killing and destroying as much of each other as possible. 
In the end, the surviving \cuezcans{} want to revive their empire and destroy the Shroud. 

The \cuezcans{} were a race of feather-clad \saurians{}, closely related to \nycans. 
At the time of the \thirdbanewar, there were still close ties between the surviving \cuezcans{} and (certain) \nycans. 

\Cuezcans{} and \nycans{} were always tied together, but they fell out and grew apart near the end of the time of \Cuezca, before the Apocalypse. 
Most \nycans{} \hr{Nycans forgot Cuezcans}{since forgot the \cuezcans}. 















\section{\Gnomphil}
\target{Gnomphil}
\index{\gnomphil}
The \gnomphilim were a race of simian \humanoids related to \nephilim.
They were of about the same size as \nephilim, but farther from \human form.
They had long fur and long, baboon-like faces (but no tail).
They had tribes and Stone Age technology.

They were well-adapted to cold climes.
They lived mostly in \UltimaThule and similar arctic regions, where the weather was too cold for the less hardy \humans and \scathae.

They worshipped \xss godlings. 

Compare them to the Gnophkeh from the Cthulhu Mythos. 















\section{\Human}
\target{Human}
\Miithian{} \humans{} are like Earth \humans{}. 
They are widespread especially in \Velcad{} and in the Far North. 

\Humans{} were originally created as a slave race by \Semiza-tachi. 
The \hr{Humans fail}{experiments failed}, and the test subjects were supposed to be killed off. 
But after \Thanatzil{} was slain the \humans{} escaped into the wild and bred true. 
The \resphain{} would later rediscover them and adopt them as their servant race. 









\subsection{Name}
As in English. The associated adjective is \emph{\human{}}. 





\subsection{Physique and metaphysique}
\subsubsection{Ethnic groups}
There are a number of races/ethnic groups of \humans{} on \Miith{}, who correspond to various ethnic groups on Earth. These include: 

\index{\Velcadians{} (race of \humans{})}
\index{Fraens{} (race of \humans{})}
\index{Kohons{} (race of \humans{})}
\index{Hoyds (race of \humans{})}
\begin{description}
  \item[{\Velcadians}] 
    have pale skin and look like the Germanic people of northern Europe (Germany, England, Scandinavia). They are the most common \human{} race, dominating most of \Velcad{} and the Northern Kingdoms. 
  \item[{Fraens}] 
    have somewhat dark skin and resemble people from southern Europe (Greece, Italy). They are common in the Imetrium, Durcac and southern \Velcad{}. 
  \item[{Kohons}] 
    have dark brown or black skin and negroid features (corresponding to Africans/Negroes). They are common in Durcac and the Far South. 
  \item[{Hoyds}] 
    have brown skin and look like Arabs. They are common in southeastern \Velcad{} (nations such as Geica) and the Orient. 
\end{description}





\subsubsection{Size}
An average \Miithian{} \human{} man is 170-175 cm tall and weighs 70 kg. A woman is 165-170 cm tall and weighs 55 kg. 





\subsubsection{\Humans suck}
\target{Humans suck}
\target{Humans are a failed slave race}
Being a failed slave race, \humans{} are sucky and measly creatures, and few of them are worth anything. 

At the time of the \hs{Murder of the Dawn}, \humans{} really sucked. 
They were smaller, weaker and more stupid than \nephilim. 
In lots of places they were the \ps{\nephilim}{} subordinates or slaves. 

The \emph{only} thing \humans{} had going for them (compared to \nephilim) was fertility. 
\Humans{} have $50\%$ women, so they breed much faster and can spread more. 

After the \resphain{} returned to assume control of the \humans, they have worked to improve them, by breeding and occult scientific experiments. 
Their efforts have met with some success, and the \humans{} of today are far better than the ones from \ps{\Merkyrah} time. 
They are still somewhat pathetic, though. 
Physically weaker than \scathae, and less well-organized. 
But more aggressive, and perhaps slightly more fertile. 

\lyricsbs{Monolith Deathcult}{Origin}{
  Risen from the seed of Enki\\
  with consent of the God-father Anu.\\
  We are the working race,\\
  created in the temples of SCH.RUPPAK.
  
  And God-slaves we are,\\
  suffering in the mines of our Masters.\\
  We populate the fields of E.DIN,\\
  the base of the Gods from Nibiru-Pha\"eton.
}

They have no culture of their own. 

\lyricsbs{Monolith Deathcult}{Deus Ex Machina}{
  Your history is not yours. \\
  I gave thee wisdom. \\
  I gave thee science \\
  and I delivered thee from bestiality.\\
}





\subsubsection{Purpose}
Unbeknownst to all, \hr{Purpose of Humanity}{\humanity{} has a dire purpose}. 









\subsection{Biology}
The average lifespan of a \human{} is 70 years, 90 in exceptional cases. 

\target{Humans are fertile}
\Humans{} are very fertile. 
More so than \scathae{} and \nephilim. 
This has allowed them to multiply a lot. 
It was probably what ensured their survival after \hr{Thanatzil dies}{\ps{\Thanatzil} fall}. 

\Humans{} can interbreed with \nephilim. 
The \human{} genes are dominant, so any children will tend to be predominantly \human. 





\subsubsection{\Demihumans}
\target{Demihuman}
There existed a number of \quo{\demihuman} races. 

\Demihumans were far more widespread before the \VaimonCaliphate. 
Back then they were not \quo{\demihumans}, but simply different variants of \humans.
There was no one race that was considered \quo{real} \humans, of which the others were deviations. 

The racist Vaimons exterminated many of the other \human races and subjugated or drove away the rest. 
Later, the race which the Vaimons represented came to be thought of as \quo{real} \humans and all the others as \demihumans. 

On other Realms than \Azmith, other \human variants dominated.

Have plenty of \demihuman slaves in Pelidor and stuff. 
Perhaps nobles (such as the Rungeran court and \ishrah) keep hot, exotic \demihuman girls as sex slaves. 

Races included:

\begin{gloss}
    
  \gitem{Standard \humans}
  \target{Men of Light}
    Standard \humans were called \quo{\truehumans}. 
    They were the \quo{Men of Light}, descended from Cordos Vaimon and Silqua. 
    More than any other \humans, they carried the genes inside them that connected them to the \hr{Lithrim}{\Lithrim} \matrix.
    
    The Men of Light \hr{Men of Light created}{originated on \Azmith}.
    Over the centuries, the Cabal slowly and covertly disseminated Men of Light to all other Realms where \humans lived, so that they would interbreed with the locals. 
    
    In time, the Men of Light interbred with other \demihuman races. 
    The \Lithrim gene was very dominant and quickly spread. 
    So in the days of the \thirdbanewar, almost all \humans carried \Lithrim inside them, even though they did not all look like Men of Light. 
    Only a few isolated \human bloodlines were \hr{Clean Humans}{clean}. 
    
  \gitem[\sheomir]{\sheomir}
  \target{Sheomir}
    \Sheomir had a fox-like furry tail and a stripe of fur running down the spine. 
    They were very rare in \Velcad, where they were often kept as exotic slaves (possibly sex slaves). 
    They were more common in the Imetrium.
    They are inspired by the Sheovins from \cite{NykiBlatchley:KaydanaandtheStaffofIshlun}. 

  \gitem{\tulan}
  \target{Tulan}
    \Tulans had skin with a reddish complexion, slightly sharp teeth, a pointed nose and face, and (for men especially) facial hair that looked a bit like whiskers. 
    Racists said they looked like rats. 
    They were the most common minority in Pelidor, where they were a subjugated lower class. 
    
    \hs{Rian} was a \tulan (as was Neina and her family).
 
  \gitem{Others}
    Some had tails or wings or horns. 
    
    Some had exotic skin \colours. 
    Some were stripy like a zebra. 
    \hs{Evith} was one of these. 

    Some \demihumans had tails like \nephilim.
    
    Some had feathers like \resphain.
    
    A few rare breeds even had wings like \resphain. 
    They could perform powered leaps with the wings, but they could not fly. 
    And the wings were fragile and did not regenerate. 
    Still, the winged \humans were considered by the \resphain to be the highest of all \humans because they resembled resphain. 
    
    Evith was a winged \demihuman. 
    And striped like a zebra. 
\end{gloss}




There were also \quo{\hr{Demiscatha}{\demiscathae}}.
The \scathae were more tolerant of their kin than \humans were. 





\subsubsection{\Resphan experiments on \humans}
The \resphain performed \hr{Resphan experiments on Humans}{experiments on \humans}. 









\subsection{Politics}





\subsubsection{\Scathae}
\Scathae and \humans \hr{Scathae and Humans hate each other at first}{hated each other when they first met}.
After some centuries they slowly learned to accept one another. 









\subsection{Psychology}
Mankind is very Shrouded, \naive{} and stupid. 
Not quite as well-meaning and {innocent} as the \scathae, but not all that far from them.
That is why it is so effective when \hr{Lithrim arises}{their inner \Erebean{} darkness is finally released in the culmination of the \Morbus{} plan}. 















\section{\Jinni}
\target{Jinn}
\index{\jinni}
The \jinn were a race of immortals.
They predated \humans and lived, among other things, in the lands southeast of Durcac. 
\Secherdamon's cult encountered the \jinn and the humanoids who served and worshipped them, and merged with them.
He waged wars with the \jinn, but eventually struck alliances with many of them, and eventually came to rule over them.
The lords of the \jinn and the greatest among them were the ifrits.
Still, some \jinn opposed Nechsain and would not serve him. 
These \quo{evil} \jinn were horrors that haunted the desert, feared by all.

The \jinn were disembodied horrors of the void.
They haunted the empty deserts, howling. 

Make clear that the \jinn are horrible alien beings, like some Cthulhu monsters.

















\section[Meccaran]{\Meccaran}
\target{Meccaran}
\target{Meccara}
\index{\meccaran{} (plural \meccara{} or \meccarans)}
\Meccara{} are amphibian humanoids. They are widespread especially in the warmer climates of the South. 









\subsection{Name}
Singular \emph{\meccaran{}}, plural \emph{\meccara{}} or \emph{\meccarans{}}. %\emph{\Meccara} is generally used when referring to the race as a whole, while \emph{\meccarans} refer to a specific group of individuals. This grammar is Imetric. 
The word is Imetric. \emph{\Meccara{}} is Imetric declination, \emph{\meccarans{}} is English declination. I will use both forms synonymously. 
The associated adjective is \emph{\meccaran{}}. 









\subsection{Physique}
\Meccara{} look like humanoid frogs. They have large, strong hind legs and are fast and agile runners, leapers and swimmers. Their forearms are dextrous, but not very strong. In combat, \meccarans{} rely more on speed than on brute force. \Meccaran{} skin is smooth and no tougher than human skin. 

A \ps{\meccaran} mouth is filled with sharp teeth that curve backwards. Adapted (among other things) to catch fish, \meccaran{} teeth are effective for biting and holding. \Meccaran{} necks are short, however, so biting is difficult in combat. 

A average, full-grown \meccaran{} female weighs 60 kg and stands about 130 cm tall in her natual posture. The male is smaller than the female, about 50 kg and 120 cm in height. \Meccarans{} do not stand fully erect but in a \quo{crouched} position with the legs bent outwards. They can increase their height by up to one third by stretching out, but this is an unnatural position and can not be maintained for long. \Meccaran{} arms are short, only around 50 cm. 

\Meccaran{} vision is inferior to that of \humans. They are near-sighted and cannot see well at great distances (this is a result of adaptation to a life in dense forests). At close range they see as well as \humans. Their sense of smell is acute\dash not as good as that of a dog, but far better than a human's. Their other senses are like those of \humans. 

\Meccara{} have the ability to regenerate lost limbs. The time this takes depends on size: A lost hand or foot can be regrown in a month, an arm in four months and a leg in six months. The new limb is generally as good as the old one, but sometimes the regeneration is faulty, making the new limb smaller, weaker and/or less agile than the old one.\footnote{Make some kind of saving throw to avoid permanent attribute loss (and possible limping, if a leg grows back too short). If the same limb has been lost and regrown more than once, the chance of faulty regeneration increases each time (so it is generally not worth the risk to cut off the new limb and hope for a better one next time). A faulty limb cannot \quo{cured} by regular magical healing, since it not an injury but the natural state of the new limb. \quo{Fixing} a faulty limb is just as hard as it would be to strengthen the original limb.} At TL7 and above, limb transplants between individuals is generally easy. Non-fatal injuries to internal organs (even the brain) can also be healed. \Meccaran{} regeneration is slow, however, and not useful in combat. 

As a special feature, \meccara{} are left-handed by default. A minority ($20\%$) are right-handed. 









\subsection{Biology}
\Meccara{}, naturally evolved from large, freshwater-dwelling, predatory frogs, are carnivorous and semi-aquatic. They are able swimmers but cannot breathe water. They prefer subtropical to tropical climates and high humidity. Their natural habitats are jungles and swamps. They naturally live as hunters. Some of the more advanced \meccaran{} communities also raise livestock for food and tools, with farms to support the animals. (This is a science they have learned from other races. The naturally carnivorous \meccara{} would be unlikely to discover farming on their own.) 

%\Meccara{} prefer to bathe in fresh water regularly (once per day at least, more in hotter or drier areas). Drying out is painful, but not fatal. 

\Meccara{} must bathe in fresh water regularly. If a \meccaran{} cannot immerse (or at least splash) himself with fresh water every day, he will weaken and die. 

\Meccara{} have two genders and reproduce by external fertilization: The female lays her eggs in water and the male fertilizes them. A female can lay several dozen eggs at a time, most of which die. The eggs are spherical, 2-3 cm in diameter. After six months, the surviving eggs hatch into tadpoles. A hatchling tadpole is only few cm in length and has very little intelligence. Gradually they tadpoles develop legs and the ability to breathe air. After around five years, they are about 50 cm long, have legs as well as a tail, and are as intelligent as a \human{} child of two years or so. At this point they mostly land-living. After one or two more years, they lose the tail and the ability to breathe water. At the age of around 10, \meccara{} are sexually mature, about 80-90 cm tall and as intelligent as a \human{} teenager. They are full-grown adults around the age of 15. \Meccara{} live to be up to 50-60 years old, 70 in exceptional cases. The females outnumber the males, making up 60 percent of the population. 

%The \meccaran{} species is closely related to the Fitteran species. It is possible (albeit rare) for the two species to crossbreed. \Meccaran{}/Fitteran hybrids are larger than \meccara{} but more intelligent than Fittera. They are sterile, like mules, and usually considered ugly freaks by both species. There are, however, communities with \meccarans{} and Fitterans living together. In these communities, crossbreeds tend to be accepted as normal members of society. 

\Meccara{} are generally not monogamous and do not mate for life. Culturally, they have few sexual taboos and tend toward promiscuity\footnote{Promiscuity in the biological sense, meaning that any two individuals in a tribe may mate.}. Indeed, some tribes are known to indulge in collective sexual orgies, often under the influence of certain drugs brewed by their witch-doctors. Bi- and homosexuality is rather common and accepted in most cultures. 









\subsection{Psychology}
\Meccara{} tend to crave independence and freedom. As such, \meccaran{} tribes are rather loosely organized and laws are few and simple. \Meccara{} are curious and inquisitive by nature and have little fear of change. Some tribes have evolved into settled farmers and built towns or even cities, but their natural lifestyle is nomadic. 

\Meccaran{} adventureres are rather common. \Meccara{} tend to be very active and energetic, with laziness being seen as abnormal. 

All \meccara{} suffer from a mild dyslexia. The part of their brain that allows them to understand writing is not as well-developed as that of most races. They can learn to read and write, but they do it less well than other creatures of equal intelligence. Of course, dyslexia is not understood at TL3, so this is usually interpreted as stupidity, which leads to \meccarans{} sometimes being looked down on as barbarians. 









\subsection{Habitat}
\Meccara{} are most widespread in the tropical lands of Uzur and in the Far South. They are relatively common in the Imetrium and Durcac and uncommon in \Velcad{}. They are rare in the North and East. Most \meccara{}-dominated communities lie in Uzur, but there are \meccaran{} tribes scattered across \Velcad{} and other places as well. 















\section{\Naiad}
\target{Naiad}
\index{\naiad}
\Naiads{} are water-dwelling \quo{spirits}. 
They are actually a kind of intelligent jellyfish with innate \hr{Telepathy}{telepathic} and telekinetic powers. 

They are traditionally referrec to as female, although this is biologically incorrect. 









\subsection{Name}
Singular \naiad, plural \emph{\naiads{}}. The word is originally Greek but the declination is English. The associated adjective is \emph{\naiad{}}, I guess. (Naiadese? Naiadean? Naiadic?) 









\subsection{Physique}
A \ps{\naiad}{} natural element is water. In the water, she appears as a barely visible patch of vaguely shimmering water. Closer inspection will reveal a network of long, spindly tendrils. These tendrils are the \ps{\naiad}{} actual body, a brain of some sort. In the water, a \naiad{} will typically spread out over an area 2-3 meters across. She is not massive and other creatures can pass through her body unharmed, perhaps without even noticing her. 

A \naiad{} must be surrounded by water to survive. On dry land, she must bring along water of her own using her telekinesis. To preserve her strength, she usually brings along only a small body of water. When travelling over land, the easiest and most confortable form to assume is that of an amorphous amoeba that slithers along. With practice and effort, a \naiad{} is able to shape her watery body into a humanoid form (or sometimes a humanoid torso with an amoeboid lower body, which is easier to control than legs). \Naiads{} are not very strong, so they will rarely drag along much more than 10-20 liters of water. So if she assumes humanoid form, it will be that of a very small humanoid. 

Cold will numb and slow a \naiad{}. If frozen, she will go into torpor but not die. If she is frozen and then shattered, she will die. Fire damage will quickly kill a \naiad{} if it can boil or evaporate the water surrounding her. Electricity does no damage. Acid does damage if it directly hits her tendrils, but this is rarely fatal unless there is a \emph{lot} of acid. Death magic has its normal effect. 









\subsection{Biology}
\Naiads{} are actually asexual and reproduce by budding. 









% \subsection{Psychology}








% \subsection{Habitat}















\section{\Nephil}
\target{Nephil}
\index{\nephil}
The \nephilim were a race of primate humanoids. 









\subsection{Physique}
\target{Nephil breasts}
\Nephilim had tails like monkeys.

\Nephilic{} women had four breasts. 
The women were fewer than the men, so they had to give birth to more young at a time to compensate. 









\subsection{Biology}




\subsubsection{\Nephil/\human{} hybrids}
\Humans, being descended from \nephilim, can interbreed with them. 
These hybrids are called half-\humans{} or half-\nephilim{} depending on whom you ask. 
They resemble their mother more than their father. 
If the mother is a \nephil, then her half-breed children look like skinny, balding \nephilim{}, while those born of a \human{} mother look like big, burly, hairy \humans. 

Either way, the hybrids are sterile. 





\subsubsection{Sexuality}
There are fewer \nephilic{} females than males. 
They make up no more than $30\%$. 

\target{Nephil misogyny}
This means that wives and daughters are very valuable property. 

In some cultures, kings and lords could have several wives, but common men had to count themselves lucky if they could get just one wife (or just sex, for that matter). 

Gay sex is common. 
In some places it is traditional for a conquering man to ass-fuck his conquered enemies, to establish his dominance. 
It shows that they are no longer real men, but slaves, property. 
Like women. 









\subsection{History}
\subsubsection{\Ophidian{} servitude}
\target{Nephilim worship Ophidians}
Originally, before the time of \Kserasshana, the \nephilim{} served  \ophidians{} and worshipped them as gods. 

In fact, the \ophidians{} \hr{Ophidians breed}{had helped create them}. 





\subsubsection{Chariots}
Back in the day, the \nephilim{} used chariots in war, because the \nephilim{} were large, and they knew of no mounts that were both fast and big enough to carry them. So they rode chariots, drawn by horses or other things.





\subsubsection{\Aryothim{} appear}
At some point, mighty \nephilim{} turned themselves into \hr{Aryoth}{\aryothim}. 





\subsubsection{Enslaved}
In some places, humans keep \nephilim{} as slaves. 









\subsection{Ogre-Magi}
A certain group of \nephilic{} sorcerers are known to others as \quo{Ogre-Magi}. There are very few of them, but they are pretty powerful.















\section[Nycan]{\Nycan}
\target{Nycan}
\target{Nycans}
\index{\nycan}
\Nycans{} are large predatory reptiles resembling real-world dinosaurs like Deinonychus or Velociraptor. 

This creature is based partially on the actual dinosaurs, partially on the Jurassic Park movies and partially pure fantasy. 

\Nycans{} are fierce predators that hunt in packs. 
They are highly intelligent and have \hr{Telepathy}{telepathic} abilities. 
They are found mostly in the Imetrium and Irokas. 
In the Imetrium, \nycans{} are domesticated and used as beasts of war. 









\subsection{Name}
Singular \emph{\nycan{}}, plural \emph{\nycans{}}. 
The word is Imetric but the declination is English. 
The associated adjective is \emph{\nycan{}}. 
%(The word is derived from \quo{nychus}, as in \quo{Deinonychus}. This is originally a Greek word for \quo{claw}.)









\subsection{Physique}
A \nycan{} is a slender, bipedal dinosaur with a long, stiff tail. Its forearms are long and strong and have sharp claws. Each hind leg has not one but two oversized claws. \Nycans{} are covered in feathers and \coloured in shades of brown, red and yellow.\footnote{There is evidence to show that some RL dinosaurs of this type were in fact feathered.} 

In combat, \nycans{} slash with the great claws on their hind legs if possible, and also attack with their foreclaws and bite. 

\Nycans{} are extremely effective predators. They are fast and agile runners and leapers and can sprint like cheetahs, reaching tremendous speeds over short distances. They are also highly intelligent, hunting in well-organized packs and displaying great cunning in their hunting behaviour. 

\Nycans{} have vision like that of \humans{} and hearing and smell like that of dogs. They are diurnal and do not see well in darkness. 

In the wild, \nycans{} are around 3 meters in length and weigh around 70 kg (like a \latinname{Deinonychus}). 
In captivity, beasts up to 7 meters long and weighing over a ton have been bred (like a \latinname{Utahraptor}). 
The female is slightly larger than the male. 

\target{Nycan endurance}
\Nycans{} are strong enough to carry riders. 
They can also sprint as fast as \hr{Relc}{\relcs} over short distances. 
But they are not suitable as mounts for longer rides. 
They lack the endurance of \relcs. 
They are built for explosive bursts of speed, not prolonged running. 






\subsubsection{Nycans are frightening}
\target{Nycans are frightening}
Describe how the \nycans{} are terrible and frightening, with their cold, reptillian eyes that know too much and hide an unnatural, inhuman intelligence. 

This is a toned-down version of \hr{Draconic appearance}{\draconic{} appearance}. 

Compare them to Sag'Churok and Gunth Mach, the two K'Chain Che'Malle that follow Redmask in \MalazanReapersGale.






\subsubsection{Magic}
\target{Nycan magic}
A few exceptionally advanced tribes/packs of \nycans{} have developed primitive magic and psionics: 
Telepathic attacks, telekinesis and maybe healing. 
Most other \nycans{} (especially tame ones, who are affected by the Shroud of Civilization) do not understand it and fear it. 









\subsection{Biology}
\Nycans{} are warm-blooded dinosaurs and a result of natural evolution. Only one species of \nycan{} is known to exist - they are so effective that they have driven all closely related species into extinction. 

A \nycan{} female lays a small cluster of one to five eggs. Equal numbers of males and females are born. \Nycans{} live in packs, each pack led by an alpha female. Males and females both participate in hunting and raising the young. They are not monogamous. 

A \nycan{} is sexually mature after 7-9 years. Their average lifespan is about 20 years in the wild and 30-35 years tame, although they can reach 50-60 years. 
%They can live up to 50 years tame, but rarely more than half that in the wild. 





\subsubsection{Races and breeds}
The Imetrians breed \nycans{} into a number of subraces and breeds. 


\index{Secca (plural Seccae)}
\index{Crycos (plural Crycoi)}
\index{Destran}
The Seccae (singular Secca) are the largest and the only \nycans{} strong enough to easily carry a rider. 
The Crycoi (singular Crycos) are the fastest, used as couriers (carrying letters) and as shock troopers. 
The Destrans (singular Destran) are bred for intelligence and sharp senses. 

The races are further divided into a number of \quo{breeds}, often named for the city, family or individual that breeds them or \quo{founded} the breed. 

\target{Mictzan}
\index{Mictzan}
Examples include the Mictzan breed of Destrans, named for the founder (an Imetric \scatha{} of Clictuan descent), and the Dorlinum breeds, of various races, bred in the Imetric city of Dorlinum. Especially renowned are the Dorlinum Secca, who are some of the largest and strongest \nycans{} known. 









\subsection{Psychology}
\Nycans{} are used to living in a pack and can be tamed and taught to recognize a non-\nycan{} as pack leader. Once tamed, \nycans{} are very loyal, and there is no risk of them \quo{going wild}. 

\Nycans{} are as intelligent as humanoids and can learn a wide variety of skills. They can learn to understand language well enough to understand simple sentences and maintain a rather large vocabulary of both simple and more abstract concepts. A skilled \hr{Nycaneer}{\nycaneer} can give his \nycans{} quite complex instructions. 





\subsubsection{Communication}
Among themselves, \nycans{} communicate using \hs{telepathy}. 
This telepathy, combined with smell, allows them to detect emotions and thoughts in other creatures. 

They also commicate using sound. \Nycan{} voices are hoarse, screeching and high-pitched. They can scream (to warn fellows of danger), yelp (if in pain), hiss (to warn or intimidate foes) and purr (when friendly). 

The reason they use sound and body language, even though telepathy might be more effective, is the same reason why \humans{} use body language and not only speech: 
It's older. 
\Nycans{} and their ancestors have had sound and body language for countless millions of years, whereas telepathy is a relatively new evolutionary breakthrough, only a million years old or so. 





\subsubsection{Compared to \nephilim}
\index{technology!\nycan}
\Nycans{} are actually just as intelligent as \nephilim. 
Both races evolved naturally on \Miith{} (unlike \scathae{} and \humans, who were engineered). 
The \nephilim{} went on to develop a technological civilization. 
The \nycans{} didn't. 
(That didn't turn out so well for the \nephilim.)

The reason for this is that the \nephilim{} were physically rather feeble, and therefore needed technology to make it big. 
The \nycans, on the other hand, were badass from birth and had no need for technical gadgets. 









\subsection{Habitat}
\Nycans{} prefer warm climates and open grasslands. They are predators and prey on all sorts of animals. 

A millennium ago, \nycans{} were widespread over much of the Old Continent. Being very dangerous creatures who may attack livestock and people, the \nycans{} were hated and feared by most intelligent creatures, and as a result, they were hunted into extinction in most lands by humanoids armed with weapons and magic. 

The only place in the West where \nycans{} survived was the land that is now the Imetrium, and centuries later, the Imetrians discovered how to domesticate them. Tame \nycans{} are now an invaluable asset in Imetric society, taking the place of dogs in RL. \Nycans{} are stronger and more intelligent than dogs, however. In the military, \nycans{} are used as trackers, battlefield shocktroopers, and even assassins. 









\subsection{Arsenal}





\subsubsection{\Armour and weapons}
Imetric \nycans{} wear \armour and weapons when going into battle. 
They wear metal vambraces on the forearms which allow them to parry blows from swords and the like. 

And they wear big-ass metal blades covering the natural claws on their feet. 

Some wield enhanced metal foreclaws, or even \hr{Skekrathuin}{\skekrathuin}-like blades. 





\subsubsection{Magic}
There exist wise \nycans{} who know some magic. 
They have a hard time studying because they cannot easily learn to read and write, and as such have little in the way of culture. 
But some of them do learn some \quo{natural} magic. 





\subsubsection{Seeing into the Beyond}
The \pps{\nycans}{} telepathic abilities also let them see into the Beyond. 

\Nycans{} fear and hate the unnatural. 
This includes such things as undead and demons, as well as \quo{black} magic, but not all magic. 
As a general rule, anything that has a Horror Effect will qualify. 
When faced with \quo{unnatural} things, \nycans{} will become enraged and want to attack it, or flee, if the menace is perceived as too powerful or horrible. 
A \bane{}, for instance, will cause \nycans{} to flee, but they might stand and hiss at it from (what they perceive to be) a safe distance. 
Something with only a Slight Horror Effect, like a dark mage (such as a \Nieur{} channeller) or a Rissitic Dominus, will cause the \nycans{} to hiss threateningly, but not immediately attack. 
An undead creature (such as a Rissitic Immortal Priest) is likely to be savagely attacked. 
If the \nycans{} are tame and well-trained, a \nycaneer{} may be able to restrain them\footnote{This would require a skill roll of some kind.}, but the instinctive hatred is strong. 

%Note that what is \quo{unnatural} is not entirely instinctive. A Rissitic Dominus, for instance, has a Slight Horror Effect, but a \nycan{} raised by Rissitic trainers and accustomed to Rissitic culture and magic 









\subsection{History}
\target{Nycans forgot Cuezcans}
The \nycans{} were originally \hr{Ophidians breed}{shaped by the \ophidians}. 

The \nycans{} were originally closely tied to the \hr{Cuezcan}{\cuezcans}. 
But after the \CuezcanApocalypse, most \nycans{} grew \quo{\Wylde} and \quo{savage}.
They forgot the true history of their people(s). 









\subsection[Nycaneers]{\Nycaneers}
\target{Nycaneer}
\target{Nycaneers}
\index{\nycan!\nycaneer}
A \nycaneer{} (plural \emph{\nycaneers{}}) is a \scathaese{} \hr{Telepathy}{telepath} who communicates with \nycans{}. 

\Nycaneer ing is actually not really a special \quo{affinity} with \nycans, as is often believed. 
Here is the deal: 
\Scathae{} are \hr{Origin of Scathae}{genetically related to \nycans}. \Nycans{} are naturally telepathic. 
\Scathae{} also possess some innate telepathic ability, but it is latent in most, and the Shroud further oppresses it. 
A few \scathae{} show special talent for telepathy. 
Since their natural telepathic \quo{voice} and communication style is \nycan-like, they find themselves inherently able to communicate with \nycans{} to some extent. 
Telepaths who are in contact with \nycans{} from an early age will, by natural means, learn to understand a telepathic \nycan{} language and thus converse with them. 
These develop into \nycaneers. 

\Scathaese{} telepaths who do not encounter any \nycans{} in their youth, or who only learned telepathy as adults, will usually never learn \nycan{} tongues. 
If they later meet a \nycan, they will be unable to communicate with it, and as adults they can no longer easily pick up new languages. 
Such people will thus have no specific affinity with \nycans{} and therefore will not be considered \nycaneers. 

It should be noted that \nycans{} do not all speak the same telepathic language. 
These are as different as spoken languages. 
So a \nycaneer{} who can converse with one \nycan{} will not necessarily understand another. 
However, an experienced and socially ept \nycaneer{} will understand not only language but also many other aspects of \nycan{} behaviour and mannerisms. 
Using telepathy and this knowledge, a \nycaneer{} will be able to communicate with a strange \nycan{} to some limited extent, much like how two humanoids who share no language can still communicate using body language and tone of voice. (In contrast, a non-\nycaneer{} and a \nycan{} will be able to communicate about as well as a \human{} and a wild wolf, ie., extremely crudely.) 



\Nycaneers, with their extra-sensory perceptive abilities, develop into \vertices{} more often than other people. 

Only \scathae{} and \rachyth{} have the \nycaneer ing talent, since they are \hr{Origin of Scathae}{genetically related to \nycans}. 
The talent is found in varying degrees, but even a weak \nycaneer{} can develop his skill, eventually being able communicate with \nycans{} almost as well as he communicates with other humanoids. 

The Imetrium actively searches for young \nycaneers{} and maintains a formal \Nycaneer{} Academy. 

\index{\melda{} (plural \meldae)}
The \nycaneer{} that commands a given \nycan{} is that \ps{\nycan}{} \melda{} (Imetric word, plural \emph{\meldae{}}). 















\section{\Nymph}
\target{Nymph}
\target{Nymphs}
There are three races of \quo{\nymphs}: 
Dryads, \naiads{} and \sylphs. 

In poetry they are considered the \quo{daughters of Mother \Miith}. 
They are portrayed as ancient and benevolent, mourning and weeping whenever \Miith{} is at war and laid waste. 

Compare to the \quo{Daughters of Beulah} in \authorbook{William Blake}{The Four Zoas}. 

\WanderersInDarknessEmph is full of references to the sorrowing \nymphs. 
















\section{Spider People}
Have a race of monstrous intelligent spider-like creatures that spin webs. 
Maybe merge them with the \hr{Weaver}{Weavers}. 

Compare to some of the wicked descriptions in \cite{Cracked:GeneticExperiments} (the spider/goat part). 
















\section{\Succubus}
\target{Succubus}
\target{Succubi}
\index{\succubus}
Maybe have a race of alien monsters that tempt people with someething they want, such as sex, and lure them out of the Shroud and into the Beyond. 
There the monsters feast on the victims's flesh and (maybe) souls. 

The monster shows people illusions, using people's dreams/fantasies/desires (sexual or otherwise) to lure them out, then manipulates their mind to draw them out through the Shroud and into the Beyond, where no one hears them scream.

Sometimes the \succubus{} will take the shape of a sex partner, seduce its victim, and then use sex to distort the victim's perception of reality and suck them into the Beyond. Then the \succubus{} transforms into a man-eating monster\dash suddenly, or gradually during sex. 

\target{Succubus sucking dick}
Maybe even have a vore-like sex scene where she devours her lover while they fuck\dash gradually sucking his life, blood and body out through his dick, leaving an empty husk. This is a super-powered version of Shereid's sperm-eating spell (see section \ref{Shereid's sperm-eating spell}).

%Compare to \authorbook{Clive Barker}{The Son of Celluloid} (\emph{Books of Blood III}). 

\lyricsbs{Clive Barker}{
  The Son of Celluloid (Books of Blood III)
}{
  You make me strong, looking at me that way. I need to be looked at, or I die. It's the natural state of illusions.
}















\section{\Vorcanth}
\target{Moon-Wolves}
\target{Vorcanth}
\index{\Vorcanth}
The \MoonWolves{}, the \quo{mystic wolves of the Frost-Moon}, are an ancient race of powerful, somewhat wolf-like creatures. They possess \human-level intelligence, but different, so they cannot easily communicate with humanoids. They are \Wylde{} creatures.

They are inspired by the song \bandsong{Bal-Sagoth}{Starfire Burning upon the Ice-Veiled Throne of Ultima Thule}, and by the Deragoth (Hounds of Darkness) in \cite{StevenEriksonIanCameronEsslemont:MalazanBookoftheFallen}. 

They are an ancient race, older than the \nephilim, and used to live side by side with the \ophidians. They used to be among the masters and rulers of the Beast Realm, but as the \dragons{} and \resphain{} have gained territory, the \moonwolves{} have declined, and there are now few of them left.

Perhaps they look down upon tame animals who denigrate themselves to serving lowly humanoids, as the slaves of slaves.







\subsection{Appearance}
A \vorcanth{} is a quadruped, six metres long (or that order of magnitude). 
They are covered in snowy white fur and look vaguely like wolves, and are typically likened to them, but they are not wolves and not closely related to them. 
A somewhat closer match in appearance are hyaenas. 
But even that is pretty far from the mark.

They are not entirely mammalian. 
They also have reptillian characteristics. 
They have snaky, reptillian tails. 

Underneath the fur they are covered in \armoured scales or plates. 
They have large \armour plates on the shoulders that kind of resemble the frills of \mulgrons. 

Compare them in appearance to the wolf-things in the movie \movie{Final Fantasy VII: Advent Children}. 

The impression they give is that of alien-looking creatures that look as if they've stepped out of the ancient pre-history. 
Compare them to Sag'Churok and Gunth Mach, the two K'Chain Che'Malle that follow Redmask in \MalazanReapersGale.





\subsubsection{Horrible}
\Vorcanths were dark, terrible, mysterious things like the Hounds of Tindalos from the Cthulhu Mythos. 
They were some of the horrors that lurked in the Realms Beyond and would sometimes prey on \Miithians. 
Even immortals were subject to their depredations. 
Even their \resphan allies never understood them very well. 









\subsection{Arsenal}





\subsubsection{Power}
Physically the \vorcanths{} are extremely powerful. 
A \vorcanth{} is easily a match for several \resphain{} in close combat, and may even challenge a young or weak \dragon{}. 

\Vorcanths{} cannot fly. 
But to compensate, they are very fast runners, insanely fast jumpers and highly stealthy stalkers (utilizing their \hr{Vorcanth travel Beyond}{power to travel Beyond}). 

Their weakness is magic. 
Some \vorcanths{} do know magic, but it is rather primitive compared to that wielded by \dragons{}, \resphain{} and \quiljaaran{}. 
The \vorcanths{} are well aware of this, and it is surprising how often they manage to offset this disadvantage using stealth, cunning and speed. 





\subsubsection{Travelling in the Beyond}
\target{Vorcanth travel Beyond}
They have the ability to \quo{fade into the mist} and travel through hidden planes, journeying to secret \quo{folds} of the Realms, which the \dragons{} and others do not understand and where they cannot follow. 

Compare to the Hounds of Tindalos from a Cthulhu story by\ldots{} Frank Belknap Long, I believe. 









\subsection{Culture}





\subsubsection{Language}
\Vorcanths{} communicate using sound and body language. 
Most elders can understand \draconic{} and \resphan{} tongues. 
A few can even speak them. 

A few mages know telepathy as well. 









\subsection{History}





\subsubsection{Enemies of \dragons}
The \vorcanths{} are ancient rivals of the \dragons{} and \ophidians. 
Once they dwelt on \Tembrae{}, but after several wars against \dragons{} and others, and being caught up in wars not having to do with them, devastated the \vorcanth{} population. 
Finally, the \secondbanewar, where the \vorcanths{} allied with the \resphain, saw most of the wiped out. 

After that they mostly retreated to Visha. 





\subsubsection{On Visha}
\target{Moon-Wolves and the Moon}
\target{Moon-Wolves and Visha}
They were called \quo{the mystic wolves of the Frost-Moon}, and their power was connected to \hs{Visha}, the Frost-Moon, the Mystic Moon. 
In fact, they dwelt \emph{on} Visha, which was a Realm in its own right. 

They had their own wars and skirmishes on Visha; against one another and against other monsters. 
They also fought agents of the Cabal and Sentinels. 





\subsubsection{Few leaders left}
In Carzain's age, few of the great \vorcanth{} leaders (whom Ramiel-tachi knew from before they became \malachim) survived. 
Many had been killed in their various wars and conflicts. 
But new leaders had arisen, and some had grown very powerful. 









\subsection{Politics}
\subsubsection{Enmity with \dragons}
\target{Moon-Wolves dislike Dragons}
The \moonwolves{} dislike \dragons, their ancient rivals who almost wiped them out. They root for the \resphain\dash although they fear the \banes. 

They see Ramiel as a saviour of sorts.





\subsubsection{Association with Ramiel}
Ramiel is a friend of the \MoonWolves{}. In the past, Ramiel helped out one of the great, venerable alpha male leaders of the \moonwolves. As thanks, a pack of them chose to remain by his side as his personal allies, following him as their alpha.

But this was before Ramiel became a \Malach{} and lost his memory (see section \ref{Ramiel becomes a Malach}). Since then, the wolves have \hr{Carzain dreams of Moon-Wolves}{tried to contact him in dreams}, but he has been entangled in the Shoud, and they have not been able to readily communicate with him. 

\target{Moon-Wolves help Ramiel in dreams}
The Shroud prevents them from just popping into the physical world. Also, if they did, the Sentinels and Cabal might hunt them down. Still, they come to his aid from time to time, in the world of dreams, at least. When he is besieged by monsters of his own imagination, at times he sees great white wolves that come and dispel or destroy the apparations. 

He senses that there is something he should know, but he doesn't understand it. But they awaken something in him, and they help him remember fragments of his past life. He realizes that the wolves are a vital clue in his quest to discover his past.

%He contacts them in dreams, and they help and guide him.
Perhaps one of his allied \moonwolves{} is sick or hurt, so the pack is questing for him, helping him and seeking his help in return. So they come to him in dreams, beckoning him to come to them. At last, he somehow manages to find the wounded wolf and help it. 

It pledges itself to him and becomes his permanent companion for the rest of the story. Ramiel now has his personal \moonwolf{} companion. At this point, he still does not understand the creatures and his link to them. He doesn't fully realize that until \hr{Ramiel's awakening in the temple}{his awakening in the temple}. 





\subsubsection{Worshippers}
There are some people who worship the \moonwolves{} as gods or demigods. Compare them to the wolf-worshipping Grey Swords and others in \cite{StevenEriksonIanCameronEsslemont:MalazanBookoftheFallen}.

\lyricsbs{Bal-Sagoth}{Naked Steel (The Warrior's Saga)}{'Neath the Moon-Wolf's gaze we shall slake our steel.}























\chapter{Monsters}















\section{\Chimaera}
\target{Chimaera}
\index{\Chimaera}
A mythical creature, related to and possibly identical to the \hr{Malgryph}{\malgryph}. 

It was \hr{Urizeth researches Chimaera}{researched by \Urizeth}. 















\section{Dark Young}
I should have a race forest monsters similar to the Dark Young of Shub-Niggurath in the RPG \emph{Call of Cthulhu}. 














\section{Flying Whales}
I should have a race of flying whale-like things. 
Compare to the windwhales in \cite{GlenCook:TheWhiteRose}. 

But they should be dark, grotesque and Cthulhu-esque. 














\section{\Malgryph}
\target{Malgryph}
\index{\malgryph}
A \malgryph was a mythological animal that looked like a giant nycan with feathered wings, a pair of backward-curving horns and the head and tail of a snake.
It was said to be wise, possessing secrets of sorcery.
It was occasionally used in heraldry. 
It was feared as a terribly destructive monster and a bringer of evil omen.
It was an ominous thing to have on your banner.

It was related to and possibly identical to the \hr{Chimaera}{\chimaera}. 

\target{Malgryph summoning}
There existed spells to conjure forth a \malgryph.
Not a \quo{real} \malgryph, of course. 
It was not a real, existing animal.
But chaos sorcerers could conjure daimonia and temporarily shape them in the form of a ghostly \malgryph that would fight for him for a time.
The \malgryph would be controlled by a \homunculus. 

It was a powerful, deadly spell of alienism.
But it was known only to \dragons and a few other sorcerers.
It was not found in regular \rethyactic textbooks.

See \emph{The Bible}, Isaiah 14:29.

\lyricsbible{Isaiah 14:29}{
  Rejoice not thou, whole Palestina, because the rod of him that smote thee is broken: for out of the serpent's root shall come forth a cockatrice, and his fruit [shall be] a fiery flying serpent.
}














\section{\NerasKirishgaith}
\target{Neras Kirish'gaith}
\target{Bladed People}
The \NerasKirishgaith{} are a race of mostrous semi-humanoid creatures with blades all over their bodies. Compare to the eponymous creatures from the anime \emph{Gilgamesh}. 

They might be \banes, but they might also be native \Miithians. 

Perhaps they have great, insect-like wings. Perhaps only some breeds have wings. 

They have hive-like societies with queens, warriors and drones. They mind-control humanoids and use the bodies of these to interact with humanoids. 

They are the ancient rivals of the \ophidians, descended from a group of terrible \quo{Progenitors}, who were some of the ones that created all life on \Miith{}, millions and millions of years ago. Compare to the Great Old Ones of H.P. Lovecraft's Cthulhu Mythos.














\section{\Salamander}
\target{Salamander}
The great \salamanders were artificial creatures, manifestations of \hr{Satha}{\RuinSatha} (a \xs). 
They had semi-solid bodies made of pure alien fire.
Using some of the mightiest spells of \hr{Ruin Satha fire magic}{\draconian fire magic}, a mage could summon (or create) a \salamander and send it to fight for him. 

A \salamander took a form vaguely like a \dragon. 
See, they were formed by spells, and those spells were devised by \dragons. 
The \dragons envisioned their fiery warriors in their own image, so the \salamanders assumed a \dragon-like form. 






















 

\chapter{Animals}















\section{Birds}














\subsection{\Grulcan}
\target{Grulcan}
\target{Diatryma}
The \grulcans{} are a species of great flightless, predatory birds. 
They are inspired by real-world animals like \latinname{Brontornis} and \latinname{Gastornis} (\latinname{Diatryma}), who lived in the Eocene. 
Maybe these can be domesticated.

The name is inspired by the bird-monster Groth-Golka that appears in \authorbook{Robert E. Howard}{The Gods of Bal-Sagoth}. 

\Grulcans{} are not as fast as \hr{Relc}{\relcs}, but they are much more \manoeuvrable (\relcs{} turn slowly because of the tails). 















\section{Invertebrates}









\subsection{Skekkok}
A large scorpion-like creature that lives in the deserts of Durcac. 

The name is an onomatopoeia for the clicking sounds the creatures make.

Compare to the scantids from \cite{VideoGame:Starcraft}. 















\section{Mammals}
\target{Saurian-dominated}
There are few large mammals on \Miith. 
\Miith{} is \saurian-dominated. 

The few large land mammals include hippopotami, hyaenas and white tigers (rare). 
Other mammals include small cats and the weasel family (carzains and wolverines). 









\begin{comment}
  \subsection{Buopoth}
  \target{buopoths}
  The buopoths are shy, forest-dwelling animals. 
  They are taken directly from some of \HPLovecraft's stories as a tribute/reference/\trope{ShoutOut}{shout-out}. 
  
  They look like elephants with smaller ears and bigger eyes. 
  Maybe they can be tamed. 
  
  \lyricstitle{\emph{Call of Cthulhu} RPG p.193}{
    \quo{%
      \ldots{} he had seen quaint lumbering buopoths come shyly from the wood to drink\ldots{}}
  }
\end{comment}













\subsection{Cats}
\target{cats}
Cats are an example of \hs{animals that can see into the Beyond}. 
They have great senses and know much of the true nature of the world, and are much more intelligent than humanoids tend to believe. 
However, a cat's mindset is quite alien, and it is difficult for humanoid \hr{Telepathy}{telepaths} to communicate well with a cat. 

Cats can see and move into the Beyond. 
That is why they are so skilled and stealthy hunters. 









\begin{comment}
  \subsection{Dogs and wolves}
  \target{dogs}
  \target{wolves}
  Wolves and other wild canines have good sight and \hr{animals that can see into the Beyond}{know quite a bit of the Beyond}. 
  %Especially the great and mighty \MoonWolves, of course (see section \ref{Moon-Wolves}). 
  They do not have a deep an insight as cats, but their minds are closer related to those of humanoids, so wolves are easier to communicate with.
  
  Tame dogs are much further removed from their \hr{Wild}{\Wylde} roots and more deeply entangled. 
  They see less than their \Wylde{} kin, and in certain regards they are more stupid.
\end{comment}















\begin{comment}
  \subsection{Gargantuan Beast}
  Have Gargantuan Beasts, like the ones in the game \emph{Diablo II}. 
  
  They look kind of like bears on two legs, but with a short neck and a flat face. 
  They are herbivorous and mostly peaceful. 
\end{comment}














\subsection{Hippopotamus}
\target{Hippopotamus}
\target{hippopotamus}
\index{hippopotamus}
The {hippopotamus} was the largest and most dangerous mammal in the world, as far as some people know.

Although there might be big whales that have the hippopotamus outclassed. 













\begin{comment}
  \subsection{Lion}
  \Miith{} has sabre-toothed lions. 
  
  They were once widespread all over \hr{Velcad}{\Velcad}, but have been hunted to near extinction by humanoids. 
  Now they are rare. 
  
  No longer a menace, they have since been romanticized and idealized (from the \hs{Vaimon age}) as a \quo{king of animals}. 
\end{comment}















\subsection{White tiger}
\index{white tiger}
\target{white tiger}
A species of tiger that lives in northern \hr{Velcad}{\Velcad} and the \hs{Northern Kingdoms}. 
It has a thick white fur. 

It is rare. 
Large \saurians{} are much more common. 















\section{Reptiles}









\subsection{\Brukath}
\target{Brukath}
\index{\brukath}
The largest species of sauropods on \Miith. 
Can exceed 45 metres in length and 200 tons in weight. 













\subsection{Caterpillar lizards}
I should have a race of caterpillar/lizard-like creatures that inhabit \hr{Machai}{\Machai} or the \hs{Veins}. 

Inspired by the drawings \quo{Curl-up} and \quo{House of stairs} by M.C. Escher. 

\lyricswikipedia{Curl-up}{Curl-up}{%
  Curl-up or Wentelteefje (original Dutch title) is a lithograph print by M. C. Escher which was first printed in November, 1951.
  
  This is the only work by Escher which consists largely of text. The text, which is written in Dutch, describes an imaginary species called Pedalternorotandomovens centroculatus articulosus, also known as \quo{wentelteefje} or \quo{rolpens}. He says this creature came into existence because of the absence in nature of wheel shaped, living creatures with the ability to roll themselves forward.
  
  The creature is elongated and \armoured with several keratinized joints. It has six legs, each with what appears to be a human foot. It has a disc-shaped head with a parrot-like beak and eyes on stalks on either side.
  
  It can either crawl over a variety of terrain with its six legs or press its beak to the ground and roll into a wheel shape. It can then roll, gaining acceleration by pushing with its legs. On slopes it can tuck its legs in and roll freely. This rolling can end in one of two ways; by abruptly unrolling in motion, which leaves the creature belly-up, or by braking to a stop with its legs and slowly unrolling backwards.
  
  The word wentelteefje is Dutch for French toast and is a contraction of \quo{wentel het eventjes}, which means \quo{turn it over briefly}. Rolpens is a dish made with chopped meat wrapped in a roll and then fired or baked. \quo{Een pens} means \quo{belly}, often used in the phrase beer-belly.
  
  There is a diagonal gap through the text containing an illustration showing the step by step process of the creature rolling into a wheel. This creature appears in two more prints completed later the same month, House of Stairs and House of Stairs II.
}















\subsection{\Corgorah}
\target{Corgorah}
\index{\corgorah{} (plural \corgoroth)}
\Gorgoroses{} are huge reptiles, terrible monsters from the \Wylde{}. 
A \gorgoros{} resembles a large theropod dinosaur like \latinname{Tyrannosaurus} or \latinname{Allosaurus}, but with far larger and stronger forearms ending in huge, wicked claws, like the Behemoths of the \emph{Heroes of Might and Magic} games. 

\Cortios{} are wild beasts, but they can be tamed, albeit with difficulty. 

Perhaps they are related to \dragons{} in some way. 

They have great mouths filled with tons of long, sharp, wicked teeth. Almost like a deep sea fish. 

Compare to Godzilla. 









\subsubsection{Name}
Singular \emph{\cortio{}}, plural \emph{\cortios{}}. 
% This declination is English. 
The adjective is \emph{\cortio{}}. 

In Rissitic they are called \emph{\tashrek{}}. 









\subsubsection{Physique}
\Cortios{} are huge reptiles, similar to dinosaurs like Tyrannosaurus or Allosaurus. 









\subsubsection{Biology}








\subsubsection{Psychology}
\Cortios{} may be tamed, but they are difficult to control. They are savage, aggressive creatures and must be brutally dominated if they are to be kept loyal. 









\subsubsection{Habitat}
\Cortios{} are most common in Durcac, southern Irokas and the Orient, but some exist in southern \Velcad{} and the Imetrium. 















\subsection{\Mezolisk}
\target{Mezolisk}
\index{\mezolisk}
Large crocodile-like or \dragon-like monster with great spikes. 

They are related to \ophidians. 
Like them, they can cool down and go cold-blooded. 

They are quite intelligent and can be tamed. 
The Rissitics use them as beasts of war. 
So do the Geicans and the people of the Orient. 
And even countries such as Runger. 

Also called \quo{dagger-drake}. 















\subsection{Ivory cobra}
\index{ivory cobra}
A cobra snake that is white in \colour and grows up to 150 cm in length. The ivory cobra is highly poisonous and its bite can easily kill a \human{} or \scatha{} (the teeth of an adult are strong enough to penetrate clothes and a \ps{\scatha} scales). 

The ivory cobra lives primarily in Durcac. The animal is sacred to the Rissitics and a symbol of their religion and empire. 















\subsection{\Lotha}
\target{Lotha}
\index{\lotha}
A \lotha (plural \emph{\lothae{}}) was a medium-sized theropod, 7-10 metres long.
They could be tamed and used as mounts and beasts of war. 

The Rissitics \hr{Rissitic monsters}{used \lothae in war}.





\subsubsection{Psychology}
\target{Lothae are skittish}
\Lothae were somewhat skittish and easily frightened by \hr{Guns}{gunfire} and other things. 
They were not evolved to stand their ground against threats.















\subsection{\Mulgrons}
\target{Mulgron}
\target{Muroc}
\index{\mulgron}
A species of large ceratopsian dinosaurs. Similar to \latinname{Triceratops}.





\subsubsection{Psychology}
\target{Murocs are steady}
Some \saurians (such as \hr{Relcs are skittish}{\relcs} and \hr{Lothae are skittish}{\lothae}) were skittish and easily frightened by \hr{Guns}{gunfire} and other things. 
Not so \murocs. 
A \muroc was rock-solid and almost never panicked. 
It was a ten-metre colossus of muscles, \armour and horns. 

\Murocs were evolved not to flee from threats (they were slow), but to stand their ground, keep their calm and \emph{fight}. 
They were warrior animals. 
This made them great \hs{beasts of war}. 















\subsection{\Relc}
\index{\relc{} (plural \relcs)}
\target{Relc}
A herbivorous, quadrupedal \hr{Saurian}{saurian}. 
It around 5-6 metres long and has an elaborate crest on its head. 
Can be tamed and used as mounts or pack animals. 

\Relcs{} vary widely in \colour. 
Some are green or brown, some are zebra-striped in black and white. 

A person who rides a \relc{} is called a \relcer{}. 

As for \hr{Domestic animals}{domestication}: 
\Relcs were very much military animals.
Armies had them, but few civilians. 
Civilians would use other and slower animals.

\meta{%
  Similar to dinosaurs like \latinname{Saurolophus} or \latinname{Corythosaurus}, but smaller.}





\subsubsection{Psychology}
\target{Relcs are skittish}
\Relcs were somewhat skittish and easily frightened by \hr{Guns}{gunfire} and other things. 
They were herd animals.















\subsection{Sauropods}
\target{sauropod}
\target{sauropods}
\index{sauropod}
Sauropods are a group of \saurians{}. 
Species include the \brukath{} and \tondra. 

Some sauropods have an elephant-like trunk. 















\subsection{\Tondra}
\target{Tondra}
\index{\tondra}
A very large, quadrupedal, herbivorous \hr{Saurian}{\saurian} with a massive body and a long neck and tail. 
It is heavily built and \coloured in shades of white, beige and tan. 
One of the largest known \saurians{}, it can reach over 35 metres in length and weigh 100 tonnes. 

\Tondras{} were domesticated by the \hr{Vaimon Caliphate}{\VaimonCaliphate} and other cultures of their time, but no longer. 

\meta{%
  Compare to sauropod dinosaurs like \latinname{Brachiosaurus} or \latinname{Apatosaurus} (aka \latinname{Brontosaurus}). 
}















\subsection{\Vreiiden}
\target{Wyverns}
\target{Vreiid}
\target{Vraiid}
\index{\vreiid}
Flying, \dragon-like reptiles. They are vicious, savage beasts. They have great mouths filled with rows of dagger-like teeth. 

The \vreiiden{} are a race of flying, reptillian monsters, like wyverns. They are savage and brutal creatures. They can be tamed to some extent, but it is difficult, and they'll never be very tame. They are used as mounts in \quo{dark} armies, like those of the Rissitics. 





\subsubsection{Mistaken for \dragons}
\Vreiid were rare and dangerous giant flying reptiles that resembled the \dragons of mythology. 
They were often referred to as \dragons by the people whom they occasionally preyed upon. 

A \vreiid was a very dangerous monster.
It was hideous and gave off a feeling of great age and evil. 
The fact that the giant creatures could fly made them even harder to defend against. 
But even so, a \vreiid had no magic and was not highly intelligent. 
They were fearsome things, but less fearsome than the \dragons described in mythology.

Some people suspected that the \vreiiden were not \dragons, and that there existed true \dragons far greater and more terrible than the \vreiiden. 

























\chapter{The Undead}















\section{Themes}
\target{Undead}
Remember to have armies of the undead.

Which side will have them? It would make most sense to have them on the Cabal side, mostly the province of the \banes, whose powers and very nature are connected with death, \hr{Entropy}{decay, defilement and parisitism} (like vampirism). 

But \hr{Rissitic pyramids}{Rissit has his pyramids}, which might house legions of the undead, waiting to be unleashed\ldots{}

I should have undead Revenants, like in Warcraft III. 

And Cold \hs{Wraiths}, that were once in Warcraft III. 

See also the section on \hs{Death}.

\lyricsbalsagoth{Arcana Antediluvia}{
  Down sixty fathoms, from stygian coral-clad tombs \\
  the pitiless abyssal sea disgorges its shambling mold-mottled dead,\\
  dank innards blackly acoil with nests of slithering things!\\
  Ghosts aglide upon the eldritch seas, \\
  unfathomed voyage to ascendancy.\\
  Traitorous blood, the surf roils red, \\
  churning crimson, thrice-cursed dead.
}







\subsection{Fallen civilizations of undead}
\target{Fallen civilizations of undead}
Have one or more fallen civilizations who were once great empires but now exist only as undead dwelling in ruined cities. 

Compare to the Tomb Kings from \emph{Warhammer}. 







\subsection{Monuments keeping the dead imprisoned}
See section \ref{Monuments keeping the dead imprisoned}.









\subsection{Undead guardians}
Maybe have undead guardians who are sacrificed and killed and have their souls and bodies bound in undeath to guard some object or place for eternity. 

Inspired by \authorbook{Poppy Z. Brite}{The Sixth Sentinel}. 









\subsection{Undead warrior/assassins}
Maybe I should have some creepy, disgusting undead warrior/assassins. 

Compare to the Autoj\"ager from the anime \emph{Trinity Blood}: \ta{They are corpses\ldots{} the corpses of vampires\ldots{}}












\begin{comment}
\section{Death Knight}
\target{Death Knight}
\index{Death Knight}
The Rissitic Death Knights are a kind of Wights. 

\subsection{Name}
\subsection{Physique}
\subsection{Biology}
\subsection{Psychology}
\subsection{Habitat}















\section{\Leech}
\target{Leech}
\index{\leech}
\Leeches{} are not to be confused with the similarly-named Liches, nor with regular leeches (non-capitalized), wormlike animals that live in swamps and the like and drink blood. \Leeches{} (capitalized) are a lesser form of \Reavers{} (see section \ref{\Reaver}): Intelligent undead who drain the life-force of others to survive. 

A \Reaver{} is created more-or-less voluntarily when a mage uses \hr{Life drain}{life-draining magic} extensively. A person cannot turn himself into a \Leech{}. \Leeches{} are created by \Reavers{} as lovers, companions or servants (or all three). Most \Leeches{} serve a \Reaver, and many dream of becoming \Reavers{} themselves. 







\subsection{Physique}
\Leeches{} look like living people of their race. There is nothing to put your finger on, but sensitive people will notice a certain savage, bestial aura about the \Leech. 

\Leeches{} have most of the powers of \Reavers{}, but weaker. They are supernaturally strong, fast and agile and can regenerate almost all wounds. They can see in almost total darkness and see auras around living creatures. A \Leech{} may take seemingly-mortal wounds yet rise again. If the brain is separated from the heart (usually by severing the head) or either brain or heart is smashed or torn apart, the \Leech{} is permanently destroyed. 

They also have the vulnerabilities of \Reavers{}, albeit to a lesser degree. Their vulnerability to wood is the same, but they are more resistant to sunlight. A \Leech{} will be destroyed in a few minutes if fully exposed to daylight, but normal shade, such as that provided by a broad-brimmed hat or umbrella, is enough to preserve them from harm. Even if exposed to sunlight, they can suffer it for up to a minute with no more than moderate pain and no visible damage. This makes them well-suited as agents for the \Reavers.

Unlike \Reavers, a \Leech{} does not drain life by spells but by drinking the blood of her victim. The victim must be alive or very recently killed for their blood to have any value to the \Leech. 

A \Leech{} retains all her former skills, including magic. But unlike \Reavers, \Leeches{} need not be mages, and most know no magic. 









\subsection{Biology}
To create a \Leech, a \Reaver{} must drain a victim to the brink of death and then cast a special spell on her, called the \quo{\Reaverz Kiss}, that transfers a portion of his life force to her, in effect causing her to drain some of his energy. The spell is also an enchantment that grants the \Leech-to-be a limited ability to drain life force without the use of spells, namely by drinking their blood. 

The \Leech-to-be will awaken with a tremendous thirst for blood. She can drink the blood of the \Reaver, but he is unlikely to allow her to sate her thirst on his own blood. Rather, he will send her to feed on mortals (perhaps keeping prisoners or servants ready, perhaps sending her off to hunt her own). 

At this point, the prospective \Leech{} has the supernatural powers of a \Leech{} to some extent and a great psychological craving for blood, but she is not fully one of the undead and not yet physically dependent on blood to survive. Resisting the desire to feed is difficult, but it can be done. If she refrains from drinking blood, the effects of the spell will gradually wear off. She will become sick and weak, but most likely she will survive, and after 5-7 days the spell has worn off completely and she will be back to normal. (In such a case, however, the \Reaver{} is likely to feel betrayed and attempt to reclaim or punish her.) 

If, on the other hand, the would-be-\Leech{} gives in to her desire and feeds on blood, she will find a great erotic pleasure in doing so, and she will find herself gradually transforming into a full-fledged \Leech{}. \Leeches{} can sustain their strength by drinking mortal blood, but they are not true \Reavers, and their method of draining life force is imperfect. Consequently, they cannot survive on this fare alone, but must also drink the blood of a \Reaver. At least once a month, the \Leech{} must feed on \Reaver{} blood or she will perish. Drinking mortal blood solidifies the \Leechz undead state, but drinking \Reaver{} blood cements it. A \Leech who has drunk \Reaver{} blood once (beyond the initial Kiss) can still fight the curse of undeath and return to life, although it becomes much harder, and even after having drunk twice the curse may be broken, although at this point magical healing and exorcism will be needed. After drinking \Reaver{} blood thrice after the Kiss, the \Leechz fate is sealed. She is now fully of the undead and can never return to life. 

As with a \Reaver, she needs not kill her victims, but unlike a \Reaver, a \Leech{} cannot absorb her victim's soul. The blood of another \Leech{} is particularly savoury and nourishing, but no substitute for \Reaver{} blood. 

The goal of many a \Leech{} is to become a \Reaver{} herself. To do this, she must learn spells of life-draining or acquire an enchanted item that duplicates the effect. This will let her drain true life force on her own, eliminating her need to feed on her master and thus eliminating her dependence on him. After draining life with magic for a while (usually a matter of months, since she will likely be feeding a lot), the \Leech{} will transform into a true \Reaver, with all the powers and weaknesses described in section \ref{\Reaver}. 

Like \Reavers{}, \Leeches{} do age, but they can arrest and reverse aging by drinking blood, thus living forever. 









\subsection{Psychology}
Some \Leeches{} love their \Reaver{} master and serve him willingly. \Reavers{} often turn their lovers into \Leeches{}. A \Reaver{} may have a single \Leech{} lover and treat her as an equal or near-equal, or he may have a whole harem of them vying for his affection. The \Reaver{} may, of course, also turn non-lovers into \Leeches, but only those people he believes he can trust. A \Reaver{} who truly cares for his \Leech{} might even teach her magic so that she may become a \Reaver{} herself. 

A \Reaver{} has no supernatural control over his \Leeches{}, beyond the fact that they need his blood to survive. Usually, he will dominate his \Leech{} servants through physical and magical power and sheer force of will. After all, the \Reaver{} will usually be not only much older and more experienced, but also a great mage, whereas most \Leeches{} know no magic. 

Still, a \Leech{} might hate her \Reaver{} master and plot against him. Of course, unless she also wants to end her own existence, she cannot simply kill him. But there are ways of dealing with this dependence:

First, a \Leech{} must drink \Reaver{} blood, but it needs not be that of the \Reaver{} that created her, so a disgruntled \Leech{} who encounters another \Reaver{} might defect. 

Second, a \Leech{} might be able to overthrow her master, keeping him as a prisoner on whom to feed at leisure. Usually, the \Leech{} will be no match for her \Reaver{} master (he will see to that), but she might be exceedingly cunning. Alternatively, if a \Reaver{} has several \Leeches{} under his control, they might band together to overthrow him. For this reason, a \Reaver{} who does not fully trust his \Leech{} slaves might play them against each other, turning them into rivals and enemies, ensuring they they do not ally against him. 

Third, the \Leech{} might learn of the fact that it is possible for her to become a \Reaver{} herself. If she manages to gain access to life-draining magic, she is likely to flee. 

To ensure her obedience, a \Reavers{} will often lie to his \Leech{} slave, convincing her that only his blood (not that of another \Reaver) will keep her alive and keeping her ignorant of how she might become a \Reaver. 

A \Reaver{} may create any number of \Leeches{} he desires, but each of them must drink his blood every month or die, and letting the \Leech{} drink his blood temporarily weakens the \Reaver, so he will not want too many mouths to feed. More importantly, a large group of \Leeches{} may mutiny against their master (as described above), so a \Reaver{} will not want to create more of them than he can control. 









\subsection{Habitat}
\Leeches{} always live near their \Reaver{} master (or they won't live long). But since they are less vulnerable to sunlight, they are more easily able to lead normal-seeming lives and not arouse suspicion. 
\end{comment}
















\section{Lich}
\target{Lich}
\index{Lich}
Liches (not to be confused with \Leeches, who are weaker undead) are arguably the most powerful type of undead. A Lich is a powerful mage who has transformed himself into one of the undead by means of an occult spell of terrible power. 

What characterizes the Lich is that it is fully self-sustained and immortal. The spell that creates the Lich opens a conduit to a source of dark energy somewhere in the Beyond. From now and evermore, this power sustains the Lich. Thus, the Lich needs to external power source. It requires no magical rituals to sustain it, nor must it feed on the life-force of others, like \Reavers{} do, but will sustain itself and exist forever.

A Lich is fully intelligent and retains all the skills and knowledge it possessed in life, including magic, and it will continue honing its skills and learning more magic throughout its immortal unlife. A mage must be of formidable skill and knowledge in order to become a Lich in the first place, and will grow only stronger as the centuries pass, so an old Lich is a mighty creature indeed. 









\subsection{Name}
Singular \emph{Lich}, plural \emph{Liches}. 









\subsection{Physique}
Liches retain the form they had in life\ldots{}









\subsection{\XulGann}
The \hr{Xul-Gann}{\XulGann} were a Rissitic form of \Liches. 















\begin{comment}
\section{\Reaver}
\target{\Reaver}
\index{\reaver}
Using dark magic, some people are able to \hr{Life drain}{drain the life force of others} to empower themselves. This is a very potent ability, but it carries a price: Repeated use of life-draining magic is addictive, physically as well as psychologically, and the mage will develop a growing craving for it. As the addiction grows, the mage will gradually transform into one of the undead. Such people are called \Reavers{}. 

\Reavers{} are usually powerful mages, but non-mages who rely on enchanted items to drain life force are also affected and may become \Reavers{}. 

Becoming a \Reaver{} is a gradual process. As a person succumbs to the addiction, his \Reaveric{} traits will grow stronger and more pronounced. The transformation is considered complete when the \Reaver{} can no longer sustain himself by natural means (food and drink) and must live off the life force of others alone. 

%\Reavers{} are people who 









\subsection{Name}
%\emph{\Reaver}, plural \emph{\Reavers} as in English. 
As in English. 
%The adjective is \emph{\Reaveric}. 









\subsection{Physique}
\Reavers{} look much like living people of their race. In time, their skin will grow somewhat pale, but they can still pass for normal people. Only people or creatures with special empathic skills will be able to detect the \Reaver{} at a glance. 

As a \Reaver{} grows in power, he will learn to drain more life force than he needs, storing it in his body to make himself supernaturally strong and fast. Powerful \Reavers{} are able to perform astonishing feats of superhuman agility and strength. Such power, however, is costly to maintain, and a \Reaver{} who wishes to remain strong must drain great amounts of life. \Reaver{} mages can use their drained life force to power their spells, making them very formidable spellcasters. 

\Reavers{} gain the ability to see in darkness, their vision unnaturally sharp even in minimal light. They cannot see in total darkness, but they see as well by starlight (under a clear sky) as in broad daylight. They also gain the ability to see the life force auras surrounding living creatures. This sense is similar to infravision, except that it shows life, not heat. It \emph{will} work in complete darkness and can see through up to 10-15 cm of earth, wood or stone or 1-2 cm of metal. This sense will \emph{not} detect undead or artificial constructs. It may or may not show alien creatures, depending on how alien they are. 

As a \Reaverz{} mastery grows, he can drain energy from opponents by a mere touch without having to cast an elaborate spell. Some \Reavers{} know spells that let them drain life through clothes or \armour, through a weapon or even at a distance. 

A \Reaverz{} greatest weakness (apart from his need to feed on life) is that he cannot abide the light of the Sun. Direct sunlight chars the \Reaverz{} flesh like fire. This vulnerability grows gradually as the \Reaver{} undergoes his transformation. A fully transformed \Reaver{} will die and crumble to ashes within a minute if exposed directly to full daylight (a few minutes if the he is large, like a \dragon{})\footnote{Only the body parts exposed to sunlight will burn. If the \Reaver{} is naked, his entire body will burn. But even if only the head is exposed, the \Reaver{} will still die when his head burns and crumbles.}. Light cast back from a strongly reflecting surface (such as a mirror or a lake) is almost as dangerous as direct sunlight. Light reflected from a bright surface (like snow, white marble or shiny metal) is less dangerous, wheras light cast back from dark surfaces (like wood or earth) is mostly harmless and can be endured for minutes with little harm. 

\Reavers{} mostly avoid going outside by day and travel only by night. If a \Reaver{} must travel by day, he will cover his entire body in clothing, using a hood, hat or mask to cover his head. There are subtle spells that offer some protection, but none are known that let a \Reaver{} withstand full daylight. A \Reaver{} unafraid of detection may use spells to cover himself in a cloud of darkness, providing further protection. 

Most non-sunlight (moonlight, bonfires etc.) is harmless to \Reavers{}. As for magical light, the spell description will sometimes state that the light works like sunlight. In this case, it will burn and can kill \Reavers{} as described above. 

\Reavers{} are vulnerable to weapons made of wood. Wooden weapons that hit bare skin or thin clothes will cause extra damage. Wood that strikes a \Reaver{} wearing \armour has no special effect. If a piece of wood pierces a part of the \Reaverz{} body, that body part will be paralyzed until the wood is removed, and numb for a while even then. If wood pierces the head or torso  (not necessarily through the heart), the \Reaverz{} entire body will be immobilized. This will work even with a wooden arrow with a metal head. 

\Reavers{} have the ability to heal almost any wound, reattach severed limbs or even regrow them from scrath. Burns, such as from sunlight, and wounds from wooden weapons can also be healed, albeit slower than most damage. Using powerful necromancy, \Reaver{} mages can sometimes even resurrect themselves after being killed and mutilated. This will leave the \Reaver{} drained and weak, however, and he must quickly feast upon great amounts of life or perish forever. If the \Reaverz{} body is burned (in fire or in sunlight), destroyed with acid or eaten and digested, it is destroyed beyond hope of resurrection. (Such a \Reaver{} might, however, still be raised as one of the incorporeal undead, such as a Wraith.) 

Certain animals can detect \Reavers{}. This includes all canines, all small cats (but not all large cats) and \nycans{}. They do this by a combination of smell and empathy. These animals will fear and hate the \Reaver{} and are likely to either attack or flee. 

It is possible for persons to learn the mystic, empathic skill of recognizing \Reavers{} and other undead. Imetric Paladins and the priests of \NishiS{}, certain Vaimon Templars and Clerics and the Ashenclaw knights of \KhothSell are all taught this ability. 









\subsection{Biology}
All races can become \Reavers{}. There are different ways to become a \Reaver{}, because there exist several varieties of life-draining magic. It is often said that \Reavers{} drink the blood of their victims, but in fact this is only one of several means of draining energy. %Some \Reavers{} know different ways to drain life, but some know only one. A \Reaver{} who only knows to drain life by drinking blood is called a Blood \Reaver{} by scholars. Other varieties are the Shadow \Reaver{} (using Rissitic Shadow magic), Nieur \Reaver{} (using Vaimon magic) and Chaos \Reaver{} (using \draconic{} Chaos magic). The different types of \Reavers{} will differ in their array of spells and skills, and likely also in culture and habits. 

The Rissitic Ashenoch cannot become \Reavers{}. Using life-draining magic, the Ashenoch will develop a physical and mental addiction, but they will not transform into undead or gain any of the \Reaveric{} traits described above, no matter how much energy they drain. 

Contrary to popular belief, those drained and killed by \Reavers{} do not rise as \Reavers{} themselves. A \Reaver{} cannot create other \Reavers{}. A person can only become a \Reaver{} through use and abuse of life-draining magic. However, it is possible for a \Reaver{} to create \Leeches{}, undead creatures similar to \Reavers{} but weaker (see section \ref{\Leech}). 

Unlike most undead, \Reavers{} do age, and their bodies will decay and weaken. In fact, \Reavers{} age much faster than the living and may age a decade in a single month. However, a \Reaver{} can use drained life force to rejuvenate himself, thus staying young and immortal potentielly forever. The \Reaver{} can regulate how much he wants to rejuvenate the surface of his body. Most \Reavers{} choose a certain age and maintain their appearance to fit this age. He may change this at any time by allowing himself to age or spending more power to rejuvenate himself. (This is not immediate, but may take several months.) Thus, regardless of his true age, a \Reaver{} may have the look of a youth or an ancient man. Whatever their skin looks like, however, all \Reavers{} make sure to keep their inner body healthy and strong, so the \quo{old man} \Reaver{} may be just as physically strong and agile as the \quo{young man} \Reaver{}. Only when a \Reaver{} is deprived of energy to drain will he begin to weaken. 

A \Reaver{} will die after 10-50 days of not feeding, depending on his strength and how much he exerts himself. But they prefer to feed every day. A \Reaver{} needs not kill his victim, but draining a victim provides much more energy for the \Reaver{} to absorb than merely draining her to the brink of death, because then the \Reaver{} is able to absorb her very soul. A victim thus absorbed is destroyed forever and can never be resurrected. 

\Reavers{} particularly savour the life force of \Leeches{}, or better yet, other \Reavers{}. (These are the only undead that can be drained.) Draining another of the undead is exceptionally nourishing and enjoyable to the \Reaver, but moreover, draining another undead to destruction and absorbing her will cause the \Reaver{} to gain a portion of her power, thus making himself permanently stronger. Occasionally, this has the side effect that the \Reaver{} will adopt some bits of the absorbed one's personality. 

A \Reaver{} cannot eat food nor drink normal drinks. He can swallow it, but his atrophied digestive system rejects it, but will have to regurgitate it soon after (up to ten minutes at most). 









\subsection{Psychology}
It is sometimes said that \Reavers{} can feel no pleasure or happiness and are forever tormented by their foul, unnatural state. It is true that there are some \Reavers{} who come to hate what they have become and succumb to self-loathing and depression. 

But \Reavers{} can actually feel lots of pleasure. Unlike most undead, \Reavers{} retain their sexual drive and can still have and enjoy sex in their undead state (although a fully transformed \Reaver{} is sterile). Draining energy is, to some extent, viewed as a sexual act, and \Reavers{} prefer to drain attractive people of their own race and the opposite sex (or whatever attracts them). Reavers can live off unintelligent animals if they must, but it is much less nourishing and distasteful to them. Draining the life of a beast gives none of the pleasure that comes with draining an intelligent creature. 

%As mentioned, they enjoy sex just as they did in life, and 

Most other passions from life remain in undeath as well. They can also taste food and drink, but this pleasure is marred by the fact that they'll have to vomit it up again. 

%Unlike most undead, \Reavers{} retain their sexual drive and can still have and enjoy sex in their undead state. Once the transformation into undead is complete the \Reaver{} is sterile, though. 

It is uncommon but not unheard of for \Reavers{} to have relationships with other \Reavers{}. More commonly, a \Reaver{} will create \Leeches{} to act as his companions and servants. 









\subsection{Habitat}
Some \Reavers{} live in civilization, where they find ways to avoid the sunlight and lead ordinary-looking lives. Others live removed from civilization and only seek out other people when they must feed. 









\subsection{Myths}
The story of the \Reaver{} is well-known throughout \Miith{}, and myths and superstition about them about in all lands. The vast majority of the storytellers who spread such myths have never met a \Reaver{} (though they may claim otherwise), so it should come as no surprise that much of what is \quo{known} about \Reavers{} is pure superstition. 

According to the tales, \Reavers{} cannot cross running water, cast neither shadow nor reflection and cannot enter a home without invitation. Garlic or a holy symbol of Good repels them. Wolves, rats and bats serve the \Reavers{}, and they can assume the form of any of these animals. 
 
\end{comment}
















\section{Wraiths}
\target{Wraiths}
\target{wraiths}
\index{wraith}
Have a race of black wraiths with blank faces, hair-like protuberances like the Protoss from \cite{VideoGame:Starcraft} and tails like Genie from Disney's \emph{Aladdin}. 

Maybe they are a type of \banes. 























\chapter{Plants and Other Life Forms}
\section{Coral reefs}
\target{coral reefs}
\index{coral reefs}
A coral reef is a collective organism, a colony made up of multitudes of smaller creatures. 
It is hiveminded and intelligent. 
But this intelligence is alien and malevolent to humanoids. 

The \nagae{} know the true nature and power of the coral reefs and fear them. 
Perhaps they worship them. 















\section{Trees}
\target{trees}
Remember to have some mysticism about trees. Trees are cool. They can live for thousands of years and accumulate wisdom and secrets. But they are cold, alien and terrifying. 

Trees hate civilized humanoids, and with good reason, for humanoids cut down trees and destroy the \Wylde{} to carve out their detestable, parasitic cities. 























