
\part{The Cosmos}
\chapter{Nature of the Universe}
\section{\Dweomers}
\target{Dweomer}
\Dweomers{} are not physical objects but abstract sources of power\dash lifeforce and magical energy. 
They are disembodied forces that exist in the Web of Realms; distributed systems spread out on myriad threads. 








\subsection{\Dweomer{} filters}
\target{Dweomer filters}
\target{front-end}
There exist small \dweomers{} that are dependent on a mother \dweomer. 
Compare to how small streams branch off from a large river, which in turns branches off from the ocean. 
These small \dweomers{} can act as filters, \quo{front-ends} for a stronger but less user-friendly \dweomer. 
A good example is \iquin, which is a filter on \itzach, itself a front-end for the \hr{Heart}{Heart of \Miith}. 









\subsection{The Sun}
\target{Sun}
\index{Sun, the}
The Sun is a \dweomer{} connected to the \hr{Heart}{Heart of \Miith}. 
%It may be \hr{\voyagers{} create the Sun}{created by the \voyagers}.

The \banes{} want to conquer and control the Sun. 

\lyricsdimmuborgir{Vinder Fra En Ensom Grav}{
  Se der i fjellbrisens indre,\\
  hvor dets liv har s\oe{}kt ly for en sol\\
  som omsider vil m\oe{}rkne med den evige natt\\
  og hindre en verden i (\aa{}) blomstre og gro.
}









\subsection{The Heart of \Miith}
\target{Heart of Miith}
\target{Heart}
\index{Heart of \Miith{}, the}
\index{\Miith{}!Heart of \Miith{}, the}
The \quo{Heart of \Miith} is a \dweomer{} which many factions seek to access and control. 
It is, in a sense, the source of all life and power on \Miith{}. 
It is the centre of the Shroud, and whoever controls the Heart has the power to reshape the Shroud. 

The name \quo{\Miith} belongs to the Heart. 
It ought to have a longer name in \draconic, of which \quo{\Miith} is a contraction. 

Also, remember to name the Realms that are part of \Miith{}. 

\Erebos{} is \emph{not} a part of \Miith{}, but \Nyx{} is. \Machai{} might or might not be. 

The Heart of \Miith{} is connected with the \hs{Sun}. The Sun is the most obvious conduit for the Heart's life-giving power\dash but not the only one, nor the strongest. 

Perhaps the Heart was \hr{Voyagers create the Heart}{created by the \voyagers}. 









\subsubsection{Poetry about evolution}
Such is the history of \Miith{}, a crucible of life, a fierce and bright flame amid the darkness of the Cosmos. Whatever is given to her, she will smelt down and reforge into something greater. 

The void gave her lowly chemicals, and she gave birth to the \krakens{}, immortal and terrible. 

The \voyagers{} gave her fish and reptiles and mammals, and she made them into \caderyns, \nycans{}, \cortios{}, whales and elephants, creatures wild and strong. 

\Moroch{} took the fish-reptiles from the sea and created \nagae{}. \Miith{} took the \nagae{} and turned them into \dragons{}, majestic and proud. 

The \banes{} took the monkeys from the trees and created \humans{}. \Miith{} took the \humans{} and turned them into Vaimons, noble and arcane. 

Only time will tell what further wonders and horrors will spring from the womb of Mother \Miith{}, for time stands never still, and in her great forge, the fires burn ever hot\ldots{} 





\subsubsection{Fixed total power of the Heart}
Perhaps there is a certain fixed upper bound to him much power can be drawn from the Heart at one time. 

The \ophidian{} peoples are the prime consumers of this power. But since the \hr{Origin of Satharioth}{coming of the \satharioth}, the \resphain{} have been digging into it as well. The \dragons{} are bitter about this, because this steals some of the power that should be theirs. 

This is the real reason for the \hr{Feud}{\feud}: The two species fight for control of the power of the Heart. Both need it to keep their species alive and evolving. But they need the complete Heart. As long as it is divided, it is not strong enough. Therefore, they must fight until one side is annihilated. 

\Tiamat-tachi managed to expand the total power pool by drawing more \xsic{} power. But they also hogged a lot of it for themselves. 





\subsubsection{Mass extinctions}
Some of the great wars (like the two \Banewars) have tampered with the Heart and caused mass death among plants and smaller animals, because they were cut off from the Heart's life. 
This has caused more than one ecological catastrophe that laid waste to the planet. 





\subsubsection{Strong after the \firstbanewar}
\target{Heart after First Banewar}
After the \firstbanewar, \Miith{} was depopulated, but the Heart was still in a good shape. 
So there was energy enough to spawn many tens of thousands of \resphain. 
But near the end of \ps{\hr{Merkyrah}{\Merkyrah}} time, the \hr{Merkyrah strained the Heart}{Heart was getting over-strained}. 





\subsubsection{Weakening and dying after the \secondbanewar}
\target{Heart weakened}
At the time of the \thirdbanewar, the Heart was weakening and dying. 

This had multiple causes.

\begin{enumerate}
  \item 
    \hr{Immortality is an abomination}{Immortal souls steal power}. 
    The mere existence of the immortal and \hs{soul-eating} \resphain{} (and, to a lesser extent, \dragons) is an atrocity against the Heart and drains its power. 
    
  \item 
    The \draconic{} and \resphan{} \hr{Matrix}{\matrices} are fundamentally opposed, and their metaphysical struggle tears the Heart in two. 
    
  \item 
    The \hr{Crystal Sphere}{\CrystalSphere} kept \Miith{} \hr{Crystal Sphere isolates Miith}{hermetically sealed and isolated}. 
    This was constricting and strangling. 
    
  \item 
    \target{Shroud harms fertility}
    The \hs{Shroud} constricts and suffocates the Heart, making it hard for it to beat and flow and live. 
    
  \item 
    \hr{Iquin}{\Iquin} makes everything worse by gobbling up souls and hoarding them. 
\end{enumerate}

The \Miithians had fucked their homeworld over several times, and all the harm was cumulative. 
The Heart could not take much more. 

There were much bigger populations back in the days of \hr{Tembrae}{\Tembrae} (especially the immortal populations, but also the mortal ones). 
Nowadays the Heart is weaker and cannot sustain as many living creatures. 

\lyricslimbonicart{Last Rite for the Silent Darkstar}{
  Feel the heartbeat of the earth,\\
  as you're on the threshold to death or birth.\\
  Minds are tormenting, head is gone wild.\\
  Release yourself from what you can't abide.\\
  A star will also die \\
  on the beautiful moonlit sky.\\
  As the life is ended,\\
  the Goddess mother weeps her child.
}

The \banelords{} are not so concerned. 
True, the Heart is faltering. 
But it doesn't take much to stir up an ecosystem and make it unpleasant to live in. 
The \banes{} know that even though it seems like all of \Miith{} is drying up and dying and will be a dead wasteland within a thousand years, there is actually plenty of juice left in the Heart. 
If \Miith{} were to die of starvation, it would be a very long and slow death that could take millions of years. 
Civilizations and species are easy to wipe out; Realms are hard to kill. 
The \banelords{} know that there will be plenty of Heart power left for the \hr{Voidbringer}{\Voidbringer} to gobble up when he comes to devour the Heart (which he is supposed to do at the culmination of the \hr{Sephirah plan}{\sephirah{} plan}). 

Compare this to the game \cite{VideoGame:FinalFantasyVII}, where the evil corporation Shinra is draining the Lifestream dry by extracting all the \emph{Mak\=o}. 

\lyricstitle{
  \href
    {http://finalfantasy.wikia.com/wiki/Final_Fantasy_VII}
    {Final Fantasy Wiki: Final Fantasy VII}
}{
  Shinra [\ldots{}] profits from the use of machines known as \quo{Mako Reactors}. These reactors siphon a special type of energy\dash called \quo{Mako}\dash out of the Planet and convert it into electricity. One of the by-products of the extraction and refinement of Mako energy is materia, a concentrated form of Mako which allows the owner to use its magical properties. [\ldots{}]
  
  In actuality, Mako energy is drawn from the Lifestream, a flow of life-force beneath the surface of the Planet. All life originates from the Lifestream, and returns to it upon death. In short, the Lifestream is the sum of all the life that has ever and will ever walk upon the Planet. The process of extracting Mako energy literally drains the life of the Planet in order to generate electricity. This can be seen quite clearly in the Shinra's capital city of Midgar, where the eight Mako Reactors have sucked out so much of the Planet's life-force that the area is covered in perpetual darkness and no plants can grow.
}













\section{\Iquin and the \Sephiroth}
\target{Iquin}
\Iquin{} is a \hr{Dweomer}{\dweomer} that drawn power both from \Erebos{} and from the Heart of \Miith{}. 

It is the force of \quo{Light} in \hs{Vaimon} metaphyics. By the \hs{Iquinian Church} viewed as the source of all good and worshipped as a divine force. Its manifestations are the \hr{Sephiroth}{\Sephiroth}. 

In a sense, \iquin{} is a \emph{face}: 
A humanoid, benevolent-looking mask strapped on to hide the true underlying darkness, which is as faceless and inhuman as a \bane. 








\subsection{History}
\subsubsection{\Merkyran{} incarnation of \iquin}
\target{Old Good Iquin}
When \hr{Origin of Resphain}{the \resphain{} were first spawned}, \nieur{} was developed and improved by \hr{Semiza}{\Semiza} and the early \resphain{} (who were \hr{Entropy}{more creative than the \banes}). 

Then \hr{Resphain lose contact with Banelords}{disaster struck} and the \resphain{} lost not only the contact with \Semiza{} and the \banelords, they also lost their memory. 
They founded \hr{Merkyrah}{a good empire}, and they were noble and good. 

So they shaped their own \dweomer. 
It was just \hs{front-end} for \nieur, and drew all of its energy from \Nyx{} and the \banelords, but it was much more user-friendly and encapsulated all the evil and \hs{Entropy} of \nieur. They called it \iquin, \quo{the Light}. 
It was filtered through the Shroud and encased in all the \ps{\resphain}{} denial and idealism, which made it feel good and benevolent and concealed its true, destructive, vampiric, parasitic nature. 
The good guys' filter suppressed the \Nyxian{} aspect of their power and kept it in check, like a mini-Shroud, but it lay latent just underneath the surface. 

In time, they even came to \hr{Good Resphain and God}{personify \iquin{} as \quo{God}}. 

They knew about the existence of \nieur, but forbade its use. Only the \hr{Early Resphan fallen ones}{fallen ones} used it. 

After the \hr{Resphan rebellion}{\resphan{} rebellion}, \iquin{} was abolished and \nieur{} was embraced.





\subsubsection{\Kezeradi{} incarnation of \iquin}
\target{Kezeradi Iquin}
When \Kezerad{} was formed, the \Kezeradi{} took up \iquin{} again. 
But they improved and refined it by incorporating technology gleaned from the \ophidians{} and the \hr{Ophidian power source}{natural \Miithian{} power source they used}. 

This new \iquin{} was designed by genius scientists such as \hr{Eryal}{\Eryal}. 
\hr{Sithiyacaan}{\Sithiyacaan} helped by providing brute-force backup.  

\hr{Daggerrain}{\Daggerrain} saw this \dweomer{} and realized it could be useful to him. 
Eventually \hr{Fall of Kezerad}{\Kezerad{} was destroyed} and their \iquin{} became twisted by the \sephiroth{} into the \iquin{} that the Vaimons would later use. 





\subsubsection{Creation of the \sephiroth}
The sixteen \Sephiroth{} were originally created from the souls of sixteen rebellious \resphan{} lords, each merged with the soul of a sacrificed \banelord. 

These \resphain{} were the lords of the rebellious \Kezeradi. After the rebels' defeat, the \banes{} (or \bane-allied \resphain) decided that using their leaders as \Sephiroth{} would not only be an effective and useful tool, but also a suitable punishment: They would live on to see the virtues they fought for corrupted and perverted, and their own power would be used to achieve it. 

A theme throughout the series is that some of the characters gradually discover the nature of the \sephiroth. 
Some work actively to destroy the \sephiroth, thus freeing the souls of the captive \Kezeradi{} lords. 





\subsubsection{Religious beliefs}
For Iquinian religious/mystical beliefs about the \sephiroth, see the section on the \hs{Iquinian Church}. 









\subsection{How it feels}
For \humans{}, \iquin{} is the ultimate: 
Highest, deepest, brightest, darkest.
Most beautiful, most terrible. 
Most alluring, most abhorrent. 

\citebandsong{DeathspellOmega:CrushingtheHolyTrinity}{%
  Deathspell Omega
}{
  Diabolus Absconditus
}{
  God is nothing if He is not, in every sense, the surpassing God;\\
  In the sense of common everyday being, in the sense of dread,\\
  Horror and impurity, and, finally, in the sense of nothing\ldots{}
}





\subsubsection{How it feels to die}
Dying and going into the light feels like this:

\citebandsong{BeyondTwilight:FortheLoveofArtandtheMaking%
}{%
  Beyond Twilight%
}{%
  For the Love of Art and the Making%
}{
  Close your eyes, fall into deep\\
  Within this place where secrets hide\\
  The perfect heart, it's behind closed doors\\
  I don't want it to hide\\
  SOS. Help me, I'm falling\\
  into the perfect heart\\
  The perfect heart
}

When a \human{} meets \iquin, he sees a brief, vague glimpse of mankind's true nature, origin and purpose. 

\citebandsong{BeyondTwilight:FortheLoveofArtandtheMaking%
}{%
  Beyond Twilight%
}{%
  For the Love of Art and the Making%
}{
  Whispering thoughts of ages untold\\
  Think of what it tells you\\
  Lost tales of ages past\\
  The whispering winds are roaming fast\\
  Oh so fast
}





\subsubsection{How it feels to learn the truth}
\target{How it feels to learn Iquin is evil}
When someone discovers the truth about how \iquin{} is evil, describe the disappointment and disillusioning they feel to discover that they have been betrayed and misled.
That the religion which they followed and obeyed all their lives, that which they thought to be the source of all good, turns out to be evil; a parasitic bloodthirsty abomination, a \trope{CosmicHorror}{Cosmic Horror}. 

\citebandsong{BeyondTwilight:TheDevilsHallofFame}{Beyond Twilight}{%
  Perfect Dark%
}{
  Follow the philosopher\\
  Is he God or Lucifer?\\
  It's written in the sky\\
  A prophecy for you and I 
  
  Electrical insects\\
  With nowhere to go 
  
  Asking the question why\\
  Religion might just be a lie\\
  Believer, take your fix\\
  Crawl up on your crucifix 
  
  I've found the answer\\
  I have the key\\
  Experience reality\\
  And the world will see \\
  That it's perfect dark\\
  It's perfect dark here\\
  In the Devil's hall of fame 
  
  Before me now I see a dying planet\\
  Captured souls with microchips implanted\\
  And it's perfect dark\\
  In the Devil's hall of fame \\
  There must be a way\\
  I'm waiting, waiting for the day\\
  Hellfire. Hellfire. \\
  Higher. Higher. \\
  Only blood remains\\
  There is nothing more to gain
}








\subsection{Parasitism}
\target{Parasitic Archons}
The \Archons{} are parasites. 
The \sephiroth{} \hr{Life drain}{drain life force} from worshippers, collecting it as a kind of \quo{tax}. 
Some of that force they allow Vaimon mages to channel. 
So Vaimon magic is powered by other people's stolen life force. 

Similarly, the reason why Vaimons \hr{Vaimon healing}{have such powerful healing magic} to offer is that they unwittingly feast on stolen lifeforce. 

This has global effects. 
In times where Vaimons expend great amounts of magical energy, this toll must be collected from the populace. 
This results in epidemies. 
People will fall sick with leprosy, limbs rotting and falling off. 









\subsection{The \Iquin plan}
\target{Sephirah plan}
\target{Iquin plan}
The \sephiroth{} were envisioned after the \hr{Cabal}{Cabalists} were disappointed in \humans. 
The \human{} race \hr{Humans are a failed slave race}{was a failure to begin with}, and the Cabal's many attempts to refine and improve them had met with little success even after thousands of years. 

So they devised a new and better way to use the \pps{\humans} soul power. 
They designed the \sephiroth{} to \hr{Sephirah soul prison}{act as a \carcer} that would suck up \human{} souls, and also bind living \humans{} to it. 
This would enable the Cabal to use the collective willpower of all the \humans{} as a \vertex. 
Granted, each \human{} will is pathetic, but bundle enough of them together and you have a formidable \vertex. 

\Iquin{} is thus a giant machine using living and dead humanoids as its nuts, bolts and cogs. 
It gives the \human{} race a purpose, a use, a meaning to their otherwise pointless, worthless existence. 

\Iquin{} was designed with \humans{} in mind, but it turned out to be simple to generalize it to work on \scathae{} and other humanoids as well. 
This was a bonus effect. 





\subsubsection{The \Resphan{} Purpose}
Apart from the planned climactic finale with \hr{Lithrim}{\Lithrim}, the \iquin{} plan also tied into the \hr{Resphan purpose}{\resphan{} \quo{purpose}}. 
It helped their race grow stronger and stronger, towards perfection. 





\subsubsection{The \Morbus is the next step}
The \hr{Morbus}{\Morbus} is the next phase of the masterplan of which the \sephiroth{} are a part. 
In a sense, the \Morbus{} is the next \quo{generation}: 
A faster, even more efficient version of \iquin, a new and improved way to harvest \humans{}. 

It is still experimental, though. 





\subsubsection{Final purpose}
\target{Final purpose of the Sephiroth}
The {\Sephiroth} have a grand purpose. Like every other aspect of the Cabal's moves orchestrated by \hr{Daggerrain}{\Daggerrain}, the \sephirah{} project has all been in preparation for the final climax and the opening of the gate to \hr{Erebos}{\Erebos}. 

\Daggerrain{} is planning a great final ritual in which all of the combined Shroud-weaving energy contained within the \Sephiroth{} will be unleashed in a cosmic explosion of \vertex{} power that will tear the Shroud asunder and fling wide the dimensional gateway to \Erebos.

I need to build up to this throughout the entire series. Cabalists drop hints here and there, while the Sentinels suspect and attempt to piece together the puzzle.

Compare this legion of bound dead to Hood's army in \cite{StevenErikson:TolltheHounds}. 

\lyricsbs{Emperor}{Night of the Graveless Souls}{
  When night comes creeping in, \\
  dark restless shades arise. \\
  Graveless souls are gathering. \\
  Seem to ignite the flame. 
  
  Creeping shadows roam the night, \\
  as a merciless rain is closing upon.\\
  Demons coming to sever the Sun. \\
  Empty eyes come to life. 
  
  Darkness reigns in the Cosmos. \\
  Forces of the Devil's lair. \\
  Innocent or brave, \\
  they appear to rape your world. 
  
  Children of the evil and demonic\\
  curse his love again. \\
  Graveless souls awake, \\
  seem to ignite the flame. 
  
  When night comes creeping in, \\
  dark restless shades arise. \\
  Prepare to crawl and run. \\
  The Black is here tonight.\\
}






\subsubsection{\Carcer}
\target{Sephirah soul prison}
\index{\carcer!\iquin}
%The \sephiroth{} are a soul prison. 
\Iquin{} is a \carcer{}. 

\lyricslimbonicart{Grace By Torments}{
  Underneath the ice of winter\\
  there is a stream so powerful.\\
  A dark river that will swallow your soul.
  
  As it twists into form,\\
  you dwell in a cold dark void.\\
  Frozen and lifeless dreaming,\\
  deep down in the darkness screaming.
  
  When pain comes to power,\\
  all the sorrows of the heart\\
  leading your soul astray\\
  to a realm of agony.
  
  All you have left is death's desire for you.
  
  A black hole.\\
  Well of souls.\\
  Icy breath.\\
  Stone cold death.\\
  Darkness calls\\
  as Heaven falls to Earth.
}

\lyricslimbonicart{Seven Doors of Death}{
  Asylum of neurotic minds.\\
  Devour all consciousness.\\
  Insanity is there to find.\\
  Dominion of darkness.
}

\lyricsdimmuborgir{Mourning Palace}{
  Daylight has finally reached its end,\\
  as evenfall strikes into the sky.\\
  Far away in the dark glimpsing moonlight,\\
  sickening souls cry out in pain.
  
  Hear the cries from the Mourning Palace.\\
  Feel the gloom of restless spirits.\\
  Hear the screams from the Mouring Palace.\\
  Feel the doom of haunting chants.
  
  Eternal is their lives in misery.\\
  Eternal is their lives in grief.\\
  Abandoned in a void of nothingness.\\
  A chain of anger, a fetter of despair.
  
  In this garden of depraved beings,\\
  this unsacred place of helpless ones,\\
  Satan blessed the creatures,\\
  enswathed them in endless night.
}

\Iquin{} turns \humans{} into hapless slaves from birth. 
They have no control over their lives and exist only to be used. 

\lyricsbs{Hate Eternal}{Spiritual Holocaust}{
  Born aware, yet destined not to be.\\
  To an entity of pain you serve.\\
  Take from the light, nourishment through spite.\\
  Depleting human decree. \\
  Chronic infection through a sickening will, \\
  the many that stain this Earth.\\
  It doesn't matter who must suffer\\
  when the sickness claims rebirth. 
}

\citebandsong{DeathspellOmega:FasIteMaledictiinIgnemAeternum}{%
  Deathspell Omega
}{
  The Shrine of Mad Laughter
}{
  A sensation of everlasting rot and those frantic wails.\\
  No, it is not a fall into the abyss.\\
  The defiance of descent, a coronation beyond liberty and slavery.\\
  The cry of woe and deliverance exudes a flame, \\
  evasive as sound and ether.\\
  An instant of collusion with death, without hope nor prospect.
}









\subsection{\Sephiroth}
\target{Sephirah}
\target{Sephiroth}
The \Sephiroth{} are the \Archons{} of Iquin.
 They can be invoked to cast magic. 
There are sixteen of them. 
The \Sephiroth{} are associated with the four classical elements; four for each element. 
Each \Sephirah{} also personifies a virtue that is considered sacred to the \hs{Iquinian Church}. 

What is not taught in Iquinian theology is that these virtues are actually meant not to benefit the people but to control them. 
They are tools designed by the \banes{} to enslave the \humans{} and use them in their war against the \dragons. 

The \Sephiroth{} are not \vertices{} individually, because they are half-mindless slaves, lacking the force of personality to become true \vertices. 
Rather, \iquin{} as a whole is one immense \vertex{}. 

Each \Sephirah{} is associated with a \quo{virtue}. 
These virtures are actually evil and used to control men. 

\lyricsbs{Exmortem}{Fix of Negativity}{
  The world is swept in a curtain \\
  from the rottenness of Man. \\
  Fixing decadent values. \\
  Miserable mortals, open your eyes. 
}

The \Sephiroth{} are divided into four \quo{elements}. These are four integral components of the mind, and, in a sense, of the universe as a whole. 
The elements are things that can be hyped up in propaganda as being virtues, the source of all good, but can also easily be subverted and twisted into a source of evil. 

The elements are: 

\begin{description}
  \item[Passion:] 
    The deep, heartfelt belief in a cause and the burning desire to fight for it. 
  \item[Eye/Vision:] 
    The Eye lets you perceive and interpret the world: In a \quo{true}/\quo{good} way, or as a veil of lies, seeing only what you want to see (or what your masters want you to see).
  \item[Voice:] 
    The Voice that lets you communicate with your fellow beings, to offer wisdom and comfort, or to condemn and spread hate and lies. 
  \item[Tears:] 
    Tears of joy, or tears that release sorrow and help you deal with it. Or tears that confirm your sorrow and only serve to pull you deeper into a mire of suffering. 
\end{description}

Another schema might be that of four basic emotions:

\begin{description}
  \item{Anger:} Defense, expansion.
  \item{Fear/pain:} Self-preservation.
  \item{Hunger:} Self-improvement.
  \item{Lust:} Procreation. This one can be sublimated into an urge to create other things than progeny, such as art, or an empire.
\end{description}








% Each Sephirah has: A name, a classical element, a power, a virtue and an explanation. 
\newcommand{\sephitem}[5]{#1 (#2) & #3 & #4 & #5 \\\hline}

\newenvironment{sephirahlist}[1]{%
  \begin{figure}
  \caption{\Sephiroth{} of #1}
  \begin{tabular}{|c|p{4cm}|c|p{6cm}|}
  \hline
  \textbf{Name} & \textbf{Power} & \textbf{Virtue} & \textbf{Description} 
  \\
  \hline
  \hline
}{%
  \end{tabular}
  \end{figure}
}





\begin{comment}
\subsubsection{Eye}
\end{comment}
\begin{sephirahlist}{Eye}
\sephitem
  {\Cushed}
  {\male}
  {Used to shape, move and Sculpt earthen objects.}
  {Lawfulness}
  {Helps the rulers keep people in check. }
\sephitem
  {\Omariel}
  {\female}
  {Cures pain (but doesn't heal wounds).}
  {Acceptance}
  {Makes people accept hardship and oppression.}
\sephitem
  {\Yemared}
  {\female}
  {Cause paralysis.}
  {Tradition}
  {Prevent rebellion and keeps the social order stable and static.}
\sephitem
  {\Yeziel}
  {\male}
  {The Coronet: Protects against magical attacks.}
  {Chastity/Purity}
  {Sex is a dangerous thing, because it may open people's eyes to the \hs{Beyond}. Also, it makes them harder to control. Also, \human{} sexuality is something the \banes{} very much want to harness and control, so they need \Yeziel{} to keep people's sexuality in check. }
\end{sephirahlist}




\begin{comment}
\subsubsection{Passion}
\end{comment}
\begin{sephirahlist}{Passion}
\sephitem
  {\Barion}
  {\male}
  {The Shield: Protects against melee attacks.}
  {Courage}
  {Encourages the people to fight against the enemies of the Church.}
\sephitem
  {\Hoshied}
  {\male}
  {The Battlement: Protects against targeted ranged attacks.}
  {Loyalty}
  {Keeps people under control and discourages them from asking questions.}
\sephitem
  {\Razilah}
  {\male}
  {Create lightning.}
  {Righteousness}
  {Opposes any kind of heresy, blasphemy and unorthodoxy; crusades against the heathens.}
\sephitem
  {\Teshiron}
  {\female}
  {The Rampart: Protects against area-of-effect attacks.}
  {Faith}
  {Keeps people loyal to the Church and not asking unwanted questions.}
\end{sephirahlist}





\begin{comment}
\subsubsection{Tear}
\end{comment}
\begin{sephirahlist}{Tear}
\sephitem
  {\Feazirah}
  {\female}
  {The gentle wind.}
  {Humility}
  {Keeps people pacified and keep them from complaining.}
\sephitem
  {\Gamishiel}
  {\female}
  {}
  {Sacrifice}
  {Make people work hard for their masters and not expect any rewards. (Her month is only 20 days long.)}
\sephitem
  {\Hapheron}
  {\male}
  {}
  {Solidarity}
  {Hate of outsiders. Turns the Iquinians, and all \humans, into a united front against their enemies.}
\sephitem
  {\Ishiel}
  {\female}
  {Healing}
  {Patience}
  {Makes people accept hardship and oppression.}
\end{sephirahlist}





\begin{comment}
\subsubsection{Voice}
\end{comment}
\begin{sephirahlist}{Voice}
\sephitem
  {\Atzirah}
  {\male}
  {The carrying wind. Used for lifting objects or flying.}
  {\Honour}
  {Keeps people inside the system for fear of dishonour while encouraging them to strive to please their masters and the Church.}
\sephitem
  {\Izion}
  {\male}
  {Creates blasts of fire. Perhaps the \Sephirah{} most commonly used in combat. }
  {Justice}
  {Destroys the wicked: Sinners, and the enemies of the church.}
\sephitem
  {\Keshirah}
  {\female}
  {The powerful wind. Used to create controlled gusts of wind. }
  {Dilligence}
  {Makes people work hard for their masters.}
\sephitem
  {\Thimared}
  {\female}
  {Limited mind control.}
  {Obedience}
  {Turn people into humble slaves. Keeps the masses from rebelling and rulers from sympathizing.}
\end{sephirahlist}







\subsection{The \Morbus: The next phase}
\target{Morbus}
\target{Disease}
A long-running theme: 
There is a ghastly, leprosy-like disease spreading like an epidemic through slums all over \Azmith, transforming people into the living dead. Or maybe into \hr{Lictors}{Lictor-like people}. 

The \Morbus{} strikes the hopeless, the disenfranchised, the weak-willed. It is a manifestation of the newest stage/phase of the \hr{Sephirah plan}{\sephirah{} plan}, and of the \hr{Return of the Banelords}{return of the \banelords}. 

In some parts of \Azmith, the \Morbus was simply called \quo{the Disease}. 
Rian \hr{Rian sees the Morbus}{sees it in \Malcur}, and Carzain \hr{Carzain sees the Morbus in Forklin}{sees it in \Forklin} and \hr{Carzain sees the Morbus in Redce}{later in \Redce}. 

Compare to Calcutta in \authorbook{\PZBrite}{Calcutta, Lord of Nerves}. And also \authorseries{Paolo Eleuteri Serpieri}{Morbus Gravis}. 

The \Morbus{} is strongest in the city slums, because these provide an easier link to the cities of \Nyx, and through them, \Erebos. Compare to the RPG \emph{Kult}, where slums are gateways to Metropolis, the eternal city\dash a morbid, decaying ruin. 

The \hs{Cult of the Worm} seek to utilize the \Morbus{} and twist it for their own purposes. 

\lyricsdimmuborgir{Blessings upon the Throne of Tyranny}{
  Infected by invalid behaviour,\\
  while capturing the stench of divine putrefaction.\\
  Confess to slavery for the world saviour.\\
  Give praise and inhale the corruption.
  
  The Enfeebled provides the fool.\\
  The Disabled provides the tool.\\
  The Apathetic demands the affection\\
  to those suffering from their own satisfaction.
}

\lyricsdimmuborgir{Architecture of a Genocidal Nature}{
  Devouring their flesh with a razorblade smile.\\
  Genes would still blindly carry on smouldering ember of Hell.\\
  Limned with gold leaf, the scarlet brush\\
  that sweeps all traces of time, place and pattern.
}

\lyricsbs{Exmortem}{Fix of Negativity}{
  We live in the noise and the dirt, the real life angst, \\
  filth and desperation, a portal of misery. \\
  The ultimate demonic playground.
  
  The world is swept in a curtain \\
  from the rottenness of Man. \\
  Fixing decadent values. \\
  Miserable mortals, open your eyes. 
}

The \Morbus{} feeds on \human{} emotions (\hs{Chaos} lifeforce, from the Heart of \Miith{}). It draws strength from \human{} evil and suffering. 

\lyricsdimmuborgir{Puritania}{
  I am war, I am pain.\\
  I am all you've ever slain.\\
  I am tears in your eyes.\\
  I am grief, I am lies.
}

\lyricsdimmuborgir{Eradication Instincts Defined}{
  Human depravity is at our disposal \\
  as the perfect tool to destroy mankind, \\
  the worst kind of them all. \\
  Modern times' ignorance, the world disease.\\
  Appeal to death of every man.
  
  A living hate smoldering abyss,\\
  nurtured through centuries with quietly exercised wrath, \\
  seeks the easiest way to the feed the engine, \\
  praising the final bloodbath.\\
}

\lyricsbs{Exmortem}{Human Rape Symphony}{
  I'm free floating cyanide, \\
  infesting the corpus collectivus. \\
  Admiring the obscure human rape symphony. 
  
  Ripened by their own corruption. \\
  Drown in your own shitty lives. \\
  Promoter of decadence. \\
  Distributor of pestilence.
  
  The pathetic stupidity of mankind. \\
  Humanity crawling on its knees. \\
  Corrupt and parasitic elements. \\
  All idylls must end!
}

\lyricsbs{Monolith Deathcult}{
    1917 - Spring Offensive (Dulce Et Decorum Est)}{
  The mindrape of the spectacle of dehumanization.\\
  Viral god wields his festering scythe of gangrene.\\
  By endless flashes of mortar or the cadaveric stench.\\
  Ghostly faces peer through the veil of celluloid.
}





\subsubsection{The plan}
The disease is called the \Morbus. 
It is a plan envisioned by \Daggerrain{} as a means to harvest \human{} souls and lifeforce and harness them as a \vertex. 
It is a further development of the \sephiroth: 
A faster, meaner, uglier version of \iquin. 

The \Morbus{} is meant to practically exterminate the \human{} race, engulf them, suck them dry and leave them \hr{Life drain}{drained} and empty of all their life and \vertex{} willpower. 
Then use them as a big \vertex{} under \ps{\Daggerrain}{} control. 

\lyricsdimmuborgir{Puritania}{
  Let chaos entwine on defenseless soil.\\
  Remove errors of man and sweep all the weakening kind.
  
  Bygone are tolerance and presence of grace.\\
  Scavengers are set out to cleanse the human filth parade.
}

The \banelords{} realize that having the \humans{} live and then die (and be absorbed into \iquin) is a waste. 
They might as well kill them, or turn them into semi-undead. 
Enter the \Morbus. 

\lyricsbs{Hate Eternal}{Spiritual Holocaust}{
  Upheaving desire, heinous in its wake.\\
  Abandonment of life, this baneful mistake. \\
  Sickness now unfolds in you. To death you beseech.
}

The \Morbus{} is a prelude to a planned global epidemy of madness and mind-wrought destruction that is intended to sunder the Shroud and the barriers between \Miith{} and \Erebos. 

\lyricsbs{Vital Remains}{Born to Rape the World}{
  Untouched by grace. Fallen are the messengers. \\
  Hymns of the profane unheard by the pious. \\
  Revere the inverted hypocrite\\
  as we breed to the sound of the hammering nail.
  
  Befoul the teachings of divinity. 
  
  Words shall unveil the silent truth,\\
  like razors through the flesh of the divine,\\
  severing the artery from God.\\
  Abandoned and devoured by the larvae of those who abhor. \\
  Distant lull deep in the subconscious.\\
  Unstoppable urge, uncontrolled, now coming to the surface.\\
  Their eyes are blind, but the dead shall always see\\
  the coming entity.
  
  All will witness as the undeniable truth unfolds. \\
  For they now know the truth that Heaven is no more. 
}

They mean to bring madness, suffering and ecstasy and force people's minds out of their comfort zones. 
Over the course of \booktitle{\SentinelsofMith}. 







\subsubsection{The afflicted}
When you look at the leper-like people afflicted by the \Morbus, they radiate all the world's emptiness, cruelty, lifelessness and neverending cold darkness. 

When you look into their eyes, the darkness and emptiness goes on forever with no bottom. Like abysses\dash portals to the true Abyss that is the endless universe. 





\subsubsection{Superstition}
Some of the afflicted turn to superstition. 
They believe that the \Morbus{} is a divine punishment, and that they must pray and believe extra hard in order to atone. 
Of course, this only makes matters worse (or better, from the Cabalists' POV). 

\lyricsbs{Hate Eternal}{Whom Gods May Destroy}{
  Afflicted by this scourge, \\
  the great plague that shall befall upon us,\\
  we are of disease, of destiny.\\
  We must become one with our gods, \\
  whose massive fury shall scorch this Earth,\\
  for we are all destined to oblivion.
}







\subsection{\Lithrim: The purpose of Man}
\target{Purpose of Humanity}
\target{Lithrim}
\index{\CrystalSphere}%
\index{\Lithrim}%
\Iquin{} and the \Morbus{} were stages in the plan, but what it all revolved around was \emph{\humanity{} itself}. 
The \human{} race was created not just to be slaves and breeders. 
They were made to form one huge species-wide \matrix{} that, when the time was right, would break open the Shroud and shatter the \CrystalSphere. 

\Humanity{} had a hidden \matrix{} that connected and empowered them all. 
This \matrix{}, feeding off the Heart of \Miith{} and drawing some power from \Erebos, gave the \humans{} their lifeforce, virility, fertility and zest for life. 
It also gave them their antisocial, destructive and self-destructive petty impulses (which \scathae{} did not have to quite the same degree).

\Humans{} were separated in individual bodies, but they had the potential to become one big god, \Lithrim: 

\citebandsong{DeathspellOmega:Kenose}{%
  Deathspell Omega
}{
  \Kenose
}{
  Everything, except God, \\
  has in itself some measure of privation.\\
  Thus all individuals may be graded \\
  according to the degree to which they are infected \\
  with mere potentiality.
}

\Lithrim was mostly an insubstantial, spiritual being; a \matrix. 
But it also had a physical manifestation (or more than one) as a gigantic, pale, fleshy thing. 
Amorphous, yet still somehow loathsomely \human. 
Perhaps with bits of \human faces and limbs sticking out, and huge \human or quasi-\human looking faces forming at random on its surface. 





\subsubsection{First Advent of \Lithrim}
\Lithrim was a \bane god of destruction. 
It appeared for the first time during the \hr{Cuezcan Apocalypse}{\CuezcanApocalypse}. 
This time \Lithrim was defeated, and the \dragons believed it was gone for good. 
But it was not. 

Read about this in \hr{First Advent of Lithrim}{the relevant section}. 





\subsubsection{Clean \humans}
\target{Clean Humans}
The essence of \Lithrim was carried inside \humans.
It began when the \hs{Men of Light} were \hr{Men of Light created}{created}. 
They then spread their genes all over the place. 
Soon almost all \humans carried the gene that connected them to the \Lithrim \matrix.

Not all, though. 
There were some \humans who were \quo{\hr{Clean Humans}{clean}}, in the sense that they had no blood of the {\hs{Men of Light}} and thus did not carry \Lithrim inside them. 





\subsubsection{Second Advent of \Lithrim}
In the end, in the \thirdbanewar, all \humans{} were forcibly merged into one huge mindless superbeing:
\Lithrim, the manifestation of all \humanity{} combined. 
 
Then the plan failed and \humanity{} died out. 
At the end of the \thirdbanewar, there might have been a few straggler \humans{} still alive, but the \matrix{} that empowered their race was destroyed, so the survivors would most likely waste away\ldots{} or devolve into inhuman vermin. 

Compare to the anime \cite{Anime:NeonGenesisEvangelion}, where Ikari \Gendou's \quo{Human Instrumentality Project} aims to dissolve the AT Fields that keep \human{} beings apart as separate beings with individual bodies and minds, so as to merge all of \humanity{} into one Angel-like creature. 

\citebandsong{DeathspellOmega:SiMonumentumRequiresCircumspice}{%
  Deathspell Omega
}{
  \Hetoimasia
}{
  For man is the key and man is the device\\
  And out of his ranks shall arise the saviour \\
  draped in the blood of the unborn
  
  But mankind was the prism to the quintessence of corruption\\
  Contemplate and say, what is earth, \\
  else than a frenetic psalmody for His venue?\\
  What joy and glory shouts he who bears the mark of the Beast\\
  Consumed and eaten have been the abundant abortions of mankind, \\
  but now, none of them, humans, shall remain \\
  but what birds could not carry off in their claws!
}





\subsubsection{How it feels}
This strange nature as pieces of a divided god was one of the reasons why \humans{} often felt emo and out of place. 

\citebandsong{DeathspellOmega:Kenose}{%
  Deathspell Omega
}{
  \Kenose
}{
  Man, lost somewhere between the restrictive force\\
  Of Cain and the expansive force of Abel,\\
  Falls from his median position between Angel and Beast\\
  Each time he ceases to desire a being superior to himself\\
  Adam's descent into materiality,\\
  May it be questioned\ldots{}\\
  The separating line between the Saved and the Damned,\\
  May it be questioned\ldots{}
}

Instinctively, deep down, \humans{} had a vague feeling of their true nature, bond and purpose. 
They feared it. 

\citebandsong{DeathspellOmega:CrushingtheHolyTrinity}{%
  Deathspell Omega
}{
  Diabolus Absconditus
}{
  If there is nothing that surpasses \\
  our powers and our understanding,\\
  If we do not acknowledge something greater than ourselves,\\
  Greater than we are despite ourselves,\\
  Something which at all costs must not be,\\
  Then we do not reach the insensate moment \\
  towards which we strive with all that is in our power \\
  and which at the same time\\
  We exert with all our power to stave off
}





\subsubsection{Mythology}
\Lithrim was the truth behind the \hr{Iquinian eschatology}{Iquinian beliefs about an Apocalypse}. 





\subsubsection{Secrecy}
\target{Lithrim was secret}
\Lithrim was a secret. 
Its true purpose was known only to the inner circles. 
The name was known to many Cabalists, but most thought \Lithrim was just a metaphor or abstraction. 
Most did not suspect so cataclysmic a result. 
















\section{\Itzach and the \Qliphoth}
\target{Itzach}
\target{Qliphoth}
\target{Qliphah}
\Itzach{} is a \hr{Dweomer}{\dweomer} created by the \hr{Banes}{\banes} and \hr{Resphan}{\resphain}. 
Is it purely a \hs{front-end} for \Erebos? 
\Itzach was raw \bane power, whereas \iquin was filtered \bane power. 

Or does it also draw power from the \hr{Heart}{Heart of \Miith}?

The force of \quo{Outer Darkness} in \hs{Vaimon} metaphysics. 
By the \hs{Iquinian Church} reviled as the source of all evil. 
Its manifestations are the \hr{Qliphah}{\Qliphoth}.

Originally, the \banes{} relied on magic that drew its power from \Erebos{}. But when \Miith{} and \Nyx{} was sealed off from \Erebos{} at the end of the First \Banewar, the \banes{} lost the source of the magical power and were left crippled, and the \dragons{} and their minions were easily able to slaughter the millions of \banes{} still on \Miith{}. Daggerrain and some other \banelords{} survived and were able to hide. Deprived of their magic, they had to develop a new magic theory. So they came up with \nieur{} theory, which is some principles from old \bane{} magic theory combined with some Chaos theory. So \nieur{} is a chaotic force, and the \Kliffoth{} are more or less the same as the \hr{Daemon}{\daemons} which Chaos magicians invoke. 

But the \banes{} wanted a better tool to control their \humans{}. The problem was that if you taught \nieur{} magic to \humans{}, it tended to reinforce their chaotic nature and make them difficult to control. But without magic at all, the \humans{} were wimps. 
And so they created \hr{Iquin}{\Iquin}. 








\subsection{List of \Qliphoth}
\target{Circle of Noon}
\target{Noon Circle}
\index{Noon Circle}
\index{Circle (in Vaimon mysticism)}
At least 100 \qliphoth{} are known and named. 
They are categorized into a number of \quo{Circles}. 
(Following this system, the \hr{Sephiroth}{\Sephiroth} are sometimes considered the Circle of Noon.) 

The darkest circles are feared as bringers of madness.

\lyricsbs{Aleister Crowley}{%
  Liber Aleph vel XCI 
  (Chapter 92: 
   De Aquila Sumienda/On Consuming the Eagle)
}{
  \ldots{} then is thy Danger fearful and imminent, for it is the Edge of the Abyss of Choronzon, where are the lonley Towers of the Black Brothers.
}





\begin{subgloss}
  \begin{comment}
  \paragraph{Circle of Dusk}
  \end{comment}
  \gitem{Circle of Dusk}
  \target{Circle of Dusk}
  \target{Dusk Circle}
  \index{Dusk Circle}
  The most harmless ones. 
  
  \begin{sephlist}
    \seph{\Gavron}
    \target{Gavron}
    \index{\Gavron}
    Can cut through small things. 
    Some people use him as a shaving razor. 
    
    \seph{\KorRashad}
    \target{Kor-Rashad}
    \index{\KorRashad}
    The guide through the \empyrean{}. 
    \KorRashad{} represents, and is an extension of, the individual's \emph{will}. 
    He is depicted as a \mulgron: 
    Massive, immovable and unstoppable. 
    
    He is inspired by Ra-Hoor-Khuit, an entity that appears in Aleister Crowley's Thelema. 
    
    \seph{Thaid}
    \target{Thaid}
    \index{Thaid}
    A guardian \qliphah. 
    He is represented as a snake: 
    Part phallus, part residue of ancient \ophidian{} mythology. 
    
    \seph{Thuin}
    \target{Thuin}
    \index{Thuin}
    A guardian \qliphah. 
    She is represented as an arch: 
    Part vagina, part heaven, part doorway. 
   
  \end{sephlist}




  \begin{comment}
  \paragraph{Circle of Twilight}
  \end{comment}
  \gitem{Circle of Twilight}
  \target{Circle of Twilight}
  \target{Twilight Circle}
  \index{Twilight Circle}
  The more dangerous ones. 
  This was the largest circle. 
  
  \begin{sephlist}
    \seph{\Djerzad}
    \target{Djerzad}
    \index{\Djerzad}
    A \Qliphah{} that causes bones to break. 
    
    \seph{\Iphicoss}
    \target{Iphicoss}
    \index{\Iphicoss}
    The \Kliffah{} of the Treacherous Gale, creates strong gusts of wind. 
    
    \seph{\Kithvaz}
    \target{Kithvaz}
    \index{\Kithvaz}
    A \qliphah{} that attacks the mind. 
    
    \seph{\Shurreem}
    \target{Shurreem}
    \index{\Shurreem}
    A \qliphah that can surround the Vaimon in an aura of transparent black \quo{fire}. 
    The black flames were painful to touch unless you were accustomed to them. 
    They felt a bit like regular fire, but more mental, as if the fire was inside you and burning away at your very mind. 
    But they caused no real harm apart from the psychological shock effect. 
    
    \Shurreem was mostly useful for intimidation purposes. 
    It made the Vaimon look wicked and sorcerous. 
    
    The \qliphah was not very widely known. 
    This made it more effective as a bluff. 
  \end{sephlist}




  \begin{comment}
  \paragraph{Circle of Midnight}
  \end{comment}
  \gitem{Circle of Midnight}
  \target{Circle of Midnight}
  \target{Midnight Circle}
  \index{Midnight Circle}
  The most powerful and dangerous of all. 
  
  Includes the following: 
  
  \begin{sephlist}
    \seph{\Bozchul}
    \index{\Bozchul}
    \target{Bozchul}
    A \qliphah{} who brings death. 
    
    \seph{\Horvaleth}
    \index{\Horvaleth}
    \target{Horvaleth}
    The \Kliffah{} of the Cruel Winter. 
    Can create blasts of cold air and shards of ice. 
    
    \seph{\Nyxachel}
    \index{\Nyxachel}
    \target{Nyxachel}
    A \Kliffah{} of lightning. 
  \end{sephlist}
\end{subgloss}









\subsection{Realm}
According to Iquinian mythology, \Itzach{} is also the Realm of the Damned, where the souls of the wicked go after death. 

The Realm of \Itzach{} is a labyrinthine dungeon of endless dark corridors, where the weeping and morbid cries of the damned echo and resound forever. 

Compare to Iril, the Maze-Hell of \authorseries{Alan Campbell}{Deepgate Codex}. 








\subsection{Origin of \Itzach}
\target{Origin of Qliphoth}
Originally, when \hr{Origin of Nyx}{\Nyx{} was created},
the \banes{} created \nieur{} as an imitation of their native \erebean{} power. \Nieur{} is based in \Nyx{} and draws its power from that dimension. It doesn't draw power from \Erebos, it just utilizes the same technology. 

But \nieur{} isn't pure \Erebean{} style magic. 
It also incorporates some \hr{Chaos}{\Chaos}\ldots{}

The \qliphoth were created to serve who conduits to \Erebos{} through which magical energy could be channelled. 









\subsection{Religious beliefs}
For Iquinian religious/mystical beliefs about the \qliphoth, see the section on the \hs{Iquinian Church}. 













\section{Maintaining the Mask of Civilization}
\target{Maintaining the Mask of Civilization}
\target{Maintaining the Shroud of Civilization}
\target{Mask of Civilization}
\target{Shroud of Civilization}
%This section shouldn't be called \quo{maintaining the \quo{Shroud} of civilization}. I need to think of a better word to describe a sub-pattern of the Shroud created by local people's thoughts and beliefs. 

%Maybe \quo{frame}, as in psychology and The Game theory? 

A people's collective belief in the king's power and splendour create a a \hs{Mask}, a mental frame, a collective \vertex, which maintains the civilization's Shroud. 
This frame strengthens the citizens and allows them to work together more effeciently, and it also keeps away the \Wylde{}. 
The people's strong belief that their settlement is separate from the \Wylde{} actually pushes back the \Wylde{}.

In a sense, the Shroud is like money: 
A collective illusion. 









\subsection{Monuments}
\target{Monuments}
All successful civilizations build great monuments: Huge castles, temples, statues, pillars, obelisks and ornaments. These are particularly effective because of the Shroud: They send a message of power, glory and greatness. They create a frame which strengthens the subjects (mentally, in their resolve, and thus also physically) and intimidates enemies. 

%Furthermore, they keep the \Wylde{} at bay. The people's collective belief in the king's power and splendour strengthen the frame of civilization, and thus the civilization's Shroud. Almost like a \vertex. A collective \vertex. This should have a name! Anyway, this power keeps away the \Wylde{}.
This also serves to keep away the \Wylde{}.

The \hs{Cabal} and \hs{Sentinels} both use these monuments to keep people locked in the Shroud.

Monuments and idols that are cherished and worshipped will gradually acquire the power of \nexi. They are invested with mental power siphoned from the adoration and faith of their worshippers.

In order to overthrow/usurp a ruler or dynasty, a powerful weapon is iconoclasm, the destruction or desecration of monuments. Alternatively, you can steal the monuments and usurp them for yourself. 

If you destroy or desecrate the monuments, it is important to build/consecrate new ones of your own. Otherwise, you invite \hs{Chaos} and the \Wylde{}. People will degenerate into anarchy, order and civilization will crumble, and eventually barbarism will take over, and the city will sink into the \Wylde{}.

I need to remember to have these monuments in all cities and towns (some grand, some more lowly).

When people worship, admire or love something\dash an icon, a place or a living being\dash they invest not only mental energy but also a small part of their soul in the thing. This makes them yearn back to the thing, and they will keep coming back. An addiction is born.





\subsubsection{Monuments keeping the dead around}
\target{Monuments keeping the dead imprisoned}
By building a monument to the dead and venerate them, it is sometimes possible to keep the dead soul around as an incorporeal ghost, retaining some measure of consciousness and communication ability. 

\index{\carcer!monument}
Sometimes the dead soul wants this. Other times it is imprisoned against its will, often by hapless relatives who think they are doing the right thing by honouring their dead.
The monument thus becomes a small \carcer.

Sometimes the dead soul can take possession of its mummy or some kind of effigy, thus becoming \hr{Undead}{undead}.

\hr{Silqua as a ghost}{Silqua gets stuck after her death} as a powerless ghost. She suffers.







\subsection{Roads through the \Wylde}
\target{Roads through the Wild}
Roads, paths and passes through the \Wylde{} are maintained through the Shroud. If they fall into disuse, they become longer, narrower, more crooked, more sinister, as the chaotic forces of the \Wylde{} work to reclaim the territory. 

If roads are regularly used, protected and maintained, they will become nicer, shorter and more direct. In the best of cases, you can stand in one city, look out at the road leading to a neighbouring city beyond the \Wylde{}, and actually \emph{see} the other city in the distance, at the end of a broad, straight road. The legends say that all cities were like this during the time of the \hr{Vaimon Caliphate}{\VaimonCaliphate} (not true), and it was still like this in a few cases in \hr{Great Velcad}{\theBelkadianEmpire}. 

On a well-kept road, the \Wylde{} is less terrifying. It is still \Wylde{}, but it's farther away, and try as it might to extend into the road, it has much less power to do so. The mind will pick up on this, and so the \Wylde{} will seem less threatening. Also, the order and niceness of the road is contagious, so the closest edge of the \Wylde{} may grow slightly more \quo{tame}\dash brighter, cleaner and with no \wildfog{}, no peering eyes and no scary noises. 





\subsubsection{Road-keepers}
\target{road-keepers}
Every self-respecting kingdom had road-keepers hired to maintain every road. 
They were soldiers or tough woodsmen or \rangers.
Their duties includes:
\begin{itemize}
  \item 
    Patrolling the roads and making sure the \eidola were in place and that the \wylde was not creeping in.
  \item 
    Repairing whatever problems they could.
  \item 
    Alerting the nobles and/or the church if there were any problems they could not solve. 
\end{itemize}

Road-keepers often acted as guides, scouts and messengers as well. 










\subsection{\Wylde{} border}
\target{Wylde border}
\target{Eidolon}
It was well known that monsters and wickedness lurked in the \wylde, so people would erect wards and totem poles and statues and stuff at the border between \wylde{} and settlement, in order to ward off evil and keep the border well-defined. 

To keep the \wylde at bay, people used \eidola. 
And \eidolon was a magical totem blessed and enchanted to protect civilization and keep away the \wylde and its denizens. 
Every religion had its way of creating \eidola. 
Have them in every \wylde scene. 

\citeauthorbook[p.84]{RobertEHoward:KullUntitledDraft}{Robert E. Howard}{%
  Untitled Draft%
}{
  \ta{Nay. 
    Grondar ends here.
    Here is the end of the world; beyond is magic and the unknown.
    Here is the boundary of the world; there begins the realm of horror and mysticism.}
  
  \ldots 
  
  \ta{Remember, they who ride beyond the sun-rise, return not!
    For of all the thousands who have crosses the Stagus, not one was returned.
    Three hundred years have passed since first I saw the light, king of Valusia.
    I ferried the army of King Gaar the Conqueror when he rode into World's End wit hall his mighty hosts.
    Seven days they were passing over yet no man of them came back.
    Aye, the sound of battle, the clash of swords clanged out over the waste lands for a long while from sun to sun, but when the moon shone all was silence.
    Mark this, Kull, no man has ever returned from beyond the Stagus.
    Nameless horrors lurk in yonder lands and terrible are the ghastly shapes of doom I glimpse beyond the river in the vagueness of dusk and the grey of early dawn.
    Mark ye, Kull.}
}















\section{Realms}
\target{Realm}
\target{Realms}
\index{Realm}
Remember to have a lot of stuff about the Web of the Realms. 
All the various Realms, their true names, their history, the barriers and pathways between them. 









\subsection{The Beyond}
\target{Beyond}
\index{Beyond}
There really exist more than three spatial dimensions. 
The different Realms exist beside each other and (potentially) intersecting each other. 
But regular creatures can't see all the dimensions at once. 
Neither mortals nor immortals can, but immortals find it easier to learn to compensate. 
With skill and magic, it is possible to \quo{turn your senses around} and so see some other dimensions than usual. 
The Realms Beyond are accessed through those other dimensions that people cannot normally see. 









\subsection{Bodies and souls}
See the section on \hr{extended soul}{extended souls}. 









\subsection{Realms within \Miith}
Many Realms are contained within \Miith{}. 
\hr{Azmith}{\Azmith} is but one of them. 
\Nyx{} and \Machai{} are other such mini-Realms, although further removed from the Heart. 

%\subsection{Large world}
The point of this is that world is large. 
There are other battlefields in the \hr{Feud}{\feud}{} than \Azmith, which despite the name is \emph{not} \quo{all of \Miith}. 

Have references to this when some immortals are absent from the action. 









\subsection{Storm beacons}
\target{Storm beacon}
\target{storm beacon}
\index{storm beacon}
Storm beacons are a kind of magical constructs or spells. 
They are cast on an area and stay there. 
They create a \quo{storm} that ravages the Realms, making it dangerous and difficult to travel through the Beyond in that area. 

But the trick is that the Shroud blocks these magical storms, so the Shrouded world remains mostly unaffected. 
So people can move about in the Shrouded world like normal, but much magic (including \hs{submerging} and summoning) is suppressed. 









\subsection{Summary of the Realms}
\target{Immortal Realms}
\target{Shrouded Realms}
\target{Shrouded Realm}
\target{Chthonic Realms}
\target{Chthonic Realm}
\target{Aquatic Realms}
\target{Aquatic Realm}
\index{Immortal Realms}
\index{Shrouded Realms}
At the time of the \thirdbanewar, \Miith contained a lot of Realms. 
The \Miithian{} Realms were informally divided into two classes: 

\begin{description}
  \item[The Immortal Realms] 
    were the ones where the immortals preferred to live. 
    
    There were at least six Immortal Realms of \Miith. 
    They were: 
    \hr{Nyx}{\Nyx}, \hs{Visha}, \hr{Dun}{\Dun} and three parts of \hr{Machai}{\Machai}.
    
    It was well-known in mythology and legend that the planets and moons were worlds with their own inhabitants. 
    
  \item[The Shrouded Realms \textnormal{aka} Telluric Realms]
    are more deeply Shrouded, hence the name. 
    But the Immortal Realms are also Shrouded, just not so much. 
    \hr{Azmith}{\Azmith} is one such Realm, and \hr{Azmith is important}{the most important one of them}. 
    
    The Telluric Realms were all on dry land. 
    There are 8-10 Shrouded Realms in all. 
  
  \item[The Chthonic Realms] 
    were underground Realms. 
    This class included \hr{Kai-Leng}{\KaiLeng}. 
    There were 3-6 of them. 
    
    These Realms might have been Shrouded or Immortal. 
    Perhaps there were some of each. 
    Perhaps they were neither. 
    Maybe the Shroud did not have the same kind of hold underground. 
  
    The underground Realms were thinly populated, and most immortals attributed to them less importance than they should.
    Only the wise \dragons and the \banelords and a few \resphain understood the significance of the underground, Chthonic Realms.

  \item[The Aquatic Realms \textnormal{or} Pelagic Realms] 
    were \hr{Deep Realm}{Realms of the Deep} and of \hr{Sea}{the sea}. 
    There were 6-8 of them. 
    
    These Realms might have been Shrouded or Immortal. 
    Perhaps there were some of each. 
    Perhaps they were neither. 
    Maybe the Shroud did not have the same kind of hold under water.
\end{description}

The Shrouded Realms are much smaller and narrower than the Immortal Realms, because of the Shroud. 
When one moves \quo{sideways} through the Shroud in a Shrouded Realm, one \hr{submerging}{gradually submerges into an Immortal Realm}. 
These Immortal Realms are full of monsters. 











\subsection{\NaathKurRamalech: He who is the Key and the Gate}
\hr{Naath-Kur-Ramalech}{\NaathKurRamalech} is a \hr{XS}{\xs} who \emph{is} the dimensional barriers around \Miith{}. 
















\section{The Shroud}
\target{Shrouded}
\target{Shroud}
\index{Shroud}
\target{Lie Sublime}
\citetitle[Lords of Cthul]{Misc:Monsterpocalypse}{Monsterpocalypse}{
  Beyond the veil of our own universe exist myriad dimensions teeming with unknown threats. 
  For eons, practitioners of the occult have dared to peel back that fragile layer that separates our world from a vast realm of darkness to glimpse the ancient powers that lurk within. 
  From time to time, those horrors have slipped through the void.
}

A major theme is that of the Shroud and the Lie Sublime. 

The Shroud is a phenomenon that not only makes the Realms less penetrable and more difficult to traverse, but also a mental illness of a sort, a collective delusion that has crept into the minds of all creatures of the known Realms, hiding the true nature of the Realms. Within the Shroud, creatures can only see a narrow sliver of their own Realm; they cannot see nor move into the deeper layers of their Realm nor into other Realms, except in special places, following \emph{threads} in the Shroud, or at special occasions where there occurs a temporary \emph{tear} in the Shroud. (The latter is what happens to Catrian in \TwilightAngelRememberEmph in the chapter \quo{Silenced}.)
For that matter, even the immortals had to travel along certain safe pathways between the Realms, lest they fall prey to the \hr{Horrors of the Void}{horrors of the void}. 

Occasionally, we see evidence of the Beyond, but our brain explains it away and we make up lies and illusions to cover it up. 
Those few who do see into the Beyond and acknowledge what they find are branded as madmen and heretics. 
Young children have not yet fully developed such illusions, so they see the real world somewhat clearer (although they don't always understand what they see). That is one of the reasons children often claim to see monsters and faeries\dash they really do. 

But this is cultural. A culture who believe certain things about the Beyond will see things that way. They will see certain snippets of the Beyond that fit their beliefs. Some cultures deliberately encourage a knowingly false view of the world to further their own ends. Through magic and religious rituals, they glimpse the Beyond, but often they see only the small, controlled parts of the Cosmos that their shadowy overlords wish them to see. 

At any rate, sometimes the barriers between the worlds break down and people can see or move into another world. This might happen in situations of stress/shock or through the use of meditation, magic or mind-affecting drugs. The wisest and mighties of mages know of the planes and can see into them. 







\subsection{History}





\subsubsection{Before the Shrouding}
Before the Shrouding, everyone knew much more about the Realms. Even the unlearned understood more and could see more of the true reality of the universe. But the Shrouding sent ripples through the fabric of the entire universe, leaving the inhabitants scarred in body and soul. 

The weak minds lost their memory, or lost the ability to understand their memories of the richer world they had lost, and so felt the need to repress their memory and make up false \hs{myths} about the nature of the world. 

Only the strongest minds remembered and still understood the nature of the universe. Even the \vertices, such as \Ishna, were profoundly shocked by the traumatic vehemence of the Shrouding. 





\subsubsection{The Shrouding}
\index{Shrouding}
The \Shrouding{} actually depended on more than one previous Shrouding-like spell: 

\begin{enumerate}
  \item The \hr{Crystal Sphere}{\CrystalSphere}, forged by \Tiamat-tachi.
  \item The \hr{Shroud of Girigor}{Shroud of \Girigor}, woven by \Daggerrain. 
  \item The \hr{Shroud of Nyx}{Shroud of \Nyx}, also woven by \Daggerrain. 
\end{enumerate}

The \hr{Shrouding}{\SecondShrouding} itself marked the end of the \secondbanewar. 

Make it clear that the Shroud was a patch. 
It was a hasty emergency solution to patch up the cracks in the much more well-designed \CrystalSphere. 
Everyone knowledgeable, including \Ishnaruchaefir, knew that the Shroud was only a temporary measure and could never last. 
Sooner or later it would collapse.
Everyone had better be ready when that happened. 








\subsubsection{The Unspoken Covenant}
\target{Charade}
In addition to the metaphysical effect of the \SecondShrouding{}, there was also the code of the \charade. By unspoken mutual consent, the \dragons{} and \banes{} agreed that from now on, their great war would be fought not in public but in secret, hidden from the eyes of mortals. 

\target{Unspoken Covenant}
This agreement is also called the Unspoken Covenant. 

\target{Founding of the Sentinels}
\target{Founding of the Cabal}
The organizations of the Cabal and the Sentinels of \Miith{} were founded at this time.

With the \charade, the master races intend to rule by fear, guilt, manipulation and psychological terror. Compare to the Demiurge from the RPG \emph{Kult} and the portrayal of Christianity and other religions in \DIBiggestSecret. 

\lyricsdimmuborgir{Heavenly Perverse}{
  For this is your empire. \\
  This is your intrigue. \\
  Here you own them all. \\
  Here you seal the deed. 
  
  Contaminated from the spree of self salvation, \\
  to keep the fever flowing in the veins. \\
  Prominently manipulating Heaven and Hell. \\
  Does your sophistically discreet interlude \\
  maintain stories not to be revealed?\\
  In your search for redemption, \\
  greed and lies become the saviour. \\
  Through the lecherous eyes disgust withstands. \\
  For are not these the windows to your soul?
}

The Shroud and the Unspoken Covenant were great tools for keeping mortals enslaved. 
Both sides of the \Feud benefited from this.
Mortals turned out to be more useful than the immortals had initially thought.
Back in the day when there were many immortals, they relied on brute force.
In the \hs{Age of the Shroud}, the immortals relied on stealth, cunning, science and long-term planning to gain Ascendancy or their \matrices.

\citebandsong{Nile:AnnihilationoftheWicked}{Nile}{
  Sss'Haa Set Yoth
}{
  And thus they keep us Ever Subjugated, Hopeless, Fearful\\
  And despairing in the Deep Hidden Abysses of our Souls.\\
  Helpless, Unable to Escape the Unending Quiet Desperation.\\
  We have Been Broken down\\
  Conditioned to Accept Unconscious Slavery.
}



\subsubsection{The Shroud and the \CrystalSphere}
\target{Shroud and Crystal Sphere}
\index{\CrystalSphere}%
The Shroud was ultimately based on the \CrystalSphere, the \psp{\banelords}{} prison created by \TyarithXserasshana. 
So to destroy the \CrystalSphere{} one had to destroy the Shroud. 

\target{thawing}
And indeed, as the Shroud began to \hr{unravelling}{unravel}, the \CrystalSphere{} also began to thaw. 
The \hr{Return of the Banelords}{\banelords{} could walk on Mith again}. 





\subsubsection{The Unravelling}
\target{Unravelling}
\target{unravelling}
In \SentinelsofMithEmph, the Shroud was beginning to unravel. 
Officially, it was because the Shroud drew its energy from the Heart of \Miith{}, and \hr{Heart weakened}{in its weakened, faltering state}, the Heart could not sustain such a complex construction. 

But that was what the \banelords{} wanted everyone to believe. 
In truth, the far more important reason was their \hr{Sephirah plan}{\sephirah{} plan}, which was wearing the Shroud thin. 
Another reason was \ps{\Secherdamon} \quo{\hr{Rissitic creativity}{Shroud-drilling}}. 

It was an omen of things to come. 

Things were slipping through the Shroud. 
People would see things they would not have been able to see in previous centuries. 
And monsters, storms, madness, disease and space-warping phenomena would leak in from the Beyond. 

Compare to the vast flowing streams of Dust which Mary Malone discovers in the world of the \emph{mulefa} in \cite{PhillipPullman:TheAmberSpyglass}. 





\subsubsection{It will be horrible}
\target{Unravelling brings chaos}
The \hs{unravelling}, and the ultimate breakdown of the Shroud, will be horrible and will plunge the world into chaos. 

\lyricstitle{\cite{HPLovecraft:TheCallofCthulhu}}{
  The most merciful thing in the world, I think, is the inability of the human mind to correlate all its contents. We live on a placid island of ignorance in the midst of black seas of infinity, and it was not meant that we should voyage far. The sciences, each straining in its own direction, have hitherto harmed us little; but some day the piecing together of dissociated knowledge will open up such terrifying vistas of reality, and of our frightful position therein, that we shall either go mad from the revelation or flee from the deadly light into the peace and safety of a new dark age.
  
  [\ldots{}]
  
  That cult would never die till the stars came right again, and the secret priests would take great Cthulhu from His tomb to revive His subjects and resume His rule of earth. The time would be easy to know, for then mankind would have become as the Great Old Ones; free and wild and beyond good and evil, with laws and morals thrown aside and all men shouting and killing and revelling in joy. Then the liberated Old Ones would teach them new ways to shout and kill and revel and enjoy themselves, and all the earth would flame with a holocaust of ecstasy and freedom.
}








\subsection{Living under the Shroud}
\subsubsection{Class differences}
It is a global tendency that the rich, the nobles, people of the upper classes in general, are stronger and more intelligent, healthy and beautiful than the people of the lower classes.

This is caused by the Shroud: It is perceived as the natural order of the world that the rulers are all-round superior to the subjects. The poor are expected to be weak, stupid and sickly, and so they become. The prophecy is self-fulfilling.

Remember to have references to this in the story.

\lyricsdimmuborgir{Heavenly Perverse}{
  Devoted to your own opiate \\
  in escapades from discontentment. \\
  Are you shutting off from the outside world \\
  to reflect on your mind shallow gutter?
  
  For this is your empire. \\
  This is your intrigue. \\
  Here you own them all. \\
  Here you seal the deed. 
  
  Contaminated from the spree of self salvation, \\
  to keep the fever flowing in the veins. \\
  Prominently manipulating Heaven and Hell. \\
  Does your sophistically discreet interlude \\
  maintain stories not to be revealed?\\
  In your search for redemption, \\
  greed and lies become the saviour. \\
  Through the lecherous eyes disgust withstands. \\
  For are not these the windows to your soul?
}

\lyricsbs{Exmortem}{Fix of Negativity}{
  The world is swept in a curtain \\
  from the rottenness of Man. \\
  Fixing decadent values. \\
  Miserable mortals, open your eyes. 
}









\subsubsection{Immortals inside the Shroud}
\target{Immortals inside the Shroud}
Immortals are weakened when they venture inside the Shroud. 
\Dragons{} cannot assume their true form, \resphain{} cannot fly and magic is weakened. 

The \hs{dead garden} in \Malcur is OK; the Shroud is weak there.
The crypts under \Malcur, where Needle summons her \banes, is also OK. 
But \Forklin{} (where \Nzessuacrith{} and \Achsah{} duke it out) is another matter. 

The \dragons{} have pulled the shortest straw here. 
They are greatly weakened by the Shroud, since they must assume feeble humanoid forms (and in addition to the loss of powers, they are often less skilled at fighting in such a form, to boot). 
The \resphain{} are not so weakened. 
A powerful \resphan{} might even have a fighting chance against a \dragon{} deep in the Shroud.  









\subsubsection{Keeping people ignorant}
\target{Destroying information}
An important part of the \charade{} is keeping the general populace as ignorant as possible. They mustn't know of the war that goes on behind the scenes, and the best way to keep them from knowing that is by keeping them from knowing \emph{anything} that they don't need to know. 

To this end, the master races tend to encourage the destruction of libraries and other culture products, to repress knowledge and expression. The \hs{Iquinian Church} is used for this purpose, among other things.

Somewhat like George Orwell's \emph{1984}\ldots{} maybe. Or \emph{Fahrenheit 451}. 









\subsubsection{Maintaining the Mask of Civilization}
See section \ref{Maintaining the Mask of Civilization}







\subsubsection{People don't believe in things}
\target{People don't believe in things}
Most people don't believe in \dragons{} or \banes{} or other monsters. Some know that \resphain{} and \rachyth{} exist, and some know the \hs{myths} that they are descended from \dragons{} and \banes, but they don't believe those myths. 

Have lots of references to this: 
Someone knows, and tries to warn people around her, but they disbelieve and mock her for her \quo{superstition}. 
Preferably, these people should then die. 

\target{People do not believe in Dragons}
Even the \rethyaxes doubted whether \dragons really existed.
They had not been seen since the \hr{Human Age}{\Human Age} began.
Perhaps the Vaimons had destroyed them all, or perhaps they had never existed.
Perhaps the Vaimons had made the \dragons up in order to make the accounts of their own deeds seem more impressive.





\subsubsection{People forget things}
After having been involved in something supernatural, ordinary people tend to quickly forget the incident. Their minds block it out and rationalize it away. 

Entangled in the Shroud, people are unwilling to accept any version of reality that differs from what they believe to know, so if their memories disagree with their preconceived beliefs, then those memories must be repressed, and new ones must be fabricated.

Remember to have references to this: Carzain (or whoever) rescues people from supernatural monsters, and then they forget it and seem to recall him only helping them with something trivial\dash or being wholly unwanted. 

Carzain becomes frustrated by this more than once. Perhaps he even kills the ungrateful fool.







\subsubsection{Technology repressed}
\index{technology!Shroud}
The \hs{Shroud represses technology}. 







\subsubsection{The world needs a hero}
People are stupid, blind sheep who long for a shepherd to follow. Any shepherd. Or just a sheep dog. 

A leader, a hero, someone who can paint a semblance of meaning upon the cold, cruel, meaningless universe. As per the trope \trope{DyingLikeAnimals}{Dying Like Animals}.







\subsubsection{Light and Darkness}
\target{Light and darkness}
Many people see the Sun/light as good things, because it shows a clear world, unlike the dark, unknown, frightening night.

Others (wiser) see the light as evil, because it makes everything black-and-white, hiding all the subtle nuances of the true world, thus confirming the Shroud and the \hr{Lie Sublime}{Lie Sublime}.







\subsubsection{Time}
For regular people, time is a soft, malleable concept. People remember up to one lifetime \emph{at best}. Everything before that is just \quo{the past}. 

Maybe I should have some hints that time is semi-meaningless and vague. 

However, all of this is purely psychological. It is not true. The \emph{real} time is hard and linear. It is the Shroud, the \Sephiroth{} and people's own stupidity that prevent people from seeing it clearly. 









\subsection{Breaking through the Shroud}
Breaking through the Shroud, even for a brief moment, is a shocking, traumatic experience that leads to madness and suffering.

\lyricslimbonicart{A Cosmic Funeral of Memories}{
  I watch the dying Sun\\
  as it fades into the horizon of crimson fire.\\
  The paranormal darkness is now descending.\\
  Invasion of my mind, heart and soul.\\
  As reality shatters around me,\\
  I feel the changing, the metamorphosis.
}

\citeauthorbook[p.250]{HPLovecraft:TheBlackTomeofAlsophocus}{H. P. Lovecraft}{%
  The Black Tome of Alsophocus%
}{%
  For he who passes the gateways always wins a shadow, and never agains can he be alone.
  I had evoked\dash and the book was indeed all I had suspected.
  That night I passed the gateway to a vortex of twisted time and vision, and when morning found me in the attic room I saw in the walls and shelves and fittings that which I had never seen before. 
  
  Nor could I ever after see the world as I had known it.
  Mixed with the present scene was always a little of the past and a little of the future, and every once-familiar object loomed alien in the new perspective brought by my widened sight.
  From then on I walked in a fantasic dream of unknown and half-known shapes; and which each new gateway crossed, the less plainly could I recognize the things of the narrow sphere to which I had so long been bound.
  What I saw about me, none else saw; and I grew doubly silent and aloof lest I be thought mad.
}





\subsubsection{Chains}
\target{chains}
People who look through the \hs{Shroud} sometimes see other people as bound by chains that trail off into nothingness. 
This is the mind recognizing on a subtle level that people are bound and enslaved. 
The mind tries to translate this into familiar concepts, and this often takes the form of a vision of chains. 

These chains are a representation of people's binding to a \hr{Matrix}{\matrix}. 

Immortals who know the Shroud see these bindings all the time. 
They typically visualize them not as crude chains but as fine, silken threads connecting people to the vast universe of \matrices{} around them. 
Where Shrouded mortals see only slavery and horror, the immortals see meaning and purpose. 
Therefore the bindings look nicer to them. 





\subsubsection{Dreams}
\target{dreams}
\target{dreaming}
When asleep, people are less bound by the Shroud. 
Their minds can wander more freely.
That is why people saw such strange visions in their dreams. 

This was why \LocarPsyrex \hr{Tiroco contacted in dreams}{contacted \Tiroco Pelidor in dreams}. 

It was \hr{Encountering Dragons in dreams}{possible to encounter} the \hr{Aloof Dragons}{sleeping \dragons} when {dreaming}.






\subsubsection{Madness}
People who see through the Shroud, but are too ignorant, weak or stupid to understand or control it, acquire traumas and go mad. (Note that \hr{Madness}{madness} can at times be a good thing.)







\subsubsection{Slums and torture chambers}
In the most horrible of places, in slums, torture chambers, prisons and asylums, the Shroud is worn thin and you can more often perceive the \hr{Horrors of the Void}{horrors that lurk} in the \hs{Beyond}. 

Remember to have plenty of horrors that lurk in the Beyond! 

See, the Shroud is woven to a large extent by the beliefs of people. And beliefs are based in passion, emotion. But as with every force, if emotion becomes too strong a force it becomes destructive. So very strong feelings, such as fear and pain, can wear through the Shroud. 

This \quo{thinning} most strongly affects those people already \quo{vulnerable} and liable to break the Shroud, such as madmen. But it also extends to other people in the area or in near those people. 

The \hs{dead garden} in \Malcur is a mild version of one such place, inhabited by an old, mad woman.  









\subsection{Masks}
\target{Mask}
\target{Masks}
The Shroud enforces a number of collective beliefs about how the world looks. 
These individual beliefs (fragments or manifestations of the Shroud) are called Masks. 
They are similar to the psychological concept of \quo{frames} in seduction theory. 

An example is the \hs{Mask of Civilization}. 

\quo{Mask} can also mean a Shroud around a person or place to hide its true form and/or nature. 
All immortals must wrap themselves in Masks when travelling inside the Shroud. 
These Masks can be stronger or weaker. 









\subsection{The Realm of the Veins}
\target{Veins}
\index{Veins, the}
At the very densest part of the Shroud, the \quo{core}, \quo{closest} to the Heart, there lies a Realm called the Veins (ie., the Veins of the Heart). 
It is made of metallic corridors. 
People who wander in there are Shrouded beyond recognition and transformed into metal statues or robots. 

A \resphan{} once ventured in there. 
He lost his mind and locked himself into a \human{} body. 
After a \human{} lifetime his body died and his maddened soul wasted away. 















\section{Souls}
\index{souls}
All living creatures have a soul. The soul is actually their true \quo{body}, stretching beyond the physical body and into the Beyond. Thanks to the Shroud, the souls are deeply bound to their physical bodies, which are really shallow shadows of the actual souls, existing in the Realm of the Shroud. 

Souls are mortal: They can be created and destroyed. When \Miithian{} creatures reproduce, they invest some of their own life force and also tap into the creative power of the Heart of \Miith{} in order to create a new soul for their offspring. 

Sex taps into the power of the Heart of \Miith{}! Remember this. This is a useful theme. 

For the more powerful races, creating new souls is more difficult, requiring more power than the Heart can easily provide. So procreating is a complex, supernatural process, and the parents must permanently invest some of their own life force in the child. Such children are typically immortal or very long-lived. 

In contrast, regular creatures can be created cheaply, and they usually die soon after. The natural cycle of things is that when a creature dies, its soul lingers for a while but then gets absorbed by the Heart, its essence recycled to produce more life in the future. 

But civilizations have fucked this up. With all of their religion and magic, they try to make their souls immortal. This means that a lot of life energy which the Heart needs gets locked up in all these immortal souls, who may be bound in the \Sephiroth{} or whatever. So by virtue of their culture and religion, humanoids suck the life of of \Miith{}. 









\subsection{Bodies and souls}
\target{extended soul}
All creatures really extend beyond the physical plane. 
Their physical body is only a small part of their existence. 
Their complete being (often called the \quo{soul}) extends into the Realms Beyond, even into \hs{Chaos}. 


Most creatures are unaware of their true self and therefore cannot consciously use it to affect other planes beyond the physical.%
\footnote{%
  Except to a small, subconscious degree. 
  Empathy is actually the act of reaching out through the Beyond and sensing others' souls, and \quo{frame control} (the act of affecting other people's thoughts simply by acting a certain way\dash used in leadership and seduction) is reaching out and poking people.} 

The art of learning to control your true self and use it to perceive and affect the world is \emph{\hs{magic}}. 





\subsubsection{Disembodied souls}
\target{Disembodied souls}
\index{soul!disembodied}
A disembodied soul can be warded off, blocked, driven away or imprisoned. 
This is easy if the soul is weak, but hard if the soul is powerful. 

But a disembodied soul cannot be destroyed or eaten. 
At least not by any means known to \Miithians. 

On the other hand, a soul cannot do shit without a body. 
It can barely retain consciousness. 
It may be able to communicate and use some feeble magic. 
But not much. 

A disembodied soul is not worth much.
A disembodied soul cannot think and exists only in a barely-conscious, dreamlike state.
In order to be fully conscious and interact with the world, a soul needs to be housed in a \quo{brain} of some sort. And a brain needs to be supported by a body.
Furthermore, in order to cast magic, you need a body to channel the magical energy.

In many stories, ghosts (for example) are insubstantial. They can walk through walls but cannot affect physical objects.
In my world, disembodied souls are even more limited than that.
It partially derives from a question I have pondered from time to time: What is the relationship between the soul and the brain?
This has turned out to be a non-trivial question to answer. 
See, I want souls to have the opportunity to reincarnate into a new body after death. This new body may be genetically unrelated to the previous body.
But I also want a creature's intelligence and other mental traits to be affected by genetics.
This means that a living creature's psyche cannot depend on the soul entirely, but must also depend on the brain (which is part of the body). 
So I have decided that a soul cannot maintain consciousness without a brain of some kind.

The \dragons, with their superior immortality, were able to maintain consciousness even though their bodies were dead or destroyed. 
\Sethicus and \Tiamat did this after their deaths.
\Resphain could not do that. 
Only \Shiaraid achieved it (during Carzain's lifetime). 
It was a monumental achievement for her, only possible because she was a \sathariah \malach with a \carcer.









\subsection{\Carcers}
\target{Carcer}
\index{\carcer}
A \carcer{} is a soul prison. 
In it, souls can be stored to use as an energy source. 
Such souls can be molded into a powerful \vertex, or perhaps even a micro-\dweomer. 









\subsection{Death}
\target{Death}
\index{death}
Have some stuff about death. Remember to think about how death works in the \Miith{} universe. 

See also the section on \hs{Undead}. 

The \hr{Sephiroth}{\Sephiroth} are initimately tied to death. 

The \hs{Cult of the Worm} worship death. Worms are their symbols.

\target{Crows and ravens}
Crows and ravens (associated with death by virtue of the scavenger nature) are associated with the \hs{Cabal} and the \resphain. See the section about \hr{Harbeth}{\Harbeth}. 

Mortals do not step out of their body unscathed when they die. They are closely tied to their bodies and lose much of their power when they die. This is partly natural, partly caused by the \hs{Shroud}. 

\lyricstitle{\authorbook{Stephen Marley}{Mortal Mask} p.133}{
  \ta{What's copulation? 
    A brief, meaningless touching of flesh. 
    A birth is nothing more than an admission of death. 
    Mortals must give birth because all mortals must die, yet their race must continue. 
    Immortals have no need of birth. 
    Mortals should wear death masks when they make what they like to call love.}
}









\subsection{Immortality}
\index{immortality}
\index{soul!immortal}





\subsubsection{Degrees of immortality}
\target{Immortality}
\target{immortality}
\target{immortal}
\target{True Immortals}
\target{True Immortality}
\target{Lesser immortality}
\target{Lesser Immortality}
Some creatures are \quo{immortal}. 
There are two degrees of \quo{immortality}

\begin{description}
  \item[Lesser Immortality:] 
    These creatures never die of \quo{natural causes} and are very hard to kill manually, but they can die of violence. 
  \item[True Immortality:]
    In addition to the powers above, true immortals don't actually die even when they are killed. 
\end{description}

The last part warrants more explanation. 
A true immortal's body can be killed, but the body is only a small part of his being (as with all beings). 
The soul lives on, and it retains all its memories and most of its power (although physical death may leave the soul knocked temporarily unconscious). 
The soul can use a bit of magic, and it can communicate, and with time it can resurrect the body, or regrow a new body from scratch, if the old one is destroyed or unavailable. 
This is easier if the soul has allies to help it by \quo{healing} and building a body. 

But the soul can still be \hr{soul destruction}{destroyed} or \hr{soul-eating}{eaten}. 

\quo{True immortality} was invented by \TyarithXserasshana, based on some \xsic{} technology. 
The \banelords{} spied on this technique and copied it when designing their \resphain, so the \resphain{} also possess it. 
But \quiljaaran, \vorcanths{} and \aryothim{} only have \quo{lesser immortality} and can be killed. 

No one is quite sure how immortal \banes{} are\ldots{}





\subsubsection{Kinds of immortality}
\target{Kinds of immortality}
The different kinds of immortality include: 
\begin{itemize}
  \item \hr{Ophidian immortality}{\Ophidian immortality}. 
  \item \hr{Draconic immortality}{\Draconic immortality}. 
  \item \hr{Resphan immortality}{\Resphan immortality}. 
  \item \hr{Malach immortality}{\Malach immortality}. 
  \item \hr{Naga immortality}{\Naga immortality}. 
\end{itemize}








\subsubsection{Immortals don't succumb to emotions}
Unlike mortals, the master races do not easily succumb to emotions such as pain and fear. 
Rather, it motivates them further, makes them stronger (albeit perhaps more irrational). 





\subsubsection{Immortality is an abomination}
\target{Immortality is an abomination}
The natural way of things is that souls are formed by the \hr{Heart}{Heart of \Miith} and incarnated in a body, live for a while and then die and return to the Heart. 
But the master races have broken this law of nature by making themselves immortal. 
Each \dragon{} and \resphan{} refuses to die and instead lingers on forever (or tries to), hogging some vital life force that the Heart needs and wants. 
Thus, power is permanently stolen from the heart, and the biosphere's cycle of life is impoverished. 

It gets worse: 
Immortal creatures cannot die of natural causes, but they can be killed by applying enough destructive magic. 
But this kind of death does not release the soul back into the Heart, it \emph{annihilates} it. 
Its fragment energies are scattered to the cosmic winds and cannot be reclaimed by the Heart. 
And so, some of the planet's heartblood is permanently stolen and lost. 

As long as the creature lives, it is still tied to the Heart, and so part of the biosphere. 
It both takes energy from the Heart and gives some back, in the cycle of life. 
But when the creature is destroyed, the Heart feels the pain of the permanent loss of soul power. 

Now, this isn't as bad as it sounds. 
While the master races have some heavy-ass souls, their populations are also pretty low. 
New \dragons{} and \resphain{} are born and die quite rarely. 
It was a major problem during the \secondbanewar{} and the \resphanwars, where the master races were being killed in droves. 
This actually causes planetwide devastation and created deserts and wastelands that have never recovered and perhaps never will. 
But otherwise, the master races rarely die, and so it's not so big a deal. 
The Heart can cope with it. 

But it gets worse. 
The \banelords{} conquered \iquin{} and turned it into a prison of souls. 
Suddenly, all mortals under the mantle of the Iquinian religion were granted immortality (albeit immortality as a powerless ghost in a nightmarish prison). 
The problem with this is, of course, that mortals (as the term suggests) die all the time, and are born all the time. 
Now we have a huge \quo{net} sucking up a ton of newborn souls and hoarding more and more of them as time goes by. 
This is disastrous for the Heart. 
It is slowly being sucked dry, more and more of its vital lifeblood being locked away by the \sephiroth. 
And recently the \hr{Morbus}{\Morbus} has come along and sped up the process, making things even worse. 

So, immortality for immortals is not so bad. 
They breed slowly \emph{because} they are immortal, and the Heart can cope with it. 
They are made for it. 
But immortality for mortals is \emph{bad news}. 
There are just too many of them.
They breed like rats and ruin everything. 

It is starting to show: 
\Miith{} is growing less lively by the year. 
Greater expanses of the \Wylde{} are decaying into dead wastelands. 
And it's harder to procreate. 
More and more children are stillborn, or misshapen wretches that die soon after birth. 
This is especially bad for the master races themselves, since they have a hard enough time procreating as it is. 

The hideous \sephiroth{} are gradually killing the Heart of \Miith{}. 
It is a vital part of the \psp{\banelords}{} evil plan. 
The next stage of this plan is already in motion: 
The \hr{Morbus}{\Morbus}. 





\subsubsection{Sustenance for immortals}
\target{Immortal voracity}
All immortals were voracious carnivores, if not soul-devourers. 
They needed the energy of other living creatures to maintain their own lives. 





\subsubsection{How it ends}
The \hr{Daggerrain falls}{fall of \Daggerrain} finally puts a stop to the \ps{\banes}{} evil plans and causes the \sephiroth{} and the \Morbus{} to fail. 
Then \hr{Respite for the Heart}{the Heart can recover somewhat}. 









\subsection{Reincarnation}
\target{Reincarnation}
The Imetrium and Rissitics, and other religions, taught that the souls of the faithful dead would be reincarnated and return to the world, and that the best of them would be gathered unto the gods and given some blessed position.
The last part was true (although the \quo{blessed position} might be no more than a mindless niche in some \matrix).

The first part was also true, although perhaps not in the way people imagined it. 
The gods could shepherd souls and make sure they got recycled in the correct places, by means of prayers and spells cast on the dying (to catch the soul and store it in the \matrix) and on the conceiving, pregnant and newborn (to make sure a suitable soul would be injected into the mother and merge with the child). 
But this process was error-prone. 
Souls, however pious, might wander off and get lost, or get eaten by aethereal predators in the Beyond.
And souls are difficult to maintain in a disembodied state.
Great and powerful and skilled souls might be able to maintain themselves and even keep some measure of consciousness and power while disembodied.
But regular mortal souls lost consciousness and entered a dreamy state.
Here the souls became malleable and fluid.
They would break apart and merge into one another; lose soul-matter to the void and absorb new soul-matter from the void.
So an incarnated soul might not be an exact copy of any dead soul, but could be a mix of fragments of many souls.
Still, close enough.










\subsection{To destroy or eat a soul}
\target{soul destruction}
\target{soul-eating}
\index{soul!destroying}
\index{soul!eating}
Souls as usually semi-immortal. 
But they can be permanently destroyed. 
This is difficult. 

To destroy a soul, you must cast some spells on the body while it remains alive.
Then kill the body in a special, magical way. 
Then quickly cast some more spells to to capture the soul, and cast some more spells to destroy it. 
This is difficult, not only because the spells are difficult to learn and cast, but also because it has to be done in advance, while the victim still lives. 
The victim will probably notice and realize that he is in danger, and so fight extra hard or try to escape. 
It is not possible to capture a soul if the body dies before you are prepared for it. 
Then it escapes into the Beyond, at which point it is very difficult to track and capture. 
Typically, this means the soul-destroying business has to be planned and prepared in advance. 

Of course, it is harder when the victim is more powerful. 
If the killer is much more powerful than the victim it is pretty easy, but if they are equal in power it takes a lot of preparation and skill\ldots{} and trickery, because if the victim suspects he will take countermeasures. 

Even more insidiously, is is possible to not merely destroy a soul but \emph{consume} it. 
This is the most extreme form of \hr{Life drain}{life force drain}. 
It is even harder, trickier and more time-consuming than simply destroying and dissipating a soul. 

Destroying souls contributes to the \hr{Heart weakened}{weakening of the Heart of \Miith}. 

Some creatures (such as the \hr{Resphan}{\resphain}) \emph{live} by draining souls. 





\subsubsection{Shroud prevents soul destruction}
\target{Shroud prevents soul destruction}
It is very hard to eat or destroy a soul from deep within the Shroud. 
See, the Shroud allows physical bodies to interact, but it places harsh restrictions on how souls may interact with each other. 
This weakens many of the powers that strong souls otherwise possess (ie., magic), but it also protects them from being eaten. 
It can still be done, though. 
It is just very difficult. 

Examples: 
\hr{Silqua dies}{\Eryal{} allowed herself to be eaten}. 
\hr{Shiaraid dies}{So did \Shiaraid}. 
 

























\chapter{Magic}















\section{General}
\target{magic}
Magic exists in the \Miith{} universe and is an important part of it. 

\citeauthorbook{LinCarter:TheNecronomiconTheDeeTranslation}{Lin Carter}{
  The Necronomicon: The Dee Translation (part II.I)
}{
  Knowest thou this, that of all the arts and crafts and sciences whereunto may mortal men aspire, supremest and most potent of them all be the practice of Magic.
  Yet indeed, as \emph{Ibn Shoddathua} sayeth in his commentaries upon the Papyruses of Mum-Nath:
  Many are they who lust for the Mastery thereof, but few indeed are they who succeed therein.
  For the wise magician is the Master of Nature and the archpriest of all her Mysteries; at his command there openeth forth the Grave of Sod or the shutten Sepulchre of Stone, to admit forth they who slumber therein; before the utterance of his will shall storms becalm themselves, and floods retreat back into the secret fountains of the Deep, and conflagrations extinguish their fiery flames.
  
  Aye, and verily can he call down from beyond the stars That which abideth in the dark and freezing spaces of the Void, or forth from the Pit may he summon That which resideth in the black and frightgul abyss; spells and enchantments may be cast upon even the holiest of men or they that be purest of heart. 
  I say unto thee that such power may the accomplished Initiate command that nations shall grovel before his awesome might, and that the very Kings and Princes of the Earth shall flock to do him homande and obeisance.
  Even the very life of the Sorerer may be by his Art extended far beyond the ordinary limitations set upon mortal men, aye, and verily, for untold centuries mat he thrive, untouched by Time.
  For, behold! doth he not wield the keys of Life and Death? wherefore shall all mere mortal men exalt the Master thereof, and grovel at his feet, \emph{i\"a Nyarlathotep}! 
  The Wise Magician is a very mighty god.
}









\subsection{Cost}
\target{The cost of magic}
Magic always comes at a price\ldots{} but the currency can be stolen\ldots{}

\emph{All} magic, or at least all powerful magic, is powered by blood and life force. 

Different schools of magic carry different prices.

All these effects fall most strongly upon those mages of weaker mind. The more strong-willed mages are able to resist it better and tend to keep their personality more intact in the long run. 

If you practice two or more different kinds of magic, each with different side effects, these side effects may pull in different directions. This is the case with a Vaimon who uses both Iquin and Nieur: Iquin works to make him principled and virtuous, whereas Nieur works to make him savage and uninhibited. In such cases, these opposing forces may sometimes balance each other out, so that the mage can remain stable. On the other hand, they may also cause the mage to go schizophrenic or otherwise mad (perhaps even developing a genuine split personality) from the strain of having his mind pulled in two opposite directions. 






\subsubsection{Addictiveness}
One problem with magic is that it is addictive. Once you begin casting magic, you will have an urge to keep doing it, casting more spells and drawing more magical power. This addiction in itself is perhaps not so serious, but magic has other effects which are amplified by this compulsion to keep using it. 





\subsubsection{Overdose}
\target{Magic overdose}
When people use an overdose of magical power, their bodies suffer for it.
They get pain, bleedings\dash{}inner or outer\dash{}and scars; signs of disease and decrepitude.
They go around in pain and cough up blood.
This also happens to immortals (although they can heal more wounds than mortals can).

If you draw too much magical power, more than your skill and strength can handle, then you burn yourself up from the inside. 
Your brain, heart and blood burns to ash.

Compare to Beak in \cite{StevenErikson:ReapersGale} p. 757. 





\subsubsection{Humanoid sacrifice}
The sacrifice of living beings is an effective means of gathering magical energy that can be used in spells. 

In many cases, this is more effective if the victim in question is \quo{innocent}, ie., naive and idealistic, deeply entangled in the Shroud\dash preferably sexual virgins as well, since sexuality can be a road out of the Shroud. 
This way, you can shock the shit out of them in their very last moments, just before you kill them. When finally confronted with all the horror of the true universe, their shock is much more profound. 
This makes for a stronger \quo{jolt} in the Shroud (remember, the Shroud is maintained by people's beliefs), which releases more energy. 
This is especially useful in situations where the spell in question needs to re-weave the Shroud (as is the case with the summoning of \hr{Nith'dornazsh}{\Nithdornazsh} in \TwilightAngelRememberEmph).





\subsubsection{Wounds in the world}
Teleportation, portals, Realm travel and Shroud-weaving magic tear holes in the Shroud and leave bleeding wounds in the universe. 

Compare to the bleeding wounds in Kurald Emurlahn in the prologue to \MalazanReapersGale.










\subsection{Energy}
All magic works by reaching out with the mind and touching the world in ways not otherwise possible. 

In most cases, the mage does not perform all the effects of magic himself. 
Rather, with his spell he calls upon some external force or creature, binds it to his will and uses its power to affect the world. 
The Vaimons use the \Sephiroth{} and \Kliffoth{} for this purpose while Chaos mages call upon various \hr{Daemon}{\daemons}. 
Some of these forces are mindless or mostly mindless, while others are living creatures with more or less inteligence.

These external forces are normally unlimited (although they might be destroyed). 
But when casting magic, the mage must use some energy of his own to control the forces of magic, and a mortal mage's supply of energy is finite. 
As the mage expends energy, he will grow tired and exhausted and need to rest. 

Furthermore, in order to call upon the forces of the supernatural, the mage must spend some time attuning himself to these forces. 
This takes the form of meditation or prayer, where the mage contacts the occult forces and builds a connection to them. 
As he casts magic, his connection to the forces will become worn and decay, so every mage must periodically meditate to re-attune himself to the sources of his magic. 









\subsection{How to cast it}





\subsubsection{Circles of mages}
\target{circle of mages}
An \hr{Ishrah}{\ishrah}\dash a group of mages\dash can band together in a circle to perform spells together. In this way they can do more as a whole than as a group of individuals. A whole that is greater than the sum of the parts. 

Such circles need a name. How about \quo{concert}?

A circle of mages \hr{Circles are Matrices}{is a small, temporary \matrix} and obeys (or should obey) the rules for \matrices. 





\subsubsection{Laboratory magic}
Some magic is slow, taking hours or days, and may even require all sorts of material components. Such magic is best performed in a laboratory of some sort, and will here be called \quo{laboratory magic}. Laboratory magic typically includes all spells that are to have a permanent effect, including the creation of enchanted items. Major summonings are also often laboratory spells, as are major shape-changing spells (from \dragon{} to humanoid form). Major healings (for deep injuries or serious diseases) are also laboratory work. 

Laboratory work sometimes requires ingredients: Exotic minerals, plants or parts of creatures, or even living creatures. 





\subsubsection{Mortal vs. immortal magic}
It is difficult for Shrouded mortals to cast spells because the Shroud suppresses magic and the energies and natural forces on which it depends. 
Therefore, Shrouded mages must perform rituals and stuff in order to reach out to and touch those cosmic forces. 

People (mostly) free of the Shroud, such as the master races, can cast magic far easier, since they can readily see and perceive the energies they all around them. 





\subsubsection{Spellcasting aids}
Most magic, except laboratory magic, requires no items to be cast. But you can have items that help you cast spells. Typical spellcasting aids are staves, wands, rings and amulets. Such items should be made from special materials (exotic sorts of wood, occult metals or body parts of creatures) and enchanted with arcane sigils. Then they will help the mage channel his magical energy, so that he can cast stronger spells and expend less energy. It might also make it easier to contact the occult forces, so spells can be cast faster and with less risk of failure. 

There are also casting aids that are consumed and destroyed when the spell is cast. The most common example is the various herbs used in healing and in minor blessings and curses. (It should be noted, however, that sometimes that which is called \quo{herbal magic} is not magic at all, but simply natural medicine.) In most cases, these components are not strictly needed but serve to boost the spell's effect\dash{}or, in some cases, the spell boosts the herbs' natural effect. 









\subsection{Life force drain}
\target{Life drain}
It is possible to drain life force from others and use it to sustain yourself. 
This is actually what all animals do when they eat other animals or plants, but mages have more effective and nasty ways to do it. 

You can drain just a little force and leave the victim alive. 
Or you can drain the victim to death but let the soul escape. 
Most wicked of all, you can \hr{soul-eating}{\emph{consume} the soul itself}, permanently destroying the victim. 
This contributes to the \hr{Heart weakened}{weakening of the Heart of \Miith}. 
Some creatures (such as the \hr{Resphan}{\resphain}) \emph{live} by draining souls. 





\subsubsection{Voluntary sacrifice}
\target{Voluntary drain}
Life drain is more effective if the victim willingly lets herself get eaten. 
The \resphain{} employ this in their \hs{Communion}. 









\subsection{Mages}









\subsubsection{Learning magic}
It is sometimes said that to become a mage one must possess some unique, inborn talent or gift, but this is actually not true. The only character traits needed to learn magic are the intelligence to understand the theory of magic and the strength of will to bind the forces of the occult to your bidding. 

In addition, one must have access to teaching materials and a teacher. It is possible to learn magic by self-study from books and scrolls alone, but this is a slow, difficult and dangerous process\dash{}magic is dangerous work, and it is easy to hurt yourself if you don't know what you're doing. Similarly, it is possible to learn magic directly from a teacher with no written materials available\dash{}there are mages among pre-literate cultures, and their magic can be potent, if primitive. But learning from a teacher with books and scrolls available is the best and most common method. (Note that in order to use books one must of course be able to read, and very few people on \Miith{} are literate.) 

Learning magic is a long and difficult process. It usually takes years to learn to cast but the simplest spells and a decade or more to become a competent mage. Like all skills, magic is most easily learned at an early age. Many mages\dash{}and most of the skilled ones\dash{}were apprenticed as children, before the age of ten, and studied the art for at least fifteen years. 





\subsubsection{Losing one's magical ability}
There are only a few things that can permanently weaken or destroy a mage's ability to cast magic. These include brain damage, amnesia, insanity/mental illness and extreme shock/mental trauma. 

Most mages also use physical gestures to cast magic, so a mage who is suddenly maimed will find it harder to cast his spells. This is not an absolute hindrance, however, and in most cases you can learn to cast your spells anyway without relying on physical gestures. (There are exceptions, though. Some spells require a physical ritual, such as the drawing of occult symbols, and such spells cannot be cast if the ritual cannot be completed. Still, sometimes it is possible to have an assistant perform these parts of the ritual instead.)





\subsubsection{Mages vs. \vertices}
See the section on \hr{Vertices vs. mages}{\vertices vs mages}. 





\subsubsection{Magical strength}
Some mages are more powerful than others. There are a number of factors determining the magnitude of effects that a mage can accomplish with magic. These include: 

\begin{itemize}
  \item Knowledge of spells. 
    Working magic is complex work, and knowing a clever spell will let you accomplish your work much faster, more effectively and with less expenditure of energy than a simple, naive spell. 
  \item Willpower. 
    Strength of will alone is important, as it lets you bind more power to your will. This one is open-ended and constrained solely by the will of the mage, but even so, even with infinite willpower there are limits to what you can do.
  \item Innate power of the soul. 
    This is a fuzzy concept as of yet, but the idea is that some creatures, even if their willpower is the same, have more mental \quo{muscle} and can handle more magic power than others. 
  \item   
    For a Chaos mage, self-\hs{Gnosis} is an important source of strength. 
\end{itemize}

Every person has a certain magical \quo{power} or \quo{strength}. 
This works mostly like muscle strength: 
A person is born with a certain natural amount of power. 
This can be increased by training and experience. 

But power can also be stolen from others. 

And power can be gained by insight (or \hr{Madness}{madness}): 
Enlightenment, epiphany, revelation. 
This is kind of like lifting yet another layer of \hs{Shroud}: 
It reveals a deeper, truer world. 
Once you realize that this world exists, you can begin to interact with it and manipulate it. 

As a mage casts spells, he will expend energy and become tired. This is similar to physical fatigue, but not quite the same. 

Perhaps to regain your magical power you need to meditate and attune yourself to the mystic forces that you channel. In a religiously oriented magic theory, this meditation will take the form of prayer\dash{}to the \Sephiroth{} or to \hr{Rissit}{\RissitNechsain} or whoever. 









\subsection{Mages and society}





\subsubsection{Mage clans and rogue mages}
\target{Master races seek to control magic}
Many mages are organized in \ishroth, mage clans and other organizations. This serves the master races' purpose. Magic is necessary in the world, but dangerous, and the master races want to control it. 

\Nieur{} and Chaos mages are frowned upon (and sometimes actively hunted down) by the Cabal and Sentinels, unless they stay within established and controlled mage clans. The mage clans themselves hunt down defectors and runaways. \hr{Takestsha}{\Takestsha} employs this in \hr{Takestsha on the run}{her fake background story}.

\hr{Geica is embattled}{Geica is embattled by the master races} for this reason. 




\subsubsection{Mages are often nobles}
\target{Most mages are nobles}
Most mages are nobles. 
Commoners rarely get the chance to be educated and study the sciences, least of all magic. 





\subsubsection{Old mages sequester themselves}
Older mages have a tendency to lock themselves up in their towers.
There they continue their occult studies, hidden away from the world. 
It is a kind of tradition. 
This works well, because the world would rather see their heel than toe. 
People are happy to see them go, because old mages tend to be rather mad. 
And they can get pretty powerful, too. 





\subsubsection{Sorcerer-kings are uncommon}
\target{Sorcerer-kings}
\index{sorcerer-kings}
In the \hr{Scatha Age}{\Scatha Age}, in \Azmith, sorcerer-king were less common than one might think. 
In some \scathaese cultures, where the \hr{Ortaican religion}{\Ortaican religion} still dominated, \rethyax-dominated \hr{Baccon}{\baccons} still held power. 
But in Iquinian countries, sorcerer-kings were rare. 

\begin{prose}
  \ta{%
    Have you ever wondered why sorcerer-kings are so rare? 
    Mages are clearly superior to the common folk, in power, intellect and wisdom alike. 
    So why do they not rule \Miith{}? 
    Why do they, in most countries, bend the knee to mundane kings?
    
    The answer is: 
    Because we do not allow it. 
    We of the Cabal, and the Sentinels as well, fear that a sorcerer-king might become to powerful. 
    So we pull strings and see that they fail before they gain too much power, or pull them down if they do.}
\end{prose}

The organizations have also founded mage orders, with which they hope to keep future generations of mages under their control, or at least surveillance. 

Also, people from royal families and powerful nobles are rarely allowed to study magic, even if talented. 
The master races do not want them getting that kind of power. 





\subsubsection{Useful purpose}
Mages would be potentially useful for all sorts of menial labour: 

\lyricslimyaael{377933}{
  Wind and water mages would be in demand not only on ships, where some authors do put them, but to give crops good weather, provide pleasant days for large festivals, turn aside or dissipate storms, purify drinking water, move streams around or dam them, clear away this blasted fog, put this blasted fog in place so that it can blind the enemy or keep rival ships from sailing, calm this stream from flooding, coax that stream into flooding so that it can provide rich soil, make this hydraulic system work\ldots{} 
  There's lots and lots of uses. 
  Earth mages could restore soil, make it richer, help crops grow faster, call animals in for slaughter, prevent earthquakes and mudslides or clean up after them, insure that transported plants survive and acclimate to new soil, and do other things depending on what power you've given them.
}

They sometimes do. 
Especially the \hs{Rissitic} mages are well-integrated in society, and to a lesser degree the \hs{Imetric} ones. 
But they don't do as much honest work as they could, mostly because mages are snobs. 
They know they are highly valued, so they assume high places in society and shy away from menial labour. 

They do work on ships, though. 
\hr{Sea}{Travel by sea is fucking dangerous}. 









\subsection{Occultism/mysticism}
I should have a lot of magic that looks \quo{occult}, \quo{mystical}. 





\subsubsection{Astrology}
See the section on \hs{astrology}.





\subsubsection{Geometry}
\target{Occult geometry}
\target{occult geometry}
Have some occult geometry. 
\hr{Malcur}{\Malcur} is built using a lot of this.

Have some \emph{Feng-Shui}-like principles of building construction. 

Especially the Rissitics \hr{Rissitic architecture}{employ it}. 
The \Ortaicans{} also \hr{Ortaican architecture}{used it}. 
And \ps{\Ishnaruchaefir} \hs{glaive} is built with it. 





\subsubsection{Numerology}
Have some occult numerology. This could be connected to geometry, forming a unified whole of occult mathematics.





\subsubsection{Sexual mysticism}
\target{sexual mysticism}
And I should have stuff about sex and magic. 







\subsection{Suppressing magic}
It is possible to suppress magic. Suppression works on a person or an area and prevents the person, or everyone in the area, from casting magic of one or more types. 





\subsubsection{Suppression spells}
One way is by casting a suppression spell on one or more persons, thus suppressing any spellcasting on their part. There are two problem with such spells. One problem with such spells is that they must be continuously maintained by one or more mages. Another problem is that the suppression can be broken if the subject is a stronger or more crafty mage than the one holding the spell. 





\subsubsection{Suppression drugs}
\target{Witchbane}
Another way is using drugs. The most well-known of such are the toadstools called \quo{witchbane}. When eaten, drunk or injected into the veins, after being properly prepared and distilled (into a soup-like fluid), witchbane affects the victim's mind, making it difficult to think clearly and making spellcasting all but impossible. (Even if the afflicted victim manages to access his magical power, he is unlikely to be able to cast a coherent spell, but may be able to unleash some random magical mayhem.) 

The advantages of witchbane is that the toadstools are rather widespread and the concoction is easy to make. The drawback (if the victim must be kept alive and mostly unharmed) is that ingesting it in large quantities will cause permanent brain damage, potentially resulting in insanity, loss of intelligence or loss of motor skills (stuttering, uncontrollable shaking or paralysis). (At this time, no magical or mundane cures for any of these effects are known.) A victim can be pacified for at most 4-6 hours without risking serious permanent damage. (Of course, all these effects will affect anyone, whether mage or not.)

There are several species of witchbane toadstools, which can be more or less potent and more or less dangerous. 





\subsubsection{Inhibitors}
%A more potent method is using special items, called inhibitors. 
An inhibitor is a specially crafted and enchanted item made from exotic crystals and precious metals that suppresses spellcasting in an area. Some inhibitors cover only a small area, such as one person. These may take the form of collars or shackles. Others are strong enough to cover an entire room. These may have any form. 

The main advantage of inhibitors is that they are stable. An inhibitor functions constantly, with no operator necessary, until it is worn out. Typically they last many years. 

The disadvantage of inhibitors is that they are prohibitively expensive. Creating them is very difficult and time-consuming laboratory work, and the ingredients are rare and exotic. Another problem is that the inhibitor can never be turned off, which is inconvenient to the user. 

Inhibitors can be destroyed with physical force, but they are always enchanted to make them extremely durable, to prevent a prisoner from simply smashing them.










\subsection{Uses of magic}
Magic has many uses. 









\subsubsection{Magical healing}
\paragraph{Naturalist healing}
Magical healing might work by touching the recipient's \hs{Chaos} body, providing it with some energy and helping it to heal itself. In such cases, a portion of the energy comes from the healer and a portion comes from the recipient's own body. Such healing is rather easy, but crude, and cannot heal difficult injuries or badass diseases\dash things the body couldn't heal on its own. The healing might be assisted by herbs.

The Vaimons and most primitive cultures use naturalist healing. Most Chaos mages also use naturalist healing. This is because the theory of Chaos magic was developed by \dragons{}, and \dragons{} have formidable natural healing capabilities, so this kind of healing is very effective on them. For the \draconic{} Chaos mages, healing their servitors (\scathae{} or others) was never a high priority\ldots{}





\paragraph{Surgical healing}
Another approach is to use magic like surgery. Instead of relying on the body's natural ability to heal itself, the mage might rely on his own skill and knowledge to directly control the healing process. This requires the mage to know a lot about anatomy and surgery, so it is difficult to learn. The process itself is also difficult and consumes time and energy from the caster. The upside is that this healing is more sophisticated and can heal more grievous wounds and diseases. 

This kind of healing might also involve herbs or drugs, and the mage/doctor might use mundane surgical tools in conjunction with his spells. 

The Imetrians and Rissitics use surgical healing. Recently, some Vaimons in Geica have begun experimenting with it. 





\subsubsection{Magic in war}
Magic has a number of uses on the battlefield. 

A Vaimon can call down the elements to kill his opponents. He can also fly over walls or use Earth magic to break the walls. 

A Rissitic or Chaos mage can summon \hr{Daemon}{\daemons} or other creatures to his aid. 

In war, (mortal) mages would bombard the enemy army with lightning and rain of fire, all the while trying to dispel and negate the enemy sorcerers' attempts to do the same. 





\subsubsection{Magical transportation}
You can use magic for travelling purposes in a number of ways. 



\paragraph{Super-speed}
  There are spells that can be cast on a creature to temporarily (or even permanently) increase its speed or endurance. 
  Such spells may come with a side effect: 
  After the effect wears off, the subject might experience extreme fatigue and/or weakness or even permanent damage, aging or death. 



\paragraph{Flight}
  \target{Flying magic}
  \target{Vaimon flight}
  A Vaimon cannot fly. 
  But with the help of \Atzirah a Vaimon can perform great powered jumps and walking-on-walls and stuff. 
  This is not easy, though. 
  
  Compare to the crazy acrobatics seen in Asian martial arts movies such as \cite{Movie:CrouchingTigerHiddenDragon}.
  
  The Imetrians have a similar spell. 
  It gives better control and \manoeuvrability but less speed, and it is more difficult to learn. 
  
  Rissitic and Chaos magic have no flying spells. 
  
  For immortals (such as \quiljaaran and \bezed \resphain), it was possible to fly with magic.
  But it was difficult to learn and taxing to do. 
  They only used it when they had to. 
  



\paragraph{Summoning}
  A Rissitic mage or Chaos mage might be able to conjure a creature and coerce it to carry him as a rider. 
  Such a creature might be able to run, swim, fly or perhaps even teleport. 



\paragraph{Teleportation}
  Chaos magic spells exist that teleport the mage from one teleporter to another. 
  There are several of these teleporters scattered across the world, but the secrets of their making have been lost. 





\subsubsection{Telepathy and empathy}
See the section \ref{Telepathy}: \quo{\hs{Telepathy}}





\subsubsection{Magical items}
Items can be enchanted in many ways. These include:

\begin{itemize}
  \item 
    Any kind of item can be made stronger, more durable. (This is quite simple.)
  \item 
    Weapons can be made to hit harder and more accurately. (This is quite complicated. I don't know how it works yet.)
  \item 
    Items can be enchanted to act as a focus for a spell, mystically as well as physically. An example might be a cannon made to fire fireballs or lightning bolts. Creating a cannon that can cast these spells on its own is extremely difficult, but you can make a cannon that can be operated by a mage, allowing him to cast spells through the cannon more powerful than he could otherwise cast. The physical shape of the cannot could also aid in shaping the spell, making it easier and safer for the mage. 
  \item 
    It is \emph{not} possible to make enchanted items with arbitrary superpowers. For instance, no \quo{Frying Pan of Ultimate Gourmet Cooking} or \quo{Comb of No Bad Hair Days Ever}.
  \item 
    You can enchant an item to carry a limited (small) number of \quo{charges} of a certain spell (or more than one). Examples are potions that are ingested to activate the spell. These may utilize the ordinary chemical properties of the ingredients in addiction to or in conjunction with the magic, to create more powerful effects (healing or whatever) than otherwise possible. 
\end{itemize}

\paragraph{Making magical items}
Making a magical item is generally difficult and time-consuming laboratory work and often requires exotic ingredients. If a magical items is to be really effective, you cannot just enchant an existing item. Rather, it must be created from scratch, with the magic being addded from the beginning, weaving the magic in at every stage while it's being forged or assembled. 

Even creating a trivial magical item is tremendous work, so mages tend to make a few powerful ones rather than many little ones. The exception is one-off items like potions, which are relatively easy to make. 









\subsection{Visuals}
\target{Magic visuals}
This section gives ideas for how to describe the effects of magic. 

Immortal mages can unleash awesome powers. 

\lyricsbalsagoth{The Obsidian Crown Unbound}{
  And even as this transpired, the Emperor's Prime Sorcerer, emissary of the Imperial Court and master of those arts which speak to man in narcotic dreams from the darkest and most silent places, summoned forth that black potency which lay entwined in stygian tendrils within his mind\ldots{} an ireful power born of they who writhed upon the shores of Pangaea before man's progenitors ever erected their lofty spires to the restless skies.\\
  And yet Vyrgothia's Master Wizard, unrivalled Arch-Mage and adept of that lost Eastern order who journey beyond the boundaries of time and space upon those nebulous wings born of the sacred Azure Lotus, rose to meet this power which lapped at the periphery of his mind like a midnight tide, and stood firm against its insistent siren call.\\
  And upon that arid field of war, the sentinels of light and shadow spoke to each other in tongues dormant since the Third Moon fell burning from the heavens, and not sweet were the words they uttered.
}

There is dark art, despair and beauty. 

\lyricslimbonicart{Beyond the Candles Burning}{
  I am a dark star rising on the raveous bleaky sky,\\
  a black diamond slunning so deep within the night.
  Maliciously I dwell in a bluish shaded beam\\
  with a stonecold heart into the core of my being.
  
  Beyond the candles burning, beyond all minds eye.\\
  A vast emperic enigma awaits me as I die.\\
  In a graceful dance obscene, in a ring of fire,\\
  I obtain my majesty as flames caressing higher.
  
  Release my spirit, unleash my soul.\\
  From the darkest dungeon, oblivion calls.\\
  In the phallic halls of ancient forlorn\\
  a cold sanctuary in doom is born.
}

\lyricslimbonicart{Solace of the Shadows}{
  I require the solace of the shadows,\\
  so the night can be redeemed.\\
  As the winds of darkness whispers my name,\\
  a kiss of death I receive.
  Nocturnal enchanter, to thine art I yield.\\
  Within the candlelight a rapture is now revealed.
}





\subsubsection{Curses of destruction}
\target{Curses of destruction visuals}
\target{Draconic curses of destruction}

Remember that \hr{Ruin Satha fire magic}{\draconic/\rethyactic fire magic} is \hr{Ruin Satha and fire}{associated with the \xs \RuinSatha} and produces many-\coloured spiritual flames.

When a \dragon (such as \Ishnaruchaefir or \Nzessuacrith) kills, he does it by invoking \xss and channelling their destructive power.

\citebandsong{Nile:InTheirDarkenesShrines}{Nile}{
  Destruction of the Temple of the Enemies of Ra
}{
  The Fire of the Eye of Horus is Upon You\\
  Searching You Consuming You\\
  Setting you on Fire Burning you To Ashes

  Unemi The Devouring Flame Consumes You\\
  Sekhmet The Blasting Immolation of the Desert Maketh an End of You\\
  Xul Ur adjugeth you to Destruction\\
  Flame Fire Conflagration Pulverize You

  Your Souls Shades Bodies and Lives Shall Never Rise Up Again\\
  Your Heads Shall Never Rejoin your Bodies\\
  Even The Words of Power of The God Thoth\\
  The Lord of Spells\\
  Shall Never Enable you to Rise Again
}

Curses as a \dragon destroys his enemies in the names of the \xss:

\citebandsong{Nile:AnnihilationoftheWicked}{Nile}{
  Lashed to the Slave Stick
}{
  Ra Pronounceth the Formulae Against Thee.\\
  The Eye of Horus is Prepared to Attack Thee.\\
  Sekhmet Uttereth Words of Flame Against Thee \\
  and Pierceth Thy Breast.

  Abui, The Gods Who Burns the Dead.\\
  Shall Leave You Smoldering in Exile from the Netherworld.\\
  Abati, the Gorer, causes You to Howl like a Jackal in Anguish.
}

\citebandsong{Nile:Ithyphallic}{Nile}{
  Ithyphallic
}{
  Let the shades of my fathers curse their faces\\
  Let the eye of Sekhmet\\
  Send the violence of the sun down upon their heads\\
  Let searing torrents of fire descend upon their brow\\
  Let flames immolate their places of sleeping

  Let the eye of Sekhmet\\
  cause their hearts to burst into flames

  Let my curses be heard\\
  Let my will be as Menthu the bull\\
  Potent to create\\
  And savage to slay those whom I hath cursed\\
  Let my wrath be terrible\\
  And my vengeance unmerciful
}

\citebandsong{Nile:Ithyphallic}{Nile}{
  Laying Fire Upon Apep
}{
  Fire be upon thee Apep\\
  Ra maketh thee to burn\\
  Thou who art hateful unto him\\
  Ra pierceth thy head\\
  He cutteth through thy face\\
  Ra melteth thine countenance\\
  Lo your skull is crushed in his hand\\
  Thy bones are smashed in pieces

  Burn thou fiend\\
  Before the fire of the eye of Ra\\
  The hidden one hath overthrown thy words\\
  The gods have turned thy face backwards\\
  Thy skull is ripped from thy spine

  The lynx hath torn open thy breast\\
  The scorpion hath cast fetters upon thee\\
  Maat hath sent forth thy destruction\\
  Thou shalt burn

  [solo: Karl]

  The god Aker hath condemned thee to the flames

  Fire be upon thee Apep\\
  Thou enemy of Ra\\
  Let flames gnaw into thee\\
  And sear thy flesh\\
  Fall down Apep\\
  I hath set torch upon thee\\
  Taste thou death Apep\\
  The burning is upon you\\
  Thou art consumed\\
  I hath lain fire upon thee\\
  I hath smeared thy remains with excrement\\
  I hath spat on thin ashes\\
  Taste thou death
}

\target{Resphan curses of destruction}
Curses when the \resphan rebels devour the souls of their enemies:

\citebandsong{Nile:AnnihilationoftheWicked}{Nile}{
  Lashed to the Slave Stick
}{
  Your Corruptible Bodies Shall be Cut to Pieces.
  Your Souls Shall have No Existence.
  Ye Shall Never Again See Ra as He Journeyeth in the Hidden Land.
  The Doom of Ra is upon You.
}





\subsubsection{Dangers}
A warning about the dangers inherent in seeking magical power:

\citeauthorbook[p.175]{LinCarter:TheNecronomiconTheDeeTranslation}{Lin Carter}{
  The Necronomicon: The Dee Translation (part I.VII.III)
}{
  Aye, be thou warned, for in all such voyages and venturings of mind or soul or spirit there be very great and terrible dangers, by mortal men undreamt-of and unknown. 
  Beware then, lest thou penetrate too deeply into the blackest backward and depthless abysm of the womb of infinite time. 
  For beyond the very Beginning thereof, and on the Other Side thereof,there dwelleth That of which man suspecteth not; and there thou wilt find a strange and ominous Realm where hidden horrors lurk and naked Terror hunts unseen; which dim, uncanny bourn hath the seeming and the semblance of a pale, and grey, and indefinite shore, lapped by the sluggish waves of unmeasured and unthinkable Time.
  And it is eve there, in an awful Light that is beyond all darkness, amidst a profound Silence that shieketh beyond all sound, that \emph{They} slink and prowl in all their ghastliness, slavering with a loathsome and ana unspeakable hunger for all that is clean and whole and unsullied.
}




\subsubsection{Invocation of \daemons}
\target{Daemon invocation visuals}
Chaos sorcerers invoke \hr{Daemon}{\daemons}, or even the \xss. 

Read the section about the \hr{XS}{\xss}. 

\lyricsbalsagoth{%
  The Splendour of a Thousand Swords Gleaming Beneath the Blazon of the Hyperborean Empire% 
  \dash Part III: 
  Cry Havoc for Glory, and the Annihilation of the Titans of Chaos
}{
  Writhing tendrils of night-dark, coruscating energy lanced from the surface of the blade, entwining the King in a pulsating chrysalis of searing sorcerous power. \\
  His eyes shone deep crimson with an illuminatory radiance not born of this world, and forces which had lain dormant since before the fall of the Third Moon stirred at last from their aeons-old slumber\ldots{}
}

\lyricsbs{Nile}{What Can Be Safely Written}{
  It was him and his spawn that defeated the Elder Things,\\
  who had long possessed sovereignty of this world,\\
  before he descended on his gray and leathern wings\\
  through the upper gate opened by Yog-Sothoth.
}

\lyricsbs{Aeternus}{The Essence of the Elder}{
  Leaving the dismay of the world behind.\\
  Traveling in the essence of chaos.\\
  Time and matter dissolving about me.\\
  A journey beyond death.
  
  Your blood pulses with a thousand years of existence.
  
  Standing proud before those whose ageless blood\\
  flows through my veins.\\
  Their spiraling souls empower me.\\
  Visions of chaos.
  
  A union of blood, a union of souls.\\
  An endless bond sworn and upheld.
  
  I walk with you\\
  over the fields you bled red,\\
  through fires that finally devoured you,\\
  through your dreams and your nightmares.
}

\Tiamat{} and the \firstgendragons{} (who are \hr{Elder Dragons worshipped}{dead but still worshipped}) are invoked in Chaos magic. 
She was the one who made the original pacts with the \xss, so she must be invoked when one seeks aid from the \xss. 

\lyricsauthorbookpage{Anton Szandor LaVey}{The Satanic Bible}{116}{%
  In the name of Satan [\TyarithXserasshana?], 
  Ruler of the earth, 
  King of the world, 
  I command the forces of Darkness to bestow their Infernal power upon me.  
  Open wide the gates of Hell and come forth from the abyss in answer to your most Unholy names\ldots{}
}

\lyricsauthorbookpage{Anton Szandor LaVey}{The Satanic Bible}{147}{%
  \textbf{Invocation Employed Towards the Conjuration of Lust}
  
  Come forth, Oh great spawn of the abyss and make thy presence manifest. I have set my thoughts upon the blazing pinnacle which glows with the chosen lust of the moments of increase and grows fervent in the turgid swell. 

  Send forth that messenger of voluptuous delights, and let these obscene vistas of my dark desires take form in future deeds and doings. 
  
  From the sixth tower of Satan there shall come a sign which joineth with those saltes within, and as such will move the body of the flesh of my summoning. 
  
  I have gathered forth my symbols and prepare my garnishings of the is to be, and the image of my creation lurketh as a seething basilisk awaiting his release. 
  
  The vision shall become as reality and through the nourishment that my sacrifice giveth, the angles of the first dimension shall become the substance of the third. 
  
  Go out into the void of night (light of day) and pierce that mind that respondeth with thoughts which leadeth to paths of lewd abandon. 
  
  (Male) My rod is athrust! The penetrating force of my venom shall shatter the sanctity of that mind which is barren of lust; and as the seed falleth, so shall its vapours be spread within that reeling brain benumbing it to helplessness according to my will! In the name of the great god Pan, may my secret thoughts be marshalled into the movements of the flesh of that which I desire! 
  
  Shemhamforash! Hail Satan! 
  
  (Female) My loins are aflame! The dripping of the nectar from my eager cleft shall act as pollen to that slumbering brain, and the mind that feels not lust shall on a sudden reel with crazed impulse. And when my mighty surge is spent, new wanderings shall begin; and that flesh which I desire shall come to me. In the names of the great harlot of Babylon, and of Lilith, and of Hecate, may my lust be fulfilled! 
  
  Shemhamforash! Hail Satan!
  
  \textbf{Invocation Employed Towards the Conjuration of Destruction}
  
  Behold! The mighty voices of my vengeance smash the stillness of the air and stand as monoliths of wrath upon a plain of writhing serpents. I am become as a monstrous machine of annihilation to the festering fragments of the body of he (she) who would detain me. 

  It repenteth me not that my summons doth ride upon the blasting winds which multiply   the sting of my bitterness; And great black slimy shapes shall rise from brackish pits 
  and vomit forth their pustulence into his (her) puny brain. 
  
  I call upon the messengers of doom to slash with grim delight this victim I hath chosen. Silent is that voiceless bird that feeds upon the brain-pulp of him (her) who hath tormented me, and the agony of the is to be shall sustain itself in shrieks of pain, only to serve as signals of warning to those who would resent my being. 
  
  Oh come forth in the name of Abaddon and destroy him (her) whose name I giveth as a sign. 
  
  Oh great brothers of the night, thou who makest my place of comfort, who rideth out upon the hot winds of Hell, who dwelleth in the devil's fane; Move and appear! Present yourselves to him (her) who sustaineth the rottenness of the mind that moves the gibbering mouth that mocks the just and strong!; rend that gaggling tongue and close his (her) throat, Oh Kali! Pierce his (her) lungs with the stings of scorpions, Oh Sekhmet! Plunge his (her) substance into the dismal void, Oh mighty Dagon! 
  
  I thrust aloft the bifid barb of Hell and on its tines resplendently impaled my sacrifice through vengeance rests! 
  
  Shemhamforash! Hail Satan!
}





\subsubsection{Protection}
\Merkyran religious spell/ritual to ward off umbrae:

\citebandsong{Nile:Ithyphallic}{Nile}{
  Papyrus Containing the Spell to Preserve Its Possessor Against Attacks from He who is in the Water
}{
  Amun\\
  Lord of the gods\\
  Thou who art of the four rams heads upon thy neck\\
  Thou standest upon the spine of the crocodile fiends\\
  To thine sides are the dog headed apes\\
  The transformed spirits of the dawn

  Drive away from me the lions of the wastes\\
  The crocodiles which come forth from the river\\
  The bite of poisonous reptiles\\
  Which crawl forth from their holes

  Be driven back crocodile thou spawn of Set\\
  Move not by means of thy tail\\
  Work not thy feet and legs\\
  Open not thy mouth\\
  Let the water which is before thee\\
  Turn into a consuming fire

  I possess the spell to\\
  Preserve me from he who is in the water

  Thou whom the thirty seven gods didst make\\
  And whom the serpent of Ra didst put in chains\\
  Thou who wast fettered with links of iron\\
  In the presence of Ra\\
  Be driven back thou spawn of Set

  Drive away from me the lions of the wastes\\
  The crocodiles which come forth from the river\\
  The bite of poisonous reptiles\\
  Which crawl forth from their holes
}





\subsubsection{Spellwords}
\target{Spellwords}
Remember to have mystic spellwords and incantations. 
\quo{Forgotten spells not uttered in many aeons\ldots{}}

\lyricsbalsagoth{Shackled to the Trilithon of Kutulu}{
  Rise o' spawn of Chaos and elder night.\\
  With these words (and by the sign of Kish), I summon thee.\\
  Slumbering serpent, primal and serene,\\
  Great Old One, hearken to me!
  
  When the stars align in the Chaosphere, \\
  then the time of awakening shall be at hand!
}

\lyricsflnv{3}{
\ta{Zarrhahull aknesh zain taushaark!}

\ta{Aggrhul shai zain hoorghth.}}

\lyricsflnv{5}{
\ta{Necrain thanasthos eheirrheinnhenn rais!\\
    Deshtor neshan tharrh!\\
    Berrhell nerrhell rhul!\\
    Bezhul tharrh!\\
    Rais!\\
    Rais!
    
    Aalheen dorth haaarth zhhuuull!}
}

\lyricsflnv{10}{
\ta{Hierremm haarhzhuul haashthorn n herh thorr!\\
    Herrint sheen!
    
    Shemterrann!}
}

\lyricsbs{\FMFroideval}{Succubus (Volume 1)}{
  \ta{Hieram zaraoth, Desdemona! Hierem haarth!}
}





\subsubsection{Summoning magic}
\target{Summoning magic visuals}
\lyricsbs{Bal-Sagoth}{
  As the Vortex Illumines the Crystalline Walls of Kor-Avul-Thaa
}{
  The sky rent asunder. \\
  Black-winged devils surge forth from the void.\\
  A maelstrom of crimson fire burns above us.\\
  What carnage hast thou wrought?
  
  And beyond the Vortex, the churning black waters of the Void did disgorge the Dwellers in Eternal Shadow.\\ 
  And upon a horde of winged horrors, brandishing swords of ebon flame, they rode out from the gate. \\
  And a terible silence fell upon Kor-Avul-Thaa.
}





\subsubsection{Visualizing the \daemons}
\target{Visualizing Daemons}
Just \hr{Visualizing Archons}{like the \Archons}, the various \hr{Daemon}{\daemons} or classes of \daemons{} and gods feel differently. 

\lyricslimyaael{427806}{%
  \textbf{3) Beckon the grotesque.} 
  
  I've wondered lately why descriptive passages on magic in so many fantasy novels do nothing for me anymore. 
  There are doubtless multiple reasons, but I think part of it is that, even when the authors are writing about destructive magic or evil inhuman creatures like the Unseelie Court, they describe the effects of magic as beautiful, or pretty. 
  That tends in the direction of fluff if the author isn't careful. If she is, it'll still call up very similar pictures from a lot of other fantasy books.
  
  I've been thrilled and felt wonder from descriptions of the grotesque, however. 
  I still haven't managed to finish \emph{Perdido Street Station}, but the descriptions of New Crobuzon, especially the beetle-headed khepri, are a lot more intriguing than yet another scene of moonlit pools and silver wolves and unicorns. 
  And my candidate for most awe-inspiring magical talent I've read about this year isn't the king-and-the-land magic in \emph{The Fall of the Kings}, although it was beautifully described. 
  It's the ability to grow cocoons on one's palms and hatch insects from them that I read about in \emph{The Etched City}. 
  I'm also enjoying the three brothers nested in each other like Russian dolls from \emph{Someone Comes to Town, Someone Leaves Town}, although that's been slow reading for other reasons.

  Many fluffy magical systems that blur into each other across fantasy books share common touchstones\dash\quo{beautiful} animals like horses and wolves, images of light from moon and sun, natural elements like water and fire that we've been trained to admire, brilliant \colours. Replacing even a few of those touchstones may lead to the sense of the strange, the weird, the alienness that we don't understand and recoil from. Insects, disease, filth, blood, and mutated and decaying bodies are much less often terms of fluffy magic. Try beckoning the grotesque into your magical system and see what it does.
}















\section{Chaos magic}
\target{Chaos magic}




\subsection{Aenigmata and Gnosis}
\target{Aenigma}
\target{Aenigmata}
\target{Gnosis}
\index{Aenigma}
\index{Gnosis}
Chaos magic is based on certain \emph{Aenigmata} (singular \emph{Aenigma}). 
They are a kind of cosmic mysteries or \quo{equations} that must be understood and solved. 
The solution to an Aenigma is a \emph{Gnosis} (plural \emph{Gnoses}). 

To possess the Gnosis of an Aenigma is not a binary thing, but something gradual. 
One explores an Aenigma and gradually gains more Gnosis, but a perfect Gnosis is feasible only in the case of the simpler Aenigmata. 
Under the reigning axioms of Aenigma theory: 
\begin{itemize}
  \item \ldots{} certain Aenigmata \emph{have} no Gnosis. 
  \item \ldots{} certain Aenigmata have a Gnosis, but this Gnosis can provably never be known to perfection. 
  \item \ldots{} and many more Aenigmata are suspected of being unknowable. 
\end{itemize}

A Gnosis is something deep and intuitive, and can be told and taught only with difficulty. 
They are not simply bits of trivia but skills that must be studied, experienced and learned. 

\quo{Aenigma} is a broad term covering many things. 
Among other things, every living creature has a particular, unique Aenigma associated with it. 
This Aenigma can be compared to a DNA code (and indeed, the DNA code is one aspect of the Aenigma). 
To possess the Gnosis of a being's Aenigma is to have occult power over that creature. 

\index{name!true name (Gnosis)}
\index{true name}
The Gnosis of a person is sometimes poetically referred to as the person's \quo{true name}. 

Any Chaos mage worth his salt will have it as one of his life's goals to attain the Gnosis of his own Aenigma. 
Self-Gnosis is the key to much power. 

Similarly, one must possess at least some measure of Gnosis of a god or \hr{Daemon}{\daemons} in order to \hr{Chaos invocation}{invoke it}. 






\subsection{Cost}
A Chaos sorcerer must sacrifice blood\dash your own \emph{and} that of others!\dash to the \hr{Daemon}{\daemons}. See \hr{Telderain}{\Telderain} for an example. Much Chaos magic involves explicit pacts with dark powers, such as \hr{Rissit}{\RissitNechsain} or the \hs{Sentinels} and their masters.

If the sorcerer is not skilled or strong-willed enough to handle it, Chaos magic can take a terrible toll on his body and mind, leaving his mind maddened and his body twisted, deformed and sickened\dash rotting from the inside out due to the influence of unnatural, otherworldly power (\quo{unnatural} in the sense that it is alien to the world of the Shroud). Compare to Hannan Mosag and his K'risnan in \cite{StevenEriksonIanCameronEsslemont:MalazanBookoftheFallen}. Also, the disgusting priests in the movie \emph{300}.

Chaos tends to twist the mind into something more chaotic and bestial. It brings forth primal urges of lust, greed and aggression and encourages the mage to act on these instincts. As a result, a Chaos mage tends to become warped and mad with time. Most \dragons{} exhibit these traits due to practicing Chaos magic; indeed, these traits are so pervasive among their race that it is considered their natural state (although a \dragon{} might turn out differently if it refrained from using Chaos magic all the time), and people whose minds are twisted through Chaos magic are said to acquire \draconic{} minds. 





\subsubsection{Sacrifices}
A weak Chaos mage can enable himself to cast more powerful spells that he otherwise could by making sacrifices not just to the gods, but directly to the \hr{Daemon}{\daemons}. 
It does not have to be something of his own. 
You can feed the \daemons{} with the blood, flesh or even souls of others. 

This is not actual bargaining. 
Remember, the \daemons{} are barely sentient and certainly not intelligent. 
The real issue is that the mage must pay of his own energy to invoke the \daemons. 
You need not pay everything up front. 
You can have the \daemons{} do their task and then collect the rest of their reward from your body afterwards. 
But if you have something else to offer, you can compel the \daemons{} to feed on that instead, leaving your own body energy untapped. 
This means you can call in more or bigger \daemons{} than you could otherwise afford. 

If you call in too many \daemons, expecting to have something to offer them, and it turns out you cannot pay up, then they will take what they are owed from your body. 
This can kill the mage. 









\subsection{Gods and \daemons}
Chaos magic was based on pacts with gods. 
A mage would often learn spells and be given energy from one or more patron gods. 
In return, the mage would have to perform services or pay tribute (in the form of sacrifices) to the gods. 

The gods could bestow power and energy to the mage, or they could just grant knowledge.
Gods often preferred the latter, because it was cheaper for them.
(A god would not want to lose out on a deal by giving more energy to the mage than it got back.)
A gift of knowledge had a drawback, though: 
It was harder for the god to revoke. 
The mage could, in principle, take the knowledge and run away without repaying the god. 









\subsubsection{\Daemons}
\target{Daemon}
The \daemons were mythical \quo{creatures} that chaos mages summoned in order to cast magic. 

In reality, the \daemons were not true creatures.
They were insubstantial machines, robots or programs created by the \xss to serve them as tools. 

There were endless hordes/masses/swarms of the mindless, nameless \daemons. 

See also the sections on \hr{Daemon invocation visuals}{\daemon invocation visuals} and \hr{Visualizing Daemons}{visualizing \daemons}. 

I should have a big pantheon of \daemons{} from which to draw. 







\subsection{How to cast it}
Chaos sorcerers must invoke the names of \hr{Daemon}{\daemons} and dark gods when casting their magic. The amount of invocation depends on the power of the spell/entity compared to that of the sorcerer. An immensely powerful sorcerer, such as one of the \shaeeroths, can perform feats of tremendous magic without a word, and need invoke aloud only the greatest of \hs{cosmic gods} and \hr{XS}{\xzaishanns}. A lesser sorcerer will need to call out the names of lesser \daemons{} as well. 

\target{Invoking Daemons}
Every Chaos spell should invoke one or more \daemons{}. 
Sometimes \xss, sometimes just lesser \daemons. 
And it should invoke \Sethicus and \Tiamat. 

Chaos magic, compared to Vaimon magic, relies much more on words. 
A spell is a sentence in the \Draconic{} tongue which invokes the \daemon{} or \daemons{} you want and (superficially) describes the effect. 
Chaos magic also uses occult symbols. 
For simple spells, the symbol is usually already encarved on some item, such as a staff or amulet, or as a tattoo, and the item is used in the spellcasting. 
In more complex (and longer) spells, symbols are drawn while casting the spell, on the ground, on paper or on one's own body. 
Chaos magic spells are usually cast by a single caster, although they may be long, requiring a complex ritual. 




\subsubsection{Invocations}
\target{Invoking the XS}
\target{Chaos invocation}
Chaos sorcerers invoked \hr{Daemon}{\daemons} and gods to cast spells.
\quo{Gods} might be \hr{Taorthae}{\taorthae}, \dragons or even \hr{XS}{\xss} (\hr{Primordial}{\Primordials}). 

Some \hs{Gnosis} of a god or \daemon{} was required before you could invoke it. 
(For a \rethyax, this Gnosis came in the form of \hr{Arcanum}{\arcana}.)

\Tiamat{} and the \firstgendragons{} (who are \hr{Elder Dragons worshipped}{dead but still worshipped}) are invoked in Chaos magic. 
She was the one who made the original pacts with the \xss, so she must be invoked when one seeks aid from the \xss. 

See also the sections on \hr{Daemon invocation visuals}{\daemon invocation visuals} and \hr{Visualizing Daemons}{visualizing \daemons}. 

Read the section about the \hr{XS}{\xss}. 





\subsubsection{Magic types and \xss}
Spells and types of magic were associated with specific \xss.

\begin{description}
  \item[Fire magic:]
    \target{Ruin Satha fire magic}
    For example, fire magic was {associated with \RuinSatha}. 
    They would invoke him (with his descriptions and titles) whenever they had to cast such magic. 
    
    The fire magic of \RuinSatha was some of the most immediately destructive magic the \dragons had at their disposal.
    It created huge-ass flames of all sorts of \colours (depending on the specific spell and the effect you wanted). 
    These flames were not just physical, but spiritual as well. 
    
    Whenever such magic is cast, remember to look up the visuals associated with \hr{Curses of destruction visuals}{destructive magic}.
    
    \RuinSatha represented change through creation and destruction. 
    His flames could sometimes cause mutation, or cause new beings to spring to life. 
    Compare to the Chaos god Tzeentch from \cite{RPG:Warhammer40000}. 
    Also a bit like the All-Spark from \cite{Movie:Transformers2007}.
\end{description}









\subsection{Initiation}
\target{Chaos mage initiation}
Maybe have scenes where Chaos mages are initiated. 
Perhaps \hr{Moro Cornel}{Moro \Cornel}.

\lyricsbs{Hate Eternal}{Rising Legions of Black}{
  Mark thy masters wrath. \\
  The scrolls now entangled.\\
  I offer my blood in chants of disgust.\\
  Enter a dimension of hate.\\
  Rejoice in flaming circles.\\
  Temptation of ones blinding faith.
  
  Rings of fire engulfing the earth, now brought to a blaze.\\
  Bound by the shadows that dwell from within.\\
  Awaken the beasts now speaking in tongues, invoking despair.\\
  I am the grace of the rising legions of black.
}

\lyricsbs{Emperor}{Alsvartr}{
  Hark, O'Nightspirit,\\
  father of my dark self.\\
  From within this realm, wherein Thou dwelleth,\\
  by this lake of blood, from which we feed to breed,\\
  I call silently from Thy presence, as I lay this oath.
  
  May this night carry my will\\
  and may these old mountains\\
  forever remember this night.\\
  May the forest whisper my name\\
  and may the storm bring these words\\
  to the end of all worlds.
  
  May the wise moon be my witness\\
  as I swear on my \honour,\\
  in respect of my pride and darkness itself,\\
  that I shall rule by the blackest wisdom.
  
  O'Nightspirit!\\
  I am at one with thee.\\
  I am the eternal power.\\
  I am the Emperor.
}





\subsection{Spells}
\begin{gloss}
  \gitem{khestni}
  \target{khestni}
  A \hr{True Draconic}{\TrueDraconic} word of power meaning \quo{die}. 
  As a spell, it kills. 
  \word{Khestni}, however, is not necessarily an instant-kill-spell, except when used on foes far weaker than the caster. 
  It is like a dagger's thrust: 
  Easy to defend against if you are prepared, but quick and easy, and very deadly in the right circumstances. 
  
  \word{Khestni} is not effective if used several times in a row against the same foe. 
  For some reason. 
  So just spamming \word{khestni} is not smart. 
\end{gloss}





\subsubsection{Summoning}
There were spells that could summon a \malgryph or a \salamander. 
The \salamander was the ultimate manifestation of \hr{Ruin Satha fire magic}{\draconian fire magic}. 















\section{\Matrix Theory}







\subsection{Astrology}
\target{Astrology}
\target{astrology}
\index{astrology}
\Matrices{} are connected with {astrology}. 
See, the stars have mystic power. 
They are alive and some of them are aware and intelligent. 
They are a kind of \hs{cosmic gods}. 
They can provide information if one knows how to decipher it. 

The common folk tend to believe that such astrological insight is gained by measuring the \emph{positions} and \emph{movements} of stars and other celestial bodies. 
Many wannabe astrologers draw up horoscopes in this manner. 
But they are frauds or fools. 
The positions of stars and planets tell nothing mystical. 
See, the stars in the \Miith{} universe are real stars. 
They are many light years away and move according to regular astronomical laws in perfectly predictable orbits that have nothing to do with what happens on \Miith. 

Instead, to gain esoteric insight one must study the \emph{\colours} of the stars. 
And not just their physical light that is visible to the ordinary eyes. 
That light, after all, is many years old (if not thousands of years), since the stars are so far away, so it tells nothing about what is happening \emph{today}. 
No, it is the stars' \quo{spiritual light} that is important. 
Specifically, the spiritual \colours and subtle psionic signals (visualized as blinking or flickering) that can only be seen with the mind's eye by one who can see through the Shroud (at least to some extent). 
This \quo{light} is only visible to the adept who has prepared himself for the task by entering a special trance, induced by spells, meditation and/or drugs. 
In this state, the astrologer is able to reach out into the \hs{Beyond} and pick up signal from the stars that are being sent out \emph{now}, faster than the speed of light. 
And it is these mystic signals that tell what the stars are thinking. 

The most common use of astrology is to gauge the status quo of the \matrices, their balance of power. 
Each \matrix{} is represented by a constellation in the sky. 
These constellations are given traditional names that symbolically refer to their associated \matrix. 
The \matrices{} and constellations are identified to such an extent that they are virtually synonymous in daily speech. 
The constellations only \emph{very} vaguely resemble the things they are named after; the names are poetic and traditional. 

By observing the subtle mystical signals the stars send out, one can gain a certain understanding of how the \matrices{} stand: 
Which \vertices{} are aligned with which \matrices, how much power does each \matrix{} have, etc. 
It is difficult to interpret, though, since \quo{balance of power} is not an easily quantifiable (meta)physical property. 

When an astrologer looks at the sky with his extended psychic senses, the stars appear to blaze with a mighty fire\dash many-coloured, and at times cold-looking.

\target{Vertices in the sky}
Unlike physical constellations, the mystic constellations \emph{move} in the sky. 
They can shrink, grow, twist and writhe. 
The same goes for individual \vertices.
They have a position in the sky, indicating their position in one or more \matrices, but they can move around.  

\target{intersecting}
When two \matrices{} (or \vertices) combine their powers, they are said to \emph{intersect}.
But only if they work together in a way that is metaphysically measurable, such as by pooling their magical power for a joint spell. 





\subsubsection{Moons}
The \hs{Moons} have mystical power and are inhabited by sinister gods and creatures, such as the \hr{Moon-Wolves}\moonwolves.









\subsection{\Matrix}
A single \vertex is rarely powerful enough to affect dramatic change on a great scale. 
So \vertices{} band together for power. 
They form a \matrixx, an organized group of \vertices{} bound together. 

A \matrix{} is actually more than just a bunch of \vertices. 
A \matrix{} is a front-end for a \dweomer{} (or more than one \dweomer). 
It is an occult infrastructure that a whole group (or even a whole civilization) can draw on. 
It strengthens their magic and thus makes the group greater than the sum of its parts. 

A \matrix{} may contain smaller sub-\matrices. 

A \matrixx{} can \quo{eclipse} another \matrixx. 





\subsubsection{\Apex}
\target{Apex}
\index{\apex}
To be truly effective, a \matrixx{} must have an \apex, a leader \vertex. An \apex{} can be chosen unanimously, or several candidates may fight over the title. It is possible for a \vertex{} to force himself into a position of \apex, binding the rest of the \matrixx{} to his will (for a time, at least). 

\target{Cardinal point}
\index{\cardinalpoint}
Below the \apex, the important, pivotal \vertices{} of a matrix are called \cardinalpoints. 
For maximum effect, the \cardinalpoints must have the correct number and be correctly arranged (in accordance with \hs{occult geometry} and the \hr{Matrix formation}{formation of the \matrix}). 





\subsubsection{Circles of mages}
\target{Circles are Matrices}
A \hs{circle of mages} is {a small, temporary \matrix}. 
As such, it obeys (or should obey) the rules of \hr{Matrix formation}{\matrix formations}, \hs{occult geometry} and numerology. 





\subsubsection{Formations}
\target{Matrix formation}
A \matrix can assume a number of different formations. 
Each formation has different properties, in accordance with \hs{occult geometry} and numerology. 

Each formation has a different optimal number and arrangement of \hr{Cardinal point}{\cardinalpoints}. 





\subsubsection{List of \matrices}
The \matrices{} include (remember to give them all shapes and names!):

\begin{gloss}
  \gitemthe{Midnight Bat} 
    \target{Midnight Bat}
    \hr{Mystraacht Matrix}{\Mystraacht{} \matrix}. 
    Sub-\matrix{} of the Paths of Ice. 
    
    \Apex: 
    Once \Zachirah. 
    Then none, for a long while. 
    Ultimately Ramiel. 
    
    \CardinalPoints: 
    \Nathrach, \Shiaraid, Ramiel, \Dasteron, \Kishiel. 
  
  \gitemthe{Paths of Ice} 
    The \banelords.
    The constellation gives associations of a thin sheet of ice (like the one covering a lake) that is slowly but surely cracking and breaking up. 
    The \quo{paths} are these cracks. 
    
    The Paths of Ice contain the Silver-Shining Rose and the other dynasty \matrices{} as sub-\matrices. 
    
    \Apex: 
    \Daggerrain. 
    
    \CardinalPoints: All \banelords. 
  
  \gitemthe{Pyre} 
    \target{Pyre}
    The \dragons{} and their \xs-born power. 
    
    \Apex: 
    Once \TyarithXserasshana. 
    Later \Vizsherioch. 
    \Nexagglachel{} and \Secherdamon{} both strove for this position but never achieved it. 
    
    \index{Dagger, the}%
    There is also a \quo{\hs{Dagger}} position. 
  
  \gitemthe{Silver-Shining Rose} \CiriathSepher. 
    Sub-\matrix{} of the Paths of Ice. 
    
    \Apex:
    \Azraid. 
    
    \CardinalPoints: 
    \Teshrial{} is a \cardinalpoint{} in a sub-\matrix.
  
  \gitemthe{\Malgryph}
    \target{Malgryph constellation}
    Contains the stars representing \hr{Zaz}{\Zaz and \Urzaz}. 
    Was \hr{Urizeth researches Malgryph constellation}{researched by \Urizeth}. 
  
  \gitemthe{Salamander}
  
  \gitemthe{Torch}
  
  \gitem{Unnamed} 
    \Iquinian{} \matrix.
    Comprised of the sixteen \sephiroth. 
    
    \Apex: 
    None. The \sephiroth{} are equal in status. 
  
  \gitem{Unnamed} 
    Imetric \matrix. 
    
    \Apex: 
    Salacar. 
  
  \gitem{Unnamed} 
    Rissitic \matrix. 
    
    \Apex: 
    \HriistN{} (\Secherdamon). 
  
  \gitem{Unnamed} 
    \target{Vorcanth Matrix}
    \Vorcanth{} \matrix.
    Closely associated with \hs{Visha}. 
    
    \Apex: 
    Unknown. 
    Might be some great \vorcanth{} leader. 
\end{gloss}





\subsubsection{Visualizing a \matrixx}
\target{Visualizing a matrix}
In \hs{dreams} or using magic, it is possible to visualize a \matrixx{}. It can appear as a city. The \apex{} will be a castle or throne (possibly empty), the \vertices{} will be immense columns or statues, and the streets below will be filled with slaves, toiling as directed by the \vertices, swarming about like ants. 










\subsection{\Nexus}
\target{Nexus}
\index{\nexus}
A physical place with great power, where it is possible to manipulate the Web of the Realms, is called a \nexus. 





\subsubsection{Ley lines}
Perhaps there are ley lines of \nexus{} energy crisscrossing the planet and the universe. These form the key threads of the Web of Realms. 

\target{Myths of creatures beneath the earth}
There are \hs{myths} of creatures dwelling beneath the earth. Supposedly, their movement and supernatural power affects the surface world in a Feng Shui kind of way. These creatures might be \trueophidian{} lords or even more ancient monsters. 









\subsection{Scientific}
\Matrix theory is more scientific than, say \hr{Sethican philosophy}{\Sethican mysticism}. 
It was more universally accepted because it really \emph{is} more true. 
And because was was less horrible to think about than \Sethicus's \xs mysticism. 

The \dragons and the \resphain had very similar conceptions of \matrix theory. 









\subsection{\Vertex}
\target{Matrix}
\index{\matrix}
The Shroud is the phenomenon that makes creatures unable to see the true universe, seeing only a narrow slice of it. Thus, the Shroud is the force that creates and separates the illusory worlds which mortals inhabit. Only mighty creatures, like \dragons, \banes{} and \resphain, and those learned in occult lore, even know of the Shroud and can see it.

\target{Vertex}
\index{\vertex}
People with the power to see into and influence the Shroud are called \vertices. 
Ordinary people cannot see the Shroud and are slaves of it. 
Even those mages who know it exists must usually use magic to even see it, and use elaborate spells to affect it. 
One becomes a \vertex{} through force of will, through a refusal to believe in the illusion that one's brain tries to impose. 
Some \vertices{} are feeble and not even mages (such as Lica, the \quo{clairvoyant} girl introduced in \emph{\LicaBook}). 

\Vertices{} are those individuals who somehow reach outside the Shroud and as such are able to understand and influence the world, to some extent. 

Normal people who are not vertices are sometimes called \quo{specks} (as in \quo{specks of dust}). 





\subsubsection{Detecting \vertices}
\target{Detecting Vertices}
A person skilled in \matrix{} theory, or simply a mage or telepath, can detect \vertices{} when they come physically near. 
They emit \quo{vibrations} through the Web of Realms, like tremours that can be felt. 
These tremours are stronger for more powerful \vertices. 
And they are especially obvious within the \hs{Shrouded Realms} because these Realms are normally rigid and free of strong \vertex{} influence (whereas the Immortal Realms have \vertices{} all over the place). 
So when a mammoth \vertex{} such as \QuessanthIshnaruchaefir{} suddenly appears in \Azmith, it can be felt for miles away. 





\subsubsection{\Vertices vs. mages}
\target{Vertices vs. mages}
Not all mages are \vertices, although some are. 
A \vertex{} is (potentially) much more powerful than a mage, just like a king is more powerful than a sauropod. 









\subsection{Zenith and Nadir}
\target{Zenith}
\target{Nadir}
\index{Zenith}
\index{Nadir}
The \quo{Zenith} is the metaphysical \quo{place} where the \hr{Heart}{Heart of \Miith} waits. 
All \matrices{} strive to reach the Zenith and control the Heart. 

A Nadir is a \quo{low point} for a \matrix{}, constellation or \vertex. 
It is a state where the \matrix{} is abnormally weak. 
A \matrix{} can be forced into a Nadir by external forces (usually temporarily). 
Some \matrices{} have natural periods of Nadir for astrological or other reasons. 















\section{Psionics}
\target{psionics}
\target{psionic}
Psionics is the art of reaching out with your \hs{extended soul} into the \hs{Beyond} and thus detect or affect other things in the world. 

The term \quo{psionics} is only known and used by a few researchers. 
Most people know only the concept \quo{\hs{magic}}. 

One of the possible uses of psionics is to contact otherworldly entities and enlist their aid. 
This manifests in \hs{invocations} and \hs{orisons}, which form the basis of most magic. 
So magic is based on psionics. 

Some people (including certain \hr{Ophidians}{\ophidians}) revile \hs{sorcery} (summoning-based magic) as evil and use only psionics. 









\subsection{Telepathy}
\target{Telepathy}
\target{telepathy}
A few creatures are naturally telepathic. 
This includes \hr{Nycan}{\nycans} and the rare \hr{Nycaneer}{\nycaneer} \scathae. 
Other creatures (immortal and mortal alike) can learn telepathy, but it is not a natural part of them nor their culture. 

Telepathy is a kind of \hs{psionics}. 
It allows you to communicate silently, mind-to-mind at a distance. 

The distance is still limited. 
The best telepaths can manage a range of a few kilometres. 

If the people communicating have no shared language, only pictures, sensations and vague emotions can be transmitted. This is slow and difficult to make sense of. 

Sending thoughts is easy. 
Reading them is much harder. 
Even a skilled telepath can only read surface thoughts and feelings, and only vaguely. 
Smooth communication requires that both parts be telepaths. 

The Imetrians have the most advanced telepathy. 
The \banes{} also have it, but they don't teach it to any but their trusted servitors. 















\section{Schools of magic}
Every self-respecting society has mages of some kind. But magic is not just magic. There are many different approaches to magic. Pretty much every civilization has its own magic theory, its own tradition of magic. 

But why does this diversity persist? Why do mages from all over the world not come together, exchange knowledge, do some research and find out which theory of magic is correct? Well, for several reasons: 

\begin{enumerate}
  \item Science is difficult. 
    Each theory has its pros and cons, and it is not always possible to \quo{combine} them. 
  \item Pride. 
    Often, scientists are very proud of their own theory, the one they were brought up with or may have helped develop. People want their own view to be correct. This makes them inflexible, irrational and unwilling to compromise and acknowledge possible flaws in their theory. 
  \item Distrust. 
    \Miith{} is not a friendly place where everyone gets along. Wars are fought between nations every day, and even in peace time there is the threat of war, old hatreds, racism, feelings of cultural superiority, and the fear of the unknown, all of which inhibit communication and the free exchange of scientific knowledge. 
  \item Secrecy. 
    Apart from the irrational \quo{distrust} above, many cultures have dreams of world domination. These are unwilling to share their knowledge, because knowledge is power, and they do not want to share power. 
  \item Religion. 
    Some magic is directly connected to some religion and weaves ties between the spellcaster and the gods of that pantheon. Typically, no one wants to be dependent upon the gods of another religion. And even in the cases where the magic is directly connected to specific gods, the theory is often intertwined with the dogmas and world view of some religion, which nonbelievers will be hesitant to accept. 
  \item Complexity. 
    The magical Universe is vast, dark and mysterious. At TL3, no civilization is close to understanding more than the tiniest fraction of it. The truth of how magic works in the Universe is perhaps unknowable, and at any rate far too complex to have been even glimpsed. But magic is a strange thing and manifests itself in many forms. 
  \item Fear. 
    This is a consequence of the fear of the unknown. The Universe of magic is vast and dark, and much magic will be horrible and frightening to the uninitiated. There is, of course, the very real risk that a spell or ritual might go horribly wrong and cause an explosion, transform the caster into an undead monster or accidentally conjure some demon from an alien world, but even apart from that, the supernatural tends to instill people with irrational fear and loathing unless they are very strong-willed or have been brought up to accept it. Therefore, most people will accept the magic of their own culture, but view most foreign magic as something foul and evil. 
\end{enumerate}

The result of all these factors is that every culture has its own magic, its own methods for casting spells. 

Schools of magic on \Miith{} include: 





\begin{gloss}
  
  
  
  
  \begin{comment}
    \paragraph{\Aryoth magic}
  \end{comment}
  \gitem{\Aryoth magic}
  \target{Aryoth magic}
  The \ps{\aryothim} own magic style was animistic, drawing on \ps{\Miith} native, animalistic forces of nature. 
  
  \target{Aryothim hate sorcery}
  They had a cultural hate of sorcery, because \hr{Aryothim enslaved by sorcery}{they had once been enslaved by sorcery}. 
  So they eschewed sorcery and used psionic/animist magic. 
  They also practiced the natural sciences and \hr{Aryoth inventors}{invented much technology}. 
  
  But as the \hs{Heart weakened}, the \aryoth style of magic also weakened. 
  This was one of the reasons why the \aryoth{} people waned and dwindled. 
  
  Some of them went over to other schools of magic. 
  
  
  
  
  
  
  
  
  
  
  \begin{comment}
    \paragraph{\Bane magic}
  \end{comment}
  \gitem{\Bane magic}
  The original \bane{} magic drew its power directly from \Erebos{}. But since the conduit (vortex) to \Erebos{} has been sealed, they needed to find a new \hr{Dweomer}{\dweomer}. Aided by the \nephilic{} sorcerer \Semiza, they \hr{Origin of Qliphoth}{created the \Qliphoth}, who serve as conduits to \Erebos{} through which magical energy can be channelled. 
  
  (How were the \Qliphoth{} created? 
  From sacrificed \banelords{} or from enslaved \pdaemons{}/\mdaemons{}?)
  
  \Bane/\nieur{} magic sometimes works like defiler magic (from \emph{Dungeons and Dragons: Dark Sun}), sucking the life energy out of nearby creatures, leaving plants withered and dead. This is because the \ps{\banes}{} power is based on \hr{Entropy}{death, stagnation and parasitism}. is based on \hr{Entropy}{death, stagnation and parasitism}. 
  
  
  
  
  
  
  
  
  
  
  \begin{comment}
    \paragraph{Chaos magic}
  \end{comment}
  \gitem{\hs{Chaos magic}}
    As used by the \dragons.
  
  
  
  
  
  
  
  
  
  
  \begin{comment}
    \paragraph{Imetric magic}
  \end{comment}
  \gitem{\hs{Imetric magic}}
  
  
  
  
  
  
  
  
  
  
  \begin{comment}\paragraph{\QuilJaaran magic}\end{comment}
  \gitem{\QuilJaaran{} magic}
  \target{QJ magic}
  \QuilJaaran{} magic was based on occult logic and philosophy, originally learned from some cosmic gods. 
  Or perhaps from the \voyagers. 
  
  \hs{Imetric magic} was very much inspired by the \quiljaaran{} tradition.
  
  
  
  
  
  
  
  
  
  
  \begin{comment}
    \paragraph{\Rethyactic magic}
  \end{comment}
  \gitem{\hr{Rethyax magic}{\Rethyactic magic}}
  
  
  
  
  
  
  
  
  
  
  \begin{comment}
    \paragraph{Rissitic magic}
  \end{comment}
  \gitem{\hr{Rissitic magic}{Rissitic magic}}
    The theory of the Body, Spirit and Shadow Worlds. 
  
  
  
  
  
  
  
  
  
  
  \begin{comment}
    \paragraph{Shamanistic magic}
  \end{comment}
  \gitem{\hs{Shamanistic magic}}
  
  
  
  
  
  
  
  
  
  
  \begin{comment}
    \paragraph{Vaimon magic}
  \end{comment}
  \gitem{\hs{Vaimon magic}}
    The theory of \hr{Iquin}{\iquin} and \hr{Itzach}{\itzach}.
 
\end{gloss}
 

























\chapter{The Realms}
This is about individual Realms. 
For general information on how the Realms work, see the section on \hs{Realms}. 















\section{\Erebos}
\target{Erebos}
\index{\Erebos}
Deep in the vastness of space, many thousand light years from \Miith{}, lies the sinister world of \Erebos{}. 
Thousands of years before the awakening of the \dragons{}, \Erebos{} had been visited by the \voyagers{}. 
There they \hr{Banes are created}{created the \banes}. 

Whenever I write about \Erebos, be sure to also read about \hr{Nyx}{\Nyx}.









\subsection{Denizens}
Denizens of \Erebos included:
\begin{itemize}
  \item \Banes.
  \item \Flyingpolyps.
  \item \Noggyaleth (maybe).
  \item \Screamers.
  \item \Umbrae.
  \item Weavers.
\end{itemize}










\subsection{Scenery}
On \Erebos, the sky is pitch black, but down in the depths beneath the twisted spires and towers, that stand thousands of metres tall, there is a dimly luminescent mist down in the depths, which sometimes lights up the place a little bit with its ghastly, spectral glow. 

In the brooding darkness of the skies dwell the \hr{Umbra}{\umbrae} and other horrors. 

Perhaps there are flying fortresses drifting around up there, remnants of the \hr{Voyager}{\voyagers} or an extinct \erebean{} civilization that was destroyed by the \banes. 

\lyricsbible{Job 10:21--22}{
  Before I go [whence] I shall not return, [even] to the land of darkness and the shadow of death; \\
  A land of darkness, as darkness [itself; and] of the shadow of death, without any order, and [where] the light [is] as darkness.
}





\subsubsection{Burrowers beneath}
In the mystic gloom of the deep abyss, you can sometimes hear the writhing of the horrible \hr{Ghobal}{\ghobaleth} and/or \hr{Flying polyps}{\flyingpolyps}, or feel the tremours of their passing and their burrowing.









\subsection{History}
\Erebos{} was not always a Realm of darkness. 
Once it had a sun that shone as brightly as any other. 
But for tens of thousands of years the \banes{} have sucked the life out of their \hr{Dweomer}{\dweomer}, the Heart of Erebos. 
And now their sun has turned black. 









\subsection{The Heart of \Erebos}
\target{Erebos undead}
Perhaps \Erebos{} is, in some sense, an undead planet, a vampire and scavenger.

The Heart of \Erebos{} still beats, but the blood it pumps is ashen and all but lifeless. That is why the \banes{} invade \Miith{}: They seek the Heart of \Miith{} so that they may live on. 















\section{\KaiLeng, the Underworld}
\target{Kai Leng}
\target{Kai-Leng}
\target{KaiLeng}
\index{\KaiLeng}
\KaiLeng{} was the underworld of \Miith{}. 
It was a \hs{Chthonic Realm}, accessible from \Azmith{} through deep tunnels. 









\subsection{Entrances}
\KaiLeng was accessible from \Azmith{} through deep tunnels. 
One system of such tunnels lay underneath Mount \hr{Shrun}{\Shrun} near \hr{Yormis}{\Yormis}. 









\subsection{Inhabitants}
In \KaiLeng dwell monsters and Great Old Ones, perhaps \xss. 

Perhaps the Tyrant Worms. 
Perhaps these are somehow related to the \hr{Ghobal}{\ghobaleth}. 
Perhaps they were engineered by the \voyagers, like the \banes{} were. 
Or perhaps the Tyrant Worms were indigenous to \Miith{}, and the \voyagers{} created cheap copies of them on \Erebos, which evolved into \ghobaleth. 

There also dwell \hr{Troglodyte}{\troglodytes} that worship these monsters. 

\KaiLeng{} contains relics of the \psp{\voyagers} civilization, and of the \xss. 





\subsubsection{Gods}
Gods that dwelt in \KaiLeng include \hr{Yolbaoth}{\Yolbaoth} and \hr{Ubloth}{\Ubloth}. 









\subsection{Someone explores \KaiLeng}
Have a scene where someone explores \KaiLeng. 

\lyricsbalsagoth{Invocations Beyond the Outer-World Night}{
  These darkling subterrene dominions, astir with strange and terrible beings, sired by entities whose genesis was far beyond the nighted void of our own outer-world! \\
  The legacy of the First Ones, spawn of the Mera!
}

Someone delves deep and finds ancient \voyager{} artifacts, connected to the origin of all life, and of the \hr{Heart}{Heart of \Miith}, and the \hs{Sun}. 

\lyricsbalsagoth{Invocations Beyond the Outer-World Night}{
  Behold, a vast plasma-fueled crystalline illuminatory orb\ldots{} \\
  a vril-sun rising!\\
  And marvel at the colossal terra-forming machines of the First Ones!
  
  Far, far beneath the surface of this coruscating sphere, at the very heart of our mysterious globe, lies the true path to man's dark destiny beyond the heavens\ldots{}
}















\section{\Machai}
\target{Machai}
\index{\Machai}
\Machai was originally a Realm of \Miith. 
After the \hr{Shrouding}{\Shrouding}, \Machai was torn apart and became split into three Realms. 

\Machai was always closer to the Realm of the \xss than the rest of \Miith. 
So it was inhabited by strange and exotic creatures and monsters. 

After (or before) the \hs{Draconian Supremacy} began, \Tiamat and many of her fellow \dragons moved into \Machai and made it their base. 

After the \secondbanewar and the \Shrouding, the three fragments of Machai came to be considered \quo{Immortal Realms}. 
They were dominated by the Sentinels. 









\subsubsection{Fantastic landscape}
Some \hs{myths} paint \Machai{} as Hell, but it actually isn't. 
It is a diverse and exotic Realm, terrible at times, but also beautiful and fascinating. 
It features such things as:

\begin{itemize}
  \item Rivers that run with liquid metal.
  \item Flying trees. These float at a constant altitude and are huge. You can build houses and towns in them. 
  \item Storms and geysers of fire. 
  \item Flying islands.
  \item Enourmous pyramids and domes. 
  \item Moon-like craters and rocks.
\end{itemize}

Compare to \authorbook{Clark Ashton Smith}{The Door to Saturn}, and the middle-illustrations of \bandalbum{Limbonic Art}{Ad Noctum - Dynasty of Death} and \bandalbum{Limbonic Art}{The Ultimate Death Worship}. 

\citeauthorbook[p.57--58]{RobertEHoward:TheMirrorsofTuzunThune}{Robert E. Howard}{%
  The Mirrors of Tuzun Thune%
}{
  Gray fogs obscured the vision, grea billows of mist, ever heaving and changing like the ghost of a great river; through these fogs Kull caught swift fleeting visions of horror and srtageness; beasts and men moved there and shapes neither men nor beasts; great exotic blossoms glowed through the grayness; tall tropic trees towered high over reeking swamps, where reptillian monsters wallowed and bellowed; the sky was ghastly with flying dragons and the restless seas rocked and roared and beat endlessly along with muddy beaches.
  Man was not, yet man was the dream of the gods and strange were the nightmare forms that glided through the noisombe junlges.
  Battle and onslaught were there, and frightful love.
  Death was there, for Life and Death go hand in hand. 
  Across the slimy beaches of the world sounded the bellowing of the monsters, and incredible shapes loomed through the steaming curtain of the incessaint rain.
}











\subsection{Fallen \draconic{} homeland}
\target{Dragonland}
\target{Fallen Dragonland}
There is a place that once used to be the capitol and homeland of the proud \draconic{} empire. 
Perhaps it's a whole Realm, or at least a pocket Realm (like \hr{Nyx}{\Nyx}). 

Now it lies in ruins, a dead necropolis haunted by monsters and the ghosts of dead \dragons. 
Perhaps there are monsters who once served the \dragons{} and now roam free as the \dragons{} have \hr{Dragons have forgotten}{forgotten how to command them}. 

There should be huge rocks that jut up like claws coming out of the earth. 
Compare to certain Limbonic Art album art pieces, or the realm of \Juujinkai{} in the anime \emph{\Urotsukidouji}. 





\subsubsection{\Dathka}
\target{Dathka}
Here we find \Dathka: 
A colossal, cyclopean palace/temple where \Tiamat{} and her \firstgendragons{} once dwelt. 
Now it serves as their necropolis. 

Compare to:

\begin{itemize}
  \item Kadath in \cite{HPLovecraft:TheDreamQuestofUnknownKadath}.
  \item R'lyeh in \cite{HPLovecraft:TheCallofCthulhu}. 
  \item The Antarctic city in \cite{HPLovecraft:AttheMountainsofMadness}. 
  \item Starvald Demelain in \cite{StevenErikson:ReapersGale}. 
\end{itemize}

\citebandsong{DeathspellOmega:SiMonumentumRequiresCircumspice}{%
  Deathspell Omega
}{
  Carnal Malefactor
}{
  Below the lid of a vast rounded monument\\
  Trickling of gristly vestiges and whacked hopes\\
  Enhanced by the horrible excess of fetid exhalation\\
  And uterine strangulation by the wreaths\\
  Of the herds astray, arid in despair, blessed\\
  With dilated flakes of fire, slowly wafting down\ldots{}
}

The tomb of \Sethicus: 

\citebandsong{Nile:AnnihilationoftheWicked}{Nile}{
  Annihilation of the Wicked
}{
  The Dominion of Seker.\\
  Barren Desert of Eternal Night.\\
  Shunned by Ra.\\
  Behind the Gate Aha-Neteru.\\
  The Wastelands of Seker.\\
  Eldest Lord of Impenetrable Blackness.\\
  Death God of Memphis.\\
  He of the Darkness and Decay of the Tomb.\\
  He of Rosetau, the Mouth of the Passage to the Underworld.\\
  Closely Guarded by Terrible Serpents\\
  who Careth Not for His Own Cult of Worshippers.
  
  Seker, Ancient and Dead,\\
  Primeval Master of the World Below,\\
  Remaineth Unwitnessed, Unseen.\\
  Hidden in His Secret Chamber.\\
  His Primitive Graven Image like as a Hawk-headed Man.\\
  Shrouded and Swathed in Tomb Wrappings.\\
  Standing Between a Pair of Wings \\
  which Issue Forth from the Back of a Monstrous Serpent,\\
  Having Two Heads, Having Two Necks \\
  and Whose Tail Terminates in a \human Skull.
}





\subsubsection{Knowledge}
There was much arcane knowledge hidden here in the \draconic/\ophidian ruins. 
Some brave seekers would come seeking it. 
But it was guarded. 

\lyricsbalsagoth{Unfettering the Hoary Sentinels of Karnak}{
  The Coptic papyrus states that, upon the walls of the pyramids and the temple were inscribed the mysteries of science, astronomy, geometry and physics; inscriptions of unknown peoples and lost civilizations whose lore was carved into the stone to preserve it from the ravages of the great deluge.\\
  The surviving knowledge of long forgotten antediluvian races!\\
  Aye, prudent Surid, heeding the warnings of his priests, erected certain repositories of long forgotten knowledge to withstand the first great flood, and then an all-consuming fire which was prophesied would come from the sky.\\
  Masoudi, in the tenth century, described automata; titanic guardians of stone and metal which were placed to guard the treasures and the entombed lore, and which were tasked to destroy all those deemed unworthy, all those who dared enter the chambers unbidden.\\
  I see them!\\
  The hoary sentinels of Karnak are unfettered!\\
  Rising from their sandy tombs to smite the intruder, the raider and the interloper with righteous fury!\\
  And what is this\ldots{} was there once a glimmer of life within the sightless stone eyes of the Theban guardian?\\
  Does the silent watcher of Giza even now descend from its granite dais to once more stalk the shifting sands on carven claws?
}





\subsubsection{Undead machines}
The \draconic{} necropolises were protected and maintained by \hs{undead machines}. 










\subsection{Blood Red Sun}
\target{Blood Red Sun of Machai}
In some parts of \Machai, the Sun looks monstrous and strange. 
It is not only huge, filling most of the sky; it is also a dark blood red \colour. It is dreadful to behold. 

Compare to the \hr{Black stars of Nyx}{black stars of \Nyx}. 

This is not all of \Machai, though. Perhaps this only applies in the \hr{Fallen Dragonland}{fallen \dragon-land}. 









\subsection{Living buildings}
\target{Living buildings of Machai}
Buildings in \Machai{} are colossal. But where \hr{City of Nyx}{the edifices of \Nyx} are tall, skeletal and dead, the structures on \Machai{} are bloated and squat. They resemble grotesque, overgrown living creatures\dash which is exactly what they are!

Perhaps the buildings are worshipped as mindless demigods by the lesser beings of \Machai. Kind of like the living Zerg buildings in the game \cite{VideoGame:Starcraft}. 

\lyricsbalsagoth{Beneath the Crimson Vaults of Cydonia}{
  Colossal shapes etched against the moons. \\
  Supine obeisance 'fore the mound.\\
  Accursed fiends hail the Slitherer. \\
  Abhorrent jaws drooling lunacy.
}









\subsection{Moons}
\Machai{} has its own evil moons.

\lyricsbalsagoth{Beneath the Crimson Vaults of Cydonia}{
  Phobos, Deimos! \\
  The moons' rays liquefied in these blood red pyramids.\\
  In the shrines of abomination, black tongues rapt with blasphemy.\\
  Chaosphere, watchtowers, genesis, Cydonia\ldots{}\\
  The Abyss yawns wide!
}









\subsection{Stephen Marley's description of Hell}
Here is a description of the Chinese Hell.

\lyricstitle{\authorbook{Stephen Marley}{Mortal Mask} p.247}{
  Red-robed Yen-Lo, Lord of the Sad Dead, took his ease on a red lacquer throne as he surveyed his ten regions of afflication.
  In the hell of desecrators and cannibals, yellow-eyed demons and black dogs drove the condemned souls into a river of boiling blood. 
  In the hell of dismemberment, the guilty hung in bits and pieces from hooks and chains.
  In the upside-down hell, the damned were suspended by their heels from the ceiling, their condition reflecting their scale of values in life.
}















\section{Mirage Asylum}
\target{Mirage Asylum}
The Mirage Asylum is \ps{\Ishnaruchaefir} citadel. 

\Ishnaruchaefir{} uses the Asylum to explore and research the far reaches of the cosmos. 
Remember, he has connections to the \hs{cosmic gods}, forces beyond the heavens. 
Unlike, for instance, \Secherdamon, who is obsessed with \Machai{} and the \xss. 









\subsection{Culture}





\subsubsection{\Dragons lie and think}
\target{Ishnaruchaefir lies thinking in Mirage Asylum}
When \Ishnaruchaefir and the other \dragons lie still and think in the Mirage Asylum, they reach out into the Beyond with their minds.
Here they not only think and speculate and theorize, but also practice their magic and do their research and experiments.
Often you can see storms of power around them as physical symptoms and manifestations of their otherworldly magical experiments.
The inhabitants of the Asylum see this as proof of the \dragons' terrible divinity and cower down and worship them.





\subsubsection{Language}
The Mirage Asylum was almost completely isolated from the rest of the world from its creation till its destruction. 
So naturally, the population developed their own language: 
Issikulik. 
Based on Greenlandic (Kalaallisut). 





\subsubsection{Population and daily life}
The Asylum has a population of between 10,000 and 50,000 people. 
Mostly \scathae. 
Like a big town or a very large castle. 

The Asylum is self-providing. 
There are some large \quo{gardens} where exotic plants grow and exotic animals feed. 
These are farmed by the civilian inhabitants. 
Compare them to the legendary \quo{Hanging Gardens of Babylon}. 

Only the four \dragons{} plus \hr{Criseis}{\Criseis} and \hr{Najarod}{\Najarod} possess the Gnosis required to open the portals that let them leave or enter the Asylum. 
And they seldom do. 

The \dragons{}, bored with farmed food, sometimes leave the Asylum to hunt. 









\subsection{History}





\subsubsection{Origin}
The Mirage Asylum was built on fragments of \Ishnaruchaefir's old tomb in which he had slept for a million years.

\target{Mirage Asylum became a hideout}
After the Shrouding, \Ishnaruchaefir tore the tomb loose and cast it out \hr{Mirage Asylum orbit}{to orbit \Miith}. 
He also used Shroud spells to hide it and thus create a permanent hideout. 
A side effect of this was that the place became twisted into its current form. 





\subsubsection{Destroyed}
At some point in the \thirdbanewar, the Asylym was \hr{Mirage Asylum destroyed}{breached and destroyed}. 

The Asylum was protected by the Shroud. 
It was only because the Shroud was \hs{unravelling} that it was possible for the \resphain{} to breach it. 









\subsection{Nature}





\subsubsection{Alive}
\target{Mirage Asylum lives}
The Mirage Asylum was a living structure (like \hr{Living machines}{living machines}). 
It looked sort of like a humongous conch. 

There were great crystalline growths that somehow grew from the living flesh of the Asylum. 
These crystals were vital to the economy and ecology of the place. 
Compare them to horns, nails or teeth: 
Dead mineral structures that grow from a living body. 





\subsubsection{Appearance}
The Mirage Asylum was like a half-open castle ruin or space hulk drifting afloat in the vast, empty void of space. 

Its architecture was insane, featuring all sorts of impossible geometry. 
It floated free in space with barely any walls. 

The Asylum had no central \quo{body}. 
It was composed entirely of winding stairways and bridges and branches and the occasional big bulging hub. 
It was sort of like a tree, but cyclic, often doubling back upon itself. 

Compare it to the Arcane Sanctuary from \emph{Diablo II}. 
Also compare to the cover of \cite{LimbonicArt:InAbhorrenceDementia}.
And the art of M.C. Escher.

See also the sections on \hr{Resphan architecture}{\resphan architecture} and \hs{dark ancient cities}. 





\subsubsection{Gravity}
The gravity on the Asylum always pointed \quo{down} towards the surface of the Asylum. 
There were plenty of places where one could go all the way around some narrow branch, but on all sides the gravity would point straight down. 

The Asylum had lower gravity than \Miith. 
This was not the gravity of the Asylum itself, but that of \Miith. 

\target{Mirage Asylum orbit}
The place orbited \Miith at a quite low orbit. 
\Miith was never visible from the Asylum. 
\Miith was always \quo{down} under the ground. 
Said \quo{ground} twisted around.
This was a result of the extremely strange geometry that abounded in the Asylum. 
The Asylum was \hr{Mirage Asylum became a hideout}{twisted into a hideout} for \Ishnaruchaefir. 
\Miith was not visible from the Asylum because it always lay \quo{underground}, and similarly, the Asylum was not visible from \Miith. 





\subsubsection{Shroud nature}
The Asylum was so twisted and insane because it was close to one of the sources of the Shroud itself. 
\Ishnaruchaefir was \hr{Ishnaruchaefir maintains the Shroud}{actively maintaining the Shroud} from within the Mirage Asylum. 
He had made it extra twisted. 





\subsubsection{Sky}
The Mirage Asylum drifted afloat in the vast, empty void of space. 
Stars and nebulae were visible above. 

What about the Sun?
Was there day and night?
From where did the place get its energy?









\subsection{Name and significance}
The name is taken from \bandsong{Limbonic Art}{Deathtrip to a Mirage Asylum}. 





\subsubsection{Reputation}
The Mirage Asylum had a reputation as a dark, formless throne of evil. 

\citeauthorbook[p.251]{HPLovecraft:TheBlackTomeofAlsophocus}{H. P. Lovecraft}{%
  The Black Tome of Alsophocus%
}{%
  \quo{%
    Nyarlathotep [\Ishnaruchaefir] rules in Sharnoth, beyond space and timeM in his gignatic ebony palace he awaits his second coming, served by his minions he broods and festers in blackest night.
    Let none meddle with spells and enchantements concerning him, for he is quick to trap the unwary.
    Let the ignorant beware, heed the \emph{Black Tome}, for terrible indeed is the wrath of Nyarlathotep.}
}





\subsubsection{Symbolic meaning of the name}
\target{Mirage Asylum symbolism}
\ps{\Ishnaruchaefir} choice of the name \quo{Mirage Asylum} is self-deprecating. 
\quo{Mirage} because it is an illusion of peace and isolation behin which he hides to avoid having to deal with his own self, his emotions and the world. 
And \quo{Asylum} because he sort of sees himself as a dangerous madman who must be sequestered and hidden away from the world, for both his sake and the world's. 
He has explored the universe to gain insight, but in a sense he has also fled out into the outer universe to escape having to face difficult questions and answers about himself. 
He likes to tell himself that he wants to learn the answers to the \hs{Aenigmata} of the universe in order to put his own Aenigma into a context and thus understand it better. 
But perhaps that is just an excuse, procrastination. 









\subsection{Politics}





\subsubsection{\Resphain resent it}
\target{Resphain resent Mirage Asylum}
\Ishnaruchaefir hid in his Mirage Asylum, a small secret Realm which only he knew how to find.
It was one of his most valuable resources.
It was how he has kept himself hidden and alive all these millennia.

The \resphain hated him for it. 
They saw it as cowardice. 
If he would only come out of his hole and fight like an honourable warrior, the \resphain could have dealt with him millennia ago (or so many felt). 

Of course he knew this, which was exactly why he kept his Asylum so secret and guarded.





\subsubsection{Threatened by horrors}
The Mirage Asylum is located in a place far removed from the beaten paths of \Miith, so the \resphain and other \Miithians cannot find it.
But this also means the Asylum is partially outside the protective \hs{Palisades}.
This means various mindless or intelligent creatures of the void can enter and attack them.

So the Asylum is regularly attacked by \hr{Horrors of the Void}{horrors of the void}.
Then the inhabitants need the magic that flows from the \dragons and their blood in order to repel the invaders and defend their home.
In especially bad cases, when large swarms of horrors attack, the \dragons themselves must arise and fight.
The inhabitants know this, so they worship the \dragons as their protectors.
And they willingly surrender some of their own as soul sacrifices to the \dragons when needed. 

The \dragons also feast on those horrors of the void when they can.
The depredations of the horrors is a necessary evil.

A pro of the Asylum's this is that they can grow exotic plants and things by drawing on some occult, dark energy streams that flow in from the void (but are blocked inside the Palisades), and combining these with the life-giving energy that flows from the Heart.









\subsection{Scenes}
Maybe have a scene where someone is dropped into the Asylum. 
They explore the place and lose tons of sanity points. 
It also gives some insight into how warped \ps{\Ishnaruchaefir} mind must be. 

















\section{Moons}
\target{Moons}
\target{moons}
\index{Moons}
Remember that the Moons have their own Realms. 
The \moonwolves{} live there, as do other things. 

The Moons play a large role in the \feud. There are \nexi{} there.

They have \hr{Astrology}{mystic astrological properties}. 

The Moon-Realms are \quo{sort of} part of the Realm of \Miith{}, although commonly not considered as such in everyday speech. 
They are still connected to the Heart of \Miith{}. 

Only at certain special times and in special places can one travel between \Azmith{} and the Moons. 







\subsection[Dun]{\Dun} 
\target{Dun}
\index{\Dun}
\index{Gray Moon}
\Dun, called the Gray Moon, was the larger of \Miith{}'s two moons. 

It was well-known in mythology and legend that the planets and moons were worlds with their own inhabitants. 
There might dwell special \demihuman and \demiscatha races on the two Lunar Realms.

\Dun{} was a world of its own, full of life. 
Unlike \hs{Visha}, which was mostly desolate wasteland. 

Remember that the moons had low gravity. 





\subsubsection{Astrology}
In \hs{astrology}, \Dun{} is considered mostly benevolent. 





\subsubsection{Astronomy}
\Dun{} is about the same size as Earth's Moon. It is closer to \Miith{} than Earth's Moon is to Earth, so it appears larger in the sky. \Dun{} has a dark gray \colour. 

\Dun{} is large enough to cause a solar eclipse. It is too large to cause the \squo{ring} effect known from solar eclipses on Earth, however. \Dun{} is larger than the Sun in the sky, so during an eclipse, the Sun is completely swallowed. 

\Dun{} itself is eclipsed when it passes behind the shadow of \Miith{}. This is called a \Dun{} eclipse.\index{\Dun{}!\Dun{} eclipse} 

\Dun{} circles \Miith{} once every $23.5$ days, roughly corresponding to a month of the \hr{Vaimon Calendar}{\VaimonCalendar}. 

\Dun and Visha were both completely full on the very first day of the year 88 years before the beginning of the \hr{Runger war}{Pelidor-Runger war}. 









\subsection{Visha}
\target{Visha}
\index{Visha}
\index{Pale Moon}
Visha, called the Pale Moon, is the smaller of \Miith{}'s two moons. 

It was well-known in mythology and legend that the planets and moons were worlds with their own inhabitants. 
There might dwell special \demihuman and \demiscatha races on the two Lunar Realms.

Where \Dun{} was teeming with life, Visha was mostly wasteland (in the \hs{Age of the Shroud} at least).
Visha's Realm was a mostly tranquil but eerie place. 
Haunted by terrible predators, the \vorcanths. 
It was also inhabited by ghosts, cruel \hs{cosmic gods} and perhaps the corpses of some \hr{Dead XS}{dead \xss}. 

Compare to \cite{HPLovecraft:TheDoomthatCametoSarnath}.

The \hr{Vorcanth}{\vorcanths} originally came from Visha, but their realm was destroyed and they were driven out. 
Compare to \authorbook{\HPLovecraft}{The Doom That Came to Sarnath}.  

Remember that the moons had low gravity. 

\citeauthorbook[p.175]{LinCarter:TheNecronomiconTheDeeTranslation}{Lin Carter}{
  The Necronomicon: The Dee Translation (part I.VII.III)
}{
  Aye, be thou warned, for in all such voyages and venturings of mind or soul or spirit there be very great and terrible dangers, by mortal men undreamt-of and unknown. 
  Beware then, lest thou penetrate too deeply into the blackest backward and depthless abysm of the womb of infinite time. 
  For beyond the very Beginning thereof, and on the Other Side thereof,there dwelleth That of which man suspecteth not; and there thou wilt find a strange and ominous Realm where hidden horrors lurk and naked Terror hunts unseen; which dim, uncanny bourn hath the seeming and the semblance of a pale, and grey, and indefinite shore, lapped by the sluggish waves of unmeasured and unthinkable Time.
  And it is eve there, in an awful Light that is beyond all darkness, amidst a profound Silence that shieketh beyond all sound, that \emph{They} slink and prowl in all their ghastliness, slavering with a loathsome and ana unspeakable hunger for all that is clean and whole and unsullied.
}

\lyricslimbonicart{Moon in the Scorpio}{
  A mirror blank ocean above me decoy.\\
  Superior forces that heal or destroy.\\
  Take me astray into the moonlight above
  through twilight eyes as a spectre shadow.
  
  In an atmosphere supreme\\
  forces dwell in domancy.\\
  The essence of its spirit is evil,\\
  as a curse upon thy name.
  
  Midnight is the shepherd of mysterious powers\\
  and moving shadows in the corner of the eye.\\
  Moon's blazing intuition\\
  contains what death requires.
  
  Behold the sky above \\
  when the moon is in the Scorpio.\\
  A cold bleak light
}





\subsubsection{Astrology}
Astrologically, Visha is considered malevolent and a bringer of ill omens. 

Visha is closely associated with the \hr{Vorcanth Matrix}{\vorcanth{} \matrix}. 





\subsubsection{Astronomy}
Visha is only half the diameter of \Dun{}. 
It is also farther away. 

Visha is not large enough to eclipse the Sun. It is much smaller, so when Visha moves in front of the Sun, it is visible as a dark hole in the Sun. This phenomenon is known as a \squo{Sunhole}\index{Sunhole}. 

Visha also sometimes casts a shadow on \Dun{}. This is called a \squo{\Dun{} hole}.\index{\Dun{}!\Dun{} hole}\index{Visha!\Dun{} hole} 

Visha itself is eclipsed when it passes behind the shadow of \Dun{} or \Miith{} iself. This is called a Visha eclipse\index{Visha!Visha eclipse}. 

Visha circles \Miith{} once every 50 days, roughly two months in the \hr{Vaimon Calendar}{\VaimonCalendar}.





\subsubsection{\Vorcanths{} and wolves}
The \hr{Vorcanth}{\vorcanths} dwelt on Visha. 

In symbolism, Visha was associated with wolves. 
The \quo{Mystic Wolves of the Frost-Moon} were well-known in mythology, but very few people knew what the \vorcanths{} were really like, or even that they existed. 
















\section{\Nithdornazsh}
\target{Nithdornazsh}
\target{Nith'dornazsh}
\index{\Nithdornazsh}
\Nithdornazsh{} was the ancient \draconic{} fortress of the \dragonking{} \Nexagglachel. 

It lay near where \Malcur lies today, because it was built upon the ruins of the same \hr{Wild}{\Wylde}{} fortress of which \Malcur's foundations are also a branch. 

\Nithdornazsh{} has been sealed off from \Miith{} for 10,000 years, maybe 20,000. It's so old that even most Sentinels and Cabalists don't know about it, but \Malcur is connected to \Nithdornazsh{} and has always been. Because of the Shroud, it is hard to use this connection for anything. 

At the time of \TwilightAngelRememberEmph, \HriistD{} has a plan to bring \Nithdornazsh{} back. See, after the death of \Nexagglachel, \Nithdornazsh{} fell into disuse and disrepair\dash all \dragons{} abandoned the place because of the painful memories of the king's fall. In fact, it has withered and died. But its soul lingers and can be reborn. It just needs to feed. Resurrecting a living fortress has very rarely been done, so no one suspects that this is what \HriistD{} is up to. 

For \HriistD, this project is not only a part of the war, but also a prestige project in \honour of the memory of his brother. 

The gambit to bring \Nithdornazsh{} to \Miith{} is a blatant breach of the \charade\dash the unspoken mutual agreement that the Sentinel-Cabal conflict should be kept underground and hidden from mortal eyes. It is uncharacteristic of \HriistD{}, who usually does not cheat this much. But for the sake of his brother, he is willing to do a lot. 

The whole \Nithdornazsh{} gambit is \ps{\HriistD}{} personal project, secret and hidden from his fellow Sentinels. 
Officially, he merely intends to invade Pelidor\dash partially to gain a foothold in central \Velcad{}, and partially as a part of his plan to awaken the \Haskelek{} (more about the Haskelek in \CarzainWithRedcorBook). 
One of their targets is the \hs{Ghost Tower} in northern Pelidor, which is a potent \nexus{} point in its own right. 
\Nzessuacrith{} is personally responsible for this last part. 

But unbeknownst to \Nzessuacrith, \HriistD{} has his own ideas and intends to use her attack as a distraction to further his own plan. He wants to resurrect \Nithdornazsh{} in \Malcur, transforming \Malcur into a reincarnation of the \draconic{} fortress. 









\subsection{Appearance}
\Nithdornazsh{} was cyclopean, gigantic. 
It was built to accomodate some of the vastest \dragons{} that ever lived. 
It conformed to the aesthetics of \hr{Draconic architecture}{\draconic{} architecture}. 

In the centre of the city there lay a temple, in which was kept a powerful relic: 
A mummified claw of \TyarithXserasshana. 

When it was summoned \Nithdornazsh{} was guarded by \hs{undead machines}. 

See also the sections on \hr{Resphan architecture}{\resphan architecture} and \hs{dark ancient cities}. 





\subsubsection{End of the world}
In \Nithdornazsh, when it arose in \Malcur, there were \quo{holes} where one could see out of the Shroud, into the Beyond, to the \quo{end of the world}, where the recognizable world gave way to unshaped chaos.
This was a \hr{XS}{\xsic} Realm. 

\citetitle[p.45--47]{RHCharles:BookofEnoch}{The Book of Enoch XVIII.11--}{
  And I saw a deep abyss, with columns of heavenly fire, and among them I saw columnds of fire fall, which were beyond measure alike towards the height and towards the depth.
  \\
  And beyond that abyss I saw a place which had no firmament of the heaven above, and no firmly founded earth beneath it: there was no water upon it, and no birds, but it was a waste and horrible place. 
  \\
  I saw there seven stars like great burning mountains, and to me, when I inquired regarding them, 
  \\
  The angel said: 
  This place is the end of heaven and earth: this has become a prison for the stars and the host of heaven.
  
  \ldots
  
  And I proceeded to where things were chaotic.
  \\
  And I saw there something horrible:
  I saw neither a heaven above for a firmly founded earth, but a place chaotic and horrible.
  \\
  And there I saw seven stars of the heaven bound together in it, like great mountains and burning with fire.
  \\
  \ldots
  \\
  And from thence I wence to another place which was still more horrible than the former, and I saw a horrible thing: a great fire there which burnt and blazed, and the place was cleft as far as the abyss, being full of the great descending columns of fire; neither its extent or magnitude could I see, not could I conjecture. 
}









\subsection{History}
\target{Nithdornazsh was Nexagglachel's tomb}
\Nithdornazsh was the old tomb of \Nexagglachel. 

When \Nexagglachel awakened he turned \Nexagglachel into his citadel. 

Out of respect for him it became deserted after his death. 
It became a necropolis in his memory, instead of being taken over and used for some other purpose. 
Thus the \resphain forgot about it. 
It was deep in \draconic territory and seemed to have little strategic significance, so the \resphain felt they had better things to do that attempt to conquer or raid it.
Until, finally, \Secherdamon felt it was time to revive it. 















\section{\Nyx}
\target{Nyx}
\index{\Nyx}

The purpose of \Nyx{} is to serve as a conduit between \Erebos{} and \Miith{}. 
\Nyx{} was initially connected to \Erebos{}, but the \dragons{} successfully sealed it off from \Erebos{}. 
Today the \banes{} use \Nyx{} as a base of operations and a place to hide away from the \dragons{}. 



\lyricsduana{thisistheend}{This is the End}{
  I dream of a host cockroaches \\
  crawling, swarming over charred flesh \\
  up bitter blackend walls \\
  in sewer pipes now purgd \\
  the cries of many amplifd \\
  then swallowd by silence \\
  echoes of a lost world \\
  of a foul smoke that lingers like stale breath \\
  of steel/glass/concrete fusion-statues \\
  rising high in a pale and infinite twilight \\
  shining beacons like bleachd bones 
}

Whenever I write about \Nyx, be sure to also read about \hr{Erebos}{\Erebos}.









\subsection{Denizens}
Denizens of \Nyx included:
\begin{itemize}
  \item \Flyingpolyps.
  \item \Noggyaleth (maybe).
  \item \Ophanim. 
  \item \Resphain. 
  \item \Umbrae.
  \item Weavers.
\end{itemize}









\subsection{Endless dark city}
\target{City of Nyx}
\Nyx{}, created as a shadow of \Erebos, is a world of immense, dead, decaying cities. Buildings of stone and metal reach thousands of metres up into the sky. At times a traveller finds himself on a ledge or causeway, in which case the ground might be hundreds or thousands of metres below\dash the buildings seeming to go on forever both up and down. 

The buildings are mostly empty and decaying, and the whole city feels like a necropolis. 
Monstrous scavengers lurk in the corners, and you may encounter the occasional \bane{} or \resphan{} or a \human{} slave. 
There are many buildings who were once inhabited by \resphain{} but have since been abandoned as their population dwindled. 
\Nyx{} is full of dangerous monsters, always ready to reclaim an undefended tower when the \resphain{} leave. 

The buildings in \Nyx{} are tall, spindly, emaciated. 
And dead; macabre corpses or skeletons. 
Compare them to the bloated \hr{Living buildings of Machai}{living buildings of \Machai}.

See also the sections on \hr{Resphan architecture}{\resphan architecture} and \hs{dark ancient cities}. 

\lyricsbs{Arcane Wisdom}{Misanthropic Horror Magnified}{
  Senseless oblivious towers,\\
  monuments to absurdity.\\
  Oceans of ignorance.\\
  Humanity at its cruel best! 
  
  When all noble values are (shamelessly) inverted, \\
  and brainwashed minds endure.\\
  When animal essence is forgotten. \\
  Misanthropic horror\ldots{} magnified.
  
  Decadence is all I see. \\
  Triviality lies before me. \\
  Nihilism\dash the only solution. \\
  Misanthropy\dash the final escapism. 
  
  Your world sinks to vomiting proportions. \\
  Visions of decay spring everywhere. \\
  Entire lost civilizations \\
  succumb to a \quo{worldly} oddity. 
  
  Woe and torment be behind me\ldots{} \\
  the maze of such reality. 
}

\lyricsbs{Exmortem}{Grand Dome of Destruction}{
  Naked chambers so cold and grim.\\
  A last gasp for air.\\
  A smell of funerals to come.\\
  Icecold Ugliness.
  
  Here I saw the Lord of Death,\\
  and his eyes flashed with rage.
  
  Gruesome Icons. Demonic Tokens. \\
  Images of a defuct future.\\
  Funeral fests. Nocturnal Chill.\\
  A mirror of the underworld.
}





\subsubsection{Chimneys}
Some places in \Nyx{} there are tall chimneys spewing out yellow and brown smog. 

The yellow is more repugnant than the black that is otherwise prevalent in \Nyx. 
The black is at least pure darkness. 
The yellow, on the other hand, is pure poison, rot, disease, corruption. 





\subsubsection{Dust-like moss}
The walls of \Nyxian{} buildings are often covered in a layer of what looks like coarse dust or ash. 
It is actually alive, a moss-like thing. 





\subsubsection{It was once nicer}
\Nyx{} was once far more civilized. 
Back in \ps{\Merkyrah}{} days, and even after that. 
\Nyx{} was not a ruin, but beautiful and thriving.

Back then, there were many \resphain, and they had tons of slaves and could maintain the gigantic city. 
But after the \secondbanewar{} and the \resphanwars, the \resphain{} and their slaves were decimated. 
There were not enough of them to maintain the enormous cities, and they fell into disrepair.
Today \Nyx{} is a decaying ruin, overrun by hideous monsters. 
Only small parts of the humongous buildings are inhabited. 





\subsubsection{Light}
\Nyx was dark, but the \resphan buildings were well-lit. 
In the days of \Merkyrah as well as afterwards. 
This was because \hr{Umbrae dislike light}{\umbrae disliked light}, so a brightly lit building was slightly less likely to be attacked by an \umbra.

The interiors of buildings were usually beautifully decorated with \hr{Resphan crystal technology}{crystal} and \hr{Glowmoss}{\glowmoss} as well as darker things. 

For visuals, see the section on \hr{Beauty of dark ancient cities}{the beauty of dark ancient cities}. 









\subsection{The deep}
\target{safe zone}
Only the tops of the spires are inhabitable. 
This top layer is called the \quo{safe zone} or \quo{safe belt}. 
The deep below is monster-haunted and too dangerous to venture into. 

In \Nyx, you can sometimes feel, infinitely far below the endless city of steel and stone, a living planet. 
Or, rather, a once-living planet, now writhing in its last convulsions. 
\FatherErebos{} was once alive and vibrant, savage and full of energy like \Miith{}. 
But the \banes{} conquered the planet and changed that. 
With their endless greed and cruelty they have sucked all life out of their homeworld. 
All that is left of their creator is a tortured, enslaved, withered husk, weeping in its endless torment and grief, weeping over the betrayal by its own children. 

\Nyx{} is not a living world and has no heart. It is merely an artificial shadow world built on top of \Erebos. 

Occasionally, when people are cast into \Nyx, they find that the earth is covered in bones and corpses\dash faces and skulls grin back with their toothy mouths and empty eyes, yet the images are so flighty and dreamlike, the faces flowing into one another, that the viewer doesn't recognize any, nor is he even sure what species the skulls were. 

\target{Nyx is above Erebos}
\Nyx{} actually (sort of) exists in the skies high above \Erebos. 
\Erebos{} is full of huge towers, many thousands of metres tall. 
The towers are thicker nearer the ground, but up in \Nyx{} they are spindly, emaciated, skeletal. 

The deep below \Nyx{} continues all the way down to \Erebos, but it is monster-infested and dangerous and not a viable route for anything, not even legions of \banes. 
The \CrystalSphere{} makes it even worse. 







\subsection{Illumination}
The \resphan{} safe zone is lit mostly by the flashes of lightning from \hr{Thunder in Nyx}{the thunderclouds far below}. 
There is also starlight from above, \hr{Sky in Nyx}{but no moon or sun}. 
Not since the \hs{Murder of the Dawn}, anyway. 





\subsubsection{\Glowmoss}
\target{Glowmoss}
\target{Glow-moss}
\index{\glowmoss}
When the \resphain{} need light, they use a luminescent moss- or coral-like lifeform that is native to \Nyx. 

The \glowmoss{} grows in the deep, drawing its nourishment from the thunderstorms. 
The \resphain{} have slaves that go down and harvest it. 
It is then put into glass lamps. 

It gives off a dull blue light. 









\subsection{Sky}
\target{Sky in Nyx}
The sky above \Nyx{} similar to that above \Azmith.
But there is no sun and no moons. 
And you can see the black stars of \Erebos. 





\subsubsection{Black stars}
\target{Black stars of Nyx}
\target{black stars}
\index{\Nyx!black stars}
Some of the stars in the \Nyxian{} sky are black. But you can clearly sense that they are there. Somehow they stand out from the blackness of the sky, by virtue of their unearthly radiance. 

Compare to the various stories in \authorbook{Robert W. Chambers}{The King in Yellow}.

\lyricsbs{Robert W. Chambers}{The Repairer of Reputations}{
  \ta{%
    I\ldots{} wept and laughed and trembled with a horror which at times assails me yet. 
    
    This is the thing that troubles me, for I cannot forget Carcosa where black stars hang in the heavens; where the shadows of men's thoughts lengthen in the afternoon, when the twin suns sink into the Lake of Hali; and my mind will bear forever the memory of the Pallid Mask.%
  }
}

The black stars are a characteristic of \Nyx{} and \Erebos. They are powerful and terrible energy sources that \quo{shine} through from \Erebos{} with their dark light. 

When you pierce the Shroud into \Nyx, the black stars always become visible. They are often one of the first things you notice\dash so strong is the menacing power they exude\dash and they are horrible to behold. 

The black stars are mentioned in \emph{\hr{Wanderers in Darkness}{\WanderersInDarkness}}. 









% \subsection{\Similuth}
% \Similuth{} is the part of \Nyx{} that the \resphain{} inhabit. 
% \Merkyrah{} lay here. 









\subsection{Parasitic Realm}
\target{Nyx is a parasite Realm}
\Nyx{} is a parasitic sub-Realm. 
In a sense, it is a cancerous growth on \Miith{} that cannot sustain itself but must suck nourishment from the rest of \Miith{}. 
Compare to the \hr{Resphain's parasitic economy}{\ps{\resphain}{} parasitic economy}. 





\subsubsection{\Nyx{} was killed}
\target{Nyx was killed}
Perhaps it was different under \hr{Merkyrah}{\Merkyrah}, which lay in \Nyx. 
Perhaps the rebels had to somehow \quo{kill} the Realm in order to establish the connection to \Erebos{} that they needed, so that they could draw on \Erebean{} power and summon \banes{} and monsters. 
Perhaps this was the \hs{Murder of the Dawn}. 




\subsubsection{It is hard to feel the Heart}
\target{Nyx is far from the Heart}
\Nyx{} is farther removed from the Heart of \Miith{} than the rest of \Miith{}. 
The Heart cannot be felt to the same extent in \Nyx{} as in \hr{Tembrae}{\Tembrae} and its fragments. 
Therefore, \Nyx{} feels barren and dead to some. 

\lyricsdimmuborgir{Stormbl\aa{}st}{
  Undring og angst samler seg i natten,\\
  i m\oe{}rket som ruver om spiret.\\
  For ingen dag kan veien hit.\\
  Intet lys kan luske frem.
}





\subsection{Weather}





\subsubsection{Thunder}
\target{Thunder in Nyx}
Thunder is common in \Nyx.
It is seen not above, but in the deeps below the towers where the \resphain live. 
\hr{Glowmoss}{\Glowmoss} gets its nourishment from the thunder. 









\subsection{Monsters}
\target{Nyx monsters}
\Nyx{} was full of nightmarish horrors. 
The \resphain{} prey on them and eat them. 
In the days of \Merkyrah, the \resphain{} hunted some monsters to extinction and thinned the populations of others. 
But in the millennia after the \hs{Murder of the Dawn}, most of the monster populations (that were not completely extinct) recovered, now that the \resphain{} themselves were declining in population for \hr{Heart weakened}{other reasons}. 





\subsubsection{Dark gods}
\target{Gods in Nyx}
There dwelt some dark gods in \Nyx{} that were independent of the \banes. 
\Daggerrain{} tried to keep them out of his pocket dimension, but he could not stop them all, and some slipped in. 

Some of these gods came to be \hr{Early diabolist Resphain}{worshipped by the early non-\Merkyran{} \resphan{} tribes}. 

Some of them were later \hr{Bael'Zerach diabolism}{worshipped by some of the \Baelzerach}. 





\subsubsection{Lurkers in the deep}
There dwell monsters in the deep underneath the cities of \Nyx. 
No one (except maybe the \hr{Banelord}{\banelords}) know what these monsters eat to live. 





\subsubsection{Spider-like monsters}
In \Nyx{} and/or \Erebos, I should have a monster that looks like a giant, monstrously misshapen black spider. 

Inspired by the cracks and blotches in the display on Jeppe's cell phone. (Jeppe is a gamer whom I met at Bjarke Lassen's workshop in February 2008.) 
















\section{The Pandaemonium}
The Pandaemonium is an alien Realm. 
It is similar to \Machai, but more Chaotic and more alien, farther removed from \Miith{}. 
Even the \dragons{} fear the Pandaemonium and the creatures that haunt it. 

Perhaps it is the true homeworld of the \xss. 
Some rumours say that the \xss{} not only came from here, they \emph{fled} from the Pandaemonium, chased by a race of even greater, more horrid gods. 















\section{The Realm of the Deep}
\target{Deep Realm}
The \quo{Realms of the Deep}, also known as the \hr{Aquatic Realm}{Aquatic or Pelagic Realms}, were the seas and the underseas, where the \nagalords{} and the \nagae{} dwelt.

\lyricsbs{Ancient Rites}{Het Verdronken Land van Saeftinge}{
  Here one can hear \\
  the call of the sea, \\
  while a deadwhite moonlight \\
  is creating the ultimate unlight. \\
  Or at night, or at night\ldots{} 
  
  O sad and beautiful night, \\
  filled with melancholy. \\
  When the silent dark waters \\
  are inviting the lonely souls. 
  
  Of mourning lost ones\ldots{} like me. \\
  Of mourning lost ones\ldots{} like me. \\
  Like me.
}















\section{The Sea}
\target{The Sea}
\target{Sea}









\subsection{Denizens}





\subsubsection{Myths about merfolk}
There were \hr{Myths about Nagae}{myths about merfolk}.
They were really \nagae.









\subsection{In ancient times}
In ancient times there was plenty of ships, seafaring and naval battles. 

The \hs{Deep Realm} and the \hr{Fragments of Tembrae}{fragments of \Tembrae} are closely intertwined.

Even in the time of the great \dragon-\resphan{} wars there was lots of seafaring and ships. 
The seas, as well as the air above them, were dangerous and treacherous, even for \dragons. 
There dwell cruel gods. 
Possibly the \hr{Kraken}{\krakens}, rivals of the \hr{XS}{\xss}.

\Dragons{} and \resphain{} could fly above the seas, but it was hard and exhausting to fly long distances, especially through the often fierce ocean winds. 
Besides, \dragons{} and \resphain{} were built with the \emph{ability} to fly, but they were not built for a \emph{life} in the air. 
So it was infeasible to fly all the way across the sea, or to fly around patrolling for longer periods of time, or fly to and from naval battles without landing. 

So it was easier and safer to sail \emph{on} the Sea (in sufficiently badass ships) than it was to fly above it for longer periods. 
So they did. 
Of course, in battles they often flied around to fight, but after the battles they would settle down on their ships. 

Besides, many immortals could not fly: 
\QuilJaaran, \aryothim, \vorcanths, \bezedeth. 

The master races, of course, built super-powered mega-ships, forged by great sorcery from exotic materials and nigh-indestructible. They could be humongous in size, several hundred metres long. 

The \resphain used ships a lot.
Sailors were often \bezedeth, since they could not fly.
Purebloods had great pride in their wings, so they did not like to admit the fact that they were dependent on ships, and so they did not become professional sailors.

The \dragons depended less on ships because \dragons were great swimmers, so if they tired of flying they could just swim.







\subsection{Circumventing the sea}
If you're cool, you can take a route through the Beyond around the seas and bypass them entirely, so you don't have to cross the sea at all. 
This is the long way around, though. 

An an example, one might go through \ps{\QuessanthIshnaruchaefir} \hs{Mirage Asylum}. 









\subsection{Placating the powers of the sea}
When sailing on the sea, you need to have priests/mages along to placate gods and spirits and keep at bay sea monsters and \Wylde{}-related natural disasters (storms and stuff). 
All cultures have some kind of sea gods to help them cross the sea. 

At times, humanoid sacrifices are needed to placate the wicked spirits of the \hs{Deep Realm}.

Remember to have gruesome things lurking in the deeps. 

\lyricsbalsagoth{Atlantis Ascendant}{This terror in the Astral Seas\ldots{}}









\subsection{Ships and navies}
In the \hs{Shrouded Realms}, mortal navies tended to be small. 
\hr{Wood is precious}{Wood was a precious resource}. 

Warships would be enchanted to protect them from fire and other calamities. 
A big ship was too precious to lose, so it had to be protected well. 
They were too expensive to let them go to waste. 

The mightiest ships were indestructible dreadnoughts, hundreds if not thousands of years old. 















\section{\Tembrae}
\target{Tembrae}
In the days of the \secondbanewar, \Tembrae{} was the \quo{main} Realm, the principal part of \Miith{}. 
\Machai{}, \Nyx{} and the Deep Realm were its neighbouring Realms. 








\subsection{Fragmentation}
\target{Fragments of Tembrae}
After the \SecondShrouding{}, \Tembrae{} was fragmented into a number of smaller Realms. 
\Azmith{} is one of them. 

\lyricsxkcd{240}{
  I had a dream that I met a girl in a dying world. 
  
  It was all coming apart. \\
  Hairline cracks in reality widened to yawning chasms. \
  Everything was going dark and light all at once, and there was a sound like breaking waves rising into a piercing scream at the edge of hearing. \\
  I knew we didn't have long together. 
}









\subsection{Less land than before}
There is not as much habitable land in \Tembrae{} (i.e., all the fragments combined) than there used to be. 
This is because \hr{Heart weakened}{the Heart is weakened} and cannot easily restore the land after all the destruction wrought on it. 
For example, the entire Realm that once housed \hr{Cuezca}{\Cuezca} is now uninhabitable, dreary wasteland. 















\section{The Voids Between the Worlds}
\target{Horrors of the Void}
The voids between planets were filled by all sorts of horrors.
When the \voyagers settled \Miith hundreds of millions of years ago, they \hr{Voyagers erect Palisades}{erected dimensional barriers around the Realms of \Miith} to keep it safe from these horrors.

In a sense, the \xss were some of these \quo{horrors} that the \voyagers feared.
They were some of the greatest \quo{horrors}.

The voids between the Realms were fucking dangerous. 
There existed blind, mindless, slavering things of which even the \resphain lived in fear. 
Whenever \resphain had to travel between the Realms, they could not just travel through the Beyond.
They only travelled along well-known pathways and had to have magic to keep them safe (more cosmic versions of the \eidola and \wylde charms that mortals used).

\citetitle[Lords of Cthul]{Misc:Monsterpocalypse}{Monsterpocalypse}{
  Beyond the veil of our own universe exist myriad dimensions teeming with unknown threats. 
  For eons, practitioners of the occult have dared to peel back that fragile layer that separates our world from a vast realm of darkness to glimpse the ancient powers that lurk within. 
  From time to time, those horrors have slipped through the void.
}

\citebandsong{DarkEmpire:DistantTides}{Dark Empire}{Distant Tides}{
  The deep of the gaping void\\
  Will surely swallow you whole\\
  Return is no option, only death\\
  By the shadows of the dark
}







\subsection{Travel between Realms}





\subsubsection{Only at certain points}
It is not feasible to travel between Realms at any arbitrary point. 
It is only safe to travel from the Immortal Realms into the Shrouded Realms and back at certain special points. 
You have to be aligned with a \matrix{} that has a connection/alignment with some \nexus{} point in the Shrouded Realm in question. 

In other places, the voids between the worlds are full of monsters and storms, making them terribly dangerous even for immortals. 
Only really badass immortals (like \QuessanthIshnaruchaefir) dare to travel the voids. 
Most immortals travel along the safe \quo{caravan routes} mapped out by the \matrices. 





\subsubsection{Submerging and surfacing}
\target{submerging}
\target{surfacing}
\index{submerging}
\index{surfacing}
It is possible to stray close to and far from the border between Realms. 
To \quo{surface} is to move closer to a given Realm, and to \quo{submerge} is to move further away from it. 
Surfacing and submerging is often seen relative to the Shrouded Realms. 

In order to see into the other world you have to move close to the border, and that makes you easier to detect. 

The Shrouded Realms are much smaller and narrower than the Immortal Realms, because of the Shroud. 
When one moves \quo{sideways} through the Shroud in a Shrouded Realm, one gradually submerges into an Immortal Realm. 

The Immortal Realms are full of monsters. 
When you submerge out of a Shrouded Realm and into an Immortal Realm, you risk running into the monsters that dwell there. 
The monsters cannot easily enter the Shrouded Realms because the Shroud keeps them out. 
They are bound to their homes because their Shrouded minds are accustomed to living there. 
Just like mortals, they cannot see into other Realms. 

But if some hapless mortal from a Shrouded Realm ventures outside, the monsters are free to eat him. 














\section{The \Wylde}
\target{Wild}
\target{Wylde}
%\subsection{\Wylde{} versus civilization}
The \Wylde{} is the uncharted wilderness between cities and villages, inhabited by dangerous beasts and monsters. 

Humanoids dwelling in the wild are labelled as savages and barbarians. 

A subtheme of the whole story is that of \Wylde{} vs. civilization. Neither of them is \quo{good}, both are evil in their own way. 

The \Wylde{} is the true, natural state of the world (the Realm of Beasts at least). 
It is a state of some sort of chaotic balance, the state to which the Realm always seeks to revert. 
But the \Wylde{} is also cruel and violent, a world of bloodshed, conflict and competition. 
Creatures of the \Wylde{} may live in packs or even tribes, but these groups are xenophobic of outsiders, and oppressive and merciless towards insiders. 

Civilization, ie., towns, cities and farmland, is an abomination in the eyes of \hs{Nature}. 
Farming is a parasitic process that sucks life out of the land, humiliating and enslaving the proud, \Wylde{} land to serve the lowly but arrogant humanoids. 
Sensitive souls can sometimes feel the suffering of the land as it cries out in anguish against its tormentors. 

For the above reason, farmland must be left barren occasionally to recuperate. 
If farmed too hard, it dies, letting nothing grow. 
Dead land, when left alone for a while, will gradually return to life as a \Wylde{} desert-like area, and may later become fertile again. 
Perhaps the deserts of the South and Orient are a result of over-farming. 

Civilized creatures are an abomination, outsiders, anathema to the world. 
They feel at home in their cities\dash the blunted, dumbed down, tortured, enslaved world that they created. 
But in the \Wylde{}, they feel out of place, unwelcome, hated. 
Like a unnatural disease that the world is striving\dash rightfully\dash to expel.

People fear the \wylde.
The \wylde is fucking dangerous.
It is a place of twisting, crawling chaos, where nightmarish horrors\dash{}giant animals and worse\dash{}lurk and feed on hapless travelers.





\subsection{Beasts, monsters and plants}
Remember to have flying monsters in the \Wylde{}! 
\hr{Vreiid}{\Vreiiden}, giant bats, giant birds of prey, pterosaurs\ldots{} 

There needs to be a race of quadrupedal, theropod-like reptiles. 
They are bigger than \nycans, smaller than \cortios{} and can be tamed. 
The Rissitics use them. 





\subsubsection{Animals see into the Beyond}
\target{animals that can see into the Beyond}
Some of the more intelligent animals are less entangled in the Shroud than humanoids and can see deeper into the Beyond. 
Examples include \hs{cats} and \hr{Nycan}{\nycans}.  





\subsubsection{Humongous creatures}
Have some truly humongous prehistoric creatures. 
Many times larger than even \dragons. 
Hundreds of metres long. 

Before the \ophidians{} came, they ruled. 
Now their indestructible bones remain. 

Perhaps these no longer exist. 
Perhaps they only exist in the seas now. 

Have buildings and things made from the bones of gargantuan monsters. 
Like in the movies \emph{Pitch Black} or \emph{Red Sonja}. 

The bones of the bigger of these creatures are so large that they can be used to construct palaces that even \dragons{} can dwell in. 

One reason why many of these are extinct is that \hr{Heart weakened}{the Heart of \Miith{} is weakened} and therefore has a harder time supporting these massive, powerful beings. 
Smaller creatures are cheaper and easier to keep alive for the Heart. 





\subsubsection{Trees}
Remember to have some mysticism about \hs{trees}. Trees are cool. 





\subsubsection{Degeneration}
Berserkers are prone to degenerating into beastly forms if they, in their mind, embrace their \Wylde{} power too much. See section \ref{The price of madness}. 









\subsection{Dark, unexplored places}
\target{Unexplored places}
Have many dark, unexplored \quo{here-there-be-\dragons} places in the \wylde. 
Even in \Velcad. 
In the \thirdbanewar period as well as in earlier periods. 
Compare to places from the Cthulhu Mythos:

\begin{itemize}
  \item The Vale of Pnoth.
  \item The Forest of Zoogs.
  \item The Peaks of Throk.
  \item The Vaults of Zin.
  \item The Tower of Koth. 
  \item Kadath in the Cold Waste.
\end{itemize}

Among other things, have a dark valley of naked, black basaltic pillars, inhabited by Gug-like monsters. 
And have places where the \quiljaaran live. 
The Serpent Men were known from legends and feared. 

See also the section on the \hs{dark universe}.









\subsection{Imagery}





\subsubsection{The majesty of Nature}
%\subsection{The land itself}
\target{Nature}
\target{Majesty of nature}
Be sure to fill \Miith{} up with vast mountains, valleys, raging rivers, deep clefts, haunted swamps, dense forests with trees over a hundred metres tall. Between these, even armies of hundreds of thousands of men cannot help but feel insignificant, dwarfed by the vastness and majesty of the world that surrounds them\dash a world they cannot hope to conquer or even move. 

The \hs{Mask of Civilization} can hide Nature's terrifying cruelty, but not its awe-inspiring majesty and power. 

\citeauthorbook[p.70]{RobertEHoward:KullUntitledDraft}{Robert E. Howard}{%
  Untitled Draft%
}{
  Along the broad white streets of Valusia swept the king and his horsemen, out through the suburbs with their spacious estates and lordly palaces; on and on until the golden spires and sapphirean towers of Valusia were but a silver shimmer in the distance and the green hills of Zalgara loomed majestically before them. 
  
  \ldots 
  But Kull walked apart, beyond the glow of the campfires to gaze out across the mystic vistas of crag and valley.
  The slopes were softened by verdure and foliage, the vales deepening into shadowy realms of magic, the hills standing out bold and clear in the silver of the moon.
  The hills of Zalgara had always keld a fascination for Kull.
  They brought to his mind the mountains of Atlantis whose snowy heights he had scaled as a youth, ere he fared forth into the great world to write his name across the stars and make an ancient throne his seat.
  
  Yet there was a difference. 
  The crags of Atlantis rose stark and gaunt; her cliffs were barren and rugged.
  The mountains of Atlantis were brutal and terrible with youth, even as Kull.
  Age had not softened their might.
  The hills of Zalgara rose up like ancient gods but green groves and waving verdure laughed upon their shoulders and cliffs and their outline was soft and flowing.
  Age\dash age\dash thought Kull; many drifting centuries had worn away their craggy splendor; they were mellow and beautiful with antiquity.
  Ancient mountains dreaming of bygone kings whose careless feet had trod their sward.
}





\subsubsection{The world eaten by maggots}
At times, people can look into the Beyond and see the world as if alive. 
It is a decaying piece of carrion, crawling with horrid, bloated maggots. 

These maggots are actually caused by civilization. 
They are a physical manifestation of the corruption caused by humanoids and their parasitic leeching, a symptom of the disease that is humanoids. 









\subsection{Maintaining the Mask of Civilization}
See section \ref{Maintaining the Mask of Civilization}









\subsection{People connected to the \wylde}





\subsubsection{Berserkers}
\target{Berserkers}
I need to same some berserker people who become superhumanly strong by channelling the raw power of their inner, \chaotic{} self. 

Maybe these people are halfway lycanthropes. They don't entirely shapeshift into animals, but in their enraged state they have shed their humanity so much that they almost appear like beastly half-men. 

\KarsaOrlong{} might be one of these: A mighty warrior from a \cregorr{} tribe, only recently turned to Rissitism. 





\subsubsection{Druids}
\target{Druids}
Druids are a special order of nature mages who channel the power of the \Wylde{}. 





\subsubsection{\Rangers: Hunters, pathfinders, explorers}
Some humanoids are \rangers: Hunters, explorers and pathfinders who have learned to live in a sort of pact with the \Wylde{}. They are in a special kind of contact with their inner, primal, \chaotic{} self, and as such can move in the \Wylde{} more safely: They understand the \Wylde{} better, and the \Wylde{} does not actively hate them. 

Actually, as a \ranger{} you can sometimes travel through the \Wylde{} \emph{faster} than on a road. This is because the \Wylde{} is more malleable and can be bent and reshaped through willpower and cunning. 

But ordinary folk sometimes hate the \rangers{}, seeing them as strange, half-savage outsiders. Compare them to the Wolfbrothers in \emph{Wheel of Time}. 

\Nycaneers{} all have some \ranger{} talent. Ilcas Northstar is a \ranger. 

\Rangers{} are prone to degenerating into beastly forms if they, in their mind, embrace their \Wylde{} power too much. See section \ref{The price of madness}. 

Compare this with the Wolfbrothers from Robert Jordan's \emph{Wheel of Time}, who run the risk of losing their humanity, goin mad and ending up running with the wolves.







\subsection{Travelling through the \Wylde}
\target{Travelling through the Wylde}
Most humanoids, when they had to travel through the \wylde, used roads, as described in the section on \hr{Roads through the Wild}{roads through the \wylde}. 

When mortal humanoids had to travel through the \wylde itself, without roads, they needed magical protection to ward away the predatory monsters and inimical magic of the \wylde. 

In the \hs{Iquinian religion}, at least two \sephiroth were specifically charged with safeguarding the sancta of civilization and keeping the \wylde at bay.
These were invoked and/or prayed to in all \wylde scenes.
Other religions had their own gods dedicated to protecting people from the \wylde. 
People walking in the \wylde almost always carried \wylde talismans blessed by these \sephiroth or other divine beings.

Preferably, people would have Vaimons and assistant priests with them (see the section on the \hs{Iquinian clerical hierarchy}).
They would pray, sing hymns to \Iquin and Silqua and burn incenses and carry sacred totems.
All this is to keep the \wylde at bay.
The common soldiers would join in the songs and sing the chorus lines.
(Have some songs like \quo{Gregoriansk Datalogi}.)

\citebandsong{Nile:Ithyphallic}{Nile}{
  Papyrus Containing the Spell to Preserve Its Possessor Against Attacks from He who is in the Water
}{
  Amun\\
  Lord of the gods\\
  Thou who art of the four rams heads upon thy neck\\
  Thou standest upon the spine of the crocodile fiends\\
  To thine sides are the dog headed apes\\
  The transformed spirits of the dawn

  Drive away from me the lions of the wastes\\
  The crocodiles which come forth from the river\\
  The bite of poisonous reptiles\\
  Which crawl forth from their holes

  Be driven back crocodile thou spawn of Set\\
  Move not by means of thy tail\\
  Work not thy feet and legs\\
  Open not thy mouth\\
  Let the water which is before thee\\
  Turn into a consuming fire

  I possess the spell to\\
  Preserve me from he who is in the water

  Thou whom the thirty seven gods didst make\\
  And whom the serpent of Ra didst put in chains\\
  Thou who wast fettered with links of iron\\
  In the presence of Ra\\
  Be driven back thou spawn of Set

  Drive away from me the lions of the wastes\\
  The crocodiles which come forth from the river\\
  The bite of poisonous reptiles\\
  Which crawl forth from their holes
}









\subsection{Terrain types}





\subsubsection{Deserts}
Have giant worms in the desert. Not quite as big as the ones from \authorseries{Frank Herbert}{Dune}, but a clear reference. 





\subsubsection{Jungles}
\target{Jungle}
Have deep, dark, humid jungles. 
Mystic and exotic. 

Compare to the film \cite{Movie:IceAge:III}. 





\subsubsection{The sea}
See section \ref{The Sea}









\subsection{Wildfog}
\index{\wildfog}
Areas of \Wylde{} are sometimes shrouded in \wildfog, a fog-like substance that obscures vision. 
(\Wildfog{} is not regular fog. It is not made of water and may be found even in deserts.) 










\subsection{\Wylde border}
See the section on \hr{Wylde border}{\wylde borders and \eidola}. 







































\chapter{The Feud}
In almost all \Miithian{} cultures, stories and myths are told about two secret organizations that exist in the shadows and wage an eternal war for dominion over the planet\dash{}the Cabal and the Sentinels of \Miith{}. They are said to have existed for many thousands of years, if not forever, and the Sentinels are associated with \dragons{} while the Cabal is said to be connected with the mysterious \banes{}, but other than that, little is known about them. Some say that the Sentinels are good, nobly defending \Miith{} against the predations of the evil Cabal, whereas others claim that it is in fact the Sentinels who are evil, and yet others assert that both orders are uncaring manipulators that prey on the hapless people of \Miith{}. 

It is said that the conflict between these obscure forces has shaped the entire history of \Miith{} and that all major heroes, leaders and rulers though time have been manipulated by one (or both) of the orders\dash{}and many are the men and women in history who have been publicly accused, even convicted, of being Cabalists or Sentinels. Both organizations are said to be of otherworldly origin and to work though dark sorcery and \daemonic{} allies. 

But few and far in between are the actual, visible indications of their existence, and indeed, today many people believe that the dark orders are but myth and superstition. Most \Velcadians{} comfortably believe that all \dragons{}, or as near all as makes no difference, dwell in remote, legendary Irokas, and that the wicked \banes{} have vanished from \Miith{} many thousand years ago, if indeed they ever existed at all. Even major kingdoms and organizations know of little evidence that the twin orders exist, and so they are for the most part relegated to the realm of legends. 

Few know even fragments of the truth, and fewer still know the whole truth, but the Sentinels and the Cabal are very real and have existed for nearly ten thousand years, and their importance in \Miithian{} history, even today, can scarcely be overestimated. 















\section{The Cabal}
\target{The Cabal}
\target{Cabal}
\index{Cabal, the}
\index{Cabal, the!Cabalist}
The Cabal is an underground organization that secretly serves the \resphain{} and the \banes. Its members are called Cabalists.

There are thousands of actual Cabalists scattered across \Miith{}, but tens of thousands more unknowingly serve the purposes of the Cabal, and millions more are being subtly manipulated by the Cabalists. Even within the Cabal, most who call themselves Cabalists know very little of their actual purpose, for the Cabal's \emph{modus operandi} is secrecy above all else. 

The supreme leader of the Cabal on \Miith{} is the \banelord{} Daggerrain, who ultimately serves the \baneking{} called \Voidbringer. The majority of Cabalists are \human{}, but they also count \scathae{} and \meccara{} among their ranks, and more than once in history has a renegade \dragon{} turned to the Cabal. 

%The Cabal is an underground order that run by and serving the \banes{} and their loyal \resphain.

The Cabalists are arranged in a strict hierarchy. They can and do backstab each other, but will do so only after careful planning and conspiration. Diplomacy and social relations are very important when one means to advance.

In contrast, the Sentinels are far more chaotic and violent. Their structure is more anarchistic, more overtly based on bullying and intimidation. Brute force and violence means more, and open combat between rivals is more common.









\subsection{Activities}





\subsubsection{Encouraging heroes}
The Cabal actively encourages would-be heroes to go out and do heroic deeds. See section \ref{Cabal encourages heroes}. 





\subsubsection{Fake resistance within the Cabal}
For new Cabalists, there is a period of \quo{trial membership}, which can last for several years. 

Also, to weed out potential traitors, there exists a fake \quo{resistance} within the Cabal. They allegedly work against the Cabal and try to foil its evil plans and do good instead. But in reality, the resistance is a hoax. It is controlled by the Cabal and used to lure out and identify potential dissidents within the Cabal, who might consider joining or forming an actual resistance. They are introduced to the fake resistance as a test, and if they join, or just neglect to inform their superiors, they are usually judged unreliable and killed off. 

Compare to the fake resistance in \authorbook{George Orwell}{1984}.





\subsubsection{The Missionarium}
The Missionarium is the division of the Cabal charged with controlling people's thoughts, beliefs, morals and emotions. They manipulate religions and the like. 

Compare to the Bene Gesserit Missionaria Protectiva from \authorbook{Frank Herbert}{Dune}. 









\subsection{Organization}





\subsubsection{Circles}
\target{Cabalist circles}
The hierarchy of the Cabal is arranged in \quo{circles}. The innermost circles are the highest ranks. 

The first circle counts as members only the very highest ranking \resphain{} and the \banelords. All \satharioth{} are of the first circle. 

Almost everyone in the first seven circles are \resphain{} or \banes. Almost all \resphain{} belong to the first nine circles. 





\subsubsection{Not everyone is a member}
The Cabal was the part of the \resphain{} alliance that took care of \bane/\resphan{} affairs in the \hs{Shrouded Realms}, as per the scriptures of the \hs{Unspoken Covenant}. 

Not all \resphain{} were members of the Cabal. 
Many restricted their attention to affairs in the \hs{Immortal Realms}: 
Direct war, science, politics, art, education (i.e., passing on the fruits of science and experience). 





\subsubsection{Ordo ab Chao}
According to \DIBiggestSecret, the Freemason Society (who are, allegedly, run by the reptillian Babylonian Brotherhood) have as their motto \quo{Ordo ab Chao}, \quo{Order from Chaos}.

Maybe the Cabal should have a similar motto. 









\subsection{Religion and eschatology}
The mortal Cabalists had religious beliefs about the \resphain. 

The Cabalists did not see themselves as evil.
They were revolutionaries who used harsh means to achieve a noble higher goal. 

The Cabalists agreed with the \rethyaxes on one thing: 
Illumination was only for the select few, the worthy. 
Not for the masses. 
(See the section on \hr{Arcanum}{\arcana}.)





\subsubsection{Eschatology}
\target{Cabal eschatology}
The Cabalists believed in, and waited for, the \hr{Second Advent of Lithrim}{Second Advent of \Lithrim}. 
They believed the great transcendent god \Lithrim would descend to \Miith together with the \resphain.
Then the righteous (Cabalists) would inherit \Miith and the wicked would be punished, either destroyed or condemned to serve as slaves. 
The Cabalists looked forward to this.
It would be the time of great salvation for them. 

See also the sections on \hs{Iquinian theology} and \hs{Iquinian mythology}, and see \hs{Needle's motivation}. 

Compare to the Christian myth of the apocalypse. 
Also compare it to the apocalypse of the Cthulhu Mythos, where some cultists believe the Great Old Ones will reward them for their faith and destroy all unbelievers. 

\citeauthorbook[p.124]{BrianLumley:TheSecondWish}{Brian Lumley}{The Second Wish}{
  Now Julia sobbed and threw herself once more into Harry's arms, clinging to him as he gazed in astonishment and revulsion at the monstrous carvings.
  The central theme of these was an octopod creature of vast proportions\dash winged, tentacled, and \dragon{}like, and yet with a vaguely anthropomorphic outline\dash and around it danced all the demons of hell.
  Worse than this main horror itself, however, was what its attendant minions were doing to the tiny but undeniably \human figures which also littered the walls.
  And there, too, as if directing the nightmare activities of a group of these small, horned horrors, was a girl\dash with a leering dog-toad abortion that cavorted gleefully about her feet!
}





\subsubsection{\Humans and \scathae}
The Cabalists believed that only \humans could enter the divine realm. 
\Scathae were doomed to be slaves in the divine realm, since they were not part of \Lithrim. 





\subsubsection{Iquinianism}
They believed Iquinianism was a twisted, corrupted version of their true beliefs. 
In this they were partially correct.
The rejected the Iquinian idea that the divine realm was some faraway spiritual thing. 
They believed that the divine realm would come on \Miith in a real, physical, literal sense. 





\subsubsection{\Lithrim}
The Cabalists believed in \Lithrim and waited for its Second Advent. 
\Lithrim was an androgynous entity.
It was not merely a god.
It was the eternal, immortal, cosmic soul of the \human race.
The god above all gods. 









\subsection{Lictor-like people}
\target{Lictors}
Some people who serve the \banes{} decay and shrivel, coming to resemble living corpses, dead and rotting for many years. Or pale, bloated maggots. Or sickly lepers. 

Compare to the hearse-driving man in \authorbook{\RWChambers}{The Yellow Sign}, or the Lictors in the RPG \emph{Kult}, or the priests in the movie \emph{300}.

Their voices are hoarse and inarticulate, almost bestial.

Note that they only look like this in their true, un-\hs{Shrouded} forms. The Shroud makes them look like healthy, normal \humans.

These Lictor-like people are thoroughly loathsome and repulsive. 

Maybe they degenerate because they are weak of mind and \hr{The cost of magic}{cannot bear the pure energy of \nieur} that they channel.





\subsubsection{Charcoal hates them}
\hs{Charcoal} has to deal with the Lictor-men. He hates them and is disgusted by them. But some of them are of as high a rank as he. 

He thinks to himself: \tho{\Qliphoth, I'm glad I haven't degenerated into one of them. I'm not weak like those people.}















\section{Chaos and Entropy}
\target{Chaos}
\index{Chaos}
\index{Entropy}
Chaos is a basic force in the universe. 
It is one of the driving forces of all life, and perhaps all existence. 
Chaos creates movement, emotions, conflict\ldots{} life. 
And death. 

\Machai{} is sometimes called the \quo{Realm of Chaos}, and it is assumed that Chaos somehow originates from there. 
That is a misconception. 
Chaos permeates all Realms, all planes of existence. 
The forces of \Machai{} merely embody, in a more overt and visible manner, a kind of \quo{Chaos} that is psychologically easily recognizable as such, and so the Realm is considered especially Chaotic. 
The Pandaemonium has this to an even greater degree. 

\lyricsbs{Arcane Wisdom}{Symphonia Chaos}{
  Chaos, ruler of Time. \\
  Chaos, infinity is Thine. \\
  Shadowy inner essence, \\
  cosmic tapestry and sparkling \\
  spheres of a circular reason. \\
  Chaos, ruler of Time. \\
  Chaos, infinity is Thine. 
}

\hs{Entropy} is another such force. 
Entropy creates stagnation, death and decay. 
In a sense, Entropy is opposed to Chaos, since Chaos \quo{promotes} life and Entropy \quo{promotes} death. 
But all life must end in death, and all movement must eventually grind to a halt, and all complex structures must eventually collapse. 
As such, Entropy can be seen as the \quo{brother} of Chaos; its necessary consequence and counterpart; its \quo{child}, if you will. 















\section{The \Feud}
\target{Feud}
The \feud{} is conflict that is at the heart of the entire story: 
The eternal war between the \dragons, natives of \Miith{}, and the \banes, alien invaders. 

Since the adoption of the \hr{Charade}{\charade}, the \feud{} has not been fought in the open, but in secret, behind the scenes. It is waged today by the two secret organizations the Cabal and the Sentinels of \Miith{}. 

\lyricslimbonicart{Suicide Commando}{
  In the shadows of the world's ambitions \\
  I see life and death, in an enormous collision. \\
  Light destroys what dark creates. \\
  Forever floating, the dark rivers of the heart. \\
  Lifeless drifting, in an orbital decay. \\
  A vicious circle of extinction. \\
  The soul reaches the state: Oblivion.
}





\subsubsection{Darkest magic wins}
\target{Darkest magic wins}
There is a tendency that in war, whichever side is willing to break more moral rules and use darker magic than the other side, wins. 
This is a bit of a theme. 






\subsubsection{Fighting for survival}
\target{Fighting for survival}
There are many, on both sides of the \hr{Feud}{\feud}, who are not evil but simply fight for the future and survival of their people. The two peoples cannot coexist, and so the only option is to wage a war of genocide. 

They are bound by Entropy and must constantly fight on, destroy and devour, lest their hunger consume them from within. 
This is especially true of the \resphain, due to their \hr{Resphan parasitism}{parasitic nature}. 

Also, \hr{Races love war}{they actively enjoy war and conflict}, which only makes it worse.

\lyricsbs{Marduk}{Scorched Earth}{
  We must win to save us from the plague's grasping jaws.\\
  No enemy can bring forth our death.
}

\lyricsdimmuborgir{In Death's Embrace}{
  Without the wit or will to end this journey,\\
  we continue travelling towards our faith,\\
  harvesting helpless Christian spirits,\\
  raping the sanctity of saints.
  
  For with the sign of the pentagram,\\
  hellfire rage is for us to come,\\
  as we shall wander the Pit,\\
  unhallowed by the Infernal One.\\
  We are forever captured\\
  by the embrace of Death.
}

\lyricsdimmuborgir{Progenies of the Great Apocalypse}{
  The battle raged on and on, \\
  fuelled by the venom of hatred for Man. \\
  Consistently, without the eyes to see, \\
  by those who revel in sewer equally.
  
  Are we not the undisputed prodigy of warfare, \\
  fearing all the mediocrity that they possess? \\
  Should we not hunt the bastards down with our might, \\
  reinforce and claim the throne that is rightfully ours?
  
  Consider the god we could be without the grace. \\
  Once and for all. \\
  Diminish the sub principle and leave it's toxic trace. \\
  Once and for all.
}





\subsubsection{The \resphain{} were winning}
For a long while, it looked as though \hr{Resphain are winning the Feud}{the \resphain{} were slowly winning the \Feud}. 















\section{Heroes}
\target{Cabal encourages heroes}
At times, the Cabal, and perhaps also the Sentinels, work to actively encourage would-be heroes to show themselves, stand up and do their heroism.

The Cabalists want to see that potential heroes, or \vertices, or people who can see through the Shroud, are discovered early on. If left to their own devices, such people might pose a threat, but if discovered while young and weak, they can either be killed off or brought into the fold. This is one of the ways they recruit new members. 















\section{Sentinels of \Miith}
\target{Sentinels}
\target{Sentinels of Miith}
\index{Sentinels of \Miith}
The Sentinels of \Miith{} are an organization founded by \dragons{}. Its purpose is to oppose the Cabal and thwart its advances. Its members are called Watchers and consider themselves the defenders of \Miith\dash{}true, native \Miithians protecting the planet from an alien menace who seek to invade the planet and destroy or enslave her people. In truth, however, much of the Sentinels' work is geared not towards opposing the \banes{}, but towards furthering the cause of \draconic{} dominance on \Miith{}. 









\subsection{Summoning the \xss: A deadly balance}
\target{Summoning the XS: A deadly balance}
The Sentinels, led by \Secherdamon{} and \Vizsherioch, among others, seek to awaken the \xss, but only partially. Enough to draw upon the \ps{\xss}{} power and use it to vanquish the \banes. But they don't want them to awaken so much that they will take control and subjugate or wipe out their \draconic{} spawn. 

To this end, the Sentinels \hr{To break or preserve the Shroud}{work to weaken the Shroud}. 

\target{Dragons want to usurp the XS}
In the best of worlds, the Sentinels hope to \quo{revive} the \xss{} by usurping them\dash transforming themselves into new \xss. \Vizsherioch{} comes closer to this than anyone else.

This is a delicate balance. Some, such as \Ishnaruchaefir, believe that the Sentinels will fail, and therefore \hr{Ishnaruchaefir fights to preserve the Shroud}{strive to preserve the Shroud}.









\subsection{Fractions among the Sentinels}
There were various fractions among the \dragons. 
The \dragons disagreed and fought against each other.

The Sentinels had a dark council who worked to ensuer the \quo{Imperial Court of \Draconian{} Sovereignty}. 
This title is taken from \DIBiggestSecret{} p.44.









\subsection{New Sentinels are initiated}
New Sentinels are initiated into the order. 

\lyricsbs{Arcane Wisdom}{
  A Winter Solstice\dash Hymn to the Fallen
}{
  Gathered to find the Ancient Spirit. \\
  United to \honour the Elderly Ones. 
  
  On the top of this misty white mountain. \\
  Aye\ldots{} our bones are now freezing cold. \\
  Let the ice burn our souls \\
  and the flames\ldots{} freeze our hearts. 
  
  For the Pagan circle of Fire must now be alit.
  
  Once men of great deeds did gather. \\
  Protected were they by ancestral deities. \\
  And now we welcome the arctic season \\
  in our souls that since long turned to ice, 
  and praise them in sheer rapture. \\
  A majestic winterspell \\
  under these cloudy angst-driven skies.
}









\subsection{The Cult of the Worm}
\target{Cult of the Worm}
\index{Cult of the Worm}
The Cult of the Worm is a cult that worships a form of \hr{Khoth-Sell}{\KhothSell}, the \firstgendragon{} goddess of Death. The cult is underground but not exactly secret. Everyone knows it exists. Governments, the \hs{Iquinian Church} and the \hs{Cabal} try to suppress it because it is subversive. 

Worshippers of the cult at times sacrifice their own limbs and organs, mutilating themselves for \quo{salvation}. 
This is probably misguided, since \KhothSell{} is a \xs. 

The cult knows about the \hr{Morbus}{\Morbus} and seek to twist and use it for their own purposes. 

\lyricsbs{Marduk}{Imago Mortis}{
  Today king, tomorrow worms and cold in mouth.\\
  A reach for purity, through decay, through black soil excrements.\\
  None shall stand before the Lord of the Death-Winged Dart.
}

The Cult of the Worm worshipped worms and decay as a manifestation of \KhothSell or other dark gods of death and decay.


\citebandsong{Nile:Ithyphallic}{Nile}{
  Eat of the Dead
}{
  The highest fulfillment of man\\
  Is to become food for the crawling things\\
  That burrow and slither in human flesh\\
  Unceasing in mindless hunger\\
  Remorseless undefiled by reason\\
  The worms of the tomb they are pure

  Their purity elevates them\\
  Above the putrefying pride of our race

  The destiny of man is\\
  Merely to be\\
  The nourishment of the worm\\
  Yet their excrement bestows higher wisdom

  From decay arises new life\\
  Fill myself with that which rots\\
  And I shall be reborn

  By writhing upon my belly like a mindless worm\\
  I shall rise up in awareness of truth\\
  I gnaw upon my own decaying flesh\\
  And my mind is forever purged\\
  Of the corruption of faith
}





\subsubsection{Reapers}
\target{Worm Cult reapers}
The priests of the cult function as \quo{reapers}. They haunt battlefields, slums, \hr{Asylums}{asylums} and other places where people die or suffer, and offer solace and salvation in exchange for loyalty and worship. Thus they prey on and recruit from the ranks of the dead and dying. The dying are easy to enlist. The dead less so, because they are harder to communicate with and less able to think clearly\dash but then, enlistment needs not necessarily be voluntary. Dead souls can be bound and enslaved.

Carzain \hr{Carzain sees reapers near Forklin}{sees some of these} after the battle at \Forklin.

If they must fight, the reapers kill with disease. Like the Silver Brethren from Stephen Marley's books \emph{Spirit Mirror} and \emph{Mortal Mask}. 









\subsection{The Dark Crescent}
\target{Dark Crescent}
\index{Dark Crescent}
A an \Azmith-spanning underground order or cult led by \LocarPsyrex. 

It is less covert than the Cabal or Sentinels. Most people know the order exists. The order is widespread and has much power in certain parts of \Velcad{}, despite the best efforts of the \hs{Iquinian Church} to suppress it. The cult is capable of raising a quite impressive army when it needs to (but its enemies do not know this, only suspect). 

It was an underground religious organization of mages and barbarians and stuff.
Many minor \rethyaxes and petty mages were secretly part of the Crescent.
Many heathen tribes in \Velcad were affiliated with the Crescent.

Compare it to the organization of the Black Moon in \FLuneNoire, led by Haazeel Thorn. 

Is the Dark Crescent connected to the actual \hr{Moons}{moons} of \Miith{}? 





\subsubsection{The physical Dark Crescent}
The Dark Crescent is actually a physical thing. It is a flying ship shaped somewhat like a crescent moon. \hr{Psyrex's throne}{\ps{\Psyrex}{} throne} is located on it. 

Perhaps it is shaped like an ashen claw, giving rise to the name of the Dark Crescent knights. 

Sometimes the Dark Crescent can be seen in the sky. But only with difficulty, since it is black. 





\subsubsection{Corruption}
Like the rest of \ps{\Secherdamon} minions, the Dark Crescent was \hr{Secherdamon plagued with corruption}{plagued by corruption and infighting}. 





\subsubsection{Dark knights}
\target{Dark Crescent Knights}
Have some sinister knights of the Dark Crescent. Like \hr{Dark knights of Mystraacht}{the dark knights of Mystraacht}. 

Compare to the Lords of Negation in \FLuneNoire. Maybe give them a similar title. 

Maybe they are the \quo{Ashenclaw knights}. 

Maybe they ride \hr{Vreiid}{\Vreiiden}.

Maybe there are undead knights who ride undead \vreiiden.

Maybe they resemble Revenants from the game \emph{Warcraft III}. 







































\chapter{Atmosphere}
\section{A dark, sinister world}
\target{The dark universe}
\target{dark universe}
I need to create the impression of a dark, mystic, cruel and savage world. In this regard I am very much inspired by the band Bal-Sagoth, especially their first two albums, \emph{A Black Moon Broods Over Lemuria} and \emph{Starfire Burning Upon the Ice-Veiled Throne of Ultima Thule}. 

I want to create a Bal-Sagoth-esque atmosphere. How do I do that? 

Plenty of landscape descriptions: 
Mist-shrouded heaths, dark and impenetrable forests, cyclopean and towering cities and citadels. 

Hints of a mystic past: 
Old, lost temples and castles. 
Forgotten gods dreaming in the void beyond the world. 

Dreams, visions and hallucinations. 

In war, mages darken the sky with witch-storms. 

Things sunken beneath the sea.

Have a green, aethereal, algae-overgrown-looking building with tall spires. Like the one scene in episode 8 of the anime \emph{Trinity Blood}. 

Also, the concept of \quo{innocence} sucks. Innocence is a lie that weaklings and cowards cling to because they are too weak and afraid to face the true world. Classical symptom of being mired deep in the Shroud. 







\subsection{Carrion worms}
Have a species of carrion worms or vultures, who feed on the suffering and destruction caused by the eternal \hr{Feud}{\feud}. 

Maybe these are intelligent, or maybe they have a cult worshipping them. These will want to sunder the Shroud and the \charade{} in order to escalate the war and create more destruction on which to feed. 







\subsection{The cold darkness of the infinite universe}
Have multiple scenes where a character sees glimpes of the beyond and is terrified but awestruck at the vistas of the cold darkness of the infinite universe, its horrors, mysteries and wonders. 

Compare to \bandsong{Bal-Sagoth}{As the Vortex Illumines the Crystalline Walls of Kor-Avul-Thaa} and \bandsong{Bal-Sagoth}{Summoning the Guardians of the Astral Gate}. 

These visions are terrifying, but also fascinating and compelling. They can instill ambition, or they can crush your will to live. (In \hr{Moro Cornel}{Moro \ps{\Cornel}} case, they sort of did both.)

\lyricsbs{Bal-Sagoth}{Summoning the Guardians of the Astral Gate}{
  Such searingly terrible stellar majesty\ldots{} \\
  my sanity is lashed like a vessel on a storm-wracked sea. \\
  What price this invocation? \\
  Shall the singing stars claim my very mind?
  
  To countless worlds we travel, \\
  riding the endless black seas 'twixt the stars\ldots{} \\
  the ebon oceans of infinity\ldots{} \\
  flying through a thousand suns, then watching their light fade, as if it were but a flickering candleflame snuffed by the wind. As beings of pure energy we become one with the vastness, transcending the ethereal walls of time, spanning at once this celestial eternity, and yet existing as no more than a mote of dust within the vista of its endlessness\ldots{} \\
  Journeying beyond\ldots{} 
  
  The threshold looms, \\
  the star-way between dimensions stretches before me\ldots{} \\
  The Gate To That Which Lies Beyond yawns wide\ldots{} 
}

They might also see stars, planets and civilizations born and die, like lowly fireflies. But pretty\ldots{}

\lyricstitle{Zeratul from \cite{VideoGame:Starcraft}}{
  I have journeyed through the darkness between the most distant stars. I have beheld the births of negative suns and borne witness to the entropy of entire realities.
}

And, of course, you see hints of the dreadful, loathsome, awe-inspiring creatures that dwell in the far reaches of the Beyond. 
\hs{Cosmic gods}, some of which are \quo{evil}. 

\lyricsbs{Bal-Sagoth}{Summoning the Guardians of the Astral Gate}{
  Unspeakable forces gibber and pulsate in the Outer Darkness\ldots{} \\
  Elder horrors dwell here, things which were ancient and revelled in sublime galactic malevolence when even Xuk'ul was naught but a bloated cosmic maggot, writhing and suckling at the breast of its amorphous mother\ldots{} \\
  They-Who-Lurk-And-Breed-In-Limbo\ldots{} \\
  the squamous sovereigns of the elder void!
  
  Primal terror drags my essence \\
  screaming back from the threshold. \\
  The ichor of pestilent tongues clings to me, \\
  tendrils probing, the ire of fiends!
  
  The ravening black worms of madness are devouring the shredded remnants of sanity as I return to my slumbering steel-clad body\ldots{}
}







\subsection{The mind, the body, and sex}
Our minds and bodies are so close and so familiar to us, and yet so labyrinthine and unknowable. 
Perhaps, when you dig deep, they as dark and gruesome and terrible to explore as the cold, alien universe that surrounds us. 







\subsection{Horror and beauty}
The Beyond, the master races, \daemons{} and black magic are all dreadful, hideous and loathsome at first sight. 
This is an effect of the Shroud. 
But when you learn more of the true nature of the universe, you begin to see and appreciate the beauty and splendour of these mighty, exalted entities. 

The \resphain, and perhaps especially the \resviel, inspire in \humans{} an urge to kneel, grovel and worship at their feet. 
Compare to Queen Azshara in the \emph{Warcraft: War of the Ancients} books by Richard Knaak. 









\subsection{Religion}
\target{Religion is dark}
The various gods of Azmith were usually seen by their followers as vast, powerful and frightening, but still mostly benevolent.
But most of them were really horrible elder entities. 
At best, they are \dragons, which is pretty terrifying already.
At worst, they are \xss (like \hr{XS Taorthae}{some \taorthae}) or abominations like the \sephiroth. 

It was gruesome for the people to realize the dark, sinister, occult nature of their gods.
It made some people realize how little they really knew, how powerful the occult unknown was and how much they were the slaves of the powers of primordial darkness.

A few Vaimons and other sorcerers suspected the truth, having learned much about the dark universe through their sorcerous practices.
These people generally kept their suspicions to themselves.
They were having a hard enough time adjusting to it themselves, and they knew most people would not believe them, or even burn them as heretics. 
Besides, it was a terrifying truth that people were better off not knowing at all.
It is better not to speculate, for down that road lies heresy, horror, corruption, evil and madness.
Ignorance is bliss.







\subsection{Survival of the fittest}
The world is a merciless crucible. 
The weak are broken down, succumbing to fear, apathy and stupidity, hammered down into a life of suffering and slavery. 

Only the strong of will are able to face the challenges of life, learn from them, emerge stronger, and thus rise above their original station in life. 
These are the people who have the potential to become \vertices. 

Note that \quo{potential} is not about inborn talent. 
It is all about making the right choices. 
Physical strength, intelligence, a talent for magic\dash all those things people are born with in various amounts. 
But all of them can be developed through sheer will, by \emph{choice}. 
It is about facing fear, accepting challenges, and embracing your inner ferocity. 

It is not about loving your neighbour. 
Altruistic motives can sometimes bring you a long way, but a selfish lust for power and ultimate pleasure can be just as effective, if not more. 
The key is \emph{ambition}. And \emph{perseverance}. 

That is what makes a man. 

That is what makes a hero. 

That is what makes a \vertex. 

(See also the section on how \hr{Races love war}{some races derive pleasure from war and conflict}.)















\section{Blurb}
\begin{blurb}
\Miith{} is a world of islands\ldots{}

dark and mystical\ldots{}

swathed in moon-fogs\ldots{}

Fevered dreams of gleaming black onyx spires and forgotten, brooding gods. 

Vast and dark is the world, yet the people huddle in their towns and cities and close their eyes, comprehending nothing of the cruel universe that surrounds them. 
\end{blurb}















\section{Buildings and ruins}









\subsection{Dark ancient cities}
\target{Dark ancient cities}
\target{dark ancient cities}
\Draconian and \bane buildings alike violated the laws of three-dimensional geometry.
They extended into the dark, hidden dimensions of the Beyond. 
Alien dimensions that no one wanted to contemplate. 

The shapes of their walls, towers, statues and ornaments stretched out into the Beyond in a way that was not only physically painful for the eye to follow, but also horrible because it drew attention to hidden things that no mortal wanted to consider. 

See also:
\begin{itemize}
  \item \hr{Draconic architecture}{\Draconic architecture}.
  \item \hr{QJ architecture}{\QuilJaaran architecture}.
  \item \hr{Resphan architecture}{\Resphan architecture}.
  \item \hr{Nyx}{\Nyx} and \hr{Erebos}{\Erebos}.
\end{itemize}

Ancient immortal cities reached out into the Beyond. 
They were built in a time before the Shroud, when the barriers between the Realms were much more permeable. 
Their streets and corridors and towers were built so they criscrossed the Realms. 
This was why \hr{Nithdornazsh is a useful gateway}{\Nithdornazsh was so useful as a gateway} between \Machai and \Azmith. 





\subsubsection{Beauty}
\target{Beauty of dark ancient cities}
The dark cities were not only horrible, but also magnificent. 

\citetitle[p.41--42]{RHCharles:BookofEnoch}{The Book of Enoch XIV.9--19}{
  And I went in till I drew nigh to a large house which was built of crystals: and the walls of the house were like a tesselated floor made of crystals, and its groundwork was of crystal.\\
  Its ceiling was like the path of the stars and the lightnings, and between them were fiery cherubim, and their heaven was clear as water.
  A flaming fire surrounded the walls, and its portals blazed with fire.\\
  And I entered into that house, and it was hot as fire and cold as ice: there were no delights of life therein: fear covered me, and trembling gat hold upon me.\\
  And as I quaked and trembled, I fell upon my face.
  And I beheld a vision,\\
  And lo! there was a second house, greater than the former, and the entire portal stood open before me, and it was built of flames of fire.\\
  And in every respect it so excelled in splendour and magnificence and extent that I cannot describe to you its splendour and its extent.\\
  And its floor was of fire, and abive it were lightnings and the path of the stars, and its ceiling also was flaming fire.\\
  And I looked and saw therein a lofty throne; its appearance was as crystal, and the wheels thereof as the shining sun, and there was the vision of cherubim.\\
  And from underneath the throne came streams of flaming fire so that I could not look thereon.
}





\subsubsection{Dark things from the depths}
Sometimes one could see or imagine dark things rising up from the depths. 

\citeauthorbook[p.133--141]{JohnGlasby:TheOldOne}{John Glasby}{The Old One}{
  \ldots{} we caught a fragmentary glimpse of something which rose from those benighted depths, clawing up from the unseen floor. [\ldots{}]
  
  To me, they held ineffable suggestions of a blasphemous structure and architecture utterly unlike anything I had ever seen. 
  [\ldots{}] the searchlight beam only touched their topmost regions.
  But even this was enough to show the sheer alienness of their general outlines.
  Had they been mere conical towers, it would not have offended our sense of perspective to such a degree. 
  But there were bulbous appendages and truncated cones which intermeshed in angles bearing no relation to Euclidean geometry and I felt my eyes twist horribly as I tried vainly to take in everything I saw. 
  
  [\ldots{}] 
  I could not help feeling there was something evil about those nightmarishly misshapen spires and pinnacles with their bizarre curves and planes; yet it was not an evil associated with Earth but rather with the endless gulfs of space and time, with dimensions other than those we know. 
  
  The majority were smashed and broken with harsh, gaping orifices showing blackly against the sickly grey.
  What beings had once moved within these structres it was impossible to visualize. 
  Certainly no hand of man had erected them and carved their cruel, hideous contours.
  [\ldots{}]
  The obscure quality of menace in their weird symbolism made me shudder and long for the sanity and safety of the ship.
  
  [\ldots{}]
  
  [I watched] for he first indication of the vast grey-stone city.
  And then I saw them for the second time, rising out of the slime of the ocean floor, clawing upward for hundreds of feet; row upon seemingly endless row of fantastically symmetrical columns, the nearer ones blindingly clear in the harsh actinic light, with countless others stretching away into the black immensity. 
  
  [\ldots{}]
  
  The effect of that monstrous labyrinth which stretched away from us into inconceivable distrances was indescribable for it was apparent at once that whatever stood on this undersea plateau had never been fashioned by nature, even in her wildest and most capricious moments. 
  And it was equally obvious that whatever hands had erected these edifices had been far from \human. 
  
  [\ldots{}]
  
  I dreamed of the long-dead city under the sea.
  But before my dreaming gaze it now stood unbroken and untarnished by time on dry land and there was no sign of the ocean.
  On an incredibly ancient plateau, wreathed in clouds of steam and noxious vapours, the Cyclopean buildings streched away in all directions as far as the eye could see and high into the lowering clouds where the topmost spires were lost to sight.
  There was something terribly un\human about the geometry of its massive grey-stone walls, and the mind-wrenching alienness of its angles and intermeshing structures went against all reaon, all known laws of mathematics, logic and architecture.
  I knew, by some weird instinct, I was seeing it as it had been perhaps several millions of years ago when it had been newly built by that race from the stars. 
  
  [\ldots{}] now I saw the inhabitants, those hideous and, if the \emph{Book of K'yog} was to be believed\dash{}artificially\dash{}created abominations that had built it!
  I saw them as vague shapes ion the vast avenues and squares, saw them clinging limpet-like to the sides of the buildings or oozing jelly-like from the grotesque apertures and doorways.
  What insane blasphemy had bred those \emph{things} I could not conceive, but the mere sight of them woke me, yelling incoherently, from my dream. 
}






\subsection{Gargoyles}
\target{Gargoyles}
\index{gargoyles}
Gargoyles are awesome. 
I should have them. 

Maybe have some in \Malcur, some statues that are believed to be dead stone, but suddenly come alive. Perhaps in the crypts beneath, letting Rian and Moro encounter them. 

Are they Sentinel- or Cabal-created?

The \hr{Ortaican gargoyles}{\Ortaicans had gargoyles}. 









\subsection{Monuments}
\target{Dark monuments}
Have old, dark, foreboding statues and cemeteries.

When active, \hr{Monuments}{monuments} are important for \hr{Maintaining the Mask of Civilization}{maintaining the Mask of Civilization}. 

Have lots of old monuments built by Elder Races. 
Superior works of superior architecture that could only have been built by superior Elder Races.

Some of them are \hr{Atmosphere:Ruins}{ruined}. 










\subsection{Pyramids}
\target{pyramids}
\target{Dark pyramids}
\target{Black pyramids}
And have dark pyramids. 
Pyramids are cool. 
Perhaps with monsters or undead spirits slumbering entombed beneath them. 
And full of cryptic hieroglyphs. 

Old, creepy cemeteries, and morbid, Nazgul-like statues. 









\subsection{Ruins}
\target{Atmosphere:Ruins}
Remember to have lots of ancient civilizations and ruins of them. 

Have ancient, decayed, crumbling, ruined cities with great empty buildings, temples and monuments. 
People can go here to fight. 
Maybe these buildings lie in \hr{Nyx}{\Nyx}. 
Maybe they are the ruins of ancient \resphan{} cities, perhaps even \Merkyrah. 

See also: \hs{Fallen civilizations of undead}. 

\lyricsbs{Exmortem}{Funerary Sculpture}{
  Huge battle machines\\
  now funerary sculptures.\\
  \Armoured divisions\\
  no longer a threat.
  
  Isolated war scenarios.\\
  Rusty iron deserts.\\
  Human remains\\
  in the sand and the wrecks.
  
  Unaffected of the grim disease\\
  I walk these fields of flesh piles.\\
  Desolated, lifeless.\\
  Slowly decaying.
  
  Huge battle machines\\
  now funerary sculptures.\\
  \Armoured divisions\\
  now plague eaten filth.
}

\lyricsbs{Exmortem}{Grand Dome of Destruction}{
Naked chambers so cold and grim.\\
  A last gasp for air.\\
  A smell of funerals to come.\\
  Icecold Ugliness.
  
  Here I saw the Lord of Death,\\
  and his eyes flashed with rage.
  
  Gruesome Icons. Demonic Tokens. \\
  Images of a defuct future.\\
  Funeral fests. Nocturnal Chill.\\
  A mirror of the underworld.
}

\citebandsong{Nile:BlackSeedsofVengeance}{Nile}{
  To Dream of Ur
}{
  Desolate and Forsaken \\
  Eerily Moaning Dark Winds\\
  Murmur Incantations\\
  Dusk Calls Forth Shadows

  Spirits of the Glorious Dead \\
  Lingering, Bound to this Place\\
  They Whisper of Untold Sagas, of Long Dead Cities\\
  the Seven Shining Cities Sacred to the Aphkhallu

  Of Ages Past when the World was Young\\
  When Babylon was Blessed of Marduk\\
  and the Sound of her Armies was the Blare of Ominous War Horns\\
  and the Clash of Immortal Cymbals\\
  of Bronze Gates Arrayed in Splendour\\
  and Magnificent Walls of Sunbaked Brick \\
  Of Temples of Marble and Bloodstained Altars

  Long Before the Jeweled Throne of Ur\\
  Fell Silent and Turned to Dust\\
  Beneath the Endless Shifting Sands\\
  and the Inevitable Vengeance of the Elements
}















\section{Epic feel}
Remember that while \Miith{} is a dark, horrible and cruel world, it is also epic, glorious and filled with splendid wonders. 









\subsection{Ancient history}
Be sure to have lots of references to ancient history, and the wars of old. 
Let the immortals talk about it. 
Compare to Drusas Achamian's dreams in \authorbook{R. Scott Bakker}{Prince of Nothing}. 







\subsection{Armies}
\target{Glorious armies}
Whenever there's a war or an army, remember to portray it in an epic manner, emphasizing the splendour and glory of the soldiers, and of war as a whole. Describe the vast size of the forces, the shining, deadly steel, their mighty war machines, their billowing banners and awesome fanfares.

This is also a way to combine darkness and epicness. 

Compare to \bandsong{Bal-Sagoth}{And Lo, When the Imperium Marches Against Gul-Kothoth, Then Dark Sorceries Shall Enshroud the Citadel of the Obsidian Crown}. 

\citeauthorbook[p.69]{RobertEHoward:KullUntitledDraft}{Robert E. Howard}{%
  Untitled Draft%
}{
  The cavalcade moved forward at an easy trot.
  The people of Valusia gazed curiously from their windows and doorways and the throngs on the streets turned as the clatter of silver hoofs resounded through the babble and chatter of trading and commerce. 
  The steeds flung their caparisoned manes; the bronze armor of the warriors glinted in the sun, the pennons on the long lances streamed backward.
}









\subsection{Mystic names and places}
\target{Mystic names}
I should spam throwaway references to mystic places and names all over the place, like Shung in this passage:

\citeauthorbook[p.254--255]{HPLovecraft:TheBlackTomeofAlsophocus}{H. P. Lovecraft}{%
  The Black Tome of Alsophocus%
}{%
  I travelled faster than thought, past unlit planets and vistas of unknown realms which swirled and shifted across immensurable distances; the stars flashed by so rapidly that they appeared as gossamer-fine threads of brightness interlaced across the universe, minute shooting stars of brilliance shining against black aether that was darker than the fabled depths of Shung.
}

And dark, mystic, violent background events, like in \bandsong{Bal-Sagoth}{Summoning the Guardians of the Astral Gate}.









\subsection{Very large castles and cities}
Cities and castles should be huge. Such as the \hr{Malcuric city gates}{\Malcuric city gates}.

The ancient citadels are near-impregnable. They are as good as indestructible and can only be captured by a lengthy siege or by treachery.















\section{Quotes}
This is a collection of quotes from music, literature and other things that might be cool and serve as inspiration.









\subsection{Bal-Sagoth}
\subsubsection{A Black Moon Broods Over Lemuria}
\lyricsbs{Bal-Sagoth}{Dreaming of Atlantean Spires}{The Topaz Throne is beckoning, the jewelled sword awaits my grasp\ldots{}}

\lyricsbs{Bal-Sagoth}{Spellcraft and Moonfire Beyond the Citadel of Frosts}{
  Black stone stummoning the eternal power of the winter Moon\ldots{}
  
  Sorcery opens fiend-haunted pathways before me\ldots{}
  
  Dark spellcraft summons the Black Gate before me\ldots{}
  
  Icy waters whispering.\\
  Tower of Silence hides the shadow-key.\\
  Ember-trees haunt my fevered dreams.\\
  Moon-Bride, sing thine dark enchantment.
  
  Elder shadows writhing before the silvern gate of eternal winter.\\
  Dark shapes entwine the mist-veiled cromlech.\\
  Dynig torchlight gleams on silent black waters.\\
  Fen-wolves sing to the gibbous Moon\ldots{}
}

\lyricsbs{Bal-Sagoth}{A Black Moon Broods Over Lemuria}{
  As a black Moon broods over Lemuria,\\
  ebon witchfire enshrouds the gleaming citadels,\\
  Sinistrous shadows rise from the vaults of the dreaming elder gods.\\
  Ophidian eyes glimmer through the icy whispering Moon-mist\ldots{}
  
  Winter Moonlight gleams through crooked boughs,
  \emph{the icy caress of night entwines the eon-veiled Obsidian Tower},
  the whisperings of ancient tongues are borne upon the winds,
  dark time-lost spells hold the key to the frost veiled Gate of the Black Moon\ldots{}
}

\lyricsbs{Bal-Sagoth}{Enthroned in the Temple of the Serpent Kings}{
  Deep within the glacial, ice-veiled temple,\\
  ancient enchantments summon the shades of the dreaming Serpent Kings,\\
  and the Ophidian Throne once again draws power from the Moon-shrouded crystal.
  
  Storm-borne bride of winter's fire, serpent-witch of the whispering fens.
  
  Frost-garlanded, the mind-binding \emph{glimmer of tear-filled Ophidian eyes}.
  
  The wyrm-horn sounds 'cross Dagon's mere\ldots{}
}

\lyricsbs{Bal-Sagoth}{Shadows 'Neath the Black Pyramid}{
  I am enraptured by Ophidian eyes.
  
  In the vaults of the dreaming gods, \\
  shackled to the slime-smeared bleeding stone\ldots{}
}

\lyricsbalsagoth{Witch-Storm}{
  The skyqueen of the dead rides forth,\\
  black storm-borne steeds, immortal blood.
  Hark to the striking of the winds, \\
  the moon burns black as slaughter reigns.\\
  Witch-Storm!
}

\lyricsbs{Bal-Sagoth}{%
  Into the Silent Chambers of the Sapphirean Throne 
  (Sagas from the Antediluvian Scrolls)%
}{
  Grim-eyed legions wait brooding\\
  'neath the banner of the Serpent King.
  
  The Atlantean sword beckons me,\\
  and I descend from Moon-shrouded skies into the Tower of the Black Serpent.
  
  And lo, I hear the beat of black leathern wings from Moonless gulfs. \\
  Dark spirits wander the silent halls of the Sapphirean Throne.\\
  And in dreams I see the ocean rise to devour the gleaming spires\\
  as the shades of immortals guide me to the Valley of Silent Paths\ldots{}
  
  The Topaz Throne of Kings is crack'd,\\
  aeon-veiled, enrob'd in black.
  
  Black wings above the land of dreams\ldots{}
  
  The ivory worm now sleeps entombed.
}





\subsubsection{Starfire Burning Upon the Ice-Veiled Throne of Ultima Thule}
\lyricsbs{Bal-Sagoth}{
  The Splendour of a Thousand Swords Gleaming Beneath the Blazon of the Hyperborean Empire
}{
  I see a land far to the north\ldots{} \\
  a vast empire of dark endless moors and snow-crowned mountains\ldots{} \\
  a land of brooding citadels and warrior-kings who hail to grim gods.
}





\subsubsection{The Chthonic Chronicles}
\lyricsbalsagoth{The Fallen Kingdoms of the Abyssal Plain}{
  We know of the hidden and silent places, the places which reside in between the veils of reality, the places which mankind was never meant to see.\\
  All this we know\ldots{} we who survive, we who are descended from those First Ones, and who give thanks to the gods-who-are-not-gods, for our creation, our genesis, for the breath of life that was forced into our progenitors during the early epochs of this cratered globe.\\
  Hearken, children of the Ersatz gods, sons and daughters of the New Earth, for here is truth\ldots{}
}









\subsection{Warhammer 40,000}
\lyricsauthorbookpage{Ben Counter}{Galaxy in Flames}{114-118}{
  \ldots{} a vision of broken steppes spread out before Loken, expanses of desolation and plains of rusted machinery like skeletons of extinct monsters. 
  A hive city on the distant horizon split open like a flower, and from its broken, burning portals rose a mighty tower of brass that punctures the pollution-heavy clouds. 
  
  The sky above was burning and the laughter of Dark Gods boomed from the heavens. 
  
  \ldots{}
  
  He wrenched his mind away from the dying world, and suddenly he was soaring through the galaxy, tumbling between the stars. 
  He saw them destroyed, bleeding glowing plumes of stellar matter into the void. 
  A baleful mass of red stars glowered above him, staring like a great and terrible eye of flame. 
  An endless tide of titanic monsters and vast space fleets vomited from that eye, drowning the universe in a tide of blood. 
  A sea of burning flames spat and leapt from the blood, consuming all in its path, leaving black, barren wasteland in its wake. 
  
  \ldots{}
  
  \emph{You are seeing the galaxy die.}
  
  \ldots{}
  
  Loken could see it where the flames burned through, the endless churning mass of the warp at the heart of everything and the eyes of dark forces seething with malevolence. 
  \ldots{}
  Bloated monsters, their bodies heaving with maggots and filth, devoured dead stars as a brass-clad giant bellowed an endless war cry from its throne of skulls and soulless magicians sacrificed billions in a silver city built of lies. 
  
  \ldots{}
  
  Sindermann had been right. 
  Loken was hearing the music of the spheres, and it was a terrible sound that spoke of corruption blood and the death of the universe. 
}







\lyricsauthorbookpage{Ben Counter}{Galaxy in Flames}{211}{
  Images tumbles through his brain, dark and monstrous, and he fought to hold onto his sanity as visions of fpure evil assailed him. 
  
  Death, like a black seething mantle, hung over everything.
  
  Tendrils of darkness wound through the air, destroying whatever they touched. 
  He screamed as he saw the flesh sloughing from Mersadie's bones, looking down at his hands to see them rotting away before his eyes. 
  His skin peeled back, the bones maggot-white. 
}







\lyricsauthorbookpage{James Swallow}{The Flight of the Eisenstein}{262-265}{
  She screamed and flailed at him, beating her hands on his chest plate. 
  He could see now where her hands were bloody with self-inflicted scratch marks. 
  \ta{Eyes and blood,} she wailed. 
  \ta{But inside the pestilence!}
  
  \ldots{}
  
  \ta{Unclean, unclean! \ldots{} 
  I have seen it! Inside the eyes!} 
  She tore wildly at her face, ripping the skin and drawing blood. 
  \ta{You see it too!}
  
  \ldots{}
  
  The adjutant\ldots{} pressed into him and raked bloody fingers over his torso. 
  \ta{You see!} she gasped. 
  \ta{Soon the end comes! All will wither!}
  
  \ldots{} The woman's upturned face became paper, aged and crumbling. 
  She slid away from him\ldots{} turning into rags of meat and dead flesh, ash and then nothing. 
  
  \ldots{}
  
  The resplendent marble-white of their \armour bled away to become dis\coloured by a feeble, sickly green, the shade of new death. 
  The ceramite warped and became rippled, merging with their flesh until it stained and throbbed. 
  Parasites and bloated organs pulsed within, and in some places wounds opened like new mouths, red-lipped with tongues of distended bowel and duct. 
  
  \ldots{} The malformed shapes of his warriors crowded in, words falling from their cracked, lisping maws. 
  \ldots{} 
  Beyond the men he saw a ghostly form towering above them, too tall to fit into the cramped corridor yet there before him, beckoning with skeletal claws. 
  
  \ta{Mortarion?} he asked. 
  
  The twisted image of his primarch nodded, the figure's blackened hood dipping in sluggish acknowledgement. 
  What Garro could see of his primarch's \armour was no longer shining with steel and brass, but dis\coloured and corroded like old copper, wound with soiled bandages and soiled with rust. 
  The Death Lord was no more and in his place stood a creature of pure corruption. 
  
  \ta{Come, Nathaniel.} 
  The voice was a whisper of wind through dead trees, a breath from a sepulchre. 
  \ta{Soon we will all know the embrace of the Lord of Decay.} 
  
  \tho{The end comes.} 
  The words tolled in his mind like a bell, and Garro looked down at his hands. 
  His gauntlets were powder, flesh was sloughing off his fingers, bones emerging and turning into blackened twigs. 
  \ta{No!} he forced the denial from his throat. 
  \ta{This will not be!} 
}









\subsection{Others}





\subsubsection{Ancient Rites}
\lyricsbs{Ancient Rites}{Dim Carcosa}{
  Black stars shine on the ancient fortified town.\\
  The sun invisible or since long down?\\
  Over the dismal landscape\ldots{}\\
  Above Carcosa
  
  No sound, only the wind sighed.\\
  Behind mysterious moons, strange towers hide.\\
  But even more distant is\ldots{}\\
  Lost Carcosa.
  
  Tales that the Hades will sing,\\
  vague stories of a yellow king,\\
  must die untold in\ldots{}\\
  Strange Carcosa.
  
  Mysteries hidden by lake Hali's nebulous depths.\\
  A presence of bizarre beauty and dread\\\
  remains unrevealed\ldots{}\\
  In Carcosa.
  
  Above the desert high,\\
  twin suns circle the sky.\\
  Nevertheless dim still\ldots{}\\
  Is Carcosa.
  
  My voice turns weak, lost is my mind.\\
  I see, but I am blind.\\
  And no sign of life in\ldots{}\\
  Dim Carcosa.
}





\subsubsection{Clive Barker}
\lyricsbs{Clive Barker}{The Son of Celluloid (Books of Blood II p.24)}{
  \ldots{} the illusion of life created from a perfect sequence of little deaths.
}

(In Barker's story, this is a reference to movies. It can also describe comics or tapestries, which are more common in Carzain's time than movies\ldots{})





\subsubsection{\RWCTKIY}
\lyricstitle{Cassilda's song in \emph{The King in Yellow}, 
act 1 scene 2}{
  Along the shore the cloud waves break.\\
  The twin suns sink beneath the \hr{Lake in Malcur}{Lake}.\\
  The shadow lengthens\\
  in Carcosa.
  
  Strange is the night where \hr{Black stars of Nyx}{black stars} rise\\
  and strange \hr{Moons}{moons} circle through the skies.\\
  But stranger still is \\
  Lost Carcosa.
  
  Songs that the Hyades shall sing,\\
  where flap the tatters of the King,\\
  must die unheard in \\
  Dim Carcosa.
  
  Song of my soul, my voice is dead.\\
  Die though, unsung, as tears unshed\\
  shall dry and die in\\
  Lost Carcosa.
}





\subsubsection{Slayer}
\lyricsbs{Slayer}{Raining Blood}{Raining blood from a lacerated sky.\\
  Bleeding its horror.\\
  Creating my structure.\\
  Now I shall reign in blood!
}







































\chapter{Miscellaneous}















\section{Ages of the World}
\target{Ages of the World}
\target{Age of Gods}









\subsection{Immortal view}
The immortals dealt with a number of \quo{ages}, including:

\target{Age of the Shroud}
\begin{itemize}
  \item The age of the \Ophidian Civilization.
  \item The \hr{Saphyrae}{\Saphyraean} age.
  \item The age of \hr{Draconian Supremacy}{\Draconian Supremacy}.
  \item The age of the \hr{Resphan wars}{\resphanwars}.
  \item The age of the \hs{Shroud}. 
\end{itemize}










\subsection{Mortal view}
Among mortal scholars, the history of \Miith{} is sometimes semi-formally divided into a number of \quo{ages}: 

\begin{description}
  \item[The Age of Gods:] 
    Everything before the advent of the \VaimonCaliphate. 
    Allegedly, \hr{Myths of vanquished monsters}{\Miith{} was ruled by Elder Races and inhuman monsters}.
  
  \item[The \Human Age:]
    \target{Human Age}
    Called the \quo{Vaimon Age} by \human{} scholars and the \quo{\Human{} Age} by nonhumans. 
    This age began with the founding of the \VaimonCaliphate by Cordos Vaimon and ended with the \Darkfall. 
  
  \item[The \Scatha Age:]
    \target{Scatha Age}
    Everything after the \Human{} Age, according to some \scathaese{} scholars living in the period. 
    Named so because most of the major empires in this period were \scatha-dominated:
    \Ortaica, Durcac, the Imetrium, \Tepharae{}, \Velcad. 
    
    Others prefer to subdivide it into some more, shorter ages:
    
    \begin{description}
      \item[The \Ortaican Age.]
      \item[The \Tepharin Age.]
      \item[The \Velcadian Age.]
    \end{description}
    
    Whatever you preferred to call it, it was true that at this point the age of \human{} dominion was over. 
    \Redce, Geica and a few other strongholds were the last bastions of the \human-dominated world. 
    
    But \humanity{} still had an important part to play: 
    The Advent of \hr{Lithrim}{\Lithrim}. 
\end{description}

Even in the \Human Age and the \Scatha Age, the ancient elder peoples still rule \Miith:

\citebandsong{Nile:AnnihilationoftheWicked}{Nile}{
  Sss'Haa Set Yoth
}{
  Lurking Among Us Hidden in Obscurity,\\
  Descended from the Dawn of the Ages,\\
  The Children of Yig And Set, Serpent Volk.\\
  Whose Civilaztions were Ancient, Long Dead And Forgotten\\
  Before The Eldest Pyramids were Built.\\
  Man Was Not the First to Walk Upright Upon the Earth.

  Cruel, Pitiless, of a Cunning Mind.\\
  Unwilling to Relinguish Dominion of the Earth.\\
  They Are Cold, Knowing No Remorse.\\
  They hath Made UnWitted Slaves of a Blinded Humanity.
}
















\section{Domestic animals}
\target{Domestic animals}
Many cultures domesticated animals and monsters and used them as mounts, beasts of burden and in war. 

\paragraph{Cultures}
\begin{description}
  \item[\hs{Imetrium}:] 
    Used \hr{Nycan}{\nycans} and \hr{Nycaneer}{\nycaneers}, as well as \hr{Imetric monsters}{various monsters, especially sea monsters}. 
    
  \item[\hs{Rissitics}:] 
    \hr{Rissitic monsters}{Used all sorts of monsters} and dinosaurs, including \lothae, \mezolisks and \corgoroth.
    
  \item[\hs{Vaimons}:]
    Used mostly herbivorous dinosaurs, such as \hs{sauropods} and hadrosaurs and pachycephalosaurs. 
\end{description}

\paragraph{Animals}
\begin{description}
 \item[\hr{Relc}{\Relcs}:] 
    Mostly military animals.
    Armies had them, but few civilians. 
    Civilians would use other and slower animals.
\end{description}


















\section{\Draconian Calendar}
\index{\Draconian{} Calendar}
The calendar used by \dragons. 
The \quo{first year of \Draconian{} Supremacy} (1 \DS) was the year where \Tiamat{} awakened the \xss. 

Year \yic{VC} is equal to year \yds{VC}. 















\section{Forbidden books}
Have some forbidden books and the like. Compare to the sizable library of fictional books in the Cthulhu Mythos (by H.P. Lovecraft and others), of which the most well-known is the Necronomicon. 

Some of them are written in blood.









\subsection[The Bath Shem Torradj]{The \BathShemTorradjErebossha}
\target{Book of Erebos}
\target{Bath Shem}
The \BathShemTorradjErebossha{} is 
%There is a \BookofErebos, 
a book (or something) written by \hr{Semiza}{\Semiza}, containing all the dark secrets of \hr{Erebos}{\Erebos}. It details the gruesome history of the \hr{Bane}{\bane} people, how \hr{Banes are created}{they were created} (possibly by \hr{Evil Cosmic God creates Banes}{an evil cosmic god}), how they conquered \Erebos{} and wiped out all other civilizations on the planet, and eventually even \hr{Banes destroy Voyagers}{destroyed their own creators}, the \hr{Voyagers}\voyagers.
% the \banes{} and the origin of \resphain{} and \humans. 

It goes on to describe how the \banes{} ravaged their way across the universe, finally coming to \Miith{}, and how they enlisted \Semiza{} and his sorcerers and created the \resphain{} as their successors and the \humans{} as their slaves. 

It is comparable to the Empyreal Lexicon from \bandalbum{Bal-Sagoth}{The Power Cosmic}.

\lyricsbalsagoth{The Empyreal Lexicon}{
  Behold the cosmic codex! The tome of the astral abyss!
  
  Such diabolical evil\ldots{}\\
  sublime macrocosmic malevolence!
  
  Such carnage wrought with your malevolent tongue, dark one\ldots{}\\
  What unfathomable horrors dwell within the lightless corners of your cursed soul?\\
  Heed not the voice of the Lexicon\ldots{} lest its whispers drive you mad!
}

\Semiza{} also describes the \psp{\banelords}{} thousand-year-long plans that were laid, and their many weapons to control \humanity.
There are things which one, as a \human, can recognize when reading them. It is frightening reading, because it reveals a lot of things that a \human{} would not want to know. No one wants to be told that he is created as a slave of a race of cruel aliens. 









\subsection{The Star-Maps of the Ancient Cosmographers}
The \hr{Star-Maps of the Ancient Cosmographers}{ancient cosmic maps created by \Tiamat{} and \ApepNesthra}. Compare to \bandsong{Bal-Sagoth}{Star-Maps of the Ancient Cosmographers}. 

\lyricsbalsagoth{Six Score and Ten Oblations to a Malefic Avatar}{
  Lore dating from time immemorial; lore surviving in the records of long extinct civilizations, be it inscribed upon parchment now crumbled to dust, etched into the sand-whipped, glyph-scored stone of hoary temples, or committed to verbal traditions long since ingrained into some collective tribal memory.\\
  This is no globally common myth cycle, no collection of universally allegorical folk tales; it is all cold, pitiless truth!
  
  What titanic demi-gods once strode the boiling surface of the young earth, treading the shattered surface of mighty Pangaea beneath their ersatz feet?\\
  What fearsome entities were already inestimably ancient when mankind himself was naught but a collection of mindless random atoms, a viscous puddle of gelid protoplasm teeming with the raw materials of life, transient cells of primordial slime, all naught but malleable and tractable clay to be worked at by unimaginable sculptors, immortal star-spanning fiends!\\
  What inhuman eyes even now watch the inconsequential toilings of man from afar?\\
  The answers to these questions of denied primacy and direful cosmogony were too repulsively horrific to contemplate, and yet\ldots{}\\
  I knew the truth!\\
  And more terrifying still\ldots{}\\
  The Z'xulth and their villainous agents of depravity even now walk among us!
}









\subsection{Wanderers in Darkness}
\target{Poem}
\target{Wanderers in Darkness}
There exists a dark, cryptic poem, entitled \emph{\hr{Wanderers in Darkness}{\WanderersInDarkness}}. 
It is filled with cosmic dread and beauty\ldots{} and wicked truth. 

Alternate titles:
\begin{itemize}
  \item 'Neath the Black Stars. 
  \item 'Neath the Cruel Stars. 
  \item Wanderers of the Formless Land (like \bandsong{Blind Guardian}{Punishment Divine}). 
  \item Shapeless Paths. 
  \item The Nocturnal \hs{Aenigmata}. 
  \item The Black Stars' Aenigmata. 
\end{itemize}





\subsubsection{Contents}
It is originally written in \hr{True Draconic}{\TrueDraconic} with \hr{Draconic runes}{runes}. 
There is much Gnosis contained in those runes and words. 

It is really about the three sons of \hr{Khoth-Sell}{\KhothSell}. 

In some sense it is the ultimate art. But it is \emph{too} deep, too wise, too transcendent. It is gruesome and horrifying for mortals, and even for immortals.

Remember that \Nexagglachel, one of the three brothers, lives on in the \satharioth. Thus, the poem is also about them, and they are \quo{bound} by it. 

It is partially about \Nexagglachel{} and \hr{Curse}{his \quo{ghost} that still haunts the \satharioth} and wanders in their inner darkness.

All sorts of characters appear in the poem. They have titles that hide their true identities, which are known only to the wise. These identities include \Ishnaruchaefir, \Secherdamon{} and more. 

A mystic star appears: Kya'hsim. 
Compare to the Hyades mentioned in \RWCTKIY. 

It has many references to the mourning \hr{Nymph}{\nymphs}. 





\subsubsection{History}
\WanderersInDarknessEmph was written in the early centuries after the \secondbanewar and the \Shrouding. 

The poem was attributed to \quo{\hr{Melcryth}{\Melcryth}}, which was probably a pseudonym (compare to \quo{Homer}). 
There existed several different, conflicting versions. 

It was translated into the \resphan tongue by \hr{Essenai}{\Essenai}. 





\subsubsection{\Ishnaruchaefir}

\Ishnaruchaefir{} feared the poem, because it revealed a deep, sinister truth about him, his nature and his likely destiny. 
Something which he was fleeing from and did not want to acknowledge. 

The poem contained insights that could potentially imperil \Ishnaruchaefir by revealing his weaknesses. 
So \Ishnaruchaefir tried to sabotage the poem. 
At first he tried to destroy or suppress the poem, but this failed and he had to give up. 
He did, though, manage to eradicate some of the fragments that he feared. 
Then he took up another strategy. 
He planted some fake, apocryphal fragments, so there would be several versions contradicting each other, thus muddying the big picture and undermining the credibility and deeper value of the poem. 





\subsubsection{Inspiration}
Compare to \RWCTKIY.

\lyricsbs{Robert W. Chambers}{The Repairer of Reputations}{
  \ta{%
    I\ldots{} wept and laughed and trembled with a horror which at times assails me yet. 
    
    This is the thing that troubles me, for I cannot forget Carcosa where black stars hang in the heavens; where the shadows of men's thoughts lengthen in the afternoon, when the twin suns sink into the Lake of Hali; and my mind will bear forever the memory of the Pallid Mask. 
    
    I pray God will curse the writer, as the writer has cursed the world with this beautiful, stupendous creation, terrible in its simplicity, irresistible in its truth\ldots{} 
    
    \ldots{} all felt that human nature could not bear the strain, not thrive on words in which the essence of purest poison lurked.}
}

Compare to the Deck of Dragons and the Tiles of the Cedance in \cite{StevenEriksonIanCameronEsslemont:MalazanBookoftheFallen}. 
 














\section{Glossary}









\subsection{Creatures}









\begin{gloss}







\begin{comment}
\paragraph{\drake}
\end{comment}
\gitem{\drake}
\index{\drake}
\target{Drake}
A group of reptile-like animals. 
They are half warm-blooded and half cold-blooded. 

Species include \hr{Ophidians}{\ophidians}, \hr{Dragon}{\dragons}, \hr{QJ}{\quiljaaran} and \hr{Mezolisk}{\mezolisks}. 

Related to, but considered distinct from, \hr{pteran}{pterans}, \hr{saurian}{\saurians{}} and reptiles. 

\meta{%
  Dinosaurs. }









\begin{comment}
\paragraph{pteran}
\end{comment}
\gitem{pteran}
\index{pteran}
\target{Pteran}
\target{pteran}
Flying reptile-like creatures related to \hr{saurian}{\saurians{}}. 
Species include the \ravcor{} and the \quilrai. 

\meta{%
  Compare to pterosaurs like \latinname{Pteranodon}.}







\begin{comment}
\paragraph{\saurian}
\end{comment}
\gitem{\saurian}
\index{\saurian}
\target{Saurian}
\target{saurian}
Any of a diverse group of reptile-like animals. 
Many species are large. 
Species include the \hr{Nycan}{\nycan}, \hr{Relc}{\relc}, \hr{Corgorah}{\corgorah}, \hr{Mulgron}{\mulgron} and \hr{Tondra}{\tondra}. 
Related to \hr{pteran}{pterans} and \hr{Drake}{\drakes}. 

\meta{%
  Dinosaurs. }















\end{gloss}










\subsection{Miscellaneous}









% \section{Other stuff}
\begin{gloss}
  
  
  
  \begin{comment}
  \subsubsection{A-G}
  \end{comment}
  
  
  
  \begin{comment}
  \paragraph{Becallios}
  \end{comment}
  \gitem[Becallioi]{Becallios}
  Imetric term for \quo{devil}. Used as an exclamation, it is a strong curse and not to be used in finer company. 
  
  
  
  \begin{comment}
  \paragraph{\dai}
  \end{comment}
  \gitem{\dai-}
  In Imetric, \quo{\dai-} can be prefixed to a name, title or pronoun to form a polite vocative. (\quo{\Dai-} is capitalized if prefixed to an already capitalized word, such as a name, but otherwise not.)
  
  
  %\gitem{Dorlinum breed}
  %Dorlinum \nycans{} are those bred in the Imetric city of Dorlinum. Especially renowned are the Dorlinum Secca, who are some of the largest and strongest \nycans{} known. 
  
  
  
  
  \begin{comment}
  \paragraph{gness}
  \end{comment}
  \gitem[gnesses]{gness}
  \index{gness}
  \target{gness}
  \target{gnesses}
  Rissitic slang for the \scathaese{} hemipenis. A male \scatha{} does not have a \human{}-like penis but two small hemipenes. 
  
  
  
  
  
  
  
  
  
  \begin{comment}
  \paragraph{god}
  \end{comment}
  \gitem{god}
  \index{god}
  \target{god}
  \target{gods}
  Generic term for \hs{immortal} beings of great power. 
  Examples include the \hs{Imetric} and \hs{Rissitic} gods. 
  
  
  
  \begin{comment}
  \subsubsection{H-N}
  \end{comment}
  
  
  
  
  
  
  
  \begin{comment}
  \paragraph{invocation}
  \end{comment}
  \gitem{invocation}
  \index{invocation}
  \target{invocation}
  \target{invocations}
  A spell that calls upon one or more entities and attempts to compel said entities to do the mage's bidding. 
  Invocations are more difficult to learn than \hs{orisons}, but more powerful. 
  
  
  
  
  
  
  \begin{comment}
  \paragraph{\ishrah}
  \end{comment}
  \gitem[\ishroth]{\ishrah}
  \target{Ishrah}
  \index{\ishrah{} (plural \ishroth)}
  A group of mages working together. 
  
  
  
%   \gitem{Mictzan breed}
%   The Mictzan is a sub-breed of the Destran race of \nycans{}. It is named for a Clictuan \scatha{} who founded the breed. 
  
  
  
  \begin{comment}
  \paragraph{\MotherTiamat}
  \end{comment}
  \gitem{\MotherTiamat}
  \Narkiza's flagship, named after the \Dragon{} Mother of myth. She is 40 metres long, carved in the likeness of a many-headed \dragon{} and armed with ten ballistae. 
  
  
  
  \begin{comment}
  \paragraph{nicca}
  \end{comment}
  \gitem{nicca}
  The hips on a \scatha{}. The Niccas are considered a sexual body part, important in determining attractiveness, and many cultures dictate that they must be covered in public. 
  
  
  
  \begin{comment}
  \paragraph{North Star}
  \end{comment}
  \gitem{North Star}
  \target{North Star}
  A star lying close to \Miith{}'s rotational axis in the northern direction. 
  It is a white star of spectral class A and one of the brighter stars of the northern sky. Often considered by astrologers to have important occult qualities. 
  The Northstar clan, in addition to their Imetric religion, venerate the North Star. 
  Called \word{Telcastora} in the \hr{Ortaican language}{\Ortaican language}. 
  (Note that since \Miith{} is not Earth, the North Star is \emph{not} the same as Polaris, which is Earth's north star at the moment.)
  \also{Northstar clan}
  
  
  
  \begin{comment}
  \subsubsection{O-U}
  \end{comment}
  
  
  
  
  
  
  
  \begin{comment}
  \paragraph{orison}
  \end{comment}
  \gitem{orison}
  \index{orison}
  \target{orison}
  \target{orisons}
  A simple spell. 
  Consists of calling upon one or more entities and making a petition. 
  Orisons are far easier to learn and cast than \hs{invocations}, but not nearly as powerful or flexible. 
  
  
  
  \begin{comment}
  \paragraph{\raebar}
  \end{comment}
  \gitem[\raebari]{\raebar}
  \target{Raebar}
  \index{\raebar}
  A member of a \baccon. 
  
  
  
  
  
  
  
  \begin{comment}
  \paragraph{Real Life}
  \end{comment}
  \gitem{Real Life}
  \index{RL}
  \quo{Real Life}, abbreviated RL, refers to the real world, as opposed to \Miith{}. I use the term when comparing \Miithian{} phenomena to those of the real world. 
  
  
  
  
  
  
  
  \begin{comment}
  \paragraph{shade pearl}
  \end{comment}
  \gitem{shade pearl}
  A pearl harvested from a rare oyster in the Far Orient, dark gray in \colour and around 2 cm in diameter. They are useful as an ingredient in a spell that creates a \quo{shroud} which conceals magic cast inside it from outside detection. 
  
  
  
  
  
  
  
  \begin{comment}
  \paragraph{sorcerer}
  \end{comment}
  \gitem{sorcerer/sorceress}
  A practitioner of \quo{sorcery}. 
  \also{sorcery}
  
  
  
  
  
  \begin{comment}
  \paragraph{sorcery}
  \end{comment}
  \gitem{sorcery}
  \target{sorcery}
  \index{sorcery}
  A word for magic. 
  Has multiple possible meanings: 
  
  \begin{enumerate}
    \item 
      Used by laymen for magic that is considered \quo{dark} or \quo{forbidden}. 
    \item 
      Used by the \hs{Iquinian Church} as a disparaging term for non-Iquinian magic. 
    \item 
      Used by the few scholars who know what \hs{psionics} are to refer to magic that involves the summoning of otherworldly entities, as opposed to pure psionics. 
      Sorcery is potentially more powerful, but also far more dangerous and risky. 
  \end{enumerate}
  
  
  
  
  
  
  
  \begin{comment}
  \paragraph{tacupien}
  \end{comment}
  \gitem{tacupien}
  A Pelidorian dance.
  
  
  
  
  
  
  
  \begin{comment}
  \subsubsection{V-Z}
  \end{comment}
  
  
  
  
\end{gloss}
















\section{Myths}
\target{Myths}
\target{myths}
Many people all over \Miith{} have myths about their history and pre-history. Some of these myths are close to the truth, others are very far from it.









\subsection{Cautionary tales}
The \Iquinian church had longer, elaborate myths telling about the things that man did not know, and admonishing man that he ought not to know and should not try to find out.
Compare to \cite[\quo{Yonath the Prophet}]{LordDunsany:TheGodsofPegana}.









\subsection{Creatures beneath the earth}
There are myths of \hr{Myths of creatures beneath the earth}{creatures beneath the earth} whose power and movement affect the surface world. 









\subsection{\Dragons}
\target{Myths about Dragons}
Mortals had myths about \dragons. 

\citebandsong{Nile:Ithyphallic}{Nile}{
  What Can Be Safely Written
}{
  On the walls of lost cities\\
  And in the carvings of madmen\\
  Who have glimpsed him in their dreams\\
  Is his image delineated\\
  Within a tomb protected by great seals he lies in death\\
  Under the weight of the dark waters of the deep\\
  Yet he dreams still, and in his dreams continues to rule this world\\
  For his thoughts master the wills of lesser creatures
}

See also the sections on \hr{Myths of vanquished monsters}{how \human heroes vanquished elder monsters} and on \hr{Iquinian myths about Dragons}{Iquinian myths about \dragons}. 









\subsection{Forgotten a lot}
People remember certain things, but there are also tons of things they don't remember. 
Such as the \banes, the \resphain, the \dragons{} and the \xss{} (most people have never heard the name \quo{\xs}). 

\lyricsbs{Monolith Deathcult}{Deus Ex Machina}{
  My name has sunk into oblivion. \\
  Thou art blinded by false gods \\
  with a ludicrous past of two millennia.
}









\subsection{The \Feud}
In almost all \Miithian{} cultures, stories and myths are told about two secret organizations that exist in the shadows and wage an eternal war for dominion over the planet\dash{}the \hs{Cabal} and the \hr{Sentinels}{Sentinels of \Miith}. 
They are said to have existed for many thousands of years, if not forever, and the Sentinels are associated with \dragons{} while the Cabal is said to be connected with the mysterious \banes{}, but other than that, little is known about them. Some say that the Sentinels are good, nobly defending \Miith{} against the predations of the evil Cabal, whereas others claim that it is in fact the Sentinels who are evil, and yet others assert that both orders are uncaring manipulators that prey on the hapless people of \Miith{}. 

It is said that the conflict between these obscure forces has shaped the entire history of \Miith{} and that all major heroes, leaders and rulers though time have been manipulated by one (or both) of the orders\dash{}and many are the men and women in history who have been publicly accused, even convicted, of being Cabalists or Sentinels. Both organizations are said to be of otherworldly origin and to work though dark sorcery and \daemonic{} allies. 

But few and far in between are the actual, visible indications of their existence, and indeed, today many people believe that the dark orders are but myth and superstition. Most \Velcadians{} comfortably believe that all \dragons{}, or as near all as makes no difference, dwell in remote, legendary Irokas, and that the wicked \banes{} have vanished from \Miith{} many thousand years ago, if indeed they ever existed at all. Even major kingdoms and organizations know of little evidence that the twin orders exist, and so they are for the most part relegated to the realm of legends. 

Few know even fragments of the truth, and fewer still know the whole truth, but the Sentinels and the Cabal are very real and have existed for nearly ten thousand years, and their importance in \Miithian{} history, even today, can scarcely be overestimated. 









\subsection{Gods that came from the stars}
Many \humans{} have myths of gods that came from the stars long ago. This refers to the \resphain{}, who \hr{Resphain come down in spaceships}{descended in spaceships from \Merkyrah}.

\lyricsbalsagoth{
  Starfire Burning Upon the Ice-Veiled Throne of Ultima Thule
}{
  They spoke of a prophecy foretold, \\
  an ancient and glorious legacy.\\
  A quest for the realm of legendry\\
  lost to Man since before even the Star-Lords descended.
}









\subsection{\Humans vanquishing monsters}
There were \hr{Myths of vanquished monsters}{myths about how great \human{}  heroes fought against and vanquished the evil pre-\human{} monsters that had dominated \Miith{} in the ages past}. 









\subsection{\Isphet the Destroyer}
\index{\Isphet}
\hr{Isphet}{\Isphet was an evil figure in \Iquinian mythology}, loosely based on \QuessanthIshnaruchaefir. 











\subsection[Merkyrah]{\Merkyrah}
Some \humans{} have myths about \Merkyrah, where they lived and walked side-by-side with gods and angels. And some unclear, conflicting, dark stories about something evil that crept in and poisoned their paradise. 

The Iquinians \hr{Iquinian myths of Merkyrah}{had such myths}.









\subsection{Merfolk}
There were \hr{Myths about Nagae}{myths about merfolk}.
They were really \nagae.








\subsection{Monsters}
Have myths of monsters and mythological creatures that are twisted versions of the master races and other beings. 

People very rarely see these creatures directly, and few live to tell the tale. Most people see only vague signs that the creatures exist, and so the stories twist and mutate with each re-telling. They end up as fantastic tales whom no one believes, no matter how true. The \hs{Shroud} helps this effect along by making people stupid and more closed-minded. 

\hr{Malcur}{\Malcur}, for instance, might have myths of worms, based on the \hr{Teshrial's creatures}{\ghobaleth{} beneath the city}. 





\subsubsection{Vampires}
As an example, there are myths of vampires that drink people's blood and transform people into new vampires. These are based on sightings of \hr{Resphan vampirism}{blood-drinking \resphain}. 

Of course, in truth, \resphain{} cannot transform people into new \resphain{} by drinking their blood. 

A \resphan: \ta{It is \human{} folly and vanity to believe that you could ever become like us.}









\subsection{Mother \Miith turned away}
There is a myth that Mother \Miith{} once saw that her children waged wars and destroyed, betrayed and devoured one another. 
She was angered and saddened, so she turned her back to them and closed her Heart.
That is why the world today is going downhill: 
It is deprived of the love of Mother \Miith{}. 

This is a corruption of the true story of the \hr{Heart weakened}{weakening of the Heart of \Miith}. 

Compare to the turning away of Mother Dark in \cite{StevenEriksonIanCameronEsslemont:MalazanBookoftheFallen}. 
And the story of Ayen's closing the gates of Paradise in \authorseries{Alan Campbell}{Deepgate Codex}. 









\subsection{People don't believe in things}
See the section \quo{\hs{People don't believe in things}}. 









\subsection{Sky falling down}
There is a myth about how the sky once fell down upon \Miith{}. It refers either to the \secondbanewar{} or to the \CuezcanApocalypse. 

Probably the \secondbanewar. This separates the \quo{age of myths} from the \quo{age of legends}. The \CuezcanApocalypse{} separates the \quo{age of legends} from historical time. 

\lyricsbalsagoth{In Search of the Lost Cities of Antarctica}{
[The Testament of the Winds:]\\
Before the Third Moon fell from orbit, before the\\
Nine Continents were formed from Pangaea's shattered surface\ldots{} 

On the day before the Third Moon's fall from the darkened sky, \\
on the day before the Cataclysm the stars align.
}









\subsection{Wars among the gods}
Many people have myths of wars fought between different factions of gods. This may refer to the \banewars, the \resphanwars{}, the \Merkyran{} war or even the \hr{Feud}{\feud}. 















\section{Psychology}









\subsection{Differences from real life}
\target{Races love war}
Make clear that \Miithian psychology differs from RL psychology.
In RL it is often asserted that all true happiness derives from \quo{love}, and that all other desires are hollow and unfulfilling in the end.
This may or may not be true in RL, but it is not true on \Miith.
\Resphain and \dragons alike derive genuine pleasure from conflict, from living out \quo{negative} emotions such as anger, hate and pride.

That is one reason why there has been so much war:
Peace is not happy for these warrior races.

In a sense, this means the \resphain and \dragons were inherently \quo{insane}, since they naturally tended towards behaviour that was destructive for them in the end.
They also had an unfortunate tendency to idolize and be attracted to such destructive, \quo{insane} behaviour.
That was one reason why the \hr{Sathariah social status}{\satharioth were so popular} despite their derangement (at times very obvious, as with \Zachirah). 

Ramiel, for one, \hr{Ramiel scored as Sathariah}{attracted more \resviel} because of his inner demons and darkness and insanity.

See also the section on how \hr{Resphain love conflict}{the \resphain enjoyed conflict}. 









\subsection{Inventiveness}
\target{Inventiveness}
\quo{Inventiveness} is not an intrinsic quality of intelligent minds. 
Many races lack it, such as the \hr{QJ not inventive}{\quiljaaran} and \hr{Entropy}{\banes}. 








\subsection{Madness and insight}
\target{Madness}
In some cases, madness is the key to insight. Of course, if you are weak\dash or worse, stupid\dash madness is likely to crush you, and you will never recover. But if you are strong, or more importantly, clever, \quo{madness} can be a good thing.

Remember: To be \quo{mad} simply means that your way of thinking is sufficiently aberrant that you cannot function in society. Madness is the breakdown of conventional reasoning. But recall that within the Realm of the Shroud, conventional reasoning is wrong! 

As such, the madness caused by extreme stress, despair and shock can awaken some of the primal mind within you, the pure animal mind, still (mostly) untainted by the \hs{Lie Sublime}. This can allow you to see through the Shroud and into true reality, and thus attain insight that would be otherwise hidden. 

\lyricsbalsagoth{The Dreamer in the Catacombs of Ur}{
  \ldots{} a tragic shell of the man I once knew, a man beset by imagined terrors and ever wary of the immemorial horrors which he claimed lurked at the periphery of humanity's perceptions.
}





\subsubsection{Madness and power}
Madness can be the key to mystic power. The glimpses into the Beyond that madness can grant you may reveal secrets of power. This can be instinctive, animal power such as that employed by berserkers (see section \ref{Berserkers}), or it can be sophisticated arcane power. In order to use the latter, you need to either be a mage already, or be extremely intelligent and quick to grasp new things. 

Moro \Cornel{} (see section \ref{Moro Cornel}) is an example of a character who found power through madness. 

\lyricslimbonicart{A Cosmic Funeral of Memories}{
  The night belongs to the predator\\
  to the one who dares crossing the threshold,\\
  the axis of dreams and wonders and black miracles.\\
  I see my self in the mirror of your eyes:\\
  A dark star on the celestial beautiful midnight sky.\\
  Release the inner radiance of what you have become.\\
  Brighten the night with your sacrifice.
  
  I have returned to life to speak of clairvoyance.\\
  I am the voice from the grave land of memories.\\
  Life can be only an illusion, and death a temptation,\\
  a final destiny.
}

\lyricsbs{Aeternus}{Burning the Shroud}{
  Within you it lies since the rise of your time.\\
  You've feared it, ignored, but truth never lies.\\
  Its existence marked at the birth of time.\\
  The darkness inside.\\
  The shadows that dance in your soul.
  
  To deny is to fear.\\
  Embrace the darkness.\\
  Reveal in its power.\\
  Within it lies the harmonious truth.\\
  A voyage into the cave of shadows\\
  yields the jewels of knowledge and wisdom.
  
  Does not the night\\
  and the silver moon\\
  enchant your being?
  
  The caressing whisper of the night.\\
  No harm will come to those who are true of heart.\\
  Look to the sun and the brightness shall blind you.\\
  Stare into the black embrace,\\
  the sight deepened with wisdom.
  
  Free your consciousness.\\
  Accept the reality of darkness.\\
  It has never been a place of dread.\\
  Knowledge and wisdom\\
  is given to those who seek.
  Achieve the empowerment\\
  of endless mysterious nights.
}







\subsubsection{The price of madness}
\target{The price of madness}
The power and/or insight granted by madness sometimes comes at a price. Your humanoid body is a thing of the Shroud, remember. So if you embrace madness and the Beyond, your mind will become twisted out of its Shrouded form, and your body will follow suit. 

This is traumatic in itself, so it will tend to inspire further shock, despair and madness. Which will cause your body and mind to become further twisted. 

If the power or frame you have embraced is one of the \Wylde{}, you will find your body degenerating into a more bestial form. If you are attuned to a specific kind of animal or monster, your body is likely to mutate into a form resembling this beast. Berserkers and \rangers{} are prone to this. 

Compare to Declan from Clive Barker's \emph{Rawhead Rex} (\emph{Books of Blood III}), who ends up worshipping the monstrous Rawhead as a deranged madman.





\subsubsection{Asylums}
\target{Asylums}
Have some nasty asylums where the Shroud is worn thin. These are gateways to the Beyond, where all sorts of horrors can slip through. 

Compare to the \emph{Kult} RPG, and the trope \trope{BedlamHouse}{Bedlam House}. 









\subsection{Passion}
\Dragons, \resphain, \cuezcans{} and, to some degree, \nephilim, draw much of their strength from their passions, their emotions. They are strong, assertive and ruling through and by virtue of their passions. 

But the slave races (\scathae{} and \humans) are enslaved, numbed and imprisoned as crippled weaklings in the Shroud, precisely \emph{because} of their emotions. Feelings of fear, devotion, loyalty, pride and even love, all these are things that weave the Shroud, keep it in place, and keep people imprisoned there.

Such is the deviousness of the master races: They have taken the source of their own greatest strength and perverted it, turned it into their slaves' greatest weakness. By harnessing people's emotions, they turn their own nature against them and keep them subjugated. 















\section{Slavery}
\target{slave}
Many nations have slaves. Pelidor and Runger are among them, as is Belek. 
\Redce{} and the Imetrium are perhaps the only major kingdoms that outlaw slavery. 

The \Velcadian{} church approves of slavery, because there is a natural hierarchy in the world, which people should just accept. Remember, humility and servitude are important \hs{Iquinian} virtues. 

The Redcor oppose nominal slavery, but as it turns out, the Redcor world view places pretty much everyone else as a rightful slave of the Redcor. They see themselves as the natural ruling caste because of their superior insight and virtue, and as a result, everyone should bow down to them, revere them as demigods and obey their least whim. At least, this is the impression that Carzain gets of them. Which makes him hate their guts. 

The Imetrians have \hr{Imetric slavery}{enforced labour for criminals}, which is practically slavery. 

Geica has a system where \hr{Geican slavery}{deeply indebted people end up as practically slaves}. 















\section{Sterilization}
\target{Sterilization}
There exists a Chaos spell that can permanently sterilize a \human{} woman. 
It works by attaching a \hr{Daemon}{\daemon} to the woman's womb. 
Whenever she gets pregnant, the \daemon{} will kill and devour the child, feeding on its lifeforce. 
This is horrible\dash the mother feels her baby's pain as it dies. 

There is also a \qliphah{} that can achieve the same result, but nicer and without the pain. 
(%
  Maybe the child still feels the pain. 
  It's just \hr{Dweomer filters}{filtered away} though the \Itzach{} Shroud so the mother doesn't realize it.%
) 

All this works only on \humans{}. 
Other races require different spells. 























\section{Story ideas}
\subsection{Names}
Some cool names that I might use:

\begin{description}
  \item[From the Hebrew Bible:]
    \ 
    \begin{itemize}
      \item Athraim (from the Bible, Numbers 21:1).
      \item Azariah/Azariel. 
      \item Bethsaida.
      \item Cushan-Rishathaim (from the Bible, Judges 3:8).
      \item Ishbaal (from the Bible; 2 Samuel 2:8)
      \item Issachar (from the Bible; son of Jacob and Leah).
      \item Mahalath (from the Bible; daughter of Ishmael and wife of Esau). 
      \item Shadrach. 
      \item Shephatiah (from the Bible; 2 Samuel 3:4)
    \end{itemize}
    
  \item[Others:]
    \ 
    \begin{itemize}
      \item Capharnaum.
      \item Corozain.
      \item 
        Dante.
        With his two sons: Dexter and Sinister.
      \item 
        Urax: 
        A cruel warrior god. 
        Compare him to the Egyptian Unas, as featured in the song \bandsong{Nile}{Unas, the Slayer of the Gods}. 
    \end{itemize}
\end{description}











\subsection{Attacking a city to draw out your enemies}
Someone attacks a city in an attempt to force their enemies to come out of hiding and face them. 

Perhaps in connection with \Malcur in \TwilightAngelRememberEmph. But maybe not. 









\subsection{Body-snatcher}
Maybe have a creature that slowly steals a person's soul and causes his body to rot and decay, as the creature steals his form and identity. 

Inspired by the monster from Clive Barker's \emph{Human Remains} (from the \emph{Books of Blood III}). 









\subsection{Collector of souls}
Have a character who goes around seeking out sorrowing lost souls, people who have nothing left to live for and only suffer. He kills them, then absorbs and binds their souls. He taps their sorrow, tears and pain. 

\ta{They were suffering anyway, but in this way, their suffering has a purpose. That is my gift to them.}









\subsection{Colour vision}

\target{Curiet's colour vision}
Curiet is a fop who cares very much about his clothes. Remember to have a scene where he talks about some \hr{Scathaese colour vision}{\colour nuances that only \scathae{} can tell apart}.









\subsection{Crying for your lost loved ones}
Have scenes where people cry for their killed loved ones. 

Be cruel! Be evil! Be cynical! Be grim! Be necr0! Be simultaneously grim and necr0 if at all possible. 









\subsection{Ghostly scene with a monster}
Have a scene where suddenly everything turns quiet, misty, ghost-like. And then a monster appears from out of the Beyond. 

Like in the anime \emph{Devil May Cry}, episode 2. 









\subsection{Girl is unable to kill}
Have a scene where Guy tells Girl: \ta{Kill me!}

Girl: \ta{No! I can't!}

Guy: \ta{You must! It's our only chance!}

Girl: \ta{No!}

In the end, she doesn't. As a consequence, they both die a horrible death. 

This is a subversion of the scene in episode 18 of the anime \cite{Anime:TrinityBlood}, where Esther refuses to kill Ion.









\subsection{Killed and dismembered victims are left to be found}
A villain kills his enemy's family and friends and leaves their mutilated, tortured corpses where his enemy will find them. Psychological terror, to fuck with the enemy's brain. 

The victim ends up going half-mad and seeing into the Beyond, in the style of the \emph{Kult} RPG. 









\subsection{Lying oracle}
Maybe have a lying oracle, hired by the Cabal or Sentinels to lead people astray, or onto the paths that their masterminds desire. 









\subsection{Mutilation}
Have a scene where a monster kills and mutilates people, leaving behind horribly mangled corpses. Perhaps hung on stakes or hooks from the ceiling, or burned, or with their skin peeled off. Blood, rivers of blood.









\subsection{Possession}
Something with monsters that possess people and take over their bodies would be cool. Weak people can be possessed easily, but strong people can fight back, being controlled only imperfectly and with difficult. Compare to the story of \quo{The Madness of the Black King} in \emph{Diablo}. 









\subsection{Regrettable sacrifice}
Someone badass, possibly \hr{Ishnaruchaefir}{\Ishnaruchaefir}, wants to sacrifice an annocent for some purpose. Someone objects. 

\Ishnaruchaefir: 
\ta{Regrettable, yes. 
  But necessary. 
  Very well. 
  Let us divide the workload between us. 
  I will do the deed, and you can regret it. 
  How does that sound?}





\subsubsection{\Resvil{} vore sex}
Have an erotic vore scene where a \resvil{} has passionate sex with a food slave, all the while gradually devouring him/her. 
Perhaps even feeding bits and pieces of him/her to her other sex slaves, as a special boon to those who have served her well. 

See section \ref{Resphan food slaves}. 

Compare with the \emph{Morbus Gravis} comics.















\section{Technology: By field}









\subsection{\Armour and wards}
\target{Armour and wards}
On the higher levels of technology and/or power, wards and force fields become more effective than \armour. 

High-tech and magical \armour (including \hs{ward runes} and \hs{Archon Wards}) are most effective against magic and ranged weapons. 
It is easier to protect oneself against such attacks. 
Immortals tend to be less well-protected against \melee{} attacks.
In \melee, the attacker has more direct control over his attack and the techniques and force he puts into it, so it is easier to circumvent or penetrate a static defense (such as \armour or spells). 

Among mortals, \armour was \hr{Armour in the TBW}{still in use even up to the \thirdbanewar}. 

The \resphain sometimes wore \hr{Glass armour}{\armour made of glass or crystal}. 





\subsubsection{Ward runes}
\target{ward runes}
\index{ward rune}
Ward runes are worn especially by \dragons. 
They are enchanted runes, carved on their scales or on charms and talismans, containing protective spells that act as \armour. 
In combat, they light up (or darken) with Chaotic energy and release \hr{Daemon}{\daemons} that protect against attacks/magic or heal wounds. 

The ward runes are magical \hr{True Draconic}{\TrueDraconic} \hr{Draconic runes}{runes}. 
They are often written on \hs{graph-glass}. 
The glass can represent the runes more accurately and reliably, and channel power through them more effectively, than a simple carved surface. 

Ward runes must be empowered by pouring energy into them. 
When empowered, the runes will leak and gradually lose power. 
When they are empty they have to be charged again. 
You cannot just accumulate energy in them and make them arbitrarily strong. 
Of course, they deplete faster if they see combat. 

Ward runes are \hr{Armour and wards}{most effective against magic and ranged weapons}. 

Ward runes are very mystical, dark and occult. 
\XzaiShann-themed. 
Each ward rune is empowered by a specific, named \xs. 

\citeauthorbook[p.19]{RPG:Warhammer:DarkElves}{Gav Thorpe et al}{%
  Dark Elves%
}{
  Shaped into the leering face of an Ice Daemon, the Shield of Ghrond is imbued with the power of the north wind which robs attacks of their force. 
}

Some \xs ward runes work by fucking with the attacker's mind, projecting frightening visions of the \xs.

\citeauthorbook{RPG:Warhammer:HighElves}{Andy Chambers}{%
  High Elves%
}{
  The Stone of Midnight exudes an impenetrable mist of darkness, and anyone trying to strike at the possessor will be confronted by his worst nightmares, vion of his own death and failure of all the works of his life.
}









\subsection{Domestic animals}





\subsubsection{Beasts of war}
\target{Beasts of war}
\target{beasts of war}
A lot of animals were used as beasts of war, including \hr{Corgorah}{\corgoroth}, \hr{Grulcan}{\grulcans}, \hr{Lotha}{\lothae}, \hr{Muroc}{\murocs}, \hr{Nycan}{\nycans} \hr{Relc}{\relcs} and more. 









\subsection{Fuels}
The \ophidian civilization used up a lot of the planet's fossil fuels. 
The \quiljaaran and \aryoth civilizations \hr{Aryothim used up fossil fuels}{used up most of the rest}.
By the \Human Age, there was almost no fossil fuels left.
This made it hard to develop any industry. 
So the mortals didn't. 









\subsection{Living machines}
\target{Living machines}
\target{living machines}
The \ophidians and \voyagers and other high-tech civilizations often used machines of flesh instead of metal and plastic. 
In some ways, super-enhanced flesh was superior to metal.

\begin{itemize}
  \item Flesh was more flexible.
  \item Flesh could more easily be made to regenerate itself. 
  \item Flesh could wax stronger through training and exercise.
    Metal would only weaken from use and require repair. 
  \item Creatures of flesh could eat and replenish themselves. 
    Machines had to be built.
  \item Flesh is more horrible than metal. 
\end{itemize}

See the section on \hr{Weapons of mass destruction}{weapons of mass destruction} for some visuals. 

The \noggyaleth, in a sense, were such machines, \hr{Voyagers train Noggyaleth}{trained by the \voyagers} as they were. 

The \dragons \hr{Dragon living technology}{used living technology a lot} (as \hr{Resphan dead technology}{opposed to the \resphain}). 
The \hr{Dragons are living machines}{\dragons themselves were also living machines}, in a sense.

Living machines should be dark and gruesome, \trope{CosmicHorror}{Cosmic Horror}. 





\subsubsection{Symbiotes}
\target{Symbiotes}
The \dragons and \ophidians used symbiotes. 
These were living things that could be attached (temporarily or permanently) to the body and serve as tools. 
They were living machines. 





\subsubsection{Undead machines}
\target{undead machines}
The \draconic{} necropolises were protected and maintained by undead machines: 
Shambling, creaking hulks of rusty steel and desiccated flesh. 

\lyricsauthorbookpage{Alan Campbell}{Iron Angel}{394}{
  Machines with human skin and faces crowded among the gears and chains?
}









\subsection{Mage-smiths}
\target{Mage-smith}
\target{Mage-smiths}
\target{mage-smith}
\target{mage-smiths}
\index{mage-smith}
In all ages and cultures there were mage-smiths. 
They could make very powerful \armour that protected even against muskets.
So those who could afford it still wore plate \armour.
But lower grades of \armour were uncommon.

Mage-smiths were especially valued when it came to making melee weapons and \armour. 
They were less useful in making \hs{guns}. 
A mage-smith could pour much magic into a sword and it would make a huge difference.
Guns were easier to mass-produce, and handcrafting did not make so great a difference.
Still, some powerful guns were handcrafted.










\subsection{Shroud represses technology}
\target{Shroud represses technology}
The \hs{Shroud} makes it worse. 
It makes everyone stupid and narrow-minded and less capable of devising new ideas and inventions.
It affects even the master races. 

Worse yet, the Shroud can cause immortals to \emph{forget} things they once knew. 
So in some areas the technology actually \emph{dropped} after the \secondbanewar, even among immortals. 

Mortals are even more severly affected. 
They are many in number and reproduce quickly, but the Shroud makes everyone so god-damned stupid that inventive geniuses are \emph{very} far in between. 
And many of their ideas get squashed by their Shrouded neighbours who are unwilling to accept new things. 

The Rissitics \hr{Rissitic research loophole}{have a loophole}.









\subsection{Technology that looks like magic}
Have lots of technology that looks like magic. 
Such as \hs{graph-glass}.
Compare to the Cthulhu Mythos. 









\subsection{Weapons}
\target{Weapons}
\target{weapons}
\index{weapons}





\subsubsection{Guns}
\target{Guns}
\target{guns}
Guns had the side effect that the noise (and maybe smell too) scared animals. 
\hr{Relcs are skittish}{\Relcs were skittish}.
So \hr{Lothae are skittish}{were \lothae}.
\Murocs \hr{Murocs are steady}{were rock-steady}.  





\subsubsection{Ranged \melee attacks}
There existed combat techniques that let the fighter make a ranged attack with a \melee weapon. 
For example, a warrior might slash with a sword and launch a wave of destructive energy from the blade. 

For this reason, immortal fighters tended to use \melee weapons more than ranged weapons, and immortal martial arts (such as the \hr{Resphan martial arts}{\resphan \quo{Paths}}) focused on techniques with \melee weapons.
Ranged weapons were important too, but \melee weapons were often more effective overall.
A \melee weapon like a swords was versatile; you could shoot with it and also use it to defend yourself in close combat. 

A huge weapon like a \senaan could fire lancing blasts of lightning-like energy. 

Compare to the powered attacks in \cite{VideoGame:FinalFantasyVII}, \cite{Anime:FinalFantasyVII:LastOrder} and \cite{Anime:FinalFantasyVII:AdventChildren}. 





\subsubsection{Superstition}
\target{Gun fiends}
There was superstition surrounding guns. 
Commoners did not really understand how they worked. 
Superstition said that there lived little fiends inside the guns that made them fire (or misfire).
Soldiers would pray to those fiends to placate them. 















\section{Technology: By Realm}
Some Shrouded Realms had higher or lower technology than others.














\section{Technology: By time period}
\target{Technology}
\target{technology}
\index{technology}
Technology advanced slowly on \Miith{}, and it was often deliberately set back or repressed. 







\subsection{\Ophidian Golden Age}
\target{High-tech civilization}
\target{interstellar civilization}
\index{technology!interstellar civilization}
In the far past, hundreds of thousands of years ago, there existed an \ophidian interstellar civilization. 
Using magic and other science, creatures could travel between Realms and through the vastness of space to \cooperate, trade or wage wars. 

\Sethicus \hr{Sethicus brought innovation}{helped build this civilization}. 

There were also \hs{living machines}. 





\subsubsection{Weapons of mass destruction}
\target{Weapons of mass destruction}
In the \Ophidian Age, there were weapons of mass destruction made with black magic. 
Such as an \quo{Azathoth Bomb} (see below). 
The \ophidians used them. 

\citeauthorbook[p.201]{BruceSterling:TheUnthinkable}{Bruce Sterling}{The Unthinkable}{
  \ta{And what of the Unthinkable, eh? 
    What price have you paid for \emph{that} business?}
  
  [\ldots{}] 
  \ta{We bear any burden in the defense of freedom.}
  
  [\ldots{}]
  \ta{%
    To deliberately contact an utterly alien entity from the abyss between universes\ldots an ultrademonic demigod whose very geometry is, as it were, an affront to sanity\ldots that Creature of nameless eons and inconceivable dimensions [\ldots]
    That hideous Radiance that bubbles and blasphemes at the cneter of all infinity.}
  
  \ta{You're being sentimental. 
    [\ldots]
    We must recall the historical circumstances in which the decision was made to develop the Azathoth Bomb.
    Giant Japanese Majins and Gojiras crashing through Asia.
    Vast squadrons of Nazi juggernauts blitzkrieging Europe\ldots and theur undersea leviathans, pretying on shipping\ldots}
  
  [\ldots]
  \ta{Yes, I witnessed one\ldots feeding.
    At the base in San Diego.}
  Doughty could recall it with an awful clarity\dash the great finned navy monster, the barnacled pockets in its vast ribbed belly holding a slumbering cargo of hideous batwinged gaunts.
  On order from Washington, the minor demons would waken, slash their way free of the monster's belly, launch, and fly to their appointed targets with pitiless accuracy and the speed of a tempest.
  In their ralongs, they clutched triple-sealed spells that could open, for a few hideous microseconds, the portal between universes.
  And for an instant, the Radiance of Azathoth would gush through. 
  And whatever that \colour touched\dash wherever its unthinkable beam contacted earthly substace\dash the Earth would blister and bubble in cosmic torment.
  The very dust of the explosion would carry an unearthly taint.
}









\subsection{\Saphyrae and Draconian Supremacy}
Before the rise of the \resphain, there existed several intelligent races, but many of them were individually powerful, thus not having developed strong cultures, and/or very long-lived, thus slow to produce new ideas. For this reason, technological advances were slow. 

Still, the time between the \firstbanewar{} and the \secondbanewar{} was something of a \hr{Golden age of technology}{golden age of technology}. 

The \banes{} brought some \hr{Bane technology}{technology with them from \Erebos}, most of which was pirated from the \voyagers{}.









\subsection{The Incursion}
The \resphain{} were quite creative and made many innovative pieces of technology. Gradually, however, they succumbed to \hr{Entropy}{\bane{} stagnation} and their own immortality, and their scientific advances slowed. Still, they are in possession of many scientific marvels. 

\target{SBW technology}
At the time of the \secondbanewar, technology on \Miith was high on both sides. 
\Miithians and \resphain each had some weapons that the other side lacked, making the first battles full of nasty surprises for both sides. 

The \hr{Cuezcan}{\cuezcans} were also advanced and boasted many high technology items and inventions. 
But they were destroyed and left in ruins. 





\subsubsection{Graph-glass}
\target{Graph-glass}
\target{graph-glass}
\index{graph-glass}
Graph-glass was a \resphan{} invention. 
It was a flat piece of glass that could display text. 





\subsubsection{Machines}
\index{technology!machines}
The \dragons{} and \banes/\resphain{} each have their machines and their primitive industry. 

Remember that the Cabal and Sentinels have at their disposal far higher technology than the mortal \Miithians. 
Part of the \charade{} is keeping the level of technology low, so people don't come to know too much and begin to suspect.

\Draconic{} machines are cyborgs, living hulks of metal and flesh. 
Some of them are called \colossi. They are hot, chaotic, almost savage things. 

Cabal machines are dead, articial, mechanical things of metal, stone and glass. They are cold, hard and exude an aura of unyielding, cruel hardness. 

The \resphain, among other things, had a kind of machine that copied \hs{graph-glass}. 

At the time of the \thirdbanewar{} no one was close to inventing computers, though. 









\subsection{Immortals in the {Age of the Shroud}}
The immortals had a printing press, both for paper and \hs{graph-glass}.
They also had industry, which they used to produce many of their weapons, including most guns. 
(But some of them were \hr{Mage-smith}{handcrafted by mage-smiths}.)

The immortals made sure that their industry stayed in the Immortal Realms and did not trickle down into the Shrouded Realms.
They had to consider the Unspoken Covenant.





\subsubsection{Cameras}
There were no cameras on \Miith. 
They could not be built like on Earth because of the different way \Miithian physics worked. 
An eye picked up images from many dimensions and the brain sorted out most of it (because of the Shroud and other things), so that the mind saw only a few of them. 
A photograph would be a mishmash of all those images from all those dimensions. 
Without a bigger context, the eye would be unable to make sense of the images on a photograph, so it would just be unrecognizable static. 

The immortals had high enough technology that they might reasonably be expected to create cameras, but they never did. 
The \ophidians, though, could make holograms using magic. 
These holograms extended out into all dimensions and were just like physical objects. 











\subsection{The Vaimon Age}
In \Azmith after the \hr{Cuezcan Apocalypse}{\CuezcanApocalypse}, the \hr{Shrouding}{\SecondShrouding} and the introduction of the \charade, technological advances were brutally repressed among the slave races. 
This repression was partly done intentionally by the master races, partly \hr{Shroud represses technology}{enforced by the Shroud itself}.
Only the master races were allowed to do serious research, and they were few in number and slow to reproduce, so very little progress has been done since then. 
Still, the master races know many technological marvels and will occasionally utilize them. 





\subsubsection{Railroads}
\target{Vaimon railroads}
The Vaimons built railroads. 
These did not have automotive-driven trains on them, but they did have big wagon trains drawn by huge dinosaurs.





\subsubsection{Weapons}
\hr{Vaimon guns}{The Vaimons made guns}. 
Most of them were ordinary, but some were magical and super-powerful. 
Later, \hr{Magical Vaimon guns survive}{only the most powerful ones survived}. 









\subsection{The \Thirdbanewar}
\target{TBW technology}
In \Azmith at the time of the \hr{TBW}{\thirdbanewar}, many high-tech items were old artifacts. 
The master races had \emph{forgotten} how to make them, due to so many dying and \hr{Shroud represses technology}{the Shroud making the survivors stupid}. 
But the technology was still fairly high. 

The \hr{Vaimon technology taboo}{Vaimons had a taboo against technology}. 
Technology still marched on, but slowly.
Much technology had been lost. 
Technology was being deliberately repressed by the immortals. 





\subsubsection{\Armour}
\target{Armour in the TBW}
Among mortals, \armour was still in use even up to the \thirdbanewar. 
There were guns, true, but sturdy metal \armour still offered good protection. 

\hr{Mage-smiths}{Mage-smiths could make powerful \armour} that protected against even musket fire. 
So those who could afford it still wore plate \armour.
But lower grades of \armour were uncommon.





\subsubsection{Languages}
\target{Languages in the Scatha Age}
In \Velcad, the \hs{Vaimon language} was used as a \emph{lingua franca} among the learned. 
There were many vernaculars, including \hr{Pelidorian language}{Pelidorian} and \hr{Rungeran language}{Rungeran}. 





\subsubsection{Railroads}
\target{TBW railroads}
There were railroads, \hr{Vaimon railroads}{as in the Vaimon Age}. 

These did not have automotive-driven trains on them, but they did have big wagon trains drawn by huge dinosaurs.

In the days of the \thirdbanewar, the trains were ancient colossi. 
The technology to build them had been lost. 
They were revered as relics and seen with superstitious awe. 
They were intimidating infernal machines. 
Legends said that once the trains had been able to move on their own (without animals to pull them), powered by the fire of \daemons. 

The technology to maintain the railroads themselves was not lost, thankfully. 
They were still being maintained. 





\subsubsection{Weapons}
\index{guns}
\index{technology!guns|see{guns}}
At the time of the \thirdbanewar, mortals in the Shrouded Realms had lower technology than in the \VaimonCaliphate. 
They still had the technology to make guns, though. 
Just somewhat less powerful. 

There were muskets and pistols. 
(Full-fledged muskets.) 
Each gun held only a single shot, though. 
And pistols were inaccurate. 

Near the \thirdbanewar, many quality guns were \hr{Rissitic economy}{Rissitic-made}. 

\target{Magical Vaimon guns survive}
History told of \hr{Vaimon guns}{the \uber-powerful guns that existed in the Vaimons' time}. 
At the time the secrets of their making had been lost, and only a few remained. 
The remaining old guns were mostly \uber-powerful magical guns. 
This led some people to think that all older guns were \uber-powerful and magical. 
This was not the case. 
Most Vaimon guns were ordinary. 
It's just that the ordinary guns had all perished, and only the \uber{} ones had stood the test of time. 

Have plenty of nasty weapons, such as stuff that fires shrapnel.
And some equivalent of scythed chariots, mounted on big dinosaurs. 
One advantage of higher technology is that I can wreak more havoc and make battles more bloody. 

See also the general section on \hs{weapons}. 



\begin{gloss}
  \gitem{arquebus}
    The arquebus was a primitive gun. 
    \citetitle{GarrettsBridges}{%
      \href{http://www.garrettsbridges.com/links/special8.html}{Garrett's Bridges}%
    }{
      The early arquebus fired only about 100 yards and often would not pierce traditional armor.
    }
    
    Other sources claim the arquebus had the same range as a crossbow, and a faster rate of fire.
  
  \gitem{crossbow}
    \citetitle{GarrettsBridges}{%
      \href{http://www.garrettsbridges.com/links/special8.html}{Garrett's Bridges}%
    }{
      The crossbow, which had an effective range of about 250 yards\ldots
    }
\end{gloss}




















