








\section{The Fall of the \Ophidians (\yds{First Banewar ends})}
\target{Fall of the Ophidians}
\index{technology!\ophidian}
Even though the \ophidians{} were victorious in the \firstbanewar{}, it  heralded the downfall of their civilization. 









\subsection{\Miith was nuked}
The war and the destructive forces (magical and non-magical) unleashed by both sides caused not one but multiple ecological catastrophes. 
This cataclysm heralded a mass extinction wiping out a ton of species. 

The \firstbanewar nuked the planet to smithereens. 
After the \banes had been banished, \Miith was bathed in fallout, both natural and supernatural. 
And ravaged by summoned monsters (many of whom still dwelt in the dark corners of \Miith millennia later). 





\subsubsection{Rampaging monsters}
After the \firstbanewar, the \ophidians lost control of many of their monsters. 
They ran amok and caused much havoc and destruction.
This was one of the many reasons why the \ophidian civilization declined and would not rise again for tens of thousands of years. 






\subsection{\Ophidian civilization collapsed}
\target{Ophidians lose telepathy}
It struck the \ophidians{} themselves hard. 
A backlash by the \banelords{} bathed the planet in destructive radiation. 
This radiation was bad enough for regular animal and plant life, but it was worst for the \ophidians, who were highly telepathically sensitive. 
The radiation attacked their highly-developed minds and resulted in permanent and debilitating brain damage. 
They suffered from amnesia and insanity\dash apathy or irrational aggression. 
The brain damage also harmed their intelligence and burned out much of their telepathic and psionic abilities. 

Only a few survived. 
And with the loss of telepathy, their civilization quickly broke down, for all their technology was built to be operated telepathically. 
The vast majority of \ophidians{} died in the war and its long aftermath. 
With their newfound insanity they used what was left of their science and magic to wage war on each other, until their race was all but extinct. 

The \banes' radiation also destroyed most of the \ophidians' computers, leaving their robots as immobile and useless hulks of metal and plastic and whatever.
Their civilization was very dependent on computers and robots, so this was another catastrophe for them.

The surviving \ophidians were few. 
They tried to rebuild their civilization, but they failed. 
Their empire declined over the next several thousand years until there was nothing left. 
To make it worse, they waged wars against each other, too. 
The great war had made the survivors more violent and xenophobic. 





\subsection{\Noggyaleth strike back}
\target{Noggyaleth suppress Ophidians}
After \hr{Noggyaleth remain passive in the FBW}{having been neutral in the \firstbanewar}, the \noggyaleth finally became convinced that it was in their interest to work with the \banes. 
They now attacked the remaining \ophidians. 
This was another factor that destroyed the \ophidian civilization. 

Since then and up till the \thirdbanewar, the \noggyaleth would hunt \ophidians.
Whenever the \ophidians tried to build greater cities and make any effort towards rebuilding their civilization, the \noggyaleth would attack. 
The \noggyaleth were insidious and cunning.
No matter how much the \ophidians tried to keep them out, they would worm their way in and find some way to fuck the \ophidians up:
\begin{itemize}
  \item Dig underneath the city and make it collapse.
  \item Destroy roads and disrupt logistics. 
  \item Poison water or food supplies.
  \item Dig tunnels and allow vermin to enter. 
  \item Generally make life hell. 
\end{itemize}








\subsection{\Ophidians divided}
The \ophidians{} (not counting \dragons) became divided into two groups. 
One group developed hyper-aggression, the other group developed hyper-apathy. 
In time, all the hyper-aggressive ones killed each other and died out, leaving only the apathetic ones. 
The latter evolved into \quiljaaran. 
They had not the will nor energy to maintain or rebuild their civilization, so it continued to decline. 
The \ophidian{} culture fell into barbarism, a process that took over a thousand years. 

Many \ophidians began to worship brutal \xs godlings that cared nothing for technology nor progress. 
Under the rule of these harsh gods, the \ophidians descended into barbarism.
They waged wars against one another and destroyed what little remained of their civilization and the planet's natural resources.

Compare to the Serpent People from the Cthulhu Mythos, who turned to the worship of Tsathoggua and were punished by their ancestral patron god Yig, causing them to degenerate. 

\target{Technology low after FBW}
In the next millennia, \Miith was populated by a lot of different barbaric, \xs-worshipping peoples. 
They would wage war a lot and destroy each other.
Technology remained low.
Until \hr{Origin of Aryothim}{the \aryothim came}.
Then \hr{Technology rises with Aryothim}{technology began to rise}. 









\subsection{\Ophidians tried to rebuild}
Throughout the ages, the \ophidians \hr{Noggyaleth plague Ophidians}{tried to rebuild their civilization, but always the \noggyaleth would ruin it}.










\section{New World Order}
The survivors had forgotten much of their old culture.
They had no technology and barely any magic. 
They had only legends. 

The technology used to be very high, but after the war they forgot their civilization and science, and how to control their slave races and \daemons. 
Compare to the Azghouls from the RPG \emph{Kult}, who were once the slaves of \humans. 

Two branches of \ophidians{} survived: 

\begin{enumerate}
  \item 
    The \dragons.
    
    The \quiljaaran{} saw them as freaks and did not trust them. 
    They, in turn, saw the \quiljaaran{} as inferior weaklings and cowards. 
    So they fought rather than working together. 
    
    The \dragons{} were strong, but few in number. 
    So they did not rise to dominate the world\prikker yet. 
  \item 
    Some regular \ophidian{} civilians, who eventually became \quiljaaran. 
\end{enumerate}









\subsection{\QuilJaaran}
The \quiljaaran{} kept on dwelling in the desolate and crumbling \ophidian{} cities. 
Each \quiljaar{} was usually able to keep a few dozen servants or slaves (mostly \loculs, sometimes \cregorrs{} and \nephilim). 
But this still meant that they ended up with dozens to hundreds of people living in \hr{Ophidian architecture}{Cyclopean} ruined cities that were built to house tens of thousands if not millions. 

The \ophidians/\quiljaaran had lost most of their culture and technology and been set back tens of thousands of years.
So they were now less powerful than before. 

The \nephilim{} were less friendly towards their \quiljaaran{} masters than the reptilian races. 
They would often run away or commit mutiny. 
Few \nephilim{} wanted to live near a \quiljaar. 









\subsection{\Cregorr{} barbarians}
\target{Cregorr did not dominate}
The \cregorrs{} \hr{Origin of Cregorr}{had been created at this point}. 
After the \firstbanewar{} and the collapse of the \ophidian{} civilization, the \cregorrs{} escaped into the \wylde. 

But they did not become a dominant race. 
They were \emph{too} aggressive and disorganized. 
They failed to develop much civilization on their own. 
They remained barbarians. 

Some \cregorrs{} ended up serving the \quiljaaran. 
But not so many. 
The \quiljaaran{} did not like the warlike \cregorrs. 
They made them feel uncomfortable. 

Many \cregorrs{} \hr{Cregorr worship XS}{worshipped the \xss}.









\subsection{Millennia pass}
Many thousands of years passed from the \firstbanewar{} to the rise of the \aryothim. 
\Dragons and \quiljaaran{} were immortal, so \quo{many tens of thousands of years} was only a few generations for them. 
Many were alive whose grandparents or even parents had lived during the \banewar. 
Also, they had long memories, which was one reason why it took their civilization so long to decline. 

But decline it did.
The \quiljaaran were too apathetic to rebuild their empire. 
The \quiljaaran had a lust for knowledge and philosophy, but \hr{QJ not inventive}{they were not inventive}. 









\subsection{\Dragonkings}
The surviving \ophidians idolized the \dragons who had fought and sacrificed themselves for their sake. 
They worshipped the \dragons and patterned themselves after them. 
The true stories about \Sethicus-tachi were twisted through the many millennia. 

Eventually a new aristocracy arose. 
They claimed descent from \Sethicus (plausible since \Sethicus was an \ophidian, but difficult to prove) and called themselves \quo{\dragonkings}. 

Many wars were fought between these \dragonkings. 









\subsection{Kush}
\target{Kush}
\index{Kush}
Kush was an \ophidian nation. 
It flourished 10.000s of years before the \dragons returned from Durance. 





\subsubsection{Language of Kush}
The language of Kush was a mystic and powerful arcane language related to \hr{True Draconic}{\TrueDraconic}. 
Arcane Kush runes could be found on the walls of the \hr{Tower of Aamon}{Tower of \Haamon}. 









\subsection{\Saphyrae (\yds{Saphyrae dominates})}
\target{Saphyrae}
\index{\Saphyrae}
\Saphyrae{} was the greatest and mightiest \quiljaaran{} empire ever. 
It rose after the \firstbanewar. 
The \Saphyraeans{} considered themselves the heirs of the \ophidian{} empire, which was \trope{ShroudedInMyth}{Shrouded in Myth} and legend already back then. 

\Saphyrae{} was the only \quiljaaran{} empire which the \aryothim{} could never really threaten. 
Although that did not stop them from trying. 
The \aryothim{} waged many wars against \Saphyrae, but they never achieved much. 

But though \Saphyrae{} held its ground, they did not expand. 
The \quiljaaran{} were already then a people in decline. 
One reason was that they were apathetic, \hr{QJ apathy}{as \quiljaaran{} were ever wont to be}. 

\Saphyrae was the closest the \ophidians came to a new empire. 
It was a grand city.
But it was never completely thriving. 
The \noggyaleth besieged it from the shadows. 
So the \ophidians had little surplus to rebuild or innovate. 
Just keeping the city up and running was hard enough. 

The time between the \firstbanewar{} and the \hs{Draconian Supremacy} is sometimes called the \Saphyraean{} Age. 

When the \dragons awoke, they took control of \Saphyrae. 
\Nexagglachel set about rebuilding the \ophidian empire, and he was doing well. 
For two thousand years or so. 

\Saphyrae was destroyed in the \secondbanewar. 





\subsubsection{Saphyr language}
\target{Saphyr}
The \caisith of \Saphyrae spoke the language of Saphyr.















\section{The \Aryoth Invasion (\yds{Aryoth invasion})}
\target{Aryothim kill QJ}
The \aryothim{} showed up. 





\subsection{Origin}
\target{Origin of Aryothim}
\target{Aryothim enslaved by sorcery}
The \nephilim lived as slaves of the \ophidians (\hr{Ophidians create Nephilim}{who had created them}). 
They \hr{Nephilim worship Ophidians}{worshipped the \ophidians}.
Some \ophidians noticed that the \nephilim, while normally docile, had a bestial ferocity underneath.
This fury, alien to the cold and dispassionate \ophidians, made a \nephil remarkably strong when released. 
This reminded the \ophidians of the \dragons, who were also powered by rage and ferocity.

The \ophidians experimented with harnessing this primal fury and strengthening it.
So they modified the \nephilim, injected them with \daemonic blood and mystic power.
They created the \aryothim:
A race of super-powered \nephilim blessed with \quiljaar-level intelligence.

The \ophidians used the \aryothim as fighters in their internecine wars. 
The \aryothim served their creators as slaves for thousands of years.
Meanwhile the \ophidians kept working on the \aryothim, making them even stronger and fiercer. 

Eventually the \aryothim grew too powerful and too volatile to control.
They rebelled. 

The \aryoth rebellion might have been engineered by the \hr{Gods Beneath}{Gods Beneath} (or the \hr{Noggyal}{\noggyaleth}).









\subsection{\Aryoth technology}
\target{Technology rises with Aryothim}
Technology on \Miith \hr{Technology low after FBW}{had been low ever since the \firstbanewar}. 
Now, with the coming of the \aryothim, technology began to rise again. 

\target{Aryoth technology}
The \aryothim developed technology that, in some areas, was superior to that of the \quiljaaran. 
The \aryothim did not have nearly as powerful magic, but they had high-quality guns and stuff.
Perhaps they even had 20th century technology. 

See the section on \hr{Aryoth}{\aryothim} for more about their technology.





\subsubsection{\Aryoth seamanship}
\target{Aryoth seamanship}
\target{QJ seamanship}
One advantage the \aryothim{} had over the \quiljaaran{} was that they were great sailors. 

The \quiljaaran{} had a taboo against challenging their long-lost kin the \nagae{}, who had long held dominion over the seas by default. 
So the \quiljaaran{} kept to the land, even though with their technology they could have made themselves a sea power had they tried. 
\hr{QJ apathy}{\QuilJaaran{} were ever apathetic}. 

The \aryothim{} had no such scruples, so they sailed out and explored. 
They quickly gained built great sea-spanning empires. 
The \nagae{} resented this and opposed the \aryothim. 
\Naga-\aryoth{} wars were fought. 
The \aryothim{} suffered losses in these wars, but these losses were outweighed by the gains of seafaring. 

The \aryothim{} gained naval superiority, which remained unbroken until the \dragons came and kicked their asses. 





\subsubsection{Fuels}
\target{Aryothim used up fossil fuels}
The \aryothim developed technology that ran on fossil fuels. 

After they depleted all fossil fuels on the planet, they \hr{Aryothim used dark magic}{began relying on dark magic for their energy needs}. 








\subsection{Rampage}
They hated the \quo{loathsome serpent men}. 
So they waged war against the \quiljaaran{} and their servitors and slaughtered many. 
They invaded the \quiljaaran{} cities, killed the \quiljaaran{} and their servitors and laid waste to the cities. 
Their rebellion was brutal and bloody.

\citebandsong{KarlSanders:SaurianExorcisms}{Karl Sanders}{%
  Impalement and Crucifixion of the Last Remnants of the Pre-Human Serpent Volk%
}{%
  I will give you a war engine, and with it ye shall smite those that still slitehr and crawl.
  The time of the great and terrible reptiles is over.
  The age of the two-legged plague is at hand.
  Only in the primitive race memories and dreams of Man will there still exist the souls of those who first claimed dominion of the earth.
}


Later the \aryothim{} would invent legends of how they had fought a valiant crusade to rid the world of the wicked reptilian menace. 
The \quiljaaran, on the other hand, considered themselves the victims of murder and vandalism at the hands of bloodthirsty barbarians. 

\target{Nephilim kill Locul}
The \loculs{} were especially hard hit. 
They were \hr{Locul weak}{weak and fearful fighters}, so they died in the droves at the hands of the rampaging racist \nephilim. 
Moreover, they were \hr{Locul friendly}{friendly and trusting}, which made it even easier for the ungrateful \nephilim{} to betray, enslave and kill them. 

To the \nephilim, the \loculs{} were loathsome reptilian creeps, but \hr{QJ miss Locul}{the \quiljaaran{} missed them when they were gone}. 

Compare to the way the Mabden kill off the Vadhagh and Nhadragh in \cite{MichaelMoorcock:Corum}. 
And the way barbarians slaughter the \quo{Peaceful Ones} in \cite{TadWilliams:MemorySorrowandThorn}. 

\citeauthorbook[p.278]{RobertEHoward:TheValleyoftheLost}{Robert E. Howard}{%
  The Valley of the Lost%
}{
  Then a new people came upon the scene.
  Over the hills came wild men clad in hides and feathers, armed with bows and flint-tipped weapons.
  They were, Reynolds knew, Indians, and yet not Indians as he knew them.
  They were slant-eyed, and their skins were yellowish rather than copper-\coloured. 
  Somehow he knew that these were the nomadic ancestors of the Toltecs, wandering and conquering on their long trek before they settled in upland valleys far to the south and evolved their own special type and civilization.
  They were still close to the primal Mongolian root-stock, and he gasped at the gigantic vistas of time this realization evoked.
  
  Reynolds saw the warriors move like a giant wave on the towering walls.
  He saw the defenders man the towers and deal death in strange and grisly forms to the invaders.
  He saw them reel back again and again, then come on once more with the blind ferocity of the primitive.
  This strange evil city, filled with mysterious people of a different order, was in their path and they could not pass until they had stamped it out.
  
  Reynolds marveled at the fury of the invaders who wasted their lives like water, matching the cruel and terrible science of an unknown civilization with sheer courage and the might of man-power.
  Their bodies littered the plateau, but not all the forces of Hell could keep them back.
  They rolled like a wave to the foot of the towers.
  They scaled the walls in the teeth of sword and arrow and death in ghastly forms.
  They gained the parapets.
  They met their enemies hand-to-hand.
  Bludgeons and axes beat down on the lunging spears, the thrusting swords.
  The tall figures of the barbarians towered over the smaller forms of the defenders.
  
  Red hell raged in the city.
  The siege became a street battle, the battle a rout, the rout a slaughter.
  Smoke rose and hung in clouds over the doomed city.
}





\subsubsection{Dark magic}
\target{Aryothim used dark magic}
The \aryothim used dark magic in their initial campaigns of genocide against the reptilian races.
They had to, since \hr{Aryothim used up fossil fuels}{they had no fossil fuels left}. 

In fact, it is an ongoing theme on \Miith that in a war, \hr{Darkest magic wins}{the side that is willing to employ the darkest sorcery is usually the side that wins}. 

(The Vaimons' magic was disguised as something nice and good, but it was really even more insidious than \xs magic\dash{}more insidious than anyone had imagined, as it turns out in the end, with \Lithrim.)

\Aryoth battle song:

\citebandsong{Nile:RamsesBringerofWar}{Nile}{
  Der Rache Krieg Lied Der Assyrische
}{
  Nergal, dread God of War and Plague\\
  Avenge the shades of our Fallen ones\\
  Blacken the sun with Fierce Winds\\
  Bring forth your Terrible Storms!\\
  Yea! Burn the flesh of our Enemies\\
  Gash their throats with Weapons of Iron\\
  Destroy Them utterly\\
  With Locusts and Disease
}




\subsubsection{Racism and Nazism}
Have plenty of Nazi-evoking race theory and racism surrounding the \aryothim. 
The similarity of \quo{\aryoth} and \quo{Aryan} is not coincidental. 

And consider the \hr{Resphan-Aryoth relationship}{relationship between \aryothim and \resphain}. 










\subsection{Conquest}
The \aryothim conquered the world and dominated much of it for thousands of years.
At first they were stone-age barbarians.
Over the millennia they developed a high-tech civilization.

They reached level of technology similar to the 20th century or above, albeit different, owing to the different laws of nature in the \Miith universe.

The \aryothim waged terrible and blody wars against the \ophidians. 
The \aryothim and their \quiljaaran enemies developed weapons of mass destruction. 
See the section about \hr{Aryoth warfare}{\aryoth warfare} and \hr{Ophidian warfare}{\ophidian warfare}. 

There was at least one holocaust in those days comparable to a nuclear holocaust. 
After each major war, the \aryothim were quicker to get back on their feet than the slow-living \quiljaaran were, so they slowly gained the upper hand. 
The \ophidians were immensely powerful, but they were not aggressive enough.
They should have wiped out the \aryothim as soon as the \aryothim began to turn on them, back when they had the chance.
And they should have wiped out the \nephilim with them.
But the \ophidians hesitated and grew complacent, and that was their downfall, for the \aryothim's hate and determination knew no bounds. 
They kept on breeding and rebuilding and forging new weapons and forming new armies and attacking and destroying and conquering until they had won. 

Perhaps it was one such war of mass destruction that roused \Nexagglachel-tachi from their sleep. 
\Nexagglachel felt the catastrophic loss of life and decided he had to do something to fix it. 
So he spent the next century or more slowly picking the locks of his tomb, until finally he was able to rise from the dead and wake his brothers. 









\subsection{The \aryoth empire}





\subsubsection{Ibthek}
\target{Ibthek}
\index{Ibthek}
Ibthek was an \aryoth city in the age before the Draconian Ascendancy. 

The gods of Ibthek were immortal elder \aryothim.
Its chief gods were \hr{Gorgomon}{\Gorgomon}, \hr{Klaad}{\Klaad} and \hr{Murru}{\Murru}.

In the age of the \VaimonCaliphate Ibthek still existed, but it had become a ruin, a necropolis. 
The gods had become undead wretches.
The city was inhabited by \humans and \nephilim that had degenerated into things resembling the ghouls from the Cthulhu Mythos; blind and mindless, without the power of speech.





\subsubsection{Relations with \nephilim}
See the section about \hr{Aryothim and Nephilim}{\aryothim and \nephilim}. 












\section{Draconian Ascendancy (\yds{Draconian Ascendancy})}
\target{Draconian Ascendancy}
\target{Draconian Supremacy}
The \quo{Draconian Ascendancy} was the event when the \dragons conquered \Miith. 
It heralded the beginning of the Age of Draconian Supremacy. 









\subsection{\Nexagglachel awakens}
\Nexagglachel still slept in durance, dead but dreaming.
\Sethicus \hr{Sethicus lets Nexagglachel sleep}{had let him sleep} but rigged him with spells so he would awaken millenia later. 
Finally \Nexagglachel awakened. 

When \Nexagglachel awoke, he immediately went out to awaken his two brothers.

When \Ishnaruchaefir was awakened by \Nexagglachel, he immediately went out to awaken \Rystessakhin. 





\subsubsection{The \dragons and their knowledge}
When the \dragons awoke, they had spent millennia pondering mysteries of science and magic. 
They had gained much insight into the occult, so their magic and mystic skills were immense. 
But they knew little of physical technology, because they had had no opportunity to experiment with building physical tools. 
So when they awoke, they could not just establish a new technological civilization. 
Furthermore, their skill was based on occult revelations and deep Gnosis which they could not easily teach to anyone else. 
So they remained monoliths of arcane power instead of building a new civilization. 

\Nexagglachel-tachi \hr{Nexagglachel could not rebuild Ophidian civilization}{tried to rebuild the \ophidian civilization but failed}. 





\subsubsection{The \ophidians were pressed}
\Nexagglachel saw that the \quiljaaran, the descendants of the \ophidians, were pressed. 
They were a \quo{fallen} people, after millions of decline from the golden age of the \hr{Ophidian civilization}{\ophidian{} civilization}. 

\target{The struggle for Dominion over Miith}
The \ophidians had been displaced as the Dominators of \Miith{}.%
\footnote{%
  Perhaps this is what the title of the series should refer to: 
  The struggle for Dominion over \Miith{}.%
} 
The \quiljaaran were retreating before the advance of the \aryothim.
They lived in half-empty ruined cities\dash a painful testament to the heights from which they had fallen, the greatness which they had lost. 
They were being pushed back hard. 
Some even feared their races would become extinct. 

Compare to \bandsong{Demilich}{The Planet that once used to Absorb Flesh in order to Achieve Divinity and Immortality (Suffocated to the Flesh that it Desired)}. 

\Nexagglachel decided that if the \ophidian golden age was to be restored, the task must fall to him and his \dragons. 

The \dragons thus set out to restore the world. 









\subsection{\Draconian Supremacy}
The \dragons were psycho-badass. 
They swept aside any \quiljaaran, \aryothim and \vorcanths that opposed them. 
They conquered \Miith and continued to dominate it for thousands of years. 
All the way up till the \secondbanewar. 





\subsubsection{\Aryothim conquered}
\target{Dragons destroy Aryothim}
The \dragons waged brutal war on the \aryothim, almost eradicating them. 
In their hatred, the \dragons destroyed much of the \aryoth civilization, including much of their hated \hr{Aryoth inventors}{technology} that had so long been a threat to the Serpentine peoples.
They destroyed universities, libraries and all sorts of containers of this hateful \aryoth knowledge, to ensure that \dragons would once again reign supreme. 

In the coming millennia, \hr{Aryothim as aloof gods}{\aryothim would be rare and keep a low profile}. 
So \hr{Aryoth-blooded rulers}{\aryoth-blooded \nephilim would rule in their stead}. 





\subsubsection{\Sardathrion}
\target{Sardathrion}
\index{\Sardathrion}
\Nexagglachel founded the city of \Sardathrion and made it his shining capital. 

\Sardathrion was later destroyed in the \hs{Incursion}.
It was never rebuilt.
It became a mythical icon of a bygone golden age.





\subsubsection{Golden age did not return}
\target{Nexagglachel could not rebuild Ophidian civilization}
The awakened \dragons could not rebuild the \ophidian civilization. 
\Nexagglachel tried his best, but it was hard. 
They were only a few individuals.
They had much scientific knowledge, but it was incomplete.
They remembered much of their science, but there was much technology that they could not replicate because they lacked the infrastructure necessary to produce it in any remotely efficient manner. 
Besides, they lacked the \hr{Draconian energy sources}{occult energy sources} which the \ophidian/\draconian civilization had used. 

The \ophidians would like to rebuild, too, but they had lost much knowledge over the many millennia, and besides, \hr{Noggyaleth suppress Ophidians}{they were being hounded by the \noggyaleth} whenever they tried to rebuild.

After \Nexagglachel died, only \Secherdamon retained the ambition to rebuild the \ophidian empire any time soon. 

The \dragons did make some progress, though. 
They introduced a lot of technology among their followers.





\subsubsection{Technology}
\target{Dragons repress technology}
\index{repression of technology!\dragons}
The \dragons repressed technology. 
For more than one reason:

\begin{enumerate}
  \item 
    \Nexagglachel saw the horrible destruction that high-tech weapons had caused.
    He decided that \humanoids were not intelligent enough to handle technology or powerful magic responsibly, so he repressed technology and kept it in the hands of the wise elite. 
    After his death, the remaining \dragons respected this aspect of his philosophy.
    It was part of the inspiration for the Unspoken Covenant. 

  \item 
    The \dragons cared primarily about sorcery, since this was what had saved their people and returned them to the throne of \Miith. 
    So that was what they researched and developed. 
    They disdained natural science as a poor man's substitute for true power, so they discouraged research into that area. 

  \item 
    The \dragons did not want lesser races to be able to gain control and dominate their betters. 
    \Dragons had a great natural talent for sorcery. 
    They had no such advantage in the area of natural science. 
\end{enumerate}

So they suppressed many sciences, and the overall technology dropped in a lot of areas. 





\subsection{\Caisith Autocracy}
\target{Caisith Autocracy}
The \Caisith Autocracy was a coalition of \ophidians who wanted to independent of the \dragons and not submit to their Supremacy. 
Each member was a \Caisith Autocrat.









\subsection[Scathae are created (\yds{Scathae created})]{\Scathae are created (\yds{Scathae created})}
\target{Origin of Scathae}
It is at this point that the \scathae were created. 
The \dragons{} wanted a servitor race to serve them. 
So they took some of their own \ophidian genes and mixed them with the genes of various \saurians, including \hr{Nycan}{\nycans}. 

The \scathae were bred from \hr{Locul}{\locul} stock. 
They were created with bits of the original \ophidian people, salvaged from mummies. 
And also with \cregorr genes. 

They performed many experiments. 
Most failed. 
But a few succeeded. 
These became the \scathae{}. 

\target{Cregorr came to serve Dragons}
The \dragons{} also tamed many of the \cregorrs{} and made them serve them. 

Both \scathae{} and \cregorrs{} had \xsic{} genes, which gave rise to the \hr{Scatha fury}{primal fury that lurked even in the peaceful \scathae}. 

Different aspects of the \scathaese mind \hr{Primordials and the Scathaese mind}{were associated with different \xss}.














\section{Draconian Supremacy waning}
\subsection{\Firstgendragons{} died}





\subsubsection{\Kserasshana{} and \Nexagglachel}
\target{Nexagglachel gets stewardship}
\index{stewardship}
Before her death, \Kserasshana{} talked to \Nexagglachel. 
He was her eldest and most skilled son. 

\begin{prose}
  \Kserasshana: 
  \ta{%
    After I and the other elders are gone, thou wilt be the greatest and best \dragon{} on \Miith.
    I charge thee with the stewardship of \Miith.
    Carry on our legacy.
    Lead our race, rule our world and keep it safe for our people.
    Your two brothers are also strong and have much potential.
    But you must lead them.}
\end{prose}






\subsubsection{The \firstgendragons{} died}
\target{Elder Dragons die}
\Tiamat{} eventually died in some war. 

But the heroic role they played in the war overshadowed their cruel tyranny. 
After the war their corpses came to be \hr{Elder Dragons worshipped}{worshipped as dead gods}, close to the \xss{} in status. 










\subsection{\Nexagglachel-tachi rise}
After the elder \dragons{} had all perished, it now fell to younger \dragons{} to lead their people (to the extent that the \draconic{} people is willing to be led). 
\Nexagglachel{} and his brothers were now the de facto top-of-the-pop. 

Without the \firstgendragons, the \dragons{} were weakened, but they still pulled through, under the capable leadership of \Nexagglachel{} and other great \dragonlords, and were able to strike back and exterminate the \banes{} that remained on \Miith. 









\subsection{\QuilJaaran{} supremacy}
\target{QJ supremacy}
The \quiljaaran{} were always rivals of the \dragons{} and \ophidians, but they sometimes allied. 

The \quiljaaran{} fought alongside the \dragons{} in the \firstbanewar. 
They suffered great losses in that war, but not quite as devastating losses as the \dragons{} (who lost almost all their elder leaders). 

So for a couple of thousand years after the \firstbanewar, the \quiljaaran{} were pretty dominant on the face of \Miith{}. 
They ruled several Realms. 

There were conflicts between \quiljaaran{} and \dragons{} in this time, but the \dragons{} were too few, too scattered to wage full-scale war, so they were most often forced to retreat before the expanding \quiljaaran. 

The \vorcanths{} and \nagae{} were also involved in these wars. 

Their domination lasted until \Nexagglachel{} was able to unite sufficiently many \dragons{} to challenge the \quiljaaran. 
\Nexagglachel{} intended to rebuild \ps{\TyarithXserasshana} empire. 

Maybe this \quo{\draconian{} revenge} \hr{Nexagglachel and Thanatzil}{coincided with \ps{\Thanatzil} time}.









\subsection{\Dragons{} disorganized}
\target{Dragons disorganized}
The \dragons{} were terribly disorganized after the \firstbanewar. 
\Nexagglachel{} made a heroic effort to unite them, and he achieved some success, but not as much as he had hoped for. 
The \dragons{} would continue that way all the way \hr{Second Banewar unites Dragons}{up till the \secondbanewar}. 

The \dragons{} were aloof and operated behind the scenes, letting the \quiljaaran{} and others dominate the picture. 









\subsection{Golden age of technology}
\target{Golden age of technology}
\index{technology}
The time between the \firstbanewar{} and the \secondbanewar{} was something of a golden age of technology. 
Many new things were invented in this period. 























