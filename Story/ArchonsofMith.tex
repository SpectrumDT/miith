
\part{Archons of \Miith}























\chapter{\TheLieSublimeBook}
\section{New civilizations appear}
After the \secondbanewar, \resphain{} and \dragons{} alike fought on with all the spite they could muster\dash against each other, and internally. For thousands of years. Both sides had lost great leaders in the war and were now divided, left in chaos. Their civilizations declined. 

Other civilizations arose around them. The \dragons{} and \banes{} alike sought to use these mortals as pawns in their game, reigning as their gods and sending them to die for the \secretwar.









\section{\Cuezca (\yds{Cuezca rises})}
\target{Cuezca}
The greatest of these new civilizations was \hr{Cuezca}{\Cuezca}. 
At the time when the \hr{Cuezcan}{\Cuezcan} people evolved, both sides in the \secretwar{} were especially weakened, weak enough for the nascent \Cuezcan{} civilization to develop in peace. 

\lyricsbalsagoth{Draconis Albionensis}{
  It was a time of change. \\
  The descendants of the Atlantean mages had fallen before the New Praesidium, \\
  and the wolves were baying at the Empire's door.\\
  An oppressive new faith was encroaching from the east, \\
  and the sylvan liege had locked tight the gates of his arboreal realm. \\
  And so it was that towards the end of the Age of Mystery, \\
  the last of Albion's great Dragon Lords did gather \\
  for what would be their final battle...
}

Some \Cuezcans{} serve the \dragons. They adopt \Draconic{} speech, runes and sorcery. 

\lyricsbalsagoth{Draconis Albionensis}{
  [The War-song of the Dragon Lords:]\\
  Dragon-phalanx rend the sky, Albion our gleaming prize,\\
  Sentinels of land and sea, guardians of destiny.\\
  Prowling amongst the pecseatan; \\
  Draconis Bipedes, swift and furious beast of battle!
  
  Dragonfyre in the fray, faith and steel shall win the day,\\
  A god to serf and king alike, the Adamantine Hammer strikes!\\
  Devouring the infidel outlanders; \\
  Draconis Nematoda, great winged worm of war!
  
  Dragon Imperium, throne of the Ancient Gods, \\
  behold the axiom, Wyruld-Cyninga! \\
  Dragon Imperium, behold the axiom. \\
  %Imperator Draconis, await your nemesis.
  All hail to Draconis Albionensis!
}









\subsection{Different Realm}
\Cuezca{} existed in some \hs{Shrouded Realm} different from \hr{Azmith}{\Azmith}. 
The fall of \Cuezca{} took place not long before the birth of \hs{Cordos Vaimon} and \hs{Silqua}. 
In fact, \hr{Cuezca motivated Vaimons}{it motivated the inception of the Vaimon order}. 









\section{The \CuezcanApocalypse (\yds{Cuezcan Apocalypse})}
\target{Cuezcan Apocalypse}
\Dragons{} and \banes{} alike recognize that \Cuezca{} is a power factor, so they both try to sway them to their cause. 
They woo the \cuezcans{} very directly, using all means. 
They also teach their magic and science to the \cuezcans{}. 

The \cuezcans{} become megalomanic.

\lyricsbalsagoth{Draconis Albionensis}{
  There is no ignominy, there is glory.\\
  There is no servitude, there is dominance.\\
  There is no defeat, there is victory.\\
  Victory eternal!\\
  
  It is time! \\
  We shall rule, and upon our dominion the sun shall never set!
}

\lyricsbalsagoth{Return to the Praesidium of Ys}{
  Your invocations unleashed the Great Worm\\
  which compelled the devouring seas to [\Cuezca].}

\index{technology!\cuezcan}
The \cuezcans{} master magic and technology. They wage war with nukes, bio-bombs and what is worse. 

\lyricsbalsagoth{Atlantis Ascendant}{
  Prophecy carved in stone\\
  aeons past by hands unknown.\\
  Winged fiends wheel forth, attack.\\
  Carnage as the Sun burns black.
  
  Doomed, doomed... the end is nigh.\\
  Your realm is lost... it shall be devoured by the sea!
  
  The Third Great Cataclysm shall reshape the face of Creation.
  
  The worm comes, riding the ravening oceans.\\
  The Outer Darkness disgorges its horrors. \\
  It is foretold. [\Cuezca] shall be destroyed.
  
  Hear the call, [\Cuezcans], proud we stand forever, \\
  mightiest of warriors, we sail across the sea. \\
  Conquering the ancient world, a legacy eternal, \\
  raise the arcane sigil high, steel and sorcery! 
  
  Blessed with immortality, \\
  dreaming spires of majesty, \\
  glory crowns our destiny!
  
  And so it was written in the stars, \\
  astride the world would stand the children of Atlantis!
}

They use terrible weapons of mass destruction. 

\lyricsbs{Monolith Deathcult}{Deus Ex Machina}{
  He threw between the three cities of the Vrishnis and the Andakas one single projectile, that was loaded with all the power of the universe. 
  A white glowing column of smoke and fire, as bright as ten thousand suns arose in all its splendour. 
  This was the unprecedented weapon, the iron lightning, a giant messenger of death, that burnt the entire family of Vrishnis and Andakas to ashes. 
  
  (Taken from the Mahabharata one fo the two major Sanskrit epics of ancient India.)
}









\subsection{The First Advent of \Lithrim}
\target{First Advent of Lithrim}
\hr{Lithrim}{\Lithrim}, the \bane god of destruction, was spawned.
With the help of the mighty \Cuezcans and their mighty sorcery, it was summoned to \Miith.
Here it wrought havoc. 

After a terrible battle, the Sentinels defeated \Lithrim and destroyed it. 

The Sentinels believed \Lithrim was now defeated forever. 
But the \banes had more tricks up their sleeve.
\Lithrim had been blasted into many fragments. 
The \banes took the fragments of \Lithrim and put them inside \human bodies. 

But to obscure their true purpose, the \banes moved \Lithrim from the \Cuezcan Realm and into another Shrouded Realm: 
\Azmith.
The Sentinels did not suspect this, so they did not detect it. 

\target{Men of Light created}
These \humans who carried \Lithrim inside them became a particular \demihuman race:
The \hr{Men of Light}{Men of the Light}. 
Cordos Vaimon and Silqua were considered the founders of this race. 

Compare to how God incarnates in \humanity in the anime \cite{Anime:MaoDante}. 
And the way \humanity is really the angel Lilim in the anime \cite{Anime:NeonGenesisEvangelion}. 









\subsection{The world is devastated}
The result is a catastrophe: 
With their newly acquired power, the \cuezcans{} cause immense destruction, tearing great expanses of the Realms apart. 
\Cuezca{} is destroyed in the process. 

The world is violently torn apart and laid waste by the terrible destruction. 

\lyricsbible{Revelations 8:7}{
  And there followed hail and fire with blood, 
  and they were cast upon the earth.
  And the third part of trees was burnt up, 
  and all the green grass was burnt up.
  And the name of the star is called Wormwood.\\
  And the third part of the waters became wormwood.\\
  And many men died of the waters,
  because they were made bitter\\
}









\subsection{The Unspoken Covenant of the \Charade}
After the \CuezcanApocalypse, some leaders among both sides, \Ishnaruchaefir{} among them, realized that the war was going too far.

There was a fear that both sides were going down a road of \quo{Mutually Assured Destruction}, and there was a desire to contain the war.
The \dragons{} and \banes{} realize that open war\dash especially open war using mortal pawns, armed with power they can't wield responsibly\dash is no good, and that if they continue this way, they will destroy the world, leaving nothing to rule. 
Also, wise ones predicted that given the \hr{Balance of the Matrices}{balance of the \matrices}, the war was unlikely to see any resolution for thousands of years. 

And so, by silent mutual consent, the Unspoken Covenant was created.
It would separate the mortal world from the immortals' endless war. 

\target{Tiphred-Serah formulate Unspoken Covenant}
Some \resphain{} of \TiphredSerah{} were instrumental in formulating and popularizing the Covenant. 

The Cabal and the Sentinels of \Miith{} are formed and the Unspoken Covenant is adopted. The Covenant dictates that the Cabal and Sentinels strive to keep mortals ignorant of the true world that surrounds them. They must not know of the secret organizations' existence, the underground war, their own origins nor the nature of the Realms. Also, the progress of science (including magic) must be suppressed and kept in check, and the Cabal and Sentinels must not teach their arcane knowledge to outsiders. 

The \Feud ing factions agreed on the Unspoken Covenant, and the \charade{} began. 
Among the master races there was now a widespread agreement to phase out their visible dominance in favour of a more hidden dominance. 
In the Shrouded Realms, that is. 
Not the Immortal Realms. 

This was both easier and more necessary now, because there were so few immortals left. 

\lyricsdimmuborgir{Reptile}{
  And so I will take shelter\\
  in the absence of the light.\\
  Hiding like a masked miniature in the dark.\\
  A revenant without relief, it seems.\\
  For the art of becoming a progeny \\
  and to be raised in such curse\\
  is to forever creep among \naive{} mortals,\\
  Infesting the dead in herdes.
}









\subsection{Vaimons came soon after}
Not long after the \CuezcanApocalypse, the \hs{Cabal formed}. 
Not long after that, they \hr{Cuezca motivated Vaimons}{began to plan the Vaimon gambit}. 























\chapter{\SilquaBook}















\section{Premise}
This book has two parallel plots: 

\begin{enumerate}
  \item 
    The \hr{Cabal formed}{forming of the Cabal} and the \hr{Fall of Kezerad}{fall of \Kezerad} in the Immortal Realms. 
  \item 
    Cordos' empire-building-business and Silqua's religion-founding-business in \Azmith. 
\end{enumerate}









\subsection{The Forming of the Cabal}
\target{Cabal formed}
The Cabal was formed not long before this book. 
Perhaps during the \CuezcanApocalypse{} decades before, perhaps just few years before. 

I need to remember to have a prologue showing the Cabal being formed. 

Some the \banelords, after being locked out in the \hr{Shrouding}{\Shrouding}, had wormed their way back into the world and gained a stronger foothold. 
This gave them new power. 
Suddenly they had the political weight to once again coerce the \resphain{} to do their bidding. 
They decreed that the dynasties should ally and form a united block. 
Whether the dynasties liked it or not. 

\target{Tiphred-Serah form Cabal}
Some \resphain{} of \TiphredSerah{} were instrumental in forming the Cabal. 
It was they who came up with the ideas for the organization of the Cabal and approached the other dynasties and negotiated the terms of their alliance. 

\target{Consolidation}
\index{Consolidation}
The slow process of allying the various \resphan{} factions was called the Consolidation. 
It took several decades.





\subsubsection{\Cuezca motivated Vaimons}
\target{Cuezca motivated Vaimons}
The fiasco of the \CuezcanApocalypse{} was part of what motivated the formation of the Vaimon order. 
The Cabal agreed on the \hs{Unspoken Covenant}, which limited their direct involvement in the world. 
This meant they would need to have servants to interact with the world for them. 
Preferably powerful ones. 
Preferably mages. 
So they began plotting how to found a Cabal-sponsored order of mortal mages. 









\subsection{The Fall of \Kezerad}
\target{Kezeradi War}
\target{Fall of Kezerad}
\TiphredSerah, \CiriathSepher{} and \Mystraacht{} were eventually bullied into teaming up and form what became known as the Cabal. 
But most of the \Baelzerach{} tribes refused. 

More importantly, \Kezerad{} refused. 

\Daggerrain{} was not happy. 
He had his eyes on the \Kezeradi{} \dweomer. 
He recognized that it had great potential to serve his plans as a prison for \human{} souls, but he needed to gain control of it and corrupt it first. 
\Daggerrain{} decided that if the \Kezeradi{} would not heel, they must be destroyed. 
So he had the other dynasties invade \Kezerad. 

The other \resphain{} were all too eager to wage war against \Kezerad{} to \hr{Resphan vampirism}{sate their cannibalistic hunger}. 

Perhaps \Kezerad{} was already at war with the \Baelzerach{} or another foe, and their fellow \resphain{} backstabbed them.

Anyway, \Kezerad{} was invaded by \resphan{} and \bane{} armies and was destroyed. 
Compare this to the destruction of the Blue Dragonflight in the \emph{Warcraft: War of the Ancients} books by Richard Knaak. 

Their leaders were transformed into the terrible \Sephiroth{}, and their \iquin{} was corrupted and twisted into the loathsome \carcer{} it is today.

%They fought against their former masters and their fellow \resphain, and eventually they were destroyed. Their leaders were transformed into \Sephiroth. 









\subsection{\Imrath{} and the World Around It}
At this point, \Imrath{} is at war with a neighbouring \scathaese{} kingdom. Relations between \humans{} and \scathae{} are very bad throughout the continent. Some of the \scathaese{} kingdoms are ruled by \dragons{} (overtly or covertly), and most \humans{} learn to hate the \dragons{} as evil monsters, and the \scathae{} (called \squo{Creeps}) as their evil monster-spawn. Among \humans{}, all magic is scorned as evil \squo{{\dragoncraft}}. The overall technology is Bronze Age (about 500 BC in RL terms). 

Some nations (especially \scathaese{}) worship \draconic{} gods, or the \dragons{} themselves. Others worship other monstrous gods. In \Imrath{} (and several other \human{} kingdoms), all these \squo{demon gods} are scorned, and the official religion is a vague belief in \squo{the Light}, representing good and justice. This religion is based on fiction: The force of \squo{the Light} is not known to actually exist. 





\subsubsection{The Peoples}
The \Imrathi{} were \hr{Imrathi nomads}{nomads}. 

Or rather: 
Cordos Vaimon's people were a nomadic tribe, mongol-like but with lower technology. 
Silqua's people wre settled farmers, Ancient Egypt-style. 
The two peoples were occasionally at war, but also had peaceful relations. 

Cordos wanted to unite the two peoples and build a great empire. 
Or maybe his predecessor (as chieftain/king) wanted this, and Cordos inherited the ambition. 

Compare Cordos to Waldemar Selig from \cite{JacquelineCarey:KushielsDart}, except good rather than evil. 






\subsubsection{Sexuality}
\target{Sexuality at Cordos' time}
Both peoples were sexually liberal and had sex left and right. 
Silqua was a professional courtesan who had lots of sex, sometimes for political reasons. 
Compare her to \PhedreNoDelaunay{} from \cite{JacquelineCarey:KushielsLegacy}. 









\subsubsection{Technology}
At this point, technology is low. 
Compare to Ancient Egypt the like. 





\subsubsection{Slavery}
The \Imrathi keep thralls. 
They are chiefly prisoners of war. 







\subsection{\Banelords{} covet \ps{\Eryal} \carcer}
\target{Banelords wanted to use Eryal's Carcer}
The \banelords{} were formulating their \hr{Sephirah plan}{\sephirah{} plan}. 
They wanted to create a giant \hr{Carcer}{\carcer}, big enough to rip asunder the Shroud. 
But they needed to do more research. 
They wanted to study the \malachim{} and their \hr{Malachim binding souls}{power to bind souls}. 

This power was especially strong in \hr{Eryal}{\Eryal}, because of her \hr{Eryal binding souls}{attractive personality}. 
The \banelords{}, by a stroke of luck, had been able to determine from the \matrices{} that \Eryal{} would soon incarnate (as Silqua, it would turn out), so they started planning. 
They wanted to set her up to bind a lot of souls and study her to see how it worked. 

So they ordered their Cabalists to get to work manipulating Silqua and Cordos-tachi. 









\subsection{\Resphan{} Xanatos Gambit}
The book opens with some \resphain{} discussing their plans. 

A \TiphredSerah{} \resvil: 
\ta{Here is my idea: A child messiah.
  An innocent young girl chosen by the gods to save the world.}

And from there they went on to plan and railroad Silqua's entire life and career. 
She was a \trope{XanatosSucker}{Xanatos Sucker} and intended as a bit of a deconstruction of the \quo{chosen saviour} archetype. 

The Cabal wanted to seize control of the Vaimons and use them to create a \resphan-backed evil empire. 

Gradually, the \resvil{} started developing an attachment to Silqua and feeling like a mother to her. 
After all, Silqua was so sweet that once you got to know her you could not help but love her. 
The \resvil{} was devastated when \hr{Silqua dies}{Silqua was killed}. 





\subsubsection{\Cishiel{} was there}
\Cishiel{} was one of the masterminds behind the Vaimon project. 
She was slightly sadistic and enjoyed putting Silqua through torture. 
When she and the others discussed how to manipulate Silqua's fate, \Cishiel{} would often try to make her suffer some more. 
\hr{Cishiel hates Eryal}{She hated \Eryal}. 

A few times \Cishiel{} would assume \human{} form and have SM sex with Silqua, where she cruelly tortures her. 

But remember to make \Cishiel{} sympathetic, not just evil. 





\subsubsection{Early Vaimons overpowered}
\target{Early Vaimons overpowered}
The early Vaimons were more powerful than they should be. 
Their immortal patrons stood directly behind them and empowered them with direct help whenever they needed it. 
The Vaimons needed this power in order to \hr{Cordos vanquishes monsters}{vanquish the pre-\human monsters}. 

But occasionally one of the important pawns would get lost and somehow severed from her patrons, and the patrons would scramble about trying to regain touch. 

Several of the early Vaimons were secretly Cabalists. 
These acted as advisors who made sure to push Silqua and Cordos-tachi in the right direction. 





\subsubsection{It went wrong}
The Silqua gambit ended up going astray because the Cabal was still newly formed and the different factions fought much against each other. 
Plus, the Sentinels were working against them. 
This chaos gave \Delphine free reins to have her love affair with Silqua, which became more and more depraved and ended with her killing Silqua. 















\section{The Half-Men of Lom}
\target{Cordos conquers Lom}
The following story might be a short story of its own, or it might be part of the novel about Silqua and Cordos.

Cordos Vaimon wiped out the half-men of \hs{Lom}, slew them all and destroyed all icons of their gods.

Compare to \cite{HPLovecraft:TheDoomThatCametoSarnath}. 

Cordos attacked Lom in daylight, for none dared approach it at night, where eerie shapes could sometimes be seen rising from the lake, where the Lom-men would dance and giber and give obeisance. 
Even in daylight there were men who disappeared by the misty shores of the lake, snatched by unseen things. 
When Cordos attacked the great temple of Mnoth, some of his men came running out of the temple screaming, driven insane by the nameless things they had faced. 

Cordos stepped inside the temple himself.
He saw that it was much larger on the inside than on the outside. 
He fled screaming. 
It must be a gateway to Hell, for he saw glimpses of torture and slime and mutilated bodies and screams and other dim horrors inside. 

Cordos dared not send any more men into the temple. 
He ordered his men to destroy it from without using great battering rams. 
This failed.
Instead they blocked up the entrance, thus barricading the last Lom-men and their sorcerers inside. 









\subsection{Kua}
Kua was a young \human girl.
She was kidnapped by the Lom-men shortly before Cordos' attack. 
She was being held captive for a purpose she did not know. 

Then Cordos' soldiers found her.
She was naked and tied up but unharmed. 
She was happy and expected to be rescued.
Instead the evil soldiers raped her. 

Perhaps in the end the soldiers killed Kua.
Or perhaps she was rescued by Cordos Vaimon, who had the evil soldiers killed.
Or perhaps the soldiers were taken by Mnoth and died a horrible death, screaming in agony and horror.















\section{Silqua and Cordos (\yds{Silqua birth})}









\subsection{Silqua (\yds{Silqua birth})}
The history of the \VaimonCaliphate begins with \hr{Silqua}{Silqua \Delaen}, the daughter of \hr{Maegon Delain}{Maegon \Delaen}, a noble in the \human{} kingdom of \hr{Imrath}{\Imrath}. 
She has two older brothers, \hs{Arcan} and \hs{Lestor}. 

As a young girl, Silqua is very intelligent and reads many books. 
She is religious and believes in the Light and the \sephiroth.\footnote{Are the \Sephiroth{} an accepted part of dogma at this time, or does Silqua invent them?}









\subsection{Cordos woos Silqua}
The nations of \hr{Imrath}{\Imrath} and \hs{Calaan} were at war. 
\Imrath won and Calaan was subjugated. 

\hr{Maegon}{Maegon \Delaen}, king of Calaan, came to the Vaimon king and offered him his virgin daughter Silqua as a peace offering. 
Silqua was to marry Belandos Vaimon to mark the peace between them, and Calaan would be subject to \Imrath as a wife is subject to her husband. 
At this time Silqua was 18 or so.

Belandos was not impressed. 
Silqua was beautiful.
Still, he had demonstrated that he could defeat the Calaanites in war.
So he had half a mind to just take Silqua and assfuck the Calaanites anyway. 

Maegon grabbed his daughter and pointed a dagger to her. 
He threatened to kill her himself rather than let her be violated. 
Silqua was deathly afraid, but she forced herself to face her fear. 

The king laughed.
Silqua was very lovely, true, but he had many beautiful women already and did not really need another. 
But then his son Cordos, who had long stared at the beautiful Silqua with desire, stepped forward. 

Cordos:
\ta{Father!
  Give me the girl.
  Remember that you owe me a boon for \hr{Cordos conquers Lom}{my conquest of the half-men of Lom}.
  I ask now for my reward: 
  Give me this girl. 
  I will take her as wife.}

The king agreed and Silqua married Cordos.









\subsection{Cordos and his title}
\quo{Vaimon} is a title that Cordos is awarded. 
It is given to him by \Delphine{} (the oracle-type mystic) after some major accomplishment. 
It means \quo{conqueror} or something like that. 









\subsection{Cordos vanquishes monsters}
\target{Cordos vanquishes monsters}
At this time, \Miith{} was still dominated by \quo{inhuman} powers: 
Gods, immortals, aliens and monsters. 
The \hs{Unspoken Covenant} was still a new thing and not everyone followed it. 
It was still an \quo{\hs{Age of Gods}}. 

On \Azmith, Cordos Vaimon got chosen by the Cabal as a \quo{champion} to drive away the dominant monsters and nonhumans and usher in a new aera: 
A \quo{\Human{} Age}. 
In time he came to see himself as such a champion, too. 

Cordos and Silqua would spearhead the newborn Iquinian religion, using their formidable Vaimon magic. 
In the end, \Azmith{} ended up under \human{} dominance, with the \scathae{} being mostly subjugated. 
The Vaimons fought and vanquished the evil pre-\human{} monsters. 

In those days, the early Vaimons were \hr{Early Vaimons overpowered}{overpowered} because the \resphain helped them. 

Compare to many of Robert Howard's stories, where triumphant Man vanquishes the evil pre-\human{} monsters. 
Especially \cite{RobertEHoward:TheShadowKingdom}. 

Also compare to how the T'lan Imass wage a war of genocide against the Jaghut in \cite{StevenEriksonIanCameronEsslemont:MalazanBookoftheFallen}. 

Cordos was a Conan/Kull-like hero. 
Silqua was also a Robert-Howard-esque princess/heroine. 

Remember, though, that the true story of Cordos and Silqua can be \emph{vastly} different from the official story later told by the Iquinians. 
The Cabal-backed church can have twisted and forged the story like crazy. 









\subsection[Byakun]{\Byakun}
\Byakun was a dark priest who lived in Silqua's time. 
He was a dark mage and very feared. 
Silqua's people were afraid of him. 

Compare him to various Lovecraft figures. 

At some point, Silqua (somehow) heroically volunteered to be sold to \Byakun as a slave. 
She did some hero work, and Cordos came in and saved her. 

Compared to the Mahrkagir in \cite{JacquelineCarey:KushielsAvatar}, who takes \Phedre as a slave. 










\subsection{\Delphine}
\Delphine, despite what I may have written earlier, was not an enemy of Cordos and Silqua. 
She was an aloof, oracle-like mentor type whom Cordos-tachi had to ask for help. 






\subsubsection{\ps{\Delphine}{} sexuality}
A theme of these books is the contrasting sexuality of the two Scions, Silqua and \Delphine{}. 
% : Where Silqua has principles of love and marriage and fears sex to some extent, 

\Delphine{} uses sex as a meants to become free, contact her true, inner power and escape the devouring emptiness that haunts her. 

This haunting is a manifestation of the \pps{\banes}{} fear of \hr{Entropy}{decay and stagnation}. 
Sex is a way to escape from stagnation and bring change and new creation. But the entire existence of the \banes{} is evil, destructive and parasitic. 
The \resphain{} have inherited this nature, and this is reflected in their sexuality. \Shiaraid{} is just a particularly obvious example. 

Note that natural sex, based on the chaotic power of the Heart of \Miith{}, is still wild and violent. But \resphan{} sex is worse. 









\subsection{Silqua's marriage and Vaimon career}
Silqua was a sweet and innocent girl who tried her best to do good in the world with the power she had. 
At first, she believed her power to be a beautiful gift from the gods. 

\citebandsong{Nightwish:AngelsFallFirst}{Nightwish}{Astral Romance}{
  Macrocosm poured its powers on me\\
  and the hopes of this world I now must leave...
}

At some point, Silqua was given as a gift to Cordos Vaimon to unite the two peoples. 
This was \emph{after} she had discovered her magic. 

She tried to pull Cordos in a good direction, but many forces were pulling him (and her), and he leaned in a darker direction that she would have liked. 
She became sad. 

She loved Cordos, but found herself more and more estranged from him. 
She grieved.

\citebandsong{Nightwish:AngelsFallFirst}{Nightwish}{Astral Romance}{
  The distance of our bridal bed\\
  awaits for me to be dead.\\
  Dust of the galaxies, take my hand, \\
  lead me to my beloved's land.
}









\subsection{Silqua discovers \iquin{} and \itzach}
Somehow, Silqua discovers \iquin{} and \nieur{}. 
She believes that \iquin{} is the Light which her religion worships. 
She learns how to channel \iquin{} and thus cast magic. 
When this is discovered, Silqua is accused of practicing evil \dragoncraft. 
But she convinces Cordos that she is really good, that the magic she has discovered is truly the manifestation of the Light. 
She teaches a number of people to channel \iquin{}. 
These include Arcan, Lestor, \Delphine and possibly Cordos Vaimon. 
She also teaches them to detect and recognize \nieur{}, but not to channel it. 

Silqua herself believes that she is the Prophet of the Light, born to bring knowledge of the Light to \Miith{}. She also discovers \nieur{}, the force of Darkness. She believes that \nieur{} contains all that is evil and must be shunned. 





\subsubsection{Why Silqua awakens}
Silqua's burgeoning powers are awakened by her sex life, her career as a prostitute. 
Her continuous sexual submission to the whims of her masters awakens dim memories of her past life, where she was \ps{\Shiaraid} submissive lover. 
But only vague, emotional memories. 
Never any concrete thoughts. 
Silqua never remembers her identity. 

But a link to her true self has been established. 
She has vague visions and \dejavus. 
And her powers grow. 
She slowly approaches \hr{Apotheosis}{\Apotheosis}. 





\subsubsection{Healing: Her first magic}
The first magic that Silqua learns is healing. 
She needs it to heal herself after sexual torture. 





\subsubsection{First invocation near Rubellah}
\target{Place of Silqua's first invocation}
\target{Rubellah}
Once Silqua was fleeing from some enemies of hers. 
They were slavers who wanted to sell her to the \Saruns. 
She hid in a cave near the rock of Rubellah.
Here she used magic in an attempt to fight them off.
It was not very efficient, not was it the first time she ever used magic.

History distorts things.  
In later legends, Rubellah became known as the first place where Silqua ever invoked the \sephiroth (which was not true). 
It became a holy site for the Vaimons (\hr{Redcor control Rubellah}{controlled by Clan Redcor}).










\subsection{Silqua feels angels}
\target{Silqua feels angels}
She feels the presence of some good angels who watch over her and protect her. 

\citebandsong{Nightwish:AngelsFallFirst}{Nightwish}{Angels Fall First}{
  An angel face smiles to me\\
  under a headline of tragedy.\\
  This smile used to give me warmth.
}

These angels were actually the \Kezeradi. 
They knew that she was an amnesiac \malach, but they also knew that she was not fully evil and had been somewhat sympathetic to their cause. 
They hoped to sway her to their cause and, through her, create a good empire on \Miith{}, where peace and goodness could reign. 

The \Sephiroth{} that Silqua felt were the original \Kezeradi{} \Sephiroth, before they were corrupted by the other \resphain. 

But the plan was interrupted. The evil \resphain{} invaded and destroyed \Kezerad, twisted the \Sephiroth{} into abominations, and usurped the \ps{\Kezeradi}{} gentle quest. 

\lyricsbs{Vital Remains}{Dechristianize}{
  Like cancer our hate consumes the light of Elysium. \\
  Unstoppable force of demonic supremacy. \\
  All destroying, all devouring.\\
  Heaven now ravaged, scarred and empty. 
  
  Strike the death knell of the Pandemonium. \\
  Imbrue one's hands in the blood of Christ. \\
  Washing away all filth of righteousness, \\
  the dimming of the light, \\
  engulfing the trinity.
  
  Our poisonous truths. \\
  Divinity drowning in impurity. \\
  Unconquerable, unstoppable. \\
  Sanguilent in your agony.
}

This happened late in Silqua's life. 
Before, she had had premonitions, evil dreams and visions that this was going to happen\dash although she didn't understand it, since she didn't quite know who the angels were. 
She only knew that there was a conflict between light and dark angels. And she was afraid of her premonitions. 




\subsubsection{She sees hints of darker powers}
She also receives vague visions telling about how there exist powers far darker and far more evil than the angels. 

\citeauthorbook{EdgarAllanPoe:TheConquerorWorm}{Edgar Allan Poe}{The Conqueror Worm}{
  Mimes, in the form of God on high,\\
      Mutter and mumble low,\\
  And hither and thither fly\dash\\
      Mere puppets they, who come and go\\
  At bidding of vast formless things\\
      That shift the scenery to and fro,\\
  Flapping from out their Condor wings\\
      Invisible Woe!
}





\subsubsection{Premonitions of the fall of \Kezerad}
She sees premonitions of the \hr{Fall of Kezerad}{fall of \Kezerad}. 
She sees them at war with the \Baelzerach{} (commanding \xsic{} powers and \daemonic{} servants), and later with the Cabalist \resphain{} (with \banes{} in their armies). 

\citeauthorbook{EdgarAllanPoe:TheConquerorWorm}{Edgar Allan Poe}{The Conqueror Worm}{
  But see, amid the mimic rout\\
      A crawling shape intrude!\\
  A blood-red thing that writhes from out\\
      The scenic solitude!\\
  It writhes!\dash it writhes!\dash with mortal pangs\\
      The mimes become its food,\\
  And seraphs sob at vermin fangs\\
      In human gore imbued.
}

She sees what she interpets as the \quo{end of the world}:

\citeauthorbook{EdgarAllanPoe:TheConquerorWorm}{Edgar Allan Poe}{The Conqueror Worm}{
  Out\dash out are the lights\dash out all!\\
      And, over each quivering form,\\
  The curtain, a funeral pall,\\
      Comes down with the rush of a storm,\\
  While the angels, all pallid and wan,\\
      Uprising, unveiling, affirm\\
  That the play is the tragedy, \quo{Man},\\
      And its hero the Conqueror Worm.
}

What she sees is actually the fall of \Kezerad{} and the fall of \humanity{} into slavery under the yoke of the corrupt, Cabal-controlled \VaimonCaliphate. 
A defeat for freedom and a failure for the \Kezeradi{} attempt to break free of the eternal \Feud. 

But people after Silqua's time don't understand this. 
They believe that her visions refer to some \quo{Doomsday} that will come in the far future. 
They don't know that it has \emph{already happened}. 

I ought to write Silqua's prophecy in archaic language, like the \emph{King James Bible}. 









\subsection{Silqua and \Delphine}
\target{Silqua and Delphine were lovers}
Silqua and \Delphine{} met and fell in love. 
They sort of had to; Silqua was \Aryal{} and \Delphine{} was \Shiaraid. 
They were lovers before, so when they met again those feelings awakened. 

They did not have sex at first. 
But at some point, Cordos Vaimon desired \ps{\Delphine} allegiance. 
\Delphine{} demanded to get Silqua as a gift in return. 
Cordos would not \emph{give} her away. 
But Silqua \emph{wanted} to be given away. 
(This was partially her self-sacrificing heroism and partially because she was attracted to \Delphine{}.)
So after some negotiation, \Delphine{} was haggled down until she accepted simply \emph{borrowing} Silqua as a sex slave for a while. 

They became lesbian lovers. 

Silqua was attracted, but she also feared \Delphine. 
She subconsciously remembered how abusive \Delphine{} had become, and \hr{Shiaraid and Eryal break up}{how ugly their break-up had been}. 

\Delphine{} genuinely loved Silqua. 
But her mind was being twisted by \hr{Curse}{\NexagglachelsCurse}, and her \hr{Self-destructive Shiaraid}{self-destructive madness} made her abusive and evil and drove them apart. 











\subsection{Silqua and \Delphine{} have visions}
Both Silqua and \Delphine{} have dark visions of the angels.

\citelimbonicart{MoonintheScorpio}{Beneath the Burial Surface}{
  Ancient black, silent gloom.\\
  Cathedral bells are calling doom.\\
  In velvet dreams I am touched by sin.
  
  As night arrives in its purple shades\\
  I drift across the shallow graves.\\
  The soul is streaming in the wind.
  
  Dark is the blessing that I am in.
  
  As darkness falls and the cold silence reigns,\\
  the nocturnal void shall become my faith.\\
  I'll transcend unto where shadows dance
  
  A gentle kiss and like a bird I'll fly\\
  into the spheres of demise,\\
  desireously in dark romance.
} 

\citelimbonicart{MoonintheScorpio}{Moon in the Scorpio}{
  It is a time of great light\\
  and of great darkness.\\
  Can you feel the presence\\
  of its phenomenon?
  
  Midnight is the shepherd of mysterious powers\\
  and moving shadows in the corner of the eye.\\
  Moon's blazing intuition\\
  contains what death requires.
  
  Cleanse the doors of perception.\\
  See things appear in its true art.\\
  The cold hands of divinity\\
  will tear thy soul apart.
}









\subsection{\Delphine{} dreams of her past}
\Delphine{} dreams of her past life as a \resvil. 

\citelimbonicart{LegacyofEvil}{Twilight Omen}{
  I had a vision in a dark cryptic dream.\\
  Floating in the universe, ethereal cosmic streams.\\
  My soul had black wings and triumphant I did fly.\\
  I rode the storms and the midnight sky.\\
  I saw the thousand lights from cities underneath me,\\
  as I ventured deeper into the night,\\
  to that sacred place beyond the twilight zone.
}





\subsubsection{She wants to know more}
\Delphine{} wants to know more of these visions, and of her own identity. 

\citelimbonicart{LegacyofEvil}{Twilight Omen}{
  Open the mind to the dark secrets of the soul.\\
  The atmosphere is electrical, condition paranormal.\\
  Watch the setting sun and the rising moon.\\
  Crossing over to a different sphere,\\
  passing through the membrane,\\
  the window to the other side.
  
  Twilight omen.
}









\subsection{Silqua becomes sick}
Silqua uses Vaimon magic. 
Recall that \hr{Parasitic Archons}{the \Archons{} are parasitic}. 
There are no Iquinians at this time, so Silqua herself bears the full brunt of the evil inherent in \Iquin. 
This means that she is often sick and weak and in pain. 














\section{Imperial Ambitions}
At some point, Belandos II dies and Cordos becomes king. He is quick to see the potential of Silqua's discovery and goes about organizing all believers of the Light into a single religious state. He manages to unite a large kingdom. Religious fervor is the key here, not the power of the magic itself. Silqua is a scientific genius and develops a lot of spells and techniques, but \iquin{}-\nieur{} magic is newly discovered, so all their spells are still primitive and somewhat weak. But the really important part is that Silqua's \iquin{} magic proves that the Light truly exists, and that she is its Prophet. 

Cordos desires to rule an empire, so he wages war on neighbouring kingdoms, \scathaese{} and \human{} alike. %Silqua initially approves of his war against the \scathae{}, since she has been raise to believe that the Creeps are evil monsters. 
Silqua initially approves, but along the way she begins to question whether war and conquest is the will of the Light. 









\subsection{Diplomacy and conquest alike}
The \VaimonCaliphate was not built through conquest alone, but even more so through diplomacy and religious propaganda.
History and legend later emphasized both Cordos' heroism in war and conquest and Silqua's gentle charisma. 
The truth was much more complex than that. 

















\section{It Goes Downhill (\yds{Silqua death})}
\subsection{\Kezerad{} falls, \iquin{} rises}
Fair \Kezerad{} is destroyed and the cruel \iquin{} is created. 

The last \Kezeradi{} behold the newborn, evil \iquin. 

\citebandsong{DeathspellOmega:FasIteMaledictiinIgnemAeternum}{%
  DeathspellOmega
}{
  The Repellent Scars of Abandon and Election
}{
  Nurtured by the multitude of man's misfortune, \\
  a thousand halos like torches in the night of the spirit, \\
  a thousand traps, pitalls of brimstone and the empty sky, \\
  prostrated face against the earth in frantic laughter...
}









\subsection{Silqua is raped and broken}
Silqua is taken captive by her enemies. 
She is cruelly raped and tortured. 
Her enemies want to break her spirit and turn her into their slave, make her worship her rapists and tormentors. 
Compare to \cite{GeorgeOrwell:1984}. 

This is necessary in order to fully and thoroughly corrupt and infest \iquin. 
See, Silqua is bound to \iquin. 
She is the mightiest of all the souls bound to \iquin. 
\Aryal{} is \Kezeradi{}, remember. 
Perhaps also a \sathariah, and perhaps one of the mightiest \resviel{} ever. 

They almost succeed, but she is rescued in the end. 
She is badly traumatized after this, though. 

Or maybe she channels \nieur{} and kills a bandit, but is horrified by her actions. 
This leaves her traumatized, and she develops an irrational fear of both \nieur{} and sex. 

\citelimbonicart{TheUltimateDeathWorship}{Suicide Commando}{
  Tyrant in soul and flesh, pain is the unholy mistress.\\
  Rites in earthly death, for darkness you confess.\\
  Follow the voices of the night, \\
  in endless sleep you'll hide from light.
  
  Dark was the day you were born, \\
  even darker on that eve you were torn (from life).\\
  Die in martyrium, for darkness' endless mysterium.\\
  Die in silent spasms of screaming. \\
  Life is infected by no meaning.
}





\subsubsection{The fall of \Kezerad{} scars her}
When \Kezerad{} finally fell and the \Sephiroth{} were corrupted and enslaved, Silqua went mad. 
Perhaps this happened while she was captured by the enemy, or perhaps her madness \emph{made} her fall into enemy hands. 
After this, she was raped, tortured and finally killed. 








\subsection{Silqua becomes more morbid}
Later in her career, Silqua's visions take a turn for the darker, more morbid. 
This has two reasons: 

\begin{itemize}
  \item She begins to have premonitions of the bloody \hr{Fall of Kezerad}{fall of \Kezerad}.
  \item She has suffered terrible things in her life (rape and torture) which have made her traumatized and warped, so she sees everything through her own gradually more twisted mind. 
\end{itemize}

\lyricsbs{\Duana}{
  \href{http://www.necrobabes.org/duana/chrysallis.html}{Chrysalis}
}{
  In the dimension of a dream \\
  she floats through walls \\
  and empty places \\
  where hungry voids \\
  like mouths \\
  swallow, chew and spit her \\
  fragments to the solar winds \\
  forked tongues \\
  speak to her in flavors \\
  and whispered halos \\
  with breath as soft as \\
  pussy \\
  willows
  
  she dwells in black holes of \\
  non-existence \\
  on the edge of the \\
  event horizon \\
  slow-dancing with death \\
  and pressing his body close \\
  to fill her hungry void \\
  silent syllables of laughter \\
  ripple from her rose-petaled lips \\
  like ether and watery kisses \\
  like sewers and the plague
}

She begins to hate her visions, seeing them as a curse rather than a blessing.

\citelimbonicart{AdNoctum}{The Supreme Sacrifice}{
  Thought are tyrants that always return\\
  to rape and torment the heart.\\
  As darkness sweeps the face of Earth,\\
  I enter the chambers of bleeding art.
  
  I receive a black picture of the future.\\
  Shockwaves attack from a nihilistic universe.\\
  Another icon shattered, drowned in gory failure.\\
  The supreme sacrifice done in hatred's curse.
}

\citemovie{StarWar:BackstrokeoftheWest}{%
  Star War: The Backstroke of the West%
}{
  [Yoda:]\\
  Pregnancy? Pregnancy?
}





\subsubsection{Silqua blames herself}
\target{Silqua blames herself}
She blames herself for having founded the \hs{Vaimons}, who are now being corrupted into a force of evil, secretly run by the Cabal. 
She fears that all idealism is doomed and that all life will be corrupted.

\citelimbonicart{AdNoctum}{The Supreme Sacrifice}{
  When you seek the dawn of light,\\
  from the cold dungeons of night,\\
  the world is caught in a spell,\\
  where dreams have become Hell.
}

At last, she hopes to die and take everything with her. 
She prays that it will die with her.

\citelimbonicart{AdNoctum}{The Supreme Sacrifice}{
  No anxiety, no pain,\\
  just everlasting sleep.
  
  Psycological autopsy.\\
  Spirally depression.\\
  Darkness takes its prey.\\
  Psycological autopsy.\\
  Spirally depression.\\
  End the life and earthly mission.
}









\subsection{Silqua dies}
\target{Silqua dies}
\hs{Silqua and Delphine were lovers}, but \hr{Curse}{\NexagglachelsCurse} fucked them up. 
Eventually \Delphine{} was driven to kill Silqua and \hr{soul-eating}{eat her soul}. 

Silqua submitted completely and gave herself up willingly. 
She loved \Delphine{} and would do anything to make her happy. 
So she said: 
\ta{Kill me. And devour my soul. That way we will always be together.}

(\hr{Shroud prevents soul destruction}{It would not have been possible} for \Delphine{} to eat Silqua's sould without permission. She would have to be more powerful.)

But devouring Silqua brought \Delphine{} no happiness. 
It only added to her tragedy and quickened her descent into madness. 





\subsubsection{Holy place where Silqua died}
\target{Silqua died at Yeshimon}
The site of \Delphine's palace, Castle \hs{Yeshimon}, later became a holy site since it was the place where Silqua died. 
(It came to be \hr{Zether control Yeshimon}{controlled by \ClanZether}.)

Later Vaimons believed that Silqua had valiantly offered herself up to be tortured to death in order to save her people and defeat the \hr{Delphine is Sarun}{evil \Sarun sorceress} \Delphine. 
In reality Silqua let \Delphine torture her in a sexual orgy because Silqua was a self-destructive masochist. 









\subsection{Silqua's afterlife}
\target{Silqua as a ghost}
At some point, Silqua dies a martyr, killed by her enemies.

After her death, \hr{Monuments keeping the dead imprisoned}{her soul lingers on as a powerless, suffering ghost}, mourning how what she thought to be her work has been taken from her and corrupted into something dark.

\citebandsong{Nightwish:AngelsFallFirst}{Nightwish}{Astral Romance}{
  Departed by the guilloutine of death, \\
  I received a letter from the dead...
}

She had hoped that her death would help and somehow set things right. 
But it didn't. 

\citelimbonicart{AdNoctum}{The Supreme Sacrifice}{
  As I found peace in death's challenge\\
  the world remained in a rotten stench.
}

In fact, she is doomed to stand by and watch as the Vaimons discover more hideous (to her) knowledge and grow in power and wickedness. 
Now she suffers in a Hell of her own making. 
Compare to the RPG \emph{Kult}.

\citelimbonicart{AdNoctum}{The Supreme Sacrifice}{
  I stand within the flame,
  watch the wisdom be discovered.
  When life and death is the same,
  I am devoured.
}

But she still loves Cordos. 

\citebandsong{Nightwish:AngelsFallFirst}{Nightwish}{Astral Romance}{
  Above the universe,\\
  beneath the Great Eye,\\
  I shall desire you forevermore.
}

\citebandsong{Rage:EndofAllDays}{Rage}{Fading Hours}{
  Why didn't you answer?\\
  You looked through me, it seemed.\\
  It all's changed but I can't swear,\\
  the whole scene's so unreal.
  
  I'm still an unbeliever, won't leave you, still live here.\\
  There must have been a reason for all, she said.
  
  Fading hours of plesure and pain,\\
  trust me now, it wasn't in vain.\\
  In this life some things will remain\\
  from fading hours.
  
  Fading hours, but my time stood still.\\
  My missing shadow can't explain.\\
  But the pictures on your table will,\\
  they're showing me pale white,\\
  the coffin's opened wide...\\
  I've forgotten fading hours.
}





\subsubsection{Move this to \Delphine}
If \hr{Silqua dies}{\Delphine{} devoured Silqua's soul}, then Silqua had no afterlife. 
She was simply \emph{gone}. 

If that was the case, then it was \Delphine{} who, in her new grief, started to see the evil that the Vaimon order and the \VaimonCaliphate were being turned into. 
She mourned even as the Vaimons raped, tortured and killed her. 









\subsection{Lestor rapes \Delphine}
\target{Lestor rapes Delphine}
This whole scene below is portrayed from \ps{\Delphine} point of view. 

\Delphine{} was responsible for Silqua's death. 
Cordos Vaimon sent Lestor \Delaen out to capture her alive and bring her justice. 
Lestor brought her alive, but not unharmed. 

Lestor's men defeated \ps{\Delphine}{} soldiers and broke into her sanctum. 
They barged into her room. 
She was outnumbered. 
But she held her head up high. 
She was wracked with guilt and sorrow, but she was not about to show it in the presence of these people, and she had never liked Lestor in the first place. 
So she met him defiantly. 

Lestor ordered his men to hold her suppressed so she could cast no magic. 

Lestor (to his men, with his eyes meeting \ps{\Delphine}): 
\ta{Do you have her suppressed?}

Vaimon soldiers: \ta{Yes, sir.}

Lestor went up to \Delphine{} and bitchslapped her across the face, fucking hard. 
She knew Lestor was rough, but she was unprepared for this direct brutality. 
She fell over. 

Lestor: 
\ta{What? 
  Are you going to crawl on the floor before me like the dog you are? 
  [\hr{Shiaraid's dogs}{\Shiaraid{} has always liked dogs}, remember.]
  Or will you stand and face me?}

\Delphine{} climbed to her feet. 

Lestor struck her again. 
This time she was more prepared. 
She staggered but remained standing. 

Lestor told \Delphine{} to kneel. 
She obeyed, and remained kneeling through their discussion. 
But she sat up as straight as she could. 
As proud and unbowed as any queen. 
Or so she hoped. 

\Delphine: 
\ta{Does it really give you such pleasure, beating a defenseless woman?
  You are no better than I, Lestor \Delaen. 
  This childish display of power proves it.
  So do not play holy and righteous with me.}

They had a longer discussion about who was right and wrong, good and evil. 
Compare to the discussion between Melisande Shahrizai and Ysandre de la Courcel in \cite[p.622]{JacquelineCarey:KushielsAvatar}. 

At last Lestor gave up: 
\ta{Maybe you are right, \Delphine. 
  Maybe I am as bad as you, or worse.
  Maybe we are both animals.
  All right, you win.}

She smiled. 
She felt happy inside. 
She had won a small victory. 

He looked at her with a vicious, twisted, hungry look in his eyes. 
She became uneasy, began to fidget. 
\tho{What is wrong with him? Why is he looking at me like that?}
He sat there for a long moment. 
She became afraid. 

Lestor: 
\ta{In fact, let me show you how right you are.
  Let me show you what an animal I am.}

He then had her stripped naked and tied to her bed. 
Then he raped and tortured her. 

\ta{You tortured Silqua and enjoyed it. 
  I intend to do the same to you.}


Lestor raped her multiple times, and let his men do the same. 
She was badly broken when he finally brought her to Cordos. 

\target{Delphine's guilt}
\Delphine{} did not fight back as much as she could have done. 
She blamed herself for killing Silqua. 
She knew that was wrong. 
She did not want to admit it in front of Lestor-tachi, so she put up a proud and defiant front (brought on, again, in part by her pride and self-destructive perversion), but inside she hated the monster she had become. 
So she was almost glad to be brutally raped, tortured, humiliated and ultimately killed (in a slow, painful and humiliating manner). 
\tho{It is no more than I deserve.} 










\subsection{Legacy of Silqua and \Delphine}
It became known that Silqua willingly let \Delphine{} kill her. 
This got posthumously interpreted as a divine sacrifice. 
Compare to the fake sacrifice of Serena Butler in \cite{BrianHerbert:LegendsofDune}. 
But in reality Silqua did it for \Delphine{} and out of her own weakness and sexual desire, not out of any large-scale moral (let alone religious) motivation. 

Silqua became mythed up to be an incarnation of \iquin{} on \Miith{}. 
No one knew what a Scion was back then.
Later, when it was discovered that Scions existed, Iquinians still believed Silqua was no Scion but something greater. 
(And that even though \Aryal{} was a measly \thelyad.) 

\Delphine{} got the status of an evil manifestation of \itzach, the antithesis of Silqua and a terrible Dark Lord. 





\subsubsection{Vaimons become scared of sex}
Silqua died under perverted circumstances, killed by the sexy \Delphine. 
And then Lestor \Delaen brutally raped \Delphine{} in retaliation. 
These nasty sex acts scared and scarred the Vaimons (Cordos, Arcan and others). 

They developed traumata surrounding sexual liberated-ness. 
As the years and generations went by, this only became worse. 
The \VaimonCaliphate became sexually restrictive. 















\section{\Caliphate (\yds{VC})}
Silqua dies. 

Cordos Vaimon continues his conquest and eventually comes to rule a great empire. This becomes the \VaimonCaliphate. 









\subsection{The \vclans}
The \hr{Vaimon clan}{\VaimonClans} were founded here. 
























\chapter{Vizicar's Time}
These are stories that take place in and around the life time of \hr{Vizicar}{\VizicarDurasRespina}. 















\section{Queen of the Caverns}









\subsection{The Hills of Uldor}
In Vizicar's time there was a region called the hills of \hs{Uldor}. 
These hills were ill-reputed and feared in the region around them, for they were home to tribes of marauding and abominable half-men. 









\subsection{Vizicar comes to Uldor}
\VizicarDurasRespina ended up in the hills of Uldor.





\subsubsection{War}
Vizicar was at war, fighting to subjugate a rival faction of Vaimons in the \caliphate, that of Shah \hs{Xerxes}. 
The day before he was sitting in his royal hall listening to tales about the dangerous tribesmen in the region, but did not pay them much heed. 
The next day he was on the battlefield, leading an \ishrah of Vaimon warriors. 
Things went wrong on the battlefield and Vizicar and a few retainers were forced to seek refuge in the \wylde woods on the hills of Uldor. 

\ta{The shah's scouts will spot us any moment.}
The speaker was \VizicarDurasRespina, \vaimoncaliph. 
They had hoped to strike at the heart of the shah's forces, but they had been surprised and forced into battle. 
Their force had been splintered, and now Vizicar only had two warriors attending him.
They could not hope to fight the shah's warriors.
They must slip past their pursuers and rejoin Vizicar's army. 

\ta{We cannot stay here and risk discovery. 
  We will go through the \wylde.}

\ta{But, Blessed One! Those are the hills of Uldor!}

\ta{We must risk them. Come!}
Vizicar rode into the woods.

Vizicar soon felt how eerie and oppressive the woods were.
Full of sinister voices and furtive shapes that lurked and crept at the edge of his vision. 
He recalled the stories told about the hills. 
He did not know how much to believe. 
He knew that tales were often distorted and exaggerated in the telling, but he also knew that deadly and horrid things dwelt in the \wylde. 

Here they were haunted by the eerie sounds of the forest and the rustling and slinking savages.
Many times Vizicar half-saw a \human-like shape slink by in the shadows, and thought he felt small glittering eyes peering evilly at him from behind the thick ivies. 





\subsubsection{Savages attack}
Suddenly the silence was broken. 
Arrows flew out of the woods. 
(Steal some prose from the discarded chapters with Carzain and Curwen in the woods.)
They could not see the attackers.
Vizicar used his sorcery to scare the attackers away.
It worked for a while, but they always came back.
Soon Galba's and Lieron's \relcs were dead.
(Maybe all three \relcs were dead.)
Then Galba and Lieron themselves were hit. 
Poisoned darts. 
They fell unconscious.
Then the savages attacked.
They held Vizicar at bay while their fellows snatched his two half-unconscious followers and dragged them away. 

The savages were naked but wore barbaric fetishes: 
Bones, ears, fingers and phalli of slain enemies. 

Vizicar killed some of his foes and the rest fled. 
He wanted to pursue.
He knew it was smarter to flee. 
He was \caliph, and his life was far more valuable than that of either soldier.
He should not endanger himself. 
But his anger burned.
As Ramiel he had a certain self-destructive urge to prove himself.
He would not flee.
His pride would not let him abandon his fellows. 
So he went in after his companion. 

He snuck after the fleeing savages, taking care not to be seen.

He came just in time to see the savages drag Galba down into a tunnel. 
(Maybe they sealed the entrance.)
When the coast was clear he went in after them. 






\subsubsection{Into the tunnels}
Vizicar's search led him into the dark tunnels beneath the earth. 

\target{Vizicar finds Uruthar totem}
In the hills and by the entrances to the tunnels Vizicar noticed some great and ancient totems, carved with reptilian motifs and throbbing with sorcerous power. 
He drew fearfully back from these grim and imposing totems that stood as if raised by \dragons in a bygone age.
Which they were.
They were the totems erected by the \ophidians to \hr{Totems against Uruthar}{forever imprison the \resphain of Uruthar beneath the earth}. 
Vizicar discovered that a small stave of such a totem pole had broken off and fallen to the ground.
He picked it up and examined it with revulsion mixed with fascination.
Then the hillmen came upon him and he had to flee. 
He took the broken stave with him.
Later, when he gained a respite from his enemies, he tucked it into his deep pocket. 





\subsubsection{Underground city}
Once in the tunnels, Vizicar discovered that the realm of the hillmen extended far below the hills.
There was an entire kingdom beneath those hills, carved into the earth and the rock. 
There were hundreds of the half-men living down here, but he could see that the subterranean city was far greater than that and could house tens of thousands if not millions. 
It was too vast and too splendid for those savages to have built it on their own.
Either they had fallen from a higher level of civilization, or they were but the servitors of a greater ruling power, or they had moved into a city built by an elder race. 









\subsection{\Lethiarch}
Vizicar encountered \hr{Lethiarch}{\Lethiarch}. 

\hr{Lethiarch wants sex}{\Lethiarch wanted sex}. 
Vizicar was the first really promising lover that had come into the the underground city since \Lethiarch slew her last fellow \resphan.
She was enamoured with him.





\subsubsection{\Lethiarch's story}
\Lethiarch told Vizicar her story, and that of the city.
\hs{Uldor} had once been \hs{Uruthar}, a mighty and eldritch kingdom ruled by \resphan gods. 
These gods had \hr{Uruthar Resphain cast out}{chosen to dwell apart from their race} and instead \hr{Uruthar founded}{founded this kingdom} where they ruled over its people.
But the kingdom was defeated in war and driven underground, where \hr{Totems against Uruthar}{they were imprisoned by powerful totems}. 
Here its gods turned upon each other and slew one another.
Finally \Lethiarch \hr{Osra dies}{slew her last fellow \resphan} and was alone. 

She also told him, in vague and general terms, about the history of her race and his.
She told him that her race was the creators from whom \humanity was descended, and in whose image \humanity was shaped. 
She told him something about how she needed food and energy in order to keep her calm and maintain control of her own mind and body, lest her \emph{true nature} break its bonds and overcome her. 





\subsubsection{Guided tour}
\Lethiarch led Vizicar around in the city while she talked to him.
He saw the \hr{Shunned tunnels in Uruthar}{entrances to the deeper tunnels}, and noticed that everyone shunned them.

Do not explain to the reader why these tunnels are shunned.
Merely let Vizicar guess what even more dreadful horrors might lurk down there in the deep.





\subsubsection{Worship}
Maybe Vizicar saw \Lethiarch worship the \SitraAchras. 
She told Vizicar that she had once turned away from the \SitraAchras and tried to deny them, but as she grew weaker she realized that she could not deny her true nature, for no living being can escape its legacy. 
She turned back to worshipping them, for she knew that she was bound to them forever, just as her people were bound to her. 

When \Lethiarch cast her spells it was as if the city awakened.
Vizicar could suddenly imagine the city in its heyday, and it was almost as if he could see it before his waking eyes.
Full of people. 
These people were abnormal and strange of face and form, but still men\dash the men of a vanished age.
A greater age. 
A more monstrous and monumental age, when colossal inhuman forces ruled the world and mortals trembled at their footsteps. 
Before the Vaimons, before the \archons, before mortals found their own power and mastered sorcery; back when mortals were wholly and fully at the mercy of foreign and superhuman forces. 

\citeauthorbook[p.278]{RobertEHoward:TheValleyoftheLost}{Robert E. Howard}{%
  The Valley of the Lost%
}{
  The drums rustled, the strange light leaped and shimmered, and before the altar came one who seemed in authority\dash an ancient monstrosity whose skin was like the whitish hide of an old serpent, and who wore on his peaked skull a golden circlet, set with weird gems.
  He bent and made suppliance to the feathered snake.
  Then with a sharp implement of some sort which left a phosphorescent mark, he drew a cryptic triangular figure on the floor before the altar, and in the figure he strewed some sort of glimmering dust.
  From it reared up a thin spiral which grew to a gigantic shadowy serpent, feathered and horrific, and then changed and faded and became a cloud of greening smoke.
  This smoke billowed out before John Reynolds' eyes and hid the serpent-eyed ring, and the altar, and the cavern itself.
  All the universe dissolved into the green smoke, in which titanic scenes and alien landscapes rose and shifted and faded, and monstrous shaped lumbered and leered.
  
  Abruptly the chaos crystallized.
  He was looking into a valley which he did not recognize.
  Somehow he knew it was Lost Valley, but in it towered a gigantic city of dully gleaming stone.
  John Reynolds was a man of the outlands and the waste places.
  He had never seen the great cities of the world.
  But he knew that nowhere in the world today such a city reared up to the sky.
  
  Its towers and battlements were those of an alien age.
  Its outline baffled his gaze with its unnatural aspects; it was a city of lunacy to the normal human eye, with its hints of alien dimensions and abnormal principles of architecture.
  Through it moved strange figures, human ,yet of a humanity definitely different from his own.
  They were clad in robes, their hands and feet were less abnormal, their eats and mouths more like those of normal humans, yet there was an undoubted kinship between them and the monsters of the cavern.
  It showed in the curious peaked skull, though this was less pronounced and bestial in the people of the city. 
  
  He saw them in the twisting streets, and in their colossal buildings, and he shuddered at the inhumanness of their lives.
  Much they did was beyond his ken; he could understand their actions and motives no more than a Zulu savage might understand the events of modern London.
  But he did understand that these people were very ancient and very evil.
  He saw then enact rituals that froze his blood with horror, obscenities and blasphemies beyond his understanding.
  He grew sick with a sensation of pollution, of contamination. 
  Somehow he knew that this city was the remnant of an outworn age\dash that this people represented the survival of an epoch lost and forgotten.
}





\subsubsection{They have sex}
\Lethiarch began to seduce Vizicar in earnest.
He realized that she was pushing for sex. 

Vizicar was not sure about having sex with this half-insane sorceress that dwells beneath the earth as the goddess of the deformed savages, but he knew she was very powerful.
And she was really sexy.
He decided that if she wanted him, it would be best to take her and enjoy it.

When Vizicar gave her a number of orgasms, \Lethiarch involuntarily let her self-control slip.
Vizicar saw a glimpse of \hr{Lethiarch's body decays}{her body decaying}.
For a brief instant her face faded to a blank, smooth, inhuman mask.
And for a brief instant her skin felt like sticky, foetid slime. 

She quickly covered herself and regained her composure.
Soon her body was all lovely and sexy again.
But Vizicar was not reassured. 
He pulled away in horror and loathing.

Vizicar did not want to fuck a woman who could turn into a thing of slime at any moment.
And her face! 
That faceless mask had been too horrible to behold.
It awakened some primal fears in his head, as if from some dim abyss of ancient ancestral memory that he shivered to recall. 
There lay some deep meaning in that.
It was the primal fear of some ancient horror that this race had feared since the dawn of time.
This he felt, but he could not explain it.
(Read about \hs{ancestral memory}.)

Moreover, he remembered what she had told him before about her \quo{true nature} and how it might break loose and overpower her.
If this foetid and loathsome thing was her true nature, then he wanted nothing to do with her.
Suddenly he wanted to get away from this awful sorceress, the faster the better. 





\subsubsection{Vizicar fights back}
\Lethiarch said some words.
She cursed the \ophidian people who drove her kind underground and reduced her to this degrading state. 
She cursed their vile spells and totems.

Then Vizicar remembered something. 
The totem! 
That reptilian totem \hr{Vizicar finds Uruthar totem}{he had found in the woods}.
And the small stave that he still carried on him.
He reached for his clothes and found the stave.
He also picked up his sword.

Vizicar took a gamble.
He knew \Lethiarch was weak in her current condition, and right now she was distraught and distracted. 
If he was ever to have a chance against her, it was now. 
The stave was in two pieces. 
He put the two pieces together and brandish the stave in front of \Lethiarch.

It worked.
When put together as one the stave regained some fraction of its power. 
\Lethiarch recoiled, first in horror.
Then her horror turned to rage. 
Vizicar knew he had to act fast. 
He quickly called upon his \archons and blasted her with magic. 
She fell back again, howling in pain.
Vizicar lashed out against her again, then quickly turned tail and ran. 
Behind him \Lethiarch rose to her feet.

\Lethiarch:
\ta{You ungrateful, treacherous swine! 
  Do you think that twig can protect you from me?}
She cast a spell, and the totem stave shattered in his hand.
He winced at the splinters of wood, for he was still naked. 
He tossed the fragments of the stave away and ran on. 
He encountered several of the hillmen, but they were completely unprepared for his savage and furious attack, and he tore them to pieces. 

Behind him he could hear \Lethiarch's curses. 
She came at him. 
He glanced over his shoulder and saw her.
She was mutating.
A massive shadow was taking form around her. 
She was shedding her \human skin and melting into a faceless, many-limbed thing of glistening black slime.
Her \facade of angelic beauty and refined sophistication gave way, and Vizicar gazed upon the raw essence of cosmic lunacy and nightmare taken physical form. 









\subsection{Vizicar flees}
In the end Vizicar managed to wound \Lethiarch, slip away, kill the guards and escape from caves and back into the open. 





\subsubsection{Vizicar considers the implications}
Vizicar was critical of her story. 
Even though she might be a superhuman sorceress, he could tell that she was not quite sane.
Besides, he noticed that she was always the hero of the story and never did anything wrong. 
He suspected she was twisting the story a lot.
But he still shuddered.
What if what she was telling him the truth?
What if it was true that \Lethiarch was the least evil of her race?
The race that made mankind?
What abhorrent gods might then still dwell in the world, and what black and hideous taint of ancient evil might mankind still carry in their very veins? 

Vizicar was shaken. 
Was her race really the gods in whose image mankind was created?
And was this black and gibbering horror really her true nature?
If so, was it also mankind's nature?

He had seen the creeping wormlike things that once were men, so he knew what depths mankind had the potential to sink to. 
Was this their true nature? 
Were these really the depths of bestiality from which his ancestors crawled, and to which they are destined to revert? 
What was it she had said? 
That no living thing could escape its legacy. 

And there was one riddle he never understood.
What were these things that \Lethiarch described as the ultimate cosmic horror and evil? 
Who, or what, were the \SitraAchra?

Vizicar did not want to dwell any further on this.
He longed to get back to the battlefield where men fought and died honestly, face-to-face in open combat under the Sun. 
Naked, he stumbled out of the \wylde to rejoin his army. 
To drown himself in slaughter. 
To fight and kill and forget. 

\citeauthorbook[p.287--288]{RobertEHoward:TheValleyoftheLost}{Robert E. Howard}{%
  The Valley of the Lost%
}{
  John Reynolds walked slowly away, and suddenly the whole horror swept upon him and the earth seemed hideously alive under his feet, the sun foul and blasphemous over his head.
  The light was sickly, yellowish and evil, and all things were polluted by the unholy knowledge locked in his skull, like hidden drums beating ceaselessly in the blackness beneath the hills.
  
  He had closed on Door forever but what other nightmare shapes might lurk in hidden places and dark pits of the earth, gloating over the souls of men?
  His knowledge was a reeking blasphemy which would never let him rest, for ever in his soul would whisper the drums that throbbed in those dark pits where lurked demons that had once been men. 
  He had looked on ultimate foulness, and his knowledge was a taint because of which he could never stand clean before men again or touch the flesh of any living thing without a shudder.
  If man, moulded of divinity, could sink to suck verminous obscenities, who could contemplate his eventual destiny unshaken?
  And if such beings as the Old People existed, what other horrors might not lurk beneath the visible surface of the universe?
  He was suddenly aware that he had glimpsed the grinning skull beneath the mask of life and that that glimpse made life intolerable.
  All certainty and stability had been swept away, leaving a mad welter of lunacy, nightmare and stalking horror.
  
  John Reynolds drew his gun and his horny thumb drew back the heavy hammer.
  Thrusting the muzzle against his temple, he pulled the trigger.
}















\section{The Ape-Gods}
\VizicarDurasRespina was waging war against his nemesis, Shah \hs{Xerxes}.

The shah had turned away from the worship of the \sephiroth because he had discovered the ancient ape-gods of \hs{Ibthek}, the bestial and barbaric pre-\human \aryoth gods of the \nephilim.

Xerxes had found the ruins of ancient Ibthek. 
Vizicar saw them and was horrified.

Vizicar had won a battle against Xerxes and conquered one of his cities.
He took two of Xerxes' young daughters captive:
Kes and Tiri, both virgins.
Vizicar took them as concubine slaves.
He took both their virginities. 

Tiri was submissive and quickly became Vizicar's obedient slave, but Kes was willful and fought back. 
She refused to submit, and Vizicar had to violently rape her. 
Vizicar tried to have Kes whipped, but it only made her resist him all the more.
He also tried whipping Tiri for Kes' misbehaviour.
But he felt sorry for poor Tiri who was being all sweet and obedient and innocent, so he stopped. 

Finally Kes escaped. 
It turned out that she was a sorceress. 
She unleashed much deadly magic.

Tiri told Vizicar that she had always been afraid of Kes.
She knew that Kes was a sorceress, but Kes had threatened her not to tell anyone.
Tiri also knew a little about Ibthek, and it scared her senseless. 

Vizicar followed her.
She led him to Ibthek. 
Here Vizicar fought ancient horrors and was forced to flee.

Vizicar saw the statues of Ibthek, and its undead charnel gods.

\citeauthorbook[p.13]{RobertBloch:TheSuicideintheStudy}{Robert Bloch}{%
  The Suicide in the Study%
}{%
  Out of the darkness nightmare came; start, staring nightmare\dash a monstrous, hairy figure; huge, grotesque, simian\dash a hideous travesty of all things human.
  It was black madness; slavering, mocking madness with little red eyes of wisdom old and evil; leeround snout and yellow fangs of grimacing death.
  It was like a rotting, living skull upon the body of a black ape.
  It was grisly and wicked, troglodytic and wise. 
  
  A monstrous thought assailed Allington.
  Was \emph{this} his other self\dash this ghoul-spawned, charnel horror of corpse-accursed dread?
}

He even encountered the bloody god \hr{Gorgomon}{\Gorgomon}. 
The god spoke to him in its primal, barbaric language:
\ta{\tongue{Golza ud ua romo \Gorgomon, blachru nod thrum \aryoth-wa rurum!}}
Vizicar guessed that it meant:
\hypota{Aye, I am the bloody \aryoth god \Gorgomon!}

Kes returned to her father and became a great and powerful sorceress. 

Perhaps Kes cast a curse on poor Tiri for betraying her secrets, and Tiri died a horrible death, slowly transforming into an inhuman monstrosity. 





















\chapter{\HundredScourgesBook}
The \Darkfall, the destruction of the \VaimonCaliphate, begins with \Belzir{} daughter of \Cormin, a young Vaimon girl of \ClanGeican. As a girl, \Belzir{} is a nerd, an skinny, unattractive girl who fails to attract the attentions of boys and thus devotes her life to science and magic. She is a genius, with great talent for magic, especially \nieur{}, and already at a young age she discovers several new \qliphoth and new techniques to channel \iquin{} and \nieur{}, making her very powerful. 

At some point, \Belzir{} is chosen to become \Calipha. So she is schooled in all the things that an \Calipha must know, including all sorts of social relations. Despite the shortcomings she used to display in these areas, she learns quickly and becomes an adept communicator and a sly manipulator. But she still has problems attracting men, and the men she does attract seek her out for her position or her power. This is not enough for \Belzir; she wants them to desire her body. So she does a lot of research, and eventually manages to magically and permanently change her body, sculpting herself into the most beautiful and desirable woman she can imagine. 

What is \Belzir's relationship with the Cabal? Is she affiliated with it? Maybe the Cabal manipulates her into destroying the \caliphate because the Iquinian Church is a threat to them... 















\section{The State of the \Caliphate}









\subsection[Scatha rebellion]{\Scatha rebellion}
\target{Scatha rebellion under Belzir}
The \VaimonCaliphate \hr{Vaimon Caliphate oppressed Scathae}{oppressed \scathae}. 
In \Belzir's time there was a \scathaese (\Ortaican) uprising that had been brewing for centuries. 
Everything \Belzir did interacted with this rebellion. 
The rebellion was as important as \Belzir, if not more so, in bringing about the end of the \caliphate. 
But afterwards \Belzir was demonized because the surviving \VaimonClans wanted a scapegoat. 









\subsection{Technology}
\target{Vaimon technology}
At this point, technology is high. 
Renaissance level. 
Think \quo{Pirate Age} or \quo{Musketeer Age}, or even \quo{Zorro Age}. 





\subsubsection{Flying machines}
They even have magical flying machines. 





\subsubsection{Weapons}
\target{Vaimon guns}
Stories tell of how mighty and glorious the Vaimons were with their magic, but they were actually more vulnerable back then than they are in Carzain's day. 

See, the technology was higher back then. 
There were guns, which could quite easily snipe off a Vaimon from a distance. 
(Protective measures like \hr{Archon Ward}{Archon Wards} were extremely expensive and rare.) 

After the \HundredScourges, knowledge of magic (seen as the highest and noblest of all sciences) was mostly preserved, but all other technology suffered. 
This had the side effect of making Vaimons more effective compared to non-mages. 









\subsection{\VaimonClans}
\target{Vaimon Clans at Belzir's time}
At \Belzir's time there were four \VaimonClans:
Geican, \Zether, \Delaen and Redcor. 















\section{\Belzir{} and Her Life (\yds{Belzir birth})}





\subsection{\Belzir sexually haunted}
As a young teenage girl (of legal age), \Belzir was haunted by dark shades and monsters. 
They abused her sexually. 
They might be ghosts from her own \carcer. 

\citeauthorbook[p.158]{RobertEHoward:BlackColossus}{Robert E. Howard}{%
  Black Colossus%
}{
  But princess Yasmela lolled not on that silken bed. She lay naked on her supple belly upon the bare marble like the most abased suppliant, her dark hair streaming over her white shoulders, her slender fingers intertwined. She lay and writhed in pure horror that froze the blood in her lithe limbs and dilated her beautiful eyes, that pricked the roots of her dark hair and made goose-flesh rise along her supple spine.
  
  Above her, in the darkest corner of the marble chamber, lurked a vast shapeless shadow. It was no living thing of form or flesh and blood. It was a clot of darkness, a blur in the sight, a monstrous night-born incubus that might have been deemed a figment of a sleep-drugged brain, but for the points of blazing yellow fire that glimmered like two eyes from the blackness.
  
  Moreover, a voice issued from it\dash a low subtle inhuman sibilance that was more like the soft abominable hissing of a serpent than anything else, and that apparently could not emanate from anything with human lips. Its sound as well as its import filled Yasmela with a shuddering horror so intolerable that she writhed and twisted her slender body as if beneath a lash, as though to rid her mind of its insinuating vileness by physical contortion.
  
  \ldots 
  
  The ghostly hissing dwindled off in lustful titterings, and Yasmela moaned and beat the marble tiles with her small fists in her ecstasy of terror.
  
  \ldots
  
  \ta{But thou shalt be my queen, oh princess! I will teach thee the ancient forgotten ways of pleasure. We\dash} 
  Before the stream of cosmic obscenity which poured from the shadowy colossus, Yasmela cringed and writhed as if from a whip that flayed her dainty bare flesh.
   
  \ta{Remember!} whispered the horror. 
  \ta{The days will not be many before I come to claim mine own!}
   
  Yasmela, pressing her face against the tiles and stopping her pink ears with her dainty fingers, yet seemed to hear a strange sweeping noise, like the beat of bat wings. Then, looking fearfully up, she saw only the moon that shone through the window with a beam that rested like a silver sword across the spot where the phantom had lurked. Trembling in every limb, she rose and staggered to a satin couch, where she threw herself down, weeping hysterically.
}





\subsubsection{Angst}
The young \Belzir angsted over being haunted by the dead souls of her \carcer. 

\citebandsong{BlindGuardian:IFTOS}{Blind Guardian}{I'm Alive}{
  Was it really me I saw in the mirror screaming? \\
  I swallowed hate and lies through a thousand cries\\
  Someone's sucking out my energy
  
  \ldots

  We're not alone, there's someone else, too\\
  From the mirror's other side\\
  Reflecting the cruel part of your soul\\
  It's time for your choice
  
  What can I do on this road to nowhere?\\
  Heart of dragon lies\\
  At the edge of time
  
  And the story ends\\
  Insanity said coldly\\
  Still waiting for the chance\\
  So out of nowhere it will rise\\
  Oh, and another journey starts\\
  By the call of the moon
}









\subsection{\Belzir disgruntled}
\Belzir{} was not always pleased with her position as \Calipha and people's expectations of her. 

She tried hard all her life to reconcile herself with the Iquinian religion, but it always felt distasteful to her. 

\citebandsong{Ihsahn:TheAdversary}{Ihsahn}{Homecoming}{
  I have tried so hard\\
  to regain my faith in man.\\
  Yet, I fail again\\
  to uphold this deception.\\
  His collective truths are far too salt.
}

Instead she found herself drawn towards the darkness, the unknown, the scary and exciting. 

\citebandsong{Ihsahn:TheAdversary}{Ihsahn}{Homecoming}{
  I return instead \\
  to the Heart and Self\\
  and say:\\
  \quo{Unutterable and nameless\\
    is that which maketh \\
    my soul's pain and sweetness,\\
    and which my bowels yearn upon.}
}

When she realized that the Iquinian religion was the Cabal's work, she hated the \resphain{} for having resurrected a version of the hated \Merkyran{} faith, which they had fought so hard to overthrow. 





\subsubsection{Flees from her castle}
She flees from her castle at night and explores the countryside and her magical skills. 

\citebandsong{CradleofFilth:CrueltyandtheBeast}{
  Cradle of Filth
}{
  Thirteen Autumns and a Widow
}{%
  So with windows flung wide to the menstrual sky,
  Solstice Eve She fled the castle in secret.\\
  A daughter of the storm, astride Her favourite nightmare,\\
  On winds without prayer.\\
  Stigmata still wept between Her legs,\\
  a cold bloodedness which impressed new hatreds.
}

She meets dark sorcerers and mystic thingies. 
They turn out to be Sentinels working to seduce her. 

\citebandsong{CradleofFilth:CrueltyandtheBeast}{Cradle of Filth}{Thirteen Autumns and a Widow}{
  She sought the Sorceress\\
  through the snow and dank woods to the sodomite's lair.
}

She returns bearing terrible insights. 

\citebandsong{CradleofFilth:CrueltyandtheBeast}{Cradle of Filth}{Thirteen Autumns and a Widow}{
  And under lacerations of dawn She returned,\\
  like a flame unto a deathshead with a promise to burn.\\
  Secrets brooded as She rode\\
  through mist and marsh to where they showed\\
  her castle walls wherein the restless\\
  counted carrion crows.
}

She has nightly escapades and adventures. 

\citebandsong{CradleofFilth:CrueltyandtheBeast}{Cradle of Filth}{The Twisted Nails of Faith}{
  To the Sorceress and Her charnel arts\\
  she swept from ebon towers at the hour of Mars.\\
  'Neath a star-inwoven sky latticed by scars,\\
  to unbind knotted reins that kept in canter, despair,\\
  shod on melancholy, fleet to sanctuary there,\\
  in netherglades tethered where onyx idols stared.
}









\subsection{She learns vampirism and regains some power}
\Belzir{} re-learns the skill of absorbing souls. 
This helps her regain a portion of her \sathariah{} powers and even some of her memory. 





\subsubsection{Vampirism makes her beautiful}
She becomes extra beautiful when she drinks blood. 
This is because her \resvil{} self shines through, and \humans{} pick up on it and feel an inbred urge to serve and worship her. 

\citebandsong{CradleofFilth:CrueltyandtheBeast}{Cradle of Filth}{Beneath the howling stars}{
  Restored to jaded bliss
  this evisceratrix\\
  descended to the ball
  with painted blood upon Her lips,\\
  passing like a comet so white
  as to eclipse\\
  the waltz wound down, transfixed.
  
  Devoid of all breath in the air,\\
  even Death paled to compare\\
  to the taint of Her splendour,\\
  so rare and engendered\\
  'pon the awed throng gathered there...
  
  Beneath the howling stars.
  
  She danced so macabre,\\
  Men entranced divined from Her gait,\\
  that this angel stepped from a pedestal\\
  had won remission from fate\\
  by alighting to darker spheres,\\
  delighting in held sway,\\
  for She was not unlike the Goddess\\
  to whom the wolves bayed.
}





\subsubsection{She gains enemies}
But her evil magic gains her enemies. 

\citebandsong{CradleofFilth:CrueltyandtheBeast}{Cradle of Filth}{Beneath the Howling Stars}{
  \quo{%
    Whilst envy glanced daggers
    from court maidens, arboured,\\
    who whispered in sects
    of suspicions abroad,\\
    that Elizabeth bewitched;
    see how even now the whore casts\\
    her spells upon the Black Count
    whom Her reddened lips hold fast.%
  }
}

Her reputation grows bad. 

\citebandsong{MonolithDeathcult:Triumvirate}{Monolith Deathcult}{Wrath of the Ba'ath}{
  Babel, Mother of Harlots and all Abominations of the Earth. \\
  A metropole of incest where the fornicators rule. \\
  Bacchanal orgies drenched in cum undaunted by sodomy. \\
  The greatest whore of the ancient world shivers in ecstasy.
}









\subsection{Her memories slowly return}
\ps{\Belzir}{} memories slowly returned to her. 
She did not develop a split personality like Ramiel would have done. 
She just remembered vague emotional impressions of her past lives. 
She has strange dreams, thoughts and fantasies which she shouldn't have. 
She feels there must be a bigger picture, a meaning behind it all. 

\citebandsong{Emperor:IXEquilibrium}{Emperor}{Of Blindness and Subsequent Seers}{
  Ever behind me\\
  rises a shadow.\\
  Taller than I,\\
  yet with a certain resemblance.
}

She feels she is missing out on some greatness and pleasure that could and should rightly be hers.

\citebandsong{Emperor:IXEquilibrium}{Emperor}{Of Blindness and Subsequent Seers}{
  How many times do I \\
  have to contemplate my own reflection\\
  and say: I have been blind?
  
  I have been blind,\\
  yet I saw the search and dreams\\
  of my rejection\\
  walking behind me.
}

Through the story she comes closer to a revelation of it all. 
But her own fear and religious conditioning hold her back. 

\citebandsong{Emperor:IXEquilibrium}{Emperor}{Of Blindness and Subsequent Seers}{
  Every time I am bound \\
  to have been granted\\
  the gift of better sight.
  
  But my anxiety built one more brick.\\
  Fearing again to choose the wrong step.
}

There is power in her memories, but also weaknesses. 
Fears. 
Vulnerability. 
Even gods have vulnerabilities. 
Realizing this helps her identify with \Shiaraid{} and achieve her \apotheosis. 

\citebandsong{Emperor:IXEquilibrium}{Emperor}{Of Blindness and Subsequent Seers}{
  Vaguely I remember\\
  the blurred eyes of someone small.\\
  These strangers often come as blind.\\
  A troubled mind I left behind.
  
  Yet, was it I or my shadow\\
  walking in the past?
}









\subsection{Caught between two plans}
There was a Cabal plan unfolding around her at court. 
The Cabalists were trying to reawaken \Shiaraid. 
But at the same time she was being corrupted and seduced by the Sentinels. 

When she learns of the Cabal plan she feels betrayed and turns on them. 

When she regains some of her memory she sees the \VaimonCaliphate as a new \Merkyrah, a betrayal of the principles she fought for in the original rebellion. 
As a \Mystraacht, in touch with Chaos, she sees it as her duty to rebel once again against oppression and stupidity. 

Slowly, her half-awakening and her chaotic memories drive her mad. 
The \Malach{} reawakening experiment fails. 





\subsubsection{\Secherdamon and the \taorthae}
\target{Taorthae manipulate Belzir}
\target{Rissit helps set up Belzir}
The war surrounding \Belzir and the ensuing \HundredScourges were partially masterminded by the \taorthae, who were just emerging at this time. 
They had a long-term plan to destroy the Vaimons, and they realized they could employ this rebellious \Calipha to their own ends. 

Chief among the \taorthae who manipulated \Belzir was \Secherdamon.
In fact, he masterminded much of \Belzir's whole life and the fall of the \caliphate. 

Shortly before this, he had \hr{Secherdamon takes the name Nexagglachel}{taken the name \quo{\Nexagglachel}}.

\Nzessuacrith aided \Secherdamon. 
She walked on \Miith in humanoid disguise and infiltrated \Belzir's court. 

\Ortaican mythology would later \hr{Ortaican myths about the Hundred Scourges}{exaggerate their role}. 
As \hr{Rissitic myths about Belzir}{would the Rissitics}. 









\subsection{She has evil dreams}
Just like Carzain and Vizicar, \Belzir{} is haunted by nightmares of her past. 

\citebandsong{CradleofFilth:CrueltyandtheBeast}{Cradle of Filth}{Desire in Violent Overture}{
  Frights came wailing from the Darkside.\\
  Haunting lipless mouths a fugue of arcane diatribes.\\
  Velvet, their voices coffined Her in slumber,\\
  bespattered and appeased.\\
  as pregnant skies outside bore thunder.
  
  How sleep the pure?\\
  Desire in violent overture.
}









\subsection{She likes nature}
\Belzir{}, and \Shiaraid{} in general, has always been fascinated by nature. 
Young \Belzir{} sneaks out of the castle to experience nature. 
Through this she discovers part of her own true nature. 

\citebandsong{Emperor:IXEquilibrium}{Emperor}{Into the Infinity of Thoughts}{
  As the Darkness creeps over the Northern mountains of Norway \\
  and the silence reach the woods, I awake and rise... \\
  Into the night I wander, like many nights before, \\
  and like in my dreams, but centuries ago. \\
  
  Under the Moon, under the trees. \\
  Into the Infinity of Darkness, \\
  beyond the light of a new day, \\
  into the frozen nature chilly, \\
  beyond the warmth of the dying Sun. \\
  Hear the whispering of the wind, \\
  the Shadows calling... 
  
  I gaze into the Moon which grants me visions \\
  these twelve full Moon nights of the year, \\
  and for each night the light of the holy disciples fades away. 
  
  Weaker and weaker, one by one. 
  
  I gaze into the Moon which \\
  makes my mind pure as crystal lakes, \\
  my eyes cold as the darkest winter nights, \\
  but yet there is a flame inside. 
  
  It guides me into the dark \\
  shadows beyond this world, \\
  into the infinity of thoughts... \\
  thoughts of upcoming reality.
}

She dreams of her past. 

\citebandsong{Emperor:IXEquilibrium}{Emperor}{The Burning Shadows of Silence}{
  Into the shadows (so dark) I hear the choirs of evil, \\
  a \quo{joy} in blasphemy beyond my darkest fantasies. \\
  
  The Gate is open... 
  
  Into the silent shadows I crawl, \\
  upon the throne so cold, \\
  atmosphere of melancholy. \\
  I will forever burn...
}









\subsection{She goes mad}
Her \Malach{} memories drive her somewhat mad. 
She begins to have morbid fantasies. 

\citebandsong{Emperor:IXEquilibrium}{Emperor}{Into the Infinity of Thoughts}{
  The lands will grow black. \\
  There is no Sunrise yet to come \\
  into the wastelands of phantoms lost. 
  
  May these moments under the Moon be eternal. \\
  May the infinity haunt me... In Darkness.
}









\subsection{She rebels against religious oppression}
Much of the story is about \ps{\Belzir}{} rebellion in the face of religious oppression and bigotry. 
In a sense, a \quo{light} version of the \Merkyran{} rebellion. 

She becomes disgruntled and disillusioned with the religious Vaimons around her. 

\citebandsong{Emperor:IXEquilibrium}{Emperor}{Decrystallizing Reason}{
  Reason, this demigod unto which you cluster. \\
  Sacrilege. You sacrifice \\
  the purity of the air beneath your wings.\\
  A slow and painful ritual\\
  to burn the youth you lost. 
  
  Demigod!\\
  Blasphemers walk among your flock.
  
  Despite your blindfold, \\
  proudly you carry the stone on your back.\\
  Disillusioned,\\
  you plant your feet safely to the ground.
  
  Demigod!\\
  Blasphemers walk among your flock.
  
  As the stone you have become.
  
  Not once did you cry\\
  for the lost ones of your world.\\
  Your care is limited to this demigod\\
  onto which you cluster.
  
  Reason! Decrystallize me.\\
  Demigod! I do blaspheme.
  
  The fallen you condemn.\\
  Your heart even free of hate.\\
  Yet, scared to death by their disbelief\\
  to ordinary common sense.
  
  And with autumn closing in, \\
  forcing life away,\\
  no mercy wil impale your sin.\\
  Dead are your tears.
}

\target{Aburun inspires Belzir}
She reads the heretic works of \hs{Aburun Dol Cuma}, and other works that are much darker and more horrible.

She discovers the truth about the Vaimons and the way the Cabal is controlling them. 
She even has a clue about \iquin. 

\citebandsong{Emperor:IXEquilibrium}{Emperor}{The Loss and Curse of Reverence}{
  Alas, this agony, the emptiness of earthborn pride\\
  hath stirred my faithful heart which guided me to darker paths.\\
  Far away from their pestilent ways\\
  cleansed was I from deceitful grace.
  
  Yet, put to scorn was I\\
  by those unclean.\\
  Enslaved by ignorance\\
  they blindly spat upon\\
  the deity of hate.\\
  Awake is the darkest fiend.
  
  By the fallen one I shall arise.
  
  Upon bewildered masses,\\
  to whom the indulgence of my soul portray as sin made god,\\
  I shall revile and quell the source\\
  whence mockery of my kind derive.
  
  This I know: Facile shalt my quest not come to pass.\\
  Deathwish be my gift to all at last.
  
  Believer, speak not to me of justice,\\
  for none have I ever seen.\\
  By God, I shall give as I receive.\\
  Betrayer, speak not to me at all.\\
  You and this world ripped my fucking heart out.
}

\citebandsong{Emperor:IXEquilibrium}{Emperor}{Depraved}{
  he looked upon the world\\
  and saw that it was still depraved
  
  as him though not like him\\
  for him, as all but closed\\
  like them, not him
  
  unaware he was already \\
  caught within their cycle\\
  with them outside
  
  depraved
  
  instinctive rebellion\\
  possessed by bestial despair\\
  he fought this law\\
  and broke the spell
  
  \quo{So, this is why I built my empire\\
    fled from the truths unto which they conspire}
  
  in spirit born a stranger\\
  a counterpart in mind\\
  the seeker raised to bury\\
  a freak among the blind
}

She feels she has a higher purpose than this. 

\citebandsong{Emperor:IXEquilibrium}{Emperor}{Depraved}{
  encountered with mysteries\\
  on his mission to reverse their ways\\
  he sought a life among the dead\\
  with refuge in an endless maze
  
  ...before the silence...
  
  \quo{This can not be the purpose of my presence.\\
    What then is the call of my return?}
}




\subsubsection{Geicans were religious}
\target{Geicans were religious}
At this point, \ClanGeican was religious and Iquinian.
It was \Belzir{} that introduced the atheist philosophy, and her son \hs{Zacrias} carried it on. 





\subsubsection{Geicans are greedy and materialistic}
Apart from being religious, \ClanGeican (and perhaps all Vaimons) are also hypocritical. \ShiinMerodar{} is corrupt, a cesspool of hypocrisy, bureaucracy and greed. I need to underline the avarice, pettiness and treacherousness of the Geicans. 

Have a named guild of politicians, lawyers and the like (\quo{\DJOF'ers}\footnote{\quo{\DJOF} is Danish for \quo{Danske Juristers og \O konomers Forbund} (or something like that), the \quo{Danish Lawyers' and Economists' Association}.}), who are made the scapegoats and blamed for the corruption of Geica, by virtue (or vice) of the materialistic world view they spread. 

Compare to the Letherii culture in \SEMalazan, and especially Chancellor Triban Gnol and Invigilator Karos Invictad. 

Perhaps the Geicans own no slaves, only life-long indentured serfs. Compare to the Indebted in Lether. 

Belzir wants something greater, something nobler and more beautiful than this. 

\citebandsong{Emperor:IXEquilibrium}{Emperor}{An Elegy of Icaros}{
  The fear is not the fate I seek.\\
  My destiny will build upon\\
  the mighty turbulence beyond.\\
  If I fall I will rise again.
}





\subsubsection{She makes social improvements}
She is a cruel and decadent tyrant, but she also introduces some genuine improvements in society. 
She removed some evil, repressive rules and tried to make the empire more free. 
She was popular in certain circles among the people. 









\subsection{\ps{\Belzir}{} idols}
She is inspired by some of the mighty and controversial Dark Vaimons in history and legend. 

She is especially fascinated with \Delphine. 
Only later does she realize that she \emph{is} \Delphine. 
This puts everything into perspective and further strengthens her conviction. 

\citebandsong{Emperor:IXEquilibrium}{Emperor}{An Elegy of Icaros}{
  Icaros, your voice once melted\\
  into the choir of the fallen ones.\\
  I have heard, I have seen\\
  the purity of their song.
  
  Icaros, your fate embrace\\
  a manifold of angels. \\
  I summon thee from shattered graves\\
  and call upon the wind.\\
  Recieve my bow of reverence.\\
  Then spread your wings\\
  and fly into oblivion.
}









\subsection{\Belzir{} reads Vizicar's notes}
\target{Belzir reads notes}
\Belzir{} discovered and read the \hr{Vizicar's notes}{notes about \Malachim} that \VizicarDurasRespina{} had written. 
She read them. 
They helped her a lot in achieving her \apotheosis. 









\subsection{\Belzir{} betrays her subordinates}
Maybe have scenes where \Belzir{} betrays her subordinates and allies and \hr{Resphan cannibalism}{eats them}. 

These should be loyal people who trust and love her, making her crime all the more heinous. 

Compare to Akasha in the movie \cite{Movie:QueenoftheDamned}.









\subsection{\Belzir{} comes to hate it all}
\Belzir{} gradually remembered her tragic past. 
She recalled how she, as \Shiaraid, once the high princess of \Mystraacht, fell into disgrace and was reviled as a villain and traitor. 
And how she then saw the \emph{same} thing happen again in her life as \Delphine. 

\Belzir, who was already heavily influenced by the Curse and the self-destructive madness it brought, became defiant, thinking: 
\ta{%
  If it is my destiny that History should curse me by all my names, then so be it. 
  By the \qliphoth, I will give History a reason to curse the name \Belzir!}

When she finally died, she was almost relieved to leave it all behind, happy to die. 
She had grown very bitter and hateful. 
Waging war against the world, hated by everyone, she did not enjoy life anymore. 
The Curse was part of the reason. 

She thought back to the time when she killed \Eryal/Silqua. 
She reflected on all the destruction and pain she had caused for everyone in her life as \Belzir. 
She reflected that the world would probably be a better place without her. 

\citebandsong{Emperor:Prometheus}{Emperor}{Thorns On My Grave}{
  I hereby commit my body to the ground.\\
  Sterilised and wrapped in plastic foil.\\
  Being an object of this space and time,\\
  this body should remain concealed.
  
  For it holds every disease \\
  ever exposed.\\
  It holds all pain and death\\
  I could ever unleash.
  
  Beneath deceiving, fragile skin\\
  breathes the ever growing hate within.
}

She felt her life had been meaningless. 
That she had caused nothing but harm. 

\citebandsong{Emperor:Prometheus}{Emperor}{Thorns On My Grave}{
  Since the first glimpse of my existence,\\
  I have fed this greedy infection.
  
  An aimless search for potential persistence\\
  found no escape from the fatal injection of life.
}

With her last strength she called out to Zacrias-tachi. 
She asked them to keep the memory of her good deeds alive. 
She did not want to be remembered solely as an evil queen, but also for her open-mindedness and her love of life. 

\citebandsong{Emperor:Prometheus}{Emperor}{Thorns On My Grave}{
  I am the [mother]. I am the [daughter].\\
  My refugee soul has escaped. This body depraved. \\
  Of final wishes I ask none, but one:\\
  Now that I am gone, lay thorns on my grave.
}









\subsection{\Belzir goes into \bane ruins below \TopazChateau}
\Belzir discovered that there was a system of \bane ruins hidden beneath the Redcor's \TopazChateau.
She \hr{Belzir goes below Chateau}{went there}, hoping to gain its power, but failed and died and was imprisoned by the Redcor. 









\subsection{\Belzir{} awakens as she dies}
\target{Belzir's Apotheosis}
At the end of the \HundredScourges, \Belzir{} is finally killed. 
Her death is so traumatic that it shakes her out of her \malach{} amnesia and awakens a lot of memories from her true life as a \resvil.
She finally achieves her \hr{Apotheosis}{\Apotheosis}! 

\citelimbonicart{MoonintheScorpio}{Beyond the Candles Burning}{
  Beyond the candles burning, beyond all minds eye.\\
  A vast emperic enigma awaits me as I die.\\
  In a graceful dance obscene, in a ring of fire,\\
  I obtain my majesty as flames caressing higher.
  
  Release my spirit, unleash my soul.\\
  From the darkest dungeon, oblivion calls.\\
  In the phallic halls of ancient forlorn\\
  a cold sanctuary in doom is born.
}

She is killed but her soul is not destroyed. 
So the Vaimons dismember and imprison her soul. 
Using powerful magic, her enemies successfully banish her soul from \Miith{} (into the darkness of \nieur{}, as the Redcor tell the story), but they fail at destroying her. In time, she learns to communicate with people on \Miith{} again and plots to return. 

As she died, she reached out, semi-consciously. 
She ended up being partially responsible for the \HundredScourges. 

\citebandsong{BeyondTwilight:FortheLoveofArtandtheMaking%
}{%
  Beyond Twilight%
}{%
  For the Love of Art and the Making%
}{
  See the black widow senseless falling\\
  Hear her now in horror calling\\
  Silent thoughts magically sing
}

Compare to back when \Ishnaruchaefir{} laid waste to the world in his anger. 















\section{Sseju Story Arc}
\subsection{Sseju discovers new gods}
\target{Sseju finds gods}
Sseju and his companions discover, contact or free some mighty creatures who end up becoming the \hr{Ortaican gods}{gods of \Ortaica}. 















\section{The \HundredScourges (\yds{Darkfall})}
\target{Hundred Scourges}
The fall of the \VaimonCaliphate was \quo{the \HundredScourges}. 
The civil war surrounding the dethroning of \Belzir{} led to a number of calamities befalling the \caliphate. 

In the fighting, Vaimons expended huge amounts of magical energy. 
Because of \hr{Parasitic Archons}{the parasitic nature of Vaimon magic}, this resulted in a number of catastrophes. 
The Heart of \Miith{} was temporarily exhausted and could not bear to create much new life. 

\target{Belzir's Carcer unleashed}
Another cause of the Scourges was the unleashing and destruction of tons of bound souls from \hr{Belzir binding souls}{\ps{\Belzir} (formidable) \carcer}. 

The Scourges included:
\begin{enumerate}
  \item Immorality (blamed on \Belzir). 
  \item Civil war. 
  \item Crime (ie., banditry).
  \item Disease (an \caliphate-wide epidemy of a leprosy-like disease). 
  \item Stillbirth. 
  \item Infertility. 
  \item Ash (defiling magic caused some vegetation to turn to ash). 
\end{enumerate}

The Scourge of Immorality is blamed on \Belzir. 
Allegedly it was her sins that brought down the anger of the \Archons, which caused all the other Scourges. 

The defeat of the Queen does not end the war. Weapons of mass destruction are used. The \caliphate is destroyed and \Miith{} bombed back several TLs. Thus end the Fourth Age. 

This wiped out many of the faithful. 

\quo{Fortunately}, the heathens had had their asses heavily kicked by \VizicarDurasRespina{} and others, so they were in no shape to sweep in and conquer the remnants of the \caliphate. 





\subsubsection{\Ortaican view}
\Ortaican mythology has \hr{Ortaican myths about the Hundred Scourges}{opinions about the \HundredScourges}. 





\subsubsection{Epilogue}
The book should end with \hs{Sseju}'s heirs predicting that their time of greatness is at hand. 
















\chapter{Aftermath of the \HundredScourges}
\section{Sseju}
After the \hr{Hundred Scourges}{\HundredScourges}, the world splintered into a number of small princedoms fighting each other. 

But \hs{Sseju} had founded a religious tradition that would later become the \hr{Rethyax}{\rethyaxes} and the \hr{Ortaica}{\Ortaican} line of sorcerer-kings. 













\section{The Rise of \Ortaica}
They conquered their neighbours and grew big over a period of 200 years. 









\subsection{\Ishicah{} enslaved}
The \Ortaican{} \rethyaxes{} \hr{Ishicah enslaved}{capture and enslave \Ishicah}. 













\section{\Tepharae succeeds \Ortaica (\yds{Fall of Ortaica})}
\target{Tepharae succeeds Ortaica}
The \Ortaicans{} eventually became too evil. 
At last their Scion slave was destroyed, which lost them some of their power. 
The \Iquinian{} \hr{Tepharae}{\Tepharites} rose up in rebellion against them, and \Ortaica{} was overthrown. 









\subsection{Peaceful takeover}
The \Telcra/\Tepharin takeover was remarkably peaceful by \Miithian standards. 
There were battles and wars fought, but it was mostly a quasi-peaceful religious reformation followed by lots of diplomacy. 
For a century or two, the \Tepharin people dominated a large part of \Azmith. 
































\chapter{Iquinian creation myth}
In the beginning there was the Light. The Light was, the Light is, and the Light will be, for the Light is eternal and everlasting, the Light is without beginning and without end and without boundaries. 

Before the world there was the Light. The Light was, the Light is, and the Light will be. And the Light give birth unto the \Sephiroth{}, for the \Sephiroth{} are of the Light, and the \Sephiroth{} are comprised of the Light. Four times four is their number, and the number of the \Sephiroth{} is sixteen. 

Before the world there were the \Sephiroth. The \Sephiroth{} were, the \Sephiroth{} are, and the \Sephiroth{} will be. And the \Sephiroth{} gave birth unto the Light, for the Light is of the sixteen \Sephiroth{}, and the Light is comprised of the \Sephiroth{}, whose number is four times four. 

The \Sephiroth{} are of the Light, and the Light is of the \Sephiroth{}. The many are one, and the one is many. Four times four is the number of the \Sephiroth, and the number of the \Sephiroth{} is sixteen. The sixteen are one, and the one is sixteen. 

Before the world there was the Light which is the \Sephiroth{}. Before the world there were the \Sephiroth{} which are the Light. 

And they created the world. 

And they created the First Men. And the first men worshipped the Light and gave praise unto the \Sephiroth{}. 

But the First Men grew decadent, and they grew negligent in their worship. They became arrogant and said: \ta{We are the greatest, the fairest and the wisest creatures of \Miith{}. Why should we fear the Light? Why should we serve the \Sephiroth{} and give prayer and thanks unto them, we who are the greatest, the fairest and the wisest of all creatures of \Miith{}?} 

And so the First Men turned away from the Light, and they abandoned the worship of the \Sephiroth{}, and they, who were the greatest, the fairest and the wisest of all creatures, they fell into wickedness, and they fell into sin, and they fell into blasphemy, and they fell into idolatry. 

The First Men turned to idolatry, and they took up the worship of evil powers, and they did sin, and they did blaspheme. 

And the \Sephiroth{} saw that the First Men had broken the covenant and fallen into wickedness, and they spake, saying: \ta{The First Men are turned from us. They have broken the covenant and abandoned the Light. In turn, the Light shall abandon them. And the First Men shall know the wages of sin, and the price of their wickedness shall lay heavy upon them.}

This the \Sephiroth{} spake: \ta{As they court evil, so evil shall come upon them, and evil shall engulf them. As the First Men worship the idols of evil, so shall evil enter into the world and cause them untold woe. And the First Men shall suffer and weep and be destroyed, because they turned to idolatry. Evil shall come into the world, and evil shall fall upon them and engulf them. Evil shall torment them, and destroy them, and death shall take them. Thus the First Men shall be destroyed, because they turned from the Light. Thus have spoken we of the Light.}

And so the \Sephiroth{} did turn from the First Men. As the First Men had turned from the Light, so the Light turned from them, and they were abandoned. 

And as the Light abandoned the First Men, so it was that evil was allowed to enter the world. And forces of evil did enter the world: \Daemons{} and monsters, they came into the world that they might wreak evil. 

Of the forces of evil that came into the world, the most terrible were the \dragons. The mightiest they were, and the foulest, and the most hideous, and the most evil, and they came into the world that they might wreak evil. Foremost among the \dragons{} were their gods. They were the \Dominators, and the greatest and the most evil of the \dragons{} were they, and they were the gods of the \dragons. Five in number were they, and the number of the \Dominators{} was five, and these are their names: 

First of them is Tiamat. Vast of body is she, so that the greatest of \dragons{} is as naught beside her. Terrible is she to look upon, and six heads hath she, and from her six mouths she breatheth fire and ice and lightning and poison. Among the \dragons{} she is the Queen. Of Chaos she is the Queen. She is evil itself, and she is Chaos itself. Of all creatures of evil she is the greatest, the mightiest and the most evil. Of the creatures of evil she is the Mother, and all manner of monsters and \daemons{} and accursed things do spring from her tained womb. She is Tiamat, and of the \Dominators{} she is the first. 

Second of them is \ApepNesthra. Of colour he is white, deathly white like unto a corpse, sickly white like unto the maggots that feast upon the dead. 
