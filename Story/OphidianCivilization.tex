















\chapter{\FirstbanewarBook}















\section{Pre-History}
In ancient times there were three peoples that dominated \Miith{}: 
The \krakens, the \xss{} and the \voyagers. 

Only the \krakens{} were native to \Miith{}. 
Billions of years ago they ruled the Realm. 
Then they became tired and went dormant. 









\subsection{World-Gods}
\target{History of the World-Gods}
The \quo{World-Gods} were a race of cosmic gods.
A World-God could merge with a planet and \emph{become} the planet.
In this way they colonized much of the universe.
Two such planets were \Miith and \Erebos. 

The \voyagers waged a neverending war against the World-Gods.

\target{Voyagers slay the World-God of Miith}
After a titanic war the \voyagers slew the World-God of \Miith.
To prevent it from ever rising again the \voyagers created a parasitic \dweomer, the \hr{Heart of Miith}{Heart of \Miith}.
This \dweomer drained energy from the World-God's own \dweomer, the \hs{Dark Heart} of \Miith. 
The Dark Heart remained undead and rotting precisely because the Heart feasted on it like a leech.
And all life on \Miith was spawned by the Heart and lived in symbiosis with the Heart.
What this meant was that all life on the planet really existed only to keep the World-God chained in death. 
All living creatures were prison wardens. 
The blood that coursed through their veins were conduits for the power of the Heart that kept the dead World-God in check.

Thus the World-God created its own monsters to destroy this hated life and allow the World-God to rise again and resume its awful, monstrous life.
These monsters were the \hs{Masters of Negation}, and their servitors were the \hr{Noggyal}{\noggyaleth}.

The Masters of Negation waged war against the \voyagers. 
Eventually the \voyagers left \Miith. 
Billions of years later, this life that the \voyagers{} had created would become the \ophidians. 









\subsection{\Voyagers come to \Miith}
\target{Voyagers come to Miith}
The \voyagers{} came from space in their spaceships. 
They created life in their Genesis Pits. 

\target{Voyagers create the Heart}
The \voyagers{} created the \hr{Heart}{Heart of \Miith} from some Chaos power. 
They shaped \Miith{} into something that looked like modern-day \Miith{}. 

\target{Voyagers create mother-mass}
They tampered with the \noggyal \hs{mother-mass} (which was \hr{Noggyaleth were the first life}{the first life form on \Miith}). 
From it they shaped all sorts of life. 

They enslaved the \noggyaleth and used them as machines and slaves and even houses.
The \voyagers taught the \noggyaleth much. 
Thus was sown one of the seeds that would eventually cause the \pps{\voyagers} downfall. 

When the \voyagers{} shaped all their life-stuff, they unwittingly tapped into a sinister force; the energy of a dark god of life and death: \KhothSell. 
This entity infested their \dweomers{} and their \hs{mother-mass}. 
The \voyagers{} did not discover this until very late. 
From there they could only hope to contain the taint, not cleanse it. 
Later their creations rebelled against them. 









\subsection{\Voyagers erect Palisades}
\target{Voyagers erect Palisades}
\target{Palisades}
The voids between planets were \hr{Horrors of the Void}{filled by all sorts of horrors}.
When the \voyagers settled \Miith hundreds of millions of years ago, they erected dimensional barriers around the Realms of \Miith to keep it safe from these horrors.

In a sense, the \xss were some of these \quo{horrors} that the \voyagers feared.
They were some of the greatest \quo{horrors}.

These dimensional barriers, known as the \Voyagers' Palisades, were an early Shroud-like construction, but much better made (by the super-advanced \voyagers), and so had none or few of the destructive side-effects that the later Shroud had.

The Palisades also kept the \banes out, but eventually the \banes managed to drill through.

The \ophidians had some limited understanding of the \voyagers' work, and they drew on it.
They built the \hr{Crystal Sphere}{\CrystalSphere} to patch up the holes in the Palisades that the \banes had made.









\subsection{Poetic version}
\subsubsection{The \voyagers{} and the \krakens}
Once there were the \voyagers, a race of mighty beings (gods, if one wills) who \travelled across the Universe, seeding worlds they found with life. In primordial time, billions of years ago, they came to \Miith{}, desiring to make the world theirs, a breeding ground for their creations. 

But \Miith{} was not barren. Before the \voyagers{} came, \Miith{} had her own children, her own indigenous life, and the greatest of them were the \krakens, the native overlords of \Miith{}. Terrible creatures they were, and immortal. And they were jealous and cared not for invaders, but craved \Miith{} as their birthright, and when the \voyagers{} sought to conquer their world, they fought. Few were the \krakens{}, numbering scarcely a dozen against the thousands of \voyagers{}, great and mighty in their own right. Yet in their wrath, the \krakens{} proved more than a match for the invaders, and in every conflict they would prevail. With valour and fury the \voyagers{} would fight, and with despair, for the \krakens{} were immortal and wielded the primal power of \Miith{} herself. Their leader and mother, the great \Kraken{} Queen, was the mightiest of all the children of \Miith{}, and none could stand before her when she rose in fury. 

But the \krakens{} were indolent, and they would often sleep, their entire race, for aeons at a time - millions, even hundreds of millions of years. And so, whenever the \krakens{} slept, the \voyagers{} would invade \Miith{}, to populate the planet with myriad life forms of their devise. And every time, after an aeon, the \krakens{} would awake to vanquish the \voyagers{}, scour the world of their creatures and repopulate it with their own spawn. But inevitably, the \krakens{} would once more go dormant, and the \voyagers{} would return to begin their work anew. 

For the \voyagers{} seek ever to perfect life, and \Miith{} is precious to them, for \Miith{} is an anvil on which to forge and shape their creations, their works of art. And a most excellent anvil is Mother \Miith{}, for it is her nature to take what is given to her and break it down and hammer it into a new shape, greater and mightier than before. The \voyagers{} have \travelled the universe for billions of years, and they have known many planets. Many worlds are hostile, wastelands of death that will destroy life and cause it to decline and decay. Other worlds are indifferent, possessed of no powerful energy, positive or negative. They will allow life to exist, but they will not support it. On such worlds, life will persist, but it will stagnate and remain humble and never know greatness. Not so Mother \Miith{}. She is possessed of a soul, and she is strong and fierce and indomitable. A loving mother is \Miith{}, but also cruel. The weak among her children she will destroy, and they shall have no legacy 
and know only oblivion. But the strong among her children she will cherish and glorify. Forged in fire and tempered in blood they shall rise, growing ever stronger, and they shall know greatness. For \Miith{} is a crucible of life, and her like is seldom found in the universe, and the \voyagers{} knew to cherish her. 





\subsubsection{\Moroch{} and the \nagae}
So the history of \Miith{} is divided into such cycles of creation and destruction in the cosmic struggle between these factions, the native \krakens{} and the alien \voyagers{}. 

Our story begins in one such cycle. The \voyagers{} had filled \Miith{} up with their creations, but they were fled, for the stars moved, and they knew that the \krakens{} would soon awaken. And one of the \krakens{} did awaken; in this day and age he is called \Moroch{}. What is his aim we cannot know, but when he awoke he began not to destroy, but to create. From the sea he took primitive beings, creatures of the \voyagers{}, and reshaped them in his own design. He gave them strength of body and mind, and bestowed upon them the power of thought, and he made them his servitors. 

The years passed, a million years and more, and still the \Kraken{} Queen and her brethren slumbered, and even \Moroch{} grew sleepy and fell dormant. But his spawn lived, and they grew and prospered, and they recalled their sire and offered him prayer and tribute. Hybrid children they were, born from the womb of the \psp{\voyagers} creation but fathered by a \kraken{} lord. Lowly they were, yet in this age they were the lords of \Miith{}, and their built their cities in the deep oceans across the globe, and they were the \nagae{}. 















\section{The \Ophidian{} Civilization}
\target{Ophidian humanoids}
Hundreds of millions of years after the \voyagers{} had left \Miith{}, the \ophidians{} evolved. 
From the ashes of the devastated world they rose up to make themselves its masters.
They went on to form great empires and develop marvelous science and sorcery.









\subsection{Origin of the \ophidians}
\subsubsection{Metabolism}
\target{Ophidians evolved}
\target{Ophidians half warm-blooded}
\target{metabolism control}
\index{metabolism control}
\index{warm-blooded}
\index{cold-blooded}
The \ophidians{} descended from reptiles who were making the transition from \quo{cold-blooded} to \quo{warm-blooded}. 
Crocodiles, lizards and the like remained cold-blooded, while synapsids, dinosaurs and pterosaurs evolved and become warm-blooded. 

Another group of animals were stranded midways. 
They developed the curious ability to switch back and forth, controlling their own metabolism to \quo{cool down} and become cold-blooded or warm-blooded as they wished. 
In a sense, this gave them the best of both worlds: 
The dynamic speed of warm-blooded creatures coupled with the patience of cold-blooded creatures. 

\citeauthorbook[p.80]{RobertTBakker:TheDinosaurHeresies}{%
  Robert T. Bakker%
}{%
  The Dinosaur Heresies%
}{
  The great serpents succeed by being something a warm-blooded mammal could never be\dash a hunter of infinite patience\ldots{}
}

The early \ophidians{} were crocodile-like in form. 
They developed relatively small bodies (like \humans or smaller). 

Despite their advantages, they were still not quite as effective as the more specialized true cold-bloods and true warm-bloods, so they remained a niche group. 
For many millions of years they evolved to fill special niches. 
Instead of huge size like some \saurians{} (or tiny size, like mammals), the proto-\ophidians{} were forced to take a different evolutionary route: 
They evolved psionic abilities, most notably psychometabolism. 

\target{torpor}
\index{torpor}
The \ophidians{} learned that if they made sure to be as inactive as possible, they could conserve their bodily energy, keep it in reserve and use it to power their brains (and, later, their psionics and sorcery). 
They could go into a state of torpor where their bodies would hibernate and operate at a very low metabolism, but their brains and minds could still work at full capacity. 
By utilizing their torpor strategically, \hr{Ophidian Imperial}{Imperials} would later learn to hoard vast amounts of energy and redirect it for magical purposes. 





\subsubsection{Telepathy and telekinesis}
Their psychometabolistic abilities were linked to a large and powerful brain. 
As their brains grew, this opened up the possibility for them to learn simple telepathy. 
Telepathic sensitivity was a useful survival trait that helped them hunt their prey and escape their own predators. 

They also learned simple telekinesis. 





\subsubsection{Intelligence}
In ancient times, the \ophidians were the underdogs in a vast, hostile world of gruesome alien races and lightly slumbering Elder horrors.
The early \ophidians were not at the top of the food chain. 
They were hunted as pests by the more powerful monsters. 
Only the most cunning \ophidians survived. 

They evolved cunning, stealth and dexterity. 
They learned to manipulate objects, both with their limbs and with telekinesis. 

Living in the ruins of elder civilizations and fighting for their lives, the early \ophidians gradually learned to live together in packs and cooperate. 
Social structure developed. 
They developed the use of tools. 
Gradually they developed intelligence and culture. 









\subsection{\Ophidian rise to power}
\target{Ophidians as underdogs}
From their very inception as a race, the \ophidians were pitted against elder creatures. 
They fought to survive. 
The nascent \ophidians were cunning and swift, and they reproduced quickly. 
They developed science, technology, psionics and sorcery.
Gradually, over the course of many millennia, the tide turned in their favour. 
They were able to push back the elder races and establish their own strongholds.
\Ophidian cities began to emerge. 

The \ophidians were underdogs, but they persevered. 
They managed to steer clear of the worst threats. 
As they learned more about the true nature of the world, they kept from going insane and instead increased in power. 
Their science and sorcery grew, aided in part by alien relics that they discovered. 
There were still horrid things that the \ophidians feared to challenge, but these were dormant threats. 
Those alien forces that actively threatened were ultimately defeated, and the \ophidian race won dominance over the planet. 
They were now a mighty race of scientists and sorcerers.

They grew to become the dominant civilization. 





\subsubsection{Paragons/Imperials}
The \ophidians were individually too weak to challenge the elder races. 
But they had social behaviour, and they had their psionic powers. 
Gradually they learned to take pack cooperation a step further. 
They learned to lend the pack's collective mental power to the leader, allowing the leader to grow much more powerful than an average \ophidian by drawing upon the essence of his pack.
These leaders became known as \emph{paragons}, and their supporters as \emph{vassals}.
This was a form of \emph{\nexus}, with the paragon as an \emph{\apex}. 
In time, the \ophidians also learned that a paragon could act in turn as vassal to a greater paragon. 

The \ophidians discovered that\dash at great cost\dash they could imbue an egg with extra essence and thus breed a prodigy who could wield paragon power to greater effect than a regular \ophidian. 
These \quo{born paragons} gradually became a superior subrace, a ruling caste. 
They came to be known as \hr{Ophidian Imperial}{\emph{Imperials}}.

The \ophidian Imperials \hr{Ophidian immortality}{achieved immortality}. 





\subsubsection{Terrible elder world}
They were the most recent in a long line of civilizations on the planet. 
Almost like \humans in the Cthulhu Mythos. 

Compare to the people of Mu in \cite{HPLovecraft:OutoftheAeons}, who fear Ghatanothoa and other terrible gods and placate them with sacrifices. 

There were mountains and great basaltic pillars wherein dwelt the horrid \hs{flying polyps}. 
There were great areas that the \ophidians dreaded and learned to stay away from.
Even in their time of greatness they feared these relics of an elder age. 





\subsubsection{Wars against the Nibir}
\target{Nibir}
\index{Nibir}
Before the \ophidians, the Nibir dominated \Miith. 
The Nibir were aliens who came from the world of Phaeton. 

A Nibir looked like a giant \emph{Anomalocaris}, but with many long spindly jointless legs like the spines on a \emph{Hallucigenia}, and with many soft \emph{Hallucigenia}-like tentacles on its back. 

They were served by a lesser race called the Phaetonians. 
A Phaetonian was roughly human-sized or slightly larger. 
It resembled a \emph{Marrella} (a trilobite-like creature from the Burgess Shale), but with fewer limbs: 
Four long multi-jointed crab- or spider-like legs and four slimmer \quo{arms}. 

For thousands of years the cruel Phaetonians used the early \ophidians as slaves or hunted them for sport. 
But the \ophidians fought. 

The fledging \ophidian civilization waged bitter wars against the Nibir and eventually prevailed. 
They destroyed all Nibir and Phaetonians on \Miith. 

Compare this to the wars waged by humanity against the elder races in \cite{RobertEHoward:TheShadowKingdom}. 





\subsubsection[Wars against the Shugul]{Wars against the \moonthings}
\target{Ophidians drive out Shugul}
The \ophidians waged war against the \shugul \moonthings and their \moongods.
They destroyed the \shugul cities and drove them out.
Soon only the \shugul capital of \Nom was left.

Later \hr{Sethicus invades Nom}{\Sethicus invaded \Nom} and built his own temple-city of \Baltherium.





\subsubsection{Wars against the \noggyaleth}
\target{Ophidian-Noggyal wars}
\target{Noggyaleth lose a war before the FBW}
The \ophidians knew that the \noggyaleth existed. 
They did not know the \noggyaleth's true nature, although a few occultists suspected.
The \ophidians feared the \noggyaleth just as they did the \xss, but the \ophidian magic was mostly powerful enough to keep the chaotic, bestial \noggyaleth at bay or even destroy them.
In earlier days, the \ophidians waged great wars against the \noggyaleth and drove them underground.





\subsubsection{Early \ophidian culture}
\target{Early Ophidian culture}
The \ophidians developed culture and civilization early on, many millions of years before \Tiamat. 
But they had the infinite patience of reptiles, so \hr{Ophidians are slow}{they lived very slowly}, due in part to their \hs{torpor}.
Hence their culture also developed extremely slowly. 
Their civilization lasted for millions of years. 

This also meant that they did not fuck up the ecosystem around them. 
Every new invention for the \ophidians{} might take many thousands of years, giving surrounding species time to adapt and not just be wiped away. 

At first the \ophidians knew nothing of the \xss except vague legends handed down from prehistoric alien races. 
But they knew of various monstrous gods that dwelt on \Miith, such as \Ubloth. 
These \Miithian gods are comparable to the Great Old Ones of the Cthulhu Mythos, where the \xss are comparable to the Outer Gods. 
The \ophidians lived in fear of the ancient gods and worshipped them. 

They still knew little about the \xss. 
Only hints and myths.
What little they knew, they feared.
Only a few mad cultists worshipped them, and with little success.
\Sethicus was the first who was strong and brave and daring and heretic enough to \hr{Sethicus understands XS}{\emph{understand} the \xss}. 





\subsubsection{Lords of the Deep}
\target{Ophidians against Lords of the Deep}
The \hs{Lords of the Deep} had fought against the \moonthings. 
After the \ophidians had weakened the \moonthings, the Lords of the Deep returned, now more powerful. 
They conquered the planet and enslaved some \ophidians. 

The \ophidians fought back. 
They remembered that they had prevailed against elder races before. 
They vanquished the Lords of the Deep and cast them out, back into the deep sea. 
The \ophidians were now truly the rulers of \Miith. 





\subsubsection{Religion taboo}
After the \ophidians had won their freedom from the Nibir and other elder races, they promised to never again serve or worship a foreign race. 
They pledged themselves to the dream of \ophidian supremacy:
The \ophidians should rule and bow to no masters. 

A strong cultural taboo arose. 
It was forbidden to worship anything as a god. 
The \ophidian culture was highly \hr{Ophidian philosophy}{atheistic and rational}. 

They also established a taboo against certain types of sorcery. 
\Sethicus would later challenge that taboo. 









\subsection{Golden age}
\target{Ophidian Golden Age}
\target{Ophidian golden age}
\target{Ophidian civilization}
\index{technology!\ophidian}
The \ophidians{} built a civilization which was great, mighty, proud, enlightened and technologically advanced. 



\lyricstitle{\emph{Call of Cthulhu} RPG p.118}{
  [The Serpent People] built black basalt cities and fought wars, all in the Permian aera or before. 
  They were great sorcerers and scientists, and devoted much energy to calling forth dreadful \daemons{} and brewing potent poisons.
}

\citeauthorbook[p.44--45]{ClarkAshtonSmith:UbboSathla}{Clark Ashton Smith}{Ubbo-Sathla}{
  At length, after aeons of immeroial brutehood, it became one of the lost serpent-men who reared their cities of black gneiss and fought their venomous wars in the world's first continent.
  It walked undulously in antehuman streets, in strange crooked vaults; it peered at primeval stars from high, Bebelian towers; it bowed with hissing litanies to great serpent-idols.
}





\subsubsection{Breeding work}
\target{Ophidians breed}
They had several entire races as their servants and slaves. 
And they controlled \daemons{}, whom they commanded to build their great cities. 

They learned bio-technology and biomancy. 
They started manipulating other races, breeding and tweaking them. 
They created their own servitor races and animals. 

That was the reason why, later on, there existed so many animal species on \Miith{} that could be domesticated: 
The \ophidians{} had been breeding them, millions of years before. 

Among other things, they helped shape the \hr{Nycan}{\nycans}. 
They also \hr{Ophidians create Nephilim}{created the \nephilim as slaves}. 





\subsubsection{\Ophidians create \nephilim}
\target{Ophidians create Nephilim}
Perhaps the \ophidians bred the \nephilim as slaves.
The \nephilim would remain slaves until \hr{Origin of Aryothim}{the advent of the \aryothim}. 









\subsection{Wars against the \vorcanths}
After the victories \hr{Ophidians drive out Shugul}{against the \shugul} and \hr{Ophidians against Lords of the Deep}{against the Lords of the Deep}, the \ophidians fought against the \vorcanths. 
The \ophidians wanted to colonize all the Realms of the system. 
The \vorcanths would not let them. 









\subsection{\Ophidian subraces}
\target{Enslaved Ophidian subraces}
The \caisith made several genetically engineered subraces, especially of the Worm caste.
These \caisith variants were mostly enslaved and subjugated. 

Compare to the thunder warriors and astrotelepaths of \emph{Warhammer 40,000}. 









\subsection{Dreadnoughts created}
The \hr{Dreadnought}{Dreadnoughts} were created. 
They were war machines in the shape of gigantic reptiles. 
They were created to fight against the \vorcanths and other enemies of the \ophidians.
And also used in wars between \ophidian nations. 









\subsection{Interstellar civilization}
\target{High-tech civilization}
\target{interstellar civilization}
\index{technology!interstellar civilization}
The \ophidians built an interstellar civilization. 
Using magic and other science, creatures could travel between Realms and through the vastness of space to \cooperate, trade or wage wars. 

\Sethicus \hr{Sethicus brought innovation}{helped build this civilization}. 

They built colonies on other planets. 
Here they found remnants of extinct alien civilizations.
Some of the colonies, however, were destroyed when those remnants rose up in vengeance against the \Miithian interlopers.

There were also \hs{living machines}. 





\subsubsection{\Ophidian expedition to Visha}
\target{Ophidians invade Visha}
The \ophidians once tried to conquer Visha.
They fled in horror before the horrid and unfathomable \moongods. 
They abandoned the plan to settle Visha. 















\subsection{Ramarxes and the Concord}
\target{Concord}
\index{Concord}
\hs{Ramarxes} was a genius \caisith inventor and entrepreneur. 
His great invention was the \emph{Concord}. 
The Concord was a telepathic union allowing all participants to communicate and remain in contact.
The Concord also functioned as a global repository of knowledge and data\dash a telepathic internet. 
A universal \caisith supermind. 

When the Concord was invented, it was quickly adopted and spread across the world. 
It strengthened the \caisith civilization.
It allowed them to build stronger \nexuses, so the paragon leaders could grow stronger. 
Ramarxes, as the inventor of the Concord, became the most famous and popular \caisith alive\dash perhaps ever. 





\subsubsection{New Concord}
Ramarxes kept developing and refining the technology of the Concord. 
He hungered for personal power and greatness, as did any ambitious \caisith. 
He wanted to be the greatest. 
So he developed \quo{deeper} Concord contracts, which individuals could then enter into (like license agreements). 
This enhanced set of Concord contracts was called the New Concord. 
Gradually, the New Concord became able to extract more and more information from individuals and influence their minds more and more.
More and more, the New Concord became a tool of mind control. 

The development of the Concord and the New Concord took centuries. 

Ramarxes was not \apex of the entire Concord. 
(The rest of his people would never willingly agree to make him global \apex.)
There were other powerful \apexes, and some had a greater pyramid of vassals than Ramarxes did. 
So Ramarxes did not \emph{look} like the most powerful \caisith in the world.
But Ramarxes understood the Concord better than any other.
He knew many tricks and shortcuts.
So he had a lot of power he could abuse. 

From Ramarxes' perspective, his motives were pure. 
Ramarxes wanted to conquer the other races in the universe. 
He wanted the \caisith to rule the universe\dash and he wanted to rule the \caisith.
He wanted to be the greatest of all, as any self-respecting \caisith would. 
He wanted to be a god, the incarnation of the soul of the \caisith race and their \hs{dweomer}. 

Gradually Ramarxes began to draw on the resources available to him.
It dawned on the rest of the population that Ramarxes was growing more powerful.
By this time the entire \caisith civilization was built around the Concord.
It was virtually impossible to \quo{opt out}. 





\subsubsection{War against Ramarxes}
More and more, people began to distrust Ramarxes. 
Other leaders banded together and demanded that Ramarxes relinquish power.
Ramarxes contended that he was not holding any unlawful power, and that there was nothing to relinquish.

Others began to actively attack Ramarxes and try to force him to relinquish power.
Ramarxes fought back with the weapons at his disposal. 
In doing so, Ramarxes \quo{tightened} the New Concord and utilized its potential for mind control. 
The more Ramarxes fought, the more people began to fear him.
Battle lines were drawn. 
Boundaries were crossed.
It came to war. 

Ramarxes became afraid. 
Afraid of losing everything he had built.
Afraid of being destroyed. 
So he began taking drastic measures.
He delved into dark sorcery. 
As the master of the Concord, he had access to vast resources of knowledge, so he was able to reconstruct ancient outlawed sorceries and perfect them further. 
He descended into darkness. 
And with his power over the New Concord, he dragged the minds of the \caisith race with him down into darkness. 

Horror reigned. 
Many \ophidians succumbed to madness\dash and their madness unleashed alien horrors upon the world. 





\subsubsection{Fall of Ramarxes}
Ramarxes' empire was crumbling, and enemies were at his very door, coming to slay him. 

In his desperation, Ramarxes reached out into the great void searching for power.
He looked to the Umaja Tablets. 
This drove him mad. 

When his enemies found him, Ramarxes was a wretch. 
He was in total despair. 
He laughed madly and made no effort to save himself. 
They slew him. 





\subsubsection{\Sethicus kills Ramarxes}
\target{Sethicus kills Ramarxes}
The team of champions sent to slay Ramarxes included Valcan (\hr{Sethicus}{\Sethicus}) and the parents of \Nexagglachel, \Iscrafel and \Secherdamon. 
Perhaps Valcan delivered the killing blow. 










