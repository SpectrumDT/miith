\section{Overview}
There are three main plot threads in this book. 
Each of them takes place in the dwelling place of one of the three fragments of the \Haskelek{} and deals with the people fighting the \Haskelek. 

\begin{enumerate}
  \item 
    Carzain goes to \Redce{}, trains there, and ultimately goes to fight the \Haskelek{} in \Redce.
  \item
    Lica and Sir James fight the \Haskelek{} in their country.
  \item
    Sentinels and Cabalists battle for control of the Ghost Tower. 
    Ultimately the Sentinels, led by \Nzessuacrith{} and aided by the \Haskelek{} fragment, succeed in driving away the Cabalists, led by \Achsah. 
\end{enumerate}









\subsection{Carzain in \Redce}
Vizicar, at this point, has regained most of his memories from his life as Vizicar. 
He knows he is a Scion, an incarnation of a \malach. 
He knows he is a superhuman badass. 
He is not \human, but something bigger and better. 
Now he wants to break his feeble \human{} shell and reawaken his true potential. 

Early on, he is joined by Shereid. 
She learns that he is a Scion. 
She joins him and supports him, like Mele supporting Rio in \cite{Tokusatsu:Gekiranger}. 
Together they do some research. 
They want to find out how to achieve his \Apotheosis{} and become the Twilight Angel he was meant to be. 
To do this they must work together with the Redcor. 

In the meantime, they discover that something abnormal is afoot in \Redce.
Something evil. 
Or is it? 

The Redcor want Carzain/Vizicar's help to fight this evil. 
(They mostly ignore and discount Shereid.) 
He is reluctant to ally himself with them too closely, but he would like to work with them to learn more. 
So he lets them think he is obeying their orders, all the while manipulating them. 
When he wants to, Carzain can look like a stupid, brash, impulsive boy who is easy to manipulate into doing stuff. 
But Vizicar, who lies and lurks in the back of Carzain's head, is much shrewder and often outsmarts the Redcor. 
(Be careful not to make the Redcor too stupid, though.)

The Redcor know they have a traitor in the higher rungs of their society, someone who leaks their secrets and breaks into their inner sancta. 
Carzain helps them to find this traitor.
But in fact it is Carzain who is the traitor. 

Carzain commands wraiths and other horrid creatures of the Beyond to do his bidding.
He uses them to hide his agenda and misdirect his enemies.
Sometimes he makes his monsters attack him and his companions so that Carzain can be seen saving them from the monster.





\subsubsection{Carzain is a Sephiroth-type}
Remember that \hr{Carzain is Sephiroth}{Carzain is meant to resemble Sephiroth} from \cite{VideoGame:FinalFantasyVII}. 





\subsubsection{Carzain regains \Tydesmos's memories}
\target{Carzain remembers Tydesmos}
During this book, Carzain gradually regains \Tydesmos's memories, \hr{Carzain cannot remember Tydesmos}{which he had otherwise been unable to remember}. 
\Tydesmos was \hr{Tydesmos power}{a mighty and wise dark mage who possessed much Mythos knowledge}.
As Carzain gains these memories, it drives him a bit more mad. 
\quo{Mad} in the sense that he gains a non-\human perspective and begins thinking more like an immortal. 

Like Sephiroth from \cite{VideoGame:FinalFantasyVII}. 





\subsubsection{Carzain as an unwilling ally}
Carzain is unwilling, but somehow circumstances force him to work with the Redcor. 
He struggles to maintain the frame of an independent ally, while they try to turn him into a servant who should rightfully obey. 

I need to make up a good reason for this. 
Perhaps I can use a story similar to that found in the anime \cite{Anime:Dragonaut}, where Kamishina Jin is pulled into the Dragonaut organization without understanding what is going on. 

Also compare to the movie \cite{Movie:MonstersVersusAliens}, where Susan and the other monsters are locked up and treated as shit. 





\subsubsection{Redcor need Carzain}
\target{Redcor need Carzain}
The Redcor needed Carzain. 
They knew that strange, occult, dangerous things were happening in \Redce, and a few suspected that \hr{Belzir awakening}{\Belzir was gaining power}. 
\Esmerel deduced that Carzain was a Scion and decided that having him around (as an ally as well as a study object) would be a great asset, since \Belzir is also a Scion. 





\subsubsection{Carzain uses his Redcor blood to coerce them}
Carzain uses his Redcor blood to coerce the Redcor into accepting him and thus gain access to stuff in \Redce. 
He exploits their snobbery and nepotism.





\subsubsection{Carzain is forced to use violence}
At first, Carzain intends to work inside the system.
He plans to go along with the Redcor while secretly manipulating them to get what he wants. 

Some of the Redcor are nice.
Others are \quo{good} in a somewhat annoying way. 
Yet others are complete assholes and use the rules to bully and oppress people, even Carzain if they can. 
Compare them to Principal Snyder and the Watcher Council in \cite{TV:Buffy}. 

Carzain puts up with it for a while, with all the pride and disdain he can muster.
Eventually, though, he feels forced to resort to violence.
The \hr{Racel dies}{he kills \Racel} in a very Sephiroth-esque scene (\cite{VideoGame:FinalFantasyVII}). 

All the way through the book, Carzain has had a sinister master plan.
This has to do with his \hr{Carzain's Sephiroth epiphany}{Sephiroth epiphany} in \TwilightAngelRememberEmph. 
He has been secretly working towards this dark goal for the whole book, in front of the reader's nose yet behind his back. 
The big surprise comes at the end. 
Carzain shows his true \colours, kills \Racel and a bucketload of other Redcor. 
He unveils what he has been secretly working on the whole time, and does a \trope{FaceHeelTurn}{Face Heel Turn} and goes to join \Belzir. 

Also make him resemble Anakin Skywalker/Darth Vader in \cite{Movie:StarWars:III}. 

Carzain has been contacted by \Belzir. 
She tries to seduce him.
Through this, he learns a little of her plans. 
He tells this to the Redcor. 
Officially, he is horrified by \Belzir's evil and wants to work with the Redcor to stop her. 
Unofficially, he is thinking of defecting to \Belzir's side. 
So he \quo{spies} on \Belzir for the Redcor. 
He tries to figure out what \Belzir is up to.
Whenever he finds out something and explains it to the Redcor, he deliberately misinterprets his data so as to mislead them and cast a smokescreen over himself to cover for his own shady plans and dealings. 





\subsubsection{Redcor suspect Carzain}
Soon after Carzain arrives in \Redce, the dangerous occult occurrences seem to escalate. 
Things get worse than before. 
The Redcor suspect that there is a mole who is betraying the Redcor from the inside and killing them and disrupting their spells and leaking information and resources to the evil rebels (whoever they are\dash the Redcor have not identified the enemy).  





\subsubsection{Carzain crippled}
\hr{Ramiel crippled}{Carzain/Ramiel} becomes more and more crippled as the story progresses.









\subsection{Redcor and the underworld}
Redcor tradition dictated that the underground should be kept secret and sealed, for it was the \vclan's sacred task to keep this evil imprisoned and hidden, lest its foul influence spread evil and madness and corruption all over the world. 

But the threat grows. 
The Redcor are haunted by evil dreams and madness. 

Some believe the right solution is to block up the underground with even more stones, more spells, more prayers and more denial; essentially rely on tradition and faith. 
Others believe that his path is not viable. 
The corruption is too insidious, and it will keep seeping out, especially because the Redcor do not understand it and hence cannot defend against it properly. 
Instead they must dare to explore and research and send expeditions down in order to find out what is going on, what the nature of this evil is and how they can fight it. 

Some Redcor, especially \hr{Kimon}{\PatriccoKimon}, believe that they should study the threat some more because it could teach them important things, deep insights about the nature of the world, \humanity, good and evil, \iquin and \itzach. 
Most Redcor believe that \Kimon is a dangerous heretic. 
The conservatives argue that it is imperative that they \emph{not} try to \quo{learn new insight}. 
They argue that \ClanRedcor already knows all they ought to know about the nature of man and the One Light and Outer Darkness. 
\Kimon's plan is madness and will only bring corruption. 
It will turn out that they are sort of right: 
\Kimon, in his eagerness to challenge tradition and learn new insight, he ends up a \trope{XanatosSucker}{Xanatos Sucker} and helps the evil forces (including Carzain) destroy \Redce. 

Many in \Redce can feel the corrupting evil in their minds. 
They pray for forgiveness and prepare for a religions doomsday.
Later \Belzir tells Carzain that these people are right: 
The Eschaton really is near. 
The walls of the world are about to crumble, and horror and madness will pour into the world from Beyond. 

Some commit suicide from the strain and horror. 
Even high-ranking Redcor. 
(These might be \trope{AssholeVictim}{Asshole Victims} or \trope{KarmicDeath}{Karmic Death}, but they might also be tragic victims.) 
It is very shocking to the Redcor that even high-ranking wise Vaimons (who should be strong-willed and protected by the One Light) can succumb to corruption and despair and take their own lives. 
Other Redcor and civilians begin to pray extra hard. 

\Humans, when they encounter the \banes or feel their presence, recognize something deep within themselves that resonates with the alien things. 
This is the ultimate horror.
Even Carzain feels this. 
It is part of his \hr{Carzain is Sephiroth}{Sephiroth-style awakening}. 

Occasionally there can be heard or felt rumbling noises or tremours from below, like vast, slimy footsteps. 
Compare to \cite{AugustDerleth:QuestforCthulhu}. 

\citeauthorbook[p.199-202]{TimCurran:Hive}{Tim Curran}{Hive}{
  The idea of getting bit wasn't what bothered him, it was the idea of the venom itself.
  And the sort of venom he might get stuck with in those blasphemous ruins was the sort that could erase who and what he was and birth something invidious and primal implanted in his genes a hundred-thousand millennia before. 
  
  \tho{You don't know that, you really don't.}
  
  Yet, he did.
  Maybe whatever it was had hid itself in the primal depths of the \human psyche, but it was there, all right.
  Waiting. 
  Biding its time.
  A ghost, a memory, a reventant hiding in the dank and dripping crypt of the \human condition like a pestilence waiting to overtake and infect.
  A cursed tomb waiting to be violated, waiting to loose some eldritch horror upon the world.
  An in-bred plague that festered in the wormy charnel depths of the subconscious, waiting to be woken, activated by the discordant piping of alien minds. 
  
  Dear Christ, there could be not nothing as horrible as this.
  
  Nothing. 
  
  He did not and could not know the ultimate aim of awakening the sleeping \dragon the Old Ones had implanted in the minds of men\ldots  but it would be colossal, it would be immense, it would be the end of history as they knew it and the beginning of something else entirely.
  The continuation of that primordial seeding, the vast outer extremity of that tree, the ultimate objective. 
}

\citeauthorbook[p.227]{TimCurran:Hive}{Tim Curran}{Hive}{
  \thought{%
    Christ, look at that old ice and what it holds.
    Like every dark and nameless secret of antiquity is locked up in that frozen sarcophagus.
    All of mankind's primal fears, cabalistic myths, and evil sorceries given flesh.
    The archetype that inspired every nightmare and twisted racial memory, every witch-tale and every legend of winged demons.
    All the awful, unthinkable things the race had bred and purged from the black cauldron of collectiv ememory, all the obscene things it could not acknowledge nor dare admit to\ldots it was here.
    This horror.
    The engineer of the race and of all races.
    And it had been waiting down here in the eon-old ice.
    Waiting and waiting, dead but dreaming, consciously forgotten but grimly remembered in the subconscious and dark lore of \human{}kind.
    But all along, they were dreaming of us just as we dreamed of them\ldots because they were us and we were them and now, dear God, millions upon millions of years later, they were waking up, they were rising to claim their children and their children's intellect. 
  }
}





\subsubsection{Gateway to \Erebos}
The ruins underneath the \TopazChateau gradually transition to the underworld of \Erebos. 
When the heroes venture down there they find bits and pieces of \Erebos. 

\citeauthorbook[p.223]{TimCurran:Hive}{Tim Curran}{Hive}{
  \ldots much of ti was nothing but huge boulders, some of them as big as two-story houses, lots of loose rocks and stacked wedges aof sandstone.
  But not all of it was of natural origin, for there were other shapes down there, ovals and pillars, assorted masonry that had been cut into those shapes.
  
  And there was no doubting where it had come from.
  
  For to either side of the gully, they could see the remains of the ancient city climbing up sharp slopes into the murk above.
  It was enormous, what they could see of it, for it climbed much higher than their lights could reach.
  A sleeping fossil, a mammoth city from nightmare antiquity.
  
  \ldots 
  
  Like everything about the Old Ones, this city\ldots it lived in the race memories of all men.
  And there was nothing remotely good associated with it.
  Just horror and pain and madness. 
}





\subsubsection{Corpse}
Exploring the cellars they uncover the corpse of one of their own.
It may be Vitor Bercerac, Carzain's enemy. 

\citeauthorbook[p.228]{TimCurran:Hive}{Tim Curran}{Hive}{
  He was curled up in sort of a fetal position, knees to chin, his face white as new snow and contorted into a grimace of absolute horror.
  Blood had trickled from the rictus of his mouth.
  His eyes were spilled down his cheeks in gelatinous trails like squashed jellyfish. 
  
  \ldots 
  And that death had been a dark matter, mindless and perverse and ghastly. 
  No man should have had to go like Gates did\ldots alone and mad in that suffocating darkness, dying a crazy and hopeless death like a rat stuck in a drainpipe.
  Screaming as his eyes boiled to soup and splashed down his face.
  As his brain went to sauce and his soul was burnt to ash.
  
  Gates had paid the final price for his curiosity. 
}









\subsection{Sentinels}
Meanwhile, \Narkiza-tachi are advancing north. 
They have conquered some stuff. 
They want to control more land. 
They need some strategic locations in southeastern \Velcad{} in order to resurrect \Belzir. 

See, the Sentinels have been communing with \Belzir. 
They want her back. 
She has grown to hate the Cabalists.
If she returns, she will want to take revenge on them. 
But they also know that she is a \Mystraacht{} patriot. 
So maybe, if she returns, she will be able to go to \Mystraacht{} and conquer it. 
This will leave a third of the Cabal's strength under the command of a maverick with a giant grudge against the rest of the Cabal. 
The Sentinels want this. 
So they are supporting \ps{\Belzir} \quo{Royalist Faction}, who work to see her restored. 

But, but, but. 
The Sentinels must not be seen as actively supporting \Belzir.
That would be bad. 
Instead, the Sentinels act as if they know nothing of the Royalists and are just pursuing their own gambits.
Then, by sheer chance, they \quo{unwittingly} end up accidentally helping the Royalists achieve their goal. 
The Sentinels obstruct the Cabalists and Redcor who were trying to obstruct the Royalists, and so the Sentinels \quo{unwittingly} end up accidentally helping the Royalists achieve their goal. 

(Oh, yes. 
\ps{\Secherdamon} \trope{XanatosGambit}{Xanatos Gambits} can be complicated.) 

The Sentinels now have \Nithdornazsh{} set up. 
So far, so good. 
This will allow \ps{\Vizsherioch} to spread his \vertex{} influence and become the \hs{Dagger}. 

Meanwhile, the Royalists are abroad. 
They have to solve some Aenigmata in order to free their queen. 
The Sentinels covertly help them do it. 

The Rissitic invasion also drains Cabal and Redcor resources away from \Redce, where a vital piece of \Belzir{} lies. 
This means that when the \Redcean{} Royalist infiltrators strike, there are few Vaimons to oppose them, and mostly weak ones. 
(Obviously this casts no suspicion on the Sentinels. 
 They have plenty of good reasons for wanting more land in \Velcad.
 No one would suspect them of secretly supporting an obscure cult serving a rogue \resphan.) 





\subsection{Royalists}
The Royalists do not have a fixed plan. 
In order to bring \Belzir{} back they need the blood (or, more generally, lifeforce) of a \resphan. 
This is not so easy to come by. 
\Shiaraid{} \hr{Shiaraid unpopular}{does not have many immortal allies}. 
So \quo{lots} of Royalists are out in the world, questing, seeking for a source of high-quality immortal blood or lifeforce. 

One group has the mummy of a powerful \quiljaaran{} king, unearthed from an ancient tomb. 
They think this mummy might contain some magic that will help them, so they are bringing it to their headquarters. 

Then Ramiel appears.
In Pelidor he \hr{Ramiel scares Nzessuacrith}{semi-unconsciously unleashes his \sathariah{} power} and scares \Nzessuacrith. 
\Shiaraid{} immediately detects and recognizes him. 
She knows Ramiel better than any other and can recognize his \vertex{} signature in a moment. 

This is a godsend for \Shiaraid. 
For centuries she has been searching for a way to come back to life. 
Now her old lover is suddenly reborn. 
This is an opportunity she cannot afford to waste. 
She must secure his alliance. 
She then sends some Royalists hunting for Ramiel (including Shereid). 

In addition to Ramiel's help, \Shiaraid{} also needs a strong link to the Midnight Bat \matrix. 
This is where \hr{Sentinels help Shiaraid with Ghost Tower}{the Sentinels can help}. 





\subsubsection{Breakthrough}
\target{Royalist breakthrough in Redce}
\Belzir \hr{Belzir keeps in touch with Royalists}{kept in touch with her Royalists}.
They drilled holes in her prison so \hr{Belzir awakening}{she could gradually awaken}. 

Maybe the book should open with a scene that shows the Royalists drilling an important hole. 
They had now drilled so many holes in \Belzir's prison that she could now also contact the non-Royalists in dreams.
She began to be able to use mind-controlling magic on the Redcor. 

This was why \hr{Redcor need Carzain}{the Redcor needed Carzain}. 





\subsubsection{\Banes}
The Royalists may have \banes on their side. 
\Banes are nasty because they have nasty powers. 
They can see and move through the Beyond, and as such they can spy on people from Beyond, sneak through impossible crevices, walk through walls and even \quo{disappear} by \hs{submerging} or pounce from nowhere by \hs{surfacing}.

The \banes tempt and lure people into their clutches using mental attacks, then corrupt and mind-control them. 
The \humans can feel that their nature is connected to these monsters. 
It is an awful thought, but also alluring and fascinating. 
Horrified, the victims cannot look away. 
The thought makes people doubtful and afraid and even desperate, which makes it all the easier for the \banes to manipulate and mind-control people. 
The people thus touched sometimes go mad afterwards. 

Compare to how the Elder Things drive people mad and control them in \cite{TimCurran:Hive}. 

There are mostly \lesserbanes, but also a few \greaterbanes. 

Those \humans who are really badly possessed look awful. 

\citeauthorbook[p.199--202]{TimCurran:Hive}{Tim Curran}{Hive}{
  \ldots at that moment it would have been hard to picture a more dangerous man than Holm.
  There was something cold and remorseless about him.
  
  \ldots
  
  Holm was looking at him and his eyes were filled with a chill blankness-
  There was nothing in them.
  Nothing \human at any rate.
  He surveyed Hayes with a flat indifference, that pallid face punched with two black eyes that made something go liquid in Hayes' belly.
  You didn't want to spend too much time looking into those eyes.
  They wer elike windows looking through into some godless, deadEnd of space.
  You could see yourself there, suffocating in that deranged, airless void. 
  
  Hayes swallowed.
  
  Those eyes drilled into him, sucking him dry.
  
  There was power in those eyes, something immense and malignant and ancient.
  The way Hayes was feeling at that moment was how he felt looking tino those glassy red orbs of the aliens in Hut \#6. 
  They got inside you, owned you, crushed your free will like a spider under a boot.
  At some primary level, they consumed and swalled you.
  And you could feel all that you were sliding down into some black soundless gullet. 
  
  \ldots
  
  Holm looked up at them with that same almost insipid blankness.
  His black eyes like those of a grasshopper consider a stalk of grass.
  That's how they looked\ldots unintelligent, completely vacant.
  At least at the moment.
  But Hayes knew those eyes and what they could do.
  One moment they were dead and empty, the next overflowing with all the knowledge of the cosmos. 
}





\subsubsection{\Bane statues}
The Redcor discover statues of \banes. 
The statues are small, stylized but gruesome. 
The worst part is the faceless head. 

The statues are elastic and seem to be made of flesh, which is a horrible realization. 

The statues mess with people's heads and make them crazy or mind-control them. 
The statues are also useful for transforming people into \banes. 
(\Banes have to \hr{Banes possess Humans}{come to \Miith through \human bodies}.)

People who find the statues will find themselves dreaming or hallucinating about the aeons-old prehistory of the \banes, how they waged wars of destruction against \dragons and other monsters, how they created \humanity as slaves or toys.





\subsubsection{\Lithrim}
The \humans see hints of their true nature, as parts of \hr{Lithrim}{\Lithrim}. 
(Read about \Lithrim.) 
They envision themselves as \banes:
Small slimy cogs in a vast, hideous, faceless machine-hivemind of pure evil. 
(\Banes are, after all, a thing of the \hs{dead universe}.)





\subsubsection{\Belzir's skull}
The Redcor have \Belzir's skull or something.
She needs it if she is to regain a physical body. 
The Redcor know this, so they keep it safe in some vault. 

Carzain learns about the legend of the skull.

Then the skull gets stolen.
Carzain gets hired to help find it. 
Much of the book is spent searching for the skull. 

There is a scene where the thief goes to deliver it to his employer.
The employer takes the skull and sends the thief away.
The thief is scared, maybe even killed. 

In the end it turns out that Carzain has the skull.
It was he who masqueraded as the thief's employer and took it from him.
Now Carzain goes to Geica with the skull.





\subsubsection{Shereid}
\hs{Shereid} is the first Royalist to find Ramiel. 

She wants to \quo{recruit} him and make him ally himself with \Belzir{} and the Royalists.
But Shereid cannot reveal her identity too soon. 
If Carzain were to refuse her offer, she would be in danger. 

She and Carzain initiate a subtle dance of evasion. 
He (Vizicar) quickly picks up on the fact that she knows something she is not saying; that she is more than she appears. 
He tries to ferret it out of her. 
But she is a skilled deceiver herself. 
She does not want to tell him before she is sure he is on her side. 

So she tries her best to turn him against the Redcor and get him to like the idea of a Geican revolution, and personal power and glory and sex. 
She talks about how bad the Redcor are and how great it would be to be free of them, to see how he reacts. 
He tentatively agrees, but he is also wary of her and does not tell her too much. 
It is a delicate game of mutual distrust. 

At last she lets slip to him the possibility that \Belzir{} is still sort of alive. 
This makes Vizicar very interested. 
He very much wants to meet this other Scion and exchange knowledge with her. 
He has read some claims that she achieved \apotheosis. 
This idea gives him a boner. 

But he still does not trust Shereid. 
Carzain and Vizicar have an internal dialogue about how they will not let her control them. 
I should also have plenty of scenes where he criticizes her\dash to her face, to others or inside his head. 
This is meant to fool the reader into thinking that he does \emph{not} want to go along with Shereid's ideas. 

But in fact he does. 
He likes Shereid. 
Much better than the Redcor. 
So, in the end, he betrays the Redcor (including \Racel) and goes with Shereid to Geica. 





\subsubsection{Tentacle rape}
Shereid or Needle is there. 
She serves the \banes.
Every night she feels like she gets raped by tentacles. 
It is awful and painful, but she craves it and cannot live without it.
She is already mad and a pawn of dark forces. 





\subsubsection{\Belzir gets a body}
\Belzir possesses the body of a Redcor\dash a woman of 50 or so, but still healthy and beautiful. 
She is mean and evil, like the evil Faith in \cite[season 3]{TV:Buffy}. 
At the end the reader should really hate her. 
It is an awful shock for the reader when Carzain betrays his allies and joins \Belzir, but the reader gets some measure of closure when Carzain also betrays \Belzir. 
(Unless he doesn't betray her yet and waits for the end of the next book.)









\subsection{Underground cult in \Redce}
\target{Cult in Redce}
In \Redce{} there is an underground cult of hopeful, gullible youths who worship dark powers and hope for some kind of reward or insight. 
It is very much an ecstasy cult, with sex-orgies. 
The young people get their sexual urges satisfied, and more so, with drugs and dark magic. 
Maybe the dark powers feed on their sex. 

Have sexy priestesses and erotic rituals, including live sacrifice. 
The sacrifices may be unwilling captives, or willing, or mind-controlled. 
It works best, I suppose, if it's sexy girls who willingly let themselves be killed in a sexual ritual of pleasure and pain. 

The sexy priestesses have power, but they ultimately crawl at the feet of the master of the cult. 
The master is a badass gangster, like Lex from the movie \cite{Movie:GargoylesRevenge}. 

\lyricslimbonicart{Twilight Omen}{
  I salute thee, baptizer of my soul.\\
  Dear pagan master, let thy universe unfold.\\
  Show me the sign of the midnight sky.\\
  I will forever follow until dawn's early light.\\
  From the ashes a fire shall be woken.\\
  From the shadows words shall be spoken.\\
  A star constellation, \\
  a burning circle of serpent eyes.\\
  Esoteric mysteries of unknown life.
}

The Cabal actually know about the cult. 
At first, they allowed it to exist.
Back then, it was clear that the cultists were not Sentinels.
The Cabalists did extensive research and found \emph{no} links from the cultists to any Sentinel-aligned \matrices. 
In fact, the cult's magic was not even real, just smoke and mirrors. 

Moreover, the cult serves a useful purpose as an outlet for the people's evil nature. 
The Cabal knows that people have evil lusts. 
The Redcor try their best to suppress those lusts. 
For some people, the internal pressure becomes unbearable and they must have an outlet. 
From the Cabalists point of view, this underground cult is as good an outlet as any. 
So they let the cult exist, but keep it under surveillance. 

(Note that only the Cabal know about the cult. The Redcor know nothing.) 

For a while this went fine for the Cabal. 
But then, unbeknownst to them, \ps{\Shiaraid} Royalists found out about the cult. 
In deepest secrecy, she had her people infiltrate the cult. 
Now she has quite a lot of influence in the cult. 
She does not rule the cult, but she has enough power to use the cult for her purposes, as long as she is crafty. 
She can run her own business in the shadows, and whenever her people are discovered she can frame the cult and have it take the heat for it. 
The cult is not so well-organized, so the cult leaders and the Cabal spies do not have an overview of everything the cult members do. 
\Shiaraid{} herself encouraged this infighting and chaos and division, because she knew that chaos and outlawry would be the perfect milieu in which for her to work her shady schemes. 

The Redcor have no idea that \Shiaraid{} is back. 
To their minds, she is an ancient enemy they defeated millennia ago, and (despite what they may say to frighten children and outsiders) they do not suspect she will come back to haunt them any time soon. 

Under the cover of a \quo{harmless} ecstasy cult, \Shiaraid{} has sent her spies crawling all over the place under the \TopazChateau{}. 
She knew the Redcor had a piece of her soul prison in their keeping. 
Now, after decades of spying and searching, she knows where it is. 
But she cannot get at it. 
She does not have any contacts in the \Chateau{} that are sufficiently high-placed or sufficiently powerful. 

Hence, it is extremely fortunate for her that Carzain is now in \Redce. 
If she can get him converted to her cause, he could help her get her soul-jar thingy from the Redcor. 





\subsubsection{Scenes}
Have erotic scenes, like in \cite{GaryMyers:TheHorrorShow}, where a hot girl is stripped naked, tied up, beaten halfway to death and then sacrificed. 
In the end, she changes her mind and screams and begs for mercy, but in vain, and she is devoured. 
All the while, the cult chant their songs to their dark gods. 









\subsection{\Kezerad}
The \Kezeradi help the Redcor fight the \banes. 
They appear as tragic, tortured, weeping angels. 
Full of suffering but still brave and enduring and fighting to save others. 
They want to stop \Belzir. 

Remember that the \Kezeradi mourn the loss of their \hr{Beacons of Kezerad}{\beacons}. 
The \beacons have been taken from them and given to these mortal Vaimons. 
But do not mention that \iquin is evil. 

\target{Sithiyacaan in Redce}
\Sithiyacaan is in \Redce, in the guise of a \human man named \hr{Herette}{\MoriceHerette}. 
(Read about \Herette.)

When Ramiel destroys \Redce \Sithiyacaan finally sort of awakens. 
He \hr{Sithiyacaan goes north}{goes north after him}.










\subsection{Ghost Tower}
\target{Sentinels help Shiaraid with Ghost Tower}
In addition to Ramiel's help, \Shiaraid{} also needs a strong link to the Midnight Bat \matrix, and through it the \Erebean{} \dweomer. 
Ramiel cannot provide this. 
His \kenosis{} weakens his link to the \matrix{} and \dweomer. 

But the Sentinels can. 
The \hs{Ghost Tower} is a link from \Azmith{} to \Nyx, remember. 
\Secherdamon-tachi plan to conquer the Tower, then \cooperate{} with \Shiaraid{} and Ramiel to cast a complex spell that will penetrate through the Tower into \Nyx{} and tap into the Midnight Bat and the \dweomer, then tap the energy and channel it through the Tower, through a complex \quo{series of tubes} devised by \Vizsherioch{} and built with assistance from the Royalist Faction, all the way from Pelidor through the Beyond and into Geica, where \ps{\Belzir} body is. 
This energy, combined with Ramiel's help, will let \Shiaraid{} rise from the dead. 

But there is a problem. 
This could be detected. 
If the Sentinels start siphoning energy from \Nyx, everyone will know. 
And the Sentinels still do not want it known that they are in bed with \Shiaraid. 
So they do something sneaky: 
They make it collide with another long-term plan of theirs, namely that of \hr{Vizsherioch becomes Shaeeroth}{raising \Vizsherioch{} to \shaeeroth{} status}. 
They will perform both \ps{\Vizsherioch} \shaeeroth{} ritual and \ps{\Shiaraid} resurrection ritual at the same time. 
Most of the energy will be consumed by \Vizsherioch, but some of it will be siphoned away and used by \Shiaraid. 
That way, if this is discovered at all, it will look as if some clever Royalists infiltrated the Sentinels and installed a \quo{back door}, stealing some of the Sentinels' energy and siphoning it away to use for their own nefarious purposes. 
No one will ever know. 

Of course, in order for the Sentinels to accomplish any of this, they must first capture the Ghost Tower. 
This is not easy, for the \resphain{} have turned it into a bastion and are fighting to defend it. 






\subsection{Telcastora Ilcas}
Telcastora Ilcas is in \Redce with Carzain and the rest. 
He is braver and more efficient than most Redcor. 
Partly because he does not have their restrictive religion and taboos, and partially because as a \scatha he is resistant to some \human-specific mental attacks and instinctive fears. 

He is \hr{Ilcas badass}{a badass Tisamon-type character}.

The Redcor suspect Ilcas of being the mole because he is an outsider, a heathen. 
And also arrogant and aloof to boot. 
Besides, he is a \scatha, where the Redcor bosses are \human. 
(Though \hr{Redcean demographics}{there were \scathae in \Redce}.)

\target{Ilcas suspects}
The Imetrium is taking an active interest in \Velcad. 
They are ancient rivals of the Rissitics and do not want the Rissitics to conquer and gain too much power (be it secular or metaphysical power). 

Telcastora Ilcas escorts Carzain-tachi to \Redce. 
Then he turns back to southern \Velcad. 
There he works with other Imetrians and allies to oppose the Rissitic plan. 

He has to deal with the sinister \nagae{} that are the allies of the Imetrium. 
Gradually, he discovers more about the Imetrium's involvement with \nagae{} and other wicked powers. 
He becomes slowly disillusioned with his religion. 
He is learning \hr{The truth about the Imetrium}{the truth about the Imetrium}, and it is a darker truth than he would have liked. 

Gradually he turns away from his commanders, and even his gods. 
But only a little bit. 
He has served the Imetric gods all his life. 
He remains true to the ideals he has always upheld. 
But he becomes more critical and distrustful. 

From the Imetrians' point of view, Ilcas serves an important purpose in this battle. 
His sword, \Telderain, is a powerful \vertex{} in the Imetric \matrix. 
So when they find out he has gone prancing off to \Redce, they quickly send for him and fetch him back to southern \Velcad{} where he is needed to form part of the metaphysical bulwark against \ps{\Secherdamon} Sentinels. 

(Note that the Imetrians are not necessarily opposing \emph{all} Sentinels. But they are certainly opposing \Secherdamon.) 

In the process, Ilcas will encounter \Narkiza. 
They find out they have a lot in common. 
But they are destined to be enemies. 









\subsection[Mystraacht]{\Mystraacht}
Some of the \Mystraacht{} have learned that Ramiel is alive. 
They want to kill him, \hr{Mystraacht rival goes after Ramiel}{so they hunt him}. 
But \hr{Azraid protects Ramiel}{\Azraid{} protects Ramiel}, as \hr{Ramiel meets Cishiel in dreams}{does \Cishiel}. 





\subsubsection{When does Ramiel meet \Cishiel?}
\target{When does Ramiel meet Cishiel?}
When does Ramiel meet \Cishiel?
There are some possibilities. 

\begin{enumerate}
  \item 
    \Cishiel{} should not discover that Ramiel lives until late in the story. 
    She and \Dasteron{} need to be introduced before that point. 
    
    In fact, maybe \Cishiel{} should be postponed to the next book. 
  \item 
    If I present the backstory of \Cishiel and \Dasteron \hr{Flashback with Cishiel and Dasteron}{as a flashback}, I can introduce \Cishiel at any time I wish. 
    She could \hr{Ramiel meets Cishiel in dreams}{appear in Carzain's dreams}.
\end{enumerate}












\subsection{Rissitics}
We follow \Narkiza{} and his Rissitic army, who are fighting their way through southern \Velcad{} in a campaign of conquest. 

In the \Narkiza chapters, make it clear that the \hs{Rissitics value their lives}. 

Their attack is partially meant as a diversion, drawing Redcor and Cabal attention away from more important matters such as \Nithdornazsh{} and the Royalist Faction's resurrection project. 

But the invasion also serves a useful purpose. 
\Secherdamon{} knows (or suspects) that the final breaking of the Shroud is near. 
He knows \iquin{} has a Hell of a lot of power. 
He wants to try to break that power. 
To this end, he wants to destroy the Redcor. 
He hopes to enlist the help of \ps{\Shiaraid} Royalists. 

The plan is that the Rissitics will move in to attack some Iquinian key strongholds and nations. 
If these fall, it will cause Iquinianism to falter on an \Azmith-wide scale. 

The Redcor know this, so they will set in as many resources as they can to fight off the Rissitics. 
This does not merely mean sending in a bunch of Vaimons. 
It also means calling in favours, pulling strings, drawing deep of the Redcor credit coffers and bullying allied rulers into banding together and fighting the Rissitic menace and defending the Iquinian world.

Meanwhile, the Royalists will resurrect \Shiaraid. 
This will be done in some place relatively close to \Redce. 
\Shiaraid{} will harbour a great hatred against the Redcor and will want to take revenge on them. 
With her \sathariah{} powers, her loyal Royalists and her covert Sentinel backing, she should be able to wreak a lot of havoc on the Redcor, which would seriously harm the Iquinian church. 
(And, with luck, this will not implicate the Sentinels at all and thus not expose \ps{\Secherdamon} long-term plan.)

If the Iquinian church can be destabilized, it will harm the global Shroud of Civilization that the church exerts over \Azmith. 
This will allow \Secherdamon{} to use gain more power in \Azmith{} and use the Realm to complete the forging of his Dagger. 





\subsubsection{\Secherdamon{} and Ramiel}
\Secherdamon{} does not know that Ramiel is back. 
And \Shiaraid{} does not tell him, even though they do communicate every once in a while. 
She keeps Ramiel a secret from \Secherdamon.

\Secherdamon{} discovers the secret near the end of the book, though. 
When someone helps \Shiaraid{} rescue her stuff from \Redce, Ramiel releases some powerful energy. 
\Secherdamon{} has been paying attention to what \Shiaraid{} is up to\dash she is his ally, but he trusts her only as far as he throw her\dash and this sudden burst of new energy interests him. 

\Secherdamon{} sees clear signs of activity in the Midnight Bat \matrix, and he hears some rumours. 
He puts two and two together and deduces that she has an ally who is a \resphan{} in disguise\dash perhaps a Scion. 

\Secherdamon{} or \Vizsherioch{} then talks to \Shiaraid. 
She is mentally unstable and has a hard time concealing her feelings, so \Secherdamon{} begins to suspect that her old lover, Ramiel, is back. 

At last the cat comes out of the bag. 
Ramiel kills \Shiaraid{} during her would-be resurrection. 
He can no longer hide his existence. 

(At this point, Ramiel knows who he is. 
 Maybe he has even taken contact with \Cishiel{}. 
 But maybe not.
 See the section about the question of \hr{When does Ramiel meet Cishiel?}{when Ramiel meets Cishiel}.) 

\Secherdamon{} is nonplussed and \hr{Secherdamon wants to off Ramiel}{wants to off Ramiel}. 









\subsection{\Ishnaruchaefir and \Azraid}
\target{Ishnaruchaefir and Azraid plot together, early in TBW}
Have a scene in the beginning of this book or the next.

\Ishnaruchaefir and \Azraid meet (\hr{Ishnaruchaefir and Azraid develop empathy}{as they have done before}). 
Gradually and ever so subtly through \SentinelsofMithEmph, \Ishnaruchaefir becomes convinced that \Azraid is on the level, and he also begins to suspect the general nature of \Azraid's plan. 

So when Azraid finally \hr{Ishnaruchaefir and Azraid plot together, late in TBW}{approaches \Ishnaruchaefir and asks for his cooperation in a secret venture}, \Ishnaruchaefir has expected it. 









\subsection{\Banes and \xss}
I should gradually build up suspense between the \banes and the \xss throughout the whole series. 
In the beginning, the players are smaller forces.
Gradually the reader learns more about the big picture. 

Make it slowly come clear that \Miith is caught between Scylla and Charybdis, and that \Miith's walls are slowly crumbling. 
Beyond the mundane wars of the \Miithians, beyond even the age-long war of the \dragons and \resphain, greater menaces threaten. 
Madness and horror are poised to pour into the world like a destructive tidal wave from the endless darkened voids Beyond. 



















\section{Prologue}









\subsection{The Fall of \Kezerad}
\target{Sithiyacaan despairs after Fall}
Have a flashback to the \hr{Fall of Kezerad}{fall of \Kezerad}, or the time just after it. 
\Sithiyacaan is desperate and cracking apart with sorrow. 
His world has crumbled, the \beacons are lost, \Essenai is destroyed (or so he thinks). 
He rages and raves while his brethren try to make him pull himself together. 
They know it is a catastrophe for morale if their great, brave leader breaks down into a wreck. 
But \Sithiyacaan is caught in the claws of \NexagglachelsCurse, which makes him extra vulnerable and mad. 









\subsection{Carzain is born}
Have a flashback to the day Carzain was born. 
Or, rather, the day he was conceived and became a foetus.

Before his birth, Carzain is a disembodied, barely conscious thing. 
In this state, he is less constricted by the Shroud. 
He can see beyond it.
He is also less afflicted by his usual \malach amnesia.
He knows and remembers his true self (in a semi-conscious way). 

He is an angel.
A god. 
He is the lord of tons of souls (in his \carcer).
He is the king of the world. 
He remembers his power as a \sathariah and a \malach.
Limitless power is at his fingertips. 

He is free of the Shroud. 
He is one with the universe. 
He can see and touch the entire Cosmos.
All bliss and glory is his. 
Everything is beautiful and perfect. 

But then it is torn away. 
He snatched out of his blissful rest. 
He is squeezed and hammered and compressed into a tiny, weak, constricting body. 
He is cut off from his divinity and immortality and power. 
He feels his mind slipping away.
His memories are being taken from him. 
He cannot remember who he is.
He can feel only loss. 
He is bewildered. 
He is afraid.
Desperate. 
His very being is being torn from him, and he can do nothing about it. 

He is blind. 
Mindless. 
Drowning in wet slime. 
He cannot breathe.

Flash forward to his birth.

Carzain is pushed out of his mother's womb. 
And suddenly he feels cold. 
There is blood.
A voice screams.
It is his own. 

Change POV to his parents. 
Nishain comes in. 
The midwife tells him it's a boy. 
Nishain touches his son and speaks his name: 
\ta{\CarzainShireyo.} 



















\section{Carzain in \Redce}









\subsection{Carzain-tachi go to \Redce}
\target{Carzain goes to Redce}
Carzain-tachi go to \Redce. 





\subsubsection{They meet Shereid}
They meet Shereid. 

Carzain notices her \hr{Geican green eyes}{green eyes}. 
He \hr{Carzain wants green eyes}{wishes he had green eyes}. 




\subsubsection{\TopazChateau}
They see the \hr{Topaz Chateau}{\TopazChateau}.
It looks as though it is really made of topaz.
Vizicar remembers his old Rainbow Palace, which was also crystalline but even more awesome. 









\subsection{Alliance with the Redcor}
\target{Carzain with Redcor}
At length, Carzain reaches \Redce{} and joins up with the Redcor. His mother is Redcor-born, and her name, \Deracille, is a Redcor name of the lesser nobility, so Carzain, grudgingly acquiescing to using his mother's last name, is accepted. He is trained as a Gandierre by Lacquasse, whom he comes to like and respect. 

He quickly learns to distrust the Redcor and not submit to them too easily. When his training as a Gandierre apprentice is complete, he is released from his bonds as an apprentice, and it is assumed and expected that he will take his vows as a Gandierre. But Carzain surprises them all by refusing to join the Gandierre. He has studied the Redcor laws and found a loophole, saying once free of apprenticeship, he has no more duties to the Redcor. However, he graciously offers to remain as their ally. The Redcor are less than happy about this, but they have to accept. 










\subsection{\Redcean climate}
Remember what the \hr{Redce climate}{\Redcean{} climate} is like. 
This will influence the story. 









\subsection{\Esmerel has won a victory}
When \Esmerel{} brings Carzain back with her to \Redce{} and can prove that he is a Scion, it is a professional victory over some of her rival scholars.
Many of her colleagues refused to believe her theory that the \spike{} was a Scion. 
There were many other reasonable explanations, after all. 
But \Esmerel{} was a bigger Scion expert than any other, and she was certain her theory was correct. 
Now she has proven it. 

Carzain and Vizicar are not happy to be paraded as her trophy. 









\subsection{Carzain was wrong about \Esmerel}
Carzain sees \Esmerel{} in a new dress: Much lower cut, showing off a bit of cleavage. 

He tells her: 
\ta{\Matron{} \Esmerel, I was wrong about you. 
I understand now that underneath your gruff, arrogant and bitchy exterior, you are actually a real woman\ldots{} with a pair of very nice boobs.}









\subsection{Carzain is secretive}
Throughout the book, Carzain is being secretive and furtive and suspicious. 
He is often absent at critical times and refuses to tell people where he has been, or tells partial truths, or lies. 
Later it turns out that he is keeping mistresses. 
He has been sneaking out and having sex with them, and that is why he is so often absent. 

This is part of his plan.
He has seduced several Redcor women and thus forced them to help cover for him and keep his doings secret. 
He also compels them to do things for him that help his secret plans.
His mistresses never learn much about what he is up to.

Ostensibly Carzain is doing it for the sex. 
In reality he is just using the women as tools. 
They are his alibis while he conducts his shady business.
He is manipulating everyone and setting up the stage so he can betray the Redcor, help \Belzir and then later betray \Belzir as well.

Along the way Carzain acts as a detective and unearths other plots and secrets. 
He appears to be helping the Redcor, but he always has a hidden agenda.









\subsection{Carzain gains an enemy}
Carzain is placed as a subordinate to a Redcor. 
She bullies and humiliates him, treating him like a stupid child. 

He tells her: 
\ta{%
  You have gained an enemy. You have me outnumbered today, but know this: 
  One day I will kill you.} 









\subsection{Carzain sees the \Morbus{} in \Redce}
\target{Carzain sees the Morbus in Redce}
Carzain walks around in \Redce. He ventures into the slums, where he sees the \hr{Morbus}{\Morbus} at work.







\subsection{Look at the zwongas on that one}
Carzain is in \Redce{} with a friend. 
A girl with very large boobs walks by. 

The friend exclaims: \tho{Look at the zwongas on that one!}

\target{zwonga}
\target{zwongas}
A \quo{zwonga} is actually an oriental fruit, a soft thingy like a very large plum. 
It is also slang for a woman's breasts, because a zwonga feels kind of like a breast. 









\subsection{I have zwongas}
Carzain is at a ball in \Redce. 
A serving girl comes up to him. 

Girl: 
\ta{Good day, sir. 
I have zwongas. 
Very nice. 
Would you like to try one?}

Carzain spewed and, for the next many moments, could only stare at the girl\dash and her zwongas\dash with a dumbfounded grin. 
He later learned that the girl was selling zwongas, which were actually a kind of fruit. 









\subsection{Carzain rescues ungrateful Redcor}
Carzain rescues a Redcor. 
She is a bitch from moment one, trying to order him around. 

Carzain: 
\ta{I haven't rescued you yet. 
If I were in your position, I wouldn't be mouthing off.}








\subsubsection{Carzain watches his language}
Later, she orders him: \ta{Watch your language, boy!}

Carzain: \ta{Fuck you. You can suck my big, sweaty cock until you choke on my cum and die.}







\subsection{Carzain practices archery}
From time to time in \Redce, Carzain practices archery in his spare time. 
He is a sucky archer, rarely ever hitting the target. 

It's a habit he inherited from Vizicar. 
In Vizicar's time they had guns, so archery was a sport, a harmless hobby. 
Vizicar was a sucky archer, too. 







\subsection{Carzain is lazy}
The Redcor try to get Carzain to practice dilligently, but he is lazy and would rather be a \boheme. 

But, from time to time, Carzain has flashes of mastery. This is typically when Vizicar (or an even older incarnation) awakens to help him, or take control. This frustrates the Redcor. 

It also serves as a subversion of the oft-seen \trope{AnAesop}{Aesop} that you must train and practice all the time to achieve greatness: Carzain pulls greatness out of his ass. (But not all the time. That would make him an annoying \trope{MartyStu}{Marty Stu}. Just once in a while.)









\subsection{Someone disbelieves Carzain's talk of monsters}
Carzain talks about monsters. Someone, a Redcor, immediately laughs. \ta{Surely you are imagining things.} Compare to episode 2 of the anime \cite{Anime:Gilgamesh}.

Vizicar immediately sees through the guy. 
\vizicar{%
  Too fast a reaction. He is trying too hard to conceal what he knows, to avoid suspicion. We need to keep an eye on him, Carzain.}









\subsection{Telcastora Ilcas faces off with a Trickster Mentor}
There is an old Redcor woman who tries to play the \trope{TricksterMentor}{Trickster Mentor} and be all hysterical and demanding. She tries to get people to jump through hoops for her before she'll help them. 

Telcastora Ilcas calls her out on it. 
\ta{Listen to me. 
  We are not supplicants begging a favour of you. 
  We come to you as equal allies and call upon you to do your \emph{duty}. 
  No more, no less. 
  And now you wish to hold the future of the world hostage to your own petty desires, your perversions? 
  Kindly explain to me, then, what makes you any whit better than the Rissitics? 
  What makes you any less of an enemy of the world?}

Carzain is impressed by Ilcas' ballsy-ness. 









\subsection{Ilcas comments on worms}
Some people eat food with \Durcaci spice. 
They talk about the rumour that \hr{Spice and worms}{the spice is produced by giant sandworms}. 
They ask Ilcas about it, since he has \travelled in \Durcac. 

He says he \emph{thinks} it is true. 
But he has never seen these worms. 









\subsection{Carzain and Vizicar work together}
Carzain and Vizicar gradually learn each others' strengths and weaknesses and how to best combine them. 

Vizicar's practiced diplomacy combined with Carzain's irreverent daring make them a great detective and manipulator, akin to Vlad Taltos from \authorseries{Steven Brust}{Dragaera}. 
(Vizicar on his own is a bit too constricted by principles and etiquette after having spent his whole life at court. Carzain teaches him to loosen up and be more open-minded and undaunted.)







\subsection{Vizicar surprised by map}
At one point Carzain looks at a map. 
Vizicar is in his head, reading along. 

\begin{prose}
  Vizicar: 
  \vizicar{Damn. This looks nothing like the maps I know.}
  
  Carzain: 
  \ta{Huh? Oh, right. Makes sense. 
    You lived before the \Darkfall, after all.}
  
  Vizicar: 
  \vizicar{The what?}
  
  Carzain summarizes what he knows of the \Darkfall. 
  
  Vizicar: 
  \vizicar{Interesting.} 
  
  Vizicar is especially intrigued by this \hr{Belzir}{\Belzir} character. 
\end{prose}







\subsection{Carzain hates the Redcor}
The Redcor are bitches. Carzain puts up with it for a while, but gradually, their arrogance makes him grow to hate them. 

The Redcor know that the Rissitics are up to something. Gradually it turns out that they mean to resurrect a \daemonic{} demigod, a \Haskelek{}. 
%See section \ref{Haskelek story} for details. 
%The \Haskelek{} is an old ally of the Sentinels and a terribly evil \daemon. It was brought to \Miith{} once about 1000 years ago (after Vizicar's reign but before \Belzir's) where it conquered and controlled a kingdom in central \Velcad{}, current day Pelidor and surroundings. But after a prolonged war, the \Haskelek{} was defeated by the Vaimons. 

%Unable to destroy it, the Vaimons imprisoned the \Haskelek{} in his temple, located in the middle of a huge, wild forest south or southeast of Pelidor. There it lay dormant, but its evil seeped out and corrupted the beasts and people in the surrounding area. The forests around its temple became dark, twisted and hostile (even more so than the regular \Miithian{} nature, which can be harsh enough) and the humanoid tribes to become degenerate savages who now worship the \Haskelek, or the memory of it. 

%There might be some planar stuff going on here, with the temple residing in another plane of reality. I should give this more thought. 

%Somehow the Rissitics reach the \Haskelekz{} temple and awaken it. It has grown stronger in the meantime by feeding off the prayers and sacrifices of his primitive worshippers, and is now almost as strong as before its death, but it needs powerful help to break free of its prison. The Rissitics free it. 

%The Redcor now oppose the \Haskelek{} and attempt to destroy it. It comes to a climactic battle, where they weaken \Haskelek{}, forcing it to retreat to its temple to recuperate. Redcor forces, including Carzain\dash and possibly Curwen\dash pursue him there, intending to reactivate the old spell seals and reseal the \Haskelek{} in its old prison\dash or something like that. 





\subsection{The Redcor leaders are incompetent}
The Redcor fight against the Rissitcs and Sentinels and the \Haskelek. But some of their leaders are incompetent fools. So, at times, Telcastora Ilcas or Carzain/Vizicar have to take command and lead their forces to victory. (In case of Carzain/Vizicar, it's more of Vizicar, since he is better versed in the art of war and commanding armies). 

Compare to Jarod Shadowsong in Richard Knaak's \emph{The Sundering}. When he leads the warriors into battle, compare to the king of Hyperborea in \bandsong{Bal-Sagoth}{The Splendour of a Thousand Swords Gleaming Beneath the Blazon of the Hyperborean Empire}. 







\subsection{Carzain runs from battle}
Carzain and some Redcor are in trouble, threatened by enemies. 

A \gandierre: \ta{We must fight! \Honour demands no less!}

Carzain: \ta{I have a better idea: Run like Hell!}

Telcastora Ilcas: \ta{I must agree with \Shireyo. I have my own \honour, but it does not require me to let myself die if I can do more good alive.}

Perhaps this is another instance of Ilcas inspiring Carzain to value his life above that of others\dash which, by utilitarian reasoning, is perfectly justified. 







\subsection{Carzain fails at etiquette}
Carzain gets owned regularly in \Redce{} because he does not know Redcor etiquette, or because he refuses to acknowledge it. 

He bitches and disses the hypocrisy of the Redcor rules and taboos. Perhaps he discusses it with Vizicar. 







\subsection{Carzain discusses philosphy with Shereid}
Carzain discusses \hs{Redcor philosophy} and \hs{Geican philosophy} with Shereid. 








\subsection{Carzain is fencing, no Vizicar}
Carzain is in a sword-fight. He tries to \quo{summon} Vizicar, who is a superior swordsman, and get him to come out and take control. But Vizicar is submerged in dormancy at that moment. 

Carzain is not a good enough swordsman to win on his own, so he flees. The Redcor are nonplussed by his cowardice. 










\subsection{Vizicar takes crazy chances}
Vizicar takes over their shared body and takes some crazy chances that nearly kills them both. 

\begin{prose}
  Carzain: \ta{You crazy fuck, are you trying to kill us?}
  
  Vizicar: 
  \ta{Well, I've already died once. 
    I got better. 
    That puts things into perspective, you know.}
\end{prose}









\subsection{Carzain sacrifices himself by fleeing}
Carzain has the choice of rescuing someone\dash\Racel?\dash at great risk to himself, or flee and save himself. Inspired by Telcastora Ilcas's utilitarianism, he chooses the latter. 

\Racel{} survives and later confronts him: \ta{What did you think you were doing? Why did you not come and rescue me? You knew I was in danger! Have you no \honour? As a man, it is your Light-given duty to risk your life to save a woman! Or are you no man at all? Are you truly so selfish, so cowardly, so petty?}

Carzain slaps her up. \ta{Listen to me, you cunt. \emph{I} am the Scion! \emph{I} am the one whom you Redcor rely on to save you. I am the most important person on this expedition. Without me, there is no mission. \emph{I am the mission}. Without me, all that you have fought to accomplish will be for nothing. It follows that my life is worth more than any of yours. My foremost duty, then, is to remain alive, that I may fulfill the mission.

Nay, it is you who are selfish. It is you who are petty. It is you, you Redcor fuck, who don't know the meaning of sacrifice. I would love to save your pretty little ass\dash and with it, it would seem, my \honour and manhood\dash but I know my duty. For that, I am willing to sacrifice my \honour among the Redcor if need be. 

That is one of the many things that separate us: You are a fool, a blind fool, trapped in dogma and hypocrisy. Whereas I am wise enough to know right from wrong without such stupid principles. That is why I am better than you!}







\subsection{Carzain reveals his friend's secret}
Carzain has a Redcor friend who confides a secret to him. He makes Carzain solemnly swear that he will not reveal the secret to anyone. 

Later Carzain suspects that something fishy is going on, and tries to solve it. He works together with some allies\dash possibly Shereid. 
In order to solve this mystery\dash and save humanoid lives and potentially avert disaster\dash he deems it necessary to reveal the friend's secret. 

This is one of the first of Carzain's many broken promises.







\subsection{Carzain's \human{} enemy}
Carzain is standing and hating some enemy, possibly a Redcor. 

A friend of his defends the enemy, saying: \ta{When you look closer at her, she's really very \human{}. Look, she has all these nice qualities and does all these sympathetic, \human{} things\ldots{}}

Carzain: \ta{So? How does that excuse her crimes? In fact, it makes it worse. If she was a piece of shit all the time, then she might have the excuse of stupidity or a bad upbringing or whatnot. But her \quo{\humanity} only shows that she can be worthwhile if she chooses. Alas, she chooses to be a stupid bitch.}







\subsection{Carzain learns to control his souls}
Carzain slowly learns to control the dead souls bound to him, and to channel them as a weapon. (See section \ref{Bound souls as a weapon}.)

Carzain unleashes this power by accident one or more times in the latter half of this story, in a fit of extreme anger, anguish or other emotion. He is horrified by what he has done, as is Vizicar. 

Describe the victim's fear and horror. Inspired by Clive Barker's \emph{Jacqueline Ess: Her Will and Testament}. 

Afterwards, Carzain is like: \tho{What the fuck? That was no \Qliphah, and it sure as fuck wasn't a \Sephirah!}







\subsection{Carzain disgruntled}
After their victory, some of the Redcor are praised for their great worth, but Carzain is neglected and gets no praise, only reprimands. This causes him to resolve to turn on the Redcor. Shereid urges him on in that direction. Ultimately, while out in the wilderness somewhere\dash perhaps on a mission\dash Carzain betrays his \Redcean{} companions and kills at least one Redcor, then flees to Geica with Shereid. 

%The Redcor are bitches. Carzain puts up with it for a while, but eventually, their arrogance has bred such hatred in him that he turns on them. So, while on a mission, he betrays his companions and kills them, including one or more Redcor. 

%that he willingly turns to evil. In his experience, those that call themselves \squo{good} are no better. So, while on a mission, he betrays his companions and kills them, including one or more Redcor. 

%He goes to Geica and learns to channel \itzach like the Geican mages do. When he returns, he is far more powerful than before. 















\section{Carzain and his Dreams}









\subsection{Carzain dreams of \ophidian{} eyes}
Carzain dreams of \ophidian{} eyes. When he wakes he wonders who the \ophidians{} are and how he knows of them. How does he know the name? How did it come to him?

Maybe move this to the next book. 

\lyricsbs{Bal-Sagoth}{
  A Black Moon Broods Over Lemuria
}{
  As a black Moon broods over Lemuria,\\
  ebon witchfire enshrouds the gleaming citadels.\\
  sinistrous shadows rise from the vaults of the dreaming elder gods.\\
  \Ophidian{} eyes glimmer through the icy whispering Moon-mist\ldots{}
}

\lyricsbs{Bal-Sagoth}{
  Enthroned in the Temple of the Serpent Kings
}{
  Storm-borne bride of winter's fire,\\
  serpent-witch of the whispering fens.\\
  Veils of scarlet and sable,\\
  blood spilled in the vault of night,\\
  Frost-garlanded, the mind-binding \\
  glimmer of tear-filled ophidian eyes.\\
  The gleam of winter Moonlight upon black waters.\\
  Nighted spells of the enchantress.
}

\lyricsbs{Bal-Sagoth}{
  Shadows 'Neath the Black Pyramid
}{
  I hearken to the grisly murmur of nameless fiends,\\
  black jaws drooling blasphemy.\\
  Beyond the witch-song, darkly sweet,\\
  The wyrm-horn sounds cross Dagon's mere.\\
  Shadow-gate (portal to the Black Pyramid) yawns wide, beckoning\ldots{}\\
  Spells scrawled in blood and frosty rime.\\
  Squamous god encoils the onyx shrine.\\
  (By the bleeding stone) I am enraptured by \ophidian{} eyes.
}









\subsection{Ramiel's \Kenosis}
Ramiel has undergone the \hr{Kenosis}{\Kenosis} and become a Scion. 
But the \Kenosis{} is imperfect and has split his personality and memories into Carzain and Vizicar. 

Vizicar tells Carzain that before they can hope for \hr{Apotheosis}{\Apotheosis} and regain their true power, they must first master and perfect the \Kenosis. 
They must become one. 









\subsection{Dreaming of \Cuezcan{} Spires}
\target{Dreaming of Cuezcan Spires}
Carzain/Ramiel has amnesia and does not remember his past as a \resphan{} lord. But he remembers scattered pieces and fragments. His memory loss pains him, and he makes it his great quest to discover who he is. 

Carzain falls sick. During this time, he has fevered dreams and visions where hidden presences whisper to him about his purpose, his quest, his destiny of greatness. 

He sees visions of terrible creatures, macabre gods and ancient, ruined cities. 

Later, we have another scene where Carzain/Ramiel broods. He searches his mind for clues about his past and identity. He takes poisonous drugs and uses dangerous magic and meditation to take himself into trance.

\lyricsbs{Bal-Sagoth}{A Black Moon Broods Over Lemuria}{
  Slumbering upon the throne of Moon-caressed ice\\
  I have supped deep the draught of white vapours.\\
  Shimmering upon the gleaming garlanded marble,\\
  a single strand of glimmering gossamer.\\
  In brooding and sombre visions I hear cries. \\
  Enthralling cries 'neath this frost-Moon rising.\\
  The whisperer in crystals speaks in dreams,\\
  of silken shadows and the softest breath of dark enchantment,\\
  of ancient cyclopean temples, raising jewelled spires to the stars.\\
  There is witchcraft in the moon,\\
  and a brooding silence reigns over the woods.\\
  My storm-forged sword,\\
  ensorcelled by eon-veiled incantations.\\
  Dark wizards' spells entwine me in ravening shackles,\\
  and black roses draw my blood with thorns as sharp as serpent's tooth. \\
  I fall into the rapturous embrace of sloe-eyed witches, \\
  the Moon gleaming upon their ivory bosoms,\\
  and descend into the still, icy waters of the lakes.\\
  Beyond the veil of the north-winds\\
  I await the emissaries of the tyrant.\\
  The wind whispering across the everlasting snows.\\
  My slumber is light as a wolf's\ldots{}
}






\subsubsection{Ramiel's dark secrets}
Ramiel has dark secrets and dark dreams. See section \ref{Ramiel dreams}.





\subsubsection{He dreams of \Nyx}
In his dark dreams he sees \Nyx. 

\lyricslimbonicart{Deathtrip to a Mirage Asylum}{
  To the abyss' morbid enigmas.
  
  Far beyond the earthly grave,\\
  the soul betakes to drift astray.\\
  Infinite horizons of pale grey skies.\\
  A still born heart within there lies.
}

He somehow understands and remembers that even though \Nyx{} is scary and horrid, it was once his home, and thus remains his home. 

\lyricslimbonicart{Deathtrip to a Mirage Asylum}{
  The sanctuary that once was lost\\
  is streaming endlessly in holocaust.\\
  The astral corpse is still pulsating,\\
  in a mirage asylum awaiting.
}









\subsection{Carzain gets lost in strange Realms}
Maybe have a scene where Carzain somehow stumbles into another Realm and is led astray. 

He encounters creatures that are sworn to serve his kind, but have been alone and abandoned for centuries or more. Perhaps they are incorporeal, but he imagines them in humanoid form. And because of his power over them, they assume the form the thinks up for them.

Perhaps this is where he finds \Belzir, and she finds him.







\subsection{\Belzir contacts Carzain in dreams}
Carzain is visited in his dreams by a mysterious woman. In his dreams, they have sex. The wraith-woman tempts him with promises of power, glory and revenge upon the Redcor. It turns out that the woman is \Belzir, the last \Calipha of the ancient \VaimonCaliphate and called the Dark Prophet of \itzach by some. She wants him to join her and her servitors, who are working to resurrect her. 

She also approaches Vizicar\dash maybe both at the same time, maybe one at a time. 

She urges them to join her, tempting them with promises of power, glory, sex\ldots{} and, best of all, knowledge. 
She knows he is a Scion, amnesiac and questing to master the \hr{Kenosis}{\Kenosis} and eventually the \hr{Apotheosis}{\Apotheosis}. 
She tells him that she is a Scion herself, the only Scion ever to have attained the \hr{Apotheosis}{\Apotheosis}. 
This gives Carzain and (especially) Vizicar a boner. 
Vizicar very much wants to go to her. 
He does not know of her and has not been indoctrinated to hate her (since he lived before she did), and he doesn't like the Redcor, so he roots for her. 
Carzain is a bit more \skeptical, with his heroic ideals and all. 

\Belzir{} tells him of how she was a rebel against the Vaimons and their oppressive religion. 
This impresses Vizicar. 
He wanted to do the same, but never had the will to actually do something about it. 
He admires her a bit. 

\Belzir claims that she is not evil.
She is working to create a better world. 
It is the Redcor that are the evil ones, with their repressive and bigoted religion. 
They are hypocrites.
She is the real deal. 

Carzain is stuck by the \hr{Belzir's sorrow}{immense sorrow and pain} she radiates. 
He notices the lines of tears under her eyes.

He is also attracted and enraptured by her \hr{Geican green eyes}{emerald green eyes}. 
He can feel the Geican eagle in her. 

When Carzain/Ramiel first encounters \Belzir, he is struck by her commanding, seductive being. 
It fills him with conflicting emotions: 
\begin{itemize}
  \item Sexual lust for her. 
  \item Intimidation; an inclination to submit and obey her. 
  \item A masochistic lust to let her sexually dominate him. 
  \item Shame at his own weakness and unmanliness. 
  \item Anger at her for making him feel shame. 
  \item Desire to punish her. 
  \item Lust to sexually dominate her. 
  \item Excitement at this challenge. 
  \item Respect for her strength, sexiness, seductiveness and boldness for making him feel all of the above. 
\end{itemize}

Later and gradually, he comes to realize this: 
\begin{prose}
\tho{I want to rule, to dominate her.
  Any opposite feelings I may experience are alien, unnatural things she plants in me through her manipulation and wiles, and which I have to combat with the power of my true self. 
  
  But \emph{she} has both: 
  She wants to submit to me, and she wants to enslave me.
  Perhaps both at the same time. 
  
  The last part is a scary thought. 
  Underneath the surface, who is really dominating whom? 
  If one of us pulls the other by a chain, who truly leads and who follows? 
  I am simple, but she is complex. 
  Damnably complex. 
  And that makes her even more dangerous. 
  I must not let down my guard with her.}
\end{prose}





\subsubsection{Chasing \Belzir}
\Belzir does not give herself to Carzain immediately.
She kisses him and lets him grope her, but does not let him have sex with her. 
She flees. 
He has to chase her. 

Compare to Atali from \cite{RobertEHoward:TheFrostGiantsDaughter}, who seduces Conan and flees. 

Only late in the story does he catch her. 
In this dream, he sprouts great wings from pure willpower. 
This is a great victory for him and a significant step towards Apotheosis.
With the wings, he easily catches her.
They have sex for the first time in this life. 
It is passionate and wonderful. 





\subsubsection{Sex with \Belzir}
Remember that \Belzir{} is herself a Scion, and she knows a lot about the \Malachim. She uses this knowledge to drop hints to Carzain/Vizicar and lead them on.

The sex scenes with Carzain and \Belzir{} should be Bal-Sagoth-esque: 

\lyricsbs{Bal-Sagoth}{
  Enthroned in the Temple of the Serpent Kings
}{
  Ruby-lipped and midnight-tressed, \\
  eyes as black as raven's wing.\\
  Flesh so pale as dawn-frost gleaming,\\
  nighted spells of the enchantress.
}

Or \Duana-esque (in the following, replace \quo{he} with \quo{she}):

\lyricsduana{nightsong}{Night Song}{
  at night he slips through my window \\
  cool air caressing me like a lover's moist tongue 
  
  I shiver new awakening \\
  moaning a rhapsody of vowels \\
  and consonants like crosses in a row \\
  his lips of bittersweet agony \\
  bruising tender flesh \\
  as I lay impaled by lust \\
  and hear the sound of my heart beating \\
  druming blood through my viens 
  
  his love is wild like wolf-song \\
  howling at the pregnant moon \\
  his voice is smooth as liquid velvet \\
  as he sings his soothing lullaby \\
  hush hush hush my pretty pretty one 
  
  until the dawn bleeds from darkness \\
  tangerine rays grazing my cooling flesh \\
  and he flees like the shadow of a dream \\
  leaving silence in his wake
}

\lyricsbalsagoth{The Hammer of the Emperor}{
  [She Came Bearing Dark Portents (The Foreshadowing):]\\
  Fever-dreams, dark omens and auguries. Prophecy!\\
  Why, when I meet your narcotic sloe-eyed gaze, \\
  does the image of a viper nestling in a bed of blossoms fill my mind's eye?\\
  Why, when you come to me by the pale light of a waning moon, \\
  do I glimpse the sheen of ophidian scales through the veils of sable?\\
  Why, when you enrapture me with your envenomed kisses, \\
  does the flicker of a serpent's tongue score my flesh?\\
  Enthralled by the vitreous lustre of your rubicund lips, \\
  your snow-pale skin musky with the intoxicating scent of night\ldots{} \\
  but such wicket thorns beneath this rose.\\
  Come witch, fly to me!\\
}

Gradually through the book, his sex with \Belzir{} helps him remember fragments of his past life. 

\lyricslimbonicart{Symphony in Moonlight and Nightmares}{
  This night belongs to us,\\
  my dear princess of death.\\
  We have a destiny together\\
  as we meet in hungry caress.
  
  A deadly kiss under moonlit sky,\\
  as I stare into thy dark and hollow eyes.\\
  You take me down where cold silence dwells.\\
  Unconscious darkness realms of demise.
  
  You are born at my grace to serve only me.
  
  Blood paint on the wall is the ancient dream I recall.\\
  Predictions in rainstorm as tears dark red cascaded.\\
  In cryptic depths of imprisoned rage where I succumbed to temptation,\\
  in the laughter of cruel gods, demonic wrath and devastation.
  
  A monstrous enemy demands to be set free.\\
  Frigid evil games. Black art and blasphemy.\\
  Prophecies in blood.\\
  A deadly kiss unto darkened bliss.
  
  We have a gift of shining\\
  by knowing the history\\
  behind obscure mysteries.
  
  The mysterious source of true art and experience.
}





\subsubsection{From \ps{\Belzir}{} perspective}
We see one of the scenes from \ps{\Belzir}{} point of view. 

\lyricsbs{Emperor}{The Ancient Queen}{
  Dark rivers run though me.\\
  Darkness follows everwhere. \\
  Kingdoms falling again and again.
}

In a dreaming state, she flies to him. 

\lyricsduana{thedreaming}{The Dreaming}{
  Midnight wakes beneath a mystic moon \\
  as ominous mists waltz dewy down \\
  the nocturnal dance of mothwings beating \\
  soft staccatos and whispered wind \\
  the spirit of a shape transforming \\
  like a ghost of velvet shadows \\
  through a lattice silently creeping \\
  into a window open to the night \\
  steals between the gauzy drapes \\
  to find the sleeper \\
  lost in slumber \\
  dreaming, dreaming, dreaming. 
  
  She watches as his soul lies hid \\
  slack-jawed and lashes fluttering \\
  lost in the nebulous expanse of dreams \\
  the rise and fall of his shallow breathing \\
  she sighs sweet syllables in his ear \\
  from bloody lips smiling deeply \\
  whispers caress his naked skin \\
  moonpale gooseflesh shivering \\
  his mouth opens to her poisoned kiss \\
  his breath captured as he is waking \\
  eyes open wide to the terrible sight \\
  dreams to night terrors transforming \\
  she drinks his spirit red like wine \\
  as his voice is \\
  screaming, screaming, screaming.
}





\subsubsection{Geican principles}
Carzain believes it's a Geican thing to accept your sexuality and revel in it. 
\hr{Carzain is disappointed in Geica}{He is disappointed} when he gets to Geica\ldots{}





\subsubsection{Shereid}
Carzain hangs out with \hr{Shereid}{Shereid}, a Vaimon woman of \ClanGeican. She is secretly a follower of \Belzir. She and Carzain end up having a sexual relationship.  





\subsubsection{He is scared when he learns \ps{\Belzir}{} identity}
When Carzain first learns \ps{\Belzir}{} true identity he is scared. 
But she convinces him that she is really good, and merely the victim of evil propaganda and lies. 
At this point Carzain doesn't like the Redcor, so he is willing to believe much. 





\subsubsection{\Belzir{} and Vizicar exchange memories}
\Belzir{} and Vizicar exchange memories of life, death and afterlife as \malachim. 

\lyricsbs{Emperor}{The Tongue of Fire}{
  the soul is never silent\\
  but wordless, held imprisoned in a cursed tomb\\
  wherein reflections never fade, never die\\
  slowly maddened by the emptiness
  
  left to perish in the ever-dark coil\\
  yet, always alert it its slumber\\
  scorn by the drops of light\\
  piercing through the surface\\
  and it screams
}

They both long to regain their true and rightful power.

\lyricsbs{Emperor}{The Tongue of Fire}{
  teach me the tongue of fire\\
  so that I may set the world ablaze\\
  for it is cold\\
  and this blindness can no longer give me shelter\\
  teach me the tongue of fire\\
  so that I may cry out loud my wrath\\
  and my passion\\
  or else my coil will blister and decay
}









\subsection{Carzain wonders what \Belzir is planning}
A theme in the book is that Carzain is always trying to puzzle out what \Belzir is planning. 
Ostensibly this is because he wants to stop her.
In reality it is because he is in league with her and wants to know how her plans affect him, so he can decide how far he wants to follow along and how soon he wants to betray her.
(Carzain first betrays the Redcor, then \Belzir.)









\subsection{Carzain has evil daydreams}
Carzain's nightmares escalate. He begins to suffer from daymares as well: Daydreams that pop up every now and then, where he sees into the Beyond or into his own imagination. He sees people's true faces, or they are horribly distorted. He sees monsters from the Beyond, or the ghosts that are bound to him, or spectres from his imagination. 

A horrific theme is the blurring of the border between reality and fiction. Or rather, Carzain gradually realizes how blurred the border has always been. Some of the visions he sees in his daydreams are lies, his half-mad subconscious mind twisting the Shroud further to fit its own fears. But some of the visions he sees are truths, with his terrified mind momentarily breaking through the Shroud and gazing into Reality. 

See, when a mind is under great stress, afflicted with powerful, stressful emotions, it may be able to draw upon reserves of primal strength that are otherwise latent and sealed away from the dumb, Shrouded mind. This may sometimes enable the person to penetrate the Shroud. Although sometimes, the stressed mind may be in an exceptionally vulnerable state, and the experience of seeing into true Reality might be extra traumatic. 

In Carzain's case, he is not immediately traumatized. Vizicar helps a lot here, since he has had to live with this for a longer time and knows more of how to deal with it. But Carzain begins to doubt both the reality around him, and his own sanity. And Vizicar's. Because we more and more learn how demented Vizicar is. 

\lyricsbs{Arcane Wisdom}{Symphonia Chaos}{
  Ghastly ghostly presences.  \\
  Insanity takes its place, \\
  while an effigy fades astray \\
  and memories still withstand. 
}









\subsection{Carzain is repulsed, but also attracted}
At first, when Carzain learns of the horrors that fill the universe, he is repulsed. He is swayed by the Redcor's talk of the Light and all that is good, and almost renounces the \quo{black} Geican magic with which he is brought up. 

Vizicar objects to Carzain's idealism, and Carzain lashes back by repressing Vizicar and shutting him out. Compare to Rand in \emph{Wheel of Time} or Gollum in \emph{Lord of the Rings}, who represses his evil side. 

But the longer he stays among the Redcor, the more he is disgusted by them, their hypocrisy and arrogance. When his faith in the Light falters, Vizicar creeps back and whispers in his ear. 

Gradually through the book, Carzain becomes disillusioned and bitter, and more and more fascinated by the dark powers of \nieur{} and other forbidden things. Having to fight against the Redcor's moral principles, he develops a distaste for all morals and comes to enjoy breaking rules and reveling in his power, obeying only his own will and whim. 

This is gradual, tho. Whenever Carzain discovers a new power of his or Vizicar's, he is repulsed and horrified at first, and only later comes to like it. These powers become guilty pleasures to him: They are useful, and he likes the intoxicating rush they give, but he also tries to deny it\dash on the outside, to keep it secret, but also on the inside, because he wants to maintain his own illusion of himself as a principled and righteous hero. 

At the end of this book, of course, he completely abandons and betrays the Redcor and goes to side with Geica. 









\subsection{\Belzir{} awakens}
\Belzir{} slowly awakens. 
She is still groggy and half-delirious from her long slumber\ldots{} and only halfway sane. 
The first times she came to Carzain and had sex with him, she was in a hazy dream, too. 
She doesn't quite understand what is going on herself. 

In dreams, she teaches Carzain secrets. 
She speaks in riddles, because she thinks in riddles. 
It's all still dreamlike and cryptic to her. 
Besides, she's half mad from her millennia in limbo. 

\lyricslimbonicart{Deathtrip to a Mirage Asylum}{
  In limbonic life I've slept,\\
  dreamt that cryptic seals me kept\\
  sheltered from all visual sight\\
  'cause shadows chased every sign of light.
  
  Awakened from an ancient slumber,\\
  recalled to act of strife.\\
  My spirit old and body cold\\
  remain in dark demise.\\
  Immortality beloved am I to dying,\\
  and the whispering of secrets to my soul.\\
  I am born to sing this sorrow.\\
  Evil has no rest in me.
}

She still suffers and has horrible dreams from her time in limbo. 

\lyricslimbonicart{Deathtrip to a Mirage Asylum}{
  The invisible addiction of darkness, emptiness.\\
  And the ghastly silence devours.
  
  Deathtrip to a mirage asylum.
}

\lyricslimbonicart{The Dark Paranormal Calling}{
  The black funeral uniform gives me powers. \\
  I'm catching without warning the dark paranormal calling.\\
  A putrefied shadow that in purgatory dwells,\\
  imprisoned by the cold anxious pits of Hell.
}

She is a \hr{Disembodied souls}{disembodied soul} and hence weak. 
She wants a body back. 
To do this she has to break some spells. 









\subsection{Something inside Carzain mourns his loss}
Something inside Carzain/Vizicar mourns his loss. Perhaps this is a servant or servants (\humans? \resphain?) who willingly let themselves be killed and have their souls bound to him, to serve him as ghosts. But he lost his ability to hear them.

\lyricsbalsagoth{%
  In the Raven-Haunted Forests of Darkenhold, Where Shadows Reign and the Hues of Sunlight Never Dance
}{
  Can you not remember? \\
  Have you forgotten the magic?\\
  Sing to us your spells once more, \\
  and the ancient forest shall dance to your words\ldots{}
}









\subsection{Carzain dreams of his destiny of greatness}
Carzain dreams of the greatness for which he believes himself destined, namely as Overlord of \Mystraacht{} (although he doesn't understand that yet). 
These dreams are more optimistic, but still dark. 

\lyricsdimmuborgir{Reptile}{
  Black uneartly void. Creatures crawling.\\
  Forbidden, forgotten, fairly underrated.\\
  Bastards in the shape of angels holding my hands,\\
  passing me what is left of the wine.
}

\lyricsbs{Marduk}{Infernal Eternal}{
  As I looked into the mirror, \\
  and saw the creation which was fading, \\
  I sailed the darkened waters of my soul \\
  on the ship of flaming hate\\
  towards the land of the damned.
  
  The cold winds of the darkness blow strongly.\\
  The cemetery glows in the dark.\\
  A thousand times, thousand voices \\
  are screaming in pain from beyond.
  
  A number of faceless shapes \\
  march forth from the darkness within.\\
  Life is slowly passing away, poisoned by guilt and sin.
  
  Embraced in Black I lie,\\
  only waiting to die. 
  
  The greatest of life's events,\\
  and I begin my journey\\
  as hands of greater characters unveil the world.\\
  Plunging through space and time, my prison has now been slain.
  
  The reaper purifies my soul.
  
  Perdition and death. \\
  I was in ages so dark and far away,\\
  and I will always be.
}

\lyricsbs{Emperor}{Cosmic Keys to my Creations and Times}{
  All these landscapes are timeless, \\
  and this is all just a part of cosmos, \\
  (but) all is mine and past and future is yet to discover\ldots{} \\
  Much have been discovered, but tomorrow \\
  I will realise I existed before myself.
  
  I will realise planets ages old, \\
  created by a ruler with a crown of dragon claws, \\
  arrived with a stargate\ldots{} \\
  a king among the wolves in the night\ldots{} \\
  An observer of the stars.
}









\subsection{Dreams of his father}
Carzain/Vizicar dreams of his true father, \hr{Zachirah}{\Zachirah}, who urges him on: 
\ta{\hr{Ramiel can do better}{You can do better!}} 

When he wakens, Carzain talks to Vizicar. 
They discuss who this \quo{father} figure might be.  









\subsection{SM sex with \Belzir}
In dreams, \Belzir/\Shiaraid{} plays her sex-games with Carzain/Vizicar/Ramiel. 
She is a \hr{Shiaraid's sexuality}{sado-masochist}. 

She hopes to achieve one of two things: 
Either coerce Ramiel into submission or provoke him so much that his rage and pride will awaken some of his inner power and let him dominate her. 
Either way, she wins. 

Ramiel has his pride and refuses to submit, because he \hr{Ramiel is nothing}{fears being nothing}. 
He fights with all he has in him. 

She presses him to his limit. 
Her attempts at domination are not merely physical, but also psychological and social. 
She convinces him that he is indebted to her and that he has to serve her and submit to her if he is to keep in her good graces, so she will continue to graciously help him. 
She hints that if he is disobedient or incompetent or otherwise displeases her, she will punish him and make his life miserable (not just physically, but socially). 

At last he explodes in a mighty roar. 
The Shroud is broken, and for a moment she sees him in his true \resphan{} form, black-skinned and winged. 
But only for a moment. 
His \hr{Kenosis}{\Kenosis} once again takes over, and he lapses back into \human{} form. 
But some memories remain. 
She has brought him a great leap closer to \hr{Apotheosis}{\Apotheosis}. 
He has \trope{TookALevelInBadass}{Gained a Level in Badass}. 

Up until this point, \Belzir{} has been quite arrogant and always teased and baited Carzain. 
But now he is Ramiel, and just as strong as she. 
She is exhilarated. 

\begin{prose}
  \Belzir: 
  \ta{You have long said that you would make me pay for my insolence.
    Come and take your revenge then, if you have the strength and the courage for it. 
    Punish me\ldots{} my lord Ramiel!}
\end{prose}

He does. (In dreams.)

Compare with the sex scenes in \authorseries{Jacqueline Carey}{Kushiel's Legacy} with \Phedre{} and Melisande, or with \Phedre{} and Severio Stregazza. 

\lyricsauthorbookpage{Jacqueline Carey}{Kushiel's Chosen}{100}{
  [Severio Stregazza:] \\
  \quo{You\ldots{} will\ldots{} acknowledge\ldots{} my\ldots{} sovereignty!}
}





\subsubsection{\Belzir{} thinks}
Afterwards, Belzir thinks to herself: 

\begin{prose}
  \Belzir:
  \tho{He is good. And evil. I can use him. 
    Yes, he and I will go on to a fruitful and delightful partnership. 
    In bed, and out of it.
    And as for the \quo{acknowledge my sovereignty}, well\ldots{} we will see about that.}
\end{prose}

But later, she has doubts. 
She does not remember everything of her former lives, but she knows there is something \quo{wrong} with her. 
She is \hr{Self-destructive Shiaraid}{self-destructive}. 
It worries her. 

She cannot recall the details, but she remembers one name: 
\Nexagglachel. 
He has cursed her somehow. 
She knows her lusts are dangerous to her. 
\tho{It is madness. But it feels so sweet\ldots{}}









\subsection{Ramiel meets \Cishiel in dreams}
\target{Ramiel meets Cishiel in dreams}
\Cishiel \hr{Azraid tells Cishiel about Ramiel}{has heard from \Azraid} that Ramiel is incarnated. 
She seeks him out.
At last she finds him and contacts him in dreams. 

Carzain is surprised that there are now \emph{two} mystical, beautiful women hounding him in his dreams. 

At first, \Cishiel speaks in riddles. 
She wants to try to lure out of Carzain how much he knows, before she tells him something she may regret. 
Also, she hopes to manipulate him. 
She has worked with manufacturing religions in the past, so she is used to manipulating mortals in this way.
Carzain does not buy it.
He tells her:

\begin{prose}
  Carzain:
  \ta{Speak plainly, woman.
    I don't know if you are a ghost or god, but I am no servant of yours.
    My time is too precious to waste on interpreting your riddles.
    If it is not worth your time to express your message properly, then I can only assume it is not worth my time to learn it.}
\end{prose}










\subsection{Vizicar remembers the \angels}
Vizicar remembers the time of the \VaimonCaliphate, when everyone thought that the \Sephiroth{} and the \angels{} were good and all-powerful, and that evil in the world would soon be defeated. See section \ref{Vaimon Caliphate naivete}.

Later, he is very much shocked to learn that the \resphain{} are evil.







\subsection{Carzain encounters \vorcanths{} again}
Carzain has bad dreams, perhaps even manifesting in the real world. 
And \hr{Moon-Wolves help Ramiel in dreams}{the \vorcanths{} come to help him}. 

The \vorcanths{} actually need his help. 
He owes them favours for their having helped him in the past (distant and \hr{Vorcanth help Ramiel}{near} past alike), so they are \emph{demanding} that he come and help them. 









\subsection{\Belzir{} tempts Carzain}
\Belzir{} tempts Carzain to delve deeper into the darkness. 
This is after he has become embittered with the Redcor. 

\lyricsbalsagoth{Return to the Praesidium of Ys}{
  Return with me, beyond the stars.\\
  Rule with me a thousand worlds.
}

\lyricslimbonicart{Path of Ice}{
  Journey into your darkened secrets.\\
  Feel the burning flame inside.\\
  Admit the ecstacy of the extreme.\\
  Just close the eyes and enjoy the override.\\
  When we adrift through the sensual streams\\
  the enchanted pains are so divine.
  
  There are thorns everywhere,\\
  but along the path of ice rose bloom above.\\
  Blood is the rose of mysterious unions,\\
  the symbol of potency.\\
  A taste of erotic sins of lust.\\
  The entrance of immortality.
  
  There are such sights to see,\\
  Adventures and pleasures to feel.
}

\lyricsdimmuborgir{The Maelstrom Mephisto}{
  Dwell in depths of the darker self, at any shore of infinity,\\
  and watch the relentless paint the soil black.\\
  What is being formed echoes throughout eternity,\\
  as the painter chooses \colour no more.
}

She teaches him some magic. 

\lyricslimbonicart{Path of Ice}{
  The dormant seeds of suffering.\\
  The art of mortal flesh that bleed.\\
  Indulged in desire, the forbidden soul empire.
  
  Path of ice.\\
  The entrance to immortality.
}

He sees glimpses of his past as a mighty \resphan{} warlord, and glimpses of a possible future as a great sorcerer-king with \Belzir{} by his side. 

This also foreshadows the fact that he \emph{is} on a path to greatness. The book ends with Carzain contemplating his path from now on. He is becoming a character like Rio from \JuukenSentaiGekiranger, obsessed with the pursuit of power and greatness.

\lyricslimbonicart{Path of Ice}{
  Cold winds pierce through me\\
  as my dark star unfolds.\\
  I ascend the throne of fantasies\\
  where the beautiful abyss recalls.
}

She tells of how the \quo{pious} Vaimons betrayed her, even though she fought for freedom and enlightenment. 
Therefore she wants him to reject the Redcor, their religion and all conventional morality. 
(The Sun is a symbol of the Redcor and the Iquinian church.) 

\lyricsbs{Monolith Deathcult}{Den Ensomme Nordens Dronning}{
  Curse the Sun and all that's holy. \\
  Turn back. I dwell around. \\
  Come to me as a shadow. \\
  Engrave the Sun and the Blackest Moon shadow. 
}





\subsubsection{She tells him of the \Feud}
\Belzir{} tells Carzain a few secrets about the \feud{} and the master races. 

\Belzir: 
\ta{Mages are so obviously superior to the common people. 
Why, then, do they not rule the world? 
Why have there been no great sorcerer-kings since me, do you think? 

I will tell you: 
\emph{They} fear ones such as us. 
Whenever a mage starts climbing to power, they stab him in the back and take him down. 
}

Carzain: \ta{\quo{They}?}

\Belzir: (Dodges the question.) 





\subsubsection{\Belzir tells Carzain about Scions}
\Belzir tells Carzain that they are both \quo{True Spirits}, immortal angels trapped in \human form.
She believes they are fragments of the sundered soul of the Creator \Dragon, the cosmic spirit of death and life.
This is a twisted myth, a vague memory of how the \satharioth derive their soul from \Nexagglachel, combined with something about Scions. 
(\Belzir knows that she and Carzain are both \satharioth and \malachim, but she does not understand the difference between the two concepts.)

\Belzir thus argues that she and he are worth more than ordinary humanoids, who do not have a True Spirit. 
Carzain has to agree that she has a point.

Compare to the story of Simon and Helen from \cite{RichardTierney:ScrollofThoth}.






\subsubsection{Carzain sees \Belzir{} as a kindred spirit}
Carzain instinctively recognizes that \Belzir{} is a kindred spirit of sorts. 
This is vague memories from their time as \resphain, where they were both \Mystraacht, close friends and sometime lovers. 

He lets her seduce him and allies with her. 

\lyricsdimmuborgir{United in Unhallowed Grace}{
  We embrace the madness gathered as one.\\
  Mourning dead passion\ldots{} she comes to me.\\
  A fate awaits us in the night.\\
  In the ruins of creation we will unite.
  
  I am smitten by forbidden fruit.\\
  Possessed by moments of dark splendour.\\
  To walk the nightmare terrains forever.\\
  The enigma lies broken.
}

\lyricsbs{Emperor}{The Ancient Queen}{
  Crawl to thee over again. \\
  The Ancient Queen, the darkest woman.\\
  Stepping through her shadow. \\
  She who sees the soul. \\
  Black is her blood. \\
  Keeper of the fury. 
  
  Stepping through her shadow. \\
  Do you see her soul?\\
  Black is her blood. \\
  Keeper of the fury. 
  
  Dark under the shadow. \\
  Thy destiny lies with her. \\
  The Ancient Queen, \\
  ruler of the domain. \\
  Keeper of the fury. \\
  In shadow\ldots{} shadow.
}

It also brings back memories of the time when \hr{Zachirah seduces Ramiel}{\Zachirah{} tempted Ramiel into joining \Mystraacht}. 
This is what \Belzir{} is planning. 
She remembers their time in \Mystraacht{} together (if not perfectly then at least partially), and she knows how powerful and effective those memories are. 















\section{Immortals and the \Feud}









\subsection[Vizsherioch and Psyrex]{\Vizsherioch and \Psyrex}
Have a scene with \Vizsherioch and \Psyrex, or perhaps \Vizsherioch and \Secherdamon. 
\Vizsherioch is introduced as a sinister, menacing figure whom even \Psyrex fears. 
This should be a prologue to the book. 


The summoning of \Nithdornazsh{} is part of \ps{\Secherdamon} plan to bring \maybehr{Vizsherioch}{\Vizsherioch} into Ascendancy. 
\Nithdornazsh{} is to become \ps{\Vizsherioch} citadel, a \nexus{} from which he can grow strong and spread his tendrils (politically and metaphysically) into the Realm of the Shroud. 
This is a vital step in the forging of the \maybehs{Dagger}. 
When the \Nithdornazsh{} project is complete, \Vizsherioch{} is more Dagger-y than ever. 

Previously, \Secherdamon{} had kept \Vizsherioch{} sequestered and hidden. 
He is his only son and the fruit of thousands of years of hard work, so \Secherdamon{} is very protective and does not want to lose him. 

\Vizsherioch approaches \LocarPsyrex.
\Vizsherioch appears as a \dax in his prime, with pearly white scales, wearing a loose robe of white, silver and gold. 

\Psyrex fears him. 
Where \Secherdamon is fiery bright, his son \Vizsherioch is dark and sinister. 
Not in \colour, but in feel. 
A vast darkness follows behind him and around him. 

\Psyrex fears to look into his eyes. 
\ps{\Secherdamon} eyes are terrible enough, but \Psyrex is used to them. 
There is passion, fervour and desire in the eyes of \Secherdamon, and anger and hate, too. 
But in \ps{\Vizsherioch} eyes there are hints of otherworldly madness. 

\Vizsherioch asks \Psyrex about his progress. 
\Psyrex tells him that there have been some setbacks. 
The Cabalists are sneaky.
He does not know who leads them, now that Charcoal is gone from the city. 
But whoever is in charge must be someone capable. 
They have managed to screw up some of his operations and kill some of his important Sentinels. 
But it is not so bad. 
He has planned for some amount of Cabal interference and taken precautions. 
He is importing more manpower. 
He will be ready on time. 

\Vizsherioch: 
\ta{The beacons are in place? Show me.}

\Psyrex shows him. 
\Vizsherioch sees the slender aethereal tendrils, grown from the Pyre \matrix.
They reach up from \Nithdornazsh to twist around the ley lines and converge upon the \nexus point in Pelidor, where they grab on and hold fast, holding the \nexus in a constricting iron grip. 
He sees the \matrix subtly reaching out to fasten upon the souls of the mortals in the city. 
Binding them.
They will be part of the invocation, he thinks. 

\Vizsherioch: 
\ta{Aye, I see it.
  You have done well, \Psyrex.}
He smiles to himself.
\ta{My father has confined me to our Realms for too long.
  I long to at last set foot in my new citadel.
  And to exert my power in \Azmith; supposedly the most pivotal of the Shrouded Realms.}

\Vizsherioch becomes distant.
\ta{The constellations are falling into place.
  I can feel the tension in the Pyre. 
  The Dagger is taking shape.
  Very soon now...}
He becomes present again.
\ta{%
  What of the \resphain?}

\Psyrex:
\ta{I have detected some \resphan activity in both \Malcur and \Forklin.
  I still have every reason to believe they will swallow the bait.
  But of course, it will depend on \Nzessuacrith and her task.}

\Vizsherioch:
\ta{She will not fail us.}

\Psyrex:
\ta{Yes.
  It is not \Nzessuacrith nor the \resphain that make me uneasy.}
\Psyrex pauses and hesitates. 

\Vizsherioch: 
\ta{You mean the Exile.}

\Psyrex:
\ta{Yes. He has been uncharacteristically... \emph{active} recently.
  In the Pelidor region.}

\Vizsherioch:
\ta{But he has not antagonized us?}

\Psyrex:
\ta{Not that I can determine. 
  And that worries me.
  So far he seems to have acted only against the \resphain, but I cannot guess his motives.
  The Exile cannot be trusted.}
\Psyrex looks at his star-charts. 
Concentrates to shift his vision, so he can bypass the roof and see the sky. 
He looks up at the real stars.
He speaks some spells of divination to read them.  
He stares at the star representing the Exile. 
It is nowhere near the Pyre, nor any other known \matrix. 

Metaphysically, that is.
Physically, the Exile is much too close. 
He is made uneasy by the thought of the rogue \vertex. 
\Ishnaruchaefir is a mystery. 
A wanderer in darkness who can appear anywhere at any time, and from whom no one is safe. 

\Vizsherioch:
\ta{\QuessanthIshnaruchaefir.}
\Psyrex is taken aback. 
He had never heard \Vizsherioch speak the Exile's name before. 
He thought \Vizsherioch shunned the name like his father did.

\Vizsherioch: 
\ta{Called Exile and Destroyer.}
To \Psyrex: 
\ta{You fear him.}

\Psyrex:
\ta{Yes, I fear him. 
  Any creature lesser than a \dragon has cause to fear the Exile.}
\tho{And many a \dragon should fear him, too.}

\Vizsherioch:
\ta{Hm.
  He is in the Pelidor region.
  So it is possible he has caught wind of our doings.
  He must not be allowed to interfere.}

\Psyrex:
\ta{The \resphain feel likewise.
  Have you heard, Lord \Vizsherioch, about this \resphan, \Teshrial, who talks about confronting the Exile?}

\Vizsherioch:
\ta{Yes. 
  Allegedly the Exile has promised to face him.}

\Psyrex:
\ta{I wonder what will come of this challenge. 
  Will the Exile really return to meet the \resphan?
  I do not think \Teshrial is a fool.
  He has a plan.
  But will it be enough?}

\Vizsherioch:
\ta{Interesting prospect, this duel.}
He becomes distant.
\ta{Perhaps I should seek out \Ishnaruchaefir.
  After all, I have never met my uncle face to face...}

\Psyrex tries to imagine such a confrontation.
Would it end in violence?
\Vizsherioch was powerful, but young. 
Would he be able to stand before \Ishnaruchaefir?

\Vizsherioch reads his thoughts.
\ta{Fear not, \Psyrex.
  I will not challenge the Exile to single combat.}
Distant.
\ta{No, that would be wise.
  Not at this time.
  Not at this time...}









\subsection{\Azraid{} protects Ramiel}
\target{Azraid protects Carzain}
\target{Azraid protects Ramiel}
Ever since \hr{Azraid learns of spike}{\Azraid{} learned the details of the \vertexspike{}}, he has taken on the role of Ramiel's \quo{guardian angel}, subtly pulling threads and protecting him from harm while he grows towards his \apotheosis. 

This is necessary. 
Many forces conspire to kill Ramiel's Scion.

\Azraid muses about this. 
He knows that great things are happening these days. 
He hopes Ramiel can finally break free of his prison as a Scion. 
He hopes to unravel the mystery of the Scions. 

Have many references to the mystery of the Scions. 
No one quite knows how they work and why they exist and what their purpose is\dash including the Scions themselves.







\subsection{Cabalists hold out in the Ghost Tower}
The \hs{Ghost Tower} is the Cabalists' bastion in Pelidor after they lost \Malcur. They use it as a staging area to mount offensives. Due to \hr{Charcoal at the Ghost Tower}{Charcoal's clever spell}, later improved and strengthened by \Achsah, the Tower is easily defensible. 

\Achsah{} and Charcoal hold out in the Tower, command their Cabalist minions, and maybe \hr{Achsah rewards Charcoal}{have sex}.





\subsubsection{\Achsah{} is inherited by a new mistress}
After \ps{\Teshrial} death, \ps{\Achsah} allegiance passes to another \resvil, a \Mystraacht{}, to whom \Teshrial{} was indebted. 

\Mystraacht: 
\ta{%
  You belong to me now.}

The \resphan{} tongue has many words for \quo{belong to} and \quo{have/possess/own}. 
She used one of the more respectful ones, but her tone and body language practically scream: 
\hypota{You are my slave.} 

\Achsah: 
\tho{%
  These \Mystraacht{} have no etiquette. 
  I miss \Teshrial{} already.}

Even so, it's not so bad, because the \Mystraacht{} are less snobbish and don't look down on her so much for her blood. 
They respect her more for her strength and skill. 









\subsection{Charcoal leaves the Ghost Tower}
To begin with, Charcoal is in the Ghost Tower with \Achsah. He helps her fight off the Sentinels, and she rewards him with sex. 

But eventually \Achsah{} sends him away on some mission. 









\subsection{Cabalists oppose the \Haskelek{} fragment in \Redce}
\Redcean{} Cabalists fight against a fragment of the \Haskelek. 
It is difficult, because they have to be covert and work indirectly under the nose of the nosy Redcor. 









\subsection[Mystraacht rival goes after Ramiel]{\Mystraacht rival goes after Ramiel}
\target{Mystraacht rival goes after Ramiel}
A \resphan{} of \Mystraacht, allied with a faction that hates Ramiel and does not want him to return, is pulling strings in \Redce. 
Trying to oppose Carzain and get him killed. 

This guy is approached by \Azraid, who asks him to stop. 
\Azraid{} \hr{Azraid protects Carzain}{has his own plans for Ramiel} and does not want this guy to interfere. 

\Azraid{} is very polite, but in a sinister way, with an undertone of threat. 

\begin{prose}
  \Azraid:
  \ta{%
    You are not under my command, of course, so I cannot force you to do anything.
    But I can \emph{request} that you desist and cease interfering with Ramiel.
    I hope you can be persuaded to acquiesce to my request. 
    Otherwise I should be\ldots{} saddened. 
    
    And by desist, I mean desist completely.
    Do not try to pull any strings.
    Do not interfere with him at all.
    
    Please consider my proposal.}
\end{prose}

\target{Azraid tells Cishiel about Ramiel}
After this, \Azraid{} figures that his veiled threats are probably not quite enough. 
So he contacts \Cishiel. 
Tells her about the affair. 
She is very interested. 

He sits back and is happy.
He knows \Cishiel{} will help keep Ramiel safe from meddling \Mystraacht{} busibodies. 
\Azraid{} was not confident he could keep them all away. 
But \Cishiel{} is very skilled. 
She has had to become sly and cunning just to survive. 
She has quite some insight into \Mystraacht, and she knows very well who Ramiel's enemies are. 
The two can work together and supply each other with information, which should be enough to keep Ramiel alive and guide him to a path where he can regain his memories and power. 

Alternatively, this might happen earlier.
See the section about the question of \hr{When does Ramiel meet Cishiel?}{when Ramiel meets Cishiel}.









\subsection{\Vizsherioch{} and \Secherdamon}
Have scenes with \Vizsherioch{} and \Secherdamon{} at home, and show their relationship. 
Perhaps \Vizsherioch{} recommends that his father send him out, after his other minions have failed. This is the first time we see \Vizsherioch. 








\subsection[\Dasteron cleans up Mystraacht]{\Dasteron cleans up \Mystraacht}
\Dasteron does much good work to clean up \Mystraacht.
It had degenerated into a gangster-like den of petty greed and brutality. 
Barbarism.

\Dasteron has a long-term vision.
He unites \Mystraacht and restores it to \honour, glory, dignity and purpose.









\subsection{\Humans{} are special}
Some high-up Cabalist or Sentinel is talking to a \human.

\ta{The old days are gone, and so are the old master races. They may not know it, but their time is up. They had their time and now they are decline. 

Oh, they may still be strong, they may still have raw power. But they are locked in place. They are slaves to their own game, unable to break from it. 

But you \humans\ldots{} you are special. You are unique. You alone possess true freedom of will. And it is this special \human{} quality that, in the end, will decide this war. All the power of the elder races is nothing compared to the gift that is \emph{\humanity}. This is the age of men, and the future lies in \human{} hands.}

Afterwards, he walks away, thinking to himself. \tho{By the \Voidbringer, that was disgusting. I almost feel dirty for spinning such an outrageous lie. I can hardly believe that he bought it. Heh. But that is \humans{} for you, is it not? So easily manipulated, so easily tricked into believing anything you tell them.}

\tho{Hah. \quo{True freedom} indeed. \quo{The gift of \humanity}. Hah.}

\tho{Foolish worms.}









\subsection{A \Haskelek{} fragment possesses a mortal host}
A mortal \scatha{} or \human{} is possessed and used as a host body by one of the fragments of the \Haskelek. 
This might be \hs{Lica}.

The fragment is only half-awake, so the mortal carries it around for a long while. 

The victim is slowly driven mad by the presence of the \Haskelek{} inside her. 
It is alien and terrifying. 
She tries to interpret and put into words those alien thoughts, emotions, images and visions that the \haskelek{} sends into her mind. 
The only recognizable emotion that comes close is\ldots{} hate. 

In the end, the fragment breaks out of her, and her body explodes. 
Compare to the people possessed by Nyak in \authorbook{Stephen Marley}{Spirit Mirror}. 

\lyricslimbonicart{From the Shades of Hatred}{
  A thousand years time dimension \\
  in subconscious incarceration. \\
  My hatred to man, has transformed me \\
  into a habitation for demons. \\
  A devil incarnation. \\
  In the forgotten past, ages ago, \\
  beast became my alter ego.
  
  The god in me, infernal black divinity.\\
  After years of agony and pain \\
  hatred is all that remains.
}









\subsection{\Dzasselid{} explores the Beyond and gains knowledge}
\Dzasselid-tachi are on a mission. He needs more knowledge. 

Have a scene where he casts divination spells, explores the Beyond and gains some kind of enlightenment. See section \ref{The dark universe}.

\lyricslimbonicart{Dynasty of Death}{
  I'm launching into the abysmal universe.\\
  Disembodied I enter the cosmic cataclysm.
  
  I discover stairways to celestial dreamscapes\\
  A dark unknown conjunction within an immoral dimension.
}

Remember that the Rissitic \humans{} are dark-skinned. 
They should remark that \Velcadian{} \humans{} are much paler. 









\subsection{Someone finds a desecrated corpse}
Someone, investigating the strange happenings that have occurred lately in \Redce, finds a corpse that has been killed and desecrated by the underground cult. It is mutilated, perhaps with the skin torn off, and bears clear signs of sexual abuse (perhaps covered in sperm after bukkake). 

Maybe something a la Clive Barker's \emph{The Midnight Meat Train} (from \emph{Books of Blood I}). 









\subsection{\Narkiza{} fights and Belgrim almost dies}
\hr{Narkiza}{\Narkiza} fights in a great battle. \hs{Belgrim} takes grievous wounds and almost dies. But with massive, heroic effort, Belgrim rises again, roars its defiance and fury, then charges and causes immense destruction. He collapses again. 

\Narkiza{} fights off the enemies to keep Belgrim safe. He is not skilled enough to heal him, but he has enough spells to send Belgrim into a deep, ensorcelled sleep, stabilizing his wounds until they can find mages and have him healed. 

Belgrim lives.









\subsection{Raising the \Haskelek}
\target{Raising the Haskelek}
The Sentinels begin raising the \Haskelek. 

There are three parts, called the Hand, the Eye and the Heart. 

The strongest part is the Heart. It is imprisoned in a forgotten temple deep in Heropond forest. It's in the southern part of Heropond, where the forest is thickest. It is the hate, madness and evil of the \Haskelek{} Heart that has corrupted Heropond and made it a dark, haunted place of monsters and diabolist savages. (There might be some planar stuff going on here, with the temple residing in another plane of reality. I should give this more thought.) 

The Hand is in Sumian, buried in a secret, shunned tomb. 

The Eye is in northern Pelidor, near \Forklin. It lies in an enchanted tower which exists in \Nyx{} but can be glimpsed from the physical world. It is called the Ghost Tower, and the locals fear to come within sight of it. 

%Somehow the Rissitics reach the \Haskelekz{} temple and awaken it. 
The \Haskelek{} has grown stronger in the meantime by feeding off the prayers and sacrifices of his primitive worshippers, and each part is now stronger than after its initial defeat at the Vaimons' hands. But still, each part is not strong enough to break its prison, and so they need pwoerful help to escape. In come the Rissitics, who attempt to free it. 
%and is now almost as strong as before its death, but it needs powerful help to break free of its prison. The Rissitics free it. 

The Eye is the first part to be released. Soon after the Rungeran forces conquer \Forklin, the Rissitic mages move in and begin the spells to awaken the Eye. It takes humanoid form\dash a humanoid is sacrificed to it, and it occupies his/her body. 

The Hand is the second part to be released. It happens near the end of the first book. After a long quest, Sir James and Lica fail and are killed. The Sentinels successfully awaken the Hand. They use its power to scour the area of opposition, securing a strong foothold in Sumian. Then the Hand goes to Pelidor to join the Eye. 

Now Carzain and the Redcor come in to stop them. A lot of stuff happens. 

%The Heart is released, but eventually the resurrection process is stopped and the \Haskelek{} returned to sleep. This is not necessarily the end of the \Haskelek{}, but it will take a lot more work to awaken it again. I don't know how this goes on exactly. Perhaps they destroy some artifact that was vital to the process. 

Eventually the Heart is released, but before the three parts can be united as one again, Carzain and the Redcor storm in and attack. They succeed in breaking up the ritual, weakening (some of) the \Haskelek{} parts and forcing them to retreat.
%The Redcor now oppose the \Haskelek{} and attempt to destroy it. It comes to a climactic battle, where they weaken \Haskelek{}, forcing it to retreat to its temple to recuperate. Redcor forces, including Carzain\dash and possibly Curwen\dash pursue him there, intending to reactivate the old spell seals and reseal the \Haskelek{} in its old prison\dash or something like that. 

Ultimately, Redcor forces, including Carzain\dash and possibly Curwen\dash destroy the physical bodies of the Eye and Hand, possibly the Heart, too. This breaks them up.

See, each part is still bound to its tomb, but is able to project its power into a possessed body and thus walk abroad. But if that body is slain, the \Haskelek{} part is sent back to its prison, and a new ritual must be performed to release it again. After the defeat of the Hand and Eye, the Redcor (with Cabal aid) storm the Ghost Tower and/or the Hand's tomb and besiege them, thus preventing the three parts from easily uniting again. The book ends in a victory for the Redcor, but one that leaves Carzain bitter and resentful (see section \ref{Carzain with Redcor}). 















\subsection{\Daggerrain{} knows it all}
Have occasional references to \hr{Daggerrain}{\Daggerrain}, hinting at his status as \trope{TheChessmaster}{Chessmaster}. He pulls all the strings and knows everything. The reader should be kept guessing as to how much of the story was actually orchestrated by \Daggerrain. 

Have Cabalists going: \ta{Yes, \Daggerrain{} predicted that this would happen\ldots{}}

An interesting twist might be to have \Secherdamon{} acting as a counter-Chessmaster, at times eaving pulling \trope{XanatosGambit}{Xanatos Gamits} on \Daggerrain.

Compare him to the Crippled God from \cite{StevenEriksonIanCameronEsslemont:MalazanBookoftheFallen}. 









\subsection{\Cishiel{} and \Dasteron}
\hr{Cishiel}{\Cishiel} is allied with \hr{Dasteron}{\Dasteron}. 
He wants to be Overlord of \Mystraacht, and she wants to help him. 
They are already lovers and have had plenty of sex. 
Often in public. 
That is \hr{Mystraacht sexuality}{a \Mystraacht{} thing}. 





\subsubsection{Frame it as a flashback}
\target{Flashback with Cishiel and Dasteron}
An idea: Let the entire story with \Cishiel and \Dasteron be a flashback at the beginning of the book where Ramiel returns to \Mystraacht. 
This way, I can introduce \Cishiel into the story without having to cover her background first, and I can dodge a lot of tricky timeline issues. 
At the time when Ramiel awakens, \Dasteron could have been Overlord for many decades. 





\subsubsection{Submissive \Cishiel}
The two devise a ploy to gain publicity: 
\Cishiel{} will sexually submit to \Dasteron{} in public. 
This will garner tons of attention and make them both look interesting and cool. 
It will make \Dasteron{} more alpha-male-ish and give him a little bit of \Zachirah-image. 

He invites her to a party at his place, or she invites him. 
As soon as they meet, she kneels down without a word and sucks his dick. 
She swallows, licks it clean, then stays on her knees. 
He grabs her hair and violently pulls her head back. 
She is in pain as she says: 
\quo{Thank you, Lord \Dasteron.}

He remains quiet. 
Throws her down to the floor and walks away. 
She remains on the floor for some seconds. 
Then she rises and meekly follows him. 

Then, at dinner, they both act like it didn't happen;
as if it were the most natural thing in the world. 
(Of course, they also have sex that night.)

Another possibility is to have \Cishiel{} act like a willing slave for \Dasteron. 
During dinner she kneels by his side. 
He caresses her hair and head, at times violently. 
When he is angered (by something else), he grips her hair and yanks her roughly. 
She moans and whimpers in a sexy way. 

Slave-girl \Cishiel{} is dressed in skimpy clothes, with small breast-cups of gold. 
Not naked as \hr{Resphan slave livery}{a \Mystraacht{} slave would normally be}, but close enough to naked to invoke associations of slavery. 
And a tight necklace reminiscent of a slave's collar. 
\Dasteron{} tried to make her wear an actual slave collar, but \Cishiel{} drew the line there. 
She was willing to endure much humiliation, but not that much. 

Everyone else knows that it was planned and constructed to make him look cool and give him publicity. 
But it still works. 
\ps{\Dasteron} rivals cannot help but be affected, even though they know it is a ploy. 
And \Dasteron{} knew they would. 

\Cishiel{} is, of course, richly compensated in favours. 
And it also reflects on her standing. 
She gains a certain naughty, erotic allure: 
\ta{Imagine the daring, that she would do such a thing!}
\hr{Resphain are possessive}{\Resphain{} are possessive}, and \Mystraacht{} perhaps more than any.
Now, whenever a \resphan{} sees her he thinks \quo{submissive sex slave} inside his head, and he gets obsessed and must have her. 
This gives her a great deal of power and political influence, and she knows how to use it. 

She muses over it. 
She has broken the rules of sexual conduct. 
In \CiriathSepher{} such an act would leave her branded as a slut with no self-respect. 
But in \Mystraacht{} it has the opposite effect. 
\Mystraacht{} is founded on a spirit of rebellion and anarchy; they hunger for sensations and turmoil. 
Breaking the rules is admired. 
She has been naughty and broken the oppressive \CiriathSepher{} morality, and the \Mystraacht{} love her for it. 





\subsubsection{\Cishiel{} hears rumours about Ramiel}
\Cishiel{} hears rumours that Ramiel is returning. 
She gets excited and goes looking for him. 

\Dasteron{} learns of it. 
It drives a wedge between him and \Cishiel. 
But they still have sex, and they still cooperate. 
It adds some spice, some challenge, some excitement to their relationship. 
They are \resphain{} of \Mystraacht, after all. 
They thrive on conflict. 

They end up enjoying this game, where they both try to extract information and favours from one another. 
They are now rivals. 
They manipulate and distrust but respect each other. 
And they have exciting, dangerous rival sex. 





\subsubsection{\Dasteron{} rapes \Cishiel}
Later, \Dasteron{} suspects \Cishiel{} of having done stuff to harm him behind his back. 
(She knows Ramiel is back. He does not know yet.) 

So \Dasteron{} comes and threatens \Cishiel. 
But she is clever. 
She manipulates him and re-frames his threats as a sex game. 
It ends with him mock-raping her and her mock-pleading for mercy. 

After he is done raping her and has dressed again and is about to leave, he turns around. 
She is still naked and lying on the bed. 
He gives her a long and intense stare. 
He makes her feel even more naked than she is. 
She wants to wrap her wings around her naked body in protection, but she does not want to show such a surrender to him, so she just lies there and lets him stare. 

Without words he tells her: 
\begin{prose}
  \Dasteron: 
  \ta{You tricked me today, \Cishiel. 
    But I will get you. 
    Mark my \quo{words}.}
\end{prose}





\subsubsection{\Dasteron{} becomes Overlord}
\target{Dasteron becomes Overlord}
\Dasteron{} made himself Overlord by gathering enough support to beat down those who disagreed. 
This had taken more than a thousand years of concentrated work. 

Before \Dasteron becomes Overlord, he has to fight several battles to the death against potential rivals.
Strength in combat is not the only virtue demanded of an Overlord, but it is an important one.
He brings \Cishiel as a spectator to those battles. 
She is dressed to resemble a slave: 
A bikini of brass and a necklace that sort of looks like a slave collar, and otherwise naked.

\target{Dasteron cannot become Apex}
\Dasteron is powerful, but his power and skill is chiefly political, not metaphysical. 
He lacks the \vertex{} strength required to make himself \apex{} of the \Mystraacht{} \matrix. 

At the point when he becomes Overlord, \Cishiel{} has learned of Ramiel's return and is working behind \ps{\Dasteron} back to support her father. 









\subsection{\Vizsherioch{} becomes a \shaeeroth}
\target{Vizsherioch becomes Shaeeroth}
\Vizsherioch{} becomes a \shaeeroth.
This is the culmination of the whole \Nithdornazsh{} gambit. 
\Nithdornazsh{} was a stepping stone for \Vizsherioch, to enable him to contact the \xss{} and harness the power that would make him a \shaeeroth.

No one anticipated this. 
Most people have forgotten that \shaeeroth{} could be created, since {no more had been created} in the last thousands of years. 
Plus, most people do not know \Vizsherioch{} even exists. 

The kind of invocation needed to become a \shaeeroth{} is different from most of the invocations that summon forth monsters or other stuff to do the mage's bidding. 
Therefore, when \Vizsherioch-tachi prepared and performed the deed, it was not seen as such an apocalyptic thing, so the Cabal did not really move out in force against them. 
They did not understand the magnitude of what was happening, because it was so atypical, and it looked relatively benevolent (or, at worst, self-destructive). 

After he becomes a \shaeeroth, he becomes feared by all.
He grows so much in power as to rival Secherdamon and \Ishnaruchaefir. 
He is, after all, (sort of) an incarnation of the \xs. A \xs in draconic form.















\section{Ramiel's Defection}
Ramiel betrays the Redcor and defects to the Royalist Faction. 







\subsection{Carzain sacrifices \Racel} 
\target{Racel dies}
Somehow Carzain and the Redcor, fighting against the evil Royalist cultists, set up a gambit where Carzain poses as a turncoat who wants to join \Belzir. 

He tells the Redcor of his plan. 
\Belzir{} has long tired to seduce him, and he has let himself be seduced to an extent, but he has always defied her in the end. 
(We have seen this.) 
But now, he says, he thinks he can fool her. 
And here is how\ldots{} 

The reader must not know his entire plan. 
Then the genre-savvy reader will know it will fail. 
\trope{UnspokenPlanGuarantee}{Unspoken Plan Guarantee}, remember. 

He sneaks in in order to infiltrate the Royalists so the Redcor can stop them. 
He now has his hands on \hr{Belzir}{\Belzir} soul-jar, which the Redcor had been keeping secret and hidden from everyone. 
He has persuaded them to give it to him because he says he can destroy \Belzir{} with it. 

\Racel{} is with him, for some reason. 

The Royalists do a \trope{ShootYourMate}{Shoot Your Mate}: 
They tell Carzain that if he is with them, he should kill \Racel. 

Carzain walks towards \Racel. 
She is nervous, but knows he would never kill her.
He moves closer.
She becomes afraid.
\tho{What is he planning?}

He kills her by chopping her head off. 

See, Carzain had been playing the Redcor for fools. 
He wanted to get in on their gambits and know as much of their insider knowledge as he could. 
But now he thinks he has learned as much from them as they are willing to let him, so he seeks greener pastures. 
He betrays them and joins \Belzir. 
He knows she can be of tremendous help to him.
He gives her the soul-jar, and she is now almost ready to return to life. 







\subsection{Carzain flees to Geica} 
Carzain flees to Geica, together with Shereid and possibly a few others. 
(Maybe even Ilcas?) 
They meet up with other members of the Royalist Faction (the followers of the Dark Queen), including Senator Hayad. 
From here, the next step is to conquer Geica from within. 
(Geica has a democracy; they mean to seize it by a coup.) 

\emph{New idea:} 
At this point, Carzain is not evil, albeit disillusioned. 
Shereid convinces him that there are forces in Geica that are better than the Redcor. 
She introduces him to Hayad and some others, who make a good impression on him. 
It takes a while before he learns that the one behind it all is \Belzir. 
He balks at this, because he has always been told that \Belzir{} is evil, but they all argue that she was less evil than the stories say, that the Redcor and other wicked folks deliberately blackened her reputation\dash and that it was in fact their rebellion that caused the empire's fall, not \Belzir's just retaliation. 
After a while, Carzain accepts this. 
After all, he hates the Redcor and is willing to believe much bad about them. 
He also balks when he learns that they mean to conquer Geica by a coup, but after spending some time among the Geicans, he is convinced that they are little better than the Redcor. 







\subsection{Carzain allies himself with \Belzir} 
Carzain allies himself with the Dark Queen, and she sends demons to aid him. 
She wants him to help resurrect her, that she may return to \Miith{}. 
To this end, she gives him a magical orb. 
Carzain agrees, but plots against her. 

Carzain joins \Belzir's armies. 
He is equipped with the sabre and the \hs{Archon Ward} that belonged to her son Zacrias (the most powerful of \Belzir's many children, who succeeded her as Lord of \ClanGeican after her death). 
He also gets two antique \hr{Vaimon guns}{Vaimon pistols} (magical and powerful). 

\Belzir{} also summons a Peryton (a hind, whom Carzain names Venom) to serve him. 

\lyricsbs{Emperor}{Witches' Sabbath}{
  Shifting shadows guide my way in the autumn. \\
  From this fell alliance eternal night shall arise. \\
  Hear the scream of the wolves calling again. \\
  Legions are wreaking destruction upon the fortress.
} 







\subsection{Ramiel's tragedy} 
\target{Ramiel's tragedy}
\Belzir has a mad plan to conquer the world using Elder sorcery.
Ramiel joins her. 
Eventually he betrays and usurps her and takes over her mad plan. 

Some heroes try to stop Ramiel.
Try to set up Ramiel's story as a tragedy where he will ultimately fail and be defeated.
The heroes strive against the odds, and as such they are bound to win. 

Portray Ramiel as a villain who becomes more and more mad and evil. 
His sanity is suffering. 
(He will later learn from his experience and grow saner again.) 
The reader believes that Ramiel has finally fallen from grace and will eventually be defeated, or perhaps (at best) turn good and sacrifice himself in the end. 

But Ramiel surprises everyone by pulling through and winning. 
He does not conquer the world, but he gains enough power to make himself a god. 
This is a subversion of the usual \trope{Tragedy}{Tragedy} trope. 

\begin{itemize}
  \item 
    Perhaps the Redcor and their allies invade Geica in order to stop Ramiel and \Belzir. 
  \item 
    Or perhaps it is his mad search for the ancient temple which the heroes strive to avert. 
    Perhaps they follow him there and try to kill him, to prevent him from becoming a dark god. 
\end{itemize}

Compare him to Kane in \cite{KarlEdwardWagner:DarknessWeaves,KarlEdwardWagner:Bloodstone,KarlEdwardWagner:DarkCrusade}. 















\section{Carzain in Geica}
\subsection{Geica is corrupt}
\target{Carzain is disappointed in Geica}
Carzain has been filled with stories of how glorious Geica is. He has heard and read stories of the splendid, enlightened Geica as it looked during the time of the \hr{Vaimon Caliphate}{\VaimonCaliphate}. He sees it as a proud, enlightened, freedom-loving and free-thinking culture where all men are equal and free to achieve their goals and live for their ideals, and also a Mecca of learning and wisdom. 

But he is sorely disappointed. The real Geica is corrupt, a cesspool of hypocrisy, bureaucracy and greed. I need to underline the avarice, pettiness and treacherousness of the Geicans. 

Have a named guild of politicians, lawyers and the like (\quo{\DJOF'ers}\footnote{\quo{\DJOF} is Danish for \quo{Danske Juristers og \O konomers Forbund} (or something like that), the \quo{Danish Lawyers' and Economists' Association}.}), who are made the scapegoats and blamed for the corruption of Geica, by virtue (or vice) of the materialistic world view they spread. 

Compare to the Letherii culture in \cite{StevenEriksonIanCameronEsslemont:MalazanBookoftheFallen}, and especially Chancellor Triban Gnol and Invigilator Karos Invictad. 

Perhaps the Geicans own no slaves, only life-long indentured serfs. Compare to the Indebted in Lether. 

This is a part of Carzain's process of disillusionment with all that is good. 

According to \Belzir, it has been going downhill ever since her death. She represented the pinnacle of Geican civilization, but history has posthumously vilified her. Her son \hs{Zacrias}, who succeeded her, did his best to carry on her legacy. His successors were less dilligent and less faithful. 

Perhaps Geica's decay is a result of being vilified, demonized and hated by the rest of the world, especially the \hs{Redcor}, who blamed the Geicans for \Belzir{} and the \hr{Hundred Scourges}{\HundredScourges}. 

Compare to House Harkonnen in \authorbook{Brian Herbert \& Kevin J. Anderson}{The Battle of Corrin}, who descended into evil after being abandoned by Vorian Atreides. 

After all this corruption and hypocrisy, \ps{\Belzir}{} bloody and violent coup almost seems an improvement. The \hs{Royalist Faction} have ideals of \quo{awakening to life the true Geican soul and legacy} and overthrow the corruption and decadence that has festered like a disease, a vermin infestation, in the absence of the true Geican spirit and way of life. They fight for the true Geica and seek to awaken the people. 

Compare them to Bruthen Trana and his Tiste Edur in \MalazanReapersGale. 





\subsubsection{True ruler is not good}
\ps{\Belzir} brutal coup is intended as a subversion of the \cliche{} of \quo{rightful ruler returns to the throne and all is good}. 





\subsubsection{Limyaael's ideas}
See \href{http://limyaael.insanejournal.com/205202.html}{Limyaael's rant on political systems} for inspiration and ideas on how to flesh out the Geican political system. 

See also \href{http://limyaael.insanejournal.com/205202.html}{Limyaael's rant on \quo{council scenes}} for inspiration regarding political intrigue and debates. 







\subsection{Royalists}
In Geica there is a \quo{Royalist Faction} who support \Belzir{} and want to restore her to life and throne. 
They love their queen and fight fiercely for her. 

Vizicar has similar feelings. 
He believes \Belzir{} is the key to his own goals. 
Besides, his feelings remember their old love for her. 

\citebandsong{BeyondTwilight:FortheLoveofArtandtheMaking%
}{%
  Beyond Twilight%
}{%
  For the Love of Art and the Making%
}{
  Through the marshland\\
  And through the rain and mud\\
  Now listen\\
  I march to youm my sleeping beauty\\
  I alone\\
  I cut myself on the moonlight beams\\
  Hordes of wolves follow behind me\\
  I alone\\
  I'm all alone\\
  My face brightens in the flash of the lightning
  
  Through the morning mist\\
  And through the frozen land\\
  I feel your presence\\
  I march to you, my sleeping beauty\\
  You alone\\
  I cut myself on the moonlight beams\\
  So that you can drink from my chest\\
  You alone\\
  The wind gripping my hair\\
  My teeth are grinding\\
  As we meet in the flash of the lightning\\
  It's like a dream
}







\subsection{Vizicar picks up an instrument}
In Geica, Vizicar sees a flute or something lying around. He picks it up. \ta{I can play that.}

He pipes some notes. It sounds horrible. He puts it down. \ta{I didn't say I could play it \emph{well}.}









\subsection{Carzain is drawn by the moon}
Carzain feels somehow that the moons are a key to unlocking the secrets of his past.

\lyricslimbonicart{Enthralled by the Shrine of Silence}{
  Restless days and sleepless nights.\\
  Drawn in the direction of the moon.\\
  Infernal magnet to mysterious destiny,\\
  beyond the grave of doom.
}









\subsection{Carzain goes on a quest to find a \vorcanth}
\target{Carzain dreams of Moon-Wolves}
Carzain is sought out in dreams by the \vorcanths. 

When he finally meets the \vorcanths, he also meets some of their \hr{Aryoth}{\aryoth} companions. 
But it is the \vorcanths{} who lead, not the \aryothim. 

\lyricsbalsagoth{Of Carnage and a Gathering of the Wolves}{
  [VOICE OF THE NIGHT:]\\
  Who are you, wanderer?
  
  [WANDERING SPIRIT:]\\
  I can't remember\ldots{}
  
  [VOICE OF THE NIGHT:]\\
  The wolves are gathering,\\
  the stars are shifting\ldots{}\\
  come, join us in the hunt.
  
  [VOICE OF THE NIGHT:]\\
  Who are you, wanderer?
  
  [WANDERING SPIRIT:]\\
  I have the scent\ldots{}
  
  [VOICE OF THE NIGHT:]\\
  Gaze into the mists\ldots{}\\
  feel the earth thawing beneath your feet.\\
  Come, bring down the prey.
}

Turns out there is a wounded \vorcanth{} who cries out for his help. 
He goes on a quest to find and save it. 
See section \ref{Moon-Wolves help Ramiel in dreams}. 

They guide him, employing occult astrology and \hr{Moon-Wolves and the Moon}{their mystic connection to the moon Visha}.

\lyricsbalsagoth{Of Carnage and a Gathering of the Wolves}{
  [THE SYLVAN ORACLE:]\\
  The wolves are gathering.\\
  The stars are shifting.\\
  This spectre at the feast.\\
  This nectar of the vine.
}

They guide him towards his true identity. He begins to understand his true nature: A terrible creature, cruel yet glorious.

\lyricsbalsagoth{Of Carnage and a Gathering of the Wolves}{
  [VOICE OF THE NIGHT:]\\
  Look at the power you possess\ldots{}\\
  See the might which you wield!\\
  You know who you are, do you not?
  
  [WANDERING SPIRIT:]\\
  Yes, I am the scythe in the field at summer,\\
  I am the thunder that awakens the earth,\\
  I am that which gives the night air its chill.
  
  [VOICE OF THE NIGHT:]\\
  Who are you, wanderer?
  
  [WANDERING SPIRIT:]\\
  I am far beyond the ken of men\ldots{}\\
  my gaze shall make the night tremble!
}

The \vorcanths{} \hr{Moon-Wolves dislike Dragons}{do not like \dragons}, their ancient rivals. They see Ramiel as their saviour.

\lyricsbalsagoth{Of Carnage and a Gathering of the Wolves}{
  He is the scourge, the thanatos, \\
  the cleansing fire, the purifying storm\ldots{}\\
  he is the cataclysm given corporeal form!
}

Later, he remembers that he is on a mission of genocide against the \dragons{} and their spawn.

\lyricsbalsagoth{Of Carnage and a Gathering of the Wolves}{
  [VOICE OF THE NIGHT:]\\
  Who are you, my son?
  
  [WANDERING SPIRIT:]\\
  Father\ldots{} I am annihilation incarnate!
}







\subsection{Someone admits to rationalizing}
Someone, possibly \Ishnaruchaefir, tells about his own history. He portrays himself as the hero and most everyone else as misguided or corrupt. At some point, he confesses that he is subconsciously rationalizing and shining up the story to fit his own desired interpretation. 







\subsection{Fake rebellion is beaten down}
The Geican power holders have caught the scent of the Royalist Faction rebellion. 

So the Royalists stage a fake coup. They are discovered because of information they themselves leaked, and are beaten down. Some of their \quo{leaders} are caught and killed. Apparently the faction is destroyed. The intention is to make the enemy lower their guard, so they can resume their plotting and be more effective for it. 

Compare to the plot of \FLuneNoireVol{5-6}. 

Hayad fights against the rebellion and thus proves himself \quo{loyal} to the government (although, of course, this is just a cover). 

Carzain is unsuspecting and only barely survives thanks to Vizicar's cleverness. \Belzir{} then tells him that it was a test of him somehow. 

Carzain is bitter over having been deceived and almost killed. 

Only Hayad apologizes to Carzain for using him. 







\subsection{Carzain becomes really badass}
During the course of this book, Vizicar awakens some more, and he and Carzain learn to better control their powers. They become more badass, in power and attitude alike. 

He begins to be more of a Dark Knight. Perhaps, at times, his eyes glow with some spectral \colour. 





\subsubsection{He remembers more of Ramiel}
He gradually remembers more and more of his life as Ramiel.

\lyricslimbonicart{Behind the Darkened Walls of Sleep}{
  Behind the darkened walls of sleep, \\
  as body rest and mind goes deep. \\
  A door opens in my heart. \\
  A dark euphoria.
  
  The soul is longing to escape \\
  the tyranny of the body. \\
  Dark winds now embrace \\
  the spiritual entity.
}

He remembers the wars he fought with the legions of \Mystraacht. 

\lyricsbs{Arcane Wisdom}{Maelstroms of Majestic Night}{
  Torrents of blasphemous fire enrage forth, \\
  thus allowing my soul to devour \\
  feeblish souls with sickening power, \\
  as the ancient warcrafts proudly descend from the North. 
  
  Oh the blood of my foes and fiends. \\
  How I beheld thy corpses, \\
  raptur'd by virulent winds. 
}









\subsection{Carzain drinks \ps{\Belzir}{} blood}
Carzain drinks \ps{\Belzir}{} blood during sex. 
It is \hr{Resphan cannibalism}{the greatest rush he's ever tasted}. 
It also brings back a little bit of \resphan{} memories, since it's \sathariah{} \Malach{} blood he's drinking. 

Compare to the movie \cite{Movie:QueenoftheDamned}. 







\subsection{Carzain goes into battle}
Carzain is about to go into battle. He dons his \armour. 

Vizicar complains. 
\vizicar{This is primitive.} 
Vizicar is used to wearing an \hs{Archon Ward}, a magical super-\armour so expensive that pretty much only the \VaimonCaliphs could afford it. 















\section{War Against the Redcor}
\subsection{Carzain has become Ramiel}
During the course of the Geican coup, \hs{Carzain} regains almost all of \hs{Vizicar}'s memories, and much of \hs{Tydesmos}'. 

By the time of the beginning of this book, Carzain and Vizicar have merged enough to be one person, with access to all of their shared memories (or all that they remember of them). (Although they still have their internal dialogue from time to time.)

He now uses the name Ramiel. 

He has become a badass dark lord who wears black \armour and robes. With swords and magic he kills anyone who stands in his way. At times his eyes glow red or some other evil \colour. 

They ally with \hr{Psyrex}{\ps{\Psyrex}} cult and wage war against the Redcor. 

Compare to Wismerhill from \FLuneNoire. It is interesting that Geica's \colours are green and black, like those of the Black Moon.

So the man who puts himself in command of the Geican army is far more than Carzain \Shireyo. He has all the life-long experience of a \VaimonCaliph and an archmage at his disposal. 
Vizicar is a much better general than \Belzir, but she is a far more skilled mage. 

Carzain is sort of like Rio from \emph{\JuukenSentaiGekiranger}. His obsession is not only to become stronger, but also to learn his true identity. 





\subsubsection{Archon Ward and other equipment}
\Belzir{} finally provides an \hs{Archon Ward} for Ramiel. Vizicar is happy. Ramiel also gets a black cape and robe, enforced with metal, to serve as backup \armour. Now, at last, he looks like a true Vaimon dark knight. 

At Vizicar's command they also produce for Ramiel an imperial coronet. It is enchanted and socketed with magical jewels. Perhaps these jewels were previously acquired on a quest.







\subsection{\Belzir controls Geica}
\Belzir{} now controls Geica. She and her associates use a mix of sincerity, propaganda and Shroud-weaving magic to make the people adore them as liberators. 









\subsection{\Belzir's resurrection nears}
\Belzir{} is looking forward to her impending resurrection. 

\lyricslimbonicart{As the Bell of Immolation Calls}{
  In a timeless departure from the flesh,\\
  Drifting the cold ether streams of death.\\
  By the altar of sacrifice, as I call upon the night\\
  to take and give me life beyond the shores of light.
  
  I glorify the hour of blackness\\
  as the bell of immolation calls.
  
  A black heart will adorn\\
  the wings when I'm reborn.\\
  Engraved on my memory\\
  is whom hatred made me.\\
  The ravages of time.\\
  Battles on in my mind.\\
  There are still wounds that bleed\\
  deep in the soul of mine.
}

She is happy to leave behind the horrible limbo.

\lyricslimbonicart{As the Bell of Immolation Calls}{
  A life among the dead and sorrowful,\\
  the endless voids where spirits are mournful.\\
  From the pale of agonising light \\
  I cross the bridge to crystal night,\\
  as the bell of immolation calls.
}







\subsection{Carzain turns against \Belzir} 
It turns out that Ramiel and \Belzir{} were manipulated by \Secherdamon{} and \Psyrex. 

\Belzir becomes more and more irrational and hysterical, like Efrel in \cite{KarlEdwardWagner:DarknessWeaves}. 
Ramiel begins to resent her. 

Ramiel becomes disgruntled. 
Ramiel grows to hate \Belzir{} and desires to usurp her, because she is an insane bitch. He also grows to hate the Sentinels for trying to manipulate him. 

He resents Shereid when he learns that she was part of the plan. He punishes her and means to kill her, but she pleads for her life. She claims that she loves him above all else and swears to serve as his devoted slave. So he lets her live. 

Ramiel's betrayal should be kept secret from the reader until the very end, but subtle hints should be dropped here and there, as he plots, schemes and prepares. The reader should understand that Ramiel is \quo{up to something}, but kept guessing as to what.









\subsection[Overlordship of Mystraacht]{Overlordship of \Mystraacht}
Ramiel and \Shiaraid{} both covet the throne of the \hs{Overlord} of \Mystraacht. 

\Shiaraid{} is the daughter of \hr{Zachirah}{\Zachirah} and feels that she is the rightful heir and should be Overlord, with Ramiel as her Prince Consort. 
She accidentally lets a suggestion of this slip. 
Ramiel picks up on it. 

He is not happy. 
He wants to rule, over \Mystraacht{} as well as over her. 
\emph{Especially} over her. 

Later she offers him the position of Overlord, with herself as his Consort. 
But by that time he has gotten paranoid and fears her ambition. 
He suspects she will cheat him and try to manipulate him so she can rule behind the throne with him as a marionette. 
She has done it before. 
She is good at exercising power even when she seems submissive. 
He does not trust her. 

\Shiaraid{} feels it is OK to submit to him but still exercise power. 
In her mind, it would be a great arrangement. 
Ramiel has his pride and she has her submissive side, so she is perfectly willing to let him have the fancy title, with her as his nominal inferior. 
But she still sees herself as his equal in worth, so she wants an equal share of the power. 
So in a sense her sado-masochistic way of thinking is more sane and healthy, since she can accept a equal relationship where she is sometimes dominant and sometimes subservient. 

Ramiel does not want to share. 
He has his (almost hysterical) \hr{Ramiel is nothing}{fear of being nothing} and cannot stomach the thought of submitting to anyone. 
He must dominate all the time. 

\hr{Curse}{\NexagglachelsCurse} plays them against each other and makes them suspect and plot against each other. 
They argue and fight and lash out in anger, with words or actual violence. 





\subsubsection{\Belzir{} is a bitch}
I need to make sure that \Belzir, under the influence of the Curse, acts like a real bitch at times. 
She may silently regret it afterwards. 
But she is bad enough that the reader understands Ramiel's growing hate and sympathizes with him (to some extent) when he betrays her. 









\subsection{\Belzir cries out \ps{\Aryal} name}
While having sex with Ramiel, \Belzir{} moans the name of \Aryal. 
Ramiel says nothing, but he is not happy. 
He knows that \Aryal{} was and is the greatest love of her life, and that he will always rank below her.
He does not like the reminder that he stands below someone else in some respect. 
It triggers his trauma about \hr{Ramiel is nothing}{being nothing}. 
It makes him jealous and resentful. 

It also makes him mistrust \Shiaraid. 
He fears she is more loyal to \Aryal{} than she is to him. 

Then he finds out the story of what happened to \Aryal. 
\Shiaraid{} \hr{Silqua dies}{betrayed and killed her}. 
If she would do that to her true love, what would she not do to him? 
The second-rate usurper who \quo{thinks he can take \ps{\Aryal} place in \ps{\Shiaraid} heart}? 

He resolves that he can't trust her and begins to plot against her. 

\Shiaraid{} also thinks about it. 
She knows Ramiel dislikes \Aryal, thinking of her as a coward and a weakling. 
This makes \Shiaraid{} resent Ramiel. 








\subsection{Shereid in distress}
Shereid slowly goes mad. Partially out of her obsessive love for Carzain/Ramiel, but even more so because she learns the cruel truth of \ps{\humanity} history. 

She keeps herself halfway sane by clinging to Ramiel and her love for him. 
They have incredible sex. Ramiel's love confirms to her that her life has a purpose. As long as she can be with him, she can live with the fact that her entire race is created to be the slaves of monstrous aliens. 

Compare her to Mele, with her love for Rio, from \emph{\JuukenSentaiGekiranger}. Come to think of it, Ramiel is also quite like Rio. 









\subsection{Carzain doubts his evil, and finally embraces it}
Throughout the book, Carzain is plagued by the memories of his evil life as Ramiel. At first he is horrified, repelled by what he finds inside him. But in the end, he embraces that legacy as his true self.

\lyricslimbonicart{Sources to Agonies}{
  Through the mirror of the soul\\
  I'm staring deep within\\
  To see what dwell behind the wall,\\
  The beauty of pale skin.
  
  The aura that surrounds me\\
  is not of noble kind.\\
  The blackness of the heart\\
  is all that's left to find.
  
  A dark river runs silent through my life\\
  like a floating nemesis.\\
  A dark shadow of what that used to be\\
  drifts now in lifeless misery.
  
  Live only to witness what I've become.\\
  Midnight is my shallow home.\\
  Soon to enter the last deed of mine.\\
  I'm forced to follow the streaming bloodline.
  
  When the wine of life is shed\\
  and dark cosmic space consumes\\
  I bring the memories back from the dead.\\
  Sources to agonies, a devouring monsoon.
}









\subsection{Black magic ritual}
\Belzir orders a black and perverse ritual in the dungeons underneath the \TopazChateau. 
The ritual includes the self-sacrifice of over a hundred prisoners that are hypnotized and mind-controlled by means of a mummified \ophidian head which \Belzir controls. 
Carzain shudders at this, for he knows of the \ophidian race and their powers of mind control. 

The head is undead, plundered from a Durance tomb. 
After he betrays \Belzir, Carzain destroys the \ophidian head out of respect for the dead (remembering his ally \hr{Zeethan Kraal}{\ZeethanKraal}). 

Compare the ritual to Carathis's spell with the 50 deaf-mute one-eyed negresses in \cite[p.100]{WilliamBeckford:Vathek}. 









\subsection{\Belzir is resurrected}
\Belzir{} is resurrected and has her body back. 
Perhaps she even gets a \resvil{} body. 

But part of her soul is imprisoned and sealed somewhere in \Redce. So they ally with \hr{Psyrex}{\ps{\Psyrex}} cult and conquer \Redce. 

Then they perform a magical ritual to heal \Belzir{} and return her true power to her.
Perhaps this ritual is also meant to restore her memories and her full \Malach{} power. 
But in the end, Carzain betrays \Belzir{} and kills her during the ritual. 

\lyricsbs{Cradle of Filth}{Cruelty Brought Thee Orchids}{
  Maleficent in dusky rose.\\
  Gathered satin lapped Her breasts\\
  like blood upon the snow.\\
  A tourniquet of Topaz\\
  glistened at Her throat.\\
  Awakening, pulled from the tomb,\\
  Her spirit, freed, eclipsed the Moon\\
  that She outshone as a fallen star.\\
  a regal ornament from a far flung nebular.
}







\subsection{\Vizsherioch{} is strengthened}
Have an arcane ritual where \Vizsherioch{} is given more power. 

Compare to \FLuneNoireVol{9}, where Wismerhill becomes the Prince of Negation. 

\lyricsflnv{9}{
  Haazeel Thorn: 
  \ta{%
    Life was truly torn from you, but I have replaced it with something better. In your veins now flows Negation.}
}







\subsection{\Redce{} falls}
Under the onslaught of the \hs{Geican} and \hs{Dark Crescent} armies, \hr{Redce}{\Redce} finally falls. 

In the end, Ramiel has to fight a boss monster: 
\Matriarch{} \hr{Dominice}{\Dominice}, a really badass Redcor mage. 
%, who is a Scion of the \Malach{} \hr{Eryal}{\Eryal}. She is older and more experienced that he, and knows more of her true nature and power. She might not be a \sathariah, but he cannot wield his full \sathariah{} power yet. 
She is far older and more experienced than he, and while a mere \human, she knows a lot about him and knows spells that can counter his \resphan{} powers. 

%She actually defeats Ramiel. But he is not quite dead. She becomes overconfident, and he rises to strike her down. Compare to Wismerhill from \FLuneNoireVol{10-11}. 
Ramiel knowingly holds back and lets \Dominice{} defeat him and \emph{almost} kill him. 
%\Dominice{} defeats Ramiel, almost killing him. 
He contacts \Belzir{} and requests that she use her power to keep him alive. 
She cannot afford to refuse. 

Meanwhile, \Dominice{} has become overconfident. 
He rises to strike her down. 
He kills her, then eats her and absorbs her power. 

Have a bloody scene where he tears out her heart, brain and liver and eat them. 

% He insisted on facing her alone because he wanted to eat her alone. 
% This gives him the strength he needs to face \Belzir{} and betray her. 
% Also, he has forced \Belzir{} to deplete her strength to heal him. A dangerous \trope{XanatosGambit}{Xanatos Gambit}, but Ramiel has always been reckless. 
This was a part of Ramiel's plan: 
He let himself be wounded, thus forcing \Belzir{} to pour her power into him and depleting her own strength to heal him. 
Now \Belzir{} is weakened, while he, having eaten \Dominice, is strong.
A dangerous \trope{XanatosGambit}{Xanatos Gambit}, but Ramiel has always been reckless. 












\subsection{The Resurrection} 
\target{Shiaraid dies}
Ramiel makes Shereid his willing sex slave, and together they plot to usurp the Dark Queen. 
Ramiel is bitter at \Belzir{} for treating him badly. 

\citebandsong{BeyondTwilight:SectionX}{Beyond Twilight}{%
  Ecstasy Arise%
}{
  I've learned to cry and crawl away\\
  But at the closing of our circle you will pay\\
  I see the hope vanish in your eyes\\
  I feel the ecstasy in me arise
}

\Belzir's resurrection is a ritual performed by many mages. 

She must be reborn through a \human{} body. 
So a young, healthy, beautiful Royalist woman (named Shadira or somesuch) willingly sacrifices her body and soul to her beloved queen. 

Carzain and Shereid plan to interrupt the ritual and drain \Belzir's power. 
In order to do this, they must have their own agents participating in the ritual in the key spots. 
So they begin to subvert other members of the Royalist Faction to their cause. 
When the war progresses and the day of the resurrection nears, they take measures to assassinate the mages that have remained loyal to \Belzir, so that they may control the ritual. 

At the end of the story, there is a big ceremony to resurrect the Queen, in which Carzain must channel her power through the orb. 
But he betrays her by smashing the orb, thus keeper her power for himself. 

After this, Carzain sets himself up as \caliph. 
But unbeknownst to him, \Belzir{} is not destroyed. 
After the betrayal, she contacts a group of Royalists in Geica, who (hidden from Carzain) were prepared to take over the ritual if anything should go wrong. 
They succeed, and she is resurrected. Some of the Royalists side with Carzain. 

Hayad and his children are among the Royalists who side with \Belzir. However, after a while Hayad notices that \Belzir{} is becoming dangerously insane. 
One night, she tortures and kills a lover who fails to satisfy her sexually. 
Hayad, fearing that he is next in line, flees to join Carzain. 





\subsubsection{\Shiaraid{} dies}
So Carzain and \Belzir{} clash and fight.
And in the end he is victorious. 
He not only kills her, but also \hr{soul-eating}{consumes her soul}. 

She fights back. 
It is a hard battle. 
Ramiel is in danger, so he has to draw really fucking deep of the power within his \carcer. 
He digs deep into his inner darkness in order to unleash as much as he possibly can of his wicked \sathariah{} power. 
He forces the souls within his \carcer{} to submit and serve him. 
This is tough, because there are immortals in there. 
Even \dragons. 

\citebandsong{BeyondTwilight:SectionX}{Beyond Twilight}{%
  Shadow Self%
}{
  Blood on a legion of nameless shores, I beseech you!\\
  Skill of a thousand nameless whores, I invoke you!\\
  Holding the light away\\
  Twisting in my mind
  
  Love of a legion of begging slaves, I exalt you!\\
  Screams of a thousand burning souls, I unleash you!\\
  Holding the light away\\
  Twisting in my mind\\
  Into the shadow's waking eyes
  
  Power of a legion of nameless lords, I command you!\\
  Locks of a thousand nameless doors, I unlock you!\\
  Lash of the six soul-taking strokes, I will feed you!\\
  Twisting in my mind\\
  Into the shadow's waking eyes
  
  \quo{Spiritus Pervertus Nymph}\\
  \quo{Spiritus Pervertus Femme}\\
  Twisting in my mind
  
  I know he steals, my body is haunted\\
  Wicked voice in my spirit has control\\
  Blinded by the beauty of my darkness\\
  Makes me fly, takes me higher
  
  I invoke thee, blackened lord.\\
  Night begins to crawl the skies\\
  Kneel before your master, shadow self\\
  Taste the pain caressed by chains
}

But in the end, \Shiaraid/\Belzir{} submits and lets him kill her, like \hr{Silqua dies}{\Aryal{} once let \Shiaraid{} kill her}. 
For a number of reasons:
\begin{enumerate}
  \item 
    \Shiaraid{} feels guilty over having killed and eaten her (submissive) lover, \Eryal, so she feels she deserves to have her new (dominant) lover do the same to her. 
  \item 
    \NexagglachelsCurse{} makes her self-destructive and suicidal and emo. 
  \item 
    She is very sad because \Mystraacht{} is in turmoil and has been tearing itself apart from the inside for thousands of years. 
    \Mystraacht{} is her father's legacy, so she is strongly attached to it and wants it to live and be strong. 
    She has almost motherly feelings towards her dynasty and will sacrifice herself for it.
    When she lets Ramiel absorb her power and unify it with his own, \Mystraacht{} will be one step closer to uniting and restoring its former glory. 
    
    After all, \ps{\Zachirah} great strength as a leader derived partially from the fact that \hr{Zachirah's slave Resviel}{his \resviel{} submitted to him}. 
    That made him the ultimate alpha male, respected and envied by all. 
    \Shiaraid{} hopes that if she submits to Ramiel and lets him eat her, it will make him (in some small way) a new \Zachirah. 
\end{enumerate}

She feels she deserves to die for her wicked \sathariah{} lusts.

\citebandsong{DeathspellOmega:CrushingtheHolyTrinity}{%
  Deathspell Omega
}{
  Diabolus Absconditus
}{
  He only will grasp me aright \\
  whose heart holds a wound that is an incurable wound,\\
  Who never, for anything, in any way, would be cured of it\ldots{}\\
  And what man, if so wounded, \\
  would ever be willing to \quo{die} of any other hurt?
}

In the end, she is a little bit happy. 
Now she can be sort of reunited with \hr{Shiaraid and the ghost of Eryal}{the ghost of \Eryal}. 
Previously, \Eryal{} had been locked away in \ps{\Shiaraid} \carcer. 
Now they will both be absorbed into Ramiel's \carcer. 
This is as close as they can come to a reunion. 

She tells herself that by sacrificing herself she has atoned for her crimes against the world, against \Eryal, and even against Ramiel. 
She has done what she thinks \Eryal{} would have wished. 

\begin{prose}
  \Shiaraid:
  \tho{Ramiel is saner than I.
    In all my life I did nothing but harm with my tremendous powers, but now, in Ramiel's hand, my powers can perhaps do some good.}
\end{prose}

She is also happy to be free of the burden of her \carcer, and all the loathsome undead souls haunting her. 
Now it is up to Ramiel to carry that burden. 

\citebandsong{BeyondTwilight:FortheLoveofArtandtheMaking%
}{%
  Beyond Twilight%
}{%
  For the Love of Art and the Making%
}{
  When struck I rise\\
  I'm finally free\\
  Mark the bitter tear descending\\
  No more heavy burden\\
  And this is my eternal music
}


Her last words to Ramiel are: 
\begin{prose}
  \ta{%
    Promise me you will reunite \Mystraacht{} and make it strong and mighty again!
    You now have me inside you, and with me the last remnants of \ps{\Zachirah} bloodline.
    That makes you his true heir. 
    And our last \sathariah.
    
    You are the only one!
    \ps{\Mystraacht} future rests on your shoulders, Ramiel.
    \ps{\Zachirah}, \ps{\Nathrach} and mine. 
    We all live on in you now. 
    You are all of us. 
    You carry our hopes, dreams and legacy. 
    You are the soul and heart of \Mystraacht!}
\end{prose}








\subsubsection{\ps{\Shiaraid} tragedy} 
\hr{Shiaraid's tragedy}{\ps{\Shiaraid} tragedy} is \NexagglachelsCurse, which forces her to destroy herself and those she loves. 
For she genuinely loves Ramiel. 

When she finally perishes, it is \ps{\Nexagglachel} revenge. 
She imagines she hears him laughing inside her head, mocking him as she once mocked him in his captivity. 
She hates him, but still some part of her recognizes the justice in it. 





\subsubsection{Ramiel is sad}
\target{Ramiel blames both sides for the tragedy}
Ramiel is victorious, but his victory is bittersweet. 
He is somewhat sad. 
He loved \Shiaraid, and now he has betrayed her. 
He repents. 
He knows it was the Curse that turned them against one another. 

He tries to tell himself it is justice, because \Delphine{} \hr{Silqua dies}{did the same to her beloved \Aryal}, and \Shiaraid{} also \hr{Shiaraid betrays Zachirah}{turned her back on \Zachirah}.
But he cannot convince himself. 

The affair leaves him bitter and disillusioned, and he becomes more evil and brutal. 

He also speculates that maybe it is just and natural for him to kill and eat her. 
After all, it is inherent in the nature and purpose of the \resphain{} to be \hr{Shiaraid betrays Zachirah}{parasitic and consume one another}. 
But that doesn't make it any better. 
He concludes that both sides in the \hs{Feud} are to blame for this tragedy: 
The \banelords{} for having bred the \resphain{} into the destructive things they are, and \Nexagglachel{} and his \dragons{} for having twisted them further. 
He grows to hate both sides more. 

\citebandsong{BeyondTwilight:SectionX}{Beyond Twilight}{%
  Section X%
}{
  Blood on my fingers\\
  Night in my soul\\
  Your breath I will hear no more\\
  I'm slowly growing cold\\
  Your eyes are open\\
  Dead to the light\\
  I hear my soul cry\\
  On the wings of death I ride
}

Eating \Shiaraid{} strikes Ramiel as a hard blow, not just psychologically, but also physically and metaphysically. 
She is a humongous soul: 
\Sathariah{} and \malach, carrying behind a huge \carcer{} of her own. 
It is difficult to gulp her down. 
Her \carcer-souls attack him and haunt him. 
He has to struggle to subdue them. 

\citebandsong{BeyondTwilight:SectionX}{Beyond Twilight}{%
  Section X%
}{
  Hunting the shadows\\
  Knowing they are one\\
  Whispering voices\\
  They stalk me through the dark\\
  My eyes are open\\
  Black, all is black\\
  I am desperately cold\\
  And upon my knees I crawl
}






\subsection{Ramiel is closer to regaining himself}
Ramiel's betraying \Belzir{} and draining her power\dash perhaps even eating her flesh\dash is a key step towards regaining his full power. Remember, \hr{Resphan power and hunger}{a powerful \resphan{} needs to eat}. 

By absorbing her, he is able to \quo{inherit} some of her knowledge. 
He also absorbs her \hr{Fragments of Nexagglachel}{\Nexagglachel{} fragment}, making him \hr{Ramiel is overpowered}{\uber-powerful}. 

\lyricslimbonicart{As the Bell of Immolation Calls}{
  A black heart will adorn\\
  the wings when I'm reborn.\\
  Engraved on my memory\\
  is whom hatred made me.\\
  The ravages of time.\\
  Battles on in my mind.\\
  There are still wounds that bleed\\
  deep in the soul of mine.
  
  I behold the beginnings of sorrow\\
  and predict the omens of cruelty.\\
  In the plague's shadow I follow,\\
  as tormenting winds sweeps\\
  through the cathedral halls,\\
  as the bell of immolation calls.
  
  In embers of infernal greed,\\
  feeding the fires unholy.\\
  Apocalypse was born\\
  when hell brainstormed through me.
}

















\section{Random ideas}









\subsection{Things to remember}
Remember that the books should not be too \human-centered. 
Have more focus on the \scathaese{} societies. 
This book should have lots of focus on the Rissitics and Imetrians. 
Perhaps even with Carzain as a mere subplot. 









\subsection{The \xzaishanns{} and the deepest Chaos}
Remember to have references to the \xzaishanns{} and the deepest recesses of Chaos, from which the \dragons{} draw their power. 

Compare to the First Warren of Starvald Demelain in \cite{StevenEriksonIanCameronEsslemont:MalazanBookoftheFallen}.

Have references to \quo{the blood of the \xzaishanns} and \quo{the power of the \xzaishanns} and \quo{the legacy of the \xzaishanns}. 









\subsection{A sexy girl}
In this book or in one of the next, maybe I should have as a main character a sexy girl who regularly gets owned, dominated, humiliated and sexually harassed by other characters, villains and heroes alike. 





\subsubsection{Erotic scene with girl being killed}
Have an erotic scene with a girl being stripped naked and killed. As in the movie \emph{House of the Dead}. 










\subsection{Sentinels battle to the death}
Have a scene where Sentinels (\rachyth?), servitors of \Secherdamon{} or \Vizsherioch, fight to the death in ritual combat. This is a ceremonial contest, where the winner will be granted great power and status. 

Inspired by one of the early episodes of the tokusatsu series \emph{Go Go Sentai Boukenger}, where \Ryuuoun{} has some of his \Jaryuu{} Tribe battle to the death for the \honour of being selected and transformed into a \Daijakuryuu, a powerful monster.







\subsection{Expedition to the lost \Dragonland}
Have a short excursion to \Dragonland, the lost realm of \dragons. Perhaps \Nzessuacrith{} or another major \draconian{} character. 







\subsection{Betrayal}
I need to have a character who seems good and nice and dandy, helping out the heroes. But in the end, he betrays them, and it turns out he was evil all along. 

Compare to the priest Father Soren from the movie \cite{Movie:GargoylesRevenge}. 
At the end it becomes apparent that he was actively working to help the gargoyles awaken. 







\subsection{Someone is betrayed and imprisoned}
Someone is betrayed and taken prisoner. By whom? The Redcor? Or the Rissitics?

And who? Perhaps Carzain, but perhaps rather someone else. I have enough Carzain material already. 

Anyway, the prisoner awakens some kind of inner power, perhaps one that had previously lain dormant and hidden. He/she strikes back and causes great destruction. 

Compare to the scene in \authorseries{Robert Jordan}{Wheel of Time} where Rand al'Thor is taken prisoner and stuffed in a chest by the Aes Sedai. Or a scene in volume 2 of \FLuneNoire{} where Wismerhill is captured by a wizard who wants to drink his blood. 







\subsection{A \succubus{} shows people illusions}
\index{\succubus}%
A \hr{Succubus}{\succubus}-like monster, or group of monsters or villains, lurk in the Topaz \Chateau{} in \Redce{} and lures people in, to devour or to capture for later use.







\subsection{\Dzasselid-tachi fight for good}
\Dzasselid{} and \Narkiza{} and the rest still fight for \Nechsain, but they try as best they can to bend it in a good direction, to work for a kind of \quo{Rissitism with a \human{} face}.

Compare to lots of good guys in \cite{StevenEriksonIanCameronEsslemont:MalazanBookoftheFallen}, and maybe \authorseries{Michael Moorcock}{Elric of \Melnibone}, where Elric serves Arioch but tries to subvert him. 







\subsection{Ilcas and \Narkiza}
Telcastora Ilcas and \Narkiza{} meet. 
They criticize each other's countries and religions. 

\begin{prose}
  \Narkiza: 
  \ta{Yous sword drinks blood. (Subtext: That is evil.)}
  
  Ilcas: 
  \ta{I am a warrior. 
    I kill my enemies. But I have never killed an innocent.}
  
  \Narkiza: 
  \ta{Innocent how?}
  
  Ilcas: 
  \ta{Give it up. 
    You are trying to get me to doubt my religion and my cause. 
    That is \emph{not} going to happen.}
\end{prose}








\subsection{Someone is overly philosophical}
Some dude is being overly philosophical and sees symbolism everywhere. A wiser character (\hs{Criseis}{\Criseis}?) calls how down on it. 

\Criseis: \ta{Dude, knock it off. Understand this: The world is not full of symbols. The world is not full of meaning. The vast majority of what goes on in the world is due to chaos, chance, meaninglessness. I have lived twenty thousand years. I know this.}

Lamer: \ta{But I am \emph{giving} it meaning, here in my head!}

\Criseis: \ta{Then let it stay inside your head. And know that the world is infinitely larger than your head, and it cares nothing for how you \quo{interpret} it, how you \quo{give it meaning}.}







\subsection{Cool conversations with \Ishnaruchaefir}
\subsubsection{Immortality}
Someone comments that immortality must suck. 

\hr{Ishnaruchaefir}{\Ishnaruchaefir}, or \hr{Criseis}{\Criseis}, or another immortal: 
\ta{Hell no! If immortality sucked I would have killed myself long ago. Immortality owns ass!}

Subvert the trope \trope{WhoWantsToLiveForever}{Who Wants to Live Forever}.





\subsubsection{Peace with my past}
\Ishnaruchaefir: \ta{I have long since made peace with my past.}

\Criseis: \tho{You lie, my lord. You have never made peace. I see it, deep in your eyes. You merely hide it well and have developed skill at ignoring insults.}





\subsubsection{Fallibility}
\Ishnaruchaefir: \ta{Acknowledge my own fallibility? Ha. I think not. Perhaps in another ten thousand years.}







\subsection{A villain disses the younger races}
A Cabalist or Sentinel is accused of being evil. He does a long rant (perhaps a \trope{HannibalLecture}{Hannibal Lecture}) about how the younger races have not only inherited all of the elder races' evil, but also made it worse through stupidity. 

The elder races were and are cruel, true, but theirs is a cultured cruelty, evil with a purpose, intelligent malice. The evil of the younger races, on the other hand, is petty, vicious and, above all, senseless. Where they act there is more waste, more chaos, more senseless destruction and suffering. 

He recites many examples of how \humans{}, \scathae{} and \meccara{} are capable of horrible crimes, completely on their own initiative. This includes all kinds of people: The downtrodden Shroud slaves, the nobles who are still Shroud slaves, the enlightened who fight in the \secretwar, and those few who try to subvert the \secretwar. All of them commit terrible crimes, for what they believe is right, or for no reason at all.







\subsection{Social interaction sucks}
Someone, perhaps Carzain, or even \Ishnaruchaefir, complains about social interaction and all of the hypocrisy which is a necessary part of it. It stems from Carzain's being tired of running political games in Geica.







\subsection{Someone discusses sorcerer-kings}
Have some high-up villain comment on the subject of \hr{Sorcerer-kings}{sorcerer-kings}, and why there are not more of them. 







\subsection{\Ishnaruchaefir{} berates a whiner}
\Ishnaruchaefir{} is talking to someone who whines. He berates him: 
\ta{Get over it. 
  I have suffered more than you ever will. 
  For ten thousand years. 
  So do not come whining to me.}






\subsection{\Banes{} were not the first invaders}
Someone remarks that the \banes{} were not the first aliens to invade \Miith. 
There have been others. 
The \xss{} were one such invading force. 
They \emph{won} and ruled \Miith{} for millions of years. 
But then they grew sleepy\ldots{}







\subsection{\Ishnaruchaefir{} fights during Nadir}
At some point in the series, I need to have a big, climactic event coincide with \hr{Ishnaruchaefir's Nadir}{\ps{\Ishnaruchaefir} Nadir}. 
He is weak, but forced to fight anyway, and not through a \XanatosGambit{} of his own this time (unlike that time with \Teshrial). 

This new Nadir is much worse than the previous one (during \TwilightAngelRememberEmph). 
The \hs{Nadirs get worse}, remember.








\subsection{A \bane{} talks about viruses}
A \bane, perhaps even \Daggerrain: 
\daggerrain{%
  There exist diseases that spread (like viruses). 
  (\Daggerrain{} knows what a virus is, but the listener would not know, so he has to explain.) 
  You \humans{} are the same in the way you spread all over the world, destroying and enslaving the land around you. And yes, we \banes, too, are the same, only on a \cosmic{} scale.}

Or\ldots{} maybe the \banes{} are not the virus. Maybe everything else, such as the \dragons, is the virus, and the \banes{} are the cosmic antibody, created to scour the infestation of life from the universe. See section \ref{Banes as an antibody}. 







\subsection{\Vizsherioch}
\subsubsection{\Rathyon{} and \Vizsherioch}
\hr{Rathyon}{\Rathyon} meets \hr{Vizsherioch}{\Vizsherioch}. 

\Vizsherioch: \ta{Greetings, nephew.}

\Rathyon: \ta{\quo{Nephew}? Who are you?}

\Vizsherioch: 
\ta{You do not know me? 
  I am \Vizsherioch, son of \Secherdamon.
  I am the \xss{} given flesh!}

\Rathyon: \ta{\quo{\Vizsherioch}? That does not sound like much.}

\Vizsherioch: 
\ta{[Smile.]
  I need to litany of names. \Vizsherioch.
  Speak my name. 
  Hear it.
  Perceive its Aenigma.
  And you will understand.}





\subsubsection{\Ishnaruchaefir{} and \Vizsherioch}
\Ishnaruchaefir{} meets \Vizsherioch{} for the first time. 

\Ishnaruchaefir: 
\ta{Aaah. 
  So \emph{you} are the new \vertex{} I sensed in \ps{\Secherdamon} constellation.}







\subsection{Immortals curse the Shroud}
Some immortals curse the Shroud. 
It makes them stupid and forgetful and \hr{Shroud represses technology}{represses technology}. 






















