\section{Concepts}









\subsection{Ramiel}





\subsection{Ramiel's enemies}
\target{Secherdamon wants to off Ramiel}
Ramiel had managed to keep his existence secret from the Sentinels for a while, but no longer. 
After he killed \Shiaraid, he could no longer hide. 

\Secherdamon{} is not happy about this new development. 
He liked \Shiaraid{} because she was so mad. 
She was dangerous to him, true, but he suspected she would be just as dangerous to the Cabal. 
But Ramiel is another matter. 
\Secherdamon{} does not know how \malach-hood and the Curse have affected Ramiel, but he remembers Ramiel as a formidable and competent adversary. 

So \Secherdamon{} decides he wants to off Ramiel. 
Throughout this book, therefore, Ramiel is hounded by Sentinels and Cabalists alike. 

Fortunately, Ramiel has allies: 
\Cishiel, \Azraid{} and the \vorcanths. 





\subsubsection{\Azraid{} helps Ramiel}
\Azraid{} \hr{Azraid protects Carzain}{has his own plans for Ramiel}, so he protects him. 
He also subtly guides Ramiel on his quest and pokes him in the direction of the things he might need to regain his power. 

In this way, \Azraid{} acts as a \trope{StealthMentor}{Stealth Mentor} to Ramiel. 

When Ramiel finally learns of \ps{\Azraid} involvement, he is amused and slightly grateful, but not surprised. 
That is the kind of thing \Azraid{} has always done. 

Later it turns out that \Azraid wants Ramiel to become a \neoresphan (\hr{Ramiel becomes Neo}{which he does}).
From \Azraid's point of view, the whole \malach project \hr{Azraid turns Malachim into Neo}{ties in with his own \neoresphan project}. 





\subsubsection{Heroes chase Ramiel}
\target{Sithiyacaan goes north}
Throughout the book, the \quo{heroes} chase Ramiel all over the world, trying to stop him from achieving his mad goal of becoming a god. 
Compare to how Cloud-tachi chase Sephiroth in \cite{VideoGame:FinalFantasyVII}. 

Telcastora Ilcas is one of these heroes.
\Sithiyacaan is another. 
\Sithiyacaan is still a madman and has most of his powers locked away. 









\subsection{Ramiel goes mad}
Ramiel's sanity is suffering.
Things are happening to his body.
Unbeknownst to him, he is \hr{Ramiel becomes Neo}{slowly mutating into a \neoresphan}. 
He has nightmares and even daymares where he feels his body transforming into something loathsome and unnatural and inhuman. 
He sees into the Beyond, even moves into it, and sees glimpses of his expanded \neoresphan body.
It scares him. 
He fears what is happening to him. 
Part of him is horrified at the though of what will happen if he goes through with his mad plan to become a god. 
Maybe he should turn back and live a quiet life as a normal \human. 

But he goes on. 
He banishes any scruples and fears he has. 
He stubbornly and fanatically continues with his mad quest to become a god, even though his own body and mind beg him to stop. 

This is a part of \quo{\hs{Ramiel's tragedy}}. 
It is a hoax to make the reader think that Ramiel is damned by his own corruption and fanaticism and doomed to fail. 
I hope to surprise the reader when Ramiel conquers and actually becomes a god. 





\subsubsection{Voices}
Ramiel heard voices in his head. 
There was one voice that was particularly loud, deep and menacing.
It lay at the bottom of it all.
Perhaps the mightiest and most dangerous of them all. 
He could not subdue it, and he could not flee from it. 
That was the voice of \Nexagglachel. 

\begin{prose}
  \Nexagglachel:
    \quo{%
      You will fail, \resphan.
      Your whole race will fall.
      You will destroy yourselves.
      You will devour yourself from within.
      You will all give penance for your crimes.}
\end{prose}
 







\subsection{The \Feud}
\target{Resphain are winning the Feud}
From the \Shrouding{} and up until this time, the \feud{} had been pretty much a war of attrition. 
And the \resphain{} were winning. 
Their numbers were dwindling, true, but the \dragons{} were dwindling faster. 
Besides, those \resphain{} who survived were the strongest, \hr{Resphain grow stronger}{and they kept growing stronger}. 







\subsection{To break or preserve the Shroud}
\target{To break or preserve the Shroud}
The \dragons{} originally created the Shroud. They did not care for progress. The \resphain{} \emph{do} care for progress, so they want to get rid of the Shroud. But they are afraid that it will give their \draconic{} foes access to more \xsic{} power, so they will not unravel the Shroud before their \matrix{} has control. 

Both \Daggerrain{} and \Secherdamon{} plan to sunder the Shroud and bring in their forces to lay waste to \Miith{} (see section \ref{Daggerrain's master plan}). For both parties, this is \hr{Summoning the XS: A deadly balance}{a delicate balance}.

Our \quo{heroes}\prikker wait, who are our heroes? \Narkiza{} and \Dzasselid? \MoroCobrel? \Cuezcans? Imetrians? \Ophidians? Remnants of the Redcor and Geicans?

Anyway, our heroes initially want to destroy the Shroud and free the people of \Miith{} from its cruel oppression and their imprisonment as witless slaves and defenseless victims, easy prey for all the horrors that lurk in the universe.

But towards the end, it becomes apparent that unravelling the Shroud is not a good idea. It will enable \Daggerrain{} or \Secherdamon{} or both to carry out their plan, and whoever wins that struggle in the end, the consequences for \Miith{}'s inhabitants will be cataclysmic. So the heroes realize that, hateful as it may be, if they wish to protect \Miith{} they will have to fight to preserve the Shroud.

The whole picture of alliances is very chaotic at this point, with factions splintering and turning on each other at the drop of a hat.

Remember to have some sad philosophy about how tragic it is that while the Shroud is a wicked abomination that keeps people down as ignorant slaves and hapless victims of the world around them, the same Shroud is still the world's best defense against all-out invasion and another apocalypse. 
The heroes have to fight to keep the world as a prison. 





\subsubsection{\Ishnaruchaefir wants to preserve the Shroud}
\target{Ishnaruchaefir fights to preserve the Shroud}
% \Ishnaruchaefir{} , but only in secret at first. Later he openly declares his opposition to \Secherdamon{} and overtly fights against him. Perhaps he becomes a leader of sorts for the opposition, like Elric in Michael Moorcock's \emph{Stormbringer}.
\Ishnaruchaefir fights against \Secherdamon.
He wants to preserve the Shroud. 
He has fought so hard and sacrificed so much to establish the Shroud and keep it alive, and he is not going to let it crumble now without a fight. 
The Shroud is a horrible, suffocating curse, but it is the world's best hope, for without it \Miith would be at the mercy of the world-devouring horrors of the Beyond. 

\Ishnaruchaefir: 
\ta{%
  A balance has reigned over the \Matrices{} of the \Feud{} for thousands of years, but it is slipping. 
  The \banes{} are gaining strength. 
  I cannot have that. 
  There are times when I have held a stalemate preferable to any end to the war. 
  But if any side is to win ascendancy, I want it to be my own people.}

\Ishnaruchaefir recruits mortal heroes to help him in his many plots\dash for though he is first and foremost renowned as a warrior, \Ishnaruchaefir can scheme with the best of them when he must. 

He negotiates with the Imetric gods and maybe recruits \hs{Telcastora Ilcas}. 

\Ishnaruchaefir is not stupid enough to put all his eggs in one basket, though.
He knows that for all his efforts, his enemies might yet succeed in destroying the Shroud.
And so he begins to \hr{Ishnaruchaefir and Azraid plot together, late in TBW}{conspire with \Azraid} and prepare the \hs{Ark} project. 





\subsubsection{Mortals fight against \iquin}
Some mortals have learned the evil purpose behind \iquin. 
They fight to stop it. 

Describe \hr{How it feels to learn Iquin is evil}{how it feels to learn that your religion is really evil}. 









\subsection{The final purpose of the \Sephiroth}
The \hr{Sephiroth}{\Sephiroth} have \hr{Final purpose of the Sephiroth}{a grand purpose}. 
It ties in with the \hs{Morbus} and the \hr{Purpose of Humanity}{purpose of \humanity}. 







\subsection{\Kezerad}
But the \hr{Kezerad}{\Kezeradi} are a danger to the \Sephirah{} project. They could potentially break the project, because they still have a \hr{Kezeradi telepathy}{deep connection to the \Sephiroth}.

By the end, some \Kezeradi{} have become main characters. Or at least one, namely \hr{Sithiyacaan}{\Sithiyacaan}, the last \Kezeradi{} prince. He and his followers work together with \Ishnaruchaefir{}\dash to some degree at least, although \Ishnaruchaefir{} was never exactly a friend of \Kezerad. 

The Cabal knows that the \Kezeradi{} have the power to fuck up their plan, so they do everything they can to hunt down the survivors.





\subsubsection[Sithiyacan]{\Sithiyacaan}
Introduce \hr{Sithiyacaan}{\Sithiyacaan}, the last \hr{Kezerad}{\Kezeradi} lord. 





\subsubsection{An undercover \Kezeradi}
A long-running \resphan, perhaps \Achsah, turns out to be an undercover \Kezeradi. Perhaps she is half \Kezeradi{} and openly condemned her heritage while secretly sympathizing with her people.









\subsection{Eschaton drawing nigh}





\subsubsection{Epidemies ravage \Azmith}
Epidemies ravage all of \Azmith, now that the \Iquin/\hr{Morbus}{\Morbus} plan is nearing its culmination. 
Entire countries are depopulated and collapse. 

It is a consequence of the life-drain inflicted by \Iquin, and also a side effect of the growing influence of the \banelords and their Entropy.
Disease is a manifestation of Entropy.

\citebandsong{Nile:AmongstheCatacombsofNephrenKa}{Nile}{
  Pestilence and Iniquity:
}{
  Heralds of Pestilence\\
  Blackest Plague Rusheth through the Land\\
  Burning Evil Winds\\
  Carry Sickness\\
  Invoking the Bitter Venom of the Gods

  Loathsome Sickening Stench of the Defiled\\
  A Cesspool Breeding the Unclean\\
  Hordes of Locusts\\
  Fiends of the South Winds\\
  Cleanse the Earth from the Impure

  The Daemon That Siezeth the Body\\
  The Daemon that Rendeth the Body

  Ruthless and Profane\\
  Lord of all Fevers and Plagues\\
  Grinning Dark Angel of the four Wings\\
  Spawn of Eng\\
  Horned God with rotting Genitalia\\
  Pazuzu
}









\subsubsection{Cataclysmic events}
\target{Eschaton: RamielsAwakeningBook}
There should be some apocalyptic events before the last book of \SentinelsofMithEmph where much of \Azmith is devastated and the truth of the \xs, \Iquin and perhaps even the \banes is revealed to the mortal masses. 

Hordes of Chaos and/or Darkness invade \Azmith, like in the \emph{Warhammer} game. 

The surviving mortals suffer.
They know their world has been destroyed and they are now living in a Hell on Earth.

Have scenes of madness and destruction like in \cite{HPLovecraft:TheCrawlingChaos}. 

See also the general section about the \hs{Eschaton} and the section about how \hr{The world goes mad}{the world goes mad}. 

\citebandsong{Nile:FestivalsofAtonement}{Nile}{
  Extinct
}{
  Paradise Lost\\
  Dreaming of Extinction\\
  We wander through the Walls of Sacrifice\\
  Sick winds brush against my Skin

  Power of Extinction\\
  Growth Intelligence No More\\
  Mud Rot Skeletal Earth\\
  Drown thy spirit of Kings

  Society's walls break Down\\
  \Humans Pound Down\\
  Dig I must dig Out\\
  Surviving puts all tools in Place\\
  Only peace comes in Death

  As I sleep I dream of Death\\
  Only Peace comes in Death
}





\subsubsection{Omens of the Eschaton}
Have mystic omens of the impending end of the world. 
Compare to the jade shards that rain from the sky in \cite{StevenErikson:DustofDreams}. 









\subsection{Armies of the undead}
The Sentinels manage to raise great armies of the undead. 
This is a continuation of what happened during \hr{Mephilex rises to power}{Suthis Mephilex's rise to power in \Yormis}. 
In this context, the servitors of \Thessulax struggle against those of \Secherdamon, for the two have different ideas of how to enact this \quo{zombie apocalypse}. 

The end result is that vast armies of undead\dash mortal and immortal alike, as well as beasts and monsters\dash march across the land.









\subsection{Looming confrontation between Ramiel and \Dasteron}
Throughout the book, the upcoming conflict between Ramiel and \Dasteron looms. 
It is a source of tension. 
Everyone knows it will happen and is waiting for it to erupt. 
Compare to the promised battle between Karsa Orlong and Rhulad in \cite{StevenErikson:ReapersGale}. 














\section{Ramiel's Search}





\subsection{\Vorcanths{} are mad at Ramiel}
The \hr{Vorcanth}{\vorcanths}, previously Ramiel's allies, had grown mad at him. 
They liked \Shiaraid{} and were nonplussed that Ramiel had backstabbed and killed her. 
They demanded he make it up to them and do them some favours before they would help him again. 





\subsection{Ramiel searches moons for answers}
Ramiel searches the moons, hoping to find answers and peace from his inner Chaos (and \hr{Curse}{\NexagglachelsCurse}). 
Especially \hs{Visha}, the moon associated with the \hr{Vorcanth}{\vorcanths}. 

\lyricslimbonicart{Enthralled by the Shrine of Silence}{
  Cold jewel Moon, I found shelter in your shade \\
  as wind and time took me astray. \\
  I'm floating down the river of forgotten names \\
  into a dark and calm water that devours my flame.\\
  There I find rest and glorious peace, \\
  a sanctuary of eternal bliss.
}

\lyricslimbonicart{Seven Doors of Death}{
  Restless days and sleepless nights.\\
  Drawn in the direction of the moon.\\
  Infernal magnet to mysterious destiny,\\
  beyond the grave of doom.
}

He wants to be rid of his frail \human{} body and return to his true form. 

\lyricsbs{Hate Eternal}{I, Monarch}{
  In this being, in flesh, there can be no absolution.\\
  Therefore I must shed my skin, in a world so fabled, so false.\\
  
  In this shell, in flesh, there can be no solitude.\\
  I will not live in this facade, in a world of contradiction.\\
  
  I, Monarch, master of what shall be.\\
  I, Monarch, captor of what I seek.\\
  I, Monarch, victor of all battles.\\
  I, Monarch, sovereign of this domain.
}









\subsection{Ramiel sees through \Iquin}
Ramiel realizes the true nature of \iquin. 
He is himself a \carcer, and his growing self-understanding allows him to understand \iquin{} as well, and see it for the carcer it is. 

\citebandsong{DeathspellOmega:FromtheEntrailstotheDirt}{%
  Deathspell Omega
}{
  Mass Grave Aesthetics
}{
  The dimension of ethereal totalitarianism discloses itself \\
  And takes possession of the quintessential human soul \\
  Like a nail hammered through most tender flesh \\
  Aeons separate the one \\
  whose eyes have seen through the night of the spirit \\
  The king, the Lord of hosts, draped in terrifying magnificence \\
  From the gleaming clot of trembling vermin
  
  This is you, nourishing \\
  the grand Mass Grave Aesthetics!
}









\subsection{Psychics see portents of the end of the world}
Gradually, as \ps{\Daggerrain}{} master plan unfolds (including the spread of the \hr{Morbus}{\Morbus}), psychics (people who can see through the Shroud) all over \Miith{} see portents of the end of the world: 
Death and emptiness, \hr{Morbus}{disease} and horror, Entropy and torturous decay. 









\subsection{Ramiel and \Ishnaruchaefir}
Ramiel meets \Ishnaruchaefir. 

\begin{prose}
  \Ishnaruchaefir:
  \ta{Ramiel. I hear you destroyed \Shiaraid.}
  (He studies Ramiel's reaction.)
  \ta{You loved her.}
  
  \Ishnaruchaefir{} fingers his glaive.
  He also destroyed his beloved once, remember. 
  A wordless \quo{\trope{NotSoDifferent}{Not So Different}}-conversation passes between them. 
  They go their separate ways with a newfound respect and mutual understanding. 
\end{prose}









\subsection{Ramiel meets \Cishiel}
\target{Ramiel meets Cishiel in person}
Ramiel is re-united with his long-lost daughter, \hr{Cishiel}{\Cishiel}. 
At first he does not recognize her. 
She was a young girl when he last saw her.
Now she has grown into a mature and very powerful \resvil. 

She has searched for him. 
She wants him back and will support his bid for the throne. 
But she also resents him for abandoning her to fend for herself back when she was a defenseless little girl. 

\target{Ramiel regains guns}
She gives him back \hr{Ramiel's guns}{\Strith{} and \Currah}, the two pistols he once used to wield. 









\subsection{Ramiel alone at sea}
On his journey north, Ramiel gets separated from his companions.
He drifts alone on the sea.
He driven half mad by his terror of the sea.
He ends on a dark and slime-covered island.
Compare to \cite{HPLovecraft:Dagon} and \cite{EdwardPickmanDerby:Azathoth}.









\subsection{Ramiel journeys to \UltimaThule}
\target{Ramiel journeys to Thule}
Ramiel sails north. 
He journeys to \hr{Thule}{\UltimaThule} in the hope of regaining his memory and power. 

\UltimaThule is in a different Realm. 

His goal was a particular lost temple city in \UltimaThule that had once been a sorcerous bastion (of some race), but was now abandoned and ruined.
Perhaps this was even a \voyager citadel.

\lyricsbs{Bal-Sagoth}{%
  Starfire Burning Upon the Ice-Veiled Throne of Ultima Thule
}{
  For how long have we \travelled?\\
  The memory grows dim, lost in the cruel, searing storm-winds.\\
  With the blessings of the Elders we began our journey, \\
  beyond the great veil of shadowed glaciers.\\
  They spoke of a prophecy foretold, an ancient and glorious legacy.\\
  A quest for the realm of legendry, \\
  lost to Man since before even the star-lords descended.
}

He gazes out over the vast, black sea and the darkened, stormy sky, dimly lit by a faint, mystic, spectral luminescence, from the clouds or the northern lights. 

Compare to \bandsong{Bal-Sagoth}{Journey to the Isle of Mists (Over the Sunless Depths of Night-Dark Seas)}. 

\lyricsbalsagoth{Invocations Beyond the Outer-World Night}{
  Seeking answers to the cryptic riddles of the universe,\\
  Secrets of the blackest, most impenetrable deeps of the umbra,\\
  Wreathed in frozen shadow and ice-bound peril,\\
  Subterrene halls of horripilated wonderment\prikker
  
  Tatsumaki Maru voyage north, ever north!\\
  Cleave a path through the massing Arctic ice!\\
  Agleam with all the \colours of the aurora,\\
  Far beyond Ny Alesund lies our goal.
  
  Wreathed in frozen shadow and ice-bound peril.\\
  Agleam with all the \colours of the aurora.\\
  The portal to the tenebrous cryptic core \\
  of this world's subterrene inner sanctums.\\
  Quaere verum\prikker \\
  Sic itur ad astra!
  
  Invocations and ideograms (dreams of the Xytaxehedron?),\\
  Conjuration of the inner world's (tenebrous) denizens,\\
  And their star-spanning progenitors, \\
  spawned beyond the outer-world night.
}

He feels deep inside him that he is nearing enlightenment. 

\lyricslimbonicart{Beneath the Burial Surface}{
  Water from a thousand tears floats in streams.\\
  The feeling from a thousand years flow over me.\\
  As I once again return to the cemetery gate\\
  I hear the dismal call from the hollow grave.
}

They encounter vast, violent storms of ice and snow. 

\citeauthorbook[p.208--209]{TimCurran:Hive}{Tim Curran}{Hive}{
  Hayes could see it out there in that haunted blackness, the headlights clotted with snow thick as a fall of flower petals, thick as the dust blowing through the decayed corridors of a ghost town.
  It was more than just a Condition One storm with near-zero visibility and winds approaching a hundred miles an hour and snow falling by the bails, pushed into frozen crests and waves.
  No, this was bigger than that.
  This was every storm that had ever scraped across the Geomagnetic graveyard of that white, dead continent.
  Pacific typhoons and Atlantic hurricanes, Midwestern tornadoes and oceanic white squalls, tempests and blizzards and violent gales\prikker all of them converging there, bled dry of their force and suction and devastation, reborn at the South Pole in a screaming glacial white-out that was sculpting the rugged landscape in canopies of frost, leeching warmth, driving blood to freon, and pushing anything alive down into a polar tomb, a necropolis of black, cracking ice. 
  
  \prikker 
  
  But there were other things on the storm.
  
  Things funneling and raging in that vortex that you could only feel in your soul, things like pain and insanity and fear.
  Maybe wraiths and ghosts and all those demented minds lost in storms and whirlwinds, creeping things from beyond death or nameless evils that had never been born\prikker the gathered malignancies and earthbound toxins of that which was \human and that which was not, writhing shadows blown from pole to pole since antiquity.
  Yes, all of that and more, the collected horrors of the race and the sheared veil of the grave, coming together at once, breathing in frost and exhaling blight, a deranged elemental sentience that howled and screeched and cackled in the shrill and broken voices of a million, a million-million lost and tormented souls. 
  
  Hayes was feeling them out there on that moaning storm-wind, enclosing the SnoCat in a frozen winding sheet.
  Death.
  Unseen, unspeakable, and unstoppable, filling its lungs with a savage whiteness and his head with a scratching black madness.
  He kept his eyes fixed on the windshield, what the headlights could show him: snow and wind and night, everything all wrapped and twined together, coming at them and drowning them in darkness.
  He kept blinking his eyes, telling himself he wasn't seeing death out there.
  Wasn't seeing spinning cloven skulls and the blowing ,rent shrouds of deathless cadavers flapping tlike high masts.
  Boiling storms of sightless eyes and ragged cornhusk figures flitting about.
  Couldn't hear them calling his name or scraping at the windows with white skeletal fingers. 
}





\subsubsection{Ramiel is wounded}
Ramiel becomes wounded and crippled. 
He gets a bad leg or a bad arm or something like that.
He needs help to travel.

He keeps over-exerting and over-pushing himself with magic.
He wears down his body, and his condition gets ever worse.
When at last he reaches the temple he looks horrible.

Ramiel is meant to look like a villain who gets what is coming to him. 
But then he triumphs.
When he regains all his \resphan power and immortality he is able to heal his body.





\subsubsection{I know you are in there somewhere}
In a fight or the like, someone tells Ramiel this: 

\begin{prose}
  Someone: 
  \ta{Carzain! I know you are in there somewhere!}
  
  Ramiel: 
  \ta{%
    Oh, no. You have gotten it mixed up. See, it was never Carzain who was \quo{in there}. Back in the day, Carzain was up-front, and I, Ramiel, was \quo{in there} within him.
  
    You cannot break through to the \quo{real me}. 
    I \emph{am} the real me.
    I am as real as I can get.}
\end{prose}















\section{At the Temple}
\subsection{Ramiel reaches the temple}
\target{Ramiel's awakening at the temple}
\target{Ramiel's awakening in the temple}
\target{Ramiel's awakening}
Ramiel reaches the temple, far to the north. 





\subsubsection{First impressions of the temple}
It is a \Mystraacht{} citadel, but built in \ophidian{} style, since \Mystraacht{} has conspired with the \ophidians. 
It is not only a temple, but a cosmic gateway to the Throne of the \Mystraacht{} Overlord, ensorcelled by \resphan{} and \ophidian{} power alike. 

It is frightening, slime-covered, alien. 
Compare to R'lyeh from \cite{HPLovecraft:TheCallofCthulhu} and other Cthulhu Mythos stories. 
Also compare to \cite[p.22--25]{TanakaHirofumi:TheSecretMemoiroftheMissionary}.

It is full of ancient alien machinery. 
Compare to the Krelran temple in Arellarti in \cite[p.49]{KarlEdwardWagner:Bloodstone}. 

\lyricsbs{Bal-Sagoth}{
  Starfire Burning Upon the Ice-Veiled Throne of Ultima Thule
}{
  Swathed in Moon-frosts, in icy winds our blazon flying.\\
  Iron gleaming 'neath the stars, black skies ablaze with astral fire.\\
  White wolves (like silent spirits) haunt us, ever northwards.\\
  The ice-gem leads us, glimmering.\\
  Powerful spells entwine the shrine of legendry,\\
  mighty gates of frozen splendour looming.\\
  When the moon and stars shine as one upon the snows, \\
  the ancient ice-gate opens, the prophecy is fulfilled!
}

\lyricslimbonicart{Lycanthropic Tales}{
  As the storm looms over the frozen landscape:\\
  A call from the beginning of time,\\
  out of darkness through the mist.
  
  Requiem in blood and fire.\\
  Symbols of occult desire.\\
  Through the cosmic vortex,\\
  the gate to unknown darkness.\\
  An esoteric voyage to eminent discovery.\\
  I am the deathlike shadow,\\
  architect of black mysteries.
}





\subsubsection{They fight their way in}
He and his companions fight their way to the temple, past its terrible guardians. 

\lyricsbs{Bal-Sagoth}{
  Starfire Burning Upon the Ice-Veiled Throne of Ultima Thule
}{
  Towering, ice-encrusted forms lumber forth from the freezing mist,\\
  their eyes shimmering with a fiendish, eldritch malevolance\prikker\\
  Our steel is raised against their weapons of gleaming crystal.\\
  And the virgin snow is rendered crimson by bloodshed 
  in a searing storm of slaughter.\\
  Wounded, dying, my flesh rent by weapons no human ever forged or wielded, \\
  I am beckoned forward by a strange, alluring force \\
  from beyond the veil of swirling mists\prikker
}





\subsubsection{They explore it}
He explores the mystic temple. 

\lyricslimbonicart{Infernal Phantom Kingdom}{
  After years of dormancy\\
  in a cosmic mausoleum,\\
  arcane cemetery.\\
  The soil is cursed and sour.\\
  A rotten landscape draped in horror.\\
  Evil has a way of returning.\\
  You can not hide from hell's eye.\\
  It is always burning;\\
  a way of returning.
}

\lyricsbs{Bal-Sagoth}{
  Starfire Burning Upon the Ice-Veiled Throne of Ultima Thule
}{
  Shadows, images form \\
  in the glittering rune-carved walls of this glacial chamber,\\
  aecrets frozen within the timeless vaults of eternity.\\
  The throne of the time-lost ice realm, \\
  entwined in the mantle of such searing star-born power.\\
  This frozen, aeon-cloaked seat of immortal majesty,\\
  of an empire forged long before the vast seas rose in devouring fury!
}





\subsubsection{Insight comes}
Here he regains some of his memory. 
He remembers his life as a \resphan{} lord. 
He recalls the march of the \Mystraacht{} \hr{Glorious armies}{armies}, the incounterable legions that once forced the world to its knees. 
Back in those days, the \banes{} were weakened and mostly absent after their great defeat in the \secondbanewar{}, so the \resphain{} were left to their own designs, and they fought among one another. 
\Mystraacht{} was the most powerful dynasty, but they were betrayed, their Overlord assassinated, and they sank into the mire of civil war. 
(See section \ref{Mystraacht betrayed}.)

\lyricsbs{Bal-Sagoth}{
  Starfire Burning Upon the Ice-Veiled Throne of Ultima Thule
}{
  What shimmering swords raised in combat \\
  once sang with the glorious clamour of steel on steel?\\
  What splendid banners, billowing in the icy gales, \\
  once heralded the march of these invincible silver-clad legions \\
  to the blood-swathed embrace of epic battle?\\
  The glory of untold thousands of years past\prikker \\
  this ethereal legacy of mighty Ultima Thule.\\
  The frozen eyes of immortal kings watch me\prikker such a dark splendour!
}

Perhaps Ramiel reads the story of his people on the walls, like in H.P. Lovecraft's \emph{At the Mountains of Madness}. 

\lyricsbs{Bal-Sagoth}{
  Starfire Burning Upon the Ice-Veiled Throne of Ultima Thule
}{
  These ancient carvings in a time-veiled tongue,\\ 
  etched into the primeval ice countless aeons ago, \\
  now bathed in diaphonous incandescence \\
  by this storm of lucent stellar power, \\
  their mindsearing meaning at last becomes known to me.\\
  Their cosmic secrets unfold.
}





\subsubsection{Ramiel remembers his past}
Ramiel remembers the bloody wars he fought against other \resphain{} and against \dragons{} and \ophidians. 

He remembers fragments of \Ishnaruchaefir. 
The two have a history together. 
They are enemies, but respecting enemies. 
Almost like friendly rivals. 

\lyricsbs{Bal-Sagoth}{
  Starfire Burning Upon the Ice-Veiled Throne of Ultima Thule
}{
  And then, enlightenment comes, gleaming down upon my consciousness as the bright moon gazes down upon this auroral vista\prikker From my mind is lifted an obscuring veil, a veil induced by sorcerous arts, and I realize I have been merely a vassal of another's twisted will, a pawn in a game which is entwined in treachery and malign aspirations to thresholds of great power. 
  
  Such a traitorous web has been spun! The elders of my kingdom bow in obeisance to the vile priests of Xothan'kur, and it is their diseased machinations which have urged me here, to the very heart of the far-fabled ice realm\prikker for they seek to usurp the power of the Conjunction, stealing the vast energies of the Ice-Veiled throne and absorbing them into their own leprous, undead bodies, perpetuating the adoration of their abhorrent liege for countless ages, liberating his vile will and enslaving the realms of the world\prikker
}

He invokes \banes, \xss{} and cosmic gods. 

\lyricsbs{Hate Eternal}{Beyond Redemption}{
  Awaken from below I call thee,\\
  thy master pure in darkness,\\
  the one of hate before thy paths\\
  were chose to be forsaken.\\
  From depths of time in portals,\\
  through empty eyes of light,\\
  I searched the texts for knowledge.\\
  Now I bleed the sacrifice.
  
  I am of purest power. \\
  Strengths of a thousand souls.\\
  I hear the voices of\\
  the oldest ones burning within.\\
  Reveal the truth unto me, \\
  for I'll command the worthy.\\
  March through the ruins of diminished tribes.
}

He swears to retake the throne and reclaim power. 

\lyricsbs{Hate Eternal}{Beyond Redemption}{
  I summon winds of fire.\\
  I summon winds of disease.\\
  I summon torture upon\\
  all my rivals who shall bleed.\\
  Confronting all before me\\
  to tempt what I have in store.\\
  Don't tempt my fury for\\
  the flames shall burn eternally.
}







\subsection{\Azraid{} begins stockpiling \Erebean{} power}
\target{Azraid stockpiles Erebean power}
\Azraid{} sees portents (mystic or mundane?) that the \banes{} are reeling and about to be defeated. So he begins stockpiling \Erebean{} power, draining it away from \Erebos{} and the \banelords{} and hoarding it for himself. This is effective, because \Daggerrain{} is not expecting \Azraid{} to betray him. \Azraid{} has always been one of the most loyal \resphain. 

Remember to have subtle references to \ps{\Azraid} evil hand in all \Azraid{} chapters. 









\subsection{Ilcas dies}
\target{Ilcas dies}
\target{Sithiyacaan awakens}
Telcastora Ilcas charges heroically into battle and dies.

\citebandsong{BlindGuardian:NIME}{Blind Guardian}{Time Stands Still (at the Iron Hill)}{
  He gleams like a star and the sound of his horn's\\
  Like a raging storm\\
  Proudly the high lord challenges the doom\\
  Lord of slaves he cries

  Lord of all Noldor\\
  A star in the night and a bearer of hope\\
  He rides into his glorious battle alone\\
  Farewell to the valiant warlord
}

This is the catalyst that makes \Sithiyacaan awaken. 
But not in time to save Ilcas. 
\Sithiyacaan now realizes that he has to use his full powers to save the world. 

\citebandsong{BlindGuardian:NIME}{Blind Guardian}{Time Stands Still (at the Iron Hill)}{
  Finally I've found myself in these lands\\
  Horror and madness I've seen here\\
  For what I became a king of the lost?\\
  Barren and lifeless the land lies
}









\subsection{Ramiel is still bound}
Even after his enlightenment at the \Mystraacht{} temple, Ramiel is still bound. He is not entirely free of the Shroud, still trapped in a \human{} body, and without access to his full \sathariah{} power. He has a lot of power now, but he does not completely understand it. 

And not only that: He is under the effect of spells that chain his mind. He has enemies who want to manipulate him, use his power as a weapon of their own. They don't want him to awaken completely. Ultimately, they want to use him in their plan, where he is to expend all his power and be destroyed himself. Perhaps they plan to directly usurp his power, absorbing it for themselves and destroying him\prikker

No! Better idea: They want to create a new kind of \Sephirah, a more powerful one, but under their control. They want to transform Ramiel into one such. In order for that to work, he must be awake and have access to his full \sathariah{} power\prikker but he must awaken while under their control, so that immediately upon his awakening, they can chain and spellbind him and transform him into their new \Sephirah. 

Perhaps this is connected to the destruction of the original \Sephiroth. Perhaps Ramiel is to become the first of a new generation of \Sephiroth{} that will take over once the current ones have served their purpose. 

At any rate, Ramiel learns too much at the temple. Or, rather, he has learned too much before he came to the temple, so he is able to piece together more of the big picture than they expected him to. He suspects their scheme and takes active measures to undermine them, lay traps for them, infiltrate their ranks, so that in the end, he will be stabbing \emph{them} in the back. 







\subsection{Ramiel's enemies}
Ramiel has \hr{Ramiel's enemies}{enemies among the \resphain}. 
Have a Scabandari Bloodeye-like character. 







\subsection{Ramiel is broken down and rebuilt}
\target{Ramiel's final awakening}
He has regained much of his memory, but he is still trapped in the Shroud and must break free and be reconciled with his true self before he can regain his true \resphan{} form and his \sathariah{} power. 

He undergoes several mental torture. I don't know if this is voluntary or if he's trapped and imprisoned. Anyway, his Shrouded personality must be torn apart and broken down before his true self can re-emerge. 

\Cishiel{} and \Gilchad{} help him. 

Ramiel does not like \Gilchad. 
He is an obnoxious, condescending snob. 
He thinks he is smarter than everyone else and does not give Ramiel the respect he thinks he deserves. 

During the process of Apotheosis, Ramiel gradually comes to remember all the bits of cosmic knowledge he has gleaned over his many lives, the many Aenigmata whose Gnosis he has glimpsed. 
Back then he usually understood very little of it, but now, with the full knowledge of his many lives, he is able to see this knowledge from many exciting perspectives, and suddenly it all makes sense.
So immense amounts of horrid, sinister insight crash down upon his mind like a tidal wave.
With his multi-faceted experience he is able to understand much more than he otherwise would.
He attains Gnosis that no \resphan has ever held before him, save perhaps \Azraid.

And it is hard on him,
He screams in terror and anguish and has to fight a desperate and bloody battle against the many inner \daemons of his thoughts and fears to avoid losing his mind entirely.
His sanity is lashed like a vessel on a storm-wracked sea.
\hr{Ramiel is traumatized from awakening}{Afterwards he is traumatized and shaken}. 





\subsubsection{\Cishiel is worried}
\Cishiel is worried.
She fears her father has gone mad, and that his new power will make him even more mad and more dangerous.
But she knows it is too late to turn back.
She has cast in her lot with Ramiel and cannot hope to betray him, even if she would\prikker which she would not.
He is still her father and she loves him. 
He saved her from the horrible fate that befell him.
She knows that had he not forbidden her to become a \malach herself, she would have had to undergo the same torture and madness that Ramiel has endured - or she might have been destroyed long ago, or still trapped in the body of an amnesiac human.
No, she owes everything to Ramiel and still has more to repay him.

She feels more anguish for \Dasteron, her other ally, for she fears Ramiel will destroy him.
Ramiel looks terrible in his fury (when he \hr{Ramiel kills a Resphan after awakening}{eats a \resphan}), and \Cishiel fears no one will be able to stand against him, not even the formidable and resourceful \Dasteron.





\subsubsection{Voyage into inner worlds}
Perhaps the awakening process should be presented as a dream-quest, a journey into imaginary worlds that exist/are formed within his mind. 

Compare to Elric's journey beyond the Shadow-Gate in \cite{MichaelMoorcock:ElricofMelnibone}. 





\subsubsection{Facing his fears}
He is confronted with all his fears, sorrows, regrets, shame and pain. He must face, learn and accept the entirety of the brutal truth of his people's cruel nature and monstrous origins, the nature of \humans{} as a slave race, and his own history as a cruel conqueror and betrayer. 

Compare with the \emph{thetalos} cave in \authorbook{Jacqueline Carey}{Kushiel's Chosen} or Chia's caves in \authorbook{Stephen Marley}{Shadow Sisters}. 

Before he begins he promises himself that he will win. 

\citebandsong{BeyondTwilight:SectionX}{Beyond Twilight}{%
  The Path of Darkness%
}{
  I smell the gasoline\\
  I smell the fire\\
  I'll make it all so clear\\
  This is my mentor
}

\citebandsong{BeyondTwilight:LurkingFantasia}{Beyond Twilight}{Rage}{
  Trying to face my sanity. \\
  But I'm too afraid to confront with it. \\
  Reaching out. \\
  For what I do not know.\\
  There's no way to explain this Hell I'm in. \\
  Can you? Can you? 
  
  No source of power to take from and wake \\
  from this nightmarish lunacy \\
  Is there a meaning or is this \\
  just a test of destiny? \\
  Creating these monsters for self purposed riches \\
  and glory in the end \\
  Thousands of victims with fresh schizophrenia \\
  isn't out here to pretend.
}

\citebandsong{BeyondTwilight:TheDevilsHallofFame}{Beyond Twilight}{%
  Godless and Wicked%
}{
  Parts of my memory have been erased.\\
  You've destroyed my files\\
  Oh but I'm getting nearer\\
  It's all getting clearer 
  
  Look for a passage way\\
  Wait for the judgement day\\
  The truth will be told\\
  And my dreams unfold \\
  I'll find a way 
  
  I'm half alive here\\
  But I have no fear\\
  Cause I'll find the light yeah 
  
  Time to changes\\
  Everything re-arranges
  
  I'm getting ready\\
  At the break of dawn \\
  The fight is on 
  
  Godless and wicked\\
  Creepy and cold
}

With all his might he clings to the good memories of grandeur. 

\citebandsong{BeyondTwilight:SectionX}{Beyond Twilight}{%
  The Path of Darkness%
}{
  Send yourself to get back night\\
  I see women crawl\\
  It's dusty and mid in mind\\
  This is my intention
}

He wants to prove that he is the boss. 

\citebandsong{BeyondTwilight:SectionX}{Beyond Twilight}{%
  The Path of Darkness%
}{
  I've seen me set the light\\
  This is the mentor\\
  I met cold on this ride\\
  This is the centre
}

But dangers lurk within his mind. 

\citebandsong{BeyondTwilight:SectionX}{Beyond Twilight}{%
  The Path of Darkness%
}{
  Don't let it fall\\
  You're not you, you're me\\
  Don't let it get a taste of your blood\\
  Yeah you're feeling small\\
  Don't look at me, don't look at me\\
  You're dancing with the Devil in my hall
}

He almost wanders lost in the dark corridors of his mind. 

\citebandsong{BeyondTwilight:SectionX}{Beyond Twilight}{%
  The Path of Darkness%
}{
  Now am I lost? Deep in my soul. \\
  There will be no forever.\\
  Through time I'm tossed. A wish without hope. \\
  Into the never.\\
  As time stands still, my spirit grows cold.\\
  I fall into darkness, as black as my soul.

  Now am I lost?\\
  Without hope into the never\\
  Deep in my soul
}

He holds tight. 

\citebandsong{BeyondTwilight:SectionX}{Beyond Twilight}{%
  The Path of Darkness%
}{
  Crawl for your master\\
  You know I'm in control
}

The whole process takes a long time. 

\citebandsong{BeyondTwilight:SectionX}{Beyond Twilight}{%
  The Path of Darkness%
}{
  Smell the gasoline\\
  I light the fire inside\\
  Smell the gasoline\\
  I sway all through the night
}

Finally, slowly he awakens. 

\citebandsong{BeyondTwilight:SectionX}{Beyond Twilight}{%
  Ecstasy Arise%
}{
  From dark waters I rise\\
  My once trembling shell conceals a God
  
  Power surging through the void I soar\\
  Intoxicating visions whispering more\\
  In my radiance every \colour seems to fade
  
  I speak only in thought\\
  Life is not forever nothing is forever\\
  In dark waters I find\\
  Beauty beyond oceans of time
  
  In my radiance every \colour will soon fade away\\
  In the emptiness I will no longer stay
}

It is much like \hr{Shaeeroth ritual}{the process that turns an \ophidian into a \dragon}. 
Ramiel gains insight into many horrible truths concerning his people, the \banes, their \matrices and their purpose.
This is only possible because of the wisdom he has already attained, and it makes him even wiser and stronger, but also less sane, and it brings him a pain and a despair that will continue to haunt him.

\citebandsong{Nile:Ithyphallic}{Nile}{
  Language of the Shadows
}{
  Abandon hope\\
  And I shall become free\\
  And with freedom acquire emptiness

  With the mind cleansed and empty\\
  There is the void known as despair\\
  A gateway upon an emptiness endless and vast

  In despair the language of the shadows is intelligible\\
  In madness all sounds become articulate

  Terror and despair they guide me\\
  Into nightmares that follow one upon the other\\
  Like windblown grains of sand

  [solo: Dallas]

  I have become as the wastelands\\
  Of unending nothingness\\
  Now shall the night things\\
  Fill me with their whisperings\\
  And the shadows reveal their wisdom
}





\subsubsection{My Name Is Legion (For We Are Many)}
This chapter (or perhaps the whole part or even the book) should be named \quo{My Name Is Legion (For We Are Many)}. 
This represents the fact that Ramiel fights his legion of inner demons.





\subsubsection{Fights his old victims}
Maybe Carzain has a dream where he faces all the people he has killed and/or betrayed. 
These are souls conjured from his \carcer.
He has to fight them all. 

Compare to \cite[p.85--86]{RobertEHoward:Ghor}, where Ghor faces all the people he has killed. 





\subsubsection{Regains his memory}
Ramiel regains all memories of all his lives. 
They come crashing back to him like a torrent. 

\target{Ramiel accepts Vizicar's death}
Among other things, he realizes that he has been deluding himself about \hr{Vizicar dies}{Vizicar's death}. 
There were no evil sea gods out to get him. 
It was just a regular accident. 





\subsubsection{Triumphant}
His traumas, his emotions and all his personality as Carzain, Vizicar and others is broken down. 
He is reconciled with his true nature as a \sathariah{} and his \malach{} \hr{Malachim bind souls}{power of binding souls} (and \hr{Ramiel binding souls}{his \carcer}).  
He embraces the enslaved souls and uses them and their power. 
He becomes the true Ramiel. 

He says to \Cishiel:
\ta{Bring me my sword and my guns.}
She swiftly obeys. 

\lyricsbs{Bal-Sagoth}{
  Starfire Burning Upon the Ice-Veiled Throne of Ultima Thule
}{
  And so, enrob'd by tendrils of starfire and the raiments of lunar mist, the immortal liege whose sceptred empire is eternity sits enthroned and brooding over his dark realm once more.
  
  Swathed in moon-frosts, in icy winds our blazon flying. \\
  When the moon and stars shine as one upon the snows, \\
  the ice-gate opens, the prophecy is fulfilled!
}

He tears asunder his \human{} guise and strides forth from the Shroud into true Reality, resplendent in all of his \sathariah{} glory. 

He summons his faithful servants, which are wolf-like creatures. See section \ref{Ramiel's wolves}. 





\subsubsection{Stronger than before}
When he awakens, \hr{Ramiel is wiser from walking the earth}{Ramiel is stronger and wiser than ever before}. 





\subsubsection{Becomes a \neoresphan}
\target{Ramiel becomes Neo}
Ramiel learns that in his long process of Apotheosis and awakening, he has not only returned to his old self. 
He has gradually metamorphosed (fully or partially) into a \neoresphan.
He is not only a beautiful and mighty \resphan. 
He is now a gruesome, slimy, loathsome, abhorrent, inhuman monster. 
This is a horrible truth. 
\Azraid has long been manipulating him, and now Ramiel's tranformation into a \neoresphan is complete. 

The temple has played a big part in this.
It was restored by \Azraid with new machines and spells designed to do exactly this:
Transform a \malach into a \neoresphan. 

This is \hr{Azraid turns Malachim into Neo}{what \Azraid planned all along}. 
Now Ramiel can be inducted into the \hr{Neo-Resphan conspiracy}{\neoresphan conspiracy}. 










\subsection{Kills a \resphan}
\target{Ramiel kills a Resphan after awakening}
Ramiel wants to test his new powers. 
As a full-fledged \malach he has \hr{Malachim binding souls}{enhanced soul-eating powers}. 
He tests it by attacking an eating a \resphan. 

Ramiel zaps his foe with dark lightning. 
Then shoots him with his pistols.
Then blasts him repeatedly with sorcery and lightning.
Gradually he tears his foe apart and, turning himself into a devouring, hungry vortex, sucks in the victim and devours him body and soul.
It is like witnessing a black hole and its accretion disc.





\subsubsection{Kills \Gilchad}
\target{Ramiel kills Gilchad}
Perhaps the first \resphan that Ramiel kills is \Gilchad. 

\Gilchad{} was responsible for guiding him through the torture process. 
When Ramiel finally awakens and emerges, he tells him: 
\ta{I owe you thanks for helping me regain my memory. 
  But I can tell by looking at you that you enjoyed the process. 
  So, with that in mind, I also owe you \emph{this}\prikker} 

And then Ramiel kills \Gilchad. 
And eats him. 

He had wanted to do this for a while, but he needed him. 
Now, for the first time ever in his life, Ramiel has full control over his \Malach{} powers and the associated \carcer. 
This means that he can eat (weak) souls far more easily than any normal \resphan{} or \dragon. 

He also wants to exercise and test that power. 
That is part of the reason for killing \Gilchad. 















\section[Return to Mystraacht]{Return to \Mystraacht: The Host Reborn}
\target{Ramiel returns to Mystraacht}
Ramiel returns to \Mystraacht. 

\citebandsong{BeyondTwilight:FortheLoveofArtandtheMaking%
}{%
  Beyond Twilight%
}{%
  For the Love of Art and the Making%
}{
  He is the prince of darkness\\
  Returning home to take his kingdom
}

Ramiel is now \hr{Ramiel is overpowered}{immensely powerful}.
When he comes back, others \hr{Ramiel is underestimated after Apotheosis}{underestimate his strength}. 





\subsection{Ramiel and \Dasteron}





\subsubsection{Ramiel hears of \Dasteron}
At first, Ramiel's motivation was just to get his powers and memory back. 

Now that he has achieved that, he spends many days doing research, learning everything about the state of the world from records and from \Cishiel's Cabalist contacts.
He covets power and greatness and feels it is his right by succession to become Overlord.

But not only that.
He also doubts \Dasteron's ability.
He returns to \Mystraacht and studies \Dasteron in person.





\subsubsection{Ramiel gets to know \Dasteron}
Ramiel goes to \Mystraacht and meets \Dasteron.
Ramiel hangs out for a while and learns how thing work nowadays.
He greets \Dasteron with respect (of sorts), but no subservience.
\Dasteron reciprocates. (Ramiel is, after all, a \sathariah, and thus in a sense \Dasteron's superior.)
Ramiel comes to \Mystraacht intent on hating \Dasteron, this pathetic usurper who thinks he can be Overlord.
But Ramiel finds, much to his chagrin, that \Dasteron is a good \resphan, a worthy leader and even a potential friend.
This complicates matters.





\subsubsection{Ramiel must dethrone \Dasteron}
Ramiel learns that \hr{Dasteron cannot become Apex}{\Dasteron{} lacks the \vertex{} strength to ascend to the position of \apex{} of the \Mystraacht{} \matrix}. 
This is no good. 
It is one of the reasons why Ramiel knows he has to dethrone \Dasteron. 
He might be a good leader for a while, but if he cannot become \apex{}, his reign is stillborn and without true prospect. 

Ramiel has learned from his travels, and from the insight he has gathered over his many lifetimes, that something big is coming. 
Some cataclysmic metaphysical event. 
And the \Mystraacht{} \matrix{} has to be strong to deal with it. 
Therefore, a true \apex{} is needed. 
Ramiel feels this \apex{} should be him. 

He becomes convinced that, while \Dasteron is a good leader, he is not good enough.
He is a skilled politician and fighter, but he does not have the cosmic insight or \vertex strenght that \Mystraacht needs.
He finds out that \Dasteron shies away from all the darker writings and fears dealing with the \banes.
\Dasteron tends to obey the \banes quickly and with great fear.
\Dasteron is fearless when it comes to his fellow \resphain, but Ramiel fears \Dasteron is not capable of managing \resphan affairs in a larger cosmos.

This makes up Ramiel's mind.
He must dethrone \Dasteron and take the throne himself.
Only he is strong enough to rule \Mystraacht. 

Conversely, \Dasteron finds that Ramiel acts hysterical and manic and irrational.
\Dasteron concludes that Ramiel is insane and unfit to rule. 





\subsubsection{Ramiel laments having to kill \Dasteron}
There is a quiet understanding between Ramiel and \Dasteron. 
They both know that each intends to murder the other rather than submit and subordinate. 

Ramiel grieves. 
He has come to see \Dasteron as almost a friend. 
Moreover, Ramiel is critical of himself. 
He has been traumatized by Daggerrain (as a First Discoverer), driven to madnes by the blood of \Nexagglachel, and has lost himself after thousands of years as an outcast \malach. 
He thinks:
\tho{Who amongs us is more insane than I, more unfit to rule? And yet there is no other who can form the \apex.}

Ramiel also knows that however good \Dasteron is, Ramiel has to dethrone him.
\Dasteron knows this as well.
Sooner or later, Ramiel will challenge him for the throne.
Here is the real tragedy:
Ramiel knows that if he loses their match, he will not submit to \Dasteron's rule.
He will challenge him again and again until he wins, for Ramiel must win.
He also comes to know \Dasteron well enough to know that \Dasteron will do the same if Ramiel wins.

They are both full of pride.
Each believes that he is the better leader, that only he is worthy and capable of leading \Mystraacht.
\Dasteron believes that Ramiel, for all his brute force, is too insane and unstable and dangerous to have as Overlord.
So they conflict is doomed to continue forever, until one of them perishes.

So Ramiel makes a very hard decision.
He will destroy and eat \Dasteron as soon as he wins (if he can, that is). 
This is tragic.
Ramiel will miss \Dasteron.
He is a great \resphan and a friend.
It will be a great loss for \Mystraacht and for the \resphan race, and for Ramiel, and for \Cishiel.
But it must be done.










\subsection[Mystraacht has degenerated]{\Mystraacht has degenerated}
In Ramiel's absence, \Mystraacht{} has degenerated into a crowd of decadent barbarians striving for maximal evil and perversity for its own sake, and out of some misguided ideological idea that this is their true nature and the purpose of \Mystraacht. 

Compare to the Dark Eldar of \emph{Warhammer 40,000}.

Ramiel: 
\tho{%
  We of \Mystraacht{} boast of our independence and free thinking.
  But we are just as bound by tradition as the \KiriathSepher.
  We go with the flow just as much as they do.
  Just look at how I followed \Zachirah.}

Ramiel intends to bring order and restore the true \Mystraacht. 
Bring them back on track and introduce meaning in all the madness. 

Ramiel: 
\ta{%
  I am \ps{\Zachirah}{} only son. 
  I alone am heir to his legacy and his designs. 
  I alone comprehend the scope of his original vision. 
  I alone understand \ps{\Mystraacht}{} true purpose.}
\tho{%
  And I alone understand the flaws in my father's vision and know how to correct them.}

The \Mystraacht{} do a lot of eating and binding souls. 
This is the source of \hs{Ramiel's bound souls}. 

\lyricstitle{Warhammer 40,000: Codex Dark Eldar}{
  As the Dark Eldar died, the air was filled with escaping souls. 
  A roiling mass of blackness hovered on the edge of vision, the screams of spirits in eternal torment sounded on the edge of hearing.
  
  \prikker
  
  The ebbing and flowing of released souls slithered around the Dracon, drawn to her blood and fear. 
  She could feel their wispy tendrils sliding over her, probing gently into her mind. 
  She felt like screaming, but she gritted her teeth and was silent. 
  
  \prikker
  
  The Dark Eldar Lord's eyes glazed over, as he took a long, deep breath. 
  Khirareq felt the spirits around her drifting away, pulled towards the gulf inside the Lord's own soul. 
  The Lord's body twitched spasmodically as he absorbed the freed life essence of his followers. 
  As the spirits of the dead were consumed, Akhara'Keth's spasms increased and a thin dribble of saliva trickled from the corner of his slack lips. 
  With a shuddering sigh, the Lord finished and slumped back in the chair. 
  When he sat forward once more his eyes burned more brightly, his skin was less wrinkled, his hair darker with more lustre. 
  
  \prikker
  
  She stared straight back at the Lord, looking deep into the ancient pits of evil that were his eyes. 
  
  \prikker
  
  \ta{%
    You need to rule?
    What do you know of needs 
    You are young, the Thirst has a shallow hold on you. 
    I will tell you of need; a deep, unfaltering emptiness that grows larger and more demanding with every passing of the night. 
    You have heard tales of how I consume a hundred souls a day. 
    That number is but the morsel to whet my appetite. 
    A hundred times that number die every day to quench my desire, my need.
    Spirits unnumbered are distilled in agony and torture to the peak of exquisite taste to fill the chasm of my soul. 
    Do not confuse needs with ambitions.}
}









\subsection{Ramiel confronts \Dasteron}
There is an upstart, \hr{Dasteron}{\Dasteron}, who is the closest thing \Mystraacht{} has to a leader these days and thus counts as having usurped Ramiel's place. 

\target{Ramiel uses macho rhetoric against Dasteron}
Ramiel uses traditional \Mystraacht \hr{Mystraacht philosophy}{macho rhetoric}. 
But he is actually \hr{Ramiel is critical of Mystraacht ideology}{critical of the \Mystraacht ideology}. 

\Dasteron{} is taller than Ramiel, and very powerful. 
(You have to be powerful as fuck to aspire to lead \Mystraacht.)

He mocks Ramiel: 
\ta{%
  I believe you will find that things have changed in your absence, Lord Ramiel.}

Ramiel:
\ta{%
  You will find that things are about to change again following my return.}

\Dasteron{}: 
\ta{%
  You have been out of the game for thousands of years, Lord Ramiel.
  You drifted around down there playing \human, 
  while I was devouring souls and waxing strong. 
  I have even devoured \dragons.
  \Dragons, Lord Ramiel!
  Times have overtaken you.
  I am your superior now.}

Ramiel: 
\ta{%
  Then I challenge you.
  I invoke my right of Trial by Combat.}

Remember to compare Ramiel's and \ps{\Dasteron} \hr{Dasteron's appearance}{appearance}. 

The battle itself, despite what they try to show the spectators, is a grim, sad affair.
There is no hate or anger, so there is no enjoyment.
It is just a grim and gritty fight to the bitter end.





\subsubsection{Motivation}
Ramiel knows he has to prove his own worth by defeating \Dasteron. 
He must impress the \Mystraacht{} so they will accept him as their Overlord. 
He has to entertain in order to lay a foundation for his future reign. 

\citebandsong{Ihsahn:TheAdversary}{Ihsahn}{Panem et Circenses}{
  Awake, O' serpent of my heart. It is time.\\
  The sun stands high, and unfaithful crowds await thee.\\
  Redemption in their eyes and stone at hand.\\
  The arena hungers for your venom.\\
  Let the games begin.
  
  Bring in the lions. Bring in the beasts.\\
  It is time to confront the masses with their fears.\\
  A sober moment. A shred of truth.\\ 
  To gaze into an honest mirror. \\
  A disturbance of their sleep.
  
  Violent teeth and claws, untamed and fierce, \\
  reaches far and cut deep into the empty eye.\\
  It is time to let the bitter venom flow\\
  through this embodiment of emptiness.
  
  And the blood shall run free like words.\\
  And the bones shall form stairs to the future.
  
  Now, unfaithful spectator. Are you satisfied?\\
  Did you come close enough to feel the lion's breath?\\
  On day soon your shall be the sacrifice.\\
  A nameless grave of the past.
  
  Protagonist!\\
  Your time is now.
}





\subsubsection{Combat}
\target{Dasteron dies}
\target{Dasteron stronger than Ramiel}
\Dasteron{} is more skilled than Ramiel. 
\hr{Dasteron's skill}{Much more skilled}. 
He has faced down \satharioth{} before and knows how to turn an opponent's strength against him. 
Besides, \hr{Dasteron's smithing}{he is a great weaponsmith} and armed with the sword \hr{Scaleron}{\Scaleron} and many other powerful magical items. 

\ps{\Scaleron} design was inspired by \hr{Ascaril}{\Ascaril}, Ramiel's mother's sword. 
\Dasteron{} tells Ramiel this. 

Ramiel is taken unawares when \Dasteron{} displays just \emph{how} skilled he is at all three \hs{Paths}; not only \hr{Path of Light}{Light}, but \hr{Path of Ice}{Ice} and \hr{Path of Darkness}{Darkness} as well. 
This nearly overpowers Ramiel. 
The two combatants are sort of evenly matched at the Path of Light, but \Dasteron{} is far better than Ramiel at Darkness and Ice. 

One reason why \Dasteron{} is stronger than Ramiel is that he has \hr{Dasteron's upbringing}{lived his entire life as a \Mystraacht{} warrior}.  
Ramiel spent his formative years in the wussy and semi-pacifist and effeminate \Merkyrah{}, and then spend thousands of years \trope{WalkingTheEarth}{Walking the Earth} as a Scion.  

Ramiel is \hr{Ramiel is overpowered}{\uber, because he has eaten \Belzir}. 
But even so, he is hard pressed. 
He is not confident he can win. 
So he \emph{cheats}. 

The fight is to the death, but \emph{not} to destruction. 
But after Ramiel wins he breaks the rules and eats his fallen foe's soul, destroying him. 
Dark Eldar style. 

\ta{Stop him!} people around him cry. 
This was not part of the deal.
But no one is brave enough to challenge the \sathariah{} who has just splattered their leader and is now eating his soul and radiating fucking wicked-sick \sathariah{} power in all directions. 

Ramiel: 
\ta{%
  I am Overlord now. That was the prize. I can do whatever I want. That is my right. Will anyone deny me?}

Someone dares call him out on the dishonourable methods he used, both before, during and after the fight: 
\ta{You cheated!}

Ramiel:
\ta{Have you become like the \KiriathSepher{} in my absence?
  Have you become fops and cowards, slaves of etiquette?
  Ruthlessness has always been the \Mystraacht{} credo. 
  I saw my chances, so I took them. 
  Just as I saw my throne and now seize it.}





\subsubsection{Claims \Scaleron}
After slaying \Dasteron, Ramiel claims the sword \Scaleron. 
From that point he wears and uses the sword. 
It is his way of paying tribute to \Dasteron. 
He was, after all, a great and brave man and a worthy opponent. 
A true warrior of \Mystraacht. 

\Scaleron{} is a more than worthy replacement for Ramiel's old weapon, \Ascaril. 









\subsection{Ramiel claims the throne}
Ramiel lays claim to the throne of \Mystraacht. 
At this time he is not only the son of the first and greatest Overlord, but also the last surviving \sathariah{} in \Mystraacht{} (\Shiaraid{}, whom he killed, was the penultimate one). 
So, try as they might, no one can can really deny that he is the best candidate for the throne. 

\begin{prose}
Critic: 
\ta{What gives you the right to claim Overlordship?}

Ramiel: 
\ta{I am Ramiel.} 

(He could list off titles, lineage and accomplishments: 
One of the original discoverers of \Semiza, \sathariah, co-founder of \Mystraacht, sole heir of the last Overlord, and a mega-badass dude. 
But he needs say no more.) 

Critic: 
\ta{I believe you will find, Lord Ramiel, that things have changed in your absence.} 

Ramiel: 
\ta{And I believe you will find, [Critic's name], that things are about to change again.} 
\end{prose}

Perhaps he challenges his rivals by invoking his right of \hr{Mystraacht trial by combat}{trial by combat}, as is a \Mystraacht{} tradition.

\emph{\RamielsAwakeningBook} ends with Ramiel proclaiming his dominion and claiming his ancestral throne, cementing his position as the unchallenged Overlord of \Mystraacht. 

\lyricsbs{Emperor}{Moon Over Kara-Shehr}{
  Master! We ride with the storm \\
  in his name, the sire, wolves' king. \\
  Enter the power coursing \\
  through veins of the night.
  
  Power ripples through me. \\
  Armageddon's thunder will bring others to my side. \\
  The throne is mine. \\
  A blackened storm of evil. 
  
  Master! We ride with the storm to take \\
  revenge in the sky upon the one \\
  who cast thee from his side.
}

\lyricsbs{Bal-Sagoth}{
  Starfire Burning Upon the Ice-Veiled Throne of Ultima Thule
}{
  Such power! I am the Chosen. \\
  The secrets of the earth and the stars are unlocked before me.\\
  I am destined to reign forever\prikker \\
  to reign from the Ice-Veiled Throne of Ultima Thule!
}



After Ramiel has regained his full \malach powers and become Overlord, he lets himself hail as the saviour and champion of the \resphan race. 
He has mastered his \carcer and the dark powers of the \banes.
He is the ideal of all \resphain who crave power and greatness and perfection.

The new \Thanatzil, even.

\citebandsong{Nile:InTheirDarkenesShrines}{Nile}{
  Churning the Maelstrom
}{
  Hail To He Who Is In The Duat, Who Is Strong\\
  Even Before The Servants of Serpents\\
  He Gathers The Power From Every Pit of Torment\\
  From They Who Hath Burnt in Flames\\
  From Words of Power Uttered By the Darkness Itself

  Hail To He in The Pit, Who Is Strong\\
  Even Before the Terrors of The Abyss\\
  Who Gathers The Power from the Wailing And Lamentations\\
  Of The Shades Chained Therein\\
  From He Who Createth Gods \\
  From The Silence Alone
}





\subsubsection{Ramiel is sad}
After having slain \Dasteron and claimed the throne, Ramiel is very serious and determined and weighed down with responsibility.
Not at all the reckless adventurer he was in his youth.

Compare to William Adama from \cite{TV:BattlestarGalactica}.





\subsubsection{Contacted by \Daggerrain}
At the end (or perhaps at the beginning of the next book), Ramiel is contacted by \Daggerrain. 

\daggerrain{Ramiel. 
  Thou hast united \Mystraacht. 
  That is good. 
  But forget not whom thou servest.}















\section{Sentinel story thread}










\subsection{\Vizsherioch{} versus \Ishnaruchaefir}
\Vizsherioch{} is strong, powerful enough to challenge even \Ishnaruchaefir. Despite his young age. 





\subsubsection{\Vizsherioch{} becomes the Dagger}
\target{Vizsherioch becomes the Dagger}
\index{Dagger, the}%
\Vizsherioch{} finally becomes the \hs{Dagger} and takes his rightful place in the \matrix. 
His confrontation with \Ishnaruchaefir{} (and his \hs{Fulcrum}, \Rystessakhin) was important in this process. 











\subsection{\Secherdamon{} communicates with Tiamat}
\target{Rissit communicates with Tiamat}
\Secherdamon, seeking power and guidance in his war against the \banes, contacts the \firstgendragons. He feels the presence of \Tiamat. She appears as a great mass of swirling Chaos, almost too amorphous to be a person. She seems almost mindless. Compare her to Azathoth from H.P. Lovecraft's stories, such as \emph{The Dream-Quest of Unknown Kadath}.

Of course, \Tiamat{} is not mindless. But she has absorbed so much Chaos power and \xzaishannic{} essence that she has become more like \xzaishann{} than a \dragon. Her mind is so Chaotic as to be completely alien and unknowable, even to \Secherdamon, who is otherwise one of the wisest among the \dragons{} in the arts of Chaos, one of those who knows the most and has done the most research. (For instance, he\dash along with subordinates\dash created the Rissitic magic theory of the Three Worlds.)

\Secherdamon wants to bring the \xss back. 

\citeauthorbook[p.68]{VengerSatanis:CthulhuCult}{Venger Satanis}{Cthulhu Cult}{
  The Great Work has one end result: to bring the Old Ones back.
  Once They return to this reality, an apocalypse of bilious green fire will burn the foolish and the weak. 
  Blood spilt in ritual sacrifice shall consecrate the ground, opening the gateways.
  Temples built to \honour the Dark Gods shall alert the vigilant.
  Hideous visions shal refresh the dead imaginations of the faithful. 
  
  Some distant night shall see the unspeakable entities from Outside descend\prikker shall see Them break free, lowered into this world\prikker shall see Them destroy as they recreate their shuddersome paradise.
  Those among us who are strong, wise and diabolic shall enter the void and become like the Old Ones.
}





\subsubsection{\Secherdamon{} visits \Dathka}
\Secherdamon{} visits the fallen city of \hr{Dathka}{\Dathka}. 

\lyricsbs{Emperor}{Ye Entrancemperium}{
  Drawn towards these lands again.
  Seeking death and sacred soil.
  I ride the longing winds of my blackened soul,
  growing stronger once I enter my empire beyond.
  
  Emperium!
  Behold my coming.
  
  The fullmoon rise above me,
  enlightening my realm in a silvery glow.
  Yet the shadows crawl beneath my storming sky,
  guarding treasures from forbidden light.
}

He remembers \hr{Secherdamon's rise to power}{the time when he became a god}. 

\lyricsbs{Emperor}{Ye Entrancemperium}{
  I still remember,\\
  though ages ago it seems,\\
  the first time I entered the gates,\\
  the revelation of ritual death\\
  by which I became divine.\\
  Sacrifice of the life I had\\
  among the flesh of the light.
  
  And now I enter again.\\
  Even stronger, yet amazed by what I see.\\
  In ecstasy I mock the world.
  
  Suddenly I memorize,\\
  asking what I left behind.\\
  Nothing.
  
  Can I ever comprehend?\\
  Will my longing ever end?\\
  Never.
}





\subsubsection{\Secherdamon{} seeks out \KhothSell}
\Secherdamon{} also hopes to invoke \KhothSell{} and obtain her power of death to wield it against the \sephiroth{} and the \Morbus. 

\lyricsbs{Exmortem}{Death deceiver}{
  Ancient image of death.\\
  The deepest depts of horror.\\
  A diabolic master plan.\\
  Not to succumb to the damned.
  
  Mummified in slumber.\\
  Awaiting to appear.\\
  Death deceiver.\\
  Exmortem.
}







\subsection{\Ishnaruchaefir{} seeks power}
\target{Ishnaruchaefir seeks out cosmic gods}
\Ishnaruchaefir{} fears \ps{\Secherdamon} plotting. 
He plans to go up against his brother, but he is not powerful enough. 
He needs more knowledge and power. 

In order to build and release the \hs{Ark}, \Ishnaruchaefir requires the help of some \hr{Aloof Dragons}{aloof Elder \Dragons} that sleep dormant.

Maybe replace the gods in this scene with Elder \Dragons. 
Or maybe it is \Secherdamon who contacts them.

In the end, a few Elder \Dragons manage to awaken (now that the Shroud is broken) and come to \Ishnaruchaefir's aid.
Not many, though. 
Perhaps 4--6.
10 at the very most. 




\subsubsection{Navel gazing}
He gazes into himself, and into his glaive, \Triestessakhin. 

\lyricsdimmuborgir{The Insight and the Catharsis}{
  Oh, dreadful angel of mine.\\
  Enrich me with the vastness of your being.
  
  Rigid father, teach me to comprehend.\\
  I'll commit myself to understand.\\
  To be the faithful and the instrument,\\
  so that the ones with blindfold can see what I have seen.
}





\subsubsection{\KhothSell}
He seeks out \KhothSell, his \quo{mother}. 
She tells him stuff. 

\lyricsdimmuborgir{The Insight and the Catharsis}{
  What more do you need of proof?\\
  Human hands conforming clooven hooves.\\
  For I know the secrets and lies behind all truths.\\
  Knowlege is power and the power is mine.\\
  It's all mine.
}





\subsubsection{Cosmic gods}
He seeks out some \hr{Cosmic gods}{cosmic gods}, hoping to gain the power and knowledge he needs to thwart his brother. 

Their dark halls are splendid and overwhelming, but also dangerous, filled with mighty guardians that strike even the legendary \Ishnaruchaefir{} with terror. 

\lyricsbalsagoth{Invocations Beyond the Outer-World Night}{
  These stygian pitch-black vaults are filled with batrachian devils,\\
  Dire crystalline watch-dogs of the chasmed deeps,\\
  (For the gleaming jewels of truth are not without their protection\prikker)\\
  Vril-gorged adamantine fiends of the threshold,\\
  Spawn of the ersatz interior sun.
}

The cosmic gods turn him away. 

\lyricsbalsagoth{
  The Hound of Chaos Transcends the Nebulous Palisades of Z'xulth
}{
  [THEY-WHO-LURK-AND-BREED-IN-LIMBO:]\\
  Silence, godling! \\
  The fact that you stand before us is testimony to your mettle, 
  but your ambitions shall yet outreach your abilities. 
  
  We have long watched thee and your struggles with amusement, Zurra, 
  Sire of Angsaar. But this audience is now at an end. The path you 
  must traverse lies before you, and you must follow it to its 
  ultimate destination.
}

But \Ishnaruchaefir{} is sneaky, and he manages to surprise and out\manoeuvre them, if only for a moment. He successfully gleans enough of their secret knowledge to achieve what he needs. And so, \Ishnaruchaefir{} confronts some of the mightiest creatures in the cosmos, entities who could crush him underfoot with no effort at all, and he \emph{still} comes out looking like a total stud. \trope{Badass}{Badass}.  







\subsection{\Ishnaruchaefir{} laughs}
At some point, \Ishnaruchaefir{} experiences something funny. 
He laughs with sincere mirth for several seconds. 
\Criseis{} (who is with him) laughs along. 
Seeing him enjoying himself is intoxicating. 
She becomes ecstatic and blissfully happy. 
Her entire world lights up, and everything seems good and bright. 

While it lasts. 

Then he gets over it, and his usual brooding, grim, sardonic state-of-mind takes over. 
Suddenly \ps{\Criseis} world again turns dark and brutal and full of war and pain.
She understands that his mirth made him forget his gruesome burden and all his traumata and hate against the world for a brief moment. 
And his feelings rub off on her, who is closer to him than any other. 
And she is more impressionable and vulnerable than he, so it rocks her world more strongly. 





\subsection{Sentinels seeks out \voyagers}
Some Sentinels, perhaps \Ishnaruchaefir, seek out the \hr{Voyagers today}{surviving \voyagers} to gain knowledge useful against the \banes. 








\subsection{\Ishnaruchaefir{} begins to realize what he must do}
\target{Ishnaruchaefir begins to realize what he must do}
\Ishnaruchaefir{} slowly begins to realize \hr{Ishnaruchaefir realizes what he must do}{what he must do} to overcome his own mental blocks and reach the Gnosis he so desperately needs. 

\citebandsong{Ihsahn:TheAdversary}{Ihsahn}{The Pain Is Still Mine}{
  A distant cry arose\\
  from the fathomless well that is my soul.\\
  I can not hear the words,\\
  so I throw my heart in like a coin\\
  and wish that it would sink forever.
}

He realizes that \hr{Mirage Asylum symbolism}{he is hiding}. 

\citebandsong{Ihsahn:TheAdversary}{Ihsahn}{The Pain Is Still Mine}{
  A purpose, a sacrifice,\\
  or merely temptation?\\
  Is my solitude anything but a perversion\\
  of my vanity?
}

He defends/rationalizes his actions to himself. 

\citebandsong{Ihsahn:TheAdversary}{Ihsahn}{The Pain Is Still Mine}{
  I never cared for this weak inclination,\\
  this paranoid tendency to flock.\\
  And in between all the noise, all the guilt,\\
  a silence would carry my spirit away\\
  from diminishing obsessions.\\
  Away from fools and poisonous flies.\\
}

But deep down he knows his rationalizations do not cut it. 

\citebandsong{Ihsahn:TheAdversary}{Ihsahn}{The Pain Is Still Mine}{
  The birth of a dreamer.
}









\subsection{\Ishnaruchaefir{} takes a crazy risk too much}
\QuessanthIshnaruchaefir{} is known to be reckless and take crazy risks. 
In this book, he takes one crazy risk too many. 
It goes wrong. 
He fucks up and accidentally leaves a back door open so the \resphain{} can sneak into his Mirage Asylum. 







\subsection{Mirage Asylum destroyed}
\target{Mirage Asylum destroyed}
At some point \ps{\Ishnaruchaefir} \hs{Mirage Asylum} is breached, invaded by \resphain{} and \banes, overrun and destroyed.
 
It had been a living thing, but now it died. 
It bled and felt pain as it died. 
It was horrible for \Criseis to feel it die.
It was her home. 

\Ishnaruchaefir{} now has no safe hiding place. 
He cannot just skulk and let things happen. 
He is forced to take a more active role in the Feud. 
And he does. 
Oh, boy, does he. 
With a vengeance. 
The \resphain{} soon regret provoking him. 

Maybe this is part of \ps{\Azraid} \trope{XanatosGambit}{Xanatos Gambit}. 

The Asylum was protected by the Shroud. 
It was only because the Shroud was \hs{unravelling} that it was possible for the \resphain{} to breach it. 

Make sure to have plenty of story leading up to this, making it plausible that the Asylum could be breached now but not just at any point in the last seven thousand years. 

This is an omen of things to come. 
Soon, the whole world will lose its Mirage Asylum, the Shroud itself. 

Compare to where Chia loses her Black Dragon Valley in \authorbook{Stephen Marley}{Shadow Sisters}. 









\subsection{\Ishnaruchaefir resurrects and learns about \iquin}
A very heroic or very innocent mortal is tragically killed. 
\Criseis, who had formed ties to this person, is devastated. 
As an immortal, she has long experience telling \quo{good} deaths from \quo{bad} deaths, and this one is a bad one. 
She rages at the injustice of it.

Someone:
\ta{It is no use. We cannot rule over life and death.}

\Criseis:
\ta{You are right.
  You and I cannot rule over life and death.
  But I know who can.}

Have some references to \Sethicus's lore and his great mastery of the forces of the world, and to the sorcery and pacts of \KhothSell. 

\Criseis prays. 
\ta{Master \Quessanth.
  \Quessanth \Melechet \Nierzshah \Tzeorossh \Ishnaruchaefir.
  Rarely have I asked you for anything. 
  I beg you now, help me.}

\Ishnaruchaefir:
\ta{I cannot. His/her soul has already passed into the afterlife.
  I cannot reach inside \iquin.}

\Criseis (desperate, tearful):
\ta{You lie! I know you can. I have seen you defy forces as great as \iquin before.}
She yells and curses at her master and all but orders him to do this for her. 

\Ishnaruchaefir is impressed by \Criseis's passionate outburst.
\Dragons respect strength and aggression more than humility. 
\ta{Very well, \Criseis. 
  You speak the truth.
  I do owe you this boon.
  So be it.
  For you I will defy the \sephiroth.}

\Ishnaruchaefir reaches into \iquin.
It is hard work.
Very hard. 
He struggles. 
He breaks through layer upon layer of \iquin. 
Finally he finds the soul he is looking for and tears it free. 
In the process he learns much about \iquin. 

\Ishnaruchaefir is horrified.
He had known \iquin was a \dweomer powered by souls of the dead, but it is a bigger, more cosmic and more ambitious project than he had dreamed. 
He had thought \iquin existed primarily as a power source for the Vaimons. 
He realizes it has bigger potential than that.
It is much more integrated into the world and has roots that go deeper. 
It is \hr{Iquin and Noggyaleth}{connected with the burrowing of the \noggyaleth}.
He begins to guess the \Lithrim plan.









\subsection{\Ishnaruchaefir{} confronts \Secherdamon}
\Ishnaruchaefir{} confronts \Secherdamon.

Perhaps, in the distant past, \Ishnaruchaefir{} warred against \Nexagglachel{} in an attempt to bring the \dragons{} together. 
They discuss this.

\begin{prose}
  \Ishnaruchaefir: 
  \ta{That is something you never grasped.}
  
  \Secherdamon: 
  \ta{You think you know so much about me. 
      Yes, I know and understand perfectly, as I always have. 
      But your mission is misaimed (forfejlet?). 
  
      You are so arrogant. 
      You do not understand what I have become; indeed, you never understood what I was. 
      You have power, yes, and you may best me in combat, but in the world I wield power beyond your wildest dreams.%
  }
  
  \Ishnaruchaefir: 
  \ta{Unlike you, I have no wildest dreams\prikker}
  
  \Secherdamon: 
  \ta{\prikker and that, brother, is your failing. 
      It is why you have become a relic of the past. 
      The future is mine! 
      I am the future!
  
      Heh. 
      Perhaps I ought to thank you and \Nexagglachel. 
      If not for his death, I might still live in his shadow. 
      The events of those days are what have me the push to rise above what I was. 
      And today I am by far the greatest of us three.%
  }
  
  \prikker
  
  \Secherdamon: 
  \ta{%
    The Gnosis which \hr{Ishnaruchaefir refuses to tell Secherdamon Gnosis}{thou stole from me}\prikker
    I have \hr{Secherdamon gains Gnosis}{reclaimed it} by mine own craft.
    I am ready.
    My plan will come to fruition, and thy hideous work will be undone!}
\end{prose}

They fight a great duel. 
It is savage, murderous business, but cathartic. 
They slash and rip one another apart and let loose all the aggression and hate that they have pent up for one another and left to brood and fester for thousands of years. 

For once, \Ishnaruchaefir{} talks about feelings and motives. 
And he defends his actions, something he has always refused to do. 
He shouts in anger, which helps, because it is what \dragons{} do when they are being candid. 
This open anger is cathartic and brings them closer to each other (even though, between \resphain{} or \humans{} or \scathae, such anger could easily work the other way and drive them further apart). 

They learn to understand each other better, through the words and curses they speak, and through the deeper truths their bodies tell, with claws, weapons and magic. 
They understand each other's Aenigmata. 

They come a step closer to resolving the endless feud between the two brothers. 
They learn to respect each other. 

At the end, \Secherdamon{} admits that he is closer to forgiving \ps{\Ishnaruchaefir} crimes against them all. 
This is hard to say. 
Equally hard, \Ishnaruchaefir{} \emph{accepts} this step towards reconciliation. 
Previously he has responded to the idea of forgiveness with nothing but scorn. 
But this time he accepts and respects it (but wordlessly, not explicitly). 

This makes \Secherdamon{} more ready for his final mission. 
\hr{Secherdamon's sacrifice}{His sacrifice}. 

% They fight. 
% Ultimately, \Ishnaruchaefir{} slays \Secherdamon.









\subsection{\Secherdamon{} dies}
\target{Secherdamon dies}
\Secherdamon{} has two possible deaths: 





\subsubsection{Version 1: \Vizsherioch{} kills him}
Somehow, \Secherdamon{} get into combat and is mortally wounded. \Vizsherioch{} comes to him. 
\Secherdamon{} asks his son to save him, but \Vizsherioch{} instead kills him. 

First, \Vizsherioch{} holds a monologue: 
\ta{Yes, father. 
  You created me to be loyal to you. But more importantly, you created me to be loyal to our quest, our legacy.

  You created me from your own essence. I am part of you, and you are part of me. In the long run, this is a weakness. You must understand this. You made me well, but you have held too much back.
  
  You are fallen. To save you would be\prikker inefficient.
  
  I will strike you down, father, and subsume your essence into me. I will absorb your \xzaishannic{} blood, and in doing so I will become greater than any of us.
  
  I will become the ultimate \draecchonosh. I will become the ultimate heir to the \firstgendragons\prikker and to the \xzaishanns. I will become\prikker \emph{perfect}. And I swear to you this, father: I will filfull our destiny. I will scour \Miith{} of the \Erebean{} infestation and restore the glory of the \draconian{} race.}

\Secherdamon{} understands his son's vision, so he is happy when he dies. 

After this, \Vizsherioch{} becomes the ultimate god. 

He remembers the \quo{his} past as a \xzaishann{}.

\lyricsbalsagoth{Return to the Praesidium of Ys}{
  I was spawned deep beneath the Pre-Cambrian sea, \\
  the scion of a far distant sun\prikker\\
  I have traversed the endless stars, \\
  and journeyed to a myriad galaxies\prikker\\
  The dimensional gates of the multiverse are mine to voyage effortlessly beyond,\\
  cosmic infinity is naught to one such as I\prikker \\
  I am as one with celestial eternity\prikker\\
  Clad in gleaming pentlandite \armour, \\
  on a whim I may reshape entire worlds,\\
  or extinguish the blazing light of a sun\prikker \\
  and I remain forever enchanted by sylphs\prikker
}

He has memories of the \xzaishanns{} beyond anything \Secherdamon{} and even the \firstgendragons{} had achieved, and he also carries within him the stolen essence of the \banelords. 

\lyricsbalsagoth{Return to the Praesidium of Ys}{
  Wielding this Power Cosmic,\\
  the Omniverse is mine to conquer.\\
  I am a god.
  
  Arcane power lances forth from my fingertips,\\
  life withers before my baleful gaze.
}

He remembers the whole civilizations that the \xzaishanns{} once destroyed.

Have some evil flashbacks with the \xzaishanns{} destroying civilizations. 

\lyricsbalsagoth{Return to the Praesidium of Ys}{
  The proud citadels of great antediluvian empires\\
  have been razed to the ground by my zircon blade.
  
  Riding the screaming crest of fettered ions,\\
  I shall bring my crystalline Chaos where order reigns.
}

He looks down at his father's corpse. 
\ta{I'm sorry, father, but I was not telling the whole truth. 
  I am not driven by our legacy or quest, but by the Will to Power.} 
Compare with Friedrich Nietzsche's philosophy.





\subsubsection{Version 2: Heroic sacrifice}
\target{Secherdamon's sacrifice}
\ps{\Secherdamon} death is a heroic sacrifice for his people. 
It has cosmos-spanning metaphysical ramifications and ripples. 

Compare to Anomander Rake's sacrifice in \cite{StevenErikson:TolltheHounds}. 















\section{Random scenes}









\subsection{The power of love}
A \human{} talks to a \resphan{} about how love is the greatest, most important thing in the world. 

\Resphan: \ta{Ah, yes. \quo{Love}. Insertion of dick into cunt, a splash of liquid, followed by cuddling. Believe me, I know what love is. I and my fellows were the ones that created it. We designed your people, programmed your obsession with sex and sexual bonding. Because that it what we need you for. Your \human{} \quo{love} is a pawn on our game, and your idolization of it is a keystone to our \hr{Daggerrain's master plan}{Master Plan}.}









\subsection{The master races did not infiltrate}
A Cabalist or Sentinel talks to a mortal about how his people supposedly infiltrated the kingdoms of mortals.

Big guy: 
\ta{%
  Ha! We never \quo{infiltrated} your world. We \emph{created} it! Your civilization is and was ever our tool. We never conquered you or couped you out. You were created, born and bred as our slaves.}









\subsection{\Criseis{} apologizes for \Ishnaruchaefir}
\Ishnaruchaefir{} has hurt some people. 
\Criseis{} comes in secret and of her own initiative to apologize and ask forgiveness on her master's behalf. 
\hr{Ishnaruchaefir never apologizes}{\Ishnaruchaefir{} never apologizes}. 















