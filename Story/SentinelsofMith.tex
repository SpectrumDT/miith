
\part{Sentinels of \Miith}























\chapter{The Amorphous God}
\target{Mephilex' initiation story}
\index{Innermost \Arcana, The}
This short story tells the back-story of \hs{Suthis Mephilex}. 















\section{Summary}









\subsection{Mephilex must learn \arcana}
At the beginning of this story (or another story), Mephilex was young and ignorant.
She was 26 years old. 
As per the traditions of the \Yormissian \rethyaxes she had spent her first thirteen years learning basic skills such as literacy, mathematics and meditation, and her next thirteen years mastering the first six \hr{Arcana}{\arcana} and their spells. 
Now she was ready to learn the True \Arcana that lay beyond the basic \arcana. 
She was ready to descend further through the \hs{Dreaded Gateways}. 
She must also learn the secrets of her family and what went on in the hidden lodge halls in cellars beneath \Yormis. 

She must learn the three \hr{Suthis Innermost Arcana}{Innermost \Arcana of the Suthis clan}.

The story should start like \cite{RobertBloch:TheSecretintheTomb}, with the reader learning that Mephilex was a descendant of a line of sorcerers.






\subsubsection{Mephilex remembers Dristan}
\target{Suthis Dristan dies}
Mephilex had an older cousin, \hs{Suthis Dristan}. 
He was also a \rethyax and had had to learn these same revelations that Mephilex now faced. 

Dristan proved not strong enough.
The revelations drove him insane, so insane that he died of fright.
Mephilex remembered well the time of her cousin's death.
She had been close to him, and the loss of him struck her hard.
Especially because she knew she must one day go through the same trials.
It made her sick to think about it.
She remembered how all food and drink seemed to taste sickly for weeks after Dristan's death. 
That was how bad she felt.

Shortly after Dristan's death Mephilex had been given a special dish, allegedly made from the flesh of a mosca. 
She later remembered that every time she had had mosca, at night she would have strange nightmares. 
She would see \Ubloth.
There would be horrid slime everywhere. 
Sometimes she would be immersed in the slime.
Sometimes it was as if the slime was inside her, pumping through her veins.
It was as if she drowned in the slime, was absorbed by it, became one with the slime. 
It was in her and she was in it. 

\citeauthorbook[p.45]{HenryKuttner:TheBlackKiss}{%
  Henry Kuttner%
}{%
  The Black Kiss%
}{%
  And in that kiss strangeness flooded him.
  He felt a shock and a tingling go through him, and then a sthrill of sudden ecstasy, and swift on its heels came horror.
  Black loathsome foulnees seemed to wash his brain, indescribable but fearfully real, making him shudder with nausea.
  It was as though unutterable evil were pouring into his body, his mind, his very soul, through the blasphemous kiss on his lips.
  He felt loathsome, contaminated.
  He fell back.
  He sprang to his feet. 
}





\subsubsection{Mephilex meditates}
The evening before her initiation, Mephilex sat and meditated. 
She felt a colossal formless power stirring, like a humongous living thing far down in the deep. 
She woke, and it was as if she could physically feel the stone tower quake underneath her, as if shaken by some thing moving far beneath.
She imagined she could hear a spectral groaning from underground.
But she convinced herself that she was just imagining it. 

Later in the story she would feel these reverberations again. 
But every time, when she looked around, it was as if only she could feel it. 

The reason was that her family had been feeding her the slime of \Ubloth for weeks, and they had been feeding her blood to \Ubloth, so the two were already forming a metaphysical bond. 









\subsection{The secret of the \ophidians}
Mephilex's father, \hs{Suthis Cruan}, was her mentor and guide.
He, together with other \rethyaxes, took her down into the deep. 
She saw the marvellous and sinister underground city. 

They took no guards with them, but as they descended further down Mephilex could see furtive shapes creeping in the darkness around them.
They were \hr{Suthis monster slaves}{the clan's monstrous half-\scathaese slaves}. 
She was afraid, but Cruan told her not to worry.
They were not enemies.
The creatures were here to protect and escort them.
She would soon see them up close.
Mephilex did not look forward to that.
The creatures were awful.
Hideously \scatha-like, and yet alien.
Monstrous.
Misshapen. 

It was very dark down here. 









\subsection{The Master}
\target{Mephilex is supposed to meet the Master}
Her father told her that she was to meet the \quo{Master}, the mysterious non-\scathaese lord that ruled the underworld of \Yormis.
This Master was really \hr{Ishtacca}{\Ishtacca}, the eldest \ophidian lord. 

Cruan (or another wise \scatha) told her that the Master and his people had dwelt in darkness for a million years and learned to live without light. 
That was why the city was so dark.

Make it clear that the Master was \emph{not} \Ubloth. 

Mephilex was shocked and repulsed by the sight of the half-\scathaese guard-slaves.
She was also scared by the \ophidian sarcophagi.

The Master was just about to possess a slave body (\hr{Ophidian possession}{as \ophidians did}) so he could talk to her.
Then the enemies attacked. 
They killed the slave body, dispelling the Master. 










\subsection{The secret of cannibalism}
Mephilex learned another horrifying secret: 
Her cousin \hr{Suthis Dristan}{Dristan} had not \hr{Suthis Dristan dies}{died on his own}, but had been killed by his family. 
Worse, they cut up Dristan's body, cooked it and fed it to Mephilex. 
Over the course of several weeks they made her eat all his flesh and drink his blood and fluids. 
That way his body did not go to waste. 
He was full of invaluable genetic material which must be preserved for the good of the bloodline.
He had been their prime scion, but now he had failed, and now Mephilex was their greatest hope.
They prepared his body with powerful spells and made her eat him so that she would inherit the power and purity in his veins. 
They even bound his very soul in his flesh and made her ingest it. 
That way he would live on through her. 

Mephilex was horrified to learn this. 
But later she accepted it.
She began to feel the closeness of Dristan's soul.
She could feel him inside her head, whispering to her.
At first this horrified her, but she learned to accept it.
She realized that she had not lost her beloved cousin.
He was still with her.
It was up to her to triumph where he had failed.
Her family expected it of her.
Dristan expected it of her.
She came to look to Dristan's ghost as a source of strength. 
She took comfort in it, and in the dark marvels of her clan. 

Compare to the main character in \cite{HPLovecraft:TheShadowOverInnsmouth}, who in the end learns to love and embrace his Deep One legacy.









\subsection{The secret of \Ubloth}
Mephilex was now ready for the third \arcanum: 
That \hr{Suthis and Ubloth}{the Suthises worshipped \Ubloth}, and that \hr{Suthis souls and Ubloth}{their souls were bound to \Ubloth}. 

Mephilex had not yet made a pact with any god. 
Her family forbade it. 
So her magic was still weak and immature.
She would soon learn why:
She was to enter into a pact with \Ubloth, her ancestral patron god. 









\subsection{Enemies attack}
But the Suthises had enemies. 
Mephilex herself had a mortal enemy in a rival clan. 
These enemies knew about the subterranean city, for there were other clans than Suthis that served the \ophidian \liches.  
Now, while Mephilex was being guided through her \arcana, the enemies attacked.
They wanted to assassinate Cruan and Mephilex while they were preoccupied and off their guard. 
And they wanted to do it down here in the deep where no one would ever know (for the existence of the underworld city was a secret from the public). 

The enemies attacked.
They killed several.









\subsection{Mephilex flees into the tunnels}
Mephilex was separated from her father and their guards and forced to flee for her life into the dark tunnels below. 
She had only heard half of the story and knew only bits and pieces about \Ubloth, not nearly enough to piece together the awful truth. 
But now, running blind through the black corridors, she stumbled on hidden altars, statues, inscriptions, laboratories and notebooks.
And on one occasion she lay hidden in a corner and listened to her enemies as they walked by her.
And she found evidence that her family sacrificed flesh and souls to \Ubloth. 

She also began to hear the voice of Dristan in her head. 
At first she fled from him and tried to shut him out.
But she realized that he was helping her, and she slowly accepted him as an ally. 
He helped her realize the truth. 

She had a scroll containing the spell she was to use to make the pact with \Ubloth.
Her father had given it to her along with strict orders not to open it until the right time came. 

From all these clues she slowly pieced together the truth.
She realized that her family served some hidden god named \Ubloth that had its shrine in the deeps. 
She realized that she had to make a pact with this god in order to gain her full power. 





\subsubsection{Mephilex encounters horrors}
Mephilex encountered horrors. 
She met awful undead slaves that went about their silent business.
They did not harm her and indeed completely ignored her, but she was still deathly afraid of them. 
She also saw living buildings, and she saw undead monsters and undead buildings that used to be alive. 









\subsection{Into \Ubloth's chamber}
Then she was chased into the deepest central chamber (or cavern).
Here she laid eyes upon the seething, bubbling pit of pure vileness and filth that was \Ubloth. 
She remembered Dristan's cries when he went mad.
He raved about \quo{the amorphous god}. 
She realized that this abomination was \Ubloth. 
This was the god that she was destined to serve. 
A hideous thing of slime that was fed with the flesh and souls of mortals. 
She screamed and cried. 

Her enemies came after her.
Self-preservation took over, and she fought to defend herself.
She managed to hide in a place where her enemies could not easily come at her.
This gained her a respite.
But she realized that she had to act fast. 
She could not fight her enemies.
She was too weak.
She was not yet a full \rethyax.
She realized what she must do: 
She must make her pact with \Ubloth, here and now.

She realized that she had been drinking the effluvium of \Ubloth for weeks, and \Ubloth had been drinking her blood. 
She was already inexorably bound to the loathsome god, whether she liked it or not.
She felt that she had no choice. 
She must bond with \Ubloth.

Fortunately she had the scroll.
She opened it and began the spell. 
She finished the spell and felt \Ubloth's filth seep into her mind. 
And she felt all the souls contained within \Ubloth; those who had given themselves to it, and those whom it had consumed. 

Mephilex had realized the nature of her clan's \hr{Suthis and Ubloth}{horrid symbiosis with \Ubloth}.
She knew she had been eating humanoid flesh and imbibing their souls, and that \Ubloth had been fed her blood, and that she had then been fed \Ubloth's effluvium. 
Finally Mephilex gave herself up to \Ubloth. 
She stepped or leapt into the pool.
She surrendered herself and let \Ubloth flow around her and into her and feed.
In return \Ubloth filled her with sorcerous power. 

Now filled with the power of \Ubloth, Mephilex immediately used it to cast a spell.
She drew energy directly from \Ubloth's body and blasted the minds of her assailants.
She then watched in horror as \Ubloth itself rose up to snatch and engulf her foes and drag them down into its slimy maw to devour their bodies and souls.
She shivered to think that it was she who brought this ghastly fate upon them.
And she shivered to think of the slavering inhuman evil of this loathsome shapeless thing which she is now forever pledged to serve, in this life and the next. 









\subsection{Aftermath}
Afterwards, Mephilex thought back to what happened. 
We hear that she later met the \quo{\hr{Mephilex is supposed to meet the Master}{Master}} and learned to love and worship him. 
Where \Ubloth was abhorrent, the Master was beautiful and splendid and terrible. 
He was not harmed when his host body was slain, for the Master was immortal.
He was the mightiest being in \Yormis save \Ubloth alone, for he is the messenger of \Ubloth and the eternal overlord of \Yormis. 
Praise the Master's name!















\section{Changes}
\begin{changes}
    
  \begin{comment}
  \paragraph{Mikkel Kjær Jensen}
  \end{comment}
  \changesitem{Mikkel Kjær Jensen}
    1. Det er mig ikke helt klart hvordan personerne i din historie ser i
    den underjordiske by. Ja, der bliver nævnt fakler nu og da, men de
    syntes at være der allerede da personerne ankommer og i det første
    tilfælde bliver det ikke nævnt at de bliver tændt, og i den andet
    tilfælde kan det ikke være sket.

    Taler vi her om de klassiske evigt-brændende Hollywood Torches (som
    har brændt siden nogen var der sidste), eller er der tale om andet?

    Hovedpersonen kan i hvert fald ikke se i mørke, ellers giver følgende
    sætning ikke mening:
    "There were few torches, so soon I was crawling on my knees in
    pitch darkness through an inch-thick layer of goo."

    2. Nær slutningen af "The Natural Order" har du et "Blam!". Jeg syntes
    at dette er en øjenbæ, som sagtens kunne fjernes, så "The sound of a
    gunshot ripped through the air." står alene. Jeg syntes det ville give
    et bedre overblik.

    3. “Yuck! This tunnel is full of goo, and it stinks!”
    Denne sætning, i starten af sektion 6, er fuld af Narm. I forhold til
    den sindstilstand som hovedpersonen har, og som bliver vidregivet til
    os. Hvis du ikke kan/vil fjerne sætningen, så overvej i det mindste at
    fjerne den sidste del "...and it stinks", det vil gøre det meget
    bedre, og mindre ufrivilligt komisk.

    Ellers syntes jeg den er fin, godt gået.
    
  \begin{comment}
  \paragraph{Worldbuilding}
  \end{comment}
  \changesitem{Worldbuilding}
    Mephilex should tell us what spells she is studying. 
    In her nightmare she passes some well-known milestones (see the section about \hr{Rethyax meditation}{\rethyax meditation}). 
    Then she is lost. 
    \Ubloth pulls her off her path and into its own clutches. 
    Later, when Mephilex throws herself into \Ubloth's embrace, she also passes through well-known stages of supernatural consciousness (albeit now on a more intuitive level, less analytical and controlled). 
  
    Mephilex mentions that the foundations were built by Suthis Ondra, her ancestor.
    (Add Suthis Ondra to the glossary.)
  
    Mention that the subterranean city is as old as Su-Gelba and Ibthek.
    Perhaps even so old that it was built while mythical \Aamon still stood (where now only the forbidding Tower of \Aamon stands).
    
    The underground temple where Cruan tells Mephilex about the \Arcanum of the Flesh is called the Shrine of the \hr{Uzul-Kaya}{\UzulKaya}.
    
    Make the reference to the war with the \banes more obscure, more vague.
    Just a hint. 
    
    Shorten the trip through the dead city. 
    
    When Mephilex prays to \Nasshikerr, she should not be speaking an easily understandable prayer.
    She recites the \hs{Lesser Shkormi Formula} in the name of the goddess \Nasshikerr who slithers unseen in the shadows.
    
    In the laboratory Mephilex finds a magical symbol that reminds her of the enigmatic and powerful Symbol of \Aamon.
    Some of the spells she sees are written in the language of \hs{Kush}.
    Mephilex knows no Kush, but she recognizes the curved sinuous script. 
    They are very different from the sharp angular \Draconic runes.
\end{changes}
























\chapter{The Master}
\target{Moro and Sperra story}
\target{Mephilex rises to power}
\index{Amorphous God, The}
The novel or novella tells the background story of \hr{Moro}{\MoroCobrel}.

\MoroCobrel and \TulionSperra were young mages in \hr{Yormis}{\Yormis}. 
They were lesbian lovers, with \Sperra as the dominant of the two. 















\section{Intrigue in \Yormis}
During Moro's time in \Yormis there was a big plot brewing.
Some factions fought against one another using ancient magic.
The Sentinels knew nothing of this whole thing and did not learn about it until it was over.
Sentinel resources were spread thin at this time, and they could not keep an eye on everyone.

A rebel faction, which included many \Ubloth cultists, had discovered the \hr{Mummies in Yormis}{crypt full of mummies} and wanted to resurrect them and thus gain an army of undead warriors and conquer the world. 

Another group, which included \hs{Suthis Mephilex}, wanted to stop them. 

\hr{Uldraan Kerross}{\UldraanKerross} was the most politically powerful mage in \Yormis at the time. 
He was horribly evil (read about him). 









\subsection{The Master}
The true ruler of \Yormis was the \ophidian lord \hr{Ishtacca}{\Ishtacca}, called the \quo{Master}. 
The story revolves around him.
Moro and Sperra heard tales of this mysterious and menacing \quo{Master}. 
They gradually discovered who the Master and his people were, and what they were doing in the dungeons below \Yormis.
They learned about the \ophidian quest for immortality.









\subsection{Suthis Mephilex}
A younger, up-and-coming mage of fine breeding and great talent was \hr{Suthis Mephilex}{Suthis Mephilex}. 




\subsubsection{Rise to power}
She wanted to oust \Uldraan and seize power for herself. 
But \Uldraan was still very powerful.
So Mephilex deftly manipulated various people into helping her defeat \Uldraan. 















\section{Sperra and Moro investigate}
\target{Moro and Sperra investigate Mephilex}
\Sperra and Moro suspected that Mephilex was up to something sinister.
They investigated. 










\subsection{Fighting or helping Mephilex}
They wanted to stop Mephilex's evil, but in reality they were being manipulated by Mephilex into helping her defeat her rival, \UldraanKerross. 

Moro started out inadvertently helping the rebels without knowing what they were up to.
Soon she switched sides. 
She helped Mephilex stop the rebels. 
Then Mephilex gained a lot of evil power and became one of the leaders of \Yormis. 

Moro was not quite happy about this. 
She did not trust Mephilex.
Moro feared Mephilex was evil.
She did not want her in power.
But she inadvertently helped Mephilex gain power.
Moro felt a bit guilty about this. 
She also feared she had done more harm than good in her attempts to save \Yormis.

Mephilex now planned to do the same thing as the rebel faction had planned to do.
Only somewhat different. 
She hoped to extract power from the mummies. 









\subsection{In the underground}
Sperra and Moro explored the dark caves underneath Mount \hr{Shrun}{\Shrun} near \Yormis.
They found the \hr{City beneath Yormis}{ruined underground \ophidian city}. 

Here they faced great danger and uncovered untold horrors. 
They saw glimpses of \hr{Degenerate Ophidians}{creeping degenerate \ophidians} and their servitors and monsters, but they were never sure exactly what they saw. 
There were also degenerate \scathae and \humans down there, \hr{Suthis cannibalism}{which were used a food} by the Suthis clan.  

They encountered undead in the underworld: 
\hr{Undead Ophidians}{\Ophidian mummies} and their zombie-like servitors.

They experienced ancestral dread.
Read about \hs{ancestral memory}.





\subsubsection{\Ubloth}
They found the great cavern-temple where lay the loathsome amorphous god \Ubloth. 
They saw the monstrous freaks that had been mutated by \hr{Effluvium of Ubloth}{\Ubloth's corrupting effluvium}.
In a climax of crowning horror they stumbled into the great hall and lay eyes upon \Ubloth itself.









\section{Sperra dies}
\target{Tulion Sperra dies}
\TulionSperra should be the main character and hero of this story.
More than once she risked her life and sanity to save Moro. 
In the end, \Sperra was killed and devoured by the god \Ubloth.
(This should come as a surprise for the reader.)

Moro many many mistakes and was clearly to blame for \Sperra's death.
Moro never forgave herself for this, and remained unhappy and guilt-wracked for the rest of her life. 
The only thing Moro could console herself with was that she had helped remove a gruesome tyrant and replace him with a somewhat milder tyrant. 
Moro only hoped that Suthis Mephilex's long-term evil plans would not prove her an even greater threat than \Uldraan was. 

After this incident, Moro left \Yormis. 
She had seen too many horrible things in the dungeons beneath. 
She wanted to get away from it all.
The whole city reminded her of horror and evil. 
She could not bear it anymore. 

Compare to the entombed saurians in \cite{Nile:InTheirDarkenesShrines}. 



























\chapter{\CarzainPrequelBook}
This book tells Carzain's back story, before he joined the Runger War. 
It details Carzain's travels, Vizicar's awakening and their joint path to \kenosis. 















\section{Overview}















\section{The Mutiny}
\target{Mutiny}
\target{Mutiny chapter}
Carzain and Nishain were betrayed by mercenaries in Heropond Forest. 
Vizicar awakened (for the first time in Carzain's life) and fought, killing the mutineers. 









\subsection{Old description}
Carzain goes to Martinum with his father. Then back.

It is only autumn as they travel through Heropond, but in the \Wylde it feels almost like winter: Hoarfrost lies on the ground, and cold mist fills the air.

The younger trees hate them and want the intruders out of their forest. Carzain finds an older tree that is indifferent, cold, uncaring\dash it knows that the humanoids will die soon enough anyway. 

(To what extent can Carzain read the thoughts of trees? Much of it is probably just guessing based on vague emotional impressions.)

He also sees the \hr{Moons}{moons} in the forest. They look spectral and spooky. Have some Bal-Sagoth-esque imagery about them.






\subsubsection{Carzain dreams of being nothing}
On the way back to \hr{Redglen}{\Redglen}, Carzain dreams \hr{Ramiel is nothing}{his recurrent nightmare of being nothing}. 

He sees visions of humongous gods: 
\Human-like, \dragons, amoebae with pseudopods\prikker





\subsubsection{Betrayed by mercenaries}
On the way back to \hr{Redglen}{\Redglen}, their hired guards mutiny against them. Carzain fights back using magic and suddenly feels a strange ecstasy, a possession. He fights with power he did not know he had and slays the thugs. He is puzzled by the possession experience. 

The magic should be more bloody. I should show how the skin blackens and seethes, let people scream their agony while they boil and burn. They get cramps and spasms and perform a sickly, morbid \quo{dance} before they die from being zapped. 

Maybe fire magic should cause the victim to burn up \emph{from the inside}, instead of just tossing fireballs. Or perhaps tossing fireballs is the beginner's tactic: Easier and cheaper, but less powerful and less accurate. Causing your victim to combust from within is harder, but much more deadly and accurate. It uses the victim's own life force as fuel, burning him to ashes. 

Describe how blisters boil and burst. 





\subsubsection{The Redcor discover the \vertexspike}
A Redcor \matron\dash\Esmerel?\dash is sitting in the Topaz \Chateau, performing divination though \index{aquamancy}. Insert some Bal-Sagoth imagery here. 

\lyricsbs{Bal-Sagoth}{A Black Moon Broods Over Lemuria}{
  And in the dark ethereal mists of winter dreams,\\
  the ebon waters of enlightenment gleam 'neath the Black Moon,\\
  and the Valley of Silent Paths beckons\prikker
}

She discovers the \vertexspike{} that is caused by Vizicar's brief awakening, and realizes something has happened. I am not sure how much the Redcor know about \vertices, but she definitely knows something is up. She goes into research, and eventually they determine that very likely a Scion has awakened. 

\Esmerel{} goes off to wander the lands of \Velcad{} in search of the Scion. A year or so later she finds him, in the midst of the Pelidor-Runger war\prikker






\subsection{Changes}
The mercenaries have guns (pistols or muskets).

Maybe I should edit the date. 

Carzain and Nishain have not been to the Imetrium, merely some big port city in southern \Scyrum. 
They do not know so much Imetric. 
Nishain only knows a few rudiments, and Carzain none at all. 

When Vizicar fights, he ignores the \hr{Itzach pain}{pain that \Itzach{} brings}. 
\Itzach{} Vaimons have ways of using power economically, thus minimizing the pain and risk. 
Vizicar does not care about this. 
He is groggy and sleepy and his senses are dulled, so he thinks he is completely fresh and healthy and has plenty of bodily energy to spare. 
He does not want to be careful and conservative. 
He wants to be flamboyant and teach these damned peasants a lesson. 

After all, he is \VaimonCaliph. 
He is used to being one of the most powerful and skilled Vaimons in the \caliphate, and a Scion to boot. 
He does not stop to consider that he is out-of-touch with his body, or that his new body is weaker and less trained than his old one. 

Remember to have references to the \hr{Shechinah}{\shechinah}. 

After the battle, Carzain has nasty open wounds caused by drawing too much \qliphah{} power. 
He uses these heroic wounds to score Mirai. 
Then he gets healed by Imetric healers. 

The chapters with \ChyrieEsmerel{} and \LocarPsyrex{} do not happen at all. 
Get rid of them. 
At this point in time, no one knows about the \vertex. 
If they did, that would just cause all sorts of trouble and plot holes. 















\section{Ilcas Northstar}
\target{Ilcas Northstar chapter}
\target{Ilcas introduction}
Carzain and Nishain met Ilcas Northstar in \Bryndwin. 









\subsection{Old description}
Carzain and Nishain go back out of Heropond forest to the town of \Bryndwin.

\Bryndwin has its share of monuments (see section \ref{Monuments}): Poles with flags and lamps on them, carved images of gods, heroes and creatures, meant to scare the \Wylde and intruders away.

They encounter \hs{Telcastora Ilcas}, his two \nycans, Razor and Countess, and a bunch of other Imetrians. 

Remember to describe \hs{Telcastora Ilcas's appearance}.

They see a big \hr{Mulgron}{\mulgron}. Carzain is impressed by how the \mulgron is even bigger than an elephant. 

Remember to \hr{Nycans are frightening}{describe the \nycans as frightening and alien}.

Carzain reaches out his hand and tries to touch Razor's snout. Razor's eyes narrow slighty, and he retracts the hand. 

The two Vaimons hitch a ride back to \Redglen with Telcastora Ilcas and his Imetrians. Carzain befriends Ilcas and his \nycans, Razor and Countess. 

They part ways somewhere in Pelidor. 

Carzain has a wound from his battle with the thugs, but he suppresses the pain, as he doesn't want to show weakness in front on Telcastora Ilcas. I should try to portray him as somewhat over-proud and silly. 

During the trip back, he is treated by an Imetric healer-surgeon. But he retains the scars: One in his leg, one in the side of his stomach. 
The surgeon's treatment is more bloody than the Iquinian healing he is used to, but it is at least as effective. The Imetrians have a scientific approach to medicine, whereas the Redcor repress real science in favour of traditional mysticism. 
This makes Carzain doubt the value of Vaimon tradition.









\subsection{Changes}
Write about Carzain's meditation. 

Replace \quo{resonance} by \quo{\shechinah}.
Add \quo{\shechinah} to the glossary. 

\Bryndwin has a \hr{Wylde border}{\Wylde border} with charms and totem poles. 

Carzain introduces himself to Ilcas: 
\ta{I am \VizicarDurasRespina.} 

Ilcas carries a gun. 

When negotiating with the priest: 
Carzain has no interest in theology, but he is skilled at faking interest, if he does say so himself. 
It is a skill he has honed when picking up women. 

On their first day on the caravan, Carzain talks to Ilcas. 
Ilcas tells stories of his heroic and violent deeds, and of his heroic and violent sword. 
He especially tells of how he has fought against the Imetrium's arch-nemeses, the Rissitics. 
\hr{Rissitics and Runger}{This is important. The Rissitics are important to the story.} 
Make sure to paint the Rissitics as threatening and imposing foes.
Tell of how they wear animal masks and have warrior orders named Order of the \hr{Order of the Crocodile}{Crocodile} and Jackal and Scorpion and \hr{Order of the Hippopotamus}{Hippopotamus} and so on. 

These associations of war makes Carzain see visions of his past, but still very vague, little more than regular daydreams. 

Then Carzain asks to hold the sword. 
Ilcas refuses. 
But he shows him the sword and swings it around a bit himself. 
And tells him about it. 

Carzain scores Mirai faster (already on the first night) and more solidly. 
I need to show the reader how cool he is. 
He pulls off a pick-up comparable to the way Tony Stark scores the reporter girl in the movie \cite{Movie:IronMan}: 
\ta{I'd be willing to lose a few hours of sleep with you.}
She is impressed already after their very first meeting. 

Have more Vizicar in the sex scene with Mirai. 
And after it. 
He talks to Carzain in his head. 
Perhaps from Vizicar's POV. 

The night after sex, he lies awake thinking wildly about what in all of \Iquin and \Itzach is happening to him. 
When he falls asleep, he meets Vizicar in dreams. 

\target{Vizicar drives Carzain to war}
After this night, Vizicar drifts away. 
Carzain feels lost. 
Like something is missing inside him. 
Some important clue in some important mystery. 
Some part of himself. 
He feels like half a man. 

Carzain goes off to war hoping to find Vizicar again.  
He seeks out battle in the hope of provoking more Vizicar action in his head, so he can figure out who Vizicar is and what the deal is. 
Maybe Vizicar subconsciously imparts in him a desire to fight and seek glory. 
















\section{Visions and dreams}
Carzain had visions and dreams. 
As did Vizicar. 

They met in dreams and talked to one another, but neither understood what was going on. 
Both thought the other man was a figment of their own imagination. 

The dreams were triggered by combat and war, or the expectation of it. 









\subsection{Vizicar's motivation}
Vizicar knew he is more than a \human, and he told Carzain this. 
Vizicar wanted to attain \kenosis and regain his body, power and memory. 
Then he wanted to achieve \apotheosis and learn his true nature.
Then he would be a god.












\section{Carzain in the War}
\target{Carzain joins the army}
Pelidor went to war against Beirod. 
A trivial border dispute, but bloody enough. 

At this point, Carzain knew there was something odd going on with him. 
Carzain joined the army because he hoped it would help him find out what was happening to him, who he was and who Vizicar was. 
He could hear Vizicar stirring in his head at the prospect of war and glory: 

\begin{prose}
  \tho{I thought so. 
    It is this sort of thing that brings you out.
    \quo{Vizicar}.
    Whoever you are.
    I will find out what the deal is with you.
    And how I can get rid of you.}
\end{prose}







\subsection{War is Coming}







\subsubsection{Carzain joins the army}
News of the impending war with Runger reaches \Redglen. 

Carzain's parents hate war. Perhaps they have even done stuff to get him to dodge the draft - without his knowledge. Any any rate, Carzain goes to the barracks and signs up. See, Carzain is bored as fuck. He's a big boy in a small town. He has a craving for greatness, fame and power - things he cannot possibly find in \Redglen. He wants to explore and wants something to happen. So he goes off to war. At this point, Carzain is maybe 22. 

In the Pelidorian army, Carzain is welcomed because of his magical knowledge. Pelidor has few mages, since neither the Imetrium nor the Iquinian church support them. He is assigned to serve under Captain Archibald Curwen, the commander of the \ishrah of the Pelidorian army.  
Carzain also learns a lot of combat technique. As he studies combat, he finds all sorts of things coming naturally to him, as if he already knows them but has just forgotten he knew. 

On the way to \Malcur, Carzain calls dibs on a horse\dash a gray stallion named Arrow\dash rather than a \relc. Later, having trouble riding the beast, he ponders his decision. Seeing some of the others getting easily ahead on their \relcs, he realizes that a horse is not necessarily better. It is just that horses are more Vaimon-like. In stories, Vaimons always ride horses, so when he pictures himself in his head as a great Vaimon hero, he sees himself astride a horse, not a \relc or other reptile. Carzain realizes that he is a status-fixated snob. 





\subsubsection{Carzain and the Goldsmiths}
He talks to the Goldsmith brothers. 
He talks of how he is destined for greatness. 
At this point he remembers \hr{Ramiel dreams of being nothing}{his recurring nightmare of being nothing}, and he tries desperately to convince himself that he is not nothing. 
He tells himself that he is big and important, because he fears that dream. 









\subsection[To Malcur]{To \Malcur}
Carzain \travelled to \Malcur to join the Pelidorian \ishrah.

On the way he met Adrian Testor. 

In \Malcur he met Archibald Curwen. 

Have plenty of Vizicar scenes early on, including before, during and after his signing-up to the army. 
(If \quo{The Mutiny} and \quo{Ilcas Northstar} are kept in \TwilightAngelRememberEmph, then include lots of Vizicar scenes there. Otherwise move them to the prequel story about Carzain.)

Vizicar (when he gained some \kenosis) liked the life of an adventurer. 
It was a refreshing change of pace from the court life he was used to, full of responsibility and chains. 
This was a new kind of freedom, and Vizicar loved it. 





\subsubsection[Carzain meets Delph and Tsekkect]{Carzain meets Delph and \Tsekkect}
\target{Carzain meets Delph}
Carzain meets \hs{Delph} and \hr{Tsekkect}{\Tsekkect}.

They are cheerful guys. 
They befriend Carzain and show him around the barracks, where, among other things, Delph utters his classic line: 
\ta{%
  Life in the army is great. 
  We get food, money, a shot at glory, and the best pussy that money can buy. 
  Well\prikker the best pussy that \emph{our} money can buy, anyway.} 

To which Carzain replies: 
\ta{And all they ask in return is that we fight to the death.}






\subsubsection{Carzain's idealism}
At the beginning of the story, \hr{Carzain's idealism}{Carzain is quite idealistic}.





\subsubsection{Vizicar awakens a little bit}
At some time at the beginning of the war, before Carzain's first battle in the army, Vizicar surfaces a little bit. 









\subsection{To battle}
They go to battle the enemy. 





\subsubsection{Carzain and Delph practice with swords}
Carzain and Delph practice their swordsmanship. 

When the two duel, Carzain mostly loses. He knows some technique, taught to him by Sir Guy in \Redglen, and he is taller, which gives him a better reach. But the shorter and cruder Delph is faster and more agile, and he fights dirty. 

Carzain: \ta{You fight dirty, you know that?}

Delph: 
\ta{No, I just fight like a \meccaran. 
  That attitude of yours is just \human{} provincialism.}





\subsubsection{Carzain won't smoke}
Carzain hangs out with some soldiers. The others smoke some kind of strong drug, like hash or marijuana. 

\begin{prose}
  Soldier: \ta{Come on, \Shireyo, get over here and smoke.}
  
  Carzain: \ta{No, I don't touch that. It's filth.}
  
  Soldier: \ta{Why do you say that?}
  
  Carzain: 
    \tho{%
      Why \emph{do} I say that? 
      Because my father told me so, but I am not going to tell them that. Think, Carzain, think! 
      I can't lose face in front of these guys. 
      Say something witty.} 
    \ta{Something witty.} 
    \tho{Phew. Nicely salvaged, Carzain.}
\end{prose}





\subsubsection{Curwen and Carzain eat}
The officers of the Pelidorian army are eating a luxurious meal. This includes the \ishrah. 

Curwen: \ta{Ah, by \nieur, I love food.}

Carzain: \ta{Food is good, but I like sex better.}

Curwen: \ta{Meh. Most women cook better than they fuck.}

Both: \ta{LOL.}









\subsection{The \Caliph Inviolate}
The sortie encounters a small Rungeran force. They fight. 

Carzain fights and kills. 
But he is not as great a warrior as he fancies himself. 
So he gets clobbered and takes wounds. 

He calls out to Curwen-tachi for help. 
They save his ass. 

Then Vizicar awakens. 
It's due to a combination of factors: 

\begin{itemize}
  \item The pure rush of combat. 
  \item The anger and shame over being defeated in combat by such lowly grunts. 
  \item The wounds leave Carzain's own conscious mind weakened, making it easier for Vizicar to take over. 
\end{itemize}


% Recall that Carzain is a Scion, an incarnation of the \Malach{} Ramiel. 
% Occasionally, when he fights and kills, Ramiel awakens in him, and he hears the voice of \VizicarFull, the previous incarnation of Ramiel. (This is inspired by, and similar to, how Rand al'Thor in Wheel of Time hears the voice of Lews Therin Telamon. But where Lews Therin is mad, annoying and even dangerous, Vizicar is sane and a powerful ally.) 

% Anyway, Carzain fights and kills, and Vizicar comes along and takes over. 

Curwen immediately senses something fishy, probably a \vertexspike{}. 

This time there is a struggle between Carzain and Vizicar, as Carzain tries to get back in control of his own body. 

%After the battle, Carzain falls unconscious as Vizicar retreats. The prolonged, unaccustomed possession and the struggle between the two personas was too much for his consciousness to take. 

After the battle Curwen probes, prodding Carzain into telling him stuff. Carzain tells a little of what he knows, but it's not much. 

Carzain's body is in a bad shape. 
Vizicar is a harsh master and takes poor care of it. 
This is because Vizicar is only half-awake. 
He's still in a half-dreamy state, and his senses are weakened. 
He doesn't entirely register the pain, so he just pushes on. 
So after the fight Carzain is a useless wreck, barely conscious. 

Maybe Carzain asks the others: \ta{Did I save the da?} 

The others: 
\tho{Save the day? You crazy fuck, you nearly killed yourself, and maybe us, too.} 
\ta{Um, yeah. Sure. You saved the day.}

Delph is scared by Carzain's weirdness. Maybe insert an inner monologue with Delph about how Carzain is a crazy weirdo: 
\tho{Typically. 
     Finally I meet a guy who seems trustworthy, who might become a true friend, and turns out he is a madman. 
     Or possessed by \mdaemons{}, or whatever. 
     Huh. 
     That'll teach me to trust a sorcerer.}

During the fight, Carzan does something stupid and reckless. 
Perhaps he sees an enemy Vaimon and goes in to capture his sabre. 
Afterwards, Curwen bitchslaps him and scolds him: 
\ta{You crazy fuck, \Shireyo!
    What the fuck were you trying to do?
    You very nearly got yourself and the rest of us killed!}
He grabs Carzain by the collar. 
\ta{Let me tell you something, boy.
    Your insubordination stopped being charming a long time ago. 
    Now either you get in line or I'll have you beaten!}





\subsubsection{Vizicar dreams}
Have another scene that describes the battle from Vizicar's point of view. 










\subsection{Dark Crypts of the Mind}
Now that Vizicar is closer to the surface, Carzain begins to \hr{Ramiel dreams}{share his visions and nightmares}. 

He is haunted by the tortured, hateful faces of shackled ghosts. They show him their world: Cold, cruel, dreadful. They hate him and desire to drag him down into their prison\dash groping for him with their clammy, hideous hands. 

He awakens full of terror, and also a tinge of guilt. He has a feeling that he is somehow responsible for their torment. 

He also has vague dreams of lost greatness\dash \hr{Visualizing a matrix}{unconsciously visualizing} the \hr{Mystraacht Matrix}{\Mystraacht{} \matrix}. Feelings of having been betrayed, having lost his power and his memory, upon which his enemies descended upon his weakened self like vultures, usurping his rightful place and \honour and forcing him into flight and hiding. 
%, having his rightful place, power and \honour usurped

When he wakes he thinks about the whole thing. 
Part of him is horrified at the madness that is consuming him. 
But another part is excited. 
Because the possession brings \emph{power}. 
And a feeling of being destined for greatness and glory. 
This is what he has always dreamt of. 

But he has doubts. 
Maybe these feelings are fake. 
Maybe the possessing \Archon{} is brainwashing him, all the while making him feel like it is the right thing to do. 
Maybe he is being mentally deceived. 

But it still feels good. 

He thinks of the legend of \hr{Lestor}{Lestor \Delaen}. 
He dealt with \qliphoth{} despite everyone's advice. 
It was dangerous. 
He became a dark and controversial figure, but also a hero who achieved great things. 





\subsubsection{Carzain thinks more on his madness}
Part of him is horrified at the madness that is consuming him. 
But another part is excited. 
Because the possession brings \emph{power}. 
And a feeling of being destined for greatness and glory. 
This is what he has always dreamt of. 

But he has doubts. 
Maybe these feelings are fake. 
Maybe the possessing \Archon{} is brainwashing him, all the while making him feel like it is the right thing to do. 
Maybe he is being mentally deceived. 

But it still feels good. 

He thinks of the legend of \hr{Lestor}{Lestor \Delaen}. 
He dealt with \qliphoth{} despite everyone's advice. 
It was dangerous. 
He became a dark and controversial figure, but also a hero who achieved great things. 

Maybe move this to \quo{The \Caliph Inviolate}. 





\subsection{Vizicar}
Vizicar fights to gain control of his own body. 

\lyricsbs{Marduk}{Voices From Avignon}{
  Speak through me. Speak through me.\\
  The sin must be washed away with blood.\\
  Dream through me. \\
  Called to a new life through death.\\
  Called to a new life through death.\\
  But what shall you reach for when all \colours fall?\\
  Overwhelmed with maledictions. \\
  Feel the rays of redemption of a brand new Sun.
  
  Choking. Asphyxia. \\
  Inhale the Darkness, lungs filled up. \\
  Asphyxia. Asphyxia. \\
  But who shall you reach for when all \colours fall?\\
  Long-drawn moans and piercing cries \\
  blend with prayers and litanies. \\
  Faces bone dry. Condemned as the river everybody drank of. \\
  Inhale! Inhale!
}

\ta{I will win the power and glory I deserve. 
  Not just as \VaimonCaliph, but higher yet. 
  I am a Scion. 
  I am a \Malach. 
  I will achieve the \apotheosis{} and become one with \Iquin{} and \Itzach.
  That is my purpose as a Scion. 
  
  But first I must gain control of my own body. 
  Right now I live out an ignominious existence as a ghost in a boy's head. 
  I am nothing\prikker}

This makes him remember his nightmare of being nothing. 
He drifts away. 

He philosophizes. 

\lyricsbs{Emperor}{Sworn}{
  Four eyes as two in one.\\
  The forward circular view that never ends.\\
  An orbital voyage throughout the endless sphere of all,\\
  where time is lost and everything transcends.
  
  A graceful presence at stolen time
  
  As ghosts to the world.\\
  Ghosts to the world.
  
  For ice, outside, are we apart.\\
  As cold and eerie mist to the hand.\\
  Ever floating on its course\\
  towards the heights of shadowland.\\
  
  Thus appear the truly sworn.
  
  To be seen.\\
  To be feared.\\
  Yet, not to be reached.
}





\subsubsection{Lost in the labyrinth}
He is trapped in the labyrinths of \itzach{}, lost and searching for those precious and ephemeral windows through which he can peer out through Carzain's eyes and affect the world through Carzain's body. 

\lyricsbs{Emperor}{Ensorcelled By Khaos}{
  In an endless maze begotten,\\
  to dead ends led by fools.\\
  I sought a plague for those\\
  who smiled at walls in humble fear.\\
  
  Yet, belligerent ways came to an end.\\
  No war could ease my hunger.\\
  By craving death I was to darkness reborn.
  
  No sacrifice too great.\\
  Caught in another maze.\\
  Truly endless.\\
  Still this maze is mine.
  
  No peace for me. No peace I seek.\\
  My quest goes far beyond.
  
  No sacrifice too great.\\
  Caught in another maze.\\
  Truly endless.\\
  Still this maze is mine.
  
  I hail the greatest fall.\\
  Dark is the spirit of my token.\\
  Dark is my call.
  
  Still craving, still torn.\\
  Hear me, master of this realm\\
  to which I was reborn.
  
  \ta{Bearer of my stigma. Thou art of me.\\
  Yet, claim no gift, but guidance.\\
  For I am the giver of temptation.\\
  Thou art the executor of Thy rewards.}
  
  Dark is the spirit of my token.\\
  Dark is my call.
  
  The cosmic forces driving me,\\
  more noble, more free.
  
  Take me.

  Love and hate and all in between,\\
  I greet them all in ecstasy.\\
  Venus, seduce me.\\
  Mars, possess me.\\
  Tear my soul from sanity.
  
  No sacrifice too great.\\
  Caught in another maze.\\
  Truly endless.\\
  Still this maze is mine and I am thine.
}









\subsection{Reads about Scions and Vizicar}
Maybe Carzain overhears someone (\Esmerel{} and \Racel{}?) talking about Scions. 
Maybe he is half-unconscious and they are standing over him, talking about him, trying to decide his future. 

He asks someone about the story of the Scions. 
Who? 
Perhaps \Racel, but more likely some other mage in the \ishrah. 
Perhaps \Sanyor. 

Anyway, he is told the story of the Scions. Or a version of it. Remember to have many false details, which can be brought into question later. 

Later, he goes into a library in \Forklin{} and reads about \VizicarDurasRespina. 
There isn't a big public library, but there are churches, and they have books. 
He goes into a church, introduces himself as an \ishrah{} mage and pulls a \trope{BavarianFireDrill}{Bavarian Fire Drill}.
He knows he has to use the same body language, frame control and social engineering skills that he masters when picking up women.
It works.
He gets his hands on the writings.

He also finds references to \hr{Vizicar's notes}{Vizicar's notes about \Malachim}, but can find no fragments of the notes themselves. 

He also finds references to \hr{Iolivine's notes}{\ps{\Iolivine} notes}. 
He has heard before that these notes are of exceptionally high quality, but he had been unable to find any copies of them. 
\hr{Redcor bogarted Iolivine's notes}{The Redcor are bogarting the notes}. 
This is one of the reasons he wants to go to \Redce. 

Among other things, he learns that \Belzir{} was an incarnation of an \quo{evil} \malach, and that she caused the \HundredScourges{} with her wicked \malach{} powers. 

More importantly, he learns that he is not mad. 
There are no \qliphoth{} controlling him. 
Instead, he is a Scion! 
He is an \uber-dude! 

He is happy. 









\subsubsection{Carzain reads about Scions}
Carzain is always researching what he is, always looking for more knowledge of Scions. 
He hopes it will lead him to \apotheosis. 

So he goes into a library in \Forclin{} and looks around. 
There isn't a big public library, but there are churches, and they have books. 
He goes into a church, introduces himself as an \ishrah{} mage and pulls a \trope{BavarianFireDrill}{Bavarian Fire Drill}.
He knows he has to use the same body language, frame control and social engineering skills that he masters when picking up women.
It works.
He gets his hands on the writings.

He also finds references to \hr{Vizicar's notes}{Vizicar's notes about \Malachim}, but can find no fragments of the notes themselves. 

He also finds references to \hr{Iolivine's notes}{\ps{\Iolivine} notes}. 
He has heard before that these notes are of exceptionally high quality, but he had been unable to find any copies of them. 
\hr{Redcor bogarted Iolivine's notes}{The Redcor are bogarting the notes}. 
This is one of the reasons he wants to go to \Redce. 

Among other things, he finds references to \Belzir. 
He already knows she was a Scion, allegedly an incarnation of an \quo{evil} \malach, and that she caused the \HundredScourges with her wicked \malach powers. 









\subsection{Why are you in my head?}
Somewhere along the way, Carzain-tachi get into a fight. Vizicar awakens. This time, after the battle Carzain falls unconscious. He dreams and meets Vizicar. 

He sees before his eyes a man who looks to be perhaps his father's age. 
But still strangely similar to himself. 
His complexion is much lighter, though. 
His long hair is blonde rather than black, and he wears a matching moustache. 

Carzain: \ta{Who are you?}

Vizicar: 
\vizicar{%
  Who am I? How can you not know who I am? I am \VizicarDurasRespina! The \caliph!}

Carzain: 
\tho{%
  \quo{\Caliph}\prikker is he the reason why that word has been popping up in my head?} 
(Maybe he doesn't think this until after he awakens from the dream-conversation.)

Carzain: \ta{Why are you in my head?}

Vizicar: \vizicar{Why are you in \emph{my} head?}

Carzain: \ta{I was here first.}

Vizicar: 
\vizicar{%
  Unless\prikker oh, \qliphoth\prikker unless I\prikker oh, by \iquin{} and \nieur. I must be dead.

  Yes, I remember now. Those vile worms killed me. \quo{Disease}, they called it. Hah!}

Carzain: 
\tho{%
  Fuck. 
  Not only is there a madman in my head, but now the madman is talking to himself!}

Carzain wonders what the deal is. 
How does this fit with the theory that it was the Midnight \Qliphoth{} that took control of him? 

Later, Vizicar apologizes to Carzain for putting him through Hell by his own recklessness, by ignoring the \hr{Itzach pain}{pain of \itzach}. 
Vizicar admits that he has been crazy and has fucked their body up. 









\subsection{Changes}
\begin{changes}
    
  \begin{comment}
  \paragraph{War is Coming}
  \end{comment}
  \changesitem{War is Coming}
    Remember to have references to the \hr{Shechinah}{\shechinah} in Nishain's speculations. 
    
    Rewrite the \quo{I want to fight} section from the officer's POV. 
    
    Mention Carzain's age (\quo{a young man, around 25\prikker}). 
    
    Make sure the soldiers notice that Carzain is a bit swarthy (being half Geican). 
    
    This might be the place where Carzain introduces himself:
    \ta{I am \VizicarDurasRespina. No! Wait\prikker I am Carzain \Shireyo\prikker}

  \begin{comment}
  \paragraph{To \Malcur}
  \end{comment}
  \changesitem{To \Malcur}
    Rewrite the scene where Curwen and Carzain meet, so it is from Curwen's POV. 
    
    Curwen should be smoking his pipe. 
    
    Mention Carzain's age (\quo{a young man, around 25\prikker}). 
    
    This might be the place where Carzain introduces himself:
    \ta{I am \VizicarDurasRespina. No! Wait\prikker I am Carzain \Shireyo\prikker}
  
  \begin{comment}\paragraph{\Forklin}\end{comment}
  \changesitem{\Forklin} 
    Get rid of \Racel. 
    
  \begin{comment}
  \paragraph{The \Caliph Inviolate}
  \end{comment}
  \changesitem{The \Caliph Inviolate}
    Again, Vizicar ignores the \hr{Itzach pain}{pain of \Itzach}, and Carzain gets burnt and wounded as a result. 
    
    Remember the \hr{Itzach pain}{pain of \itzach} when Curwen casts magic. 
    
    Remember to have references to the \hr{Shechinah}{\shechinah}. 
    
    Curwen needs to tell the reader that the \ishrah{} mages are equipped with plate \armour. 
    They just do not wear it when \travelling, nor on stealth missions like this. 
    But when the battle comes, he misses it. 
    
    Maybe Vizicar holds back his power and magic. 
    He does not want to arouse suspicion and get caught. 
  
  \begin{comment}\paragraph{\Tsekkect{} and Delph}\end{comment}
  \changesitem{\Tsekkect{} and Delph} 
    Make sure people like \Tsekkect{} and Delph have inner lives of their own and don't spend all their time gaping at Carzain. 
\end{changes}
















\section{Mercenary}
After this short war ended, Carzain left the army and became a mercenary and adventurer.
























\chapter[Twilight Angel, Remember]{\TwilightAngelRemember}
\target{Twilight Angel, Remember}

















\section{Overview}
\begin{quote}
  \Miith{} is a cruel world. 
  
  \new
  For thousands of years a war has raged between the \dragons{} and the \resphain\dash a race of dark angels\dash and their mortal agents and pawns. 
  
  \new
  \VizicarDurasRespina, a long-dead sorcerer, awakens and finds himself a ghost trapped in the body of a strange man. 
  Vizicar must reconcile himself with his new body\dash and with a world that has changed in the many centuries since his death\dash before he can hope to complete his life's quest: 
  To discover who he is and why he is condemned to die and be reborn. 
\end{quote}




\begin{comment}
\bookchapter{Dramatis Personae}



% \section{Immortals and their servants}
\section[Dragons]{\maybehr{Dragons}{\Dragons}}
\begin{dramatispersonae}
  \dramitem[Sethicus]{\VardredSethicus}{\dragon}{male}, 
    founder of the \draconian race, 
    deceased
%     perished before the \secondbanewar{}
  \dramitem[Tiamat]{\TyarithXserasshana}{\dragon}{\female}, 
    ancient \dragon queen,
    deceased
%     perished before the \secondbanewar{}
  \dramitem[Nexagglachel]{\RaemythNexagglachel}{\dragon}{\male}, 
    \maybehr{Shae'eroth}{\shaeeroth}, 
    deceased
%     perished before the \secondbanewar{}
%     first son of \Tiamat{} 
  \dramitem[Ishnaruchaefir]{\QuessanthIshnaruchaefir}{\dragon}{\male}, 
    \maybehr{Shae'eroth}{\shaeeroth},
    called the Destroyer and the Exile
%     second son of \Tiamat{} 
  \dramitem[Secherdamon]{\IrocasSecherdamon}{\dragon}{\male}, 
    \maybehr{Shae'eroth}{\shaeeroth}
%     third son of \Tiamat{} 
  \dramitem[Nzessuacrith]{\CryocasNzessuacrith}{\dragon}{\female}
%   \dramitem{Kelthasserai}{\dragon}{\male}, 
%     fallen in the \maybehr{Second Banewar}{\Secondbanewar}
%   \dramitem{Skelcurmaggra}{\dragon}{\female}
%     fallen in the \maybehr{Second Banewar}{\Secondbanewar}
%   \dramitem{Izvathorn}{\dragon}{\male}
%     fallen in the \maybehr{Second Banewar}{\Secondbanewar}
%   \dramitem[Vaccashyth]{\Vaccashyth}{\dragon}{\female}
%   \dramitem[Dasvedshiracht]{\Dasvedshiracht}{\dragon}{\male}
\end{dramatispersonae}

\subsection{Servitors}
\begin{dramatispersonae}
  \dramitem[Psyrex]{\LocarPsyrex}{\scatha}{\male}, 
    leader of the \maybehs{Dark Crescent}
  \dramitem[Criseis]{\Criseis}{\scatha}{\female}, 
    servant of \Ishnaruchaefir
\end{dramatispersonae}




\section[Resphain]{\maybehr{Resphan}{\Resphain}}
\begin{dramatispersonae}
  \dramdead[Thanatzil]{\Thanatzil}{\resphan}{\male},
    the first \resphan
%     perished before the \secondbanewar{}
  \dramitem[Azraid]  {\Azraid}{\resphan}{\male}, 
    High Lord of \maybehr{CS}{\KiriathSepher}
  \dramitem[Teshrial]{\Teshrial}{\resphan}{\male}, 
    \maybehr{Ketheran}{\ketheran} of \CiriathSepher
    \begin{subdramatispersonae}
      \dramitem[Zereth]{\Zereth}{\resphan}{\female}, 
        daughter of {\Azraid}, mother of {\Teshrial}
      \dramitem[Tuerdal]{\Tuerdal}{\resphan}{\male}, 
        father of {\Teshrial}
    \end{subdramatispersonae}
  \dramitem[Firaxel] {\Firaxel}  {\resphan}{\female},
    \ketheran of \maybehr{Tiphred-Serah}{\TiphredSerah}
  \dramitem[Urizeth] {\Urizeth}  {\resphan}{\female}, 
    \maybehr{Thelyad}{\thelyad} of \CiriathSepher
  \dramitem[Ganethed]{\Ganethed} {\resphan}{\male},   
    \maybehr{Thelyad}{\thelyad} of \CiriathSepher
  \dramitem[Achsah]  {\Achsah}   {\resphan}{\female}, 
    \maybehr{Beuzed}{\bezed}
  \dramitem          {Lelmach}   {\resphan}{\female}, 
    \maybehr{Beuzed}{\bezed}
\end{dramatispersonae}

\subsection{Servitors}
\begin{dramatispersonae}
  \dramitem{Duma}{\human}{\female}, \naor matron
  \dramitem{Evith, Jirin, Luria}{\human}{\female}, young \naorim
\end{dramatispersonae}



\section[Pelidorians]{\maybehr{Pelidor}{Pelidorians}}
\subsection[House Pelidor]{\maybehr{House Pelidor}{House Pelidor}}
\begin{dramatispersonae}
  \dramitem[Icor]  [\Icor{} Pelidor]
    {\Rayuth[\Icor] Pelidor}{\scatha}{\male}, 
    ruler of Pelidor
    \index{Pelidor!\Icor{} Pelidor}
  \dramitem[Tiroco][\Tiroco{} Pelidor]
    {\Rinyuth[\Tiroco] Pelidor}{\scatha}{\female}, 
    his wife
    \index{Pelidor!\Tiroco{} Pelidor}
%   \begin{subdramatispersonae}
%     \dramitem[Roric]{Roric Pelidor}{\scatha}{\male}, 
%       their son
%       \index{Pelidor!\Tiroco{} Pelidor}
%     \dramitem[Frico]{Frico Pelidor}{\scatha}{\female}, 
%       their daughter
%       \index{Pelidor!\Tiroco{} Pelidor}
%     \dramitem[][egg]{An unnamed egg}{\scatha}{?}, 
%       their third child
%   \end{subdramatispersonae}
%   \dramitem[Liocai]{Liocai Pelidor}{\scatha}{\female}, 
%     \ps{\Icor}{} younger sister
%     \index{Pelidor!Liocai{} Pelidor}
  \dramitem[Sethgal][\Sethgal{} Pelidor]
    {\Rah[\Sethgal] Pelidor}{\scatha}{\male}, 
    \ps{\Icor}{} cousin
    \index{Pelidor!\Sethgal{} Pelidor}
  \dramitem[Dornaer][\Dornaer{} Pelidor]
    {\Rah[\Dornaer] Pelidor}{\scatha}{\female}, 
    \ps{\Tiroco}{} elder sister
    \index{Pelidor!\Dornaer{} Pelidor}
\end{dramatispersonae}

% \subsection{At the court in \Malcur}
% \begin{dramatispersonae}
%   \dramitem[Wulfwin Norden][\WimarNorden]{\Pater{} \WimarNorden}{\scatha}{\male}, 
%     \maybehr{Telcra}{\Telcra} \maybehr{Cleric}{\cleric} 
%   \dramitem{Iasper Bartholin}{\scatha}{\male}, 
%     ducal treasurer
%   \dramitem[][Graenell]{Baron Graenell}{\scatha}{\male}
%   \dramitem[Risvet Hemfork][\Risvet{} Hemfork]
%     {Baroness \Risvet{} Hemfork}{\scatha}{\female}
%   \dramitem[][Osphal Turmalin]
%     {Viscountess Osphal Turmalin}{\scatha}{\female}
%   \dramitem[Theal Kintair][\Theal{} \Kintaer{}]
%     {Earl \Theal{} \Kintaer{}}{\human}{\male}
%   \begin{subdramatispersonae}
%     \dramitem[Constance Kintaer]{\Constance\ \Kintaer}{\human}{\female}, his maiden daughter
%   \end{subdramatispersonae}
%   \dramitem{Charcoal}{\human}{\male}, 
%     Vaimon, 
%     Cabalist of the \charcoalcircle{} circle
%   \dramitem[Needle]{\Piacet\ (Needle)}{\human}{\female}, 
%     \ps{\Tiroco}{} handmaiden slave, 
%     novice Vaimon, 
%     Cabalist of the \needlecircle{} circle
%   \begin{subdramatispersonae}
%     \dramdead{Belya}{\human}{\female}, 
%       \ps{\Piacet}{} sister 
%   \end{subdramatispersonae}
%   \dramitem{Weyra}{\scatha}{\female}, 
%     \ps{\Tiroco}{} handmaiden slave 
%   \dramitem{Duen}{\scatha}{\female}, 
%     \ps{\Tiroco}{} handmaiden slave 
%   \dramitem{Sevac}{\scatha}{\female}, \ps{\Tiroco}{} bodyguard
%   \dramitem{Nobb}{\human}{\male}, a dungeon guard
% \end{dramatispersonae}

% \subsection{Commoners in \Malcur}
% \begin{dramatispersonae}
%   \dramitem{Rian}{\human}{\male}, a thief
% %   \dramitem[Badrick]{Patrick (\quo{Badrick})}{\human}{\male}, a thief
%   \dramitem[Bryon Carpenter]{\Bryon{} Carpenter}{\human}{\male}, 
%     an aging, childless man
%   \dramitem{Rod Baker}{\human}{\male}, a large man
%   \begin{subdramatispersonae}
%     \dramitem{Neina}{\human}{\male}, the baker's daughter
%   \end{subdramatispersonae}
%   \dramitem      {Dennick}{\human}{\male}, a thief
%   \dramitem[Uswa]{\Uswa}{\meccaran}{\female}, a drunken fortune-teller
%   \dramitem      {Jorgen}{\human}{\male}, \maybehs{Black Plague} gangster
%   \dramitem      {Ornen (Briar)}{\human}{\male}, 
%     Cabalist of the \briarcircle{} circle
% \end{dramatispersonae}

\subsection[The Ishrah]{The \maybehr{Ishrah}{\Ishrah}}
\begin{dramatispersonae}
  \dramitem[Moro Cornel]{Moro \Cornel}{\scatha}{\female},
    \maybehr{Rethyax}{\rethyax},
    academic head of the \ishrah{}
  \dramitem[Archibald Curwen][Archibald Curwen]
    {Lord Archibald Curwen}{\human}{\male}, 
    \maybehr{Telcra}{\Telcra} \maybehr{Templar}{\templar}, 
    military head of the \ishrah{}
  \dramitem[Onatol]{\Ambrose\ \Anatoli}{\scatha}{\male},
    \maybehr{Rethyax}{\rethyax}
%   \begin{subdramatispersonae}
%     \dramitem{Baernor}{\scatha}{\male}, \ps{\Onatol} apprentice
%   \end{subdramatispersonae}
%   \dramdead{Borg Zelab}{\human}{\male}, {\Telcra} {\templar}
%   \dramitem[Sanyor]{\Sanyor}{\scatha}{\male}, {\Telcra} \templar{}
%   \begin{subdramatispersonae}
%     \dramitem{Thedoro}{\scatha}{\female}, \ps{\Sanyor} apprentice
%   \end{subdramatispersonae}
%   \dramitem{Hicarro}{\scatha}{\female}, {\Telcra} \templar{}
%   \dramitem{Fiorae}{\scatha}{\female}, {\Telcra} \templar{}
\end{dramatispersonae}

% \subsection{The military}
% \begin{dramatispersonae}
%   \dramitem[][Gemadon]
%     {Captain Gemadon}{\scatha}{\male}, in \maybehr{Redglen}{\Redglen}
%   \dramitem[][Nimloc]
%     {Lieutenant Nimloc}{\scatha}{\male}, a scribe
%   \dramitem{Adrian Testor}{\human}{\male}, 
%     a wealthy \maybehr{Miksha}{\miksha} rider from \Redglen 
%   \dramitem{Rory}{\human}{\male}, a bladesmith from \Redglen 
%   \dramitem{Kreb}{\human}{\male}, a \Goyden{} boy
%   \dramitem[Tsekkect]{\Tsekkect}{\meccaran}{\female}, 
%     of the \maybehr{Thbatswa}{\Thbatswa} tribe 
%   \dramitem{Delph}{\human}{\male}, of \maybehr{Tepharin}{\Tepharin} descent
%   \dramitem{Gwelthein}{\scatha}{\female}, a \maybehr{Ranger}{\ranger}
%   \dramitem{Filcoi}{\scatha}{\female}, a \maybehr{Ranger}{\ranger}
%   \dramitem{Egian}{\scatha}{\female}
% \end{dramatispersonae}

\subsection{Others}
\begin{dramatispersonae}
  \dramitem{Claedd}{\scatha}{\female}, leader of Derwael
  \dramitem{Erowol}{\scatha}{\male}, an elder of Derwael
  \dramitem{Tur}{\scatha}{\male}, a youth of Derwael
  \dramitem{Bila}{\scatha}{\female}, a girl of Lunum
  \dramitem[Theal Kintair][\Theal \Kintaer]
    {\Rah[\Theal] \Kintaer}{\scatha}{\male}, noble and knight
\end{dramatispersonae}



\section{Imetrians}
\begin{dramatispersonae}
  \dramitem[Ilcas Northstar][Telcastora Ilcas]
    {\IlcSR{} Telcastora Ilcas \quo{Northstar}}
    {\scatha}{\male}, 
    \maybehr{Nycaneer}{\nycaneer} 
    \index{Ilcas!Telcastora Ilcas}
%   \dramitem{\Retaxis{} Raeco Mannica}{\scatha}{\female}, 
%     a \nycaneer{} and Ilcas' wife
%   \dramitem{Cassili Northstar}{\scatha}{\female}, 
%     their eldest child, Paladin-in-training 
%   \dramitem{Selcai Northstar}{\scatha}{\female}, 
%     their second child, a \nycaneer{}
%   \dramitem{Tarcus Mannica}{\scatha}{\female}, 
%     their third child, a soldier
%   \dramitem{Astor Mannica}{\scatha}{\female}, 
%     their fourth child, an apprentice mage
  \begin{subdramatispersonae}
    \dramitem{Countess}\nycan\female, companion of Telcastora Ilcas
    \dramitem{Razor}\nycan\male, companion of Telcastora Ilcas
  \end{subdramatispersonae}
  \dramitem[Ulphon Nestor][Ulphon Nestor]
    {\Ispan{} Ulphon Nestor}{\scatha}{\male}, 
    priest and mage
%   \dramitem{Equin Mirai}{\human}{\female}
\end{dramatispersonae}



\begin{comment}
\section[Rissitics]{\maybehs{Rissitics}}
\begin{dramatispersonae}
  \dramitem{\HriistN}{immortal}{\male}, supreme god (called Rissit by outsiders)
  \dramitem{\TesHanith\ \TsaltNyzleth}{\scatha}{\female}, high priestess
\end{dramatispersonae}
\subsection{In \FendorSmall}
\begin{dramatispersonae}
  \dramitem{Bantoyn \Rekkan-\Ondmyst}{\scatha}{\male}, 
    expedition leader (codename: Barrud).
  \dramitem{\Filgzed\ Hedrail Tsalt-\Sheshefkesad-\Ginfik}{\scatha}{\female}, 
    a mage (codename: Filiza).
  \dramitem{Dzavish Tsalt-\Sheshefkesad-\Bryn}{\human}{\female}, 
    a mage, her assistant (codename: Javiz)
  \dramitem{Br\^om}{\scatha}{\male} 
    (codename: Boruman), a sailor
  \dramitem{Dorm}{\human}{\male}, a Hazidi sailor
  \dramitem{Gelgein}{\human}{\male}, a Hazidi sailor
  \dramitem{Mamnik}{\human}{\male}, a trader (codename: Mamrim)
  \dramitem{Ashta}{\scatha}{\female}, a soldier (codename: Aisha)
  \dramitem{Juktat}{\scatha}{\male), a soldier (codename: Yudai) 
  \dramitem{A \Gisshorn\ agent}{\human}{\male}, codename: Mestos)
  \dramitem{Another \Gisshorn\ agent}{\human}{\male}, codename: Semphai)
  \dramitem{Ragev, Kirm and Noll}, \Filgzedz{} servants
\end{dramatispersonae}
\subsection{Near Fendor}
\begin{dramatispersonae}
  \dramitem{\Narkiza\ \Rekkan-\Neftzaid\ \Ashenoch-\Hashkfed} (\scatha{} \male), 
    general, wielder of the morning star \Femtu{}
  \dramitem{Belgrim}{\Cortio}{\male}, \Narkizaz{} mount
  \dramitem{Kufur \Rekkan-\Ondmyst}{\scatha}{\female}, 
    \Narkiza's assistant officer
  \dramitem{Geldashad \Rekkan{}-\Kozud{} \Ashenoch-\Fedza}{\human}{\male}
  \dramitem{\Dasvedshiracht{} Tsalt-\Shesshefkesad-\Ryzeyd{} \UrrGammosh}{\dragon}{\male}, 
    an undead \dragon{} mage
  \dramitem{Vekhtet Tsalt-\Shesshefkesad-\Kseinga]
  \dramitem{\Dzeredz}{\human}{\female}
\end{dramatispersonae}
\end{comment}



\section{Rungerans}
\begin{dramatispersonae}
  \dramitem[Morgan Runger][Morgan Runger]
    {King Morgan Runger}{\human}{\male}
%   \begin{subdramatispersonae}
%     \dramitem[Mathyas][Prince]  {\Mathyas}{\human}{\male}, 
%       Morgan's eldest son and heir
%     \dramitem[]       [Prince]  {Zacrias}{\human}{\male}, 
%       Morgan's younger son
%     \dramitem[Iselle] [Princess]{\Iselle}{\human}{\female}, 
%       Morgan's daughter  
%   \end{subdramatispersonae}
  \dramitem[Andros][Andros]
    {\Pater{} Andros}{\scatha}{\male}, 
    \maybehr{Telcra}{\Telcra} \cleric{} 
  \dramitem[Jirad Tantor]{\Jirad\ Tantor}{\human}{\male}, 
    \maybehr{Telcra}{\Telcra} \templar, \ishrah mage 
  \begin{subdramatispersonae}
    \dramitem[Mycah Tantor]{\Mycah{} Tantor}{\human}{\male}, 
      his son and apprentice
  \end{subdramatispersonae}
  \dramitem[Takestsha]{\Takestsha}{\human}{\female},
    \maybehr{Rethyax}{\rethyax}, \ishrah mage
  \dramitem[Orla of Fanshire]{\Orla{} of Fanshire}{\human}{\male}, \ishrah mage
%   \begin{subdramatispersonae}
%     \dramitem{Rosen Jaegwin}{\human}{\female}, his apprentice
%   \end{subdramatispersonae}
  \dramitem[Garog son of Otonn][Captain]{Garog son of Otonn}{\scatha}{\male}, soldier
  \dramitem[Enthon][\Frater]{Enthon}{\scatha}{\female}, 
    \maybehr{Telcra}{\Telcra} \cleric{} 
  \dramitem{Murein}{\human}{\female}, wise woman in \maybehs{Gedrock}
\end{dramatispersonae}




\section{Vaimons}
\begin{dramatispersonae}
  \dramitem[Carzain]{\CarzainDeracilleShireyo}
    {\human}{\male}, 
    rogue 
    \begin{subdramatispersonae}
      \dramitem{Arrow}{\relc}{\male}
    \end{subdramatispersonae}
%   \dramitem[Nishain Shireyo]{Nishain \Shireyo}{\human}{\male}, 
%     Carzain's father, \maybehs{Geican} Vaimon
%   \dramitem[Roanne Deracille]{\Roanne\ \Deracille}
%     {\human}{\female}, 
%     Nishain's wife, previously \maybehs{Redcor} Vaimon
%   \dramitem[Zacophine Vincerre][\Mater]
%     {\Zacophine{} \Vincerre}{\human}{\female}, 
%     Redcor, emissary to the court in \Malcur
%   \dramitem[Clarice Camilienne][\Soror]
%     {\Clarice{} \Camilienne}{\human}{\female}, 
%     Redcor, assistant to \Vincerre{}
  \dramitem[Chyrie Esmerel][\Matron]
    {\Chyrie\ \Esmerel}
    {\human}{\female}, 
    Redcor
%   \dramitem[Racel Galisetti][\Soror]
%     {\Racel Galisetti}
%     {\human}{\female}, Redcor, from \maybehr{Redglen}{\Redglen}
%   \dramitem[France Perival]
%     {\France\ \Perival}{\human}{\male},
%     Redcor \maybehr{Gandierre}{\gandierre}
%   \dramitem[Isacc Chiran]
%     {\Isacc\ \Chiran}{\human}{\male},
%     Redcor \maybehr{Gandierre}{\gandierre}
%   \dramitem{Soror Iselle}{\human}{\female}, Esmerel's assistant.
  \dramdead[Sylvie Dereine][\Sylvie\ \Dereine]{\Soror{} \Sylvie\ \Dereine}
    {\human}{\female}, Redcor historian
  \dramdead{Silqua Vaimon}{\human}{\female}, the first \maybehs{Vaimon}
  \dramdead{Cordos Vaimon}{\human}{\male}, 
    the first \maybehr{Vaimon Caliphate}{\VaimonCaliph}
%   \dramdead[Arcan Delain]{Arcan \Delain}{\human}{\male}, founder of \ClanDelaen
%   \dramdead[Lestor Delain]{Lestor \Delain}{\human}{\male}
%   \dramdead{Grith Ecallivan}{\human}{\male}
  \dramdead[Vizicar]{\VizicarDurasRespina}{\human}{\male}, Delain, once \caliph
\end{dramatispersonae}



\section{Others}
\begin{dramatispersonae}
  \dramdead{Catrian}{\human}{\female}, a weaver's wife in Bendaire
  \dramitem[Dorian]{Dorian}{\scatha}{\male}, in Bendaire
  \dramitem{Gaston}{\human}{\male}, a citizen of Bendaire
  \dramitem{Grum}{\human/\nephil half-breed}{\male}, bandit leader
  \dramitem{Rogg}{\human}{\male}, bandit 
  \dramitem{Ivar}{\human}{\male}, bandit 
  \dramitem{Gawl}{\human}{\male}, bandit 
  \dramitem{Faeni}{\human}{\female}, bandit 
  \dramitem[Shiaraid]{\Shiaraid}{\malach}{\female}, a dormant \maybehr{Vertex}{\vertex}
%   \dramitem{An angel}{?}{?}
%   \dramitem[Nasshikerr]{\Nasshikerr}{\Taortha}{\male}, \Ortaican god of shadows and stealth 
\end{dramatispersonae}








\end{comment}









\subsection[Malcur]{\Malcur}
Much of the book is set in or revolves around the city of \hr{Malcur}{\Malcur}, the capital city of \hs{Pelidor} and a powerful \hr{Nexus}{\nexus}.





\subsubsection{Sentinel plan}
\target{Secherdamon wants Nithdornazsh}
\target{Ghost Tower ploy}
The \hs{Sentinels} have a plan: 
\hr{Secherdamon}{\Secherdamon} wants to resurrect the \draconic{} fortress of \hr{Nith'dornazsh}{\Nithdornazsh}, and he wants to do it in \Malcur. 

The overall plan is this:

First they go for the \hs{Ghost Tower} near \hr{Forclin}{\Forclin}, besiege it and send in their Rissitic agents. 
The Cabalists immediately smell that something nasty is brewing, so they send in agents to stop them. 
They even send \banes. 

It comes to a major confrontation between Rissitic and Cabal agents. 
\hr{Nzessuacrith}{\Nzessuacrith}, disguised as \hr{Takestsha}{\Takestsha}, is forced to change to \draconic{} form to fend of the \bane{} attack. 

The Cabalists immediately detect the \dragon{} and send loads of reinforcements to the Ghost Tower, bent on fighting off the Rissitics (Sentinels?) at all costs. 

%But this was all part of the Sentinels' plan. Having drained Pelidor of Cabal resources\dash every Cabalist that matters off to defend the Ghost Tower\dash Sentinel agents in \Malcur now proceed to the cemetery. 
\target{Rissitics start Haskelek plan}
Now, \Nzessuacrith{} genuinely does want the Ghost Tower, because it contains one of the parts of the \hr{Haskelek}{\Haskelek}. But her whole attack is a diversion. She works for, and is unwittingly being manipulated by \Secherdamon, who has his own plans. Having drained Pelidor of Cabal resources\dash every Cabalist that matters off to defend the Ghost Tower\dash his agents in \Malcur now commence with their arcane ritual, to achieve the resurrection of the fortress of \hr{Nith'dornazsh}{\Nithdornazsh}. 

\target{Nithdornazsh is a useful gateway}
\hr{Dark ancient cities}{Ancient immortal cities} reached out into the Beyond. 
They were built in a time before the Shroud, when the barriers between the Realms were much more permeable. 
Their streets and corridors and towers were built so they criscrossed the Realms. 
This was why \Nithdornazsh was so useful as a gateway between \Machai and \Azmith. 

The Sentinels plan to make \Malcur ready for the great change and the Resurrection by gradually seizing control of the Shroud over the city and subtly changing people's beliefs. 
When they have enough Shroud power, they can mind control the people and channel their spells through them. 
The people of \Malcur themselves will facilitate the Resurrection. 

\target{The Change of Malcur}
The ordinary thugs do not know about the Resurrection. 
They only know about some great upcoming event called \quo{the Change}. 





\subsubsection{\ps{\Secherdamon} role}
\Secherdamon{} should be portrayed as a distant, mystic dark lord. 

Keep his name secret for a long time and have people refer to him only by various ominous titles. 

Compare him to Lucifer from \FLuneNoire. 






\subsubsection{The summoning ritual}
%There is a \vertex{} in the Shroud somewhere in \Malcur. The Sentinels want to draw a thread through this \vertex{} to form a connection to \Machai{} and pull \Nithdornazsh{} into the world. 
To this end, they need a magical ritual. \ps{\Secherdamon} servitor \hr{Psyrex}{\Psyrex} is responsible for this. He pulls the strings of the \Malcuric{} Sentinels and the criminal underworld in order to set up the ritual. 

The plan utilizes the \hr{Occult geometry}{occult geometry} that was used in the building of \Malcur.

The ritual to pull \Nithdornazsh{} from \hr{Machai}{\Machai} into \Miith{} requires \quo{beacons} to be set up at strategic locations. 
Each of these beacons must be secured and prepared for the ritual, including humanoids captured and ready to be sacrificed. 
The Sentinel-led mafia, masterminded by \Psyrex, can set up the beacons, but they need help diverting the town guards and the church. 
To do this, \Psyrex{} recruits \hr{Tiroco}{\rinyuth[\Tiroco]}. 





\subsubsection{\Secherdamon versus \noggyaleth}
I am not sure if \hr{Secherdamon}{\Secherdamon} is aware of the \noggyaleth. 
If not, then \hr{Ishnaruchaefir tells Secherdamon of the Ghobaleth}{\Ishnaruchaefir{} tells him at the end} and thus forces \Secherdamon{} to \cooperate{} with him.

If \Secherdamon{} does know, then he must be planning for it. 
He knows that the \noggyaleth{} can fuck up his resurrection ritual. 
They can even use their Shroud-weaving power to take control of the ritual and use it for Cabal purposes\dash but only if \Teshrial, their master, or \Achsah, his servant, is there to guide and control the mindless abominations. 
Thus the \hs{Ghost Tower ploy} to lure \Achsah{} out of \Malcur.
Then \LocarPsyrex{} and his Cabalists ought to be able to handle whatever \Teshrial{} can whip up. 

But the \noggyaleth{} are more strongly present than \Secherdamon{} believes. 
They have taken root and entrenched themselves much deeper than he thinks. 
So they can disrupt the spell, with catastrophic consequences for everyone. 
This fucks up \ps{\Secherdamon} plan. 

But \Ishnaruchaefir{} knows this, because only he is crazy and reckless enough to seek out the \noggyaleth{} and fight them in \melee{} combat, which \Secherdamon{} or \LocarPsyrex{} would never dream of doing. 
\Ishnaruchaefir{} finally \hr{Ishnaruchaefir tells Secherdamon of the Ghobaleth}{warns \Secherdamon{} about the problem}. 

Then \Ishnaruchaefir{} fights the \noggyaleth{} and holds them off. 
They are not all slain, but held off and weakened enough for \hr{Psyrex-tachi invoke the ritual}{\Psyrex-tachi to complete the ritual}. The sorcerers now have enough control of the local Shroud to be able to trap the \noggyaleth, and so turn the worms' Shroud-weaving power against them and their masters and use it to \hr{Nith'dornazsh rises}{resurrect \Nithdornazsh}. 

In this interpretation, the \noggyaleth{} are a vital part of the Sentinels' plan as well.





\subsubsection{\QuilJaaran in \Malcur}
\target{QJ in Malcur}
Maybe there are \quiljaaran in \Malcur, working on the side of \Secherdamon's Sentinels to further their dark plan.
They are the leaders of the Sentinel mages and the ones overseeing the dark ritual. 

Moro and Rian each see a glimpse of a \quiljaaran.
He sees through the Shroud for a moment, and its disguise slips. 
It used to just be a \scatha, but suddenly he sees it as a serpent thing. 
It is horrible to behold. 
He only sees it for a moment.
Then it slinks back into its tunnel/cellar or behind a door or whatever. 

\target{Moro and Rian rationalize QJ}
Both Rian and Moro block it out and try to deny it and rationalize it away.
They did not see a snake.
It was just a regular \scatha that looked weird in the light. 
Nothing more. 

But when they \hr{Moro and Rian realize QJ exist}{meet and compare stories}, they realize the snakes are real.





\subsubsection[Ishnaruchaefir's stealth]{\ps{\Ishnaruchaefir} stealth}
\target{Exile intersecting with Pyre}
\Ishnaruchaefir{} is stealthy. 
He helps \Secherdamon-tachi, but not too directly. 
He does not combine his powers with theirs in a straightforward manner. 
If he did that, his alliance would be astrologically detectable; astrologers would be able to see in the sky that the \hs{Exile} was intersecting with the \hs{Pyre}. 
He doesn't want that. 
So he merely helps them out without directly aiding them. 

Later \hr{Azraid muses on Exile and Pyre}{\Azraid{} will muse on this}. 





\subsubsection{Twist ending: Evil wins}
At the end, evil wins. 
\Malcur falls and \Nithdornazsh{} rises. 

\target{stop the evil}
This must be a surprise for the reader. 
Remember to have tons of references to how the heroes want to \quo{stop the evil}, so the reader thinks they will succeed. 










\subsection[Cabal plan for Malcur]{Cabal plan for \Malcur}
\target{Cabal plan for Malcur}
\target{Teshrial's creatures}
\target{Teshrial's monsters}
\target{Malcur gambit}
\target{Malcur venture}
The \hs{Cabal} have their own plans for \Malcur. 
\hr{Teshrial}{\Teshrial}, the leader of the Cabal in the Pelidor region, has brought several of the dreaded \hr{Ghobal}{\noggyaleth} to \Malcur. Perhaps these \noggyaleth{} have been there for thousands of years, or have gradually dug their way to \Malcur from Erebos over the course of thousands of years. 

Anyway, since \Malcur is a powerful \nexus, \Teshrial{} intends to use the \noggyaleth{} to open a portal to \hr{Erebos}{\Erebos} through \Malcur. 

The worms are hiding beneath the city. 
If the Sentinels try anything major, he is confident that the \noggyaleth{} can handle it. 

The purpose of the Cabal's \Malcur venture was to bore a hole from \Nyx and into the deeps of the planet \Miith.
They would build a conduct to the life-giving Heart and provide life and fertility to their race, which had long been dwindling.
It was a great and glorious undertaking.

The \resphain involved have very high expectations of this gambit and hope it will determine the future fate of \CiriathSepher, if not all \resphain. 
The ones not part of the gambit are more \skeptical. 
\Azraid has hopes for the venture, but remains \skeptical and aloof.
Many did not believe it would succeed, but \Teshrial and his associates were enthusiastic.
\Teshrial looked very much forward to it.
When the great bridge was complete, he would take \Firaxel to it, and they would have wonderful sex, and she would conceive, and he would be a father and a hero.

\target{Cabal stations near Malcur}
The \resphain working in \Malcur have some viewing stations and stuff set up in a Realm adjacent to \Azmith. 
Later \hr{Ishnaruchaefir attacks viewing station}{\Ishnaruchaefir attacks one such station}. 

There were some \noggyaleth under \Malcur.
They were part of the Cabal's plan.
They would drill the hole through the dimensions and pave the way for the bridge.
\Urizeth was there.
She was the party occultist and charged with dealing with the \noggyaleth.
There was one of several \noggyaleth.
It did not quite make sense to try to distinguish between individual \noggyaleth, for they would merge and split apart when they willed\dash{}or when their masters told them to.
\Teshrial feared the \noggyaleth and did not understand them.
He left it up to \Urizeth to manage them.
It is \Urizeth who performs the spells to direct the \noggyaleth, because \Teshrial is too scared to learn them. 

Read about how the \noggyaleth \hr{Noggyal corruption}{corrupt the planet}. 

\Teshrial \hr{Teshrial fears Noggyaleth}{is afraid of the \noggyaleth}. 

The \banes had a darker plan for \Malcur.
It was really a part of their master plan for opening the way from \Erebos to the Heart of \Miith and the \noggyal mother-mass.
The \noggyaleth played a dark, terrible role that the \resphain did not suspect.
The \resphain just used the monsters, stayed away from them and refrained from asking questions.
They feared both \noggyaleth and \banes and did not want to know any more than they already did.

The time when \Secherdamon plans to resurrect \Nithdornazsh coincides approximately with \Ishnaruchaefir's Nadir.
This is not by chance. 
In this period, the Shroud is thin, so they have a better chance of succeeding, breaching the barriers between the worlds and bringing their citadel to \Azmith. 
The Cabal's \Malcur venture is also mouthing out into some conclusion at this time. 
Or was supposed to. 

\hr{Urizeth is not a Cabalist}{\Urizeth was not a Cabalist at all}. 
She was hired for the \Malcur venture as an external consultant because of her great occult expertise. 
\Ganethed was the local occultist, but he was not good enough to do it all alone. 
He was a kinsman of \Urizeth, so he brought her in. 
\Achsah was also assigned to the project because of her occult experience, and because she was a High Telepath. 





\subsubsection{\Ishnaruchaefir is a menace}
Perhaps \Ishnaruchaefir needs not attack and destroy stuff in order to be a threat. 
I just need to clarify that if he is not stopped soon (chased away or preferably killed), he will wreck everything they have worked for in \Malcur.
When he is at his full strength, he could attack in force and drive the Cabal out of \Malcur entirely.
It is known that he takes an interest in \Malcur, so he likely has long-term evil plans there.
He gave hints of that in WSB. (Make him give hints!)
That must not be allowed to happen.
Furthermore, even now that he is weak, he might be up to something.
If \Urizeth's conclusions are correct, then these Nadirs happen to him regularly, and if so, \Ishnaruchaefir must have learned long ago to live with them and still get stuff done.
One must not assume that he is harmless in his Nadir.
(Maybe it is \Azraid who speculates the above to \Teshrial.)










\subsection[Tantor's journal and Eresh-Kal]{Tantor's journal and \EreshKal}
\target{Tantor's journal}
The Sentinels, who are secretly supporting Runger, have planted a diary (written by Rungeran mage \Jirad{} Tantor) among the possessions of \Ambrose{} \Onatol, a Pelidorian \ishrah{} mage. 

%Tantor relates how he, together with a Rungeran expedition, finds a forgotten temple that appears to be many thousands of years old, built with odd shapes that twist the eye and seem to defy geometry and reason, and adorned with carvings of horrific monsters. 
Tantor relates how he is sent on a secret expedition, led by \Takestsha, King Morgan's new advisor. \Takestsha{} is looking for the lost temple of \Rungertemple, in which is supposedly hidden a vast wealth of mystic knowledge. 

After a harrowing trek through the \Wylde{} they find the temple. It appears to be many thousands of years old, built with odd shapes that twist the eye and seem to defy geometry and reason, and adorned with carvings of horrific monsters. 

In this tower they find a wealth of mystic knowledge: Books, scrolls and artifacts. All in all a tremendous source of magic. They bring it back to the king. 

Not only do they have to hide this knowledge from the Iquinian church, but some mysterious people also appear and try to steal the magic and kill the finders. They use magic, and from their technique and behaviour Charcoal estimates that they are very likely Sentinels. But they are defeated or out\manoeuvred and the magic is brought to the king. Tantor has his reservations, but he follows the leader, a very strong-willed mage and a loyal Rungeran (perhaps a relative of the king). 

Tantor fears that Morgan Runger wants to use this terrible magic to conquer. He warns \Anatoli{}. 

It turns out that the discovery occured a year or a few years ago. Tantor and the other participants were sworn to silence, and he only recently gathered up the courage to write this. 






\subsubsection{The truth} 
The whole story about how Morgan's people discover a cache of magical knowledge is a hoax. It's meant to cover up the fact that the Rungerans are armed with Rissitic/Sentinel-provided magic. The Sentinel-looking characters that try to stop the Rungerans are completely fictional. They are there for the specific purpose to convince a Cabalist reader that the whole thing is \emph{not} a Sentinel plot (which it is). 

%The tower is real, however. Its location and nature are known to both the Sentinels and the Cabal. But it's guarded by nasty things, and the Rungerans were never there. (Or were they?) 
\Rungertemple{} is real, however, and so is the expedition. \Takestsha-tachi really were there and picked up some occult tablets and scrolls, but these are not as valuable as they are eld to believe. The actual mystic knowledge was provided by \Takestsha, who is really the \dragon{} \Nzessuacrith. 

Giles Tantor is a real person and \emph{did} correspond with \Anatoli{}. His account is real, although he was being mind-fucked to some extent by \Nzessuacrith. 
%But this story is not his work. The real Tantor was killed by the Sentinels, and his \quo{journal} was
A few fragments are concocted by the Sentinels and inserted afterwards. Tantor was later killed by the Sentinels, and his edited journal was sent to \Anatoli{} in \Malcur, ostensibly to warn him, but in reality it was meant to be leaked to Charcoal. 
Then \Anatoli{} was killed so he wouldn't spill the wrong beans. 

The \EreshKali{} magic is Rissitic magic, but not standard issue. 
It is a new experimental kind. 
\Secherdamon{} intends to kill two birds with one stone by field-testing his new spells and winning the Runger war at the same time. 
Also, the Rungeran mages are much more expendable than precious Rissitic ones. 
\Secherdamon{} loves his own people. 









\subsection{Rissitic invasion}
\target{Rissitics and Runger}
Shortly before the Rungeran attack against Pelidor, the Rissitic Empire launched invasion forces in southern \Galessan. 
They are especially targetting \hs{Sumian}. 
Or maybe \Scyrum. 

This is part of a bigger long-term strategy. 
\Secherdamon{} wants a bigger foothold in the world. 
And it also serves as a diversion, a means to distract the world from the Runger-Pelidor war. 
With Rissitics attacking from the south, they will draw all the eyes to them, and the world will not pay much attention to a little border dispute between two unseeming little kingdoms. 
The Redcor Church and the Imetrium will be far too busy to respond to Pelidor's call for aid. 

But in reality the Runger-Pelidor war is more important than the Rissitic invasion, because it sets the stage for the resurrection of \Nithdornazsh. 

Remember to make it clear that Durcac = Rissitics! 
Write this multiple times! 

And remember to advertise \hr{Narkiza}{\Narkiza}. 
Namedrop him and tell everyone how skilled and feared and powerful and cool he is. 





\subsubsection{Imagery}
Some people do not understand the fuss because they have heard that Durcac is relatively small and its warriors few in number.
The wiser people know that it is the Rissitics' superior technology and powerful sorcery that make them dangerous.

The Pelidorian court should be discussing the \hr{Rissitics and Runger}{Rissitic invasion of \Scyrum/Sumian} and what it means for all \Galessan. 
Make it clear that the Redcor Church is supporting the \Scyrics/Sumianese, and the Imetrium may throw its weight in as well. 
Everyone is very disturbed by the Rissitic invasion. 

The Rissitics are feared. 
Have references to the dreaded, mighty warlord \Narkiza, who is more than a \scatha. 
An immortal warrior mage. 
Almost a demigod. 
An \hr{Ashenoch}{\Ashenoch}. 

Be sure to \hr{Rissitic reputation}{demonize the Rissitics}. 
Rumour turns them into legions of demons from Hell, inhuman wielders of dark sorcery and evil.
The fact that \Narkiza is known to be an \Ashenoch, and that the Rissitics have always relied on magic and monsters, only makes their image worse.
They are seen as the marauding Legions of Chaos.
\Tiroco hears reports of towns and cities in the south being overrun by screaming, wicked Rissitics.

\lyricsbalsagoth{
  Behold, the Armies of War Descend Screaming From the Heavens
}{
  Behold, the armies of war descend screaming from the heavens!
}

Among other things, the Rissitics have an \quo{Order of the Hippopotamus}. 
The hippopotamus is the largest and most dangerous mammal in the world, as far as some people know.
\Tiroco has not seen a hippopotamus, of course, but she has heard of them.
She sees it as a loathsome symbol of bestial brutality. 
















\section{The Dreaming Predator}
\subsection{The \SecondShrouding}
Have a flashback to somewhere around the time of the \hr{Shrouding}{\SecondShrouding}. 
Probably seen through the eyes of \Nzessuacrith. 

We see her flying over a vast battlefield, strewn with the corpses of mortals, \dragons{}, \cuezcans, \banes{} and \resphain. 

She meets \Secherdamon{} and talks to him. 
He spews his hate for \Ishnaruchaefir{}, since this is shortly after the tragic \hr{Fall of Nexagglachel}{fall of \Nexagglachel}. 

She also talks to others, who comment on the conflict between \Ishnaruchaefir{} and \Secherdamon. 

This is before \hr{Secherdamon's rise to power}{\ps{\Secherdamon} rise to power}, but \Nzessuacrith{} or someone else\dash probably an \ophidian\dash predicts that the youngest brother is bound for greatness\dash a terrible kind of greatness. 
You can see it in his eyes, the bitterness, the avarice and the determination. 





\subsubsection{\Nzessuacrith{} flying over \Machai}
Maybe have a scene where \Nzessuacrith{} flies over the plains of \Machai{}. 

\lyricslimbonicart{Twilight Omen}{
  My soul had black wings and triumphant I did fly.\\
  I rode the storms and the midnight sky.\\
  I saw the thousand lights from cities underneath me,\\
  as I ventured deeper into the night,\\
  to that sacred place beyond the twilight zone.
}





\subsubsection{The fall of the Heart}
They talk about how most of the immortals are now dead, and of how the Heart was been weakened and is no longer strong now to bring them back to their glory days.
It cannot support such a population in its current state, with both factions pulling and tearing at the Heart. 









\subsection{Silenced}
Catrian, a Sentinel woman living somewhere near \Redce, sees a \bane{}. She is killed by the Cabal. 

During her flight, she submerges into \Nyx. She sees the \hr{Black stars of Nyx}{black stars}. 

Despite being a Sentinel, Catrian has never seen a \dragon. She has, however, seen some of the \draconian{} people, the \rachyth. 







\subsection{Wanderer in Darkness}
Have a scene where the Cabalists in the Pelidor region discuss their plans, and fear who might interfere. The company includes \Teshrial, \Achsah{} and Charcoal, who has been summoned. A few more unnamed Cabalists (perhaps named), a few lesser \banes{}. Maybe they are in communion with \Azraid{}, although he is not present in person. 



 






\subsection{What Slithers Beneath}
\target{Ishnaruchaefir attacks Teshrial's creature}
\target{Ishna fights Teshrial's monster}
\Ishnaruchaefir comes to \Malcur. 
\Teshrial comes to fight him off. 

\Criseis does not know what \Ishnaruchaefir has in mind. 
She knows what he has told her to do (i.e., reconnoitre for any supernatural presences infesting \Malcur), but no more.
She imagines he intends to draw out whatever is in the ground and destroy it.

\Ishnaruchaefir is keeping the \resphain busy.
\Criseis has taken shelter and is cowering there.
Or so it seems. 
In reality, \Criseis digs deep with her aethereal senses.
She reaches down deep below \Malcur, in the planes close to \Nyx, and she detects the \noggyaleth.

This was his plan all along. 
\Criseis is his secret weapon. 
She is \hr{Criseis's senses}{super-sensitive} and can detect things that are supposed to be hidden from everyone.
But few people suspect, because she is just a lowly \scatha. 

\target{Ishnaruchaefir fakes ignorance about Ghobaleth}
\Ishnaruchaefir{} says something like: 
\ta{What a shame that your scheme is now ruined.}
This is to fool \Teshrial{} into thinking \Ishnaruchaefir{} doesn't suspect the full extent of the \noggyal{} plan. 
\hr{Teshrial does not know that Ishnaruchaefir knows}{\Teshrial{} takes the bait}. 

\Teshrial boasts:
\begin{prose}
  \Teshrial:
  \ta{I have killed \dragons before, you know.}
  
  \Ishnaruchaefir: 
  \ta{Have you now?}
  
  \Teshrial:
  \ta{In the battle of $X$, $Y$ years ago.}
  
  \Ishnaruchaefir:
  \ta{Aye, I heard about that. So that was you.
    Hm. 
    So you killed \Zessuruch. 
    One of our youngest. 
    Huh. 
    I was not surprised. 
    \Zessuruch was prone to overconfidence.
    I hear she has grown wiser since her death.
    So you did our people a favour. 
    Well done, \dragon-slayer.}
\end{prose}

\hr{Resphain speak poor Draconic}{Like most \resphain}, \Teshrial \hr{Teshrial speaks poor Draconic}{spoke \Draconic only poorly}. 
So when \Ishnaruchaefir speaks \Draconic, \Teshrial has to make an effort to understand it. 
\Criseis can tell how \Teshrial is listening intently, all the while trying to hide it.
He does not want to show his own shortcomings. 

\Teshrial{} should not be too overconfident. 
He knows he is in over his head and does not really expect to win. 

Eventually, \Ishnaruchaefir is repelled. 
But before he leaves, he sends a challenge to \Teshrial.
He is impressed by \Teshrial's bravery.
He accepts \Teshrial's challenge and is willing to face him again.
He invites \Teshrial to find a time and place for a new duel.

\ta{\Resphain! Carry this message to your fallen comrade, \Teshrial.
  When he returneth to life, tell him this:
  Thy display of bravery hath left me impressed.
  If thou darest face me once more, then send unto me a rematch challenge, and I will humour thy request.}

\Ishnaruchaefir believes he can use \Teshrial as a \trope{XanatosSucker}{Xanatos Sucker}.
With some luck, he can manipulate everyone so that \Teshrial duelling him at the same time that \Secherdamon will cause \Malcur to come crashing down.
The whole \trope{XanatosGambit}{Xanatos Gambit} is not fully crystallized in \Ishnaruchaefir's head yet, but he has a good idea.

Afterwards, the \resphain are confident they have kept \Ishnaruchaefir from learning what he should not learn and fucking up their plans. 
When \Teshrial died, everything had been cleaned up.
They are sure he did not learn anything.
But they were not keeping an eye on \Criseis.
They do not understand quite how keen her senses are.
Under their noses she has snooped around and gained a good overview of what the Cabal are doing. 

After the battle, the \noggyal presence has retracted. 
\Criseis can still feel it, but only faintly, and only because she knows it is there.
She asks her master if he intends to pursue.

\Ishnaruchaefir:
\ta{Nay. Let them hide.}

Rian is scarred and horrified to see the evil sorcerer slay the shining god. 
\Criseis Shrouds him and makes him forget the details, but some measure of religius/existential dread remains with him. 
Remember that in all his later chapters.
Maybe even ask on a forum how to express this. 

When \Criseis{} insists they let Rian live, \Ishnaruchaefir{} goes:
\ta{Sigh. [Smile.] 
  I fear one day your sentimentality will be the death of us, \Criseis.
  But very well.
  Now come.}

At the end of the chapter, have a super-short scene where \Ishnaruchaefir tells \Criseis to describe to him what she has learned.
She begins to tell him something very interesting.
A surprisign and bold plan from the \resphain.
She talks offscreen, and he listens with great interest and a sardonic smile.

Afterwards, the \resphain are confident they have kept \Ishnaruchaefir from learning what he should not learn and fucking up their plans. 
When \Teshrial died, everything had been cleaned up.
They are sure he did not learn anything.
But they were not keeping an eye on \Criseis.
They do not understand quite how keen her senses are.
Under their noses she has snooped around and gained a good overview of what the Cabal are doing. 

Both \Ishnaruchaefir and \Criseis had to be here.
He could not just send \Criseis alone. 
He had tried that.
She has been in \Malcur before and has only managed to pick up hints.
She never dared delve too deep, for fear of detection. 
\Ishnaruchaefir had to be there for two reasons:
\begin{enumerate}
  \item 
    To scare the Cabalists and stir them up, make them do something drastic that could be detected. 
    The \noggyaleth are easier to detect when they are being moved than when they are just quietly burrowing around.
  \item 
    To take all the attention.
    If \Criseis just delves into the deep on her own, she would be detected. 
    If there is a big-ass fight going on in the dead garden, no one will notice \Criseis feeling around. 
\end{enumerate}
%Have a scene earlier on where \Ishnaruchaefir{} fights one of them and only barely prevails. Maybe this is why \Ishnaruchaefir{} knows about them. 





\subsubsection{Rian is traumatized}
Rian remembers the horrible black stars that hung in the heavens. 

\lyricsbs{Robert W. Chambers}{The Repairer of Reputations}{
  \ta{I\prikker wept and laughed and trembled with a horror which at times assails me yet. 
  
  This is the thing that troubles me, for I cannot forget Carcosa where black stars hang in the heavens; where the shadows of men's thoughts lengthen in the afternoon, when the twin suns sink into the Lake of Hali; and my mind will bear forever the memory of the Pallid Mask.}
}

He was also scared shitless from seeing \Ishnaruchaefir{} fight. He has lost sanity points (to use the terminology of the RPG \cite{RPG:CallofCthulhu}). 





\subsubsection{\Psyrex{} and \Secherdamon{} see it}
\Psyrex{} and \Secherdamon{} notice the fight and comment it. 
They remark on the fact that it is a blatant breach of the \charade. 















\section{The Runger War}
\target{Runger war}
\subsection{Daggers and \Daemons}
We go to \Malcur (capital city of Pelidor), where \rayuth[\Icor] is assassinated by the Sentinels. 
But it has to be well-concealed, so it's not obvious that it's the Sentinels that did it. 
It must look like it's just the Rungerans. 

At a ducal ball, an assassin sneaks up to the \rayuth during the dance and knifes him. She has the backing of Sentinel agents. It's their magic that lets her get close to the \rayuth. But she is brainwashed and thinks it's all thanks to her own cleverness and stealth and the aid of a few contacts. 

\Icor gets killed by a servant of \Onatol's who has been mind-controlled. 
\Onatol himself is likewise under mind control. 
The spell does not control \Onatol's entire mind, it just makes him slightly crazy and aggressive, so that he fights and gets himself killed when the soldiers come to arrest him. 
Being a mage, \Onatol has a strong mind that is hard to control. 
The servant's mind is easy to control, so the Sentinels have full control over him. 

After \Onatol dies, have a scene where \LocarPsyrex contacts his local agent. 
The agent tells \Psyrex that everything went well. 
The agent then slinks away.
The agent is one of the sorcerers who will later be responsible for raising \Nithdornazsh.

At the end of the book, make sure to reveal that \Onatol was an innocent guy, framed as a part of a circuitous \trope{XanatosRoulette}{Xanatos Roulette} with the purpose of leaking Tantor's diary to Charcoal and making him think he had discovered a vital clue. 

\textbf{Alternately}, get rid of the journal entirely.
The Tantor chapters are just told in flashback, or are relegated to a prologue part. 
In that case, Curwen does not learn of the magic and \Takestsha until Carzain comes to \Forclin and tells him about it.

\Icor's widow, \rinyuth[\Tiroco], suddenly becomes regent of Pelidor. 
We follow her and the general \hs{Sethgal}, who must lead the nation against the Rungeran invasion. 









\subsection{\MoroCobrel}
\target{Moro feels Ishnaruchaefir and Teshrial}
\MoroCobrel{} feels the battle between \Ishnaruchaefir{} and \Teshrial{} in the dead garden. 
(She does not feel the \noggyal. It is too deeply Shrouded and concealed.) 

A bit later she goes to the garden to investigate. 
She cannot find anything.
\ta{My spiritual senses are too dull.} 

She touches the ground and concentrates as much as she can. 
For a brief moment she feels the presence of some gigantic creatures, crawling around her or under her. 
But only for a brief moment, and very indistinctly. 

\tho{It might be my imagination.
  It would not be the first time my paranoia and traumata have had me see things. 
  
  If only I were more sane\prikker}








\subsection{War is Coming}
\target{Pelidor-Runger war}
The king of Runger plans to invade Pelidor after the assassination of \rayuth[\Icor]. 









\subsection{Beyond the Veil}
In \Malcur they hold a funeral for \Icor. \Tiroco{} wants his spirit laid to rest so she can concentrate on moving on. 

We follow \Icor{} from the point of his death. He gets stabbed, then bleeds to death. At first he can still speak, but he quickly grows numb and can only lie there. He can still hear \Tiroco{} and feel her touch, but even that fades. 

Camilienne comes to heal him. He feels the \Sephiroth{} enter his body. He's fought wars and been healed by the \Sephiroth{} before, and it's always felt nice, but this time it's different. Death lifts the Shroud a little bit and lets us see some of the things we don't want to see. And now the dying \Icor{} vaguely sees the \Sephiroth{} as what they truly are: Horrors created by the \banes{} to enslave \humanity{} and enforce the Lie. He fears and despairs. 

After a while, the \Sephiroth{} dissipate. Camilienne proclaims him dead. He finds that he still has a bit of feeling for the living world. He can still feel \Tiroco{} a little bit. Which is bad, because he feels her pain and despair. 

Then he starts hearing the \daemons{} of the void. There are the occasional \daemons{} who have learned \Miithian{} languages and torment the dead to pass time. He is plagued by some of them. 

The \daemons{} go away and he floats around for a while in complete solitude, unable to perceive anything. He fears he is losing his mind, not wanting to spend eternity as a mad ghost. 






\subsubsection{\Icor{} sees vision of soul prison}
In the afterlife, \Icor{} sees a vision of a prison of souls: Thousands, perhaps millions of souls, chained and enslaved, screaming and weeping, bound within a horrible cage. 

This is actually the \Sephiroth{} that he sees. 

He sees souls of the dead being sucked into this Hell, into eternal suffering and torment. Some of them beg for help. A few are able to see him. They reach out to him, imploring him to save them. But he can't. Not only has he no idea of how he could help them; he is also paralyzed by fear. When one of the dead moans its pain and stretches its ghastly fingers toward him, he can only recoil in disgust, and feel guilt at his own cowardice, weakness and cruelty towards the tortured souls. 

Maybe he hears some dead souls talking. They are halfway or wholly mad and babble nonsensically. 






\subsubsection{\Psyrex{} approaches \Icor}
Why does he see it? 
Well, he is dead, so he is drifting towards it, until he is rescued by \Psyrex. 

\Psyrex{} comes to him in the guise of a child-like Cherub. 
 
\ps{\Icor} vision of the Realms is a bit clearer than normal, because he already had cause to suspect the \Sephiroth{} due to the events surrounding his death. 
And maybe \Psyrex{} helps him, showing him visions to scare him. 
(Of course, \ps{\Psyrex}{} interpretation of what \Icor{} sees is far from reality, and \Psyrex{} knows it.)

\Psyrex{} warns him that what he sees is \Itzach, where the souls of sinners go, and that \Icor{} must help him if he is to be considered among the righteous and avoid being thrown into \Itzach{} himself. 
\Icor{} doesn't quite believe \ps{\Psyrex}{} claims of holiness, but he sees no better choice. 







\subsection{Needle}
Charcoal goes to \hs{Needle}. Since Charcoal is going off to war, he gives her instructions to take over the local Cabal in his absence. 

Charcoal is certain that something fishy is going on in \Malcur. He suspects Sentinel activity, mages preparing something of \quo{cataclysmic proportions}. He is also convinced that they know of the Cabalist presence in \Malcur, since he has had agents killed and spells broken. 

He already has men out searching for them and wants Needle to take over the investigation. If they discover anything, they should destroy them, if possible. Show them that the Cabal means business. 

Even striking without really knowing what they're up against might be a good idea, because the Cabal is usually always careful and not happy to take chances, so if they come in with full force, it might convince their enemies that they know more than they do, and thus frighten them into being more reluctant, buying the Cabalists valuable time. Plus, if their enemies are afraid, they get stressed, and when stressed they are more liable to make mistakes that give them away. 

When done giving instructions, he fucks her. 









\subsection{For War and For Glory}
\Tiroco{} and her court have received word that war is coming. 
\hs{Sethgal} is appointed Marshal and leads the armies into battle. 

\Tiroco{}, not a trained warrior, remains in \Malcur to rule. 
With her are \Vincerre{} (Camilienne being sent with the army), Norden and Bartholomy. 









\subsection{The Mystery of \EreshKal}
Charcoal reads Giles Tantor's journal. 

The whole thing might be written in some cipher code, but it's one Charcoal can easily decode. 







\subsection{Veils that Divide} 
\Icor{} comes back as a ghost. 
He finds his own funeral. 

He tries to communicate with \Tiroco, but fails.





\subsubsection{\Tiroco{} as regent}
\Tiroco{} and \Icor{} were married in an attempt to reconcile two branches of the Pelidor clan, the \Malcurians{} and the \Forcliners. 
Now that \Icor{} is dead, the clan fights over who should be the next \rayuth. 
They can't decide and vote while there's a war going on, so they postpone the decision till after the war. 
This means that \Tiroco{} is stuck as regent until then. 

This is part of \ps{\Psyrex}{} plan. 
\Tiroco{} is weak and easy for him to manipulate, especially when he has \Icor{} as leverage. 







\subsection{A Dark Angel's Gift}
Needle is sitting in her room, setting her hair, putting on makeup and making herself pretty. She is happy that Charcoal is gone. She hates Charcoal, and she loathes being his sex slave. 

Charcoal is cruel and evil and does all he can to make the sex painful to her. In the beginning he took her often, because she was beautiful. She caught on and started deliberately making herself ugly, so he would be turned on by her and thus rape her less often. He may or may not know this. Once in a while he beat her as punishment for being ugly, but he did fuck her less often. Needle accepts the trade: She would choose his whip over his dick any day of the week. 

But now Charcoal is gone and she can be beautiful again. Like any girl, she much prefers being sexy. 

She hasn't become pregnant with him, for which she is very grateful. It is the magic, she knows. All that spellcasting \quo{sucks all the power and life out} of his sperm, leaving it ashen and dead. His cum tastes bitter and dead, too. The cum of a real man, while it may not exactly taste good, at least feels alive, energetic. 





\subsubsection{\Achsah{} seeks out Needle}
\hr{Achsah}{\Achsah} seeks out Needle and gives her new instructions. 

When Needle flinches, \Achsah{} assures her: 
\ta{No need to fear, girl. I am not going to rape you.} 
She pauses, then adds: 
\ta{I have higher standards than that.}

She gives Needle a gift: She teaches her a spell that lets her summon \Achsah{}, to communicate or to send reinforcements. 







\subsection{The Gods of \EreshKal}
Charcoal reads more in Tantor's journal. Tantor-tachi reach the temple of \Rungertemple. Inside they find tablets of arcane scribblings. 

The tables show fragments of the cosmic paths that connect \Miith{} and \Machai; the paths that were \hr{Star-Maps of the Ancient Cosmographers}{mapped out by the founding \dragons millennia ago}. 









\subsection{Captured}
\target{Rian's missing girlfriend}
Introduce Rian. He is a \Malcurian boy of maybe 16-17, an orphan who used to be a thief. But at some point in his teens he helped a carpenter and his family. As thanks, the carpenter took him in as an apprentice, to help him get out of a life of crime and into a decent, \honourable career. 

Now, Rian has a girlfriend, namely Neina the baker's daughter. They are engaged to be married in a year or so, but Rian is a horny youth and can hardly wait. We meet them an evening when he is walking her home for some reason. They are kissing in a corner, and he moves in under her blouse to grope her breasts, but she stops him. He relents and kisses her goodnight, and they part ways. 

She walks to her house, but she is intercepted by a thug working for the Sentinels. He knocks her out and kidnaps her, intending to use her as a sacrifice for the \Nithdornazsh{} ritual. 







\subsection{\Forclin}
Several days have passed since the Pelidorian army set out from \Malcur. 

On their march north, the army passes through \Forclin. They camp there. 





\subsubsection{Curwen meets \Esmerel}
\target{Esmerel in Pelidor}
The Pelidorian army, led by \hs{Sethgal}, is marching out from \Malcur. 
Soon after, they are joined by Redcor \Matron{} \Esmerel. 

\Esmerel{} is in Pelidor not to search for Scions, but to oversee and monitor the war. 
She is there to provide healing and \quo{spiritual guidance}, which in practice means she is a spy and a lobbyist trying to make people owe her a debt of gratitude. 
She intends to help heal the wounded. 
And maybe intervene and give more direct help in combat if it should be required. 
The Redcor do not trust Morgan Runger. 
He has long been unwilling to heed (read: obey) his Redcor advisors. 
They suspect him of using black magic and being disloyal toward the Iquinian faith. 
They are unhappy about this. 
They feel Morgan might need to be taught a lesson about disrespecting the Redcor. 















\section{The Cancerous City}







\subsection{Trinity of Pestilences}
\target{Rian sees the Morbus}
\hs{Rian} is in the slums of \Malcur, searching for \hr{Rian's missing girlfriend}{his missing girlfriend}. 

The slums are filled with death, poverty, destitution, despair, hunger, hopelessness. 
The poor, degenerate people are almost living dead. 
They live as pitiful scavengers. 
Some of them even eat \human{} and \scathaese{} flesh. 
Some of them suffer from the \hr{Morbus}{\Morbus}. 

Compare to Calcutta in \cite{PoppyZBrite:CalcuttaLordofNerves}. 







\subsection{Where Angels Fear to Tread}
\target{Tiroco contacted in dreams}
After the funeral \Tiroco{} is visited by \ps{\Icor} ghost, first in her dreams. 
\Icor{} mourns and complains that he is trapped in limbo and unable to go into the Light. He begs \Tiroco{} to help him by doing something in the crypt.

% (Is there a closed crypt or only an open cemetery?)





\subsubsection{\ps{\Icor} point of view}
\ps{\Icor} mind is hazy from being a ghost. 
He tends to forget \ps{\Psyrex}{} instructions, and even the mere fact that he's a ghost.

He has a hard time getting through to \Tiroco{} due to the Shroud. 
He tries to appear in shadows, in mirrors, behind curtains and stuff\dash places where people's Shrouded perception of reality is weakened, where they are more open to the possibility of strange things happening. 
But if he is behind a curtain and talking to her, and she pulls the curtain aside, then he \quo{vanishes} and she will see right through him\dash the Shroud separates them. 

He is frustrated by what he perceives as \Tiroco{} being stupid. 
He is gradually going mad. 

\lyricsbs{Hate Eternal}{Two Demons}{
  I am diversity. So weary of the angst, 
  so weary of what I have become.\\
  Through this dread I will retain 
  this penance that still haunts me.
  
  I am complexity. Within these walls that hold me, \\
  my past and my present collide.\\
  In this state I must sustain
  this morphing that becomes me.
}

But he also wants to \hs{stop the evil} that is festering in \Malcur. 

It should be noted that \Icor{} sees the regular, Shrouded world and not the true world. 
This is due (in part) to a Shrouding spell cast on him by \LocarPsyrex. 





\subsubsection{\ps{\Tiroco} point of view}
In the beginning \Tiroco{} doesn't believe in it, thinking that it's just her sorrow and wishes that manifest in dreams. But the dreams become more insistent, and she begins to see \Icor{} even when awake (mostly at evening or otherwise in the dark, because the Shroud is more easily penetrable in darkness). 

She begins to believe that the visitations are real. She visits a priest of some sort, probably William Norden, to seek spriritual advice. Norden tells her that ghosts should not be listened to. For the most part, ghosts are not real, but illusions conjured by the \Qliphoth{} to trick people. And if the ghosts are real, they must be mad and twisted, and their pleas should not be heeded.

So \Tiroco{} tries to ignore the ghost, but \ps{\Icor} visitations become more frequent, and he begs and pleads, then blames her for betraying him. At last he guilts her into complying. 

The concrete things he wants her to do are quite complex and take a while to complete. She needs to consort with disagreeable people in the \Malcurian underworld and do furtive, suspicious things. At the same time, \Icor{} tells her that she must not reveal anything to the churches. He says they won't understand and will try to stop her. What she must do is suspicious and dark in nature, consorting with powers normally considered forbidden. But, \Icor{} assures her, it's the only way to free him. If the church's dogmas are to be observed, he will be trapped in limbo forever. Already, he says, he is losing strength, and soon he will be unable to contact her, doomed to float forever in pain and solitude in the cold void. 

So \Tiroco{} gets to work, contacting people and acquiring components necessary for \ps{\Icor} \quo{exorcism} ritual. 

But she also wants to \hs{stop the evil} that is festering in \Malcur. 









\subsection{The Terror of \EreshKal}
Carzain passes by a Pelidorian village. 
It has been destroyed by the Rungerans' evil magic (as a test of their spells). 
Carzain is horrified by the foul sorcery at work. 








\subsection{The \Qliphoth{} Lie Ever in Wait}
\Tiroco{} must move the guards around and divert the attention of the church and authorities. \Icor{} tells her that \quo{they would not understand}, that they, in ignorance, would sabotage their undertaking, dooming him. (Is it only \Icor{}, or are there more ghosts? I think he tells her that she is saving not only him, but many ghosts.) 

%Here's an idea: \Tiroco{} goes on a campaign against \quo{superstition}. She 
In order to do her undergound work, \Tiroco{} officially goes on a campaign against superstition. She approaches the church and convinces them to help her \quo{calm down} the population and suppress nasty rumours of ghosts, black magic and sinister crime. 
Fear is rampant in the country, after all, since the \rayuth has just been murdered and the king of Runger has declared war. So the peasants and citizens need to be reassured, need to forget their fears. 
And this is handy, because if stories of supernatural activity are suppressed and dismissed as superstition, it makes it much easier for \Tiroco{} to do her furtive work. 

And then we hear the story from the perspective of some normal citizen who sees horrible, unnatural things happening, while the church and authorities tell him to forget it, that it's all delusions. This is \hs{Rian}, the ex-thief with the missing girlfriend, Neina. 

This is good for two reasons: 
First, it's a clever move by Tiroco, which helps give her some personality. Until now she's been a very tame character with no real skills, so she needs this. 
Second, it gives me the opportunity to write some horror parts featuring some \quo{common} people who can be genuinely horrified (as opposed to the mages and heroes who are otherwise my main characters, who are less easily frightened). 

What she is doing is a cruel, brutal oppression of her people. Needless to say, she does not respect modern \human{} rights in \emph{any} form. Like the War on Terror in RL. But see it from her point-of-view, where she justifies her crimes. Her love for \Icor{} is weighed against her obligation to her country. 





\subsubsection{She wants to stop the evil}
\Tiroco{} hopes to stop the mystic evil that is festering in \Malcur. 
She hopes she is doing the right thing. 









\subsubsection{Needle notices \Tiroco}
Needle notices that \Tiroco{} is acting strange: 
Furtive, paranoid, jumpy, and keeping lots of secrets. 
Needle suspects that she is up to something. 






\subsubsection{Rian is threatened by a thug}
Rian snoops around in the dark corners of the city, trying to sniff out clues about his girl, Neina. 
He sees something suspicious, 
which might be some of the Sentinel agents, out doing their shady work. 

Then he is jumped by a thief. The thief turns out to be an old \quo{friend}, so for old times' sake he does not kill Rian. He merely slaps him up and tells him to go home and stop snooping around. 









\subsection{The Thirsty Nether}
On his snooping through the city, Rian overhears some thieves (in Sentinel employ) talking about their wicked plan. 
Possibly \Psyrex{}, or at least some of his top-level henchmen. 
He sneaks into a suspicious building. 
Here, he is witness to a terrible ritual of dark magic, involving humanoid sacrifice. 
He sees into \Machai, terrible vistas of \daemonic{} landscapes and stuff. 

He is discovered, and the mages send thugs after him. 
He runs for his life. 
He has been given a warning, so he knows that this time they will surely kill him if they can catch him. 
He nearly gets caught and killed, but \MoroCobrel{} arrives to pull him out. 

He talks to her. She warns him that he shouldn't be snooping around in places where he might get killed. He tells that that he must, for the sake of his beloved. 

Rian fears the supernatural evil that is taking place in \Malcur. 
He hopes someone will stop it. 





\subsubsection{Needle is approached by a spy}
Needle is approached by a spy of hers. He tells her that he and his fellows have discovered a lair of people who use chaos magic and seem to be up to something nasty. (This spy was at the same place that Rian was. Perhaps he actively sabotaged Rian's infiltration attempt, using him as a distraction, allowing himself to escape unseen. 

Needle begins to prepare a raid. 









\subsection{Dark Crypts of the Mind}





\subsubsection{\Tiroco{} talks to the crazy old woman}
\Tiroco, on a covert trip through the city, passes through the \hs{dead garden}, where is addressed by the \hs{crazy old woman}. The old woman is clingy and mad, desperately clutching \ps{\Tiroco} sleeve and babbling about worms, blood, death and the destruction of the city. 

\Tiroco{} herself is shaken, since she is already weak from the things she's witnessed, so her bodyguard has to shake off the old woman and drag a shocked \Tiroco{} away. But afterwards, \Tiroco{} finds herself seeing worms at every turn.

\lyricsbalsagoth{%
  In the Raven-Haunted Forests of Darkenhold, Where Shadows Reign and the Hues of Sunlight Never Dance
}{
  Can you not see the coils of the worm all about you?\\
  Can you not hear the writhing of the worm beneath you?\\
  Can you not scent the breath of the worm riding the wind?\\
  Can you not touch the skin of the worm in all that surrounds you?\\
  Can you not taste the ichors of the worm upon your tongue?\\
  Do dreams of the worm not haunt your slumber?
}

She tells stories of death, destruction and the end of the world.

\lyricslimbonicart{Legacy of Evil}{
  The night has a legacy of evil.\\
  Nocturnal dreamscapes.\\
  A wilderness of infinite disharmony\\
  in an isolated aspect eternally.\\
  A maze inside your brain\\
  leading into the insane\\
  abyss of fear, illusions and despair.
  
  Something ghastly is passing by.\\
  The full moon in the sky.\\
  See the coming hurricane of terror,\\
  ancient sadness and horror.\\
  A virulent syndrome of misanthropy,\\
  captured by the obscure mystery. 
}





\subsubsection{\Tiroco{} talks to Moro}
%What about \MoroCobrel? I don't think she has seen enough yet to truly act. But remember to have her in some scenes. 
Remember to have \Tiroco{} talk to \MoroCobrel{} a few times in the course of the book.

Moro has not yet seen enough to truly act. But remember to have her in some scenes. 

Moro hopes to \hr{stop the evil}{stop the mystic evil} that is festering in \Malcur. 
She is investigating, trying to find out what is going on so she can stop it. 








\subsection{The Bleeding Wood}





\subsubsection{Rian sees a raid}
\target{Rian sees a raid}
Rian sneaks back to a place near the one he was last time. 
He spies on some Sentinel-employed thieves and their evil talk. 

Then some Cabal agents arrive to crash the party. 
Needle is there, surrounded by bodyguards. 
And around her, more-or-less hidden in \Nyx, are \banerats{}. 
Rian is frozen in terror and just crouches still, watching the Cabal and Sentinel mages, warriors and creatures \rayuth it out. 
He sees into \Nyx{} and other realms and is scared even more. 


When he finally runs away, he stumbles into \Nyx. 
He panics, but at last sits down and tries madly to convince himself that what he is seeing is not real, that he is hallucinating. 
And it works. 
His vision readjusts to the Shroud and he sees his old World again, and manages to stumble back into it. 

He goes away shaken to the core. 





\subsection{Rian meets a living building}
Already early on, during the preparations for the great summoning of \Nithd{}, some people have begun degenerating into mutants, or living buildings. 

Rian discovers a building that has living humanoids magically merged into the foundation. Compare this to Bootstrap Bill or Wyvern, who are becoming part of the \shipname{Flying Dutchman} in the movies \cite{Movie:PiratesoftheCaribbean:II} and \cite{Movie:PiratesoftheCaribbean:III}. 












\section{Spectre of the Fray}








\subsection{Imetrians join the war}
Telcastora Ilcas and some other Imetrians approach \Tiroco. 
They offer their military aid. 

Ilcas carries a gun.

The Imetrians have many soldiers spread out over the country, to protect their temples and their pilgrims. 
200 men at least, and several of them veterans. 
They are allowed this due to some agreements made with the Pelidorian \rayuths. 

They offer their aid. 
But in return they want more privileges for their religion in Pelidor. 
Their aim is clear: They want to proselytize. 

\Tiroco{} and her advisors acquiesce, albeit reluctantly. 
\Tiroco{} wants them, but she fears it is against her \quo{crusade}, and therefore it will look suspicious if she just invites them in. 
Fortunately, Moro and others come to the rescue and vouch for the Imetrians. 
The Vaimons argue. 
Eventually, an agreement is made. 
Only minimal concessions are given. 





\subsubsection{Ilcas and Moro}
Before the Imetrians take off, Ilcas approaches \Tiroco. 
He tells her he wants to talk to an \Ishrah{} mage and asks her if there is one she trusts. 

\Tiroco{} hesitates due to her own falling-out with Moro. 
But she collects himself and reminds herself that Moro is good. 
So she directs Ilcas to Moro. 

Ilcas approaches Moro. 
He tells her about the visions Razor has been having. 
He warns her about the evil in the city. 
He and Razor give all the details they can. 
Moro thanks him. 
But she is suspicious: Why do this? 
Ilcas is not required to by their agreement. 

But Ilcas is a hero. 
Since Razor is sure this alien force is \quo{evil}, Ilcas-tachi feel it is their duty as good Imetrians to inform the people who are allies of their people. 

Moro learns some useful stuff. 





\subsubsection{Imetrians ride away and talk}
When it is settled, the Imetrians ride from \Malcur towards \Forclin. 
They ride \relcs. 
You can't ride \nycans{} over long distances (they \hr{Nycan endurance}{lack the endurance}), and \mulgrons{} are too slow. 

Only minimal concessions were given. 

The Imetrians did not drive so hard a bargain. 
They know it will be bad for them if Pelidor falls to Runger. 
Runger has never let many Imetrians in. 
The Imetrians don't like Runger. 
To top it off, the Imetrians have long suspected Runger of having covert dealings with Durcac, albeit without proof. 

Ilcas talks to his mage-priest companion about this. 
They discuss the fact that Runger is possibly in league with Durcac. 

\begin{prose}
  Ilcas: 
  \ta{Should we not have told the \rinyuth about our fears?}
  
  Mage: 
  \ta{No. Our intelligence is our own. 
    Dessali teaches that knowledge is precious.
    We cannot just give it out for free.}
\end{prose}

Ilcas is still not convinced. 
If it were up to him he would have told them. 
But at least he told them about his fears regarding the evil in \Malcur. 
He tells the mage about that. 
The mage is not happy that Ilcas blurted it out. 
Ilcas defends himself saying that it was not Imetric intelligence but his discovery. 
Razor's discovery, actually. 
So the knowledge was Razor's property to dispense as he saw fit, and Razor saw fit to tell the Pelidorian mage. 









\subsection{Carzain in \Forclin}
\target{Carzain returns to Forklin}
\target{Carzain rejoins the army}
\target{Carzain sees the Morbus in Forklin}
Carzain \Shireyo{} has heard news of the war against Runger. 
He now comes to rejoin the army, and his old acquaintance Archibald Curwen. 

Carzain comes to rendezvous with the army in or near \Forclin. 
He meets Curwen. 
It has been ten years since they last saw each other. 

Carzain brings valuable reconnaisance. 
Curwen is very interested. 
It fits what he has been reading in Tantor's diary. 

Armed with Carzain's information, Curwen convinces Sethgal that it is best to stay holed up in \Forclin{} and meet the Rungerans there. 
See, Carzain's findings further confirms Curwen's suspicion that the Rungerans want \Forclin{} for mystic reasons. 









\subsection{The Ghost Tower}
Sethgal hears that Dendrum has fallen.
Curiously, the Rungeran \ishrah did \emph{not} use their fabled doomsday weapon against Dendrum. 
The real reason is that \Takestsha does not want to draw the Cabal's attention too early.
She wants to drag the Cabal to \Forclin, but not before \Psyrex is ready to raise \Nithdornazsh.
So she has to wait for the opportune moment. 

When they return to \Forclin, Curwen sees the Ghost Tower. 
He thinks about it.
Only later \hr{Charcoal guesses Ghost Tower plan}{will he realize its importance}. 









\subsection{Rungerans besiege \Forclin}
The Rungeran army besieges \Forclin. 
Carzain marvels at the \hr{Glorious armies}{size and splendour of the army}. 

\target{Rungeran super-cannon}
The Rungerans have a big enchanted super-cannon. 
With mage support it can shoot really far and hard. 

Some approximate numbers:
The big super-cannon fires 600 m. 
Pelidor has 500,000 inhabitants. 
10--20\% of those live in the cities, the rest in the country.
(Some sources claim that 1300s England had an urbanization of around 20\%.)
Forclin has 12,000 to 20,000 inhabitants. 

See the sections about:
\begin{itemize}
  \item \hr{TBW weapon ranges}{Weapon ranges in the age of the \thirdbanewar}. 
  \item \hr{Forclin}{\Forclin}, for population data. 
\end{itemize}

There are about 20,000 Rungerans. 
Mostly \humans. 
The heavy infantry is made up of \scathae. 
The cavalry consists largely of the Rungeran nobility, which is \human-dominated. 
Conversely, the Pelidorian nobility and cavalry was \scatha-dominated, whereas most of the \humans are found in the light and medium infantry and among the gunners and archers. 

The main body of the cavalry is the 4000 \relcers. 
The hard core is 60 or so \murocs. 
They have no \grulcans, though. 
\Grulcans are mainly a \Galessan thing.

There are also some 200 \mezolisks. 
These are actually imported from Durcac. 
But they are not badly conspicuous. 
Runger has been known to use \mezolisks{} before. 
The Rissitics have just helped them beef up the number of them. 

Perhaps most gruesome of all, the Rungerans have perhaps 300 \nephil ogres. 
But they are hidden, so the Pelidorians can't see them yet. 

In contrast, the Pelidorian army is smaller, only about 12,000 men. 
They have mostly \scathae. 
Including 1500 \relcers.
And 40 \murocs. 
And a fearsome elite cavalry riding \grulcans.
600 of them. 
Sethgal wonders how the \grulcans will fare against the Rungeran \mezolisks. 

The Pelidorians also have 9,000 infantry, gunners and archers. 







\subsection{The Cannonade}
The Rungerans bombard \Forclin.
Carzain and others sally out to stop them. 

When I am to write a major battle-scene, I should remember to re-read \bandsong{Bal-Sagoth}{To Dethrone the Witch-Queen of Mytos K'Unn (The Legend of the Battle of Blackhelm Vale)}. 

The Rungerans use cannons in the siege. 

Both sides have priests leading them in \hs{prayers against disease}. 
And the Iquinians pray for deliverance from \hr{Isphet}{\Isphet} and his evil.
Remember the \hs{Iquinian clerical hierarchy}. 

Both sides have Iquinian \hs{knights}, who \hr{Knights have superpowers}{have superpowers}.





\subsubsection{Charcoal guesses Ghost Tower plan}
\target{Charcoal guesses Ghost Tower plan}
Charcoal guesses that the Sentinels want the Ghost Tower.

He contacts \Achsah and tells her about this. 
This makes her contact her fellow \resphain and request immediate reinforcements. 









\subsection{Push comes to shove}





\subsubsection{Imetrians arrive}
A small army of about 500 Imetrians fight at \Forclin. 
Not many. 
But these are not conscripts.
They are all skilled fighters.
And they have great \saurians with them. 
And a mage.
And a seasoned hero.
There are about 300 cavalry, 100 \nycaneers and 100 \nycans. 
All in all, they are a formidable fighting force, much more so than their numbers might suggest. 

Describe how skilled, disciplined and fearless they are, with their Imetric gods giving them courage and strength. 

They are supernaturally strong and effective.
As one of their priests says: 

\begin{prose}
  Priest: 
  \ta{We fight an important battle, brethren.
    There are few of us, but the power of our gods will run through us all the stronger for it.}
\end{prose}

And it is true. 
You can see their heathen gods are with them. 

We don't see this from the Imetrians' POV. 
After all, the Imetrians are just a minor player in the story so far. 
They should not get a big, dramatic role until I have had the time to develop them into something cooler, more badass, more background-rich, more well-rounded. 

We see it from Sethgal's POV. 
He is one of the only people present who speaks some Imetric. 
He overhears someone (Ilcas or an Imetric priest) giving the soldiers a peptalk. 

\begin{prose}
  Imetric priest: 
  \ta{If we die on this day, we shall live again!}
  
  Sethgal: 
  \tho{I know the Imetrians believe that they reincarnate when they die.
    I wonder if that is true.
    It is certainly not true for us Iquinians. 
    We die and go into the Light.
    But for them\prikker who can say?}
\end{prose}

The Imetrians scare Sethgal. 
They fight with great zeal, \hr{Imetrian coldness}{but their fervour is\prikker cold}. 
Calculating. 
Reptilian. 
He is quite disturbed.

(Maybe make a footnote about how the warm-blooded \scathae{} do not see themselves as \quo{reptiles}. At the very least, mention this in the glossary.)

Compare them to Haldir's Elves at Helm's Deep in the movie \cite{Movie:LordoftheRings:II} (not present in the book \cite{JRRTolkien:LordoftheRings:II}). 

Remember to read about \hs{Telcastora Ilcas} and the \nycans{} before writing this. 





\subsubsection{Sethgal and Ilcas}
After the battle, have a scene where Sethgal and Ilcas talk about leadership and strategy and experience. 

Ilcas is a good, inspiring leader because of his faith. 
He believes in his cause with great fervour, and this inspires people to follow him.
But he is no great strategist or leader. 
So he can make people follow him, but would not know where to lead them. 

Sethgal is the other way around. 
He is highly skilled as a general, but he does not have quite the same charisma, the same passion. 
His men admire and respect him, but they do not \emph{love} him. 
This is one of his weaknesses. 

Perhaps, Sethgal reflects, this is one of the reasons why he was not elected \rayuth{}. 

Also, remember to display \hs{Ilcas' racism}. 
He accidentally mentions that he and his men are of superior breeding than the Pelidorians. 
Then, when Sethgal is offended, he tries to back down and mitigate what he said. 

Sethgal then looks out over the battlefield.
He is grimly determined to rout these Rungerans. 

\lyricsbs{Bal-Sagoth}{
  And Lo, When the Imperium Marches Against Gul-Kothoth, Then Dark Sorceries Shall Enshroud the Citadel of the Obsidian Crown
}{
  A seething forest of blackened blades.\\
  A churning sea of ebon war-chariots.\\
  A searing storm of flaming shafts.\\
  All this havoc and more shall I unleash against my foe\prikker\\
  Into battle! The Legion shall march\prikker the fall of Gul-Kothoth is nigh!
}

See also \cite{RobertEHoward:KingsoftheNight}. 





\subsubsection{Ulphon Nestor dies}
\target{Ulphon Nestor dies}
Telcastora Ilcas's Imetrians have one mage with them: 
\hs{Ulphon Nestor}. 
He dies quite early on in the battle, killed by some brave Rungerans. 

Later, the Imetrians \hr{Ilcas-tachi attack the Rungeran Ishrah}{mount an attack against the Rungeran \ishrah}.
Here they need a mage for artillery support, so they ask Carzain to join them. 









\subsection{Morgan Runger sees what he has done}
We follow Morgan Runger. He has made an alliance with the Rissitics. Originally, he was thrilled at the prospect of rising to power as an ally of Durcac, but of late he has come to doubt, for two reasons. One, he fears that \Nechsain{} will subjugate him as a vassal rather than a true ally. Two, and perhaps more importantly, he beginst to ponder the ethical consequences of his actions. 

He oversees \Takestsha{} and her fellow sorcerers as they invoke the terrible spells gleaned from the tablets recovered from \Rungertemple, and he witnesses firsthand the horrible destruction they cause. 

He imagines that the \sephiroth{} are weeping, mourning his fall from grace.

\lyricslimbonicart{Beneath the Burial Surface}{
  The sky is darkening, soon the night befall.\\
  Righteously angels are weeping for my soul.\\
  All childhood dreams are soon to be lost,\\
  all innocence to be shattered.
  
  I am the fallen from grace.
  
  My face is a river.\\
  See my eyes as they drown in black.\\
  My sacred doom and nemesis\\
  beneath the burial surface\\
  To the final act of the immortal sin\\
  I am lead by funeral winds.
}

He sees the terrible wickedness of the \EreshKali{} magic.

\lyricslimbonicart{Darkzone Martyrium}{
  Black energies in the twilight space\\
  come shivering through the shallow haze.\\
  Into darkness so impure divine.\\
  A bloodshed emotion to evil wine.
}

Cannibalize the scenes with \Takestsha from \quo{The \Caliph Inviolate}! 
(Look in the \quo{Carzain Prequel} folder.)

\begin{prose}
  Morgan Runger had approached her from behind and was few steps away now. 
  She had smelled and heard him approach from a mile off, of course. 
  \ta{They were Pelidorian scouts,} she told him, not turning. 
  \ta{We have chased them off. I say we let them run.}
  
  \ta{Mhm. Someone will deal with it,} said the king. 
  Untroubled. 
  Morgan had seen the fire, as had everyone, but evidently it did not scare him. 
  He was confident that his bolstered \ishrah{} could counter any sorcerous attack. 
  Just as she wanted him to be. 
  
  He came up close behind and wrapped his arms around her. 
  Began to grope her body. 
  \tho{%
    Heh. Delegating responsibility so you can have your pleasure? How very kingly.}
  Morgan had grown quite shameless about their affair, despite the fact that, as a nominal \Iquinian, he was theoretically obliged to be faithful to his wife. 
  But he was a king and could do whatever he wanted. 
  Do \emph{whomever} he wanted. 
  \tho{%
    And, as any man might, he wants to remind everyone how beautiful a woman he is fucking.}
  
  % She doesn't mind, of course. 
  % As a \dragon{} she has no sexual shame. 
  \Takestsha{} reflected that perhaps she ought to fake some shame and modesty to make her guise as a \human{} woman more believable. 
  But then again, she was playing the role of the mysterious and erotic sorceress, and part of her allure was her rejection of conventional morals. 
  
  \tho{%
    Oh, yes. 
    I suppose I had better have sex with Morgan. 
    Have to keep my pet king compliant. 
    
    Tee-hee.
    The discovery of the Scion has made me in a good mood. 
    Who knows? 
    I might even enjoy it tonight.}
\end{prose}






\subsubsection{The evil magic}
\target{Rungeran temple magic}
The \Rungertemple{} magic might be defiler magic (as in \emph{Dungeons and Dragons: Dark Sun}), sucking life out of the world and leaving it gray, dusty and dead. Alternately, it might be bestial, destructive and chaotic magic, apalling in its sheer hate, ferocity and inhumanity. 

The magic involves the conjuration and binding of terrible \daemons{} from \Chaos. These should be as horrible, inhuman and Cthulhu-like as possible. The \daemons{} are the source of their power; they're the ones wreaking the destruction. 

It is hinted that the \daemons{} are not really bound; they are just playing along for their own unfathomable reasons, and may decide to turn on the mages any time. Have at least one scene where the sorcery suddenly backlashes on one of the mages and he dies a horrible death, his flesh boiled and burnt and his soul consumed by the \daemon{} he unleashed. 

Make it clear that the foolish \humans{} are playing with powers far beyond their understanding. 

It is very hard, taxing and traumatic work for the mages. The sorcerers, being ill-informed and ill-educated in the use of this great power, are twisted by it. Their bodies become warped and misshapen, and they go more and more mad. Compare to Hannan Mosag and his K'risnan in \cite{StevenEriksonIanCameronEsslemont:MalazanBookoftheFallen}.

\lyricsbs{Monolith Deathcult}{%
    1917 - Spring Offensive (Dulce Et Decorum Est)
}{
  Creeping like a snake from a can, \\
  the slithering stench of yellow death.\\
  Chemical flame of decay\\
  burning skin and intestine.\\
  Regurgitating the bloody guts.\\
  Spewing last life from a wretched soul.
}






\subsubsection{Morgan angsts}
Morgan grieves and is plagued with guilt and doubt? Should he continue along this path? But what else can he do? 

\target{Morgan has sex with Takestsha}
He has sex with \Takestsha{} (he has few compuctions against cheating on his wife, who is at home in Runger). \Takestsha{} comforts him and assures him that he is doing the right thing. 

Morgan has converted to the Rissitic faith, but he keeps it secret from people for the time being. He still catches himself praying to the \Sephiroth{} out of habit. 





\subsubsection{Morgan's history}
Morgan has a history. His father, Uther Runger, had something to do with High King \LastHighKing{} and the fall of \hr{Great Velcad}{\GreatVelcad}. 

Later (after \TwilightAngelRememberEmph), Morgan will turn against the Rissitics and attempt to right what he has wronged. 















\section{The Immortals}







\subsection{The immortals and their background}
Remember, in chapters featuring the immortals, to have references to their history and relationships. 
And flashbacks, and dream sequences.

The above chapter with \Achsah{} is a good example of where I might do this. 

Does \Achsah{} or \Teshrial{} share a back story with \Ishnaruchaefir? 
That might be cool. 

Compare to the many flashbacks and cryptic back story references in \cite{StevenEriksonIanCameronEsslemont:MalazanBookoftheFallen}. 









\subsection{\Achsah{} and \Teshrial}
Remember to read about \hr{Achsah}{\Achsah} and \hr{Teshrial}{\Teshrial} before writing this! 

Have one or more scenes with \Achsah{} and \Teshrial. She is subordinate to him in rank, so she humbly knees before her master. But inside \hr{Achsah hates Teshrial}{she despises him}. 

Maybe she tells him about new developments with \Tiroco, which she's heard from Needle. 

\target{Brains will triumph over brawn}
We see \Teshrial{} training with various weapons. 
He is an \hr{Weapon master Teshrial}{expert weapons master} with both melee weapons, firearms and magical devices. 
Therefore he wants to test his mettle against the infamous \Ishnaruchaefir. 
He wants to prove that technique triumphs over brute force ({unlike} last time). 
Some \resphain{} doubt this. 
He points out that he has fought \dragons{} before (perhaps even killed some), and he has fought the \Baelzerach{} and their \daemons. 
One of them thinks back to the time after \ps{\Teshrial} first battle with \Ishnaruchaefir. 
\Teshrial{} had been so badly wounded that he had had to drain 50 \humans{} dry of blood and consume all their lifeforce to heal himself and grow his wings and legs back.  

\Menessiaraid \hr{Resurrection help}{helps} \Teshrial revive from his \hs{life-seed}.

He remembers the humiliation of that day as if it were yesterday. 
It fires his pride. 

He thinks about {\ps{\Ishnaruchaefir} rematch offer}. 

\Teshrial{} admits to himself (if not to her) that he was overconfident and careless when fighting \Ishnaruchaefir{}. 
He promises himself to be better prepared next time. 
When he thinks back to the fight, he can see how reckless and stupid it was. 
What chance did he think he stood against the Destroyer? 
But he rationalizes it away, telling himself that he didn't go in to win. 
Rather (allegedly), the fight was a sort of reconnaisance. 
He did it to test \Ishnaruchaefir, see how he fought, so that he could better figure out how to counter it. 
It was just warm-up to make him better prepared for the real fight that is to come. 

\Achsah{} is somewhat disturbed to hear him talk of \quo{next time}. 
She would have thought that one battle against the terrible \Ishnaruchaefir{} must be enough. 
She is worried by the obsession her lord has developed. 

When \Teshrial{} mocks \ps{\Achsah} birth, she thinks to herself: 
\tho{%
  I was born like \Thanatzil, our founder. 
  In blood. 
  The same blood of life which, in other parts of our lives, we hail as something great and sacred.}
  
If \Thanatzil{} is mentioned, then remember to add him to the Dramatis Personae. 

Remember to read the sections about \hr{Teshrial}{\Teshrial} and \hr{Achsah}{\Achsah} before writing the chapter!








\subsection{\ps{\Teshrial} farm}
\target{Teshrial's farm}
Remember to read about \hr{Teshrial}{\Teshrial} before writing this! 

\Teshrial{} breeds \humans. 
Not just slave \humans, no. 
Top quality \humans{}, for the \hs{Communion}. 
He breeds them for beauty, health, obedience and good taste. 

\tho{%
  Most \resphain{} don't understand \humans.
  To them, a \human{} is just an automaton that serves their needs.
  They do not understand all the care and hard work that does into breeding them. 
  
  \Humans{} must be bred with love.}

And \Teshrial{} loves his \humans. 

He has a number of farms (breeding villaes) where he breeds \humans. 
He flies to one such farm to inspect it. 

Mention that \hr{Resphain enjoy flying}{\resphain{} enjoy flying}. 

\begin{prose}
\Teshrial: 
\tho{Maybe mortals would appreciate walking more if only nobles had legs.

  Brr. 
  What a grotesque and repulsive idea. 
  Legless \humans{} crawling around.
  Poor things.}
\end{prose}

He arrives at the farm. 

He inspects the newborn and gives them his blessing. 
He touches them with the tips of his wings to bless them, caressing them with his soft feathers. 

He looks at a young girl of around 15. 
Touches her. 
She giggles and squirms when he tickles her with his feathers. 
But she is brave enough to meet his gaze. 

\ta{%
  Look at you. 
  How fine you are. 
  How perfect. 
  As beautiful as a \resvil.}

Maybe the girl is Evith. 
Almost ripe. 

\ta{%
  Very soon, my beautiful one. Very soon.}

But he thinks to himself: 
\tho{%
  Sigh.
  If only it \emph{were} a \resphan{} child.}
There are very few \resphain{} being born these days. 
\Teshrial{} has barely ever seen a \resphan{} child. 
He has only heard stories of the golden age (\hr{Shroud harms fertility}{before the Shroud}) when his people were numerous and fertile. 
It has something to do with the Shroud, he knows. 
He does not understand the details of \dweomer{} theory. 
It has not been his specialty. 

\tho{%
  What I would not give \hr{Teshrial wants children}{to have a son or daughter of my own}. 
  Not a \bezed. 
  Not a surrogate.
  A pureblood.}






\subsubsection{Punishing lovers}
There are two young \humans{} who have broken the rules. 
They have become lovers and had sex, and the girl has gotten pregnant. 
This is a crime, because they should only breed as they are ordered. 
It is a sin, for a \ps{\human} flesh and sex is a sacred thing (\resphan{} property) not to be thus profaned. 
They must be punished. 

\Teshrial: 
\ta{%
  Know this: 
  You have caused me great sorrow. 
  Your womb and your seed are sacred treasures, but you have violated and defiled yourselves by your act of fornication.
  I would that this sin could be forgiven and washed awau, for I grieve to see my beloved children punished. 
  But some sins cannot be washed away. 
  You must die, and your flesh shall feed the beasts, for you have been tainted and can no longer achieve the blessing of the Communion.}









\subsection{\Ishnaruchaefir wreaks havoc}
Clarify that \Ishnaruchaefir is a genuine and current threat to the \resphain, not just a past threat.
He keeps fucking up their schemes, and he is endangering a long-term scheme that is vital to \CiriathSepher if they want to rise to supremacy and realize their worth.
He preys on the \resphain and kills them and their servants. 
He has been passive for a long time, but now he is becoming a serious menace, and the Cabal fear him.

And he is casting spells (storm beacons and the like) that prevent the Cabalists from doing their thing in \Malcur.
They have a constructive goal in \Malcur. 
Clarify that.
They think \Ishnaruchaefir is the biggest threat against that goal, but it turns out Secherdamon is a worse threat. 

\Ishnaruchaefir is not so big a threat that all \resphain in the world are after him, though.
So far he is just endandering the \hr{Cabal plan for Malcur}{\Malcur venture}, which only a small part of \CiriathSepher really care about. 
(Although the ones that do care have very high expectations of this gambit and hope it will determine the future fate of \CiriathSepher, if not all \resphain. 
 The ones not part of the gambit are more \skeptical. 
 \Azraid has hopes for the venture, but remains \skeptical and aloof.)

(Make a section about the \CiriathSepher \Malcur Venture.)

\target{Teshrial is their best bet}
Anyway, there are several \resphain who want to do \Ishnaruchaefir in, but he is notoriously elusive.
But he has promised \Teshrial to give him a rematch, and told \Teshrial to contact him when he is ready.
In some very clear terms.
So the \resphain know that if they want to do \Ishnaruchaefir in, \Teshrial is their best bet.





\subsubsection{\Ishnaruchaefir attacks viewing station}
\target{Ishnaruchaefir attacks viewing station}
The \resphain working in \Malcur \hr{Cabal stations near Malcur}{have some viewing stations and stuff} set up in a Realm adjacent to \Azmith. 
\Ishnaruchaefir sends a horde of his \daemons to overrun one such station. 
It demonstrates to the \resphain how dangerous he is, and why he must be stopped.









\subsection{\ps{\Teshrial} banquet}
\Teshrial{} holds a small banquet and invites some \ketheran{} acquaintances. 
Among them is \Firaxel, a \resvil{} whom he has the hots for and wants to score. 
She is high status, capable and beautiful. 
A real catch. 
\Teshrial{} wants to lose his \quo{\hr{Teshrial's virginity}{pureblood virginity}}. 

Read about \hr{Teshrial}{\Teshrial} and \hr{Firaxel}{\Firaxel} before writing this chapter!

For the party, \Teshrial{} has dyed his white hair with a single stripe of magenta, fitting the pink of his eyes. 
He also wears a magenta sash. 

Another guest is \hr{Dezruth}{\Dezruth} of \Mystraacht. 
(Remember to describe \hr{Dezruth's appearance}{his appearance}.) 
He comes bringing two naked girl-slaves with him (\hr{Resphan slave livery}{\Mystraacht{} slaves often go naked}). 
This provokes some of the \CiriathSepher{} present. 
\Dezruth{} himself thinks he is being very modest and respectful of his host's culture by bringing only two slaves, where he would have liked to bring more. 
The girls kneel by his side most of the time, only occasionally doing something active to service him. 
\Dezruth{} mostly relies on the local slaves when he needs to be serviced. 
His own slaves are mostly there to show his status. 

He woos \Firaxel. 
She teases him. 
(\Resviel{} are expert teasers.)
But she is a bit impressed and aroused when she hears the story of how he stood his ground alone against the dreaded \Ishnaruchaefir{} and had the courage to challenge him. 
And even to fight to the death! 
(\Ishnaruchaefir{} has a reputation for destroying souls, so the fact that \Teshrial{} has survived with his soul intact is considered an accomplishment.
The \hs{Shroud prevents soul destruction}, but \Ishnaruchaefir{} is strong enough that he \emph{could} have eaten \ps{\Teshrial} soul if he \emph{really} wanted to.
\Menessiaraid{} later warns him about this.)

\Teshrial{} sees her breasts heavy ever so slightly in arousal. 
He knows he has won a small victory and is one step closer to his goal. 

He tells about his ambition to defeat \hr{Ishnaruchaefir}{\Ishnaruchaefir}. 
He hopes that it will win him status in the \ps{\resviel}{} eyes, especially \Firaxel. 
He thinks about \hr{Teshrial's family}{his father and mother and their great deeds}. 

\tho{I deserve \resvil{} pussy. 
  I have earned it. But I will earn it even more.}

He and \Firaxel{} softly caress each other with the tips of their feathers. 

\Teshrial{} mentally comments to himself how un-beautiful \Achsah{} is compared to \Firaxel. 
(Contrast this to how beautiful \Achsah{} appears to Needle.) 

At the end of the party, she kisses him on the lips. 
That is another victory for \Teshrial, but no guarantee yet. 
She gives him one of her feathers as a gift. 
This is arrogant of her, and he worries that he has lost value by supplicating and treasuring it. 





\subsubsection{\ps{\Teshrial} quest}
\target{Teshrial's quest}
\target{Menessiaraid's advice}
\Teshrial{} tells the others about the quest he has undertaken to bring down the mighty \Ishnaruchaefir. 
He knows it is risky and dangerous, but he wants to take risks and be bold. 
The \resviel{} love a hero. 
And he notices how \ps{\Firaxel} eyes light up when he tells about it, subtly implying his own daring. 

At first, he simply talks big about vanquishing \Ishnaruchaefir, but has yet to formulate any real plan for how to do so. 

At the party is \hr{Menessiaraid}{\Menessiaraid}, a friend of \Teshrial{} who wants him the best of success in his wooing of \Firaxel. 
He recommends that \Teshrial{} study the myths. 

There are people who believe that \WanderersInDarknessEmph holds the key to slaying the mighty \dragonlord. 
But not much research has been done. 
It is not really worth it. 
The have \resphain{} very rarely been given the opportunity to confront the elusive \Ishnaruchaefir{} on anything but his own terms\dash if at all. 

But \Teshrial{} now has such a chance. 
When they fought, \Ishnaruchaefir{} {offered to accept \ps{\Teshrial} challenge to a rematch} should he issue it. 

Given that he has this extraordinary chance, \ps{\Teshrial} friends encourage him to do research, prepare himself and lay traps that fully utilize \ps{\Teshrial} strengths and \ps{\Ishnaruchaefir} weaknesses. 
\quo{Solve his \hs{Aenigma}}, so to speak. 

A few have believed to have found the answer and then gone off to fight \Ishnaruchaefir{}. 
None returned to tell the tale. 
\Menessiaraid{} knows of one such would-be hero who challenged \Ishnaruchaefir{} and fell: 
\hr{Lothagiel}{\Lothagiel}. 
\Menessiaraid{} recommends that \Teshrial{} seek out \hr{Nemuragh}{\Nemuragh}, who was a close friend or family member of \Lothagiel. 
\Teshrial{} later \hr{Teshrial seeks out widow}{does}. 





\subsubsection{\ps{\Teshrial} motivation}
Make it clear that \ps{\Teshrial} motivation is not purely selfish. 
He wants to win glory for himself, but he also wants to do heroic things because he genuinely thinks they are right. 
He wants to destroy \Ishnaruchaefir{} not just out of a personal vendetta, but because he is an evil menace\dash so evil and insane that even his own people, even his brother and daughter, revile him as anathema. 
So evil that he was even \hr{Ishnaruchaefir and the Sentinels}{cast out of the Sentinels} (or so \Teshrial{} believes). 
And \Teshrial{} wants to avenge the many victims (warriors and civilians, mortal and immortal alike) whom the wicked Exile has slain over the millennia. 

He wants to have children, not just for his own vanity and legacy's sake, but because his race needs children. 
He genuinely thinks he has good genes, and it is his duty to carry them on. 





\subsubsection{Called back by \Achsah}
While out on his farm, \Teshrial{} receives word from \Achsah{} (who is a \hs{High Telepath}) that \Ishnaruchaefir{} has appeared in \Malcur. 
He hurries back home. 









\subsection{\Teshrial goes to \Urizeth}
\subsubsection{\Teshrial{} reads portents of \Ishnaruchaefir}
Have lots of astrology involved when \Teshrial{} researches the Aenigma. 

Fearing \Ishnaruchaefir, \Teshrial{} reads the stars in an attempt to understand the weave of the \matrices. 
He finds \ps{\Ishnaruchaefir} star\dash the \hs{Exile}\dash shining bright, which is a sign that \Ishnaruchaefir{} is once more active as a \vertex, after having been mostly dormant for hundreds of years (unlike \Secherdamon, who is always plotting). 

\target{Teshrial does not know that Ishnaruchaefir knows}
\Teshrial{} is confident that \Ishnaruchaefir{} does not suspect the true extent of the \noggyal{} plot. 
After all, \Ishnaruchaefir{} said: 
\ta{%
  \hr{Ishnaruchaefir fakes ignorance about Ghobaleth}{%
    What a shame that your scheme is now ruined.}}
(Actually \Ishnaruchaefir{} lied. 
 He knew full well that there were more \noggyaleth.)

\Achsah{} also fears the return of the mystic immortal. 
She never knows if \Ishnaruchaefir{} is a direct enemy, a third-party hindrance, or even a temporary ally. 
She doesn't understand him. 

\target{Achsah and Teshrial worry about Secherdamon and Ishnaruchaefir}
\Achsah{} and \Teshrial{} also worry about \Secherdamon.
They know they both have business in \Malcur, so they suspect a connection. 
Even though the two are known to be archfoes, this coincidence is conspicuous. 
But they study the constellations, and the Exile is nowhere near any of \ps{\Secherdamon} \matrices, so they cannot be in league with one another. 
The two are presumably just both drawn toward \Malcur because it is a \nexus. 





\subsubsection{\Nemuragh}
\target{Teshrial seeks out widow}
\Menessiaraid{} \hr{Menessiaraid's advice}{once recommended} that \Teshrial{} seek out \hr{Nemuragh}{\Nemuragh}, who was a close friend and gay lover of \Lothagiel, a would-be hero who challenged \Ishnaruchaefir{} and fell. 

\target{Teshrial gets notes}
\Teshrial{} seeks out \hr{Nemuragh}{\Nemuragh}. 
\Nemuragh{} tells him of his friend's quest and his findings. 
After some negotiation, he lets \Teshrial{} have \ps{\Lothagiel} notes. 
This gives him a lot of insight into the Aenigma. 

Read about \hr{Lothagiel}{\Lothagiel}!

\ps{\Lothagiel} notes are not complete. 
There are fragments missing, presumably destroyed. 
Besides, they were written for himself to understand, not for outsiders. 
They were not compiled into an easily readable form. 
So \Teshrial{} has to piece it together and try to make out what \Lothagiel{} was thinking. 

\Nemuragh{} tells him that \Lothagiel{} was studying \WanderersInDarknessEmph. 
\Teshrial{} later brings this up with \Menessiaraid. 

\begin{prose}
  \Teshrial{} wants to scoff. 
  \tho{Poetry? 
    You want me to prepare for battle by reading poems?}
  It seems ridiculous to him. 
  
  But \Menessiaraid{} likes the idea.
  
  \Teshrial: 
  \ta{Please explain.} 
  
  \ta{\WanderersInDarknessEmph is powerful.
    It is widely believed that it holds the answers to many important questions within its riddles and symbols.}
  
  \WanderersInDarkness is an epic poem, Teshrial knows.
  It deals with Ishnaruchaefir and his brothers, among other things. 
  Written after the Incursion by \Melcryth, some mad \dragon.
  Or so it was believed.
  The poem is in Draconic, but the name \quo{\Melcryth} appears to be a pseudonym, and no one knows who hid behind it. 
  
  \Teshrial{} is inclined to disbelieve. 
  But he knows \Menessiaraid{} is not stupid.
  \Menessiaraid{}'s area of study is religion and mythology.
  He has helped shape the dogma and world-view of more than one puppet religion in the Shrouded Realms.
  He knows mythology is serious business.
  So he might have a point.
  
  \Menessiaraid:
  \ta{There are many who study it.
    Not just mystics and philosophers, but also magic scholars.
    And Matrix theorists. 
    The poem is difficult to understand, but its study has yielded some remarkable insights over the centuries.
    You should not discard it.
    Nor other pieces of myth and poetry like it.}
  
  \Teshrial: 
  \ta{Hm. 
    I am \skeptical, but I know you know what you are talking about.
    Very well.
    I will do as you say and look into the myths.}
\end{prose}

After his initial defeat, \Teshrial{} has developed an obsession with \Ishnaruchaefir{} (just as \Ishnaruchaefir{} wanted him to) and now studies the records about him. 
He reads them on \hs{graph-glass}. 

He reads \hr{Urizeth}{\ps{\Urizeth}} annotated \emph{\hr{Wanderers in Darkness}{\WanderersInDarkness}}. 

There are many different versions and fragments of the epic. 
Some of them are considered apocryphal by many. 
\Teshrial{} reads them anyway. 
They are artistically successful and evocative, so \Teshrial, fancying himself an artist, likes them and believes there must be some truth in them. 






\subsubsection{\Urizeth}
\Teshrial{} does not figure out all this on his own. 
He seeks out \Urizeth{} and talks to her in person. 
He asks for her help in interpreting the Achilles Heel and the Nadir. 
She provides much help. 
He shows her \ps{\Lothagiel} notes. 
She did not previously know of it, so she is very fascinated and interested. 
\Teshrial{} learns much. 

\Urizeth{} asks to keep copies of \ps{\Lothagiel} notes and study them some more. 
She tells \Teshrial{} to come back later, and she will discuss her findings with him. 





\subsubsection{Nadir}
From \ps{\Lothagiel} notes, \Teshrial{} learns a very interesting observation: 
Apparently \hr{Ishnaruchaefir's Nadir}{\Ishnaruchaefir{} has a period of Nadir} at more-or-less regular intervals. 

From regular \matrix{} theory it is unsurprising that this would happen. 
\Ishnaruchaefir, after all, is a big-ass \vertex, and the wielder of a powerful \hs{weaving artifact} to boot. 
But apparently no one had mapped the Nadir cycle before. 
\Lothagiel{} managed this. 
He researched some passages of \WanderersInDarknessEmph (which, as it turns out, were authentic, not planted) and picked up some hints. 
He studied \matrix{} theory, interpreted the symbols and kennings and connected the dots. 

According to \WanderersInDarknessEmph, the Nadir occurs \quo{when the \hs{Exile} is engulfed by the briny waters}. 
This is an astrological sign. 
\WanderersInDarknessEmph also reveals that he is at his weakest in the middle of the period, and there are further astrological signs to mark when this happens. 
\Teshrial{} and \Lothagiel{} have no means of verifying this last part, so they have to take the poet's word for it. 

Do not mention any exact numbers regarding the period. 
Just \quo{a number of years}. 
I don't want to paint myself into a corner.

Then he formulated a hypothesis and went about gathering empirical data from \quo{sightings} of \Ishnaruchaefir. 
These were few and far in between, of course, but still, with seven thousand years of history to draw from, he was able to dig up enough sightings for a pattern to emerge. 
And \Lothagiel{} was happy, for the observations supported his hypothesis: 
There were few to no sightings in the periods where \Ishnaruchaefir{} was supposed to be in Nadir. 
And when once in a while he was forced into combat, he seemed uncharacteristically reluctant, weak and prone to fleeing\dash and he did not wield his glaive. 

\ta{And without \Rystessakhin. 
  It\dash or should I say \emph{she}\dash is otherwise perhaps his most powerful weapon.}









\subsection[Criseis in Malcur]{\Criseis in \Malcur}
\Criseis goes to \Malcur to see how things are going. 
She alights on the roof of a high tower and gazes out over the city. 
She can clearly feel that something metaphysical is afoot. 





\subsubsection{Talks to a thug}
\Criseis{} espies a thug who smells like one aligned with the Sentinels. 
She catches him and forces him to tell her what he knows, using compulsion magic. 
Then she Shrouds the dude to make him forget. 

No, she kills him. 
She feels a bit bad about killing him, but reminds herself that he is an evil thug who preys upon the weak. 
She stabs him with a dagger, then ruffles through his pockets and steals some copper coins to make it look like a realistic mugging. 

Later she dumps the coins to a beggar. 





\subsubsection{Visits \Uswa}
\Criseis{} goes to the \hs{dead garden} and talks to \hr{Uswa}{\Uswa}. 
She recognizes that, though mad, \Uswa{} knows more than most people credit her for. 

First we see \Uswa{} alone. 
She is mumbling. 
She communes with \quo{things in the ground} and \quo{things in the air}. 
She knows that there is more than one \quo{tribe} of \quo{things} vying for dominance. 
She can see the chains leading down into the deep, and the spiritual chains that tie people to \Nyx{} and the Cabal \Matrix. 

Then \Criseis{} comes. 
\Uswa{} tells her stuff. 

\Uswa{} sees through \ps{\Criseis} mortal guise and sees that she is really immortal. 





\subsubsection{Meets \Teshrial}
\Teshrial{} meets \Criseis. 

She responds to him in the \Resphan{} tongue. 

He tells her he accepts {\ps{\Ishnaruchaefir} rematch offer}. 
He \quo{makes an appointment}. 

\Criseis{} is afraid. 
From the date \Teshrial{} requests, she can deduce that he has figured out \hr{Ishnaruchaefir's Nadir}{\ps{\Ishnaruchaefir} Nadir}. 
But she doesn't know that this is part of \ps{\Ishnaruchaefir} \trope{XanatosRoulette}{Xanatos Roulette}, so she begins to really fear \Teshrial, for her master's sake. 

And, come to think of it, also for her own sake.
She can tell that he means her ill. 

\begin{prose}
  \Criseis: 
  \ta{I see the way you are looking at me, Lord \Teshrial. 
    Are you planning to use me as leverage to get to my master?
    I think you will regret it if you do. 
    So far Master \Quessanth{} has maintained a certain diplomatic code of conduct. 
    A certain mutual respect. 
    This will cease the instant you try to harm me.
    And I do not think that is what you want, my lord.}
  
  \Teshrial: 
  \ta{Are you threatening me, \scatha.}
  
  \Criseis: 
  \ta{%
    [The story of \hr{Criseis's siblings}{her sister and brother}.] 
    And this is not a myth, Lord \Teshrial. 
    If you doubt me, I suggest you look up the incident in the \CiriathSepher{} archives.}
\end{prose}

\Teshrial feels she is being rude. 
She is a mere mortal and he is a \ketheran.
She has no right to bargain with him or try to \quo{warn} him. 
So he threatens her. 

\begin{prose}
  \Teshrial: 
  \ta{Are you afraid, \Criseis?}
  
  \Criseis:
  \ta{Yes. But \emph{you} should be more afraid, Lord \Teshrial.
    I hear you breed \humans.
    How would you like to have a village of them suddenly consumed in a fireball?
    That is the least of what my master might do to you if you harm me.
    Please know that this is not my threat. 
    I would not condone the slaughter of defenseless \humans, but I cannot answer for my master's anger.}
\end{prose}

\Teshrial{} knows about \ps{\Ishnaruchaefir} rages and terrorism, so he wasn't going to harm her. 
(At the beginning or at the end of the chapter we need to see \ps{\Teshrial} POV and show that he does know the story of \ps{\Ishnaruchaefir} terrorism.)
But her tale still chills him a bit. 
He thinks of how horribly evil the Destroyer is, that he would be so cruel as to let cute, defenseless \humans{} suffer for his wrath. 
This makes \Teshrial{} hate him.
He must die. 

\begin{prose}
  \Teshrial: 
  \tho{Such a monster. He must be destroyed.}
\end{prose}


As \Teshrial{} discovers, \Ishnaruchaefir{} \hr{Ishnaruchaefir's code of honour}{does have a certain code of \honour, but can go berserk if prompted}. 









\subsection{\Urizeth dies}





\subsubsection{\Ishnaruchaefir{} hears of \Urizeth}
Have a scene in the \hs{Mirage Asylum} with \Criseis{} and \Ishnaruchaefir. 

\begin{prose}
  \Criseis:
  \ta{Be careful with this \Teshrial, Master \Quessanth. 
    He is clever and resourceful. 
    Do you remember \Lothagiel? 
    I have learned from my Sentinel contacts that 
    \Teshrial{} has procured \ps{\Lothagiel} notes and is researching you. 
    Moreover, he is studying \WanderersInDarknessEmph, and even consulting an expert on the poem.
    A certain \Urizeth{} of \TiphredSerah.}
  
  \Ishnaruchaefir{} (smiling diabolically): 
  \ta{Is he now? 
    I will have to do something about that.
    \Urizeth, you say\prikker?}
\end{prose}

\Ishnaruchaefir{} is happy. 
This is exactly what he wants \Teshrial{} to do. 
But he wants to fake that he fears this research. 






\subsubsection{\Ishnaruchaefir kills \Urizeth}
\target{Ishnaruchaefir kills Urizeth}
\Ishnaruchaefir{} seeks out \Urizeth{} while she is out in a Shrouded Realm, visiting a demesne. 
He sneaks up on her as close as he can while avoiding detection (which is not very close). 
Then he charges. 
She immediately senses a huge-ass \vertex{} coming straight at her, so she tries to flee. 
But she is unprepared. 
Besides, she is no athlete and hence no fast flyer. 

He catches her. 
He blasts her with an attack spell. 
It does not kill her, but it shreds her wings, grounding her. 
Now she cannot escape. 

\begin{prose}
  \Urizeth: 
  \ta{You\prikker \Ishnaruchaefir!}

  \Ishnaruchaefir: 
  \ta{\Urizeth.
    It has come to my attention that you are\prikker researching me.
    I must interpret this as a challenge. 
    And I am not pleased.
    I fear I must make an example.
    To you and all your kind.}
\end{prose}

Then he kills her. 
She defends herself, but she knows she has no chance. 

When she dies, he makes a mock attempt at destroying her soul. 
She fights back, using all the \TiphredSerah{} stealth and cleverness at her disposal. 
She succeeds, and her soul survives and eludes his grasp.
He lets her think she outsmarted him, but in reality he wanted her to do it. 
Maybe he might have been able to destroy her, but he let her go. 
(He admits, though, that she was good. Slippery like an eel. He is not sure he could have destroyed her. Maybe he tells this to \Secherdamon{} or \Nzessuacrith{} or \Menessiaraid{} or \Criseis.) 
He wants her to go back to \Teshrial{} and redouble her efforts to help him. 

And above all, \Ishnaruchaefir{} wants the \resphain{} to think he fears them and their research. 
He wants them to think they are on the right track and redouble their efforts to divine his weaknesses from \WanderersInDarknessEmph. 

This is a gamble from his side. 
He knows there is a chance they will discover some true weaknesses, but he is willing to take that chance. 
He thinks it more likely that they will discover red herrings and thus play into his hands. 





\subsubsection{\Teshrial{} hears of it}
When \Teshrial{} returns to \ps{\Urizeth} place to discuss their research, he is dismayed when another \resphan{} there tells him that \Urizeth{} has been killed. 
She has not yet regained consciousness, so her fellows only know that she is dead but not destroyed. 
They do not know who did it. 

\Teshrial{} walks away dismayed. 
He has a good idea of who might have killed \Urizeth{}. 









\subsection{\Achsah goes to \Forclin}





\subsubsection{\Achsah{} thinks about Needle}
Needle is very handy. 
The Sentinels know about Charcoal, but they don't know about her. 
She is all close to \Tiroco{} and can spy on her and have her shadowed, and the Sentinels don't know it. 

\Achsah{} kind of likes Needle. 
She loves her for her imperfections. 
\Achsah{} herself is imperfect, too, being \bezed-born. 




\subsubsection{\Achsah{} in \Malcur}
Remember to have more scenes with \Achsah{} in \Malcur. 

She has doubts about Needle's competence, so she takes action more directly. 

Remember to portray her as cool. 





\subsubsection{\Achsah{} suspects a trap}
\target{Achsah suspects that Malcur is a decoy}
\Achsah{} is attempting to unravel what \Ishnaruchaefir{} and \Secherdamon{} are up to. 
Insert some musing about the terrible dark lord \Secherdamon, and the mystic immortal \Ishnaruchaefir, whose motives no-one knows, but whose badass-ness is universally feared. 

Late in the story, \Achsah{} becomes convinced that there is something wrong with the whole \Malcur deal. 
It is too obvious. 
The Sentinels are leaving too many clues out in the open. 
She has been a Sentinel and opposed \Secherdamon{} for thousands of years. 
She knows that being this overt is not his style. 
He is smarter and more stealthy than this. 

She becomes convinced that there is something deeper going on. 
She suspects that \Malcur is a decoy. 

This is deliberate. 
\Secherdamon{} has made his \Malcur ploy so obvious as to convince everyone that it must be a decoy. 
He draws all eyes to him and assures them that there is nothing to see. 
That way, when \Nzessuacrith{} attacks \Forclin, everyone will assume that \Forclin{} is the real deal and rush off to there to try and stop him. 
This will leave \Malcur wide open. 

So she goes to \Forclin. 









\subsection{\Urizeth revives}
\Urizeth{} is thoroughly killed, so it takes several days for her to come back. 
Six days or so after her death, she sends word to \Teshrial{} and asks her to come see her. 
He does. 

When he sees her, she is in terrible shape. 
She looks like a mummy; a shrivelled, mutilated husk of a \resvil. 
She can talk and move her arms, but her legs and wings are not yet fully regrown, so she is confined to a wheelchair for the time being. 
\Teshrial{} is grossed out, but he fully sympathizes. 
He remember how badly shape he himself was in after \Ishnaruchaefir{} had killed him. 

\Teshrial is impressed at the level-headed manner in which she bears her injuries. 
It must be a \TiphredSerah thing, he concludes. 
A \CiriathSepher would be devastated and lock himself up and let no one but his most trusted confidants see him until he was fully healed. 
\Teshrial knows; that was what he did.
A \Mystraacht, on the contrary, would probably display his wounds with pride, as a testament to his bravery or somesuch. 
The \TiphredSerah, as far as he understands, have a philosophy that \quo{looks can be deceptive} and that appearances should not be given much weight. 
\Teshrial supposes this mentality is the reason why \Urizeth is able to bear her wounds with so little emotion. 

\Urizeth{} tells him the story of how she was killed, and the warnings and threats \Ishnaruchaefir{} gave her. 
But \Urizeth{} refuses to give in to his threats. 
She also refuses to wait for her body to recover. 
She wants revenge on the evil \dragon. 
She wants to get back to work as soon as possible, so she can help \Teshrial{} devise a way to rid the world of this cruel monster for good. 






\subsubsection{They discover weaknesses}
With the help of \Urizeth{} and other wise, elder immortals (perhaps even non-\resphain, such as \quiljaaran), \Teshrial{} manages to figure out \ps{\Ishnaruchaefir} weaknesses. 
He studies records of \Ishnaruchaefir{} in battle and maps \hr{Ishnaruchaefir's fake weakness}{the weaknesses he displays}. 

But in truth, the weaknesses are all fake. 
They are lies that \Ishnaruchaefir{} himself planted for this exact reason: 
To fool his enemies, lure them out and destroy them. 
It wasn't in the original \emph{\hr{Wanderers in Darkness}{\WanderersInDarkness}}, so he has planted some edited and unfaithful copies with his own fake myths in them.
(Some bits he wrote himself, others he had his close allies compose.) 

\lyricsbs{William Blake}{%
  The Four Zoas (Night the Second, 24:10-24:4)
}{
  \begin{tabular}{cl}
    FZ2-23.11; E313 & 
    First he beheld the body of Man pale, cold, the horrors of death 
    \\
    FZ2-23.12; E313 & 
    Beneath his feet shot thro' him as he stood in the Human Brain 
    \\
    FZ2-23.13; E313 & 
    And all its golden porches grew pale with his sickening light 
    \\
    FZ2-23.14; E313 & 
    No more Exulting for he saw Eternal Death beneath 
    \\
    FZ2-23.15; E313 & 
    Pale he beheld futurity; pale he beheld the Abyss 
    \\
    FZ2-23.16; E313 & 
    Where Enion blind \& age bent wept in direful hunger craving 
    \\
    FZ2-23.17; E313 & 
    All rav'ning like the hungry worm, \& like the silent grave 
    \\
    &
    \\
    
      
    FZ2-24.1;   E314 & 
    Mighty was the draught of Voidness to draw Existence in 
    \\
    
      
    FZ2-24.2;   E314 & 
    Terrific Urizen strode above, in fear \& pale dismay 
    \\
    FZ2-24.3;   E314 & 
    He saw the indefinite space beneath \& his soul shrunk with horror 
    \\
    FZ2-24.4;   E314 & 
    His feet upon the verge of Non Existence; his voice went forth   
  \end{tabular}
}





\subsection{The Achilles Heel}
\target{Ishnaruchaefir's weakness}
There is a myth in \WanderersInDarknessEmph that \Ishnaruchaefir{} can only be killed under certain conditions. 
He has an Achilles Heel. 





\subsubsection{\Zaz and \Urzaz}
\WanderersInDarknessEmph spoke of a mysterious pair of entities named \hr{Zaz}{\Zaz and \Urzaz}. 
It was unclear whether these were \dragons, \xss, cosmic gods or even purely metaphorical entities, personifications of something abstract.
Compare to Gog and Magog from the \emph{Bible}.

\Urizeth discovers that \Ishnaruchaefir apparently fears them and takes damage from them.  
Some \WanderersInDarknessEmph passages describe how the \Zaz and \Urzaz are anathema to him.

The Exile feared the \quo{body of \Zaz} and \quo{that which issueth forth from \Urzaz}.
Some clues said that he feared the \malgryph (the \quo{body of \Zaz}), that he quailed before it, that it held the power to cast him down and destroy him.

\Urizeth had long had trouble interpreting \quo{that which issueth forth from \Urzaz}.
It could be the very \quo{being} or \quo{aura} of \Urzaz, or it could be his breath or something else that emanates from him.
But now that \Teshrial specifically asks her to look into the problem, she remembers some old research she has done.
It did not lead to much back then, but now she digs it up and looks at it with renewed motivation.

\target{Urizeth researches Chimaera}
She had been trying to nature of the mysterious \Zaz and \Urzaz, and had an idea they were connected with the \quo{\Chimaera}.
The \Chimaera is a creature, probably a metaphoric one. 
It is related to and perhaps identical to \Zaz and \Urzaz.
Perhaps it is the union of these two (possibly contrasting entities) that form the \Chimaera (a \chimaera is a crossbreed or mashup or combination). 

\target{Urizeth thinks Zaz and Urzaz are the Chimaera}
\Urizeth suspected a link from \Zaz and \Urzaz to the \Chimaera. 
But \WanderersInDarknessEmph is a huge poem, and she had not been looking in the right places.
Now her attention is turned towards the parts that deal with the Exile, and here there are some very clear indications that (once you thoroughly interpret them) strongly suggest that \Zaz and \Urzaz more or less \emph{are} the \Chimaera.

\target{Urizeth researches Malgryph constellation}
In connection to all this, there was \hr{Malgryph constellation}{a constellation called the \Malgryph}.
It had a very obscure meaning in ancient \draconian occultism.
It was never used in the \rethyactic tradition except in the vaguest of references, so \Urizeth-tachi had great difficulty researching it. 
They would have to consult a \dragon or \quiljaar sorcerer to learn what the \Malgryph meant.
(Maybe they tried contacting a \quiljaar, but \Ishnaruchaefir got to him first and coerced him into silence.)
But \Urizeth knows that the stars representing \Zaz and \Urzaz are part of the \Malgryph constellation.
It is possible that the \Malgryph \emph{is} the \quo{\Chimaera}.
A \malgryph is, after all, a mix of different beasts and hence a \chimaera of sorts.





\subsubsection{The \chimaera and the \malgryph}
A clever reading of the poem suggests that the body of \Zaz is the same as the \quo{\Chimaera's ichor}. 

The \Chimaera's ichor is a physical substance.
\Urizeth remembers that there exists some practical research regarding this.
It was suspected that the \Chimaera's ichor might have interesting arcane uses, so alchemists did a lot of research on its nature and composition and how to reproduce it.

\Urizeth searches in the archives and finds some material about it.
It turns out that the alchemists did indeed succeed in brewing some \Chimaera's ichor.
It seemed to satisfy the properties described in the poem, so the alchemists were confident they had the right mixture.
Sadly, they failed to find any use for it, so the research project fizzled and was forgotten.
But now \Urizeth and \Teshrial have rediscovered it, and \Urizeth believes the \Chimaera's ichor is vital to defeating \Ishnaruchaefir.

After some more reading and interpretation, \Urizeth believes they need not merely the ichor.
They need the \malgryph itself.
They have to research and discover the \hr{Malgryph summoning}{spell that lets them summon a \malgryph}, and then unleash it upon \Ishnaruchaefir.
\Teshrial is \skeptical, since he believes to know that \malgryphs do not truly exist. 
\Urizeth tells him how that works. 

\Urizeth believes that if they can brew some \Chimaera's ichor, they can use it to summon the \malgryph and have it fight \Ishnaruchaefir.
But it must be done stealthily. 
\Ishnaruchaefir already suspects what they are up to.
He knows they are researching him and reading \WanderersInDarknessEmph, so he may suspect they have uncovered this secret. 
That might in fact be the very reason why he killed \Urizeth.
So they need to summon the \malgryph in some stealthy, sneaky manner. 





\target{It is all fake}
The \malgryph idea is actually a clever ruse by \Ishnaruchaefir. 

In reality, \hr{Zaz}{\Zaz and \Urzaz} \quo{are} not the \malgryph. 
Their nature is more complex and ambiguous than that.

\Ishnaruchaefir planted the clues saying that he feared the \malgryph.
It was based on some authentic \WanderersInDarknessEmph passages that tell how \Ishnaruchaefir feared \Zaz and \Urzaz and was cursed and cast out by them.
This was based on real events.
\hr{Zaz}{\Zaz and \Urzaz} were real cosmic gods, albeit highly obscure ones. 
There was a time when \hr{Zaz denies Ishnaruchaefir}{\Ishnaruchaefir appealed to them for aid}. 
They denied him and punished him, and he was wounded and weakened by their attack.
\Ishnaruchaefir made up some more \WanderersInDarknessEmph verses that were very similar to these ones and embellished on them, telling a bogus story about how he was an enemy of \Zaz and \Urzaz and all their being. 
In reality \hr{Ishnaruchaefir and Zaz}{he is an ally of sorts of those cosmic gods}, and he can command much of their power.





\subsubsection{Heel is inaccessible}
But it is difficult for \Teshrial{} to exploit the Achilles Heel using the skills, weapons and techniques at his disposal as a regular \resphan. 

So \Menessiaraid{} advises him to seek out the \quo{research department}. 
Who knows? 
They might have an \hr{Teshrial's experimental weapon}{experimental weapon} for him\prikker

Maybe it is not \Menessiaraid{} but \Azraid{} \hr{Teshrial talks to Azraid}{who suggests this}. 








\subsection{\Teshrial talks to \Azraid}
\target{Teshrial talks to Azraid}
Have a scene where \Teshrial{} talks to \hr{Azraid}{\Azraid}. 
He remarks on how \hr{Azraid's appearance}{\Azraid{} is much shorter than he}, but looks taller because of his great presence. 

Remember to read about \hr{Azraid}{\Azraid} before writing this. 

\Azraid{} is very interested in the \vertexspike{}. 
He asks \Teshrial{} for details. 
\Teshrial{} regretfully informs him that he knows little. 
It is \ps{\Achsah} table. 
\Azraid{} \hr{Azraid learns of spike}{later learns more}. 

\Teshrial{} talks about his plan to slay \Ishnaruchaefir. 
He has several trump cards:
\begin{itemize}
  \item The Shroud.
  \item Astrology. 
  \item \Noggyaleth.
  \item The Achilles Heel. 
\end{itemize}

\Azraid{} is \skeptical about \ps{\Teshrial} plan to challenge \Ishnaruchaefir.
\Azraid{} has encountered and fought \Ishnaruchaefir{} and knows \Teshrial{} is clearly no match for the \dragon. 
\Teshrial{} assures him he has a plan. 
He will lay a clever ambush, use his \noggyaleth{} and employ the secrets of the prophecies to attack all of \ps{\Ishnaruchaefir} weaknesses. 
\Azraid{} is still doubtful, but the idea has potential. 
\Azraid knows that \Ishnaruchaefir{} is inquisitive and willing to take silly chances or be chivalrous for the sake of his curiosity. 
\Teshrial can take advantage of this. 
It can buy him time to pull off some of his gambits and traps. 
\Azraid gives \Teshrial{} his blessing to continue. 

\begin{prose}
  \Teshrial: \ta{I will lay a trap with my \noggyaleth.}
  
  \Azraid: 
  \ta{But \Ishnaruchaefir{} has already killed one of your \noggyaleth.}
  
  \Teshrial: 
  \ta{Yes, my High Lord \Sathariah, but he believes that was the end of it.
    He does not suspect the true extent of my \noggyal{} plot.
    And even if he does, the one he slew was a runt. 
    I have much bigger ones.
    And he knows naught of the cunning trap I have laid.
    I will utilize his Achilles Heel, as described in the prophecies.}
  
  \Azraid: 
  \ta{Prophecies. 
    You do not believe in such superstition, do you?}
  
  \Teshrial: 
  \ta{No, my High Lord \Sathariah, not literally, of course.
    But myths are often truth shrouded in poetry.}
  
  \Azraid: 
  \ta{True.
    \emph{\hr{Wanderers in Darkness}{\WanderersInDarkness}} has proven itself full of remarkable insight in the past, hidden in symbolism. 
    So you may be right. 
    But I hope you are interpreting this correctly, \Teshrial. 
  
    Besides\prikker 
    \Ishnaruchaefir{} is known to take reckless chances.
    You just might fool him. 
    Very well. 
    You have my blessing. 
    
    However, as well prepared as you are, I still fear that it may not be enough. 
    I have an idea\prikker}
\end{prose}

\Azraid{} is not sure he believes in the Achilles Heel. 
(This is foreshadowing of the fact that the heel is fake, so that does not seem like an \trope{AssPull}{Ass Pull}.) 

\begin{prose}
  \Azraid: 
  \ta{%
    Can it really be true, that \ps{\Ishnaruchaefir} greatest weakness was just under our noses all this time?
    It seems to good to be true.}
\end{prose}


\Teshrial{} comments (inside his head) on the fact that \hr{Azraid's appearance}{\Azraid{} has wrinkles}. 
And his monstrous hand, which he always keeps hidden\dash with good reason, as \Teshrial{} knows. 
Have subtle references to the evil hand in all \Azraid{} chapters. 

\Azraid{} asks whether \Ishnaruchaefir{} might be in league with \Secherdamon. 
But \Teshrial{} assures him he is not. 
\hr{Achsah and Teshrial worry about Secherdamon and Ishnaruchaefir}{He and \Achsah{} have previously examined this}. 

Perhaps \Ishnaruchaefir needs not attack and destroy stuff in order to be a threat. 
I just need to clarify that if he is not stopped soon (chased away or preferably killed), he will wreck everything they have worked for in \Malcur.
When he is at his full strength, he could attack in force and drive the Cabal out of \Malcur entirely.
It is known that he takes an interest in \Malcur, so he likely has long-term evil plans there.
He gave hints of that in WSB. (Make him give hints!)
That must not be allowed to happen.
Furthermore, even now that he is weak, he might be up to something.
If \Urizeth's conclusions are correct, then these Nadirs happen to him regularly, and if so, \Ishnaruchaefir must have learned long ago to live with them and still get stuff done.
One must not assume that he is harmless in his Nadir.
(Maybe it is \Azraid who speculates the above to \Teshrial.)

\Azraid thinks to himself. 
He is not sure whether \Teshrial's \Malcur venture is really so great and important a thing as \Teshrial-tachi like to believe. 





\subsubsection{Experimental weapon}
\target{Teshrial's experimental weapon}
\Azraid{} explains about some of the experiments that his scientists are working on. 
There is a mutation spell in their arsenal. 
\Teshrial{} is hesitant. 
But also excited.
\Azraid{} is good at marketing the thing. 
\Teshrial{} will bring extra glory to himself if he does his people this further service and field-tests their secret weapon. 

\Teshrial{} agrees to take this weapon (but only as a last-resort backup plan). 
\Azraid{} tells him to go to some Cabal scientists and get it from them. 

He does. 
There is a \banelord{} among them. 
Remember to make the \banelord{} use very archaic language. 

They give him this horrible parasite that crawls onto his body. 
When he activates it, it will burrow into his body and transform him into a monster. 

This tool is laboratory-tested but not field-tested\prikker and only tested at a part of its potential strength, not at full power. 
They fear that the full-powered version will have unhealthy side-effects. 

\Teshrial{} sees the mutation process demonstrated (offscreen) and shudders. 

He takes the weapon, but makes no guarantees. 
He intends to hold it back as a last-resort back-up weapon, 
He fears it and will not use it unless he has to. 

The weapon is part of the \hr{Neo-Resphan}{\neoresphan} project. 





\subsubsection{\NeoResphan{} treatment}
The below is an optional spin on the whole \quo{experimental weapon} deal. 

\Teshrial{} is not only handed a device. 
He also undergoes some alchemical and magical treatment that is meant to transform and strengthen his body. 
It works. 
When the day of the duel arrives his body has been strengthened, so he is more powerful than normal even in his \resphan{} form. 

But there are side effects. 
He gets attacks of pain and dizziness. 
Sometimes he wakes up at night with nausea. 
One day after a meal, where he had been drinking lots of blood, he finds himself bleeding afterwards when he's alone. 
The blood he has ingested is seeping out of him. 

I am not sure about the above. 
It might be too cheesy. 
Remember, I am going for \quo{heroic} side effects, not \quo{villainous} ones. 

He is afraid. 
But he bears the pain and fear. 
For his quest's sake. 
For \Firaxel. 





\subsubsection{\Teshrial must undergo a quest}
Perhaps \Teshrial must undergo some kind of quest or test or training before he is deemed ready and worthy to receive the \neoresphan treatment\dash or he must undergo training before he is able and ready to use it. 





\subsubsection{\Achsah{} fears the experimental spell}
\Teshrial{} mentions to \Achsah{} the possibility of using this experimental spell. 
\Achsah{} is frightened by the consequences of that. 
She does not like \Teshrial, but this mad sacrifice on his part is still frightening to her. 

\Achsah{} understands \ps{\Teshrial} motivation. 
He wants children. 
\Achsah{} can never have children. 
She wishes she could at least have the vain hope of once being a mother. 
So she tries to wish the best of success for him, at least. 





\subsubsection{\Teshrial sees how ugly \Nyx is}
In a scene late in the book, after \Teshrial has acquainted himself with the \noggyaleth, \neoresphain and \WanderersInDarknessEmph:
  
He flies above \Nyx.
He sees the \bane-built spires. 
Now that he gazes into the deep, he notices how twisted they are.
They look really scary and wicked.
He had always taken them for granted, but now that he has gazed deep on the dark mysteries of his people (and the even darker mysteries of the \banes), they scare him.
There is something evil about the way they twist and bulge.
They twist into alien dimensions (more alien that what he likes to consider). 
Like they are appendages of some vast monster that tries to crawl and claw its way up from the deep where it belongs. 
See also the sections on \hr{Nyx}{\Nyx}, \hr{Resphan architecture}{\resphan architecture} and \hs{dark ancient cities}. 









\subsection{\Achsah in \Forclin}
\Achsah is now in \Forclin. 
She alights on the roof of a high tower and gazes out over the city. 
She can clearly feel that something metaphysical is afoot. 
She believes she is right. 
The Sentinels are up to something here. 
Probably \Secherdamon. 
She will find out what. 

Se looks at the \hs{Ghost Tower} and comments that it is \hr{Ghost Tower history}{definitely \resphan-built}. 
Despite what the common folk may think. 
Her people once had a city here.
It was destroyed in the \resphanwars. 
Or something. 
(Read about the \hs{Ghost Tower}. 
 If no information is there, make it up and add it.)

She thinks about \Ishnaruchaefir. 
She is not comfortable with \Teshrial's idea of confronting \Ishnaruchaefir in battle. 
She thinks \Teshrial will get himself killed. 
\Teshrial is a vain and condescending snob, but he is not a bad \resphan.
He is no worse than most of \CiriathSepher. 
She cannot expect better treatment. 
She is \ashenblooded, after all. 
Her mother was a hairy, brutish \nephil. 

\Teshrial does not deserve to be destroyed by \Ishnaruchaefir. 

She remembers \Ishnaruchaefir when they met him in the dead garden in \Malcur. 
When \Ishnaruchaefir{} approached the city, \Achsah{} \hr{Detecting Vertices}{could feel him from miles away}. 
He was a behemoth \vertex. 
It felt like a humongous 200 ton sauropod stomping through the city. 
He had not only his own \vertex{} power, but also that of the glaive sending out tremours through the Shroud. 

\Achsah{} described \Teshrial{} in a way to make him seem madly overconfident. 
This is what he intends. 
He hopes to provoke \Ishnaruchaefir{} and goad him into coming to fight him, to \quo{teach him a lesson}. 

\Achsah{} lets slip that \Teshrial{} intends to use the astrological situation as a weapon against \Ishnaruchaefir. 
This is part of \ps{\Teshrial} plan: 
Leak information about one of the traps. 
That way \Ishnaruchaefir{} will be overconfident and not suspect the other traps. 

Make it clear how arrogant, dominant and strong \QuessanthIshnaruchaefir{} is. 
He all but commands \Achsah{} around. 
She realizes that if he gave her an order, she would likely obey him. 
She doesn't like that thought. 
She lived during the Incursion, after all. 
Or the \secondbanewar, as the Sentinels call it. 
She remembers his atrocities. 

\Achsah once \hr{Achsah met Ishnaruchaefir}{encountered \Ishnaruchaefir on the battlefield}. 
She saw his visage, ablaze with fierce chaotic sorcery and hatred toward her kind. 
She felt his presence, radiating a silent promise of death and vengeance. 
And she had not stood her ground. 
She had not even let him come near her. 
She had fled from his path in panic. 
The rest of the battlefield had seemed like a sanctuary, then. 

\begin{prose}
  \tho{I had no courage to face him. 
    But \Teshrial\prikker he stood his ground. 
    He met the Destroyer in single combat, and he fought to the death.
    \Teshrial{} is brave, I must give him that. 
    Much braver than I.} 
\end{prose}



She loathes, fears and despises \Ishnaruchaefir. 
But at the same time, it is hard not to be captivated by the force of his personality. 
Such will. 
Such power. 
Such glory. 
Such ferocity. 

There is raging bloodlust and savagery in him, she knows. 
But at other times, such as this, there is mournful feeling. 

He is an outcast, despised and condemned by his own people, feared and reviled by all. 
Yet he stands proud like a king. 
She can feel great pride and confidence radiating from him.
\Ishnaruchaefir is every bit the \dragonking{} that his brother \Nexagglachel{} was. 
She feel some kind of kinship with him in that moment. 
She is also an Exile of sorts herself, being \ashenblooded. 

(Not too much kinship, mind you. 
 I do not want to dilute the Cosmic Horror.
 Perhaps she sees him as an ideal or possible mentor-figure instead of a kindred spirit.
 An example of how awesome you can be as an Exile.)

\begin{prose}
  \tho{If only I had his strength.
    His disdain.
    His pride.} 
\end{prose}








\subsection{\Teshrial and \Criseis and \Ishnaruchaefir}
\Criseis is in \Malcur again. 
Again she carries \Ishnaruchaefir's presence with her, like she did the first time. 
This is deliberate, to lure the \resphain out. 
\Ishnaruchaefir suspects that \Teshrial is ready to challenge him, and he wants to force \Teshrial's hand. 
\Ishnaruchaefir does not like to be reactive. 
He wants to be proactive. 

\Teshrial is distressed. 
He comes down to \Forclin as expected to confront \Criseis. 
He brings some other \resphain with him who happened to be nearby. 
Perhaps \Menessiaraid. 

On the way down, \Teshrial thinks about the situation.
The \vertex signature is more obvious this time. 
Last time, only the High Telepath, \Achsah, was able to detect it. 
Now all their measuring apparatus can feel it.
The significance of this is not lost on \Teshrial. 
\Ishnaruchaefir is deliberately being obvious.
He wants to lure the \resphain out. 
They are doing exactly what \Ishnaruchaefir wants them to do. 
But they still have to respond.
Who knows what he will do if they do not come?

This can replace the scene where \hr{Ishnaruchaefir attacks viewing station}{\Ishnaruchaefir attacks a viewing station}. 
I just need to have something that clarifies what a great menace \Ishnaruchaefir is. 

The \resphain arrive in \Malcur and find \Criseis. 

\Criseis makes a peace sign.
She is afraid of \Teshrial.
He is more bitter and hateful towards her than last time. 
But he does not harm her. 

\Criseis tells him she comes with a message from her master.

\Ishnaruchaefir then comes up to possess \Criseis's body and speak through her.
\Criseis goes into a trance. 
\Teshrial can recognize the feeling of \Ishnaruchaefir.
He can see the suggestion of a vast, black, \draconian form around her. 
He feels much Cosmic Horror. 

They exchange some taunts. 
\Teshrial{} tells \Ishnaruchaefir{} the date when he wants his rematch. 
It is at the center of \ps{\Ishnaruchaefir} Nadir. 

\Teshrial{} acts overconfident and arrogant. 
Like a whelp who thinks he is something big. 
This is what he intends. 
He hopes to provoke \Ishnaruchaefir{} and goad him into coming to fight him, to \quo{teach him a lesson}. 

\Teshrial lets slip that he intends to use the astrological situation as a weapon against \Ishnaruchaefir. 
This is part of \ps{\Teshrial} plan: 
Leak information about one of the traps. 
That way \Ishnaruchaefir{} will be overconfident and not suspect the other traps. 

\Teshrial tells \Ishnaruchaefir the time and place where he wants to fight him.
\Ishnaruchaefir accepts. 

\Ishnaruchaefir{} fakes suicidal bravado (i.e., fakes that he is trying to fake that he is not afraid, ultimately to make \Teshrial{} think he \emph{is} afraid).
He laughs.

\begin{prose}
  \Ishnaruchaefir: 
  \ta{I see you have done your research. 
    You and that scribbler, \Urizeth.
    But very well.
    I will not renege on my word\prikker not today, at least. 
    You are smaller than I, so it is only fair to grant you this concession.
    My legend attributes to me many flaws, but cowardice is not one of them.}
  
  \Teshrial:
  \ta{Actually, it is.}
  
  \Ishnaruchaefir:
  \ta{Hm. Yes, now that you mention it, it actually is.
    But be that as it may.
    I accept your challenge, \resphan.}
\end{prose}

\Ishnaruchaefir{} makes \Teshrial{} think that he (\Ishnaruchaefir) thinks that \Teshrial{} knows only about the Nadir and not the Achilles Heel. 
He fakes being afraid that \Teshrial{} might discover the Heel, but still cocky and confident that \Teshrial{} probably won't find the Heel. 

\Ishnaruchaefir reminds him of the fact that if the Cabal ever want to kill \Ishnaruchaefir, \hr{Teshrial is their best bet}{\Teshrial is their best bet}. 
\Ishnaruchaefir makes \Teshrial promise him that he will fight only \Teshrial, no other \resphain. 
(\Teshrial smirks inside because he plans to ambush \Ishnaruchaefir with his non-\resphan allies. \Ishnaruchaefir anticipates this. \Ishnaruchaefir is smart enough to know that \Teshrial would never face him if he did not have a trap or several prepared.)

\Ishnaruchaefir also warns them that if \Teshrial should break his promise, \Ishnaruchaefir will refuse to fight, and return later to exact a terrible vengeance.
\Ishnaruchaefir reminds \Teshrial that he found out quickly and easily about \Urizeth, to remind \Teshrial how deep his Cabal spies go and how good his intelligence is.
\Ishnaruchaefir lets slip that he knows \Teshrial has feelings for \Firaxel, and hints that he will come after \Firaxel if \Teshrial betrays him. 
(This is a bluff. It was through luck that \Ishnaruchaefir was able to get at \Urizeth so quickly and efficiently. He doubts he would be able to do the same with \Firaxel.)

\Teshrial remembers the story of how \Ishnaruchaefir terrorized the \resphain after \Criseis' siblings were murdered, so he takes \Ishnaruchaefir's warning very seriously.















\section{Malcur Thread}









\subsection{Rian in church}
Have a scene with Rian in church where he attends prayer and \hr{Rituals against Isphet}{mutilates \Isphet}. 
There is an effigy of \Isphet there in the form of a black serpent. 
Every church-goer is handed a needle or stick with which to impale the monster. 
At last, the effigy is hacked into pieces and burnt. 

In the church, Rian prays for guidance from the \sephiroth, and for their help with his quest to free Neina, and perhaps even \hs{stop the evil}. 









\subsection[Malcur is going mad]{\Malcur is going mad}
\target{Malcur is going mad}
The city of \Malcur is gradually going mad, as a result of the Sentinel ritual gaining strength. 

Mages can see energy seeping, bleeding out of holes in the Shroud, like cosmic blood. 

The people are affected by this chaos, too. They go mad and irrational. Violent riots become more and more frequent, resulting in lynches, arson, gang wars and bloody murder for no reason at all. 

Chaos is unleashing people's natural, inherent viciousness and bestial savagery. 
Have some \trope{RapeTheDog}{Rape the Dog} moments where the people of \Malcur show what kind of scum they really are at heart. 
Then the reader will feel less sorry for them when they are massacred. 

For our main characters, one advantage of the chaos that is engulfing \Malcur is that it is easier to be covert. 
\MoroCobrel can skulk through the streets, cowled and furtive, with a zombie-like man trailing behind her, and not even look very suspicious. 
No one has time to notice. 
In fact, people seem to be trying hard to deny the fact that it is happening. 
This makes it easier to operate, but it is also very worrying.
Why are people insane and denying it?
(Moro may have seen it before, though. She knows \Ubloth, remember.)

Maybe, at the end, have a scene where a bunch of guys are lynching an innocent victim. Then some supernatural horror comes along and the bullies all die a gruesome death, while the victim is lucky and slinks away. 

\lyricslimbonicart{Towards the Oblivion of Dreams}{
  With the underworld's \\
  subconscious darkness I am allied. \\
  Deeper aspects, forces of nature,\\
  mind can now see the unseen. \\
  Ancient land hidden from man. \\
  An esoteric dream in the desert sand. \\
  Shimmering sparks in the darkness, \\
  as death overtakes the soul. \\
  Loose the body, and earthly conscience. \\
  The freedom of the spirit must be total.
}





\subsubsection{Mystic night}
Many of the important scenes take place at night.
Shroud-weaving magic is more potent at night because the Shroud is weaker here. 

As we approach the end, the \Malcuric night becomes more mist-shrouded and mystic. 
Read some Bal-Sagoth.





\subsubsection{\Nithdornazsh peeks out}
Have scenes where \Nithdornazsh{} peeks out of the Shroud. 
Various mortals (possibly Rian) see glimpses of living (or undead) buildings made of flesh and metal. 
Walls suddenly transform into bleeding, oozing flesh for a short instant. 
Preferably in hidden places\dash dark alleys, the slums, behind boards.

Rian sees this more than once. 





\subsubsection{The horror of \Nithdornazsh}
Someone (perhaps \MoroCobrel) sees the \maybehr{Blood-Red Vaults of Nith'dornazsh}{Blood-Red Vaults of \Nithdornazsh} and is horrified. 
She is almost mentally pulled in, but she manages to steer away. 

\lyricsbalsagoth{Beneath the Crimson Vaults of Cydonia}{
  This red charnel pit of primal horror, \\
  howling black ecstasies to the void.\\
  Ancient and divine, older than the hidden Icosahedron, \\
  now rebirthed beyond the chaosphere.\\
  Rise\prikker rise and destroy!\\
  Hatred, carnage, slaughter, havoc, chaos, murder!\\
  I am become the devourer of all life!
}


 


\subsubsection{\Humanoid-based horror}
\target{Humanoid horror}
Have more horror with humanoids that turn into monstrous forms.
Then they go mad and become wholly converted by whatever evil side has transformed them.
They laugh and taunt the remaining mortals (such as Rian and Moro) and tell them that there is no escape, that the whole world will be transformed into a nightmarish hell of madness and chaos. 

(This should also happen earlier in the story.)

Have Moro and Rian as a kind of \quo{Only Sane Men}.
They see \Malcur go mad around them.
They are badly affected, too.
They are going crazy with fear because of all the horrors they see.
Moro only manages to keep her sanity because she has seen things like this before. 
And because she has monitored the process of \Malcur's going to hell, and so is less surprised by it than the regular folks who just see it happening overnight. 
Rian does not hold up well. 
All this evil tears his world view asunder. 
The \sephiroth seem to be powerless to prevent it. 
And this evil is not like he had imagined it.
He had expected winged devils with pitchforks, but not this. 
Not people turning into warped monsters before his very eyes. 

Rian tries to interpret it all in an Iquinian way to make it fit his world view. 
These people must be caught in fetters of \itzach. 
But his rationalization fails. 
This is all too alien, too wrong. 
It is nothing like the priests have described. 
Sinners are supposed to be cast out into the Outer Darkness. 
They are not supposed to mutate like this. 
It is wrong. 
Rian's world breaks down.

Moro slaps Rian up and tells him to pull himself the fuck together. 
She forces himself to shape up.
She is the main reason why he keeps on going and doesn't break down and become a babbling lunatic. 





\subsubsection{Weak souls succumb}
\target{Weak souls go mad in Malcur}
Weak souls are easily affected by \quo{the Change}. 
They go mad and/or mutate into monsters. 
Strong souls are less susceptible. 
Moro and Rian are strong souls (that's why they are main characters).
But Neina is weak. 





\subsubsection{Comparison with Carcosa}
The following scenes from \cite{RPG:CallofCthulhu:GreatOldOnes}, a supplement to the RPG \cite{RPG:CallofCthulhu}, may give an idea of the atmosphere I want to evoke. 

\lyricstitle{\emph{The Great Old Ones} p.70-72}{
  [The prisoner of Carcosa] is free to do anything while awaiting rescue or madness in dark Carcosa. The alienness of this city of towering black buildings costs 1/1D10 SAN per day. 
  Fill the prisoner's time in the city with odd occurrences:
  
  \begin{enumerate}
    \item 
      a keening voice wailing a lonely dirge, the source of which can never be found;
    \item 
      intermittent wingbeats of great unseen things in the thick clouds overhead; 
    \item 
      a slithering wave of fog which tirelessly pursues the prisoner through the damp empty streets;
    \item 
      occasional footsteps or whispering voices in the streets of the abandoned city;
    \item 
      a glimpse of a shadowy figure down the street, where no one can be found;
    \item 
      nightmarish splashing in the waters of the lake;
    \item 
      noises whose sources elude vision because of the thich fog;
    \item 
      a glowing Yellow Sign in the waters of the lake. 
  \end{enumerate}
  
  [\prikker]
  
  As the cultists chant the ritual, thick waves of fog roll in from the lake, then the lake itself swells and grows larger, and the water takes on an oily sheen. 
  The ground gently quakes and stretches. 
  Suddenly the investigators find themselves standing on the outskirts of an alien city, at the edge of a lake much larger than the one they had been observing. 
  The swamp has vanished. 
  The night sky is dull white, and in it black stars shine in unfamiliar patterns. 
}









\subsection{\Nasshikerr returns to Moro}
\target{Nasshikerr returns to Moro}
\hr{Nasshikerr}{\Nasshikerr} returns to Moro. 
He has found out some stuff about what is going on in \Malcur. 
He tells her about how it all comes together. 

Moro goes into action. 
She joins up with \Tiroco, to a limited extent. 
They do all they can to \hs{stop the evil}. 

They find out when the evil ritual will take place, and do all in their power to stop it. 





\subsubsection{\ps{\Nasshikerr} POV}
\Nasshikerr{} does not think any real harm will come from telling Moro about \ps{\Secherdamon} plan. 
He is a rival of \Secherdamon. 

Many Cabalists and especially Sentinels are more interested in their internal power squabbles than the \hs{Feud} itself. 
They do not expect the balance of power in the Feud to change within polynomial time\prikker at least not change asymptotically. 
Millennia of pointless squabbling and minor balance-of-power fluctuations have made them complacent. 
They do not suspect that the end is near. 
Few are as keen-sighted as \Secherdamon{} or \Azraid, and few suspect the scope or nature of their \trope{XanatosGambit}{Xanatos Gambits}. 

Besides, \Nasshikerr{} does not actually believe that Moro will be able to accomplish anything real. 
She is up against the formidable \hr{Psyrex}{\LocarPsyrex}, after all. 
But \Nasshikerr{} feels it could be interesting to see her try. 
(But remember to not make \Nasshikerr{} too evil.) 

Moreover, \Nasshikerr{} hopes Moro will be able to figure out \ps{\Secherdamon} plan. 
He wants to know more about what \Secherdamon{} is up to in \Malcur. 
He cannot figure it out on his own. 
He makes Moro promise to report her findings to him. 

\Nasshikerr{} kind of likes Moro, even though he is rough with her. 
She is a bit short-sighted, and her traumata make her narrow-minded and blind, but she does her best, and he respects her for it. 





\subsubsection{Moro sacrifices to {\Nasshikerr}}
Moro must sometimes sacrifice humanoids to \Nasshikerr. 
She is ashamed of it. 
She hides it from Rian, since she remembers how horrified Rian was after he had seen the Sentinels sacrificing humanoids. 

Whenever she must, she takes some thieves or other lowlife scum that no one will miss. 

\begin{prose}
  \tho{Thieves and scum. That could have been Rian, before I knew him.}
  The thought made Moro shudder.  
\end{prose}

When Moro is out in the city, she notices that \hr{Malcur is going mad}{\Malcur is going mad}. 
    
\hr{Rian is religious}{Make Rian more religious}.
Make sure he prays in every chapter and scene that he is in.
He prays to be delivered from \Isphet's evil. 
He has lingering existential/religious dread from the day when he saw the dark sorcerer slay the shining god (even though he was Shrouded and does not remember it all). 

In all the Rian chapters, whenever it is appropriate, have references to the \hr{Myths of vanquished monsters}{myths of Iquinian heroes vanquishing inhuman Elder Races and monsters}. 
When he encounters something supernatural, he fears that the wicked Elder monsters will conquer the world. 









\subsection{Needle summons \banes and \banerats}
\target{Needle summons Banes}
Needle needs to go on a raid against some Sentinel agents. 
She is only a minor mage, but she has been taught a spell that will invoke \Achsah{}, so she can send minions to her location. 





\subsubsection{She reports to \Achsah}
Some stuff happens. 

Needle is told to go into the crypt. 

\Achsah: \ta{Go into the crypt and summon \banes.}

Needle: \ta{Gasp! The ancient Vaimon crypt?}

\Achsah: 
\ta{%
  Tee-hee (very feminine laugh). I can share this secret with you: That crypt is older than the Vaimons.}

When Needle is down there\dash \emph{deep} down\dash she thinks about this, but it doesn't mean much to her. 
She knows nothing of architecture or ancient history. 





\subsubsection{She goes into the dark crypt}
She goes into the Vaimon-built crypt beneath \CastlePelidor. 
These are dark, disused and have an \quo{antediluvian} feel to them. 
It is a secret place built by Cabalists in the \VaimonCaliphate to serve as a gateway to \Nyx. 
Everwhere, she feels as if the eyes of wicked \Qliphoth{} are upon her. 
As if her audacious steps cause sleeping powers to stir in anger. 
She's afraid. 





\subsubsection{The \banes{} need \human{} hosts}
The \banes{} cannot come to \Miith{} directly, but need to possess a \human{} host. 
So Needle brings three prisoners with her to use. 
All of them total losers. 
They're probably dungeon inmates. 
She has contacts in the dungeon and can easily make some less important prisoners \quo{disappear} in the papers. 





\subsubsection{The summoning ritual}
The ritual is dark and mildly erotic. 
She must circumvent the \Sephiroth{} and invoke \Qliphoth{}, and even nameless powers. 

Is she naked? 
She might be, but only if I can come up with a good excuse for stripping her. 
Perhaps she was simply told that she should be naked because the \resphain{} like the sight of a \human{} prostrate, naked and begging\dash completely humiliated and submissive. 
Perhaps \Achsah{} sexually abuses her a bit. 

Needle is pretty afraid, for she has never had to deal with major issues like this alone before. 
Nor has she ever seen a \bane{} or \banerat{} up close. 
She has seen \resphain, but the more horrid creatures she has seen only from a distance. 
She has always shied away from them and averted her eyes, never having to deal with them directly. So she is frightened at having to take command of them. 

Needle mentions the Shroud. 
She doesn't really understand what the Shroud is. 
But is proud that she knows it exists. 





\subsubsection{She remembers how she first became a Cabalist}
Needle is reminded of the day when she first became a Cabalist, where she was first confronted with \itzach, \Nyx{} and the horrors that dwell there. 
She feels like she is re-living it. 

\lyricsdimmuborgir{Grotesquery Conceiled Within Measureless Magic}{
  For thy presence made pleasure of pain,\\
  and thy madness turned sanity into vain.\\
  Profoundly wicked owner of souls.\\
  The mysteries of thy creation beheld by ghouls.
  
  Diabolically disguised heavenly bodies,\\
  and its atrociously desired primordial elements.\\
  Plunging through the confused beart of sulphur.\\
  In all this darkness, how can a man see?
}

She originally thought that mages were specially gifted supermen with inborn magical powers beyond the ken of the common herd. 
She was wildly surprised and more than a little scared when Charcoal-tachi told her that \emph{she} herself could learn magic. 
According to Charcoal, all it took to cast magic was a worthwhile brain and the willingness to see beyond the lies with which the common folk surround themselves (code for the Shroud) and face the dark truths of the universe. 
And then grab this truth with both hands and coerce it to spit out what you want. 





\subsubsection{Read up on \banes}
Remember to read up on \banes{} in my notes before writing this chapter. 





\subsubsection{She is afraid of the \banes}
She manages to summon two or three \lesserbanes{} and maybe some \banerats. 
The \banerats{} are not so powerful and act mostly as scouts and bloodhounds. 
But the \banes{} are terrible. 
Needle is scared shitless at commanding them. 
(She notices that they wear splint-mail-like \armour\prikker maybe.)

The \banes{} approach her and seem very menacing. 
She can't control them. 
They come ever closer. 

They seem to be directing some mental attack her at her, which she can't defend herself against. 

Needle panics, thinking that they are coming to eat her. 
She freezes up, or falls to her knees. 
She begins crying and whimpering. 
She whines and begs: 
\ta{Stay back! Go away! Don't hurt me\prikker}
She pleads with them to spare her life. 

\Achsah{} contacts her telepathically and slaps her up, telling her to get a hold of herself. 

Needle realizes that the \banes{} never intended to hurt her. 
They were simply menacing because that is how \banes{} appear to \humans: 
Hideous, cold, alien, evil. 

It turns out that the \pps{\banes}{} \quo{mental attack} was simply an attempt to communicate. 
They spoke to her in their telekinetic language and told her their names. 
\Bane{} names are not words but kinetic sensations. 
They are terribly frightening to unsuspecting mortals. 
\tho{So that's their way of speech, is it?}
To Needle it felt as if they were carving their names with daggers on her bare skin. 

Slowly she gets the hang of commanding them. 
But she is still afraid of them. 
Perhaps this gradually drives her mad. 

The three \pps{\banes}{} names are Quicksand, Undercurrent and Gallow. 
They look completely alike to her eyes, but their telepathic \quo{feel} is different. 

Seeing the \banes{} and fearing their terrible power makes Needle admire the \resphain{} even more, since they are mighty enough to control such fell, wicked beings. 
She comes to love them even more. 





\subsubsection{\Achsah{} laughs}
\Achsah{} has been observing. 
Now she sits back and laughs. 
There was no real need for Needle to be naked during the ritual. 
It was just a practical joke \Achsah{} pulled on her. 

She jokingly scolds herself for it. 
\tho{%
  That was unfair of me. 
  I shouldn't be doing that. 
  Playing tricks on poor, unsuspecting minions. 
  Naughty, wicked \Achsah.}





\subsubsection{What happens later}
The \banes{} kill many before they are finally overcome by \ps{\Psyrex}{} mages. 
Maybe \MoroCobrel{} slays one of them\dash after a difficult fight!





\subsubsection{Quotes}
\lyricslimbonicart{Moon in the Scorpio}{
  In an atmosphere supreme\\
  forces dwell in domancy.\\
  The essence of its spirit is evil,\\
  as a curse upon thy name.
  
  Midnight is the shepherd of mysterious powers\\
  and moving shadows in the corner of the eye.\\
  Moon's blazing intuition\\
  contains what death requires.
  
  Cleanse the doors of perception.\\
  See things appear in its true art.\\
  The cold hands of divinity\\
  will tear thy soul apart.
  
  Behold the sky above \\
  when the moon is in the Scorpio.\\
  A cold bleak light. 
}










\subsection{Moro and Rian and Needle}





\subsubsection{Needle finds the Sentinel lair}
Needle and her servants succeed in locating the Sentinel lair in \Malcur. She begins to make preparations to storm them. 

Needle is growing ecstatic and hysterical, with the new dark powers at her command and the terrible \banes{} that follow her.

\lyricslimbonicart{Darkzone Martyrium}{
  I perish in my own desire.\\
  I burn within lusting hate.\\
  Destructively the minds inspire\\
  the soul to terminate.
  
  I ride the ancient overture\\
  as life is torn astray.\\
  I glance the illusive spectrum\\
  and all light that fades away.
  
  Black energies in the twilight space\\
  come shivering through the shallow haze.\\
  Into darkness so impure divine.\\
  A bloodshed emotion to evil wine.
}





\subsubsection{Moro and Rian find the Sentinel lair}
Moro and Rian have gotten some tips from \Nasshikerr (chiefly) and from the thug Moro has interrogated (secondarily).
They know there are some evil people that are preparing some big, evil ritual of magic, and that it is connected to that which is destroying the city.
The captive thug tells them where the ritual will take place. 
\Nasshikerr has told them the time. 

Moro and Rian go through \Malcur and look for the place which \hr{Nasshikerr returns to Moro}{\Nasshikerr} and/or \hr{Rian and Moro interrogate a thug}{their captive thug} described to them. 
This is dangerous and traumatic for them, because \hr{Malcur is going mad}{\Malcur is going mad}. 

Rian and Moro investigate and begin to uncover the city's dark secrets. 

When Moro is out in the city, she notices that \hr{Malcur is going mad}{\Malcur is going mad}. 

Finally they find the Sentinel hideout where the ritual takes place. 
They go there to attack the ritual and try to stop them.

At the very same time, Needle also attacks.
Moro and Rian see Needle and the wicked \banes she commands. 
They are both horrified\dash{}especially Moro, because she knows what \banes are. 
They do not know that Needle is there to do the same thing as they, so they assume she is their enemy (after all, she commands the \banes).
So they decide to try to kill her. 





\subsubsection{\Banes wreak havoc}
The \banes wreak havoc on the Sentinel camp. 
They slaughter many Sentinels. 

It was an unforeseen development.
\Psyrex had not expected the Cabalists to loose \banes. 
And these \lesserbanes are small enough to be difficult to detect for a mage, but still deadly enough to be a great menace. 
\Psyrex fears he will have to go in himself and fight the \banes. 





\subsubsection{Rian and \Cobrel{} kill Needle}
\target{Needle dies}
Near the end of the story, Needle has managed to sniff out that \Tiroco{} is working for the Sentinels, and she is tracking her moves. 
Needle has her Cabalists poised to break in and interrupt the Sentinel ritual\dash which would FUBAR \ps{\Secherdamon} plans. 

\Psyrex{} suspects the raid and is trying hard to pin it down, but failing. 
Needle is not stupid. 

But Rian and \Cobrel{} are onto Needle. 
Before Needle can launch her final raid, they strike against her. 
\Cobrel{} keeps her (few) defenders (mortal or supernatural?) at bay. 
Rian distracts Needle and her companions by throwing knives at them (he has acquired skill in knife-throwing in his thieving days). 
Then Moro goes in and kills Needle. 

Remember that Rian needs to put his carpentry skills to use! 
    
\hr{Rian is religious}{Make Rian more religious}.
Make sure he prays in every chapter and scene that he is in.
When he sees \Ishnaruchaefir, he prays for deliverance from this great evil.
He prays to be delivered from \Isphet's evil. 
He has lingering existential/religious dread from the day when he saw the dark sorcerer slay the shining god (even though he was Shrouded and does not remember it all). 

In all the Rian chapters, whenever it is appropriate, have references to the \hr{Myths of vanquished monsters}{myths of Iquinian heroes vanquishing inhuman Elder Races and monsters}. 
When he encounters something supernatural, he fears that the wicked Elder monsters will conquer the world. 





\subsubsection{\Banes go out of control}
Now that Needle is dead, the \banes have no one to give them orders. 

\Psyrex arrives in person to fight the \banes. 
They are hard, even for him. 

Moro leads one \bane away and hurts it.
She cannot kill it, but she can keep it at bay and occupy its attention long enough for \Psyrex to deal with the other two \banes and complete their ritual. 
Then, when \Nithdornazsh rises, Moro gets separated from the \bane and manages to finally escape it.
She is convinced it will go elsewhere.
After all, it has no particular reason to want to kill her. 








\subsection{Needle dies}





\subsubsection{Rian is going mad}
Rian is slowly being driven mad by the things he has witnessed. 

\lyricslimbonicart{The Yawning Abyss of Madness}{
  Behind the sealed door to imagination \\
  I sense the voices of devastation. \\
  Dementia praecox. 
  
  A cascade of dark emotions. \\
  An ominous silence imprisons me\\
  with disfigured landscapes.
}

Remember that Rian now has a religious trauma and fears that there is something wrong with the \sephiroth. 
He tries rationalizing it away, but a lingering dread remains and will return to haunt him throughout the story. 

\citebandsong{DeathspellOmega:FasIteMaledictiinIgnemAeternum}{%
  Deathspell Omega
}{
  The Shrine of Mad Laughter
}{
  The idea of God is pale next to that of perdition, \\
  but of this I could have no inkling in advance.
}






\subsubsection{They sneak up on Needle}
Rian and Moro sneak into the place where Needle is. 
They are afraid of the dark powers around them\dash the \banes{} worst of all. 
They feel the darkness and evil surrounding them, like a palpable force. 

\lyricslimbonicart{A Demonoid Virtue}{
  The night has predatorial eyes,\\
  drifting in a plae of disguise.\\
  Beneath the spelling moon\\
  the spirit rises out of darkness.\\
  The spirit rises\prikker
  
  Voices call for my soul.
}





\subsubsection{Why do you do this?}
Before Needle dies, Rian asks her: 
\ta{Why do you do this?}

Needle: 
\ta{For survival. 
  For the future of my people. 
  \emph{Our} people, damn you! 
  For \human kind's survival!}

Just before she dies, Needle's last thoughts are: 
\tho{Great \resphain.
  \Achsah.
  \Teshrial.
  Etc.
  Forgive me for failing you. 
  But I have loved you and served you with all my being.
  Truly I have. 
  
  I have no regrets.
  
  No.
  That's not true. 
  There is something I regret.
  I regret\prikker}

And then she dies. 
We never get to hear what she regrets. 

Remember that Rian now has a religious trauma and fears that there is something wrong with the \sephiroth. 
He tries rationalizing it away, but a lingering dread remains and will return to haunt him throughout the story. 

\citebandsong{DeathspellOmega:FasIteMaledictiinIgnemAeternum}{%
  Deathspell Omega
}{
  The Shrine of Mad Laughter
}{
  The idea of God is pale next to that of perdition, \\
  but of this I could have no inkling in advance.
}





\subsubsection{A \bane hunts them}
One of \hr{Needle summons Banes}{the three \banes{} that Needle summoned} comes after Moro and Rian. It is too powerful for them to kill, but Moro has enough magic to be able to hold it at bay and give them time to escape. 

Moro uses the spell \word{\hs{khestni}}. 
It hurts the \bane, but does not kill it. 
It is traumatic for Moro to cast. 
It almost hurts her more than the \bane. 

Whenever Moro draws deep of her terrible magic, she fears that perverse powers might gain control of her. 

\lyricslimbonicart{A Demonoid Virtue}{
  Voices call for my soul.
  
  The cunning serpents kiss I taste.\\
  Baptised beside the ancient takes of\\
  fire, fire burning higher.\\
  Unite with me in dark desires.\\
  I perish in bliss of cruelty.\\
  Tormented souls will never rest in peace.
}

But time and time again she must risk it\dash such as when fending off the \bane. 

\lyricslimbonicart{A Demonoid Virtue}{
  In the flames an omen blaze,\\
  enforcing throught the cosmic haze.\\
  To cross the line and dare to glance,\\
  and enter cold void where death romances\\
  in mysteries.
}

After some time, it stops pursuing and goes after more important targets. Moro is relieved, because she was at her limit and could not hold it off much longer. 











\subsection{Moro's research}






\subsubsection{Astrology}
After having failed to extract much information from \Tiroco, Moro tries to do the best she can with the descriptions \Tiroco{} provided her before storming out. 
So she tries to \hr{Astrology}{read the stars} for \Tiroco, to see if that gains her any insight. 

\hs{Moro has doubts about her astrology}. 
It is hard to read anything about a single person. 
It is somewhat easier to read about an entire city. 
She knows there is something wrong with \Malcur as a whole. 
She has some descriptions and clues. 
She tries to use these and read the stars about them. 

It doesn't entirely work. 
So she contacts \Nasshikerr{} instead. 





\subsubsection{\Nasshikerr}
\target{Moro and Nasshikerr}
\MoroCobrel{} wanted to find out more. 
She could not puzzle out what was going on. 
So she called on her patron god, \hr{Nasshikerr}{\Nasshikerr}, whom \hr{Moro serves Nasshikerr}{she served}.
He appeared in his guise of an chameleon.  
She asked him for advice and knowledge. 

\Nasshikerr{} was a vain god who saw no reason to help Moro any more than necessary, so she had to plead and beg and bargain. 
Eventually \Nasshikerr{} agreed. 
He could not tell her what was happening right now, but she should call him again later, and he would have something to tell her. 
And he warned her that she had better have a humanoid sacrifice ready next time; her credit was all but spent. 

\Nasshikerr{} does not admit that he doesn't know. 
He gives evasive answers and promises to return. 
But Moro sees through him. 
\tho{Stupid, arrogant god. 
  Just admit that you don't know.}

Afterwards, \Nasshikerr{} sits and wonders. 
Should he inform \Secherdamon? 
Nah. 
He is no close ally of \ps{\Secherdamon}. 
He will play his own game. 

\Nasshikerr{} is surprised at how much Moro knows, though. 
Given how traumatized she is, she should not be able to see this much. 
The Shroud must be \hs{unravelling}. 

\Nasshikerr{} resolves to read the stars some more. 
He is an expert astrologer. 

Remember to read the section about \hr{Nasshikerr}{\Nasshikerr} before writing the chapter!

And remember that \Nasshikerr{} needs to be in the glossary. 
List his race as \quo{\Taortha}. 











\subsection{\ps{\Tiroco} visons get worse}
After the interrupted divination ritual with \MoroCobrel, \Tiroco{} suffers from an \quo{unfinished gestalt}, and her mind is left in a vulnerable state, unnaturally divorced from the Shroud. 

Her visions start to get really horrible and bloody at this point. She gains the ability to see into the future. (Of course, seeing the true future is impossible no matter what, and her \quo{ability} is exceptionally unreliable.)

\lyricsbs{Bal-Sagoth}{%
  The Splendour of a Thousand Swords Gleaming Beneath the Blazon of the Hyperborean Empire
}{
  The land awash with spilled blood \\
  and viscera torn forth from the sundered dead.\\
  Gorge the Earth with flesh darkened by the claw and fang of war,\\
  rent open to the ravenous maws of worms.
}

\Tiroco{} is traumatized.

She starts to realize that her visions are related to \Icor{} and \Psyrex.

Later it will turn out that it is the nocturnal visitations by \Icor, enabled by \Psyrex, that has slowly opened her mind to the Beyond. 
Made her see through the Shroud a little bit. 
Her encounter with \Uswa{} left her with a lot of images in her mind. 
Dwelling on those memories, her subconscious mind was able to see, in the Beyond, the \emph{true} images that inspired \ps{\Uswa} speech. 
Those images now appear in her dreams. 







\subsection{Rian seeks out \Tiroco}
Rian tries to tell the city guards about the incidents, but they throw him out. He goes to the church, and they call him a heretical madman. 

Eventually he resolves to go directly to the \rinyuth. 

So he sneaks into \ps{\Tiroco} castle, corners her and talks to her about it. 
She realizes that this must not get out, and so, while it pains her to betray a subject, she does what she must: She calls the guards and tells them to throw Rian in prison and not listen to a word the madman says. 

At some point, Needle shows up. Rian freaks out, saying that it's her, that she is one of the evil sorcerers causing the whole thing. Everyone laughs, and the soldiers slap him silly. 

Needle begins to suspect that \Tiroco{} knows more than she lets on, so she begins spying on her mistress. 









\subsection{Moro springs Rian from prison}
Rian is thrown in the dungeon and sentenced to death. 

Before he is to be hung, however, \MoroCobrel{} sneaks in and frees him. 

He tells her that Needle is one of the bad guys. 
She is intrigued by what he has discovered and agrees to let him work with her. 

Remember that Rian needs to put his carpentry skills to use! 

Moro carries a pistol. 





\subsubsection{They talk about \quiljaaran}
\target{Moro and Rian realize QJ exist}
Moro and Rian exchange their stories. 
They have both seen glimpses of the \hr{QJ in Malcur}{\quiljaaran in \Malcur}, but denied it.
Now they know they both saw them. 
So they must be real. 
This is a relief, but also a horrible discovery.

For Rian, \quiljaaran are a remnant of the \hs{Age of Chaos}.
They are Elder monsters that should by all rights have been \hr{Myths of vanquished monsters}{wiped out by Cordos Vaimon}. 

For Moro, they are a cruel, painful reminder of the horrors that lurked beneath \Yormis. 










\subsection{Rian sees vision of soul prison}
Rian asks \MoroCobrel{} for magical help in finding Neina. Moro fears that Neina is dead and cannot bring herself to show the boy his girlfriend's horrible fate, so she refuses at first. But Rian pleads and begs, and at last Moro relents. 

So she uses her necromantic magic to search for Neina. She warns him first that this spell is not highly reliable and might not show anything useful. 

How does the spell work? What kind of divination? Astrology? Aquamancy? Pyromancy? 

After looking for her in the Shroud and the \Wylde{} (which can be frightening enough) to no avail, Moro expands her search to the soul prison of the \Sephiroth{}. Rian is horrified by what he sees; souls chained, tortured and crying out in pain. 

Rian somehow recognizes the feeling of the \Sephiroth. See, he is a very religious young man. A kind priest and a lot of religious discipline were highly instrumental in bringing him out of his life of crime and converting him into a respectable citizen. Bottom line, Rian prays a lot and knows the Light and the \Sephiroth{} well. So he recognizes their feeling in the soul prison. He is horrified. He tries rationalizing it away, but a lingering dread remains and will return to haunt him throughout the story. (Maybe at the end he dies and goes to the wicked \Sephiroth?) 

However, all is not bleak. He and Moro were unable to find Neina in the soul prison, and that is good news. But where is she then? Rian assumes that means she is alive, and is relieved. Moro is less optimistic, and though she tries to hide it from him, he picks up on her unease and is in turn made uneasy. 

He walks away with twin feelings of fear: \tho{What was that Hell I witnessed? And what was that feeling? No, it cannot be\prikker No, I will not think on it!}

\tho{Neina. Be alive. Don't be dead. Please be alive. Please, \Sephiroth, let her be alive.}

\tho{The \Sephiroth\prikker oh, Light\prikker}





\subsubsection{Moro's motivation}
Moro knows that something is seriously wrong in \Malcur, and she believes that the Rian/Neina case may be a lead. 
She intends to follow this lead and hopefully find some information on what is going on. 

She helps him search the Realms for Neina, but she is afraid she might lose herself in the process. 

\lyricslimbonicart{A Void of Lifeless Dreams}{
  I close my eyes and transcend.\\
  beyond the light to a dark world without end.\\
  The spirit escapes the temple of flesh,\\
  as seductive winds of madness\\
  absorbing me into the cryogenic system.
  
  A void of lifeless dreams.
  
  Enter a galactic domain,\\
  frozen in time and space.\\
  A new infinity of serenity.\\
  Interstellar voyage.\\
  The spirit escapes the temple of flesh\\
  into the sphere of mystified gloom.\\
  A cosmic funeral of memories. 
  
  Sarcophagus panorama.
  
  A mania to explore the enigma.\\
  Isolated celestial corpse.\\
  Astral embryonic life form.\\
  Invocation of the dormant realms.
}
  
  
  






\subsection{Rian and Moro interrogate a thug}
\target{Rian and Moro interrogate a thug}
Moro and Rian go out in the city.
They hope to find a Black Plague gangster whom they can capture and interrogate.
It takes several nights of trying before they find some. 


They drag the gangster into a nearby alley. 
The gangster is a sphyle. 
They interrogate her. 
Moro uses some subtle magic to make the gangster talk. 

From the gangster they learn where the abductees are kept and where the Plague have their hideout.
At least one hideout. 
There are others, but this thug does not know where they are.
She is only a low-lewel grunt. 
She is not in on the secrets of how the organization operates. 

The Black Plague thug does not know what the big plan is about. 
He only knows about \quo{\hr{The Change of Malcur}{the Change}}. 

Rian wants to storm in and rescue Neina now that they know where she is, or at least have a good idea where. 
Moro urges caution. 
Moro is not satisfied. 
The thug did not really know anything. 
They have learned an address and the descriptions of some people.
But they have not learned details nor overview of the Black Plague's plans. 

Moro wishes \Nasshikerr would come back to her with more information.
Later \hr{Nasshikerr returns to Moro}{he does}. 















\section{The Fall of Pelidor}









\subsection{Extreme multiprogramming}
When I get to the climactic scenes, there is a technique I need to remember to use: 
\emph{Change POV often}. 
It is employed in \cite{StevenEriksonIanCameronEsslemont:MalazanBookoftheFallen}, and to great and terrific effect. 
It prolongs the tension and draws attention to the bigness and epicness of the whole thing. 

All these things should happen more-or-less in parallel:
\begin{itemize}
  \item The fall of \Forclin. 
  \item The duel between \Teshrial{} and \Ishnaruchaefir. 
  \item Moro and Rian's underground work in \Malcur. 
  \item \Takestsha-tachi's spellcasting in \Forclin. 
  \item \Psyrex-tachi's spellcasting in \Malcur. 
  \item Moro's assassination of Needle. 
  \item Rian's rescuing Neina.  
  \item The battle for the Ghost Tower. 
\end{itemize}

End chapters on a cliffhanger! 
All over the place.









\subsection{Various people pick up on the \Sephiroth}
Gradually throughout the story, more than one person gradually comes to suspect the \Sephiroth{} and unravel the truth about them. 
Rian is obviously one. 
\Tiroco{} may be another. 

At the end, strong hints will have been thrown that the \Sephiroth{} are not what they are made out to be. We still don't know much, tho, other than the fact that they are connected with the soul prison. 

Remember to have loads of prayers and religious references, where the \Sephiroth{} are described as good, indeed, as the source of all good. 

\citebandsong{DeathspellOmega:FasIteMaledictiinIgnemAeternum}{%
  Deathspell Omega
}{
  The Shrine of Mad Laughter
}{
  The idea of God is pale next to that of perdition, \\
  but of this I could have no inkling in advance.
}

\paragraph{But:} 
The \sephiroth should \emph{not} be exposed as evil in the first book. 
They should only be mildly suspected.
Overall they should be portrayed as pure and good, albeit not all powerful. 

\Psyrex tells someone: 
\ta{Your \sephiroth cannot save you, no matter how much they might \emph{love} you.
  Not any longer.
  It is too late for salvation for \Malcur.}









\subsection[Locar Psyrex]{\LocarPsyrex}
In the notes below I have three scenes with \Psyrex: 
One with \Vizsherioch, one with \Nzessuacrith and one with \Ishnaruchaefir. 

Perhaps I should merge them all into one, having just \Nzessuacrith.
And maybe \Vizsherioch, felt as a distant telepathic presence. 





\subsubsection[Psyrex and Vizsherioch]{\Psyrex and \Vizsherioch}
\target{Vizsherioch and Nithdornazsh}
\index{Dagger, the}%
The summoning of \Nithdornazsh{} is part of \ps{\Secherdamon} plan to bring \hr{Vizsherioch}{\Vizsherioch} into Ascendancy. 
\Nithdornazsh{} is to become \ps{\Vizsherioch} citadel, a \nexus{} from which he can grow strong and spread his tendrils (politically and metaphysically) into the Realm of the Shroud. 
This is a vital step in the forging of the \hs{Dagger}. 
When the \Nithdornazsh{} project is complete, \Vizsherioch{} is more Dagger-y than ever. 

Previously, \Secherdamon{} had kept \Vizsherioch{} sequestered and hidden. 
He is his only son and the fruit of thousands of years of hard work, so \Secherdamon{} is very protective and does not want to lose him. 

\Vizsherioch approaches \LocarPsyrex.
\Vizsherioch appears as a \dax in his prime, with pearly white scales, wearing a loose robe of white, silver and gold. 

\Psyrex fears him. 
Where \Secherdamon is fiery bright, his son \Vizsherioch is dark and sinister. 
Not in \colour, but in feel. 
A vast darkness follows behind him and around him. 

\Psyrex fears to look into his eyes. 
\ps{\Secherdamon} eyes are terrible enough, but \Psyrex is used to them. 
There is passion, fervour and desire in the eyes of \Secherdamon, and anger and hate, too. 
But in \ps{\Vizsherioch} eyes there are hints of otherworldly madness. 

\Vizsherioch asks \Psyrex about his progress. 
\Psyrex tells him that there have been some setbacks. 
The Cabalists are sneaky.
He does not know who leads them, now that Charcoal is gone from the city. 
But whoever is in charge must be someone capable. 
They have managed to screw up some of his operations and kill some of his important Sentinels. 
But it is not so bad. 
He has planned for some amount of Cabal interference and taken precautions. 
He is importing more manpower. 
He will be ready on time. 

\begin{prose}
  \Vizsherioch: 
  \ta{The beacons are in place? Show me.}
  
  \Psyrex shows him. 
  \Vizsherioch sees the slender aethereal tendrils, grown from the Pyre \matrix.
  They reach up from \Nithdornazsh to twist around the ley lines and converge upon the \nexus point in Pelidor, where they grab on and hold fast, holding the \nexus in a constricting iron grip. 
  He sees the \matrix subtly reaching out to fasten upon the souls of the mortals in the city. 
  Binding them.
  They will be part of the invocation, he thinks. 
  
  \Vizsherioch: 
  \ta{Aye, I see it.
    You have done well, \Psyrex.}
  He smiles to himself.
  \ta{My father has confined me to our Realms for too long.
    I long to at last set foot in my new citadel.
    And to exert my power in \Azmith; supposedly the most pivotal of the Shrouded Realms.}
  
  \Vizsherioch becomes distant.
  \ta{The constellations are falling into place.
    I can feel the tension in the Pyre. 
    The Dagger is taking shape.
    Very soon now\prikker}
  He becomes present again.
  \ta{%
    What of the \resphain?}
  
  \Psyrex:
  \ta{I have detected some \resphan activity in both \Malcur and \Forclin.
    I still have every reason to believe they will swallow the bait.
    But of course, it will depend on \Nzessuacrith and her task.}
  
  \Vizsherioch:
  \ta{She will not fail us.}
  
  \Psyrex:
  \ta{Yes.
    It is not \Nzessuacrith nor the \resphain that make me uneasy.}
  \Psyrex pauses and hesitates. 
  
  \Vizsherioch: 
  \ta{You mean the Exile.}
  
  \Psyrex:
  \ta{Yes. He has been uncharacteristically\prikker \emph{active} recently.
    In the Pelidor region.}
  
  \Vizsherioch:
  \ta{But he has not antagonized us?}
  
  \Psyrex:
  \ta{Not that I can determine. 
    And that worries me.
    So far he seems to have acted only against the \resphain, but I cannot guess his motives.
    The Exile cannot be trusted.}
  \Psyrex looks at his star-charts. 
  Concentrates to shift his vision. 
  Looks up, bypassing the roof, at the real stars. 
  Stares at the star representing the Exile. 
  It is nowhere near the Pyre. 
  Nor any other known \matrix. 
  He is made uneasy by the thought of the rogue \vertex. 
  A wanderer in darkness who can appear anywhere at any time. 
  
  \Vizsherioch:
  \ta{\QuessanthIshnaruchaefir.}
  \Psyrex is taken aback. 
  He had never heard \Vizsherioch speak the Exile's name before. 
  He thought \Vizsherioch shunned the name like his father did.
  
  \Vizsherioch: 
  \ta{Called Exile and Destroyer.}
  To \Psyrex: 
  \ta{You fear him.}
  
  \Psyrex:
  \ta{Yes, I fear him. 
    Any creature lesser than a \dragon has cause to fear the Exile.}
  \tho{And many a \dragon should fear him, too.}
  
  \Vizsherioch:
  \ta{Hm.
    He is in the Pelidor region.
    So it is possible he has caught wind of our doings.
    He must not be allowed to interfere.}
  
  \Psyrex:
  \ta{The \resphain feel likewise.
    Have you heard, Lord \Vizsherioch, about this \resphan, \Teshrial, who talks about confronting the Exile?}
  
  \Vizsherioch:
  \ta{Yes. 
    Allegedly the Exile has promised to face him.}
  
  \Psyrex:
  \ta{I wonder what will come of this challenge. 
    Will the Exile really return to meet the \resphan?
    I do not think \Teshrial is a fool.
    He has a plan.
    But will it be enough?}
  
  \Vizsherioch:
  \ta{Interesting prospect, this duel.}
  He becomes distant.
  \ta{Perhaps I should seek out \Ishnaruchaefir.
    After all, I have never met my uncle face to face\prikker}
  
  \Psyrex tries to imagine such a confrontation.
  Would it end in violence?
  \Vizsherioch was powerful, but young. 
  Would he be able to stand before \Ishnaruchaefir?
  
  \Vizsherioch reads his thoughts.
  \ta{Fear not, \Psyrex.
    I will not challenge the Exile to single combat.}
  Distant.
  \ta{No, that would be wise.
    Not at this time.
    Not at this time\prikker}
\end{prose}





\subsubsection[Nzessuacrith visits Psyrex]{\Nzessuacrith{} visits \Psyrex}
Remember to have scenes where \Nzessuacrith{} visits \Psyrex{} at his Dark Crescent throne. 

Remove any direct references to the \Malcur \nexus. 
Just refer to \quo{the \nexus{} in Pelidor}. 
We don't want to tell the reader that \Forclin{} is a decoy. 





\subsubsection{\Ishnaruchaefir visits \Psyrex}
Then he goes to see \Psyrex. 
% Maybe have a scene where \Ishnaruchaefir{} approaches \Psyrex{} in \Malcur. 
Like the scenes in \cite{StevenErikson:GardensoftheMoon} where Anomander Rake visits Baruk in Darujhistan. 

Remove any direct references to the \Malcur \nexus. 
Just refer to \quo{the \nexus{} in Pelidor}. 
We don't want to tell the reader that \Forclin{} is a decoy. 

\Psyrex{} is not happy that \Ishnaruchaefir{} can simply barge in, bypassing his guards and spells. 

\begin{prose}
  \Ishnaruchaefir: 
  \ta{So. I see you are attempting to resurrect \Nithdornazsh.}
  
  \Psyrex: 
  \ta{Yes. Do not try to stop us!}
  
  \Ishnaruchaefir: 
  \ta{I would not dream of it. Are you aware of the \noggyaleth?}
  
  \Psyrex: 
  \ta{Of course. Here is how we plan to deal with them\prikker}
\end{prose}


\Psyrex{} tells how the resurrection of \Nithdornazsh{} is a \quo{pivotal step, a decisive battle in the eternal war. One step on the long ladder of bringing the \matrix{} of our kind to ascendancy}.  

\Psyrex{} berates \Ishnaruchaefir{} for being too obvious and attracting unwanted attention. 
He feels somewhat weird yelling at an immortal who is close being the equal of his Exalted Lord, \Secherdamon. 
But \Psyrex{} is proud and knows his worth, so he sees himself as almost \ps{\Ishnaruchaefir} equal. 
So he dares treat him like one. 
Still, he calls him \quo{\Ishnaruchaefir} and not \quo{Exile}. 
\Psyrex{} is not confident enough to refuse him his name. 

Make it clear how arrogant, dominant and strong \QuessanthIshnaruchaefir{} is. 
\Psyrex{} is proud to have stood his ground. 
\Ishnaruchaefir{} tried to pry his plan out of him.
\Psyrex{} resisted, but \ps{\Ishnaruchaefir} frame was super-strong, and he was forced to give up much.
But he did not reveal as much as he could have done. 
He has not totally lost, and he takes pride in that. 
Few can stand against such force of personality, and \Psyrex, for all his skill and wisdom, is only a \scatha. 
He rarely admits that, but in this hour, in a room with the dreaded Exile, he feels the truth of it.
Immortal though he may be, he is only a \scatha. 

Compare \Ishnaruchaefir{} to Lord Asriel in \cite{PhillipPullman:NorthernLights} when he twists Iofur Raknisson around his little finger. 





\subsubsection{The Blood-Red Vaults of \Nithdornazsh}
\target{Blood-Red Vaults of Nith'dornazsh}
Perhaps they look at the underworld beneath \Malcur. 
Like the {haunted dungeons beneath \CastlePelidor}, the underworld is a place where the Shroud is thin. 
It is a gateway to the Blood-Red Vaults of \Nithdornazsh. 

\lyricsbalsagoth{Beneath the Crimson Vaults of Cydonia}{
  Ancient and divine, older than the hidden Icosahedron, \\
  now rebirthed beyond the chaosphere.\\
  Rise\prikker rise and destroy!
  
  Hatred, carnage, slaughter, havoc, chaos, murder!\\
  I am become the devourer of all life!
  
  Phobos, Deimos! \\
  The moons' rays liquefied in these blood red pyramids.\\
  In the shrines of abomination, black tongues rapt with blasphemy.\\
  Chaosphere, watchtowers, genesis, Cydonia\prikker\\
  The Abyss yawns wide!\\
  Spirit of the carrion-thronged battlefield, open wide thy gate!
  
  Unruly evil!\\
  Colossal shapes etched against the moons, \\
  supine obeisance 'fore the mound,\\
  Accursed fiends, hail the Slitherer, \\
  abhorrent jaws drooling lunacy.
}





\subsubsection[Psyrex fears Ishnaruchaefir]{\Psyrex{} fears \Ishnaruchaefir} 
{\Psyrex} fears \Ishnaruchaefir. 
He uses divination and discerns signs that \Ishnaruchaefir{} is active as a \vertex, and also near him. 
He frets, afraid of what the enigmatic immortal might do.

Remove any direct references to the \Malcur \nexus. 
Just refer to \quo{the \nexus{} in Pelidor}. 
We don't want to tell the reader that \Forclin{} is a decoy. 









\subsection{\Teshrial's trap}
\Teshrial and \Urizeth go to \Malcur to plan and set up a trap for \Ishnaruchaefir. 
Maybe it was \Azraid who suggested this. 





\subsubsection{\Teshrial{} sees \Ishnaruchaefir{} as the worst threat}
Have a scene with \Teshrial{} where he contemplates potential threats to \Malcur. 
He sees \Ishnaruchaefir{} as the worst threat, because you never know what that meddling rogue immortal might do. 
\Secherdamon{} is less of a threat, because there is no way he would enter combat personally (it's a trauma he has from back when his brother was killed by \resphain). 
But he fears \Ishnaruchaefir{} and takes precautions for his return. 





\subsubsection{Looks at his \humans}
\Teshrial{} looks at some of home of his home-bred \humans. 
He caresses a beautiful youth with his feathers. 
\Teshrial{} loves his \humans{} and will not let the evil \Ishnaruchaefir{} harm them. 
He was shocked to learn that \Ishnaruchaefir{} would be so cruel as to let cute, defenseless \humans{} suffer for his wrath. 
This makes \Teshrial{} hate him.
He must die. 





\subsubsection{\Teshrial and \Urizeth inspect \noggyaleth}
\Teshrial and \Urizeth inspect their secret weapon: 
The \noggyaleth{} hiding beneath \Malcur. 
See section \ref{Teshrial's creatures}. 

\Teshrial \hr{Teshrial fears Noggyaleth}{is afraid of the \noggyaleth}. 


\Teshrial{} is overconfident. After \ps{\Ishnaruchaefir} initial attack on \Malcur (see section \ref{Ishnaruchaefir attacks Teshrial's creature}), \Teshrial{} was left with the impression that \Ishnaruchaefir{} believes that there was only one \noggyal{}, and that the way to \Malcur is now open. But \Teshrial{} has several more, and he believes that \Ishnaruchaefir{} doesn't know about them. 

\Teshrial{} is confident in their ability to defend the city. Maybe he knows that \Ishnaruchaefir{} knows that he has one of the \noggyaleth{}, but he thinks the rest are hidden. But \Ishnaruchaefir{} sees through him. He knows that \Teshrial{} must have several more \noggyaleth{}. 

Perhaps he needs to perform an occult ritual to unchain the \noggyaleth{}. 
%Perhaps he must needs invoke powers greater than he\dash\resphan{} kings, or the \banelords, or even \Voidbringer.
He needs to invoke greater powers: 
His people's ancient pacts with the \banelords{} and their unholy might.

\lyricsbalsagoth{%
  In the Raven-Haunted Forests of Darkenhold, Where Shadows Reign and the Hues of Sunlight Never Dance
}{
  I stand now at the anvil,\\
  adamantine hammer in my hand.\\
  In thunder-song the steel I smite,\\
  a clarion heard throughout this land.
  
  Ablaze upon the Altar of Stone,\\
  the Sigil of An-rayuth, the summoning.\\
  Folk of the Mist, Dwellers in Shadow,\\
  the thrice-blessed wand of the Wood-Gods is beckoning.
  
  At the aeon-swathed Shrine of the Oak I kneel.\\
  O' Oracle of the Great Forest, hear me this night\prikker
}

Maybe this chapter should be called \quo{The Conqueror Worm}. 
Or some other chapter.

\lyricsbalsagoth{%
  In the Raven-Haunted Forests of Darkenhold, Where Shadows Reign and the Hues of Sunlight Never Dance
}{
  Swaying serpents ring my oak-hewn throne,\\
  Night and Shadow are my hunting dogs.\\
  Ravenous, they howl to be unshackled,\\
  that their maws may be glutted \\
  (with the blood of my foes).
}









\subsection{\Ishnaruchaefir contacts \Secherdamon}
\target{Ishnaruchaefir tells Secherdamon of the Ghobaleth}
Read about \hs{Chaos magic}, and remember to invoke \Sethicus and \Tiamat. 

\Ishnaruchaefir{} contacts \Secherdamon{} and tells him about what he is doing. 

% Or maybe not \Secherdamon.
% Maybe only \LocarPsyrex. 
% Maybe he and \Secherdamon{} are not on speaking terms, so \Psyrex{} has to mediate. 
\Ishnaruchaefir visits {\LocarPsyrex} in his inner sanctum at the Dark Crescent. 
\Psyrex is distressed at how easily \Ishnaruchaefir broke in. 

\Ishnaruchaefir is not physically there. 
But he is close.
He is projecting his presence into \Psyrex's mind. 
\Psyrex feels a mammoth presence.

\Psyrex first calls \Ishnaruchaefir \quo{Exile}. 
But later he crumbles under the \ps{\dragonlord} piercing stare and starts calling him by his real name. 

He tells \Psyrex{} of the \hr{Teshrial's creatures}{\noggyaleth}. 
They are a greater threat than \Psyrex{} realized, and can seriously fuck up his plans. 

\Psyrex{} curses at him for not having told him this before.

\Ishnaruchaefir{} bargains. 
He is willing to help \Psyrex, but he wants something in return. 
(I don't know what yet.) 
\Ishnaruchaefir{} will lead \Teshrial{} and his \noggyaleth{} away from the city. 
He has planted an obsession in \ps{\Teshrial} mind and manipulated him with his false myths, so he believes \Teshrial{} will be obsessed enough to leave his precious city semi-unguarded and rush off to face \Ishnaruchaefir. 
But due to the nature of the myths, on which \Teshrial{} relies, this must be done at a strategically exact point in time. 

\Psyrex{} listens while \Ishnaruchaefir{} describes his plan. 
(Offscreen, of course. 
 \trope{UnspokenPlanGuarantee}{Unspoken Plan Guarantee} and all that.)
Then:
\begin{prose}
  \Psyrex: \ta{This is madness!} 
  
  \Ishnaruchaefir: 
  \ta{
    My last gambit\dash the one for which I am now infamous\dash was also madness. 
    I will do this.
    I will do this favour to my race.
    To \Nexagglachel.
    Yes, even to \Secherdamon.
    And when I do, you will stand ready.}
\end{prose}

Remember to stress how dangerous this is for \Ishnaruchaefir.
\Psyrex knows about the Nadir. 
He knows \Ishnaruchaefir will be at his very weakest. 
Ostensibly he will only be fighting a single \resphan, but \Psyrex is too smart to underestimate the \resphain.
\Ishnaruchaefir will be going into a trap, and he knows it. 

\Ishnaruchaefir{} cannot take on all of \ps{\Teshrial} \noggyaleth{} on his own. 
Therefore, the showdown must be planned to coincide with the ritual summoning \Nithdornazsh. 
This will confuse the \noggyaleth, and they will be forced to struggle against \Nithdornazsh, which will weaken them and slow them down. 

This means \Psyrex{} must speed up his plans. 

\Psyrex{} curses some more, then ultimately agrees, knowing that it is too late to do anything else. 

This scene is inspired by the conversation between Ganoes Paran and Shadowthrone in the middle of \cite{StevenErikson:TheBonehunters}. 

Later in the scene, \Secherdamon appears, in vision form if not in physical form. 
He curses the Exile, but ultimately agrees to his demands. 

\Criseis is with \Ishnaruchaefir. 
She greets \Psyrex. 
They call each other \quo{cousin}. 

Remove any direct references to the \Malcur \nexus. 
Just refer to \quo{the \nexus{} in Pelidor}. 
We don't want to tell the reader that \Forclin{} is a decoy. 





\subsubsection{\Psyrex tells \Nzessuacrith to hurry}
\target{Psyrex tells Nzessuacrith to capture Forklin quickly}
\Psyrex passes on the message to \Nzessuacrith, telling her to hurry up and \quo{capture} the Ghost Tower already. 
He knows that \Achsah{} is resourceful and might salvage everything even if \Teshrial{} falls.
He fears \Achsah more than he fears \Teshrial, actually. 

\Psyrex also relays to \Nzessuacrith a message from \Secherdamon.
\Secherdamon has said that if it becomes necessary, \Nzessuacrith should not hesitate to break the Unspoken Covenant.
Blatantly.

\Nzessuacrith smiles. 
She understands very well what the last part means.
It means that if need be, she should transform to \draconian form. 





\subsubsection{Nadir begins}
\hr{Ishnaruchaefir's Nadir}{\ps{\Ishnaruchaefir} Nadir} is coming, and he feels its onset. 
He prepares for it. 
Sets \Rystessakhin{} in a special place where she can \quo{recharge}. 

\Ishnaruchaefir \hr{Ishnaruchaefir bleeds in Nadir}{bleeds and looks terrible} when he is in the Nadir. 

Make clear that this is of vital importance for the Shroud and to protect \Miith{} from alien menaces (the \banes). 
This makes \Ishnaruchaefir{} dreadfully vulnerable, and \Criseis{} and the grandchildren know that he runs a terrible risk by going into battle at such a time. 

But to \Ishnaruchaefir, it is worth it. 
The resurrection of \Nithdornazsh{} would be a monumental victory for the Sentinels. 
He does not tell the people around him about his plan, of course. 

He leaves instructions with \Criseis{} and \Thiencaste-tachi about what to do if he dies. 
How to take care of \Rystessakhin{} and all that. 

\Criseis{} is very worried when she sees him put down \Rystessakhin. 
\hr{Nadirs get worse}{His Nadirs are getting worse} every time. 
Something must be profoundly wrong with the Shroud. 
She has tried to pry out of him what is wrong, but he is not talking much. 
When he puts down the glaive he stands tall and arrogant and does not let anything show. 
But \Criseis{} has served him for ten thousand years and knows him better than any other. 
She can detect all the little telltale signs: 
the way he hesitates for the briefest moment; the way he stands up a few millimetres taller afterwards when he no longer carries the glaive's burden. 
It all hints of a terrible, excuciating pain that would be unendurable to any lesser person\prikker even a lesser \dragon. 

She feels his pain. 
It is unfair, she thinks, that he has to carry this terrible burden and danger, and get nothing but scorn in return from the world. 
And she fears for the future. 
\hr{Nadirs get worse}{The Nadirs are getting steadily worse}. 
What if some day the burden gets so heavy that even he can no longer shoulder it? 

Before his great duel, \Ishnaruchaefir{} draws some energy from \Rystessakhin \dash as much as he dares\dash and uses it to empower his \hs{ward runes}. 
(Remember to read about \hs{ward runes}.)

Maybe \Criseis does not see \Ishnaruchaefir up close. 
She fears to get close to him.
He is radiating powerful and dangerous energy.
A lash with one of those whirling energy threads might kill her.
So she stays far away.
She contacts him with telepathy. 
(This way, the reader also only sees \Ishnaruchaefir as a blur.)

\Criseis warns him:

\begin{prose}
  \Criseis:
  \ta{\Teshrial{} is well-prepared this time. 
    He has studied your strengths and weaknesses.
    I believe he has studied \WanderersInDarknessEmph.}
  
  \Ishnaruchaefir:
  \ta{Has he now?}
  (Smug, mysterious smile.
  He knows \Teshrial{} has taken the bait and fallen for the story of his alleged Achilles Heel.)
  
  \Criseis:
  \ta{Do not do this, master!
    I beg you.
    It is obviously a trap.
    And you are at your weakest.
    You may fall!}
  
  \Ishnaruchaefir{} (looking wistful, contemplating the possibility that he might die):
  \ta{I might.
    But know this, \Criseis:
    This battle will be of pivotal importance.
    This I predict.
    A mighty storm is brewing on the Pelidorian horizon.
    This storm will herald a \thirdbanewar.
    And that is a war I intend for my race to win.
    It is a risk, but I am willing to take it.
    For \Nexagglachel.
    For our people.
    Even\prikker even for \Secherdamon, perhaps.}
\end{prose}





\subsubsection{\Secherdamon{} thinks}
\target{Secherdamon thinks Ishnaruchaefir will sacrifice himself}
Shortly before \ps{\Ishnaruchaefir} duel is scheduled, have a scene with \Secherdamon. 
He ponders his brother's involvement, and what \Ishnaruchaefir{} intends. 
Then he realizes what is about to happen. 

\begin{prose}
  \Secherdamon;
  \tho{Fuck.
    \Ishnaruchaefir{} is going into battle.
    Now.
    Right in the middle of his deep Nadir.
    (\Secherdamon{} knows about the Nadir\dash superficially, at least.)
    
    And he is walking straight into \ps{\Teshrial} trap!
    Surely he must realize this. 
    What is he doing?
    He might actually be in danger!
    
    Why is he doing this?
    And he is helping me.
    Why?
    
    Does he intend, at last, to sacrifice his life to pay for his crimes?
    That cannot be.}
\end{prose}

For a short moment, this line of thought makes \Secherdamon{} respect \Ishnaruchaefir{}. 
He thinks and visualizes \ps{\Ishnaruchaefir} name, something he has not done for many centuries. 












\subsection{\Forclin falls}
The war lasts about a year. Eventually, Runger escalates the war by bringing in more and more mages and supernatural aid. 
The Imetrians begin to suspect that Runger has allied with Durcac and gained the aid of Rissitic mages and troops (which is true). 
The Imetrians are not certain, but these suspicions sway them to finally send some aid to Pelidor. 

The mages draw so much magical power that they \hr{Magic overdose}{overdose and damage their bodies}. 
Including Carzain and Curwen. 
And worst of all the Rungeran mages (all but \Takestsha). 





\subsubsection{Push comes to shove}
The Pelidorians are holed up in \Forclin. 
It is a strong city, and they believe they can hold off the Rungerans for long enough to hopefully summon allies from neighbouring kingdoms, or perhaps the Imetrium. 
As long as they can keep the Rungerans out, they have a chance. 

\Takestsha{} was also content to keep the siege up for a while. 
But then she receives news from \LocarPsyrex. 
He tells her that she will have to speed up her plans. 
She curses, but agrees. 
So instead of just waiting and holding the city besieged, she now forces Morgan to go all-out on the offensive and make every effort to take the city. 





\subsubsection{Rungeran reinforcements}
\Takestsha{} has lost a few mages, so she calls in some more \quo{recruits} to the \ishrah. 
These are actually Rissitic mages, disguised as rogues. 
Morgan knows this, but the rest of the \ishrah{} and army doesn't, although they may suspect. 

They have also brought on some companies of mercenaries, which are Rissitic soldiers or \Durcaci{} tribesmen in disguise. 
These mercenaries bring in big, frightening monsters: 
Stegosaurs or ankylosaurs or sauropods. 
And a few of the warriors are \cregorrs!
Those guys are fucking wicked. 

Ilcas observes the new recruits to the Rungeran army. 
No one knows the mages are new (and they don't cast Rissitic magic, so they are difficult to pinpoint as being odd-men-out). 
But scouts have seen the new mercenary bands arrive. 
They are also noticeably different from the rest. 
It is no secret that they are not native Rungerans. 
Ilcas suspects that they are \Durcaci{} tribesmen. 
And if they are \Durcaci, then it is likely that they are also Rissitic. 
And this fits nicely with the theory that the Rungerans are allied with the Rissitics. 
Ilcas goes to tell Sethgal about it. 
He also does his best to send a message to the Imetrium and inform them that there is almost definitely Rissitic involvement. 

This is true, of course. 
The tribesmen \emph{are} Rissitic, or at least Rissitic-allied. 
They were only brought to the battlefield at this late hour, because \Takestsha{} did not want everyone to know that Runger was in league with \Durcac{} too soon. 





\subsubsection{Carzain}
Near the end of the war, Carzain is stationed in \Forclin{}, a strategically important city in the northern Pelidor. There is a lengthy siege and the Rungerans bring in Rissitic reinforcements. 

Also, Redcor \Matron{} \Esmerel{} is there. \Esmerel{} suspects that a Scion has been incarnated and dwells in Pelidor. The Redcor have determined this via astrology or something. \Esmerel{} comes to investigate, together with \Racel{} and two Gandierre, \France{} \Perival{} and \Isacc{} \Chiran. \Racel{} is there because she is Pelidorian\prikker I think. 
\Esmerel{} investigates and ultimately determines that Carzain is the Scion. 

They fight bravely, but ultimately the city falls.





\subsubsection{Carzain must be awesome}
Carzain should be brave and iron-willed in the face of adversity, like Lucian from \cite{Movie:UnderworldIII}. 
Maybe he should take over leadership of the Pelidorian army after Sethgal is killed. 
They still lose the war, but he saves many of them. 
Or something.





\subsubsection{Carzain-tachi fight for survival}
After the fall of Forklin, Carzain ends up together with Telcastora Ilcas and a few other Imetrians. 
Carzain also has a few of his soldier buddies along. 
Then they meet up with \Esmerel{} and her Redcor. 
They slink around and try not to be killed by the Rungeran forces, but they are forced into a fight, and several of them are killed. 
Only Carzain, Ilcas, \Esmerel{} and \Racel{} survive. 
They slink around some more and realize that \Forclin{} is lost. 





\subsubsection{Carzain and Delph discuss archetypes}
Carzain says to Delph: 
\ta{Y'know, this is almost like a scene from the legends.
  With the heroes standing in the rubble of the fallen city and swearing revenge and all.}

Delph:
\ta{I hope we're not in a legend, 'cause if we are, you're doomed.
  'Cause I'm the hero, and you're the hero's faithful companion. 
  And the companion always dies before the end.}

At the end, he is dying and thinks: 
\tho{I can't die.
  I'm not supposed to die.
  Damn you, Carzain. 
  You're the sidekick. 
  You were supposed to die, not me.}





\subsubsection{\Tsekkect{} runs away}
\Tsekkect{} is alone. 
Delph is dead, and she has been separated from Carzain-tachi. 
She thinks:
\tho{Fuck it. I've had it with these stupid nobles and their stupid war.
  I'm going back to my tribe.
  They'll take me back.
  They have to.
  They will.
  I hope.}

This is the last we ever see of her. 
I don't need her, I just want to avert the \trope{SortingAlgorithmOfMortality}{Sorting Algorithm of Mortality} by not killing the nonhuman companion. 





\subsubsection{Carzain-tachi encounters \vorcanths}
\target{Vorcanth help Ramiel}
Have one very mysterious scene where Carzain dreams. 
His traumas and \hr{Ramiel's bound souls}{the bound souls that haunt him} take on physical shape and attack him. 
But \hr{Moon-Wolves help Ramiel in dreams}{white wolves appear and dispel them}, saving him. 
Only to disappear again, casting enigmatic glances at him. 
He senses that he knows them, and feels conflicting feelings of loss, guilt, joy and relief.

When Carzain first sees them, he compares them to wolves. But Vizicar says hyaenas. Carzain hasn't seen a hyaena, but he recognizes them from Vizicar's memories. (Vizicar is more well-\travelled, since he is not only older, but also a king.)

Already in \TwilightAngelRememberEmph, Ramiel is receiving dreams about \vorcanths. 
It is a \hr{Carzain dreams of Moon-Wolves}{distress call from a \vorcanth{} in trouble}. 

Remember to read the section about \hr{Vorcanth}{\vorcanths} before writing the chapter!

Immediately after this scene with the \vorcanth{}, have a scene where some astrologer reads the stars and notices that Visha (associated with the \hr{Vorcanth Matrix}{\vorcanth{} \matrix}) is \hs{intersecting} with the Midnight Bat (the \hs{Mystraacht Matrix}{\Mystraacht{} \matrix}). 
Or maybe this doesn't happen until later, where Ramiel is contacted by a badass \vorcanth{} leader. 

It is important to stress that it is the \vorcanths{} who take the initiative and help Ramiel out first. After that, \emph{he owes them}. 









\subsection{\Teshrial prepares to battle \Ishnaruchaefir}
\Teshrial{} suspects that \Ishnaruchaefir{} will soon return. 
He prepares himself to face him. 

\Teshrial{} is laying a trap for \Ishnaruchaefir. 
He plans to meet him in a place parallel to \Malcur, but farther away from the Shroud (perhaps through the \hs{dead garden} again). 
He has his \noggyaleth{} lying in wait and intends to call them out to ambush \Ishnaruchaefir{} while fighting him. 

\Teshrial{} has several trump cards:
\begin{itemize}
  \item The Shroud.
  \item Astrology. 
  \item \Noggyaleth.
  \item The Achilles Heel. 
\end{itemize}





\subsubsection{\Teshrial refuses ambush}
\target{Teshrial fears to break agreement with Ishnaruchaefir}
The other \resphain tell \Teshrial that they should set a bigger ambush for \Ishnaruchaefir, have some more \resphain lie and wait and attack him. 
\Teshrial very firmly declines and explains his reasons.

\Ishnaruchaefir has promised to fight \Teshrial alone.
If \Teshrial breaks their agreement, \Ishnaruchaefir will refuse to fight and exact a terrible vengeance later. 
They now have a chance to exploit the Destroyer's code of \honour. 
They would be fools to let that go to waste. 

Instead, several powerful \resphain give \Teshrial gifts of magical items for him to use in the battle, including a sword, pistols, \armour, amulets and wing braces.





\subsubsection{\Achsah contacts \Teshrial}
\Achsah contacts \Teshrial. 
She tells him about the situation in \Forclin.
He only listens half-heartedly.
She can tell that he is obsessed with his upcoming duel. 

\Achsah fears for him. 
She tries to dissuade him. 
\Teshrial brushes her off and berates her for her cowardice.

Afterwards, \Teshrial is a bit ashamed. 
He should treat \Achsah better, he reflects. 
She is worried about him.
He should respect that.
He resolves that after the battle he will try to be nicer to his \bezed subordinates.
This is a part of \hr{Teshrial's unfinished business}{\Teshrial's unfinished business}. 




\subsubsection{His brethren fear for him}
Make it clear \ps{\Teshrial} brethren fear for him and want him to succeed. 
\Menessiaraid, his close friend, comes to see him off. 

\Menessiaraid{} gives \Teshrial{} a gift: 
His own \senaan, \hr{Ossiraith}{\Ossiraith}. 
He knows \ps{\Teshrial} favourite weapon, \hr{Turishah}{\Turishah}, was destroyed. 
He entrusts his friend with \Ossiraith. 
\Teshrial{} is \honoured and vows to put it to good use. 

In addition to \Ossiraith, \Teshrial{} carries into battle two \hr{Ghijed}{\ghijedeth}. 
Before going into battle he looks down at his pistols. 
He takes them with him just in case, but he does not expect so much from them. 
\Ishnaruchaefir{} will probably wear \hs{ward runes}, and as such will be protected against guns. 
Guns are not so great in single combat with \dragons{} overall.
They are good if you can outnumber or out\manoeuvre your enemy. 
But in a fight like this, \melee{} is the best way to go. 

\Teshrial dresses for the battle so as to match \Ossiraith. 
Read about \hr{Ossiraith}{\Ossiraith} and dress \Teshrial in the same \colour. 

\Urizeth also fears for him.
She is sad that she will be nowhere near him during the battle.
She feels a bit cowardly for it.
\Teshrial is also warming towards her.
This is part of his personal development and \hr{Teshrial's unfinished business}{his unfinished business}. 





\subsubsection{Stewardship of \Malcur}
\target{Teshrial leaves Bezed in charge}
\Teshrial{} knows that \Malcur is terribly important and that it would be bad if it fell into the Sentinels' hands. 
He must be careful and guard it against \Psyrex-tachi while he himself is off fighting. 
He knows the time of his duel is coming up, so he has contacted his superiors and requested to have more aid sent to him. 
Some more \resphain{} come to relieve him. 

When he feels that \Ishnaruchaefir{} is near, he takes off and leaves some low-ranking \resphain{} in charge of \Malcur. 
At this point \Achsah{} is already in \Forclin. 

\Teshrial{} thinks he is manipulating \Ishnaruchaefir{} and forcing him to face him at the right astrological moment. 
But he is actually being strung along. 

\lyricslimbonicart{In Mourning Mystique}{
  Darkness!\\
  I seek the silence that you bring.
  
  Grant me thy sacred gifts,\\
  bestow my soul thy offerings.\\
  Let the ancient forces of nature rule.\\
  Take my blood as the sacrifice,\\
  a symbolic faithful bond of truth.\\
  I kneel in front of thy altar black.
  
  When you look into an abyss, the abyss also looks into you.
  
  Tonight I enter into obscure dreams.\\
  In darkness shelter, I am unseen.\\
  With the esoteric gifts I possess\\
  I bring damnation by enforcing death.\\
  In the beginning of the storm \\
  I'll come forth.
  
  An arrival into a twilight reverb\\
  as just a shadow of the former self.
  Sorrow is my name.\\
  My true essence is pain
  
  Hear the mourning of the mendacious\\
  from the empty halls and shafts\\
  of false blinding light.\\
  Prepare the last sacrifice (on the altar)\\
  in the temple of decay.
  
  Please spare me from the final agony of shame.
  
  I am evil from the moment of conception.
}





\subsubsection{\Achsah{} wishes him well}
As \Teshrial{} departs, he contacts \Achsah{} telepathically and briefs her. 
She expresses her fears and concern for him. 
I need to paint some measure of friendship between the two. 
She does not like him, but she does not want him to die a painful death, or mutate permanently into a monster. 
She asks him to be careful and wishes him luck. 
He thanks her. 

This is small measure of making-up for them. 
They have never liked one another.
Here at the last minute, they take steps toward each other. 
It should be a bit touching, and tragic in retrospect. 





\subsubsection{\Firaxel{} visits him}
\target{Teshrial's date}
\Teshrial{} is visited by \Firaxel. 
He has specifically asked her to come see him off to his duel. 
He wants to see her one more time before he risks his life and soul. 
She is attracted to him and wants to see him, too. 
They have an intimate, romantic date. 

\hr{Firaxel is a scientist}{\Firaxel{} is a scientist}. 
He learns that she greatly admires people who do something to advance science, especially if they take great risks to do so. 

\begin{prose}
  \Firaxel: 
  \ta{Our race exists for a purpose: 
    To improve ourselves, to bring ourselves closer to perfection.
    And our best and finest tool in this quest is science.}
\end{prose}

\Teshrial{} wants to impress her. 
So he tells her of how he has spoken with cutting-edge researchers and is participating in the \neoresphan{} project and testing a dangerous experimental weapon. 
It is dangerous, but he does it. 
So he is fighting and risking his life not only to prove himself and for the war and rid the world of an evil menace and avenge all the victims of the wicked \Ishnaruchaefir, but also for science and the betterment of their people. 

This turns her on. 
He can see it. 
Her eyes glow with desire. 
Her breasts heave. 
She licks her lips with hunger. 

They \emph{almost} have sex. 
They certainly kiss and grope. 
She promises him sex (in unmistakable terms) when he returns triumphant. 
The best sex of his life. 
And he can believe it. 
\Resviel{} are great lovers. 

He walks away happy. 
His dreams are coming true. 
Seeing her has given him courage. 
Before he was unsure if he dared use the experimental weapon. 
He was planning on holding it back as a desperate last resort, and even so he was unsure if he would have the courage to use it. 
But now he knows how much \Firaxel{} will love him for it. 
Now he dares use it. 
For \ps{\Firaxel} sake he is willing to take that risk. 

Have some from \ps{\Firaxel} POV. 
\Firaxel{} has developed genuine feelings for him and is afraid for him. 
She admires him. 

\begin{prose}
  \tho{What a \resphan. 
    What a brave, noble, beautiful soul. 
  
    But my heart aches to see you go. 
    I fear to let you do this. 
    Silly warrior \resphan. 
    Why must you risk your life and your very soul? 
    Your bravery alone was enough to bring me into his bed. 
    I can't wait to have sex with you. 
    
    But what if you fail?
    What if you die? 
    I might never see you again. 
    This might be our last moment together.
    What then? 
    
    I want to have sex with you here and now.
    If only I could. 
    But I can't. 
    I have a duty. 
    I cannot give myself to anyone less than a true hero. 
    Otherwise I would be a slut.
    I have no choice but to wait.
    
    Damn all these rules.
    
    Come back safely to me.
    My brave, beautiful hero. 
    My sweet, beloved prince.}
\end{prose}






\subsubsection{He is nervous}
\Teshrial{} is nervous and afraid to enter his battle. 
After all, he is just a \ketheran{} facing a terrible \dragon. 
But he thinks of \ps{\Firaxel} kiss. 
And her sexiness, her hot and lusty eyes, and her soft, whispered promises of erotic bliss to come. 
This gives him courage and strength. 
It is \trope{ThePowerOfLove}{The Power of Love}. 

There is also a little bit of unexpected \trope{ThePowerOfFriendship}{Power of Friendship} from \Achsah. 

Thinking of these bright things, \Teshrial{} takes heart. 
He rights himself and is now ready to face \Ishnaruchaefir. 









\subsection{The Power of \EreshKal}
In this section, \Takestsha storms \Forclin. 




\subsubsection{Curwen goes to the Ghost Tower}
\target{Charcoal at the Ghost Tower}
Throughout much of the story, Archibald Curwen (Charcoal), supposedly the sneaky master Cabalist, is duped, manipulated and played for a fool by his enemies. Sentinels and other agents seem to run circles around him. 
He's been played for a \trope{XanatosSucker}{Xanatos Sucker} the whole book. 
But at the end he finally realizes what's going on around him and strikes back. 

Near the end\dash perhaps after having discovered one or more of the people who have been cheating him, such as \Sanyor{}\dash Charcoal shows what a formidable agent he truly is. 

Curwen realizes that \Takestsha and her \ishrah are too powerful and dangerous.
He cannot just use conventional tactics against them.
Instead, he gets an idea. 
He devises a master plan to dispel the Rungerans' \EreshKali magic and strike a hard blow to their forces. 
But knows that will be hard. 
He cannot do it without help. 

So he gets another idea. 
He can use the Ghost Tower.

Curwen has been in \Forclin before. 
He knows the Ghost Tower and has even been inside it. 
He knows it is a conduit to the Realm where the \resphain live (though he has not been there). 
He formulates a plan to use the Tower in his counterspell against \Takestsha by channelling energy through the Tower's \nexus, or something like that. 

So he leaves Carzain in charge of the \ishrah and departs the battlefield, heading for the Ghost Tower. 





\subsubsection{Curwen contacts \Achsah}
On the way to the Tower, Curwen recites an orison to contact \Achsah and ask for her advice. 

\begin{prose}
  \Achsah:
  \ta{What? Do you have urgent news about the Sentinels?}
  
  Curwen:
  \ta{Well\prikker no, my Lady \Resvil. But\prikker}
  
  \Achsah:
  \ta{Then do not pester me.
    Figure it out yourself.
    I am busy.}
\end{prose}

\Achsah is unwilling to help. 
She has her hands full. 
She is stressed. 
She is sure the Sentinels are up to something really nasty here.
She tries to figure out what it is. 
She has no time to advise Charcoal. 
He is on his own.
She is sure he can solve his own problems. 
He is a skilled mage and a high-circle Cabalist. 





\subsubsection{\Achsah suspects \Takestsha}
\Achsah, who is in \Forclin, looks at the \quo{\EreshKali} spells cast by \Takestsha-tachi. 
\Achsah{} wonders when she first observes the \EreshKali{} magic. 
It does not feel like anything she would expect them to have. 
It also does not feel like what she would expect ancient \meccara{} to have. 
It is new to her, and it makes her suspicious. 
But what it \emph{does} smell like is Rissitic magic. 
That makes her even more suspicious. 
(She has heard Charcoal's account of Tantor's diary, but Charcoal has never seen Rissitic magic, so he cannot draw the connection.) 

Then she realizes that those spells are actually meant to tear the Shroud and reach into the Beyond, where it can summon\prikker stuff. 
And the Ghost Tower, which is in close promixity now that the Rungerans have breached \Forclin, acts as a catalyst. 
Those spells are tearing at the very fabric of the Shroud. 
Something fucking nasty is breaking through, or so \Achsah thinks. 
In reality the summoning spell at \Forclin is a smokescreen. 
It is meant to warp the Shroud and look big and impressive, but it doesn't actually \emph{do} anything. 
It's just meant to attract attention and convince everyone that the real stuff is happening in \Forclin, near the Ghost Tower. 

\Achsah now strongly suspects that \Malcur is a decoy and that the Sentinels' real goal is \Forclin. 

But still she keeps watching.
She does not intervene. 
Partially because she must remain ready and keep a bird's eye view of the action and cannot afford to commit herself to any narrow battlefield action.
Partially because of the Unspoken Covenant.
She will not be the first to break it. 





\subsubsection{\Takestsha attacks}
\Takestsha and her mages have to start their big attack spell. 
\Takestsha knows it is risky. 
Her mages are not holding up as well as they should. 
They are weaker than she had hoped. 

If it were up to her, \Takestsha would be patient and lay a prolonged siege. 
But \hr{Psyrex tells Nzessuacrith to capture Forklin quickly}{\Secherdamon and \Psyrex have asked her to make haste}. 

\target{Takestsha will not become Nzessuacrith too soon}
\Takestsha \emph{knows} what \Secherdamon's plan is.
She knows her own attack is a decoy.
Her mission is to attract the \resphain's attention and fool them into coming to fight her. 
Conquering \Forclin is just a means to that end. 
She can, of course, assume \draconian form right away.
But she will not break the Unspoken Covenant for no reason.
She will first go as far as she can in her \human guise.
Only when she absolutely has to will she break her disguise.
If she were to take \draconian form too early, it would be suspicious, and the \resphain might not be fooled. 
It must look like she was forced to unveil herself. 

So she has to act in haste. 
She decides she must take some risks she would otherwise not have taken. 

So she and her \ishrah begin their great spell that will bring down \Forclin. 
It may be foiled, but \Takestsha hopes it will work. 

\Takestsha must push the Rungeran mages really hard. 
Push them to their limit and beyond it. 
Her master spell is a colossus on feet of clay, and she knows it. 

\citeauthorbook[p.248]{RobertEHoward:HouroftheDragon}{Robert E. Howard}{%
  Hour of the Dragon%
}{
  He glanced up at the sky, and he glanced down at the slim white figure [of a captive virgin girl] on the dark stone. And lifting a dagger inlaid with archaic hieroglyphs, he intoned an immemorial invocation:

  \ta{%
    Set, god of darkness, scaly lord of the shadows, by the blood of a virgin and the sevenfold symbol I call to your sons below the black earth! Children of the deeps, below the red earth, under the black earth, awaken and shake your awful manes! Let the hills rock and the stones topple upon my enemies! Let the sky grow dark above them, the earth unstable beneath their feet! Let a wind from the deep black earth curl up beneath their feet, and blacken and shrivel them\dash}
}





\subsubsection{Curwen begins his counterspell}
Archibald Curwen is at the Ghost Tower. 

When Curwen enters the Ghost Tower, he finds that it is much larger on the inside than on the outside.
Curwen knows this is a Shroud phenomenon.
In the city, the repressive Shroud twists the mind and the eye and makes the tower look small, and takes a man along paths that make the tower look small.
But in here, the Shroud is weaker, so the true extent of the Tower reveals itself to him.
Or something like that.

Have a \quo{moon-shrouded crystal} or the like inside the Tower. 

\lyricsbs{Bal-Sagoth}{%
  Enthroned in the Temple of the Serpent Kings%
}{
  Deep within the glacial, ice-veiled temple,\\
  ancient enchantments summon the shades of the dreaming Serpent Kings,\\
  and the Ophidian Throne once again draws power from the Moon-shrouded crystal.
}

He begins his counterspell. 

\lyricslimbonicart{Solace of the Shadows}{
  I set the stones for invoking ceremonies.\\
  In the twilight zones arise abstract galaxies.\\
  The magic eye unveils the blackened skies.\\
  A new horizon begins to each one that dies.
}

He is frightened by the magnitude of the powers he unleashes. 
But also thrilled, exhilarated.

\lyricslimbonicart{Solace of the Shadows}{
  The desolation makes me feel\\
  so dark, so cold, the silence.\\
  So dark, so cold, the emptiness.\\
  Solace of the shadows.
  
  Night surrounds and embraces me.\\
  Darkness holds the secrets of man's fears.\\
  It captures my heart as the purgatory sears.\\
  I cast now the spell, as I cross through raging flames,\\
  into darkness, cursing names.
}





\subsubsection{Ilcas-tachi attack the \ishrah}
\target{Ilcas-tachi attack the Rungeran Ishrah}
Sethgal and his forces are strained to the utmost.
He has to hold back the mundane Rungeran army, which is enough of a problem already.
It is twice as large as his own, and the walls of \Forclin are crumbling under the Rungeran cannonade. 

Meanwhile, Telcastora Ilcas leads his Imetrians in a brave attack against the Rungeran \ishrah.
This is something Sethgal has asked them to do. 
The Imetrian mage, Ulphon, \hr{Ulphon Nestor dies}{has been killed}, so Ilcas asks Carzain to cover them. 
He does. 

\target{Ilcas injures Takestsha}
The \ishrah have massive ranks of soldiers protecting them.
But the Imetrians are fearsome fighters, and Carzain is a badass mage.
They break through and kill several mages. 
Ilcas and his \nycans even manage to seriously injure \Takestsha. 
She would have died if she were an ordinary mortal. 

This attack is won primarily by the Imetrians. 
Emphasize the courage and superhuman skill of Ilcas and his \nycans. 
They are awesome forces of destruction. 
Carzain also fights well, but he has a secondary role. 
Carzain gives them some artillery support and protects them against enemy magic, but he does not fight in \melee himself.
Carzain does not kill much.
He is mostly a distraction. 
This is the Imetrians' hour of triumph. 
Carzain's moment of glory comes later when he \hr{Carzain fights Takestsha alone}{fights \Takestsha alone}. 

\Takestsha knows this is bad.
Her spell is strained as it is.
If these interlopers kill too many of her mages, her spell will certainly fail. 

So she diverts her attention from the spell and unleashes some nasty spells against the attackers.
As nasty as she can make them in her current state. 
(She is in a weakened humanoid form, deep in the Shroud, and she is a bit exhausted, and she is stressed because she has so many things on her mind and must maintain so big and complex a spell.)

She kills several Imetrians and forces the rest to retreat.
Then her soldiers are able to close their ranks.
Carzain-tachi are overwhelmed.
They have no chance but to flee to save their own hides. 





\subsubsection{The \EreshKali magic backfires}
\target{Eresh-Kali magic backfires}
Curwen is working on his counterspell.
When Carzain-tachi attack \Takestsha, he sees the opening he needs. 
He strikes with the full force of his counterspell. 

The spell catches \Takestsha-tachi at the worst possible time. 
The Rungeran \ishrah is reduced. 
Several mages have been killed in the \hr{Ilcas-tachi attack the Rungeran Ishrah}{Imetric attack}. 
The remaining ones cannot endure all the stress and strain.
The \EreshKali spells backfire on them. 
This kills the entire Rungeran \ishrah{}.
Only \Takestsha survives, and she is badly wounded. 

This buys the Cabalists time to send in reinforcements, including \banes{} and perhaps \resphain, which forces \Nzessuacrith{} to breach the \charade{} and assume her \draconic{} form to fight them off. 

By now \Forclin{} is pretty doomed, but now \Nzessuacrith{} is weakened enough for \Achsah{}, her fellow \resphain{} and their \hr{Umbra}{\umbrae} to have a fighting chance against her. 





\subsubsection{Curwen dies}
\target{Curwen dies}
\Takestsha is not pleased to be thus thwarted. 
In the midst of the destruction, she reaches out and strikes back through the counterspell.
She grabs hold of Curwen's spells and twist them against him.
She kills Curwen.

He fought well and bravely.
He made a difference.
But he was just a mortal against an immortal, so he paid for that difference with his life. 









\subsection[Malcur must be a decoy]{\Malcur must be a decoy}
The Cabalists suspect that the Sentinels (specifically, \Secherdamon) are trying to open some kind of portal some place in Pelidor, but they don't know exactly where. 

The Cabalists believe the ploy is to create a \emph{portal} or \emph{conduit} to \Nithdornazsh{} or something like that. 
They don't imagine that the Sentinels actually intend to \emph{resurrect} \Nithdornazsh{} and bring the entire city to \Azmith, straight into the deep Shroud. 

At first, the Cabalists suspect \Malcur. 
But \ps{\Takestsha} desperate, rushed attack against \Forclin{} convinces them that the target is \Forclin{} and the Ghost Tower, and that \Malcur was a decoy. 
\Secherdamon{} is known for very elaborate decoys like this. 
\Achsah, who has long \hr{Achsah suspects that Malcur is a decoy}{suspected that \Malcur was a decoy}, is the first to believe the \Forclin{} ruse. 





\subsubsection{\Nzessuacrith shows her true \colours}
\Takestsha was badly shocked when \hr{Eresh-Kali magic backfires}{Curwen's counterspell struck} and caused her \quo{\EreshKali} master-spell to backfire. 
She had to suddenly use a lot of powerful magic just to keep alive. 
(Her \human body is much weaker than her \draconian body, and \hr{Ilcas injures Takestsha}{she was wounded by Telcastora Ilcas}.) 

In her haste, she accidentally lets her stealth slip.
Tremours of her true \vertex signature spill forth. 

\target{Takestsha bleeds power}
\Takestsha realizes her cover is blown. 
But she is also hurt. 
She is leaking arcane blood so badly that even Shrouded humanoids can see it.
She is trailing tendrils of power, and she bleeds blood that sizzles and burns and evaporates. 
Her own Rungeran soldiers flee screaming from her path.

\target{Takestsha retreats to heal}
\Takestsha has to retreat. 
She must take a little while to heal. 
But she promises that after some minutes of rest, she will back.
And this time she will hold nothing back.
The \resphain must surely know by now that she is here, so there is no point in playing any more games of deception.
As soon as she is fresh, she will break her humanoid guise and attack in her true form. 

\begin{prose}
  \Takestsha: 
  \ta{Mark my words. 
    I will not be thwarted by mere mortals.
    \Forclin will fall this day by my hand.
    Or my claw, if need be.}
\end{prose}






\subsubsection{\Achsah detects \Nzessuacrith}
\target{Achsah calls for help}
\Achsah detects it when \Nzessuacrith lets her stealth slip. 
She smells a \ps{\dragon}{} presence. 

It is a frightening \trope{CosmicHorror}{Cosmic Horror} revelation for \Achsah when she perceives and recognizes \Nzessuacrith.
\Nzessuacrith may not be \Iscrafel, but she is great a great and terrible \dragon. 

Now \Achsah{} is really frightened. 
She realizes that she can't handle \Nzessuacrith{} on her own. 
Moreover, if a \dragon{} is here, then it must be \trope{SeriousBusiness}{Serious Business}. 
And it is well known that \Nzessuacrith{} works for \Secherdamon, so \Achsah{} is convinced that she is the spearhead of his plan. 

She also becomes convinced that this attack has been deliberately timed to coincide with \ps{\Teshrial} duel, so everyone's eyes would be fixed on \Malcur. 
So \Secherdamon{} must have learned of the duel and is using it as a smokescreen. 
No one in the Cabal imagines that \Ishnaruchaefir{} is \emph{actively} and directly helping \Secherdamon, because he hasn't been doing that for millennia now. 






\subsubsection{\Achsah calls for help}
\Achsah{} needs help. 
An average \dragon is \hr{Dragons vs Resphain in power}{a terrible opponent}, even \hr{Umbra power}{if you have \umbrae} on your side.
And \Nzessuacrith is far more than an average \dragon. 
\Achsah will need \emph{many} reinforcements if she is to take down \Nzessuacrith. 
She contacts her fellow \resphain who are working on the \hr{Malcur venture}{\Malcur venture} and requisitions reinforcements. 

The other \resphain are not happy about that, because they would like to keep some in reserve for \Ishnaruchaefir.
But Achsah pleads her case, and \Teshrial agrees with her (because he \hr{Teshrial fears to break agreement with Ishnaruchaefir}{fears to break \Ishnaruchaefir's agreement}).
(This happens before \Teshrial's duel has started.)

So a dozen \resphain hurry to \Forclin to fight off \Nzessuacrith. 
This includes the \resphain{} whom \Teshrial{} \hr{Teshrial leaves Bezed in charge}{left in charge of \Malcur}. 
There are now only a few left to guard \Malcur. 

This is distressing and hurtful to the Cabal's plans because they had not planned for such an eventuality. 
Like \humans, \resphain can be shortsighted.
\Dragon attacks \emph{almost} never happen.
Therefore, many \resphain tend to assume \dragon attacks will \emph{never} happen, and hence they will not plan for them. 
\Nzessuacrith's appearance is a blatant breach of the Unspoken Covenant which none could have foreseen.

Also remember that the faction taking care of Pelidorian business is small. 
There are not many \resphain there, so they are not well-equipped to deal with such fierce \dragon attacks.

Only one \resphan{} now remains to guard \Malcur: \Paerzim. 
\hr{Psyrex kills Paerzim}{That ends badly}. 





\subsubsection{\Achsah fetches \umbrae}
\target{Achsah fetches Umbrae}
When \Achsah{} has realized \Takestsha{} is a \dragon{} (and perhaps even the mighty \Nzessuacrith), she reckons she still has a few hours of time before the shit hits the fan. 

So she calls on Charcoal (who is in \Forclin) and gives him responsiblity for keeping control of \Forclin{} for some hours until reinforcements arrive from \Malcur. 

\Achsah{} then quickly submerges and goes to \Nyx{} to fetch one or more \umbrae. 
She believes she and her companions will need them if they are to confront the \dragon\dash especially if it really is \Nzessuacrith, as she fears. 

She has asked \Teshrial{} to send \emph{one} \resphan{} directly to \Forclin{} (to relieve poor Charcoal) and send the others to \Nyx{} to rendezvous with her there. 

Remember to have a cool, dark, evocative, mystic scene where they summon the \umbrae{} from the endless dark deep. 

\Achsah{} is in awe and fear when the \umbra{} shows up. 
She is old enough to remember the days in \Merkyrah{} when \hr{Merkyrans fear Umbrae}{the \resphain{} lived in fear of the \umbra}. 
She remembers the terror and awe she felt when \hr{Rebels awestruck by tame Umbrae}{she first saw the rebel leaders riding \umbrae}, back when she was just a rank-and-file rebel acolyte. 

Nowadays she has ridden an \umbra{} countless times. 
But she still remembers the fear of them and knows to be careful. 
Many younger \resphain{} do not know this, she reflects. 
They are overconfident around \umbrae.
They do not show the terrible monsters the respect they deserve. 
They have not lived back in the days when, at any moment, an \umbra{} might swoop in from the dark sky or the dark deep and kill and eat a half-dozen \resphain. 
(Remember, \resphain{} were weaker back then.)









\subsection[Psyrex wins]{\Psyrex wins}





\subsubsection[Psyrex-tachi invoke the ritual]{\Psyrex-tachi invoke the ritual}
\target{Psyrex-tachi invoke the ritual}
\Psyrex-tachi begin the sorcerous ritual that will summon and resurrect \Nithdornazsh. 

\Psyrex, the mad sorcerer, leads the ritual. 
Several other mages support him, and loads of servants stand by to perform menial tasks, such as procuring living prisoners to be sacrificed. 

\Psyrex rejoices. 

\citeauthorbook[p.137]{RobertEHoward:TheAltarandtheScorpion}{Robert E. Howard}{%
  The Altar and the Scorpion%
}{
  \ta{The real gods are dark and bloody!
    Remember my words when soon you lie on an ebon altar behind which broods a black shadow forever!
    Before you die you shall know the real gods, the powerful, the terrible gods, who came from forgotten worlds and lost realms of blackness.
    Who had their birth on frozen stars, and black suns brooding beyond the light of any stars!
    You shall know the brain shattering truth of that Unnamable One, to whose reality no earthly likeness may be given, but whose symbol is\dash the Black Shadow!}
    
    The girl ceased to cry, frozen, like the youth, into dazed silence.
    They sensed, behind these threats, a hideous and inhuman gulf of monstrous shadows.
}

They invoke mystic, forbidden names. 

\lyricsbs{Bal-Sagoth}{%
  As the Vortex Illumines the Crystalline Walls of Kor-Avul-Thaa%
}{
  By Klatrymadon and Zuranthus,\\
  such ancient secrets we discovered \\
  within these sinistrous, worm-worn pages,\\
  Etched with darksome glyphs and sigils, \\
  bound with fearsome spells, \\
  An eldritch tide of stygian sorceries \\
  unfettered by the forbidden Tome of Shadows\prikker
  
  Now thunderous cataclysm befalls the gleaming Kor-Avul-Thaa \\
  (The mystic gate stands open!) \\
  The Xytaxehedron held to the stars\prikker \\
  the incantation uttered with eager tongues\prikker 
}

\ps{\Ishnaruchaefir} is one of the names. 

\lyricsbalsagoth{Invocations Beyond the Outer-World Night}{
  Invocations and ideograms (dreams of the Xytaxehedron?),\\
  Conjuration of the inner world's (tenebrous) denizens,\\
  And their star-spanning progenitors, \\
  spawned beyond the outer-world night.
}

They invoke Chaos. 

\lyricsbs{Arcane Wisdom}{Symphonia Chaos}{
  Chaos, ruler of Time. \\
  Chaos, infinity is Thine. \\
  Shadowy inner essence, \\
  cosmic tapestry and sparkling \\
  spheres of a circular reason. \\
  Chaos, ruler of Time. \\
  Chaos, infinity is Thine. 
}

\lyricsbs{Emperor}{Moon Over Kara-Shehr}{
  Our time is upon us. \\
  Master! Appear! \\
  Over the nocturnal sky \\
  from the mountains of black we ride. \\
  Fly! 
  
  All thy servants fly \\
  though the serpent's darkened sky. \\
  Hears the opponent cry, \\
  ravaged by his terror. 
  
  Master! We ride with the storm \\
  in his name, the sire, wolves' king.\\ 
  Enter the power coursing \\
  through veins of the night. 
  
  Who gathers winds, summons thunder, \\
  summons rain, summons might?
}

Describe how \pdaemons{} fly screaming through the sky and blood boils up through the ground. 

\lyricsbs{Bal-Sagoth}{
  Dreaming of Atlantean Spires
}{
  The sky is black with Chaos-fiends,\\
  spellcraft rides the witch-storm's wings.\\
  Beneath the vaults of time-lost tombs\\
  sorcerers summon the shadow-kings.\\
  The Topaz Throne is beckoning,\\
  the jewelled sword awaits my grasp.\\
  The dreaming gods now grimly brood\\
  in the silence of Atlantean Spires.
}

Also, describe the mages' awe and fear as they invoke the names of gods, and then see those gods actually appear!

\lyricsbs{Bal-Sagoth}{
  As the Vortex Illumines the Crystalline Walls of Kor-Avul-Thaa
}{
  What long-shackled powers of the elder dark have our conjurings loosed?
  
  By Klatrymadon and Zuranthus, \\
  the vortex blackens the stars above,\\
  A vast plague of amorphous horrors \\
  descends to rend with fang and talon,\\
  (As with torrents of blood the crystalline walls run red?)\\
  And in the glooming chambers of our shadowed sanctum, \\
  we wait, half-mad with terror,\\
  To reap the slaughterous harvest which we have sown\prikker
  
  [The Chronicler of the Cataclysm:]\\
  And beyond the vortex, the churning black waters of the void did disgorge the Dwellers in Eternal Shadow. \\
  And upon a horde of winged horrors, brandishing swords of ebon flame, they rode out from the Gate\prikker \\
  And a terrible silence fell upon Kor-Avul-Thaa\prikker
  
  [The Echoes of the Oracle:]\\
  The sky rent asunder, \\
  black winged devils surge forth from the void\prikker\\
  A maelstrom of crimson fire burns above us\prikker \\
  what carnage has thou wrought?
  
  By Klatrymadon and Zuranthus, \\
  in Kor-Avul-Thaa, darkness reigns eternal\prikker\\
  Nevermore shall the city glimmer, \\
  for now the crystalline walls gleam black\prikker\\
  Ever black\prikker
}





\subsubsection{\Paerzim killed by \Psyrex}
\target{Psyrex kills Paerzim}
After \hr{Achsah calls for help}{\Achsah{} has summoned everyone to \Forclin}, only one measly \resphan{}, \Paerzim, remains in \Malcur. 
Needle is \ps{\Paerzim} second-in-command. 

\Paerzim{} is competent enough to give \Psyrex{} trouble. 

But then Needle is killed by Moro and Rian. 
This means \Paerzim{} has to take an active role in everything and divert his attention around. 
This is just what \Psyrex{} needs. 
He is now able to out\manoeuvre \Paerzim{} and kill him. 

Now the Cabal in \Malcur is all but destroyed, and the way for \Psyrex-tachi is opened. 
\Achsah-tachi are busy in \Forclin. 
\Teshrial{} is busy with his duel. 
None of them are in a position to come back and stop \Psyrex. 
He can freely complete his summoning ritual. 
\Nithdornazsh{} will rise. 





\subsubsection{The Cabal are thwarted}
With Needle dead, the Cabal raid is thrown into disarray. 
\Psyrex{} finds their trail and figures out their moves. 
When the Cabalists finally get their act together, \Psyrex{} has his men and magic prepared for them. 
The Cabalists fight bravely and sneakily, but \Psyrex{} just manages to stitch together enough makeshift defenses to fend them off. 

\Achsah, who might be able to put a stop to the Sentinels' dastardly plan, is up near the Ghost Tower. \ps{\Secherdamon} clever plan is working. 

Just before all Hell breaks loose\dash literally!\dash\Psyrex{} appears before Rian and Moro. He congratulates them for their hard work and their success, and thanks them for the help. If not for them, he says, Needle-tachi would have taken him by surprise, he and his companions would have been slain and their gambit would have crumbled. 

%Do Rian and Moro die? I think Rian at least dies. Moro is cool enough that she might escape, but she might also die. If she dies, I might want to portray her as a bitchy Aes Sedai who bullies Rian, so the reader is not sad to see her die. 

%I need to portray Rian as semi-cool but not so cool that the audience will be outraged when I kill him. And I need to be very sure that \Psyrex{} is portrayed as kickass, so people will like it when he wins. 





\subsubsection{\Tiroco{} feels the end is near}
\Tiroco{} feels the end is near. 

\lyricsbs{Marduk}{
  Cold Mouth Prayer
}{
  Quickened lumps of Earth,\\
  A feast for fowls and greedy worms.\\
  Count your sins, the Snakes within.\\
  An eyeless leap into the Bosom of Decay.\\
  Cold Mouth Prayer.\\
  Cold Mouth Prayer.
  
  A flash, a minute, a winter's dust.\\
  A choir of fingers sings of putrefaction.\\
  Cold Mouth Prayer. Cold Mouth Prayer.\\
  A smile left to rot in the Sun of Despair.
}









\subsection{Rian finds Neina}





\subsubsection{Rian chooses to save the girl}
Rian has the opportunity to stave off \Malcur's fall for a little while and save many more lives, but that would seriously endanger Neina. 
Rian is a dumbass and chooses to \trope{AlwaysSaveTheGirl}{Always Save the Girl}. 
Moro berates him for it to his face, and later (after he has gone) curses him in retrospect for being selfish and choosing what was obviously the worse choice.




\subsubsection{Rian finds Neina and escapes}
\target{they find Neina}
In the chaos that ensues after the \banes wreak havoc in the Sentinel camp, Rian slips in and manages to find and rescue Neina.

Have a long, tension-filled scene where we hop back and forth between Neina and Rian. 
There is a thug, Blon, who wants to rape Neina.
He reaches into her cell and gropes her. 
She shies away and hates him and suffers. 
But he dares not rape her because the sorcerers forbid it. 

Later, when all hell is breaking loose, Blon decides he might as well go for it. 
So while the rest are running around confused, he goes to Neina. 
He gropes her and undresses her. 
She screams and fights.
She manages to kick him painfully in the nuts.
He retreats a bit to get over the pain. 
This gives her a short reprieve.
Then he comes back in and beats her savagely. 
He rips her clothes off.
She screams and cries. 
She is desperate. 
She does not want to believe that this is happening to her. 
At last, he penetrates her and completes his violation of her. 
She loses her virginity to an evil, ugly, foul-smelling crook whom she hates. 

All the while, we hop back and forth between Rian and Neina.
He runs around frantically looking for her. 
At last he reaches her.
She is being raped by Blon. 
Rian attacks.
Blon pulls out.
Rian kills him.
This is the first person Rian has ever killed. 
As Blon dies, sperm pours out of his dick onto the ground.
Neina has been raped, but they can take some small consolation in the fact that Blon did not come inside her. 

Rian has saved her, but at a terrible price. 
Everything else is going to hell around them. 
Moro wished he was with her.
Together they might have been able to save more people.
But no.
Rian had only eyes for Neina.
He insisted that he had to \trope{AlwaysSaveTheGirl}{Always Save the Girl}.
It was a horribly bad choice from a utilitarian point-of-view, and he did not even succeed. 
Karma by proxy! 

Neina, being a weak soul, \hr{Weak souls go mad in Malcur}{is affected by the Change}.
She may be half-mad. 
She comes somewhat to her senses when she gets out of Malcur, but she will never be quite sane again. 





%At the last minute, however, they manage to escape. 
Rian finds Neina and they run away together. 

The sight of the two together, their love and happiness at being finally reunited, reaches Moro's heart, and she is both touched and envious. It softens her a little bit, and she gains a bit of hope for the future and the world.





\subsubsection{Neina's story}
\target{Neina's story}
Neina was supposed to have been killed a while ago. But some of the Sentinel-hired thugs screwed up, and she was left to rot in a dungeon for many weeks. 

A few times she was dragged to a ritual to be sacrificed, but then not sacrificed anyway and returned to the dungeon. 
\ps{\Psyrex}{} thugs are not the most organized people in the world. 

When Rian comes to rescue her, she is half-mad and hysterical, babbling about monsters and black magic. She is quite useless. 





\subsubsection{Rian and Moro almost eaten by living house}
\target{Rian and Neina are separated from Moro}
Rian, Neina and Moro huddle inside a house. Then suddenly the Shroud opens up and the house reveals itself as a monstrous living entity. It makes a grab for them with its claws or tentacles, intending to devour them. 

Compare to a scene in the anime \cite[episode 1]{Anime:IczerOne}, where a house comes alive.

They escape, but in the chaos, Rian and Neina get separated from Moro. Now the young lovers are in deep shit, because it was Moro's skill and sorcery that was keeping them alive.

Rian develops a nasty case of claustrophobia after being almost eaten by a living house, having seen the walls literally close in around him and turn into monstrous mouths. 





\subsubsection{Maybe Rian dies}
\target{Maybe Rian dies}
Maybe I will \hr{Remove Tiroco story thread}{remove the \Tiroco story thread}. 
If so, I might kill Rian and Neina instead.

\begin{itemize}
  \item 
    Maybe Moro tries to convince Rian to help her do some real good and save as many people as they can.
    Rian instead runs off, frantically intent on saving Neina, whatever the cost to everyone else. 
    He is quickly killed as punishment for his stupidity.
    \quo{\trope{AlwaysSaveTheGirl}{Always Save the Girl}} is not a good idea on \Miith. 
    
  \item 
    Maybe Rian does find Neina.
    He rescues her while she is still a virgin.
    They run out. 
    Then they are killed. 
\end{itemize}

Maybe Rian is killed by a \bane before Moro's eyes. 
Moro gets traumatized. 
But at least she manages to do some good and save some other people. 

Maybe it is Moro who gets rescued by \Criseis in the end. 















\section{\Nithdornazsh Rising}







\subsection{Battle for the Ghost Tower}
The Sentinels and Cabal battle for control of the Ghost Tower. 

Note that \hr{Immortals inside the Shroud}{the Shroud suppresses the immortals' powers}. 
I need to deal with that somehow. 

Read about: 
  \hr{Resphan}{\resphain},
  \hr{Dragon}{\dragons},
  \hr{Umbra}{\umbrae},
  \hr{Resphan equipment}{\resphan equipment},
  \hr{Weapons}{weapons},
  \hs{technology}. 
Read about \hs{Chaos magic}, and remember to invoke \Sethicus and \Tiamat. 

\Achsah retains command of her group of \resphain. 
The \resphain who come to help \Achsah \hr{Achsah's rank}{stand below \Achsah{} in rank} and must obey her commands. 
This galls them, for some of them are purebloods. 
But \Achsah is highly talented and experienced, more powerful than many purebloods.
(Some of the others are \thelyadeth or \gessurim, others \bezedeth.)

\hr{Ashenblood lesser immortality}{\Bezedeth do not possess True Immortality}, so if they die, they are gone for good. 
This means \Achsah must be very brave and convince her fellow \bezedeth to be likewise when they have to go up against \Nzessuacrith.
    
Maybe the \resphain wear \hr{Glass armour}{\armour made of glass or crystal}.

\Nzessuacrith attacks her foes with dark curses invoking the \xss. 
See the section on \hr{Magic visuals}{magic visuals}, especially \hr{Curses of destruction visuals}{curses of destruction}.





\subsubsection{Carzain attacks \Takestsha}
\target{Carzain fights Takestsha alone}
\Takestsha \hr{Takestsha retreats to heal}{has retreated from the battle to heal}. 
But Carzain is not done with her.
He fights his way through the Rungeran ranks using might and stealth.
He has left he Imetrians behind now. 
He is alone. 
This is his moment of glory. 

He tracks \Takestsha and attacks her. 
With the great sorcerous power he and she unleash (and the way \hr{Takestsha bleeds power}{\Takestsha bleeds arcane power}), no Rungeran soldier dares come anywhere near. 
They fight. 

Fortunately, Carzain is \uber-powerful for a mortal. 
Read about \hs{Carzain's strength}. 
\Takestsha keeps underestimating him and slapping him with too little power, and he keeps climbing to his feet again. 

Carzain is overpowered, but he fights bravely. 
He was wounded already when he approached her (having taken wounds in the charge against the \ishrah) and sustains many more wounds in the battle, but by heroic willpower he keeps himself alive and fights on. 
He pushes himself to the utmost, and in his hour of need, on the brink of death, he manages to unlock some of the dark power that lies sleeping within him. 
He gets closer to his true self. 

Carzain does not understand why \Takestsha is so powerful.
He is willing to swear that her magic is \rethyactic in nature.
But she is too fast. 
Conventional wisdom has it that in close combat, a Vaimon will always defeat a \rethyax. 
The \rethyax's magic may be more powerful, but the Vaimon's magic is faster. 
But not her. 
She is fast as fuck. 

Finally she pushes him away and transforms. 
She breaks her \human form and begins to mutate and grow. 
She begins to return to her \draconian form.

This gives pause to Carzain. 
He starts to suspect he has gaped over more than he can handle. 
But he does not back away.
He keeps up his attack. 

She fights on while she is mutating into her true form. 
She badly wounds him. 
Then his \malach self begins to awaken. 
Somehow the savage battle triggers something in him.
It is the first time in all his Scion lives that Ramiel has faced a \dragon this close. 
He has many strong memories of fighting against \dragons, including \Nzessuacrith herself. 
Now they come back to him. 
The fact that his body is badly wounded helps. 
It is more traumatic, and it forces his desperate mind to dig deeper for reserves of power. 
He has to fight for his life, harder and more desperately than he has ever fought before. 
Besides, he is surrounded by immense power that rends the Shroud in tatters. 
This makes it easier for him to see through his own inner Shroud and access powers and memories that he did not know he had. 

And Carzain indeed finds new reserves of power. 
He manages to unleash sorcery beyond any \human. 
A trace of his \sathariah power that is awakening. 

This is his \quo{\hr{Carzain's Sephiroth epiphany}{Sephiroth epiphany}}. 
As part of his epiphany, Carzain sees visions of \Mystraacht.
Also read about \malachim, \carcers, Carzain, Vizicar and Ramiel.

\citeauthorbook[p.243]{RobertEHoward:KingsoftheNight}{Robert E. Howard}{%
  Kings of the Night%
}{
  The sun was sinking into the western sea; all the heather swam read like an ocean of blood.
  Etched in the dying sun, as he had first appeared, Kull stood, and then, like a mist lifting, a mighty vista opened behind the reeling king.
  Cormac's astounded eyes caught a fleeting gigantic glimpse of other climes and spheres\dash as if mirrored in summer clouds he saw, instead of the heather hills stretching away to the sea, a dim and mighty land of blue mountains and gleaming quiet lakes\dash the golden, purple and sapphirean spires and towering walls of a mighty city such as the earth has not known for many a drifting age.
}

\citebandsong{DarkEmpire:DistantTides}{Dark Empire}{A Soul Divided}{
  Since my birth, haunting visions have occurred to me. \\
  A strange power controlling my destiny. \\
  Burned in the fire, but my blackened soul has remained.\\
  My dark desire will infect this mortal plane.
  
  Lying here, left for dead. Bandaged and alone.\\
  A separate half I'll create to undermine their throne. \\
  Feel your aggression. The taste of my hate inside of you\\
  feeds my obsession, until the time you set me free. \\
  This was meant to be
  
  Remember me, your true self. Its who you must become. \\
  You and I, are one and the same, the Chosen One. \\
  Rock turn to rust. My minions rise and humanity\\
  all turns to dust. None will ever stop my insanity.
  
  Banish it to the void. Seal your fate at once. \\
  Let the light take you in. See it through. \\
  I am immortal. Even if I'm wiped away,\\
  I can't be stopped, no. My influence has spread.
  Your kind is dead
}

\citebandsong{DarkEmpire:HumanityDethroned}{Dark Empire}{Eyes of Defiance}{
  Shadows of the past are consuming all I see.\\
  Breaking through the darkness there's another chance for me.\\
  Once again inside me breathes the energy of life. \\
  Wash away my sins, look into my eyes.\\
  I will defy!
}

He wounds \Takestsha a bit. 
She curses and draws back. 
\Takestsha blasts him again. 

This time he cannot get back up.
He is struck down to the ground, writhing in pain and helpless. 

Remember to have great mind-shattering revelations of cosmic magic when \Nzessuacrith assumes her true form. 
Compare to \cite{StephenMarkRainey:Signals}. 

\Takestsha is stunned a bit and draws back to think, and to complete her transformation. 
She feels the \sathariah power radiating from him. 
(Do not mention the word \quo{\sathariah}.) 
This surprises and startles her.
She realizes that he is more than a mere \human. 
He is a Scion. 
But in a way, this also reassures her.
It would be embarassing for her to live with knowing that a mere \human could cause her such trouble. 
Now that she knows he is a Scion. 
She understands better. 

While Carzain is down, \Takestsha completes her transformation.
She now stands before him in her full, terrible glory as \Nzessuacrith. 

She is just about to finish off this insolent mortal.
Then needles of shimmering energy lance into her.
She looks up.
It is \Achsah and her \resphan kin.
They have finally arrived at \Forclin to repel her.
\Nzessuacrith welcomes them.
She forgets about the high-powered \human and leaps into the air, eager to confront her hated foes. 

Carzain \hr{Ilcas rescues Carzain after fight with Takestsha}{falls unconscious and is rescued by Ilcas}. 





\subsubsection{\Nzessuacrith in \draconian form}
\Nzessuacrith{} has finally been forced to leave the body of \Takestsha{} and assume her true, \draconic{} form and enter the fray. 

\Nzessuacrith{} wears \hs{ward runes}. 
(Remember to read about \hs{ward runes}.)

Before entering combat, \Nzessuacrith{} casts a spell that makes her natural weapons poisonous. 

Read about \hr{Nzessuacrith}{\Nzessuacrith}, \hr{Achsah}{\Achsah}, \hr{Dragons}{\dragons}, \hr{Resphan}{\resphain} and \hr{Umbra}{\umbrae} before writing this section. 
And read some RPG books for inspiration on spells and weapons and magical items they might use. 

\Nzessuacrith{} is surprised that she is able to take \draconic{} form so deep in the Shroud. 
She realizes that her own \EreshKali spells have significantly weakened the Shroud (locally and temporarily, that is).
As she spells backfired, this effect could potentially have become even stronger and more out-of-control. 
She speculates that the Ghost Tower might be exterting a further destabilizing influence on the Shroud. 
But is she not quite convinced by this explanation. 
She fears the Shroud is \hs{unravelling}. 

\begin{prose}
  \tho{The barriers between the Realms are breakdown down.
    The great cosmic Seals are leaking.
    Like water seeping through holes in a rotted dam.
    What happens if\dash when\dash the river breaks through?
    A \thirdbanewar? 
    Can \Miith{} survive a \thirdbanewar?
    
    And what is causing it?
    Is it the weakening of the Heart?
    The conflict of the \matrices?
    What?}
\end{prose}

Have some mortals who are stricken with terror and awe at the sight of the \dragon{}. 
Perhaps Carzain and his party. 
They have heard \hs{myths} of \dragons, but their true power and majesty is downplayed in the Iquinian myths. 
The sight of an actual \dragon{} blows away all the myths and faerie tales. 

Delph dies in this final, climactic battle. 
But his rat lives.

Carzain sees the \hr{Umbra}{\umbrae} that \Achsah-tachi summon. 
He likens them to bats, but \hr{Umbra like bat}{different}. 
They also \hr{Umbra sounds}{hear the \umbrae{} howl}. 






\subsubsection{\Achsah meets the challenge}
The sight of \ps{\Nzessuacrith}{} unmasked has drawn \Achsah{} $100\%$ from \Malcur to the Ghost Tower. 
Charcoal's plan\dash aided by Carzain\dash has bought the Sentinels enough time to call in reinforcements and ultimately fight off the Sentinels. 

We see \Achsah{} at the Tower. 
Perhaps she is riding a terrible but splendid monstrous steed.

\lyricsbalsagoth{When Rides the Scion of the Storms}{
  I see him\prikker \\
  grim and noble astride his great winged steed, \\
  gleaming spear crackling in his grasp, \\
  beckoning me onwards to the next life\prikker \\
  to ever more slaughter and carnage\prikker \\
  Yes, adour and brooding spirit he is, \\
  and in his burning eyes I see \\
  a great secret which I must discover,\\ 
  a powerful mystery I alone must solve.
}

Does she actually wield a spear?

\begin{prose}
  \Achsah{} mocks \Nzessuacrith{}: 
  \ta{I had expected more. I had feared I would be facing \QuessanthIshnaruchaefir.} 
  
  \Nzessuacrith{} mocks \Achsah{} in turn: 
  \ta{I had expected a \ketheran. 
    Not some pathetic half-\human{} scum. 
    Without the stolen \draconic{} blood you \resphain{} are worthless. 
    You, \Achsah, are nothing but a swarthy \human{} with delusions of grandeur.} 
  
  \Achsah: \ta{A \human, am I? We will see about that.}
  
  They fight. 
  Then, a bit later: 
  
  \Achsah: 
  \ta{If you did your research, \Nzessuacrith, you would know that I am of \Merkyrah, and as such not not half \human{} but half \nephil.} 
  
  \Nzessuacrith: 
  \ta{Hah! 
  Feebly trying to defend the last shreds of your dignity? 
  Very well, I take back my last insult. 
  You, \Achsah{}, are nothing but a pathetic, swarthy \nephil{} with delusions of grandeur.} 
\end{prose}

Notice that \Nzessuacrith{} is proven right. 
\Achsah{} is soundly beaten and only prevails when Ramiel, a \sathariah, shows up to help her. 





\subsubsection{They battle}
Enter a great battle between \Nzessuacrith{} and \Achsah. 
This battle is long, hard, bloody, brutal and dirty. 

\Achsah{} draws up her full power, which is considerable. 
What she lacks of inborn gifts she makes up for in age and experience. 

\lyricslimbonicart{Beyond the Candles Burning}{
  I am a dark star rising on the raveous bleaky sky,\\
  a black diamond slunning so deep within the night.\\
  Maliciously I dwell in a bluish shaded beam\\
  with a stonecold heart into the core of my being.
}

Perhaps they fight in humanoid form first before going into their monstrous forms. 





\subsubsection{The \umbra}
\Achsah and her \resphan cohorts summon a terrible monster to their side: 
An \hr{Umbra}{\umbra}. 
They control it with spells of binding, visualized as aethereal {chains} of mental energy. 

Compare to the monster that Durza summons in the movie \cite{Movie:Eragon} (not present in the book \cite{ChristopherPaolini:Eragon}). 

\Nzessuacrith struggles against a vast, black, shapeless cloak of massive darkness.
The \umbra \hr{Umbrae eat souls}{drains energy from everything}.
It even sucks the life out of \Nzessuacrith.
Her scales, gleaming bright as the mystic moon, fade to a dead and colourless gray, like a gray emptiness without light or colour. 

The mortals can feel rumbling and vibrations as from thunder or earthquakes.
After a while they realize that they are hearing the impossibly deep-pitched keening wails of the \umbra. 

\Nzessuacrith's battle with the \umbra should be chaotic and confusing. 
Compare to the end scene in \cite{HPLovecraft:TheShadowOutofTime} where Peaslee is almost taken by a flying polyp. 
We see a vast, black, shapeless, immaterial horror engulf the shining \dragon princess. 
She fights. 
There is a storm of lightning and chaos and sorcery, and there can be heard her enraged roars and the deep wailing of the thing.
And then, with a feat of superhuman effort, she breaks free of its hold. 

\Nzessuacrith exhibits superhuman bravery and effort in her heroic battle against this terrifying cosmic horror. 
She is already wounded from her fight against Carzain \Shachar and the \resphain.
But she fights with all the might and fury that is her \draconian heritage.
She kills some of the \resphain, weakening their control over the \umbra.
Some of the remaining \resphain, fearing that the \umbra might break loose and turn on them, panic.
They dare not fight to get the \umbra back under their control.
They turn tail and flee.
At last \Achsah is alone.
She lets go of all her spells and cowers down.
In the end \Nzessuacrith triumphs and fights off the \umbra.
She banishes it and locks it out of the Shroud.
It vanishes back to the night-black indescribable dimensions of the void which are its haunt. 

Describe the awesome forces of magic unleashed.

See the section on \hr{Magic visuals}{magic visuals}.

\Nzessuacrith{} is weakened, but still a badass motherfucker. 
She kills more than one \resphan{} (but not permanently). 

\Nzessuacrith{} does \emph{not} use the spell \word{\hs{khestni}}. 
She has nastier weapons at her disposal. 

\Nzessuacrith{} uses magic to enhance all of her moves. 
She growls words to power to punctuate every attack or parade. 
Unlike \Ishnaruchaefir. 

\Nzessuacrith summons monsters to do her bidding.

\citeauthorbook[p.344]{ClarkAshtonSmith:TheDarkEidolon}{Clark Ashton Smith}{%
  The Dark Eidolon%
}{
  Yea, the undying worms of fire and darkness have come forth like an army at thy summons, and the wings of nether genii have risen to occlude the sun when you called them.
}





\subsubsection{The \umbra escapes}
The \umbra{} escapes from the battlefield. 
\Achsah{} reflects that it will probably go into the \Wylde{} and live off whatever mortals it can catch. 

There it will live out the rest of its days. 

The rest of its days\prikker how much is that? 
Are \umbrae{} immortal? 
\Achsah{} does not know. 

And do they reproduce in the \Wylde{}?
How do they reproduce? 
Do they reproduce at all? 
\Achsah{} does not know. 

The \banelords{} probably know, she reflects. 
But she has never been in a position to ask a \banelord{} a question. 
And she would not dare even if she could. 
She is not too proud to admit that the \banelords{} make her shit herself with fear. 
She remembered feeling their evil presence during the Incursion. 
And the presence of the dreaded \Voidbringer. 
The fear has not decreased in her memory, even after all these millennia. 
She still remembers it vividly. 

That is another thing the young \resphain{} do not understand. 
They do not fear the \banes{} as they should. 
The mortals have the right idea here, she thinks. 
Many mortals see their gods as terribly frightening, alien powers around which one must tread very carefully.

\begin{prose}
  \tho{We \resphain{} could learn from that.
    We like to think of ourselves as being on top of that ladder of power.
    But we are not.
    There are things out there far older, far darker and far more powerful than we.}
\end{prose}





\subsubsection{\Nzessuacrith flees}
\target{Ramiel scares Nzessuacrith}
\Nzessuacrith was badly wounded even before she assumed \draconian form.
She is fighting several powerful \resphain mounted on their terrible \umbrae. 
She is holding up, but it is a hard fight for her. 

\Achsah is skilled. 
Eventually her determination, bravery and good combat tactics win the day. 
\Nzessuacrith is overpowered forced to flee. 

\Nzessuacrith rationalizes it, telling herself that \Nithdornazsh must be about to rise. 
The \resphain cannot return to \Malcur in time to stop the ritual now. 
So she permits herself to flee from the field of battle. 

The \resphain are themselves badly wounded.
She has killed some of them (non-fatally).
They are in no condition to pursue. 





% \Nzessuacrith{} could defeat the \resphain, but \hr{Charcoal at the Ghost Tower}{Charcoal's spell antagonizes her}, and ultimately she loses and must retreat.
% 
% Now, Charcoal's magic alone would not be sufficient to make any difference, since \Nzessuacrith{} is a powerful \dragon. But Carzain/Vizicar is there. The \nieur{} ritual that Charcoal unleashes somehow causes the \sathariah{} within Carzain to stir and awaken a little bit. Acting instinctively, Ramiel adds his power to the ritual. 
% 
% Perhaps Ramiel senses that \Nzessuacrith{} is family. 
% She is, after all, kin to \Nexagglachel, the sire of the \satharioth. 
% 
% Now, Ramiel doesn't have much power available. 
% He is still in deep sleep. 
% But he makes his \emph{presence} known. 
% \Nzessuacrith{} feels the smell of a \sathariah, and it distracts and alarms her. 
% She had not counted on the innocuous Scion being a \sathariah. 
% It throws her off balance for a moment, and this is enough for \Achsah-tachi to wound her and drive her off. 





\subsubsection{Conclusion}
At the end of the day, the Tower is in Cabalist hands. 

The battle has wrought destruction of Godzilla-like proportions to the surrounding countryside. The Tower itself is untouched, because it is built with the technology of the ancients. \Forclin{} is laid in ruins, except for the castle and certain walls and towers, who are superhuman. 

\Nzessuacrith{} (\hr{Nzessuacrith likes beauty}{who likes beautiful things}) mourns the devastation wrought on the beautiful \Forclin. 

Now \Achsah{} and \Nzessuacrith{} know who Carzain is\dash to an extent, at least.
\Achsah{} is $100\%$ sure the \vertex{} is a \sathariah{} aligned with the \hs{Midnight Bat}. 
Which means it must be a Scion. 
Which means there are only two possibilities. 
She reports this to \Azraid, who \hr{Azraid learns of spike}{muses about it}. 

\Nzessuacrith{} and \Ishnaruchaefir{} are less certain about this. 
They have not studied the \vertex{} (or the theory of \malachim) as much as \Achsah{} has. 

Some of the Cabalists muse over how they underestimated Charcoal. 
They believed to have him figured out, but he still managed to thwart their plans. 

Meanwhile, in \Malcur, \hr{Ishnaruchaefir kills Teshrial}{\Ishnaruchaefir is about to kill \Teshrial}, and \hr{Nith'dornazsh rises}{\Nithdornazsh is rising}.





\subsubsection{Consequences for mortals}
Sethgal still lives. 
Much of \Forclin has been devastated, but all is not lost. 

It has gone worse for the Rungerans. 
Carzain had chased \Takestsha deep behind the Rungeran lines before she transformed.
So the battle of the immortals took place (initially) in the middle of the Rungeran army. 
In the process, the army took bad casualties, and its morale is broken. 
The remnants have scattered. 

The war with Runger is probably over. 

Carzain has vanished.
Maybe so have the Imetrians. 
Sethgal now has to try to rebuild. 





\subsubsection{Morgan Runger}
Morgan Runger sees his army panic and begin to scatter while the immortals battle. 
He realize the invasion is lost. 
He orders his remaining forces to retreat. 

He rides away in his howdah.
Awestruck he watches the destruction behind him while absently fondling the breasts of one of his naked concubines. 

Maybe Morgan's party is hit by a stray fireball and killed. 
Maybe he escapes back to Runger. 










\subsection{Carzain heads to \Redce}





\subsubsection{Ilcas rescues Carzain}
\target{Ilcas rescues Carzain after fight with Takestsha}
After his fight with \Takestsha, Carzain falls unconscious and lies bleeding to death. 
But Ilcas Northstar has seen his heroic battle against the \dragon-sorceress. 
Ilcas fights his way through the panic and the destruction caused by the warring immortals. 
He reaches Carzain's unconscious body and hauls it to safety. 

Ilcas cannot heal Carzain himself. 
But he finds (or is found by) \Esmerel, who offers her help.
She heals Carzain's grievous wounds and broken limbs. 
She saves his life. 
It takes an awful toll on her.
She looks ten or twenty years older afterwards. 
But she does what she feels she has to do.
She suspects this man is something really special, and she will sacrifice much to secure him for \ClanRedcor.
Besides, now he owes her. 

\Esmerel convinces Ilcas to help her transport Carzain to \Redce. 
They set out while Carzain is still unconscious. 

Carzain later awakens.
He learns that \Esmerel has saved his life, at great cost to herself. 
This means he owes her.
In exchange she coerces him into promising to go with her to \Redce and help \ClanRedcor against a dark enemy. 
He does not want to serve the Redcor, but he lets her coerce him.
He lets her think he is cowed by guilt and obligation and will now serve them faithfully. 
But in truth, he has his own plans. 





\subsubsection{Razor is more wary of Carzain}
Have a scene from Razor's POV. 

Since their last meeting, Carzain has grown darker. 
Vizicar has awakened again and is now closer to the surface. 
Therefore, more of his dark, wicked \sathariah{} nature shines through. 

Razor notices this immediately, and he is now more wary of Carzain than before. 
Does Razor hide this, or does he openly display his unease? 
He probably hides it. 
Razor is a sneaky bastard. 

Carzain notices the creepy lizard staring at him. 
He distrusts it. 
Cannibalize the \quo{creepy \human}/\quo{creepy lizard} scene I wrote once. 





\subsubsection{To \Redce}
\Esmerel{} wants to flee back to \Redce{} and tries to persuade Carzain to come with her, tempting him with romises of glory and greatness if he allies himself with the Redcor. 
Also, \Esmerel{} promises him that they can help him master his \hr{Kenosis}{\Kenosis}. 
He wants this, since in his current state he is an unstable madman and a danger to everyone. 
Also, he hopes if he allies with the Redcor, he can get his hands on \hr{Iolivine's notes}{\ps{\Iolivine} notes on Scions}, which \hr{Redcor bogarted Iolivine's notes}{the Redcor are bogarting}. 

Ilcas agrees to accompany them, because the Imetrians and Redcor want to temporarily join forces to drive back the Rissitics. Carzain agrees, partly influenced by Ilcas' decision. 

I need to stress the fact that Carzain does not agree to serve the Redcor. He is somehow tricked and coerced into coming with them, and thus can claim to be a prisoner of sorts in the Topaz \Chateau. 

How exactly does this work?

They slip out of \Forclin{} amid the chaos and make their way north to \Redce{}. 
But their way goes past the Ghost Tower, so Carzain becomes involved in the events there. 

Remember that the Redcor should refer a lot to their scripture and their historical heroes, such as Silqua and Rebecca Redcor. 





\subsubsection{Carzain is like Sephiroth}
Remember that \hr{Carzain is Sephiroth}{Carzain is supposed to be like Sephiroth} from \cite{VideoGame:FinalFantasyVII}. 

At the end of \TwilightAngelRememberEmph, he has a \hr{Carzain's Sephiroth epiphany}{Sephiroth-style epiphany} and turns evil. 

The epiphany has to do with Curwen (\hr{Charcoal at the Ghost Tower}{his plan to use the Ghost Tower}) and \Takestsha.
Carzain comes into combat with \Takestsha and realizes that she is \Nzessuacrith. 
She uses dark magic against him. 
This is traumatic. 

This is the first battle of this scale he has fought in this life.
He has fought smaller battles as Carzain, and bigger ones as Vizicar.
After a battle, he is usually loath and tired of all the slaughter and bloodshed and glad to see it end. 

After this battle, Carzain realizes that he is not loath of the slaughter as usual.
Rather, he finds that he has relished it as never before. 
He realizes that his true nature is a dark angel of battle.
He feels he has taken a great step forward to finding his soul. 





\subsubsection{Carzain espies Morgan Runger}
On the battlefield, Carzain espies \hs{Morgan Runger} in the distance. 

\vizicar{%
  Feh. Kings who are not mages themselves are mere upstarts. Such a coward. He commands his sorcerers to work their dire magics, but dares not do it himself.}

Vizicar is quite the pro-mage nazi. 







\subsubsection{Carzain sees reapers on the battlefield}
\target{Carzain sees reapers near Forklin}
After the battle of \Forclin, Carzain sees some \quo{reapers} on the battlefield. They are \hs{Worm Cult reapers}, as well as \hr{Crows and ravens}{crow- or raven-like men}. 

They appear as unclear ghosts. They might be an apparition created in his mind from the fog, smoke, animal and corpses. He is in doubt: \tho{Am I seeing things?}

They pick up some dead. Carzain seems to see corpses\dash or their souls\dash rise to wander, limp or crawl away into the Beyond. 

Perhaps the Worm Cult reapers fight the Ravens. 





\subsubsection{Carzain and Vizicar talk}
Carzain and Vizicar talk. 
Vizicar theorizes that it was he who \hr{Vizicar drives Carzain to war}{subconsciously imparted to Carzain a craving to fight and seek glory}, thus causing him to go off to war. 





\subsubsection{Carzain-tachi encounter a \bane}
Carzain-tachi encounter a \bane. 
Ilcas Northstar holds up an Imetric holy symbol and recites a prayer/\hs{orison} to ward off evil and keep it at bay. 
It seems to work for a moment, but then the \bane{} advances again. 





\subsubsection{Ilcas kills prisoners}
There is a scene where Telcastora Ilcas, together with some Redcor, have taken some enemies prisoner. 
These are Pelidorian soldiers who have deserted and turned into bandits. 

Ilcas is about to kill them. 

\begin{prose}
  \Racel/\Esmerel: 
  \ta{No! Stop. If you kill them, you are no better a man that they.}
  
  Ilcas: 
  \ta{What are you talking about? Fuck that!} 
  He kills the men. 
  \ta{Of course I am better than they. 
    They were not only brigand scum, they were also traitors. 
    They were paid and armed by their kingdom and entrusted to protect their people from enemies. 
    They betrayed that trust and turned against their own people. 
    They were the worst kind of filth. 
    I am doing the world a favour.}
\end{prose}

Ilcas truly hates these men's guts. 
Traitors and malfeasants are pieces of shit. 

\begin{prose}
  \Esmerel: More whining. 
  
  Ilcas: 
  \ta{We are not under the jurisdiction of Redcor law!} 
  
  \Esmerel: 
  \ta{Nor are we under Imetric law!}
  
  Ilcas: 
  \ta{Look around you. 
    Pelidor has fallen. 
    This place is \Wylde{}. 
    We are under no law. 
    When protected by no law it falls upon each of us to act on our morality as best we can. 
    I did exactly that.}
\end{prose}

Carzain stands next to Ilcas and admires the manliness. 

Maybe Ilcas concludes with something like this: 

\begin{prose}
  Ilcas: 
  \ta{%
    Maybe we \scathae{} have a more rational outlook on death and killing than you \humans{} do. 
    I don't know.}
\end{prose}

Perhaps the last part is said in private to Carzain out of earshot of the Redcor. 

Ilcas notices that during the killing, Carzain stood by and looked on with a bloodthirsty grin. 
This disturbs Ilcas. 
He had good reasons for killing them, but he did not \emph{enjoy} it. 
And now that he thinks about it, Carzain has been overfond of killing for a long time. 
He was also too proud for his own good when he told of the mercenaries he had killed in Heropond last year. 
Ilcas fears Carzain is turning into a bloodthirsty maniac. 





\subsubsection{Ilcas wants to feed his sword}
In truth, Ilcas killed the prisoners not just for the sake of punishment, but also to feed his sword. 
\Telderain{} hungers for blood and soul energy (although it does not eat entire souls; it just leaves them somewhat drained), otherwise it goes crazy and tries to drive Ilcas crazy, too.  

But he doesn't tell this to the Redcor. 
He doesn't want them to know that he wields a black magic sword filled with blood-drinking \daemons. 





\subsubsection{Later Ilcas lets a hostage die}
Later, Ilcas lets a hostage die. 
Perhaps a child, or an egg. 
He knows that sacrificing the hostage to kill the villain is preferable to letting the villain get away. 
But some of the dumbass Redcor, such as \Racel, are unwilling to see that. 





\subsubsection{Ilcas cares for his \nycans}
At some point, Ilcas Northstar endangers the rest of his companions for the sake of one of the \nycans. 

Some of the Redcor are butthurt about this.

\begin{prose}
  \Esmerel: 
  \ta{%
    You would endanger all of our lives for the sake of an \emph{animal}?}
  
  Ilcas: 
  \ta{\Matron, maybe I owe it to you to make our position clear. 
    Let me give you a run-down of my loyalties, in descending order of priority. 
    One: the Imetrium. 
    Two: the Telcastora clan.
    Three: my wife and children.
    Four: the \nycans.
    Five: you people.}
  He looks at Curiet and jokingly adds.
  \ta{Six: Serpentin.}
  
  Curiet:
  \ta{What? Why I am at the bottom?
    All of them are Vaimons! They can defend themselves. 
    I am a defenseless civilian!}
  
  Ilcas looks at him.
  \ta{Point taken. 
    Correction. 
    Five: Serpentin. 
    Six: you Vaimons.}
\end{prose}









\subsection{\Teshrial and \Ishnaruchaefir}
\target{Ishnaruchaefir kills Teshrial}
\Ishnaruchaefir{} has taken \ps{\Teshrial} \quo{bait}. 
He arrives in the dead garden in answer to the challenge. 

\Teshrial{} has several trump cards:
\begin{itemize}
  \item Astrology. 
  \item The Achilles Heel. 
  \item The Shroud.
  \item \Noggyaleth.
  \item \NeoResphan metamorphosis. 
\end{itemize}
    
\Ishnaruchaefir{} uses the spell \word{\hs{khestni}} once during the fight. 
It hurts \Teshrial, but does not kill him. 

\Ishnaruchaefir{} wears \hs{ward runes}. 
(Remember to read about \hs{ward runes}.)

Maybe the \resphain wear \hr{Glass armour}{\armour made of glass or crystal}.

Read about: 
  \hr{Resphan}{\resphain},
  \hr{Dragon}{\dragons},
  \hr{Teshrial}{\Teshrial},
  \hr{Ishnaruchaefir}{\Ishnaruchaefir},
  \hr{Resphan martial arts}{\resphan martial arts} (\Teshrial{} walks the \hs{Path of Ice}),
  \hr{Resphan equipment}{\resphan equipment},
  \hr{Weapons}{weapons},
  \hs{technology}. 
Read about \hs{Chaos magic}, and remember to invoke \Sethicus and \Tiamat. 

\Ishnaruchaefir \hr{Ishnaruchaefir bleeds in Nadir}{bleeds and looks terrible} when he is in the Nadir. 
Every spell he casts causes more wounds to spring open on him\dash{}he pays for his magic with blood and pain.

\Ishnaruchaefir uses some magic, but not so much. 
His magical power is depleted during his Nadir; his physical strength is much more intact (although not quite intact). 
So he mostly fights using his physical strength. 

The Cabal plan is nearly complete. 
The Cabalist \Malcur venture, like \Secherdamon's plan, also coincides with \Ishnaruchaefir's Nadir. 
The \noggyaleth have grown numerous and large and powerful.

Throughout the section, have throwaway references to \hr{Mystic names}{mystic names and places}, like Shung. 




\subsubsection{\Menessiaraid looks on}
\Teshrial{} brings one spectator to the fight: 
His good friend \Menessiaraid. 

He only brings one spectator. 
Otherwise he fears \Ishnaruchaefir, suspecting an ambush, would refuse to fight. 
On the other hand, they both knew they could not fight in private. 
Rumours of this fight have been going all over the place among the dynasties. 
They would need to have \emph{someone} there to witness it. 

So \Menessiaraid{} is there alone, and he does not fight. 
He is a skilled telepath, and his head is crowded full of powerful \resphain{} who want to see the fight through his eyes. 

This means there are fewer \resphain{} left to keep an eye on \Forclin{} and \Malcur. 
Which is exactly what \Ishnaruchaefir{} and \Secherdamon{} want. 





\subsubsection{\ps{\Teshrial} perspective}
The battle should be seen from \ps{\Teshrial} perspective. 
He tries every means at his disposal to defeat \Ishnaruchaefir, and more than once he thinks he has succeeded, but \Ishnaruchaefir{} keeps getting back up. 
Describe \ps{\Teshrial} anguish when he dies. 





\subsubsection{They fight in humanoid form}
\Ishnaruchaefir{} arrives. 
He does not carry his glaive, \Rystessakhin, but only a pair of \skekrathuins. 

They meet in a fairly tightly Shrouded layer of the Realm. 
They are forced to fight wearing \quo{\hs{Masks}}. 
This means that \Teshrial{} has something of an upper hand. 
He is \hr{Immortals inside the Shroud}{not as weakened as \Ishnaruchaefir{} is by the Shroud}. 

First they fight in humanoid form. 
\Teshrial{} challenges \Ishnaruchaefir{} to come down and face him in humanoid form. 
\Ishnaruchaefir{} accepts. 

\Teshrial{} is comparatively young and inexperienced. 
He has never seen \Ishnaruchaefir{} in combat before and underestimates him. 
He mocks \Ishnaruchaefir, calling him a decayed, outdated relic of the far past. 
He believes he can out\manoeuvre the old, bitter, set-in-his-ways \dragon.
He underestimates him. 
Otherwise, \Teshrial{} would have realized how fucked he was, and would have fled. 

Even so, the battle is non-trivial, even for \Ishnaruchaefir. 

They duke it out. 
\Teshrial{} loses. 

At the beginning of the fight \Ishnaruchaefir{} is cool and calm. 
He keeps up this \facade{} as long as he is fighting in humanoid form. 
But when he reverts to \draconian{} form to battle the \noggyaleth{}, and later Mutant-\Teshrial, he lets loose all his \draconian{} fury and lets his hatred against the \resphain{} guide him. 

\Ishnaruchaefir{} does not use so much magic. 
His physical prowess in humanoid form is enough to defeat \Teshrial{}. 
His body is wicked-psycho-tough and durable on its own. 
It is part of his tactic to rely on physical strength as long as possible, thus conserving his magical reserves for when he really needs them. 
He doesn't go all-out magic until mutant-\Teshrial{} shows up. 





\subsubsection{\Noggyaleth}
\Teshrial{} is losing. 
So he pulls out one of his trump cards: 
He summons his \noggyaleth. 

The \noggyaleth have been burrowing through the ground beneath \Malcur, corrupting it.
Read about how the \noggyaleth \hr{Noggyal corruption}{corrupt the planet}. 

Now the \noggyaleth, having long hidden beneath the earth, burst forth.

When \Criseis sees a \noggyal:
\citeauthorbook[p.147]{JohnGlasby:TheOldOne}{John Glasby}{The Old One}{
  Even in retrospect it is not possible to convey in words the nature of that monstrosity which squeezed its vast bulk through the gaping abyss.
  It held a hint of noxious plasticity, of writhing tentacles which changed their number and shape.
  But more than anything, I had the impression of gigantic size, that huge as that part of it looked where it almost completely blocked the opening, there was an infinitely greater bulk mercifully hidden from us.
  
  [\prikker]
  
  But I know there was nothing imagniary of halucinatory about the black, coiling tentacle that seized Dorman around the waist and bore him, kicking and screaming frantically, into the gaping, beaked maw which appeared as if from nowhere beneath that single glaring red eye!
}

\Ishnaruchaefir{} fights them. 
\Teshrial{} stands back and supports them with spells. 

Have a description of the monsters that swarm out to attack \Ishnaruchaefir. 

The \noggyaleth cause the very earth to rise and attack \Ishnaruchaefir.

\citeauthorbook[p.290]{DavidDrake:ThanCursetheDarkness}{David Drake}{%
  Than Curse the Darkness%
}{%
  Pulsing, rising, higher already than the giants of the forest ringing it, the fifty-foot-thick column of what had been earth dominated the night.
  A spear of false lightning jabbed and glanced off, freezing the chaos below for the eyes of any watchers. 
  From the base of the main neck had sprouted a ring of tendrils, ruddy and golden and glittering overall with inclusions of quartz. 
  They snaked among the combatants as flexible as silk; when they closed, they ground together like millstones and spattered blood a dozen yards up the sides of the central columns.
}

But \Ishnaruchaefir is prepared. 
He has prepared spells that breaks their hold over the earth. 
He forces them to come out in the flesh, without the protection of the earth, and fight him naked. 
They do. 

See the section on \hr{Magic visuals}{magic visuals}, especially \hr{Summoning magic visuals}{summoning}.

\Teshrial{} has been fooled into believing that his \noggyaleth{} are a secret trap which \Ishnaruchaefir{} doesn't suspect. 
He keeps them hidden and only summons them in the last minute when he really needs them. 
But unbeknownst to him, \Ishnaruchaefir{} has predicted this move. 
And the fight has been planned so it coincides with \Psyrex-tachi's summoning of \Nithdornazsh. 
This means that the \noggyaleth{} are disoriented and weakened and slow to answer \ps{\Teshrial} call. 
This gives \Ishnaruchaefir{} plenty of time to prepare for them and pick them off one by one when they arrive. 
They cannot lie in wait and ambush him as \Teshrial{} wanted them to. 

\Ishnaruchaefir{} assumes his true, \draconian{} form to fight them. 
Insert an epic description of the mighty \dragon{} here, a la \bandsong{Bal-Sagoth}{Black Dragons Soar Above the Mountain of Shadows}. 

\Teshrial{}, desperately intent on vanquishing \Ishnaruchaefir{}, summons reinforcements, bleeding the city dry of Cabal monsters. 
This is what \Ishnaruchaefir{} wants, because it leaves \Malcur vulnerable to \ps{\Psyrex}{} spell. 

When \Ishnaruchaefir{} fights, he has bound \daemons{} that whirl around his body and act as \armour and weapons. 
They are black, gray, white, purple and blood red. 
Compare to the battle between Fulgrim and Ferrus Manus in \authorbook{Graham McNeill}{Fulgrim}. 

The \noggyaleth grab on to \Ishnaruchaefir with their sucking mouths and grasping limbs/pseudopods.
They drag him down and engulf and swallow him.
Then they try to drown and crush and devour and digest him.

\lyricstitle{Draft excerpt from the chapter \quo{What Slithers Beneath}}{
  %The crushing, drowning sensation had passed, and Rian felt like he was thinking clearly again. Yet the scene before his eyes still seemed more like a hazy dream than reality. 
  Rian was not sure if he was awake or dreaming. 
  \tho{I hope I am dreaming.} 
  Some veil like dark smoke obscured the garden, hiding the black one from view. %, giving him only vague glimpses of the black one. 
  
  Then there came a monstrous sound. 
  
  Roaring. 
  
  Shrieking. 
  
  Groaning. 
  
  Rattling. 
  
  From not one throat, but many. If, indeed, things capable of making such sounds had anything as familiar as throats. 
  
  It came from deep beneath the earth. It came from all around him. It came from inside his head. But above all else, it came from the darkened garden. From the centre of the opaque haze. 
  
  He heard a voice, then. Deep, growling, inhuman. But definitely a voice, speaking powerful words in an alien tongue. 
  %It spoke 
  \tho{The voice of the sorcerer.} Rian could not remember how he knew this, but he though that un-\scathaese{} voice was somehow that of the dark-scaled sorcerer. 
  
  And a faint, distant howling, barely audible above the monstrous roaring. \tho{Or did I imagine it?}
  
  There! Within the cloud, a flash. The glinting of metal. 
  
  \tho{The scythe\prikker} 
  
  The groaning noises intensified. 
  
  The battle had begun. 
  
  %And there! A glipse of what he thought was the black one
  Terrible sounds of combat could be heard. 
  The swish of steel through the air. 
  The grinding of huge jaws. 
  The clash of massive bodies. 
  Grunts of pain. 
  And terrible words in no \human{} tongue. 
  
  And through the fog, through the blur that veiled everything, he saw glimpses. Flashes of light\prikker \tho{Lightning? Fire?} 
  The writhing of towering things\prikker \tho{A worm? Worms?}
  And black scales. Immense dark wings. Claws. Teeth. Horns. Blades. 
  The splattering of alien blood. 
  
  Rian did not know how long the battle went on. It could have been heartbeats, or a whole afternoon. \tho{Or a whole night, if I am dreaming.} 
  
  Then suddenly, a high shriek ripped through his ears like a knife. 
  
  The shriek became a rattle. 
  
  The thrashing of a huge body, or bodies.
  
  Seething, like boiling water. 
  
  Thumping. 
  
  Then silence. 
  
  A long moment passed. 
  
  \tho{%
    Is the battle over? 
    
    Who won?}
  
  The silence stretched. Again he had to struggle to stay awake. The world swam before his eyes. \tho{Must not doze off. Must not.}
  
  Then, movement. 
  
  Rian tried to focus his blurred eyes. At the edge of the garden, something. Something was emerging. 
  
  Black. With a great bladed weapon. 
  
  \tho{The sorcerer. He prevailed.}
}





\subsubsection{\Teshrial summons \malgryph}
When \Teshrial starts to conjure the \malgryph, \Ishnaruchaefir acts afraid and tries to stop the summoning.
\Teshrial realizes it will be more difficult than he thought to summon the \malgryph, so he uses his secret weapon and transforms into his monstrous form.
This gives pause to \Ishnaruchaefir, for it is an unexpected move. 
\Teshrial is able to push back \Ishnaruchaefir and overpower him for a while. 
Long enough to buy time for himself to summon his \malgryph.
(Make sure \Teshrial looks heroic and self-sacrificing here. He does not like turning into a monster, but he is noble and selfless and does it anyway. He thinks of his beloved and hopes she will forgive him for the way he has defiled his own body.)
But \Ishnaruchaefir laughs and casts his own spells.
He takes control of the \malgryph and turns it against \Teshrial and his worms.





\subsubsection{\Teshrial mutates}
The \noggyaleth{} are vanquished. 

\Teshrial{} now uses his last resort: With an experimental spell he absorbs his servant monsters into himself and merges with them, thus mutating into a colossal monster. The spell is dangerous and quite likely irreversible\dash it might drive him mad, prevent him from changing back and dooming him to spend the rest of his life as an insane abomination. 

We see \Teshrial{} from the inside. He knows the danger and fears it, but he is consumed by his eagerness to slay the mythical \Ishnaruchaefir{} and prove his worth, so he can advance in the hierarchy of the \ketherain. 

The mutation thing is the only one of \ps{\Teshrial} traps which \Ishnaruchaefir{} failed to anticipate. 
It startles him and throws him off balance for a moment. 

\Teshrial{} suggets that they drop their \quo{\hr{Masks}{masquerade}}. 
This surprises \Ishnaruchaefir{}. 
It is rare for a \resphan{} to be the first to suggest to \quo{drop the masquerade}. 

\Teshrial begins his mutation spell. 
It is a tremendous and powerful spell. 
It is mentally traumatic for him. 

\citeauthorbook[p.254--255]{HPLovecraft:TheBlackTomeofAlsophocus}{H. P. Lovecraft}{%
  The Black Tome of Alsophocus%
}{%
  Again I made the five concentric circles of fire on the floor, and standing in the innermost one, invoked powers beyond all imagining with an incantation so inconceivably terrible that my hands trembled as I made the mystic passes and symbols.
  The walls dissolved and the great black wind swept me away through dark gulfs of space and grey regions of matter.
  I \travelled faster than thought, past unlit planets and vistas of unknown realms which swirled and shifted across immensurable distances; the stars flashed by so rapidly that they appeared as gossamer-fine threads of brightness interlaced across the universe, minute shooting stars of brilliance shining against black aether that was darker than the fabled depths of Shung.
}

The mutation takes time. 
But \Ishnaruchaefir{} does not take advantage of this. 
He politely stands back and gives \Teshrial{} time and room to transform. 
He is curious and wants to see this new weapon in action and test its mettle against his own. 
Knowledge is power, and he is willing to take even foolish chances to gain knowledge. 
He knows that if he has underestimated \Teshrial, he may well perish. 
But he chances it. 

\Teshrial{} mutates. 
He becomes a \neoresphan and comes to \hr{Neo-Resphan appearance}{look like one}. 
Then, using his newly \hs{increased vampire powers}, he absorbs the bodies of the living and dead \noggyaleth in order to grow to huge size.

\Menessiaraid gazes upon \ps{\Teshrial} mutated form. 
\Menessiaraid is horrified by the sight, but also impressed.
He sees him as an awesome, magnificent angel, both terrible and beautiful.
A vision of the greatness, potential and future of the \resphan race.

\lyricslimbonicart{Beyond the Candles Burning}{
  I am a dark star rising on the raveous bleaky sky,\\
  a black diamond slunning so deep within the night.
  Maliciously I dwell in a bluish shaded beam\\
  with a stonecold heart into the core of my being.
  
  Beyond the candles burning, beyond all minds eye.\\
  A vast emperic enigma awaits me as I die.\\
  In a graceful dance obscene, in a ring of fire,\\
  I obtain my majesty as flames caressing higher.
  
  Release my spirit, unleash my soul.\\
  From the darkest dungeon, oblivion calls.\\
  In the phallic halls of ancient forlorn\\
  a cold sanctuary in doom is born.
}

\lyricslimbonicart{Solace of the Shadows}{
  I require the solace of the shadows,\\
  so the night can be redeemed.\\
  As the winds of darkness whispers my name,\\
  a kiss of death I receive.
  Nocturnal enchanter, to thine art I yield.\\
  Within the candlelight a rapture is now revealed.
}

In his \neoresphan{} form, \Teshrial{} gains \hs{increased vampire powers}. 
He can now drain \ps{\Ishnaruchaefir} life force even from a distance. 
\Ishnaruchaefir{} realizes this and becomes more careful. 
He figures out that \Teshrial{} now also stands a better chance of absorbing his soul if he should win. 

\Ishnaruchaefir{} realizes that \Teshrial{} \quo{knows} about his Achilles Heel. 
So he pretends to fear this and makes certain to guard the Achilles Heel, to goad \Teshrial{} on. 

\Teshrial{} fights bravely and savagely. 
A battle of Godzilla-like proportions ensues. 
Perhaps \ps{\Teshrial} mutated form resembles the monster Orga from the movie \cite{Movie:Godzilla2000}. 

\target{Ishnaruchaefir impaled by spines}
\Ishnaruchaefir{} has his body impaled by two or three huge spear-like spines. 
One of his hearts is impaled, but \hr{Dragons have three hearts}{\dragons{} have three hearts}, so he can bear it. 

He can fight on fine despite the pain. 
But he doesn't play stoic. 
He growls and pants with pain of his many wounds. 
\Teshrial{} thinks he is finished, so he gets overconfident. 
He steps closer to finish off \Ishnaruchaefir{} and consume his soul to gain his power (he knows he can, with his increased vampire powers). 

\tho{%
  This will make me as great as a \sathariah. 
  Nay, greater!
  With this power I will be greater than even \Azraid!
  All glory will be mine!
  All \resviel{} will be mine!}

But this was what \Ishnaruchaefir{} wanted him to do. 
He is not dead, and now he leaps up and kills \Teshrial. 





\subsubsection{\Teshrial dies}
When \Ishnaruchaefir is just about to kill \Teshrial, he tells him:

\begin{prose}
  \ta{So you sought to use the \malgryph against me, did you? 
    Too late. 
    That would have worked five thousand years ago. 
    But I have grown stronger since then. 
    I have overcome many of my old vulnerabilities.}
\end{prose}

\Menessiaraid hears the above. 
That is deliberate from \ps{\Ishnaruchaefir} side.
He does not want to arouse suspicion. 
He does not want anyone to suspect that the relevant \WanderersInDarknessEmph passages are fakes that he planted. 
So he uses the above as a cover story. 

Have a sad scene from \ps{\Teshrial} POV as he dies. 
He thinks of his beloved, of \hr{Teshrial's date}{the amazing sex she promised him} and of the life they could have had together. 





\subsubsection{\Ishnaruchaefir and \Criseis after the battle}
After the battle, \Ishnaruchaefir heads off to \Malcur.

\Ishnaruchaefir admits to \Criseis that he is displeased with the fact that the \resphain have now learned of his Nadir and how to map it.
\Urizeth still lives, after all.
And she is not fool.
She has probably taken measures to ensure that her discoveries will live on even if she is destroyed. 

But it it of no matter. 
The \resphain were bound to find out sooner or later. 
Now that the \thirdbanewar is looming, \Ishnaruchaefir will have to take a much more active role than he has done previously. 
He could not hope to keep his Nadir cycle secret forever. 
He has taken a great chance and gone into combat during his Nadir this once.
It will not happen again. 

\Ishnaruchaefir confesses in private to \Criseis that the part about \Ishnaruchaefir's \quo{overcoming his old vulnerabilities} was a lie. 
It is \Criseis who pieces together the story and realizes that \Ishnaruchaefir has planted the fake \WanderersInDarknessEmph verses and thus masterminded \Teshrial's quest against him. 
It is she who tells the reader this.
She asks her master if what she suspects is true. 
He refuses to comment, just smiles to himself.





\subsubsection{\Menessiaraid after the battle}
\Menessiaraid goes away. 
He is very sad that his friend has died, and horrified to witness the power and cunning of the Destroyer. 
But he is also sort of hopeful. 
If \Teshrial can turn into such a magnificent monster, then it bodes well for the potential of the \resphan race. 
He goes away awestruck at his friend's courage and greatness, and with high hopes for the future and the Quest for Perfection.









\subsection{\Nithdornazsh rises}
\target{Nith'dornazsh rises}
With Needle killed and with \Achsah{} and all other high-up Cabalists fighting for the Ghost Tower, there is no one to stop \Psyrex{} and his \Malcurian Sentinels.
They successfully invoke their ritual. \Nithdornazsh{} rises into the world of \Miith{}, tearing the Shroud apart around it, and the city with it. 
\Secherdamon{} comes to claim his prize. 

The city is made of living flesh and metal. It gorges itself on the bodies of the inhabitants of \Malcur, adding their flesh to its own monstrous body. Perhaps some modicum of their consciousness remains, leaving them to bleed and moan while imprisoned as part of the city walls. 

In fact, \Nithdornazsh{} is dead, and it is resurrected by absorbing the flesh and souls of the people of \Malcur. It is born of \Malcur and devours its own mother. Remember to have lots of horror about this. Perhaps witnessed by \MoroCobrel, who escapes at the end. 

During the resurrection ceremony, the people of \Malcur are seized by a madness. The more susceptible among them become insane mutants who run amok, hunting down other people and eating their flesh. 

A witch-storm rips the sky. \Pdaemons{} and \dragon-spawn fly on the winds, raining down terror from above. 

Have a scene with some normal \Malcuric{} citizen or citizens. Possibly Rian. He runs through the city, literally seeing Hell erupt around him. The stones and the very earth become alive underneath him, coming up to swallow him. 

Hordes of warriors and \pdaemons{} swarm out of thin air\dash out of the Beyond, which is no longer Beyond, but right here! They rampage through the city, slaughtering and enslaving the people. 

\lyricsdimmuborgir{Architecture of a Genocidal Nature}{
  Emerged from the depths of the Earth, gasps.\\
  It rages against mankind, to annihilate the Earth and worse.\\
  It spills the blood like rain. The beauty of Death it represents.
}

\citeauthorbook[p.76]{RPG:Warhammer:TheEmpire}{Alessio Cavatore}{
  Warhammer: The Empire
}{
  The seething Realm of Chaos swept over the city, engulfing it, and Praag was changed forever, its stone walls and buildings melding into hellish and inhuman shapes. 
  Those citizens unlucky enough to still be alive were swept into the maelstrom, their living bodies fused into the walls of the city itself, so that it was no longer possible to tell man from stone. 
  Distorted faces leered from the walls, agonized limbs writhed from the pavements and pillars of stone shrieked in madness with voices that once came from \human lips. 
  Praag had become a living nightmare and a grave warning of what lay ahead should the Chaos armies conquer the land.
}

\lyricsbalsagoth{Witch-Storm}{
  The skyqueen of the dead rides forth,\\
  black storm-borne steeds, immortal blood.\\
  Hark to the striking of the winds, \\
  the moon burns black as slaughter reigns.\\
  Witch-Storm!
}





\subsubsection{\Malcuric{} bullies die}
Have a scene with some \Malcuric{} thugs mugging a poor guy and being mean to animals. Then the \daemons{} come and give them a horrible, painful death, absorbing their souls into even more torment. 

The victim and animal run away and survive. 

Compare to the anime \cite[episode 6]{Anime:TokyoMajin} and \cite{Anime:ElfenLied}. 





\subsubsection{Comparison with Carcosa}
The following scenes from \cite{RPG:CallofCthulhu:GreatOldOnes}, a supplement to the RPG \cite{RPG:CallofCthulhu}, may give an idea of the atmosphere I want to evoke. 

\lyricstitle{\emph{The Great Old Ones} p.70-72}{
  [The prisoner of Carcosa] is free to do anything while awaiting rescue or madness in dark Carcosa. The alienness of this city of towering black buildings costs 1/1D10 SAN per day. 
  Fill the prisoner's time in the city with odd occurences:
  
  \begin{enumerate}
    \item 
      a keening voice wailing a lonely dirge, the source of which can never be found;
    \item 
      intermittent wingbeats of great unseen things in the thick clouds overhead; 
    \item 
      a slithering wave of fog which tirelessly pursues the prisoner through the damp empty streets;
    \item 
      occasional footsteps or whispering voices in the streets of the abandoned city;
    \item 
      a glimpse of a shadowy figure down the street, where no one can be found;
    \item 
      nigthmarish splashing in the waters of the lake;
    \item 
      noises whose sources elude vision because of the thich fog;
    \item 
      a glowing Yellow Sign in the waters of the lake. 
  \end{enumerate}
  
  [\prikker]
  
  As the cultists chant the ritual, thick waves of fog roll in from the lake, then the lake itself swells and grows larger, and the water takes on an oily sheen. 
  The ground gently quakes and stretches. 
  Suddenly the investigators find themselves standing on the outskirts of an alien city, at the edge of a lake much larger than the one they had been observing. 
  The swamp has vanished. 
  The night sky is dull white, and in it black stars shine in unfamiliar patterns. 
}





\subsubsection{\Secherdamon{} rejoices}
\Secherdamon{} rejoices at the fruition of his grand plan. 

\lyricsbalsagoth{Beneath the Crimson Vaults of Cydonia}{
  This red charnel pit of primal horror, \\
  howling black ecstasies to the void.
  Ancient and divine, older than the hidden Icosahedron, \\
  now rebirthed beyond the chaosphere.\\
  Rise\prikker rise and destroy!
  
  Hatred, carnage, slaughter, havoc, chaos, murder!\\
  I am become the devourer of all life!
  
  Phobos, Deimos! \\
  The moons' rays liquefied in these blood red pyramids.\\
  In the shrines of abomination, black tongues rapt with blasphemy.\\
  Chaosphere, watchtowers, genesis, Cydonia\prikker\\
  The Abyss yawns wide!\\
  Spirit of the carrion-thronged battlefield, open wide thy gate!
}

Monstrous \daemons{} who worship \Secherdamon{} appear and attack.

\lyricsbalsagoth{Beneath the Crimson Vaults of Cydonia}{
  Unruly evil!\\
  Colossal shapes etched against the moons, \\
  supine obeisance 'fore the mound. \\
  Accursed fiends hail the Slitherer, \\
  abhorrent jaws drooling lunacy.
}

\Secherdamon{} himself manifests as a ghost-like avatar. 

\lyricsbalsagoth{Beneath the Crimson Vaults of Cydonia}{
  The Abyss yawns wide\prikker Claws sharpened on the dead.\\
  The Abyss yawns wide\prikker Ensanguined fangs agleam.
  
  Great shadow, awaken and eclipse the suns of a thousand worlds\prikker\\
  Slumbering 'neath these crimson vaults, \\
  behold the majesty of the Outer Darkness!\\
  Praise the Z'xulth!
}


\lyricsbalsagoth{Beneath the Crimson Vaults of Cydonia}{
  Fell Worm of the Black Galaxy, \\
  awaken and descend without pity upon the Tellurian sphere!\\
  Destroy the flaccid priests of the newborn usurper faiths.\\
  Sweep away the thralls of the cruciform stave!\\
  Crush the lackeys of the corrupted hexagram!\\
  Devour the slaves of the eastern crescent!\\
  Crush them, grind them, slay them all!\\
  Plague-blessed, flay them alive!
  
  Now, behold in terror what waits beneath the crimson vaults of Cydonia\prikker
}







\subsection[Psyrex thanks Tiroco]{\Psyrex{} thanks \Tiroco}
\target{Secherdamon lets Tiroco and Icor choose their fate}
\Psyrex{} approaches \Tiroco{}. He thanks her for her help and gives her an offer to serve her. Knowing that he is a Hellish enemy of the Light who betrayed her and her entire city, she refuses and curses him. He expected as much. Still feeling that he owes her a favour, he lets her choose her fate. She has proven herself skillful enough that he cannot simply let her go, so if she will not join him, she must die. %He explains that he has shown mercy to capable enemies in the past and always lived to regret it. 

He remarks that: 
\ta{%
  Were I \Ishnaruchaefir, I might fall prey to sentimentality and spare you. 
  But I am wiser than that. 
  Show mercy to a capable enemy and you will live to regret it.}

But since she did it all for \ps{\Icor} sake, she deserves at least to be with him. So he grants her a choice: He can kill her and release \ps{\Icor} soul, letting them go into the Light as they wish. Or he can destroy them permanently, so that they will know peace. He advises them that the Light is not what they believe it to be, and that he would recommend annihilation rather than a return to slavery. 

%But she does not want to listen to him and chooses the Light. He is somewhat sad to oblige her. 
\Tiroco{} does not want to listen and chooses the Light. But \Icor{} objects. He has done some thinking and come to the conclusion that the \Sephiroth{} are not as good as people think. So he votes for annihilation. After a while, \Tiroco{} consents. 

Have a sad yet somewhat happy scene where the two drift off into oblivion in each other's arms. 

Then \Psyrex{} seats himself on his throne, surveying his new conquest. He muses that the \charade{} is broken. It is also interesting that \Ishnaruchaefir{} has returned to the \secretwar. 

Afterwards, \Psyrex{} comments to \Secherdamon{} (or \Nzessuacrith, or another) that it was difficult to save them. 
The \sephiroth{} had a strong grip on them. 
Moreover, the \quo{river} of \Iquin{} has grown stronger in recent years. 
Wilder, more raging. 
As if it is breaking out of control. 
(The Shroud is \hs{unravelling}.) 

\Psyrex{} fears the \sephiroth{} and \iquin{} and Iquinianism. 
Perhaps it should just be explained and made to seem as if he dislikes the Iquinian religion and does not want to give it any more souls and thus more power. 
Just keep it as a subtle undertone that he really believes an afterlife in \iquin{} would be painful for them\dash a fate he would not wish upon them, \honourable folk as they are. 
(After all, I don't want the reader to know that \Iquin{} is evil just yet.)





\subsubsection{Why do you do this?}
Before she dies, \Tiroco{} asks \Psyrex: \ta{Why do you do this?}

\Psyrex: 
\ta{For survival. 
  For the future of my people. 
  And your people as well, \scatha. 
  For our kind's survival, against the \humans{} and their creators.}

\Tiroco{} wonders who the \pps{\humans} \quo{creators} might be. Then she dies. 







\subsection[\Ishnaruchaefir returns to Malcur]{\Ishnaruchaefir returns to \Malcur}





\subsubsection{\Criseis meets \Psyrex}
Perhaps \Criseis meets \Psyrex in \Malcur.
They might call each other \quo{cousin} (if \Psyrex is a \scatha, that is). 





\subsubsection{Saves Rian and Neina}
\target{Ishnaruchaefir saves Rian and Neina}
Rian and Neina \hr{Rian and Neina are separated from Moro}{have gotten separated from Moro}. They flee and huddle together in the collapsing, mutating city. 

Then \Ishnaruchaefir{} arrives in \Malcur to bear witness to the rise of \Nithdornazsh. 

It is \Criseis who espies Rian.
She begs \Ishnaruchaefir to go down and help Rian by killing or scaring away the monsters that are about to eat him.
\Ishnaruchaefir \hr{Ishnaruchaefir's compassion}{finds sympathy for these small creatures} and obliges her.
Rian and Neina cower in horror of the vast \dragon.
\Criseis then dismounts and leads the \humans to safety while \Ishnaruchaefir goes to talk to \Nzessuacrith or something.

Rian sees something that he recognizes in \ps{\Ishnaruchaefir} eyes, although he isn't sure what. 
He notices that the \dragon{} has huge, bleeding wounds that go all the way through his body (after being \hr{Ishnaruchaefir impaled by spines}{impaled on huge spines}). 

\Ishnaruchaefir{} descends, lands near Rian and assumes humanoid form. Rian recognizes him: It is the mighty warrior-mage who {once inspired him to become a better man}. 
\Ishnaruchaefir is aware of this to some extent, and is glad to see that Rian has indeed become a better man. 

\Ishnaruchaefir{} is wise and can read much of Rian's story in his eyes. He sees immediately that the young man has fought bravely to save his beloved. This touches him, for he wishes he \hr{Ishnaruchaefir slays his beloved}{could have done same}. He also feels some kind of responsiblity for them, since he was part of the plan that destroyed their city. 
So, in a \trope{PetTheDog}{Pet the Dog} moment, he decides to save Rian and Neina. 

Some \daemons{} or evil warriors approach. \Ishnaruchaefir{} sends them packing with a single word. He then saves Rian and Neina, sending them through a portal out of the city and into an area that should be semi-safe. 

He sees that Neina is \hr{Neina's story}{badly traumatized}. So he casts a spell on her, erasing much of her memory of the last months. This has the side effect of making her permanently dumber than before. \Ishnaruchaefir{} warns Rian that he has done this, and that many memories have been lost other than those of her captivity (this kind of magic is inaccurate and risky, and it is not \ps{\Ishnaruchaefir} specialty). 

After the spell, Neina is not quite sure who Rian is, but some part of her remembers him and the love she had for him. He is hopeful and trusts that he can nurture her back to health. 

They thank \Ishnaruchaefir{} of all their hearts. Then they go through, hand in hand. 
We see them kissing and embracing, and it is implied that, despite whatever hardships cruel \Miith{} will throw at them, they will live happily every after. 

After the escape from \Malcur, Rian and Neina are forever scarred from witnessing their home transforming into a nightmarish hell from within.

\citebandsong{Nile:AnnihilationoftheWicked}{Nile}{
  Von Unausspechlichen Kulten
}{
  I Dare Not Again Surrender to the Deep Sleep Which Ever Beckons Me.\\
  Lest I in Dread Shudder at the Nameless Things.\\
  That May at this Very Moment Be Crawling and Lurking.\\
  At the Slimy Edges of My Conciousness.\\
  Slithering Forth from the Bowels of Their Infernal Pits.\\
  Worshipping Their Ancient Stone Idols and \\
  Carving Their Own Detestable Likenesses \\
  On Subterranean Obelisks of Blood-soaked Granite.
}

Neina goes mad and remains mad ever after.

\citebandsong{Nile:RamsesBringerofWar}{Nile}{
  Howling of the Jinn:
}{
  Fiendish Insects encircle Me\\
  Howling Wind Wraiths\\
  Surround my disembodied Ka

  Dulcarnon\\
  Hideous Unseen\\
  Speaking in Tongues\\
  Heard only by the Mad

  Shrieking Insects Swarm over Me\\
  Suffocate Me\\
  Suffocate my Soul

  Majnun I am Empty\\
  Crawling Reptiles Devour my Soul\\
  They utterly and completely Annihilate Me\\
  I can hear the Howling of the Djinn\\
  Echoing in the mountains of Kaf
}





\subsubsection{Rian and Neina Rescued}
Rian has gotten Neina out, but they are both hopeless.
They are sure it is the end of the world.
Rian feels he can only pray for salvation before he inevitably dies. 
But then \Criseis comes and saves them. 
She takes them out of the city.

Rian asks \Criseis: 
\ta{But where are we supposed to go form here? 
  If it's the end of the world\prikker}

\Criseis: 
\ta{It is not the end of the world.}
She looks thoughtful and worried for a moment.
\ta{At least, not just yet.}

Rian: \ta{But the city\prikker}

\ta{It is the end of \Malcur, but not the world. 
  Listen.
  I have done what I could for you.
  You may not be unscathed, but at least you are alive.
  You have a chance to make a new beginning.
  Good luck.}





\subsubsection{Saves Moro}
Maybe \Criseis saves not only Rian and Neina, but also \MoroCobrel.

Moro has not been idle.
After realizing that the battle for the city is lost, she has worked hard to save as many people as possible. 
She has herded a lot of innocent people together and used her magic to protect them as best she could. 
Perhaps she is rejoined by Rian and Neina (if they are not dead).

But Moro is getting desperate.
She cannot get out of \Malcur. 
She fears it is only a matter of time before they all suffer a horrible death. 
But then \Criseis shows up and shows compassion. 

\Criseis cannot save them all alone.
She has to appeal to \Ishnaruchaefir (her master and god) and pray for him to grant her power so that she can open a tunnel and let them all escape. 
\Ishnaruchaefir, in a fit of compassion (a \trope{PetTheDog}{Pet the Dog} moment), indulges her and saves the people. 





\subsubsection{Meets \Nzessuacrith}
He meets \Nzessuacrith{} in \Malcur. 
She has just returned from the Ghost Tower, where she was hurt in combat and scared off by a wave of \sathariah{} power. 

She is in a bad shape after her fight. 
She has big, open, bleeding wounds. 
Her silvery scales are blackened and dirty. 
Many of her \hs{ward runes} are depleted. 
She is so wounded that she dares not go into the Shroud. 
Its constricting influence would cut off the flow of life-giving \xsic{} energy that is currently sustaining her, leaving her with a masked body too weak to cope with the bleeding wounds. 
That might (temporarily) kill her. 

\Nzessuacrith: 
\ta{The only power at their disposal that can match ours. 
    Stolen from us as it is.} 

But she is ashamed of the fact that a little \sathariah{} smell was enough to scare her so and force her to flee. 
She does not openly admit it to \Ishnaruchaefir, and only to \Secherdamon{} after treading water for a while. 

The \dragons \hr{True Draconic signifies emotion}{spoke \TrueDraconic out of emotion}. 

They do not \hr{Dragon violence}{snap and lash out at each other}, as a friendly pair of \dragons{} would. 
They just stare coldly. 
They don't like each other. 

\Ishnaruchaefir{} is intrigued by the \sathariah{} part. 
He flies out to the Ghost Tower general area to investigate. 
He is wiser than she and can see \matrices{} and \vertices{} more clearly. 
He tracks down Carzain to say hello. 

Maybe \Ishnaruchaefir{} is the one who tells \Nzessuacrith{} about \ps{\Secherdamon} true plan, and how she was just a diversion. 

In this scene, \Nzessuacrith{} for the first time calls \Ishnaruchaefir{} \quo{father}. 

\Nzessuacrith{} and \Secherdamon{} call \Ishnaruchaefir{} \quo{Exile}. 
They refuse to afford him the \honour of pronouncing his name. 

It is mentioned that the whole thing is a major breach of the \charade. 

\Ishnaruchaefir initially believes \Secherdamon's plan takes place in \Malcur.
Then he falls for the \Forclin decoy when it is unveiled. 
He is impressed to learn that both were decoys. 
When he talks to \Nzessuacrith, he admits this. 

\Secherdamon tells \Nzessuacrith that she took her damn time.
She only assumed her true form at the very last minute. 
She \hr{Takestsha will not become Nzessuacrith too soon}{was well aware of this}.
She retorts that she is not one to flaunt the Unspoken covenant. 

Remember to explain the role of \Ambrose \Onatol and \Jirad Tantor and the diary. 





\subsubsection{He meets Carzain}
While Carzain-tachi are making their way through the \Wylde{}, \Ishnaruchaefir{} appears. 

Maybe he secretly sends monsters to attack them, then to come in and rescue them. 

He talks to them.

Remember that \Ishnaruchaefir{} does not speak \Velcadian. 
He does, however, speak Imetric and Vaimon. 

\Ishnaruchaefir{} is known and infamous to the Redcor. 
\Esmerel{} recognizes him and scorns him as a evil sorcerer. 
Northstar recognizes his name, but is uncertain as to whether he is good or evil. 

(Why? 
What is \ps{\Ishnaruchaefir} back-story? 
Who knows him? 
Who likes him? 
Who hates him? 
Who fears him?)

Carzain: 
\ta{Who are you? 
    Whose side are you on? 
    Are you good or evil?}

\Ishnaruchaefir{}: 
\ta{I\prikker simply am what I am.}

Carzain: (Thinks about it for a moment.) 
\ta{What a stupid answer. 
    What a feeble way of dodging the question!}

\Ishnaruchaefir: \ta{Hm. True.} Smiles. 
\ta{Very well. 
    The truth: 
    I do not know. 
    I fear you will have to judge for yourself.} 

See, this was a test. 
\Ishnaruchaefir{} \hr{Ishnaruchaefir's inanities}{sometimes does this}; spouting inanities to see if the other guy accepts them as profound wisdom or recognizes them as inane and calls him out on it. 

\Ishnaruchaefir{} tells Carzain: 
\ta{I predict that you and I shall meet again. 
    Perhaps as allies, but more likely as enemies. 
    The world is cruel, do you not agree?}

After Carzain-tachi depart, \Criseis{} asks him: 
\ta{Why did you not kill him, my Lord, knowing what he is?}

\ta{What excitement would there be left in life if I were to slay all my enemies when they be still small?}

\Ishnaruchaefir{} thinks to himself: 
\tho{%
  He has the \sathariah{} smell. 
  I ought to deeply hate him. 
  And I do.
  
  Maybe \Criseis{} is right. 
  Maybe I should kill him. 
  But what the Hell?
  If I kill him now, there will just be another Scion in a couple of centuries. 
  Might as well let him grow up and deal with him now.} 







\subsection{Epilogue}
\subsubsection{Carzain}
Carzain and his party finally set out for \Redce. 





\subsubsection{Charcoal is rewarded}
\target{Achsah rewards Charcoal}
Charcoal is rewarded by \Achsah{} for \hr{Charcoal at the Ghost Tower}{helping to guard the valuable Ghost Tower}.

%If it's \Achsah, then perhaps she rewards him by giving him the best sex he has ever had.
Perhaps \Achsah{} rewards him by giving him the best sex he has ever had.

Inside, Charcoal is thinking: 
\tho{I'm glad she doesn't mention the things I fucked up in \Malcur. 
  But, after all, I wasn't present. I can wipe the blame off on Needle\dash and \Achsah{} herself, and even \Teshrial.}





\subsubsection{\Achsah{} curses the \dragons}
\Achsah, on her throne in \Nyx, curses at the audacity of the \dragons. The resurrection of a \Machaic{} fortress on \Miith{} has never been attempted before and is a blatant breach of the \charade. 

She is also baffled by \ps{\Ishnaruchaefir} ingenuity. She had thought that the mystic warrior remained mostly aloof from the \secretwar{} and had not thought that he would interfere so directly, much less to the aid of his brother. And least of all did she expect him to see through the plan with the \noggyaleth{} beneath \Malcur. 

\ta{Damn \Secherdamon, that cheat. And damn \Ishnaruchaefir, that anarchist.}

Still, she is grateful that \Ishnaruchaefir{} has killed \Teshrial, whom she hated. 





\subsubsection{Some people wonder about the \quo{survival}}
Some people, both \human{} and \scathaese{}, speculate on the things they were told about how the feuding factions are allegedly fighting for: 
\quo{Survival. 
  The survival of our race. The \emph{purpose} of our race. We are waging the war that we were \emph{born} to wage.} 





\subsubsection{\Azraid{} muses}
\target{Azraid muses on Exile and Pyre}
\Azraid{} sits and muses about \Ishnaruchaefir{} and his involvement. 
\Azraid{} is the \apex{} of his \matrix, so he detected it immediately when \Teshrial{} was destroyed. 
But he doesn't know the details of the fight. 

He thinks about \Ishnaruchaefir{} helping \Secherdamon. 
\hr{Exile intersecting with Pyre}{The Exile did not intersect with the Pyre}. 
And yet, it still did. 
In intent and effect, if not in the formal, metaphysical sense. 

Remember to have subtle references to \ps{\Azraid} evil hand in all \Azraid{} chapters. 

\begin{prose}
  \tho{But it is the thought that counts. 
    Let that be a lesson for all astrologers.
    The Star-Maps of the Cosmos can reveal much, but it cannot tell us all things that are worth knowing.}
\end{prose}

\target{Azraid learns of spike}
He also thinks about the \vertexspike{}. 
\Achsah{} is $100\%$ sure the \vertex{} is a \sathariah{} aligned with the \hs{Midnight Bat}. 
Which means it must be a Scion. 
Which means there are only two possibilities. 

\Azraid{} is very interested. 
He wants Ramiel back. 
\Azraid{} himself \hr{Azraid dies}{is preparing to die}, so he wants to set up an heir to take over leadership of the \resphan{} race when he is gone. 
Ramiel is a good candidate. 

So \Azraid{} \hr{Azraid protects Carzain}{assumes the role of Ramiel's \quo{guardian angel}}, subtly pulling threads and protecting him from harm while he grows towards his \apotheosis. 
(This is necessary. Many forces conspire to kill Ramiel's Scion.)

\Azraid{} is distressed about the resurrection of \Nithdornazsh, but only in a detached, relaxed way, 
It is an annoyance that will make his work more difficult in the future. 






\subsubsection{\Secherdamon{} and \Nzessuacrith}
\Secherdamon{} and \Nzessuacrith{} talk. 
She mopes over being defeated. 
He had previously promised to come to her assistance if she needed it, so she berates him for not coming. 
He explains that he did indeed come to her assistance, for his actions in \Malcur accomplished more than the capture of the Ghost Tower could possibly do. 
He thanks her for her help, for it was the failure of her initial, more covert plan which forced her to go in with full force. 
This in turn drew all Cabal eyes to her, which allowed him to carry out his plan. 
All of which he had planned in advance. 

He apologizes for having used her. 
\Secherdamon{} and \Nzessuacrith, of all people, understand very well the pain of being betrayed by their own family. 
They are close allies and should be able to trust each other. 
They are more sensitive and considerate than the irreverent \Ishnaruchaefir. 

It is mentioned that the whole thing is a major breach of the \charade. 

\Secherdamon{} muses about \ps{\Ishnaruchaefir} unpredictability. 
He also reveals that \Ishnaruchaefir{} is his brother. 
He is greatly disappointed that \Ishnaruchaefir{} did not die. 
\Secherdamon{} \hr{Secherdamon thinks Ishnaruchaefir will sacrifice himself}{had hoped \Ishnaruchaefir{} would sacrifice himself}. 

After the resurrection of \Nithdornazsh, the Sentinels have a lot of work ahead of them.
They have to restore the city and build new eidola to shape the Shroud and the gateways and facilitate the travel between the Realms that they want. 

The book closes with \Secherdamon{} musing:
\tho{The Unspoken Covenant is broken. 
  Open war is upon us. 
  Perhaps even a \thirdbanewar.
  Yes. 
  The final war is upon us.
  And it has only just begun\prikker}















\section{Changes}










\subsection{The Dreaming Predator}
\begin{changes}
  \begin{comment}\paragraph{Prologue}\end{comment}
  \changesitem{Prologue} 
    \Nzessuacrith is horrified and saddened to find that she cannot contact \Secherdamon telepathically like she could before the Shrouding. 
    She can only talk to him using words.
  
  \begin{comment}\paragraph{Wanderer in Darkness}\end{comment}
  \changesitem{Wanderer in Darkness} 
    
    In the beginning \Teshrial sees an \umbra in the distance.
    He shuddered and turned away, for he knew that there are things in the universe which it is not good to look upon or think about, even for a \resphan.
    
    Have throwaway references to \hr{Mystic names}{mystic names and places}, like Shung. 
  
    Remember that the \humans should be various exotic \demihuman varieties.
    
    Have some more \hs{astrology}. 
    
    \Achsah is not a High Telepath.
    She is merely in contact with one.
    Read about \hs{High Telepaths}.
    
    Mention this:
    \quo{Not even mighty \sathariah warriors like \Nathrach, \Netzachirah or even the great \Morcariel, founder of \CiriathSepher, were able to stand against the wrath of the Destroyer.}  
    
    \Teshrial mentions that he has life-seeds.
    
    Allan Balsgaard has the following comment for the chapters \quo{Wanderer in Darkness} and \quo{What Slithers Beneath}: 
    \ta{%
      Jeg ved godt tyskeren i dig protesterer voldsomt, men m\aa{}ske kunne du sk\ae{}rer ned p\aa{} antallet af navne. 
      Det er ikke n\o{}dvendigt at navngive hovedpersonens bror eller lign. 
      Det kan du g\o{}re i appendikset. 
      Husk at ogs\aa{} almindelige mennesker skal kunne l\ae{}se det.
      Desuden. 
      Kan de udf\o{}rlige beskrivelser af guderne, deres relationer og deres kampe ikke d\ae{}mpes lidt? 
      Hvis du virkelig ikke kan spare dig, s\aa{} lav det helst som en fort\ae{}lling om dem uden replikker. 
      Jeg er bange for at guderne bliver for opbrugte og trivielle, n\aa{}r du skal til at skrive bog nr. 4. 
      Jeg tror du risikerer at de p\aa{} et tidspunkt bliver afmystificeret og dermed kedelige for l\ae{}seren.%
    }
    
  
  
  \begin{comment}
    \paragraph{What Slithers Beneath: To Do}
  \end{comment}
  \changesitem{What Slithers Beneath: To Do} 
    Update the reference to when \Achsah met \Ishnaruchaefir.
    Add more details and make it more epic and ominous. 
    Do likewise for the reference to when \Teshrial fought \Zessuruch. 
    
    Do not show \Secherdamon, but do mention him a lot. 
    He is a terrible dark lord (the Serpentine Lord), and \LocarPsyrex is his almost-as-terrible immortal archmage. 
    Make \Psyrex more badass, more menacing, more terrifying. 
    Even the \resphain hesitate to fight him.
    
    Overall, namedrop \Psyrex and \Secherdamon more. 
    
    Have a scene with \Psyrex, the dark sorcerer.
    He thinks about \Ishnaruchaefir and the \resphain.
    He contacts his ally, who in these days goes by the name of \Takestsha. 
    
    He tells her:
    \ta{I have made a discovery.
      I have news that will greatly interest you.
      News of \emph{someone} who greatly interests you\prikker}
\end{changes}









\subsection{On the Wings of \Dragons Unseen}
\begin{changes}

  \begin{comment}
  \paragraph{Screaming in the Dark}
  \end{comment}
  \changesitem{Screaming in the Dark}
    Change the \scathae into \humans.
    
    Have references to the \hr{Wylde gods}{hideous gods of the \wylde}. 
    The bandits fear them.
    
    Grum's raiders pass by some sinister, moss-grown ruins in the \wylde.
    They remind Faeni of the fell, pre-\human cities with names like \hs{Ibthek} and \hs{Su-Gelba}, whereof legends paint a foreboding picture. 
    They stay far away from these ruins and move on.
    Instead they camp in this abandoned cemetery which, while eerie, is at least the built by the hands of mortal humanoids. 
    
    The creature that Carzain summons is a \hs{Night-Feaster}. 
    He summons it using two of the three \hr{Vymorjan Chants}{\Vymorjan Chants}. 
    (Under normal circumstances he would use all three chants, but here he just uses chants I and II. 
    Chant III is a very long and expensive spell and not really necessary here.
    Chant I is easy and cheap here in the woods at night; he does not have to chant for very long before a Feaster appears.
    Then he casts chant II and lets the Feaster roam free.
    The Feaster terrorizes the thugs a little, then flies away.
    It does not really want to stay.)
    
    In the end, Carzain and Vizicar resolve to go find a cute girl for the night. 
  
  \begin{comment}\paragraph{The Mystery of \EreshKal}\end{comment}
  \changesitem{The Mystery of \EreshKal}
    Look at \href{http://limyaael.livejournal.com/167123.html}{Limyaael's \quo{Sounds of the Jungle} rant} for ideas and advice on the forest scenes. 
    
    Replace \quo{forest} with \quo{\hr{Jungle}{jungle}}. 
    
    Read some Robert E. Howard before I write the scenes with Carzain and Tantor in the \wylde. 
    
    Tantor's diary is written in \hr{Rungeran language}{Rungeran}.
    Fortunately \hr{Curwen's languages}{Curwen speaks Rungeran}.
    
    Or maybe it is written in the Vaimon tongue, as Vaimons are wont to do.
    
    Curwen quickly figures out that \hr{Jirad Tantor}{\Jirad{} Tantor} must be a kinsman of the \scarv{} of Tantor. 
    \hs{Most mages are nobles}, after all.
    
    Remove the part where Tantor looks up in a book to read about \EreshKal.
    The fey kingdom of \EreshKal is well-known to occultists (in broad terms, at least).
    
    Perhaps \Takestsha takes the form of a strange \hr{Demihuman}{\demihuman} to add to her mystery and allure.
    
    Remember that there are \hr{TBW railroads}{railroads}. 
    Tantor-tachi go by railroad to Gedrock. 
    
    Tantor-tachi have one or more (small) sauropods as draft animals. 
    
    There is a \hr{Wylde border}{\Wylde{} border} around Gedrock, which Tantor passes by. 
    
    Tantor needs to be more racist. 
    When he hears there are \meccara{} in \EreshKal, he scoffs. 
    \Meccara{} are \quo{lower humanoids}, clearly less intelligent and civilized and worthwhile than \humans{} and \scathae. 
    He does not expect much of them, and he certainly does not fear them. 
    
    Charcoal does not call him out on this. 
    Charcoal is a racist himself.

    Everyone on \Azmith knows that monsters and evil spirits and even evil gods lurk in the Wild and in the dark, forbidden places of the world.
    \Jirad Tantor should not dismiss things as superstition.
    Rather, they know they are venturing into a dangerous place full of monsters.
    Still, the forest is not so huge, and it is in the middle of Runger, surrounded by churches and Light-fearing men, so what is the worst that could happen?
    
    When they are in Gedrock, make it clear that life in a village is dangerous. 
    They have \eidola, but even so, the \wylde is always close and threatening.
    The local priests have much work to do in keeping the \eidola blessed and keeping the \wylde out. 

    Remember to see the section about the \hr{Wylde}{\wylde}, and especially the one on \hr{Travelling through the Wylde}{\travelling through the \wylde}. 
    
    Tantor is certain that \EreshKal was not built by \meccaran hands. 
    But his version of the story is very much \coloured by his racism. 
    He looks down on \meccara. 
    
    Be sure to have a clear difference in Tantor's writing style before and after his son's death. 
    Before it, he looks down on the \meccara as \quo{lower \humanoids}. 
    They are repulsive, but he also sort of pities them. 
    And he is impressed and awed by the grandness of the temple. 
    
    After his son's death, he comes to viciously hate \meccara. 
    He curses them for their evil, stupidity, inferiority, ugliness, bad smell and everything he can think of. 
    And he now feels horror rather than awe at the temple. 
    
    Already out in the forest, he and the others see abhorrent things skulking and creeping at the edges of their camp. 
    Ugly midget \humanoids. 
    Probably \meccara. 
    Sometimes the soldiers shoot at them. 
    One soldier hits. 
    The thing shrieks and bleeds, but escapes, and no one wants to pursue the wretched thing out into the \wylde. 
    But they can see the blood. 
    They know it is a living creature that bleeds. 
    That reassures them. 
    A bit. 
    
    Inside the temple they see more of these skulking shapes.
    
    It is important that I hint at more than I show. 
    I want to evoke something like \cite{HPLovecraft:AttheMountainsofMadness}, \emph{not} something like \cite{JohnGlasby:TheBroodingCity}. 
    It is mostly Curwen who sees hints of dark things, for he knows much more of the occult and the dark, forbidden myths than Tantor does.
    He recognizes things from the sinister prehistory of the \scatha/\meccaran race.
    Curwen knew the truth (or parts of it) about how the \meccara degenerated. 
    (See the section about \hr{Meccaran Scathae}{\meccaran \scathae}.)
    
    Allan Balsgaard has this comment:
    \ta{%
      Rigtig fedt. 
      Dog kunne man godt g\o{}re det mere utydeligt hvad templet indeholdt.
      Ankomsten til skoven kunne m\aa{}ske godt v\ae{}re strukket lidt, s\aa{} man f\aa{}r en fornemmelse af rejse og afstand (lidt opvarmning). 
      Hvad med lidt naturbeskrivelser, beskrivelser af almindelige mennesker, b\o{}nner, vagabonder og andre rejsende??
      Samtalen med den gamle kone er rigtig god, dog kunne man godt v\ae{}re blevet sparet lidt for de mange detaljer. 
      Igen antydningens kunst.
      Takestsha er virkelig spooky, hun sover ikke\prikker%
    }
    
    The spooky temple makes Tantor glimpse the Beyond. 
    He sees weird and terrible visions. 
    See the section about \quo{Breaking through the Shroud} for inspiration.
  
  
  \begin{comment}
    \paragraph{The Terror of \EreshKal}
  \end{comment}
  \changesitem{The Terror of \EreshKal}
    In the forest near \Forclin, Carzain comes across a dark and foreboding place with gnarled, crumbling standing stones.
    The place is empty, but he has read sinister literature (including fragments of the reputedly pre-\human \BathShemTorradjErebossha). 
    The legends describe such cromlechs and hint at what manner of nameless, grotesque things come down from outer space to places like this to dance and feed when the moons stand right in the sky. 
    And he knows about the repellent \hr{Glithid}{\glithids} that come from the forest to answer their summons and attend them. 
    He remembers the black charnel god \KhothSell and other names which he shudders to recall.
    Carzain keeps his distance and quickly moves on. 
    
    See \cite[p.366]{HPLovecraft:TheDreamQuestofUnknownKadath}. 
    Use the physical description as inspiration. 
    Also see \cite{HPLovecraft:TheStatementofRandolphCarter}. 
    
    Read about the \hr{Moon-god}{\moongods} and other \hr{Wylde gods}{gods of the \wylde}. 
    
    When he feels the dark magic, he is reminded of that age-old sinister cromlech in the forest.
    From the \BathShemTorradjErebossha he recalls the name of black cosmic entities. 
    \emph{\KhothSell{}} and other dreadful cosmic entities.
  
    Carzain should say something about how evil Morgan Runger is.
    Remember, I want the reader to really hate Runger and root for Pelidor. 
    
    Carzain muses.
    Maybe Morgan just wants to conquer and rule like any other king.
    But his \ishrah is evil. 
    They mean to unleash forces of hideous, world-consuming evil. 
    Carzain fears them. 
    
    Compare to how evil the Wasp Empire is in \cite{AdrianTchaikovsky:ShadowsoftheApt}. 
    
    
  \begin{comment}
    \paragraph{The Gods of \EreshKal}
  \end{comment}
  \changesitem{The Gods of \EreshKal}
    
    Read some Robert E. Howard before I write the scenes with Carzain and Tantor in the \wylde. 
    
    Since the last time he read in Tantor's diary, Curwen has done some research.
    He has found out that this \Jirad Tantor is indeed a real person and a member of King Morgan's \ishrah.
    \Jirad is a younger son of a nobleman, much like Curwen himself. 
    Many mages are like that in countries where the eldest child inherits stuff.
    Firstborn sons often shy away from magic because it is scary and they would rather just rule stuff.
    Most people, even though they may worship their \Archons or gods, fear close contact with magic. 
    Younger sons sometimes decide that magic is their best chance of power and glory.
    Some of them manage to get apprenticed, and some of these manage to become true mages. 
    
    Maybe change Tantor into a \scatha. 
    If so, mention it here.
    
    Replace \quo{forest} with \quo{\hr{Jungle}{jungle}}. 
    
    As soon as Tantor steps inside the entrance to \EreshKal{} he imagines he feels the dust and darkness of uncounted millenia. 
    He believes the building is extremely old. 
    From before the \Human{} Age. 
    Older than Cordos Vaimon. 
    
    It is well-known that the \meccara{} are a younger race than \humans. 
    But Tantor believes that these \EreshKali{} are the caretakers of a tradition of sinister, forbidden wisdom far older than they themselves. 
    A legacy that reaches back to the races and empires that reigned immemorial aeons ago and were forgotten before the first \humans{} set foot on \Miith. 
    
    \EreshKal{} is the ruins of what was originally a \quiljaaran{} city. 
    Tantor refers to the \hr{Myths of vanquished monsters}{myths of Iquinian heroes vanquishing inhuman Elder Races and monsters}. 
    He sees \EreshKal{} as a leftover from the mythical \hs{Age of Gods}. 
    He daydreams of the ancient wars when \humans{} vanquished the serpent men. 

    Even \Jirad Tantor is impressed by the fallen temple of EreshKal and wonders what it was like when it was inhabited and at the height of its power and glory.
    
    The buildings are huge, being \quiljaaran-built. 
    (Read about \hr{QJ architecture}{\quiljaaran architecture} and \hr{Ophidian architecture}{\ophidian architecture}.) 

    \citebandsong{Nile:BlackSeedsofVengeance}{Nile}{
      To Dream of Ur
    }{
      Desolate and Forsaken \\
      Eerily Moaning Dark Winds\\
      Murmur Incantations\\
      Dusk Calls Forth Shadows
    
      Spirits of the Glorious Dead \\
      Lingering, Bound to this Place\\
      They Whisper of Untold Sagas, of Long Dead Cities\\
      the Seven Shining Cities Sacred to the Aphkhallu
    
      Of Ages Past when the World was Young\\
      When Babylon was Blessed of Marduk\\
      and the Sound of her Armies was the Blare of Ominous War Horns\\
      and the Clash of Immortal Cymbals\\
      of Bronze Gates Arrayed in Splendour\\
      and Magnificent Walls of Sunbaked Brick \\
      Of Temples of Marble and Bloodstained Altars
    
      Long Before the Jeweled Throne of Ur\\
      Fell Silent and Turned to Dust\\
      Beneath the Endless Shifting Sands\\
      and the Inevitable Vengeance of the Elements
    }
    
    Charcoal laughs at Tantor. 
    He knows that the \quo{Age of Gods} is still going on and that the Elder Races are very much alive and kicking. 
  
    Describe the sounds and smells inside the \EreshKali{} temple. 
    The smell is musky and very old. 
    
    There are ruins of inhuman machines. 
    It is clear to see that the primitive \meccara are not the original inhabitants/founders/builders. 
    
    \citeauthorbook[p.76 of 138]{KarlEdwardWagner:DarknessWeaves}{%
      Karl Edward Wagner%
    }{%
      Darkness Weaves%
    }{
      Shambling man-sized creatures, who looked like monstrous hybrids of man and frog, stood watching Kane in the shattered chamber of some colossal prehuman structure. 
      Great bronze swords were clutched in webbed fists as they waited in the shadows of the cracked and leaning walls. 
      Slimy water covered much of the floor, and fleshy vines stole through jagged apertures to enshroud looming machines of unguessable nature. 
      A gigantic crystal filled the center of the chamber\dash a sullen dome nearly a hundred yards across, composed of a substance that resembled bloodstone. 
      The scarlet veins of the crystal suddenly seemed to glow with life. 
      Blinding flashes of coruscant energy burst from long-slumbering pillars of machinery, driving the amphibian creatures back in fear. 
      An eerie light of green, veined with red, shot forth from the depths of the awakened crystal and bathed Kane in its fire.
    }
    
    \citeauthorbook[p.49]{LeeClarkZumpe:PassagetoOblivion}{%
      Lee Clark Zumpe%
    }{%
      Passage to Oblivion%
    }{
      Spread before them, luminous monoliths towered, their apexes scratching at the starless night skies.
      Gargantuan statues\dash too abstract to be identifiable in nineteenth century terms\dash invoked idols of undreamt religions, while altars festooned with incadescent crystals surged with curious power. 
    }
    
    The Rungeran soldiers in Eresh-Kal wield guns. 
    (Read about \hs{guns}.)
    Tantor has a gun which he uses once to kill a Meccaran.
    Tantor tells: 
    \begin{prose}
      Tantor: 
      \ta{%
        These mongrel savages jump at the sound of a musket. I imagine they might never have seen guns before and think them to be sorcery.}
    \end{prose}
    
    Tantor needs to be more racist. 
    Emphasize how \quo{filthy} and \quo{degenerate} the \EreshKali{} are, even by \meccaran{} standards\dash and that is saying a lot, because even a regular \meccaran{} is pretty dirty and stupid and ugly and inferior. 
    Tantor describes how his people win because they are of superior races. 
    (Some of his men are \scathae. 
     They are, of course, not as good as \humans, but still clearly better than \meccara.
     \Scathae{} are \quo{higher humanoids}, although not the highest there is.)
    
    Tantor believes that \meccara are small (smaller than \humans and \scathae) because they are a lower race and less well-developed. 
    It seems to him that these \EreshKali are even more stunted and backward than the average \meccarans.
    They are particularly small and degenerate and weak. 
    
    Also write in the Glossary about \quo{lower} and \quo{higher} humanoids. 
    
    When \Mycah{} dies, Tantor thinks it is the ancient curse, cast upon this loathsome place by wicked pre-\human{} gods, that has caused it. 
    The old gods are bitter to have been displaced and hate the \humans{} that have since swarmed all over the world that was once theirs. 
    So they want revenge. 
    Encroaching on their territory can only end badly, Tantor muses. 
    
    Tantor finds some writing in the temple that he cannot read. 
    The symbols certainly look nothing like Vaimon nor \Ortaican{} script. 
    They are reminiscent of Rissitic glyphs, but Tantor is pretty sure that is not it, either. 
    He has reproduced some of them in the diary. 
    Charcoal cannot read them, either. 
    
    Charcoal wonders. 
    Can it really be true that there lay such a treasure trove of magical wealth hidden in Runger, right under everyone's noses, where any moderately-armed warband could just waltz in and claim it? 
    If so, how did \Takestsha{} learn of it, and why did no one else learn of it before her\dash especially if it really is as old as Tantor guesses? 
    It seems incredible. 
    The Cabal and/or Sentinels ought to have sniffed out this treasure and looted it long ago. 
    Something does not add up here\prikker 

    On the last day, after the big battle, they hear slavering noises.
    They realize there are more \meccara, and they have brought monsters with them. 
    They fear it is their gruesome gods (whose images and statues they saw in the big chamber) which have now awakened and are hunting the interlopers. 
    They flee out quickly, pursued by monstrous horrors. 
    They do not see the things clearly, but they get an impression of gray, slimy tentacles. 
    And they can hear their repellent high-pitched piping, mewling, whistling noises.
    And they can smell them.
    
    This ultimate horror is never seen, perhaps never even glimpsed. 
    Only hinted at.
    Compare to that undescribed ultimate horror that lies beyond the highest peaks in \cite{HPLovecraft:AttheMountainsofMadness}. 
    
    Several stragglers are grabbed by the monsters.
    But anyone who stops or turns around to try and help them are also killed. 
    There is nothing to do but run. 
    
    In the end, many survive, and now they have the valuable plaques they came for. 
    
    At least one soldier turns around and sees the monsters, but is rescued and survives. 
    He is driven stark raving mad. 
    
    Get rid of the sex scene with \Takestsha and Tantor. 
    It serves no purpose. 
    And brutally cut down on the part after the escape from \EreshKal. 
    
    Tantor now warns \Onatol. 
    King Morgan has his sights on Pelidor, and Tantor believes \Takestsha is exerting some considerable influence on the king. 
    She is after something.
    Something of great mystical significance.
    Tantor believes this something is to be found in the city of \Forclin.
    Some powerful artifact of the elder ages.
  
  \begin{comment}
    \paragraph{\Forclin}
  \end{comment}
  \changesitem{\Forclin}
    Explain what a \bacconate is.
    
    When Carzain sees the gargoyles, he should speculate that they are probably magical.
    They are monstrous and scary-looking.
    
    When Carzain sees the Ghost Tower, he sees the suggestion of shapes around it.
    Winged fiends circling about it. 
    
    Curwen should not go to Carzain.
    Curwen should say: \quo{Send him in.}
\end{changes}









\subsection{Spectre of the Fray}
\begin{changes}
  \begin{comment}
    \paragraph{The Ghost Tower}
  \end{comment}
  \changesitem{The Ghost Tower}
    In the Bila scene, remember to mention that the couriers from Dendrum came by boat on the Ucarn river. 
  
    Curwen thinks back to Tantor's diary, where Tantor claimed \Takestsha was after some occult target in the city of \Forclin.
    What could he have referred to? 
    Is there some artifact buried beneath the city that Curwen does not know about?
    
    Carzain stands at the Tower.
    He looks up but cannot see the winged fiends that he saw in the chapter \quo{\Forclin}. 
    But he imagines he hears their shrill howls. 
    
  \begin{comment}
    \paragraph{A Machine of Flesh and Iron}
  \end{comment}
  \changesitem{A Machine of Flesh and Iron}
    The Rungerans start bringing in their cannons. 
    The cannons come in by railroad. 

    But the Pelidorians have disabled the railroads. 
    They have taken down the rails and dug up the ground so the rails cannot be easily repaired. 
    They have also taken down some \eidola along the Ucarn road. 
    As a result, the roads have become less safe.
    The \wylde is slowly creeping in. 

    Destroying the roads is controversial.
    The religious people are not happy about it. 
    The Iquinians see the maintenance of roads as a religious duty.
    By destroying roads, Sethgal is allowing the evil of the \wylde to seep into the world and disrupt civilization.
    It is bad etiquette.
    But Sethgal does it anyway.
    He is pragmatic. 
    This can slow the Rungerans down, and that is a good thing. 
    It gives his people more time to prepare. 

    The Rungerans are forced to haul in their cannons over the rough land. 
    They use \nephil ogres to do some of their heavy work.

    The Rungerans have to fight their way through land that is gradually becoming \wylde. 
    This is not very dangerous, for the Rungerans have a large army, but it does slow them down.
    Besides, they have to devote resources to actively maintaining the road behind them so they have supply lines. 
    This weakens their combat forces. 
    Sethgal is sneaky. 

    But the Rungerans will not be deterred. 
    They send up forces first to encircle and besiege the city.
    Then they stand there. 
    They are encamped outside cannon range from the walls. 
    Gradually they fill it up with more men. 
    At last they bring in their cannons. 
    They have ogres that serve as heavy muscle for the cannons, to haul them back into place after each shot. 
    
    Sethgal sees the ogres. 
    They are revolting things\dash half-\human monsters, monstrous but also pitiable. 
    
    Remember that the Rungerans also bring in supplies by ship, from the river. 
    Also mention that there is a naval battle going on on the river, which might be very important. 
    Just mention it a few times, do not actually show the battle.
    Both sides try to bring in supplies by sea, all the while trying to sink or capture the enemy's ships and rob them of their supplies. 
    The fortune had shifted back and forth several times, no side seeming to gain naval superiority for long.
    It looked as though this battle would be decided on land. 
    
    
  \begin{comment}
    \paragraph{The Cannonade}
  \end{comment}
  \changesitem{The Cannonade}
    Koit says: 
    \ta{%
      Now, another thing is that if the cavalry is made up of the nobility, it would be nice to mention a few more nobles, so that we'd know who is going to war. It adds this aspect of personality that's otherwise missing.
      
      One such is to mention the outriders as 'The relc cavalry of x, y and z, who are the finest relc-breeders in the land' or so.
      
      Second, each noble family should have a banner or such. To say that "Sethgal sees the banners of x and y ride out, that the D-chap who died sees the bannerman of z fall, that Carzain hears the horn of q"
      
      Third is just giving a title and a name to a few chaps, eg, the man who wants to pick up the dying D-chap is referred to as 'The master-at-arms of Sethgal, Robin Hood ( or whoever he was) wanted to pick him up but was told to bugger off'
      
      The same with the Rungeran rally. 
      It might be that the Rungeran columns of x and y, under the banner of a red dancing dragon were the first to supress the shock of assault...
      
      Dornaer is fine, though mentioning her family wouldn't be too bad. Currently, we don't know why she has the command.}
      
      Compare to how \cite{Homer:Iliad} and \cite{GeorgeRRMartin:ASongofIceandFire} has tons of named characters with just a small bit of characterization.
      It adds worldbuilding and makes the world feel more epic and complete. 
    
    
  \begin{comment}
    \paragraph{The Power of \EreshKal}
  \end{comment}
  \changesitem{The Power of \EreshKal}
    Get rid of the pillar of sorcery.
    Replace it with something more ominous, more eerie, more subtle.
    A faint howling, a shimmering in the air.
    Deep groaning or rumbling noises from beneath the earth, or giant footfalls.
    
    Immense dull booms every once in a while, that seem to come from the sky or from everywhere at once, making everything tremble.
\end{changes}









\subsection{Malcur Thread}
\begin{changes}
  \begin{comment}\paragraph{Beyond the Veil}\end{comment}
  \changesitem{Beyond the Veil}
    Revise the funeral rites. 
    Think up a better funeral tradition. 
    See \href{http://limyaael.livejournal.com/179073.html}{Limyaael's death rant} for ideas. 

  \begin{comment}\paragraph{Veils that Divide}\end{comment}
  \changesitem{Veils that Divide}
    Mention \Isphet and the \qliphoth in the scene with \Icor's funeral. 


  
  \begin{comment}\paragraph{A Dark Angel's Gift}\end{comment}
  \changesitem{A Dark Angel's Gift}
    Maybe Needle is not offered command over \banes. 
    Maybe that doesn't happen until much later. 
  
  \begin{comment}\paragraph{Captured}\end{comment}
  \changesitem{Captured} 
    
    In all the Rian chapters, whenever it is appropriate, have references to the \hr{Myths of vanquished monsters}{myths of Iquinian heroes vanquishing inhuman Elder Races and monsters}. 
    When he encounters something supernatural, he fears that the wicked Elder monsters will conquer the world. 
    
    \hr{Rian is religious}{Make Rian more religious}.
    Make sure he prays in every chapter and scene that he is in.
    Make clear how grateful to the Light he is for how he has been freed from his life of crime and allowed to make a new, honest life for himself.
    He has lingering existential/religious dread from the day when he saw the dark sorcerer slay the shining god (even though he was Shrouded and does not remember it all). 
  
  \begin{comment}\paragraph{Trinity of Plagues}\end{comment}
  \changesitem{Trinity of Plagues} 
    Change title to \quo{Trinity of Plagues}. 
    
    In all the Rian chapters, whenever it is appropriate, have references to the \hr{Myths of vanquished monsters}{myths of Iquinian heroes vanquishing inhuman Elder Races and monsters}. 
    When he encounters something supernatural, he fears that the wicked Elder monsters will conquer the world. 
    
    \hr{Rian is religious}{Make Rian more religious}.
    Make sure he prays in every chapter and scene that he is in.
    Make clear how grateful to the Light he is for how he has been freed from his life of crime and allowed to make a new, honest life for himself.
    He prays to be delivered from \Isphet's evil. 
    He has lingering existential/religious dread from the day when he saw the dark sorcerer slay the shining god (even though he was Shrouded and does not remember it all). 
    
    \Uswa is in bad shape due to malnourishment and hunger.
    When Rian comes seeking her advice, he brings food.
    
    Dennick has been like an older brother to Rian ever since Rian's parents died. 
    
    Dennick has a wife. 
    She is not well.
    She is badly sick.
    Dennick fears she has the \hs{Disease}. 
    She lies in the next room.
    They let her lie in peace.
    
    Dennick does not offer beer. 
    Rian has a decent-paying job.
    So he brings beer to Dennick as a gift. 
    
    Dennick is a thief, but an honest thief. 
    Rian is sad that Dennick is still criminal, \hr{Iquinian prostitution metaphor}{prostutiting his soul} for material gain.
    Rian prays for Dennick's good but sinful soul every day. 
  \begin{comment}
  \paragraph{The \Qliphoth Lie Ever in Wait}
  \end{comment}
  \changesitem{The \Qliphoth Lie Ever in Wait}
    
    In all the Rian chapters, whenever it is appropriate, have references to the \hr{Myths of vanquished monsters}{myths of Iquinian heroes vanquishing inhuman Elder Races and monsters}. 
    When he encounters something supernatural, he fears that the wicked Elder monsters will conquer the world. 
    He prays to be delivered from \Isphet's evil. 
    
    \hr{Rian is religious}{Make Rian more religious}.
    Make sure he prays in every chapter and scene that he is in.
    Make clear how grateful to the Light he is for how he has been freed from his life of crime and allowed to make a new, honest life for himself.
    He has lingering existential/religious dread from the day when he saw the dark sorcerer slay the shining god (even though he was Shrouded and does not remember it all). 
  
  \begin{comment}
  \paragraph{The Thirsty Nether}
  \end{comment}
  \changesitem{The Thirsty Nether}
    Remember to read about the Shroud. 
  
    The sorcerers whom Rian see talk about how \quo{\hr{The Change of Malcur}{the Change}} is coming up. 
    
    Rian sees one of the \hr{QJ in Malcur}{\quiljaaran in \Malcur}. 
    (Read about them.) 
    Moro also sees it. 
    
    Needle knows she is a poor reader. 
    But she doesn't understand that it's because she learned it as an adult. 
    She thinks it's because she is lowborn, and that nobles simply have talent for reading and other fine arts by virtue of being nobles. 
    (Maybe ask on a forum how to best represent this.)
    
    In the chapter where \Tiroco meets \Uswa, mention that \ps{\Tiroco} bodyguard carries a sword and a pistol. 
    
    In the first scene where Rian flees: 
    He has seen something of the Beyond. 
    He still suffers from aftershock. 
    His mortal mind (otherwise driven by denial) has not fully recovered. 
    He is able to see through the Shroud to some small extent. 
    The streets look different.
    Houses and landmarks which he remembers are suddenly gone, replaced by new, unfamiliar ones. 
    Nothing is as he remembered it.
    He panics. 
    This should not be happening.
    He knows these streets. 
    He has been here many times to spy. 
    He runs around frantically and becomes hopelessly lost. 
    
  
  
  \begin{comment}
  \paragraph{The Dark Crypts of the Mind}
  \end{comment}
  \changesitem{The Dark Crypts of the Mind}
    Read about \hr{Moro}{\MoroCobrel} and \hr{Nasshikerr}{\Nasshikerr}. 
    
    Read about \hr{Ortaica}{\Ortaican mysticism} and \hr{Rethyax magic}{\rethyactic magic}. 
  
  
  \begin{comment}
    \paragraph{The Bleeding Wood}
  \end{comment}
  \changesitem[The Bleeding Wood]{\hr{Rian sees a raid}{The Bleeding Wood}}
    \begin{itemize}
      \item 
        Read about the Shroud. 
      \item 
        Have a scene before the main chapter where Rian goes to the hideout. 
        He sees some people in the shadows. 
        By a trick of the light, the people look like monsters to him for a second. 
        He is already somewhat paranoid, and he is beginning to see through the Shroud and see strange things that look monstrous and freakish to him.
        So his nerves are on edge. 
        
        \citeauthorbook[p.27]{StephenKing:CrouchEnd}{Stephen King}{Crouch End}{
          Standing on the corner beside their parked motorcycles were three boys in leathers.
          They looked up at the cab and for a moment\dash the setting sun was almost full in her face from this angle\dash it seemed that the bikers did not have \human heads at all.
          For that one moment she was nastily sure that the sleek, flat and sloping heads of rats sat atop those black leather jackets, rats with beady black eyes staring at the cab. 
          Then the light shifted just a tiny bit and she saw of course she had been mistaken; there were only three boys in their late teens there, smoking cigarettes\prikker
        }
      \item 
        Mention that \hs{Needle hates the Black Plague}. 
      \item 
        Rian should not see quite so much. 
        He knows that there is black magic and wickedness and conflict, but he doesn't know who is who. 
        He sees Needle, but doesn't know which side she is on, so he just categorizes her as \quo{evil}.
        
        Rian just sees Needle-tachi near the building, armed and menacing. 
        Then a \grimrat{} startles him and chases him. 
        He sees no more. 
        He does not see the raid at all, so he does not know there are two different factions fighting each other. 
        He just knows there are evil mages and monsters. 
        
        Maybe he sees one of the \hr{QJ in Malcur}{\quiljaaran in \Malcur}. 
        (Read about them.) 
      \item 
        In all the Rian chapters, whenever it is appropriate, have references to the \hr{Myths of vanquished monsters}{myths of Iquinian heroes vanquishing inhuman Elder Races and monsters}. 
        When he encounters something supernatural, he fears that the wicked Elder monsters will conquer the world. 
        He prays to be delivered from \Isphet's evil. 
      \item 
        \hr{Rian is religious}{Make Rian more religious}.
        Make sure he prays in every chapter and scene that he is in.
        He has lingering existential/religious dread from the day when he saw the dark sorcerer slay the shining god (even though he was Shrouded and does not remember it all). 
      \item 
        There is a \grimrat{} that crawls up a beam and runs across a crossbeam to leap down on its victim. 
      \item 
        Skip the entire battle and go straight to the aftermath with Needle contemplating. 
      \item 
        Needle is horrified after seeing the \grimrats{} tear people to shreds. 
        \ta{But it had to be done.}
        
        She is also squeamish about killing the plaguers. 
        But she must. 
        She has to remind herself that thugs such as these deserve no mercy. 
        She thinks back to the beginning of her family's tragedy, where her brother was killed by someone who might very well have been a plaguer. 
        A \scatha{} dressed in black, with a hood. 
        
        \tho{Yes. It was a plaguer. It must have been. I hate them.}
        She clings to this hate. 
        It makes her job easier. 
      \item 
        Afterwards, \Psyrex{} sits on his throne and muses. 
        \begin{prose}
          \ta{I have lost one of my mages. 
            I will probably have to import a new mage to \Malcur.
            
            Sad business. 
            Still, they did their job. 
            The plan is still on schedule.}
        \end{prose}
      \item 
        The scene where Rian where sees a \sphyle that is turning into wood should be expanded.
        I want more \hr{Humanoid horror}{humanoid-based horror}, remember. 
        The \sphyle comes alive and grabs Rian. 
        He screams.
        She talks to him, or tries to.
        It is hard for her to speak, but Rian understands more than he wants to. 
        
        \citeauthorbook[p.100-102]{RobertEHoward:TheScarletCitadel}{Robert E. Howard}{%
          The Scarlet Citadel%
        }{
          Within these bars lay a figure, which, as he approached, he saw was either a man, or the exact likeness of a man, twined and bound about with the tendrils of a thick vine which seemed to grow through the solid stone of the floor. It was covered with strangely pointed leaves and crimson blossoms\dash not the satiny red of natural petals, but a livid, unnatural crimson, like a perversity of flower-life. Its clinging, pliant branches wound about the man's naked body and limbs, seeming to caress his shrinking flesh with lustful avid kisses. One great blossom hovered exactly over his mouth. A low bestial moaning drooled from the loose lips; the head rolled as if in unbearable agony, and the eyes looked full at Conan. But there was no light of intelligence in them; they were blank, glassy, the eyes of an idiot.
          
          \prikker 
          
          \ta{%
            He pent me in here with this devil-flower whose seeds drifted down through the black cosmos from Yag the Accursed, and found fertile field only in the maggot-writhing corruption that seethes on the floors of hell.}
        }
        
        Maybe it is a \human, not a \scatha. 
        
        Afterwards Rian runs out.
        Then he sees \emph{everyone} as wood-people.
        Or he sees some of them slowly mutating into demonic monsters.
        
        Maybe he runs into the Beyond and sees everyone as as chained and faceless things.
        In fact, maybe take the Catrian chapter and move it here, with Rian instead of Catrian. 
        
        Rian sees loathsome, creeping, half-undead, chained people. 
        Compare to Eallal from \cite{RobertEHoward:TheShadowKingdom}. 
        
        \citeauthorbook[p.39]{RobertEHoward:TheShadowKingdom}{Robert E. Howard}{%
          The Shadow Kingdom%
        }{
          The glow merged into a shadowy form.
          A shape vaguely like a man it was, but misty and illusive, like a wisp of fog, that grew more tangible as it approached, but never fully material.
          A face looked at them, a pair of luminous great eyes, that seemed to hold all the tortures of a million centuries.
          There was no menace in that face, with its dim, worn features, but only a great pity\dash and that face\dash that face\dash
          
          \prikker 
          
          The phantom came straight on, giving them no heed; Kull shrank back as it passed them, feeling an icy breath like a breeze from the arctic snow.
          Straight on went the shape with slow, silent footsteps, as if the chains of all the ages were upon those vague feet; vanishing about a bend of the corridor.
        }
    \end{itemize}
\end{changes}










\subsection{General}
\begin{enumerate}
  \item 
    Update the pronunciation guide!
    Fix the pronunciation of Imetric and Resphan/Vaimon words. 
    Read the \hs{Languages} appendix first. 
    
  \item 
    Fix \hs{noble titles}. 
    Update the glossary with them. 

  \item 
    Have more descriptions. 
    Describe not only looks, but also sounds and smells. 
    Describe: 
    \begin{itemize}
      \item \ps{\Tiroco} room. 
      \item Needle's room. 
      \item The temple of \EreshKal. 
      \item Sethgal's command tent. 
      \item The Pelidorian army camp. 
      \item The Rungeran army camp. 
    \end{itemize}
  
  \item 
    Rewrite the \Archon-summoning scenes to show the \hr{Visualizing Archons}{personalized feel of each \Archon}. 
  
  \item 
    Have and mention examples of \hr{Rissitic economy}{Rissitic export goods}. 
  
  \item 
    Go over the Dramatis Personae and add more titles to names. 
    Such as \quo{\Symeon{} Clerk}. 
  
  \item 
    The translation of \quo{Iquin} $\to$ \quo{Light} is cheesy. 
    Change it to \quo{One Light}. 
    
    Similarly, change \itzach to \quo{Outer Darkness}.
  
  \item 
    Make a appendix about \matrices{} and \hs{astrology}, entitled \quo{Map of the Stars} or \quo{Pattern of the Cosmos} or \quo{Star-Maps of the Ancient Cosmographers} or somesuch. 
    Read some Bal-Sagoth for inspiration. 
    
    And draw an actual map of the stars and mystic constellations, \hr{Vertices in the sky}{with \vertices{} shown}. 
    
    Remember to read the sections about \hr{Matrix}{\matrices} and \hs{astrology} before doing this. 
    
    Remember that \nexi{} such as \Malcur and \Nithdornazsh{} should also be represented in the Star-Maps. 
    
    Compare to the Deck of Dragons in \cite{StevenEriksonIanCameronEsslemont:MalazanBookoftheFallen}. 
  
  \item 
    Read the \hs{Languages} appendix and fix all names and words so they don't contain illegal phonemes. 
  
  \item 
    There are way too many of those dreams/visions of doom in \Malcur.
    Get rid of some of them. 
  
  \item 
    Get rid of elephants, rhinoceroi, dogs, wolves and lions. 
    \Miith{} is \hr{Saurian-dominated}{\saurian-dominated}. 
  
    A \belwan{} is not a mammal but a small, hornless ceratopsian. 
    
  \item 
    Have references to the \quo{\hs{Ages of the World}} early on, and in the Glossary. 
  
  \item 
    Maybe the entire first \quo{Part} should be a Prologue \quo{Part}. 
  
  \item 
    Add a \hr{Vaimon Middle-East}{Middle-Eastern style to the Vaimons}. 
  
  \item 
    Make clear in the appendix that there is a translation convention in effect.
    For example, the planet is not really called \quo{\Miith} in all languages.
  
  \item 
    Remember that cities must be designed with domestic dinosaurs in mind.
    And big-ass dinosaur-drawn trains. 
    Factor this into the descriptions of Forclin and \Malcur.

  \item 
    Make the whole world more dystopian.

  \item 
    End chapters on a cliffhanger! 
    All over the place.

  \item 
    Get rid of the \quo{kraken} or merge them with the \noggyaleth and \xss.
  
  \item 
    Have plenty of \hr{Demihuman}{\demihuman} slaves in Pelidor and stuff. 
    Read about \hr{Demihuman}{\demihumans}.
  
  \item 
    Rethink the idea of the different \scatha ethnicities.
    Now we have the concept of \hr{Demiscatha}{\demiscathae}. 
    
  \item 
    Replace all occurrences of \quo{chaos sorcerer/sorcery/magic} with \quo{\rethyax}. 
  
  \item 
    Replace \quo{\bane} with \quo{\hr{Sitra Achra}{\SitraAchra}} in \resphan contexts. 
  
  \item 
    Remember to keep track of \hr{Languages in the Scatha Age}{the languages spoken in the \Scatha Age}. 
  
  \item 
    Remember that the chapter \quo{What Slithers Beneath} has to be far enough removed in time from the main story that \Teshrial has ample time to revive and heal.
    Look how long time it took \Urizeth to heal. 
    But then, \hr{Ketherain heal faster}{\ketherain heal faster than \thelyadeth}. 
  
  \item
    Replace all forests with \hr{Jungle}{jungles}. 
  
  \item 
    Maybe get rid of \GreatVelcad and merge it with \Tepharae. 
    We are in the \Scatha Age, remember.
    
  \item 
    Maybe get rid of \ClanTelcra and merge it with \ClanZether. 
    
  \item 
    Maybe rename \Miith to Shetiyah.
    Means "foundation stone" in Hebrew; "the rock over the abyss around which the whole world was built". 
    
  \item 
    Maybe \LocarPsyrex is a \quiljaar.

  \item 
    Get rid of \ClanTelcra. 
    Merge it with \ClanZether. 
  
  \item 
    Make sure the \scathae appear as powerful as \humans and not as inferior pussies. 
    
    Koit: \ta{the scathae seemed big but a bit dumb, however usually kind-hearted.}
    
    The chapter \quo{Screaming in the Dark} has a bunch of \scathae that come off as dumb and helpless. 
    I should have a chapter before this that makes them look cooler and more well-developed and intelligent.
    In fact, maybe I should turn the villages in \quo{Screaming in the Dark} into \humans.
    Seems they cause too much trouble in the current form. 
    That way I would lose the racist aspect, but I can live with that. 
    I can fit in plenty of racism later.
    
    The villagers should be hairy \demihumans.
    Some of them comment that Carzain is a purer \human than any of them. 
    But then, so were some of the bandits, and that did not make the bandits any less evil. 
    
    \ta{an even better racist touch actually. 'why are our fellow humans attacking us, if they could take their toll on the nearby scatha village'}
  
  \item 
    Drop hints all over the place about the back histories of \dragons, \resphain and \banes.
    Especially in the Curwen and \Cobrel chapters. 
    See the section about \hr{Hints}{hints}. 
  
  \item 
    I should flesh out the ideas of \sephirah \hs{pillars} and \qliphah \hs{nebulae} and refer to them in the story where appropriate. 
  
  \item 
    Have more cliffhangers at the end of chapters. 
    Like in \cite{AdrianTchaikovsky:ShadowsoftheApt}. 
   
  \item
    The \wylde is the grim and terrible dark world where chaos reigns. 
    Make this clear in the story!

  \item
    The \resphain should be noble and heroic but dark.
  
  \item
    Atmosphere: 
    Emphasize \hr{Hints}{through hints} that the this time Rungeran invasion is more than mere land-grabbing. 
    Morgan Runger is bringing with him a fell supernatural menace, not dreaming of its true magnitude. 
    Imetrian mages can see this visions; hence they send Ilcas with an army to help fight him. 
    They fear that if Pelidor falls it will be a stepping-stone for a greater and more terrible evil which will cast its shadow over the continent and seek to devour all.
    That must not happen.
\end{enumerate}





\subsubsection{Characters}
\begin{changes}
  \begin{comment}\paragraph{Archibald Curwen}\end{comment}
  \changesitem{Archibald Curwen} 
    Rewrite Archibald Curwen. 
    He is boring because he is not emotionally involved in anything that goes on. 
    He is unlikable because he is grumpy and brutish. 
    Make him more dastardly, more sardonic. 
    Make him bitter in a sarcastic way rather than a grumpy way. 
    He should smile more. 
    He should say \quo{\Mister \Shireyo}. 
    
    He shouldn't be so fat. 
    Change \quo{lose that belly of yours} to \quo{lose some weight}. 
  
  \begin{comment}
  \paragraph{Carzain \Shachar}
  \end{comment}
  \changesitem{Carzain \Shachar} 
    Give Carzain more humour.
  
  \begin{comment}
    \paragraph{\LocarPsyrex}
  \end{comment}
  \changesitem{\LocarPsyrex} 
    Make him darker.
    \hr{Psyrex darker}{Read about him}.
    
    \Psyrex needs to set up a \hr{Draconic Matrix}{\draconic \matrix} with a Flame, a Skull and a Hollow.
    \Psyrex himself is the Archway.
    I need to introduce this early in the book.
    Then, later, have a very short scene when each \vertex is in place: 
    \hypota{The Skull is in place.} 
    Like that.
  
  \begin{comment}
  \paragraph{\MoroCobrel}
  \end{comment}
  \changesitem{\MoroCobrel} 
    Have more scenes with \MoroCobrel.
    \begin{itemize}
      \item 
        Including \hr{Moro feels Ishnaruchaefir and Teshrial}{one in the dead garden in the beginning}. 
      \item 
        And the scene with \hr{Moro and Nasshikerr}{Moro and \Nasshikerr}. 
    \end{itemize}
  
    Drop hints that she sacrifices humanoids to \Nasshikerr.
    Usually she just sacrifices animals. 
    Only once in a while does he demand a \humanoid.
    
    Maybe scratch the above. 
    Make Moro nicer, more heroic.
    Remember, she is the biggest \scatha character in the book.
    I do not want to give the reader the impression that \scathae are evil. 

  \begin{comment}\paragraph{Rian}\end{comment}
  \changesitem{Rian} 
    Rewrite Rian.
    Make him more detective-like.
    Right now he depends too much on blind luck. 
    
    And compress the whole Rian story thread. 
    It is unrealistic for Neina to be held captive and alive for so long. 
    
    Remember to fix up the Rian chronology. 
    The \quo{What Slithers Beneath} chapter is now closer to the rest of the story. 
    Maybe there is not enough time to fit in Rian's rehabilitation between them. 
    
    The dark parts of \Malcur, where Rian dares to go, feel more like nature than city.
    They are decayed and corroded so much that they have almost become \wylde. 
    The \eidola in these slums and mafia quarters have been allowed to decay, partially because all the crime makes the Vaimons and monks afraid to come in there to maintain the \eidola.
    The Shroud-weaving alienist sorcery of the Sentinels and Cabalists also has to do with it. 
    The sorcery unravels the Shroud and tears holes into the Beyond from which occult power can seep. 
    This also causes the Mask of Civilization to unravel and makes the civilized settlement areas slowly decay into \wylde.
    
    These semi-\wylde places awakens Rian's \hr{Human racial memory}{racial memory} and makes him remember his people's dark and bloody past.
    
    \citeauthorbook[p.204]{RobertEHoward:TheLittlePeople}{Robert E. Howard}{%
      The Little People%
    }{
      The horrid things that pursued her were closing in upon her.
      They would reach her before I.
      God knows the thing was horrible enough but back in the recesses of my mind, grimmer horrors were whispering; dream memories wherein stunted creatures pursued white limbed women across such fens as these.
      Lurking memories of the ages when dawns were young and men struggled with forces which were not of men. 
    }

    Maybe turn Rian into a girl, named Ria.
    
    In all Ria's chapters where she sees into the Beyond, remember to read the section about \quo{Breaking through the Shroud} first.

    He experienced ancestral dread.
    Read about \hs{ancestral memory}.
  
  \begin{comment}\paragraph{Telcastora Ilcas}\end{comment}
  \changesitem{Telcastora Ilcas} 
    Have more focus on Ilcas Northstar and the Imetrians. 
    Show Ilcas and how he is sent to Pelidor because the Imetrians fear Rissitic involvement. 
    (And have a better explanation of why the Imetrians and Rissitics are such rivals.)
  
  \begin{comment}
    \paragraph{Sethgal}
  \end{comment}
  \changesitem{Sethgal} 
    Mention more than once that Sethgal really hopes some Imetric reinforcements get here.
    As soon as the war started, Sethgal wanted to begin negotiations with the Imetrians. 
    He is quite pragmatic when it comes to religion.
    But other (more religious) forces are working against him and the Imetric alliance.
    This makes Sethgal angry and sad.
    They need those Imetrians. 
  
  \begin{comment}
    \paragraph{\Teshrial}
  \end{comment}
  \changesitem{\Teshrial}
    \Teshrial needs to have more weaknesses.
    See the section about \hr{Teshrial's failure}{\Teshrial's failure}. 
    
    \Teshrial should be more sympathetic, less of a snob.
    The \resphain should be noble and heroic but dark.
\end{changes}





\subsubsection{Geography}
\begin{enumerate}
 
  \item  
    Draw a map of Pelidor/Runger/\Scyrum. 
    And one of \Malcur. 
    
  \item  
    Mark the regions of the map as \quo{\scatha-inhabited}, \quo{\human-inhabited}, \quo{\wylde} etc.
    
  \item  
    Maybe countries should not have explicit borders. 
    After all, borders are difficult to police. 
    And most of the country is \wylde anyway. 
    See also \href{http://limyaael.insanejournal.com/373476.html}{Limyaael's \quo{International relationships rant}}. 
    
  \item  
    Have plenty of \quo{wasteland} and dark, unknown regions. 
  
  \item  
    Re-design the geography of the northern lands. 
    And the south, for that matter.
    Make the north into more of a united land mass.
    And splinter the Imetrium into an archipelago.
  
  \item
    Change geography:
    \quo{\Velcad} refers only to Galessan.
    
    There were several \Velcadian languages, all of them descended from \Tepharin (which, in turn, borrowed heavily from both \Ortaican and Vaimon).
    Pelidorian and Rungeran were different but related languages.
  
  \item 
    Get rid of \GreatVelcad.
    Replace it entirely with \Tepharae.
    Get rid of the Tigers.
    Maybe replace them with a \Tepharin order named after a dinosaur.
    Maybe the \Tepharins came from the isle of \Velcad or something like that.
    But rename the isle.

  \item 
    \hr{Unexplored places}{Have many dark, unexplored \quo{here-there-be-\dragons} places in the \wylde}.

  \item 
    Remember that some places are \hr{Baccon}{\bacconates} ruled by mages. 
\end{enumerate}





\subsubsection{Glossary}
\begin{enumerate}
  \item 
    Add the names of all \xss{} used to the glossary. 
    \Ishnaruchaefir{} invokes some in \quo{What Slithers Beneath}. 
    
  \item 
    Fix the glossary entry for \quo{\hs{sorcery}}. 
  
  \item 
    Glossary: 
    \Sethicus was the mythical founder of the \draconian race (sometimes called the \quo{race of \Sethicus}).
    
    \Thanatzil was the mythical founder of the \resphan race (sometimes called the \quo{race of \Thanatzil}).
    
  \item 
    Rewrite the glossary and remove all references to the \quo{present time}. 
    Replace by explicit years:
    \quo{In year $n$ VC, \Icor was the \rayuth of Pelidor\prikker}
  
  \item 
    Write the glossary from a \resphan point of view. 
    For example, \quo{Incursion} is an item.
    \quo{\Secondbanewar} redirects to \quo{Incursion}.
    \Sethicus and the Durance are myths.
    \Thanatzil is a legend, but not a myth.
  
  \item 
    Make a remark about Carzain and Vizicar.
    Their internal dialogue is not literally what happens.
    It is a figurative representation of how the two minds access each other's thoughts and memories. 
  
  \item 
    Fix the name: 
    \Rystessakhin{} $\to$ \AeocrithRystessakhin. 
\end{enumerate}





\subsubsection{Quotes and mythology}
\begin{enumerate}
  \item 
    At the beginning of every \quo{Part} (and possibly in the middle of \quo{Parts} too), have passages from WID, \Iquinian myths, \Ortaican myth and perhaps other stuff.
    If I can make up enough of it, have some in every chapter.
    
    Make sure they wildly contradict each other.
    Even the ones belonging to the same religion.
    Have each side openly demonize the other. 
    
  \item 
    Clarify that myths are not necessarily true. 
    Perhaps make it clear that the text we see is just one of several possible translations, and not considered canon by all (perhaps even heretical by some).
    Hint to the viewer that myth and history should not be taken as canon. 

  \item 
    To make clear to the reader that Iquinian myths are not necessarily true: 
    Have ideas and imagery that contradicts Earth mythology. 
    For example, let snakes or horned/hoofed humanoids be good instead of evil. 
    Actually, maybe certain \resphain should have horns. 
    
  \item 
    At the beginning, have some quotes that tell about the \hr{Ortaican-Vaimon relationship}{theological conflict between \Ortaicans and Iquinians}. 
    (Read the section immediately!)
    
  \item 
    Have \Ortaican \hs{pseudepigraphic Sseju quotes} that condemn Iquinianism. 
  
  \item 
    Quote the poem \quo{Gods in Darkness} from \cite{KarlEdwardWagner:DarknessWeaves}:
    
    \citeauthorbook[p.2 of 138]{KarlEdwardWagner:DarknessWeaves}{%
      Karl Edward Wagner%
    }{%
      Darkness Weaves%
    }{
      In their castle beyond night\\
      gather the Gods in Darkness,\\
      with darkness to pattern man's fate.
    }
  
  \item 
    Just before or after the chapter \quo{What Slithers Beneath}, I should have some quotes from Iquinian scripture where \Isphet is reviled for his great evil.
    But present him in a way as to make him look like a total stud; an awesome dark lord. 

\end{enumerate}






\subsubsection{\TeX nicalities}
\begin{enumerate}
  \item
    Sort the index properly! 
    
    Find out how to do \quo{see also} in the index. 
  \item   
    Fix the counter-related problem with years! 
    And make a command that checks whether a number is defined.
  \item 
    Make a macro that capitalizes a word. 
    See \authorbook{Donald Knuth}{The \TeX book} example 20.19 p.217. 
  \item 
    Reduce line spacing in Dramatis Personae and Pronunciation. 
  \item 
    Typeset the index in a smaller font.
  \item 
    Typeset the table of contents in a smaller font and in two columns. 
  \item 
    Add \texttt{$\backslash$gref} commands for \emph{all} Glossary items (incl. people), at relevant places in the text. 
    Replace all \texttt{$\backslash$hr} and \texttt{$\backslash$hs} with these. 
  \item 
    Make a global switch that determines whether a \texttt{$\backslash$gref} should act as a \texttt{$\backslash$hr} or a \texttt{$\backslash$maybehr}. 
\end{enumerate}





\subsubsection{Limyaael-inspired ideas}
\begin{enumerate}
  \item
    Maybe \Esmerel{} doesn't recognize the signs of a Scion immediately. 
    Maybe she has to go research a lot of writings and signs first (in the big-ass library at the \TopazChateau). 
    Then, later, when she is sure that it was a Scion she saw, she tells the Conclave and receives their blessing to go off and search for it. 
  
  \item 
    Have references to poems, religious dogma, history records and whatnot. 
    See \emph{Limyaael}'s \href{http://limyaael.livejournal.com/357919.html}{Non-linear narrative rant}.
\end{enumerate}





\subsubsection{Koit's ideas}
\begin{enumerate}
  \item 
    Make Carzain stronger\dash but keep up a threat that he might crumble under the power. 
    \begin{itemize}
      \item 
        \quo{%
          Why I didn't like Carzain? No one is that overconfident in his nonexistant abilities\prikker At least, that's how I saw him.}
    \end{itemize}
  \item Trim the scene with Tiroco and her egg. 
  \item Give \hs{Sethgal} more of a personality. 
  \item Write foreword and reading guide
  \item Fix \quo{Involate} $\to$ \quo{Inviolate}. 
  \item 
    Tiroco: 
    \quo{%
      She seems to be emotionally disabled from her current paragraphs. 
      The behaviour seems inhuman (and she's not human). 
      A typical human would likely do everything he or she could to spend more time with someone who has left (Icor), but she tries to flee from him when she sees him. 
      Perhaps it's right, but it would deserve a bit of mention why it's soo. 
      religious taboo set by Light? 
      Ghosts are perhaps of dark? 
      then I could understand her}
  \item 
    Make sure it is \quo{Charcoal} and not \quo{Curwen} who reads the diaries. 
  \item Give some indication that Curwen recognizes Takestsha. 
  \item Make Curwen's secret identity more secret. 
  \item Add more references to history and historic figures. 
  \item Have some Pelidor-Runger skirmishes before the main battle. 
  \item 
    \quo{have you devised any specific military tactics either side is well known of or feared for?}
  \item 
    It looks like \Ishnaruchaefir{} finds \Triestessakhin{} in the garden. It shouldn't. Fix it. 
  \item Maybe Iasper Bartholin's title shouldn't be \quo{Mister}. 
  \item Draw a map of \Malcur. 
  \item 
    page 11, paragraph 5: 
    \ta{The captain may be,} said Carzain, 
    \ta{but has less savoury characters among his men.}
  \item 
    page 11, last paragraph: \quo{even alien and uneartly minerals} I would think that 'unearthly' and 'alien' have quite the same connotation, especially when there's no spacetravel (or if you don't mean that the minerals are from some other dimension)
  \item 
    page 11, last: 
    \quo{It was a journey of many days from \Redglen{} to Martinum}.  Shouldn't it be the opposite\dash they are coming from Martinum
  \item 
    page 12, third: 
    \quo{They were five days' travel out of Martinum and still ten days from Redglen,} 
    
    really, a travel of 15 days on foot and with a cart is rather quick and means a shor distance.
  \item 
    p. 13, fourth: \quo{warring kingdoms and baronies}, You mention a 'baronry' while you'll be dealing with a 'duke' for the most of the come time forward\prikker 
  \item 
    p. 16, third: \quo{At the last minute} seriously, the guy's been running at him for *ages*
  \item 
    p. 16: the mace is above Carzain's head, if the mace falls, the head is below it.
  \item 
    p. 185 - how many moons? two? then it's 'larger' instead of 'largest' at paragraph 2 ending
  \item 
    Carzain: If you mention that he just came back from a trip to a faraway land, it's illogical that he *never* again brigns it up. 
  \item 
    The beginning of Northstar was really weird, I don't know anyone *That* arrogant (Show those Qliphoth who's boss). 
  \item 
    Add a disclaimer to the front of all the note documents:
    \begin{quotation}
      These notes are rough drafts and not fully updated. 
      They are full of anachronisms, contradictions and residues of discarded ideas. 
      They should most definitely not be considered canon. 
    \end{quotation}
  \item 
    Regarding the portrayal of \resphain in \quo{Wanderer in Darkness} and \quo{What Slithers Beneath}: 
    \ta{I like the joke you've got in it (respectable occultist. :D). well, they seem honour and respect based in many ways. that being said, we don't really learn much of them this quickly. the resurrection part is a bit puzzling\prikker the part about allying with other Resphan would take ages in interesting, and I suppose it might carry some weight, but without going into the problem sooner or later, you've just brought up an empty detail.
    we don't really get to see the Resphan outside their conference or whatever. I already said that knowing more of their battle would be nice. }
    
    So, explain the \quo{ally with other \resphain} part better.
    There are all sorts of political reasons why it will not work.
  \item 
    Regarding the portrayal of travel through the Beyond int \quo{What Slithers Beneath}:
    It seems like the \resphain come from the sky. 
    It is very unclear what the Beyond is.
    I should explain this better: 
    They come out of nowhere, out of a parallel dimension. 
  \item 
    Regarding Communion: 
    \ta{%
      I suppose you'll be describing some other communion later on, where you describe exactly what the human is hoping for? to add the info here would stall it too much, but it would be good to know, I believe}
    
    It is unclear what Evith expects will happen.
    The short answer is, she wants to go to heaven.
    And she does\prikker sort of.
    I need to clarify that.
  \item 
    \ta{Well, as I said, the duel sequence [\quo{What Slithers Beneath}] could use a bit of touch\prikker I like the Prologue of Dragons. A bit more destruction might be useful though\prikker Then I liked how Carzain slew the bandits [\quo{Screaming in the Dark}]\prikker What was perhaps inexplicable/bad was how Carzain actually travelled, or how long he did travel to the burned village, that he had the luxury to say that he had lots of time to spend there}
\end{enumerate}










\subsection{Split into two books}
If the book ends up being too long, I could split it in two halves.
Not by cutting it in the middle, but by splitting it geographically.

The first half takes place chiefly in \Forclin.
Main characters are Carzain, Curwen, \Achsah and \Takestsha.
It ends immediately after the great boss battle with \Achsah versus \Nzessuacrith.
\Forclin is in Cabal hands, but something ominous looms in \Malcur. 

Maybe have the opening of the conflict between \Teshrial and \Ishnaruchaefir already in the beginning of the first book.
Have \Achsah refer to it as a mystical, faraway conflict, but still very important. 
And then end the first book on a cliffhanger: 
\Achsah-tachi have secured \Forclin, but then they get a distress call from \Malcur:
\ta{\Malcur is under attack! By both \Ishnaruchaefir and the Sentinels!}

The other half takes place in \Malcur. 
Main characters are \Teshrial, \MoroCobrel, Rian and maybe \Tiroco. 










\subsection{Removing story threads}





\subsubsection{\Tiroco story thread}
\target{Remove Tiroco story thread}
I may decide I have too many story threads and need to cut some of them.
If so, the \Tiroco/\Icor thread is a prime candidate for cutting. 

I will let the story thread live for now. 
I can always cut it later.

\begin{enumerate}
  \item 
    \Tiroco is supposed to die at the end of her story thread. 
    If \Tiroco is removed, then I should consider killing someone else instead. 
    I could \hr{Maybe Rian dies}{kill Rian and Neina}. 
    
  \item 
    If the \Tiroco thread dies, then the \quo{Daggers and \Daemons} chapter should be reduced to a footnote.
    All that the reader needs to know is this:
    \begin{itemize}
      \item The \rayuth has been assassinated.
      \item \Ishrah mage \Ambrose \Onatol was implicated in the assassination.
      \item \Onatol resisted arrest and was killed by Archibald Curwen.
      \item Charcoal has obtained some papers from \Onatol. 
    \end{itemize}
    
    Ostensibly, the murder of \Icor was to weaken Pelidor and make it easier to invade and conquer. 
    In reality, the assassination was a ruse.
    It was just there in order to make the Rungeran invasion look more secular in motivation, to hide the Sentinel involvement. 
\end{enumerate}
























 

























\chapter{\TheKenosis}
\section{Overview}
There are three main plot threads in this book. 
Each of them takes place in the dwelling place of one of the three fragments of the \Haskelek{} and deals with the people fighting the \Haskelek. 

\begin{enumerate}
  \item 
    Carzain goes to \Redce{}, trains there, and ultimately goes to fight the \Haskelek{} in \Redce.
  \item
    Lica and Sir James fight the \Haskelek{} in their country.
  \item
    Sentinels and Cabalists battle for control of the Ghost Tower. 
    Ultimately the Sentinels, led by \Nzessuacrith{} and aided by the \Haskelek{} fragment, succeed in driving away the Cabalists, led by \Achsah. 
\end{enumerate}









\subsection{Carzain in \Redce}
Vizicar, at this point, has regained most of his memories from his life as Vizicar. 
He knows he is a Scion, an incarnation of a \malach. 
He knows he is a superhuman badass. 
He is not \human, but something bigger and better. 
Now he wants to break his feeble \human{} shell and reawaken his true potential. 

Early on, he is joined by Shereid. 
She learns that he is a Scion. 
She joins him and supports him, like Mele supporting Rio in \cite{Tokusatsu:Gekiranger}. 
Together they do some research. 
They want to find out how to achieve his \Apotheosis{} and become the Twilight Angel he was meant to be. 
To do this they must work together with the Redcor. 

In the meantime, they discover that something abnormal is afoot in \Redce.
Something evil. 
Or is it? 

The Redcor want Carzain/Vizicar's help to fight this evil. 
(They mostly ignore and discount Shereid.) 
He is reluctant to ally himself with them too closely, but he would like to work with them to learn more. 
So he lets them think he is obeying their orders, all the while manipulating them. 
When he wants to, Carzain can look like a stupid, brash, impulsive boy who is easy to manipulate into doing stuff. 
But Vizicar, who lies and lurks in the back of Carzain's head, is much shrewder and often outsmarts the Redcor. 
(Be careful not to make the Redcor too stupid, though.)

The Redcor know they have a traitor in the higher rungs of their society, someone who leaks their secrets and breaks into their inner sancta. 
Carzain helps them to find this traitor.
But in fact it is Carzain who is the traitor. 

Carzain commands wraiths and other horrid creatures of the Beyond to do his bidding.
He uses them to hide his agenda and misdirect his enemies.
Sometimes he makes his monsters attack him and his companions so that Carzain can be seen saving them from the monster.





\subsubsection{Carzain is a Sephiroth-type}
Remember that \hr{Carzain is Sephiroth}{Carzain is meant to resemble Sephiroth} from \cite{VideoGame:FinalFantasyVII}. 





\subsubsection{Carzain regains \Tydesmos's memories}
\target{Carzain remembers Tydesmos}
During this book, Carzain gradually regains \Tydesmos's memories, \hr{Carzain cannot remember Tydesmos}{which he had otherwise been unable to remember}. 
\Tydesmos was \hr{Tydesmos power}{a mighty and wise dark mage who possessed much Mythos knowledge}.
As Carzain gains these memories, it drives him a bit more mad. 
\quo{Mad} in the sense that he gains a non-\human perspective and begins thinking more like an immortal. 

Like Sephiroth from \cite{VideoGame:FinalFantasyVII}. 





\subsubsection{Carzain as an unwilling ally}
Carzain is unwilling, but somehow circumstances force him to work with the Redcor. 
He struggles to maintain the frame of an independent ally, while they try to turn him into a servant who should rightfully obey. 

I need to make up a good reason for this. 
Perhaps I can use a story similar to that found in the anime \cite{Anime:Dragonaut}, where Kamishina Jin is pulled into the Dragonaut organization without understanding what is going on. 

Also compare to the movie \cite{Movie:MonstersVersusAliens}, where Susan and the other monsters are locked up and treated as shit. 





\subsubsection{Redcor need Carzain}
\target{Redcor need Carzain}
The Redcor needed Carzain. 
They knew that strange, occult, dangerous things were happening in \Redce, and a few suspected that \hr{Belzir awakening}{\Belzir was gaining power}. 
\Esmerel deduced that Carzain was a Scion and decided that having him around (as an ally as well as a study object) would be a great asset, since \Belzir is also a Scion. 





\subsubsection{Carzain uses his Redcor blood to coerce them}
Carzain uses his Redcor blood to coerce the Redcor into accepting him and thus gain access to stuff in \Redce. 
He exploits their snobbery and nepotism.





\subsubsection{Carzain is forced to use violence}
At first, Carzain intends to work inside the system.
He plans to go along with the Redcor while secretly manipulating them to get what he wants. 

Some of the Redcor are nice.
Others are \quo{good} in a somewhat annoying way. 
Yet others are complete assholes and use the rules to bully and oppress people, even Carzain if they can. 
Compare them to Principal Snyder and the Watcher Council in \cite{TV:Buffy}. 

Carzain puts up with it for a while, with all the pride and disdain he can muster.
Eventually, though, he feels forced to resort to violence.
The \hr{Racel dies}{he kills \Racel} in a very Sephiroth-esque scene (\cite{VideoGame:FinalFantasyVII}). 

All the way through the book, Carzain has had a sinister master plan.
This has to do with his \hr{Carzain's Sephiroth epiphany}{Sephiroth epiphany} in \TwilightAngelRememberEmph. 
He has been secretly working towards this dark goal for the whole book, in front of the reader's nose yet behind his back. 
The big surprise comes at the end. 
Carzain shows his true \colours, kills \Racel and a bucketload of other Redcor. 
He unveils what he has been secretly working on the whole time, and does a \trope{FaceHeelTurn}{Face Heel Turn} and goes to join \Belzir. 

Also make him resemble Anakin Skywalker/Darth Vader in \cite{Movie:StarWars:III}. 

Carzain has been contacted by \Belzir. 
She tries to seduce him.
Through this, he learns a little of her plans. 
He tells this to the Redcor. 
Officially, he is horrified by \Belzir's evil and wants to work with the Redcor to stop her. 
Unofficially, he is thinking of defecting to \Belzir's side. 
So he \quo{spies} on \Belzir for the Redcor. 
He tries to figure out what \Belzir is up to.
Whenever he finds out something and explains it to the Redcor, he deliberately misinterprets his data so as to mislead them and cast a smokescreen over himself to cover for his own shady plans and dealings. 





\subsubsection{Redcor suspect Carzain}
Soon after Carzain arrives in \Redce, the dangerous occult occurrences seem to escalate. 
Things get worse than before. 
The Redcor suspect that there is a mole who is betraying the Redcor from the inside and killing them and disrupting their spells and leaking information and resources to the evil rebels (whoever they are\dash the Redcor have not identified the enemy).  





\subsubsection{Carzain crippled}
\hr{Ramiel crippled}{Carzain/Ramiel} becomes more and more crippled as the story progresses.









\subsection{Redcor and the underworld}
Redcor tradition dictated that the underground should be kept secret and sealed, for it was the \vclan's sacred task to keep this evil imprisoned and hidden, lest its foul influence spread evil and madness and corruption all over the world. 

But the threat grows. 
The Redcor are haunted by evil dreams and madness. 

Some believe the right solution is to block up the underground with even more stones, more spells, more prayers and more denial; essentially rely on tradition and faith. 
Others believe that his path is not viable. 
The corruption is too insidious, and it will keep seeping out, especially because the Redcor do not understand it and hence cannot defend against it properly. 
Instead they must dare to explore and research and send expeditions down in order to find out what is going on, what the nature of this evil is and how they can fight it. 

Some Redcor, especially \hr{Kimon}{\PatriccoKimon}, believe that they should study the threat some more because it could teach them important things, deep insights about the nature of the world, \humanity, good and evil, \iquin and \itzach. 
Most Redcor believe that \Kimon is a dangerous heretic. 
The conservatives argue that it is imperative that they \emph{not} try to \quo{learn new insight}. 
They argue that \ClanRedcor already knows all they ought to know about the nature of man and the One Light and Outer Darkness. 
\Kimon's plan is madness and will only bring corruption. 
It will turn out that they are sort of right: 
\Kimon, in his eagerness to challenge tradition and learn new insight, he ends up a \trope{XanatosSucker}{Xanatos Sucker} and helps the evil forces (including Carzain) destroy \Redce. 

Many in \Redce can feel the corrupting evil in their minds. 
They pray for forgiveness and prepare for a religions doomsday.
Later \Belzir tells Carzain that these people are right: 
The Eschaton really is near. 
The walls of the world are about to crumble, and horror and madness will pour into the world from Beyond. 

Some commit suicide from the strain and horror. 
Even high-ranking Redcor. 
(These might be \trope{AssholeVictim}{Asshole Victims} or \trope{KarmicDeath}{Karmic Death}, but they might also be tragic victims.) 
It is very shocking to the Redcor that even high-ranking wise Vaimons (who should be strong-willed and protected by the One Light) can succumb to corruption and despair and take their own lives. 
Other Redcor and civilians begin to pray extra hard. 

\Humans, when they encounter the \banes or feel their presence, recognize something deep within themselves that resonates with the alien things. 
This is the ultimate horror.
Even Carzain feels this. 
It is part of his \hr{Carzain is Sephiroth}{Sephiroth-style awakening}. 

Occasionally there can be heard or felt rumbling noises or tremours from below, like vast, slimy footsteps. 
Compare to \cite{AugustDerleth:QuestforCthulhu}. 

\citeauthorbook[p.199-202]{TimCurran:Hive}{Tim Curran}{Hive}{
  The idea of getting bit wasn't what bothered him, it was the idea of the venom itself.
  And the sort of venom he might get stuck with in those blasphemous ruins was the sort that could erase who and what he was and birth something invidious and primal implanted in his genes a hundred-thousand millennia before. 
  
  \tho{You don't know that, you really don't.}
  
  Yet, he did.
  Maybe whatever it was had hid itself in the primal depths of the \human psyche, but it was there, all right.
  Waiting. 
  Biding its time.
  A ghost, a memory, a reventant hiding in the dank and dripping crypt of the \human condition like a pestilence waiting to overtake and infect.
  A cursed tomb waiting to be violated, waiting to loose some eldritch horror upon the world.
  An in-bred plague that festered in the wormy charnel depths of the subconscious, waiting to be woken, activated by the discordant piping of alien minds. 
  
  Dear Christ, there could be not nothing as horrible as this.
  
  Nothing. 
  
  He did not and could not know the ultimate aim of awakening the sleeping \dragon the Old Ones had implanted in the minds of men\ldots  but it would be colossal, it would be immense, it would be the end of history as they knew it and the beginning of something else entirely.
  The continuation of that primordial seeding, the vast outer extremity of that tree, the ultimate objective. 
}

\citeauthorbook[p.227]{TimCurran:Hive}{Tim Curran}{Hive}{
  \thought{%
    Christ, look at that old ice and what it holds.
    Like every dark and nameless secret of antiquity is locked up in that frozen sarcophagus.
    All of mankind's primal fears, cabalistic myths, and evil sorceries given flesh.
    The archetype that inspired every nightmare and twisted racial memory, every witch-tale and every legend of winged demons.
    All the awful, unthinkable things the race had bred and purged from the black cauldron of collectiv ememory, all the obscene things it could not acknowledge nor dare admit to\ldots it was here.
    This horror.
    The engineer of the race and of all races.
    And it had been waiting down here in the eon-old ice.
    Waiting and waiting, dead but dreaming, consciously forgotten but grimly remembered in the subconscious and dark lore of \human{}kind.
    But all along, they were dreaming of us just as we dreamed of them\ldots because they were us and we were them and now, dear God, millions upon millions of years later, they were waking up, they were rising to claim their children and their children's intellect. 
  }
}





\subsubsection{Gateway to \Erebos}
The ruins underneath the \TopazChateau gradually transition to the underworld of \Erebos. 
When the heroes venture down there they find bits and pieces of \Erebos. 

\citeauthorbook[p.223]{TimCurran:Hive}{Tim Curran}{Hive}{
  \ldots much of ti was nothing but huge boulders, some of them as big as two-story houses, lots of loose rocks and stacked wedges aof sandstone.
  But not all of it was of natural origin, for there were other shapes down there, ovals and pillars, assorted masonry that had been cut into those shapes.
  
  And there was no doubting where it had come from.
  
  For to either side of the gully, they could see the remains of the ancient city climbing up sharp slopes into the murk above.
  It was enormous, what they could see of it, for it climbed much higher than their lights could reach.
  A sleeping fossil, a mammoth city from nightmare antiquity.
  
  \ldots 
  
  Like everything about the Old Ones, this city\ldots it lived in the race memories of all men.
  And there was nothing remotely good associated with it.
  Just horror and pain and madness. 
}





\subsubsection{Corpse}
Exploring the cellars they uncover the corpse of one of their own.
It may be Vitor Bercerac, Carzain's enemy. 

\citeauthorbook[p.228]{TimCurran:Hive}{Tim Curran}{Hive}{
  He was curled up in sort of a fetal position, knees to chin, his face white as new snow and contorted into a grimace of absolute horror.
  Blood had trickled from the rictus of his mouth.
  His eyes were spilled down his cheeks in gelatinous trails like squashed jellyfish. 
  
  \ldots 
  And that death had been a dark matter, mindless and perverse and ghastly. 
  No man should have had to go like Gates did\ldots alone and mad in that suffocating darkness, dying a crazy and hopeless death like a rat stuck in a drainpipe.
  Screaming as his eyes boiled to soup and splashed down his face.
  As his brain went to sauce and his soul was burnt to ash.
  
  Gates had paid the final price for his curiosity. 
}









\subsection{Sentinels}
Meanwhile, \Narkiza-tachi are advancing north. 
They have conquered some stuff. 
They want to control more land. 
They need some strategic locations in southeastern \Velcad{} in order to resurrect \Belzir. 

See, the Sentinels have been communing with \Belzir. 
They want her back. 
She has grown to hate the Cabalists.
If she returns, she will want to take revenge on them. 
But they also know that she is a \Mystraacht{} patriot. 
So maybe, if she returns, she will be able to go to \Mystraacht{} and conquer it. 
This will leave a third of the Cabal's strength under the command of a maverick with a giant grudge against the rest of the Cabal. 
The Sentinels want this. 
So they are supporting \ps{\Belzir} \quo{Royalist Faction}, who work to see her restored. 

But, but, but. 
The Sentinels must not be seen as actively supporting \Belzir.
That would be bad. 
Instead, the Sentinels act as if they know nothing of the Royalists and are just pursuing their own gambits.
Then, by sheer chance, they \quo{unwittingly} end up accidentally helping the Royalists achieve their goal. 
The Sentinels obstruct the Cabalists and Redcor who were trying to obstruct the Royalists, and so the Sentinels \quo{unwittingly} end up accidentally helping the Royalists achieve their goal. 

(Oh, yes. 
\ps{\Secherdamon} \trope{XanatosGambit}{Xanatos Gambits} can be complicated.) 

The Sentinels now have \Nithdornazsh{} set up. 
So far, so good. 
This will allow \ps{\Vizsherioch} to spread his \vertex{} influence and become the \hs{Dagger}. 

Meanwhile, the Royalists are abroad. 
They have to solve some Aenigmata in order to free their queen. 
The Sentinels covertly help them do it. 

The Rissitic invasion also drains Cabal and Redcor resources away from \Redce, where a vital piece of \Belzir{} lies. 
This means that when the \Redcean{} Royalist infiltrators strike, there are few Vaimons to oppose them, and mostly weak ones. 
(Obviously this casts no suspicion on the Sentinels. 
 They have plenty of good reasons for wanting more land in \Velcad.
 No one would suspect them of secretly supporting an obscure cult serving a rogue \resphan.) 





\subsection{Royalists}
The Royalists do not have a fixed plan. 
In order to bring \Belzir{} back they need the blood (or, more generally, lifeforce) of a \resphan. 
This is not so easy to come by. 
\Shiaraid{} \hr{Shiaraid unpopular}{does not have many immortal allies}. 
So \quo{lots} of Royalists are out in the world, questing, seeking for a source of high-quality immortal blood or lifeforce. 

One group has the mummy of a powerful \quiljaaran{} king, unearthed from an ancient tomb. 
They think this mummy might contain some magic that will help them, so they are bringing it to their headquarters. 

Then Ramiel appears.
In Pelidor he \hr{Ramiel scares Nzessuacrith}{semi-unconsciously unleashes his \sathariah{} power} and scares \Nzessuacrith. 
\Shiaraid{} immediately detects and recognizes him. 
She knows Ramiel better than any other and can recognize his \vertex{} signature in a moment. 

This is a godsend for \Shiaraid. 
For centuries she has been searching for a way to come back to life. 
Now her old lover is suddenly reborn. 
This is an opportunity she cannot afford to waste. 
She must secure his alliance. 
She then sends some Royalists hunting for Ramiel (including Shereid). 

In addition to Ramiel's help, \Shiaraid{} also needs a strong link to the Midnight Bat \matrix. 
This is where \hr{Sentinels help Shiaraid with Ghost Tower}{the Sentinels can help}. 





\subsubsection{Breakthrough}
\target{Royalist breakthrough in Redce}
\Belzir \hr{Belzir keeps in touch with Royalists}{kept in touch with her Royalists}.
They drilled holes in her prison so \hr{Belzir awakening}{she could gradually awaken}. 

Maybe the book should open with a scene that shows the Royalists drilling an important hole. 
They had now drilled so many holes in \Belzir's prison that she could now also contact the non-Royalists in dreams.
She began to be able to use mind-controlling magic on the Redcor. 

This was why \hr{Redcor need Carzain}{the Redcor needed Carzain}. 





\subsubsection{\Banes}
The Royalists may have \banes on their side. 
\Banes are nasty because they have nasty powers. 
They can see and move through the Beyond, and as such they can spy on people from Beyond, sneak through impossible crevices, walk through walls and even \quo{disappear} by \hs{submerging} or pounce from nowhere by \hs{surfacing}.

The \banes tempt and lure people into their clutches using mental attacks, then corrupt and mind-control them. 
The \humans can feel that their nature is connected to these monsters. 
It is an awful thought, but also alluring and fascinating. 
Horrified, the victims cannot look away. 
The thought makes people doubtful and afraid and even desperate, which makes it all the easier for the \banes to manipulate and mind-control people. 
The people thus touched sometimes go mad afterwards. 

Compare to how the Elder Things drive people mad and control them in \cite{TimCurran:Hive}. 

There are mostly \lesserbanes, but also a few \greaterbanes. 

Those \humans who are really badly possessed look awful. 

\citeauthorbook[p.199--202]{TimCurran:Hive}{Tim Curran}{Hive}{
  \ldots at that moment it would have been hard to picture a more dangerous man than Holm.
  There was something cold and remorseless about him.
  
  \ldots
  
  Holm was looking at him and his eyes were filled with a chill blankness-
  There was nothing in them.
  Nothing \human at any rate.
  He surveyed Hayes with a flat indifference, that pallid face punched with two black eyes that made something go liquid in Hayes' belly.
  You didn't want to spend too much time looking into those eyes.
  They wer elike windows looking through into some godless, deadEnd of space.
  You could see yourself there, suffocating in that deranged, airless void. 
  
  Hayes swallowed.
  
  Those eyes drilled into him, sucking him dry.
  
  There was power in those eyes, something immense and malignant and ancient.
  The way Hayes was feeling at that moment was how he felt looking tino those glassy red orbs of the aliens in Hut \#6. 
  They got inside you, owned you, crushed your free will like a spider under a boot.
  At some primary level, they consumed and swalled you.
  And you could feel all that you were sliding down into some black soundless gullet. 
  
  \ldots
  
  Holm looked up at them with that same almost insipid blankness.
  His black eyes like those of a grasshopper consider a stalk of grass.
  That's how they looked\ldots unintelligent, completely vacant.
  At least at the moment.
  But Hayes knew those eyes and what they could do.
  One moment they were dead and empty, the next overflowing with all the knowledge of the cosmos. 
}





\subsubsection{\Bane statues}
The Redcor discover statues of \banes. 
The statues are small, stylized but gruesome. 
The worst part is the faceless head. 

The statues are elastic and seem to be made of flesh, which is a horrible realization. 

The statues mess with people's heads and make them crazy or mind-control them. 
The statues are also useful for transforming people into \banes. 
(\Banes have to \hr{Banes possess Humans}{come to \Miith through \human bodies}.)

People who find the statues will find themselves dreaming or hallucinating about the aeons-old prehistory of the \banes, how they waged wars of destruction against \dragons and other monsters, how they created \humanity as slaves or toys.





\subsubsection{\Lithrim}
The \humans see hints of their true nature, as parts of \hr{Lithrim}{\Lithrim}. 
(Read about \Lithrim.) 
They envision themselves as \banes:
Small slimy cogs in a vast, hideous, faceless machine-hivemind of pure evil. 
(\Banes are, after all, a thing of the \hs{dead universe}.)





\subsubsection{\Belzir's skull}
The Redcor have \Belzir's skull or something.
She needs it if she is to regain a physical body. 
The Redcor know this, so they keep it safe in some vault. 

Carzain learns about the legend of the skull.

Then the skull gets stolen.
Carzain gets hired to help find it. 
Much of the book is spent searching for the skull. 

There is a scene where the thief goes to deliver it to his employer.
The employer takes the skull and sends the thief away.
The thief is scared, maybe even killed. 

In the end it turns out that Carzain has the skull.
It was he who masqueraded as the thief's employer and took it from him.
Now Carzain goes to Geica with the skull.





\subsubsection{Shereid}
\hs{Shereid} is the first Royalist to find Ramiel. 

She wants to \quo{recruit} him and make him ally himself with \Belzir{} and the Royalists.
But Shereid cannot reveal her identity too soon. 
If Carzain were to refuse her offer, she would be in danger. 

She and Carzain initiate a subtle dance of evasion. 
He (Vizicar) quickly picks up on the fact that she knows something she is not saying; that she is more than she appears. 
He tries to ferret it out of her. 
But she is a skilled deceiver herself. 
She does not want to tell him before she is sure he is on her side. 

So she tries her best to turn him against the Redcor and get him to like the idea of a Geican revolution, and personal power and glory and sex. 
She talks about how bad the Redcor are and how great it would be to be free of them, to see how he reacts. 
He tentatively agrees, but he is also wary of her and does not tell her too much. 
It is a delicate game of mutual distrust. 

At last she lets slip to him the possibility that \Belzir{} is still sort of alive. 
This makes Vizicar very interested. 
He very much wants to meet this other Scion and exchange knowledge with her. 
He has read some claims that she achieved \apotheosis. 
This idea gives him a boner. 

But he still does not trust Shereid. 
Carzain and Vizicar have an internal dialogue about how they will not let her control them. 
I should also have plenty of scenes where he criticizes her\dash to her face, to others or inside his head. 
This is meant to fool the reader into thinking that he does \emph{not} want to go along with Shereid's ideas. 

But in fact he does. 
He likes Shereid. 
Much better than the Redcor. 
So, in the end, he betrays the Redcor (including \Racel) and goes with Shereid to Geica. 





\subsubsection{Tentacle rape}
Shereid or Needle is there. 
She serves the \banes.
Every night she feels like she gets raped by tentacles. 
It is awful and painful, but she craves it and cannot live without it.
She is already mad and a pawn of dark forces. 





\subsubsection{\Belzir gets a body}
\Belzir possesses the body of a Redcor\dash a woman of 50 or so, but still healthy and beautiful. 
She is mean and evil, like the evil Faith in \cite[season 3]{TV:Buffy}. 
At the end the reader should really hate her. 
It is an awful shock for the reader when Carzain betrays his allies and joins \Belzir, but the reader gets some measure of closure when Carzain also betrays \Belzir. 
(Unless he doesn't betray her yet and waits for the end of the next book.)









\subsection{Underground cult in \Redce}
\target{Cult in Redce}
In \Redce{} there is an underground cult of hopeful, gullible youths who worship dark powers and hope for some kind of reward or insight. 
It is very much an ecstasy cult, with sex-orgies. 
The young people get their sexual urges satisfied, and more so, with drugs and dark magic. 
Maybe the dark powers feed on their sex. 

Have sexy priestesses and erotic rituals, including live sacrifice. 
The sacrifices may be unwilling captives, or willing, or mind-controlled. 
It works best, I suppose, if it's sexy girls who willingly let themselves be killed in a sexual ritual of pleasure and pain. 

The sexy priestesses have power, but they ultimately crawl at the feet of the master of the cult. 
The master is a badass gangster, like Lex from the movie \cite{Movie:GargoylesRevenge}. 

\lyricslimbonicart{Twilight Omen}{
  I salute thee, baptizer of my soul.\\
  Dear pagan master, let thy universe unfold.\\
  Show me the sign of the midnight sky.\\
  I will forever follow until dawn's early light.\\
  From the ashes a fire shall be woken.\\
  From the shadows words shall be spoken.\\
  A star constellation, \\
  a burning circle of serpent eyes.\\
  Esoteric mysteries of unknown life.
}

The Cabal actually know about the cult. 
At first, they allowed it to exist.
Back then, it was clear that the cultists were not Sentinels.
The Cabalists did extensive research and found \emph{no} links from the cultists to any Sentinel-aligned \matrices. 
In fact, the cult's magic was not even real, just smoke and mirrors. 

Moreover, the cult serves a useful purpose as an outlet for the people's evil nature. 
The Cabal knows that people have evil lusts. 
The Redcor try their best to suppress those lusts. 
For some people, the internal pressure becomes unbearable and they must have an outlet. 
From the Cabalists point of view, this underground cult is as good an outlet as any. 
So they let the cult exist, but keep it under surveillance. 

(Note that only the Cabal know about the cult. The Redcor know nothing.) 

For a while this went fine for the Cabal. 
But then, unbeknownst to them, \ps{\Shiaraid} Royalists found out about the cult. 
In deepest secrecy, she had her people infiltrate the cult. 
Now she has quite a lot of influence in the cult. 
She does not rule the cult, but she has enough power to use the cult for her purposes, as long as she is crafty. 
She can run her own business in the shadows, and whenever her people are discovered she can frame the cult and have it take the heat for it. 
The cult is not so well-organized, so the cult leaders and the Cabal spies do not have an overview of everything the cult members do. 
\Shiaraid{} herself encouraged this infighting and chaos and division, because she knew that chaos and outlawry would be the perfect milieu in which for her to work her shady schemes. 

The Redcor have no idea that \Shiaraid{} is back. 
To their minds, she is an ancient enemy they defeated millennia ago, and (despite what they may say to frighten children and outsiders) they do not suspect she will come back to haunt them any time soon. 

Under the cover of a \quo{harmless} ecstasy cult, \Shiaraid{} has sent her spies crawling all over the place under the \TopazChateau{}. 
She knew the Redcor had a piece of her soul prison in their keeping. 
Now, after decades of spying and searching, she knows where it is. 
But she cannot get at it. 
She does not have any contacts in the \Chateau{} that are sufficiently high-placed or sufficiently powerful. 

Hence, it is extremely fortunate for her that Carzain is now in \Redce. 
If she can get him converted to her cause, he could help her get her soul-jar thingy from the Redcor. 





\subsubsection{Scenes}
Have erotic scenes, like in \cite{GaryMyers:TheHorrorShow}, where a hot girl is stripped naked, tied up, beaten halfway to death and then sacrificed. 
In the end, she changes her mind and screams and begs for mercy, but in vain, and she is devoured. 
All the while, the cult chant their songs to their dark gods. 









\subsection{\Kezerad}
The \Kezeradi help the Redcor fight the \banes. 
They appear as tragic, tortured, weeping angels. 
Full of suffering but still brave and enduring and fighting to save others. 
They want to stop \Belzir. 

Remember that the \Kezeradi mourn the loss of their \hr{Beacons of Kezerad}{\beacons}. 
The \beacons have been taken from them and given to these mortal Vaimons. 
But do not mention that \iquin is evil. 

\target{Sithiyacaan in Redce}
\Sithiyacaan is in \Redce, in the guise of a \human man named \hr{Herette}{\MoriceHerette}. 
(Read about \Herette.)

When Ramiel destroys \Redce \Sithiyacaan finally sort of awakens. 
He \hr{Sithiyacaan goes north}{goes north after him}.










\subsection{Ghost Tower}
\target{Sentinels help Shiaraid with Ghost Tower}
In addition to Ramiel's help, \Shiaraid{} also needs a strong link to the Midnight Bat \matrix, and through it the \Erebean{} \dweomer. 
Ramiel cannot provide this. 
His \kenosis{} weakens his link to the \matrix{} and \dweomer. 

But the Sentinels can. 
The \hs{Ghost Tower} is a link from \Azmith{} to \Nyx, remember. 
\Secherdamon-tachi plan to conquer the Tower, then \cooperate{} with \Shiaraid{} and Ramiel to cast a complex spell that will penetrate through the Tower into \Nyx{} and tap into the Midnight Bat and the \dweomer, then tap the energy and channel it through the Tower, through a complex \quo{series of tubes} devised by \Vizsherioch{} and built with assistance from the Royalist Faction, all the way from Pelidor through the Beyond and into Geica, where \ps{\Belzir} body is. 
This energy, combined with Ramiel's help, will let \Shiaraid{} rise from the dead. 

But there is a problem. 
This could be detected. 
If the Sentinels start siphoning energy from \Nyx, everyone will know. 
And the Sentinels still do not want it known that they are in bed with \Shiaraid. 
So they do something sneaky: 
They make it collide with another long-term plan of theirs, namely that of \hr{Vizsherioch becomes Shaeeroth}{raising \Vizsherioch{} to \shaeeroth{} status}. 
They will perform both \ps{\Vizsherioch} \shaeeroth{} ritual and \ps{\Shiaraid} resurrection ritual at the same time. 
Most of the energy will be consumed by \Vizsherioch, but some of it will be siphoned away and used by \Shiaraid. 
That way, if this is discovered at all, it will look as if some clever Royalists infiltrated the Sentinels and installed a \quo{back door}, stealing some of the Sentinels' energy and siphoning it away to use for their own nefarious purposes. 
No one will ever know. 

Of course, in order for the Sentinels to accomplish any of this, they must first capture the Ghost Tower. 
This is not easy, for the \resphain{} have turned it into a bastion and are fighting to defend it. 






\subsection{Telcastora Ilcas}
Telcastora Ilcas is in \Redce with Carzain and the rest. 
He is braver and more efficient than most Redcor. 
Partly because he does not have their restrictive religion and taboos, and partially because as a \scatha he is resistant to some \human-specific mental attacks and instinctive fears. 

He is \hr{Ilcas badass}{a badass Tisamon-type character}.

The Redcor suspect Ilcas of being the mole because he is an outsider, a heathen. 
And also arrogant and aloof to boot. 
Besides, he is a \scatha, where the Redcor bosses are \human. 
(Though \hr{Redcean demographics}{there were \scathae in \Redce}.)

\target{Ilcas suspects}
The Imetrium is taking an active interest in \Velcad. 
They are ancient rivals of the Rissitics and do not want the Rissitics to conquer and gain too much power (be it secular or metaphysical power). 

Telcastora Ilcas escorts Carzain-tachi to \Redce. 
Then he turns back to southern \Velcad. 
There he works with other Imetrians and allies to oppose the Rissitic plan. 

He has to deal with the sinister \nagae{} that are the allies of the Imetrium. 
Gradually, he discovers more about the Imetrium's involvement with \nagae{} and other wicked powers. 
He becomes slowly disillusioned with his religion. 
He is learning \hr{The truth about the Imetrium}{the truth about the Imetrium}, and it is a darker truth than he would have liked. 

Gradually he turns away from his commanders, and even his gods. 
But only a little bit. 
He has served the Imetric gods all his life. 
He remains true to the ideals he has always upheld. 
But he becomes more critical and distrustful. 

From the Imetrians' point of view, Ilcas serves an important purpose in this battle. 
His sword, \Telderain, is a powerful \vertex{} in the Imetric \matrix. 
So when they find out he has gone prancing off to \Redce, they quickly send for him and fetch him back to southern \Velcad{} where he is needed to form part of the metaphysical bulwark against \ps{\Secherdamon} Sentinels. 

(Note that the Imetrians are not necessarily opposing \emph{all} Sentinels. But they are certainly opposing \Secherdamon.) 

In the process, Ilcas will encounter \Narkiza. 
They find out they have a lot in common. 
But they are destined to be enemies. 









\subsection[Mystraacht]{\Mystraacht}
Some of the \Mystraacht{} have learned that Ramiel is alive. 
They want to kill him, \hr{Mystraacht rival goes after Ramiel}{so they hunt him}. 
But \hr{Azraid protects Ramiel}{\Azraid{} protects Ramiel}, as \hr{Ramiel meets Cishiel in dreams}{does \Cishiel}. 





\subsubsection{When does Ramiel meet \Cishiel?}
\target{When does Ramiel meet Cishiel?}
When does Ramiel meet \Cishiel?
There are some possibilities. 

\begin{enumerate}
  \item 
    \Cishiel{} should not discover that Ramiel lives until late in the story. 
    She and \Dasteron{} need to be introduced before that point. 
    
    In fact, maybe \Cishiel{} should be postponed to the next book. 
  \item 
    If I present the backstory of \Cishiel and \Dasteron \hr{Flashback with Cishiel and Dasteron}{as a flashback}, I can introduce \Cishiel at any time I wish. 
    She could \hr{Ramiel meets Cishiel in dreams}{appear in Carzain's dreams}.
\end{enumerate}












\subsection{Rissitics}
We follow \Narkiza{} and his Rissitic army, who are fighting their way through southern \Velcad{} in a campaign of conquest. 

In the \Narkiza chapters, make it clear that the \hs{Rissitics value their lives}. 

Their attack is partially meant as a diversion, drawing Redcor and Cabal attention away from more important matters such as \Nithdornazsh{} and the Royalist Faction's resurrection project. 

But the invasion also serves a useful purpose. 
\Secherdamon{} knows (or suspects) that the final breaking of the Shroud is near. 
He knows \iquin{} has a Hell of a lot of power. 
He wants to try to break that power. 
To this end, he wants to destroy the Redcor. 
He hopes to enlist the help of \ps{\Shiaraid} Royalists. 

The plan is that the Rissitics will move in to attack some Iquinian key strongholds and nations. 
If these fall, it will cause Iquinianism to falter on an \Azmith-wide scale. 

The Redcor know this, so they will set in as many resources as they can to fight off the Rissitics. 
This does not merely mean sending in a bunch of Vaimons. 
It also means calling in favours, pulling strings, drawing deep of the Redcor credit coffers and bullying allied rulers into banding together and fighting the Rissitic menace and defending the Iquinian world.

Meanwhile, the Royalists will resurrect \Shiaraid. 
This will be done in some place relatively close to \Redce. 
\Shiaraid{} will harbour a great hatred against the Redcor and will want to take revenge on them. 
With her \sathariah{} powers, her loyal Royalists and her covert Sentinel backing, she should be able to wreak a lot of havoc on the Redcor, which would seriously harm the Iquinian church. 
(And, with luck, this will not implicate the Sentinels at all and thus not expose \ps{\Secherdamon} long-term plan.)

If the Iquinian church can be destabilized, it will harm the global Shroud of Civilization that the church exerts over \Azmith. 
This will allow \Secherdamon{} to use gain more power in \Azmith{} and use the Realm to complete the forging of his Dagger. 





\subsubsection{\Secherdamon{} and Ramiel}
\Secherdamon{} does not know that Ramiel is back. 
And \Shiaraid{} does not tell him, even though they do communicate every once in a while. 
She keeps Ramiel a secret from \Secherdamon.

\Secherdamon{} discovers the secret near the end of the book, though. 
When someone helps \Shiaraid{} rescue her stuff from \Redce, Ramiel releases some powerful energy. 
\Secherdamon{} has been paying attention to what \Shiaraid{} is up to\dash she is his ally, but he trusts her only as far as he throw her\dash and this sudden burst of new energy interests him. 

\Secherdamon{} sees clear signs of activity in the Midnight Bat \matrix, and he hears some rumours. 
He puts two and two together and deduces that she has an ally who is a \resphan{} in disguise\dash perhaps a Scion. 

\Secherdamon{} or \Vizsherioch{} then talks to \Shiaraid. 
She is mentally unstable and has a hard time concealing her feelings, so \Secherdamon{} begins to suspect that her old lover, Ramiel, is back. 

At last the cat comes out of the bag. 
Ramiel kills \Shiaraid{} during her would-be resurrection. 
He can no longer hide his existence. 

(At this point, Ramiel knows who he is. 
 Maybe he has even taken contact with \Cishiel{}. 
 But maybe not.
 See the section about the question of \hr{When does Ramiel meet Cishiel?}{when Ramiel meets Cishiel}.) 

\Secherdamon{} is nonplussed and \hr{Secherdamon wants to off Ramiel}{wants to off Ramiel}. 









\subsection{\Ishnaruchaefir and \Azraid}
\target{Ishnaruchaefir and Azraid plot together, early in TBW}
Have a scene in the beginning of this book or the next.

\Ishnaruchaefir and \Azraid meet (\hr{Ishnaruchaefir and Azraid develop empathy}{as they have done before}). 
Gradually and ever so subtly through \SentinelsofMithEmph, \Ishnaruchaefir becomes convinced that \Azraid is on the level, and he also begins to suspect the general nature of \Azraid's plan. 

So when Azraid finally \hr{Ishnaruchaefir and Azraid plot together, late in TBW}{approaches \Ishnaruchaefir and asks for his cooperation in a secret venture}, \Ishnaruchaefir has expected it. 









\subsection{\Banes and \xss}
I should gradually build up suspense between the \banes and the \xss throughout the whole series. 
In the beginning, the players are smaller forces.
Gradually the reader learns more about the big picture. 

Make it slowly come clear that \Miith is caught between Scylla and Charybdis, and that \Miith's walls are slowly crumbling. 
Beyond the mundane wars of the \Miithians, beyond even the age-long war of the \dragons and \resphain, greater menaces threaten. 
Madness and horror are poised to pour into the world like a destructive tidal wave from the endless darkened voids Beyond. 



















\section{Prologue}









\subsection{The Fall of \Kezerad}
\target{Sithiyacaan despairs after Fall}
Have a flashback to the \hr{Fall of Kezerad}{fall of \Kezerad}, or the time just after it. 
\Sithiyacaan is desperate and cracking apart with sorrow. 
His world has crumbled, the \beacons are lost, \Essenai is destroyed (or so he thinks). 
He rages and raves while his brethren try to make him pull himself together. 
They know it is a catastrophe for morale if their great, brave leader breaks down into a wreck. 
But \Sithiyacaan is caught in the claws of \NexagglachelsCurse, which makes him extra vulnerable and mad. 









\subsection{Carzain is born}
Have a flashback to the day Carzain was born. 
Or, rather, the day he was conceived and became a foetus.

Before his birth, Carzain is a disembodied, barely conscious thing. 
In this state, he is less constricted by the Shroud. 
He can see beyond it.
He is also less afflicted by his usual \malach amnesia.
He knows and remembers his true self (in a semi-conscious way). 

He is an angel.
A god. 
He is the lord of tons of souls (in his \carcer).
He is the king of the world. 
He remembers his power as a \sathariah and a \malach.
Limitless power is at his fingertips. 

He is free of the Shroud. 
He is one with the universe. 
He can see and touch the entire Cosmos.
All bliss and glory is his. 
Everything is beautiful and perfect. 

But then it is torn away. 
He snatched out of his blissful rest. 
He is squeezed and hammered and compressed into a tiny, weak, constricting body. 
He is cut off from his divinity and immortality and power. 
He feels his mind slipping away.
His memories are being taken from him. 
He cannot remember who he is.
He can feel only loss. 
He is bewildered. 
He is afraid.
Desperate. 
His very being is being torn from him, and he can do nothing about it. 

He is blind. 
Mindless. 
Drowning in wet slime. 
He cannot breathe.

Flash forward to his birth.

Carzain is pushed out of his mother's womb. 
And suddenly he feels cold. 
There is blood.
A voice screams.
It is his own. 

Change POV to his parents. 
Nishain comes in. 
The midwife tells him it's a boy. 
Nishain touches his son and speaks his name: 
\ta{\CarzainShireyo.} 



















\section{Carzain in \Redce}









\subsection{Carzain-tachi go to \Redce}
\target{Carzain goes to Redce}
Carzain-tachi go to \Redce. 





\subsubsection{They meet Shereid}
They meet Shereid. 

Carzain notices her \hr{Geican green eyes}{green eyes}. 
He \hr{Carzain wants green eyes}{wishes he had green eyes}. 




\subsubsection{\TopazChateau}
They see the \hr{Topaz Chateau}{\TopazChateau}.
It looks as though it is really made of topaz.
Vizicar remembers his old Rainbow Palace, which was also crystalline but even more awesome. 









\subsection{Alliance with the Redcor}
\target{Carzain with Redcor}
At length, Carzain reaches \Redce{} and joins up with the Redcor. His mother is Redcor-born, and her name, \Deracille, is a Redcor name of the lesser nobility, so Carzain, grudgingly acquiescing to using his mother's last name, is accepted. He is trained as a Gandierre by Lacquasse, whom he comes to like and respect. 

He quickly learns to distrust the Redcor and not submit to them too easily. When his training as a Gandierre apprentice is complete, he is released from his bonds as an apprentice, and it is assumed and expected that he will take his vows as a Gandierre. But Carzain surprises them all by refusing to join the Gandierre. He has studied the Redcor laws and found a loophole, saying once free of apprenticeship, he has no more duties to the Redcor. However, he graciously offers to remain as their ally. The Redcor are less than happy about this, but they have to accept. 










\subsection{\Redcean climate}
Remember what the \hr{Redce climate}{\Redcean{} climate} is like. 
This will influence the story. 









\subsection{\Esmerel has won a victory}
When \Esmerel{} brings Carzain back with her to \Redce{} and can prove that he is a Scion, it is a professional victory over some of her rival scholars.
Many of her colleagues refused to believe her theory that the \spike{} was a Scion. 
There were many other reasonable explanations, after all. 
But \Esmerel{} was a bigger Scion expert than any other, and she was certain her theory was correct. 
Now she has proven it. 

Carzain and Vizicar are not happy to be paraded as her trophy. 









\subsection{Carzain was wrong about \Esmerel}
Carzain sees \Esmerel{} in a new dress: Much lower cut, showing off a bit of cleavage. 

He tells her: 
\ta{\Matron{} \Esmerel, I was wrong about you. 
I understand now that underneath your gruff, arrogant and bitchy exterior, you are actually a real woman\ldots{} with a pair of very nice boobs.}









\subsection{Carzain is secretive}
Throughout the book, Carzain is being secretive and furtive and suspicious. 
He is often absent at critical times and refuses to tell people where he has been, or tells partial truths, or lies. 
Later it turns out that he is keeping mistresses. 
He has been sneaking out and having sex with them, and that is why he is so often absent. 

This is part of his plan.
He has seduced several Redcor women and thus forced them to help cover for him and keep his doings secret. 
He also compels them to do things for him that help his secret plans.
His mistresses never learn much about what he is up to.

Ostensibly Carzain is doing it for the sex. 
In reality he is just using the women as tools. 
They are his alibis while he conducts his shady business.
He is manipulating everyone and setting up the stage so he can betray the Redcor, help \Belzir and then later betray \Belzir as well.

Along the way Carzain acts as a detective and unearths other plots and secrets. 
He appears to be helping the Redcor, but he always has a hidden agenda.









\subsection{Carzain gains an enemy}
Carzain is placed as a subordinate to a Redcor. 
She bullies and humiliates him, treating him like a stupid child. 

He tells her: 
\ta{%
  You have gained an enemy. You have me outnumbered today, but know this: 
  One day I will kill you.} 









\subsection{Carzain sees the \Morbus{} in \Redce}
\target{Carzain sees the Morbus in Redce}
Carzain walks around in \Redce. He ventures into the slums, where he sees the \hr{Morbus}{\Morbus} at work.







\subsection{Look at the zwongas on that one}
Carzain is in \Redce{} with a friend. 
A girl with very large boobs walks by. 

The friend exclaims: \tho{Look at the zwongas on that one!}

\target{zwonga}
\target{zwongas}
A \quo{zwonga} is actually an oriental fruit, a soft thingy like a very large plum. 
It is also slang for a woman's breasts, because a zwonga feels kind of like a breast. 









\subsection{I have zwongas}
Carzain is at a ball in \Redce. 
A serving girl comes up to him. 

Girl: 
\ta{Good day, sir. 
I have zwongas. 
Very nice. 
Would you like to try one?}

Carzain spewed and, for the next many moments, could only stare at the girl\dash and her zwongas\dash with a dumbfounded grin. 
He later learned that the girl was selling zwongas, which were actually a kind of fruit. 









\subsection{Carzain rescues ungrateful Redcor}
Carzain rescues a Redcor. 
She is a bitch from moment one, trying to order him around. 

Carzain: 
\ta{I haven't rescued you yet. 
If I were in your position, I wouldn't be mouthing off.}








\subsubsection{Carzain watches his language}
Later, she orders him: \ta{Watch your language, boy!}

Carzain: \ta{Fuck you. You can suck my big, sweaty cock until you choke on my cum and die.}







\subsection{Carzain practices archery}
From time to time in \Redce, Carzain practices archery in his spare time. 
He is a sucky archer, rarely ever hitting the target. 

It's a habit he inherited from Vizicar. 
In Vizicar's time they had guns, so archery was a sport, a harmless hobby. 
Vizicar was a sucky archer, too. 







\subsection{Carzain is lazy}
The Redcor try to get Carzain to practice dilligently, but he is lazy and would rather be a \boheme. 

But, from time to time, Carzain has flashes of mastery. This is typically when Vizicar (or an even older incarnation) awakens to help him, or take control. This frustrates the Redcor. 

It also serves as a subversion of the oft-seen \trope{AnAesop}{Aesop} that you must train and practice all the time to achieve greatness: Carzain pulls greatness out of his ass. (But not all the time. That would make him an annoying \trope{MartyStu}{Marty Stu}. Just once in a while.)









\subsection{Someone disbelieves Carzain's talk of monsters}
Carzain talks about monsters. Someone, a Redcor, immediately laughs. \ta{Surely you are imagining things.} Compare to episode 2 of the anime \cite{Anime:Gilgamesh}.

Vizicar immediately sees through the guy. 
\vizicar{%
  Too fast a reaction. He is trying too hard to conceal what he knows, to avoid suspicion. We need to keep an eye on him, Carzain.}









\subsection{Telcastora Ilcas faces off with a Trickster Mentor}
There is an old Redcor woman who tries to play the \trope{TricksterMentor}{Trickster Mentor} and be all hysterical and demanding. She tries to get people to jump through hoops for her before she'll help them. 

Telcastora Ilcas calls her out on it. 
\ta{Listen to me. 
  We are not supplicants begging a favour of you. 
  We come to you as equal allies and call upon you to do your \emph{duty}. 
  No more, no less. 
  And now you wish to hold the future of the world hostage to your own petty desires, your perversions? 
  Kindly explain to me, then, what makes you any whit better than the Rissitics? 
  What makes you any less of an enemy of the world?}

Carzain is impressed by Ilcas' ballsy-ness. 









\subsection{Ilcas comments on worms}
Some people eat food with \Durcaci spice. 
They talk about the rumour that \hr{Spice and worms}{the spice is produced by giant sandworms}. 
They ask Ilcas about it, since he has \travelled in \Durcac. 

He says he \emph{thinks} it is true. 
But he has never seen these worms. 









\subsection{Carzain and Vizicar work together}
Carzain and Vizicar gradually learn each others' strengths and weaknesses and how to best combine them. 

Vizicar's practiced diplomacy combined with Carzain's irreverent daring make them a great detective and manipulator, akin to Vlad Taltos from \authorseries{Steven Brust}{Dragaera}. 
(Vizicar on his own is a bit too constricted by principles and etiquette after having spent his whole life at court. Carzain teaches him to loosen up and be more open-minded and undaunted.)







\subsection{Vizicar surprised by map}
At one point Carzain looks at a map. 
Vizicar is in his head, reading along. 

\begin{prose}
  Vizicar: 
  \vizicar{Damn. This looks nothing like the maps I know.}
  
  Carzain: 
  \ta{Huh? Oh, right. Makes sense. 
    You lived before the \Darkfall, after all.}
  
  Vizicar: 
  \vizicar{The what?}
  
  Carzain summarizes what he knows of the \Darkfall. 
  
  Vizicar: 
  \vizicar{Interesting.} 
  
  Vizicar is especially intrigued by this \hr{Belzir}{\Belzir} character. 
\end{prose}







\subsection{Carzain hates the Redcor}
The Redcor are bitches. Carzain puts up with it for a while, but gradually, their arrogance makes him grow to hate them. 

The Redcor know that the Rissitics are up to something. Gradually it turns out that they mean to resurrect a \daemonic{} demigod, a \Haskelek{}. 
%See section \ref{Haskelek story} for details. 
%The \Haskelek{} is an old ally of the Sentinels and a terribly evil \daemon. It was brought to \Miith{} once about 1000 years ago (after Vizicar's reign but before \Belzir's) where it conquered and controlled a kingdom in central \Velcad{}, current day Pelidor and surroundings. But after a prolonged war, the \Haskelek{} was defeated by the Vaimons. 

%Unable to destroy it, the Vaimons imprisoned the \Haskelek{} in his temple, located in the middle of a huge, wild forest south or southeast of Pelidor. There it lay dormant, but its evil seeped out and corrupted the beasts and people in the surrounding area. The forests around its temple became dark, twisted and hostile (even more so than the regular \Miithian{} nature, which can be harsh enough) and the humanoid tribes to become degenerate savages who now worship the \Haskelek, or the memory of it. 

%There might be some planar stuff going on here, with the temple residing in another plane of reality. I should give this more thought. 

%Somehow the Rissitics reach the \Haskelekz{} temple and awaken it. It has grown stronger in the meantime by feeding off the prayers and sacrifices of his primitive worshippers, and is now almost as strong as before its death, but it needs powerful help to break free of its prison. The Rissitics free it. 

%The Redcor now oppose the \Haskelek{} and attempt to destroy it. It comes to a climactic battle, where they weaken \Haskelek{}, forcing it to retreat to its temple to recuperate. Redcor forces, including Carzain\dash and possibly Curwen\dash pursue him there, intending to reactivate the old spell seals and reseal the \Haskelek{} in its old prison\dash or something like that. 





\subsection{The Redcor leaders are incompetent}
The Redcor fight against the Rissitcs and Sentinels and the \Haskelek. But some of their leaders are incompetent fools. So, at times, Telcastora Ilcas or Carzain/Vizicar have to take command and lead their forces to victory. (In case of Carzain/Vizicar, it's more of Vizicar, since he is better versed in the art of war and commanding armies). 

Compare to Jarod Shadowsong in Richard Knaak's \emph{The Sundering}. When he leads the warriors into battle, compare to the king of Hyperborea in \bandsong{Bal-Sagoth}{The Splendour of a Thousand Swords Gleaming Beneath the Blazon of the Hyperborean Empire}. 







\subsection{Carzain runs from battle}
Carzain and some Redcor are in trouble, threatened by enemies. 

A \gandierre: \ta{We must fight! \Honour demands no less!}

Carzain: \ta{I have a better idea: Run like Hell!}

Telcastora Ilcas: \ta{I must agree with \Shireyo. I have my own \honour, but it does not require me to let myself die if I can do more good alive.}

Perhaps this is another instance of Ilcas inspiring Carzain to value his life above that of others\dash which, by utilitarian reasoning, is perfectly justified. 







\subsection{Carzain fails at etiquette}
Carzain gets owned regularly in \Redce{} because he does not know Redcor etiquette, or because he refuses to acknowledge it. 

He bitches and disses the hypocrisy of the Redcor rules and taboos. Perhaps he discusses it with Vizicar. 







\subsection{Carzain discusses philosphy with Shereid}
Carzain discusses \hs{Redcor philosophy} and \hs{Geican philosophy} with Shereid. 








\subsection{Carzain is fencing, no Vizicar}
Carzain is in a sword-fight. He tries to \quo{summon} Vizicar, who is a superior swordsman, and get him to come out and take control. But Vizicar is submerged in dormancy at that moment. 

Carzain is not a good enough swordsman to win on his own, so he flees. The Redcor are nonplussed by his cowardice. 










\subsection{Vizicar takes crazy chances}
Vizicar takes over their shared body and takes some crazy chances that nearly kills them both. 

\begin{prose}
  Carzain: \ta{You crazy fuck, are you trying to kill us?}
  
  Vizicar: 
  \ta{Well, I've already died once. 
    I got better. 
    That puts things into perspective, you know.}
\end{prose}









\subsection{Carzain sacrifices himself by fleeing}
Carzain has the choice of rescuing someone\dash\Racel?\dash at great risk to himself, or flee and save himself. Inspired by Telcastora Ilcas's utilitarianism, he chooses the latter. 

\Racel{} survives and later confronts him: \ta{What did you think you were doing? Why did you not come and rescue me? You knew I was in danger! Have you no \honour? As a man, it is your Light-given duty to risk your life to save a woman! Or are you no man at all? Are you truly so selfish, so cowardly, so petty?}

Carzain slaps her up. \ta{Listen to me, you cunt. \emph{I} am the Scion! \emph{I} am the one whom you Redcor rely on to save you. I am the most important person on this expedition. Without me, there is no mission. \emph{I am the mission}. Without me, all that you have fought to accomplish will be for nothing. It follows that my life is worth more than any of yours. My foremost duty, then, is to remain alive, that I may fulfill the mission.

Nay, it is you who are selfish. It is you who are petty. It is you, you Redcor fuck, who don't know the meaning of sacrifice. I would love to save your pretty little ass\dash and with it, it would seem, my \honour and manhood\dash but I know my duty. For that, I am willing to sacrifice my \honour among the Redcor if need be. 

That is one of the many things that separate us: You are a fool, a blind fool, trapped in dogma and hypocrisy. Whereas I am wise enough to know right from wrong without such stupid principles. That is why I am better than you!}







\subsection{Carzain reveals his friend's secret}
Carzain has a Redcor friend who confides a secret to him. He makes Carzain solemnly swear that he will not reveal the secret to anyone. 

Later Carzain suspects that something fishy is going on, and tries to solve it. He works together with some allies\dash possibly Shereid. 
In order to solve this mystery\dash and save humanoid lives and potentially avert disaster\dash he deems it necessary to reveal the friend's secret. 

This is one of the first of Carzain's many broken promises.







\subsection{Carzain's \human{} enemy}
Carzain is standing and hating some enemy, possibly a Redcor. 

A friend of his defends the enemy, saying: \ta{When you look closer at her, she's really very \human{}. Look, she has all these nice qualities and does all these sympathetic, \human{} things\ldots{}}

Carzain: \ta{So? How does that excuse her crimes? In fact, it makes it worse. If she was a piece of shit all the time, then she might have the excuse of stupidity or a bad upbringing or whatnot. But her \quo{\humanity} only shows that she can be worthwhile if she chooses. Alas, she chooses to be a stupid bitch.}







\subsection{Carzain learns to control his souls}
Carzain slowly learns to control the dead souls bound to him, and to channel them as a weapon. (See section \ref{Bound souls as a weapon}.)

Carzain unleashes this power by accident one or more times in the latter half of this story, in a fit of extreme anger, anguish or other emotion. He is horrified by what he has done, as is Vizicar. 

Describe the victim's fear and horror. Inspired by Clive Barker's \emph{Jacqueline Ess: Her Will and Testament}. 

Afterwards, Carzain is like: \tho{What the fuck? That was no \Qliphah, and it sure as fuck wasn't a \Sephirah!}







\subsection{Carzain disgruntled}
After their victory, some of the Redcor are praised for their great worth, but Carzain is neglected and gets no praise, only reprimands. This causes him to resolve to turn on the Redcor. Shereid urges him on in that direction. Ultimately, while out in the wilderness somewhere\dash perhaps on a mission\dash Carzain betrays his \Redcean{} companions and kills at least one Redcor, then flees to Geica with Shereid. 

%The Redcor are bitches. Carzain puts up with it for a while, but eventually, their arrogance has bred such hatred in him that he turns on them. So, while on a mission, he betrays his companions and kills them, including one or more Redcor. 

%that he willingly turns to evil. In his experience, those that call themselves \squo{good} are no better. So, while on a mission, he betrays his companions and kills them, including one or more Redcor. 

%He goes to Geica and learns to channel \itzach like the Geican mages do. When he returns, he is far more powerful than before. 















\section{Carzain and his Dreams}









\subsection{Carzain dreams of \ophidian{} eyes}
Carzain dreams of \ophidian{} eyes. When he wakes he wonders who the \ophidians{} are and how he knows of them. How does he know the name? How did it come to him?

Maybe move this to the next book. 

\lyricsbs{Bal-Sagoth}{
  A Black Moon Broods Over Lemuria
}{
  As a black Moon broods over Lemuria,\\
  ebon witchfire enshrouds the gleaming citadels.\\
  sinistrous shadows rise from the vaults of the dreaming elder gods.\\
  \Ophidian{} eyes glimmer through the icy whispering Moon-mist\ldots{}
}

\lyricsbs{Bal-Sagoth}{
  Enthroned in the Temple of the Serpent Kings
}{
  Storm-borne bride of winter's fire,\\
  serpent-witch of the whispering fens.\\
  Veils of scarlet and sable,\\
  blood spilled in the vault of night,\\
  Frost-garlanded, the mind-binding \\
  glimmer of tear-filled ophidian eyes.\\
  The gleam of winter Moonlight upon black waters.\\
  Nighted spells of the enchantress.
}

\lyricsbs{Bal-Sagoth}{
  Shadows 'Neath the Black Pyramid
}{
  I hearken to the grisly murmur of nameless fiends,\\
  black jaws drooling blasphemy.\\
  Beyond the witch-song, darkly sweet,\\
  The wyrm-horn sounds cross Dagon's mere.\\
  Shadow-gate (portal to the Black Pyramid) yawns wide, beckoning\ldots{}\\
  Spells scrawled in blood and frosty rime.\\
  Squamous god encoils the onyx shrine.\\
  (By the bleeding stone) I am enraptured by \ophidian{} eyes.
}









\subsection{Ramiel's \Kenosis}
Ramiel has undergone the \hr{Kenosis}{\Kenosis} and become a Scion. 
But the \Kenosis{} is imperfect and has split his personality and memories into Carzain and Vizicar. 

Vizicar tells Carzain that before they can hope for \hr{Apotheosis}{\Apotheosis} and regain their true power, they must first master and perfect the \Kenosis. 
They must become one. 









\subsection{Dreaming of \Cuezcan{} Spires}
\target{Dreaming of Cuezcan Spires}
Carzain/Ramiel has amnesia and does not remember his past as a \resphan{} lord. But he remembers scattered pieces and fragments. His memory loss pains him, and he makes it his great quest to discover who he is. 

Carzain falls sick. During this time, he has fevered dreams and visions where hidden presences whisper to him about his purpose, his quest, his destiny of greatness. 

He sees visions of terrible creatures, macabre gods and ancient, ruined cities. 

Later, we have another scene where Carzain/Ramiel broods. He searches his mind for clues about his past and identity. He takes poisonous drugs and uses dangerous magic and meditation to take himself into trance.

\lyricsbs{Bal-Sagoth}{A Black Moon Broods Over Lemuria}{
  Slumbering upon the throne of Moon-caressed ice\\
  I have supped deep the draught of white vapours.\\
  Shimmering upon the gleaming garlanded marble,\\
  a single strand of glimmering gossamer.\\
  In brooding and sombre visions I hear cries. \\
  Enthralling cries 'neath this frost-Moon rising.\\
  The whisperer in crystals speaks in dreams,\\
  of silken shadows and the softest breath of dark enchantment,\\
  of ancient cyclopean temples, raising jewelled spires to the stars.\\
  There is witchcraft in the moon,\\
  and a brooding silence reigns over the woods.\\
  My storm-forged sword,\\
  ensorcelled by eon-veiled incantations.\\
  Dark wizards' spells entwine me in ravening shackles,\\
  and black roses draw my blood with thorns as sharp as serpent's tooth. \\
  I fall into the rapturous embrace of sloe-eyed witches, \\
  the Moon gleaming upon their ivory bosoms,\\
  and descend into the still, icy waters of the lakes.\\
  Beyond the veil of the north-winds\\
  I await the emissaries of the tyrant.\\
  The wind whispering across the everlasting snows.\\
  My slumber is light as a wolf's\ldots{}
}






\subsubsection{Ramiel's dark secrets}
Ramiel has dark secrets and dark dreams. See section \ref{Ramiel dreams}.





\subsubsection{He dreams of \Nyx}
In his dark dreams he sees \Nyx. 

\lyricslimbonicart{Deathtrip to a Mirage Asylum}{
  To the abyss' morbid enigmas.
  
  Far beyond the earthly grave,\\
  the soul betakes to drift astray.\\
  Infinite horizons of pale grey skies.\\
  A still born heart within there lies.
}

He somehow understands and remembers that even though \Nyx{} is scary and horrid, it was once his home, and thus remains his home. 

\lyricslimbonicart{Deathtrip to a Mirage Asylum}{
  The sanctuary that once was lost\\
  is streaming endlessly in holocaust.\\
  The astral corpse is still pulsating,\\
  in a mirage asylum awaiting.
}









\subsection{Carzain gets lost in strange Realms}
Maybe have a scene where Carzain somehow stumbles into another Realm and is led astray. 

He encounters creatures that are sworn to serve his kind, but have been alone and abandoned for centuries or more. Perhaps they are incorporeal, but he imagines them in humanoid form. And because of his power over them, they assume the form the thinks up for them.

Perhaps this is where he finds \Belzir, and she finds him.







\subsection{\Belzir contacts Carzain in dreams}
Carzain is visited in his dreams by a mysterious woman. In his dreams, they have sex. The wraith-woman tempts him with promises of power, glory and revenge upon the Redcor. It turns out that the woman is \Belzir, the last \Calipha of the ancient \VaimonCaliphate and called the Dark Prophet of \itzach by some. She wants him to join her and her servitors, who are working to resurrect her. 

She also approaches Vizicar\dash maybe both at the same time, maybe one at a time. 

She urges them to join her, tempting them with promises of power, glory, sex\ldots{} and, best of all, knowledge. 
She knows he is a Scion, amnesiac and questing to master the \hr{Kenosis}{\Kenosis} and eventually the \hr{Apotheosis}{\Apotheosis}. 
She tells him that she is a Scion herself, the only Scion ever to have attained the \hr{Apotheosis}{\Apotheosis}. 
This gives Carzain and (especially) Vizicar a boner. 
Vizicar very much wants to go to her. 
He does not know of her and has not been indoctrinated to hate her (since he lived before she did), and he doesn't like the Redcor, so he roots for her. 
Carzain is a bit more \skeptical, with his heroic ideals and all. 

\Belzir{} tells him of how she was a rebel against the Vaimons and their oppressive religion. 
This impresses Vizicar. 
He wanted to do the same, but never had the will to actually do something about it. 
He admires her a bit. 

\Belzir claims that she is not evil.
She is working to create a better world. 
It is the Redcor that are the evil ones, with their repressive and bigoted religion. 
They are hypocrites.
She is the real deal. 

Carzain is stuck by the \hr{Belzir's sorrow}{immense sorrow and pain} she radiates. 
He notices the lines of tears under her eyes.

He is also attracted and enraptured by her \hr{Geican green eyes}{emerald green eyes}. 
He can feel the Geican eagle in her. 

When Carzain/Ramiel first encounters \Belzir, he is struck by her commanding, seductive being. 
It fills him with conflicting emotions: 
\begin{itemize}
  \item Sexual lust for her. 
  \item Intimidation; an inclination to submit and obey her. 
  \item A masochistic lust to let her sexually dominate him. 
  \item Shame at his own weakness and unmanliness. 
  \item Anger at her for making him feel shame. 
  \item Desire to punish her. 
  \item Lust to sexually dominate her. 
  \item Excitement at this challenge. 
  \item Respect for her strength, sexiness, seductiveness and boldness for making him feel all of the above. 
\end{itemize}

Later and gradually, he comes to realize this: 
\begin{prose}
\tho{I want to rule, to dominate her.
  Any opposite feelings I may experience are alien, unnatural things she plants in me through her manipulation and wiles, and which I have to combat with the power of my true self. 
  
  But \emph{she} has both: 
  She wants to submit to me, and she wants to enslave me.
  Perhaps both at the same time. 
  
  The last part is a scary thought. 
  Underneath the surface, who is really dominating whom? 
  If one of us pulls the other by a chain, who truly leads and who follows? 
  I am simple, but she is complex. 
  Damnably complex. 
  And that makes her even more dangerous. 
  I must not let down my guard with her.}
\end{prose}





\subsubsection{Chasing \Belzir}
\Belzir does not give herself to Carzain immediately.
She kisses him and lets him grope her, but does not let him have sex with her. 
She flees. 
He has to chase her. 

Compare to Atali from \cite{RobertEHoward:TheFrostGiantsDaughter}, who seduces Conan and flees. 

Only late in the story does he catch her. 
In this dream, he sprouts great wings from pure willpower. 
This is a great victory for him and a significant step towards Apotheosis.
With the wings, he easily catches her.
They have sex for the first time in this life. 
It is passionate and wonderful. 





\subsubsection{Sex with \Belzir}
Remember that \Belzir{} is herself a Scion, and she knows a lot about the \Malachim. She uses this knowledge to drop hints to Carzain/Vizicar and lead them on.

The sex scenes with Carzain and \Belzir{} should be Bal-Sagoth-esque: 

\lyricsbs{Bal-Sagoth}{
  Enthroned in the Temple of the Serpent Kings
}{
  Ruby-lipped and midnight-tressed, \\
  eyes as black as raven's wing.\\
  Flesh so pale as dawn-frost gleaming,\\
  nighted spells of the enchantress.
}

Or \Duana-esque (in the following, replace \quo{he} with \quo{she}):

\lyricsduana{nightsong}{Night Song}{
  at night he slips through my window \\
  cool air caressing me like a lover's moist tongue 
  
  I shiver new awakening \\
  moaning a rhapsody of vowels \\
  and consonants like crosses in a row \\
  his lips of bittersweet agony \\
  bruising tender flesh \\
  as I lay impaled by lust \\
  and hear the sound of my heart beating \\
  druming blood through my viens 
  
  his love is wild like wolf-song \\
  howling at the pregnant moon \\
  his voice is smooth as liquid velvet \\
  as he sings his soothing lullaby \\
  hush hush hush my pretty pretty one 
  
  until the dawn bleeds from darkness \\
  tangerine rays grazing my cooling flesh \\
  and he flees like the shadow of a dream \\
  leaving silence in his wake
}

\lyricsbalsagoth{The Hammer of the Emperor}{
  [She Came Bearing Dark Portents (The Foreshadowing):]\\
  Fever-dreams, dark omens and auguries. Prophecy!\\
  Why, when I meet your narcotic sloe-eyed gaze, \\
  does the image of a viper nestling in a bed of blossoms fill my mind's eye?\\
  Why, when you come to me by the pale light of a waning moon, \\
  do I glimpse the sheen of ophidian scales through the veils of sable?\\
  Why, when you enrapture me with your envenomed kisses, \\
  does the flicker of a serpent's tongue score my flesh?\\
  Enthralled by the vitreous lustre of your rubicund lips, \\
  your snow-pale skin musky with the intoxicating scent of night\ldots{} \\
  but such wicket thorns beneath this rose.\\
  Come witch, fly to me!\\
}

Gradually through the book, his sex with \Belzir{} helps him remember fragments of his past life. 

\lyricslimbonicart{Symphony in Moonlight and Nightmares}{
  This night belongs to us,\\
  my dear princess of death.\\
  We have a destiny together\\
  as we meet in hungry caress.
  
  A deadly kiss under moonlit sky,\\
  as I stare into thy dark and hollow eyes.\\
  You take me down where cold silence dwells.\\
  Unconscious darkness realms of demise.
  
  You are born at my grace to serve only me.
  
  Blood paint on the wall is the ancient dream I recall.\\
  Predictions in rainstorm as tears dark red cascaded.\\
  In cryptic depths of imprisoned rage where I succumbed to temptation,\\
  in the laughter of cruel gods, demonic wrath and devastation.
  
  A monstrous enemy demands to be set free.\\
  Frigid evil games. Black art and blasphemy.\\
  Prophecies in blood.\\
  A deadly kiss unto darkened bliss.
  
  We have a gift of shining\\
  by knowing the history\\
  behind obscure mysteries.
  
  The mysterious source of true art and experience.
}





\subsubsection{From \ps{\Belzir}{} perspective}
We see one of the scenes from \ps{\Belzir}{} point of view. 

\lyricsbs{Emperor}{The Ancient Queen}{
  Dark rivers run though me.\\
  Darkness follows everwhere. \\
  Kingdoms falling again and again.
}

In a dreaming state, she flies to him. 

\lyricsduana{thedreaming}{The Dreaming}{
  Midnight wakes beneath a mystic moon \\
  as ominous mists waltz dewy down \\
  the nocturnal dance of mothwings beating \\
  soft staccatos and whispered wind \\
  the spirit of a shape transforming \\
  like a ghost of velvet shadows \\
  through a lattice silently creeping \\
  into a window open to the night \\
  steals between the gauzy drapes \\
  to find the sleeper \\
  lost in slumber \\
  dreaming, dreaming, dreaming. 
  
  She watches as his soul lies hid \\
  slack-jawed and lashes fluttering \\
  lost in the nebulous expanse of dreams \\
  the rise and fall of his shallow breathing \\
  she sighs sweet syllables in his ear \\
  from bloody lips smiling deeply \\
  whispers caress his naked skin \\
  moonpale gooseflesh shivering \\
  his mouth opens to her poisoned kiss \\
  his breath captured as he is waking \\
  eyes open wide to the terrible sight \\
  dreams to night terrors transforming \\
  she drinks his spirit red like wine \\
  as his voice is \\
  screaming, screaming, screaming.
}





\subsubsection{Geican principles}
Carzain believes it's a Geican thing to accept your sexuality and revel in it. 
\hr{Carzain is disappointed in Geica}{He is disappointed} when he gets to Geica\ldots{}





\subsubsection{Shereid}
Carzain hangs out with \hr{Shereid}{Shereid}, a Vaimon woman of \ClanGeican. She is secretly a follower of \Belzir. She and Carzain end up having a sexual relationship.  





\subsubsection{He is scared when he learns \ps{\Belzir}{} identity}
When Carzain first learns \ps{\Belzir}{} true identity he is scared. 
But she convinces him that she is really good, and merely the victim of evil propaganda and lies. 
At this point Carzain doesn't like the Redcor, so he is willing to believe much. 





\subsubsection{\Belzir{} and Vizicar exchange memories}
\Belzir{} and Vizicar exchange memories of life, death and afterlife as \malachim. 

\lyricsbs{Emperor}{The Tongue of Fire}{
  the soul is never silent\\
  but wordless, held imprisoned in a cursed tomb\\
  wherein reflections never fade, never die\\
  slowly maddened by the emptiness
  
  left to perish in the ever-dark coil\\
  yet, always alert it its slumber\\
  scorn by the drops of light\\
  piercing through the surface\\
  and it screams
}

They both long to regain their true and rightful power.

\lyricsbs{Emperor}{The Tongue of Fire}{
  teach me the tongue of fire\\
  so that I may set the world ablaze\\
  for it is cold\\
  and this blindness can no longer give me shelter\\
  teach me the tongue of fire\\
  so that I may cry out loud my wrath\\
  and my passion\\
  or else my coil will blister and decay
}









\subsection{Carzain wonders what \Belzir is planning}
A theme in the book is that Carzain is always trying to puzzle out what \Belzir is planning. 
Ostensibly this is because he wants to stop her.
In reality it is because he is in league with her and wants to know how her plans affect him, so he can decide how far he wants to follow along and how soon he wants to betray her.
(Carzain first betrays the Redcor, then \Belzir.)









\subsection{Carzain has evil daydreams}
Carzain's nightmares escalate. He begins to suffer from daymares as well: Daydreams that pop up every now and then, where he sees into the Beyond or into his own imagination. He sees people's true faces, or they are horribly distorted. He sees monsters from the Beyond, or the ghosts that are bound to him, or spectres from his imagination. 

A horrific theme is the blurring of the border between reality and fiction. Or rather, Carzain gradually realizes how blurred the border has always been. Some of the visions he sees in his daydreams are lies, his half-mad subconscious mind twisting the Shroud further to fit its own fears. But some of the visions he sees are truths, with his terrified mind momentarily breaking through the Shroud and gazing into Reality. 

See, when a mind is under great stress, afflicted with powerful, stressful emotions, it may be able to draw upon reserves of primal strength that are otherwise latent and sealed away from the dumb, Shrouded mind. This may sometimes enable the person to penetrate the Shroud. Although sometimes, the stressed mind may be in an exceptionally vulnerable state, and the experience of seeing into true Reality might be extra traumatic. 

In Carzain's case, he is not immediately traumatized. Vizicar helps a lot here, since he has had to live with this for a longer time and knows more of how to deal with it. But Carzain begins to doubt both the reality around him, and his own sanity. And Vizicar's. Because we more and more learn how demented Vizicar is. 

\lyricsbs{Arcane Wisdom}{Symphonia Chaos}{
  Ghastly ghostly presences.  \\
  Insanity takes its place, \\
  while an effigy fades astray \\
  and memories still withstand. 
}









\subsection{Carzain is repulsed, but also attracted}
At first, when Carzain learns of the horrors that fill the universe, he is repulsed. He is swayed by the Redcor's talk of the Light and all that is good, and almost renounces the \quo{black} Geican magic with which he is brought up. 

Vizicar objects to Carzain's idealism, and Carzain lashes back by repressing Vizicar and shutting him out. Compare to Rand in \emph{Wheel of Time} or Gollum in \emph{Lord of the Rings}, who represses his evil side. 

But the longer he stays among the Redcor, the more he is disgusted by them, their hypocrisy and arrogance. When his faith in the Light falters, Vizicar creeps back and whispers in his ear. 

Gradually through the book, Carzain becomes disillusioned and bitter, and more and more fascinated by the dark powers of \nieur{} and other forbidden things. Having to fight against the Redcor's moral principles, he develops a distaste for all morals and comes to enjoy breaking rules and reveling in his power, obeying only his own will and whim. 

This is gradual, tho. Whenever Carzain discovers a new power of his or Vizicar's, he is repulsed and horrified at first, and only later comes to like it. These powers become guilty pleasures to him: They are useful, and he likes the intoxicating rush they give, but he also tries to deny it\dash on the outside, to keep it secret, but also on the inside, because he wants to maintain his own illusion of himself as a principled and righteous hero. 

At the end of this book, of course, he completely abandons and betrays the Redcor and goes to side with Geica. 









\subsection{\Belzir{} awakens}
\Belzir{} slowly awakens. 
She is still groggy and half-delirious from her long slumber\ldots{} and only halfway sane. 
The first times she came to Carzain and had sex with him, she was in a hazy dream, too. 
She doesn't quite understand what is going on herself. 

In dreams, she teaches Carzain secrets. 
She speaks in riddles, because she thinks in riddles. 
It's all still dreamlike and cryptic to her. 
Besides, she's half mad from her millennia in limbo. 

\lyricslimbonicart{Deathtrip to a Mirage Asylum}{
  In limbonic life I've slept,\\
  dreamt that cryptic seals me kept\\
  sheltered from all visual sight\\
  'cause shadows chased every sign of light.
  
  Awakened from an ancient slumber,\\
  recalled to act of strife.\\
  My spirit old and body cold\\
  remain in dark demise.\\
  Immortality beloved am I to dying,\\
  and the whispering of secrets to my soul.\\
  I am born to sing this sorrow.\\
  Evil has no rest in me.
}

She still suffers and has horrible dreams from her time in limbo. 

\lyricslimbonicart{Deathtrip to a Mirage Asylum}{
  The invisible addiction of darkness, emptiness.\\
  And the ghastly silence devours.
  
  Deathtrip to a mirage asylum.
}

\lyricslimbonicart{The Dark Paranormal Calling}{
  The black funeral uniform gives me powers. \\
  I'm catching without warning the dark paranormal calling.\\
  A putrefied shadow that in purgatory dwells,\\
  imprisoned by the cold anxious pits of Hell.
}

She is a \hr{Disembodied souls}{disembodied soul} and hence weak. 
She wants a body back. 
To do this she has to break some spells. 









\subsection{Something inside Carzain mourns his loss}
Something inside Carzain/Vizicar mourns his loss. Perhaps this is a servant or servants (\humans? \resphain?) who willingly let themselves be killed and have their souls bound to him, to serve him as ghosts. But he lost his ability to hear them.

\lyricsbalsagoth{%
  In the Raven-Haunted Forests of Darkenhold, Where Shadows Reign and the Hues of Sunlight Never Dance
}{
  Can you not remember? \\
  Have you forgotten the magic?\\
  Sing to us your spells once more, \\
  and the ancient forest shall dance to your words\ldots{}
}









\subsection{Carzain dreams of his destiny of greatness}
Carzain dreams of the greatness for which he believes himself destined, namely as Overlord of \Mystraacht{} (although he doesn't understand that yet). 
These dreams are more optimistic, but still dark. 

\lyricsdimmuborgir{Reptile}{
  Black uneartly void. Creatures crawling.\\
  Forbidden, forgotten, fairly underrated.\\
  Bastards in the shape of angels holding my hands,\\
  passing me what is left of the wine.
}

\lyricsbs{Marduk}{Infernal Eternal}{
  As I looked into the mirror, \\
  and saw the creation which was fading, \\
  I sailed the darkened waters of my soul \\
  on the ship of flaming hate\\
  towards the land of the damned.
  
  The cold winds of the darkness blow strongly.\\
  The cemetery glows in the dark.\\
  A thousand times, thousand voices \\
  are screaming in pain from beyond.
  
  A number of faceless shapes \\
  march forth from the darkness within.\\
  Life is slowly passing away, poisoned by guilt and sin.
  
  Embraced in Black I lie,\\
  only waiting to die. 
  
  The greatest of life's events,\\
  and I begin my journey\\
  as hands of greater characters unveil the world.\\
  Plunging through space and time, my prison has now been slain.
  
  The reaper purifies my soul.
  
  Perdition and death. \\
  I was in ages so dark and far away,\\
  and I will always be.
}

\lyricsbs{Emperor}{Cosmic Keys to my Creations and Times}{
  All these landscapes are timeless, \\
  and this is all just a part of cosmos, \\
  (but) all is mine and past and future is yet to discover\ldots{} \\
  Much have been discovered, but tomorrow \\
  I will realise I existed before myself.
  
  I will realise planets ages old, \\
  created by a ruler with a crown of dragon claws, \\
  arrived with a stargate\ldots{} \\
  a king among the wolves in the night\ldots{} \\
  An observer of the stars.
}









\subsection{Dreams of his father}
Carzain/Vizicar dreams of his true father, \hr{Zachirah}{\Zachirah}, who urges him on: 
\ta{\hr{Ramiel can do better}{You can do better!}} 

When he wakens, Carzain talks to Vizicar. 
They discuss who this \quo{father} figure might be.  









\subsection{SM sex with \Belzir}
In dreams, \Belzir/\Shiaraid{} plays her sex-games with Carzain/Vizicar/Ramiel. 
She is a \hr{Shiaraid's sexuality}{sado-masochist}. 

She hopes to achieve one of two things: 
Either coerce Ramiel into submission or provoke him so much that his rage and pride will awaken some of his inner power and let him dominate her. 
Either way, she wins. 

Ramiel has his pride and refuses to submit, because he \hr{Ramiel is nothing}{fears being nothing}. 
He fights with all he has in him. 

She presses him to his limit. 
Her attempts at domination are not merely physical, but also psychological and social. 
She convinces him that he is indebted to her and that he has to serve her and submit to her if he is to keep in her good graces, so she will continue to graciously help him. 
She hints that if he is disobedient or incompetent or otherwise displeases her, she will punish him and make his life miserable (not just physically, but socially). 

At last he explodes in a mighty roar. 
The Shroud is broken, and for a moment she sees him in his true \resphan{} form, black-skinned and winged. 
But only for a moment. 
His \hr{Kenosis}{\Kenosis} once again takes over, and he lapses back into \human{} form. 
But some memories remain. 
She has brought him a great leap closer to \hr{Apotheosis}{\Apotheosis}. 
He has \trope{TookALevelInBadass}{Gained a Level in Badass}. 

Up until this point, \Belzir{} has been quite arrogant and always teased and baited Carzain. 
But now he is Ramiel, and just as strong as she. 
She is exhilarated. 

\begin{prose}
  \Belzir: 
  \ta{You have long said that you would make me pay for my insolence.
    Come and take your revenge then, if you have the strength and the courage for it. 
    Punish me\ldots{} my lord Ramiel!}
\end{prose}

He does. (In dreams.)

Compare with the sex scenes in \authorseries{Jacqueline Carey}{Kushiel's Legacy} with \Phedre{} and Melisande, or with \Phedre{} and Severio Stregazza. 

\lyricsauthorbookpage{Jacqueline Carey}{Kushiel's Chosen}{100}{
  [Severio Stregazza:] \\
  \quo{You\ldots{} will\ldots{} acknowledge\ldots{} my\ldots{} sovereignty!}
}





\subsubsection{\Belzir{} thinks}
Afterwards, Belzir thinks to herself: 

\begin{prose}
  \Belzir:
  \tho{He is good. And evil. I can use him. 
    Yes, he and I will go on to a fruitful and delightful partnership. 
    In bed, and out of it.
    And as for the \quo{acknowledge my sovereignty}, well\ldots{} we will see about that.}
\end{prose}

But later, she has doubts. 
She does not remember everything of her former lives, but she knows there is something \quo{wrong} with her. 
She is \hr{Self-destructive Shiaraid}{self-destructive}. 
It worries her. 

She cannot recall the details, but she remembers one name: 
\Nexagglachel. 
He has cursed her somehow. 
She knows her lusts are dangerous to her. 
\tho{It is madness. But it feels so sweet\ldots{}}









\subsection{Ramiel meets \Cishiel in dreams}
\target{Ramiel meets Cishiel in dreams}
\Cishiel \hr{Azraid tells Cishiel about Ramiel}{has heard from \Azraid} that Ramiel is incarnated. 
She seeks him out.
At last she finds him and contacts him in dreams. 

Carzain is surprised that there are now \emph{two} mystical, beautiful women hounding him in his dreams. 

At first, \Cishiel speaks in riddles. 
She wants to try to lure out of Carzain how much he knows, before she tells him something she may regret. 
Also, she hopes to manipulate him. 
She has worked with manufacturing religions in the past, so she is used to manipulating mortals in this way.
Carzain does not buy it.
He tells her:

\begin{prose}
  Carzain:
  \ta{Speak plainly, woman.
    I don't know if you are a ghost or god, but I am no servant of yours.
    My time is too precious to waste on interpreting your riddles.
    If it is not worth your time to express your message properly, then I can only assume it is not worth my time to learn it.}
\end{prose}










\subsection{Vizicar remembers the \angels}
Vizicar remembers the time of the \VaimonCaliphate, when everyone thought that the \Sephiroth{} and the \angels{} were good and all-powerful, and that evil in the world would soon be defeated. See section \ref{Vaimon Caliphate naivete}.

Later, he is very much shocked to learn that the \resphain{} are evil.







\subsection{Carzain encounters \vorcanths{} again}
Carzain has bad dreams, perhaps even manifesting in the real world. 
And \hr{Moon-Wolves help Ramiel in dreams}{the \vorcanths{} come to help him}. 

The \vorcanths{} actually need his help. 
He owes them favours for their having helped him in the past (distant and \hr{Vorcanth help Ramiel}{near} past alike), so they are \emph{demanding} that he come and help them. 









\subsection{\Belzir{} tempts Carzain}
\Belzir{} tempts Carzain to delve deeper into the darkness. 
This is after he has become embittered with the Redcor. 

\lyricsbalsagoth{Return to the Praesidium of Ys}{
  Return with me, beyond the stars.\\
  Rule with me a thousand worlds.
}

\lyricslimbonicart{Path of Ice}{
  Journey into your darkened secrets.\\
  Feel the burning flame inside.\\
  Admit the ecstacy of the extreme.\\
  Just close the eyes and enjoy the override.\\
  When we adrift through the sensual streams\\
  the enchanted pains are so divine.
  
  There are thorns everywhere,\\
  but along the path of ice rose bloom above.\\
  Blood is the rose of mysterious unions,\\
  the symbol of potency.\\
  A taste of erotic sins of lust.\\
  The entrance of immortality.
  
  There are such sights to see,\\
  Adventures and pleasures to feel.
}

\lyricsdimmuborgir{The Maelstrom Mephisto}{
  Dwell in depths of the darker self, at any shore of infinity,\\
  and watch the relentless paint the soil black.\\
  What is being formed echoes throughout eternity,\\
  as the painter chooses \colour no more.
}

She teaches him some magic. 

\lyricslimbonicart{Path of Ice}{
  The dormant seeds of suffering.\\
  The art of mortal flesh that bleed.\\
  Indulged in desire, the forbidden soul empire.
  
  Path of ice.\\
  The entrance to immortality.
}

He sees glimpses of his past as a mighty \resphan{} warlord, and glimpses of a possible future as a great sorcerer-king with \Belzir{} by his side. 

This also foreshadows the fact that he \emph{is} on a path to greatness. The book ends with Carzain contemplating his path from now on. He is becoming a character like Rio from \JuukenSentaiGekiranger, obsessed with the pursuit of power and greatness.

\lyricslimbonicart{Path of Ice}{
  Cold winds pierce through me\\
  as my dark star unfolds.\\
  I ascend the throne of fantasies\\
  where the beautiful abyss recalls.
}

She tells of how the \quo{pious} Vaimons betrayed her, even though she fought for freedom and enlightenment. 
Therefore she wants him to reject the Redcor, their religion and all conventional morality. 
(The Sun is a symbol of the Redcor and the Iquinian church.) 

\lyricsbs{Monolith Deathcult}{Den Ensomme Nordens Dronning}{
  Curse the Sun and all that's holy. \\
  Turn back. I dwell around. \\
  Come to me as a shadow. \\
  Engrave the Sun and the Blackest Moon shadow. 
}





\subsubsection{She tells him of the \Feud}
\Belzir{} tells Carzain a few secrets about the \feud{} and the master races. 

\Belzir: 
\ta{Mages are so obviously superior to the common people. 
Why, then, do they not rule the world? 
Why have there been no great sorcerer-kings since me, do you think? 

I will tell you: 
\emph{They} fear ones such as us. 
Whenever a mage starts climbing to power, they stab him in the back and take him down. 
}

Carzain: \ta{\quo{They}?}

\Belzir: (Dodges the question.) 





\subsubsection{\Belzir tells Carzain about Scions}
\Belzir tells Carzain that they are both \quo{True Spirits}, immortal angels trapped in \human form.
She believes they are fragments of the sundered soul of the Creator \Dragon, the cosmic spirit of death and life.
This is a twisted myth, a vague memory of how the \satharioth derive their soul from \Nexagglachel, combined with something about Scions. 
(\Belzir knows that she and Carzain are both \satharioth and \malachim, but she does not understand the difference between the two concepts.)

\Belzir thus argues that she and he are worth more than ordinary humanoids, who do not have a True Spirit. 
Carzain has to agree that she has a point.

Compare to the story of Simon and Helen from \cite{RichardTierney:ScrollofThoth}.






\subsubsection{Carzain sees \Belzir{} as a kindred spirit}
Carzain instinctively recognizes that \Belzir{} is a kindred spirit of sorts. 
This is vague memories from their time as \resphain, where they were both \Mystraacht, close friends and sometime lovers. 

He lets her seduce him and allies with her. 

\lyricsdimmuborgir{United in Unhallowed Grace}{
  We embrace the madness gathered as one.\\
  Mourning dead passion\ldots{} she comes to me.\\
  A fate awaits us in the night.\\
  In the ruins of creation we will unite.
  
  I am smitten by forbidden fruit.\\
  Possessed by moments of dark splendour.\\
  To walk the nightmare terrains forever.\\
  The enigma lies broken.
}

\lyricsbs{Emperor}{The Ancient Queen}{
  Crawl to thee over again. \\
  The Ancient Queen, the darkest woman.\\
  Stepping through her shadow. \\
  She who sees the soul. \\
  Black is her blood. \\
  Keeper of the fury. 
  
  Stepping through her shadow. \\
  Do you see her soul?\\
  Black is her blood. \\
  Keeper of the fury. 
  
  Dark under the shadow. \\
  Thy destiny lies with her. \\
  The Ancient Queen, \\
  ruler of the domain. \\
  Keeper of the fury. \\
  In shadow\ldots{} shadow.
}

It also brings back memories of the time when \hr{Zachirah seduces Ramiel}{\Zachirah{} tempted Ramiel into joining \Mystraacht}. 
This is what \Belzir{} is planning. 
She remembers their time in \Mystraacht{} together (if not perfectly then at least partially), and she knows how powerful and effective those memories are. 















\section{Immortals and the \Feud}









\subsection[Vizsherioch and Psyrex]{\Vizsherioch and \Psyrex}
Have a scene with \Vizsherioch and \Psyrex, or perhaps \Vizsherioch and \Secherdamon. 
\Vizsherioch is introduced as a sinister, menacing figure whom even \Psyrex fears. 
This should be a prologue to the book. 


The summoning of \Nithdornazsh{} is part of \ps{\Secherdamon} plan to bring \maybehr{Vizsherioch}{\Vizsherioch} into Ascendancy. 
\Nithdornazsh{} is to become \ps{\Vizsherioch} citadel, a \nexus{} from which he can grow strong and spread his tendrils (politically and metaphysically) into the Realm of the Shroud. 
This is a vital step in the forging of the \maybehs{Dagger}. 
When the \Nithdornazsh{} project is complete, \Vizsherioch{} is more Dagger-y than ever. 

Previously, \Secherdamon{} had kept \Vizsherioch{} sequestered and hidden. 
He is his only son and the fruit of thousands of years of hard work, so \Secherdamon{} is very protective and does not want to lose him. 

\Vizsherioch approaches \LocarPsyrex.
\Vizsherioch appears as a \dax in his prime, with pearly white scales, wearing a loose robe of white, silver and gold. 

\Psyrex fears him. 
Where \Secherdamon is fiery bright, his son \Vizsherioch is dark and sinister. 
Not in \colour, but in feel. 
A vast darkness follows behind him and around him. 

\Psyrex fears to look into his eyes. 
\ps{\Secherdamon} eyes are terrible enough, but \Psyrex is used to them. 
There is passion, fervour and desire in the eyes of \Secherdamon, and anger and hate, too. 
But in \ps{\Vizsherioch} eyes there are hints of otherworldly madness. 

\Vizsherioch asks \Psyrex about his progress. 
\Psyrex tells him that there have been some setbacks. 
The Cabalists are sneaky.
He does not know who leads them, now that Charcoal is gone from the city. 
But whoever is in charge must be someone capable. 
They have managed to screw up some of his operations and kill some of his important Sentinels. 
But it is not so bad. 
He has planned for some amount of Cabal interference and taken precautions. 
He is importing more manpower. 
He will be ready on time. 

\Vizsherioch: 
\ta{The beacons are in place? Show me.}

\Psyrex shows him. 
\Vizsherioch sees the slender aethereal tendrils, grown from the Pyre \matrix.
They reach up from \Nithdornazsh to twist around the ley lines and converge upon the \nexus point in Pelidor, where they grab on and hold fast, holding the \nexus in a constricting iron grip. 
He sees the \matrix subtly reaching out to fasten upon the souls of the mortals in the city. 
Binding them.
They will be part of the invocation, he thinks. 

\Vizsherioch: 
\ta{Aye, I see it.
  You have done well, \Psyrex.}
He smiles to himself.
\ta{My father has confined me to our Realms for too long.
  I long to at last set foot in my new citadel.
  And to exert my power in \Azmith; supposedly the most pivotal of the Shrouded Realms.}

\Vizsherioch becomes distant.
\ta{The constellations are falling into place.
  I can feel the tension in the Pyre. 
  The Dagger is taking shape.
  Very soon now...}
He becomes present again.
\ta{%
  What of the \resphain?}

\Psyrex:
\ta{I have detected some \resphan activity in both \Malcur and \Forklin.
  I still have every reason to believe they will swallow the bait.
  But of course, it will depend on \Nzessuacrith and her task.}

\Vizsherioch:
\ta{She will not fail us.}

\Psyrex:
\ta{Yes.
  It is not \Nzessuacrith nor the \resphain that make me uneasy.}
\Psyrex pauses and hesitates. 

\Vizsherioch: 
\ta{You mean the Exile.}

\Psyrex:
\ta{Yes. He has been uncharacteristically... \emph{active} recently.
  In the Pelidor region.}

\Vizsherioch:
\ta{But he has not antagonized us?}

\Psyrex:
\ta{Not that I can determine. 
  And that worries me.
  So far he seems to have acted only against the \resphain, but I cannot guess his motives.
  The Exile cannot be trusted.}
\Psyrex looks at his star-charts. 
Concentrates to shift his vision, so he can bypass the roof and see the sky. 
He looks up at the real stars.
He speaks some spells of divination to read them.  
He stares at the star representing the Exile. 
It is nowhere near the Pyre, nor any other known \matrix. 

Metaphysically, that is.
Physically, the Exile is much too close. 
He is made uneasy by the thought of the rogue \vertex. 
\Ishnaruchaefir is a mystery. 
A wanderer in darkness who can appear anywhere at any time, and from whom no one is safe. 

\Vizsherioch:
\ta{\QuessanthIshnaruchaefir.}
\Psyrex is taken aback. 
He had never heard \Vizsherioch speak the Exile's name before. 
He thought \Vizsherioch shunned the name like his father did.

\Vizsherioch: 
\ta{Called Exile and Destroyer.}
To \Psyrex: 
\ta{You fear him.}

\Psyrex:
\ta{Yes, I fear him. 
  Any creature lesser than a \dragon has cause to fear the Exile.}
\tho{And many a \dragon should fear him, too.}

\Vizsherioch:
\ta{Hm.
  He is in the Pelidor region.
  So it is possible he has caught wind of our doings.
  He must not be allowed to interfere.}

\Psyrex:
\ta{The \resphain feel likewise.
  Have you heard, Lord \Vizsherioch, about this \resphan, \Teshrial, who talks about confronting the Exile?}

\Vizsherioch:
\ta{Yes. 
  Allegedly the Exile has promised to face him.}

\Psyrex:
\ta{I wonder what will come of this challenge. 
  Will the Exile really return to meet the \resphan?
  I do not think \Teshrial is a fool.
  He has a plan.
  But will it be enough?}

\Vizsherioch:
\ta{Interesting prospect, this duel.}
He becomes distant.
\ta{Perhaps I should seek out \Ishnaruchaefir.
  After all, I have never met my uncle face to face...}

\Psyrex tries to imagine such a confrontation.
Would it end in violence?
\Vizsherioch was powerful, but young. 
Would he be able to stand before \Ishnaruchaefir?

\Vizsherioch reads his thoughts.
\ta{Fear not, \Psyrex.
  I will not challenge the Exile to single combat.}
Distant.
\ta{No, that would be wise.
  Not at this time.
  Not at this time...}









\subsection{\Azraid{} protects Ramiel}
\target{Azraid protects Carzain}
\target{Azraid protects Ramiel}
Ever since \hr{Azraid learns of spike}{\Azraid{} learned the details of the \vertexspike{}}, he has taken on the role of Ramiel's \quo{guardian angel}, subtly pulling threads and protecting him from harm while he grows towards his \apotheosis. 

This is necessary. 
Many forces conspire to kill Ramiel's Scion.

\Azraid muses about this. 
He knows that great things are happening these days. 
He hopes Ramiel can finally break free of his prison as a Scion. 
He hopes to unravel the mystery of the Scions. 

Have many references to the mystery of the Scions. 
No one quite knows how they work and why they exist and what their purpose is\dash including the Scions themselves.







\subsection{Cabalists hold out in the Ghost Tower}
The \hs{Ghost Tower} is the Cabalists' bastion in Pelidor after they lost \Malcur. They use it as a staging area to mount offensives. Due to \hr{Charcoal at the Ghost Tower}{Charcoal's clever spell}, later improved and strengthened by \Achsah, the Tower is easily defensible. 

\Achsah{} and Charcoal hold out in the Tower, command their Cabalist minions, and maybe \hr{Achsah rewards Charcoal}{have sex}.





\subsubsection{\Achsah{} is inherited by a new mistress}
After \ps{\Teshrial} death, \ps{\Achsah} allegiance passes to another \resvil, a \Mystraacht{}, to whom \Teshrial{} was indebted. 

\Mystraacht: 
\ta{%
  You belong to me now.}

The \resphan{} tongue has many words for \quo{belong to} and \quo{have/possess/own}. 
She used one of the more respectful ones, but her tone and body language practically scream: 
\hypota{You are my slave.} 

\Achsah: 
\tho{%
  These \Mystraacht{} have no etiquette. 
  I miss \Teshrial{} already.}

Even so, it's not so bad, because the \Mystraacht{} are less snobbish and don't look down on her so much for her blood. 
They respect her more for her strength and skill. 









\subsection{Charcoal leaves the Ghost Tower}
To begin with, Charcoal is in the Ghost Tower with \Achsah. He helps her fight off the Sentinels, and she rewards him with sex. 

But eventually \Achsah{} sends him away on some mission. 









\subsection{Cabalists oppose the \Haskelek{} fragment in \Redce}
\Redcean{} Cabalists fight against a fragment of the \Haskelek. 
It is difficult, because they have to be covert and work indirectly under the nose of the nosy Redcor. 









\subsection[Mystraacht rival goes after Ramiel]{\Mystraacht rival goes after Ramiel}
\target{Mystraacht rival goes after Ramiel}
A \resphan{} of \Mystraacht, allied with a faction that hates Ramiel and does not want him to return, is pulling strings in \Redce. 
Trying to oppose Carzain and get him killed. 

This guy is approached by \Azraid, who asks him to stop. 
\Azraid{} \hr{Azraid protects Carzain}{has his own plans for Ramiel} and does not want this guy to interfere. 

\Azraid{} is very polite, but in a sinister way, with an undertone of threat. 

\begin{prose}
  \Azraid:
  \ta{%
    You are not under my command, of course, so I cannot force you to do anything.
    But I can \emph{request} that you desist and cease interfering with Ramiel.
    I hope you can be persuaded to acquiesce to my request. 
    Otherwise I should be\ldots{} saddened. 
    
    And by desist, I mean desist completely.
    Do not try to pull any strings.
    Do not interfere with him at all.
    
    Please consider my proposal.}
\end{prose}

\target{Azraid tells Cishiel about Ramiel}
After this, \Azraid{} figures that his veiled threats are probably not quite enough. 
So he contacts \Cishiel. 
Tells her about the affair. 
She is very interested. 

He sits back and is happy.
He knows \Cishiel{} will help keep Ramiel safe from meddling \Mystraacht{} busibodies. 
\Azraid{} was not confident he could keep them all away. 
But \Cishiel{} is very skilled. 
She has had to become sly and cunning just to survive. 
She has quite some insight into \Mystraacht, and she knows very well who Ramiel's enemies are. 
The two can work together and supply each other with information, which should be enough to keep Ramiel alive and guide him to a path where he can regain his memories and power. 

Alternatively, this might happen earlier.
See the section about the question of \hr{When does Ramiel meet Cishiel?}{when Ramiel meets Cishiel}.









\subsection{\Vizsherioch{} and \Secherdamon}
Have scenes with \Vizsherioch{} and \Secherdamon{} at home, and show their relationship. 
Perhaps \Vizsherioch{} recommends that his father send him out, after his other minions have failed. This is the first time we see \Vizsherioch. 








\subsection[\Dasteron cleans up Mystraacht]{\Dasteron cleans up \Mystraacht}
\Dasteron does much good work to clean up \Mystraacht.
It had degenerated into a gangster-like den of petty greed and brutality. 
Barbarism.

\Dasteron has a long-term vision.
He unites \Mystraacht and restores it to \honour, glory, dignity and purpose.









\subsection{\Humans{} are special}
Some high-up Cabalist or Sentinel is talking to a \human.

\ta{The old days are gone, and so are the old master races. They may not know it, but their time is up. They had their time and now they are decline. 

Oh, they may still be strong, they may still have raw power. But they are locked in place. They are slaves to their own game, unable to break from it. 

But you \humans\ldots{} you are special. You are unique. You alone possess true freedom of will. And it is this special \human{} quality that, in the end, will decide this war. All the power of the elder races is nothing compared to the gift that is \emph{\humanity}. This is the age of men, and the future lies in \human{} hands.}

Afterwards, he walks away, thinking to himself. \tho{By the \Voidbringer, that was disgusting. I almost feel dirty for spinning such an outrageous lie. I can hardly believe that he bought it. Heh. But that is \humans{} for you, is it not? So easily manipulated, so easily tricked into believing anything you tell them.}

\tho{Hah. \quo{True freedom} indeed. \quo{The gift of \humanity}. Hah.}

\tho{Foolish worms.}









\subsection{A \Haskelek{} fragment possesses a mortal host}
A mortal \scatha{} or \human{} is possessed and used as a host body by one of the fragments of the \Haskelek. 
This might be \hs{Lica}.

The fragment is only half-awake, so the mortal carries it around for a long while. 

The victim is slowly driven mad by the presence of the \Haskelek{} inside her. 
It is alien and terrifying. 
She tries to interpret and put into words those alien thoughts, emotions, images and visions that the \haskelek{} sends into her mind. 
The only recognizable emotion that comes close is\ldots{} hate. 

In the end, the fragment breaks out of her, and her body explodes. 
Compare to the people possessed by Nyak in \authorbook{Stephen Marley}{Spirit Mirror}. 

\lyricslimbonicart{From the Shades of Hatred}{
  A thousand years time dimension \\
  in subconscious incarceration. \\
  My hatred to man, has transformed me \\
  into a habitation for demons. \\
  A devil incarnation. \\
  In the forgotten past, ages ago, \\
  beast became my alter ego.
  
  The god in me, infernal black divinity.\\
  After years of agony and pain \\
  hatred is all that remains.
}









\subsection{\Dzasselid{} explores the Beyond and gains knowledge}
\Dzasselid-tachi are on a mission. He needs more knowledge. 

Have a scene where he casts divination spells, explores the Beyond and gains some kind of enlightenment. See section \ref{The dark universe}.

\lyricslimbonicart{Dynasty of Death}{
  I'm launching into the abysmal universe.\\
  Disembodied I enter the cosmic cataclysm.
  
  I discover stairways to celestial dreamscapes\\
  A dark unknown conjunction within an immoral dimension.
}

Remember that the Rissitic \humans{} are dark-skinned. 
They should remark that \Velcadian{} \humans{} are much paler. 









\subsection{Someone finds a desecrated corpse}
Someone, investigating the strange happenings that have occurred lately in \Redce, finds a corpse that has been killed and desecrated by the underground cult. It is mutilated, perhaps with the skin torn off, and bears clear signs of sexual abuse (perhaps covered in sperm after bukkake). 

Maybe something a la Clive Barker's \emph{The Midnight Meat Train} (from \emph{Books of Blood I}). 









\subsection{\Narkiza{} fights and Belgrim almost dies}
\hr{Narkiza}{\Narkiza} fights in a great battle. \hs{Belgrim} takes grievous wounds and almost dies. But with massive, heroic effort, Belgrim rises again, roars its defiance and fury, then charges and causes immense destruction. He collapses again. 

\Narkiza{} fights off the enemies to keep Belgrim safe. He is not skilled enough to heal him, but he has enough spells to send Belgrim into a deep, ensorcelled sleep, stabilizing his wounds until they can find mages and have him healed. 

Belgrim lives.









\subsection{Raising the \Haskelek}
\target{Raising the Haskelek}
The Sentinels begin raising the \Haskelek. 

There are three parts, called the Hand, the Eye and the Heart. 

The strongest part is the Heart. It is imprisoned in a forgotten temple deep in Heropond forest. It's in the southern part of Heropond, where the forest is thickest. It is the hate, madness and evil of the \Haskelek{} Heart that has corrupted Heropond and made it a dark, haunted place of monsters and diabolist savages. (There might be some planar stuff going on here, with the temple residing in another plane of reality. I should give this more thought.) 

The Hand is in Sumian, buried in a secret, shunned tomb. 

The Eye is in northern Pelidor, near \Forklin. It lies in an enchanted tower which exists in \Nyx{} but can be glimpsed from the physical world. It is called the Ghost Tower, and the locals fear to come within sight of it. 

%Somehow the Rissitics reach the \Haskelekz{} temple and awaken it. 
The \Haskelek{} has grown stronger in the meantime by feeding off the prayers and sacrifices of his primitive worshippers, and each part is now stronger than after its initial defeat at the Vaimons' hands. But still, each part is not strong enough to break its prison, and so they need pwoerful help to escape. In come the Rissitics, who attempt to free it. 
%and is now almost as strong as before its death, but it needs powerful help to break free of its prison. The Rissitics free it. 

The Eye is the first part to be released. Soon after the Rungeran forces conquer \Forklin, the Rissitic mages move in and begin the spells to awaken the Eye. It takes humanoid form\dash a humanoid is sacrificed to it, and it occupies his/her body. 

The Hand is the second part to be released. It happens near the end of the first book. After a long quest, Sir James and Lica fail and are killed. The Sentinels successfully awaken the Hand. They use its power to scour the area of opposition, securing a strong foothold in Sumian. Then the Hand goes to Pelidor to join the Eye. 

Now Carzain and the Redcor come in to stop them. A lot of stuff happens. 

%The Heart is released, but eventually the resurrection process is stopped and the \Haskelek{} returned to sleep. This is not necessarily the end of the \Haskelek{}, but it will take a lot more work to awaken it again. I don't know how this goes on exactly. Perhaps they destroy some artifact that was vital to the process. 

Eventually the Heart is released, but before the three parts can be united as one again, Carzain and the Redcor storm in and attack. They succeed in breaking up the ritual, weakening (some of) the \Haskelek{} parts and forcing them to retreat.
%The Redcor now oppose the \Haskelek{} and attempt to destroy it. It comes to a climactic battle, where they weaken \Haskelek{}, forcing it to retreat to its temple to recuperate. Redcor forces, including Carzain\dash and possibly Curwen\dash pursue him there, intending to reactivate the old spell seals and reseal the \Haskelek{} in its old prison\dash or something like that. 

Ultimately, Redcor forces, including Carzain\dash and possibly Curwen\dash destroy the physical bodies of the Eye and Hand, possibly the Heart, too. This breaks them up.

See, each part is still bound to its tomb, but is able to project its power into a possessed body and thus walk abroad. But if that body is slain, the \Haskelek{} part is sent back to its prison, and a new ritual must be performed to release it again. After the defeat of the Hand and Eye, the Redcor (with Cabal aid) storm the Ghost Tower and/or the Hand's tomb and besiege them, thus preventing the three parts from easily uniting again. The book ends in a victory for the Redcor, but one that leaves Carzain bitter and resentful (see section \ref{Carzain with Redcor}). 















\subsection{\Daggerrain{} knows it all}
Have occasional references to \hr{Daggerrain}{\Daggerrain}, hinting at his status as \trope{TheChessmaster}{Chessmaster}. He pulls all the strings and knows everything. The reader should be kept guessing as to how much of the story was actually orchestrated by \Daggerrain. 

Have Cabalists going: \ta{Yes, \Daggerrain{} predicted that this would happen\ldots{}}

An interesting twist might be to have \Secherdamon{} acting as a counter-Chessmaster, at times eaving pulling \trope{XanatosGambit}{Xanatos Gamits} on \Daggerrain.

Compare him to the Crippled God from \cite{StevenEriksonIanCameronEsslemont:MalazanBookoftheFallen}. 









\subsection{\Cishiel{} and \Dasteron}
\hr{Cishiel}{\Cishiel} is allied with \hr{Dasteron}{\Dasteron}. 
He wants to be Overlord of \Mystraacht, and she wants to help him. 
They are already lovers and have had plenty of sex. 
Often in public. 
That is \hr{Mystraacht sexuality}{a \Mystraacht{} thing}. 





\subsubsection{Frame it as a flashback}
\target{Flashback with Cishiel and Dasteron}
An idea: Let the entire story with \Cishiel and \Dasteron be a flashback at the beginning of the book where Ramiel returns to \Mystraacht. 
This way, I can introduce \Cishiel into the story without having to cover her background first, and I can dodge a lot of tricky timeline issues. 
At the time when Ramiel awakens, \Dasteron could have been Overlord for many decades. 





\subsubsection{Submissive \Cishiel}
The two devise a ploy to gain publicity: 
\Cishiel{} will sexually submit to \Dasteron{} in public. 
This will garner tons of attention and make them both look interesting and cool. 
It will make \Dasteron{} more alpha-male-ish and give him a little bit of \Zachirah-image. 

He invites her to a party at his place, or she invites him. 
As soon as they meet, she kneels down without a word and sucks his dick. 
She swallows, licks it clean, then stays on her knees. 
He grabs her hair and violently pulls her head back. 
She is in pain as she says: 
\quo{Thank you, Lord \Dasteron.}

He remains quiet. 
Throws her down to the floor and walks away. 
She remains on the floor for some seconds. 
Then she rises and meekly follows him. 

Then, at dinner, they both act like it didn't happen;
as if it were the most natural thing in the world. 
(Of course, they also have sex that night.)

Another possibility is to have \Cishiel{} act like a willing slave for \Dasteron. 
During dinner she kneels by his side. 
He caresses her hair and head, at times violently. 
When he is angered (by something else), he grips her hair and yanks her roughly. 
She moans and whimpers in a sexy way. 

Slave-girl \Cishiel{} is dressed in skimpy clothes, with small breast-cups of gold. 
Not naked as \hr{Resphan slave livery}{a \Mystraacht{} slave would normally be}, but close enough to naked to invoke associations of slavery. 
And a tight necklace reminiscent of a slave's collar. 
\Dasteron{} tried to make her wear an actual slave collar, but \Cishiel{} drew the line there. 
She was willing to endure much humiliation, but not that much. 

Everyone else knows that it was planned and constructed to make him look cool and give him publicity. 
But it still works. 
\ps{\Dasteron} rivals cannot help but be affected, even though they know it is a ploy. 
And \Dasteron{} knew they would. 

\Cishiel{} is, of course, richly compensated in favours. 
And it also reflects on her standing. 
She gains a certain naughty, erotic allure: 
\ta{Imagine the daring, that she would do such a thing!}
\hr{Resphain are possessive}{\Resphain{} are possessive}, and \Mystraacht{} perhaps more than any.
Now, whenever a \resphan{} sees her he thinks \quo{submissive sex slave} inside his head, and he gets obsessed and must have her. 
This gives her a great deal of power and political influence, and she knows how to use it. 

She muses over it. 
She has broken the rules of sexual conduct. 
In \CiriathSepher{} such an act would leave her branded as a slut with no self-respect. 
But in \Mystraacht{} it has the opposite effect. 
\Mystraacht{} is founded on a spirit of rebellion and anarchy; they hunger for sensations and turmoil. 
Breaking the rules is admired. 
She has been naughty and broken the oppressive \CiriathSepher{} morality, and the \Mystraacht{} love her for it. 





\subsubsection{\Cishiel{} hears rumours about Ramiel}
\Cishiel{} hears rumours that Ramiel is returning. 
She gets excited and goes looking for him. 

\Dasteron{} learns of it. 
It drives a wedge between him and \Cishiel. 
But they still have sex, and they still cooperate. 
It adds some spice, some challenge, some excitement to their relationship. 
They are \resphain{} of \Mystraacht, after all. 
They thrive on conflict. 

They end up enjoying this game, where they both try to extract information and favours from one another. 
They are now rivals. 
They manipulate and distrust but respect each other. 
And they have exciting, dangerous rival sex. 





\subsubsection{\Dasteron{} rapes \Cishiel}
Later, \Dasteron{} suspects \Cishiel{} of having done stuff to harm him behind his back. 
(She knows Ramiel is back. He does not know yet.) 

So \Dasteron{} comes and threatens \Cishiel. 
But she is clever. 
She manipulates him and re-frames his threats as a sex game. 
It ends with him mock-raping her and her mock-pleading for mercy. 

After he is done raping her and has dressed again and is about to leave, he turns around. 
She is still naked and lying on the bed. 
He gives her a long and intense stare. 
He makes her feel even more naked than she is. 
She wants to wrap her wings around her naked body in protection, but she does not want to show such a surrender to him, so she just lies there and lets him stare. 

Without words he tells her: 
\begin{prose}
  \Dasteron: 
  \ta{You tricked me today, \Cishiel. 
    But I will get you. 
    Mark my \quo{words}.}
\end{prose}





\subsubsection{\Dasteron{} becomes Overlord}
\target{Dasteron becomes Overlord}
\Dasteron{} made himself Overlord by gathering enough support to beat down those who disagreed. 
This had taken more than a thousand years of concentrated work. 

Before \Dasteron becomes Overlord, he has to fight several battles to the death against potential rivals.
Strength in combat is not the only virtue demanded of an Overlord, but it is an important one.
He brings \Cishiel as a spectator to those battles. 
She is dressed to resemble a slave: 
A bikini of brass and a necklace that sort of looks like a slave collar, and otherwise naked.

\target{Dasteron cannot become Apex}
\Dasteron is powerful, but his power and skill is chiefly political, not metaphysical. 
He lacks the \vertex{} strength required to make himself \apex{} of the \Mystraacht{} \matrix. 

At the point when he becomes Overlord, \Cishiel{} has learned of Ramiel's return and is working behind \ps{\Dasteron} back to support her father. 









\subsection{\Vizsherioch{} becomes a \shaeeroth}
\target{Vizsherioch becomes Shaeeroth}
\Vizsherioch{} becomes a \shaeeroth.
This is the culmination of the whole \Nithdornazsh{} gambit. 
\Nithdornazsh{} was a stepping stone for \Vizsherioch, to enable him to contact the \xss{} and harness the power that would make him a \shaeeroth.

No one anticipated this. 
Most people have forgotten that \shaeeroth{} could be created, since {no more had been created} in the last thousands of years. 
Plus, most people do not know \Vizsherioch{} even exists. 

The kind of invocation needed to become a \shaeeroth{} is different from most of the invocations that summon forth monsters or other stuff to do the mage's bidding. 
Therefore, when \Vizsherioch-tachi prepared and performed the deed, it was not seen as such an apocalyptic thing, so the Cabal did not really move out in force against them. 
They did not understand the magnitude of what was happening, because it was so atypical, and it looked relatively benevolent (or, at worst, self-destructive). 

After he becomes a \shaeeroth, he becomes feared by all.
He grows so much in power as to rival Secherdamon and \Ishnaruchaefir. 
He is, after all, (sort of) an incarnation of the \xs. A \xs in draconic form.















\section{Ramiel's Defection}
Ramiel betrays the Redcor and defects to the Royalist Faction. 







\subsection{Carzain sacrifices \Racel} 
\target{Racel dies}
Somehow Carzain and the Redcor, fighting against the evil Royalist cultists, set up a gambit where Carzain poses as a turncoat who wants to join \Belzir. 

He tells the Redcor of his plan. 
\Belzir{} has long tired to seduce him, and he has let himself be seduced to an extent, but he has always defied her in the end. 
(We have seen this.) 
But now, he says, he thinks he can fool her. 
And here is how\ldots{} 

The reader must not know his entire plan. 
Then the genre-savvy reader will know it will fail. 
\trope{UnspokenPlanGuarantee}{Unspoken Plan Guarantee}, remember. 

He sneaks in in order to infiltrate the Royalists so the Redcor can stop them. 
He now has his hands on \hr{Belzir}{\Belzir} soul-jar, which the Redcor had been keeping secret and hidden from everyone. 
He has persuaded them to give it to him because he says he can destroy \Belzir{} with it. 

\Racel{} is with him, for some reason. 

The Royalists do a \trope{ShootYourMate}{Shoot Your Mate}: 
They tell Carzain that if he is with them, he should kill \Racel. 

Carzain walks towards \Racel. 
She is nervous, but knows he would never kill her.
He moves closer.
She becomes afraid.
\tho{What is he planning?}

He kills her by chopping her head off. 

See, Carzain had been playing the Redcor for fools. 
He wanted to get in on their gambits and know as much of their insider knowledge as he could. 
But now he thinks he has learned as much from them as they are willing to let him, so he seeks greener pastures. 
He betrays them and joins \Belzir. 
He knows she can be of tremendous help to him.
He gives her the soul-jar, and she is now almost ready to return to life. 







\subsection{Carzain flees to Geica} 
Carzain flees to Geica, together with Shereid and possibly a few others. 
(Maybe even Ilcas?) 
They meet up with other members of the Royalist Faction (the followers of the Dark Queen), including Senator Hayad. 
From here, the next step is to conquer Geica from within. 
(Geica has a democracy; they mean to seize it by a coup.) 

\emph{New idea:} 
At this point, Carzain is not evil, albeit disillusioned. 
Shereid convinces him that there are forces in Geica that are better than the Redcor. 
She introduces him to Hayad and some others, who make a good impression on him. 
It takes a while before he learns that the one behind it all is \Belzir. 
He balks at this, because he has always been told that \Belzir{} is evil, but they all argue that she was less evil than the stories say, that the Redcor and other wicked folks deliberately blackened her reputation\dash and that it was in fact their rebellion that caused the empire's fall, not \Belzir's just retaliation. 
After a while, Carzain accepts this. 
After all, he hates the Redcor and is willing to believe much bad about them. 
He also balks when he learns that they mean to conquer Geica by a coup, but after spending some time among the Geicans, he is convinced that they are little better than the Redcor. 







\subsection{Carzain allies himself with \Belzir} 
Carzain allies himself with the Dark Queen, and she sends demons to aid him. 
She wants him to help resurrect her, that she may return to \Miith{}. 
To this end, she gives him a magical orb. 
Carzain agrees, but plots against her. 

Carzain joins \Belzir's armies. 
He is equipped with the sabre and the \hs{Archon Ward} that belonged to her son Zacrias (the most powerful of \Belzir's many children, who succeeded her as Lord of \ClanGeican after her death). 
He also gets two antique \hr{Vaimon guns}{Vaimon pistols} (magical and powerful). 

\Belzir{} also summons a Peryton (a hind, whom Carzain names Venom) to serve him. 

\lyricsbs{Emperor}{Witches' Sabbath}{
  Shifting shadows guide my way in the autumn. \\
  From this fell alliance eternal night shall arise. \\
  Hear the scream of the wolves calling again. \\
  Legions are wreaking destruction upon the fortress.
} 







\subsection{Ramiel's tragedy} 
\target{Ramiel's tragedy}
\Belzir has a mad plan to conquer the world using Elder sorcery.
Ramiel joins her. 
Eventually he betrays and usurps her and takes over her mad plan. 

Some heroes try to stop Ramiel.
Try to set up Ramiel's story as a tragedy where he will ultimately fail and be defeated.
The heroes strive against the odds, and as such they are bound to win. 

Portray Ramiel as a villain who becomes more and more mad and evil. 
His sanity is suffering. 
(He will later learn from his experience and grow saner again.) 
The reader believes that Ramiel has finally fallen from grace and will eventually be defeated, or perhaps (at best) turn good and sacrifice himself in the end. 

But Ramiel surprises everyone by pulling through and winning. 
He does not conquer the world, but he gains enough power to make himself a god. 
This is a subversion of the usual \trope{Tragedy}{Tragedy} trope. 

\begin{itemize}
  \item 
    Perhaps the Redcor and their allies invade Geica in order to stop Ramiel and \Belzir. 
  \item 
    Or perhaps it is his mad search for the ancient temple which the heroes strive to avert. 
    Perhaps they follow him there and try to kill him, to prevent him from becoming a dark god. 
\end{itemize}

Compare him to Kane in \cite{KarlEdwardWagner:DarknessWeaves,KarlEdwardWagner:Bloodstone,KarlEdwardWagner:DarkCrusade}. 















\section{Carzain in Geica}
\subsection{Geica is corrupt}
\target{Carzain is disappointed in Geica}
Carzain has been filled with stories of how glorious Geica is. He has heard and read stories of the splendid, enlightened Geica as it looked during the time of the \hr{Vaimon Caliphate}{\VaimonCaliphate}. He sees it as a proud, enlightened, freedom-loving and free-thinking culture where all men are equal and free to achieve their goals and live for their ideals, and also a Mecca of learning and wisdom. 

But he is sorely disappointed. The real Geica is corrupt, a cesspool of hypocrisy, bureaucracy and greed. I need to underline the avarice, pettiness and treacherousness of the Geicans. 

Have a named guild of politicians, lawyers and the like (\quo{\DJOF'ers}\footnote{\quo{\DJOF} is Danish for \quo{Danske Juristers og \O konomers Forbund} (or something like that), the \quo{Danish Lawyers' and Economists' Association}.}), who are made the scapegoats and blamed for the corruption of Geica, by virtue (or vice) of the materialistic world view they spread. 

Compare to the Letherii culture in \cite{StevenEriksonIanCameronEsslemont:MalazanBookoftheFallen}, and especially Chancellor Triban Gnol and Invigilator Karos Invictad. 

Perhaps the Geicans own no slaves, only life-long indentured serfs. Compare to the Indebted in Lether. 

This is a part of Carzain's process of disillusionment with all that is good. 

According to \Belzir, it has been going downhill ever since her death. She represented the pinnacle of Geican civilization, but history has posthumously vilified her. Her son \hs{Zacrias}, who succeeded her, did his best to carry on her legacy. His successors were less dilligent and less faithful. 

Perhaps Geica's decay is a result of being vilified, demonized and hated by the rest of the world, especially the \hs{Redcor}, who blamed the Geicans for \Belzir{} and the \hr{Hundred Scourges}{\HundredScourges}. 

Compare to House Harkonnen in \authorbook{Brian Herbert \& Kevin J. Anderson}{The Battle of Corrin}, who descended into evil after being abandoned by Vorian Atreides. 

After all this corruption and hypocrisy, \ps{\Belzir}{} bloody and violent coup almost seems an improvement. The \hs{Royalist Faction} have ideals of \quo{awakening to life the true Geican soul and legacy} and overthrow the corruption and decadence that has festered like a disease, a vermin infestation, in the absence of the true Geican spirit and way of life. They fight for the true Geica and seek to awaken the people. 

Compare them to Bruthen Trana and his Tiste Edur in \MalazanReapersGale. 





\subsubsection{True ruler is not good}
\ps{\Belzir} brutal coup is intended as a subversion of the \cliche{} of \quo{rightful ruler returns to the throne and all is good}. 





\subsubsection{Limyaael's ideas}
See \href{http://limyaael.insanejournal.com/205202.html}{Limyaael's rant on political systems} for inspiration and ideas on how to flesh out the Geican political system. 

See also \href{http://limyaael.insanejournal.com/205202.html}{Limyaael's rant on \quo{council scenes}} for inspiration regarding political intrigue and debates. 







\subsection{Royalists}
In Geica there is a \quo{Royalist Faction} who support \Belzir{} and want to restore her to life and throne. 
They love their queen and fight fiercely for her. 

Vizicar has similar feelings. 
He believes \Belzir{} is the key to his own goals. 
Besides, his feelings remember their old love for her. 

\citebandsong{BeyondTwilight:FortheLoveofArtandtheMaking%
}{%
  Beyond Twilight%
}{%
  For the Love of Art and the Making%
}{
  Through the marshland\\
  And through the rain and mud\\
  Now listen\\
  I march to youm my sleeping beauty\\
  I alone\\
  I cut myself on the moonlight beams\\
  Hordes of wolves follow behind me\\
  I alone\\
  I'm all alone\\
  My face brightens in the flash of the lightning
  
  Through the morning mist\\
  And through the frozen land\\
  I feel your presence\\
  I march to you, my sleeping beauty\\
  You alone\\
  I cut myself on the moonlight beams\\
  So that you can drink from my chest\\
  You alone\\
  The wind gripping my hair\\
  My teeth are grinding\\
  As we meet in the flash of the lightning\\
  It's like a dream
}







\subsection{Vizicar picks up an instrument}
In Geica, Vizicar sees a flute or something lying around. He picks it up. \ta{I can play that.}

He pipes some notes. It sounds horrible. He puts it down. \ta{I didn't say I could play it \emph{well}.}









\subsection{Carzain is drawn by the moon}
Carzain feels somehow that the moons are a key to unlocking the secrets of his past.

\lyricslimbonicart{Enthralled by the Shrine of Silence}{
  Restless days and sleepless nights.\\
  Drawn in the direction of the moon.\\
  Infernal magnet to mysterious destiny,\\
  beyond the grave of doom.
}









\subsection{Carzain goes on a quest to find a \vorcanth}
\target{Carzain dreams of Moon-Wolves}
Carzain is sought out in dreams by the \vorcanths. 

When he finally meets the \vorcanths, he also meets some of their \hr{Aryoth}{\aryoth} companions. 
But it is the \vorcanths{} who lead, not the \aryothim. 

\lyricsbalsagoth{Of Carnage and a Gathering of the Wolves}{
  [VOICE OF THE NIGHT:]\\
  Who are you, wanderer?
  
  [WANDERING SPIRIT:]\\
  I can't remember\ldots{}
  
  [VOICE OF THE NIGHT:]\\
  The wolves are gathering,\\
  the stars are shifting\ldots{}\\
  come, join us in the hunt.
  
  [VOICE OF THE NIGHT:]\\
  Who are you, wanderer?
  
  [WANDERING SPIRIT:]\\
  I have the scent\ldots{}
  
  [VOICE OF THE NIGHT:]\\
  Gaze into the mists\ldots{}\\
  feel the earth thawing beneath your feet.\\
  Come, bring down the prey.
}

Turns out there is a wounded \vorcanth{} who cries out for his help. 
He goes on a quest to find and save it. 
See section \ref{Moon-Wolves help Ramiel in dreams}. 

They guide him, employing occult astrology and \hr{Moon-Wolves and the Moon}{their mystic connection to the moon Visha}.

\lyricsbalsagoth{Of Carnage and a Gathering of the Wolves}{
  [THE SYLVAN ORACLE:]\\
  The wolves are gathering.\\
  The stars are shifting.\\
  This spectre at the feast.\\
  This nectar of the vine.
}

They guide him towards his true identity. He begins to understand his true nature: A terrible creature, cruel yet glorious.

\lyricsbalsagoth{Of Carnage and a Gathering of the Wolves}{
  [VOICE OF THE NIGHT:]\\
  Look at the power you possess\ldots{}\\
  See the might which you wield!\\
  You know who you are, do you not?
  
  [WANDERING SPIRIT:]\\
  Yes, I am the scythe in the field at summer,\\
  I am the thunder that awakens the earth,\\
  I am that which gives the night air its chill.
  
  [VOICE OF THE NIGHT:]\\
  Who are you, wanderer?
  
  [WANDERING SPIRIT:]\\
  I am far beyond the ken of men\ldots{}\\
  my gaze shall make the night tremble!
}

The \vorcanths{} \hr{Moon-Wolves dislike Dragons}{do not like \dragons}, their ancient rivals. They see Ramiel as their saviour.

\lyricsbalsagoth{Of Carnage and a Gathering of the Wolves}{
  He is the scourge, the thanatos, \\
  the cleansing fire, the purifying storm\ldots{}\\
  he is the cataclysm given corporeal form!
}

Later, he remembers that he is on a mission of genocide against the \dragons{} and their spawn.

\lyricsbalsagoth{Of Carnage and a Gathering of the Wolves}{
  [VOICE OF THE NIGHT:]\\
  Who are you, my son?
  
  [WANDERING SPIRIT:]\\
  Father\ldots{} I am annihilation incarnate!
}







\subsection{Someone admits to rationalizing}
Someone, possibly \Ishnaruchaefir, tells about his own history. He portrays himself as the hero and most everyone else as misguided or corrupt. At some point, he confesses that he is subconsciously rationalizing and shining up the story to fit his own desired interpretation. 







\subsection{Fake rebellion is beaten down}
The Geican power holders have caught the scent of the Royalist Faction rebellion. 

So the Royalists stage a fake coup. They are discovered because of information they themselves leaked, and are beaten down. Some of their \quo{leaders} are caught and killed. Apparently the faction is destroyed. The intention is to make the enemy lower their guard, so they can resume their plotting and be more effective for it. 

Compare to the plot of \FLuneNoireVol{5-6}. 

Hayad fights against the rebellion and thus proves himself \quo{loyal} to the government (although, of course, this is just a cover). 

Carzain is unsuspecting and only barely survives thanks to Vizicar's cleverness. \Belzir{} then tells him that it was a test of him somehow. 

Carzain is bitter over having been deceived and almost killed. 

Only Hayad apologizes to Carzain for using him. 







\subsection{Carzain becomes really badass}
During the course of this book, Vizicar awakens some more, and he and Carzain learn to better control their powers. They become more badass, in power and attitude alike. 

He begins to be more of a Dark Knight. Perhaps, at times, his eyes glow with some spectral \colour. 





\subsubsection{He remembers more of Ramiel}
He gradually remembers more and more of his life as Ramiel.

\lyricslimbonicart{Behind the Darkened Walls of Sleep}{
  Behind the darkened walls of sleep, \\
  as body rest and mind goes deep. \\
  A door opens in my heart. \\
  A dark euphoria.
  
  The soul is longing to escape \\
  the tyranny of the body. \\
  Dark winds now embrace \\
  the spiritual entity.
}

He remembers the wars he fought with the legions of \Mystraacht. 

\lyricsbs{Arcane Wisdom}{Maelstroms of Majestic Night}{
  Torrents of blasphemous fire enrage forth, \\
  thus allowing my soul to devour \\
  feeblish souls with sickening power, \\
  as the ancient warcrafts proudly descend from the North. 
  
  Oh the blood of my foes and fiends. \\
  How I beheld thy corpses, \\
  raptur'd by virulent winds. 
}









\subsection{Carzain drinks \ps{\Belzir}{} blood}
Carzain drinks \ps{\Belzir}{} blood during sex. 
It is \hr{Resphan cannibalism}{the greatest rush he's ever tasted}. 
It also brings back a little bit of \resphan{} memories, since it's \sathariah{} \Malach{} blood he's drinking. 

Compare to the movie \cite{Movie:QueenoftheDamned}. 







\subsection{Carzain goes into battle}
Carzain is about to go into battle. He dons his \armour. 

Vizicar complains. 
\vizicar{This is primitive.} 
Vizicar is used to wearing an \hs{Archon Ward}, a magical super-\armour so expensive that pretty much only the \VaimonCaliphs could afford it. 















\section{War Against the Redcor}
\subsection{Carzain has become Ramiel}
During the course of the Geican coup, \hs{Carzain} regains almost all of \hs{Vizicar}'s memories, and much of \hs{Tydesmos}'. 

By the time of the beginning of this book, Carzain and Vizicar have merged enough to be one person, with access to all of their shared memories (or all that they remember of them). (Although they still have their internal dialogue from time to time.)

He now uses the name Ramiel. 

He has become a badass dark lord who wears black \armour and robes. With swords and magic he kills anyone who stands in his way. At times his eyes glow red or some other evil \colour. 

They ally with \hr{Psyrex}{\ps{\Psyrex}} cult and wage war against the Redcor. 

Compare to Wismerhill from \FLuneNoire. It is interesting that Geica's \colours are green and black, like those of the Black Moon.

So the man who puts himself in command of the Geican army is far more than Carzain \Shireyo. He has all the life-long experience of a \VaimonCaliph and an archmage at his disposal. 
Vizicar is a much better general than \Belzir, but she is a far more skilled mage. 

Carzain is sort of like Rio from \emph{\JuukenSentaiGekiranger}. His obsession is not only to become stronger, but also to learn his true identity. 





\subsubsection{Archon Ward and other equipment}
\Belzir{} finally provides an \hs{Archon Ward} for Ramiel. Vizicar is happy. Ramiel also gets a black cape and robe, enforced with metal, to serve as backup \armour. Now, at last, he looks like a true Vaimon dark knight. 

At Vizicar's command they also produce for Ramiel an imperial coronet. It is enchanted and socketed with magical jewels. Perhaps these jewels were previously acquired on a quest.







\subsection{\Belzir controls Geica}
\Belzir{} now controls Geica. She and her associates use a mix of sincerity, propaganda and Shroud-weaving magic to make the people adore them as liberators. 









\subsection{\Belzir's resurrection nears}
\Belzir{} is looking forward to her impending resurrection. 

\lyricslimbonicart{As the Bell of Immolation Calls}{
  In a timeless departure from the flesh,\\
  Drifting the cold ether streams of death.\\
  By the altar of sacrifice, as I call upon the night\\
  to take and give me life beyond the shores of light.
  
  I glorify the hour of blackness\\
  as the bell of immolation calls.
  
  A black heart will adorn\\
  the wings when I'm reborn.\\
  Engraved on my memory\\
  is whom hatred made me.\\
  The ravages of time.\\
  Battles on in my mind.\\
  There are still wounds that bleed\\
  deep in the soul of mine.
}

She is happy to leave behind the horrible limbo.

\lyricslimbonicart{As the Bell of Immolation Calls}{
  A life among the dead and sorrowful,\\
  the endless voids where spirits are mournful.\\
  From the pale of agonising light \\
  I cross the bridge to crystal night,\\
  as the bell of immolation calls.
}







\subsection{Carzain turns against \Belzir} 
It turns out that Ramiel and \Belzir{} were manipulated by \Secherdamon{} and \Psyrex. 

\Belzir becomes more and more irrational and hysterical, like Efrel in \cite{KarlEdwardWagner:DarknessWeaves}. 
Ramiel begins to resent her. 

Ramiel becomes disgruntled. 
Ramiel grows to hate \Belzir{} and desires to usurp her, because she is an insane bitch. He also grows to hate the Sentinels for trying to manipulate him. 

He resents Shereid when he learns that she was part of the plan. He punishes her and means to kill her, but she pleads for her life. She claims that she loves him above all else and swears to serve as his devoted slave. So he lets her live. 

Ramiel's betrayal should be kept secret from the reader until the very end, but subtle hints should be dropped here and there, as he plots, schemes and prepares. The reader should understand that Ramiel is \quo{up to something}, but kept guessing as to what.









\subsection[Overlordship of Mystraacht]{Overlordship of \Mystraacht}
Ramiel and \Shiaraid{} both covet the throne of the \hs{Overlord} of \Mystraacht. 

\Shiaraid{} is the daughter of \hr{Zachirah}{\Zachirah} and feels that she is the rightful heir and should be Overlord, with Ramiel as her Prince Consort. 
She accidentally lets a suggestion of this slip. 
Ramiel picks up on it. 

He is not happy. 
He wants to rule, over \Mystraacht{} as well as over her. 
\emph{Especially} over her. 

Later she offers him the position of Overlord, with herself as his Consort. 
But by that time he has gotten paranoid and fears her ambition. 
He suspects she will cheat him and try to manipulate him so she can rule behind the throne with him as a marionette. 
She has done it before. 
She is good at exercising power even when she seems submissive. 
He does not trust her. 

\Shiaraid{} feels it is OK to submit to him but still exercise power. 
In her mind, it would be a great arrangement. 
Ramiel has his pride and she has her submissive side, so she is perfectly willing to let him have the fancy title, with her as his nominal inferior. 
But she still sees herself as his equal in worth, so she wants an equal share of the power. 
So in a sense her sado-masochistic way of thinking is more sane and healthy, since she can accept a equal relationship where she is sometimes dominant and sometimes subservient. 

Ramiel does not want to share. 
He has his (almost hysterical) \hr{Ramiel is nothing}{fear of being nothing} and cannot stomach the thought of submitting to anyone. 
He must dominate all the time. 

\hr{Curse}{\NexagglachelsCurse} plays them against each other and makes them suspect and plot against each other. 
They argue and fight and lash out in anger, with words or actual violence. 





\subsubsection{\Belzir{} is a bitch}
I need to make sure that \Belzir, under the influence of the Curse, acts like a real bitch at times. 
She may silently regret it afterwards. 
But she is bad enough that the reader understands Ramiel's growing hate and sympathizes with him (to some extent) when he betrays her. 









\subsection{\Belzir cries out \ps{\Aryal} name}
While having sex with Ramiel, \Belzir{} moans the name of \Aryal. 
Ramiel says nothing, but he is not happy. 
He knows that \Aryal{} was and is the greatest love of her life, and that he will always rank below her.
He does not like the reminder that he stands below someone else in some respect. 
It triggers his trauma about \hr{Ramiel is nothing}{being nothing}. 
It makes him jealous and resentful. 

It also makes him mistrust \Shiaraid. 
He fears she is more loyal to \Aryal{} than she is to him. 

Then he finds out the story of what happened to \Aryal. 
\Shiaraid{} \hr{Silqua dies}{betrayed and killed her}. 
If she would do that to her true love, what would she not do to him? 
The second-rate usurper who \quo{thinks he can take \ps{\Aryal} place in \ps{\Shiaraid} heart}? 

He resolves that he can't trust her and begins to plot against her. 

\Shiaraid{} also thinks about it. 
She knows Ramiel dislikes \Aryal, thinking of her as a coward and a weakling. 
This makes \Shiaraid{} resent Ramiel. 








\subsection{Shereid in distress}
Shereid slowly goes mad. Partially out of her obsessive love for Carzain/Ramiel, but even more so because she learns the cruel truth of \ps{\humanity} history. 

She keeps herself halfway sane by clinging to Ramiel and her love for him. 
They have incredible sex. Ramiel's love confirms to her that her life has a purpose. As long as she can be with him, she can live with the fact that her entire race is created to be the slaves of monstrous aliens. 

Compare her to Mele, with her love for Rio, from \emph{\JuukenSentaiGekiranger}. Come to think of it, Ramiel is also quite like Rio. 









\subsection{Carzain doubts his evil, and finally embraces it}
Throughout the book, Carzain is plagued by the memories of his evil life as Ramiel. At first he is horrified, repelled by what he finds inside him. But in the end, he embraces that legacy as his true self.

\lyricslimbonicart{Sources to Agonies}{
  Through the mirror of the soul\\
  I'm staring deep within\\
  To see what dwell behind the wall,\\
  The beauty of pale skin.
  
  The aura that surrounds me\\
  is not of noble kind.\\
  The blackness of the heart\\
  is all that's left to find.
  
  A dark river runs silent through my life\\
  like a floating nemesis.\\
  A dark shadow of what that used to be\\
  drifts now in lifeless misery.
  
  Live only to witness what I've become.\\
  Midnight is my shallow home.\\
  Soon to enter the last deed of mine.\\
  I'm forced to follow the streaming bloodline.
  
  When the wine of life is shed\\
  and dark cosmic space consumes\\
  I bring the memories back from the dead.\\
  Sources to agonies, a devouring monsoon.
}









\subsection{Black magic ritual}
\Belzir orders a black and perverse ritual in the dungeons underneath the \TopazChateau. 
The ritual includes the self-sacrifice of over a hundred prisoners that are hypnotized and mind-controlled by means of a mummified \ophidian head which \Belzir controls. 
Carzain shudders at this, for he knows of the \ophidian race and their powers of mind control. 

The head is undead, plundered from a Durance tomb. 
After he betrays \Belzir, Carzain destroys the \ophidian head out of respect for the dead (remembering his ally \hr{Zeethan Kraal}{\ZeethanKraal}). 

Compare the ritual to Carathis's spell with the 50 deaf-mute one-eyed negresses in \cite[p.100]{WilliamBeckford:Vathek}. 









\subsection{\Belzir is resurrected}
\Belzir{} is resurrected and has her body back. 
Perhaps she even gets a \resvil{} body. 

But part of her soul is imprisoned and sealed somewhere in \Redce. So they ally with \hr{Psyrex}{\ps{\Psyrex}} cult and conquer \Redce. 

Then they perform a magical ritual to heal \Belzir{} and return her true power to her.
Perhaps this ritual is also meant to restore her memories and her full \Malach{} power. 
But in the end, Carzain betrays \Belzir{} and kills her during the ritual. 

\lyricsbs{Cradle of Filth}{Cruelty Brought Thee Orchids}{
  Maleficent in dusky rose.\\
  Gathered satin lapped Her breasts\\
  like blood upon the snow.\\
  A tourniquet of Topaz\\
  glistened at Her throat.\\
  Awakening, pulled from the tomb,\\
  Her spirit, freed, eclipsed the Moon\\
  that She outshone as a fallen star.\\
  a regal ornament from a far flung nebular.
}







\subsection{\Vizsherioch{} is strengthened}
Have an arcane ritual where \Vizsherioch{} is given more power. 

Compare to \FLuneNoireVol{9}, where Wismerhill becomes the Prince of Negation. 

\lyricsflnv{9}{
  Haazeel Thorn: 
  \ta{%
    Life was truly torn from you, but I have replaced it with something better. In your veins now flows Negation.}
}







\subsection{\Redce{} falls}
Under the onslaught of the \hs{Geican} and \hs{Dark Crescent} armies, \hr{Redce}{\Redce} finally falls. 

In the end, Ramiel has to fight a boss monster: 
\Matriarch{} \hr{Dominice}{\Dominice}, a really badass Redcor mage. 
%, who is a Scion of the \Malach{} \hr{Eryal}{\Eryal}. She is older and more experienced that he, and knows more of her true nature and power. She might not be a \sathariah, but he cannot wield his full \sathariah{} power yet. 
She is far older and more experienced than he, and while a mere \human, she knows a lot about him and knows spells that can counter his \resphan{} powers. 

%She actually defeats Ramiel. But he is not quite dead. She becomes overconfident, and he rises to strike her down. Compare to Wismerhill from \FLuneNoireVol{10-11}. 
Ramiel knowingly holds back and lets \Dominice{} defeat him and \emph{almost} kill him. 
%\Dominice{} defeats Ramiel, almost killing him. 
He contacts \Belzir{} and requests that she use her power to keep him alive. 
She cannot afford to refuse. 

Meanwhile, \Dominice{} has become overconfident. 
He rises to strike her down. 
He kills her, then eats her and absorbs her power. 

Have a bloody scene where he tears out her heart, brain and liver and eat them. 

% He insisted on facing her alone because he wanted to eat her alone. 
% This gives him the strength he needs to face \Belzir{} and betray her. 
% Also, he has forced \Belzir{} to deplete her strength to heal him. A dangerous \trope{XanatosGambit}{Xanatos Gambit}, but Ramiel has always been reckless. 
This was a part of Ramiel's plan: 
He let himself be wounded, thus forcing \Belzir{} to pour her power into him and depleting her own strength to heal him. 
Now \Belzir{} is weakened, while he, having eaten \Dominice, is strong.
A dangerous \trope{XanatosGambit}{Xanatos Gambit}, but Ramiel has always been reckless. 












\subsection{The Resurrection} 
\target{Shiaraid dies}
Ramiel makes Shereid his willing sex slave, and together they plot to usurp the Dark Queen. 
Ramiel is bitter at \Belzir{} for treating him badly. 

\citebandsong{BeyondTwilight:SectionX}{Beyond Twilight}{%
  Ecstasy Arise%
}{
  I've learned to cry and crawl away\\
  But at the closing of our circle you will pay\\
  I see the hope vanish in your eyes\\
  I feel the ecstasy in me arise
}

\Belzir's resurrection is a ritual performed by many mages. 

She must be reborn through a \human{} body. 
So a young, healthy, beautiful Royalist woman (named Shadira or somesuch) willingly sacrifices her body and soul to her beloved queen. 

Carzain and Shereid plan to interrupt the ritual and drain \Belzir's power. 
In order to do this, they must have their own agents participating in the ritual in the key spots. 
So they begin to subvert other members of the Royalist Faction to their cause. 
When the war progresses and the day of the resurrection nears, they take measures to assassinate the mages that have remained loyal to \Belzir, so that they may control the ritual. 

At the end of the story, there is a big ceremony to resurrect the Queen, in which Carzain must channel her power through the orb. 
But he betrays her by smashing the orb, thus keeper her power for himself. 

After this, Carzain sets himself up as \caliph. 
But unbeknownst to him, \Belzir{} is not destroyed. 
After the betrayal, she contacts a group of Royalists in Geica, who (hidden from Carzain) were prepared to take over the ritual if anything should go wrong. 
They succeed, and she is resurrected. Some of the Royalists side with Carzain. 

Hayad and his children are among the Royalists who side with \Belzir. However, after a while Hayad notices that \Belzir{} is becoming dangerously insane. 
One night, she tortures and kills a lover who fails to satisfy her sexually. 
Hayad, fearing that he is next in line, flees to join Carzain. 





\subsubsection{\Shiaraid{} dies}
So Carzain and \Belzir{} clash and fight.
And in the end he is victorious. 
He not only kills her, but also \hr{soul-eating}{consumes her soul}. 

She fights back. 
It is a hard battle. 
Ramiel is in danger, so he has to draw really fucking deep of the power within his \carcer. 
He digs deep into his inner darkness in order to unleash as much as he possibly can of his wicked \sathariah{} power. 
He forces the souls within his \carcer{} to submit and serve him. 
This is tough, because there are immortals in there. 
Even \dragons. 

\citebandsong{BeyondTwilight:SectionX}{Beyond Twilight}{%
  Shadow Self%
}{
  Blood on a legion of nameless shores, I beseech you!\\
  Skill of a thousand nameless whores, I invoke you!\\
  Holding the light away\\
  Twisting in my mind
  
  Love of a legion of begging slaves, I exalt you!\\
  Screams of a thousand burning souls, I unleash you!\\
  Holding the light away\\
  Twisting in my mind\\
  Into the shadow's waking eyes
  
  Power of a legion of nameless lords, I command you!\\
  Locks of a thousand nameless doors, I unlock you!\\
  Lash of the six soul-taking strokes, I will feed you!\\
  Twisting in my mind\\
  Into the shadow's waking eyes
  
  \quo{Spiritus Pervertus Nymph}\\
  \quo{Spiritus Pervertus Femme}\\
  Twisting in my mind
  
  I know he steals, my body is haunted\\
  Wicked voice in my spirit has control\\
  Blinded by the beauty of my darkness\\
  Makes me fly, takes me higher
  
  I invoke thee, blackened lord.\\
  Night begins to crawl the skies\\
  Kneel before your master, shadow self\\
  Taste the pain caressed by chains
}

But in the end, \Shiaraid/\Belzir{} submits and lets him kill her, like \hr{Silqua dies}{\Aryal{} once let \Shiaraid{} kill her}. 
For a number of reasons:
\begin{enumerate}
  \item 
    \Shiaraid{} feels guilty over having killed and eaten her (submissive) lover, \Eryal, so she feels she deserves to have her new (dominant) lover do the same to her. 
  \item 
    \NexagglachelsCurse{} makes her self-destructive and suicidal and emo. 
  \item 
    She is very sad because \Mystraacht{} is in turmoil and has been tearing itself apart from the inside for thousands of years. 
    \Mystraacht{} is her father's legacy, so she is strongly attached to it and wants it to live and be strong. 
    She has almost motherly feelings towards her dynasty and will sacrifice herself for it.
    When she lets Ramiel absorb her power and unify it with his own, \Mystraacht{} will be one step closer to uniting and restoring its former glory. 
    
    After all, \ps{\Zachirah} great strength as a leader derived partially from the fact that \hr{Zachirah's slave Resviel}{his \resviel{} submitted to him}. 
    That made him the ultimate alpha male, respected and envied by all. 
    \Shiaraid{} hopes that if she submits to Ramiel and lets him eat her, it will make him (in some small way) a new \Zachirah. 
\end{enumerate}

She feels she deserves to die for her wicked \sathariah{} lusts.

\citebandsong{DeathspellOmega:CrushingtheHolyTrinity}{%
  Deathspell Omega
}{
  Diabolus Absconditus
}{
  He only will grasp me aright \\
  whose heart holds a wound that is an incurable wound,\\
  Who never, for anything, in any way, would be cured of it\ldots{}\\
  And what man, if so wounded, \\
  would ever be willing to \quo{die} of any other hurt?
}

In the end, she is a little bit happy. 
Now she can be sort of reunited with \hr{Shiaraid and the ghost of Eryal}{the ghost of \Eryal}. 
Previously, \Eryal{} had been locked away in \ps{\Shiaraid} \carcer. 
Now they will both be absorbed into Ramiel's \carcer. 
This is as close as they can come to a reunion. 

She tells herself that by sacrificing herself she has atoned for her crimes against the world, against \Eryal, and even against Ramiel. 
She has done what she thinks \Eryal{} would have wished. 

\begin{prose}
  \Shiaraid:
  \tho{Ramiel is saner than I.
    In all my life I did nothing but harm with my tremendous powers, but now, in Ramiel's hand, my powers can perhaps do some good.}
\end{prose}

She is also happy to be free of the burden of her \carcer, and all the loathsome undead souls haunting her. 
Now it is up to Ramiel to carry that burden. 

\citebandsong{BeyondTwilight:FortheLoveofArtandtheMaking%
}{%
  Beyond Twilight%
}{%
  For the Love of Art and the Making%
}{
  When struck I rise\\
  I'm finally free\\
  Mark the bitter tear descending\\
  No more heavy burden\\
  And this is my eternal music
}


Her last words to Ramiel are: 
\begin{prose}
  \ta{%
    Promise me you will reunite \Mystraacht{} and make it strong and mighty again!
    You now have me inside you, and with me the last remnants of \ps{\Zachirah} bloodline.
    That makes you his true heir. 
    And our last \sathariah.
    
    You are the only one!
    \ps{\Mystraacht} future rests on your shoulders, Ramiel.
    \ps{\Zachirah}, \ps{\Nathrach} and mine. 
    We all live on in you now. 
    You are all of us. 
    You carry our hopes, dreams and legacy. 
    You are the soul and heart of \Mystraacht!}
\end{prose}








\subsubsection{\ps{\Shiaraid} tragedy} 
\hr{Shiaraid's tragedy}{\ps{\Shiaraid} tragedy} is \NexagglachelsCurse, which forces her to destroy herself and those she loves. 
For she genuinely loves Ramiel. 

When she finally perishes, it is \ps{\Nexagglachel} revenge. 
She imagines she hears him laughing inside her head, mocking him as she once mocked him in his captivity. 
She hates him, but still some part of her recognizes the justice in it. 





\subsubsection{Ramiel is sad}
\target{Ramiel blames both sides for the tragedy}
Ramiel is victorious, but his victory is bittersweet. 
He is somewhat sad. 
He loved \Shiaraid, and now he has betrayed her. 
He repents. 
He knows it was the Curse that turned them against one another. 

He tries to tell himself it is justice, because \Delphine{} \hr{Silqua dies}{did the same to her beloved \Aryal}, and \Shiaraid{} also \hr{Shiaraid betrays Zachirah}{turned her back on \Zachirah}.
But he cannot convince himself. 

The affair leaves him bitter and disillusioned, and he becomes more evil and brutal. 

He also speculates that maybe it is just and natural for him to kill and eat her. 
After all, it is inherent in the nature and purpose of the \resphain{} to be \hr{Shiaraid betrays Zachirah}{parasitic and consume one another}. 
But that doesn't make it any better. 
He concludes that both sides in the \hs{Feud} are to blame for this tragedy: 
The \banelords{} for having bred the \resphain{} into the destructive things they are, and \Nexagglachel{} and his \dragons{} for having twisted them further. 
He grows to hate both sides more. 

\citebandsong{BeyondTwilight:SectionX}{Beyond Twilight}{%
  Section X%
}{
  Blood on my fingers\\
  Night in my soul\\
  Your breath I will hear no more\\
  I'm slowly growing cold\\
  Your eyes are open\\
  Dead to the light\\
  I hear my soul cry\\
  On the wings of death I ride
}

Eating \Shiaraid{} strikes Ramiel as a hard blow, not just psychologically, but also physically and metaphysically. 
She is a humongous soul: 
\Sathariah{} and \malach, carrying behind a huge \carcer{} of her own. 
It is difficult to gulp her down. 
Her \carcer-souls attack him and haunt him. 
He has to struggle to subdue them. 

\citebandsong{BeyondTwilight:SectionX}{Beyond Twilight}{%
  Section X%
}{
  Hunting the shadows\\
  Knowing they are one\\
  Whispering voices\\
  They stalk me through the dark\\
  My eyes are open\\
  Black, all is black\\
  I am desperately cold\\
  And upon my knees I crawl
}






\subsection{Ramiel is closer to regaining himself}
Ramiel's betraying \Belzir{} and draining her power\dash perhaps even eating her flesh\dash is a key step towards regaining his full power. Remember, \hr{Resphan power and hunger}{a powerful \resphan{} needs to eat}. 

By absorbing her, he is able to \quo{inherit} some of her knowledge. 
He also absorbs her \hr{Fragments of Nexagglachel}{\Nexagglachel{} fragment}, making him \hr{Ramiel is overpowered}{\uber-powerful}. 

\lyricslimbonicart{As the Bell of Immolation Calls}{
  A black heart will adorn\\
  the wings when I'm reborn.\\
  Engraved on my memory\\
  is whom hatred made me.\\
  The ravages of time.\\
  Battles on in my mind.\\
  There are still wounds that bleed\\
  deep in the soul of mine.
  
  I behold the beginnings of sorrow\\
  and predict the omens of cruelty.\\
  In the plague's shadow I follow,\\
  as tormenting winds sweeps\\
  through the cathedral halls,\\
  as the bell of immolation calls.
  
  In embers of infernal greed,\\
  feeding the fires unholy.\\
  Apocalypse was born\\
  when hell brainstormed through me.
}

















\section{Random ideas}









\subsection{Things to remember}
Remember that the books should not be too \human-centered. 
Have more focus on the \scathaese{} societies. 
This book should have lots of focus on the Rissitics and Imetrians. 
Perhaps even with Carzain as a mere subplot. 









\subsection{The \xzaishanns{} and the deepest Chaos}
Remember to have references to the \xzaishanns{} and the deepest recesses of Chaos, from which the \dragons{} draw their power. 

Compare to the First Warren of Starvald Demelain in \cite{StevenEriksonIanCameronEsslemont:MalazanBookoftheFallen}.

Have references to \quo{the blood of the \xzaishanns} and \quo{the power of the \xzaishanns} and \quo{the legacy of the \xzaishanns}. 









\subsection{A sexy girl}
In this book or in one of the next, maybe I should have as a main character a sexy girl who regularly gets owned, dominated, humiliated and sexually harassed by other characters, villains and heroes alike. 





\subsubsection{Erotic scene with girl being killed}
Have an erotic scene with a girl being stripped naked and killed. As in the movie \emph{House of the Dead}. 










\subsection{Sentinels battle to the death}
Have a scene where Sentinels (\rachyth?), servitors of \Secherdamon{} or \Vizsherioch, fight to the death in ritual combat. This is a ceremonial contest, where the winner will be granted great power and status. 

Inspired by one of the early episodes of the tokusatsu series \emph{Go Go Sentai Boukenger}, where \Ryuuoun{} has some of his \Jaryuu{} Tribe battle to the death for the \honour of being selected and transformed into a \Daijakuryuu, a powerful monster.







\subsection{Expedition to the lost \Dragonland}
Have a short excursion to \Dragonland, the lost realm of \dragons. Perhaps \Nzessuacrith{} or another major \draconian{} character. 







\subsection{Betrayal}
I need to have a character who seems good and nice and dandy, helping out the heroes. But in the end, he betrays them, and it turns out he was evil all along. 

Compare to the priest Father Soren from the movie \cite{Movie:GargoylesRevenge}. 
At the end it becomes apparent that he was actively working to help the gargoyles awaken. 







\subsection{Someone is betrayed and imprisoned}
Someone is betrayed and taken prisoner. By whom? The Redcor? Or the Rissitics?

And who? Perhaps Carzain, but perhaps rather someone else. I have enough Carzain material already. 

Anyway, the prisoner awakens some kind of inner power, perhaps one that had previously lain dormant and hidden. He/she strikes back and causes great destruction. 

Compare to the scene in \authorseries{Robert Jordan}{Wheel of Time} where Rand al'Thor is taken prisoner and stuffed in a chest by the Aes Sedai. Or a scene in volume 2 of \FLuneNoire{} where Wismerhill is captured by a wizard who wants to drink his blood. 







\subsection{A \succubus{} shows people illusions}
\index{\succubus}%
A \hr{Succubus}{\succubus}-like monster, or group of monsters or villains, lurk in the Topaz \Chateau{} in \Redce{} and lures people in, to devour or to capture for later use.







\subsection{\Dzasselid-tachi fight for good}
\Dzasselid{} and \Narkiza{} and the rest still fight for \Nechsain, but they try as best they can to bend it in a good direction, to work for a kind of \quo{Rissitism with a \human{} face}.

Compare to lots of good guys in \cite{StevenEriksonIanCameronEsslemont:MalazanBookoftheFallen}, and maybe \authorseries{Michael Moorcock}{Elric of \Melnibone}, where Elric serves Arioch but tries to subvert him. 







\subsection{Ilcas and \Narkiza}
Telcastora Ilcas and \Narkiza{} meet. 
They criticize each other's countries and religions. 

\begin{prose}
  \Narkiza: 
  \ta{Yous sword drinks blood. (Subtext: That is evil.)}
  
  Ilcas: 
  \ta{I am a warrior. 
    I kill my enemies. But I have never killed an innocent.}
  
  \Narkiza: 
  \ta{Innocent how?}
  
  Ilcas: 
  \ta{Give it up. 
    You are trying to get me to doubt my religion and my cause. 
    That is \emph{not} going to happen.}
\end{prose}








\subsection{Someone is overly philosophical}
Some dude is being overly philosophical and sees symbolism everywhere. A wiser character (\hs{Criseis}{\Criseis}?) calls how down on it. 

\Criseis: \ta{Dude, knock it off. Understand this: The world is not full of symbols. The world is not full of meaning. The vast majority of what goes on in the world is due to chaos, chance, meaninglessness. I have lived twenty thousand years. I know this.}

Lamer: \ta{But I am \emph{giving} it meaning, here in my head!}

\Criseis: \ta{Then let it stay inside your head. And know that the world is infinitely larger than your head, and it cares nothing for how you \quo{interpret} it, how you \quo{give it meaning}.}







\subsection{Cool conversations with \Ishnaruchaefir}
\subsubsection{Immortality}
Someone comments that immortality must suck. 

\hr{Ishnaruchaefir}{\Ishnaruchaefir}, or \hr{Criseis}{\Criseis}, or another immortal: 
\ta{Hell no! If immortality sucked I would have killed myself long ago. Immortality owns ass!}

Subvert the trope \trope{WhoWantsToLiveForever}{Who Wants to Live Forever}.





\subsubsection{Peace with my past}
\Ishnaruchaefir: \ta{I have long since made peace with my past.}

\Criseis: \tho{You lie, my lord. You have never made peace. I see it, deep in your eyes. You merely hide it well and have developed skill at ignoring insults.}





\subsubsection{Fallibility}
\Ishnaruchaefir: \ta{Acknowledge my own fallibility? Ha. I think not. Perhaps in another ten thousand years.}







\subsection{A villain disses the younger races}
A Cabalist or Sentinel is accused of being evil. He does a long rant (perhaps a \trope{HannibalLecture}{Hannibal Lecture}) about how the younger races have not only inherited all of the elder races' evil, but also made it worse through stupidity. 

The elder races were and are cruel, true, but theirs is a cultured cruelty, evil with a purpose, intelligent malice. The evil of the younger races, on the other hand, is petty, vicious and, above all, senseless. Where they act there is more waste, more chaos, more senseless destruction and suffering. 

He recites many examples of how \humans{}, \scathae{} and \meccara{} are capable of horrible crimes, completely on their own initiative. This includes all kinds of people: The downtrodden Shroud slaves, the nobles who are still Shroud slaves, the enlightened who fight in the \secretwar, and those few who try to subvert the \secretwar. All of them commit terrible crimes, for what they believe is right, or for no reason at all.







\subsection{Social interaction sucks}
Someone, perhaps Carzain, or even \Ishnaruchaefir, complains about social interaction and all of the hypocrisy which is a necessary part of it. It stems from Carzain's being tired of running political games in Geica.







\subsection{Someone discusses sorcerer-kings}
Have some high-up villain comment on the subject of \hr{Sorcerer-kings}{sorcerer-kings}, and why there are not more of them. 







\subsection{\Ishnaruchaefir{} berates a whiner}
\Ishnaruchaefir{} is talking to someone who whines. He berates him: 
\ta{Get over it. 
  I have suffered more than you ever will. 
  For ten thousand years. 
  So do not come whining to me.}






\subsection{\Banes{} were not the first invaders}
Someone remarks that the \banes{} were not the first aliens to invade \Miith. 
There have been others. 
The \xss{} were one such invading force. 
They \emph{won} and ruled \Miith{} for millions of years. 
But then they grew sleepy\ldots{}







\subsection{\Ishnaruchaefir{} fights during Nadir}
At some point in the series, I need to have a big, climactic event coincide with \hr{Ishnaruchaefir's Nadir}{\ps{\Ishnaruchaefir} Nadir}. 
He is weak, but forced to fight anyway, and not through a \XanatosGambit{} of his own this time (unlike that time with \Teshrial). 

This new Nadir is much worse than the previous one (during \TwilightAngelRememberEmph). 
The \hs{Nadirs get worse}, remember.








\subsection{A \bane{} talks about viruses}
A \bane, perhaps even \Daggerrain: 
\daggerrain{%
  There exist diseases that spread (like viruses). 
  (\Daggerrain{} knows what a virus is, but the listener would not know, so he has to explain.) 
  You \humans{} are the same in the way you spread all over the world, destroying and enslaving the land around you. And yes, we \banes, too, are the same, only on a \cosmic{} scale.}

Or\ldots{} maybe the \banes{} are not the virus. Maybe everything else, such as the \dragons, is the virus, and the \banes{} are the cosmic antibody, created to scour the infestation of life from the universe. See section \ref{Banes as an antibody}. 







\subsection{\Vizsherioch}
\subsubsection{\Rathyon{} and \Vizsherioch}
\hr{Rathyon}{\Rathyon} meets \hr{Vizsherioch}{\Vizsherioch}. 

\Vizsherioch: \ta{Greetings, nephew.}

\Rathyon: \ta{\quo{Nephew}? Who are you?}

\Vizsherioch: 
\ta{You do not know me? 
  I am \Vizsherioch, son of \Secherdamon.
  I am the \xss{} given flesh!}

\Rathyon: \ta{\quo{\Vizsherioch}? That does not sound like much.}

\Vizsherioch: 
\ta{[Smile.]
  I need to litany of names. \Vizsherioch.
  Speak my name. 
  Hear it.
  Perceive its Aenigma.
  And you will understand.}





\subsubsection{\Ishnaruchaefir{} and \Vizsherioch}
\Ishnaruchaefir{} meets \Vizsherioch{} for the first time. 

\Ishnaruchaefir: 
\ta{Aaah. 
  So \emph{you} are the new \vertex{} I sensed in \ps{\Secherdamon} constellation.}







\subsection{Immortals curse the Shroud}
Some immortals curse the Shroud. 
It makes them stupid and forgetful and \hr{Shroud represses technology}{represses technology}. 














































\chapter{\RamielsAwakeningBook}
\section{Concepts}









\subsection{Ramiel and Hayad}
After Ramiel betrayed \Belzir, \hs{Hayad} remained his friend. 

But Hayad is Sentinel-allied, and Ramiel goes to join the Cabal. 
How does that work? 

I guess Hayad will die at some point, so Ramiel won't be forced to turn on his friend. 







\subsubsection{The \Bane{} Invasion}
%After defeating \Belzir{}, Carzain fights some Rissitics and some \banes{} sent by Rissit. But eventually, it becomes apparent that the \banes{} have their own agenda. 

After this, Carzain discovers that he and \Belzir{} have been manipulated by the Sentinels. He becomes angry and leaves them. 

Perhaps he travels the world for a while now, fighting for whatever cause catches his fancy. Kind of like Elric of \Melnibone{} in his mercenary years. 

Somehow, he meets the \banes. Perhaps he is taken prisoner by them. They show him how all morality is a lie, how everything in the world revolves around the struggle between \dragons{} and \banes. They convince him that, as a \Malach, he is created by the \banes{} and destined for greatness as a hero on their side. 

At this point, Carzain has encountered several different civilizations and their different moralities, and has found them all lacking\dash{}totalitarian, intolerant and contradictory. He has also read snippets of the Book of Nom, strengthening his growing suspicion that good is futile. He is near to concluding that everyone is a hypocrite, that all morals are false and that the only truth is power. 
After being shown the brutal truth by the \banes, he becomes completely disillusioned and turns to evil completely. (Perhaps it is his already high degree of disillusionment at this point that lets him keep his sanity when confronted with the truth.) 
%Carzain, completely disillusioned, abandons all his former morals and turns to evil completely. 
He abandons the name Carzain \Shireyo{} and becomes Ramiel (his \Malach{} name). 

Later, he learns that the Redcor, or at least some of them, are on the Cabal's side, and that he will have to work with them. He balks until he is reassured that he will not be put under them but beside or above them. 

%The Cabal betrays \Belzir{}, and it turns out that they actually serve the \banes. 

%Late on in the story, Carzain and a party of other heroes go on a quest to destroy some \banes. It comes to a big boss fight, but in the end, one of the heroes betrays the party and turns on them. Carzain dies. %, and the \banes conquer \Velcad{}. 

%The \banes{} conquer \Velcad{}. The Imetrians and Rissitics team up with Irokas to oppose them. I don't know what happens after that. (A nice evil ending might be to have \Tiamat{} awaken and destroy \banes{} and \dragons{} alike\prikker) 





\subsubsection{Ramiel's companions}
Maybe Ramiel has a female companion. 
Compare to Mele from \emph{\JuukenSentaiGekiranger}, with Ramiel (of course) being Rio. 

Perhaps he gathers \vorcanths{} and other creatures around him and forms his own \Rinjuuken. 









\subsection{Ramiel's enemies}
\target{Secherdamon wants to off Ramiel}
Ramiel had managed to keep his existence secret from the Sentinels for a while, but no longer. 
After he killed \Shiaraid, he could no longer hide. 

\Secherdamon{} is not happy about this new development. 
He liked \Shiaraid{} because she was so mad. 
She was dangerous to him, true, but he suspected she would be just as dangerous to the Cabal. 
But Ramiel is another matter. 
\Secherdamon{} does not know how \malach-hood and the Curse have affected Ramiel, but he remembers Ramiel as a formidable and competent adversary. 

So \Secherdamon{} decides he wants to off Ramiel. 
Throughout this book, therefore, Ramiel is hounded by Sentinels and Cabalists alike. 

Fortunately, Ramiel has allies: 
\Cishiel, \Azraid{} and the \vorcanths. 





\subsubsection{\Azraid{} helps Ramiel}
\Azraid{} \hr{Azraid protects Carzain}{has his own plans for Ramiel}, so he protects him. 
He also subtly guides Ramiel on his quest and pokes him in the direction of the things he might need to regain his power. 

In this way, \Azraid{} acts as a \trope{StealthMentor}{Stealth Mentor} to Ramiel. 

When Ramiel finally learns of \ps{\Azraid} involvement, he is amused and slightly grateful, but not surprised. 
That is the kind of thing \Azraid{} has always done. 

Later it turns out that \Azraid wants Ramiel to become a \neoresphan (\hr{Ramiel becomes Neo}{which he does}).
From \Azraid's point of view, the whole \malach project \hr{Azraid turns Malachim into Neo}{ties in with his own \neoresphan project}. 





\subsubsection{Heroes chase Ramiel}
\target{Sithiyacaan goes north}
Throughout the book, the \quo{heroes} chase Ramiel all over the world, trying to stop him from achieving his mad goal of becoming a god. 
Compare to how Cloud-tachi chase Sephiroth in \cite{VideoGame:FinalFantasyVII}. 

Telcastora Ilcas is one of these heroes.
\Sithiyacaan is another. 
\Sithiyacaan is still a madman and has most of his powers locked away. 









\subsection{Ramiel goes mad}
Ramiel's sanity is suffering.
Things are happening to his body.
Unbeknownst to him, he is \hr{Ramiel becomes Neo}{slowly mutating into a \neoresphan}. 
He has nightmares and even daymares where he feels his body transforming into something loathsome and unnatural and inhuman. 
He sees into the Beyond, even moves into it, and sees glimpses of his expanded \neoresphan body.
It scares him. 
He fears what is happening to him. 
Part of him is horrified at the though of what will happen if he goes through with his mad plan to become a god. 
Maybe he should turn back and live a quiet life as a normal \human. 

But he goes on. 
He banishes any scruples and fears he has. 
He stubbornly and fanatically continues with his mad quest to become a god, even though his own body and mind beg him to stop. 

This is a part of \quo{\hs{Ramiel's tragedy}}. 
It is a hoax to make the reader think that Ramiel is damned by his own corruption and fanaticism and doomed to fail. 
I hope to surprise the reader when Ramiel conquers and actually becomes a god. 





\subsubsection{Voices}
Ramiel heard voices in his head. 
There was one voice that was particularly loud, deep and menacing.
It lay at the bottom of it all.
Perhaps the mightiest and most dangerous of them all. 
He could not subdue it, and he could not flee from it. 
That was the voice of \Nexagglachel. 

\begin{prose}
  \Nexagglachel:
    \quo{%
      You will fail, \resphan.
      Your whole race will fall.
      You will destroy yourselves.
      You will devour yourself from within.
      You will all give penance for your crimes.}
\end{prose}
 







\subsection{The \Feud}
\target{Resphain are winning the Feud}
From the \Shrouding{} and up until this time, the \feud{} had been pretty much a war of attrition. 
And the \resphain{} were winning. 
Their numbers were dwindling, true, but the \dragons{} were dwindling faster. 
Besides, those \resphain{} who survived were the strongest, \hr{Resphain grow stronger}{and they kept growing stronger}. 







\subsection{To break or preserve the Shroud}
\target{To break or preserve the Shroud}
The \dragons{} originally created the Shroud. They did not care for progress. The \resphain{} \emph{do} care for progress, so they want to get rid of the Shroud. But they are afraid that it will give their \draconic{} foes access to more \xsic{} power, so they will not unravel the Shroud before their \matrix{} has control. 

Both \Daggerrain{} and \Secherdamon{} plan to sunder the Shroud and bring in their forces to lay waste to \Miith{} (see section \ref{Daggerrain's master plan}). For both parties, this is \hr{Summoning the XS: A deadly balance}{a delicate balance}.

Our \quo{heroes}\prikker wait, who are our heroes? \Narkiza{} and \Dzasselid? \MoroCobrel? \Cuezcans? Imetrians? \Ophidians? Remnants of the Redcor and Geicans?

Anyway, our heroes initially want to destroy the Shroud and free the people of \Miith{} from its cruel oppression and their imprisonment as witless slaves and defenseless victims, easy prey for all the horrors that lurk in the universe.

But towards the end, it becomes apparent that unravelling the Shroud is not a good idea. It will enable \Daggerrain{} or \Secherdamon{} or both to carry out their plan, and whoever wins that struggle in the end, the consequences for \Miith{}'s inhabitants will be cataclysmic. So the heroes realize that, hateful as it may be, if they wish to protect \Miith{} they will have to fight to preserve the Shroud.

The whole picture of alliances is very chaotic at this point, with factions splintering and turning on each other at the drop of a hat.

Remember to have some sad philosophy about how tragic it is that while the Shroud is a wicked abomination that keeps people down as ignorant slaves and hapless victims of the world around them, the same Shroud is still the world's best defense against all-out invasion and another apocalypse. 
The heroes have to fight to keep the world as a prison. 





\subsubsection{\Ishnaruchaefir wants to preserve the Shroud}
\target{Ishnaruchaefir fights to preserve the Shroud}
% \Ishnaruchaefir{} , but only in secret at first. Later he openly declares his opposition to \Secherdamon{} and overtly fights against him. Perhaps he becomes a leader of sorts for the opposition, like Elric in Michael Moorcock's \emph{Stormbringer}.
\Ishnaruchaefir fights against \Secherdamon.
He wants to preserve the Shroud. 
He has fought so hard and sacrificed so much to establish the Shroud and keep it alive, and he is not going to let it crumble now without a fight. 
The Shroud is a horrible, suffocating curse, but it is the world's best hope, for without it \Miith would be at the mercy of the world-devouring horrors of the Beyond. 

\Ishnaruchaefir: 
\ta{%
  A balance has reigned over the \Matrices{} of the \Feud{} for thousands of years, but it is slipping. 
  The \banes{} are gaining strength. 
  I cannot have that. 
  There are times when I have held a stalemate preferable to any end to the war. 
  But if any side is to win ascendancy, I want it to be my own people.}

\Ishnaruchaefir recruits mortal heroes to help him in his many plots\dash for though he is first and foremost renowned as a warrior, \Ishnaruchaefir can scheme with the best of them when he must. 

He negotiates with the Imetric gods and maybe recruits \hs{Telcastora Ilcas}. 

\Ishnaruchaefir is not stupid enough to put all his eggs in one basket, though.
He knows that for all his efforts, his enemies might yet succeed in destroying the Shroud.
And so he begins to \hr{Ishnaruchaefir and Azraid plot together, late in TBW}{conspire with \Azraid} and prepare the \hs{Ark} project. 





\subsubsection{Mortals fight against \iquin}
Some mortals have learned the evil purpose behind \iquin. 
They fight to stop it. 

Describe \hr{How it feels to learn Iquin is evil}{how it feels to learn that your religion is really evil}. 









\subsection{The final purpose of the \Sephiroth}
The \hr{Sephiroth}{\Sephiroth} have \hr{Final purpose of the Sephiroth}{a grand purpose}. 
It ties in with the \hs{Morbus} and the \hr{Purpose of Humanity}{purpose of \humanity}. 







\subsection{\Kezerad}
But the \hr{Kezerad}{\Kezeradi} are a danger to the \Sephirah{} project. They could potentially break the project, because they still have a \hr{Kezeradi telepathy}{deep connection to the \Sephiroth}.

By the end, some \Kezeradi{} have become main characters. Or at least one, namely \hr{Sithiyacaan}{\Sithiyacaan}, the last \Kezeradi{} prince. He and his followers work together with \Ishnaruchaefir{}\dash to some degree at least, although \Ishnaruchaefir{} was never exactly a friend of \Kezerad. 

The Cabal knows that the \Kezeradi{} have the power to fuck up their plan, so they do everything they can to hunt down the survivors.





\subsubsection[Sithiyacan]{\Sithiyacaan}
Introduce \hr{Sithiyacaan}{\Sithiyacaan}, the last \hr{Kezerad}{\Kezeradi} lord. 





\subsubsection{An undercover \Kezeradi}
A long-running \resphan, perhaps \Achsah, turns out to be an undercover \Kezeradi. Perhaps she is half \Kezeradi{} and openly condemned her heritage while secretly sympathizing with her people.









\subsection{Eschaton drawing nigh}





\subsubsection{Epidemies ravage \Azmith}
Epidemies ravage all of \Azmith, now that the \Iquin/\hr{Morbus}{\Morbus} plan is nearing its culmination. 
Entire countries are depopulated and collapse. 

It is a consequence of the life-drain inflicted by \Iquin, and also a side effect of the growing influence of the \banelords and their Entropy.
Disease is a manifestation of Entropy.

\citebandsong{Nile:AmongstheCatacombsofNephrenKa}{Nile}{
  Pestilence and Iniquity:
}{
  Heralds of Pestilence\\
  Blackest Plague Rusheth through the Land\\
  Burning Evil Winds\\
  Carry Sickness\\
  Invoking the Bitter Venom of the Gods

  Loathsome Sickening Stench of the Defiled\\
  A Cesspool Breeding the Unclean\\
  Hordes of Locusts\\
  Fiends of the South Winds\\
  Cleanse the Earth from the Impure

  The Daemon That Siezeth the Body\\
  The Daemon that Rendeth the Body

  Ruthless and Profane\\
  Lord of all Fevers and Plagues\\
  Grinning Dark Angel of the four Wings\\
  Spawn of Eng\\
  Horned God with rotting Genitalia\\
  Pazuzu
}









\subsubsection{Cataclysmic events}
\target{Eschaton: RamielsAwakeningBook}
There should be some apocalyptic events before the last book of \SentinelsofMithEmph where much of \Azmith is devastated and the truth of the \xs, \Iquin and perhaps even the \banes is revealed to the mortal masses. 

Hordes of Chaos and/or Darkness invade \Azmith, like in the \emph{Warhammer} game. 

The surviving mortals suffer.
They know their world has been destroyed and they are now living in a Hell on Earth.

Have scenes of madness and destruction like in \cite{HPLovecraft:TheCrawlingChaos}. 

See also the general section about the \hs{Eschaton} and the section about how \hr{The world goes mad}{the world goes mad}. 

\citebandsong{Nile:FestivalsofAtonement}{Nile}{
  Extinct
}{
  Paradise Lost\\
  Dreaming of Extinction\\
  We wander through the Walls of Sacrifice\\
  Sick winds brush against my Skin

  Power of Extinction\\
  Growth Intelligence No More\\
  Mud Rot Skeletal Earth\\
  Drown thy spirit of Kings

  Society's walls break Down\\
  \Humans Pound Down\\
  Dig I must dig Out\\
  Surviving puts all tools in Place\\
  Only peace comes in Death

  As I sleep I dream of Death\\
  Only Peace comes in Death
}





\subsubsection{Omens of the Eschaton}
Have mystic omens of the impending end of the world. 
Compare to the jade shards that rain from the sky in \cite{StevenErikson:DustofDreams}. 









\subsection{Armies of the undead}
The Sentinels manage to raise great armies of the undead. 
This is a continuation of what happened during \hr{Mephilex rises to power}{Suthis Mephilex's rise to power in \Yormis}. 
In this context, the servitors of \Thessulax struggle against those of \Secherdamon, for the two have different ideas of how to enact this \quo{zombie apocalypse}. 

The end result is that vast armies of undead\dash mortal and immortal alike, as well as beasts and monsters\dash march across the land.









\subsection{Looming confrontation between Ramiel and \Dasteron}
Throughout the book, the upcoming conflict between Ramiel and \Dasteron looms. 
It is a source of tension. 
Everyone knows it will happen and is waiting for it to erupt. 
Compare to the promised battle between Karsa Orlong and Rhulad in \cite{StevenErikson:ReapersGale}. 














\section{Ramiel's Search}





\subsection{\Vorcanths{} are mad at Ramiel}
The \hr{Vorcanth}{\vorcanths}, previously Ramiel's allies, had grown mad at him. 
They liked \Shiaraid{} and were nonplussed that Ramiel had backstabbed and killed her. 
They demanded he make it up to them and do them some favours before they would help him again. 





\subsection{Ramiel searches moons for answers}
Ramiel searches the moons, hoping to find answers and peace from his inner Chaos (and \hr{Curse}{\NexagglachelsCurse}). 
Especially \hs{Visha}, the moon associated with the \hr{Vorcanth}{\vorcanths}. 

\lyricslimbonicart{Enthralled by the Shrine of Silence}{
  Cold jewel Moon, I found shelter in your shade \\
  as wind and time took me astray. \\
  I'm floating down the river of forgotten names \\
  into a dark and calm water that devours my flame.\\
  There I find rest and glorious peace, \\
  a sanctuary of eternal bliss.
}

\lyricslimbonicart{Seven Doors of Death}{
  Restless days and sleepless nights.\\
  Drawn in the direction of the moon.\\
  Infernal magnet to mysterious destiny,\\
  beyond the grave of doom.
}

He wants to be rid of his frail \human{} body and return to his true form. 

\lyricsbs{Hate Eternal}{I, Monarch}{
  In this being, in flesh, there can be no absolution.\\
  Therefore I must shed my skin, in a world so fabled, so false.\\
  
  In this shell, in flesh, there can be no solitude.\\
  I will not live in this facade, in a world of contradiction.\\
  
  I, Monarch, master of what shall be.\\
  I, Monarch, captor of what I seek.\\
  I, Monarch, victor of all battles.\\
  I, Monarch, sovereign of this domain.
}









\subsection{Ramiel sees through \Iquin}
Ramiel realizes the true nature of \iquin. 
He is himself a \carcer, and his growing self-understanding allows him to understand \iquin{} as well, and see it for the carcer it is. 

\citebandsong{DeathspellOmega:FromtheEntrailstotheDirt}{%
  Deathspell Omega
}{
  Mass Grave Aesthetics
}{
  The dimension of ethereal totalitarianism discloses itself \\
  And takes possession of the quintessential human soul \\
  Like a nail hammered through most tender flesh \\
  Aeons separate the one \\
  whose eyes have seen through the night of the spirit \\
  The king, the Lord of hosts, draped in terrifying magnificence \\
  From the gleaming clot of trembling vermin
  
  This is you, nourishing \\
  the grand Mass Grave Aesthetics!
}









\subsection{Psychics see portents of the end of the world}
Gradually, as \ps{\Daggerrain}{} master plan unfolds (including the spread of the \hr{Morbus}{\Morbus}), psychics (people who can see through the Shroud) all over \Miith{} see portents of the end of the world: 
Death and emptiness, \hr{Morbus}{disease} and horror, Entropy and torturous decay. 









\subsection{Ramiel and \Ishnaruchaefir}
Ramiel meets \Ishnaruchaefir. 

\begin{prose}
  \Ishnaruchaefir:
  \ta{Ramiel. I hear you destroyed \Shiaraid.}
  (He studies Ramiel's reaction.)
  \ta{You loved her.}
  
  \Ishnaruchaefir{} fingers his glaive.
  He also destroyed his beloved once, remember. 
  A wordless \quo{\trope{NotSoDifferent}{Not So Different}}-conversation passes between them. 
  They go their separate ways with a newfound respect and mutual understanding. 
\end{prose}









\subsection{Ramiel meets \Cishiel}
\target{Ramiel meets Cishiel in person}
Ramiel is re-united with his long-lost daughter, \hr{Cishiel}{\Cishiel}. 
At first he does not recognize her. 
She was a young girl when he last saw her.
Now she has grown into a mature and very powerful \resvil. 

She has searched for him. 
She wants him back and will support his bid for the throne. 
But she also resents him for abandoning her to fend for herself back when she was a defenseless little girl. 

\target{Ramiel regains guns}
She gives him back \hr{Ramiel's guns}{\Strith{} and \Currah}, the two pistols he once used to wield. 









\subsection{Ramiel alone at sea}
On his journey north, Ramiel gets separated from his companions.
He drifts alone on the sea.
He driven half mad by his terror of the sea.
He ends on a dark and slime-covered island.
Compare to \cite{HPLovecraft:Dagon} and \cite{EdwardPickmanDerby:Azathoth}.









\subsection{Ramiel journeys to \UltimaThule}
\target{Ramiel journeys to Thule}
Ramiel sails north. 
He journeys to \hr{Thule}{\UltimaThule} in the hope of regaining his memory and power. 

\UltimaThule is in a different Realm. 

His goal was a particular lost temple city in \UltimaThule that had once been a sorcerous bastion (of some race), but was now abandoned and ruined.
Perhaps this was even a \voyager citadel.

\lyricsbs{Bal-Sagoth}{%
  Starfire Burning Upon the Ice-Veiled Throne of Ultima Thule
}{
  For how long have we \travelled?\\
  The memory grows dim, lost in the cruel, searing storm-winds.\\
  With the blessings of the Elders we began our journey, \\
  beyond the great veil of shadowed glaciers.\\
  They spoke of a prophecy foretold, an ancient and glorious legacy.\\
  A quest for the realm of legendry, \\
  lost to Man since before even the star-lords descended.
}

He gazes out over the vast, black sea and the darkened, stormy sky, dimly lit by a faint, mystic, spectral luminescence, from the clouds or the northern lights. 

Compare to \bandsong{Bal-Sagoth}{Journey to the Isle of Mists (Over the Sunless Depths of Night-Dark Seas)}. 

\lyricsbalsagoth{Invocations Beyond the Outer-World Night}{
  Seeking answers to the cryptic riddles of the universe,\\
  Secrets of the blackest, most impenetrable deeps of the umbra,\\
  Wreathed in frozen shadow and ice-bound peril,\\
  Subterrene halls of horripilated wonderment\prikker
  
  Tatsumaki Maru voyage north, ever north!\\
  Cleave a path through the massing Arctic ice!\\
  Agleam with all the \colours of the aurora,\\
  Far beyond Ny Alesund lies our goal.
  
  Wreathed in frozen shadow and ice-bound peril.\\
  Agleam with all the \colours of the aurora.\\
  The portal to the tenebrous cryptic core \\
  of this world's subterrene inner sanctums.\\
  Quaere verum\prikker \\
  Sic itur ad astra!
  
  Invocations and ideograms (dreams of the Xytaxehedron?),\\
  Conjuration of the inner world's (tenebrous) denizens,\\
  And their star-spanning progenitors, \\
  spawned beyond the outer-world night.
}

He feels deep inside him that he is nearing enlightenment. 

\lyricslimbonicart{Beneath the Burial Surface}{
  Water from a thousand tears floats in streams.\\
  The feeling from a thousand years flow over me.\\
  As I once again return to the cemetery gate\\
  I hear the dismal call from the hollow grave.
}

They encounter vast, violent storms of ice and snow. 

\citeauthorbook[p.208--209]{TimCurran:Hive}{Tim Curran}{Hive}{
  Hayes could see it out there in that haunted blackness, the headlights clotted with snow thick as a fall of flower petals, thick as the dust blowing through the decayed corridors of a ghost town.
  It was more than just a Condition One storm with near-zero visibility and winds approaching a hundred miles an hour and snow falling by the bails, pushed into frozen crests and waves.
  No, this was bigger than that.
  This was every storm that had ever scraped across the Geomagnetic graveyard of that white, dead continent.
  Pacific typhoons and Atlantic hurricanes, Midwestern tornadoes and oceanic white squalls, tempests and blizzards and violent gales\prikker all of them converging there, bled dry of their force and suction and devastation, reborn at the South Pole in a screaming glacial white-out that was sculpting the rugged landscape in canopies of frost, leeching warmth, driving blood to freon, and pushing anything alive down into a polar tomb, a necropolis of black, cracking ice. 
  
  \prikker 
  
  But there were other things on the storm.
  
  Things funneling and raging in that vortex that you could only feel in your soul, things like pain and insanity and fear.
  Maybe wraiths and ghosts and all those demented minds lost in storms and whirlwinds, creeping things from beyond death or nameless evils that had never been born\prikker the gathered malignancies and earthbound toxins aof that which was \human and that which was not, writhing shadows blown from pole to pole since antiquity.
  Yes, all of that and more, the collected horrors of the race and the sheared veil of the grave, coming together at once, breathing in frost and exhaling blight, a deranged elemental sentience that howled and screeched and cackled in the shrill and broken voices of a million, a million-million lost and tormented souls. 
  
  Hayes was feeling them out there on that moaning storm-wind, enclosing the SnoCat in a frozen winding sheet.
  Death.
  Unseen, unspeakable, and unstoppable, filling its lungs with a savage whiteness and his head with a scratching black madness.
  He kept his eyes fixed on the windshield, what the headlights could show him: snow and wind and night, everything all wrapped and twined together, coming at them and drowning them in darkness.
  He kept blinking his eyes, telling himself he wasn't seeing death out there.
  Wasn't seeing spinning cloven skulls and the blowing ,rent shrouds of deathless cadavers flapping tlike high masts.
  Boiling storms of sightless eyes and ragged cornhusk figures flitting about.
  Couldn't hear them calling his name or scraping at the windows with white skeletal fingers. 
}





\subsubsection{Ramiel is wounded}
Ramiel becomes wounded and crippled. 
He gets a bad leg or a bad arm or something like that.
He needs help to travel.

He keeps over-exerting and over-pushing himself with magic.
He wears down his body, and his condition gets ever worse.
When at last he reaches the temple he looks horrible.

Ramiel is meant to look like a villain who gets what is coming to him. 
But then he triumphs.
When he regains all his \resphan power and immortality he is able to heal his body.





\subsubsection{I know you are in there somewhere}
In a fight or the like, someone tells Ramiel this: 

\begin{prose}
  Someone: 
  \ta{Carzain! I know you are in there somewhere!}
  
  Ramiel: 
  \ta{%
    Oh, no. You have gotten it mixed up. See, it was never Carzain who was \quo{in there}. Back in the day, Carzain was up-front, and I, Ramiel, was \quo{in there} within him.
  
    You cannot break through to the \quo{real me}. 
    I \emph{am} the real me.
    I am as real as I can get.}
\end{prose}







\subsection{A \resvil{} who loves and hates Ramiel}
Ramiel has an admirer and enemy. A \resvil{} who is in love with him to the point of obsession, but also hates him for his evil and his (in her eyes) betrayal of her people. Is she \Mystraacht{} or from another dynasty?

Perhaps I can merge her with Shereid. 

Have some scenes from her perspective, where she is disgusted with her own obsession. 

In the end, he probably kills her. Perhaps she willingly lets him kill her: Stripping naked for him, kneeling down to first suck his dick, and then letting him run his blade through her breast. 

Compare her to Titus Pettifer from Clive Barker's \emph{Jacqueline Ess: Her Will and Testament} (\emph{Books of Blood II} p.81-84). Or the relationship between Elric and Stormbringer in Michael Moorcock's \emph{Elric of \Melnibone} series\dash and maybe the scenes where Elric slays Zarozinia and Moonglum. 















\section{At the Temple}
\subsection{Ramiel reaches the temple}
\target{Ramiel's awakening at the temple}
\target{Ramiel's awakening in the temple}
\target{Ramiel's awakening}
Ramiel reaches the temple, far to the north. 





\subsubsection{First impressions of the temple}
It is a \Mystraacht{} citadel, but built in \ophidian{} style, since \Mystraacht{} has conspired with the \ophidians. 
It is not only a temple, but a cosmic gzateway to the Throne of the \Mystraacht{} Overlord, ensorcelled by \resphan{} and \ophidian{} power alike. 

It is frightening, slime-covered, alien. 
Compare to R'lyeh from \cite{HPLovecraft:TheCallofCthulhu} and other Cthulhu Mythos stories. 
Also compare to \cite[p.22--25]{TanakaHirofumi:TheSecretMemoiroftheMissionary}.

It is full of ancient alien machinery. 
Compare to the Krelran temple in Arellarti in \cite[p.49]{KarlEdwardWagner:Bloodstone}. 

\lyricsbs{Bal-Sagoth}{
  Starfire Burning Upon the Ice-Veiled Throne of Ultima Thule
}{
  Swathed in Moon-frosts, in icy winds our blazon flying.\\
  Iron gleaming 'neath the stars, black skies ablaze with astral fire.\\
  White wolves (like silent spirits) haunt us, ever northwards.\\
  The ice-gem leads us, glimmering.\\
  Powerful spells entwine the shrine of legendry,\\
  mighty gates of frozen splendour looming.\\
  When the moon and stars shine as one upon the snows, \\
  the ancient ice-gate opens, the prophecy is fulfilled!
}

\lyricslimbonicart{Lycanthropic Tales}{
  As the storm looms over the frozen landscape:\\
  A call from the beginning of time,\\
  out of darkness through the mist.
  
  Requiem in blood and fire.\\
  Symbols of occult desire.\\
  Through the cosmic vortex,\\
  the gate to unknown darkness.\\
  An esoteric voyage to eminent discovery.\\
  I am the deathlike shadow,\\
  architect of black mysteries.
}





\subsubsection{They fight their way in}
He and his companions fight their way to the temple, past its terrible guardians. 

\lyricsbs{Bal-Sagoth}{
  Starfire Burning Upon the Ice-Veiled Throne of Ultima Thule
}{
  Towering, ice-encrusted forms lumber forth from the freezing mist,\\
  their eyes shimmering with a fiendish, eldritch malevolance\prikker\\
  Our steel is raised against their weapons of gleaming crystal.\\
  And the virgin snow is rendered crimson by bloodshed 
  in a searing storm of slaughter.\\
  Wounded, dying, my flesh rent by weapons no human ever forged or wielded, \\
  I am beckoned forward by a strange, alluring force \\
  from beyond the veil of swirling mists\prikker
}





\subsubsection{They explore it}
He explores the mystic temple. 

\lyricslimbonicart{Infernal Phantom Kingdom}{
  After years of dormancy\\
  in a cosmic mausoleum,\\
  arcane cemetery.\\
  The soil is cursed and sour.\\
  A rotten landscape draped in horror.\\
  Evil has a way of returning.\\
  You can not hide from hell's eye.\\
  It is always burning;\\
  a way of returning.
}

\lyricsbs{Bal-Sagoth}{
  Starfire Burning Upon the Ice-Veiled Throne of Ultima Thule
}{
  Shadows, images form \\
  in the glittering rune-carved walls of this glacial chamber,\\
  aecrets frozen within the timeless vaults of eternity.\\
  The throne of the time-lost ice realm, \\
  entwined in the mantle of such searing star-born power.\\
  This frozen, aeon-cloaked seat of immortal majesty,\\
  of an empire forged long before the vast seas rose in devouring fury!
}





\subsubsection{Insight comes}
Here he regains some of his memory. 
He remembers his life as a \resphan{} lord. 
He recalls the march of the \Mystraacht{} \hr{Glorious armies}{armies}, the incounterable legions that once forced the world to its knees. 
Back in those days, the \banes{} were weakened and mostly absent after their great defeat in the \secondbanewar{}, so the \resphain{} were left to their own designs, and they fought among one another. 
\Mystraacht{} was the most powerful dynasty, but they were betrayed, their Overlord assassinated, and they sank into the mire of civil war. 
(See section \ref{Mystraacht betrayed}.)

\lyricsbs{Bal-Sagoth}{
  Starfire Burning Upon the Ice-Veiled Throne of Ultima Thule
}{
  What shimmering swords raised in combat \\
  once sang with the glorious clamour of steel on steel?\\
  What splendid banners, billowing in the icy gales, \\
  once heralded the march of these invincible silver-clad legions \\
  to the blood-swathed embrace of epic battle?\\
  The glory of untold thousands of years past\prikker \\
  this ethereal legacy of mighty Ultima Thule.\\
  The frozen eyes of immortal kings watch me\prikker such a dark splendour!
}

Perhaps Ramiel reads the story of his people on the walls, like in H.P. Lovecraft's \emph{At the Mountains of Madness}. 

\lyricsbs{Bal-Sagoth}{
  Starfire Burning Upon the Ice-Veiled Throne of Ultima Thule
}{
  These ancient carvings in a time-veiled tongue,\\ 
  etched into the primeval ice countless aeons ago, \\
  now bathed in diaphonous incandescence \\
  by this storm of lucent stellar power, \\
  their mindsearing meaning at last becomes known to me.\\
  Their cosmic secrets unfold.
}





\subsubsection{Ramiel remembers his past}
Ramiel remembers the bloody wars he fought against other \resphain{} and against \dragons{} and \ophidians. 

He remembers fragments of \Ishnaruchaefir. 
The two have a history together. 
They are enemies, but respecting enemies. 
Almost like friendly rivals. 

\lyricsbs{Bal-Sagoth}{
  Starfire Burning Upon the Ice-Veiled Throne of Ultima Thule
}{
  And then, enlightenment comes, gleaming down upon my consciousness as the bright moon gazes down upon this auroral vista\prikker From my mind is lifted an obscuring veil, a veil induced by sorcerous arts, and I realize I have been merely a vassal of another's twisted will, a pawn in a game which is entwined in treachery and malign aspirations to thresholds of great power. 
  
  Such a traitorous web has been spun! The elders of my kingdom bow in obeisance to the vile priests of Xothan'kur, and it is their diseased machinations which have urged me here, to the very heart of the far-fabled ice realm\prikker for they seek to usurp the power of the Conjunction, stealing the vast energies of the Ice-Veiled throne and absorbing them into their own leprous, undead bodies, perpetuating the adoration of their abhorrent liege for countless ages, liberating his vile will and enslaving the realms of the world\prikker
}

He invokes \banes, \xss{} and cosmic gods. 

\lyricsbs{Hate Eternal}{Beyond Redemption}{
  Awaken from below I call thee,\\
  thy master pure in darkness,\\
  the one of hate before thy paths\\
  were chose to be forsaken.\\
  From depths of time in portals,\\
  through empty eyes of light,\\
  I searched the texts for knowledge.\\
  Now I bleed the sacrifice.
  
  I am of purest power. \\
  Strengths of a thousand souls.\\
  I hear the voices of\\
  the oldest ones burning within.\\
  Reveal the truth unto me, \\
  for I'll command the worthy.\\
  March through the ruins of diminished tribes.
}

He swears to retake the throne and reclaim power. 

\lyricsbs{Hate Eternal}{Beyond Redemption}{
  I summon winds of fire.\\
  I summon winds of disease.\\
  I summon torture upon\\
  all my rivals who shall bleed.\\
  Confronting all before me\\
  to tempt what I have in store.\\
  Don't tempt my fury for\\
  the flames shall burn eternally.
}







\subsection{\Azraid{} begins stockpiling \Erebean{} power}
\target{Azraid stockpiles Erebean power}
\Azraid{} sees portents (mystic or mundane?) that the \banes{} are reeling and about to be defeated. So he begins stockpiling \Erebean{} power, draining it away from \Erebos{} and the \banelords{} and hoarding it for himself. This is effective, because \Daggerrain{} is not expecting \Azraid{} to betray him. \Azraid{} has always been one of the most loyal \resphain. 

Remember to have subtle references to \ps{\Azraid} evil hand in all \Azraid{} chapters. 









\subsection{Ilcas dies}
\target{Ilcas dies}
\target{Sithiyacaan awakens}
Telcastora Ilcas charges heroically into battle and dies.

\citebandsong{BlindGuardian:NIME}{Blind Guardian}{Time Stands Still (at the Iron Hill)}{
  He gleams like a star and the sound of his horn's\\
  Like a raging storm\\
  Proudly the high lord challenges the doom\\
  Lord of slaves he cries

  Lord of all Noldor\\
  A star in the night and a bearer of hope\\
  He rides into his glorious battle alone\\
  Farewell to the valiant warlord
}

This is the catalyst that makes \Sithiyacaan awaken. 
But not in time to save Ilcas. 
\Sithiyacaan now realizes that he has to use his full powers to save the world. 

\citebandsong{BlindGuardian:NIME}{Blind Guardian}{Time Stands Still (at the Iron Hill)}{
  Finally I've found myself in these lands\\
  Horror and madness I've seen here\\
  For what I became a king of the lost?\\
  Barren and lifeless the land lies
}









\subsection{Ramiel is still bound}
Even after his enlightenment at the \Mystraacht{} temple, Ramiel is still bound. He is not entirely free of the Shroud, still trapped in a \human{} body, and without access to his full \sathariah{} power. He has a lot of power now, but he does not completely understand it. 

And not only that: He is under the effect of spells that chain his mind. He has enemies who want to manipulate him, use his power as a weapon of their own. They don't want him to awaken completely. Ultimately, they want to use him in their plan, where he is to expend all his power and be destroyed himself. Perhaps they plan to directly usurp his power, absorbing it for themselves and destroying him\prikker

No! Better idea: They want to create a new kind of \Sephirah, a more powerful one, but under their control. They want to transform Ramiel into one such. In order for that to work, he must be awake and have access to his full \sathariah{} power\prikker but he must awaken while under their control, so that immediately upon his awakening, they can chain and spellbind him and transform him into their new \Sephirah. 

Perhaps this is connected to the destruction of the original \Sephiroth. Perhaps Ramiel is to become the first of a new generation of \Sephiroth{} that will take over once the current ones have served their purpose. 

At any rate, Ramiel learns too much at the temple. Or, rather, he has learned too much before he came to the temple, so he is able to piece together more of the big picture than they expected him to. He suspects their scheme and takes active measures to undermine them, lay traps for them, infiltrate their ranks, so that in the end, he will be stabbing \emph{them} in the back. 







\subsection{Ramiel's enemies}
Ramiel has \hr{Ramiel's enemies}{enemies among the \resphain}. 
Have a Scabandari Bloodeye-like character. 







\subsection{Ramiel is broken down and rebuilt}
\target{Ramiel's final awakening}
He has regained much of his memory, but he is still trapped in the Shroud and must break free and be reconciled with his true self before he can regain his true \resphan{} form and his \sathariah{} power. 

He undergoes several mental torture. I don't know if this is voluntary or if he's trapped and imprisoned. Anyway, his Shrouded personality must be torn apart and broken down before his true self can re-emerge. 

\Cishiel{} and \Gilchad{} help him. 

Ramiel does not like \Gilchad. 
He is an obnoxious, condescending snob. 
He thinks he is smarter than everyone else and does not give Ramiel the respect he thinks he deserves. 

During the process of Apotheosis, Ramiel gradually comes to remember all the bits of cosmic knowledge he has gleaned over his many lives, the many Aenigmata whose Gnosis he has glimpsed. 
Back then he usually understood very little of it, but now, with the full knowledge of his many lives, he is able to see this knowledge from many exciting perspectives, and suddenly it all makes sense.
So immense amounts of horrid, sinister insight crash down upon his mind like a tidal wave.
With his multi-faceted experience he is able to understand much more than he otherwise would.
He attains Gnosis that no \resphan has ever held before him, save perhaps \Azraid.

And it is hard on him,
He screams in terror and anguish and has to fight a desperate and bloody battle against the many inner \daemons of his thoughts and fears to avoid losing his mind entirely.
His sanity is lashed like a vessel on a storm-wracked sea.
\hr{Ramiel is traumatized from awakening}{Afterwards he is traumatized and shaken}. 





\subsubsection{\Cishiel is worried}
\Cishiel is worried.
She fears her father has gone mad, and that his new power will make him even more mad and more dangerous.
But she knows it is too late to turn back.
She has cast in her lot with Ramiel and cannot hope to betray him, even if she would\prikker which she would not.
He is still her father and she loves him. 
He saved her from the horrible fate that befell him.
She knows that had he not forbidden her to become a \malach herself, she would have had to undergo the same torture and madness that Ramiel has endured - or she might have been destroyed long ago, or still trapped in the body of an amnesiac human.
No, she owes everything to Ramiel and still has more to repay him.

She feels more anguish for \Dasteron, her other ally, for she fears Ramiel will destroy him.
Ramiel looks terrible in his fury (when he \hr{Ramiel kills a Resphan after awakening}{eats a \resphan}), and \Cishiel fears no one will be able to stand against him, not even the formidable and resourceful \Dasteron.





\subsubsection{Voyage into inner worlds}
Perhaps the awakening process should be presented as a dream-quest, a journey into imaginary worlds that exist/are formed within his mind. 

Compare to Elric's journey beyond the Shadow-Gate in \cite{MichaelMoorcock:ElricofMelnibone}. 





\subsubsection{Facing his fears}
He is confronted with all his fears, sorrows, regrets, shame and pain. He must face, learn and accept the entirety of the brutal truth of his people's cruel nature and monstrous origins, the nature of \humans{} as a slave race, and his own history as a cruel conqueror and betrayer. 

Compare with the \emph{thetalos} cave in \authorbook{Jacqueline Carey}{Kushiel's Chosen} or Chia's caves in \authorbook{Stephen Marley}{Shadow Sisters}. 

Before he begins he promises himself that he will win. 

\citebandsong{BeyondTwilight:SectionX}{Beyond Twilight}{%
  The Path of Darkness%
}{
  I smell the gasoline\\
  I smell the fire\\
  I'll make it all so clear\\
  This is my mentor
}

\citebandsong{BeyondTwilight:LurkingFantasia}{Beyond Twilight}{Rage}{
  Trying to face my sanity. \\
  But I'm too afraid to confront with it. \\
  Reaching out. \\
  For what I do not know.\\
  There's no way to explain this Hell I'm in. \\
  Can you? Can you? 
  
  No source of power to take from and wake \\
  from this nightmarish lunacy \\
  Is there a meaning or is this \\
  just a test of destiny? \\
  Creating these monsters for self purposed riches \\
  and glory in the end \\
  Thousands of victims with fresh schizophrenia \\
  isn't out here to pretend.
}

\citebandsong{BeyondTwilight:TheDevilsHallofFame}{Beyond Twilight}{%
  Godless and Wicked%
}{
  Parts of my memory have been erased.\\
  You've destroyed my files\\
  Oh but I'm getting nearer\\
  It's all getting clearer 
  
  Look for a passage way\\
  Wait for the judgement day\\
  The truth will be told\\
  And my dreams unfold \\
  I'll find a way 
  
  I'm half alive here\\
  But I have no fear\\
  Cause I'll find the light yeah 
  
  Time to changes\\
  Everything re-arranges
  
  I'm getting ready\\
  At the break of dawn \\
  The fight is on 
  
  Godless and wicked\\
  Creepy and cold
}

With all his might he clings to the good memories of grandeur. 

\citebandsong{BeyondTwilight:SectionX}{Beyond Twilight}{%
  The Path of Darkness%
}{
  Send yourself to get back night\\
  I see women crawl\\
  It's dusty and mid in mind\\
  This is my intention
}

He wants to prove that he is the boss. 

\citebandsong{BeyondTwilight:SectionX}{Beyond Twilight}{%
  The Path of Darkness%
}{
  I've seen me set the light\\
  This is the mentor\\
  I met cold on this ride\\
  This is the centre
}

But dangers lurk within his mind. 

\citebandsong{BeyondTwilight:SectionX}{Beyond Twilight}{%
  The Path of Darkness%
}{
  Don't let it fall\\
  You're not you, you're me\\
  Don't let it get a taste of your blood\\
  Yeah you're feeling small\\
  Don't look at me, don't look at me\\
  You're dancing with the Devil in my hall
}

He almost wanders lost in the dark corridors of his mind. 

\citebandsong{BeyondTwilight:SectionX}{Beyond Twilight}{%
  The Path of Darkness%
}{
  Now am I lost? Deep in my soul. \\
  There will be no forever.\\
  Through time I'm tossed. A wish without hope. \\
  Into the never.\\
  As time stands still, my spirit grows cold.\\
  I fall into darkness, as black as my soul.

  Now am I lost?\\
  Without hope into the never\\
  Deep in my soul
}

He holds tight. 

\citebandsong{BeyondTwilight:SectionX}{Beyond Twilight}{%
  The Path of Darkness%
}{
  Crawl for your master\\
  You know I'm in control
}

The whole process takes a long time. 

\citebandsong{BeyondTwilight:SectionX}{Beyond Twilight}{%
  The Path of Darkness%
}{
  Smell the gasoline\\
  I light the fire inside\\
  Smell the gasoline\\
  I sway all through the night
}

Finally, slowly he awakens. 

\citebandsong{BeyondTwilight:SectionX}{Beyond Twilight}{%
  Ecstasy Arise%
}{
  From dark waters I rise\\
  My once trembling shell conceals a God
  
  Power surging through the void I soar\\
  Intoxicating visions whispering more\\
  In my radiance every \colour seems to fade
  
  I speak only in thought\\
  Life is not forever nothing is forever\\
  In dark waters I find\\
  Beauty beyond oceans of time
  
  In my radiance every \colour will soon fade away\\
  In the emptiness I will no longer stay
}

It is much like \hr{Shaeeroth ritual}{the process that turns an \ophidian into a \dragon}. 
Ramiel gains insight into many horrible truths concerning his people, the \banes, their \matrices and their purpose.
This is only possible because of the wisdom he has already attained, and it makes him even wiser and stronger, but also less sane, and it brings him a pain and a despair that will continue to haunt him.

\citebandsong{Nile:Ithyphallic}{Nile}{
  Language of the Shadows
}{
  Abandon hope\\
  And I shall become free\\
  And with freedom acquire emptiness

  With the mind cleansed and empty\\
  There is the void known as despair\\
  A gateway upon an emptiness endless and vast

  In despair the language of the shadows is intelligible\\
  In madness all sounds become articulate

  Terror and despair they guide me\\
  Into nightmares that follow one upon the other\\
  Like windblown grains of sand

  [solo: Dallas]

  I have become as the wastelands\\
  Of unending nothingness\\
  Now shall the night things\\
  Fill me with their whisperings\\
  And the shadows reveal their wisdom
}





\subsubsection{My Name Is Legion (For We Are Many)}
This chapter (or perhaps the whole part or even the book) should be named \quo{My Name Is Legion (For We Are Many)}. 
This represents the fact that Ramiel fights his legion of inner demons.





\subsubsection{Fights his old victims}
Maybe Carzain has a dream where he faces all the people he has killed and/or betrayed. 
These are souls conjured from his \carcer.
He has to fight them all. 

Compare to \cite[p.85--86]{RobertEHoward:Ghor}, where Ghor faces all the people he has killed. 





\subsubsection{Regains his memory}
Ramiel regains all memories of all his lives. 
They come crashing back to him like a torrent. 

\target{Ramiel accepts Vizicar's death}
Among other things, he realizes that he has been deluding himself about \hr{Vizicar dies}{Vizicar's death}. 
There were no evil sea gods out to get him. 
It was just a regular accident. 





\subsubsection{Triumphant}
His traumas, his emotions and all his personality as Carzain, Vizicar and others is broken down. 
He is reconciled with his true nature as a \sathariah{} and his \malach{} \hr{Malachim bind souls}{power of binding souls} (and \hr{Ramiel binding souls}{his \carcer}).  
He embraces the enslaved souls and uses them and their power. 
He becomes the true Ramiel. 

He says to \Cishiel:
\ta{Bring me my sword and my guns.}
She swiftly obeys. 

\lyricsbs{Bal-Sagoth}{
  Starfire Burning Upon the Ice-Veiled Throne of Ultima Thule
}{
  And so, enrob'd by tendrils of starfire and the raiments of lunar mist, the immortal liege whose sceptred empire is eternity sits enthroned and brooding over his dark realm once more.
  
  Swathed in moon-frosts, in icy winds our blazon flying. \\
  When the moon and stars shine as one upon the snows, \\
  the ice-gate opens, the prophecy is fulfilled!
}

He tears asunder his \human{} guise and strides forth from the Shroud into true Reality, resplendent in all of his \sathariah{} glory. 

He summons his faithful servants, which are wolf-like creatures. See section \ref{Ramiel's wolves}. 





\subsubsection{Stronger than before}
When he awakens, \hr{Ramiel is wiser from walking the earth}{Ramiel is stronger and wiser than ever before}. 





\subsubsection{Becomes a \neoresphan}
\target{Ramiel becomes Neo}
Ramiel learns that in his long process of Apotheosis and awakening, he has not only returned to his old self. 
He has gradually metamorphosed (fully or partially) into a \neoresphan.
He is not only a beautiful and mighty \resphan. 
He is now a gruesome, slimy, loathsome, abhorrent, inhuman monster. 
This is a horrible truth. 
\Azraid has long been manipulating him, and now Ramiel's tranformation into a \neoresphan is complete. 

The temple has played a big part in this.
It was restored by \Azraid with new machines and spells designed to do exactly this:
Transform a \malach into a \neoresphan. 

This is \hr{Azraid turns Malachim into Neo}{what \Azraid planned all along}. 
Now Ramiel can be inducted into the \hr{Neo-Resphan conspiracy}{\neoresphan conspiracy}. 










\subsection{Kills a \resphan}
\target{Ramiel kills a Resphan after awakening}
Ramiel wants to test his new powers. 
As a full-fledged \malach he has \hr{Malachim binding souls}{enhanced soul-eating powers}. 
He tests it by attacking an eating a \resphan. 

Ramiel zaps his foe with dark lightning. 
Then shoots him with his pistols.
Then blasts him repeatedly with sorcery and lightning.
Gradually he tears his foe apart and, turning himself into a devouring, hungry vortex, sucks in the victim and devours him body and soul.
It is like witnessing a black hole and its accretion disc.





\subsubsection{Kills \Gilchad}
\target{Ramiel kills Gilchad}
Perhaps the first \resphan that Ramiel kills is \Gilchad. 

\Gilchad{} was responsible for guiding him through the torture process. 
When Ramiel finally awakens and emerges, he tells him: 
\ta{I owe you thanks for helping me regain my memory. 
  But I can tell by looking at you that you enjoyed the process. 
  So, with that in mind, I also owe you \emph{this}\prikker} 

And then Ramiel kills \Gilchad. 
And eats him. 

He had wanted to do this for a while, but he needed him. 
Now, for the first time ever in his life, Ramiel has full control over his \Malach{} powers and the associated \carcer. 
This means that he can eat (weak) souls far more easily than any normal \resphan{} or \dragon. 

He also wants to exercise and test that power. 
That is part of the reason for killing \Gilchad. 















\section[Return to Mystraacht]{Return to \Mystraacht: The Host Reborn}
\target{Ramiel returns to Mystraacht}
Ramiel returns to \Mystraacht. 

\citebandsong{BeyondTwilight:FortheLoveofArtandtheMaking%
}{%
  Beyond Twilight%
}{%
  For the Love of Art and the Making%
}{
  He is the prince of darkness\\
  Returning home to take his kingdom
}

Ramiel is now \hr{Ramiel is overpowered}{immensely powerful}.
When he comes back, others \hr{Ramiel is underestimated after Apotheosis}{underestimate his strength}. 





\subsection{Ramiel and \Dasteron}





\subsubsection{Ramiel hears of \Dasteron}
At first, Ramiel's motivation was just to get his powers and memory back. 

Now that he has achieved that, he spends many days doing research, learning everything about the state of the world from records and from \Cishiel's Cabalist contacts.
He covets power and greatness and feels it is his right by succession to become Overlord.

But not only that.
He also doubts \Dasteron's ability.
He returns to \Mystraacht and studies \Dasteron in person.





\subsubsection{Ramiel gets to know \Dasteron}
Ramiel goes to \Mystraacht and meets \Dasteron.
Ramiel hangs out for a while and learns how thing work nowadays.
He greets \Dasteron with respect (of sorts), but no subservience.
\Dasteron reciprocates. (Ramiel is, after all, a \sathariah, and thus in a sense \Dasteron's superior.)
Ramiel comes to \Mystraacht intent on hating \Dasteron, this pathetic usurper who thinks he can be Overlord.
But Ramiel finds, much to his chagrin, that \Dasteron is a good \resphan, a worthy leader and even a potential friend.
This complicates matters.





\subsubsection{Ramiel must dethrone \Dasteron}
Ramiel learns that \hr{Dasteron cannot become Apex}{\Dasteron{} lacks the \vertex{} strength to ascend to the position of \apex{} of the \Mystraacht{} \matrix}. 
This is no good. 
It is one of the reasons why Ramiel knows he has to dethrone \Dasteron. 
He might be a good leader for a while, but if he cannot become \apex{}, his reign is stillborn and without true prospect. 

Ramiel has learned from his travels, and from the insight he has gathered over his many lifetimes, that something big is coming. 
Some cataclysmic metaphysical event. 
And the \Mystraacht{} \matrix{} has to be strong to deal with it. 
Therefore, a true \apex{} is needed. 
Ramiel feels this \apex{} should be him. 

He becomes convinced that, while \Dasteron is a good leader, he is not good enough.
He is a skilled politician and fighter, but he does not have the cosmic insight or \vertex strenght that \Mystraacht needs.
He finds out that \Dasteron shies away from all the darker writings and fears dealing with the \banes.
\Dasteron tends to obey the \banes quickly and with great fear.
\Dasteron is fearless when it comes to his fellow \resphain, but Ramiel fears \Dasteron is not capable of managing \resphan affairs in a larger cosmos.

This makes up Ramiel's mind.
He must dethrone \Dasteron and take the throne himself.
Only he is strong enough to rule \Mystraacht. 

Conversely, \Dasteron finds that Ramiel acts hysterical and manic and irrational.
\Dasteron concludes that Ramiel is insane and unfit to rule. 





\subsubsection{Ramiel laments having to kill \Dasteron}
There is a quiet understanding between Ramiel and \Dasteron. 
They both know that each intends to murder the other rather than submit and subordinate. 

Ramiel grieves. 
He has come to see \Dasteron as almost a friend. 
Moreover, Ramiel is critical of himself. 
He has been traumatized by Daggerrain (as a First Discoverer), driven to madnes by the blood of \Nexagglachel, and has lost himself after thousands of years as an outcast \malach. 
He thinks:
\tho{Who amongs us is more insane than I, more unfit to rule? And yet there is no other who can form the \apex.}

Ramiel also knows that however good \Dasteron is, Ramiel has to dethrone him.
\Dasteron knows this as well.
Sooner or later, Ramiel will challenge him for the throne.
Here is the real tragedy:
Ramiel knows that if he loses their match, he will not submit to \Dasteron's rule.
He will challenge him again and again until he wins, for Ramiel must win.
He also comes to know \Dasteron well enough to know that \Dasteron will do the same if Ramiel wins.

They are both full of pride.
Each believes that he is the better leader, that only he is worthy and capable of leading \Mystraacht.
\Dasteron believes that Ramiel, for all his brute force, is too insane and unstable and dangerous to have as Overlord.
So they conflict is doomed to continue forever, until one of them perishes.

So Ramiel makes a very hard decision.
He will destroy and eat \Dasteron as soon as he wins (if he can, that is). 
This is tragic.
Ramiel will miss \Dasteron.
He is a great \resphan and a friend.
It will be a great loss for \Mystraacht and for the \resphan race, and for Ramiel, and for \Cishiel.
But it must be done.










\subsection[Mystraacht has degenerated]{\Mystraacht has degenerated}
In Ramiel's absence, \Mystraacht{} has degenerated into a crowd of decadent barbarians striving for maximal evil and perversity for its own sake, and out of some misguided ideological idea that this is their true nature and the purpose of \Mystraacht. 

Compare to the Dark Eldar of \emph{Warhammer 40,000}.

Ramiel: 
\tho{%
  We of \Mystraacht{} boast of our independence and free thinking.
  But we are just as bound by tradition as the \KiriathSepher.
  We go with the flow just as much as they do.
  Just look at how I followed \Zachirah.}

Ramiel intends to bring order and restore the true \Mystraacht. 
Bring them back on track and introduce meaning in all the madness. 

Ramiel: 
\ta{%
  I am \ps{\Zachirah}{} only son. 
  I alone am heir to his legacy and his designs. 
  I alone comprehend the scope of his original vision. 
  I alone understand \ps{\Mystraacht}{} true purpose.}
\tho{%
  And I alone understand the flaws in my father's vision and know how to correct them.}

The \Mystraacht{} do a lot of eating and binding souls. 
This is the source of \hs{Ramiel's bound souls}. 

\lyricstitle{Warhammer 40,000: Codex Dark Eldar}{
  As the Dark Eldar died, the air was filled with escaping souls. 
  A roiling mass of blackness hovered on the edge of vision, the screams of spirits in eternal torment sounded on the edge of hearing.
  
  \prikker
  
  The ebbing and flowing of released souls slithered around the Dracon, drawn to her blood and fear. 
  She could feel their wispy tendrils sliding over her, probing gently into her mind. 
  She felt like screaming, but she gritted her teeth and was silent. 
  
  \prikker
  
  The Dark Eldar Lord's eyes glazed over, as he took a long, deep breath. 
  Khirareq felt the spirits around her drifting away, pulled towards the gulf inside the Lord's own soul. 
  The Lord's body twitched spasmodically as he absorbed the freed life essence of his followers. 
  As the spirits of the dead were consumed, Akhara'Keth's spasms increased and a thin dribble of saliva trickled from the corner of his slack lips. 
  With a shuddering sigh, the Lord finished and slumped back in the chair. 
  When he sat forward once more his eyes burned more brightly, his skin was less wrinkled, his hair darker with more lustre. 
  
  \prikker
  
  She stared straight back at the Lord, looking deep into the ancient pits of evil that were his eyes. 
  
  \prikker
  
  \ta{%
    You need to rule?
    What do you know of needs 
    You are young, the Thirst has a shallow hold on you. 
    I will tell you of need; a deep, unfaltering emptiness that grows larger and more demanding with every passing of the night. 
    You have heard tales of how I consume a hundred souls a day. 
    That number is but the morsel to whet my appetite. 
    A hundred times that number die every day to quench my desire, my need.
    Spirits unnumbered are distilled in agony and torture to the peak of exquisite taste to fill the chasm of my soul. 
    Do not confuse needs with ambitions.}
}









\subsection{Ramiel confronts \Dasteron}
There is an upstart, \hr{Dasteron}{\Dasteron}, who is the closest thing \Mystraacht{} has to a leader these days and thus counts as having usurped Ramiel's place. 

\target{Ramiel uses macho rhetoric against Dasteron}
Ramiel uses traditional \Mystraacht \hr{Mystraacht philosophy}{macho rhetoric}. 
But he is actually \hr{Ramiel is critical of Mystraacht ideology}{critical of the \Mystraacht ideology}. 

\Dasteron{} is taller than Ramiel, and very powerful. 
(You have to be powerful as fuck to aspire to lead \Mystraacht.)

He mocks Ramiel: 
\ta{%
  I believe you will find that things have changed in your absence, Lord Ramiel.}

Ramiel:
\ta{%
  You will find that things are about to change again following my return.}

\Dasteron{}: 
\ta{%
  You have been out of the game for thousands of years, Lord Ramiel.
  You drifted around down there playing \human, 
  while I was devouring souls and waxing strong. 
  I have even devoured \dragons.
  \Dragons, Lord Ramiel!
  Times have overtaken you.
  I am your superior now.}

Ramiel: 
\ta{%
  Then I challenge you.
  I invoke my right of Trial by Combat.}

Remember to compare Ramiel's and \ps{\Dasteron} \hr{Dasteron's appearance}{appearance}. 

The battle itself, despite what they try to show the spectators, is a grim, sad affair.
There is no hate or anger, so there is no enjoyment.
It is just a grim and gritty fight to the bitter end.





\subsubsection{Motivation}
Ramiel knows he has to prove his own worth by defeating \Dasteron. 
He must impress the \Mystraacht{} so they will accept him as their Overlord. 
He has to entertain in order to lay a foundation for his future reign. 

\citebandsong{Ihsahn:TheAdversary}{Ihsahn}{Panem et Circenses}{
  Awake, O' serpent of my heart. It is time.\\
  The sun stands high, and unfaithful crowds await thee.\\
  Redemption in their eyes and stone at hand.\\
  The arena hungers for your venom.\\
  Let the games begin.
  
  Bring in the lions. Bring in the beasts.\\
  It is time to confront the masses with their fears.\\
  A sober moment. A shred of truth.\\ 
  To gaze into an honest mirror. \\
  A disturbance of their sleep.
  
  Violent teeth and claws, untamed and fierce, \\
  reaches far and cut deep into the empty eye.\\
  It is time to let the bitter venom flow\\
  through this embodiment of emptiness.
  
  And the blood shall run free like words.\\
  And the bones shall form stairs to the future.
  
  Now, unfaithful spectator. Are you satisfied?\\
  Did you come close enough to feel the lion's breath?\\
  On day soon your shall be the sacrifice.\\
  A nameless grave of the past.
  
  Protagonist!\\
  Your time is now.
}





\subsubsection{Combat}
\target{Dasteron dies}
\target{Dasteron stronger than Ramiel}
\Dasteron{} is more skilled than Ramiel. 
\hr{Dasteron's skill}{Much more skilled}. 
He has faced down \satharioth{} before and knows how to turn an opponent's strength against him. 
Besides, \hr{Dasteron's smithing}{he is a great weaponsmith} and armed with the sword \hr{Scaleron}{\Scaleron} and many other powerful magical items. 

\ps{\Scaleron} design was inspired by \hr{Ascaril}{\Ascaril}, Ramiel's mother's sword. 
\Dasteron{} tells Ramiel this. 

Ramiel is taken unawares when \Dasteron{} displays just \emph{how} skilled he is at all three \hs{Paths}; not only \hr{Path of Light}{Light}, but \hr{Path of Ice}{Ice} and \hr{Path of Darkness}{Darkness} as well. 
This nearly overpowers Ramiel. 
The two combatants are sort of evenly matched at the Path of Light, but \Dasteron{} is far better than Ramiel at Darkness and Ice. 

One reason why \Dasteron{} is stronger than Ramiel is that he has \hr{Dasteron's upbringing}{lived his entire life as a \Mystraacht{} warrior}.  
Ramiel spent his formative years in the wussy and semi-pacifist and effeminate \Merkyrah{}, and then spend thousands of years \trope{WalkingTheEarth}{Walking the Earth} as a Scion.  

Ramiel is \hr{Ramiel is overpowered}{\uber, because he has eaten \Belzir}. 
But even so, he is hard pressed. 
He is not confident he can win. 
So he \emph{cheats}. 

The fight is to the death, but \emph{not} to destruction. 
But after Ramiel wins he breaks the rules and eats his fallen foe's soul, destroying him. 
Dark Eldar style. 

\ta{Stop him!} people around him cry. 
This was not part of the deal.
But no one is brave enough to challenge the \sathariah{} who has just splattered their leader and is now eating his soul and radiating fucking wicked-sick \sathariah{} power in all directions. 

Ramiel: 
\ta{%
  I am Overlord now. That was the prize. I can do whatever I want. That is my right. Will anyone deny me?}

Someone dares call him out on the dishonourable methods he used, both before, during and after the fight: 
\ta{You cheated!}

Ramiel:
\ta{Have you become like the \KiriathSepher{} in my absence?
  Have you become fops and cowards, slaves of etiquette?
  Ruthlessness has always been the \Mystraacht{} credo. 
  I saw my chances, so I took them. 
  Just as I saw my throne and now seize it.}





\subsubsection{Claims \Scaleron}
After slaying \Dasteron, Ramiel claims the sword \Scaleron. 
From that point he wears and uses the sword. 
It is his way of paying tribute to \Dasteron. 
He was, after all, a great and brave man and a worthy opponent. 
A true warrior of \Mystraacht. 

\Scaleron{} is a more than worthy replacement for Ramiel's old weapon, \Ascaril. 









\subsection{Ramiel claims the throne}
Ramiel lays claim to the throne of \Mystraacht. 
At this time he is not only the son of the first and greatest Overlord, but also the last surviving \sathariah{} in \Mystraacht{} (\Shiaraid{}, whom he killed, was the penultimate one). 
So, try as they might, no one can can really deny that he is the best candidate for the throne. 

\begin{prose}
Critic: 
\ta{What gives you the right to claim Overlordship?}

Ramiel: 
\ta{I am Ramiel.} 

(He could list off titles, lineage and accomplishments: 
One of the original discoverers of \Semiza, \sathariah, co-founder of \Mystraacht, sole heir of the last Overlord, and a mega-badass dude. 
But he needs say no more.) 

Critic: 
\ta{I believe you will find, Lord Ramiel, that things have changed in your absence.} 

Ramiel: 
\ta{And I believe you will find, [Critic's name], that things are about to change again.} 
\end{prose}

Perhaps he challenges his rivals by invoking his right of \hr{Mystraacht trial by combat}{trial by combat}, as is a \Mystraacht{} tradition.

\emph{\RamielsAwakeningBook} ends with Ramiel proclaiming his dominion and claiming his ancestral throne, cementing his position as the unchallenged Overlord of \Mystraacht. 

\lyricsbs{Emperor}{Moon Over Kara-Shehr}{
  Master! We ride with the storm \\
  in his name, the sire, wolves' king. \\
  Enter the power coursing \\
  through veins of the night.
  
  Power ripples through me. \\
  Armageddon's thunder will bring others to my side. \\
  The throne is mine. \\
  A blackened storm of evil. 
  
  Master! We ride with the storm to take \\
  revenge in the sky upon the one \\
  who cast thee from his side.
}

\lyricsbs{Bal-Sagoth}{
  Starfire Burning Upon the Ice-Veiled Throne of Ultima Thule
}{
  Such power! I am the Chosen. \\
  The secrets of the earth and the stars are unlocked before me.\\
  I am destined to reign forever\prikker \\
  to reign from the Ice-Veiled Throne of Ultima Thule!
}



After Ramiel has regained his full \malach powers and become Overlord, he lets himself hail as the saviour and champion of the \resphan race. 
He has mastered his \carcer and the dark powers of the \banes.
He is the ideal of all \resphain who crave power and greatness and perfection.

The new \Thanatzil, even.

\citebandsong{Nile:InTheirDarkenesShrines}{Nile}{
  Churning the Maelstrom
}{
  Hail To He Who Is In The Duat, Who Is Strong\\
  Even Before The Servants of Serpents\\
  He Gathers The Power From Every Pit of Torment\\
  From They Who Hath Burnt in Flames\\
  From Words of Power Uttered By the Darkness Itself

  Hail To He in The Pit, Who Is Strong\\
  Even Before the Terrors of The Abyss\\
  Who Gathers The Power from the Wailing And Lamentations\\
  Of The Shades Chained Therein\\
  From He Who Createth Gods \\
  From The Silence Alone
}





\subsubsection{Ramiel is sad}
After having slain \Dasteron and claimed the throne, Ramiel is very serious and determined and weighed down with responsibility.
Not at all the reckless adventurer he was in his youth.

Compare to William Adama from \cite{TV:BattlestarGalactica}.





\subsubsection{Contacted by \Daggerrain}
At the end (or perhaps at the beginning of the next book), Ramiel is contacted by \Daggerrain. 

\daggerrain{Ramiel. 
  Thou hast united \Mystraacht. 
  That is good. 
  But forget not whom thou servest.}















\section{Sentinel story thread}










\subsection{\Vizsherioch{} versus \Ishnaruchaefir}
\Vizsherioch{} is strong, powerful enough to challenge even \Ishnaruchaefir. Despite his young age. 





\subsubsection{\Vizsherioch{} becomes the Dagger}
\target{Vizsherioch becomes the Dagger}
\index{Dagger, the}%
\Vizsherioch{} finally becomes the \hs{Dagger} and takes his rightful place in the \matrix. 
His confrontation with \Ishnaruchaefir{} (and his \hs{Fulcrum}, \Rystessakhin) was important in this process. 











\subsection{\Secherdamon{} communicates with Tiamat}
\target{Rissit communicates with Tiamat}
\Secherdamon, seeking power and guidance in his war against the \banes, contacts the \firstgendragons. He feels the presence of \Tiamat. She appears as a great mass of swirling Chaos, almost too amorphous to be a person. She seems almost mindless. Compare her to Azathoth from H.P. Lovecraft's stories, such as \emph{The Dream-Quest of Unknown Kadath}.

Of course, \Tiamat{} is not mindless. But she has absorbed so much Chaos power and \xzaishannic{} essence that she has become more like \xzaishann{} than a \dragon. Her mind is so Chaotic as to be completely alien and unknowable, even to \Secherdamon, who is otherwise one of the wisest among the \dragons{} in the arts of Chaos, one of those who knows the most and has done the most research. (For instance, he\dash along with subordinates\dash created the Rissitic magic theory of the Three Worlds.)

\Secherdamon wants to bring the \xss back. 

\citeauthorbook[p.68]{VengerSatanis:CthulhuCult}{Venger Satanis}{Cthulhu Cult}{
  The Great Work has one end result: to bring the Old Ones back.
  Once They return to this reality, an apocalypse of bilious green fire will burn the foolish and the weak. 
  Blood spilt in ritual sacrifice shall consecrate the ground, opening the gateways.
  Temples built to \honour the Dark Gods shall alert the vigilant.
  Hideous visions shal refresh the dead imaginations of the faithful. 
  
  Some distant night shall see the unspeakable entities from Outside descend\prikker shall see Them break free, lowered into this world\prikker shall see Them destroy as they recreate their shuddersome paradise.
  Those among us who are strong, wise and diabolic shall enter the void and become like the Old Ones.
}





\subsubsection{\Secherdamon{} visits \Dathka}
\Secherdamon{} visits the fallen city of \hr{Dathka}{\Dathka}. 

\lyricsbs{Emperor}{Ye Entrancemperium}{
  Drawn towards these lands again.
  Seeking death and sacred soil.
  I ride the longing winds of my blackened soul,
  growing stronger once I enter my empire beyond.
  
  Emperium!
  Behold my coming.
  
  The fullmoon rise above me,
  enlightening my realm in a silvery glow.
  Yet the shadows crawl beneath my storming sky,
  guarding treasures from forbidden light.
}

He remembers \hr{Secherdamon's rise to power}{the time when he became a god}. 

\lyricsbs{Emperor}{Ye Entrancemperium}{
  I still remember,\\
  though ages ago it seems,\\
  the first time I entered the gates,\\
  the revelation of ritual death\\
  by which I became divine.\\
  Sacrifice of the life I had\\
  among the flesh of the light.
  
  And now I enter again.\\
  Even stronger, yet amazed by what I see.\\
  In ecstasy I mock the world.
  
  Suddenly I memorize,\\
  asking what I left behind.\\
  Nothing.
  
  Can I ever comprehend?\\
  Will my longing ever end?\\
  Never.
}





\subsubsection{\Secherdamon{} seeks out \KhothSell}
\Secherdamon{} also hopes to invoke \KhothSell{} and obtain her power of death to wield it against the \sephiroth{} and the \Morbus. 

\lyricsbs{Exmortem}{Death deceiver}{
  Ancient image of death.\\
  The deepest depts of horror.\\
  A diabolic master plan.\\
  Not to succumb to the damned.
  
  Mummified in slumber.\\
  Awaiting to appear.\\
  Death deceiver.\\
  Exmortem.
}







\subsection{\Ishnaruchaefir{} seeks power}
\target{Ishnaruchaefir seeks out cosmic gods}
\Ishnaruchaefir{} fears \ps{\Secherdamon} plotting. 
He plans to go up against his brother, but he is not powerful enough. 
He needs more knowledge and power. 

In order to build and release the \hs{Ark}, \Ishnaruchaefir requires the help of some \hr{Aloof Dragons}{aloof Elder \Dragons} that sleep dormant.

Maybe replace the gods in this scene with Elder \Dragons. 
Or maybe it is \Secherdamon who contacts them.

In the end, a few Elder \Dragons manage to awaken (now that the Shroud is broken) and come to \Ishnaruchaefir's aid.
Not many, though. 
Perhaps 4--6.
10 at the very most. 




\subsubsection{Navel gazing}
He gazes into himself, and into his glaive, \Triestessakhin. 

\lyricsdimmuborgir{The Insight and the Catharsis}{
  Oh, dreadful angel of mine.\\
  Enrich me with the vastness of your being.
  
  Rigid father, teach me to comprehend.\\
  I'll commit myself to understand.\\
  To be the faithful and the instrument,\\
  so that the ones with blindfold can see what I have seen.
}





\subsubsection{\KhothSell}
He seeks out \KhothSell, his \quo{mother}. 
She tells him stuff. 

\lyricsdimmuborgir{The Insight and the Catharsis}{
  What more do you need of proof?\\
  Human hands conforming clooven hooves.\\
  For I know the secrets and lies behind all truths.\\
  Knowlege is power and the power is mine.\\
  It's all mine.
}





\subsubsection{Cosmic gods}
He seeks out some \hr{Cosmic gods}{cosmic gods}, hoping to gain the power and knowledge he needs to thwart his brother. 

Their dark halls are splendid and overwhelming, but also dangerous, filled with mighty guardians that strike even the legendary \Ishnaruchaefir{} with terror. 

\lyricsbalsagoth{Invocations Beyond the Outer-World Night}{
  These stygian pitch-black vaults are filled with batrachian devils,\\
  Dire crystalline watch-dogs of the chasmed deeps,\\
  (For the gleaming jewels of truth are not without their protection\prikker)\\
  Vril-gorged adamantine fiends of the threshold,\\
  Spawn of the ersatz interior sun.
}

The cosmic gods turn him away. 

\lyricsbalsagoth{
  The Hound of Chaos Transcends the Nebulous Palisades of Z'xulth
}{
  [THEY-WHO-LURK-AND-BREED-IN-LIMBO:]\\
  Silence, godling! \\
  The fact that you stand before us is testimony to your mettle, 
  but your ambitions shall yet outreach your abilities. 
  
  We have long watched thee and your struggles with amusement, Zurra, 
  Sire of Angsaar. But this audience is now at an end. The path you 
  must traverse lies before you, and you must follow it to its 
  ultimate destination.
}

But \Ishnaruchaefir{} is sneaky, and he manages to surprise and out\manoeuvre them, if only for a moment. He successfully gleans enough of their secret knowledge to achieve what he needs. And so, \Ishnaruchaefir{} confronts some of the mightiest creatures in the cosmos, entities who could crush him underfoot with no effort at all, and he \emph{still} comes out looking like a total stud. \trope{Badass}{Badass}.  







\subsection{\Ishnaruchaefir{} laughs}
At some point, \Ishnaruchaefir{} experiences something funny. 
He laughs with sincere mirth for several seconds. 
\Criseis{} (who is with him) laughs along. 
Seeing him enjoying himself is intoxicating. 
She becomes ecstatic and blissfully happy. 
Her entire world lights up, and everything seems good and bright. 

While it lasts. 

Then he gets over it, and his usual brooding, grim, sardonic state-of-mind takes over. 
Suddenly \ps{\Criseis} world again turns dark and brutal and full of war and pain.
She understands that his mirth made him forget his gruesome burden and all his traumata and hate against the world for a brief moment. 
And his feelings rub off on her, who is closer to him than any other. 
And she is more impressionable and vulnerable than he, so it rocks her world more strongly. 





\subsection{Sentinels seeks out \voyagers}
Some Sentinels, perhaps \Ishnaruchaefir, seek out the \hr{Voyagers today}{surviving \voyagers} to gain knowledge useful against the \banes. 








\subsection{\Ishnaruchaefir{} begins to realize what he must do}
\target{Ishnaruchaefir begins to realize what he must do}
\Ishnaruchaefir{} slowly begins to realize \hr{Ishnaruchaefir realizes what he must do}{what he must do} to overcome his own mental blocks and reach the Gnosis he so desperately needs. 

\citebandsong{Ihsahn:TheAdversary}{Ihsahn}{The Pain Is Still Mine}{
  A distant cry arose\\
  from the fathomless well that is my soul.\\
  I can not hear the words,\\
  so I throw my heart in like a coin\\
  and wish that it would sink forever.
}

He realizes that \hr{Mirage Asylum symbolism}{he is hiding}. 

\citebandsong{Ihsahn:TheAdversary}{Ihsahn}{The Pain Is Still Mine}{
  A purpose, a sacrifice,\\
  or merely temptation?\\
  Is my solitude anything but a perversion\\
  of my vanity?
}

He defends/rationalizes his actions to himself. 

\citebandsong{Ihsahn:TheAdversary}{Ihsahn}{The Pain Is Still Mine}{
  I never cared for this weak inclination,\\
  this paranoid tendency to flock.\\
  And in between all the noise, all the guilt,\\
  a silence would carry my spirit away\\
  from diminishing obsessions.\\
  Away from fools and poisonous flies.\\
}

But deep down he knows his rationalizations do not cut it. 

\citebandsong{Ihsahn:TheAdversary}{Ihsahn}{The Pain Is Still Mine}{
  The birth of a dreamer.
}









\subsection{\Ishnaruchaefir{} takes a crazy risk too much}
\QuessanthIshnaruchaefir{} is known to be reckless and take crazy risks. 
In this book, he takes one crazy risk too many. 
It goes wrong. 
He fucks up and accidentally leaves a back door open so the \resphain{} can sneak into his Mirage Asylum. 







\subsection{Mirage Asylum destroyed}
\target{Mirage Asylum destroyed}
At some point \ps{\Ishnaruchaefir} \hs{Mirage Asylum} is breached, invaded by \resphain{} and \banes, overrun and destroyed.
 
It had been a living thing, but now it died. 
It bled and felt pain as it died. 
It was horrible for \Criseis to feel it die.
It was her home. 

\Ishnaruchaefir{} now has no safe hiding place. 
He cannot just skulk and let things happen. 
He is forced to take a more active role in the Feud. 
And he does. 
Oh, boy, does he. 
With a vengeance. 
The \resphain{} soon regret provoking him. 

Maybe this is part of \ps{\Azraid} \trope{XanatosGambit}{Xanatos Gambit}. 

The Asylum was protected by the Shroud. 
It was only because the Shroud was \hs{unravelling} that it was possible for the \resphain{} to breach it. 

Make sure to have plenty of story leading up to this, making it plausible that the Asylum could be breached now but not just at any point in the last seven thousand years. 

This is an omen of things to come. 
Soon, the whole world will lose its Mirage Asylum, the Shroud itself. 

Compare to where Chia loses her Black Dragon Valley in \authorbook{Stephen Marley}{Shadow Sisters}. 









\subsection{\Ishnaruchaefir resurrects and learns about \iquin}
A very heroic or very innocent mortal is tragically killed. 
\Criseis, who had formed ties to this person, is devastated. 
As an immortal, she has long experience telling \quo{good} deaths from \quo{bad} deaths, and this one is a bad one. 
She rages at the injustice of it.

Someone:
\ta{It is no use. We cannot rule over life and death.}

\Criseis:
\ta{You are right.
  You and I cannot rule over life and death.
  But I know who can.}

Have some references to \Sethicus's lore and his great mastery of the forces of the world, and to the sorcery and pacts of \KhothSell. 

\Criseis prays. 
\ta{Master \Quessanth.
  \Quessanth \Melechet \Nierzshah \Tzeorossh \Ishnaruchaefir.
  Rarely have I asked you for anything. 
  I beg you now, help me.}

\Ishnaruchaefir:
\ta{I cannot. His/her soul has already passed into the afterlife.
  I cannot reach inside \iquin.}

\Criseis (desperate, tearful):
\ta{You lie! I know you can. I have seen you defy forces as great as \iquin before.}
She yells and curses at her master and all but orders him to do this for her. 

\Ishnaruchaefir is impressed by \Criseis's passionate outburst.
\Dragons respect strength and aggression more than humility. 
\ta{Very well, \Criseis. 
  You speak the truth.
  I do owe you this boon.
  So be it.
  For you I will defy the \sephiroth.}

\Ishnaruchaefir reaches into \iquin.
It is hard work.
Very hard. 
He struggles. 
He breaks through layer upon layer of \iquin. 
Finally he finds the soul he is looking for and tears it free. 
In the process he learns much about \iquin. 

\Ishnaruchaefir is horrified.
He had known \iquin was a \dweomer powered by souls of the dead, but it is a bigger, more cosmic and more ambitious project than he had dreamed. 
He had thought \iquin existed primarily as a power source for the Vaimons. 
He realizes it has bigger potential than that.
It is much more integrated into the world and has roots that go deeper. 
It is \hr{Iquin and Noggyaleth}{connected with the burrowing of the \noggyaleth}.
He begins to guess the \Lithrim plan.









\subsection{\Ishnaruchaefir{} confronts \Secherdamon}
\Ishnaruchaefir{} confronts \Secherdamon.

Perhaps, in the distant past, \Ishnaruchaefir{} warred against \Nexagglachel{} in an attempt to bring the \dragons{} together. 
They discuss this.

\begin{prose}
  \Ishnaruchaefir: 
  \ta{That is something you never grasped.}
  
  \Secherdamon: 
  \ta{You think you know so much about me. 
      Yes, I know and understand perfectly, as I always have. 
      But your mission is misaimed (forfejlet?). 
  
      You are so arrogant. 
      You do not understand what I have become; indeed, you never understood what I was. 
      You have power, yes, and you may best me in combat, but in the world I wield power beyond your wildest dreams.%
  }
  
  \Ishnaruchaefir: 
  \ta{Unlike you, I have no wildest dreams\prikker}
  
  \Secherdamon: 
  \ta{\prikker and that, brother, is your failing. 
      It is why you have become a relic of the past. 
      The future is mine! 
      I am the future!
  
      Heh. 
      Perhaps I ought to thank you and \Nexagglachel. 
      If not for his death, I might still live in his shadow. 
      The events of those days are what have me the push to rise above what I was. 
      And today I am by far the greatest of us three.%
  }
  
  \prikker
  
  \Secherdamon: 
  \ta{%
    The Gnosis which \hr{Ishnaruchaefir refuses to tell Secherdamon Gnosis}{thou stole from me}\prikker
    I have \hr{Secherdamon gains Gnosis}{reclaimed it} by mine own craft.
    I am ready.
    My plan will come to fruition, and thy hideous work will be undone!}
\end{prose}

They fight a great duel. 
It is savage, murderous business, but cathartic. 
They slash and rip one another apart and let loose all the aggression and hate that they have pent up for one another and left to brood and fester for thousands of years. 

For once, \Ishnaruchaefir{} talks about feelings and motives. 
And he defends his actions, something he has always refused to do. 
He shouts in anger, which helps, because it is what \dragons{} do when they are being candid. 
This open anger is cathartic and brings them closer to each other (even though, between \resphain{} or \humans{} or \scathae, such anger could easily work the other way and drive them further apart). 

They learn to understand each other better, through the words and curses they speak, and through the deeper truths their bodies tell, with claws, weapons and magic. 
They understand each other's Aenigmata. 

They come a step closer to resolving the endless feud between the two brothers. 
They learn to respect each other. 

At the end, \Secherdamon{} admits that he is closer to forgiving \ps{\Ishnaruchaefir} crimes against them all. 
This is hard to say. 
Equally hard, \Ishnaruchaefir{} \emph{accepts} this step towards reconciliation. 
Previously he has responded to the idea of forgiveness with nothing but scorn. 
But this time he accepts and respects it (but wordlessly, not explicitly). 

This makes \Secherdamon{} more ready for his final mission. 
\hr{Secherdamon's sacrifice}{His sacrifice}. 

% They fight. 
% Ultimately, \Ishnaruchaefir{} slays \Secherdamon.









\subsection{\Secherdamon{} dies}
\target{Secherdamon dies}
\Secherdamon{} has two possible deaths: 





\subsubsection{Version 1: \Vizsherioch{} kills him}
Somehow, \Secherdamon{} get into combat and is mortally wounded. \Vizsherioch{} comes to him. 
\Secherdamon{} asks his son to save him, but \Vizsherioch{} instead kills him. 

First, \Vizsherioch{} holds a monologue: 
\ta{Yes, father. 
  You created me to be loyal to you. But more importantly, you created me to be loyal to our quest, our legacy.

  You created me from your own essence. I am part of you, and you are part of me. In the long run, this is a weakness. You must understand this. You made me well, but you have held too much back.
  
  You are fallen. To save you would be\prikker inefficient.
  
  I will strike you down, father, and subsume your essence into me. I will absorb your \xzaishannic{} blood, and in doing so I will become greater than any of us.
  
  I will become the ultimate \draecchonosh. I will become the ultimate heir to the \firstgendragons\prikker and to the \xzaishanns. I will become\prikker \emph{perfect}. And I swear to you this, father: I will filfull our destiny. I will scour \Miith{} of the \Erebean{} infestation and restore the glory of the \draconian{} race.}

\Secherdamon{} understands his son's vision, so he is happy when he dies. 

After this, \Vizsherioch{} becomes the ultimate god. 

He remembers the \quo{his} past as a \xzaishann{}.

\lyricsbalsagoth{Return to the Praesidium of Ys}{
  I was spawned deep beneath the Pre-Cambrian sea, \\
  the scion of a far distant sun\prikker\\
  I have traversed the endless stars, \\
  and journeyed to a myriad galaxies\prikker\\
  The dimensional gates of the multiverse are mine to voyage effortlessly beyond,\\
  cosmic infinity is naught to one such as I\prikker \\
  I am as one with celestial eternity\prikker\\
  Clad in gleaming pentlandite \armour, \\
  on a whim I may reshape entire worlds,\\
  or extinguish the blazing light of a sun\prikker \\
  and I remain forever enchanted by sylphs\prikker
}

He has memories of the \xzaishanns{} beyond anything \Secherdamon{} and even the \firstgendragons{} had achieved, and he also carries within him the stolen essence of the \banelords. 

\lyricsbalsagoth{Return to the Praesidium of Ys}{
  Wielding this Power Cosmic,\\
  the Omniverse is mine to conquer.\\
  I am a god.
  
  Arcane power lances forth from my fingertips,\\
  life withers before my baleful gaze.
}

He remembers the whole civilizations that the \xzaishanns{} once destroyed.

Have some evil flashbacks with the \xzaishanns{} destroying civilizations. 

\lyricsbalsagoth{Return to the Praesidium of Ys}{
  The proud citadels of great antediluvian empires\\
  have been razed to the ground by my zircon blade.
  
  Riding the screaming crest of fettered ions,\\
  I shall bring my crystalline Chaos where order reigns.
}

He looks down at his father's corpse. 
\ta{I'm sorry, father, but I was not telling the whole truth. 
  I am not driven by our legacy or quest, but by the Will to Power.} 
Compare with Friedrich Nietzsche's philosophy.





\subsubsection{Version 2: Heroic sacrifice}
\target{Secherdamon's sacrifice}
\ps{\Secherdamon} death is a heroic sacrifice for his people. 
It has cosmos-spanning metaphysical ramifications and ripples. 

Compare to Anomander Rake's sacrifice in \cite{StevenErikson:TolltheHounds}. 















\section{Random scenes}









\subsection{The power of love}
A \human{} talks to a \resphan{} about how love is the greatest, most important thing in the world. 

\Resphan: \ta{Ah, yes. \quo{Love}. Insertion of dick into cunt, a splash of liquid, followed by cuddling. Believe me, I know what love is. I and my fellows were the ones that created it. We designed your people, programmed your obsession with sex and sexual bonding. Because that it what we need you for. Your \human{} \quo{love} is a pawn on our game, and your idolization of it is a keystone to our \hr{Daggerrain's master plan}{Master Plan}.}









\subsection{The master races did not infiltrate}
A Cabalist or Sentinel talks to a mortal about how his people supposedly infiltrated the kingdoms of mortals.

Big guy: 
\ta{%
  Ha! We never \quo{infiltrated} your world. We \emph{created} it! Your civilization is and was ever our tool. We never conquered you or couped you out. You were created, born and bred as our slaves.}









\subsection{\Criseis{} apologizes for \Ishnaruchaefir}
\Ishnaruchaefir{} has hurt some people. 
\Criseis{} comes in secret and of her own initiative to apologize and ask forgiveness on her master's behalf. 
\hr{Ishnaruchaefir never apologizes}{\Ishnaruchaefir{} never apologizes}. 































\chapter{\ThirdBanewarBook}
\target{Third Banewar}
\target{TBW}
This book details what would later be retroactively called \thirdbanewar, the final showdown with the \banes. 















\section{Overview}









\subsection{The Unravelling brings chaos}
\target{Eschaton: ThirdBanewarBook}
The \hs{Unravelling brings chaos}. 
The world is slowly going mad. 
It will get much worse\prikker

See also the general section about the \hs{Eschaton}. 





\subsubsection{Ghosts escape}
Now, as the \hr{Sephirah plan}{\sephirah{} plan} is approaching fruition, the \carcer{} of \Iquin{} is filled to the bursting. 
Due to the \hs{unravelling}, \Iquin{} has grown unstable and leaking. 
Ghosts seep out and escape into the world, in droves. 

Compare to the ghosts and the Spectres in \cite{PhillipPullman:HisDarkMaterials}. 









\subsection{The world goes mad}
\target{The world goes mad}
Have a longer story where the world goes mad. 
More people discover the truth about \iquin, but by then it is too late, and their souls are too deeply bound to \iquin to break free. 

They also discover that there are even older and more malevolent things in the world than \iquin.
Things perhaps even more inimical to them.
So that even if \iquin is a parasitic soul prison, it is perhaps better than the alternative. 

The end of the world is heralded by earthquakes and floods and other natural catastrophes. 

Compare to the movie \cite{Movie:IntheMouthofMadness}, and \cite{RobertBloch:StrangeEons} and \cite{RPG:CallofCthulhu:EndTime}. 

Have scenes of chaos as in \cite{TimCurran:TheSpawning}.

See also the general section about the \hs{Eschaton} and the section about \hr{Eschaton: RamielsAwakeningBook}{the cataclysmic events} that happen at this time. 

\citeauthorbook[\quo{First Thought in Three Forms}, p.86--100]{%
  BentleyLayton:TheGnosticScriptures%
}{%
  Bentley Layton%
}{%
  The Gnostic Scriptures%
}{
  Now, when the great authorities knew that the time for fulfillment had come\dash as when labor pains are felt by a lying-in woman\dash and that it was near the door, and that just so, destruction had drawn night, all the elements together shook.
  And the foundations of Hades and the ceilings of chaos moved.
  A great fire broke out in their midsts.
  And the rocky cliffs and the earth moved as a reef is moved by the wind. 
}

Seek inspiration in the Book of Revelation in the \emph{Bible} and other religious eschatology. 

\lyricsbs{Slayer}{Raining Blood}{Raining blood from a lacerated sky.\\
  Bleeding its horror.\\
  Creating my structure.\\
  Now I shall reign in blood!
}

\lyricstitle{\cite{HPLovecraft:TheCallofCthulhu}}{
  The most merciful thing in the world, I think, is the inability of the human mind to correlate all its contents. We live on a placid island of ignorance in the midst of black seas of infinity, and it was not meant that we should voyage far. The sciences, each straining in its own direction, have hitherto harmed us little; but some day the piecing together of dissociated knowledge will open up such terrifying vistas of reality, and of our frightful position therein, that we shall either go mad from the revelation or flee from the deadly light into the peace and safety of a new dark age.
  
  [\prikker]
  
  That cult would never die till the stars came right again, and the secret priests would take great Cthulhu from His tomb to revive His subjects and resume His rule of earth. The time would be easy to know, for then mankind would have become as the Great Old Ones; free and wild and beyond good and evil, with laws and morals thrown aside and all men shouting and killing and revelling in joy. Then the liberated Old Ones would teach them new ways to shout and kill and revel and enjoy themselves, and all the earth would flame with a holocaust of ecstasy and freedom.
}

\lyricsauthorbookpage{Graham McNeill}{False Gods}{157}{
  I saw beyond and into the warp. I saw the powers that dwell there\prikker 
  
  There is great evil in the warp and I need you to know the truth of Chaos before the galaxy is condemned to the fate that awaits it. 
  
  I saw it, Warmaster, the galaxy as a wasteland, the Emperor dead and mankind in bondage to a nightmarish hell of bureaucracy and superstition. \\
  All is grim darkness and all is war. \\
  Only you have the power to stop this future. \\
  You must be strong, Warmaster. \\
  Never forget that.
}





\subsubsection{Storm in the afterlife}
At this time there is such a furious storm of chaos raging just outside \Miith's walls that it is shaking even the afterlife. 
All cultures are losing mortal souls at an alarming rate.
And even immortal souls are in danger of being swallowed up by the consuming emptiness outside, where hungry and mindless gods gnaw and rage and howl. 
Compare them to the dancers around Azathoth's throne in \cite{HPLovecraft:TheDreamQuestofUnknownKadath}.

The Shroud has created an unnatural pressure inside the walls of \Miith. 
Now that it is all about to come apart, the blind powers of the outer spaces are back with a vengeance. 
Immortality is no longer certain. 
The immortals suddenly know new kinds of fear. 









\subsection{\Voidbringer must be stopped}
The \dragons{} and other \bane-opponents know that if the Cabal manage to open the portal to \Erebos{} as wide as they did in the \secondbanewar, they will be able to let in millions of \bane{} warriors. 
There is no way in fucking Hell the \Miithians will be able to deal with that many.
They are fewer and weaker than they were last time. 
So they must stop the portal. 








\subsection{Aloof \dragons must be roused}
There were some \hr{Aloof Dragons}{aloof \dragons} who took no part in the \feud.

The aloof ones did not care that \hr{No new Dragons are born}{almost no new \dragons were born} since the \hr{Heart weakened}{weakening of the Heart} began.
The aloof ones were powerful and immortal and had the luxury to wait a few thousands or even tens of thousands of years between procreating. Many of them were over 20,000 years old and saw the war with the \resphain as a temporary nuisance. 

\Ishnaruchaefir and \Secherdamon were similarly old and would likely have thought the same, but the war came horribly close to them when some of their close family members (Nexagglachel and \Ishnaruchaefir's three sons) were destroyed by the \resphain.
Now, in the \thirdbanewar, \Secherdamon or \Ishnaruchaefir had to convince the aloof ones that the war was more than a nuisance and that its conclusion would have great repercussions for all \dragons, even the ones who had fled to \Machai.
Eventually some of the aloof ones became convinced. 
Not all, but enough.

The conclusion of the war WAS important.
The aloof ones could hide in \Machai, but they could not really get away. 
They were still bound by the Heart and would suffer its fate.
Either that, or they would have to evolve, abandoning their Draconic nature entirely and becoming something different.
\Ishnaruchaefir and \Secherdamon knew this might be an option\dash{}to flee to \Machai and abandon \dragonkind\dash{}but it was not a palatable one. 
They both wanted the Draconian race to survive.
\Secherdamon wanted to win the war and was willing to enlist the \xs, with all the associated costs.
\Ishnaruchaefir did not like that idea.
So he looked outward into the universe for other ideas.









\subsection{Ramiel is mad}
Ramiel tries to hide it, but \hr{Ramiel is mad after awakening}{his sanity has suffered badly}. 









\subsection{Clean \humans}
There are some \humans who are \quo{\hr{Clean Humans}{clean}}, in the sense that they have no blood of the \quo{\hs{Men of Light}} and thus do not carry \Lithrim inside them. 
These \humans still have \bane and \resphan blood, but no \Lithrim.
This means that they are unaffected by the \hs{Second Advent}. 

Some of these \humans are among the heroes. 
They help the immortal heroes.
They accomplish some minor things and manage to make a difference. 
Then they die at the end.
But they die a heroic death. 

















\section{The \Matrices Battle}









\subsection{\ps{\Daggerrain} master plan}
\hr{Daggerrain's master plan}{\ps{\Daggerrain}{} master plan}, planned and orchestrated throughout twenty thousand years, is unfolding. 

\Secherdamon{} and \Ishnaruchaefir{} oppose him. Ramiel ostensibly serves the \banes, but is actually planning yet another betrayal.

\Daggerrain{} intends to sunder the wall to \Erebos{} and allow the \Voidbringer{} unrestricted access to \Miith{}. 

But \hr{Daggerrain's blind spot}{he has a blind spot}\prikker

\Secherdamon, on the other hand, wants to open the gate to Chaos and awaken \Tiamat{} and the \firstgendragons. He has communicated with \KhothSell{} or another one of them. She agrees to help him, but only if he can open the way and make it easy for them to help him. (See section \ref{Rissit communicates with Tiamat}.)

But \Iurzmacul{} plots against his fellow \firstgendragons{}. He does not want to see \Tiamat{} and the others return and devastate \Miith{}. Maybe he tells \Ishnaruchaefir{} and guides him. Maybe \Ishnaruchaefir{} is the son of \Iurzmacul.





\subsubsection{\Dragons fear \Voidbringer}
The \dragons{} know that if \Voidbringer{} gains access to \Miith, they will not stand a chance against the feared \baneking. 
If that happens their only recourse will be to follow \ps{\Vizsherioch} lead and appeal to the \xss. 

\Vizsherioch{} will not accept \Ishnaruchaefir{} as an ally. 
He would prefer to kill him. 

\Vizsherioch{} \emph{wants} to see the \banes{} succeed and \Voidbringer{} brought into the world, because then all of \Miith{} will be forced to turn to him as their only saviour. 
He cannot summon the \xss{} alone, but with enough \dragons{} and \quiljaaran{} and other immortals on his side, he can pull it off. 

He relishes the thought of a war between the \xss{} and the \banelords. 
He is willing to risk the destruction of all of \Miith. 
He kind of thinks like a \xss, remember. 
Plus \hr{Immortal Vizsherioch}{he is more immortal than most}. 









\subsection{Ramiel solidifies his power}
I should let a few years pass between the end of the last book and the beginning of (the serious part of) this one. 
Ramiel needs some time to consolidate his rule as Overlord. 
Not everyone swoons and falls down to worship at his feet. 

And even after some years, he still has rivals and has trouble controlling the dynasty. 

\target{Dasteron's cousins oppose Ramiel}
Among his most vehement foes are \hr{Dasteron's cousins}{\ps{\Dasteron} cousins}, \Sargamel and \Themirod. 

\target{Ramiel uses macho rhetoric as Overlord}
Ramiel uses traditional \Mystraacht \hr{Mystraacht philosophy}{macho rhetoric}. 
But he is actually \hr{Ramiel is critical of Mystraacht ideology}{critical of the \Mystraacht ideology}. 

After his return to Overlordship, Ramiel broods over the doings of the \resphan race.
He is disgusted by their stupidity and hypocrisy. 
The \Mystraacht curse the \CiriathSepher for being shallow and dishonest, but the new, wiser Ramiel can now see that the \Mystraacht are no better themselves, with all their shallow machismo and their hollow talk of \honour and bravery and whatnot.

He thinks: 
\tho{There are times when I wish the \banelords would just come and devour everything. Or whatever it is they plan to do.}

Even among the highest tiers of the \resphain who serve them, there is no one who really knows what the \banelords intend to do.
No \resphan understands the mind of a \bane.









\subsection{\Dragons summon \xss}
\Secherdamon-tachi want to awaken the \xss and summon them to \Miith to fight against the \bane menace. 
Maybe this is mostly intended to slow down the \banes. 
The \dragons know they cannot rely on the \xss as allies.
The \xss might decide it is not worth the effort and pull out of \Miith.
But if the \dragons can at least get the \xss to hold off the \banes, they have a chance. 

The \banes have a thousand-year master plan.
The \xss have little interest in \Miith.
They are likely to give up and go home. 
They will not commit many resources to \Miith. 
It is unlikely that they will save it in the long run, but they may save it in the short run. 





\subsubsection{\Vizsherioch seeks out \daemons}
\Vizsherioch{} seeks out some great and mighty \daemons{} who once served the \xzaishanns{} but have lain dormant for thousands if not millions of years. 

He demands that they serve him. They laugh and reject him, for they have been approached by \dragons{} before. But \Vizsherioch{} is more \xzaishann{} than any \dragon{} ever before. When the \daemons{} realize that, they are impressed and overwhelmed. They agree to serve him. 

He also seeks out \hr{Cosmic gods}{cosmic gods}\dash the same ones that \hr{Ishnaruchaefir seeks out cosmic gods}{\Ishnaruchaefir{} sought out}. But they turn him down. He claims that he is a \xzaishann{} reincarnated, and they laugh.

\lyricsbalsagoth{
  The Hound of Chaos Transcends the Nebulous Palisades of Z'xulth
}{
  The Outer Darkness has cast me from its cavernous maw! 
  So be it\prikker the final pieces of this great macrocosmic puzzle shall soon fall into place. 
  It has been millennia since I last visited my beloved brother, Zuranthus\prikker 
  and this reunion shall most certainly be steeped in ire! 
  
  The annals of eternity shall soon be illumined by the glorious radiance of my deeds, and the searing fire of my long denied vengeance! Woe unto my dogged nemesis and all those who dare aspire to thwart me\prikker for my final ascendence is nigh!
}







\subsection{\Ishnaruchaefir invokes dark gods}
\Ishnaruchaefir{} is forced to invoke various dark gods to seek their power and aid. Some of the same forces that \Secherdamon{} originally invoked to gain power (see section \ref{Rissit invokes dark gods}). 

Perhaps these powers originally helped effect the \hr{Shrouding}{\SecondShrouding}, and now \Ishnaruchaefir{} needs their power to defend the Shroud (see section \ref{To break or preserve the Shroud}).

In the end, a few Elder \Dragons manage to awaken (now that the Shroud is broken) and come to \Ishnaruchaefir's aid.









\subsection{\Iquinian Church preaches}
The \Iquinian Church preaches extra hard. 
The advent of \Lithrim{} is nigh, and the Cabalists want to be sure all the myriad Iquinians are kept under tight control. 

\citebandsong{DeathspellOmega:FasIteMaledictiinIgnemAeternum}{%
  Deathspell Omega
}{
  A Chore for the Lost
}{
  Every human being not going to the extreme limit \\
  is the servant or the enemy of man \\
  and the accomplice of a nameless obscenity.
}







\subsection{Ramiel accepts \NexagglachelsCurse}
Ramiel is seduced by the blood of \Nexagglachel{} in his veins. 
He is swayed by the \ps{\dragonlord} voice and ultimately agrees to betray the \banelords, in return for peace of mind from the \hr{Madness of the Curse}{madness and dementia that the curse brings}.

\lyricslimbonicart{The Yawning Abyss of Madness}{
  The winds that carry this esoteric call\\
  emerges from the dungeons underneath my soul. \\
  As I cross the bridge to that darkness, \\
  my eyes are filled with so much death.
}







\subsection{\Ishnaruchaefir hears \Nexagglachel would be proud}
\target{Ishnaruchaefir hears Nexagglachel would be proud}
\Ishnaruchaefir{} meets some \sathariah. 
Possibly Ramiel or \Azraid. 
(Maybe when \hr{Ishnaruchaefir and Azraid plot together, late in TBW}{\Azraid{} conspires with \Ishnaruchaefir}.) 

\begin{prose}
  The \sathariah: 
  \ta{I have a piece of your brother inside of me, you know. 
    I know something of his thoughts. 
    And you know what? 
    I think if \Nexagglachel{} knew what you had done, he would be pleased.}
\end{prose}

The above is not well-meant. 
The \sathariah{} hates \Nexagglachel, because \Nexagglachel{} hates him and drives him mad and ruins his life. 
But it is sincere. 

\Ishnaruchaefir{} takes it as a confirmation that he is on the right track. 
This makes him incredibly relieved. 
He has spent millennia brooding over whether he is \hr{Ishnaruchaefir's stewardship}{carrying on \ps{\Nexagglachel} legacy} properly. 
When \Criseis{} next sees him, he is happier than she has seen him in thousands of years. 
He does not show it, but she, who knows him better than any other, can detect it in his smell and his aethereal threads. 
The pall of darkness that always hangs over him has lifted a little. 










\subsection{Ramiel allies with \Baelzerach}
In secret, Ramiel contacts the \Baelzerach{} and forges a hidden alliance between them and \Mystraacht. They will not openly fight side by side, but they will strike at the same enemies and covertly work to achieve the same goals. 

This is especially sneaky because no one expects it. \Mystraacht{} and \Baelzerach{} are traditionally enemies, having fought many and very bloody wars. 









\subsection{The \banes bring a spaceship to \Miith}
The \banelords{} succeed at bringing a titanic and ancient \hr{Bane technology}{\voyager{} spaceship} from \Erebos/\Nyx{} to \Miith{}. 
It will take all of the \psp{\dragons}{} \xsic{} power to destroy it.

Other people see signs of its approach.

\lyricslimbonicart{Nebulous Dawn}{
  Dark atmospheric winds\\
  are striking towards Earth.\\
  Nihilistic cosmic force.\\
  Asteroid curse.
}

The ship arrives. 
It is a massively powerful juggernaut and wreaks tremendous destruction. 

\lyricslimbonicart{Nebulous Dawn}{
  A blazing phenomenon\\
  launching through the skies.\\
  A deadly impact, tempest chaos.\\
  Utterly devoid of life.
  
  Fountains of fire rise from the lava chambers.\\
  Columns of smoke darken the sky\\
  and cover the landscape in ashes.\\
  Storms with the anatomy of disaster.\\
  Immense swirling winds.
}









\subsection{\Erebos \Matrix Ascendant}
The \bane{} \matrix{} is strong and has almost taken over the world. 
Its corrupting, destructive influence can be felt all over \Miith{}. 
The \hr{Morbus}{\Morbus} is but one of its manifestations. 

\lyricslimbonicart{Nebulous Dawn}{
  An epic eclipse of the Sun.\\
  A rebirth of desolation.\\
  When the cold silence reigns,\\
  darkness sweeps the face of Earth\\
  with infesting freezing torments.\\
  Across the desert wastelands.\\
  Cataclysmic events.
}









\subsection{\Rystessakhin is damaged}
\Ishnaruchaefir{} is in battle, wielding his \hs{glaive}. 
Suddenly it goes wrong, and the unthinkable happens: 
The glaive becomes damaged. 
One of the little blades snaps off. 

\Ishnaruchaefir{} is terrified. 
He knew the Shroud was \hs{unravelling} and it was becoming harder for \Rystessakhin{} to keep it in place. 
\hr{Ishnaruchaefir's Nadir}{His own Nadirs} had been growing steadily worse as a result. 
But had never imagined that it would come to this. 

He immediately unsummons the glaive. 
Then he spends all his available energy on fighting his way free and escaping from the fight. 

He realizes that something is catastrophically wrong. 
The Shroud is under attack, and the pressure on poor \Rystessakhin{} has become so great that even her physical form is chipping and falling apart. 
The status quo is crumbling, and he cannot hope to maintain it any longer. 
He needs to change his strategy.
Something drastic must be done. 








\subsection{\Azraid conspires with \Ishnaruchaefir}
\target{Ishnaruchaefir and Azraid plot together, late in TBW}
\Azraid{} conspires with \Ishnaruchaefir. 

They have \hr{Ishnaruchaefir and Azraid develop empathy}{talked before} and have a relationship as friendly rivals. 
They have even \hr{Ishnaruchaefir and Azraid plot together, early in TBW}{plotted together before}. 
They are both philosophers who feel somewhat estranged from their races, so they have a certain mutual understanding. 

Gradually and ever so subtly through \quo{Sentinels of \Miith} \Ishnaruchaefir becomes convinced that Azraid is on the level, and he also begins to suspect the general nature of Azraid's plan. 
So when Azraid finally approaches \Ishnaruchaefir and asks for his cooperation in a secret venture, \Ishnaruchaefir has a bit of a \quo{what took you so long?} attitude, and the whole interaction has a \quo{I know you know I know you know\prikker} frame.
The two have each formulated ideas that can be combined into a useful whole.

Now \Azraid{}, for the first time \emph{ever}, comes clean and reveals (part of) his master plan to \Ishnaruchaefir. 
\Ishnaruchaefir{} is the best person he can trust with his plan. 

\Azraid{} also needs Ramiel, but he cannot trust him.
Ramiel is too volatile, too unreliable. 
\Ishnaruchaefir{} \hr{Ishnaruchaefir's impulsiveness}{has spent all his impulsiveness long ago}, so \Azraid{} knows he is dependable.

\Ishnaruchaefir{} agrees to cooperate with his old enemy. 

Perhaps it is here \Azraid{} \hr{Ishnaruchaefir hears Nexagglachel would be proud}{tells \Ishnaruchaefir{} that \Nexagglachel{} would be proud}. 








\subsection{\Harbeth dies}
\target{Harbeth dies}
\Harbeth dies for some reason. 

Maybe \Zereth also dies. 
Most of the \hr{CS heirs die}{\CiriathSepher{} heirs die}. 








\subsection{\Ishnaruchaefir sacrifices \Criseis}
Near the end, \Ishnaruchaefir sacrifices \Criseis and commits other cruel but necessary acts. 
Make it horrifying but awesome. 
He calls upon the \xss and other terrible dark gods and prepares to sacrifice the entire world, as he once did in the \hs{Shrouding}. 

Compare to Bayaz in \cite{JoeAbercrombie:LastArgumentofKings} where he sacrifices Yulwei and breaks the First Law. 















\section{\Ishnaruchaefir's quest}
\Ishnaruchaefir has lost his Mirage Asylum. 
He realizes that if he is to truly make a difference in this war, he will need more power.
(He \hr{Ishnaruchaefir begins to realize what he must do}{began to realize this earlier}.)
He decides that he has to go on a pilgrimage to the ancient \draconian land of \Dragonland and the dreaded citadel of \hr{Baltherium}{\Baltherium}. 

Compare the whole quest to Randolph Carter's search for Kadath in \cite{HPLovecraft:TheDreamQuestofUnknownKadath}. 

Read about \hr{Nom}{\Nom} and \hr{Baltherium}{\Baltherium}.









\subsection{\Matrix}
Fortunately \Ishnaruchaefir has prepared for this. 
He has sought out allies and recruited heroes for his cause, both mortal and immortal. 
He now drafts some of these heroes.
He needs them to accompany him and serve as \vertices, for he will need a full \hr{Draconian Matrices}{\draconian \matrix} if he is to accomplish his task. 

His \matrix includes \hr{Criseis}{\Criseis} and \hs{Telcastora Ilcas}.
It was also supposed to include \hr{Najarod}{\Najarod}, but he was suddenly killed, and his youngest lover \hr{Shiin}{\Shiin} took his place. 

\Criseis was not originally supposed to come.
She begged \Ishnaruchaefir to take her along, and he relented.
But she was not a part of the \matrix.
Then something went wrong, and another \matrix member died.
\Criseis now had to take the last position, as much as \Ishnaruchaefir hated to see her take the risk. 









\subsection{Culmination}
At the end, \Ishnaruchaefir has to fight his way through the horrors of the \hs{dead universe}. 

\citeauthorbook[p.482--483]{HPLovecraft:TheDreamQuestofUnknownKadath}{\HPLovecraft}{%
  The Dream-Quest of Unknown Kadath%
}{%
  And Randolph Carter, gasping and dizzy on his hideous Shantak, shot screamingly into space toward the cold blue glare of boreal Vega; looking but once behind him at the clustered and chaotic turrets of the onyx nightmare wherein still glowed the lone lurid light of that window above the air and the clouds of earth's dreamland. Great polypous horrors slid darkly past, and unseen bat wings beat multitudinous around him, but still he clung to the unwholesome mane of that loathly and hippocephalic scaled bird. The stars danced mockingly, almost shifting now and then to form pale signs of doom that one might wonder one had not seen and feared before; and ever the winds of nether howled of vague blackness and loneliness beyond the cosmos.
  
  Then through the glittering vault ahead there fell a hush of portent, and all the winds and horrors slunk away as night things slink away before the dawn. Trembling in waves that golden wisps of nebula made weirdly visible, there rose a timid hint of far-off melody, droning in faint chords that our own universe of stars knows not. And as that music grew, the Shantak raised its ears and plunged ahead, and Carter likewise bent to catch each lovely strain. It was a song, but not the song of any voice. Night and the spheres sang it, and it was old when space and Nyarlathotep and the Other Gods were born.
  
  Faster flew the Shantak, and lower bent the rider, drunk with the marvel of strange gulfs, and whirling in the crystal coils of outer magic. Then came too late the warning of the evil one, the sardonic caution of the daemon legate who had bidden the seeker beware the madness of that song. Only to taunt had Nyarlathotep marked out the way to safety and the marvellous sunset city; only to mock had that black messenger revealed the secret of these truant gods whose steps he could so easily lead back at will. For madness and the void's wild vengeance are Nyarlathotep's only gifts to the presumptuous; and frantick though the rider strove to turn his disgusting steed, that leering, tittering Shantak coursed on impetuous and relentless, flapping its great slippery wings in malignant joy and headed for those unhallowed pits whither no dreams reach; that last amorphous blight of nether-most confusion where bubbles and blasphemes at infinity's centre the mindless daemon-sultan Azathoth, whose name no lips dare speak aloud.
  
  Unswerving and obedient to the foul legate's orders, that hellish bird plunged onward through shoals of shapeless lurkers and caperers in darkness, and vacuous herds of drifting entities that pawed and groped and groped and pawed; the nameless larvae of the Other Gods, that are like them blind and without mind, and possessed of singular hungers and thirsts
  Onward unswerving and relentless, and tittering hilariously to watch the chuckling and hysterics into which the risen song of night and the spheres had turned, that eldritch scaly monster bore its helpless rider; hurtling and shooting, cleaving the uttermost rim and spanning the outermost abysses; leaving behind the stars and the realms of matter, and darting meteor-like through stark formlessness toward those inconceivable, unlighted chambers beyond time wherein Azathoth gnaws shapeless and ravenous amidst the muffled, maddening beat of vile drums and the thin, monotonous whine of accursed flutes.
}







\subsection{\Ishnaruchaefir realizes what he must do}
\target{Ishnaruchaefir realizes what he must do}
At the end, in order to break free (like Randolph Carter on his way to Azathoth's throne) \Ishnaruchaefir will need true Gnosis of his own Aenigma. 
To do this he has to look deep inside himself and come to terms with himself. 
Face his inner demons. 
This means he must overcome a ton of traumata, which have festered inside him ever since he \hr{Ishnaruchaefir slays his beloved}{killed \Rystessakhin} and \hr{Ishnaruchaefir goes into the void}{fled into the void}. 

He must make peace with the memory of \Rystessakhin{}. 
Only then can he call upon all his strength. 
He must save himself before he can hope to save the world. 

\Achamoth fights against \Ishnaruchaefir and seems to want to stop him (like Nyarlathotep in \cite{HPLovecraft:TheDreamQuestofUnknownKadath}).
But in the end \Ishnaruchaefir suspects that he has been played for a \trope{XanatosSucker}{Xanatos Sucker} and that \Achamoth was really guiding him through the necessary struggle as a \trope{TricksterMentor}{Trickster Mentor}. 

\citebandsong{Ihsahn:TheAdversary}{Ihsahn}{Invocation}{
  As deep cuts of truth.\\
  As a fire that closes the wound.\\
  So is my redemption.
  
  Beyond repentance.\\
  This is the ordeal of fire.
}

It will be hard, expensive and bloody, for both him and the world. 
But it will be worth it. 

\citebandsong{Ihsahn:TheAdversary}{Ihsahn}{Invocation}{
  Come suffering, Apocalypse.\\
  Release the fires of Hell.\\
  I call upon destruction and despair.\\
  Not for vengeance, not for power.\\
  Beneath the ashes I walk.
}

At last he takes a deep breath and prepares to do it. 

\citebandsong{Ihsahn:TheAdversary}{Ihsahn}{The Pain Is Still Mine}{
  Behold, an angel of vengeance.\\
  A lion. A sword of fire. \\
  Alas, the burden of my heart\\
  is violence undone\\
  Pain unfulfilled.\\
  Silence.
}

Finally he does it. 

\citebandsong{Ihsahn:TheAdversary}{Ihsahn}{The Pain Is Still Mine}{
  When I finally cut \\
  deep into the flesh of guilt,\\
  the un-naked body of shame\\
  and the veins of repentance open wide,\\
  sending rivers of blood into my mouth.
  
  The pain is still mine.
}

\citeauthorbook[p.484--485]{HPLovecraft:TheDreamQuestofUnknownKadath}{\HPLovecraft}{%
  The Dream-Quest of Unknown Kadath%
}{%
  Turn\dash turn\dash blackness on every side, but Randolph Carter could turn.
  
  Thick though the rushing nightmare that clutched his senses, Randolph Carter could turn and move. He could move, and if he chose he could leap off the evil Shantak that bore him hurtlingly doomward at the orders of Nyarlathotep. He could leap off and dare those depths of night that yawned interminably down, those depths of fear whose terrors yet could not exceed the nameless doom that lurked waiting at chaos' core. He could turn and move and leap\dash{}he could\dash{}he would\dash{}he would\dash{}he would.
  
  Off that vast hippocephalic abomination leaped the doomed and desperate dreamer, and down through endless voids of sentient blackness he fell. Aeons reeled, universes died and were born again, stars became nebulae and nebulae became stars, and still Randolph Carter fell through those endless voids of sentient blackness.
  
  Then in the slow creeping course of eternity the utmost cycle of the cosmos churned itself into another futile completion, and all things became again as they were unreckoned kalpas before. Matter and light were born anew as space once had known them; and comets, suns and worlds sprang flaming into life, though nothing survived to tell that they had been and gone, been and gone, always and always, back to no first beginning.
  
  And there was a firmament again, and a wind, and a glare of purple light in the eyes of the falling dreamer. There were gods and presences and wills; beauty and evil, and the shrieking of noxious night robbed of its prey. For through the unknown ultimate cycle had lived a thought and a vision of a dreamer's boyhood, and now there were remade a waking world and an old cherished city to body and to justify these things. Out of the void S'ngac the violet gas had pointed the way, and archaic Nodens was bellowing his guidance from unhinted deeps.
  
  Stars swelled to dawns, and dawns burst into fountains of gold, carmine, and purple, and still the dreamer fell. Cries rent the aether as ribbons of light beat back the fiends from outside. And hoary Nodens raised a howl of triumph when Nyarlathotep, close on his quarry, stopped baffled by a glare that seared his formless hunting-horrors to grey dust. Randolph Carter had indeed descended at last the wide marmoreal flights to his marvellous city, for he was come again to the fair New England world that had wrought him.
}













\section{Climactic Battle}
Near the end there is a great, climactic battle between the \bane/\resphan{} forces and some \quo{good guys}. 
There are very few \dragons{} (perhaps only \Ishnaruchaefir{} and his companions), because most \dragons{} support \ps{\Secherdamon} gambit, hoping to utilize the \pps{\banes}{} Shroud-weaving spell to bring the \firstgendragons{} into the world and have them vanquish the \Voidbringer{} and his legions.







\subsection{Rage, Rage Against the Dying of the Light}
One of the sections or chapters of this book should be called \quo{Rage, Rage Against the Dying of the Light}.
It is a reference to the poem \quo{Do not go gentle into that good night} by Dylan Thomas.







\subsection{\HothNrul}
\target{Hoth-Nrul summoned}
\Vizsherioch{} and his cohorts manage to bring a mighty ally to \Miith: 
\hr{Hoth-Nrul}{\HothNrul}, a \xsic{} godling. 
\HothNrul{} is intended to spearhead the coming \xs{} invasion of \Miith. 

Later, \Ishnaruchaefir-tachi communicate with \HothNrul{} and convince the alien god that it is in its interest to attack \Daggerrain.
So \HothNrul{} seeks out the \banelord. 

\HothNrul{} has difficulty attacking \Daggerrain, for he is well-hidden-away. 
But it manages to do some damage. 
When \Daggerrain{} is distracted by the \xs{}, \Azraid{} and Ramiel strike and fuck up \Daggerrain. 







\subsection{\Ishnaruchaefir leads \dragons}
\target{Ishnaruchaefir leads Dragons to war in TBW}
\Ishnaruchaefir{} does not believe in this gambit. He has no faith in \Tiamat{} and her brethren and does not think they will come to help. 
So he works to persuade the \dragons{} (including some \hr{Aloof Dragons}{aloof ones}) to fight, to fly with him against \ps{\Daggerrain}{} legions. 

At the last moment, he succeeds. 
Have a massively epic scene where \Ishnaruchaefir{} flies at the head of a massive flight of a \dragons, flanked by hundreds of \rachyths, \drakes{} and \pdaemons, to engage the \bane{} armies and seize victory.
They are awesome to behold as they return to \Miith in full force. 
Dozens of Elder \Dragons. 

Compare to: 

\begin{itemize}
  \item 
    \bandsong{Bal-Sagoth}{Black Dragons Soar Above the Mountain of Shadows}. 
  \item 
    The scenes in \authorseries{Richard Knaak}{War of the Ancients} where everyone holds their breath for the \dragons{} to come to their aid, and where they finally do come and save the day.
  \item 
    The scenes in \authorbook{Graham McNeill}{Fulgrim} p.472, where the Alpha Legion, Word Bearers, Night Lords and Iron Warriors come to the \quo{rescue}. 
\end{itemize}

\citeauthorbook[p.344]{ClarkAshtonSmith:TheDarkEidolon}{Clark Ashton Smith}{%
  The Dark Eidolon%
}{
  Yea, the undying worms of fire and darkness have come forth like an army at thy summons\prikker
}







\subsection{The Advent of \Lithrim}
\target{Lithrim arises}
\target{Second Advent}
\target{Second Advent of Lithrim}
\Lithrim{} finally arises. 

Mankind is very Shrouded, \naive{} and stupid. 
Not quite as well-meaning and {innocent} as the \scathae, but not all that far from them.
That is why it is so effective when their inner \Erebean{} darkness is finally released in the culmination of the \Morbus{} plan. 

\citebandsong{DeathspellOmega:SiMonumentumRequiresCircumspice}{%
  Deathspell Omega
}{
  Sola Fide
}{
  Some rejoice of the children's innocence and smiles\\
  Others of their shameful defloration \\
  by the ministers of Christ, swollen by arrogance and lust\\
  Jesus spoke words of wisdom: \quo{Let the children come to me}\\
  What's more enthralling indeed \\
  than ruining through them the genesis of life itself?\\
  Dive deep into the blue eyes of the newborn \\
  and thou shalt get a glance of heaven\\
  When this dive reveals despair, \\
  hideous ruins of the glory of Hell\\
  The serving sons of disobedience proclaim \\
  Vengeance belongeth unto Him\prikker
}

\citebandsong{DeathspellOmega:SiMonumentumRequiresCircumspice}{%
  Deathspell Omega
}{
  Jubilate Deo (O Be Joyful in the Lord)
}{
  Innocence sacrificed and heaven denied\\
  Iniquity divine\dash no repentance\dash over mankind\\
  Mind-raped, self-devoured, empty god-eyed\\
  Transgression coercive, Redemption never to be found

  Under the altar of cannibalism\\
  Man hast (thou) confessed the spiritual schism\\
  And without pride mourned the loss of freedom,\\
  Our blessed Lord have condemned you to martyrdom
  
  Let everything that hath breath praise the Lord, \\
  praise ye the Lord.
}

It is the culmination of that which began with \ps{\Thanatzil} noble sacrifice. 

\citebandsong{DeathspellOmega:Kenose}{%
  Deathspell Omega
}{
  \Kenose
}{
  The advent of Plerosis is the destiny of Man \\
  and shall shatter up to the Heavens,\\
  A savage aperture to the High Mass of the Comforter
}

It is cataclysmic and destructive. 
It unleashes all the wicked, destructive Entropy that, as it turns out, were the building blocks of Man all along. 

\citebandsong{DeathspellOmega:Kenose}{%
  Deathspell Omega
}{
  \Kenose
}{
  Instigating manifold quadrants of industrialized death,\\
  An avid Moloch, never satiated, an endless Feast,\\
  Following the principle of reversibility of merits,\\ 
  shattering up the Word\\
  As Pillars of grayish Soulfire spurt out to a bereaved firmament
}

Man instinctively welcomes the advent of \Lithrim.

\citebandsong{DeathspellOmega:CrushingtheHolyTrinity}{%
  Deathspell Omega
}{
  Diabolus Absconditus
}{
  The unreservedly open spirit, \\
  open to death, to torment, to joy,\\
  The open spirit, open and dying,\\
  Suffering and dying and happy, \\
  stands in a certain veiled light:\\
  That light is divine.\\
  And the cry that breaks from a twisted \\
  mouth may perhaps twist him who utters it,\\
  But what he speaks is an immense alleluia, \\
  flung into endless silence, and lost there.
}

But it is also scary. 

\citebandsong{DeathspellOmega:CrushingtheHolyTrinity}{%
  Deathspell Omega
}{
  Diabolus Absconditus
}{
  From very high above a kind of stillness \\
  swept down upon me and froze me\\
  It was as though I were borne aloft \\
  in a flight of headless and unbodied angels\\
  Shaped from the broad swooping of wings, \\
  but it was simpler than that\\
  I became unhappy and felt painfully forsaken, \\
  as one is when in the presence of God
}

Some \humans{} fight against it. 

\citebandsong{BeyondTwilight:TheEdge}{Beyond Twilight}{Salem}{
  Entombed by the will of the night.\\
  All your thoughts deflowered.\\
  Waived all of your beliefs\\
  in this midnight hour. \\
  Search your past, try and make it back.\\
  All your strength overpowered.\\
  Unknown forces run you through \\
  in the shadow of the black tower.
  
  The witches of Salem hold your fate. 
  
  Relief obtained in sacred mass.\\
  All your sins forgiven. \\
  The ultimate sign of witchery.\\
  The total lack of sanity.
}

\Humans{} lose their individual identity. 

\citebandsong{DeathspellOmega:Kenose}{%
  Deathspell Omega
}{
  \Kenose
}{
  And the victim, blind to the radiating Light of Truth, \\
  stuttering, repeats:\\
  Lamma Sabacthani?\\
  Lamma Sabacthani?\\
  Lamma Sabacthani?
  
  Consummatum est.
}

Mankind becomes a god: \Lithrim. 

\citebandsong{DeathspellOmega:Kenose}{%
  Deathspell Omega
}{
  \Kenose
}{
  I am the son of Man, \\
  and this in erring reason's spite, is my pride\prikker\\
  War, be enthroned, a form of divine retribution!\\
  Execution, be sacred, agent of divine Providence!
  
  The stillness of Contemplation \\
  is allowed in billions of woeful cries,\\
  So astonishingly simultaneous and in unison,\\
  Each and every second, \\
  They defuse each other in such a perfect manner,\\
  Equaling the most inscrutable of all silences\\
  Doctrine of Mystical Substitution, Mystical Body, \\
  Sanctorum Communionem,\\
  Celebrate the Sin of one reflecting, \\
  tectonic forces alike, upon the multitude
  
  The purest of all Holocausts shall be perpetrated\\
  By a loving hand, never knowing if it provided felicity\\
  Or the vilest of everlasting torments\\
  No man can see Me and live!
}

When the \humans{} become the fell, abhorrent \Lithrim, they (or the smarter of them) at last realize the purpose of their otherwise empty and wretched lives. 

\citebandsong{DeathspellOmega:CrushingtheHolyTrinity}{%
  Deathspell Omega
}{
  Diabolus Absconditus
}{
  The act whereby being\dash existence\dash is bestowed upon us\\
  Is an unbearable surpassing of being,\\
  An act no less unbearable than that of dying.\\
  And since, in death, being is taken away from us \\
  at the same time it is given us,\\
  We must seek for it in the feeling of dying,\\
  In those unbearable moments \\
  when it seems to us that we are dying\\
  Because the existence in us,\\
  During these interludes,\\
  Exists through nothing but a sustaining and ruinous excess,\\
  When the fullness of horror and that of joy coincide.
  
  As I waited for annihilation, all that subsisted in me\\
  Seemed to me to be the dross over which man's life tarries\prikker
}

But they lose their \humanity{} in the process. 

\citebandsong{DeathspellOmega:FromtheEntrailstotheDirt}{%
  Deathspell Omega%
}{
  Mass Grave Aesthetics
}{
  The black Idol fills the veil of flesh with noxious smoke, \\
  Depicting primal human experiences indifferently, \\
  Contemptuous of moral concerns, dehumanized \\
  The howling of wolves and the destructive sword \\
  are portions of Eternity, \\
  Too great for the eyes of merely a man
}

\citebandsong{BlindGuardian:FTB}{%
  Blind Guardian%
}{
  Hall of the King
}{
  You're the birth and you're the end.\\
  You've been hurt but you're not dead.\\
  Discovering you is what should never be.\\
  Poisoned are our souls and dark our hearts.\\
  Ruins we have left to rule the world.\\
  Destructive are our minds. It's much too late.\\
}

But they gain new insight as they leave their \human{} forms:

\citebandsong{DeathspellOmega:FromtheEntrailstotheDirt}{%
  Deathspell Omega%
}{
  Mass Grave Aesthetics%
}{
  Violence exists I the moment \\
  when the eye turns upwards into the head, \\
  When inversion is complete and total \\
  The darkness of the upturned eye is not the absence of light \\
  But the process of seeing being taken to its limit \\
  That thorough derangement of the senses, \\
  Way beyond the deceptive conflict between darkness and light \\
  Opens perceptions to the tyranny of the Chekhinah
}

\Lithrim{} is a slave under the \psp{\banelords}{} command. 

\citebandsong{DeathspellOmega:CrushingtheHolyTrinity}{%
  Deathspell Omega
}{
  Diabolus Absconditus
}{
  I can utter no word, O my God, \\
  unless I be permitted by Thee\\
  And can move in no direction \\
  until I obtain Thy sanction\\
  It is Thou, O my God, \\
  who hast called me into being through the power\\
  Of Thy might, and has endued me \\
  with Thy grace to manifest Thy Cause
}

\citebandsong{DeathspellOmega:FasIteMaledictiinIgnemAeternum}{%
  Deathspell Omega
}{
  Bread of Bitterness
}{
  From a supplication without response, \\
  the essence of man, his ground giving way, \\
  comes illumination by a sun of great evil \\
  that sets aflame the inner core \\
  and enthrones suffocation and the intolerable without respite \\
  as the joyful reward for a million aborted truths.
  
  This silence that among all man has charged with sacred horror, \\
  it becomes sovereign, in repugnant nativity, \\
  and detaches itself from the bonds \\
  which paralyse a vertiginous movement towards the void.
  
  Breathless ecstatic experience,\\
  it opens the horizon a bit more,\\
  this wound of God;\\
  it is the assassination of the abyss of possibilities, \\
  the depths of being left to holy vultures. 
}





\subsubsection{Sentinels realize \Lithrim}
The Sentinels finally realize what \Lithrim is.
(Do not reveal this until the \humans have already coalesced into \Lithrim.)

\begin{prose}
  \ta{\Lithrim, the hideous that feeds on the souls of the \humans.}
  
  \ta{No. 
    It is worse than that.
    \Lithrim does not feed on the \humans.
    \Lithrim \emph{is} the \humans!
    Picture it.
    Every \human every born into the \iquinian faith, for thousands of years, has been gathered for this purpose.
    Each \human soul is puny, of course, but they number in the hundreds of millions, if not billions.
    All coalesced together to form a nightmare god of colossal power!}
\end{prose}





\subsubsection{\Lithrim awakens the \resphain's latent madness}
When \Lithrim awakens, the madness latent in the \resphain also awakens. 
Even in places where they believe themselves safe the \resphain are seized by madness and begin to fight and kill each other.
They almost ruin everything. 










\subsection{The portal opens}
\Lithrim{} opens the way for the \Voidbringer. 

\citebandsong{BeyondTwilight:TheDevilsHallofFame}{Beyond Twilight}{%
  Perfect Dark%
}{
  Hold the candle\\
  Pass the flame\\
  We're all in this together \\
  Cosmic winds are gathering\\
  Coming your way\\
  Telling you you are only a fragment\\
  A fragment of time 
  
  And it's perfect dark\\
  It's perfect dark here\\
  Hear the angels screaming
}





\subsubsection{People feel the \Voidbringer}
As the Cabal plot nears fruition and they are opening the gate to \Erebos, \Miithians can feel the presence of the \Voidbringer: 
An ever hungry, ever consuming emptiness, completely inhuman. 
The older immortals remember it from the \secondbanewar. 





\subsubsection{\Banes{} slip through}
Many \lesserbanes{} slip through into \Miith. 
Not in the same millions that got through in the \secondbanewar, but still quite a lot. 
But not so many that the combined \dragons{} and \resphain{} cannot deal with them\prikker hopefully. 

















\section{Ramiel the Betrayer}
\target{Ramiel betrays Voidbringer}
\target{Ramiel betrays Banes}









\subsection{Summary}
\hs{Ramiel} has decided that he will no longer serve the \banelords. 

\lyricsbs{Dream Theater}{In the Presence of Enemies}{\obeylines%
  My soul grows weaker.\\
  He knows and he waits.\\
  He watches over me,\\
  standing at the infernal gates.\\
  In the hour of darkness,\\
  the moment I feared has passed,\\
  the moment I lost my faith.\\
  Promising salvation.\\
  My soul is my own now.\\
  I do not fight for you,\\
  Dark Master!
}

\citebandsong{BeyondTwilight:TheDevilsHallofFame}{Beyond Twilight}{%
  Hellfire%
}{
  Before me now I see a dying planet.\\
  Captured souls with microchips implanted.\\
  When warning came you stayed the same.\\
  When warning came you stayed the same.\\
  I am crumbling over the obstacles.\\
  Nothing left to entertain. \\
  The smell of burning corpses. \\
  A new world domain.
  
  Before me now I see a future\\
  A future black\\
  Oh come and save me\\
  I'm under soul attack
  
  I will break out from this madness\\
  There is nothing here but sadness\\
  I'm burning in the hellfire\\
  There must be a way\\
  Waiting for the day\\
  
  Hellfire burns higher\\
  Higher, higher. \\
  Only blood remains\\
  There is nothing more to gain \\
  Take me.
}

He now plans to betray the \banes. 
He intends to siphon energy from the \Voidbringer, thus empowering himself enough to overpower \Daggerrain{} and slay or banish him. 
It is a mad scheme, but Ramiel has always been one for mad schemes. It stems from his sublime arrogance and megalomania (see section \ref{Ramiel's megalomania}). 

It is a good plan, but \Daggerrain{} is too clever. 
The \banelord{} has always known that Ramiel was a loose cannon and could not be trusted, so he had taken precautions. 
When Ramiel finally betrayed the \banes, they were ready for him. 

So Ramiel's plan fails\prikker

\prikker or would fail, if not for the intervention of \Azraid. 
\Azraid{} had always played the loyal Cabalist, faithful servant of the \banelords, but \hr{Azraid hates Banes}{in secret he has always hated them} and \hr{Azraid plots against Banes}{plotted against them}. 
He anticipated Ramiel's betrayal and planned for it. 
So when Ramiel fails, \hr{Azraid helps Ramiel betray Banes}{\Azraid{} steps in and helps him}. 

\ps{\Azraid} gambit cannot succeed without Ramiel's help. 
But he cannot tell Ramiel about it before the very last minute, because he doesn't trust Ramiel's ability to keep the secret. 
(It is \emph{hard} to keep secrets from \Daggerrain.) 
It is thus something of a \trope{XanatosRoulette}{Xanatos Roulette}. 

So, in the end, it turns out that \Azraid{} was \hr{Daggerrain's blind spot}{\ps{\Daggerrain}{} blind spot}. 





\subsubsection{The Ark}
\target{Ark}
\index{Ark}
\Azraid{} and \Ishnaruchaefir plan to use an ancient artifact to destroy \Daggerrain: the Ark. 
The Ark is based on the ruins of the \hr{Sethicus tomb}{tomb of \Sethicus}, which lies in the Realm of \hr{Neevrai}{\Neevrai}. 

It will be difficult, but \Azraid{} is ready. 
He has spent millennia preparing for this day, and he is prepared to sacrifice his life to see it through. 
And he has \ps{\Ishnaruchaefir} awesome Gnosis helping him. 

In the end \Azraid{} destroys himself.  

The Ark was actually an unfinished relic created by \Sethicus. 
It was a part of \hr{Sethicus plan}{his masterplan} to escape \Miith before \hr{Sethican eschatology}{it was destroyed}. 
As was to be expected of an element of \Sethicus's plan, the Ark required a tremendous sacrifice of blood and life and Heart-power to launch it, and more sacrifice still to sustain it on its journey. 





\subsubsection{It all takes place in the \matrix{} world}
Note that all of this takes place in the world of the \matrices. 
The three may not physically meet at all. 
They simply pull at cosmic strings using their \vertex{} abilities. 
Any any \quo{portals} are not physical portals, but a ubiquitous thinning of the Shroud allowing passage into \Erebos{} or somewhere else. 





\subsubsection{Why they wait so long}
One might think \Azraid{} could have betrayed the \banes{} long ago. 
But that would not have worked. 
He was not confident he could have done enough damage before he was caught, eliminated and replaced. 

No, this is a unique opportunity. 
\ps{\Azraid} betrayal has to happen \emph{after} the advent of \Lithrim. 
They need to rely on the Shroud-wide chaos and fluctuations that \Lithrim{} creates.





\subsubsection{Ark fueled by dead \ophidians}
The Ark is a big-ass \hs{living machine}. 

The Ark is built from \Sethicus's old tomb. 
\Sethicus was buried with many other \ophidians and perhaps even \dragons.
Many of them still lie there, dead but dreaming. 
These can be used as fuel to power the Ark and keep it alive. 
When the dead learn about this plan, they are nonplussed. 
They see it as betrayal and fight back. 
They would rather see everyone perish with them than be betrayed and sacrificed for someone else's gain. 

The undead cannot move, but they still have arcane power. 
\Ishnaruchaefir has to expend more of his already badly strained power to overcome the dead and force them to become fuel. 
This is also psychologically hard for him: 
It is another horrible betrayal of his own people, even violating the \dragons' important \hr{Dragons do not eat Dragons}{cannibalism taboo}. 
It is another gruesome scar on his soul. 
But it must be done, and \Ishnaruchaefir knows he is the only one hard enough to do it. 

This should be a climactic thing, another \quo{everything almost going awry at the last minute} twist at the end of the last book. 









\subsection{Ramiel drains the \ps{\Voidbringer}{} power}
Ramiel tries to betray the \Voidbringer. 
This is similar to how Astronema betrays Dark Specter and drains his power in \emph{Power Rangers in Space}. 

He absorbs some of the \ps{\Voidbringer}{} power. 





\subsubsection{How it feels}
He feels himself swimming adrift in an endless, bleak ocean of dark, Entropic power, feels how it almost devours him and his sanity.

\lyricsbs{Bal-Sagoth}{Summoning the Guardians of the Astral Gate}{
  The ravening black worms of madness 
  are devouring the shredded remnants of sanity 
  as I return to my slumbering steel-clad body\prikker \\
  but as the dream-veil lifts, I feel my limbs transform, 
  flesh becoming cold stone\prikker \\
  enshrouded by a dark mantle of obsidian. \\
  And the laughter of the Guardians echoes, 
  carries upon the winds of this spectral eve.
}

Actually, he began doing this much earlier, but it takes a while before \Daggerrain{} \quo{discovers} his treachery. 

Ramiel plans to use his \carcer against the \Voidbringer.

\citebandsong{BlindGuardian:IFTOS}{Blind Guardian}{%
  The Script for My Requiem (demo version)%
}{
  Battle rages higher and higher. \\
  Akron starts to feel my hate. \\
  There's no escaping for the dark souls \\
  I can't control anymore, anymore. \\
  What started in the underworld, \\
  it should be finished here and now. 
  
  I know his tricks, black magic power. \\
  But now it's time for him to say goodbye. \\
  There's only one thing left, the dawn shall arrive. \\
  We've got to get it past, fear the dark side. \\
  Turn off his light. \\
  Turn off his light.
}





\subsection{Ramiel fails}
\target{Ramiel as Xanatos Sucker}
\Daggerrain{} is too clever for Ramiel. 
He had anticipated Ramiel's betrayal and easily stops him. 
He is about to kill and destroy Ramiel\prikker 

Before his death, Ramiel's life flashes before his eyes. 
He realizes that \Daggerrain{} must have anticipated his every move and planned accordingly. 
He has been playing straight into \ps{\Daggerrain}{} hands the whole time. 

He truly \hr{Ramiel is nothing}{is nothing}\prikker 
or so he fears. 

Compare a \hr{Azraid as Xanatos Sucker}{similar scene with \Azraid{} much earlier}. 
Also compare to the scene in \emph{\JuukenSentaiGekiranger} episode 46 (or so) where Rio discovers that his entire life has been masterminded by Long. 

We see inside Ramiel's head as he struggles against \Daggerrain{} and almost dies. 

\lyricsbs{Hate Eternal}{Path to the Eternal Gods}{
  On this journey into death, \\
  I am beside myself in tremendous bliss,\\
  For this allegiance has been made. \\
  Shall my sins be absolved, \\
  washed away by the blood of the sacred lambs? \\ 
  Yet I am not amongst the flock.
  
  The remnants of my life now become my vast illusions,\\
  whilst exiled from your grace. 
  
  With fortitude, with courage, 
  I face all my fears, \\
  on my path to the eternal gods. \\
  With my wrath, with my disdain, 
  I face all my fears, \\
  on my path to the eternal gods.\\
  As I cross over into my final place, 
  I face all my fears, \\
  on my path to the eternal gods.\\
  As I bear trials in my final resting place, 
  I face all my fears, 
  on my path to the eternal gods.\\
}





\subsection{\Azraid helps Ramiel}
\target{Azraid helps Ramiel betray Banes}
At the last moment, before \Daggerrain{} would have destroyed Ramiel, \Azraid{} steps in and backstabs \Daggerrain. 

Ramiel realizes that he had been overreacting before, overestimating \ps{\Daggerrain}{} omniscience and underestimating his own capability. 
Some parts of his plan actually \emph{were} successful, not covered by the \ps{\banelord}{} \trope{XanatosGambit}{Xanatos Gambit}. 
He regains some faith in himself. 

Together, the two \satharioth{} push \Daggerrain{} back. 





\subsubsection{Ramiel and \Azraid{} as kindred spirits}
\Azraid{} criticizes Ramiel for being too overt and honest (\hr{Mystraacht philosophy}{and would be expected of a \Mystraacht}), making him predictable. 
In contrast to \Azraid{} himself, who has managed to creep under even \ps{\Daggerrain}{} radar for twenty thousand years. 

But he is also sympathetic, because he recognizes that Ramiel's situation is like a mirror of his own. 
Back when \Daggerrain{} manipulated the young \Azraid{} into betraying and murdering his brother \Damiarch, he was similarly shocked and disillusioned. 
He shares this with Ramiel. 
This is a rare \trope{PetTheDog}{Pet the Dog} moment for \Azraid. 

A new understanding and empathy suddenly exists between the two \resphan{} lords. 
They are now kindred spirits. 
After this battle they part, not as friends, nor allies, but with a deeper mutual respect than previously. 

\begin{prose}
  Ramiel: 
  \ta{\Azraid. Were you among those who betrayed \Zachirah{} and me?}

  \Azraid: 
  \ta{LOL. Ramiel. 
    I have deceived the \banelords{} for thousands of years. 
    I have deceived everyone. 
    Among all the \resphain, there is no greater fraud than I.
    If I were to answer \quo{no}, would you believe me?
    I would not believe me.}
\end{prose}






\subsubsection{Ramiel imagines hearing \ps{\Daggerrain}{} voice}
Ramiel jokingly imagines hearing \ps{\Daggerrain}{} voice in his head: 
\daggerrain{A minor setback. 
I have planned for this contingency. 
It will take us another 3651 years to get out plan back on schedule\prikker} 
(The number 3651 is a random one that Ramiel pulls out of his ass, but it sounds like something \Daggerrain{} might say.)








\subsection{\Azraid dies\prikker maybe}
\target{Azraid dies}
Maybe \Azraid{} dies. 
He sacrifices himself in order to violently disrupt the \Erebean{} \matrix{}, thus causing it to implode and collapse faster than \Daggerrain{} can restore it to order. 

\ps{\Azraid} death leaves \CiriathSepher{}, its hierarchy and \matrix, in chaos. 
(\hr{CS heirs die}{Not much is left} of the \hr{CS order of succession}{\CiriathSepher{} order of succession} at this time.)
It is now up to Ramiel to pick up the pieces and lead the \resphan{} race, if they are to have a chance of surviving. 

Compare to Anomander Rake's sacrifice in \cite{StevenErikson:TolltheHounds}. 

\citebandsong{BeyondTwilight:TheDevilsHallofFame}{Beyond Twilight}{%
  The Devil's Hall of Fame%
}{
  Surrender\\
  Lost in the dark\\
  Without angels to guide me 
  
  Hands of death reaching out for me\\
  I'm longing to rest\\
  I see myself move in need of power
}

He knows the \banes{} better than any other. 

\citebandsong{BeyondTwilight:TheDevilsHallofFame}{Beyond Twilight}{%
  The Devil's Hall of Fame%
}{
  I've looked straight into the eyes of demons rising\\
  So close.\\
  I've had one foot on the other side\\
  For so long. \\
  I've seen evil so evil that all living died\\
  The devil is near.\\
}

He wants to stop the \iquin{} plan. 

\citebandsong{BeyondTwilight:TheDevilsHallofFame}{Beyond Twilight}{%
  The Devil's Hall of Fame%
}{
  You may fool the people, but you can't fool me\\
  No more, not anymore. 
  
  And all the souls turned black\\
  There is no way back
}

He dies. 

\citebandsong{BeyondTwilight:TheDevilsHallofFame}{Beyond Twilight}{%
  The Devil's Hall of Fame%
}{
  Stranded with the darkness\\
  It's so cold in here \\
  I feel the rush within me
}

He hopes that \Miith{} (or a fragment of it) now has a future. 

\citebandsong{BeyondTwilight:TheDevilsHallofFame}{Beyond Twilight}{%
  The Devil's Hall of Fame%
}{
  Time will mend\\
  Ease the sorrow\\
  Take my hand\\
  There's no tomorrow
}











\subsection{\Daggerrain falls}
\target{Daggerrain falls}
\Daggerrain{} has fallen. 
And the worst part is that the link between \Erebos{} and \Nyx{} fell with him. 
\hr{Daggerrain is the gate}{\Daggerrain{} \emph{was} the link}. Without him, the way is closed permanently. 
Now, the only way for the \banes{} to return from \Erebos{} would be to send spaceships, like they did the first time around. 
That might take millennia, because \Erebos{} and \Miith{} are very distant from each other. 















\section{The Fall of the Shroud}
\subsection{\ps{\Ishnaruchaefir} glaive is destroyed}
\target{Ishnaruchaefir's glaive is destroyed}
\target{Glaive is destroyed}
Throughout most of the story, \hr{Ishnaruchaefir fights to preserve the Shroud}{\Ishnaruchaefir{} has fought to preserve the Shroud}. But at the end, he lets himself persuade to assist in unravelling the Shroud. \Triestessakhin{} is the one who persuades him. 

Ultimately, \hr{Ishnaruchaefir's glaive}{\ps{\Ishnaruchaefir} glaive} is destroyed, its Shroud-weaving effect (as a \hs{weaving artifact}) is broken, and the soul of \Triestessakhin{} is released. 
\hr{Glaive must be destroyed}{This had to happen}. 

She is quite insane after these many thousand years inside the sword, but he manages to lead a conversation with her. She is bitter and tries to make him repent. 

\Ishnaruchaefir: 
\ta{%
  I mourn what I have done, and at times I even regret. 
  But I would not change it, and given the chance, I would do the same again.}

Compare to the scene with Lord Soth and his wife at the end of \emph{Spectre of the Black Rose} by James Lowder and Voronica-Whitney Robinson. 









\subsection{The Shroud is torn}
In the process of this whole thing, the Shroud has been ripped and torn and now lies greatly weakened. A climax of this was the \hr{Glaive is destroyed}{breaking of the glaive \Triestessakhin}. 

The \charade{} is breaking apart and the old world order is crumbled. 

The \hs{Unravelling brings chaos}. 
The world will descend into chaos and anarchy. 

\lyricstitle{\cite{HPLovecraft:TheCallofCthulhu}}{
  The most merciful thing in the world, I think, is the inability of the human mind to correlate all its contents. We live on a placid island of ignorance in the midst of black seas of infinity, and it was not meant that we should voyage far. The sciences, each straining in its own direction, have hitherto harmed us little; but some day the piecing together of dissociated knowledge will open up such terrifying vistas of reality, and of our frightful position therein, that we shall either go mad from the revelation or flee from the deadly light into the peace and safety of a new dark age.
  
  [\prikker]
  
  That cult would never die till the stars came right again, and the secret priests would take great Cthulhu from His tomb to revive His subjects and resume His rule of earth. The time would be easy to know, for then mankind would have become as the Great Old Ones; free and wild and beyond good and evil, with laws and morals thrown aside and all men shouting and killing and revelling in joy. Then the liberated Old Ones would teach them new ways to shout and kill and revel and enjoy themselves, and all the earth would flame with a holocaust of ecstasy and freedom.
}

In a way, this is \hr{Ishnaruchaefir sees three futures}{\ps{\Ishnaruchaefir} nightmare} come true. 

\lyricsauthorbookpage{Graham McNeill}{False Gods}{157}{
  I saw beyond and into the warp. I saw the powers that dwell there\prikker 
  
  There is great evil in the warp and I need you to know the truth of Chaos before the galaxy is condemned to the fate that awaits it. 
  
  I saw it, Warmaster, the galaxy as a wasteland, the Emperor dead and mankind in bondage to a nightmarish hell of bureaucracy and superstition. \\
  All is grim darkness and all is war. \\
  Only you have the power to stop this future. \\
  You must be strong, Warmaster. \\
  Never forget that.
}















\section[The Return of Tyrasshana]{The Return of \Tiamat}
\target{Ungod}
\index{Ungod}
In secret, \Tiamat was behind the \resphain. 
She was the Ungod of the Outer Darkness.
Her goal was to absorb all intelligent life on \Miith and unite them all in one great Concord. 
She found the \banes and conquered them and made them her slaves, part of her Concord hivemind. 
That was one of the reasons why the \banes were so horrid: 
They were not individuals. 
They were, in a sense, soulless\dash but also part of something greater and terrifying. 

\Tiamat masterminded the fall of Semiza and the rise of the \resphain.
She made Semiza corrupt the \resphain to her service. 
She made them invade \Miith and wage war against the \ophidian races. 

\Tiamat wanted to grow strong and escape from her awful prison. 
The advent of \Lithrim finally freed her. 
She was now free. 
Ramiel, \Azeraid and \Iscrafel managed to destroy \Daggerrain and the \bane hordes, but they could not stop the Mother. 
\Tiamat was now awake, and she would devour the world. 

(\Daggerrain was a servant of the \hr{Voidbringer}, the cosmic god of the \banes. 
The Voidbringer is separate from the Ungod, although they are often conflated by the ignorant.)

But all was not completely bleak. 
\Tiamat's merger with the \banes made her more stagnant and decaying and destructive than she might otherwise have been.
The heroes had destroyed this, the worst part of her. 
And in their own quests they had grown strong.
They now represented a mighty and virtuous force of positive change. 
\Tiamat absorbed them, and in doing so she absorbed their virtue and strenght and made it part of her.
So they would live on in her, and their deeds and legacy would help steer the great Mother. 
Now \Tiamat would devour everything on the planet and then move on to conquer the universe. 

And the Mother spread her wings. 













































\chapter{Random Ideas}












\section{\Shilred{} Story Arc}





\subsubsection[Ishnaruchaefir and his fake mission]{\Ishnaruchaefir{} and his fake mission}
\target{Ishnaruchaefir's fake mission}
\Ishnaruchaefir{} is clever. He knows about \hr{Teshrial's creatures}{the \noggyaleth{} in \Malcur}, or at least suspects them. So he \hr{Ishnaruchaefir attacks Teshrial's creature}{goes there to see for himself}. 

\Ishnaruchaefir initially believes \Secherdamon's plan takes place in \Malcur.
Then he falls for the \Forclin decoy when it is unveiled. 
He is impressed to learn that both were decoys. 

He has also seen through \ps{\Secherdamon} plan to resurrect \Nithdornazsh{} in \Malcur. 
He has more Sentinel contacts than most people realize, so he has spies telling him of \ps{\Secherdamon} plans. 
And he is an expert astrologer and \matrix-theorist, so he has picked up subtle clues concerning \Nithdornazsh{} and \Malcur. 

He knows that the \noggyaleth{} may put a stop to the plan. 
He wants to help his brother.
He does not fully approve of \ps{\Secherdamon} plan to resurrect \Nithdornazsh, but he helps out of respect for \hr{Nexagglachel}{\Nexagglachel}, since he still feels guilt and obligation towards him and his memory. 
\Nithd{} was \ps{\Nexagglachel}{} fortress, after all, so it can stand as a monument of sorts to the fallen \dragonking.

\target{Ishnaruchaefir's creatures}
So \Ishnaruchaefir{} plans to free some of his own servitors\dash\pdaemons, \dragons{} or otherwise\dash who are imprisoned in a cave or lost temple somewhere in the eastern \PelidorContinent. 

But he can't just go off and do that. He would be discovered and foiled by \hr{Teshrial}{\Teshrial}, the Cabalist in charge of the area around the cave. 

So \Ishnaruchaefir{} develops a circuitous plan: 
\hr{Ishnaruchaefir recruits Shilred-tachi}{He recruits \Shilred-tachi} and sends them off to fulfill an ill-defined quest for him, in the general area of the cave in question. He tells them to search for a cave or something like that. 

But whatever he alludes that he wants them to do is a lie. The purpose of their entire mission is to look suspicious. \Ishnaruchaefir{} shows up to help them just often enough that the sneaky Cabalists cannot help but notice, yet rarely enough that it still looks like he is trying to lay low. 

What he is actually looking for is another cave, where his servitors are imprisoned. 

The Cabalists in the area where \Shilred-tachi are going are led by \Teshrial{} (see section \ref{Teshrial}). We follow him in several chapters, where he fears \Ishnaruchaefir{} and tries to guess his plan. 






\subsection{Rissitic invasion}
Some time after the Rungerans cross the Pelidorian border, Rissitic fleets sail north to besiege and conquer cities in \Scyrum. 
%march north from Gaznor into Beirod. They besiege and conquer Beirodi cities. 

%Or\prikker wait\prikker is it Beirod they attack? Or is it \Scyrum? \Scyrum{} might make more sense. 

For the Rissitics, controlling more land in \Velcad{} is nice, but the chief purpose of this attack is to draw the Redcor away. See, in intra-\Velcadian{} wars the Redcor traditionally take little part, acting mostly as negotiators and healers. But in case of a major attack on an Iquinian kingdom by a heathen invader, the Redcor typically offer more direct aid, participating in battles with their magic. So when the Rissitics invade, the Redcor are forced to send Vaimons from the entire region to help oppose them. 
This makes the invasion harder, but it has the nice bonus feature that it leaves \Malcur drained of Redcor, making it easier for the Sentinels to operate there. 

We follow Narkiza on his march north through \Scyrum. 

%Is Geldashad with him, or is he off on a mission of his own?

%What about \Shilred? 

Remember to describe the Rissitic army as \hr{Glorious armies}{magnificent and glorious}.







\subsection{\Shilred-tachi get lost in \Nyx}
\Shilred{}, \Dzasselid{} and \KarsaOrlong{} (see \ref{Karsa Orlong}) take part in the initial attack on a port city in southern \Scyrum. Maybe they are together with Mestos and Semphai. Anyway, during the battle they become separated from the main army and nearly get killed by \banes. %Maybe they even flee out of the mortal Realm and stumble into \Nyx. 

Maybe \Dzasselid{} deliberately takes them into \Nyx. Perhaps he has a back-story as a double agent for the Cabal, and thus knows his way around in \Nyx. Or something like that. He needs some skeletons in his closet, remember?

At any rate, they end up in \Nyx{} and get surrounded by \banes{} or other monsters. 






\subsubsection{\Ishnaruchaefir{} approaches}
Just when \Shilred-tachi are about to be killed, \Ishnaruchaefir{} appears. He appears in humanoid form. Using magic (incl. the spell \quo{\kingstongue{khestni}}) and his mighty glaive (see section \ref{Ishnaruchaefir's glaive}), he slays the monsters and saves them. The monsters should be huge, monstrous and menacing, like in the anime \emph{Devil May Cry}. 

Given that they owe him their lives, he claims them as his own. Perhaps one of the \Gisshorns{} knows who he is and convinces the others to be respectful. Somehow or other, \Ishnaruchaefir{} beats it into their heads that he is a high-ranking ally of Rissit and that they should obey him. 

Maybe \Ishnaruchaefir{} first approaches \Narkiza{}, gives a little help and asks to borrow some of his men. 






\subsubsection{\ps{\Ishnaruchaefir} mission}
\target{Ishnaruchaefir recruits Shilred-tachi}
\Ishnaruchaefir{} sends them on a private, secret mission. He wants them to find a cave somewhere in western \Velcad{}, where they must do fulfill some as-of-yet unspecified task for him. 

\Ishnaruchaefir{} tells them that he cannot go off searching for the cave himself, as he would be detected and attract Cabal attention (and Sentinel attention, for that matter). But he teaches them an invocation that will summon him, to use when they reach the cave or are in dire need of his help. 

\hr{Ishnaruchaefir's fake mission}{The above is actually a lie.}







\subsection{\Shilred-tachi go into the \Wylde}
On their mission for \Ishnaruchaefir, \Shilred-tachi must go into the \Wylde{}. \Ishnaruchaefir{} throws them into some place which has a tenuous link to \Nyx. Perhaps a cave system that was once inhabited, or a ruined castle. It must be something sort of urban area, since \Nyx{} is city-like. 

From there, they go out into the \Wylde{}. They fight their way through monster-infested jungles. 

Maybe they encounter a native tribe. Maybe they help them, like how Karsa Orlong helps the Anibar in \emph{The Bonehunters}. Maybe they just coerce them into accepting them. Maybe they hire native guides, who later die. 

They are chased through the \Wylde{} by savage monsters, maybe zombies or lycanthropes. Monstrous almost-humanoids, at any rate. Occasionally they are harassed by the Cabal. 







\subsection{\Shilred-tachi find a ruined city}
\Shilred-tachi find a ruined city out in the wild, with broken walls, toppled towers, defaced monuments and statues and the like. It is decrepit, dead, moss-grown and spooky. 

Compare to the cemetery in the movie \cite{Movie:HouseoftheDead}. 

Remember to have some references: 
\ta{%
  There are many places of power in the world, places of convergence. Especially out in the \Wylde{}. After all, the \Wylde{} is far vaster than the settled world.}

Remember to have them fight monsters. Have a scene where \Dzasselid{} kicks the butts of loads of monsters. 







\subsection{\Shilred-tachi find a laboratory}
\Shilred-tachi fight past monsters and end up in the laboratory of some necromancer or witch-doctor. It is filled with mutilated corpses and deformed creatures\dash living, dead or undead. Like in the movie \cite{Movie:HouseoftheDead}. 







\subsection{\Shilred-tachi crawl around underground}
\Shilred-tachi find themselves underground in a system of dark, creepy, monster-infested caves. Like in the movie \cite{Movie:HouseoftheDead}. 








\subsection{\Shilred{}-tachi see great monsters in the distance}
%Remember \Shilred{} and her party. They need to do something intelligent. 
\Shilred-tachi, \travelling in the \Wylde{}, see some colossal creatures far off in the distance. They need to be at some special place or in special circumstances, being able to see into the Beyond, in order to be able to see them. 

Maybe they see the creatures on a background of \Nyx{} or \Machai{}, or maybe just the great mountains and forests of the Beast Realm (these are larger than the ones that can be seen from within the Shroud). 

Perhaps the creatures are \dragons{} and \banes{} fighting. 

Perhaps, in the end, they see \Ishnaruchaefir{} vs \Teshrial-tachi this way. 

Remember to drop hints along the way about how powerful and terrible \Ishnaruchaefir{} is. Perhaps when they see these monsters, they reflect about \Ishnaruchaefir{} and his awesome power. 







\subsection{\Narkiza{} on the move}
%Early on, we follow \Filshed{} and her party as they invade Fendor. 

We follow \Narkiza{}, first in the port city in \Scyrum, and later as his army pushes north into \Velcad{}. His invasion is one of several \quo{prongs} in the Sentinel scheme.

In \TwilightAngelRememberEmph, only little time is devoted to \Narkiza{} and the other Rissitics. They will be given much more screen time in \emph{\CarzainWithRedcorBook}. 
%in the scheme to awaken the \Haskelek, 

%\Filshed{}, maybe together with the two \Gisshorn, are sent as a second \quo{prong} on covert missions to revive parts of the \Haskelek. 

%A third \quo{prong} are the Rissitics that are with Morgan Runger. 

%There might be a fourth \quo{prong}, consisting of some non-Rissitic Sentinels. They might include \Cryocas{} and her lover, Nisgzarchiev. Or maybe Vlad Racul. 










\subsection{\Shilred-tachi reach the cave}
\target{Shilred-tachi reach the cave}
\Shilred-tachi finally reach their destination. 





\subsubsection{They see fantastic and terrible things}
They near the cave and travel through fantastic, alien Realms (perhaps parts of \hr{Machai}{\Machai}). Compare to \authorbook{Clark Ashton Smith}{The Door to Saturn}. 

They even see glimpses of the \xss. 
\Shilred{}, \hr{Shilred is a nerd}{being something of a nerd}, is amazed by what she sees. And horrified.

\lyricslimbonicart{When Mind and Flesh Depart}{
  I'm captured by the sound of the tempest winds, \\
  and my demonvoice that dwells within.\\
  The ground I walk is sour.\\
  A spectrum shines so genuine.\\
  While the earthquake starts to devour,\\
  I behold the shrine in the ruin.
}

She sees horrors and is disillusioned. 

\lyricslimbonicart{When Mind and Flesh Depart}{
  The human way of life, \\
  an inferior state of mind.
  The third stone from the Sun\\
  had become a stillborn illusion.
  
  I'm under siege of anger and fury.\\
  There is no values, no faith or glory.\\
  Antagonising mortal flesh.\\
  Life on Earth I do oppress\\
  with infernal bleedings of bitterness.
}






\subsubsection{\Teshrial{} moves out}
\Teshrial{} fails in his attempts to guess \ps{\Ishnaruchaefir} thoughts. 
When \Shilred-tachi ostensibly reach their destination\dash the main entrance to the cave\dash they are attacked by \ps{\Teshrial} monsters and must call on \Ishnaruchaefir. 
He appears in humanoid form to fight the monsters. 
\Teshrial{} is now convinced that \Ishnaruchaefir{} means business, so he rushes to his location in person, leading a group of mighty \banes{} and monsters. 





\subsubsection{Epic description of \Ishnaruchaefir{} as a \dragon}
\Ishnaruchaefir{} assumes his true, \draconian{} form. Insert an epic description of the mighty \dragon{} here, a la \bandsong{Bal-Sagoth}{Black Dragons Soar Above the Mountain of Shadows}. 

This is the first time we are shown a \dragon. 
Describe the mortals' awe at the sight. 
\Dzasselid{} might have seen one before, but \Shilred{} sure hasn't. 
She has heard \hs{myths} of \dragons{}, but reality surpasses the myths. 





\subsubsection{\Dzasselid-tachi are sent to the real cave}
\Ishnaruchaefir{} draws out all the Cabal monsters in the area. 

In the meantime, \Criseis{} sees to it that \Dzasselid-tachi are sent to the \emph{real} cave he is after\dash which is now mostly unguarded. 

They encounter some creatures that try to block their way. 
\Criseis{} zaps them.

\lyricsbalsagoth{
  The Splendour of a Thousand Swords Gleaming Beneath the Blazon of the Hyperborean Empire - Part II: The Dark Liege of Chaos is Unleashed at the Ensorcelled Shrine of A'zura-Kai
}{
  [THE KING:]\\
  By the darkling powers of the Shadow-Sword, \\
  I call forth the fury of the storm\\
  to rend the massed legions of Chaos!
  
  [ALTARUS:]\\
  And at the sound of his baleful Words of Power, \\
  the sky split wide in fury, \\
  and searing tendrils of ruinous lightning \\
  lanced inexorably forth from the heavens \\
  to rake and reave the massed hordes of Chaos\prikker
}

They find the creatures they are looking for and set them free. Thus \Ishnaruchaefir{} has tricked \Teshrial{}, and his circuitous plan has succeeded. The freed monsters now rush to \Malcur to assist in the summoning of \Nithdornazsh. 

See \hr{Ishnaruchaefir kills Teshrial}{here} for the end of the fight.






\subsubsection{\Shilred{} is left behind near the cave}
\Shilred{} accidentally gets separated from the rest and stays behind near the cave. 
She gets to see the fight between \Ishnaruchaefir{} and \Teshrial{} in all its gory glory. 















\section{Sir James story arc}
Sir James is a \human{} Iquinian knight living somewhere in \Velcad{}. 
It might be in the east, if I want him connected to the king-thing and Vlad Racul, or it might be in the west (Ontephar or \Scyrum) if I want him connected to Telcastora Ilcas. 
I think it's going to be in the east. 

There is also Sir Lawrence, another knight. 
Perhaps Sir James' companion. 

And Prince Alistair, who later becomes King Alistair. 
I think. 

And Vlad Racul, a mysterious sorcerer who is actually the \dragon{} Candrazor (a high-ranking Sentinel) in disguise. 

Sir James discovers a Rissitic plot somewhere. 
The Rissitics work on resurrecting a part of the \Haskelek{} \daemon. 
He goes to the Redcor for help. They set out to thwart the Rissitics. 
It turns into a big quest with all sorts of stuff going on. 
There's an evil Rissitic mastermind (who?) who keeps outsmarting them, but somehow the reader is assured that good will prevail. 
And then, in the end, I assfuck the reader by having the Rissitics win, and Sir James and his party of heroes are killed. 




\subsection{The beginning}
Sir James, as a knight and minor lord of sorts, is approached by the girl Lica, a commoner who's had visions, seen suspicous things. 
His advisors scoff at the girl's fanciful tales and persuade him to dismiss her, but he does so with doubts. 
Later stuff happens to him that convinces him she was right, so he seeks her out and brings her back. 

James or his companion asks Lica: 
\ta{How old are you, girl?}

\ta{I am sixteen, my lord.}

Later, James comments to his companion: 
\ta{She is pretty, though.}

The companion laughs. 
\ta{She is sixteen, James. 
  Even my warty cackling crone of a great-grandmother was pretty when she was sixteen.}

%Turns out that Lica has seen a \bane. The \banes{} are working on some big thing, so they are getting bolder and more careless. (It's for the same reason that Catrian saw a \bane{} earlier.) Or maybe she hasn't seen an actual \bane, but merely a Cabalist working magic. Or maybe some \Nyxian{} monster. 
Turns out that Lica has seen either a Sentinel mage or some \daemon. 
And it's a good thing that Sir James starts to believe her, because it turns out the Sentinels is after her. Assassins come for her and he has to protect her. They flee back to the castle or whatever, and the authorities (Redcor church?) finally believe that something is up. 

Lica is gifted in seeing into the worlds beyond. Since her childhood she has had trouble accepting the illusion that everyone else believes in. She always wanted to see through the lie, doubt what she was told. This has continued to the current day (she is now 16 or so) and lets her see many things that others can't, but it also makes her strange (she is unmarried and unbetrothed, and her parents worry) and affects her sanity. 

Alarmed by Lica's vision, the Redcor send a delegation to investigate. Lica is brought along as a guide of sorts, and Sir James goes as her protector. 









\subsection{Lica sees strange visions}
Lica sees grim and horrid visions. 
She hates this \quo{gift}. 

\lyricslimbonicart{The Supreme Sacrifice}{
  Thought are tyrants that always return\\
  to rape and torment the heart.\\
  As darkness sweeps the face of Earth,\\
  I enter the chambers of bleeding art.
}









\subsection{Lica sees the \Morbus}
Lica sees the \hr{Morbus}{\Morbus} at work. 
She has dreams of its evil consequences. 

She sees visions of \Nyx{} and mistakes them for visions of the future. 
She is scared shitless. 

\lyricsduana{thisistheend}{This is the End}{
  I dream of a host cockroaches \\
  crawling, swarming over charred flesh \\
  up bitter blackend walls \\
  in sewer pipes now purgd \\
  the cries of many amplifd \\
  then swallowd by silence \\
  echoes of a lost world \\
  of a foul smoke that lingers like stale breath \\
  of steel/glass/concrete fusion-statues \\
  rising high in a pale and infinite twilight \\
  shining beacons like bleachd bones 
  
  I see field upon field upon field \\
  of bloatd-cattle-corpses bleeding intestines \\
  staring with sightless opaque eyes \\
  of waves drowning mountains \\
  and the earth's flesh torn with angry fissures \\
  of molten-spewing-spittle \\
  and hot lashing tongues \\
  of a planet roaring/quaking/quivering \\
  with unbound vergence 
  
  shivers erupt through space and time \\
  and images flashing faster faster\prikker \\
  until finally all is annihilated 
  
  like a vacuum space sucks the debris \\
  into a black empty hole \\
  and I fall down into the void \\
  this is the end
}









\subsection{Lica suffers}
Lica feels like she is losing her innocence and being corrupted from the inside by the terrible truths she sees.

\lyricslimbonicart{Beneath the Burial Surface}{
  The sky is darkening, soon the night befall.\\
  Righteously angels are weeping for my soul.\\
  All childhood dreams are soon to be lost,\\
  all innocence to be shattered.
  
  I am the fallen from grace.
  
  My face is a river.\\
  See my eyes as they drown in black.
  My sacred doom and nemesis\\
  beneath the burial surface\\
  To the final act of the immortal sin\\
  I am lead by funeral winds.
}









\subsection{Lica can curse people}
Lica discovers at a young age that she can curse people. 
When people are mean to her or refuse to believe her, she can stare at them and hate them and wish them ill\dash and sometimes it happens.
This is a manifestation of her (albeit weak) \vertex{} powers. 























