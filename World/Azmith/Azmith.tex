
\part{The Realm of \Azmith}
\index{\Azmith}
\target{Azmith}
\Azmith{} is a Realm on \Miith{}, a fragment of the shattered realm of \Tembrae. 
The name \quo{\Azmith} also refers to the politically most important continent of the Realm. 

The continent consists of \hr{Velcad}{\Velcad}, the \hs{Northern Kingdoms}, \hs{Threll}, the \hs{Imetrium}, \hs{Uzur}, \hs{Durcac}, the \hs{Near Orient} and the \hr{Serpentines}{\Serplands}. %Some scholars also consider Irokas a part of \Azmith{} while others do not. 

\quo{\Azmith} is \hs{Archaic Vaimon} for \quo{all the world} (containing the word \quo{\Miith}, meaning \quo{the world}). 

\target{Azmith is important}
\Azmith was the most important of the \hs{Shrouded Realms}. 
This was the place where \hs{Silqua} was incarnated, and therefore place where the \hr{Vaimon Caliphate}{\VaimonCaliphate} first took root. 
As a result, \iquin was more strongly rooted here than in any other Realm (although religions parallel to \hs{Iquinianism} \hr{Iquinianism exported}{were exported} to other Shrouded Realms). 

This might be the reason why so many Scions incarnated in \Azmith. 
They were drawn to it as a \quo{centre of mystical convergence}. 

And it is the reason why it was in \Azmith{} that \Secherdamon{} chose to resurrect \Nithdornazsh. 








































\chapter{The Birth of the \VaimonCaliphate}















\section{\Asaathur}
\target{Asaathur}
\index{\Asaathur}
\index{Eryx!Ring of Eryx}
\index{Eryx!labyrinth}
\Asaathur was a fallen \ophidian nation. 
It was said that \Asaathur had fallen into ruin because the high \hs{Mysteriarch} went mad from sitting and gazing into the \hs{Ring of Eryx}.
This ring had been found in the mysterious invisible labyrinth of Eryx.















\section{Calaan}
\target{Calaan}
\index{Calaan}
A \human nation that existed at the time of Cordos Vaimon. 
A rival of \Imrath. 
The Calaanites might have been nomads. 

\hs{Silqua} was a princess of {Calaan}.

The name is inspired by \quo{Canaan}, a land mentioned in the Bible.















\section{Ematheroc}
\target{Ematheroc}
\index{Ematheroc}
A nation that existed at the time of Cordos Vaimon. 
A rival of \Imrath. 

The name is inspired by \quo{Kadatheron}, a city that is mentioned in \cite{HPLovecraft:TheDoomThatCametoSarnath}. 















\section{\Imrath}
\target{Imrath}
\index{\Imrath}
The \human kingdom to which \hs{Silqua} and \hs{Cordos Vaimon} belonged. 

\target{Imrathi nomads}
\Imrath{} was a nomadic people, like the Mongols. 
They waged wars against their rivals of \hs{Calaan}, \hs{Ematheroc} and \hs{Tulya}. 















\section{Lom}
\target{Lom}
\index{Lom}
Lom was a lake that lay near the place where the \hr{Imrath}{\Imrathi} dwelt.

By the shores of the mist-shrouded and murky lake of Lom there dwelt a race of repellent, half-\human things with huge hairless heads and fishlike, stupidly staring eyes.
The Lom-men prayed to the strange and ugly gods of the lake, chief among whom was \hs{Mnoth}, whose ichthyic features were awful to recall. 









\subsection{History}





\subsubsection{Destroyed by Cordos Vaimon}
The \humans of \Imrath had long hated the Lom-men and feared their wicked sorcery, until at last prince \hs{Cordos Vaimon} led an army against the Lom-men and \hr{Cordos conquers Lom}{defeated them}.





\subsubsection{Still haunted}
\target{Mnoth haunts Lom}
After Cordos Vaimon wiped out the Lom-men, the lake was still haunted and feared.
Once in a while the god Mnoth would still visit the lake, entering through arcane secret tunnels through the Beyond known only to the \hs{Lords of the Deep}. 
















\section{Nemaag}
\target{Nemaag}
\index{Nemaag}
A \scathaese nation that existed at the time of Cordos Vaimon. 
A rival of \Imrath. 

Nemaag was the most feared of \Imrath's rivals. 
It was a mystic realm of sorcery and shadows. 
Here the \ophidians still held sway, and Nemaag was said to be ruled by a \dragon-king. 















\section{Su-Gelba}
\target{Su-Gelba}
\index{Su-Gelba}
Su-Gelba was an \ophidian city.
It survived long as a bastion of sorcerers, even into the age of the \VaimonCaliphate. 









\subsection{The Tower of \Haamon}
\target{Aamon}
\target{Haamon}
\target{Tower of Aamon}
\target{Tower of Haamon}
\index{Tower of \Haamon}
The Tower of \Haamon was a mystical site existing somewhere in the dimensions. 
In ancient times it was the central citadel of Su-Gelba from where its terrible lords reigned. 

After the downfall of Su-Gelba the tower wandered the dimensions. 
It was co-terminous with outer spheres where the cosmic winds howled. 

In the tower dwelt a council of three immortal \ophidian philosopher-guru-mystics, mysterious and terrible. 

On the tower walls was encarved the powerful Symbol of \Haamon, as well as powerful runes in the language of \hs{Kush}. 

The tower had a mythical status among the \rethyaxes.
They hoped to catch a glimpse of it in their dream-meditations.
They hoped to glean a little of the wisdom of the terrible dark philosophers that dwelt in the tower. 















\section{Tulya}
\target{Tulya}
\index{Tulya}
A \scathaese nation that existed at the time of Cordos Vaimon. 
A rival of \Imrath. 















\section{Yaru}
\target{Yaru}
\index{Yaru}
A \scathaese nation that existed at the time of Cordos Vaimon. 
A rival of \Imrath. 








































\chapter{The \Human Age}
















\section{\Masthenon}
\target{Masthenon}
\index{\Masthenon}
A \hs{Tassian} \scathaese{} people that flourished during the time of the \hr{Vaimon Caliphate}{\VaimonCaliphate}. 

A single person is a \quo{\Masthen}, plural \quo{\Masthenon}, adjective \quo{\Mastheno}. 
\quo{The \Masthenon} is also the name of their civilization as a whole. 

The \hr{Ortaica}{\Ortaicans} descend from the \Masthenon. 
The \hr{Shurco}{\Shurco} have some \Masthenon{} heritage in their blood and culture, but also other things. 















\section[Sarun]{\Sarun}
\target{Sarun}
\index{\Sarun}
The \Saruns were a race of \humans that wielded great power in Cordos Vaimon's time. 









\subsection{Culture}





\subsubsection{Mages}
\Saruns \hr{Sarun intelligence}{were intelligent}. 
They valued learning, so they had a tradition of educating many mages.
With their power and wealth, they could also afford to educate mages. 





\subsubsection{Religion}
\target{Sarun religion}
The \Saruns worshipped brutal \draconian gods.
The Vaimons hated them for this (and for their cruelty). 





\subsubsection{Sultanate}
\target{Sarun sultan}
The \Saruns were ruled by a sultan.









\subsection{History}





\subsubsection{Empire}
At the time of \hs{Cordos Vaimon} the \Saruns wielded great political power. 
They were cruel and decadent rulers and \hr{Sarun religion}{worshipped evil gods}, so they were hated. 





\subsubsection{Defeated by Vaimons}
The Vaimons waged war against them. 
Eventually the Vaimons won.
They destroyed the \Sarun empire and wiped out almost the entire \Sarun race.




\subsubsection{Survivors}
After the fall of their empire only few \Saruns survived. 
They lived in hiding. 
They still worshipped their brutal and decadent gods. 









\subsection{Physique}
\Saruns had very pale skin, elongated heads with high foreheads, and white or gray hair. 
Due to their pale skin they were sensitive to sunlight.









\subsection{Psychology}
\target{Sarun intelligence}
\Saruns were often highly intelligent. 
But they also had a tendency to become insane and sadistic.
Compare to House Targaryen in \cite{GeorgeRRMartin:ASongofIceandFire}. 















\section{\Shurco}
\target{Shurcarie}
\target{Shurco}
\index{\Shurco}
\index{\Shurcarie}
The \Shurcarie{} was a \scathaese{} \hr{Mekrii}{\Mekrii} \hr{Bacconate}{\bacconate} that existed during the time of the \hr{Vaimon Caliphate}{\VaimonCaliphate}, controlling much of \hr{Durcac Continent}{\DurcacContinent}. 
It was defeated in a series of wars by \VizicarDurasRespina{} and others and did not last after the \hr{Hundred Scourges}{\darkfall}. 

Their heritage was carried on by the \Ortaicans{} and others. 

Modern-day \hs{Durcac} owes much of its cultural heritage to the \Shurco.









\subsection{Culture}





\subsubsection{Egyptian}
\Shurco culture should be compared to that of Ancient Egypt. 





\subsubsection{Mysteriarchs}
\target{Mysteriarch}
\target{Mysteriarchs}
\index{Mysteriarch}
The rulers of \Shurco were the dark and feared Mysteriarchs, a cabal of sorcerers with \ophidian blood. 
The greatest of them was called the Last Mysteriarch or the Ultimate Mysteriarch. 










\subsection{History}
The \Shurco{} were \Mekrii, but they also owed parts of their culture to the \hs{Tassian} \Masthenon, to whom they were also in part related. 















\section{\Tepharae}
\target{Tepharae}
A \hr{Bacconate}{\bacconate}, dominated by the \hr{Tepharite}{\Tepharite} people, that existed after the \hr{Hundred Scourges}{\darkfall}. 
It rose to prominence after the fall of \hr{Ortaica}{\Ortaica} but before the advent of \hr{Great Velcad}{\theBelkadianEmpire}. 
At the height of its power, it covered modern-day \hs{Pelidor}, \hs{Runger}, \hs{Ontephar}, \hr{Scyrum}{\Scyrum}, \hs{Beirod} and some more. 
The \Tepharin{} tongue is still spoken in parts of these countries. 

\Tepharae{} was \hs{Iquinian}. 
It was they who \hr{Telcra founded}{founded} the Vaimon \hr{Telcra}{\ClanTelcra}. 
\also{\hr{Tepharite}{\Tepharite}}









\subsection{Culture}
\subsubsection{Names: Western order}
\Ortaican{} traditionally have two names: 
A family/clan name and a personal name. 
The \emph{personal} name comes \emph{first}. 

This is like \Velcadian{} and Vaimon, but unlike \Ortaican{} and Imetric. 









\subsection{History}
\Tepharae (together with \hr{Telcra}{\ClanTelcra}) \hr{Tepharae succeeds Ortaica}{rose to prominence after the fall of \Ortaica}.















\section{Sturia}
\target{Sturia}
\index{Sturia}
Sturia was a land that existed during the time of the \VaimonCaliphate.
It was part of the \caliphate, but like many other lands it would sometimes try to secede or rebel and have to be beaten into submission.

\target{Sturiac}
\index{Sturiac}
The chief language spoken in Sturia was Sturiac. 









\subsection{Uruthar}
\target{Uruthar}
\index{Uruthar}
Before the mortal kingdom of Sturia rose, there lay in the same region the eldritch realm of Uruthar.
It was ruled by a group of \bezed \resphain, including \hr{Lethiarch}{\Lethiarch} and \hr{Osra}{\Osra}. 





\subsubsection{\Resphain outcasts}
\target{Uruthar Resphain cast out}
These \resphain originally dwelt with their brethren among the dynasties, but they grew dissatisfied with their subjugated position. 
They desired to rule as gods rather than serve as scorned half-breeds. 
They were also afraid of the sinister cosmic sorcery practised by the dynasties. 
They felt that their brethren had gone too far in their dealings with the horrible \SitraAchras.
\Lethiarch-tachi would have no part in this evil worship of soul-corrupting, world-devouring macrocosmic madness.
So they fled (or were cast out, however one chooses to see it). 





\subsubsection{Founding of Uruthar}
\target{Uruthar founded}
They set up shop in the Shrouded Realms and built an underground city for themselves.
Here the Shroud was weak, so they could dwell here in peace and still have access to their powers.
(It was easier for a \bezed than for a pureblood to live inside the Shroud.)
It was easy to coerce the nearby \humans to serve and worship them.
So the \resphan outcasts founded the horrid kingdom of Uruthar. 
There they reigned as gods and devoured the flesh and souls of their worshippers. 

Aesthetically, they were a kind of \hr{Resphan vampire lords}{vampire lords}. 





\subsubsection{Uruthar driven underground}
In the millennia before \hs{Cordos Vaimon}, Uruthar would wage war against the rival nations of \hs{Yaru} and \hs{Tulya} and \hr{Nemaag}. 
And the \resphain would clash with the \hs{Cabal}, who resented having rogue \resphain running around and causing trouble and undermining the Cabal's authority over the \resphan race. 

Eventually the brave \scathae of Yaru and Tulya conquered the vile \humans. 
And the \resphain of Uruthar fought against the \draconian gods of the \scathae, and they fought with treachery and rage and the dark sorceries of the unplumbed void, but they prevailed not.

\target{Totems against Uruthar}
In the end the gods and men of Uruthar were overpowered. 
They withdrew to their dark caves under the earth, and the serpentine gods of the reptile folk cast mighty spells and erected mighty totems that forced the \resphain to huddle underground forever, for they had not the strength left to sunder the mighty spells of the reptilian gods.





\subsubsection{Shunned deeper tunnels}
\target{Shunned tunnels in Uruthar}
The people of Uruthar had lived underground for thousands of years, nourishing their hate. 
But there were deeper tunnels where they dared not go. 
There were dark entrances to tunnels that appeared to lead down.
These entrances were blocked up with stone and metal and carved with warding symbols.
People walked in great circles around these blocked entrances, superstitiously afraid to come near them. 
These were entrances to the Realm of \hr{Thessulax}{\Thessulax}, the \Dragon of Death, the charnel Queen of the Underworld.





\subsubsection{The gods of Uruthar died}
\target{Gods of Uruthar died}
\target{gods of Uruthar died}
One by one the gods of Uruthar were slain.

\target{Osra dies}
Finally only two \resphain were alive:
\hr{Lethiarch}{\Lethiarch} and \hr{Osra}{\Osra}. 
They waged war against each other. 
\Osra was the more powerful, and eventually \Lethiarch was forced to submit and surrender. 
\Lethiarch had sex with \Osra in order to prove her submission to him. 
But she betrayed him and knifed him while they were having sex.
She slew him. 
Then \Lethiarch was the last \resphan in Uruthar. 









\subsection{The Hills of Uldor}
\target{Uldor}
\index{Uldor}
In the time of \VizicarDurasRespina, the kingdom of dark Uruthar was lost to the mists of legendry. 
Only the hills of Uldor remained, marking the site of the heart of fallen Uruthar.
(The name \quo{Uldor} was a derivative of \quo{Uruthar}.) 

These hills were ill-reputed and feared in the region around them. 
The forest-covered hills were home to tribes of abominable, degenerate, half-\human barbarians. 
They would attack wayfarers, and they would raid nearby villages and steal food and livestock and even kidnap people. 
It was said that they ate their captives and/or sacrificed them to their wicked gods.

These barbarians were descendants of the \humans of Uruthar. 
They were now sunk deep into savagery. 
Many hundreds dwelt in the hills and in the subterranean city beneath them, but they were a feeble shadow of the kingdom's former glory.





\subsubsection{The hillmen had supernatural powers}
The people dwelling nearby had tried to defend themselves, but the slinking savages seemed to have supernatural powers. 
They could disappear in the forest and reappear anywhere they wanted, and they even seemed able to walk through walls. 

This was because the degenerates had had their brains twisted from centuries of dwelling underground. 
They had lost much of their sanity and humanity and intelligence, but had gained some new powers. 
They could see through the Shroud now, which enabled them to take shortcuts. 
But they were insane and retarded and depraved.

Nobles had tried to send soldiers into the hills to wipe out the hillmen. 
The soldiers were wiped out with treachery and poison and sorcery and awful monsters. 







































\chapter[The Scatha Age]{The \Scatha Age}















\section{Andras}
\target{Andras}
\index{Andras}
A kingdom southwest of \hr{Velcad}{\Velcad}. 
It borders the \hs{Imetrium}, \hs{Threll}, \hs{Ontephar}, \hr{Scyrum}{\Scyrum} and the \hr{Risvael Sea}{\Risvaelsea}. 
The Threll Mountains mark its northern border with Threll and the rivers Bron and \hr{Pylor}{\Pylor} mark its eastern border with \Scyrum{}. 

Andras is ruled by King Tiberius Andras. 
The kingdom is allied with the Imetrium, and the Imetric religion thrives alongside \hs{Iquinianism}. 





\subsubsection{Demographics and languages}
Languages and ethnic groups included \hr{Samurin}{\Samurin}, \hs{Imetric} and the occasional pockets of \hr{Tepharite}{\Tepharites} and \hr{Ortic}{\Orticans}. 















\section{Beirod}
\index{Beirod}
\target{Beirod}
A kingdom in the southern \hr{Velcad}{\Velcad}. 
It borders \hs{Pelidor} to the northwest, \hs{Runger} to the northeast, \hr{Scyrum}{\Scyrum}{} to the west (marked by Heropond Forest), \hs{Gaznor} to the south, the \Risvaelsea{} to the southeast and the \hs{Lorn Sea} to the east.  





\subsubsection{Demographics and languages}
Languages and ethnic groups included \hr{Velcadian language}{\Velcadian} and \hr{Ortic}{\Ortic}. 















\section{Belek}
\index{Belek}
\target{Belek}
A kingdom in western \Velcad{}. 
Belek is also the name of the royal house of the land. 
Its most famous member was \hs{Uther the Tiger}. 

Borders \hs{Sumian} in the south and \hr{Thyrin}{\Thyrin} in the north. 









\subsection{Lendamere}
\target{Lendamere}
\index{Lendamere}
The capital city of Belek. 















\section{Clictua}
\target{Clictua}
\index{Clictua}
A \hr{Meccaran}{\meccaran}-dominated kingdom in northeastern \hs{Uzur}. 
One of the largest known kingdoms in Uzur. 
It borders the \hs{Imetrium} to the north and \hs{Durcac} to the east. 





\subsubsection{Demographics and languages}
The Clictua language is meant to resemble Nahuatl. 















\section{Geshiba}
\target{Geshiba}
\index{Geshiba}
One of the \hs{Northern Kingdoms}. 
It borders \hs{Rosval} to the west and the \hr{Serpentine Sea}{\Serpsea} to the east. 















\section{Gonarod}
\target{Gonarod}
\index{Gonarod}
A nation in northeastern \Velcad{}. 
It borders the \hs{Gwendor Sea} to the west, \hr{Marcil}{\Marcil} to the north and the \hr{Serpentine Sea}{\Serpsea} to the east. 















\section{\GreatVelcad}
\target{Great Velcad}
\index{\GreatBelkade}
A great dominated empire that existed from \yic{Founding of Belkade} to \yic{Fall of Belkade}. 
At the height of its power, it covered all the lands that are now \Velcad{}. 
The founder of the Empire was \hs{Cuthran the Victorious}, who became its first High King. 
The rulers of the Empire were the High Kings of House \Velcad. 

The Empire had strong ties to \hr{Telcra}{\ClanTelcra} from the beginning. 
They also had some ties to the Redcor. 

In \yic{Fall of Belkade}, the Empire collapsed for some reason. 
This shouldn't be too many years before our story begins. Perhaps only 5-10 years before Carzain is born. 
\also{\Velcadian{} (language)}





\subsubsection{Demographics and languages}
The official language of the Empire was \Velcadian{}, still spoken in large parts of \Velcad{}. 










\subsection{History}
\subsubsection{The fall of \theBelkadianEmpire}
It is still unclear and mysterious what actually happened at High King \ps{\LastHighKing} court. 

Perhaps \LastHighKing{} or his court Vaimons found out about the \Charade{} and the underground war, and acted to foil the factions' plans. 

Perhaps \LastHighKing{} or one of his advisors was planted by the \cuezcans{}. If so, he is still alive (or maybe undead) and at large, working underground to oppose the warring factions. 

Moro \Cornel, the Pelidorian archmage, might be in league with the fallen High King\ldots{}

The \hr{Kezerad}{\Kezeradi} might have had a hand in the fall of the Empire.

It might have had to do with \hr{Semiza}{\Semiza}. 





\subsubsection{\Velcad{} today}
\Velcad{} should be divided into a small number of large kingdoms, each of which is subdivided into multiple duchies, baronies and whatnot. Pelidor is one of the latter. 

The kingdoms need to be bigger so we can have bigger armies and wars on a more epic scale. 










\subsection{Noble titles}
\target{noble titles}
\index{noble titles}
The noble titles used in \Velcad{} include:

\begin{gloss}
  \gitem{King} 
    Used by some sovereign rulers, but not all. 
  \gitemnoindex{\Rayuth{}} 
    Was a vassal lord's title in \GreatVelcad. 
    Comparable to a duke. 
    After the fall of \GreatVelcad, some now-sovereign rulers retained the title. 
  \gitemnoindex{\Rinyuth{}}
    The husband or wife of a regent \rayuth{}. 
  \gitemnoindex{\Scarv{}}
    Below a \rayuth{}. 
    Comparable to a count. 
  \gitemnoindex{\Rah{}} 
    Used for \hs{knights}. Equivalent of \quo{sir}. 
\end{gloss}

Some of these titles are suffixed to names: 
\Rayuth[\Icor] Pelidor, \rah[\Sethgal] Pelidor. 








\subsubsection{High King}
\target{High King}
\index{High King}
\quo{High King} was the title taken by the kings of \theBelkadianEmpire. 
The first High King was \hs{Cuthran the Victorious}. 
The High Kings belonged to House \Velcad. 

A High King's wife was called High Queen. The Empire never had a ruling High Queen.















\section{Hazid}
\target{Hazid}
\index{Hazid}
A nation in the Near Orient, near Geica. Ruled by a Sultan. 















\section{Kochu}
\target{Kochu}
\index{Kochu}
A \hr{Meccaran}{\meccaran} nation in northwestern \hs{Uzur}. 





\subsubsection{Demographics and languages}
The Kochu language is meant to resemble Japanese. 
















\section{\Marcil}
\index{\Marcil}
\target{Marcil}
A nation in northeastern \Velcad{}. 
It borders \hr{Redce}{\Redce} to the west, \hs{Rosval} to the north, \hs{Gonarod} to the south and the \hr{Serpentine Sea}{\Serpsea} to the east. 









\subsection{Bendaire}
\target{Bendaire}
\index{Bendaire}
A city in \Marcil. 
















\section{Ontephar}
\index{Ontephar}
\target{Ontephar}
A kingdom in southern \Velcad{}. 
It borders Pelidor to the south, Runger to the east and \hr{Scyrum}{\Scyrum}{} to the south-west. 
It is ruled by an Archduke. 
Ontephar was previously the heartland of \hr{Tepharae}{\Tepharae}. 

Its capital city is \Tephar. 





\subsubsection{Demographics and languages}
Languages include \Velcadian{}, \Tepharin{} and occasionally Imetric.















\section{Pelidor}
\target{Pelidor}
\index{Pelidor}
A duchy in the central \hr{Pelidor Continent}{\PelidorContinent}.



\lyricstitle{Draft excerpt from the chapter \quo{What Slithers Beneath}}{
  Their path had led near to the centre of the city. The colossal \CastlePelidor{}\dash home of the \rayuth, ruler of Malcur and of all Pelidor\dash was visible even from the slums, looming forebodingly in the distance as a reminder of all that separated the lowlives from their highborn lords. But here, it towered over everything. Even though it was still distant and would take a long walk to reach\dash Rian guessed a thousand paces or more\dash, already its white marble walls seemed to span almost the entire horizon. Immense towers rose up to touch the sky, their brass-coated roofs gleaming like small suns. 
  Friezes and statues adorned the walls, of armoured knights, angels, beasts and dreadful monsters; some of them huge, many times taller than a man. 
  Rian found himself in the line of sight of a giant angel of marble\dash androgynous, winged, brandishing aloft a flaming sword, as if to strike with righteous anger down upon the thief's sinful head. He hurried on, eager to put some buildings between him and the avenging stone angel. 
}





\subsection{Demographics and languages}
\target{Pelidorian demographics}
\target{Pelidorian language}
The main language spoken in Pelidor was Pelidorian. 
It was related to \Ortaican and \Tepharin. 

Almost the entire nobility was \scathaese. 
Most slaves (but not all) were \humans.

Before the \thirdbanewar Pelidor had a population of maybe 500,000 to 1,000,000 people. 
It had a standard \hs{urbanization} of about 20\%. 







\subsection{House Pelidor}
\target{House Pelidor}
\index{House Pelidor}
\index{Pelidor!House Pelidor}
House Pelidor, a \scathaese{} noble family of \Tepharin{} descent, are the current rulers of \Malcur and the nation of Pelidor. 
Their head carries the title of \quo{\rayuth}.

The Pelidors were once worshippers of a group of lost gods (the Dreaming Gods?), perhaps \Tepharin{} gods. 
They were keepers of the occult secrets of what lay hidden underneath \Malcur. 
Charged to ensure that what lay beneath would not be awakened. 

But when the \Velcadians{} invaded, most of the Pelidors were killed as dangerous heretics and heathens, and a child prince was made \rayuth, now a \Velcadian{} vassal. The country was forcibly converted to the Iquinian faith and the secrets of \Malcur were lost. Many records were destroyed because the people should not know that kind of heresy. 

\Tiroco{} is also of House Pelidor. She is \ps{\Icor} half-cousin. She knows nothing of her family's legacy. 

But the secrets are not all forgotten. A select few keepers among the Pelidors have been handing down oral remembrance\dash always behind the \rayuth's back\dash all the while doing research and trying to rediscover more. 

Liocai Pelidor, the sister of \rayuth[\Icor], is one such keeper. She knows a shocking amount of things. Things that even the Sentinels don't know. 
 
This might be a good place to tie in the story of the mysterious fall of \hr{Great Velcad}{\theBelkadianEmpire}. 









\subsubsection{Symbol: \Grulcan}
\target{Pelidor symbol}
The symbol of House Pelidor is a bronze-\\coloured \hr{Grulcan}{\grulcan} bird on a blue background. They also \hr{Pelidorian war Grulcans}{use \grulcans{} as beasts of war}. 









\subsection{House Turmalin}
The second-most powerful noble house in Pelidor, after House Pelidor. 
They control much of southern Pelidor. 

Their head is Viscountess Osphal Turmalin. 

Their symbol is three parallel black spears on an orange background. 









\subsection{Besuld}
\index{Besuld}
\index{Pelidor!Besuld}
A port city in southeastern Pelidor. 











\subsection{Dendrum}
\index{Pelidor!Dendrum}
\index{Dendrum}
\target{Dendrum}
A city in eastern Pelidor. 
Lies on the \hs{Ucarn}, approximately halfway between \hr{Forclin}{\Forclin} and \hs{Ucarnum}. 











\subsection{Equipment and creatures}






\subsubsection{\Grulcans}
\target{Pelidorian war Grulcans}
The \hr{Pelidor symbol}{symbol of house Pelidor} is a \hr{Grulcan}{\grulcan} bird. 
They tame these birds and use them as beasts of war. 
The bird also live in the \wylde in Pelidor and surrounding lands. 









\subsection{\Forclin}
\target{Forklin}
\target{Forclin}
\index{Forklin}
\index{Pelidor!Forklin}
A city in northern \hs{Pelidor}.
It was built by the \hr{Ortaican}{\Ortaicans}. 

In accordance with \hs{Pelidorian demographics} and standard \hs{urbanization}, \Forclin might have had a population of 12,000 to 20,000 before the war with Runger.

It was built on top of the ruins of an ancient \resphan{} fortress, of which the \hs{Ghost Tower} was the only surviving remnant. 

Describe the colossal walls and towers of \Forclin.
It was a majestic city, but dark and mystical. 
To the eyes of an Iquinian it was not as nice and clean as the Vaimon-built \Malcur. 
It was more martial and barbaric and sorcerous and intimidating. 

But it was much younger than \Malcur. 
Only the Ghost Tower was truly ancient. 

The gargoyles on the walls were magical.
Maybe they could spit fire or napalm or other nasty stuff. 
Some gargoyles crawled on the walls, others sat on top of them. 

\citeauthorbook[p.24]{RobertEHoward:TheShadowKingdom}{Robert E. Howard}{%
  The Shadow Kingdom%
}{
  The \colour and the gayety of the day had goven away to the eery stillness of night.
  The city's antiquity was more than ever apparent beneath the bent, silver moon. 
  The huge pillars of the mansions and palaces towered up into the stars.
  The broad stairways, silent and desrted, seemed to climb endlessly until they vanished in the shadowy darkness of the upper realms.
  Stars to the stars, thought Kull, his imaginative mind inspired by the weird grandeur of the scene. 
  
  Clang! clang! clang! sounded the silver hoofs on the broad, moon-flooded streets, but otherwise there was no sound.
  The age of the city, its incredible antiquity, was almost oppressive to the king; it was as if the great silent buildings laughed at him, noiselessly, with unguessable mockery.
  And what secrets did they hold? 
  
  \ta{You are young,} said the palaces and the temples and the shrines, 
  \ta{but we are old.
    The world was awild with youth when we were reared.
    You and your tribe shall pass, but we are invincible, indestructible.
    We towered above a strange world, ere Arlantis and Lemuria rose from the sea; we still shall reign when the green waters sigh for many a restless fathom above the spires of Lemuria and the hills of Atlantis and when the isles of the Western Men are the mountains of a strange land.
    
    How many kings have we watched ride down these streets before Kull of Atlantis was even a dream in the mind of Ka, bird of Creation? 
    Ride on, Kull of Atlantis; great shall follow you; greater came before you.
    They are dust; they are forgotten; we stand: we know; we are. 
    Ride, ride on, Kull of Atlantis; Kull the king, Kull the fool.}
  
  [\ldots]
  
  Glow, moon: you light a king's way.
  Gleam, stars; you are torches in the train of an emepror!
  And clang, silver shod-hoods; you herald that Kull rides through Valusia. 
}









\subsection{Ghost Tower}
\target{Ghost Tower}
\index{Ghost Tower}
\index{Pelidor!Ghost Tower}
A spooky tower in \hr{Forclin}{\Forclin} in \hs{Pelidor}

There are a bunch of ancient ruins near the Ghost Tower\dash parts of a larger complex, of which the tower is the only complete remnant. 






\subsubsection{\Haskelek{} myth}
\target{Haskelek myth}
There are myths about how there allegedly lies a terrible \daemon{} imprisoned somewhere in the Ghost Tower\dash a \hr{Haskelek}{\Haskelek}. 





\subsubsection{Haunted}
\index{\succubus}%
It is probably haunted by monsters. 
These monsters could only rarely get through the Shroud before, but it is getting easier now with the \hs{unravelling}. 
Every now and then they manage to slink out of the Beyond and eat some people from \Forclin. 
Or lure the people to them, \hr{Succubus}{\succubus} style. 





\subsubsection{History}
\target{Ghost Tower history}
Many people assume the Tower is Vaimon-built, since it is older than the \Ortaican{} buildings around it. 
After all, the Vaimon age was long and featured diverse building styles. 
But this is actually not true. 
The Tower and the ruins are \resphan-built.
The Tower was originially the central tower of a larger fortress complex, but the rest has since decayed. 
The Tower, being virtually indestructible, has remained, but is now empty and abandoned. 





\subsubsection{Invisibility}
The Tower is invisible from within \Forclin, because the people repress the fact that it exists. 
Their collective denial creates a Shroud that renders it invisible, even to outsiders. 

The Tower can be clearly seen from outside \Forclin, though. 
Out there the Shroud of Civilization does not hold the same sway. 

When you are up close to the Tower, it can be seen. 
It stretches up above, but after about 5-10 metres it gets shrouded in fog so that you can't see the top or estimate how tall it is. 
It is actually over 100 metres tall. 





\subsubsection{Mystic lake}
\target{Mystic lake near the Ghost Tower}
Maybe there is a mystical, magical lake near the Tower. 
There might be \hr{Mystic lake in Malcur}{a similar lake in \Malcur}.

Compare to Lake Hali in \RWCTKIY. 





\subsubsection{Portal to \Nyx}
\target{From Ghost Tower to Nyx}
The upper levels of the Tower submerge into \Nyx. 











\subsection{Gilwaed}
\index{Pelidor!Gilwaed}
\index{Gilwaed}
\target{Gilwaed}
A village in eastern Pelidor, north of the \hs{Ucarn}, somewhere between \hr{Forclin}{\Forclin} and \hs{Ucarnum}. 
The name is \Tepharin. 

Population: Approximately 100 \scathae. 











\subsection{Kenshaer}
\index{Pelidor!Kenshaer}
\index{Kenshaer}
\target{Kenshaer}
A \hr{Wild}{\Wylde} forest in eastern Pelidor, north of the \hs{Ucarn} between \Forclin{} and \hs{Dendrum}. 









\subsection[Malcur]{\Malcur}
\target{Malcur}
\index{\Malcur}
\index{Pelidor!\Malcur}
\Malcur was the capital city of Pelidor, ruled by House Pelidor from Castle Pelidor in the heart of \Malcur. 

\Malcur was an ancient city. 
\CastlePelidor was built by Vaimons, not long after Cordos's time.

And the catacombs beneath were are thousands of years older. 
They were disused and forgotten, and the Pelidors barely remembered that they existed. 





\subsubsection{Ancient history}
\target{Ancient history of Malcur}
\Malcur was originally a \hr{Wild}{\Wylde}{} stronghold, and the crypts date back to this time. Or maybe it was built on top of a \xzaishann{} tomb, containing powerful arcane glyphs and spells forming a part of the seal that separates \Machai{} (the \hr{Wild}{\Wylde}{} homeworld) from \Miith\dash the very spells that formed the first foundation of the Shroud. The magic of these seals can be tapped for other purposes and used to weave the Shroud, making \Malcur and its underground a potent \nexus{} point. 

Some terrible power lies entombed beneath \Malcur. After the \Darkfall, someone attempted to release this power, but it failed. House Pelidor took over the castle and have been its stewards ever since. 

\hr{Teshrial's creatures}{\ps{\Teshrial} terrible monsters} are connected to \Malcur's mystic past somehow.





\subsubsection{Black Plague}
\index{Black Plague}
\index{plaguer}
\target{Black Plague}
\target{plaguers}
A thieves' guild in \hr{Malcur}{\Malcur}. 
Rumoured to have dealings with sorcerers. 
Their members are sometimes nicknamed \quo{plaguers}. 





\subsubsection{Churches and mausolea}
\Malcur had a great \Telcra{} cathedral and a smaller Redcor church, each with a mausoleum attached. 
These mausolea were huge. 

There was also a smaller Imetric church. 





\subsubsection{Description of the scenery}
The great buildings of \Malcur, including the city walls, guard towers and the central \CastlePelidor, are built of shining white stone\dash think of a name for this fictional type of stone. 

It has plenty of \hr{Monuments}{monumental structures}. 
In many places it is adorned and embossed with pink quartz and black opals. And there are enormous statues, obelisks, gargoyles or \hr{Iquinian angels}{angels}, friezes on the walls, ornate cathedrals and towers, and a behemoth central castle.

Describe it so it sounds exotic, epic and Bal-Sagoth-like. Almost like Kor-Avul-Thaa (but not quite).

\Malcur is built utilizing a lot of \hr{Occult geometry}{occult geometry}. 





\subsubsection{Rich and poor quarters}
\target{Malcur rich and poor}
\Malcur has a rich quarter, the High City, and a poor quarter, the Low City. 

The High City is made of very tall houses, palaces and towers. There are bridges between the towers so that the rich need never tread on the filthy ground where the poor live. 

The Low City spans everything from comfortable craftsmen to the worst slums. \hs{Rian} lives here.

There was an inner city, encircled by walls, which was mostly safe from monsters, and an outer city outside the walls, with only walls and \eidola for protection.
The slums were, of course, outside.

The inner city was very densely populated and crowded, since everyone who could afford it wanted to live there.
So there were also slum-like places in there. 





\subsubsection{Recent history}
\Malcur has never fallen in war and never been taken in a siege. At least, not since the Vaimons destroyed the city and rebuilt it. When the \Velcadians{} invaded, the \rayuth of Pelidor surrendered peacefully after lengthy negotiations. 





\subsubsection{\Nexus{} status}
\Malcur is an important \nexus{}. This is why \hr{Teshrial's creatures}{\Teshrial{} has brought \ghobaleth{} to \Malcur}. 

Perhaps these \ghobaleth{} have been there for thousands of years, or have gradually dug their way to \Malcur from Erebos over the course of thousands of years. 

But apart from the \ghobaleth, there is also \hr{Wild}{\Wylde}{} darkness lurking beneath \Malcur. And perhaps degenerate creatures who live in caves underground and worship them. 

Maybe there dwell degenerate \troglodytes, digging tunnels under the crypts and worshipping their detestable gods, perhaps including \dragons.Or perhaps these are \nagae, living in undergound lakes and the like.





\subsubsection{Dead garden}
\target{Dead garden in Malcur}
\target{dead garden}
\index{\Malcur!dead garden}
Somewhere in the centre of \Malcur, near Castle Pelidor (but not too near), there lies a garden of crooked, dead trees. 
It is one of the mystic centres of the city, somehow connected with \Malcur's occult past. 

The twisted trees are black and gray, leafless and withered-looking. They look dead, but they are alive. They radiate some kind of malice, and people steer clear of them. Only this one strange old woman lives near them. 

In or near the garden there lives a \hs{crazy old woman}. 
She knows quite a lot about the Beyond and the supernatural, but is also quite \hr{Madness}{mad}.

The garden is inspired by the Azath house and cemetery in \cite{StevenErikson:MidnightTides}. 

The dead garden is a small patch of \wylde in the middle of the city, which everyone tries to forget about (because otherwise they would be living in fear of it all the time).

The garden is the core of the \nexus. It is here that one can most easily feel the \hr{Teshrial's creatures}{worms}. 

Despite their wicked look and feel, the trees are not necessarily evil, although they may tend to be inimical to humanoids. They possess some measure of natural power and wisdom. The crazy old woman can hear their thoughts (because she is mad and more free from the Shroud), and she relays their warnings to anyone who will listen (which is not many). The twisted trees may not love humanoids, but they fear the \ghobaleth{}, and they fear \Nithdornazsh.

The Shroud is thin in the garden. 
This allows the immortals to use their full powers there, \hr{Immortals inside the Shroud}{which would otherwise be impossible}.

\lyricsbalsagoth{
  In the Raven-Haunted Forests of Darkenhold, Where Shadows Reign and the Hues of Sunlight Never Dance
}{
  Can you not see the coils of the worm all about you?\\
  Can you not hear the writhing of the worm beneath you?\\
  Can you not scent the breath of the worm riding the wind?\\
  Can you not touch the skin of the worm in all that surrounds you?\\
  Can you not taste the ichors of the worm upon your tongue?\\
  Do dreams of the worm not haunt your slumber?
}

There is a \hr{Wylde border}{\Wylde{} border} around the dead garden. 
It is, after all, a small patch of \wylde{} in the middle of \Malcur.
    
The dead garden is twisting and crawling with chaos.
It is monstrous and haunted.
Being \wylde, the garden twists and mutates.
Paths are never the same.

When describing the garden, remember to see the section about the \hr{Wylde}{\wylde}, and especially the one on \hr{Travelling through the Wylde}{\travelling through the \wylde}. 

There was a great hole in the middle of the dead garden.
Simply called \quo{the hole}. 
Not capitalized. 
The dead garden and things in it did not have names. 

The hole appeared to be bottomless and just went on down in the darkness. 
It was a passage into the Beyond. 

In the foremath of the \thirdbanewar, \Ishnaruchaefir tried to use the hole as a passage into \Azmith. 

\citeauthorbook[p.196]{RobertEHoward:SwordsofthePurpleKingdom}{Robert E. Howard}{%
  Swords of the Purple Kingdom%
}{
  But as night fell and the shadows merged, Delcarted grew nervous.
  The night wind whispered grisly things through the branches and the broad palm leaves and the tall grasses and the stars seemed cold and far away.
  Legends and tales came back to her, and she fancied that above the throb of her pulsing heart, she could hear the rustle of unseen black wings, and the mutter of fiendish voices. 
  
  \ldots 
  
  She stole along a broken pave, and the whispering palm leaves brushed against her like ghostly fingers.
  About her lay a pulsation gulf of shadows, vibrant and alive with nameless evil.
  There was no sound.
}







\subsubsection{Mystic lake}
\target{Mystic lake in Malcur}
\target{Lake in Malcur}
Maybe there is a mystical, magical lake near the garden. Maybe it lies between the garden and \CastlePelidor. 
There might be \hr{Mystic lake near the Ghost Tower}{a similar lake near the Ghost Tower}.

Compare to Lake Hali in \RWCTKIY. 









\subsection{Politics}





\subsubsection{\Meccaran conscripts}
The Pelidorian army can draft some local tribes of \meccara{} that live in eastern Pelidor. 
They are mostly independent, but nominally they owe allegiance to the \rayuth. 
They are employed as guerillas, since they know the terrain. 

The various tribes each speak their own languages, and only few understand \Velcadian{} or \Tepharin. 
They are ill-integrated and have little love for the \rayuth or \rinyuth or whatever. 
They have their own languages and cultures and customs and do not feel like a part of Pelidor. 









\subsection{\Redglen}
\target{Redglen}
\index{\Redglen}
\index{Pelidor!\Redglen}
A town in eastern Pelidor, near \hs{Heropond Forest} and west of \hs{Torgin}. 
It is the hometown of \hr{Carzain}{Carzain \Shireyo} and his family. 









\subsection{Sulcanum}
\target{Sulcanum}
\index{Sulcanum}
\index{Pelidor!Sulcanum}
A city north of \Malcur. 
Ruled by the family of \hs{Sethgal Pelidor}. 









\subsection{Torgin}
\target{Torgin}
\index{Torgin}
\index{Pelidor!Torgin}
A city in southern Pelidor, east of \Redglen. 









\subsection{Ucarn}
\target{Ucarn}
\index{Ucarn}
A river in eastern \hs{Pelidor}. 
Runs from \hr{Forclin}{\Forclin} in the west out to the port city of \hs{Ucarnum} at its mouth in the east, where it meets the (much wider) \hs{Nerim}. 









\subsection{Ucarnum}
\target{Ucarnum}
\index{Ucarnum}
A city in eastern \hs{Pelidor}. 
Lies at the mouth of the river \hs{Ucarn} where it runs into the \hs{Nerim}. 















\section{Rosval}
\target{Rosval}
\index{Rosval}
One of the \hs{Northern Kingdoms}. 
It borders the \hs{Gwendor Sea} to the west, \hr{Redce}{\Redce} to the southwest, \hr{Marcil}{\Marcil} to the south, the \hr{Serpentine Sea}{\Serpsea} to the east and \hs{Geshiba} to the northeast. 















\section{Runger}
\target{Runger}
\index{Runger}
A kingdom in central \hr{Velcad}{\Velcad}. 
It borders \hs{Beirod} to the south, \hs{Pelidor} to the west (marked by the river \hs{Nerim}), the \hs{Gwendor Sea} to the north and the \hs{Lorn Sea} to the east. 

%It borders the Hirum Gulf to the northwest, Pelidor to the southwest and Beirod to the south. 

It is ruled by the kings of House Runger. 
The current ruler is \hr{Morgan Runger}{King Morgan I son of Uther I}. 
Morgan has two daughters but no sons, so the heir to the throne is Prince Matthias, the husband of Morgan's eldest daughter, Estelle. Recently, King Morgan has allied himself with the Rissitics and plots to conquer Pelidor and perhaps other nearby kingdoms as well. 

The banner of Runger is two brown wolverines chasing each other in a circle, against a tan background.









\subsection{Demographics and languages}
\target{Rungeran language}
The main language spoken in Runger was Rungeran. 
It was a \human language resembling English. 

Or maybe they spoke the \hr{Vidran language}{same language as in \Vidra}.









\subsection{Dormina}
\index{Dormina}
Capital city of Runger. 









\subsection{\EreshKal}
\target{Eresh-Kal}
\index{\EreshKal}
A \meccaran{} civilization that once existed in modern-day Runger. Allegedly, a lost \EreshKali{} temple lies hidden in \hs{Waythane Forest}.  









\subsection{Gedrock}
\target{Gedrock}
\index{Gedrock}
A village in Runger, near \hs{Waythane Forest}. 









\subsection{Manburg}
A city in Runger. 









\subsection{Waythane Forest}
\target{Waythane Forest}
\index{Waythane Forest}
\index{Runger!Waythane Forest}
A large forest in southern Runger. Allegedly hides the last temple of \hr{Eresh-Kal}{\EreshKal}. 















\section[Scyrum]{\Scyrum}
\target{Scyrum}
\index{\Scyrum}
A kingdom southeast of \Velcad{}. 
Borders the Imetrium to the south, near Martinum, and borders Pelidor to the north-east. 
Much of the eastern part of the kingdom is made up by the large Heropond Forest. 

The capital city is \hr{Pylandos}{\Pylandos}. 
Other cities include Icconos. 





\subsubsection{Demographics and languages}
Languages and ethnic groups include \Tepharin, \Samurin{} and \Ortic. 









\subsection{\Bryndwin}
\index{\Bryndwin}
\target{Bryndwin}
A village in eastern \Scyrum, near \hs{Heropond Forest}. 









\subsection[Pylandos]{\Pylandos}
\index{\Pylandos}
\target{Pylandos}
Capital city of \Scyrum. 















\section{Shamanistic cultures}
\target{Shamanistic magic}
\index{Shamanism}
Shamanism is a religion, or, more properly, a collective term for a class of religions, practiced primarily by the \meccaran{} tribes, but also in some communities of other creatures. %as well as some communities of Humans. A few communities of Scathae have adopted Shamanistic beliefs, and some Troglodytes practice religions similar to Shamanism. 

Shamanism exists in many diverse forms and is not in any way a centralized or even homogeneous religion. Some forms of Shamanism are very peaceful (or even pacifist) and resemble Animism\footnote{Needless to say, my usage of the terms \quo{Shamanism} and \quo{Animism} in a \Miithian{} context is meant to be similar to, but also different from, the RL religions of the same name. In RL, shamanism is perhaps more properly considered a form of animism than merely something similar.}, wheras others are very dark and sinister, employing \quo{black} magic and consorting with monstrous powers. 

Shamanism is based on the belief that the world is full of spirits, that all things have a soul. Shamanistic magic works by communing with and influencing these spirits. 

Beliefs include: 

All people, animals and things have a soul. These souls can be communed with, and done other stuff to\ldots{} 

%Shamanistic magic is a tradition used primarily by the \meccaran{} tribes, especially in Uzur, but also in the Middle Lands. It is often considered primitive and barbaric by other cultures, and it is true that most Shamans have little theoretical understanding of how their magic works. Still, the Shamans are not to be underestimated, for their magic can be very effective. 

%Some aspects of Shamanistic magic and religion: 

There are Shamans and Witch-Doctors. (The distinction is not always rigid.) A Shaman is a religious authority whose magic is seen as something holy. A Witch-Doctor is an alienist who deals with dark, occult powers. He is primarily a mage, and perhaps secondarily a priest. In many tribes, Witch-Doctors are feared and mistrusted (but they are usually still there, because their magic is potent). 









\subsection{Examples of cultures}
Shamanistic cultures include: 

\begin{itemize}
  \item \hs{Clictua}. 
  \item \hr{Goyden}{\Goyden}. 
  \item \hs{Kochu}. 
\end{itemize}









\subsection{Shamanistic magic}
Shamans usually cannot cast \quo{spells} in a regular sense. Their magic relies either on rituals or on herbs and drugs. Witch-Doctors can usually cast regular spells. 

Their rituals revolve around dance and music. Some spells can be performed by the Shaman alone, but most require many people. 

\Meccaran{} singing usually uses a chanting choir and a lead singer. The chanting is rhythmic but not melodic. It may be with understandable words, strange, magical words or just nonsense. A lead singer is typically a shaman. \Meccaran{} lead singing is high-pitched and screaming, often wild and ecstatic. This often sounds extremely chaotic and un-musical to outsiders. 

Typical rituals are war dances, hunting dances and exorcisms. 

In a hunting ritual, the people seek out the great spirit of the prey animal. In some interpretations, the shaman peacefully communes with the spirit and obtains permission to kill some of the animals (this is granted because it is the nature of prey animals to get eaten). In other interpretation, the shaman duels with the spirit and takes his \quo{permission} by force. 

In a war ritual, the people seek out the guiding spirit of their enemies and combat it. If successful, the enemy is weakened and the tribesmen get bonuses to fight them. 

They eat the bodies of their dead (under a magic ritual), so that their souls return to the tribe. 

They eat the bodies of their enemies (under a different ritual) to gain their strength. (Only the blood and important organs, especially the heart and brain, have occult properties. The rest of the body is just eaten for nourishment.) 

Living people may drink each other's blood in a ritual, or even eat each other's flesh\footnote{For a \meccaran, it is easy to remove a finger or so, since it will regenerate.}, creating an arcane blood bond between them. 

Warriors have a totem animal and draw power from it. 

They use drugs a lot, to help the soul wander and to access spiritual power. Warriors sometimes take drugs to go berserk. 









\subsection{Shamanistic spells}
%Some sample spells: 

Here is a list of sample Shamanistic spells. Note that since Shamanistic communities are a very heterogeneous group, any given Shaman or Witch-Doctor need not have access to all of the spells below. 

\spellsha{Control Weather}
Shamans can control the weather to a limited extent. Requires a tribal ritual. 

\spellsha{Curse of Despair}
Causes depression and despair in the victim. 

\spellsha{Curse of Sloth}
Makes the victim lazy. 

\spellsha{Curse of Disease}
Causes disease\ldots{} 















\section{Sumian}
\target{Sumian}
\index{Sumian}
A nation in eastern \hr{Velcad}{\Velcad}, east of the \hs{Lorn Sea} and west of the \hr{Serpentine Sea}{\Serpsea}. 
Borders \hs{Belek} in the north. 















\section[Thyrin]{\Thyrin}
\target{Thyrin}
\index{\Thyrin}
A nation in eastern \hr{Velcad}{\Velcad}, east of the \hs{Lorn Sea} and west of the \hr{Serpentine Sea}{\Serpsea}. 
Borders \hs{Belek} in the south. 















\section{\UltimaThule}
\target{Ultima Thule}
\target{Thule}
\index{\UltimaThule}

\UltimaThule was an Immortal Realm. 
It was cold and full of ice.
It was closer to the \CrystalSphere than most any other place in \Miith, and a particularly good place for the \banelords to exert their influence, and for \Miithians to contact them or to draw power from Erebos.
The \resphain feared and disliked the place (just as they feared the \banelords), so while they had bastions there, they did not like to spend time there. 

It was to \UltimaThule that \hr{Ramiel journeys to Thule}{Ramiel journeyed to regain his memory and power}. 

The nothern regions of \Azmith gradually transitioned to \UltimaThule.
The lands were full of monsters and cold magic.
The inhabitants (\human or \scatha) were fierce barbarians who prayed to dark gods. 
Not only the \banelords had influence, but also the \xs. 
The Shroud was thin in the north, because the land gradually transitioned to an Immortal Realm. 

\Gnomphilim dwelt here. 

The Realm was ruled by cold, sleeping Ice-Gods (who might be \xss or \dragons) and cruel, cold \hs{Vampire Lords} or \quo{Cold Wraiths}. 

There were also \quiljaaran that dwelt there. 

In \UltimaThule one could feel the cold winds of darkness from Erebos.
They spoke of desolation, despair and entropy.

\lyricsbalsagoth{
  Starfire Burning Upon the Ice-Veiled Throne of Ultima Thule
}{
  The grim Ice-Gods sleep in these frost-bound tombs,\\
  illumined by the caress of lunar fire\\
  and the kiss of star-gleam from the Stygian void.
}

\citebandsong{Arcturus:AsperaHiemsSymfonia}{Arcturus}{
  To Thou That Dwellest in the Night
}{
  Here is so desolate;\\
  Times, they are dark\\
  Words ceas'\\
  to end as echoes rolling afar\\
  Empathy asseth\\
  whilst thou drapest this world in black;\\
  The only \colour that can paint my soul
  
  Clad in the shades of night\\
  Thou reflectest the pure of heart
  
  Amidst all the grief this winter unfoldeth\\
  The thorn in my side\dash{}thou retainest\\
  Thy breeze maketh me shiver\\
  Maimeth me with its frozen malice
  
  Thou minglest with the dense night\\
  I hearken to the voice of thy winds\\
  They are the saddest of all sounds of thine\\
  Never will I take leave from thy haunt
  
  Hast thou ever desired me?\\
  I receive no answer, \\
  thou letst it pass in silence\ldots{}
}















\section{\Vidra}
\index{\Vidra}
\target{Vidra}
A kingdom in northern \Velcad{}. 
\hr{Great Velcad}{\GreatVelcad}, during its time, was ruled from \Vidra{} by the \hr{High King}{High Kings} of House \Velcad. 
House \Velcad{} still rule the kingdom of \Vidra, but they no longer use the title High King. 
\also{\hs{High King}, \hr{Great Velcad}{\GreatVelcad}}









\subsection{Demographics and languages}
\target{Vidran language}
The main language spoken in Vidran was Vidran. 
It was a \human language resembling English. 

Or maybe they spoke the \hr{Rungeran language}{same language as in Runger}.















\section{\Yormis}
\target{Yormis}
\index{\Yormis}
\Yormis was an \Ortaican city.
After the fall of \Ortaica and the rise of \Iquinian \Tepharae, \Yormis did not surrender and remained a bastion of \hr{Rethyax}{\rethyaxes} that maintained the \Ortaican religion.
It became known as a dark city of sorcerers, and there were many unwholesome rumours about them.









\subsection{Appearance}
There were some dark, scary catacombs underneath \Yormis.
These led to \KaiLeng where dwelt \Ubloth and \Thessulax. 

\citeauthorbook[p.225]{RobertEHoward:HouroftheDragon}{Robert E. Howard}{%
  Hour of the Dragon%
}{
  He followed his awful guide through blackness that loomed before and behind them and was filled with skulking shapes of horror and lunacy that cringed from the blinding glow of the Heart.
}









\subsection{Geography}
\Yormis lay in northern \Velcad, near Pelidor.

It lay near the mountain of \hr{Shrun}{\Shrun}.





\subsubsection{\Shrun}
\target{Shrun}
\index{\Shrun}
\Shrun was a mountain upon whose slopes \Yormis was built.
The founders built their city here for two reasons:
\begin{enumerate}
  \item 
    The mountain was full of valuable ore for them to mine.
  \item 
    It was a great defensive position. 
    The mountain was covered in dangerous \wylde infested with \hr{Werloc}{\werlocs}. 
    (The people of \Yormis greatly feared these \werlocs.)
\end{enumerate}


Under the mountain was a system of tunnels that led down into the underworld of \hr{Kai-Leng}{\KaiLeng}. 
Here dwelt the \xs godling \hr{Ubloth}{\Ubloth}.





\subsubsection{Well}
\target{Well in Yormis}
Near Mount \Shrun there lay a \quo{lake} or \quo{well} of viscous black slime. 
The slime was an emanation of the loathsome god \hr{Ubloth}{\Ubloth}. 
By touching the slime, drinking it or submerging into the well, one could commune with \Ubloth and learn its secret \arcana. 

\citeauthorbook[p.103]{RobertEHoward:TheScarletCitadel}{Robert E. Howard}{%
  The Scarlet Citadel%
}{
  \ta{%
    [Tsotha] did descend into a well he found, and came out with a strange expression which has not since left his eyes.
  
    I have seen that well, but I do not care to seek in it for wisdom. 
    I am a sorcerer, and older than men reckon, but I am human.}
}









\subsection{Culture}
\target{Cults in Yormis}
\Yormis was ruled by a bit of an anarchy of cults and mage-schools.
Some worshipped the \Ortaican gods.
Others worshipped darker powers, such as \Ubloth (\hr{Ubloth cult}{who had a cult there}).
A few knew that the \Ortaican gods were \xss.
The Dark Crescent had much power in \Yormis.









\subsection{Equipment}





\subsubsection{Mummies}
\target{Mummies in Yormis}
Beneath \Yormis were catacombs full of mummies. 
Most of them were mortal \scathae, dating from \Ortaican times or from the first \Yormis. 

A few of the mummies were ancient \hr{Ophidian mummies}{\ophidian mummies}.
Most of those were from the first \Yormis, but a few were true ancients who had lain in Durance for a million years. 
These Elders were donated by \Thessulax from her mummy-filled tomb. 

The mummies were \hr{Ophidian mummy worship}{worshipped} because they could still impart wisdom. 









\subsection{History}





\subsubsection{Age of Chaos}
In the \quo{\hs{Age of Chaos}}, \Yormis was the site of an \ophidian city, ruled by \hr{Thessulax}{\Thessulax} from her tomb-citadel in \KaiLeng.

This city was destroyed by Cordos Vaimon. 
With \hr{Early Vaimons overpowered}{their superpowers in those days}, the early Vaimons were a match even for \ophidians. 





\subsubsection{Vaimon Age}
In the days of the \VaimonCaliphate, there was no \Yormis. 
The land around \hr{Shrun}{\Shrun} was just \wylde. 
This was partially because of the \xss that dwelt beneath \Shrun, spreading their evil and corruption. 





\subsubsection{\Ortaican Age}
\index{Suthis Ondra}
\Yormis was re-founded some time after the \hr{Hundred Scourges}{\HundredScourges} as an \Ortaican city. 
It was built by \hs{Suthis Ondra}, one of the great founders of \Ortaica and a mighty \hr{Rethyax}{\rethyax}. 

\Thessulax and her Sentinels guided Suthis to the place. 
Here Suthis discovered the mystic things beneath Mount \Shrun:
The entrance to \KaiLeng, the \xs \Ubloth and the \hr{Well in Yormis}{black well}. 
So he chose to build his sorcerous fortress here. 
The \ophidians helped him do it. 





\subsubsection{Post-\Ortaican Age}
\target{Yormis remained a power factor}
After the \hr{Fall of Ortaica}{fall of \Ortaica}, \Yormis survived and remained a \rethyax-dominated power factor. 
The reasons for \Yormis's success were both \hr{Yormis economy}{economical} and \hr{Yormis military}{military}. 





\subsubsection{Underground}
\target{City beneath Yormis}
The ruins of the ancient \ophidian city still lay beneath \Yormis.
There were vast caverns full of houses and towers. 

The giant city could house millions, but in the time of \MoroCobrel it was almost entirely dead and deserted and lifeless.
But it still felt as if it lived on with a terrible, unnatural, inhuman unlife. 
Half-seen ghastly shapes slithered, and sinister corpse-like eyes stared out from the whispering shadows.
A few \ophidians still dwelt there.
It was a horrible charnel corpse of a city, haunted by the lingering malevolence of a thousand aeons. 

There were also degenerate \scathae and \humans down there, \hr{Suthis cannibalism}{which were used a food} by the Suthis clan. 










\subsection{Inhabitants}





\subsubsection{Notable citizens}
\begin{itemize}
  \item 
    \index{Suthis Ondra}%
    \hs{Suthis Ondra} was the founder of \Yormis. 
    
  \item 
    \hs{Suthis Mephilex} was the most powerful mage in \Yormis at the time of the \thirdbanewar. 
    
  \item 
    \hr{Moro Cornel}{Moro \Cornel} was raised and educated in \Yormis. 
\end{itemize}









\subsection{\Ophidians}
\target{QJ ride salamanders}
There were \hr{QJ in Yormis}{some \quiljaaran in \Yormis}.

The \quiljaaran under \Yormis rode on grotesque salamanders. 
Or maybe on \hr{Lindworm}{\lindworms}.
(\hr{QJ often ride}{\QuilJaaran often rode}.)
Moro \Cobrel had seen them do it. 





\subsubsection{Search for immortality}
\target{Ophidians in Yormis search for immortality}
The \ophidians of \Yormis were ruled by a small council of elders who had survived the Durance.
They were now \hr{Undead Ophidians}{undead mummies}.
Their bodies were weak and frail, and they searched for a way to become strong again.
They were working on an elixir of immortality, brewed from the \hr{Effluvium of Ubloth}{effluvium of \Ubloth}. 
But it was still imperfect.
So they used their \scathaese servitors as guinea pigs.
Many \scathae desired immortality and were eager to accept whatever risks it entailed.









\subsection{Politics}





\subsubsection{Economy}
\target{Yormis economy}
\Yormis lay near Mount \hr{Shrun}{\Shrun}.
They had access to lots of valuable metals and other minerals. 
They had many great mines where ore was extracted. 
These minerals were a great export.
This kept \Yormis's economy strong. 

This economical factor was one of the reasons why \hr{Yormis remained a power factor}{\Yormis remained a power factor} in the region even after the fall of \Ortaica. 





\subsubsection{Military}
\target{Yormis military}
Other kingdoms had tried to invade \Yormis many times. 
But they defended themselves using sorcery, monsters and the \wylde. 

\Yormis also built up a reputation as fearsome sorcerers.
This was useful, because it made many people too afraid to invade them. 

This military/reputation factor was one of the reasons why \hr{Yormis remained a power factor}{\Yormis remained a power factor} in the region even after the fall of \Ortaica. 









\subsection{Suthis clan}
\target{Suthis}
\index{Suthis}
The Suthis clan was a powerful \scathaese clan. 
It traced its descent from \hs{Suthis Ondra}, one of the founders of \Yormis.
The clan remained one of the most powerful factors in \Yormis.

Another notable scion of the clan was \hs{Suthis Mephilex}.





\subsubsection{\Arcana}
\target{Suthis Arcana}
\target{Suthis Innermost Arcana}
The \rethyaxes of the Suthis clan had their own secret \hr{Arcanum}{\arcana}.
These they called their \quo{Innermost \Arcana}. 
They included these facts:

\begin{itemize}
  \item 
    That the \ophidian \liches existed underneath \Yormis, and that they ruled the Suthis clan as mentors.
  \item 
    That the Suthis clan \hr{Suthis cannibalism}{ate \scatha flesh} in order to refine their bodies and their bloodline.
  \item 
    That the Suthis \rethyaxes \hr{Suthis and Ubloth}{drew their sorcerous power from the hideous god \Ubloth}, and that \hr{Suthis souls and Ubloth}{their souls and reincarnation were bound to \Ubloth}.
\end{itemize}





\subsubsection{Cannibalism}
\target{Suthis cannibalism}
The Suthis clan kept degenerate \scathae and \humans as slaves in secret pens in the underground beneath \Yormis. 
These slaves were bred to serve as food. 
It was a part of the Suthises' \hr{Suthis research}{research and quest}.





\subsubsection{Descent}
The Suthises had \ophidian blood in them.
They repeatedly interbred with the \ophidians.
Some of them even had \xsic blood in them, like the \dragons. 
Suthis Mephilex \hr{Mephilex has XS blood}{was one such}. 
It was a part of the Suthises' \hr{Suthis research}{research and quest}.





\subsubsection{Monstrous slaves}
\target{Suthis monster slaves}
The Suthises cohorts had raised up some horrid quasi-humanoid things to serve them.
These were once slave races of the \ophidians and had been entombed with their masters.
The Suthises had resurrected some of them (having been taught the necessary spells and formulae by the \ophidians), but imperfectly.
Many of the things they raised came back as misshapen monsters. 
They were still obedient slaves, but they were crippled and awful to look upon, and some howled in constant pain. 





\subsubsection{Oath}
Here is a traditional Suthis oath/invocation to \Ubloth.


\begin{diary}%
  I pledge my soul to thee, \Ubloth.
  My lifeblood is in thee, and thine essence is in me. 
  I am one with thee. 
  My soul is thine in this life and the next. 
  Cleave unto me, great one, and grant me thy power. 
  Feed me with strength as I pledge myself to feed thee in turn. 
  
  My body and soul are one with thine, for I am Suthis. 
  As thou gave birth to me so shall I return to thee, and as I return to thee so shalt thou give birth to me again. 
  In this cycle is our clan sworn to thee forever. 
  
  This is our covenant.
  This is the pact which I now swear. 
  Receive my soul and give me thine!
  \Ubloth!
\end{diary}





\subsubsection{Politics}
The Suthis clan knew about the \hr{City beneath Yormis}{\ophidian city beneath \Yormis}. 
They had dealings with the \ophidians and with \Thessulax, and even forbidden powers: 
\Ubloth and the \xss. 





\subsubsection{Research}
\target{Suthis research}
The Suthis clan were conducting arcane research, under the supervision of their \ophidian masters.
They aimed to revive the \ophidian race.
They wanted to create a new \ophidian race from \scathaese stock, and they wanted to secure immortality for \hr{Undead Ophidians}{the old \ophidians who were undead from the Durance}.

The children of the Suthises were test subjects, spawned with \ophidian and \draconic blood.
Suthis Mephilex was a \hr{Mephilex is an experiment}{particularly successful experiment}.





\subsubsection{\Ubloth}
\target{Suthis and Ubloth}
Some \rethyaxes in \Yormis worshipped \Ubloth.
They were able to draw much sorcerous power from the demigod, for \Ubloth dwelt near to them and was a fairly generous and unambitious god. 
So the \Ubloth cult had much power in the city. 
They did not serve \Ubloth's designs (the thing had few designs at all, and \hr{Ubloth's mind}{might even be mindless}) but their own.
They just sacrificed to \Ubloth. 

All \rethyaxes must have a pact with one or more gods in order to cast really powerful magic. 
The Suthises used \Ubloth.

\target{Suthis souls and Ubloth}
The Suthis pact also bound their souls to \Ubloth. 
When they died their souls would return to \Ubloth. 
It was then possible to distil a soul from \Ubloth's \hs{effluvium}. 
This effluvium was fed to pregnant \sphyles in order to reincarnate the souls.
This was all part of the Suthises' \hr{Suthis research}{research and quest}.
(Souls would not necessarily reincarnate in one piece. 
 They could easily fragment and recombine during reincarnation.)

The Suthises lived in a gruesome symbiosis with \Ubloth.
\Ubloth did not feed directly, but \emph{through} them.
The Suthises devoured the flesh of humanoids, preparing their meals with sorcery so as to gain their victims' souls (or at least some of their soul-energy).
\Ubloth would then drink the blood of the Suthises and thus gain nourishment and souls.
The Suthises became physically weakened and decrepit from being drained by \Ubloth, but in exchange they were allowed to drink \Ubloth's effluvium (or potions distilled from it), which gave them a link to the god through which they could channel its magical energy. 









\subsection{Other clans}
\target{Jaslar}
\target{Kish}
\target{Torshen}
\target{Vrael}
Clan Jaslar was a hereditary enemy of Clan \hs{Suthis}.

Other clans included Kish, Torshen and Vrael.























\chapter{The Imetrium}
\target{Imetrium}
\target{Imetric}
\index{Imetrium, the}









\section{Culture}





\subsection{Aesthetics}
The symbol of the Imetrium is a white, four-spoked star, edged with blue within a circle of black against a white background. 

Imetrians often dress in shades of blue and green, like the sea. 
With stripes of white or copper.









\subsection{Caste system}
\target{caste}
\target{Caste system}
\index{caste system}
\index{Durcac!caste system}
\index{Rissitics!caste system}
The Rissitics are organized in a system of castes.
The castes are, in descending order of status:

\begin{enumerate}
  \item \Nyzlet, the priests, sorcerers and scientists.
  \item \Reken, the elite warriors and commanders. 
    Typically translated \quo{knights}. 
  \item \Bedhin, craftsmen.
  \item \Kyth, warriors. 
  \item \Hok, labourers and peasants.
\end{enumerate}

(%
  In earlier versions I had the \quo{Gzend}, a slave caste. 
  I have gotten rid of them. 
  The \hok{} are already serfs.%
)






\subsubsection{Purpose}
\target{Purpose of the caste system}
The purpose of the caste system was to curb the damaging side-effects of the Shroud-weakening that allowed the \hs{Rissitic creativity}. 
The tight, rigid caste system was intended to create a \hs{Mask of Civilization} and thus form a stable frame around the \quo{drilled hole into Chaos}. 
That way, the Chaos could be (hopefully) contained and controlled, and the Rissitics could gain some bird's-eye-view of it, so they could utilize its beneficial effects but minimize its damaging side-effects. 





\subsection{Clerical ranks}
\index{Imetrium!clerical ranks}
The titles used by the Imetric clergy are (in descending order): 

\target{Laccorin}
\target{Ispan}
\target{Telphan}
\target{Amra}
\begin{itemize}
  \item \Laccorin{} (similar to a cardinal).
  \item \Ispan.
  \item \Telphan. 
  \item \Amra{} (a regular priest).
\end{itemize}

A generic word for \quo{priest} is \Stracos, plural \Stracoi. 
% \also{The Imetrium}




\subsection{Crime slavery}
\target{Imetric slavery}
You cannot officially own slaves in the Imetrium. 
But criminals can be put to forced labour under slave-like conditions. 
Usually temporarily, but sometimes for life. 
(This is not passed on to their children, though.)





\subsection{Etiquette}
\target{Imetric etiquette}
The Imetrians have an etiquette rule that one should not talk openly about one's own weaknesses. 
That is considered feeble and pandering for sympathy. 
One should, of course, still be aware of one's limitations, and it is occasionally OK to tell others about theirs. 

On the other hand, it's OK to be proud and tell people of your own virtues and accomplishments. 
The Imetrians believe people should be proud of their accomplishments and discuss them openly, because it will inspire and motivate others to also achieve great things for the empire. 





\subsection{Language}
\index{Imetrium!Imetric language}
The language spoken in the Imetrium. 
Imetric is descended from \hs{Ortaican} and written using the \Ortaican{} alphabet. 

Imetric is intended to look similar to Greek and Latin, but pronounced similar to English. 





\subsection{Military ranks}
\index{Imetrium!military ranks}
The military ranks used by the Imetric army are (in descending order): 

\target{Deccor}
\target{Retaxis}
\target{Salican}
\target{Vexstra}
\target{Corphin}
\target{Inclan}
\begin{itemize}
  \item \Deccor{} (equivalent to a general). 
  \item \Retaxis{}. 
  \item \Salican{}. 
  \item \Vexstra{}. 
  \item \Corphin{}. 
  \item \Inclan{} (equivalent to a private). 
\end{itemize}



A generic word for \quo{soldier} is \Rengos, plural \Rengoi. 
% \also{The Imetrium}









\subsection{\Naga skin clothes}
\target{Imetric Naga skin clothes}
The Imetrians made clothes from \hr{Nagae shed skin}{shed \naga skins}, freely given. 
It was a great \honour to wear them. 

Outsiders sometimes disbelieved the stories that the clothes were made from the skin of mermen and mermaids. 
Others assumed that the Imetrians killed the merfolk for their skins. 
Both notions were offensive to the Imetrians. 









\subsection{Names: Asian order}
Imetrians traditionally have two names: 
A family/clan name and a personal name. 
The clan name comes \emph{first}. 
This is \Ortaican{} tradition. 









\subsection{Sexuality}
Here are some excerpts of \hs{Telcastora Ilcas} trying to explain the Imetric sexual philosophy to \hr{Carzain}{Carzain \Shireyo}. 


\begin{prose}
  Northstar nodded. \ta{So, how was she?}
  
  \ta{Not bad. Not bad.} Carzain nodded, his expression indifferent. Then he rolled eyes at himself. 
  \ta{Pff. Whom am I trying to fool. Seriously, she was good! She displayed techniques I had heard about but never actually encountered. Where do Imetrian girls learn all this?} 
  
  \ta{Haha. Well\ldots{}}
  
  \ta{Do your people spend your entire youth having sex? 
    Practicing your moves?}
  
  \ta{Haha. No. Maybe. I don't know\ldots{} 
    We Imetrians do own decency. We do not belive in sexual anarchy like the Rissitics do\dash or the Geicans, perhaps. I am not so familiar with Geican culture. But despite that, we do not share all the unreasonable sexual taboos that the Iquinians have. We practice a\ldots{} \quo{golden middle road}, some might say. Now, I do not know Equin Mirai, but she is a comely young woman\dash if I am any judge\dash and from my impression of her disposition, I would not be surprised to learn that her sexual experience was extensive. And I know that there exist\ldots{} \quo{techniques}, as you say, that in certain milieus are passed around and practiced.} 
  Carzain shot him a questioning glance, which Northstar answered with a secretive, almost teasing one. 
  \ta{%
    I have no more than the most superficial experience with this sort of thing. I only know it exists.}
  
  \ldots{}
  
  
  
  \ta{So, tell me\ldots{}} said Carzain, changing the subject. \ta{Explain again how the whole fake secrecy deal works.}
  
  Northstar chuckled. \ta{Well, I believe it is a consequence of\ldots{}} He paused for words. \ta{See, despite what I said before about a \quo{golden middle road}, the truth is that 
  %the Imetric theology's teachings 
  %regarding sexual morals are notoriously unclear. 
  the Imetric view of sexual morals is notoriously unclear. 
  Promiscuity is traditionally considered taboo, but no such prohibitions are actually given in the \Imetriad{}.
  
  I believe this has caused us to develop a strange, perhaps hypocritical mentality: Sex is at once both permitted and forbidden. I don't know\ldots{} perhaps we have some instinctive need to establish sexual taboos. Or perhaps eroticism itself thrives on the\ldots{} allure of the forbidden. I don't know. But whatever the reason, we seem to have created a half-imaginary prohibition against wanton sex.}
  
  \ta{How so?}
  
  \ta{Well\ldots{}} again Northstar had to pause to put his thoughts into words. \ta{In a way, promiscuity\dash meaning sex outside of marriage or otherwise challenging tradition\dash in a way, it is forbidden, so that if one practices it, it must be kept hidden and secret from everyone. But on the other hand, the actual consequences of discovery are minimal. There is no punishment by law, no social stigma. No cruel expulsion, like Iquinian societies practice.
  
  The result is a game of false secrecy, of conspiracy against a fictitious menace. It is a strange thing, and I admit I never truly understood it. I was never a great player of the game, back in the day.} He regarded the \human. \ta{Nonetheless, you seem to have absorbed it readily enough, \Shireyo.}
  
  Carzain shrugged. \ta{I've learned that seduction doesn't need to make sense. I suppose the Imetric game of hiding is no stranger than the games we play in Pelidor.}
  
  Northstar's reply to that was a shrug and a \ta{hm}. 
\end{prose}





\subsection{Technology}
\target{Imetric technology}
Imetric technolog and economy were not bad, but they were inferior to those of the Rissitics. 
One reason for this was that the Imetrians spent so much time and energy \quo{dancing around totem poles} and building monuments and worshipping their gods (building up their \hr{Imetric fanaticism}{fanaticism}).
This sapped time and energy which they could potentially have used to grow food and make weapons.





\subsection{Water}
\target{Imetrium and water}
The Imetrium is very much a water-based civilization. 

The Imetrium is an kingdom of islands. 

Their gods are water-based. 

The Imetrians have become great seafarers. 

They live off fishing to a large extent. 

\hr{Imetric water magic}{Much of their magic} and \hr{Imetric monsters}{many of their monsters} are water-based. 

This is partially because of their \hr{Imetrians and Nagae}{ties to the \nagae}. 

Imetrian sailors and sea-mages are so renowned that they are often hired as mercenaries by others\dash{}those who dare associate with such dangerous sorcerers and their terrible monsters, that is.








\section{Geography}





\subsection{Falcus Aira}
\index{Falcus Aira}
\target{Falcus Aira}
The capital city of the Imetrium, built at the northern coast of the \hs{Naemor Strait}. 





\subsection{Fendor}
\index{Fendor}
\target{Fendor}
An island in the \hr{Risvael Sea}{\Risvaelsea}. 
It is currently controlled by the Imetrium. It has two major towns: Fendacor in the east and \Cicora{} in the west.





\subsection{Martinum}
\index{Martinum}
A city in the Imetrium, built on both sides of the narrow \hs{Martinum Strait}. 
It is famous for the great Martinum Bridge, spanning the Strait and considered a marvel of architecture. 

Martinum was previously the capital of a kingdom by the same name. It is one of the largest, most prosperous and most strategically important cities of the Imetrium, militarily as well as economically. 

Martinum is ruled by a \hr{Laccorin}{\Laccorin}.
The current ruler is Vian Martin, a descendant of the old royal line of Martinum. 





\subsection{Martinum Strait}
\index{Martinum Strait}
\target{Martinum Strait}
The narrow strait that separates the \Samure{} Gulf (west) from the Risvael Sea (east). 





\subsection{Tugan}
\index{Tugan}
\target{Tugan}
A small island in the \Risvaelsea. It is currently controlled by the Imetrium. It has one major town, Pandex. 















\section{History}
\target{Imetrium founded}
The Imetrium appeared near the end of \ps{\Ortaica}{} reign. 
They \hr{Fall of Ortaica}{conflicted with the \Ortaicans}. 

This is one of the reasons why the Rissitics and Imetrians hate each other. 









\subsection[Nycaneer alliance]{\Nycaneer{} alliance}
\target{Imetric-Nycaneer alliance}
Before the Imetrium there existed some \nycaneer{} tribes in \Ortaica, who were allied with \nycans. 
During \ps{\Ortaica} time they were still pretty \Wylde{}. 

Some of these tribes allied with the the \Ortaicans, but they were never assimilated into the \Ortaican{} culture and never really became a part of the \bacconate; they remained external sometime allies. 

\hs{Eoncos}, a \nycan{} god, succeeded in tying all (or most of) the tribes together and allying them with the nascent Imetrium. 









\section{Magic and metaphysics}
\target{Imetric magic}
The Imetrians have their own magic theory. 

Imetric magic theory deals with the \quo{Aether}, a mystical plane where magic resides.
Or maybe they call it the \quo{Astral Plane}. 

Imetric magic, in effect, might resemble \hs{psionics} more than traditional magic. 
It has a lot of telekinesis, telepathy and clairsentience. 
It is meant to look less \quo{occult} and more, I dunno, more \quo{clean} and \quo{nice} and \quo{scientific} than most schools of magic. 

Imetric magic is very mystical and strange and occult. Chaotic in a way, but a cold way.
In contrast, Rissiic magic is logical and scientific and rational.
Rissitic magic can also be chaotic, but in a controlled way, even though it may be hot and fiery.
Rissitics learn to channel, control and utilize their emotions.
Imetric mages come off as in\human, robotic, alien, devoid of emotions.

\citeauthorbook[p.26]{RPG:Warhammer:HordesofChaos}{Gavin Thorpe et al}{
  Warhammer: Hordes of Chaos
}{
  Raising his staff high into the air, he roared at the night sky. 
  Dark lightning flickered at the corners of his midnight eyes as the heavens answered his call, deep, rumbling echoes sounding over the plains.
  The darkness gathered around him like a whispering cloak, shadows coiling out rom the surrounding gloom to hover behind him.
  The twisting blackness reached out with amorphous tendrils, their icy touch sucking the warmth from the warriors' bodies, and making the fire shrink out of existence.
  
  In darkness the Sorcerer spoke, his voice as though there were a dozen people speaking the same words from his throat.
}





\subsection{Mages}

High-ranking Imetric sorcerers and priests are often those with much \naga blood, for these are more highly trusted by the gods and the predominant Imetric intelligentsia.
As they grow in age, power and occult \quo{forbidden} knowledge, their minds twist and become un-\scathaese and more \naga-like. 
As they channel lots of magic, their bodies twist as well, as the \naga sorcery awakens their \naga blood and strengthens and brings out the \naga part of them. 
They mutate and come to resemble \nagae.
Those with enough \naga blood and enough skill and experience (and luck) eventuality become \nagae and go to join their gods in the sea.

The Imetric gods are highly aquatic! 
Make this clear!

The older mages are strangely mutated and look deformed.
They are highly honoured for this in Imetric society, but outlanders use this as proof that the Imetrians are monsters that worship evil things.

Imetrian mages not only feel inhuman, they also look it physically.
And they think inhuman. 
Their minds become \naga-like, and they lose their more \scathaese emotions.
Outsiders sometimes interpret this as \quo{insanity}, but it is really just a change of perspectives, the acquisition of a new mindset, new emotions, new motivations, new ways of thinking.





\subsection{Power}
\target{Imetric vs Rissitic magic}
Imetric magic was slightly weaker than Rissitic magic, but still pretty powerful. 
They \hr{Imetrians rely on magic}{relied on it a lot in war}. 





\subsection{Water}
\target{Imetric water magic}
Much Imetric magic is water-based. 
They are, after all, \hr{Imetrium and water}{a sea power}. 















\section{Military}





\subsection{Armies}
\target{Imetric armies}
\index{technology!Imetrium}
One of the Imetrium's strengths is their soldiers. 
The Rissitics may have higher technology and magic, and certain empires have fielded larger armies, but none have had soldiers more well-trained, disciplined and dedicated as the Imetrians. 

They are backed by their gods: 
Salacar gives them the faith to fight for what is right. 
Eoncos strengthens their bodies. 
Dessali strengthens their minds. 
And so on. 





\subsection{Differences from other kinds of armies}
\lyricstitle{Draft excerpt from the chapter \quo{The Cannonade}}{
  The \nycaneers are super-effective with their telepathy. 
  Sethgal and Carzain are surprised at this. 
  They have heard of \nycaneers, but they still think of telepathy as a thing for mages, not armies of soldiers. 
  And the \nycaneers use it to awesome effect. 

  An Imetric priest says: 

  \begin{prose}
    Priest: 
    \ta{We fight an important battle, brethren.
      There are few of us, but the power of our gods will run through us all the stronger for it.}
  \end{prose}

  And it is true. 
  You can see their heathen gods are with them. 

  We don't see this from the Imetrians' POV. 
  After all, the Imetrians are just a minor player in the story so far. 
  They should not get a big, dramatic role until I have had the time to develop them into something cooler, more badass, more background-rich, more well-rounded. 

  We see it from Sethgal's POV. 
  Or maybe \Dornaer. 
  He is one of the only people present who speaks some Imetric. 
  Sethgal is happy to have the Imetrians on his side in the coming siege.

  \begin{prose}
    Sethgal: 
    \ta{I knew we could rely on you.}
    
    \tho{That's actually not true. 
      I thought we couldn't. 
      I \maybehr{Sethgal curses Imetrians}{cursed the Imetrians earlier} for being faithless allies.
      But they proved me wrong.} 
  \end{prose}

  Sethgal overhears someone (Ilcas or an Imetric priest) giving the soldiers a peptalk. 

  \begin{prose}
    Imetric priest: 
    \ta{If we die on this day, we shall live again!}
    
    Sethgal: 
    \tho{I know the Imetrians believe that they reincarnate when they die.
      I wonder if that is true.
      It is certainly not true for us Iquinians. 
      We die and go into the Light.
      But for them\ldots{} who can say?}
  \end{prose}

  The Imetrians scare Sethgal. 
  They fight with great zeal, \maybehr{Imetrian coldness}{but their fervour is\ldots{} cold}. 
  Calculating. 
  Reptilian. 
  He is quite disturbed.

  (Maybe make a footnote about how the warm-blooded \scathae{} do not see themselves as \quo{reptiles}. At the very least, mention this in the glossary.)
}





\subsection{Icy \armour}
Have Imetric warriors clad in \armour made of what looks like transparent crystal, glass or ice. 
And matching weapons: 
Spears and swords of ice. 

Compare to the Stormriders in \authorbook{Ian Cameron Esslemont}{Night of Knives}. 





\subsection{Monsters}
\target{Imetric monsters}
The Imetrians employed a wide variety of beasts and monsters in war (\hr{Rissitic monsters}{just like the Rissitics}).

The Imetrians especially used sea monsters, which made them so powerful and dangerous in naval battles. 
They were, after all, \hr{Imetrium and water}{a sea power}. 

See also the section on \hr{Domestic animals}{domestic animals}. 





\subsection{Pros and cons}
\target{Imetric military pros and cons}
\target{Imetrians rely on magic}
The Imetrians relied a lot on magic in war. 
They had \hr{Imetric technology}{slightly lower technology} and \hr{Imetric vs Rissitic magic}{slightly weaker magic} than the Rissitics, but they made up for it with better trained and disciplined men, and superior morale.
The Imetric gods \hr{Imetric fanaticism}{really knew how to make their worshippers fanatic}. 





\subsection{Seafaring power}
The Imetrium is \hr{Imetrium and water}{a sea power}. 















\section{Mythology}
\target{Imetric mythology}









\subsection{\Ishnaruchaefir}
\Ishnaruchaefir \hr{Ishnaruchaefir and the Imetrium}{appeared in Imetric mythology}.















\section{Politics}
The Imetrium is traditionally hostile towards \hs{Durcac}. 

They have sometimes allied with the \hr{Iquinian}{\Iquinian} Vaimons





\subsection{Ideology}
The Imetric system is a primitive communist plan economy. 
People are comparatively wealthy, well-fed, healthy and secure. 
But the system is also rather totalitarian, obsessed as they are with their \quo{justice}. 
There is much control and repression. 





\subsection{Reputation}
\target{Imetric reputation}
The Imetrians were feared for being dark and inhuman, \hr{Rissitic reputation}{like the Rissitics}.
But they were less menacing and aggressive than the Rissitics, and so the \Velcadians sometimes made deals and alliances with them for mutual protection against the Rissitics.
The Rissitics rarely tried to invade the Imetrium.





\subsection{Ties to the \nagae}
\target{Imetrians and Nagae}
The Imetrium is tied to the \nagae. 
Perhaps Salacar is actually a \nagalord{} in disguise. 

Also, Dessali is a \naiad, a water-dwelling \quo{spirit}. 

The Imetrians have prayers that invoke the \nagae. 

\lyricsbs{Monolith Deathcult}{Den Ensomme Nordens Dronning}{
  Eternal Father, strong to save, \\
  whose arm hath bound the restless wave, \\
  who biddest the mighty ocean deep \\
  its own appointed limits keep; \\
  oh, hear us when we cry to Thee, \\
  for those in peril on the sea!
}








\section{Religion}
The Imetric religion has some adherents in surrounding countries, especially \hs{Andras} and \hr{Scyrum}{\Scyrum}, and even some of eastern \hr{Velcad}{\Velcad}. 

The religion is polytheistic. 
The five major gods make up \quo{the Tribunal}, which governs the Imetric pantheon and nation. 
The Tribunal members are: 
  
\begin{gloss}
  \gitemlink{Salacar} 
  \index{Salacar}
    God of justice, head of the pantheon.
  \gitemlink{Dessali} 
  \index{Dessali}
    Goddess of reason, science and knowledge. 
  \gitemlink[Nishi-Settias]{\NishiS}
  \index{\NishiS}
    Goddess of life and death. 
  \gitemlink{Eoncos}
  \index{Eoncos}
    God of strength, bravery and war. 
  \gitemlink[Hiothrex]{\Hiothrex}
  \index{\Hiothrex}
    God of vengeance. 
\end{gloss}

Other gods worshipped by the Imetrians include: 

\begin{gloss}
  \gitemlink{Maegon} 
  \index{Maegon}
  A sea god, actually a \nagalord. 
\end{gloss}





\subsection{Cold fanaticism}
\target{Imetric fanaticism}
The Imetric gods really knew how to make their worshippers fanatic. 
It was an \quo{Elder World} thing, learned from the \nagae. 

That was also why Imetrians spoke so highly of their \quo{justice}. 
It was because the gods had good publicity and marketing. 
It \hr{Imetric military pros and cons}{worked great for military purposes}. 
It was actually pretty sinister.

The Imetrians as a people were very fanatic and devoted. 
Frighteningly so. 
To outsiders they could come off as religious madmen. 
They could look quite ecstatic as they fought for what they believed in. 

\target{Imetrian coldness}
\target{Imetric coldness}
The Imetrians fought and worshipped with great zeal and fervour, but it was a fervour that was somehow\ldots{} cold. 
Calculating. 
Reptilian. 
\Ophidian. 

This was due to the fact that the Imetrians were affiliated with the cold \nagae{} and even had \naga{} blood in their veins. 
Their religious faith was always cold and sinister, not hot and passionate like that of the Iquinians, \Ortaicans{} and \Tepharites. 

The cold fervour was a manifestation of the \hr{Scatha fury}{primal fury that lurked in all \scathae}. 







\subsection{The \Imetriad}
\index{\Imetriad}
The \Imetriad{} is the holy scripture of the Imetrium, containing the core of their beliefs and morals. 









\subsection{Reincarnation}
The Rissitics believed in \hr{Reincarnation}{reincarnation}. 















\section{The truth about the Imetrium}
\target{The truth about the Imetrium}
The Imetrium is secretly backed by the \nagae. 

Salacar is a saviour-figure of a kind. 
Similar to the \human{} \quo{Messiah} that is mentioned in \cite{StevenErikson:TheBonehunters}, who might be Shadowthrone. 

Salacar is a \quo{promised child}, born to return his own people to glory and, in his creators' eyes, purge the world of evil. 
As such, he carries in him the souls of many of his forebears in him, and their many thousands of years worth of memories. 
He is an powerful \vertex, the \apex{} of a \cuezcan/\naga{} \matrixx. 

But all of it in secret. 
Although the Cabal and Sentinels might suspect.





\subsection{Ilcas suspects}
At some point in the story, \hs{Telcastora Ilcas} learns something disquieting and \hr{Ilcas suspects}{begins to suspect} that the Imetric gods are more than they seem to be. 









\section{Northstar clan}
\index{Northstar!Northstar clan}
The Northstars are a \scathaese{} clan. 
The Northstar name is famous and prestigious (known throughout the Imetrium and parts of southern \Velcad{} and northern Durcac), and the clan has produced several renowned heroes. 
The Northstars trace their history at least 1000 years back. 
Today they are Imetrians, but they retain certain pagan traditions, of which the most well-known is the veneration of the \hs{North Star}. 

The Northstars have their own city, Telcarmium, in the western Imetrium somewhat east of Martinum, where many of them live. The vast majority of Northstars live in the Imetrium. 
They are a tightly knit family and maintain contact between all their members. 
There are a few people outside the Imetrium who bear the name Northstar but maintain no contact with the family. 
These are considered traitors or impostors by the clan. 
The official Northstar clan has about a thousand members. 

The name \quo{Northstar} is actually \word{Telcastora}, which is the name of the North Star in the \Ortaican{} tongue. 
Throughout my writings, however, Telcastora has been translated into English \quo{Northstar}. 

Well-known Northstars include Ilcas and Cassili the Condor. 
\also{North Star} 
























\chapter{\Ortaica}
\target{Ortaican}
\target{Ortaica}
\index{\Ortaica}
A great \hr{Scatha}{\scatha}-dominated \hr{Bacconate}{\bacconate} that flourished after the \hr{Hundred Scourges}{\HundredScourges}. 

The \Ortaican religion and the \hr{Rethyax}{\rethyax} mages continued to flourish up to the \thirdbanewar. 









\section{Culture}





\subsection{Architecture}
\target{Ortaican architecture}
\index{architecture!\Ortaican}
The \Ortaicans{} used \hs{occult geometry} in their architecture. 
But primitively.
They were not as skilled as their successors the \hs{Rissitics}. 





\subsection{Asian name order}
\Ortaican{} traditionally have two names: 
A family/clan name and a personal name. 
The clan name comes \emph{first}. 

The Imetrium has inherited this, but the \hr{Tepharite}{\Tepharites} did not. 




\subsection{\Baccons}
\target{Bacconate}
\target{Baccon}
\index{\bacconate}
\index{\baccon}
A \baccon was a ruling council of \rethyaxes in \Ortaica. 
Every \Ortaican province was ruled by a \baccon, and the entire \Ortaican empire was ruled by a supreme \baccon. 

An area ruled by a \baccon{} was called a \bacconate. 

After the fall of \Ortaica, the concept of \baccons and \bacconates survived. 
Up till the \thirdbanewar, many non-Iquinian nations were ruled by \rethyaxes.
It was only in the Iquinian lands that sorcerer-kings were prevented. 

Originally, a \baccon was a \quo{council of the wise}, and wise meant mages, \rethyaxes.
In some post-\Ortaican \bacconates, though, the concept was watered down and the \baccons degenerated to mere aristocratic juntas. 

A member of a \baccon{} was called a \hr{Raebar}{\raebar}.





\subsection{Gargoyles}
\target{Ortaican gargoyles}
\Ortaican buildings were often adorned with \hr{Gargoyles}{gargoyles}.
They looked hideous to \human eyes, less to to \scathaese eyes.
\Forclin was an example of this.





\subsection{Language}
The \Ortaican{} language is now dead, but several modern languages are descended from it and still written using \Ortaican{} alphabet. 

\Ortaican looks a little bit like Nahuatl (the language spoken by the Aztecs and related peoples), but only a little.





\subsection{Mesoamerican asthetics}
Aesthetically, \Ortaica is partially based on Mesoamerican cultures such as the Aztecs.
But not too much.









\section{History}
\subsection{Etymology}
The name \quo{\Ortaican} derives from \quo{Noratai ca}, meaning \quo{Norat's tribe}. 
Norat was an ancient chieftain of a \Mastheno{} tribe that would later adopt his name and become the \Ortaicans. 





\subsection{Rise to power}
The \Ortaicans{} originated from the south, in modern-day Durcac. 
They are a \Tassian{} people, descended from the \Masthenon. 
Under the influence of \hs{Sseju} and many others, they became a big \bacconate. 

At the height of its power, \Ortaica{} covered all of southern \hr{Azmith}{\Azmith}, including the lands that are now {Durcac}, the \hs{Imetrium}, southern \hr{Velcad}{\Velcad} and even \hs{Uzur} and \hs{Geica}. 
\Scathaese{} cultures today, especially the Imetrium, owe much of their culture to the Ortaicans. 

The Ortaicans, alongside their kin the \Shurcos, warred against the \hr{Vaimon Caliphate}{\VaimonCaliphate} and were beaten by \hr{Vizicar}{\VizicarDurasRespina}. 
After the \hr{Hundred Scourges}{\darkfall} the \Shurco{} were in decline, but the \Ortaicans{} were on the rise. 
Their empire kind of rose from the ashes of the \Shurco. 
They built a great \hr{Bacconate}{\bacconate} and dominated much of the world for a while. 

Hardened by their wars against the Vaimons, the Ortaicans were a tough warrior people. 
They had a heathen religion with their own pantheon of gods, the \hr{Taortha}{\Taorthae}.  





\subsection{\Ishicah{} enslaved}
\target{Ishicah enslaved}
At some point the \rethyaxes{} managed to capture a Scion while she was still young and weak. 
This Scion was \hr{Iolivine}{\Iolivine}, an incarnation of the \Malach{} \hr{Ishicah}{\Ishicah}. 

Maybe \Ishicah{} had awakened to some of her powers and tried to set herself up as a goddess. 
She failed and was taken captive by the \Ortaicans. 

They kept her imprisoned and enslaved, and for over 100 years they used her magical power to power their own sorcery. 
They fed her with living creatures to keep her alive and strong\dash even \resphain, when they could procure them. 

The \rethyaxes{} had Sentinel backing and knew about the \ps{\resphain}{} existence. 
\Ishicah{} was a \sathariah{}. 
The Sentinels loved the irony of putting her through some of the same torture and \hr{Life drain}{power drain} that \Nexagglachel{} had undergone. 
But \Ishicah{} was not as strong as he. 
She succumbed and begged and became their obedient slave for the rest of her existence. 

They forced her to \hr{Apotheosis}{\Apotheosis}, but still kept her imprisoned. 

\target{Ortaican Matrix}
This was particularly effective because \Ishicah{}, \hr{Malachim binding souls}{being a \sathariah{} \malach}, had a very powerful \carcer{} of souls bound to her. 
Her \carcer{} would be torn from her and used as the foundation of the nascent \Ortaican{} \matrix. 
This \matrix{} was the source of the \ps{\rethyaxes} power. 
It bound a lot of \daemons{} to the \Taorthae{} and thus allowed the \rethyaxes{} to cast magic. 
It would continue to empower the \rethyaxes{} up to the \hs{Unravelling}, if not further.

The \Ortaican{} \matrix{} would also form the basis for the later \hr{Rissitic Matrix}{Rissitic \matrix}. 





\subsection{Unpopular}
The \Ortaican{} \baccons{} became unpopular. 
Their magic was no more evil than that of the Vaimons, but they failed to win the population's hearts in the same way the Vaimons did, because the Vaimons had the brainwashing power of \Iquin. 

Their remembrance has since been further blackened by propaganda from the \hr{Iquinian}{\Iquinian} \hr{Tepharite}{\Tepharites}. 
They are now remembered as an evil empire, whereas the \VaimonCaliphate is idealized as a golden age. 





\subsection{The Absconding of the \Taorthae}
The \taorthae were busy fighting wars on other Realms and amongst themselves.
They had little time to keep track of what their mortal followers were doing. 

That did not help \Ortaica. 
They fought and argued and had no surplus time nor energy to keep a good overview of what was happening in \Ortaica. 
This means they were not fully aware of the earthly political turmoil at the time. 
On top of that, the gods were too preoccupied to advise their worshippers properly, so the mortals were left to their own designs. 

Without the direct leadership of their gods, \Ortaica became fractured and unstable. 
This made it easier for their enemies to destroy them. 

This period where the gods were aloof would be known as \quo{the Absconding of the \Taorthae}.
In fact, the Absconding preceded and caused the splintering of \Ortaica.
Post-\Ortaican \rethyactic scholars would interpret it as the other way around.
Scholars would also blow the Absconding out of proportion.
Myths arose that back in the \quo{golden age} of \Ortaica, the gods had walked the earth amongst their followers and fought the enemies of the Bacconate in person.
The Vaimons' records had plenty of stories of \daemons fighting for the \Ortaicans, so it was not too hard to believe that the gods had been there. 
And now they were gone. 

The \rethyaxes mourned this.
Some lost faith in the \rethyactic religion, and the remnants of \Ortaica were weakened further.
This allowed the Imetrians and the Iquinian Tepharins to take over.
In fact the Absconding was much more relaxed. 
The gods had never been very active in the first place.
They obeyed the \hs{Unspoken Covenant}, after all.





\subsection{Cabal intrigue}
It was, for a large part, Cabal intrigue and treachery that drove the \Ortaicans apart and destroyed their empire.





\subsection{Fall}
\target{Fall of Ortaica}
The \Ortaicans{} eventually became too evil. 
At last \Ishicah was somehow destroyed, which lost them some of their power. 
The \hr{Iquinian}{\Iquinian} \Tepharites{} rose up in rebellion against them, and \Ortaica{} was overthrown. 
They also had conflicts with the \hr{Imetrium founded}{fledgling Imetrium}, which hastened \ps{\Ortaica}{} downfall. 

It came to bloody infighting between the \Ortaican{} \raebari. 
One of them broke away from his fellows in bloody rebellion. 
Compare to the insurrectionist Malekith (who founded the Dark Elves) in \cite{RPG:Warhammer:DarkElves}. 





\subsection{Rissitics survived}
The Rissitics \hr{Rissitics were Ortaican}{were originally a \rethyactic{} sect inside \Ortaica}. 
After the fall of the \bacconate{} the Rissitics lived on and formed their own kingdom. 









\section{Mythology}
\target{Ortaican mythology}









\subsection{\Dragons}
The \Ortaicans worshipped a number of \dragons. 
In mythology, these \dragons were once the rulers of the world and upholders of the cosmic order. 
They were then killed or sacrificed their lives, but even in death they continued to uphold and maintain the world. 

\target{Ortaican blood sacrifice}
In order to keep doing this, the dead \dragon gods must be fed with sacrifices of blood. 
Ergo the \Ortaicans sacrificed animals and humanoids to their hungry dead gods. 

These dead \dragons included the fallen \Nechsains: 
\hr{Settras}{\Settras} (\Sethicus) and \hr{Mezzagrael}{\Mezzagrael} (\Nexagglachel). 

\target{Iquinian sacrifices too}
The Iquinians \hr{Iquinian criticism of Ortaica}{criticized \Ortaica} for their tradition of blood sacrifice. 
But the Iquinian religion {was just as bad}. 
Their \sephiroth were fed by the sacrifice of souls.
They just kept it secret. 
The \Ortaicans were at least honest about their hunger for blood. 

Compare them to the Immortal Emperor of Mankind in \cite{RPG:Warhammer40000} and to various mythological figures, including Osiris (Egyptian), Jesus (Christian), Inanna (Sumerian) and Quetzalcoatl (Aztec). 

See also the section about \hr{Myths about Dragons}{\dragons in art and mythology}. 









\subsection{History of the world}





\subsubsection{Nature of the world}
\Ortaican mysticism was more naturalistic than Iquinian (and closer to the truth). 
It taught that the world was a natural thing. 
It existed and ought to exist. 
It was the soul's own responsibility and choice to seek enlightenment and greatness. 
Every mortal soul (at least, every \scathaese soul) had \hr{Ortaican potential for greatness}{potential for greatness}, but only if they approached it with virtue.

See also the section about \hr{Ortaican myths about Iquin}{\Ortaican myths about \iquin}, which describes some important myths. 





\subsubsection{\Banes}
\target{Ortaican myths about Banes}
Originally, the world was ruled by the \Ortaican gods.
This included the standard pantheon of \Taorthae and some additional ones. 
The ruler of the gods, the first \Nechsain, was \hr{Settras}{\Settras} (whose real identity was \Sethicus). 

\target{Ortaican myths about Sethicus}
There existed a dark force of evil, hungry and greedy and hateful. 
This dark force was not native to \Miith but invaded it. 
There was a great war, and the righteous gods defeated the invaders. 
In the process, some great and mighty gods (\Settras and \Tiamat) sacrificed their lives.

The invaders were imprisoned. 
But one day they would return. 
Then the \hr{Ortaican eschatology}{Eschaton} would begin. 
(See that section.) 

These dark invaders were based on the \banes. 






\subsection{\Iquin}
\target{Ortaican myths about Iquin}
The \Ortaicans had their own myths about \Iquin and the \sephiroth.

The \hr{Ortaican myths about Banes}{dark invaders} were defeated but not destroyed. 
Now they sent down their agents in disguise. 
These wicked gods in disguise conspired to overthrow the rightful rulers of the world. 

The evil gods were sly, so they feigned friendship and deceived the other gods. 
They backstabbed some of the righteous gods, took them captive and burned them in a huge pyre. 
They cast their spells on the pyre so that it looked nice and beautiful. 
The righteous gods, unsuspecting, were impressed by the bright pyre and remained friendly to the new gods. 
But in secret, the evil gods kept backstabbing more gods to feed their pyre. 

Eventually, when the righteous gods discovered the evil ones' intention, it was too late. 
Too many had been sacrificed to the pyre.
The betrayers, who drew their power from the pyre, had become too powerful to stop.
So the good gods retreated and went into hiding.
But they swore that one day they would return and take their revenge and liberate the world from the evil conquerors. 

The pyre could burn forever, fuelled by the bodies and souls of gods. 
The pyre became \iquin. 
It was masked with deceitful spells so that it looked beautiful to divine and mortal eyes alike.
The new gods now called themselves \sephiroth. 
Mortals took to worshipping the \sephiroth and their \iquin, and gradually the mortals forgot that the world had ever been ruled by other gods. 

Rissitics claimed that \hr{Rissit is a saviour}{\HriistN} (being the brother of \hr{Mezzagrael}{\Mezzagrael}) was that promised saviour who came to free the world of the \iquinian tyranny. 





\subsubsection{The \HundredScourges}
\target{Ortaican myths about the Hundred Scourges}
According to \Ortaican myths, their gods brought down the wicked Vaimons and drove away their hideous \Archons. 
This was \hr{Taorthae manipulate Belzir}{partially based on history}. 

\Ortaican myth saw the \HundredScourges as just punishment upon \Miithians for worshipping the wicked \Archons and not opposing the wicked Vaimons.





\subsubsection{Eschaton}
\target{Ortaican eschatology}
The \Ortaicans had myths about a great Apocalypse, an Eschaton. 
Here the \hr{Ortaican myths about Banes}{dark forces} (\banes) that had once been defeated by \Sethicus-tachi would finally break the chains of their prison and arise again.
They would invade \Miith and the gods would have to fight them. 

The \taorthae were not all-powerful. 
(\hr{Omnipotence of Iquin}{Unlike in Vaimon mythology}, their was no doubt among the \rethyaxes about this fact.)
So if the \taorthae were to win this last war, they would need help.
Therefore, it was the duty of every \rethyax to do his utmost to help the gods. 
\emph{This} was the true purpose of the \Ortaican religion. 

This purpose was an \arcanum, however. 
The commoners did not know of it.
The masses were just told that the evil forces would break free and that the \taorthae would fight them and prevail.
The wicked would be punished. 

The \Ortaican myth would turn out to be quasi-true:
\hr{The world goes mad}{The Apocalypse really did happen}! 
Compare to \hr{Vaimon eschatology}{Vaimon eschatology}. 
See also the general section about the \hs{Eschaton}. 

\citeauthorbook[On the Origin of the World, p.178]{%
  MarvinWMeyer:TheNagHammadiLibraryinEnglish}{Marvin W. Meyer et al}{
  The Nag Hammadi Library in English
}{
  Before the consummation of the aeon, the whole place will be shaken by a great thunder.
  Then the rulers will lament, crying out on account of their death-
  The \hr{Iquinian angels}{angels} will mourn for their men, and the demons will weep for their times, and their men will mourn to <\ldots{} and> they will be disturbed.
  Its kings will be drunk from the flaming sword and they will make war against one another, so that the earth will be drunk from the blood which is poured out.
  And the seas will be troubled by that war.
  Then the sun will darken and the moon will lose its light.
  The stars of the heaven will disregard their course and a great thunder will come out pf a great power that is above all the powers of Chaos, the place where the firmament of woman is situated.
  When she has created the first work, she will take off her wise flame of insight.
  She will put on a senseless wrath.
  Then she will drive out the gods of Chaos whom she has created together with the First Father.
  She will cast them down to the abyss.
  They will be wiped out by their own injustice.
  For they will become like the mountains which blaze out fire, and they will gnaw at one another until they are destroyed by their First Father.
  When he destroys them, he will turn against himself and destroy himself until he ceases to be. 
  And their heavens will fall upon one another and their powers will burn.
  their aeons will also be overthrown.
  And the First Father's heaven will fall and it will split in two.
  Likewise the place of his joy, however, will fall down to the earth, and the earth will not be able to support them.
  They will fall down to the abyss, and the abyss will be overthrown.
  
  The light will cover the darkness, and it will wipe it out.
  It will become like one which had not come into being.
  And the work which the darkness followed will be dissolved.
  And the deficiency will be plucked out at its root and thrown down to the darkness.
  And the light will withdraw up to its root.
  And the glory of the unbegotten will appear, and it will fill all of the aeons, when the prophetic utterance and the report of those who are kinds are revealed and are fulfilled by those who are called perfect.
  Those who were not perfected in the unbegotten Father will receive their glories in their aeons and in the kingdoms of immortals.
  But they will not ever enter the kingless realm.
  
  For it is necessary that every one enter the place from whence he came.
  For each one by his deed and his knowledge will reveal his nature.
}









\subsection{\Ishnaruchaefir}
\target{Ishnaruchaefir in Ortaican mythology}
\Ishnaruchaefir was not mentioned at all in \Ortaican mythology.
This was deliberate from \Secherdamon's side. 
He wanted to completely deny \Ishnaruchaefir and not even deign to acknowledge his existence.
\Secherdamon wanted \Ishnaruchaefir to be a true Exile. 

\target{Ortaicans reject Isphet}
The \Ortaicans rejected the myth of \hr{Isphet}{\Isphet}, whom the Iquinians feared so much.
According to \rethyactic lore, \Isphet did not exist and was merely an Iquinian fiction, like the notion of \hr{Omnipotence of Iquin}{the omnipotent \iquin}. 
This was partially because \Secherdamon would not grant \Ishnaruchaefir the honour of a place in his mythology and preferred to deny his existence entirely. 









\subsection{\Mezzagrael}
\target{Mezzagrael}
\target{Ortaican myths about Nexagglachel}
In \Ortaican mythology, \Mezzagrael was one of the oldest and greatest of the \taorthae. 
In ancient times he had been the just and rightful ruler of \Miith. 
(Even the inner \arcana acknowledged \Mezzagrael as the ruler of \Miith.
 He was less powerful than the \Primordials, but his rule was blessed by them.
 He held the mandate of the \Primordials.)

In the \hr{Ortaican myths about Iquin}{myth about \iquin}, he was one of the gods whom the evil usurper gods captured and sacrificed to their pyre. 
When \Mezzagrael realized the evil ones' plan, he sacrificed his immortal life in order to send a light of hope into the future. 
In this way, \Mezzagrael granted salvation unto the faithful (\Ortaican or pre-\Ortaican) people.
He gave a promise that one day, they would be delivered from the evil ones. 

\Mezzagrael was a type like Osiris from Egyptian mythology. 

\Mezzagrael, as ruler of the world, held the title \Nechsain. 
He had inherited it from \Settras, his father. 

See also the section about \hr{Myths about Dragons}{\dragons in art and mythology}. 





\subsubsection{Rissitic view}
Rissitics claimed that \hr{Rissit is a saviour}{Rissit} (being the brother of \hr{Mezzagrael}{\Mezzagrael}) was that promised saviour who came to free the world of the \iquinian tyranny and inherit the title of \Nechsain. 





\subsubsection{Truth}
The myth of \Mezzagrael was based on the true story of \hr{Nexagglachel}{\Nexagglachel}. 

\target{Ortaicans use Nexagglachel's name}
Both \quo{\Mezzagrael} and \quo{\Nechsain} were derived from the name \quo{\Nexagglachel}. 
(\hr{Nexagglachel is a powerful name}{It was a powerful name}, and the Sentinels who founded the \Ortaican mythology wanted to milk the name for everything it was worth.)









\subsection{Sseju}
Sseju's legacy was twisted into something unrecognizable.





\subsubsection{Pseudepigraphic quotes}
\target{pseudepigraphic Sseju quotes}
\Ortaican scripture had many pseudepigraphic quotes attributed to Sseju that condemned Iquinianism.

{\VizicarDurasRespina} \hr{Vizicar doubts Sseju quotes}{doubted the veracity of these quotes}. 









\subsection{Tower of \Haamon}
The \hr{Tower of Aamon}{Tower of \Haamon} had an important place in \rethyax mysticism. 















\section{Philosophy and religion}
\target{Ortaican religion}









\subsection{\Arcana}
\target{Arcanum}
\target{Arcana}
\index{\arcanum}
A central dogma was that of the \arcana (singular \arcanum). 
There existed a great number of \arcana. 
An \arcanum was a piece of secret knowledge. 
Every \rethyax spell was part of an \arcanum.
Each god presided over one or more \arcana and could teach them to its followers. 

\Arcanum theory was a simplification of the Aenigma theory the \dragons used. 
Each \arcanum was a piece of well-known Gnosis.

The more \arcana a \rethyax knows, the more powerful he is. 

\citeauthorbook[\quo{True Kabbalah Inspiration}, p.123--125]{%
  AlanUnterman:TheKabbalisticTradition%
}{%
  Alan Unterman%
}{%
  The Kabbalistic Tradition%
}{
  He knew the Mysteries of Creation [\emph{Maaseh Bereshit}], the Mysteries of the Divine Chariot [\emph{Maaseh Merkabah}], the languages of birds, the conversation of palms, trees and herbs, \ldots the flaming embers of charcoal and the conversation of \hr{Iquinian angels}{angels}.
}





\subsubsection{Secrecy}
Enlightenment was only for the select few, not the masses.
The \arcana were secret for a reason.
Knowledge was very dangerous in the wrong hands. 
Only a strong soul could manage such knowledge. 
Glancing an \arcanum prematurely could lead to disaster for you or your surroundings. 

Only a \rethyax should even try to learn any \arcana.
Commoners should leave the esoteric mysteries to their betters.
Some \arcana were deeper than others and had prerequisite \arcana. 

The act of learning or glancing an \arcanum which one was not ready for was called \quo{stealing} the \arcanum. 

\Ortaican mythology was full of fables where someone stole an \arcanum that they were not ready for and came to a gruesome end as result. 
A commoner would go mad or be eaten by some \daemon or monster.
Or the gods would smite him for insolence. 

Even a \rethyax who tried to steal \arcana would come to disaster.
He would go mad, or call down some calamity when his magic went out of control.
Or his patron gods would punish him. 





\subsubsection{The Eight Dreaded Gateways}
\target{Dreaded Gateways}
The \rethyaxes had eight dreaded gateways.
Each gateway led to new enlightenment and opened the way to new \arcana and new magical power.
But there was a price to pay.
The initiate lost some of his humanity and attracted the cruel attention of sinister forces.

\citeauthorbook[p.415]{HPLovecraft:TheBook}{\HPLovecraft}{The Book}{
  For he who passes the gateways always wins a shadow, and never again can he be alone.
}





\subsubsection{The First \Arcanum}
\target{First Arcanum}
The First \Arcanum was the first \arcanum that any \rethyax was taught. 
It was the \rethyax code that laid down the basic rules:
\begin{itemize}
  \item How a \rethyax should interact with gods, commoners and other \rethyaxes.
  \item The basics (descriptive and moral) principles of magic.
  \item The rules for deference between \rethyaxes. 
  \item 
    The First \Arcanum also taught that \hr{Ortaican planes of existence}{the world was composed of a number of layers or planes}. 
\end{itemize}





\subsubsection{The Deeper \Arcana}
The deeper \arcana contained dark knowledge of the horrors of \Miith and Beyond. 

\citeauthorbook[p.248]{RobertEHoward:AWitchShallBeBorn}{Robert E. Howard}{%
  A Witch Shall Be Born%
}{
  \ta{But the life in me was stronger than the life in common folk, for it partakes of the essence of the forces that seethe in the black gulfs beyond mortal ken.}
  
  \ldots 
  
  \ta{He said I was but an earthly sprite, knowing naught of the deeper gulfs of cosmic sorcery.}
}





\subsubsection{Writing}
The \hr{Ortaican letters}{\Ortaican alphabet} were almost an \arcanum, but not quite. 

\Rethyaxes sometimes also used \hr{Draconic runes}{\draconic runes} for their spells.
The runes were an \arcanum. 

\citeauthorbook[\quo{Mysticism of the Talmud}, p.126]{%
  AlanUnterman:TheKabbalisticTradition%
}{%
  Alan Unterman%
}{%
  The Kabbalistic Tradition%
}{
  All the words of the Talmudic Torah are like firebrands, mysteries of the Torah which they hid from the eyes of simple people who are not fit to learn their secrets.
}








\subsection{Commandments}
\target{Ortaican religion has few laws}
The basis of \Ortaican morality was a set of \leges (singular \lex) or pillars. 
These were basic laws. 
A religious constitution of sorts.
\Ortaican dogma did not lay out much morality beyond the bare necessities.
For example, there was no general law against theft.
Instead, \Ortaican dogma merely dictated a social order.
It would be up to secular rulers to formulate fitting laws.
The Iquinians used this fact as \hr{Iquinian criticism of Ortaica}{an argument to show that \Ortaicanism was evil}.

Some basic rules and principles of the \Ortaican religion were:

\begin{itemize}
  \item 
    Only the strong and wise should try to be \rethyaxes.
    The \rethyaxes should be the leaders of society.
    Many non-Iquinian nations were ruled by \rethyaxes.
  \item 
    The strong and wise had a duty to protect the weak and ignorant.
    Not least, they had to be protected from themselves.
    This meant that the wise had to lead and create laws to keep the masses safe. 
    And prevent the masses from learning what they should not. 
  \item 
    The world was full of evil and horrors that hunger for the souls of mortals. 
    They would devour all the souls of the dead if they could and condemn them to everlasting suffering. 
    \Iquin was one such soul-devouring horror.
    (Have juicy myths about these horrors.)
    Mortals could not hope to defend themselves from such monstrosities.
    So they must serve the gods. 
    In return the gods would safeguard their souls from the horrors so that they could pass through the void unharmed. 
    The souls would then reincarnate to new life and be given a new chance to rise to greatness and enlightenment. 
  \item 
    Everyone had the potential for greatness in them, but not everyone should strive for greatness. 
    Some people's lot in life was to serve. 
    They should just live a virtuous life.
    If they did that, they would be reborn into a better life.
    Their chance would come soon enough if only they honoured the gods and their betters.
    One day, in another life, they might themselves become \rethyaxes. 
  \item 
    The masses should serve the \rethyaxes and the \rethyaxes should serve the gods.
    The gods, in turn, would grant power and knowledge to the \rethyaxes, and the \rethyaxes would protect the people and keep them safe. 
    A cosmic feudalism. 
  \item 
    No one should ever try to steal an \arcanum. 
    Instead, people should strive to be virtuous.
    Knowledge of \arcana did not make a soul better.
    Only virtue made a soul better. 
    Those who were worthy would be granted access to the \arcana in due time. 
\end{itemize}








\subsection{Gods and angels}
The \rethyax gods were divided into two categories: 
The \Primordials and the Younger Gods. 





\subsubsection{\Primordials}
\target{Primordial}
\target{Primordials}
\target{XS Taorthae}
The \Primordials were really \xss.
In \Ortaican mythology they were credited with the creation of the world and living creatures.
The \Primordials came from the stars and shaped the world into its current form. 
Except \Nelxurra, who never set foot on \Miith but stayed among the stars. 

They had since absconded and now slept and kept themselves aloof from the world. 
But they were still sources of great power. 
The \Primordials were alien and incomprehensible to mortals. 

There were four \Primordials.
(The \Ortaicans did not know of \hr{Satha}{\RuinSatha}.)
They were known to the \Ortaicans by variant names.
The \quo{true} names of the \Primordials (the names the \dragons used) were deep and secret \arcana.

The mortals did not know they were actually \xss. 
(Most \rethyaxes did not know what a \quo{\xs} was.)
But their descriptions in \Ortaican mythology were fairly close to their true natures. 
Although they were more anthropomorphized in their \Ortaican versions.
The descriptions were close enough that a \rethyax who knew of the \Primordials would recognize them when he encountered the \xss (and be shocked and horrified by the revelation). 

Each \Primordial was \hr{Primordials and the Scathaese mind}{associated with a part of the \scathaese body and psyche}. 

\citeauthorbook[p.137]{RobertEHoward:TheAltarandtheScorpion}{Robert E. Howard}{%
  The Altar and the Scorpion%
}{
  \ta{The real gods are dark and bloody!
    Remember my words when soon you lie on an ebon altar behind which broods a black shadow forever!
    Before you die you shall know the real gods, the powerful, the terrible gods, who came from forgotten worlds and lost realms of blackness.
    Who had their birth on frozen stars, and black suns brooding beyond the light of any stars!
    You shall know the brain shattering truth of that Unnamable One, to whose reality no earthly likeness may be given, but whose symbol is\dash the Black Shadow!}
    
    The girl ceased to cry, frozen, like the youth, into dazed silence.
    They sensed, behind these threats, a hideous and inhuman gulf of monstrous shadows.
}



\begin{gloss}
  \begin{comment}
  \subparagraph{\Costorul}
  \end{comment}
  \gitem{\Costorul (\hr{Khoth-Sell}{\KhothSell})} 
  \target{Costorul}
  \index{\Costorul}
  \index{\KhothSell}
  Goddess of life, death, rebirth and immortality. 
  
  Associated with the earth, the groin and sexuality. 
  
  
  
  \begin{comment}
  \subparagraph{\Kythraxas}
  \end{comment}
  \gitem{\Kythraxas (\hr{Kyaethem Chrei Az}{\KyaethemChreiAz})} 
  \target{Kythraxas}
  \index{\KyaethemChreiAz}
  \index{\Kythraxas}
  God of\ldots{} something. 
  
  Associated with \hr{Kyaethem Chrei Az associated with air or water}{the air and wind, or maybe with the sea}. 
  
  
  
  \begin{comment}
  \subparagraph{\Nelxurra}
  \end{comment}
  \gitem{\Nelxurra (\hr{Nerrhan-Koss}{\NerranKoss})} 
  \target{Nelxurra}
  \index{\Nelxurra}
  \index{\NerrhanKoss}
  God of the occult and the distant Realms. 
  
  Associated with the stars and the questing, contemplative, philosophical mind. 
  
  
  
  \begin{comment}
  \subparagraph{\Rammasul}
  \end{comment}
  \gitem{\Rammasul (\hr{Naath-Kur-Ramalech}{\NaathKurRamalech})} 
  \target{Rammasul}
  \index{\NerrhanKoss}
  \index{\Rammasul}
  God of\ldots{} something. 
  
  Associated with the firmament (sky), the cranium and the firm, righteous thinking mind. 
\end{gloss}



The existence of the \Primordials was an \arcanum known only to the \rethyaxes. 
The masses only knew of and prayed to the \Taorthae. 
The \rethyaxes believed that the \Primordials were so dark and so dangerous that no commoner should interact with them at all. 
No uninitiated mortal should desire to look upon the faces of the \Primordials or hear their voices, lest his very mind be blasted and destroyed by the sheer awesome power of the eldest gods. 
The \Primordials dwelt in the Elder Darkness far beyond the fields of mortals, for their visages were too great and terrible for the world to behold. 

\citebandsong{Nile:Ithyphallic}{Nile}{
  As He Creates So He Destroys
}{
  No living creature can look upon his face\\
  And endure its terrible heat and black radiance\\
  That is like the reverberating unseen rays of molten iron\\
  Which strike and burn the skin of those who would dare\\
  Gaze into the countenance of the idiot god
}




\subsubsection{\Taorthae}
\target{Taortha}
\target{Taorthae}
\target{gods of Ortaica}
\target{Ortaican gods}
The Younger Gods, the \Taorthae, were depicted in mythology as descendants of the \Primordials.
While the \Primordials slept, the \Taorthae had inherited rulership of \Miith.
The Younger Gods were more humanoid in aspect and easier to worship. 
It was they who were charged with the daily maintenance of the world and with protecting their worshippers from malevolent powers.
  
They were \hr{Sseju finds gods}{discovered, contacted or freed by Sseju}. 
Some of these gods were \dragons. 

Some of them were later killed by the \hr{Tepharite}{\Tepharites}. 
Others went into the background and into hiding. 
They still exist and are still worshipped by occasional heathens. 
The \rethyaxes{} still swear by and invoke them. 

Some of them are \hr{front-end}{front-ends} for the dead \hr{Elder Dragons}{\firstgendragons}. 

Maybe the name should be different. 
Tabortha? 
Tarbotha? 
Thwerta?
Dwertha? 
Toartha?

Maybe have both an original \Ortaican{} form (more Greek) and a newer \Tepharin{} form (more Celtic). 

Some of the \taorthae appeared in the form of \dragons. 
After the Absconding, \rethyactic theologians would discuss whether they actually were \dragons or whether they just took on their likenesses. 
Even the \rethyaxes \hr{People do not believe in Dragons}{doubted whether \dragons really existed}.

The \taorthae included: 

  
\begin{gloss}
  \begin{comment}
  \subparagraph{Daxian}
  \end{comment}
  \gitemlink{Daxian} 
  \index{Daxian}
  God of weather and the \Wylde{}. 



  \begin{comment}
  \subparagraph{Ishcatla}
  \end{comment}
  \gitem{Ishcatla}
  \index{Ishcatla}
  
  
  
  \begin{comment}
  \subparagraph{Isxae}
  \end{comment}
  \gitemlink{Isxae} 
  \index{Isxae}
  Goddess of law and rulership. 
  
  
  
  \begin{comment}
  \subparagraph{\Mezzagrael}
  \end{comment}
  \gitemlink[Mezzagrael]{\Mezzagrael}
  \index{\Mezzagrael}
  An ancient god. 
  He was the second \Nechsain (rightful ruler of \Miith), inheriting the title from \hr{Settras}{\Settras}. 
  He was \hr{Ortaican myths about Nexagglachel}{killed by traitor gods}.
  His real identity was \hr{Nexagglachel}{\Nexagglachel}.  
  
  
  
  \begin{comment}
  \subparagraph{\Nasshikerr}
  \end{comment}
  \gitemlink[Nasshikerr]{\Nasshikerr}
  \index{\Nasshikerr}
  God of shadows and the hidden. 
  Patron of spies, assassins and thieves. 
  
  
  
  \begin{comment}
  \subparagraph{Rissit}
  \end{comment}
  \gitemlink[Rissit]{Rissit}
  \index{Rissit}
  God of sorcery and science. 
  Also a \hr{Rissitic gods}{Rissitic god}.
  His real identity was \hr{Secherdamon}{\Secherdamon}. 
  
  
  
  \begin{comment}
  \subparagraph{\Settras}
  \end{comment}
  \gitemlink[Settras]{\Settras}
  \index{\Settras}
  \target{Settras}
  An ancient god. 
  He was the first \Nechsain (the ruler of \Miith).
  Sacrificed his life in the \hr{Ortaican myths about Sethicus}{first war against the dark forces}. 
  His real identity was \hr{Sethicus}{\Sethicus}. 
  
  The Rissitic name \quo{\hr{Sesstra}{\Narkiza}} was a variant of \quo{\Settras}. 



  \begin{comment}
  \subparagraph{Shellagh}
  \end{comment}
  \gitemlink{Shellagh}
  \index{Shellagh}
  God of the sea.  



  \begin{comment}
  \subparagraph{Sithra}
  \end{comment}
  \gitem{Sithra}
  \index{Sithra}



  \begin{comment}
  \subparagraph{\Thessulax}
  \end{comment}
  \gitemlink[Thessulax]{\Thessulax}
  \index{\Thessulax}
  Goddess of the earth, the underworld and the undead. 
  
  
  
  \begin{comment}
  \subparagraph{\Usherain}
  \end{comment}
  \gitemlink[Usherain]{\Usherain}
  \index{\Usherain}
  A goddess with a portfolio overlapping that of \hr{Nasshikerr}{\Nasshikerr}. 
  Also a \hr{Rissitic gods}{Rissitic god}.
  Her real identity was \hr{Nzessuacrith}{\Nzessuacrith}. 



\end{gloss}





\subsubsection{Appearance}
\target{Appearance of Ortaican gods}
The \Ortaican gods and angels took grotesque forms.
They could look like combinations of many creatures:
Humanoids, \nycans, \lothae, \murocs, \hr{Pteran}{\pterans} and more. 

Legends had it that the gods would sometimes walk among mortals. 
They would travel in disguise, for ordinary mortals were not ready to behold the true faces of the gods. 
The faces of the gods were \arcana. 









\subsection{Planes of existence}
\target{Ortaican planes of existence}
\Rethyax metaphysics taught that the world was composed of a number of layers or planes. 

This was inspired by \Sethicus's \hr{Sethican planes}{theory of planes}. 

Commoners could only perceive the most shallow layer, the Material Plane. 
\Daimonia dwelt in the deeper planes. 
\Rethyaxes learned how to feel and interact with the deeper planes. 
They needed the help of the gods to do so. 

The mere existence of the planes was a part of the \hr{First Arcanum}{First \Arcanum}.
You had to learn other \arcana in order to actually feel the planes. 

The \Primordials alone knew all the planes, and how many there were. 









\subsection{Potential for greatness}
\target{Ortaican potential for greatness}
Unlike Iquinians, \Ortaicans were expected to strive for greatness. 
Humility was a virtue, but it was not given nearly as great emphasis as it was in Iquinian theology. 
An \Ortaican was expected to know his own limitations.
He must strive for purity before attempting anything that was out of his league.
But he should always hope for more. 

\Ortaicanism was a self-development religion. 
This was ultimately rooted in \hr{Sethican philosophy}{\Sethican philosophy}.

The \rethyaxes took this to a higher leve. 
They sought to become closer to the gods and eventually become gods or demigods themselves.
This quest, however, was in itself a secret \arcanum.
The masses knew nothing of this ambition.

In the early days of \Ortaica, only \scathae were considered to have this potential for greatness. 
This had to due with the fact that \hr{Ortaican religion was Scatha supremacist}{\Ortaica was a \scatha supremacist culture} and tied in with the (semi-true) mystic doctrine of \hr{Primordials and the Scathaese mind}{the \Primordials dwelt in the \scathaese mind}. 









\subsection[The Scathaese mind]{The \scathaese mind}
\target{Primordials and the Scathaese mind}
According to \rethyax mysticism, \scathae were created by the \hr{Primordial}{\Primordials}.
The {\Primordials dwelt in the minds and bodies of \scathae}. 

Each \Primordial created a part of the body and and a part of the \scathaese psyche. 
Each was also associated with a part of the world. 
See the section about \hr{Primordial}{\Primordials} for more. 

In truth, \hr{Origin of Scathae}{\scathae were created by the \dragons} using bio-magic learned from the \xss.
The process of creating the \scathae was very much inspired by the way the \dragons had created themselves.
The \scathae were designed as \dragons in miniature.
As such, \scathae \hr{Scathae have potential for greatness}{genuinely did have more individual potential for greatness} than \hr{Humans suck}{\humans did}.

Parts of a \scatha's consciousness felt like different gods or monsters. 
This was one of the \quo{truths} behind the myths that the various gods created various parts of the \scathaese mind. 
\Rethyax magic employed these mind aspects. 
A \rethyax had to activate his inner snake/worm/whatever to cast spells. 

It was horrifying when people recognized what other dark things were connected to things inside their minds.
Including various grotesque \xss such as \Ubloth.














\section{Politics}









\subsection{Dark}
The \Ortaican culture and religion should be portrayed as dark and sinister and occult and scary. 
They did horrible things. 
For example, they had \hr{Ortaican blood sacrifice}{blood sacrifice}

But the \Ortaican religion was not necessarily evil.
The Iquinians \hr{Iquinian sacrifices too}{needed sacrifices too}, they just kept it hidden. 









\subsection{\Humans}
\target{Ortaican religion was Scatha supremacist}
The \rethyax religion was designed with \scathae in mind.
It was very anti-\human at first. 
The \hr{Primordials and the Scathaese mind}{\Primordials dwelt in the minds and bodies of \scathae}. 

Later, theologians made concessions and allowed that the \Primordials might also dwell in \humans.









\subsection{Vaimons and Iquinianism}
\target{Ortaican-Vaimon relationship}
The \Ortaican religion was formed in the days of the \VaimonCaliphate. 
As such, it was very much a reaction against the Vaimon faith, and many of its teachings were devoted to telling how bad Iquinianism was and how it should be opposed.

\target{Iquinian criticism of Ortaica}
The Iquinians used the fact that \hr{Ortaican religion has few laws}{the \Ortaican religion had few laws} as an argument to show that \Ortaicanism was evil.
The Iquinians themselves \hr{Iquinianism has many laws}{had many laws}.
They argued that there were huge amounts of obviously evil deeds that the religion did not forbid and thus tacitly allowed and thus encouraged.
(Notice the nice straw man argument.)

The \Ortaicans retorted that Iquinianism was totalitarian where \Ortaicanism left some room for free will and humanoid common sense. 

The Iquinians also criticized \Ortaica because the \Ortaicans \hr{Ortaican blood sacrifice}{practiced blood sacrifice}, even that of humanoids.
But the Iquinian religion \hr{Iquinian sacrifices too}{was just as bad}. 

The \rethyaxes and others believed that Iquinianism was dangerous.
It was an apocalyptic religion whose declared goal was the destruction of the world (in the form of the dogma of \hr{Tikkun}{\tikkun}).
It should be obvious to anyone that they were dangerous and should be fought.
The Vaimons commanded powerful magic. 
Who knew if the Vaimons might one day succeed in their crazy dreams to destroy the world?

The \Ortaicans had \hr{Ortaican myths about Iquin}{myths about how \iquin came to be} and how evil it was. 














\section{\Rethyaxes and magic}
\target{Rethyax}
The \Ortaicans{} founded their own order of Chaos sorcerers, the \rethyaxes. 
They were ruled by \rethyax{} sorcerer-kings. 

\target{Rethyax magic}
\target{Rethyactic magic}
\Rethyactic magic was a variant of \Draconic{} Chaos magic. 
It involved the summoning of \hr{Daemon}{\daemons}. 

It did \emph{not} have the concept of \hs{Aenigmata}. 

But it did have plenty of astrology. 
The \daemons{} were associated with planets, \hs{moons} and stars. 

The \hr{Ortaican gods}{\Ortaican{} gods} were also associated with celestial bodies. 









\subsection{Demography}
\target{Rethyax demography}
There are fewer \rethyaxes{} than Vaimons. 
This is partially because the you need to understand some deeper, darker mysticism in order to become a \rethyax, and go through \hr{Rethyax initiation}{harder, more mystic initiation tests}. 

Vaimons just go through the \Archons, which are pretty user-friendly in comparison to the Chaotic powers the \rethyaxes{} wield. 
This power is filtered through the \Ortaican{} gods and originates from the \xss{} (although many \rethyaxes{} do not know the last part). 









\subsection{How to cast it}





\subsubsection{Shemrod}
\Rethyaxes{} sometimes smoke shemrod leaf in order to be able to channel more magic. 
It puts them in a faint trance-like state similar to the Vaimons' \hr{Shechinah}{\shechinah}. 
It protects them against the physical pain and damage of the virulent Chaotic magic. 

But shemrod is pretty unhealthy on its own. 
An overdose can be lethal. 









\subsection{Initiation}
\target{Rethyax initiation}
A \rethyax{} has to go through a tough, dark, mystic initiation in order to gain his powers, open his body and soul as a vessel of Chaotic power and bind the \hr{Daemon}{\daemons} to his will. 

\citeauthorbook[p.16]{RPG:Warhammer:DarkElves}{Gav Thorpe et al}{%
  Warhammer: Dark Elves%
}{
  A Sorceress must walk the dark paths of the Realm of Chaos, the deep pits of the oceans, and the raging bowels of the fiery mountains in her quest for knowledge. 
  The channelling of the raw Winds of Chaos and the binding of these forces give the Sorceress her power. 
  The creatures of the Chaos Hells will bow to her will in the end. 
  Such power is vast but dangerous, and the aspirant to the Dark Convent of the Sorceresses must be courageous and strong. 
}









\subsection{Meditation}
\Rethyaxes meditated to contact their gods and gain power. 
When a \rethyax wanted information from his god, he also did a meditation like this. 

\citeauthorbook[p.107--111]{StephenSennitt:LiberKoth}{Stephen Sennitt}{Liber Koth}{
  The dark wheel shifts to the next Aeon and the Temple is in readiness for the next phase.
  
  You squat, cowled in black before the flame which reveals the sigil of Hastur.
  You send your consciousness through the portal into the blood-red freezing desert night; into the wasteland of ancient Hali, which once might have been a paradise, but is now as dry and yellow-black as a corpse, with the dripping blood suns that never fully rise and never fully set.
  
  Moving across this deathly scene is a black titan windstorm, curiously silent in its merciless frenzy.
  It is the beating wings, the fluttering eyelid, of Great Hastur, the voice of the Old Ones, the Cry of Silence, He who stalks the spaces between the stars.
  As the cyclone that is His being approaches, you are sucked into the vortex and taken into the Higher Dimensions of reality.
  Here, the howling silence speaks to you, the eyelid beats in rhythm to the pulsation of the stars, the abyss of space is charged with a terrible sneitnece; the Great God blows through you, and you crack open like a lightning-struck tree.
  
  \ldots
  
  The aeonic wheel reaches its nadir and there is a gigantic shifting in the Black AByss beyond the portal.
  
  The sigil of Cthulhu is illuminated by one pale candle.
  As the shifting and pulsing becomes increasingly violent you feel yourself pulled through the vortex of the portal into the outer darkness.
  For a time there is nothing, a resounding absence of awareness, until, like someone awakening from a long, long sleep, you find yourself in a hellish terrain of ink-black ichor; the deep slime and mud of an island raised from the ocean-bed. 
  
  You follow a turgid path of sorts between ghastly ruined stones and pillars which depict strange aquatic lifeforms, until you reach a Temple, its ruined entranceway allowing you access into its dark interior.
  Inside the stench of aeons long death assails the nostrils, and the cloying decay is suffocating\dash and yet something stirs and vague thought-patterns begin to formulate into a haunting invokatory refrain:
  \quo{\emph{That is not Dead which can Eternal lie; And with strange Aeons even Death may Die.}}
  
  The sibilant voice echoes in your mind as you wind your way further into the lair of darkness.
  Finally you come to a rough, stone door upon which is emblazoned the sigil of Cthulhu. 
  It writhes, strangely sentient, and trident-like.
  At the whim of Something in the dark, the door awkwardly slides open and you enter into a cavernous chamber, beautiful in its ferocious asymmetry, flistening in deepest jade ran through with jet-black veins of marble.
  And it the centre of this terrible majesty sits Great Cthulhu Himself, Priest of the Old Ones.
  The dimensional strain He pulls on your perceptions seem to distort and displace the Angles of Space and Time, each displacement causing a shift in your perception, a radical change in consciousness, so that every facet of Self is revealed and it is understood that in reality \quo{Death} does not \quo{Die}; that Self is Deathless and Eternal.
  This realisation is a multi-faceted Jewel in the mind which liberates you from the sleep of Aeons, awakening you to the full knowledge of past-present-future self-reality.
}









\subsection{Was more powerful once}
The \rethyaxes{} were more powerful in the past during the reign of \Ortaica. 
This is because their gods had more worshippers back then and therefore more power to share. 
It was also partially because of \hr{Ishicah}{\Ishicah}. 









\subsection{What it can do}
\index{energy claws}
A few \rethyaxes knew how to create \hs{energy claws}.
This was a rare and powerful spell.






















\chapter{The Rissitics}
\index{Durcac}
\index{Durkhak}
\index{Rissitics}
\target{Durcac}
\target{Rissitic}
\target{Rissitics}
A collection of peoples south in \hr{Durcac Continent}{\DurcacContinent}, united by the Rissitic religion \hr{Rissitic tribes divided}{but not always much else}. 

The largest Rissitic nation was Durcac. 

The whole area was also called the Rissitic Dominion or the Rissitic Empire, which was not always a fitting description.

The Rissitic peoples claimed a huge area as their territory, but \hr{Durcac desert}{some of it is desert} and sparsely populated. 















\section{Culture}









\subsection{Aesthetics}
Rissitics often dress in reds, yellows and browns. 
Tinged with black and gold.









\subsection{Architecture}
\target{Rissitic architecture}
\index{architecture!Rissitic}
The Rissitics utilize tons of \hs{occult geometry} in their architecture. 
Their magic is very advanced in this area. 
It is related to their \hs{glyph} magic, which is also occult-geometric. 





\subsubsection{Pyramids and ziggurats}
\target{Rissitic pyramids}
The Rissitics have some great pyramids. All Rissitics who die have their bodies interred in the pyramids, as dictated by their religion. 

In truth, these corpses are used in secret to amass vast legions of the \hr{Rissitic undead}{undead}. Like in \emph{Warhammer: Tomb Kings}. 









\subsection{Dagger sign}
\index{sign of the dagger}
\index{dagger sign}
The \quo{sign of the dagger} or the \quo{dagger sign} is a Rissitic gesture of greeting, the dagger sign consists of a flat hand held up before one's face with the thumb-edge of the hand near the nose and the fingers pointing up. The hand is then moved down to form a fist touching the center of the chest, fingers inward. 

The sign is an imitation of Rissit's symbol of a snake coiled around a dagger. 
% \also{\HriistN}






\subsection{Units of measurement}
\index{Rissitics!units of measurement}
The Rissitics have their own system of measurement units. 

\index{thumb (unit of measurement)}
\index{finger (unit of measurement)}
\index{stride (unit of measurement)}
A thumb is 3 cm. Three thumbs make a finger (9 cm), twelve fingers make a stride (108 cm).















\section{Demographics}

The \scathaese{} population of Durcac is largely \hr{Mekrii}{\Mekrii}, descended from the \hr{Shurco}{\Shurco}. 

The area claimed by the Rissitics was home to a number of nomadic peoples, some of which were Rissitic in name only, or not at all. 















\section{Geography}





\subsection{Desert}
\target{Durcac desert}
Abroad, it was said that virtually all of Durcac was covered by desert. 
This was actually not true, but mostly just hearsay and rumours, painting the Rissitics as inhuman monsters. 
In reality, only parts of eastern Durcac were desert. 

Remember to have an explanation for this desert. 
Durcac was devastated and defiled in some terrible war in ages past. 





\subsection{Gaznor}
\index{Gaznor}
\target{Gaznor}
A peninsula in southern \Velcad{}, controlled by the Rissitic Dominion. It borders Beirod to the North and the \Risvaelsea{} on all other sides. 

Languages spoken include Rissitic, \Velcadian{} and \Ortic. 















\section{History}





\subsection{Originally part of \Ortaica}
\target{Rissitics were Ortaican}
The Rissitics were originally a part of \Ortaica. 
They were a \rethyactic{} sect concerned with research and science. 
They worshipped \Secherdamon as their most important god. 

\target{Rissitics split from Ortaica}
Rissitism appeared after a religious/political schism among the \Taorthae. 
\Secherdamon, who had \hr{Secherdamon takes the name Nexagglachel}{taken the name \quo{\Nexagglachel}}, wanted more power among the \taorthae.
The other \taorthae did not want to give him more power. 
They fought, and eventually \Secherdamon's followers split off and formed their own religion. 
Note that there was no full-scale war between the Rissitics and the orthodox \Ortaicans. 
There was violence, but overall it was a relatively peaceful religious split. 

In \hs{Rissitic mythology} this schism became represented as a myth of a battle between the gods. 
Compare to the Egyptian myths about Set, Osiris and Horus, and how the various cults used the myths to demonize each other. 

\Nzessuacrith was one of those who opposed \Secherdamon in the schism.
Later, after \Ortaica had dissolved, \Secherdamon persuaded her to ally herself with him again. 









\subsection{Lived on after \Ortaica}
After the fall of the \bacconate{} the Rissitics lived on. 
\Secherdamon{} made the Rissitics his main venture. 
He invested his full support in them. 









\subsection{Merging with tribes}
\target{Rissitics merge with tribes}
At first they existed as an underground cult, or even fugitives, hiding from the dominant Iquinians and Imetrians. 
This went on for centuries. 

The surviving Rissitics, persecuted, fled south into Durcac. 
Here they allied and mixed with some indigenous tribes from the south and east. 
(This was why their language was so far from \Ortaican.)

The Rissitics and the tribesmen learned much from each other, for some of the tribes were \xs{} worshippers and knew some occult Gnosis that the Rissitics did not know. 

Some of the tribes had \hr{Cregorr}{\cregorr} among their numbers. 

\Secherdamon{} and his Rissitics allied with several other gods, who were eventually all subordinated under \Secherdamon. 
Some of them were weaker and consented to serve him. 
Others had to be slain. 
Yet others were aloof minor \xss{} who cared nothing for their \quo{rank} in some \Miithian{} pantheon.





\subsection{Creativity}
\target{Rissitic creativity}
\target{Rissitic research loophole}
\index{technology!Rissitic}
The Rissitics were very creative. 
They had many scientists and made quite a lot of scientific discoveries and breakthroughs. 
They had more success devising new technology than most people on \Miith{} \hr{Shroud represses technology}{ever since the \SecondShrouding}. 

% \Secherdamon{} has discovered a special \quo{loophole} that allows his people to bypass the Shroud and invent new things. 
% I need to work on what this could be. 
\Secherdamon{} had his own plan to break out of the Shroud. 
It was built upon the \hr{Secherdamon gains Gnosis}{\xs{} Gnosis} that \Secherdamon{} discovered after millennia of searching. 
At the time of \booktitle{\hr{Twilight Angel, Remember}{\TwilightAngelRemember}}, it had met with some success. 
It was the reason why the Rissitics could be such free-thinkers. 
The \hr{Rissitic Matrix}{Rissitic \matrix} was working like a giant drill, gradually drilling a \quo{hole} through the Shroud and out into Chaos, thus letting the creative (and destructive) influence of Chaos into their people's minds, allowing them to think new, creative thoughts (or go mad). 

This \quo{drilling} was dangerous business, though. 
It destabilized the barriers between the Realms. 
At times, the barriers would spring open and dangerous things would leak in from the Beyond: 
Monsters, storms, madness, disease, catastrophes that warp the fabric of the world and cause people, buildings and places to become deformed. 
This sort of thing happened distressingly often in Durcac. 

To curb this effect, the Sentinels \hr{Purpose of the caste system}{devised the Rissitic caste system}. 

On a more long-term scale, \ps{\Secherdamon} Shroud-drilling interacted with the Cabal's \hr{Sephirah plan}{\sephirah{} plan} in unpredictable and dangerous ways. 
This destabilized the entire Shroud, causing it to \hr{unravelling}{unravel} and become full of holes. 

\index{Dagger, the}%
The goal of this hole Shroud-drilling business was to create the \hs{Dagger}. 





\subsection{Divided}
\target{Rissitic tribes divided}
Before the coming of \hr{Narkiza}{\Narkiza}, the Rissitics were not a unified nation, but a loose collection of tribes.
Durcac was the largest, most powerful Rissitic nation, but not the only one, and it controlled less than half of the Rissitic peoples.

They were united by the Rissitic religion, but Rissit/\Secherdamon was not the only Rissitic god, and he did not have complete control of them.
Other gods struggled against him, as did the \jinn. 
And Cabal intrigue was a constant threat. 

The Rissitic Dominion was supposed to be a unified empire, but due to \hr{Rissitic infighting}{corruption and infighting and other reasons}, it splintered.

In the last many decades before \SentinelsofMithEmph, \Secherdamon had been very much preoccupied with strengthening his \matrix, getting it into top shape before the coming \thirdbanewar.
And preparing \Vizsherioch to become the \hs{Dagger}. 
He had little time and energy left to devote to political stability, and he realized too late that the Rissitic tribes were badly fractured.
Getting them all back together was a very non-trivial task, even for a god. 

\Secherdamon realized the problem too late. 
He could not solve it in a top-down manner. 
He had to do some bottom-up work, too.
So he found \Narkiza. 

\hr{Narkiza unites tribes}{\Narkiza united the tribes}. 





\subsection{United by \Narkiza}
\target{Narkiza unites tribes}
\Narkiza was already an \Ashenoch and a recognized hero.
\Secherdamon spoke directly to \Narkiza and bade him conquer and unite all the Rissitic tribes. 
It was hard work, but \Narkiza did it.
\Secherdamon heaped many rewards on \Narkiza for it. 


It was \hr{Narkiza}{\Narkiza} who united the Rissitic peoples into a single kingdom and paved the way for \Secherdamon's planned invasion of \Velcad.
Everyone feared this super\human overlord who had subdued the fearsome Rissitics under his banner. 
He had to first convince everyone that Rissit and the other gods were with him.

He got a \hr{Narkiza's dark warlord reputation}{reputation as a dark warlord}. 

















\section{Language}
\index{Durcac!language}
\index{Rissitics!Rissitic language}
The Rissitic peoples did not all speak the same tongue.
There was a whole family of Rissitic languages. 
\quo{High Rissitic} was spoken in Durcac.
Some of the tribes spoke languages completely different from and unrelated to Rissitic.

High Rissitic was far from \Ortaican. 
It had some roots in \Ortaican, but was based more on the languages of the \hr{Rissitics merge with tribes}{indigenous Durcaci tribes that assimilated the Rissitic cult}. 

Rissitic is harsh and guttural, meant to resemble Danish, Dutch and German. 
Some words are taken from Egyptian or Mesopotamian languages, however, since \hs{Rissitic mythology} is inspired by those of Egypt and Mesopotamia. 















\section{Magic}
\target{Rissitic magic}
\target{Rissitics!Rissitic magic}
The Rissitics practiced the magic theory of the \quo{Three Worlds}, which was developed by \RissitNechsain{} from millennia of research. 
It was based on \hr{Rethyax magic}{\rethyactic magic theory}, but very different from it.

The power they wield is great, and many of them, being mortal humanoids, are not up to the task. 
Their minds and bodies succumb to the pressure and begin to \hr{The cost of magic}{twist and warp under the influence of the Chaotic power} that they feebly try to master. 
They become scarred and misshapen, their bodies rotting inside. 
Compare to Hannan Mosag and his K'risnan in \cite{StevenEriksonIanCameronEsslemont:MalazanBookoftheFallen}. 
For this reason, mages wear full-body-covering robes and masks. 
At least the \Shessefkesad, and likely the \Dzeyrgvin{}, too. 

They work hard to preserve their sanity through prayer and mental exercises. 
Sometimes it fails in the long run. 

Some of the priests are offered the gift of undeath to preserve them, since a dead mummy does not mutate. 

The \hr{Ashenoch}{\Ashenoch} are resistant to physical mutation, since their bodies are superhumanly empowered. They may still go mad, tho. 

Rissitic magic (which is a further development on Chaos magic) uses long incantations, and also uses symbols extensively. 
Most spells require an occult \hs{glyph} to be drawn. 
Rissitic magic focuses more on elaborate but powerful arcane rituals rather than quick spells. 
Quick spells exist, but they are difficult and their casting consumes much energy from the caster's body. 
Quick spells are used especially by the \hr{Ashenoch}{\Ashenoch}, who have plenty of physical strength to draw upon. 
But most spells are cast as rituals taking rather a long time and requiring several mages, usually a lead mage and some assistants. 









\subsection{Glyphs}
\target{glyphs}
\target{glyph}
\index{glyph}
Glyphs are writing symbols used in Rissitic. 
Occult glyphs are used in much Rissitic magic. 
Glyphs often consist of pictures of animals, plants and objects as well as geometric shapes.

Most spells require an occult {glyph} to be drawn. 
Glyphs can be drawn in the air with your fingers, or (better) with a special stylus, or they can be drawn on a surface. 









\subsection{Glyph heads}
\index{glyph head}
A glyph head is the severed head of a creature, mummified and enscribed with occult glyphs. Glyph heads are used for storing spells or magical energy for later use. The creation of a glyph head requires the sacrifice of a living creature, and the soul of the head's former owner is bound in the process, used to seal the magical energy in the head. 

When the spells contained are released the \hr{soul destruction}{soul is permanently destroyed} and the head becomes useless (it cannot be enchanted to work as a glyph head again). 









\subsection{\Matrix}
\target{Rissitic Matrix}
\Secherdamon{} designed a special Rissitic \matrix, somewhat separate from the various \draconic{} matrices but still connected. 
It was ultimately based on the \hr{Ortaican Matrix}{\Ortaican{} \matrix}, which was built upon \ps{\hr{Ishicah}{\Ishicah}} \hr{Carcer}{\carcer}. 

The Rissitics improved this \matrix{} and bound new, cooler, more advanced constellations of \daemons{} to it. 

\Secherdamon's \matrix was very powerful and very ambitious.
He tried to build it up to be the vastest and mightiest \matrix ever.
This meant that the \matrix required much maintenance.
So he had enire pyramids and ziggurats and the like erected whose purpose was to channel magical energies and act as \nexi and keep the \matrix alive and stable.
These buildings required sacrifices and spells performed daily to maintain them and keep them running.
They would grow and hum and cracle with sorcerous energy at all times.

Compare to the College of Light, from \cite{RPG:Warhammer:TheEmpire}, whose purpose is to maintain the sorcerous Wind of Hysh.










\subsection{Name theory}
Rissitic magic uses, among other things, some \quo{true names} that, when known, can be used in spells (spoken or written) to give the caster power over the named subject. 

Each creature (and perhaps object?) has a Body-name, a Spirit-name and a Shadow-name. 

These names are most efficient against mortals. 
Mortals are generally \quo{simple} beings whose names are short and manageable. 
Immortals have true names, too, but these typically consist of very long and complex spells, or even rituals. 
Sometimes so complex that they are not worth the effort. 

No Rissitic mage has ever been able to use the true names of a \bane{} to any worthwhile effect. 
Attempting to use a \ps{\bane}{} name (or even learn it) tends to badly damage the caster's sanity. 

This name theory is based on Chaotic \hs{Aenigma} theory. 
It is an attempt at a feasible application of Aenigmatic name theory, which is very heavy and impractical in its raw form. 





\subsubsection{Cursing names}
The Rissitics use the practice of cursing people's magical names:

\citebandsong{Nile:InTheirDarkenesShrines}{Nile}{
  Execration Text
}{
  Mut The Dangerous Dead\\
  Trouble me No Longer\\
  I Inscribe Thy Name\\
  I Threaten Thee With The Second Death\\
  I Kill Thy Name\\
  And Thus I Kill Thee Again\\
  In The Afterlife

  Bau Terror of the Living\\
  Angry Spirits of the Condemned Dead\\
  I Write thy Name\\
  I Burn Thy Name In Flames\\
  I Kill Thy Name\\
  And Thus Thee Are Accursed\\
  Even Unto The Underworld

  Mut The Troublesome Dead\\
  Plague Me No Longer\\
  Thou Art Cursed\\
  Thy Name Is Crushed\\
  Thine Clay is Smashed And Broken\\
  Thy Vengeance Against The Living\\
  Shall Come to Naught
}





\subsubsection{New names}
\target{Rissitic new names}
\hr{Ashenoch}{\Ashenoch} and other \uber-dudes receive new names when they are initiated into their \uber-order. 
The new names and their \hs{glyphs} have mystic significance and are tied to their superpowers. 








\subsection{Oven juice}
\target{Oven juice}
\index{oven juice}
Among other uses, the \hs{ovens} have this property: 
From them can be distilled an energy-rich, spirits-like drink. 
The \nyzlet{} drink this. 
It helps them with their magic. 

But it is quite addictive. 
The more heavy magic you cast, the more you get addicted to the drink. 
Rissitic magic draws its energy both from the caster's body/mind and from the Rissitic \matrix. 

The mages die if deprived of their drink. 
The really old or just really addicted ones can die in as little as 24 hours. 









\subsection{Undead}
\target{Rissitic Liches}
\target{Xul-Gann}
\target{Rissitic undead}
\index{\XulGann}
The Rissitics had undead among their number, including many elder priests. 
They were immortal \hr{Lich}{\Liches}, called the \XulGann.  
This had great religious significance, and they were much revered. 

\hr{Psyrex}{\LocarPsyrex} was \hr{Psyrex' undeath}{one such \Lich}\ldots{} sort of. 

The \hr{Ashenoch}{\Ashenoch} were another type of Rissitic undead. 

All types stayed undead by drinking the blood of \hr{Secherdamon's Resphan slaves}{captive slave \resphain}. 

\citebandsong{Nile:BlackSeedsofVengeance}{Nile}{
  Chapter for Transforming into a Snake
}{
  I am a Long Lived Snake\\
  I Pass the Night and Am Reborn Every Day\\
  I am the Snake which is in the Limits of the Earth\\
  I Pass the Night and am Reborn, Renewed and Rejuvenated Every Day
}









\subsection{Texall Axioms}
\target{Texall Axioms}
The Texall Axioms are a series of magical laws, discovered and formulated by \hs{Texall}. 
They were a great breakthrough and became a central foundation stone in Rissitic magic theory. 















\section{Military}





\subsection{Monsters}
\target{Rissitic monsters}
The Rissitics employ a wide variety of beasts and monsters in war (\hr{Imetric monsters}{just like the Imetrians}).
Including \lothae, \mezolisks and \corgoroth. 

They used the \hr{Hippopotamus}{hippopotamus} in their imagery, but they did not domesticate them. 

See also the section on \hr{Domestic animals}{domestic animals}. 





\subsection{Value of life}
\target{Rissitics value their lives}
The Rissitic population was relatively small. 
This meant that in war, lives were precious to them because of their small numbers.
They used tactics, technology and sorcery to make up for it.















\section{Orders}
The Rissitics have a number of \quo{Orders} of warriors, mages and whatnot. 
Many orders are named after an animal. 
They often carry masks carved in the likeness of their animal. 





\subsection{\Ashenoch}
\target{Ashenoch}
\index{\Ashenoch{} (plural \Ashenoch)}
The \Ashenoch{} (plural \Ashenoch) were a Rissitic order of superhuman undead warrior-mages. 
The name was derived from a longer name meaning \quo{the Order of the Rusty-Red \Corgorah}. 

They received \hr{Rissitic new names}{new names} when they became \Ashenoch. 





\subsubsection{Initiation ritual}
\target{Ashenoch ritual}
The initiate is killed and reborn as an \Ashenoch. 
The process is similar to, and derived from, the \hr{Shaeeroth ritual}{process that turns an \ophidian} into a \dragon. 

\citebandsong{Nile:BlackSeedsofVengeance}{Nile}{
  The Black Flame
}{
  Open, For Me the Gates Shall Open\\
  Over the Fire of the Spirit, The Breath Drawn by the Gods. \\
  Arise Apophis Return, That I Might Return, \\
  Borne by the Flame Drawn by the Gods Who Clear the Way that I Might Pass.\\
  The Gods Which Sprang from the Drops of Blood \\
  which Dripped From the Phallus of Set\\
  That I might be Reborn\\
  For I am Khetti Satha Shemsu, Seneh Nekai\\
  And Will Become Set of a Million Years
  
  Akhu Amenti Hekau\\
  I shed My Burnt Skin and am Renewed
}





\subsubsection{Undeath}
The \Ashenoch were undead.
Like the several other types of \hs{Rissitic undead}, 









\subsection{Order of the Crocodile}
\target{Order of the Crocodile}
Slow-moving heavy cavalry. 
Especially useful as siege troops. 
(Because the crocodile lies in wait for its prey.)





\subsection{Order of the Hippopotamus}
\target{Order of the Hippopotamus}
Heavy shock troopers. 

The \hr{Hippopotamus}{hippopotamus} is the largest and most dangerous mammal in the world, as far as some people know.
Mention this when the Rissitic Hippopotami are mentioned.






\subsection{Order of the Jackal}
Perhaps have an Order of the Jackal, whose members wear jackal-like metal masks and \armour. Compare to the movies \emph{The Scorpion King} and \emph{Stargate}. 

\lyricsbs{Monolith Deathcult}{Deus Ex Machina}{
  For thy blasphemy thou shalt be punished \\
  The Systemlords bring thee thy avengers \\
  Your soul will be possessed and brutally mentally slaughtered \\
  Resistance shall be butchered by Jackal-headed iron warriors \\
  The demons nestled themselves in your bodies \\
  While you stare in the red gleaming eye of the serpent mask \\
  Falcon claws will slit your throat to pieces \\
  Thou shall be enslaved again
}















\section{Politics}




\subsubsection{Corruption}
\target{Rissitic infighting}
Like the rest of \ps{\Secherdamon} minions, the Dark Crescent was \hr{Secherdamon plagued with corruption}{plagued by corruption and infighting}. 

Durcac, the Rissitic Dominion, is ruled by \HriistN, who is actually \HriistD{} in disguise. 
But Durcac is infiltrated. 
Chiefly by agents of other Sentinel factions, but also by Cabalists. 

Have some backstabbing like among Sha'ik's army in \cite{StevenErikson:HouseofChains} or the Malazans in \cite{StevenErikson:TheBonehunters}. 

Remember that the Cabal is more numerous and better organized than the chaotic Sentinels. 





\subsection{Economy and trade}
\target{Rissitic economy}
\index{technology!Rissitic}
Durcac was not merely a warrior nation. 
They were also a major power by virtue of trade. 
Their geographical position gave them access to exotic goods from the south and southeast. 
And their high technology let them produce many unique and valuable export goods. 

Among other things, many quality guns were Rissitic made. 

Have more examples of Rissitic-made things! 

Durcac's soul was not very fertile, but they made up for it with technology and industry. 
The standard of living was relatively high. 
Even slaves were rather well off. 

There was lots of pollution, though. 





\subsubsection{Spice}
\target{Spice and worms}
Durcac exports a pepper-like spice. 
Rumours have it that this spice is produced by some enormous giant worms that live in the \hr{Durcac desert}{desert}. 









\subsection{Ideology}
The Rissitic system is somewhat chaotic and anarchistic on a small scale. 
At least, it is not nearly as centrally controlled as the Imetrium (which is a primitive communist plan economy). 
The Rissitic commoners are poorer, less healthy and less secure than Imetrians, but they are also more \quo{free}, in the sense that the laws and, more importantly, customs are less restrictive. 
For example, sexual morals are liberal. 





\subsection{Reputation}
\target{Rissitic reputation}
Outsiders often saw the Rissitics as murderous and evil, led by a violent, cruel religion. 

\lyricsbs{Monolith Deathcult}{Desecration of the Black Stone}{
  They conspire in their caves in their envious world of emptiness,\\
  preparing for Jihad to obey their desert god.\\
  They murder for belief to deflower 70 virgins.\\
  These \quo{brave} Mudjahedeen brought Gaza in darkness.\\
  Their God is a worm baptised in the blood of the prophet.
}

Be sure to demonize the Rissitics. 
Rumour turned them into legions of demons from Hell, inhuman wielders of dark sorcery and evil.
The fact that their warlord \Narkiza was known to be an \Ashenoch, and that the Rissitics had always relied on magic and monsters, only made their image worse.
They were seen as the marauding Legions of Chaos.

The Rissitics were seen as demonic hordes of evil, feared by the \Velcadians.
They used dark sorcery liberally and all the time, making them seem inhuman and supernatural monsters.

Durcac was a nightmare place where laughing evil sorcerers kidnapped beautiful virgins and cast them screaming into great \hs{ovens} of fire to be burned alive as food for the sorcerers' terrible gods. 

They were seen as \hr{Imetric reputation}{worse than the Imetrium}.

\lyricsbalsagoth{
  Behold, the Armies of War Descend Screaming From the Heavens
}{
  Behold, the armies of war descend screaming from the heavens!
}





\subsection{Threat of destruction}
Perhaps the Rissitics, at least the upper echelons, are desperate to conquer for fear of being devoured from the inside by the Chaotic power that they utilize. 

\lyricsbs{Marduk}{Scorched Earth}{%
  We must win to save us from the plague's grasping jaws\ldots{}}
















\section{Religion}
\target{Rissitic mythology}
\index{Durcac!religion}
\index{Rissitics!religion}
The Rissitic religion, also called Rissitism, is based around the worship of the god \HriistN, also called Rissit. 
\quo{Rissitics} is also the term used by nonbelievers for adherents of the religion. 









\subsection{\Dragons}
\Dragons featured prominently in Rissitic myth. 

See also the section about \hr{Myths about Dragons}{\dragons in art and mythology}. 









\subsection{Funeral ovens}
\target{Rissitic ovens}
\target{ovens}
\index{ovens}
After death, Rissitics are cremated in large, special, enchanted ovens. 
These ovens are designed to melt bodies and souls and extract energy from them to feed the Rissitic \matrix. 

The ovens are made with advanced technology. 
They are more sophisticated and more effective as a means of extracting life energy than just a raw digestion system, or the \psp{\resphain} \hs{Communion}. 

It is a great \honour to be burnt in the ovens.

The oven-burning is more energy-effective if done \emph{before} death, so people are encouraged to commit themselves to the ovens when they grow old, sick or invalid. 
This is an extra-great \honour; giving one's life for \Nechsain. 
Many faithful Rissitics choose this as a good way to die. 

It is almost expected for the \nyzlet{} to do this. 
Their bodies and souls are especially energy-rich after a whole life casting magic and channelling the power of the Rissitic \matrix. 

Sacred animals are sometimes sacrificed in the ovens. 
Also, criminals and prisoners of war are given to the ovens as sacrifices. 
This is seen as an \honour for them. 
It is a merciful redemption, absolving them of their sins as infidels or sinners, and accepting them into the Rissitic afterlife. 









\subsection{Gods}
\target{Rissitic gods}
The Rissitics worshipped a number of gods. 


\begin{gloss}  
  \begin{comment}
  \subparagraph{Rissit}
  \end{comment}
  \gitemlink[Rissit]{Rissit}
  \index{Rissit}
  God of sorcery and science. 
  Also a \hr{Taortha}{\taortha}.
  His real identity is \hr{Secherdamon}{\Secherdamon}. 

  
  
  \begin{comment}
  \subparagraph{\Usherain}
  \end{comment}
  \gitemlink[Usherain]{\Usherain}
  \index{\Usherain}
  A goddess with a portfolio overlapping that of \hr{Nasshikerr}{\Nasshikerr}. 
  Also a \hr{Taortha}{\taortha}.
  Her real identity is \hr{Nzessuacrith}{\Nzessuacrith}. 



\end{gloss}









\subsection{\RissitNechsain}
\target{Rissit is a saviour}
In Rissitic mythology, \hr{Rissit Nechsain}{\RissitNechsain} was seen as a saviour figure. 

Rissit was the brother of \hr{Mezzagrael}{\Mezzagrael}, the great god and original rightful ruler of \Miith who had sacrificed himself to save the world.
The ghost of \Mezzagrael once came to Rissit and showed him a revelation.
\Mezzagrael appointed Rissit as his successor and heir and charged him with the task of overthrowing the false gods. 

Unto Rissit, \Mezzagrael bestowed all his power and kingship. 
In fact, the ghost of \Mezzagrael merged with Rissit and let itself absorb into him. 
Thus Rissit gained and earned the title \Nechsain, which had originally belonged to \Mezzagrael. 
In a sense, \Nechsain was not merely the brother and heir of \Mezzagrael but his reincarnation. 
As such, it was \Nechsain's duty and right to conquer the world and rule it as the new \Mezzagrael. 

At the time this myth was founded, \Secherdamon/Rissit \hr{Secherdamon takes the name Nexagglachel}{really did take the name of his brother}. 

This apotheosis of Rissit happened at the time when the \hr{Rissitics split from Ortaica}{Rissitics split from \Ortaica}. 
Unfortunately, according to Rissitic mythology, some of the other \taorthae in their blindness and stubbornness refused to acknowledge Rissit as \Nechsain.
Therefore the Rissitics had to split from \Ortaica and become rivals.

Where \Mezzagrael is comparable to Osiris from Egyptian mythology, Rissit is comparable to Horus, who is both the son and the reincarnation of Osiris. 



\citeauthorbook[\quo{First Thought in Three Forms}, p.86--100]{%
  BentleyLayton:TheGnosticScriptures%
}{%
  Bentley Layton%
}{%
  The Gnostic Scriptures%
}{
  But then for my part, I descended and got as far as chaos.\\
  And I dwelt with my own who were there, hidden within them, bestowing power and imparting image unto them.\\
  And down to the present [\ldots{}] those who [\ldots{}], i.e. the offspring of the light.\\
  It is I who am their parent. \\
  And I shall tell you a mystery that is ineffable and indescribable by any mouth.\\
  For you I loosed all the fetters and broke the bonds of the demons of Hades, bonds that were bound to my limbs and worked against them. \\
  And I threw down the high walls of the darkness,\\
  And I broke open the solid gates of the merciless and split their bolts.\\
  And the evil agency, who strikes you, who impedes you, the tyrant, the adversary, the ruler, the real enemy\dash as for all of these, I taught them about my own, the offspring of the light:\\
  So that they might become loosened from all these and rescued from all the fetters, and might enter the place where they had been in the beginning.\\
  It is I who am the first to have descended, for the sake of that part of me which remained, \\
  Namely, the spirit that exists within the soul and which has come to exist out of the water of lige and out of the baptism of the mysteries.\\
  I myself spoke with the rulers and with authorities, for I had descended deep into their language;\\
  And I uttered my mysteries to my own\dash a hidden mystery\dash \\
  And the fetters were loosened, as was eternal forgetfulness.\\
  And within them I bore fruit, namely, the thinking that concerns the unchangeable eternal realm (aeon) and my house and their parent.\\
  \ldots{}\\
  All who were in me shone bright.\\
  And for the ineffable lights within me I prepared an manner of appearance.\\
  Amen!
}





\subsubsection{\Belzir}
\target{Rissitic myths about Belzir}
In Rissitic mythology, Rissit took credit for having shaped \Belzir into the destroyer-of-the-\caliphate she ended up becoming.
There was \hr{Rissit helps set up Belzir}{some truth in this}. 










\subsection{Reincarnation}
The Rissitics believed in \hr{Reincarnation}{reincarnation}. 









\subsection{\Resphan slaves}
There were rumours about \hr{Secherdamon's Resphan slaves}{\Secherdamon's \resphan slaves}. 









\subsection{Sacred animals}
A number of animals are \quo{sacred} and have religious significance to the Rissitics. 
The most important of these are cobras, \hr{Mezolisk}{\mezolisks} and \hr{Corgorah}{\corgoroth}. 








































\chapter{The Vaimons}















\section{\Delaen}
\target{Delaen}
\target{Delain}
\index{\Delaen}









\subsection{History}





\subsubsection{Secular}
\target{Delaen was secular}
At the time of \hr{Vizicar}{\VizicarDurasRespina}, \ClanDelaen was the most secular and non-religious of the \VaimonClans.
This helped Vizicar grow up as a free-thinking man. 















\section{Geican}
%\section{\ClanGeican}
\target{Geican}
\target{Geica}
\index{Geica}
Geica, located in the southeastern corner of \Velcad{}, is the homeland of the Vaimon \ClanGeican. 

Unlike the Redcor, who are purely a \VaimonClan, \ClanGeican is known to accept non-Vaimons \quo{converts}, granting them status of full \vclan members. Once achieved, Geican status lasts for life and is passed on to all children\ldots{} or is it? 

\index{Geicanese}
The Geicans, ie., the members of \ClanGeican, are the upper class of Geica. The regular citizens of Geica are called \emph{Geicanese} (the term is singular, plural and adjective). About $60\%$ of the Geicans are Vaimons. 

They are scientists and study all sciences, including magic. They are known to use Nieur and actively pursue the study of it. 
They are believed to dabble in all sorts of black and evil magic and to consort with evil powers. 
For this reason they are seen as evil diabolists by some, especially their ancient rivals, the Redcor. 









\subsection{Aesthetics}
The traditional \colour of \ClanGeican is green, and many Geicans wear green robes of some sort. 
The symbol of Geica and \ClanGeican is an eagle in flight (a symbol of freedom), green on a black background. 
The eagle symbolizes freedom, and the black background signifies that without freedom (i.e., outside the eagle) there is only darkness and evil. 

The throne of Geica is of Emerald. 









\subsection{Culture}





\subsubsection{Debt slavery}
\target{Geican slavery}
The Geicans adore \quo{freedom}, so they don't have slaves. 
In name. 
But if you owe too much money, you can be indentured under slave-like conditions. 









\subsection{Geography}
\subsubsection{Fallen Emerald Palace}
Originally the Geican seat of power was the splendid Emerald Palace. 
But it was destroyed in the \darkfall. 

All \VaimonClans used to have such palaces of crystal. 
But today only the Redcor \hr{Topaz Chateau}{\TopazChateau} remains. 








\subsection{History}
The founder of \ClanGeican was Tiraad Geican, son of Cordos Vaimon. 









\subsection{Philosophy}
\target{Geican philosophy}
As a culture, the Geicans are atheistic and anarchistic and hail the freedom of the individual as their highest ideal. 

The Geicans (or, at least, some Geicans) believe that philosophy and science should be free and unhindered by morality. 

Some of them believe that philosophy and science \emph{should} be dark, scary and boundary-crossing. Only by abandoning everything known and safe and throwing yourself out in the deep water of the unknown, in the vast, dark emptiness of the inhuman universe, can you gain new insight. 

They believe that if you do not confront the frightening and unknown, you will continue to live in its shadow, on its mercy. As its slave or as its prey. 

Or are they a rare example of the triumph of mortals? Then again, the Geican democracy is a pretty corrupt, mafia-like nepotist system.

The modern-day Geican philosophy was founded by \hs{Zacrias}, son of  \Belzir{} and the first leader of the \vclan after the \Darkfall. 





\subsubsection{Religion}
The Geican culture is atheistic. They reject the personification of Iquin and Nieur that the Redcor believe in. According to the Geicans, the \hs{Iquinian} religion is a lie: The Spirit of the Light does not exist, Iquin (like Nieur) is just an impersonal, amoral force. All Iquinian \quo{miracles} are regular magic and the Iquinian metaphysics is based on a magic theory that is flawed and contradictory. The Redcor world view is a supremely naive fairy tale that blatantly ignores and denies many elementary facts known to any serious scholar. The Geicans see the Redcor as cowards who close their eyes to the real world because they are afraid to face the dark and terrible truth. In turn, they view themselves as a superior and wiser people, true scientists who pursue the truth without fear. 

Natually, this belief is blasphemous to the Iquinians. To make matters worse, \ClanGeican is known to have utilized black magic, and the Dark Prophet \Belzir{}, who caused the Darkfall, was a Geican. As a result, the Redcor believe that the Geicans are all malicious diabolists and alienists bent on conquering or destroying the world. (The Geicans themselves claim that the Redcor are intolerant bigots and prejudiced against them because of a select few genuine villains in Geican history. They are also quick to quote a record of Redcor atrocities in return.) 





\subsubsection{They were once religious}
The Geicans have not always been atheists. 
Before \ps{\Belzir}{} time, \hr{Geicans were religious}{they were Iquinian}. 










\subsection{Politics}
\subsubsection{Government}
Geica has a democracy and is governed by a Senate of 30 Senators. Only Geican \vclan members can vote. 

Originally the \vclan was led by a Grandmaster. 
\Belzir{} and Zacrias both held this title. 
Some centuries after the \Darkfall, the Grandmaster came to be democratically elected. 
Later it disappeared entirely and was replaced by a Senate. 





\subsubsection{Free thinking: An embattled kingdom}
\target{Geica is embattled}
Why are the Geicans allowed to be such free thinkers? 
%Are they under someone's protection? Perhaps \Kezerad{} or the \Cuezcans? 

Well\ldots{} Geica is embattled by the master races. Both factions seek to control the \vclan \hr{Master races seek to control magic}{and its magic}, and they antagonize each other, so that in the end no one controls Geica. This is one of the reasons why there is so much free thinking going on there. 









\subsection{Physique}
\target{Geican green eyes}
Green eyes were considered the hallmark of Geican \humans.
They were still rare among Geicans, but much more common than among \humans as a whole.
And a number of prominent figures in Geican history had those eyes, including \hr{Belzir}{\Belzir}. 
(\hs{Shereid} also had them.)









\subsection{The Royalist Faction}
\target{Royalist Faction}
\target{Royalist}
The underground faction that serves \hr{Belzir}{\Belzir} and aims to restore her to life and power. 

The faction is actually being manipulated by the \hs{Sentinels}. 

Notable members include \hs{Hayad}, \hs{Shereid} and, ultimately, \hr{Carzain}{Carzain \Shireyo}. 















\section{\Iquinian Church}
\target{Church of the Light}
\target{Iquinian}
\target{Iquinian Church}
\target{Iquinianism}
\target{Iquinian religion}
\index{Iquinian Church}
\index{Church of the Light}
\index{Church of \Iquin}
\index{Iquinianism}
The Iquinian Church, also called \emph{Iquinianism} or \emph{the Church of the Light}, was a religion founded by the Vaimons. 
Iquinianism was based on the worship of \iquin{}, the One Light, seen as the source of all good, and of the \Sephiroth{} as its manifestations. 

There were two major branches of the Iquinian church: 
The \hs{Redcor} branch and the \hr{Telcra}{\Telcra}. 

\index{Iquinian Church!Redcor branch}
\index{Spirit of the Light}
The Iquinian Church, also called the {Church of Iquin} or the {Church of the Light}, is based on \Iquin-\Nieur theory (see section \ref{Vaimon magic}), the idea that the universe is based on two primal forces: 
\Iquin (the One Light) and \Nieur (the Outer Darkness). 
The Iquinians worship \Iquin and believe that it is the primary, the first and most important of the two, representing all that is just and good. 
The One Light (sometimes personified as a god) is viewed as a supreme, perfect being, all-good and very powerful, \hr{Omnipotence of Iquin}{perhaps even all-powerful}. 
The One Light is seen as something special because it is \quo{not of this world} but something \quo{transcendent}. 
The Church tends to look down on religions that worship \quo{earthly gods} (such as the Imetrium), which are seen as inferior, false gods. 

The Iquinians see themselves and their church as the champions of good. 
They seem to combat and destroy evil wherever they encounter it. 
The Church of \Iquin is widespread throughout most of \Velcad{}. 
The Church does not rule directly outside \Redce{}, but the Redcor are skilled manipulators, pulling strings and directing events from behind the thrones. 

The Iquinian religion was also called the \quo{Vaimon religion}.
Post-\caliphate \hr{Geican}{Geicans} were not happy about this, since they had abandoned much of the religion but were still Vaimons.







\subsection{Culture}





\subsubsection{Clerical hierarchy}
\target{Iquinian clerical hierarchy}
The Iquinian church contained a core of true Vaimons who made up the higher tiers of the priesthood.

Then there was a larger number of assistant priests who could not invoke the \sephiroth on their own, but who assisted the Vaimons in prayer and magical rituals.
They were deacons, cantors, sextons, monks and nuns. 
These only knew simple orisons.





\subsubsection{Destroying information}
The Church is fond of destroying books and other materials dealing with the occult and other \quo{evil} things. They portray it as bad knowledge that people were not meant to know, and whose existence can only cause harm. But in reality, the Church is being manipulated by the Cabal, who wants \hr{Destroying information}{to keep people ignorant}.





\subsubsection{Funerals and preserving the dead}
\target{Iquinian funerals}
\target{mausoleum}
\index{mausoleum}
The Iquinians preserve the dead as mummies in great mausolea, or even \hs{pyramids}. 

Remember, \hr{Sephirah plan}{the purpose of Iquinianism is to harvest mortal souls}. 
The mausolea are designed to better keep the souls under control and reclaim as much soul-energy from the bodies as possible. 





\subsubsection{Laws}
\target{Iquinianism has many laws}
Iquinianism had many laws. 
In addition to the sixteen virtues of the \sephiroth there were official lists of sins and explicit rules of penance and punishment. 

This stood in contrast to the \Ortaican religion, which \hr{Ortaican religion has few laws}{had few laws}. 
This Iquinians \hr{Iquinian criticism of Ortaica}{used this to argue that \Ortaicanism was evil}. 

The real reason is this: 
Iquinianism was \hr{Iquin plan}{designed as a way of brainwashing the population}, and thus had to be mentally intrusive. 
\Ortaicanism was designed by the \taorthae as a political convenience, and so it was intentionally kept vague and flexible. 










\subsection{History}





\subsubsection{Exported as parallel religions}
\target{Iquinianism exported}
Seeing the success of the Iquinian religion in \Azmith, the Cabal since \quo{exported} the concept to other \hs{Shrouded Realms}. 
They constructed some similar religions, also worshipping \iquin{} in some form. 









\subsection{\Isphet the Destroyer}
\target{Isphet}
\target{Destroyer myth}
\index{\Isphet}
\Isphet was an evil figure in \Iquinian mythology. 
He was an evil god, an enemy of the \sephiroth. 
Sometimes described as a \qliphah. 
A nameless \qliphah of the Midnight Circle. 
Sometimes he was considered the king of all \qliphoth and the master of all that is evil. 
(A few doubted that he was a \qliphah at all.)

\Isphet was in continual battle with the \sephiroth.
In legendary times, he had waged a war against everyone and tried to destroy the world. 
He tore the world apart and killed millions upon millions. 
The \sephiroth and their \hr{Iquinian angels}{angels} had attacked him in force. 
They fought against the Adversary and his legions of \qliphoth. 

After \Isphet had destroyed the world, the \sephiroth had to banish him and rebuild it. 
The \sephiroth prevailed and cast out the Adversary. 
\Isphet since became the \hs{Exile}, feared and hated by all. 
A Satan type. 

From then the craven villain would hide in his dark pit of evil and only rarely dare to venture forth into the world of mortals. 
To this day, the myth said, the \sephiroth were locked in a cosmic battle with \Isphet. 
That was why the \sephiroth did not show themselves in the mortal world. 
And their believers \hr{Rituals against Isphet}{had to help them combat him}. 

Compare to Egyptian mythology, where there were spells to overthrow the \dragon Apep. 

See also the section about \hr{Myths about Dragons}{\dragons in art and mythology}. 

\citetitle{KingJamesBible}{%
  The Bible (Revelation 12:3--4,12:7--10)%
}{
  And there appeared another wonder in heaven; and behold a great red dragon, having seven heads and ten horns, and seven crowns upon his heads.\\
  And his tail drew the third part of the stars of heaven, and did cast them to the earth: and the dragon stood before the woman which was ready to be delivered, for to devour her child as soon as it was born.
  
  And there was war in heaven: Michael and his \hr{Iquinian angels}{angels} fought against the dragon; and the dragon fought and his angels, \\
  And prevailed not; neither was their place found any more in heaven. \\
  And the great dragon was cast out, that old serpent, called the Devil, and Satan, which deceiveth the whole world: he was cast out into the earth, and his angels were cast out with him.\\
  And I heard a loud voice saying in heaven, Now is come salvation, and strength, and the kingdom of our God, and the power of his Christ: for the accuser of our brethren is cast down, which accused them before our God day and night.
}

\lyricsbalsagoth{
  Enthroned in the Temple of the Serpent Kings
}{
  Scourge of Angsaar, wielder of the Black Sword,\\
  Immortal Lord of Darkmere, Serpent-Witch ensorcel me.
}





\subsubsection{Appearance}
\Isphet was depicted as a black \dragon. 
\Dragons symbolized evil in Iquinian mythology.

\Isphet was described as black with burning eyes and wreathed in fire and smoke. 
Sometimes he is said to have multiple heads and to breathe fire. 





\subsubsection{Names and titles}
\ps{\Isphet} titles included:
\begin{itemize}
  \item The Adversary.
  \item The Scourge.
  \item The Destroyer.
  \item Lord of Chaos.
  \item Lord of the Outer Darkness.
\end{itemize}

His name also appeared in the variant Iscraphet or Iscraphel. 

\quo{Isfet}, as far as I know, is an Egyptian word that means \quo{chaos} and is associated with the serpent Apep, the eternal enemy of Ra, the Sun god. 






\subsubsection{History}
The myth of \Isphet was made up somewhere in the Vaimon Age. 
It was unknown in Cordos Vaimon's time. 





\subsubsection{Rituals}
\target{Rituals against Isphet}
In churches they performed rituals at regular intervals that were meant to keep \Isphet at bay.
He was immortal and would not perish until the end of the world, but he could be wounded and weakened and mutilated.

People would gather in a church to attend \hr{Iquinian prayers}{prayer} and mutilate \Isphet. 
In one popular variant there would be an effigy of \Isphet in the form of a black serpent. 
Every church-goer was handed a needle or stick with which to impale the monster. 
At last, the effigy was hacked into pieces and burnt. 





\subsubsection{Truth}
\Isphet did not really exist. 
His myth was a twisted mash-up of the true stories of several \dragons, including: 
\begin{itemize}
  \item 
    \hr{Ishnaruchaefir}{\QuessanthIshnaruchaefir} and his role in the \hr{Shrouding}{\SecondShrouding}, where he \quo{sort of} destroyed the world.
    
    \Isphet's name and appearance was based on \Ishnaruchaefir. 
  \item 
    \hr{Secherdamon}{\IrocasSecherdamon} as the schemer who wanted to overthrow the \sephiroth. 
  \item 
    \hr{Tiamat}{\TyarithXserasshana} with her monstrous, multi-headed appearance. 
\end{itemize}









\subsection{Mythology}
\target{Vaimon mythology}
\target{Iquinian mythology}





\subsubsection{Angels}
\target{Iquinian angels}
Angels were beings of \iquin, lesser than the \sephiroth. 
They were represented as winged humanoids resembling \resphain. 

In the beginning (during the \caliphate), all angels looked like winged \humans. 
After \hr{Telcra integrates Scathae}{\ClanTelcra integrated the \scathae}, \scathaese angels were also sometimes depicted. 
Artists disagreed on whether these angels should have feathered wings or \hr{Pteran}{\pteran}-like wings. 
Feathered wings tended to win out, because \pteran wings were associated with \dragons, \hr{Iquinian myths about Dragons}{who were creatures of evil in Iquinian mythology}. 

The angels were said to live atop vast, beautiful towers that rose through the clouds and touched the sky. 
This was a twisted image of \hr{Nyx}{\Nyx}, where the true angels (\resphain) lived. 
It was a terrible shock for an Iquinian humanoid to come to \Nyx and see the hideous truth behind the myths about angels. 





\subsubsection{Creation}
\target{Iquinian creation myth}
Iquinian myth held that \iquin had always existed. 
The nature of \itzach was more ambiguous. 
Some interpretations said \itzach had always existed as the twin of \iquin.
Others said \itzach was an emanation of \iquin and existed at its mercy. 

In the beginning, the world was in balance, nicely delimited into a Dark half and a Light half. 
But then \itzach invaded \iquin, under the leadership of the dark god \hr{Isphet}{\Isphet}, the Lord of the Outer Darkness. 
The hordes of \itzach brought chaos. 

The essence of \iquin and the fetters of \itzach intertwined to create the material world. 
The \qliphoth of \itzach stole the essence that emanated from \iquin and forced it into unnatural shapes, thus creating the false illusion that the world was made of separate \quo{things} and \quo{individuals}.
Thus the Defiled world of \hr{Gehinnom}{\Gehinnom} was created. 

In order to keep the \qliphoth in check so they would not overrun the world, the \sephiroth created the \hr{Vaimon Abyss}{Abyss} as a moat around \Atziluth. 





\subsubsection{Dark age}
According to myth, \Iquin created the world long ago.
The \dragons and other forces of evil also existed, for good cannot exist without its opposite.

The first generation of mortals (pre-\humans) were sinful and displeased the One Light.
And the dark powers taught them forbidden skills and sorcery and knowledge which the One Light did not want them to have. 
So the \sephiroth abandoned them and allowed the \dragons and other evils to overthrow the mortals, overrun the world and rule it with terror for an Age.

\citetitle[p.88]{RHCharles:BookofEnoch}{The Book of Enoch LXV.6--10}{
  And a command hath gone forth from the presence of the Lord concerning those who dwell on the earth that their ruin is accomplished because they have learnt all the secrets of the angels, and all the violence of the Satans, and all their powers\dash the most secret ones\dash and all the power of those who practice sorcery, and the power of witchcraft, and the power of those who make molten images for the whole earth\ldots{}
  
  \ldots

  \quo{Because of their unreighteousness their judgement has been determined upon and shall not be withheld by Me for ever.
  Because of the sorceries which they have searched out and learn, the earth and those who dwell upon it shall be destroyed.}
}

After an Age of the World had passed, the \sephiroth took mercy on the world, and so they once again shown themselves.
They revealed themselves to Silqua and made her create the Vaimon order, so that \humans (having been punished enough) would now drive out the forces of Elder evil and rule the world again.

\citetitle[p.37--38]{RHCharles:BookofEnoch}{The Book of Enoch X.2--13}{
  [\ldots{}]
  and reveal to [Noah] that the end is appraoching; that the whole earth will be destroyed, and a deluge is about to come upon the whole earth, and it will destroy all that is on it.
  \\  
  And now instruct him that he may escape and his seed may be preserved for all the generations of the world.
  \\
  And again the Lord said to Raphael:
  Bind Az\^az\^el hand and foot, and cast him into the darkness; and make an opening in the desert, which is in D\^ud\^a\^el, and cast him therein,
  \\
  And place upon him rough and jagged rocks, and cover him with darkness, and let him abide there for ever, and cover his face that he may not see light.
  \\
  And in the day of the great judgement he shall be cast into the fire.
  \\
  And heal the earth which the angels have corrupted, and proclaim the healing of the earth, that they may heal the plague, and that all the children of men may not perish through all the secret things that the Watchers have disclosed and have taught their sons. 
  \\
  \ldots 
  \\
  In those days they shall be led off to the abyss of fire; and to the torment and the prison in which they shall be confined for ever.
}


Those Elder mortals were \humans in some versions of the story.
Others held that they were the \nephilim, thus justifying hate and persecution against the \nephil race.

This myth is based, to some limited extent, on the true story of how the \aryothim once ruled the world before the return of the \dragons.





\subsubsection{\Dragons}
\target{Iquinian myths about Dragons}
In \Iquinian mythology (and in the view of many \humanoids of the \Human Age and Scatha Age), \dragons were godlike Elder horrors who ruled \Miith with terror in the dark days of chaos before the \sephiroth took action and created the Vaimons to throw them out.
The \dragons were the spawn of chaos, the kin of loathsome alien gods and wielders of the blackest sorcery ever conceived.

\citebandsong{Nile:Ithyphallic}{Nile}{
  What Can Be Safely Written
}{
  On the walls of lost cities\\
  And in the carvings of madmen\\
  Who have glimpsed him in their dreams\\
  Is his image delineated\\
  Within a tomb protected by great seals he lies in death\\
  Under the weight of the dark waters of the deep\\
  Yet he dreams still, and in his dreams continues to rule this world\\
  For his thoughts master the wills of lesser creatures
}

See also the sections on \hr{Myths of vanquished monsters}{how \human heroes vanquished elder monsters} and on \hr{Myths about Dragons}{general myths about \dragons}. 





\subsubsection{\Humans vanquishing monsters}
\target{Myths of vanquished monsters}
There were myths about how great \human{} (sometimes \scathaese) heroes \hr{Cordos vanquishes monsters}{fought against and vanquished the evil pre-\human{} monsters that had dominated \Miith{} in the ages past}. 
Especially \hs{Cordos Vaimon} got this role: 
A Conan-esque hero, a frontiersman that ushered in the \quo{\hr{Human Age}{\Human{} Age}}. 

Stories tell how these brave, Iquinian, Light-fearing \human{} heroes overthrew and drove out the evil pre-\human{} Elder Races and monsters.
At last the wicked monsters were exterminated by the grace and power of the Light. 

Modern \humans were glad that these terrible, loathsome creatures no longer existed. 

Another source of these stories, buried deep down in \hr{Human racial memory}{\human{} racial memory}, is the vague recollection of \hr{Aryothim kill QJ}{the far older wars between the \aryothim{} and the \quiljaaran}. 

See also the section on \hr{Myths about Dragons}{myths about \dragons}. 

\citeauthorbook[p.35--36]{RobertEHoward:TheShadowKingdom}{Robert E. Howard}{%
  The Shadow Kingdom%
}{
  [Kull] stopped short, staring, for suddenly, like the silent swinging wide of a mystic door, misty, unfathomed reaches opened in the recesses of his consciousness and for an instant he seemed to gaze back through the vastnesses that spanned life and life; seeing through the vague and ghostly fogs dim shapes reliving dead centuries\dash men in combat with hideous monsters, vanquishing a planet of frightful terrors.
  Against a gray, ever-shifting background moved strange nightmare forms, fantasies of lunacy and fear; and man, the jest of the gods, the blind, wisdomless striver from dust to dust, following the long bloody trail of his destiny, knowing not why, bestial, blundering, like a great murderous child, yet feeling somewhere a spark of divine fire\ldots Kull drew a hand across his brow, shaken; these sudden glimpes into the abysses of memory always startled him.
  
  \ldots
  
  \ta{%
    Long and terrible was the way, lasting through the bloody centuries, since first the first men, risen from the mire of apedom, turned upon those who then ruled the world.
    
    And at last mankind conquered, so long ago that naught but dim legends come to use through the ages.
    The snake-people were the last to go, yet at last men conquered even them and drove them forth into the waste lands of the world, there to make with true snakes until some day, say the sages, the horrid breed shall vanish utterly.    
    Yet the Things returned in crafty guise as men grew soft and degenerate, forgetting anceint wars.
    Ah, that was a grim and secret war!
    Among the men of the Younger Earth stole the frightful monsters of the Elder Planet, safeguarded by their horrid wisdom and mysticisms, taking all forms and shapes, doing deeds of horror secretly.}
}

\target{Cordos began fighting Wylde}
Before Cordos, the world was a scary and monstrous place. 
Back then, all the world was \wylde. 
Cordos began the holy task of forcing back the \wylde and building up \human civilization in its place.
Therefore, it was the sacred duty of later Iquinians to carry on this grand work, for the sake of \humanity and the One Light. 

\citeauthorbook[p.57--58]{RobertEHoward:TheMirrorsofTuzunThune}{Robert E. Howard}{%
  The Mirrors of Tuzun Thune%
}{
  Gray fogs obscured the vision, grea billows of mist, ever heaving and changing like the ghost of a great river; through these fogs Kull caught swift fleeting visions of horror and srtageness; beasts and men moved there and shapes neither men nor beasts; great exotic blossoms glowed through the grayness; tall tropic trees towered high over reeking swamps, where reptilian monsters wallowed and bellowed; the sky was ghastly with flying dragons and the restless seas rocked and roared and beat endlessly along with muddy beaches.
  Man was not, yet man was the dream of the gods and strange were the nightmare forms that glided through the noisombe junlges.
  Battle and onslaught were there, and frightful love.
  Death was there, for Life and Death go hand in hand. 
  Across the slimy beaches of the world sounded the bellowing of the monsters, and incredible shapes loomed through the steaming curtain of the incessaint rain.
}





\subsubsection{Idolization of individuals}
\target{Cordos and Silqua in mythology}
\target{Silqua in Iquinian mythology}
In later Iquinian theology, Cordos and Silqua were highly exalted holy characters. 
They were the founders of modern mankind. 
Compare them to Adam and Eve from Judeo-Christian mythology.
Or Adam Kadmon from \Cabbalah.
Or Albion from William Blake's mythology.

Silqua was a saviour figure sent to give salvation to mankind.
But she was still fallible, and \hr{Iquinian myths about Silqua and sex}{sex was her weakness and downfall}. 

\citeauthorbook[\quo{First Thought in Three Forms}, p.86--100]{%
  BentleyLayton:TheGnosticScriptures%
}{%
  Bentley Layton%
}{%
  The Gnostic Scriptures%
}{
  But then for my part, I descended and got as far as chaos.\\
  And I dwelt with my own who were there, hidden within them, bestowing power and imparting image unto them.\\
  And down to the present [\ldots{}] those who [\ldots{}], i.e. the offspring of the light.\\
  It is I who am their parent. \\
  And I shall tell you a mystery that is ineffable and indescribable by any mouth.\\
  For you I loosed all the fetters and broke the bonds of the demons of Hades, bonds that were bound to my limbs and worked against them. \\
  And I threw down the high walls of the darkness,\\
  And I broke open the solid gates of the merciless and split their bolts.\\
  And the evil agency, who strikes you, who impedes you, the tyrant, the adversary, the ruler, the real enemy\dash as for all of these, I taught them about my own, the offspring of the light:\\
  So that they might become loosened from all these and rescued from all the fetters, and might enter the place where they had been in the beginning.\\
  It is I who am the first to have descended, for the sake of that part of me which remained, \\
  Namely, the spirit that exists within the soul and which has come to exist out of the water of lige and out of the baptism of the mysteries.\\
  I myself spoke with the rulers and with authorities, for I had descended deep into their language;\\
  And I uttered my mysteries to my own\dash a hidden mystery\dash\\
  And the fetters were loosened, as was eternal forgetfulness.\\
  And within them I bore fruit, namely, the thinking that concerns the unchangeable eternal realm (aeon) and my house and their parent.\\
  \ldots{} \\
  All who were in me shone bright.\\
  And for the ineffable lights within me I prepared an manner of appearance.\\
  Amen!
}

\target{Delphine in mythology}
Conversely, \hr{Delphine}{\Delphine} was reviled and seen as pure evil incarnated in \human form. 
Compare her to Lilith from Christian mythology, or the serpent in Paradise.





\subsubsection[Merkyrah]{\Merkyrah}
\target{Iquinian myths of Merkyrah}
The \Iquinian{} mythology went back to \Merkyrah. 
Here, it was a magical realm where mortals lived and walked side-by-side with gods and \hr{Iquinian angels}{angels}. 
And some unclear, conflicting, dark stories about something evil that crept in and poisoned their paradise. 

There was an \quo{original sin} of sorts. 
The Iquinians blamed themselves for the fall of \Merkyrah, in the same way that divorce children in RL sometimes blame themselves for their parents' divorce. 

A twisted version of the story of \hr{Thanatzil}{\Thanatzil} also featured. 
Allegedly, it was the sinful mankind's own fault that he they failed to be saved, even after his noble sacrifice. 

\citebandsong{DeathspellOmega:SiMonumentumRequiresCircumspice}{%
  Deathspell Omega
}{
  \Hetoimasia
}{
  For man [\nephil] is the key and man [\nephil] is the device\\
  And out of his ranks shall arise the saviour \\
  draped in the blood of the unborn
  
  For he will grow from child to man and extirpate\\
  souls in a devilish whirl from your cursed bosom.\\
  Fraught voices rise to the sky \\
  and beseech god to avert the incarnation\\
  But mankind was the prism to the quintessence of corruption
}





\subsubsection{\Qliphoth}
Some Iquinian Vaimons (especially those who accepted the \hr{Omnipotence of Iquin}{idea that \iquin was omnipotent}) believed that the \qliphoth had no life of their own.
They had to steal some essence from the One Light in order to live and exist. 
They were hollow, empty shells; undead things with the semblance of life but lacking any true essence. 

The \qliphoth were not pure evil.
They were animated by a core of stolen Light.
There was some good in them. 
Therefore it was OK to invoke the \qliphoth and use them to cast magic. 
(The \qliphoth were better than the heathen gods, which were of course completely forbidden.)

Some believed that the One Light, in its endless mercy and compassion, had allowed even these wretched half-beings to exist.
Not all \Telcras believed this, but they did, overall, believe in an \Iquin more forgiving than the \Iquin the Redcor believed in. 









\subsection{Philosophy}
\target{Iquinian theology}
\target{Iquinian philosophy}





\subsubsection{All are one}
The Iquinian church preached that \quo{we are all one}. 
All humanoids were part of \iquin, even though bound in fetters of \itzach.
They just had to realize it. 
Commoners could contact the rest of \iquin (and, thereby, other humanoids) through \hr{Iquinian prayers}{prayer} and mass. 
The Vaimons were wiser than commoners and had access to more knowledge. 
They learned how to break down the walls that separated them from the rest of the \iquin and thus draw upon the awesome power of \iquin to cast magic. 

In reality (the church said), all things and beings were one, made from the same indivisible One Light. 
All shapes and all individuality was a result of the \hr{Iquinian fetters}{fetters} that \itzach cast upon the world. 
The task of all living beings was to live in virtue and not sin.
Virtue would dissolve the fetters, but sin would forge new fetters. 





\subsubsection{Animal sacrifice}
\target{Iquinian animal sacrifice}
The \iquinians sacrificed animals \hr{Wylde totem sacrifice}{in order to keep their \wylde totems alive}. 
But unlike most religions they sacrificed only animals, not humanoids.
They supplied the rest of the energy via intense prayer.
Iquinians had a much more intense prayer schedule than most religions. 
They had to recite long prayers many times every day.





\subsubsection{\Atziluth}
\target{Atziluth}
\target{Kingdom of the One Light}
\Atziluth was the abstract \quo{place} where the \sephiroth dwelt.
And it was the place where believers hoped to go when they died.
They wanted to \quo{go into the One Light} and become one with \iquin. 

\Atziluth was also called the Divine Realm and the Kingdom of the One Light. 

\Atziluth could be reached through the \hr{Empyrean}{\empyrean}. 
It lay enclosed on all sides by the \empyrean and was, in a sense, a part of it. 
(Although the last part was a point of theological debate.)

A Vaimon could not move into the innermost circles of the \empyrean as long as he had a physical body.
The body fettered him to the lower, material world.
He could not become one with the \sephiroth, but he could touch them. 

\target{The Beyond horrible to Iquinians}
Iquinians believed that the true world, \Atziluth, which lay beyond \Gehinnom, was bright and pure and good and beautiful. 
Therefore it came as such a dreadful shock when they glanced into the Beyond or otherwise learned that there existed true worlds Beyond that were far worse than \Gehinnom. 
Worse even than \itzach.
The traditional image of \itzach was an \quo{Outer Darkness}; distant and abstract and unreal. 
The Beyond, once you got to know it, was very close, full of horrors awfully physical and slimy and smelly and slavering and \emph{real}. 





\subsubsection{Canon controversy}
The \VaimonClans had each their prophets and founding fathers who claimed revelations from the \sephiroth and laid down laws and commandments and spiritual \quo{wisdom}.
Some of these revelations were genuine.
Others were induced by \qliphoth or other immortals.
The \vclans did not agree on which prophetic works and scriptures were canon, so there was plenty of division in the Vaimon religion already in the time of the empire.

The immortals could not easily unite the empire.
First of all, there was the Unspoken Covenant to consider.
Second, the Sentinels and Cabal were always fighting and trying to fuck each other's plots up.
Third, each faction had plenty of infighting and disagreements, so how could they be expected to prevent the same thing from happening in the Shrouded Realms?




\subsubsection{Eschaton}
\target{Iquinian eschatology}
\target{Vaimon eschatology}
Iquinian eschatology held that there would come a Doom's Day, an Apocalypse, an Eschaton.
On this day, the \sephiroth would descend to \Miith and all fetters would be broken. 
All those souls that were found to be virtuous would be taken in and become one with the One Light.
All those sould that were found to be sinners would be cast down and bound in \itzach, where they would dwell in pain and chaos and suffering forever more. 

This was sort of true. 
It was planned that one day \hr{Lithrim}{\Lithrim} would come. 

In fact, I should have more myths and prophecies about the Advent of \Lithrim. 
Perhaps the Eschaton should be named \quo{the Advent}.
 
Some Vaimons believed the Advent was not a future event but the end goal for each individual's personal development. 

\hr{Lithrim was secret}{The truth about \Lithrim was a secret}, unknown even to most Cabalists. 
Most just thought it was a metaphor or abstraction. 

Compare to the \hr{Ortaican eschatology}{\Ortaican eschatology}. 
See also the general section about the \hs{Eschaton}. 





\subsubsection{Fetters}
\target{Iquinian fetters}
Iquinians believed that humanoids had \quo{fetters} of darkness that bound them to \itzach. 
These fetters were made of sin. 
All humanoids were born with fetters, for they were imperfect and flawed beings in a flawed material world. 
All every time they sinned, they forged more fetters. 
An Iquinian strove to break the fetters of darkness through \hr{Iquinian prayers}{prayer} and by following the virtues of the \sephiroth. 
Many people chose one \sephirah or a few, and then strove to emulate them and submit completely to the virtues they stood for. 

\target{Iquinian mercy}
When a person broke all his fetters, he would be free of \itzach and able to transcend into the One Light. 
But this was only possible for the great Vaimon saints like Silqua and Cordos. 
It was held to be impossible for regular people to break all their fetters. 
They were flawed creatures, after all.
But when a person died with any fetters left, he could still appeal to the mercy of the \sephiroth, and they might help him and bless him and free him of his last fetters so he could at last be one with the One Light. 
This would only happen to those who were truly faithful, though. 
The souls deemed unworthy would be cast out into the Outer Darkness to suffer forever in the chaos of \itzach.






\subsubsection{\Gehinnom}
\target{Gehinnom}
The physical world was held to be a consequence of \itzach. 
\Iquin represented unity, but \itzach represented diversity. 
\Iquin was the essence of all things, but it was \itzach that gave those things shape and thus made them into \quo{things}.

The physical world was Defiled because it was \hr{Iquinian creation myth}{created by the mingling of \iquin and \itzach}. 
This Defiled world was called \Gehinnom. 
All living creatures were part of \Gehinnom and were thus also themselves Defiled; impure, unworthy, lowly sinners. 

The Iquinians believed they had a sacred duty to purify and unmake the Defiled world of \Gehinnom. 
This work was called \hr{Tikkun}{\tikkun}.





\subsubsection{Numerology}
\target{Vaimon numerology}
The Vaimon alphabet had sixteen consonants.
Every \sephirah's name began with a different consonant. 
Thus the letter corresponded to the \sephirah and its virtue. 
Each consonant also had a number, the same as the number of the month of the corresponding \sephirah.
The consonants are, in order: 
\begin{enumerate}
  \item Shil
  \item Fir\footnote{Fir is variously romanized as F or Ph for aesthetic effect.}
  \item Ker\footnote{Ker is variously romanized as C, K or Q for aesthetic effect.}
  \item Raith
  
  \item Bal
  \item Chaid
  \item Sum
  \item Ten
  
  \item Zod
  \item Tzad
  \item Thul
  \item Mor
  
  \item Vod
  \item Luth
  \item Nuz
  \item Yith
\end{enumerate}


Other consonant sounds were spelled as \quo{voiced} versions of the base consonants, or as combinations of letters:
\begin{itemize}
  \item Voiced Tzad became J
  \item Voiced Chaid became H
  \item Voiced Ker became G
  \item Qu was spelled as Ker and the vowel U.
\end{itemize}


Each vowel also needs to have some meaning\ldots{}

Every person's name thus had a theological meaning. 





\subsubsection{Omnipotence of \Iquin}
\target{Omnipotence of Iquin}
Vaimon theologians disagreed over whether or not \iquin was all-powerful. 

Some Vaimons believed that \Iquin was all-powerful and could vanquish \itzach any day. 
\Iquin refrained from doing so because it was disillusioned and disappointed by the sin and wickedness of living creatures, who gave themselves over to \itzach when they should worship \iquin alone. 
Therefore \iquin allowed the Defiled world of \Gehinnom to exist.
People had to be good and really deserve it before \iquin would grant them salvation.
Those Vaimons who believed in omnipotence had a slightly more laissez-faire morality.
After all, in the end everything would be all right. 
The One Light would save the world. 
Some of these even believed that \iquin was all-knowing and that everything was predestined. 

Some Vaimons believed that \Iquin was \emph{not} all-powerful, that \iquin and \itzach were equal foes. 
These endorsed a stricter morality.
If \iquin was not all-powerful, then \itzach was a real menace and might even one day win.
Then it was every mortal's duty to do his utmost to fight evil in all its forms. 
The non-omnipotentialists saw the omnipotentialists as a threat because they embraced false complacency. 
The non-omnipotentialists all denied the idea of predestination. 

The issues of omnipotence and predestination were perhaps the two greatest points of contention in \iquinian theology.
Wars had been fought between \vclans (and civil wars fought within \vclans) over such religious differences.

The issue of omnipotence was also a point of contention among the \resphain who masterminded the Iquinian church.
Some of them believed the dogma of omnipotence was a great propaganda trick.
Others believed it was too outrageous a lie, and that the followers would not keep buying it. 

\ClanTelcra \hr{Telcras believe in omnipotence}{believed in omnipotence}. 
\ClanRedcor \hr{Redcor do not believe in omnipotence}{did not}. 
This was one of the reasons why they disliked one another, and one reason why \hr{Telcra is more popular than Redcor}{\ClanTelcra gained more popularity with the people}.

\paragraph{To Do:}
  Place each \vclan along these lines!
  What do the Redcor, Geicans and \Telcras believe?
  Remember that each \vclan may have changed its mind over the course of history.





\subsubsection{Prayers}
\target{Iquinian prayers}
Iquinian prayers were seen as a simple form of magic. 
Praying really changed the world. 
It brought the Divine Realm and the spiritual union closer. 

In prayer, even commoners could experience and feel the \hr{Shechinah}{\shechinah}. 





\subsubsection{Sex}
\target{Iquinian prostitution metaphor}
The \hr{Iquinian creation myth}{mingling of \iquin and \itzach} that created \hr{Gehinnom}{\Gehinnom} was a sexual thing. 
The innocent, virginal core of Light was raped and violated (in some versions, seduced) by the wicked, lustful Outer Darkness.
The Light became fragmented. 
Little specks of Light succumbed to seduction and became whores of the darkness.
These cosmic sluts made up the physical world and all living beings. 

Thus, \trope{SexIsEvil}{Sex Is Evil}. 

Every time a mortal sinned, he prostituted himself to Darkness for mere material gain. 

\target{Iquinian myths about Silqua and sex}
In fact, sex was so seductive and evil that even the noble Silqua was vulnerable to it. 
This was how the evil \hr{Delphine}{\Delphine} was able to seduce, torture and kill her. 
Even though \hr{Silqua in Iquinian mythology}{Silqua was \uber} and the divinity of the One Light was manifest in her, a part of her was still just a weak, foolish, sinful, horny woman that could be brought down by her sexuality. 
Even divinity could not overpower feminine frailty. 
(Notice the sexism?)

This proved how insidious sex was, and how much it should be feared and hated. 

\target{Redcor myths about Silqua and sex}
The Redcor did not accept this idea. 
\hr{Redcor feminism}{They were feminists}.
They believed that Silqua remained sexually virtuous till the very end. 
She was not seduced by \Delphine. 
She was forcibly kidnapped. 





\subsubsection{\Tikkun}
\target{Tikkun}
The Iquinians believed that the diverse physical world was an evil.
It was a false world, unlike the true world of the One Light (\hr{Atziluth}{\Atziluth}). 
The duty of Iquinians was to purify \Gehinnom and return it to its state of purity and oneness.
This holy duty to cleanse and thus unmake \Gehinnom was called \tikkun.
\begin{itemize}
  \item 
    They sought to purify their own souls through righteous living and thus bring themselves back to \iquin.
  \item 
    They sought to spread the true faith so that other souls would be freed from \Gehinnom and return to \iquin.
  \item 
    They killed heathens so that their sould would return to \itzach where they belonged, and the heathens would not be able to spawn descendants nor spread their evil beliefs, which would otherwise condemn more souls and more spritual matter to maintain \Gehinnom. 
  \item 
    They fought evil religions to prevent their evil gods from keeping Defiled souls bound in \Gehinnom. 
\end{itemize}

Animals were also Defiled, but they were not intelliget enough to be able to save themselves, so humanoids must do it for them.
By butchering an animal with the correct spells and prayers and then eating its flesh, its essence would be freed from \Gehinnom and returned to \iquin. 
So slaughtering and eating animals was a sacred, religious act, transforming Defiled matter into pure spiritual oneness. 
The eater would become \quo{one} with the animal, thus bringing the world a bit closer to complete oneness.

People would also say prayers before eating and drinking anything, even if not butchering it.
(Add this when Rian drinks with Dennick.)

It was part of the sacred duty of \tikkun to fight the \hr{Wild}{\wylde}.
The \wylde was a manifestation of Defilement.
It sought to encroach on the habitations of humanoids and create even more defilement.
It must be held back. 
The Iquinians also believed that taking materials (such as wood and metal) from nature and using them to build things (such as houses, tools and weapons) was a holy thing. 
The process took materials of pure Defilement from the \wylde and shaped them into things with a purpose, thus bringing them and the world closer to \iquin. 

\Tikkun was a continuation of the work \hr{Cordos began fighting Wylde}{begun by Cordos Vaimon} when he began \quo{conquering} the world from the evil monsters.

Similarly, every living person had the sixteen \sephiroth manifest inside him, but they were enclosed by \qliphoth. 
Only through prayer and forgiveness and living out the virtues could the \qliphoth be broken so that the person could become one with the \sephiroth. 





\subsubsection{Vision of the One Light}
The One Light had kept itself hidden since the Defilement. 
Only select holy individuals had been allowed to see with the vision of the One Light.
These included Silqua and Cordos. 
















\section{Redcor}
\target{Redcor}
\target{Redce}
\index{Redcor}
\index{Redce}
The Redcor are a \VaimonClan. 
Their homeland is \Redce, a theocratic nation in northeastern \Velcad{} ruled by \ClanRedcor.

\target{Conclave}
\index{Conclave}
The entire organization of \Redce{} and \ClanRedcor is based on the Iquinian religion. 
The Redcor are a matriarchal and matrilinear people, and their kingdom is ruled by a council of women, called the {Conclave}. The Conclave gather around the Topaz Throne in the capital city of \Redce{}. 

\index{\Redcean}
Note that the Redcor are the ruling class of \Redce{}. A common citizen of \Redce{} is a \Redcean{} (plural \emph{\Redcean{}s}). A believer of the Iquinian religion is called an \emph{Iquinian}. 









\subsection{Aesthetics}
The traditional \colour of \ClanRedcor is yellow. 
The symbol of \Redce{}, \ClanRedcor and the Iquinian Church is a yellow Sun on a blue background. 









\subsection{Culture}






\subsubsection{Clerical ranks}
The Clerics are scholars and priests and the spiritual and political leaders of the Redcor. 
The lowest ranked clerics are monks, bearing the title \emph{\frater} (plural \emph{\fratres}) if male or \emph{\soror} (plural \emph{\sorores}) if female. 
Male Clerics cannot rise above the rank of Frater. 
For female Clerics only, the higher ranks are \mater, \matron and \matriarch. 





\subsubsection{Government}
The Redcor are ruled by the Conclave, which consists of all \matriarchs{} (who are all equal in status). 





\subsubsection{Language}
The official language of \Redce is the \hr{Redcor language}{Redcor dialect} of the \hs{Vaimon language}. 
\hr{Velcadian language}{\Velcadian} is also spoken. 





\subsubsection{Social classes}
The Redcor Vaimons are divided into \clerics{} and \templars{}. 

Apprentices training to become Vaimons are called \neophytes.\index{\neophyte}





\subsubsection{\Templars: \Ryzin{} and Gandierre}
\index{Redcor!\templar}
\index{\templar!Redcor}
The \templars{} are warrior mages and the defenders and holy knights of the Redcor faith. 
They are split into two orders: 
The {\Ryzin}, for women, and the {Gandierre}, for men. 
%Many such knights are Redcor, but not most. 

Traditionally, the \Ryzin{} (unlike other \Redcean{} women) cut their hair short and wear trousers rather than skirts. 









\subsection{Geography}





\subsubsection{Climate}
\target{Redce climate}
\Redce{} is quite cold. 
There is snow much of the year. 
The northern edge of the country is tundra-like. 
There are some local Inuit-like peoples, not all of which are fully Iquinianized. 





\subsubsection{Demographics}
\target{Redcean demographics}
\Redce had a more \human population than most nations, but there were still plenty of \scathae.
The \scathae typically worked as labourers, farmers or soldiers. 
\ClanRedcor itself was strictly \human-only. 





\subsubsection{Rubellah, place of the first invocation}
\target{Redcor control Rubellah}
Clan Redcor controlled the rock of \hs{Rubellah}, a sacred site which was believed to be the {place where Silqua first invoked the \sephiroth}.
This might or might not be true.

The place attracted many Iquinian pilgrims. 

The Redcor very much wanted to control Castle \hs{Yeshimon} (the {place where Silqua died}), but this place was \hr{Zether control Yeshimon}{controlled by \ClanZether}.





\subsubsection{\TopazChateau}
\target{Topaz Chateau}
\index{\TopazChateau}
The central stronghold of the Redcor. 

The capital city is that surrounding the Topaz \Chateau. This city is very much divided into rich and poor quarters. Like \hr{Malcur rich and poor}{\Malcur}, but more extreme. The \Chateau{} is extremely tall, and the rich Redcor live infinitely far removed from the hardships, poverty and suffering of the common folk. 

The \Chateau itself had crystalline walls and looked as if it really was build of topaz. 
It was a beautiful and mysterious city of sorceresses. 
Compare to Kor-Avul-Thaa from the song \bandsong{Bal-Sagoth}{As the Vortex Illumines the Crystalline Walls of Kor-Avul-Thaa}. 

\target{Bane ruin under Chateau}
Unbeknownst to most, the \TopazChateau was built on top of an ancient \bane ruin, possibly a crashed space ship. 

\Belzir \hr{Belzir imprisoned under Chateau}{was imprisoned in these ruins}.









\subsection{History}
The founder of \ClanRedcor was Rebecca Redcor, daughter of \hs{Cordos Vaimon} and \hr{Silqua}{Silqua \Delaen}. 





\subsubsection{Fought against the \banes}
\target{Redcor fight Banes}
The Redcor were not complete tools and pawns. 
They were controlled by the Cabal to some limited extent and they did use \iquin, but the Cabalists did not have complete control over them.
The Redcor were also pulled in other directions by the Sentinels and \Kezerad, and some of them were intelligent enough to think on their own. 

The Redcor were one of the few mortal agencies on \Azmith that knew that the \banes existed. 
They knew the \banes were evil and fought to oppose and contain them. 

This explains why the Redcor were so dogmatic and secretive:

\begin{enumerate}
  \item 
    The Redcor leaders knew (or at least suspected) horrible secrets about the dark nature of the world and \humankind and the \sephiroth and \qliphoth. 
    These secrets must be kept secret at all costs.
    Hence they cultivated a dogmatic tradition where inferiors were discouraged from questioning their superiors.
  \item 
    Those who knew secrets forced themselves into a strict, dogmatic mindset in order to retain their sanity and not fall to chaos and madness and moral relativism.
  \item 
    The Cabal had introduced false dogmata and false beliefs in an attempt to mislead and manipulate the Redcor and prevent them from working against the Cabal agenda. 
\end{enumerate}





\subsubsection{\Belzir imprisoned}
\target{Belzir goes into Chateau}
\target{Belzir goes below Chateau}
\target{Belzir imprisoned under Chateau}
In the final days of the \VaimonCaliphate, \Belzir found the ruin and attacked the \Chateau. 
She went into the dungeons and explored the ruin, trying to learn its secrets and gain its power, but the Redcor tricked her and captured, killed and imprisoned her in the cellars below their \Chateau. 

The Redcor leaders knew that \Belzir was a Scion, but they kept this knowledge secret and hidden. 
It was a horrible thing to know, for their dogma said that Scions were things of good, associated with \iquin. 
And \Belzir was a hideous creature. 
She had never been truly \human. 
She was a hideous dark thing of \itzach that had once taken \human form, pre-\human horror sleeping beneath their city. 
So they believed.
And they were not completely wrong, for she was a \sathariah and a \malach, and well on her way to \hr{Azraid turns Malachim into Neo}{turning into a \neoresphan}. 





\subsubsection{\Belzir awakening}
\Belzir \hr{Belzir keeps in touch with Royalists}{kept in touch with her Royalists}.

Near the time of the \thirdbanewar, \hr{Belzir awakening}{\Belzir had gained power}. 
She was now able to reach out to the people living above her and contact them in their dreams. 

This was why \hr{Redcor need Carzain}{the Redcor needed Carzain}. 









\subsection{Philosophy}
\target{Redcor philosophy}






\subsubsection{Stern and ascetic}
\target{Redcor sternness}
The Redcor (\hr{Redcor do not believe in omnipotence}{who did not believe} in the \hr{Omnipotence of Iquin}{omnipotence of \Iquin}) preached doom and danger and asceticism, which was not popular. 
Their \iquin was stern and embattled. 

This philosophy was realistic in a sense, but \hr{Telcra is more popular than Redcor}{not as popular} as \hr{Telcra forgiveness}{the more lenient \Telcra beliefs}. 





\subsubsection{Feminism}
\target{Redcor feminism}
The Redcor were feminists.
They had their own myths about Silqua, where she was stronger and \hr{Redcor myths about Silqua and sex}{not susceptible to seduction}. 





\subsubsection{Free-thinking and heresy}
The Redcor believe that philosophy should be moral, clean, pure, structured, regulated, dogmatic and wholesome. Philosophy is a means of coming to a better understanding of the truths that are already revealed and known. 

Speculation into the dark, the unknown, the frightening is forbidden and considered heresy, a direct way to corruption and damnation. Evil thoughts are the way to evil deeds.





\subsubsection{\Iquin and \Itzach}
\target{Redcor do not believe in omnipotence}
The Redcor, \hr{Telcras believe in omnipotence}{unlike the \Telcras}, did not believe the \hr{Omnipotence of Iquin}{omnipotence of \Iquin}. 
They were more paranoid and ascetic than the \Telcras. 
They feared \itzach and never invoked it. 

This was one of the reasons why they disliked one another, and one reason why \hr{Telcra is more popular than Redcor}{\ClanTelcra gained more popularity with the people}.





\subsubsection{Non-rulership policy}
\target{Redcor do not rule}
The Redcor have as a policy that they do not try to directly rule other countries. 
To outward appearances they remain aloof from politics, but in reality they manipulate people like no tomorrow and wield great power behind the throne many places. 

They have been tempted to break this rule occasionally, but \sephirah{} brainwashing makes them obey it. 





\subsubsection{Prayer}
The Redcor believed in constant prayer and scripture study in an effort to strengthen the One Light and thus combat the Outer Darkness. 

\citeauthorbook[\quo{Angels Punishing the Wicked}, p.70]{%
  AlanUnterman:TheKabbalisticTradition%
}{%
  Alan Unterman%
}{%
  The Kabbalistic Tradition%
}{
  And know that through finding new meaning in the Torah spiritual forces are created from the letters of the Torah.
  These forces are literally \hr{Iquinian angels}{angels}.
  They receive power from Edom which enables them to punish the wicked with the sword and death 
  \ldots
  These spiritual powers, namely the angels, come about because of the renewal of the Torah through new interpretations.
  The renewal of the Torah relates to the holiness which is added above.
  According to this increase in Torah so there is an increase in the number of angels.
  The contrary is also true.
  Sometimes the holiness is so little that the angels that are created through the renewal of the Torah have diminished power.
  They do not have the ability to receive the power to punishe the wicked with the sword and death.
  They only have the power to suppress the wicked and to bring fear into their hearts, but not to punish them with the sword and to remove them.
}





\subsubsection{Technology taboo}
The Redcor were the strongest supporters of the \hs{Vaimon technology taboo}. 











\subsection{Politics}





\subsubsection{\Banes}
The Redcor \hr{Redcor fight Banes}{knew that the \banes existed and fought them}. 






\subsubsection{\ClanTelcra}
\target{Redcor hate Telcra}
The Redcor kind of hate the \Telcras. 
They look down on the junior \vclan as unruly children, upstarts unworthy of the Vaimon name. 















\section{\Telcra}
\target{Telcra}
\index{\Telcra}
\Telcra{} is a \VaimonClan. 
It is the youngest \vclan, the only one founded after the \darkfall. 
It is \Iquinian{} (in name at least) and makes up one of the two branches of the \hs{Iquinian Church} (the other being the Redcor). 









\subsection{Culture}






\subsubsection{\Clerics{} and \templars}
\index{\cleric!\Telcra}
\index{\templar!\Telcra}
\index{\Telcra!\cleric}
\index{\Telcra!\templar}
Like the Redcor, \ClanTelcra{} is split into \clerics{} and \templars. 
But there is a relatively large gap between the two. 
This separation was there already from the beginning, because the \baccons{} didn't want to church to become too powerful. 
The \baccons{} wanted to keep their potential competitors scattered, so they could control them. 
So they made sure the \Telcra{} \clerics{} were relatively peaceful and the \templars{} relatively secular. 
And the fracturing of the \vclan has just made it worse. 

The \clerics{} run and maintain the \Telcra{} Church and tend to be very \Iquinian. 
The \templars{} can be overtly anti-\Iquinian{} and use \itzach{} right and left. 
The \clerics{} disapprove of this. 
Many \templars{} are more-or-less mercenaries, working in the highest-bidding \ishrah. 





\subsubsection{Demographics}
\target{Telcra demographics}
\ClanTelcra{} was the only \vclan to have an almost fifty-fifty split between \human{} and \scatha{} members. 
In terms of species, \Telcra{} was by far the most egalitarian of all \VaimonClans. 









\subsection{Geography}





\subsubsection{Castle Yeshimon}
\target{Zether control Yeshimon}
\ClanZether controlled Caste \hs{Yeshimon}, a holy site that was believed to be the \hr{Silqua died at Yeshimon}{place where Silqua died}. 
The place attracted many Iquinian pilgrims. 









\subsection{History}
\target{Telcra founded}
The \vclan was founded during the time of \hr{Tepharae}{\Tepharae}. 
Originally, the \bacconate{} had close ties to the Redcor. 
But the Redcor religion was frightening, and the common folk were not happy about accepting it. 
They wanted to make themselves less dependent on the Redcor. 

The Cabal had the great idea to create a new \Iquinian{} Church that would be less magical, less mystical and more down-to-earth. 
An \quo{Iquinianism light}. 
They also needed some new Vaimons that were not controlled by the Redcor. 
So they founded a new \VaimonClan, the \Telcra. 

And it worked great. 
The people adopted it.

But the Cabalists were being deceived by Sentinels (who had, of course, infiltrated the order). 
The \Telcra{} Church is a much weaker grip on the populace than the Redcor Church would have been. 

In the beginning the \vclan was centrally organized. 
But after the dissolution of \Tepharae{} the \vclan lost its central organization and became scattered and fractioned. 
In the days of the \bacconate, all major imperial lords kept a small but very professional \ishrah{} of \Telcras. 
After the fall of the \bacconate, the \vclan fell apart and most were killed. 
(Why were they killed? And by whom?) 

Now there are only a few, scattered \Telcras{} left, and few lords can muster a decent \ishrah. \Malcur only has a makeshift \ishrah{} of one or two \Telcras, some chaos sorcerers and hedge wizards, and Carzain \Shireyo. 





\subsubsection{Rose to power}
\Tepharae, together with \hr{Telcra}{\ClanTelcra}, \hr{Tepharae succeeds Ortaica}{rose to prominence after the fall of \Ortaica}.





\subsubsection{Passing of the \Human Age}
In terms of species, \Telcra{} was \hr{Telcra demographics}{the most egalitarian of all \VaimonClans}. 

The founding of a \VaimonClan full of \scathae{} was a symptom/consequence of the passing of the \quo{\hr{Human Age}{\Human{} Age}}, and also one of the causes it. 

\target{Telcra integrates Scathae}
One of the great successes of \ClanTelcra was integrating the \scathae. 
They preached that the \hr{Vaimon Caliphate oppressed Scathae}{\human supremacist Vaimons of the past} had been unjust oppressors and heretics, and that in truth, \scathae were the equals of \humans. 
They demonized previous Vaimons and rejected the validity of the \caliphate (at least at first).
The Redcor were not happy about that. 




\subsubsection{More popular than the Redcor}
\target{Telcra is more popular than Redcor}
The \Telcras had more popular appeal than the Redcor.
The \Telcras (\hr{Telcras believe in omnipotence}{who believed} in the \hr{Omnipotence of Iquin}{omnipotence of \Iquin}) preached that the One Light would triumph and everything would be good.
Their \iquin was \hr{Telcra forgiveness}{gentle and forgiving and peaceful}. 

The Redcor (\hr{Redcor do not believe in omnipotence}{who did not believe} in the \hr{Omnipotence of Iquin}{omnipotence of \Iquin}) preached doom and danger and asceticism, which was not popular. 
Their \iquin was \hr{Redcor sternness}{stern and embattled}. 

The Redcor were more realistic, but the masses chose hope over realism. 









\subsection{Philosophy}






\subsubsection{Forgiving}
\target{Telcra forgiveness}
\Telcra philosophy was fairly benevolent and forgiving. 

This philosophy was unrealistic in a sense, but \hr{Telcra is more popular than Redcor}{more popular} than the \hr{Redcor sternness}{stern Redcor beliefs}. 






\subsubsection{\Iquin{} and \Itzach}
\target{Telcras believe in omnipotence}
\ClanTelcra believed in the \hr{Omnipotence of Iquin}{omnipotence of \Iquin} (\hr{Redcor do not believe in omnipotence}{unlike the Redcor}. 
They believed that \itzach was no real threat and would always be overcome by \iquin.
They spoke highly of the value and power of pure faith. 
Therefore, they felt it was safe to invoke \itzach. 
According to their scriptures, \itzach existed at the mercy of \iquin as its slave. 
It was thus also given to the Vaimons, bound by the supremacy of \iquin to do their bidding. 
Channelling \itzach was not without risks, of course, but sometimes it was needed, and times of need you had to do things that would otherwise be sinful. 
Killing, for example, was normally a sin, but if you were fighting a just war, killing was acceptable if not required. 

This was one of the reasons why the Redcor and \Telcras disliked one another, and one reason why \hr{Telcra is more popular than Redcor}{\ClanTelcra gained more popularity with the people}.




\subsubsection{Non-rulership policy}
\hr{Redcor do not rule}{Like the Redcor}, \ClanTelcra{} has a rule that forbids them to try to rule other countries directly.
Still, under the \hr{Tepharae}{\Tepharin{} \Bacconate} they held quite a lot of political power. 
But after that \bacconate{} crumbled, so did the central organization of \ClanTelcra. 

The Redcor dislike the \pps{\Telcras} power and tried to antagonize them. 
Since the collapse of the \Telcra, the well-organized Redcor have been able to keep the scattered \Telcras{} from gaining too much power. 
The two \vclans \hr{Redcor hate Telcra}{kind of hate each other}. 









\subsection{Politics}
The Redcor resented them and saw them as rogues, not a true \vclan. 















\section{Vaimons}
\target{Vaimon}
\target{Vaimons}
\index{Vaimon}
The Vaimons are an order of \human{} mages, founded somewhere around the year 1 \IC{} by \hr{Silqua}{Silqua \Delaen} and \hs{Cordos Vaimon} (the first \VaimonCaliph), after whom it was named. Vaimon magic is based on the twin forces of \iquin{} and \nieur, and spells are cast by invoking the various \Archons{}. 

The Vaimons previously ruled a \hr{Vaimon Caliphate}{\VaimonCaliphate}, but it fell in \yic{Darkfall} in what is called the \quo{\Darkfall}. 

\also{Cordos Vaimon, Silqua, Redcor, Geican, Quaerin}









\subsection{Aesthetics}
Each \VaimonClan has a traditional \colour. There is no red \vclan, because red was the traditional \colour of the Vaimons' enemies back at the time when the \vclans were founded. 





\subsubsection{Middle-Eastern theme}
\target{Vaimon Middle-East}
The \VaimonCaliphate should have a Middle-Eastern theme, with domes and minarets and stuff. 









\subsection{\VClans}
\target{Vaimon clan}

The Vaimons are divided into a number of autonomous \quo{\vclans}. 
Originally there were six \vclans: 

\begin{description}
  \item[{\Zether}:] 
    Founded by \Zether Vaimon, son of Cordos and Silqua and the second \VaimonCaliph.
  \item[\hs{Redcor}:] 
    Founded by Rebecca Redcor, daughter of Cordos and Silqua.
  \item[{\Delaen}:] 
    Descended from Silqua's brothers, Arcan and Lestor.
  \item[{\Irgel}:] 
    Descended from Kerzah \Irgel, Cordos and Silqua's younger son.
  \item[\hs{Geican}:] 
    Founded by Tiraad Geican, son of Cordos and one of his other wives.
  \item[{Quaerin}:] 
    Founded by Norcah Quaerin, son of Cordos by a third wife.
\end{description}

Perhaps there was also a clad Ephrad.





\subsubsection{\Belzir's time}
By \Belzir's time, \hr{Vaimon Clans at Belzir's time}{there were only four clans left}.





\subsubsection{\Thirdbanewar}
By the time of the \thirdbanewar, only the \vclans Redcor and Geican survived, alongside a newcomer \vclan:

\begin{description}
  \item[\hr{Telcra}{\Telcra}:] 
    The only \vclan to have been founded after the
 \hr{Hundred Scourges}{\darkfall}. 
\end{description}

The \vclans \Delaen, Quaerin and \Zether had died out long before the \thirdbanewar. 

There existed also \quo{rogue} Vaimons, owing allegiance to no \vclan.









\subsection{Culture}





\subsubsection{Demographics}
Most of the \VaimonClans were \human{}-dominated. 
Some, like \ClanRedcor, were purely \human. 
Others, like \ClanGeican, admitted \scathaese{} members. 

\ClanTelcra{} was the only \vclan to have an almost fifty-fifty split between \human{} and \scatha{} members. 





\subsubsection{Domestic animals}
See also the section on \hr{Domestic animals}{domestic animals}. 





\subsubsection{Language}
\index{Vaimon!language}
\index{Archaic Vaimon (language)}
\index{Ancient Vaimon (language)}
\index{Modern Vaimon (language)}
\index{Redcor!language}
The Vaimon tongue is descended from that spoken in \hr{Imrath}{\Imrath}. 
It was once the official language of the \hr{Vaimon Caliphate}{\VaimonCaliphate} and is still spoken in \Redce{} and used by the Redcor. 
Previously used as an intercultural \emph{lingua franca}, it has been replaced by \Velcadian{} in recent centuries. 

The term \quo{Archaic Vaimon} or \quo{Ancient Vaimon} is used to describe older forms of the language, as spoken in the \VaimonCaliphate. 
This is contrasted to \quo{Modern Vaimon}, as spoken in \Redce. 
Modern Vaimon is sometimes called \quo{Redcor Vaimon} or simply the Redcor language, but the Redcor frown upon this terminology, insisting that their tongue is the true Vaimon tongue. 

The Redcor dialect is meant to resemble French, but Archaic Vaimon is meant to sound like Hebrew. 





\subsubsection{Nudity}
\target{Vaimon modesty}
The \vclans had very different ideals of modesty. 
\ClanZether was one of the least taboo-ridden. 
They had a proud tradition of making all their statues naked. 

The Redcor were much more prudish.
Their statues were always clothed.

The \Telcras inherited the Redcor prudishness.





\subsubsection{Prayers}
\target{prayers against disease}
The Vaimons have prayers against disease. 
This is necessary, because the \hr{Parasitic Archons}{\Archons{} are parasitic} and spread disease. 
In war, they need these prayers to ward off disease to prevent their armies from dying from it. 
(This causes the disease to hit random civilians instead, but helps keep the army clean.) 





\subsubsection{\Templars}
\index{\templar}
\target{Templar}
A \templar{} is a Vaimon warrior-mage and knight. 
Today, only \ClanRedcor maintains an order of \templars{}. 





\subsubsection{Technology taboo}
\target{Vaimon technology taboo}
The Vaimons, especially the Redcor, had a taboo against science and technology.
They used it, but they opposed any innovation and research.
Research that challenged traditionally held views were taboo.
It reminded the Vaimons of \Belzir's heresy, and of heathenism and other wickedness.









\subsection{Equipment}





\subsubsection{Archon Ward}
\target{Archon Ward}
\target{Archon Wards}
\index{Archon Ward}
A type of magical \armour made by the Vaimons during the time of the \hr{Vaimon Caliphate}{\VaimonCaliphate}. An Archon Ward consists of a headband, a necklace and a number of bracelets. Together, when activated by a skilled Vaimon, these can form an energy shield surrounding the user. 

Archon Wards were extremely rare and expensive even in the empire. Several \VaimonCaliphs are known to have worn them, but barely anyone else could afford them. 





\subsubsection{Sabres}
\index{sabre}
The Vaimons traditionally wield sabres. 
\quo{Sabre}, in this case, is a loose term covering a wide variety of lighter swords, usually single-edged and curved. 

Sabres are not very effective against \armoured opponents. 
Here, a Vaimon would use magic. 

All the different martial (and magical!) arts practiced by the Vaimons have names. 

\target{chandre}
\target{chatresse}
A medium-heavy sabre is called a \chandre. 
The art of wielding it is \chatresse. 
This is where \VizicarDurasRespina{} excelled. 





\subsubsection{Technology}
\index{technology!\VaimonCaliphate}
\hs{Technology} rose during the \VaimonCaliphate's time. 
But that was not anything the Vaimons could take credit for. 
They did not invent guns and Archon Wards and everything. 
These things were imported through Cabal channels. 





\subsubsection{\Truesilver}
\target{Truesilver}
\index{\truesilver}
\Truesilver{} is a powerful metal from which some old Vaimon weapons are made. 
It is not a naturally occurring mineral but an artificial alloy, the technique of whose making has been lost. 

\Truesilver{} is almost as strong as \dragonsteel, but significantly lighter. 

Its known components are \dragonsteel, iron and silver. 
There must also have been some lighter metals involved, but they are unknown. 







\subsection{Politics}





\subsubsection{\Ortaicans and \rethyaxes}
See the section about \hr{Ortaican-Vaimon relationship}{\Ortaican-Vaimon relationships}. 





\subsubsection{Reputation}
In the \hr{Scatha Age}{\Scatha Age}, Vaimons were admired and feared. 
They were larger-than-life figures, touched by \Iquin (or worse, \itzach). 









\subsection{\VaimonCalendar}
\target{Vaimon Calendar}
\index{\VaimonCalendar}
\index{\VC}
The calendar of the old \hr{Vaimon Caliphate}{\VaimonCaliphate}, still used by Vaimons and in most of \Velcad{}. 
\quo{$n$ \IC} denotes year number $n$ in the \ImperialCalendar, counting from the year when Cordos Vaimon was crowned \caliph (year 1 \IC{}). 

A year is 380 days long. 
%where year 1 \IC{} was the year when Cordos Vaimon was crowned \caliph. 

The \ImperialCalendar has sixteen months, dedicated to the sixteen \Sephiroth{}, and each week has eight days, named after the Vaimon founders. 

The end of the year is celebrated with the festival of \Camaire{} on the last day of \Gamishiel{}, midways between the winter solstice and the spring equinox. 
\also{days (\ImperialCalendar), months (\ImperialCalendar)}





\subsubsection{\Camaire}
\index{\Camaire}
In the \VaimonCalendar, \Camaire{} is the festival that marks the end of the year. It falls on the last day of the month of \Gamishiel{}, midways between the summer solstice and the spring equinox.





\subsubsection{Days of the week}
\index{days of the week (\VaimonCalendar)}
\index{\Corjin}
\index{\Zetherab}
\index{\Rebecab}
\index{\Arcab}
\index{\Norquin}
\index{\Tirjin}
\index{\Kerzab}
\index{\Siljin}
In the \ImperialCalendar, a week is eight days long. Each day is named after of the Vaimon founders. 
The days, in order, are:

\begin{enumerate}
  \item \Corjin (after Cordos Vaimon, the first \VaimonCaliph).
  \item \Zetherab (after \Zether Vaimon, son of Cordos and Silqua and the second \caliph).
  \item \Rebecab (after Rebecca Redcor, daughter of Cordos and Silqua).
  \item \Arcab (after Arcan \Delaen, Silqua's eldest brother).
  \item \Norquin (after Norcah Quaerin, son of Cordos and another wife).
  \item \Tirjin (after Tiraad Geican, son of Cordos and a third wife).
  \item \Kerzab (after Kerzah \Irgel, younger son of Cordos and Silqua).
  \item \Siljin (after \hr{Silqua}{Silqua \Delaen}, Cordos' wife). 
\end{enumerate}
\also{\ImperialCalendar, Vaimon}





\subsubsection{Months}
\index{months (\VaimonCalendar)}
\index{\Atzirah!month}
\index{\Feazirah!month}
\index{\Keshirah!month}
\index{\Razilah!month}
\index{\Barion!month}
\index{\Hapheron!month}
\index{\Izion!month}
\index{\Teshiron!month}
\index{\Cushed!month}
\index{\Hoshied!month}
\index{\Thimared!month}
\index{\Yemared!month}
\index{\Gamishiel!month}
\index{\Ishiel!month}
\index{\Omariel!month}
\index{\Yeziel!month}

The \ImperialCalendar{} has sixteen months, dedicated to the sixteen \sephiroth{}.
The four months of spring are \Atzirah{}, \Razilah, \Keshirah{} and \Feazirah{}. 
The summer months are \Barion{}, \Teshiron, \Izion{} and \Hapheron. 
The autumn months are \Thimared, \Yemared, \Cushed{} and \Hoshied. 
The winter months are \Omariel, \Yeziel, \Ishiel{} and \Gamishiel. 

Each month is 24 days long, split into three weeks of eight days each (beginning with \Corjin{} and ending with \Siljin). 
The exception is \Gamishiel{}, the last month of the year, which is only 20 days long. 
(\Gamishiel{} is the \sephirah{} of Sacrifice.) 
















\section{\VaimonCaliphate}
\target{Vaimon Caliphate}
\target{Vaimon age}
\index{Vaimon!\VaimonCaliphate}
The \VaimonCaliphate existed from the year \yic{Founding of the Vaimon Caliphate}, where it was founded by \hs{Cordos Vaimon}, and until the \hr{Hundred Scourges}{\darkfall} where it fell, during the reign of \hr{Belzir}{\Belzir}. 

It spanned much of \Azmith, if not most. 

\target{Vaimon Caliphate naivete}
Back then, people believed that the \Sephiroth{} and \hr{Iquinian angels}{angels} (\resphain) were noble and good, having defeated the evil \pdaemons{} and being well on their way to vanquishing the last remnants of evil in the world. 

But the truth was slowly revealed to the learned during the reigns of the last few \caliphs, including \hr{Vizicar}{\VizicarDurasRespina}. But Vizicar still very much believed in the faith of \iquin{} and did not discover much, only a suspicion that something is lurking beneath the surface somewhere.

It all collapses under \Belzir, who learns far more than she should.









\subsection{Culture}





\subsubsection{\VaimonCaliph}
The \VaimonCaliph was both a secular and a religious authority. 
He was addressed \quo{Your Magnificence}.









\subsection{Geography}





\subsubsection{Rainbow Palace}
\index{Rainbow Palace}
\target{Rainbow Palace}
The royal palace of the \VaimonCaliph in \hr{Shiin-Merodar}{\ShiinMerodar}. 
Now destroyed. 





\subsubsection{\ShiinMerodar}
\index{\ShiinMerodar}
\target{Shiin-Merodar}
Once the capital city of the \VaimonCaliphate, seat of the magnificent \hs{Rainbow Palace}, \Merodar has now sunk beneath the sea. It lies somewhere between \hr{Vidra}{\Vidra} and \hs{Ontephar}. 

There are only some islands left of the big place that once contained \ShiinMerodar. 
The place was destroyed in a magical catastrophe during the \hr{Hundred Scourges}{\darkfall}. 





\subsubsection{\Vymorja}
\target{Vymorja}
\index{\Vymorja}
\Vymorja was a city in the \VaimonCaliphate that became the site of heresy.
The Vaimons of \Vymorja studied demonology and the forbidden sorcery of the \ophidians.
They translated a number of \ophidian spells into Vaimon terms so that Vaimons could call upon them. 

\Vymorja was razed to the ground by the Iquinians, but the wisdom of the \Vymorjans survived.

\target{Vymorjan Chants}
The most well-known \Vymorjan spells were the set known as the \Vymorjan Chants, used to summon and control \hs{Night-Feasters}. 
There were three chants. 
The first chant summoned a Night-Feaster.
The second prevented the monster from attacking the caster. 
The third chant compelled the Feaster to obey the caster's commands. 









\subsection{History}





\subsubsection{Heretics}
\target{Vaimon heretics}
During the time of the \VaimonCaliphate there were some Vaimon heretics who rebelled against the beliefs of the \caliphate. 
They believed that \humans{} were gods and merely had to release their inner divinity through embracing a life of chaos and blasphemy. 
Some of these mystics were actually half-mad \hs{Scions}. 

One of these heretics was \hr{Iolivine}{\Iolivine}. 

Compare to Aleister Crowley, and some of the philosophy in \authorbook{Graham McNeill}{Fulgrim}. 





\subsubsection{The \Darkfall}
\index{\Darkfall}
Perhaps the \hr{Hundred Scourges}{\Darkfall} coincided with a terrible, bloody war between different \resphan factions. 
Perhaps this was even the fall of \Kezerad. 

Inspired by the album \bandalbum{Symphony X}{Paradise Lost}.









\subsection{Philosophy}





\subsubsection{\Human supremacist}
\target{Vaimon Caliphate oppressed Scathae}
The \VaimonCaliphate was a \human supremacist culture. 
They repressed \scathae. 

The \scathae rebelled many times. 
\hr{Scatha rebellion under Belzir}{The worst such rebellion happened during \ps{\Belzir} reign}. 

One of the great successes of \ClanTelcra was \hr{Telcra integrates Scathae}{integrating the \scathae}. 















\section{Magic}
\target{Vaimon magic}
Traditional Vaimon metaphysics tells that the Universe is governed by two basic forces, named \Iquin{} and \Itzach. \Iquin{} is translated \quo{Light}, and is considered gentle and preserving, whereas \Itzach{}, translated \quo{Darkness} or \quo{Shadow}, is seen as aggressive and destructive. 

Each Vaimon \vclan has their own interpretation of \Iquin{}-\Itzach{} magic theory. 

The \hs{Redcor} (and the \hs{Iquinian Church}) insist that \Itzach{} should be translated \quo{Shadow}, because, according to their religion, \Iquin{} existed before \Itzach{}, and \Itzach{} is a corruption of \Iquin{}, something secondary that exists only at the mercy of the Light.\footnote{This is comparable to the role of the Devil in certain Christian interpretations: God and Satan are not equally matched adversaries. Rather, God is seen as all-powerful, and Satan, with all his evil, exists only because God, in all his good, allows him to. Supposedly, this somehow makes sense.} The Geicans, on the other hand insist that \Itzach{} should be translated \quo{Darkness}. To them, \Itzach{} is not a feeble reflection of \Iquin{}, but a primal force, possibly even more primal than \Iquin{}. 





\subsection{\Archons}
In Vaimon metaphysics, \Archons{} were supernatural beings and forces of nature. 
The \Archons{} included the \hr{Sephirah}{\Sephiroth} and \hr{Qliphah}{\Kliffoth}, who could be invoked to cast magic.  

The \Archons of Vaimon metaphysics were sometimes worshipped like gods, but they were not gods. 
Especially Iquinians prided themselves on their Archons being worlds apart from the vulgar, earthly gods of some religions, and would take offense to the \Archons{} being labelled as \quo{gods}. 

The gods of many religions literally walked the earth and could be encountered, whereas the Archons were disembodied forces that could manifest their power but had no physical form. 

The Iquinians teach that the \Archons are godlike beings to be revered and worshipped. 
They preach submission to the \sephiroth, transforming oneself into a vessel for their will and their work. 
But \ClanGeican teaches that, powerful though they may be, the \Archons are not true intelligent creatures, and they can be dominated. 
The Vaimon must exercise his will and master the \Archons, compel them to his bidding. 
A competent Vaimon knows which \Archons and feats are within his ability and which are beyond him, and he knows to keep his mind free of the influence of dangerous \Archons. 






\subsubsection{Abyss}
\target{Vaimon Abyss}
The Abyss was an abstract, mental \quo{plane}. 
It separated \hr{Atziluth}{\Atziluth} from the dwelling places of the \qliphoth. 

It was unclear from the theory whether the \qliphoth dwelt \emph{in} the Abyss or beyond it.
But in order to invoke the \qliphoth, a Vaimon had to mentally \quo{cross} the Abyss. 

According to \hs{Iquinian mythology}, the \sephiroth created the Abyss in order to limit the \hr{Iquinian creation myth}{defilement of the world caused by the invading \qliphoth}. 
The Abyss acted as a \quo{moat} around \Atziluth. 

In reality, the Abyss had to do with the borders separating \Miith from \hr{Erebos}{\Erebos}:
The \hr{Crystal Sphere}{\CrystalSphere}. 





\subsubsection{\Empyrean}
\target{Empyrean}
\index{\empyrean}
In Vaimon metaphysics, the \quo{\empyrean} was the \quo{plane} that the \Archons{} were said to inhabit. 
The \empyrean{} was not a physical \hs{Realm} but a mental abstraction. 

\hr{Atziluth}{\Atziluth} was a part of the \empyrean. 





\subsubsection{Exact definition}
What exactly constitutes an \Archon?
It clearly includes \sephiroth and \qliphoth.
But what about \hr{Iquinian angels}{angels}? 
And \malachim?







\subsection{Cost}

Vaimon magic is more subtle than Chaos magic and warps only the soul, not the body\ldots{} mostly.

\Iquin{} brainwashes the channeller with the \quo{virtue} that the \Sephirah{} in question represents. The more the mage channels the \Sephirah{}, the more its virtue will be ingrained in his mind. So people who channel Iquin a lot tend to become zealously devoted to the Iquinian ideology (if not to the Church itself). 

Some magic schools have spells that extend one's life. 
These spells may have nasty side effects. 
(An example of this is \hr{Life drain}{life drain}, which gradually turns the caster into an undead Reaver.) 

To cast \hr{Iquin}{\Iquin} magic you must pray to the \sephiroth, which binds your soul to them and \hr{Parasitic Archons}{steals moments of your life span}. But at least the \sephiroth{} do it in an aesthetic manner, so it doesn't show immediately. 

But wait\ldots{} \hs{Vaimons} live \emph{longer} than regular people. How does that work? 

It must be because the \sephiroth{} hoard the life energy of \emph{all} worshippers, and the Vaimons tap into that pool. So they're not paying the cost themselves, everyone else is. 

When you draw power from the \qliphoth, you strengthen the connection from \Miith{} through \hr{Nyx}{\Nyx} to \hr{Erebos}{\Erebos}. 
Also, you run the risk of the Cabal or Sentinels finding you and getting rid of you. 
(The Cabal is not fond of unauthorized \hr{Itzach}{\itzach} channelling.)

The more you draw on a \qliphah, the more you bind your soul to it and the more it infects and corrupts your soul and body.
Hence mages who use \itzach are often feeble and sickly and plagued with infirmity, physical and psychic.
The more different \qliphoth you use, the worse these effects become (but the more formidable you become as a mage, because you gain a diversity of powers). 

What do the \hr{Resphan}{\resphain} do? They are inherently vampiric and parasitic in nature and must drink the blood and life-force of others (typically humans) to power their magic and their physical strength. 

Perhaps some Cabalists \hr{Lictors}{degenerate into wretches} as well.





\subsubsection{\Itzach: Pain and horror}
\target{Itzach pain}
The \qliphoth{} of the Dawn Circle is relatively harmless. 

But \qliphah-magic of the Darker Circles is painful for mortals. 
Both physically and psychologically. 

It is psychologically hard for a Vaimon to call upon and channel a \qliphah{}. 
When her mind reaches out to the nether gulfs of \Itzach, it instinctively recoils in fear and loathing. 
She gets a natural urge to flee from this inhuman pit of primal black horror and never gaze upon it again. 
She has to steel herself and block out everything except her goal, in order to keep her sanity. 

Even for strong, brave men like Carzain and Vizicar, it is hard and taxing for body and soul. 

Moreover, when channelling a \qliphah, the Vaimon often feels an illusory pain that somehow resembles the effect, she is trying to create with her magic.   
She then has to be able to endure this pain and hold the concentration and channel the power outwards. 
If she succumbs to the pain and loses her resolve and focus, the \qliphah{} will break control and attack her body, which can cause wounds springing open, nasty inner bleedings, and even death. 

Some philosophers see this as a kind of karma: 
\quo{%
  A Vaimon should not seek inflict on others what she has not first endured upon her own body.}

\Itzach{} Vaimons have devised optimized methods to achieve their desired results while minimizing the pain and risk to their own bodies and sanity. 

When a Vaimon is in pain and channelling a \qliphah, once in a while he will hear voices in her head. 
These voices do not speak clear words, but they seem to vaguely hint of things. 
They promise to make the pain go away if the Vaimon will only give in to them and let them into his soul and body. 
This is a trap. 
Letting the voices in will only bring madness and make the Vaimon lose all control of his \qliphoth. 
It is unknown if the \quo{voices} are actually the voices of \qliphoth{} or just imagination, but it is well-known that they are up to no good. 





\subsubsection{Parasitism}
Vaimon magic is parasitic, because it is powered by \hr{Parasitic Archons}{the life force stolen by the \Archons}. 





\subsubsection{Stigmata}
\target{Vaimon stigmata}
When a Vaimon expended a lot of energy this would take a toll on his body. 
Wounds would form\dash inner and outer. 
Some of these wounds never fully healed.
They remained as scars, called \emph{Vaimon stigmata}. 
Through their lifetimes, Vaimons typically accumulated these stigmata. 
Older Vaimons tended to have more. 

These stigmata were a kind of \hr{Decrepity}{decrepities}.








\subsection{Demography}
\target{Vaimon demography}
There are \hr{Rethyax demography}{more Vaimons than \rethyaxes}. 









\subsection{How to cast it}
Vaimon magic uses mostly fast, simple \quo{point-and-click} spells: 
You simply invoke the names of one or more \Archons, mentally visualize the effect you want, and perform a simple gesture such as pointing your finger. 
%Stricly speaking, neither gesture nor speech are actually needed. The name is a way of contacting the \Archon{}, which is not necessary if you are experienced and finely attuned to Iquin or Nieur. (Or is it? Maybe the invocation is always necessary.) 
The gesture is not really needed. 
It is merely a psychological aid to help shape the spell, which is unneeded if you have fine control over the \Archons. 
The invocation of the name may or may not be needed, I'm not sure. If the spell is difficult or requires much power, you may have to keep chanting the name(s) over and over. 
Vaimons can also combine into a circle to perform ritual magic. Such a circle is much less flexible and slower to react than a single caster, but can cast more powerful and complex magic. 





\subsubsection{\Shechinah{} and meditation}
\target{Shechinah}
\target{resonance}
\index{\shechinah}
\index{resonance}
The \shechinah, also called \quo{resonance} is a state of being connected and \quo{attuned} to the \Archons. 
Vaimons achieve resonance through meditation or prayer, usually practiced daily. 

Vaimons who use \Itzach{} will, before all others, call upon the \qliphah{} \hr{Kor-Rashad}{\KorRashad} to guide them through the Empyrean. 
Then some \sephiroth. 
\KorRashad{} acts to counterbalance the influence of the brainwashing \sephiroth{}, who in turn protect the Vaimon from the mind-consuming \qliphoth. 
Then the Dark Vaimon will invoke \hs{Thaid} and \hs{Thuin}. 
Then some more \Archons. 

The more \shechinah{} a Vaimon has, the easier it is to cast magic, and the cheaper (measured in personal energy) each spell is, and the greater effects you can accomplish. 
But every spell drains some \shechinah. 
Every spell has a minimum \shechinah{} level required to cast it, and big, heavy effects likewise. 

But the higher your \shechinah{}, the greater power the \Archons{} hold over your mind. 

An old an experienced Vaimon builds up a store of permanent \shechinah{} that cannot be lost. 

Commoners could also experience and feel the \shechinah, such as in \hr{Iquinian prayers}{prayer}. 









\subsection{Pros and cons}
Vaimon magic is faster and easier and safer to cast than \rethyactic{} magic. 
Chaos magic is more powerful, but only with plenty of \trope{PrepTime}{\quo{Prep Time}} to cast complex spells. 
Chaos is more versatile, but also more dangerous and volatile. 

On a battlefield, a \rethyactic{} \ps{\ishrah} place is far behind the lines, casting big and slow and terrible spells to rain down on the field. 
A Vaimon's place is out there in the fray, blasting away at the frontline wherever he can do the most harm. 

The Vaimons also have their shielding \sephiroth{} (\Barion, \Hoshied, \Teshiron{} and \Yeziel) which provide good protection against physical and magical attacks, giving them a further edge against other mages in close quarters. 

A Vaimon must spend both personal energy and \shechinah{} to cast magic. 
\Rethyaxes{} do not need \shechinah. 
On the other hand, a \rethyax{} must make explicit pacts with the dark gods and sometimes do favours, solve tasks and even undertake quests for them. 









\subsection{Spells}





\subsubsection{Nasty combat magic}
Vaimons need to have some really nasty combat spells. 
Compare to this:

\lyricswikipedia{http://en.wikipedia.org/wiki/Homunculus}{Homunculus}{
  In the visions, Zosimos mentions encountering a man who impales him with a sword, and then undergoes \quo{unendurable torment}, his eyes become blood, he spews forth his flesh, and changes into \quo{the opposite of himself, into a mutilated anthroparion, and he tore his flesh with his own teeth, and sank into himself}\ldots{}
}





\subsubsection{Flight and jumping}
Vaimons cannot fly, but they can make great powered leaps.
See the section on \hs{Flying magic}. 









\subsection{Sanity}
A Vaimon's sanity is constantly in danger of being invaded and brainwashed away by the \Archons. 
The \sephiroth{} pull the mind in a \quo{virtuous}, conservative, bigoted direction, while the \qliphoth{} pull in a chaotic, violent, manic direction. 

The Geicans believe, therefore, that the Vaimon must balance the two opposing influences and find a golden mean. 

\lyricsbs{Aleister Crowley}{%
  Liber Aleph vel XCI
  (Chapter 13: 
   De Somnis ($\delta$) Sequentia/%
   On Dreams (d) Continuation)
}{
  If then there be a Traitor in that Consciousness, how much the more is it necessary for thee to arise and extirpate him before he wholly infect thee with the divided Purpose which is the first Breach in that Fortress of the Soul whose Fall should bring it to the shapeless Ruin that is Choronzon!
}

\lyricsbs{Aleister Crowley}{%
  Liber Aleph vel XCI
  (Chapter 15: 
   De Via per Empyraeum/%
   On Travel through the Empyrean)
}{
  And therefore is Confusion or Terror in any such Practice an Error fearful indeed, bringing about Obsession, which is a temporary or even may it be a permanent Division of the Personality, or Insanity, and therefore a defeat most fatal and pernicious, a Surrender of the Soul to Choronzon.
}





\subsubsection{Failed Vaimons}
\target{Failed Vaimons}
Some people who try to become Vaimons fail and are driven halfway or wholly mad by the \Archons{} coursing through their mind and body. 
\hr{Roanne Deracille}{\Roanne{} \Deracille} is one such. 









\subsection{Uses}





\subsubsection{Healing}
\target{Vaimon healing}
\hs{Iquinian} Vaimons are some of the world's best healers. They believe it is because \hr{Iquin}{\Iquin} is the force of Light, Life, Creation and all that is good. 

But in reality, as it turns out, it is because the \sephiroth{} have plenty of lifeforce to go around, because they actively \hr{Parasitic Archons}{drain the lifeforce of worshippers}. 
Every prayer to the Light steals minutes or hours from your lifespan and binds your soul more tightly to \iquin{}. 
It also steals intelligence, free thinking and free will and binds you more tightly in the Shroud. 
And when you die, the \sephiroth{} can, if need be, dismantle and gobble up your very soul and use it as fuel for healing magic. 

Vaimons lived as long as other people.
One might think they would have shorter lifespans because the \sephiroth drained their lifeforce. 
But the Vaimons themselves took something back from the \sephiroth as well. 
They unknowingly siphoned the lifeforce and lifespan of the masses of Iquinian believers to keep themselves alive. 









\subsection{Visualizing the \Archon}
\target{Visualizing Archons}
Each \Archon{} has a distinct synesthetic feeling to it. 
It might be that of flying high above the ground, or a deep jungle, or the grinding of rocks, or whatever. 
It might also vary from person to person. 

Dark Vaimons imagine \hr{Kor-Rashad}{\KorRashad} as proclaiming their philosophy. 

\lyricsbs{Emperor}{Thus Spake the Nightspirit}{
  Close your eyes and gaze into\\
  this realm that I reveal.\\
  See where eternities are born.\\
  Close your eyes, behold the powers\\
  of the broken seal.\\
  See the liars bound in thorns.
  
  Fear, and you shall fall.\\
  Weakness suffocates your will.\\
  Dare, yet never fail.\\
  Wisdom guides the one,\\
  the strong who can defy\\
  death.
}

\lyricslimyaael{427806}{%
  \textbf{3) Beckon the grotesque.} 
  
  I've wondered lately why descriptive passages on magic in so many fantasy novels do nothing for me anymore. 
  There are doubtless multiple reasons, but I think part of it is that, even when the authors are writing about destructive magic or evil inhuman creatures like the Unseelie Court, they describe the effects of magic as beautiful, or pretty. 
  That tends in the direction of fluff if the author isn't careful. If she is, it'll still call up very similar pictures from a lot of other fantasy books.
  
  I've been thrilled and felt wonder from descriptions of the grotesque, however. 
  I still haven't managed to finish \emph{Perdido Street Station}, but the descriptions of New Crobuzon, especially the beetle-headed khepri, are a lot more intriguing than yet another scene of moonlit pools and silver wolves and unicorns. 
  And my candidate for most awe-inspiring magical talent I've read about this year isn't the king-and-the-land magic in \emph{The Fall of the Kings}, although it was beautifully described. 
  It's the ability to grow cocoons on one's palms and hatch insects from them that I read about in \emph{The Etched City}. 
  I'm also enjoying the three brothers nested in each other like Russian dolls from \emph{Someone Comes to Town, Someone Leaves Town}, although that's been slow reading for other reasons.

  Many fluffy magical systems that blur into each other across fantasy books share common touchstones\dash\quo{beautiful} animals like horses and wolves, images of light from moon and sun, natural elements like water and fire that we've been trained to admire, brilliant \colours. Replacing even a few of those touchstones may lead to the sense of the strange, the weird, the alienness that we don't understand and recoil from. Insects, disease, filth, blood, and mutated and decaying bodies are much less often terms of fluffy magic. Try beckoning the grotesque into your magical system and see what it does.
}

















































































\chapter{Ethnic groups}















\section{\Goydens}
\target{Goyden}
\index{\Goyden{} (plural \Goydens)}
The \Goydens{} are a savage people living in the \Wylde{} in southern Pelidor and northern \hs{Beirod}. 
They are \humans, but some say they are half \human{} and half beast. 
They speak their own language and pray to their own gods. 
According to some rumours they can change between \human{} and animal form. 















\section{\Orticans}
\index{\Ortican}
\index{\Ortic{} (language)}
\target{Ortic}
A \scathaese{} ethnic group and language somewhat widespread in \hs{Beirod}, \hs{Gaznor} and southern \hs{Pelidor} and \hr{Scyrum}{\Scyrum}. 

The \Ortic{} language is descended from \hr{Ortaican language}{\Ortaican}.















\section{\Samur}
\target{Samurin}
\index{\Samur}
A predominantly \scathaese{} people that originate from the area on the northern bank of the \Samure{} Gulf but later spread eastward into what is now the northeastern Imetrium, Andras and western \hr{Scyrum}{\Scyrum}. 









\subsection{\Samurin{} language}
\index{\Samur!\Samurin{} (language)}
The \Samurin{} language is still spoken in parts of Andras and \hr{Scyrum}{\Scyrum}, but repressed in the Imetrium. 















\section{\Tepharites}
\index{\Tepharite}
\target{Tepharite}
The \Tepharites{} are a people living in southern \hr{Velcad}{\Velcad}. 
Originally they were purely \scathae{} (of mixed \hr{Tassian}{\Tassian} and \hr{Mekrii}{\Mekrii} descent), but they merged with some \human{} tribes, and eventually these came to be considered \Tepharites{} as well. 

Prior to the advent of \hr{Great Velcad}{\theBelkadianEmpire} they had their \hr{Bacconate}{\bacconate}, \hr{Tepharae}{\Tepharae}. 

The \Tepharin{} language (related to \hr{Ortaican language}{\Ortaican}) is still spoken many places in \hs{Ontephar}, \hs{Pelidor} and \hr{Scyrum}{\Scyrum}. 
\also{\Tepharae}







































\chapter{Places}















\section{\Velcad}
\target{Velcad}
\index{\Velcad}
A region located in the middle of \hr{Azmith}{\Azmith} and comprising a large number of independent kingdoms (and great expanses of \Wylde{}). 
It is not well-defined or well-delimited, but is usually considered to cover (at least) all lands where the \Velcadian{} tongue is spoken. 

%It includes, among other things, the nations of Belek, \Redce, Pelidor and Runger. 

\Velcad{} includes the nations of \hs{Andras}, \hs{Beirod}, \hs{Belek}, \hs{Ontephar}, \hs{Pelidor}, \hr{Redce}{\Redce}, \hs{Runger} and \hr{Scyrum}{\Scyrum}, among others. 









\subsection{Knights}
\target{Knight}
\target{knights}
\index{knight}
In \hr{Velcad}{\Velcad}, especially honoured warriors are given the status of knights. They are blessed by the \hr{Iquinian}{\Iquinian} Church and carry the \hr{noble titles}{title} \quo{\rah}. 
Examples include \hr{Sethgal}{\rah[Sethgal] Pelidor}. 

\target{hedge knight}
Most knights are noble-born, but commoners are sometimes knighted. The latter are called \quo{hedge knights}. 

\target{Knights have superpowers}
Iquinian knights have superpowers.
They are not Vaimons, but with the help of priests they can summon the \sephiroth and perform superhuman feats of martial arts.
They also look and feel imposing and regal and commanding and holy, because they have the blessing of the \sephiroth\dash{}that blessing is what makes them knights.









\subsection{The Tigers}
\target{Tiger}
\index{Tiger (warrior order)}
The Tigers were the elite, professional soldiers of \hr{Great Velcad}{\theBelkadianEmpire}, taking their name from \hs{Uther the Tiger}. 
After the downfall of the empire the Tigers are no longer centrally organized. 
Some owe their allegiance to various lords, while others are mercenaries. 

The Tigers contain both cavalry and infantry, including archers. 
Some of them are \hr{Knight}{knights}, annointed by the \hs{Iquinian} church. 

The Tigers have no mages. 
Instead, the mages affiliated with \hr{Great Velcad}{\theBelkadianEmpire} were the \hr{Telcra}{\Telcras}. 















\section{Bron}
\index{Bron}
A river in western \hr{Velcad}{\Velcad} that, for much of its length, marks the border between \hs{Andras} and \hr{Scyrum}{\Scyrum}{}. It runs from the Threll mountains into the \hr{Risvael Sea}{\Risvaelsea} and meets the river \hr{Pylor}{\Pylor}. 















\section{Far Orient}
\target{Far Orient}
\index{Far Orient}
\index{Orient!Far Orient}
The eastern part of the \hs{Orient}. 















\section{Gwendor Sea}
\index{Gwendor Sea}
\target{Gwendor Sea}
A sea in the northen \hr{Velcad}{\Velcad}. It encloses \hr{Vidra}{\Vidra} and borders \hs{Threll}, \hs{Ontephar}, \hs{Runger} and \hr{Redce}{\Redce}, among others. To the west it flows into and becomes the \hs{Thenglain Sea}. 















\section{Heropond Forest}
\target{Heropond Forest}
\index{Heropond Forest}
A large \Wylde{} forest in \hr{Pelidor Continent}{\PelidorContinent}. Marks the border between \hs{Pelidor} and \hr{Scyrum}{\Scyrum}. 









\subsection{Leglan's Pass}
\index{Leglan's Pass}
A path through the narrowest part of Heropond forest, near \hr{Bryndwin}{\Bryndwin} (on the \Scyric{} side) and \Redglen{} (on the Pelidorian side. 
It was formerly used as a trade route, but has fallen out of use in recent decades and grown more \hr{Wild}{\Wylde}. 















\section{Hirum Gulf}
\target{Hirum Gulf}
\index{Hirum Gulf}
A gulf north of \hr{Pelidor Continent}{\PelidorContinent}. 
It meets the \hs{Gwendor Sea} to the northeast, \hs{Runger} to the southeast, \hs{Pelidor} to the southwest and \hs{Ontephar} to the west. 
The river \hs{Nerim} runs south from the Hirum Gulf. 















\section{\Durcaccontinent}
\target{Durcac Continent}
\index{\Durcaccontinent}
A continent south of \hr{Velcad}{\Velcad}, southeast of the \hs{Imetrium}, northeast of \hs{Uzur} and west of the \hs{Orient}. 
Contains \hs{Durcac}, among other things. 
















\section{Lorn Sea}
\index{Lorn Sea}
\target{Lorn Sea}
A narrow sea in eastern \hr{Velcad}{\Velcad}, marking the eastern borders of \hs{Runger} and \hs{Beirod} and the western borders of \hr{Thyrin}{\Thyrin}, \hs{Sumian} and \hs{Belek}. It meets the \hs{Gwendor Sea} to the north and the \hr{Risvael Sea}{\Risvaelsea} to the south. 
















\section{Naemor Strait}
\target{Naemor Strait}
\index{Naemor Strait}
A strait in the \hs{Imetrium}, separating the \hr{Samure Gulf}{\Samure{} Gulf} from the \hs{Thenglain Sea}. The Imetric city of \hs{Falcus Aira} is built at the Naemor Strait. 
















\section{Near Orient}
\target{Near Orient}
\index{Near Orient}
\index{Orient!Near Orient}
A collective term for the nearer part of the \hs{Orient}. 
Kingdoms in the Near Orient include \hs{Hazid}. 
















\section{Nerim} 
\index{Nerim}
\target{Nerim}
A fjord that runs south from the \hs{Hirum Gulf} and marks the border between \hs{Runger} and \hs{Pelidor}. 
















\section{Northern Kingdoms}
\target{Northern Kingdoms}
\index{Northern Kingdoms}
Generic term for the lands north of \hr{Velcad}{\Velcad} (many of which are not actually kingdoms). 
















\section{Orient}
\target{Orient}
\index{Orient}
The area southeast of \hr{Velcad}{\Velcad} (east of \hs{Durcac} and south of the \hr{Serpentines}{\Serplands}). 
Informally split into the \hs{Near Orient} and the \hs{Far Orient}.  
















\section{\PelidorContinent}
\target{Pelidor Continent}
\index{\PelidorContinent}
A continent in central \hr{Azmith}{\Azmith}. Contains \hs{Pelidor}. 
















\section[Pylor]{\Pylor}
\target{Pylor}
\index{\Pylor}
A river. 
Marks the border between \hs{Andras} and \hr{Scyrum}{\Scyrum}.















\section{\Risvaelsea}
\target{Risvael Sea}
\index{\Risvaelsea}
A large sea in the centre of \hr{Azmith}{\Azmith} that marks the southern border of \hr{Velcad}{\Velcad}, the northern border of \hs{Durcac} and the easter border of the \hs{Imetrium}. In it lie the two large isles of \hs{Fendor} and \hs{Tugan}. 















\section{\Samure{} Gulf}
\target{Samure Gulf}
\index{\Samure{} Gulf}
A gulf in the \hs{Imetrium}. It borders the \hs{Thenglain Sea} to the west at the \hs{Naemor Strait} near \hs{Falcus Aira} and the \hr{Risvael Sea}{\Risvaelsea} to the east at the \hs{Martinum Strait}. 















\section{The \Serplands}
\target{Serpentines}
\index{\Serplands}
Also called the \Serpadj{} kingdoms. Collective term for a number of kingdoms that lie between the \hr{Serpentine Sea}{\Serpsea} to the west and the \Dragonridge{} to the east. The \Serpadj{} kingdoms are diverse in culture. Some are effectively vassal states of Irokas. 















\section{\Serpsea}
\target{Serpentine Sea}
\index{\Serpsea}
A sea that marks the eastern border of \hr{Velcad}{\Velcad}. The \Serp{} is long in the north-south direction but narrow in the east-west direction, and it twists and bends like a snake, hence the name. East of the \Serp{} lie a number of kingdoms called the \hr{Serpentines}{\Serplands}. 
\also{\Serplands}

%\gitem{\Serpriver}
%A great sea that runs along the eastern border of \Velcad{}. In fact, the \Serpriver is often considered to be the easter border of \Velcad{} by definition. East of the \Serpriver lie a number of kingdoms called the \Serplands. 
%\also{\Serplands}















\section{Thenglain Sea}
\target{Thenglain Sea}
\index{Thenglain Sea}
A great sea that marks the western border of \hr{Azmith}{\Azmith}. 















\section{Threll}
\target{Threll}
\index{Threll}
A region east of \hr{Velcad}{\Velcad}, a devastated land haunted by monsters. The Threll Mountains mark the its southern border with the \hs{Imetrium} and \hs{Andras}. It borders \hs{Ontephar} to the east, the \hs{Gwendor Sea} to the north and the \hs{Thenglain Sea} to the west. 















\section{Uzur}
\target{Uzur}
\index{Uzur}
A region south of the \hs{Imetrium} and southwest of \hr{Velcad}{\Velcad}. 
Much of it is covered in \hr{Wild}{\Wylde} forests and swamps, but there are nations that thrive here. 
\Meccara{} are the dominant humanoids. 

Nations include \hs{Clictua} and \hs{Kochu}. 




































































 
