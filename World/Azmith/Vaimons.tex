\chapter{The Vaimons}















\section{\Delaen}
\target{Delaen}
\target{Delain}
\index{\Delaen}









\subsection{History}





\subsubsection{Secular}
\target{Delaen was secular}
At the time of \hr{Vizicar}{\VizicarDurasRespina}, \ClanDelaen was the most secular and non-religious of the \VaimonClans.
This helped Vizicar grow up as a free-thinking man. 















\section{Geican}
%\section{\ClanGeican}
\target{Geican}
\target{Geica}
\index{Geica}
Geica, located in the southeastern corner of \Velcad{}, is the homeland of the Vaimon \ClanGeican. 

Unlike the Redcor, who are purely a \VaimonClan, \ClanGeican is known to accept non-Vaimons \quo{converts}, granting them status of full \vclan members. Once achieved, Geican status lasts for life and is passed on to all children\ldots{} or is it? 

\index{Geicanese}
The Geicans, ie., the members of \ClanGeican, are the upper class of Geica. The regular citizens of Geica are called \emph{Geicanese} (the term is singular, plural and adjective). About $60\%$ of the Geicans are Vaimons. 

They are scientists and study all sciences, including magic. They are known to use Nieur and actively pursue the study of it. 
They are believed to dabble in all sorts of black and evil magic and to consort with evil powers. 
For this reason they are seen as evil diabolists by some, especially their ancient rivals, the Redcor. 









\subsection{Aesthetics}
The traditional \colour of \ClanGeican is green, and many Geicans wear green robes of some sort. 
The symbol of Geica and \ClanGeican is an eagle in flight (a symbol of freedom), green on a black background. 
The eagle symbolizes freedom, and the black background signifies that without freedom (i.e., outside the eagle) there is only darkness and evil. 

The throne of Geica is of Emerald. 









\subsection{Culture}





\subsubsection{Debt slavery}
\target{Geican slavery}
The Geicans adore \quo{freedom}, so they don't have slaves. 
In name. 
But if you owe too much money, you can be indentured under slave-like conditions. 









\subsection{Geography}
\subsubsection{Fallen Emerald Palace}
Originally the Geican seat of power was the splendid Emerald Palace. 
But it was destroyed in the \darkfall. 

All \VaimonClans used to have such palaces of crystal. 
But today only the Redcor \hr{Topaz Chateau}{\TopazChateau} remains. 








\subsection{History}
The founder of \ClanGeican was Tiraad Geican, son of Cordos Vaimon. 









\subsection{Philosophy}
\target{Geican philosophy}
As a culture, the Geicans are atheistic and anarchistic and hail the freedom of the individual as their highest ideal. 

The Geicans (or, at least, some Geicans) believe that philosophy and science should be free and unhindered by morality. 

Some of them believe that philosophy and science \emph{should} be dark, scary and boundary-crossing. Only by abandoning everything known and safe and throwing yourself out in the deep water of the unknown, in the vast, dark emptiness of the inhuman universe, can you gain new insight. 

They believe that if you do not confront the frightening and unknown, you will continue to live in its shadow, on its mercy. As its slave or as its prey. 

Or are they a rare example of the triumph of mortals? Then again, the Geican democracy is a pretty corrupt, mafia-like nepotist system.

The modern-day Geican philosophy was founded by \hs{Zacrias}, son of  \Belzir{} and the first leader of the \vclan after the \Darkfall. 





\subsubsection{Religion}
The Geican culture is atheistic. They reject the personification of Iquin and Nieur that the Redcor believe in. According to the Geicans, the \hs{Iquinian} religion is a lie: The Spirit of the Light does not exist, Iquin (like Nieur) is just an impersonal, amoral force. All Iquinian \quo{miracles} are regular magic and the Iquinian metaphysics is based on a magic theory that is flawed and contradictory. The Redcor world view is a supremely naive fairy tale that blatantly ignores and denies many elementary facts known to any serious scholar. The Geicans see the Redcor as cowards who close their eyes to the real world because they are afraid to face the dark and terrible truth. In turn, they view themselves as a superior and wiser people, true scientists who pursue the truth without fear. 

Natually, this belief is blasphemous to the Iquinians. To make matters worse, \ClanGeican is known to have utilized black magic, and the Dark Prophet \Belzir{}, who caused the Darkfall, was a Geican. As a result, the Redcor believe that the Geicans are all malicious diabolists and alienists bent on conquering or destroying the world. (The Geicans themselves claim that the Redcor are intolerant bigots and prejudiced against them because of a select few genuine villains in Geican history. They are also quick to quote a record of Redcor atrocities in return.) 





\subsubsection{They were once religious}
The Geicans have not always been atheists. 
Before \ps{\Belzir}{} time, \hr{Geicans were religious}{they were Iquinian}. 










\subsection{Politics}
\subsubsection{Government}
Geica has a democracy and is governed by a Senate of 30 Senators. Only Geican \vclan members can vote. 

Originally the \vclan was led by a Grandmaster. 
\Belzir{} and Zacrias both held this title. 
Some centuries after the \Darkfall, the Grandmaster came to be democratically elected. 
Later it disappeared entirely and was replaced by a Senate. 





\subsubsection{Free thinking: An embattled kingdom}
\target{Geica is embattled}
Why are the Geicans allowed to be such free thinkers? 
%Are they under someone's protection? Perhaps \Kezerad{} or the \Cuezcans? 

Well\ldots{} Geica is embattled by the master races. Both factions seek to control the \vclan \hr{Master races seek to control magic}{and its magic}, and they antagonize each other, so that in the end no one controls Geica. This is one of the reasons why there is so much free thinking going on there. 









\subsection{Physique}
\target{Geican green eyes}
Green eyes were considered the hallmark of Geican \humans.
They were still rare among Geicans, but much more common than among \humans as a whole.
And a number of prominent figures in Geican history had those eyes, including \hr{Belzir}{\Belzir}. 
(\hs{Shereid} also had them.)









\subsection{The Royalist Faction}
\target{Royalist Faction}
\target{Royalist}
The underground faction that serves \hr{Belzir}{\Belzir} and aims to restore her to life and power. 

The faction is actually being manipulated by the \hs{Sentinels}. 

Notable members include \hs{Hayad}, \hs{Shereid} and, ultimately, \hr{Carzain}{Carzain \Shireyo}. 















\section{\Iquinian Church}
\target{Church of the Light}
\target{Iquinian}
\target{Iquinian Church}
\target{Iquinianism}
\target{Iquinian religion}
\index{Iquinian Church}
\index{Church of the Light}
\index{Church of \Iquin}
\index{Iquinianism}
The Iquinian Church, also called \emph{Iquinianism} or \emph{the Church of the Light}, was a religion founded by the Vaimons. 
Iquinianism was based on the worship of \iquin{}, the One Light, seen as the source of all good, and of the \Sephiroth{} as its manifestations. 

There were two major branches of the Iquinian church: 
The \hs{Redcor} branch and the \hr{Telcra}{\Telcra}. 

\index{Iquinian Church!Redcor branch}
\index{Spirit of the Light}
The Iquinian Church, also called the {Church of Iquin} or the {Church of the Light}, is based on \Iquin-\Nieur theory (see section \ref{Vaimon magic}), the idea that the universe is based on two primal forces: 
\Iquin (the One Light) and \Nieur (the Outer Darkness). 
The Iquinians worship \Iquin and believe that it is the primary, the first and most important of the two, representing all that is just and good. 
The One Light (sometimes personified as a god) is viewed as a supreme, perfect being, all-good and very powerful, \hr{Omnipotence of Iquin}{perhaps even all-powerful}. 
The One Light is seen as something special because it is \quo{not of this world} but something \quo{transcendent}. 
The Church tends to look down on religions that worship \quo{earthly gods} (such as the Imetrium), which are seen as inferior, false gods. 

The Iquinians see themselves and their church as the champions of good. 
They seem to combat and destroy evil wherever they encounter it. 
The Church of \Iquin is widespread throughout most of \Velcad{}. 
The Church does not rule directly outside \Redce{}, but the Redcor are skilled manipulators, pulling strings and directing events from behind the thrones. 

The Iquinian religion was also called the \quo{Vaimon religion}.
Post-\caliphate \hr{Geican}{Geicans} were not happy about this, since they had abandoned much of the religion but were still Vaimons.







\subsection{Culture}





\subsubsection{Clerical hierarchy}
\target{Iquinian clerical hierarchy}
The Iquinian church contained a core of true Vaimons who made up the higher tiers of the priesthood.

Then there was a larger number of assistant priests who could not invoke the \sephiroth on their own, but who assisted the Vaimons in prayer and magical rituals.
They were deacons, cantors, sextons, monks and nuns. 
These only knew simple orisons.





\subsubsection{Destroying information}
The Church is fond of destroying books and other materials dealing with the occult and other \quo{evil} things. They portray it as bad knowledge that people were not meant to know, and whose existence can only cause harm. But in reality, the Church is being manipulated by the Cabal, who wants \hr{Destroying information}{to keep people ignorant}.





\subsubsection{Funerals and preserving the dead}
\target{Iquinian funerals}
\target{mausoleum}
\index{mausoleum}
The Iquinians preserve the dead as mummies in great mausolea, or even \hs{pyramids}. 

Remember, \hr{Sephirah plan}{the purpose of Iquinianism is to harvest mortal souls}. 
The mausolea are designed to better keep the souls under control and reclaim as much soul-energy from the bodies as possible. 





\subsubsection{Laws}
\target{Iquinianism has many laws}
Iquinianism had many laws. 
In addition to the sixteen virtues of the \sephiroth there were official lists of sins and explicit rules of penance and punishment. 

This stood in contrast to the \Ortaican religion, which \hr{Ortaican religion has few laws}{had few laws}. 
This Iquinians \hr{Iquinian criticism of Ortaica}{used this to argue that \Ortaicanism was evil}. 

The real reason is this: 
Iquinianism was \hr{Iquin plan}{designed as a way of brainwashing the population}, and thus had to be mentally intrusive. 
\Ortaicanism was designed by the \taorthae as a political convenience, and so it was intentionally kept vague and flexible. 










\subsection{History}





\subsubsection{Exported as parallel religions}
\target{Iquinianism exported}
Seeing the success of the Iquinian religion in \Azmith, the Cabal since \quo{exported} the concept to other \hs{Shrouded Realms}. 
They constructed some similar religions, also worshipping \iquin{} in some form. 









\subsection{\Isphet the Destroyer}
\target{Isphet}
\target{Destroyer myth}
\index{\Isphet}
\Isphet was an evil figure in \Iquinian mythology. 
He was an evil god, an enemy of the \sephiroth. 
Sometimes described as a \qliphah. 
A nameless \qliphah of the Midnight Circle. 
Sometimes he was considered the king of all \qliphoth and the master of all that is evil. 
(A few doubted that he was a \qliphah at all.)

\Isphet was in continual battle with the \sephiroth.
In legendary times, he had waged a war against everyone and tried to destroy the world. 
He tore the world apart and killed millions upon millions. 
The \sephiroth and their \hr{Iquinian angels}{angels} had attacked him in force. 
They fought against the Adversary and his legions of \qliphoth. 

After \Isphet had destroyed the world, the \sephiroth had to banish him and rebuild it. 
The \sephiroth prevailed and cast out the Adversary. 
\Isphet since became the \hs{Exile}, feared and hated by all. 
A Satan type. 

From then the craven villain would hide in his dark pit of evil and only rarely dare to venture forth into the world of mortals. 
To this day, the myth said, the \sephiroth were locked in a cosmic battle with \Isphet. 
That was why the \sephiroth did not show themselves in the mortal world. 
And their believers \hr{Rituals against Isphet}{had to help them combat him}. 

Compare to Egyptian mythology, where there were spells to overthrow the \dragon Apep. 

See also the section about \hr{Myths about Dragons}{\dragons in art and mythology}. 

\citetitle{KingJamesBible}{%
  The Bible (Revelation 12:3--4,12:7--10)%
}{
  And there appeared another wonder in heaven; and behold a great red dragon, having seven heads and ten horns, and seven crowns upon his heads.\\
  And his tail drew the third part of the stars of heaven, and did cast them to the earth: and the dragon stood before the woman which was ready to be delivered, for to devour her child as soon as it was born.
  
  And there was war in heaven: Michael and his \hr{Iquinian angels}{angels} fought against the dragon; and the dragon fought and his angels, \\
  And prevailed not; neither was their place found any more in heaven. \\
  And the great dragon was cast out, that old serpent, called the Devil, and Satan, which deceiveth the whole world: he was cast out into the earth, and his angels were cast out with him.\\
  And I heard a loud voice saying in heaven, Now is come salvation, and strength, and the kingdom of our God, and the power of his Christ: for the accuser of our brethren is cast down, which accused them before our God day and night.
}

\lyricsbalsagoth{
  Enthroned in the Temple of the Serpent Kings
}{
  Scourge of Angsaar, wielder of the Black Sword,\\
  Immortal Lord of Darkmere, Serpent-Witch ensorcel me.
}





\subsubsection{Appearance}
\Isphet was depicted as a black \dragon. 
\Dragons symbolized evil in Iquinian mythology.

\Isphet was described as black with burning eyes and wreathed in fire and smoke. 
Sometimes he is said to have multiple heads and to breathe fire. 





\subsubsection{Names and titles}
\ps{\Isphet} titles included:
\begin{itemize}
  \item The Adversary.
  \item The Scourge.
  \item The Destroyer.
  \item Lord of Chaos.
  \item Lord of the Outer Darkness.
\end{itemize}

His name also appeared in the variant Iscraphet or Iscraphel. 

\quo{Isfet}, as far as I know, is an Egyptian word that means \quo{chaos} and is associated with the serpent Apep, the eternal enemy of Ra, the Sun god. 






\subsubsection{History}
The myth of \Isphet was made up somewhere in the Vaimon Age. 
It was unknown in Cordos Vaimon's time. 





\subsubsection{Rituals}
\target{Rituals against Isphet}
In churches they performed rituals at regular intervals that were meant to keep \Isphet at bay.
He was immortal and would not perish until the end of the world, but he could be wounded and weakened and mutilated.

People would gather in a church to attend \hr{Iquinian prayers}{prayer} and mutilate \Isphet. 
In one popular variant there would be an effigy of \Isphet in the form of a black serpent. 
Every church-goer was handed a needle or stick with which to impale the monster. 
At last, the effigy was hacked into pieces and burnt. 





\subsubsection{Truth}
\Isphet did not really exist. 
His myth was a twisted mash-up of the true stories of several \dragons, including: 
\begin{itemize}
  \item 
    \hr{Ishnaruchaefir}{\QuessanthIshnaruchaefir} and his role in the \hr{Shrouding}{\SecondShrouding}, where he \quo{sort of} destroyed the world.
    
    \Isphet's name and appearance was based on \Ishnaruchaefir. 
  \item 
    \hr{Secherdamon}{\IrocasSecherdamon} as the schemer who wanted to overthrow the \sephiroth. 
  \item 
    \hr{Tiamat}{\TyarithXserasshana} with her monstrous, multi-headed appearance. 
\end{itemize}









\subsection{Mythology}
\target{Vaimon mythology}
\target{Iquinian mythology}





\subsubsection{Angels}
\target{Iquinian angels}
Angels were beings of \iquin, lesser than the \sephiroth. 
They were represented as winged humanoids resembling \resphain. 

In the beginning (during the \caliphate), all angels looked like winged \humans. 
After \hr{Telcra integrates Scathae}{\ClanTelcra integrated the \scathae}, \scathaese angels were also sometimes depicted. 
Artists disagreed on whether these angels should have feathered wings or \hr{Pteran}{\pteran}-like wings. 
Feathered wings tended to win out, because \pteran wings were associated with \dragons, \hr{Iquinian myths about Dragons}{who were creatures of evil in Iquinian mythology}. 

The angels were said to live atop vast, beautiful towers that rose through the clouds and touched the sky. 
This was a twisted image of \hr{Nyx}{\Nyx}, where the true angels (\resphain) lived. 
It was a terrible shock for an Iquinian humanoid to come to \Nyx and see the hideous truth behind the myths about angels. 





\subsubsection{Creation}
\target{Iquinian creation myth}
Iquinian myth held that \iquin had always existed. 
The nature of \itzach was more ambiguous. 
Some interpretations said \itzach had always existed as the twin of \iquin.
Others said \itzach was an emanation of \iquin and existed at its mercy. 

In the beginning, the world was in balance, nicely delimited into a Dark half and a Light half. 
But then \itzach invaded \iquin, under the leadership of the dark god \hr{Isphet}{\Isphet}, the Lord of the Outer Darkness. 
The hordes of \itzach brought chaos. 

The essence of \iquin and the fetters of \itzach intertwined to create the material world. 
The \qliphoth of \itzach stole the essence that emanated from \iquin and forced it into unnatural shapes, thus creating the false illusion that the world was made of separate \quo{things} and \quo{individuals}.
Thus the Defiled world of \hr{Gehinnom}{\Gehinnom} was created. 

In order to keep the \qliphoth in check so they would not overrun the world, the \sephiroth created the \hr{Vaimon Abyss}{Abyss} as a moat around \Atziluth. 





\subsubsection{Dark age}
According to myth, \Iquin created the world long ago.
The \dragons and other forces of evil also existed, for good cannot exist without its opposite.

The first generation of mortals (pre-\humans) were sinful and displeased the One Light.
And the dark powers taught them forbidden skills and sorcery and knowledge which the One Light did not want them to have. 
So the \sephiroth abandoned them and allowed the \dragons and other evils to overthrow the mortals, overrun the world and rule it with terror for an Age.

\citetitle[p.88]{RHCharles:BookofEnoch}{The Book of Enoch LXV.6--10}{
  And a command hath gone forth from the presence of the Lord concerning those who dwell on the earth that their ruin is accomplished because they have learnt all the secrets of the angels, and all the violence of the Satans, and all their powers\dash the most secret ones\dash and all the power of those who practice sorcery, and the power of witchcraft, and the power of those who make molten images for the whole earth\ldots{}
  
  \ldots

  \quo{Because of their unreighteousness their judgement has been determined upon and shall not be withheld by Me for ever.
  Because of the sorceries which they have searched out and learn, the earth and those who dwell upon it shall be destroyed.}
}

After an Age of the World had passed, the \sephiroth took mercy on the world, and so they once again shown themselves.
They revealed themselves to Silqua and made her create the Vaimon order, so that \humans (having been punished enough) would now drive out the forces of Elder evil and rule the world again.

\citetitle[p.37--38]{RHCharles:BookofEnoch}{The Book of Enoch X.2--13}{
  [\ldots{}]
  and reveal to [Noah] that the end is appraoching; that the whole earth will be destroyed, and a deluge is about to come upon the whole earth, and it will destroy all that is on it.
  \\  
  And now instruct him that he may escape and his seed may be preserved for all the generations of the world.
  \\
  And again the Lord said to Raphael:
  Bind Az\^az\^el hand and foot, and cast him into the darkness; and make an opening in the desert, which is in D\^ud\^a\^el, and cast him therein,
  \\
  And place upon him rough and jagged rocks, and cover him with darkness, and let him abide there for ever, and cover his face that he may not see light.
  \\
  And in the day of the great judgement he shall be cast into the fire.
  \\
  And heal the earth which the angels have corrupted, and proclaim the healing of the earth, that they may heal the plague, and that all the children of men may not perish through all the secret things that the Watchers have disclosed and have taught their sons. 
  \\
  \ldots 
  \\
  In those days they shall be led off to the abyss of fire; and to the torment and the prison in which they shall be confined for ever.
}


Those Elder mortals were \humans in some versions of the story.
Others held that they were the \nephilim, thus justifying hate and persecution against the \nephil race.

This myth is based, to some limited extent, on the true story of how the \aryothim once ruled the world before the return of the \dragons.





\subsubsection{\Dragons}
\target{Iquinian myths about Dragons}
In \Iquinian mythology (and in the view of many \humanoids of the \Human Age and Scatha Age), \dragons were godlike Elder horrors who ruled \Miith with terror in the dark days of chaos before the \sephiroth took action and created the Vaimons to throw them out.
The \dragons were the spawn of chaos, the kin of loathsome alien gods and wielders of the blackest sorcery ever conceived.

\citebandsong{Nile:Ithyphallic}{Nile}{
  What Can Be Safely Written
}{
  On the walls of lost cities\\
  And in the carvings of madmen\\
  Who have glimpsed him in their dreams\\
  Is his image delineated\\
  Within a tomb protected by great seals he lies in death\\
  Under the weight of the dark waters of the deep\\
  Yet he dreams still, and in his dreams continues to rule this world\\
  For his thoughts master the wills of lesser creatures
}

See also the sections on \hr{Myths of vanquished monsters}{how \human heroes vanquished elder monsters} and on \hr{Myths about Dragons}{general myths about \dragons}. 





\subsubsection{\Humans vanquishing monsters}
\target{Myths of vanquished monsters}
There were myths about how great \human{} (sometimes \scathaese) heroes \hr{Cordos vanquishes monsters}{fought against and vanquished the evil pre-\human{} monsters that had dominated \Miith{} in the ages past}. 
Especially \hs{Cordos Vaimon} got this role: 
A Conan-esque hero, a frontiersman that ushered in the \quo{\hr{Human Age}{\Human{} Age}}. 

Stories tell how these brave, Iquinian, Light-fearing \human{} heroes overthrew and drove out the evil pre-\human{} Elder Races and monsters.
At last the wicked monsters were exterminated by the grace and power of the Light. 

Modern \humans were glad that these terrible, loathsome creatures no longer existed. 

Another source of these stories, buried deep down in \hr{Human racial memory}{\human{} racial memory}, is the vague recollection of \hr{Aryothim kill QJ}{the far older wars between the \aryothim{} and the \quiljaaran}. 

See also the section on \hr{Myths about Dragons}{myths about \dragons}. 

\citeauthorbook[p.35--36]{RobertEHoward:TheShadowKingdom}{Robert E. Howard}{%
  The Shadow Kingdom%
}{
  [Kull] stopped short, staring, for suddenly, like the silent swinging wide of a mystic door, misty, unfathomed reaches opened in the recesses of his consciousness and for an instant he seemed to gaze back through the vastnesses that spanned life and life; seeing through the vague and ghostly fogs dim shapes reliving dead centuries\dash men in combat with hideous monsters, vanquishing a planet of frightful terrors.
  Against a gray, ever-shifting background moved strange nightmare forms, fantasies of lunacy and fear; and man, the jest of the gods, the blind, wisdomless striver from dust to dust, following the long bloody trail of his destiny, knowing not why, bestial, blundering, like a great murderous child, yet feeling somewhere a spark of divine fire\ldots Kull drew a hand across his brow, shaken; these sudden glimpes into the abysses of memory always startled him.
  
  \ldots
  
  \ta{%
    Long and terrible was the way, lasting through the bloody centuries, since first the first men, risen from the mire of apedom, turned upon those who then ruled the world.
    
    And at last mankind conquered, so long ago that naught but dim legends come to use through the ages.
    The snake-people were the last to go, yet at last men conquered even them and drove them forth into the waste lands of the world, there to make with true snakes until some day, say the sages, the horrid breed shall vanish utterly.    
    Yet the Things returned in crafty guise as men grew soft and degenerate, forgetting anceint wars.
    Ah, that was a grim and secret war!
    Among the men of the Younger Earth stole the frightful monsters of the Elder Planet, safeguarded by their horrid wisdom and mysticisms, taking all forms and shapes, doing deeds of horror secretly.}
}

\target{Cordos began fighting Wylde}
Before Cordos, the world was a scary and monstrous place. 
Back then, all the world was \wylde. 
Cordos began the holy task of forcing back the \wylde and building up \human civilization in its place.
Therefore, it was the sacred duty of later Iquinians to carry on this grand work, for the sake of \humanity and the One Light. 

\citeauthorbook[p.57--58]{RobertEHoward:TheMirrorsofTuzunThune}{Robert E. Howard}{%
  The Mirrors of Tuzun Thune%
}{
  Gray fogs obscured the vision, grea billows of mist, ever heaving and changing like the ghost of a great river; through these fogs Kull caught swift fleeting visions of horror and srtageness; beasts and men moved there and shapes neither men nor beasts; great exotic blossoms glowed through the grayness; tall tropic trees towered high over reeking swamps, where reptilian monsters wallowed and bellowed; the sky was ghastly with flying dragons and the restless seas rocked and roared and beat endlessly along with muddy beaches.
  Man was not, yet man was the dream of the gods and strange were the nightmare forms that glided through the noisombe junlges.
  Battle and onslaught were there, and frightful love.
  Death was there, for Life and Death go hand in hand. 
  Across the slimy beaches of the world sounded the bellowing of the monsters, and incredible shapes loomed through the steaming curtain of the incessaint rain.
}





\subsubsection{Idolization of individuals}
\target{Cordos and Silqua in mythology}
\target{Silqua in Iquinian mythology}
In later Iquinian theology, Cordos and Silqua were highly exalted holy characters. 
They were the founders of modern mankind. 
Compare them to Adam and Eve from Judeo-Christian mythology.
Or Adam Kadmon from \Cabbalah.
Or Albion from William Blake's mythology.

Silqua was a saviour figure sent to give salvation to mankind.
But she was still fallible, and \hr{Iquinian myths about Silqua and sex}{sex was her weakness and downfall}. 

\citeauthorbook[\quo{First Thought in Three Forms}, p.86--100]{%
  BentleyLayton:TheGnosticScriptures%
}{%
  Bentley Layton%
}{%
  The Gnostic Scriptures%
}{
  But then for my part, I descended and got as far as chaos.\\
  And I dwelt with my own who were there, hidden within them, bestowing power and imparting image unto them.\\
  And down to the present [\ldots{}] those who [\ldots{}], i.e. the offspring of the light.\\
  It is I who am their parent. \\
  And I shall tell you a mystery that is ineffable and indescribable by any mouth.\\
  For you I loosed all the fetters and broke the bonds of the demons of Hades, bonds that were bound to my limbs and worked against them. \\
  And I threw down the high walls of the darkness,\\
  And I broke open the solid gates of the merciless and split their bolts.\\
  And the evil agency, who strikes you, who impedes you, the tyrant, the adversary, the ruler, the real enemy\dash as for all of these, I taught them about my own, the offspring of the light:\\
  So that they might become loosened from all these and rescued from all the fetters, and might enter the place where they had been in the beginning.\\
  It is I who am the first to have descended, for the sake of that part of me which remained, \\
  Namely, the spirit that exists within the soul and which has come to exist out of the water of lige and out of the baptism of the mysteries.\\
  I myself spoke with the rulers and with authorities, for I had descended deep into their language;\\
  And I uttered my mysteries to my own\dash a hidden mystery\dash\\
  And the fetters were loosened, as was eternal forgetfulness.\\
  And within them I bore fruit, namely, the thinking that concerns the unchangeable eternal realm (aeon) and my house and their parent.\\
  \ldots{} \\
  All who were in me shone bright.\\
  And for the ineffable lights within me I prepared an manner of appearance.\\
  Amen!
}

\target{Delphine in mythology}
Conversely, \hr{Delphine}{\Delphine} was reviled and seen as pure evil incarnated in \human form. 
Compare her to Lilith from Christian mythology, or the serpent in Paradise.





\subsubsection[Merkyrah]{\Merkyrah}
\target{Iquinian myths of Merkyrah}
The \Iquinian{} mythology went back to \Merkyrah. 
Here, it was a magical realm where mortals lived and walked side-by-side with gods and \hr{Iquinian angels}{angels}. 
And some unclear, conflicting, dark stories about something evil that crept in and poisoned their paradise. 

There was an \quo{original sin} of sorts. 
The Iquinians blamed themselves for the fall of \Merkyrah, in the same way that divorce children in RL sometimes blame themselves for their parents' divorce. 

A twisted version of the story of \hr{Thanatzil}{\Thanatzil} also featured. 
Allegedly, it was the sinful mankind's own fault that he they failed to be saved, even after his noble sacrifice. 

\citebandsong{DeathspellOmega:SiMonumentumRequiresCircumspice}{%
  Deathspell Omega
}{
  \Hetoimasia
}{
  For man [\nephil] is the key and man [\nephil] is the device\\
  And out of his ranks shall arise the saviour \\
  draped in the blood of the unborn
  
  For he will grow from child to man and extirpate\\
  souls in a devilish whirl from your cursed bosom.\\
  Fraught voices rise to the sky \\
  and beseech god to avert the incarnation\\
  But mankind was the prism to the quintessence of corruption
}





\subsubsection{\Qliphoth}
Some Iquinian Vaimons (especially those who accepted the \hr{Omnipotence of Iquin}{idea that \iquin was omnipotent}) believed that the \qliphoth had no life of their own.
They had to steal some essence from the One Light in order to live and exist. 
They were hollow, empty shells; undead things with the semblance of life but lacking any true essence. 

The \qliphoth were not pure evil.
They were animated by a core of stolen Light.
There was some good in them. 
Therefore it was OK to invoke the \qliphoth and use them to cast magic. 
(The \qliphoth were better than the heathen gods, which were of course completely forbidden.)

Some believed that the One Light, in its endless mercy and compassion, had allowed even these wretched half-beings to exist.
Not all \Telcras believed this, but they did, overall, believe in an \Iquin more forgiving than the \Iquin the Redcor believed in. 









\subsection{Philosophy}
\target{Iquinian theology}
\target{Iquinian philosophy}





\subsubsection{All are one}
The Iquinian church preached that \quo{we are all one}. 
All humanoids were part of \iquin, even though bound in fetters of \itzach.
They just had to realize it. 
Commoners could contact the rest of \iquin (and, thereby, other humanoids) through \hr{Iquinian prayers}{prayer} and mass. 
The Vaimons were wiser than commoners and had access to more knowledge. 
They learned how to break down the walls that separated them from the rest of the \iquin and thus draw upon the awesome power of \iquin to cast magic. 

In reality (the church said), all things and beings were one, made from the same indivisible One Light. 
All shapes and all individuality was a result of the \hr{Iquinian fetters}{fetters} that \itzach cast upon the world. 
The task of all living beings was to live in virtue and not sin.
Virtue would dissolve the fetters, but sin would forge new fetters. 





\subsubsection{Animal sacrifice}
\target{Iquinian animal sacrifice}
The \iquinians sacrificed animals \hr{Wylde totem sacrifice}{in order to keep their \wylde totems alive}. 
But unlike most religions they sacrificed only animals, not humanoids.
They supplied the rest of the energy via intense prayer.
Iquinians had a much more intense prayer schedule than most religions. 
They had to recite long prayers many times every day.





\subsubsection{\Atziluth}
\target{Atziluth}
\target{Kingdom of the One Light}
\Atziluth was the abstract \quo{place} where the \sephiroth dwelt.
And it was the place where believers hoped to go when they died.
They wanted to \quo{go into the One Light} and become one with \iquin. 

\Atziluth was also called the Divine Realm and the Kingdom of the One Light. 

\Atziluth could be reached through the \hr{Empyrean}{\empyrean}. 
It lay enclosed on all sides by the \empyrean and was, in a sense, a part of it. 
(Although the last part was a point of theological debate.)

A Vaimon could not move into the innermost circles of the \empyrean as long as he had a physical body.
The body fettered him to the lower, material world.
He could not become one with the \sephiroth, but he could touch them. 

\target{The Beyond horrible to Iquinians}
Iquinians believed that the true world, \Atziluth, which lay beyond \Gehinnom, was bright and pure and good and beautiful. 
Therefore it came as such a dreadful shock when they glanced into the Beyond or otherwise learned that there existed true worlds Beyond that were far worse than \Gehinnom. 
Worse even than \itzach.
The traditional image of \itzach was an \quo{Outer Darkness}; distant and abstract and unreal. 
The Beyond, once you got to know it, was very close, full of horrors awfully physical and slimy and smelly and slavering and \emph{real}. 





\subsubsection{Canon controversy}
The \VaimonClans had each their prophets and founding fathers who claimed revelations from the \sephiroth and laid down laws and commandments and spiritual \quo{wisdom}.
Some of these revelations were genuine.
Others were induced by \qliphoth or other immortals.
The \vclans did not agree on which prophetic works and scriptures were canon, so there was plenty of division in the Vaimon religion already in the time of the empire.

The immortals could not easily unite the empire.
First of all, there was the Unspoken Covenant to consider.
Second, the Sentinels and Cabal were always fighting and trying to fuck each other's plots up.
Third, each faction had plenty of infighting and disagreements, so how could they be expected to prevent the same thing from happening in the Shrouded Realms?




\subsubsection{Eschaton}
\target{Iquinian eschatology}
\target{Vaimon eschatology}
Iquinian eschatology held that there would come a Doom's Day, an Apocalypse, an Eschaton.
On this day, the \sephiroth would descend to \Miith and all fetters would be broken. 
All those souls that were found to be virtuous would be taken in and become one with the One Light.
All those sould that were found to be sinners would be cast down and bound in \itzach, where they would dwell in pain and chaos and suffering forever more. 

This was sort of true. 
It was planned that one day \hr{Lithrim}{\Lithrim} would come. 

In fact, I should have more myths and prophecies about the Advent of \Lithrim. 
Perhaps the Eschaton should be named \quo{the Advent}.
 
Some Vaimons believed the Advent was not a future event but the end goal for each individual's personal development. 

\hr{Lithrim was secret}{The truth about \Lithrim was a secret}, unknown even to most Cabalists. 
Most just thought it was a metaphor or abstraction. 

Compare to the \hr{Ortaican eschatology}{\Ortaican eschatology}. 
See also the general section about the \hs{Eschaton}. 





\subsubsection{Fetters}
\target{Iquinian fetters}
Iquinians believed that humanoids had \quo{fetters} of darkness that bound them to \itzach. 
These fetters were made of sin. 
All humanoids were born with fetters, for they were imperfect and flawed beings in a flawed material world. 
All every time they sinned, they forged more fetters. 
An Iquinian strove to break the fetters of darkness through \hr{Iquinian prayers}{prayer} and by following the virtues of the \sephiroth. 
Many people chose one \sephirah or a few, and then strove to emulate them and submit completely to the virtues they stood for. 

\target{Iquinian mercy}
When a person broke all his fetters, he would be free of \itzach and able to transcend into the One Light. 
But this was only possible for the great Vaimon saints like Silqua and Cordos. 
It was held to be impossible for regular people to break all their fetters. 
They were flawed creatures, after all.
But when a person died with any fetters left, he could still appeal to the mercy of the \sephiroth, and they might help him and bless him and free him of his last fetters so he could at last be one with the One Light. 
This would only happen to those who were truly faithful, though. 
The souls deemed unworthy would be cast out into the Outer Darkness to suffer forever in the chaos of \itzach.






\subsubsection{\Gehinnom}
\target{Gehinnom}
The physical world was held to be a consequence of \itzach. 
\Iquin represented unity, but \itzach represented diversity. 
\Iquin was the essence of all things, but it was \itzach that gave those things shape and thus made them into \quo{things}.

The physical world was Defiled because it was \hr{Iquinian creation myth}{created by the mingling of \iquin and \itzach}. 
This Defiled world was called \Gehinnom. 
All living creatures were part of \Gehinnom and were thus also themselves Defiled; impure, unworthy, lowly sinners. 

The Iquinians believed they had a sacred duty to purify and unmake the Defiled world of \Gehinnom. 
This work was called \hr{Tikkun}{\tikkun}.





\subsubsection{Numerology}
\target{Vaimon numerology}
The Vaimon alphabet had sixteen consonants.
Every \sephirah's name began with a different consonant. 
Thus the letter corresponded to the \sephirah and its virtue. 
Each consonant also had a number, the same as the number of the month of the corresponding \sephirah.
The consonants are, in order: 
\begin{enumerate}
  \item Shil
  \item Fir\footnote{Fir is variously romanized as F or Ph for aesthetic effect.}
  \item Ker\footnote{Ker is variously romanized as C, K or Q for aesthetic effect.}
  \item Raith
  
  \item Bal
  \item Chaid
  \item Sum
  \item Ten
  
  \item Zod
  \item Tzad
  \item Thul
  \item Mor
  
  \item Vod
  \item Luth
  \item Nuz
  \item Yith
\end{enumerate}


Other consonant sounds were spelled as \quo{voiced} versions of the base consonants, or as combinations of letters:
\begin{itemize}
  \item Voiced Tzad became J
  \item Voiced Chaid became H
  \item Voiced Ker became G
  \item Qu was spelled as Ker and the vowel U.
\end{itemize}


Each vowel also needs to have some meaning\ldots{}

Every person's name thus had a theological meaning. 





\subsubsection{Omnipotence of \Iquin}
\target{Omnipotence of Iquin}
Vaimon theologians disagreed over whether or not \iquin was all-powerful. 

Some Vaimons believed that \Iquin was all-powerful and could vanquish \itzach any day. 
\Iquin refrained from doing so because it was disillusioned and disappointed by the sin and wickedness of living creatures, who gave themselves over to \itzach when they should worship \iquin alone. 
Therefore \iquin allowed the Defiled world of \Gehinnom to exist.
People had to be good and really deserve it before \iquin would grant them salvation.
Those Vaimons who believed in omnipotence had a slightly more laissez-faire morality.
After all, in the end everything would be all right. 
The One Light would save the world. 
Some of these even believed that \iquin was all-knowing and that everything was predestined. 

Some Vaimons believed that \Iquin was \emph{not} all-powerful, that \iquin and \itzach were equal foes. 
These endorsed a stricter morality.
If \iquin was not all-powerful, then \itzach was a real menace and might even one day win.
Then it was every mortal's duty to do his utmost to fight evil in all its forms. 
The non-omnipotentialists saw the omnipotentialists as a threat because they embraced false complacency. 
The non-omnipotentialists all denied the idea of predestination. 

The issues of omnipotence and predestination were perhaps the two greatest points of contention in \iquinian theology.
Wars had been fought between \vclans (and civil wars fought within \vclans) over such religious differences.

The issue of omnipotence was also a point of contention among the \resphain who masterminded the Iquinian church.
Some of them believed the dogma of omnipotence was a great propaganda trick.
Others believed it was too outrageous a lie, and that the followers would not keep buying it. 

\ClanTelcra \hr{Telcras believe in omnipotence}{believed in omnipotence}. 
\ClanRedcor \hr{Redcor do not believe in omnipotence}{did not}. 
This was one of the reasons why they disliked one another, and one reason why \hr{Telcra is more popular than Redcor}{\ClanTelcra gained more popularity with the people}.

\paragraph{To Do:}
  Place each \vclan along these lines!
  What do the Redcor, Geicans and \Telcras believe?
  Remember that each \vclan may have changed its mind over the course of history.





\subsubsection{Prayers}
\target{Iquinian prayers}
Iquinian prayers were seen as a simple form of magic. 
Praying really changed the world. 
It brought the Divine Realm and the spiritual union closer. 

In prayer, even commoners could experience and feel the \hr{Shechinah}{\shechinah}. 





\subsubsection{Sex}
\target{Iquinian prostitution metaphor}
The \hr{Iquinian creation myth}{mingling of \iquin and \itzach} that created \hr{Gehinnom}{\Gehinnom} was a sexual thing. 
The innocent, virginal core of Light was raped and violated (in some versions, seduced) by the wicked, lustful Outer Darkness.
The Light became fragmented. 
Little specks of Light succumbed to seduction and became whores of the darkness.
These cosmic sluts made up the physical world and all living beings. 

Thus, \trope{SexIsEvil}{Sex Is Evil}. 

Every time a mortal sinned, he prostituted himself to Darkness for mere material gain. 

\target{Iquinian myths about Silqua and sex}
In fact, sex was so seductive and evil that even the noble Silqua was vulnerable to it. 
This was how the evil \hr{Delphine}{\Delphine} was able to seduce, torture and kill her. 
Even though \hr{Silqua in Iquinian mythology}{Silqua was \uber} and the divinity of the One Light was manifest in her, a part of her was still just a weak, foolish, sinful, horny woman that could be brought down by her sexuality. 
Even divinity could not overpower feminine frailty. 
(Notice the sexism?)

This proved how insidious sex was, and how much it should be feared and hated. 

\target{Redcor myths about Silqua and sex}
The Redcor did not accept this idea. 
\hr{Redcor feminism}{They were feminists}.
They believed that Silqua remained sexually virtuous till the very end. 
She was not seduced by \Delphine. 
She was forcibly kidnapped. 





\subsubsection{\Tikkun}
\target{Tikkun}
The Iquinians believed that the diverse physical world was an evil.
It was a false world, unlike the true world of the One Light (\hr{Atziluth}{\Atziluth}). 
The duty of Iquinians was to purify \Gehinnom and return it to its state of purity and oneness.
This holy duty to cleanse and thus unmake \Gehinnom was called \tikkun.
\begin{itemize}
  \item 
    They sought to purify their own souls through righteous living and thus bring themselves back to \iquin.
  \item 
    They sought to spread the true faith so that other souls would be freed from \Gehinnom and return to \iquin.
  \item 
    They killed heathens so that their sould would return to \itzach where they belonged, and the heathens would not be able to spawn descendants nor spread their evil beliefs, which would otherwise condemn more souls and more spritual matter to maintain \Gehinnom. 
  \item 
    They fought evil religions to prevent their evil gods from keeping Defiled souls bound in \Gehinnom. 
\end{itemize}

Animals were also Defiled, but they were not intelliget enough to be able to save themselves, so humanoids must do it for them.
By butchering an animal with the correct spells and prayers and then eating its flesh, its essence would be freed from \Gehinnom and returned to \iquin. 
So slaughtering and eating animals was a sacred, religious act, transforming Defiled matter into pure spiritual oneness. 
The eater would become \quo{one} with the animal, thus bringing the world a bit closer to complete oneness.

People would also say prayers before eating and drinking anything, even if not butchering it.
(Add this when Rian drinks with Dennick.)

It was part of the sacred duty of \tikkun to fight the \hr{Wild}{\wylde}.
The \wylde was a manifestation of Defilement.
It sought to encroach on the habitations of humanoids and create even more defilement.
It must be held back. 
The Iquinians also believed that taking materials (such as wood and metal) from nature and using them to build things (such as houses, tools and weapons) was a holy thing. 
The process took materials of pure Defilement from the \wylde and shaped them into things with a purpose, thus bringing them and the world closer to \iquin. 

\Tikkun was a continuation of the work \hr{Cordos began fighting Wylde}{begun by Cordos Vaimon} when he began \quo{conquering} the world from the evil monsters.

Similarly, every living person had the sixteen \sephiroth manifest inside him, but they were enclosed by \qliphoth. 
Only through prayer and forgiveness and living out the virtues could the \qliphoth be broken so that the person could become one with the \sephiroth. 





\subsubsection{Vision of the One Light}
The One Light had kept itself hidden since the Defilement. 
Only select holy individuals had been allowed to see with the vision of the One Light.
These included Silqua and Cordos. 
















\section{Redcor}
\target{Redcor}
\target{Redce}
\index{Redcor}
\index{Redce}
The Redcor are a \VaimonClan. 
Their homeland is \Redce, a theocratic nation in northeastern \Velcad{} ruled by \ClanRedcor.

\target{Conclave}
\index{Conclave}
The entire organization of \Redce{} and \ClanRedcor is based on the Iquinian religion. 
The Redcor are a matriarchal and matrilinear people, and their kingdom is ruled by a council of women, called the {Conclave}. The Conclave gather around the Topaz Throne in the capital city of \Redce{}. 

\index{\Redcean}
Note that the Redcor are the ruling class of \Redce{}. A common citizen of \Redce{} is a \Redcean{} (plural \emph{\Redcean{}s}). A believer of the Iquinian religion is called an \emph{Iquinian}. 









\subsection{Aesthetics}
The traditional \colour of \ClanRedcor is yellow. 
The symbol of \Redce{}, \ClanRedcor and the Iquinian Church is a yellow Sun on a blue background. 









\subsection{Culture}






\subsubsection{Clerical ranks}
The Clerics are scholars and priests and the spiritual and political leaders of the Redcor. 
The lowest ranked clerics are monks, bearing the title \emph{\frater} (plural \emph{\fratres}) if male or \emph{\soror} (plural \emph{\sorores}) if female. 
Male Clerics cannot rise above the rank of Frater. 
For female Clerics only, the higher ranks are \mater, \matron and \matriarch. 





\subsubsection{Government}
The Redcor are ruled by the Conclave, which consists of all \matriarchs{} (who are all equal in status). 





\subsubsection{Language}
The official language of \Redce is the \hr{Redcor language}{Redcor dialect} of the \hs{Vaimon language}. 
\hr{Velcadian language}{\Velcadian} is also spoken. 





\subsubsection{Social classes}
The Redcor Vaimons are divided into \clerics{} and \templars{}. 

Apprentices training to become Vaimons are called \neophytes.\index{\neophyte}





\subsubsection{\Templars: \Ryzin{} and Gandierre}
\index{Redcor!\templar}
\index{\templar!Redcor}
The \templars{} are warrior mages and the defenders and holy knights of the Redcor faith. 
They are split into two orders: 
The {\Ryzin}, for women, and the {Gandierre}, for men. 
%Many such knights are Redcor, but not most. 

Traditionally, the \Ryzin{} (unlike other \Redcean{} women) cut their hair short and wear trousers rather than skirts. 









\subsection{Geography}





\subsubsection{Climate}
\target{Redce climate}
\Redce{} is quite cold. 
There is snow much of the year. 
The northern edge of the country is tundra-like. 
There are some local Inuit-like peoples, not all of which are fully Iquinianized. 





\subsubsection{Demographics}
\target{Redcean demographics}
\Redce had a more \human population than most nations, but there were still plenty of \scathae.
The \scathae typically worked as labourers, farmers or soldiers. 
\ClanRedcor itself was strictly \human-only. 





\subsubsection{Rubellah, place of the first invocation}
\target{Redcor control Rubellah}
Clan Redcor controlled the rock of \hs{Rubellah}, a sacred site which was believed to be the {place where Silqua first invoked the \sephiroth}.
This might or might not be true.

The place attracted many Iquinian pilgrims. 

The Redcor very much wanted to control Castle \hs{Yeshimon} (the {place where Silqua died}), but this place was \hr{Zether control Yeshimon}{controlled by \ClanZether}.





\subsubsection{\TopazChateau}
\target{Topaz Chateau}
\index{\TopazChateau}
The central stronghold of the Redcor. 

The capital city is that surrounding the Topaz \Chateau. This city is very much divided into rich and poor quarters. Like \hr{Malcur rich and poor}{\Malcur}, but more extreme. The \Chateau{} is extremely tall, and the rich Redcor live infinitely far removed from the hardships, poverty and suffering of the common folk. 

The \Chateau itself had crystalline walls and looked as if it really was build of topaz. 
It was a beautiful and mysterious city of sorceresses. 
Compare to Kor-Avul-Thaa from the song \bandsong{Bal-Sagoth}{As the Vortex Illumines the Crystalline Walls of Kor-Avul-Thaa}. 

\target{Bane ruin under Chateau}
Unbeknownst to most, the \TopazChateau was built on top of an ancient \bane ruin, possibly a crashed space ship. 

\Belzir \hr{Belzir imprisoned under Chateau}{was imprisoned in these ruins}.









\subsection{History}
The founder of \ClanRedcor was Rebecca Redcor, daughter of \hs{Cordos Vaimon} and \hr{Silqua}{Silqua \Delaen}. 





\subsubsection{Fought against the \banes}
\target{Redcor fight Banes}
The Redcor were not complete tools and pawns. 
They were controlled by the Cabal to some limited extent and they did use \iquin, but the Cabalists did not have complete control over them.
The Redcor were also pulled in other directions by the Sentinels and \Kezerad, and some of them were intelligent enough to think on their own. 

The Redcor were one of the few mortal agencies on \Azmith that knew that the \banes existed. 
They knew the \banes were evil and fought to oppose and contain them. 

This explains why the Redcor were so dogmatic and secretive:

\begin{enumerate}
  \item 
    The Redcor leaders knew (or at least suspected) horrible secrets about the dark nature of the world and \humankind and the \sephiroth and \qliphoth. 
    These secrets must be kept secret at all costs.
    Hence they cultivated a dogmatic tradition where inferiors were discouraged from questioning their superiors.
  \item 
    Those who knew secrets forced themselves into a strict, dogmatic mindset in order to retain their sanity and not fall to chaos and madness and moral relativism.
  \item 
    The Cabal had introduced false dogmata and false beliefs in an attempt to mislead and manipulate the Redcor and prevent them from working against the Cabal agenda. 
\end{enumerate}





\subsubsection{\Belzir imprisoned}
\target{Belzir goes into Chateau}
\target{Belzir goes below Chateau}
\target{Belzir imprisoned under Chateau}
In the final days of the \VaimonCaliphate, \Belzir found the ruin and attacked the \Chateau. 
She went into the dungeons and explored the ruin, trying to learn its secrets and gain its power, but the Redcor tricked her and captured, killed and imprisoned her in the cellars below their \Chateau. 

The Redcor leaders knew that \Belzir was a Scion, but they kept this knowledge secret and hidden. 
It was a horrible thing to know, for their dogma said that Scions were things of good, associated with \iquin. 
And \Belzir was a hideous creature. 
She had never been truly \human. 
She was a hideous dark thing of \itzach that had once taken \human form, pre-\human horror sleeping beneath their city. 
So they believed.
And they were not completely wrong, for she was a \sathariah and a \malach, and well on her way to \hr{Azraid turns Malachim into Neo}{turning into a \neoresphan}. 





\subsubsection{\Belzir awakening}
\Belzir \hr{Belzir keeps in touch with Royalists}{kept in touch with her Royalists}.

Near the time of the \thirdbanewar, \hr{Belzir awakening}{\Belzir had gained power}. 
She was now able to reach out to the people living above her and contact them in their dreams. 

This was why \hr{Redcor need Carzain}{the Redcor needed Carzain}. 









\subsection{Philosophy}
\target{Redcor philosophy}






\subsubsection{Stern and ascetic}
\target{Redcor sternness}
The Redcor (\hr{Redcor do not believe in omnipotence}{who did not believe} in the \hr{Omnipotence of Iquin}{omnipotence of \Iquin}) preached doom and danger and asceticism, which was not popular. 
Their \iquin was stern and embattled. 

This philosophy was realistic in a sense, but \hr{Telcra is more popular than Redcor}{not as popular} as \hr{Telcra forgiveness}{the more lenient \Telcra beliefs}. 





\subsubsection{Feminism}
\target{Redcor feminism}
The Redcor were feminists.
They had their own myths about Silqua, where she was stronger and \hr{Redcor myths about Silqua and sex}{not susceptible to seduction}. 





\subsubsection{Free-thinking and heresy}
The Redcor believe that philosophy should be moral, clean, pure, structured, regulated, dogmatic and wholesome. Philosophy is a means of coming to a better understanding of the truths that are already revealed and known. 

Speculation into the dark, the unknown, the frightening is forbidden and considered heresy, a direct way to corruption and damnation. Evil thoughts are the way to evil deeds.





\subsubsection{\Iquin and \Itzach}
\target{Redcor do not believe in omnipotence}
The Redcor, \hr{Telcras believe in omnipotence}{unlike the \Telcras}, did not believe the \hr{Omnipotence of Iquin}{omnipotence of \Iquin}. 
They were more paranoid and ascetic than the \Telcras. 
They feared \itzach and never invoked it. 

This was one of the reasons why they disliked one another, and one reason why \hr{Telcra is more popular than Redcor}{\ClanTelcra gained more popularity with the people}.





\subsubsection{Non-rulership policy}
\target{Redcor do not rule}
The Redcor have as a policy that they do not try to directly rule other countries. 
To outward appearances they remain aloof from politics, but in reality they manipulate people like no tomorrow and wield great power behind the throne many places. 

They have been tempted to break this rule occasionally, but \sephirah{} brainwashing makes them obey it. 





\subsubsection{Prayer}
The Redcor believed in constant prayer and scripture study in an effort to strengthen the One Light and thus combat the Outer Darkness. 

\citeauthorbook[\quo{Angels Punishing the Wicked}, p.70]{%
  AlanUnterman:TheKabbalisticTradition%
}{%
  Alan Unterman%
}{%
  The Kabbalistic Tradition%
}{
  And know that through finding new meaning in the Torah spiritual forces are created from the letters of the Torah.
  These forces are literally \hr{Iquinian angels}{angels}.
  They receive power from Edom which enables them to punish the wicked with the sword and death 
  \ldots
  These spiritual powers, namely the angels, come about because of the renewal of the Torah through new interpretations.
  The renewal of the Torah relates to the holiness which is added above.
  According to this increase in Torah so there is an increase in the number of angels.
  The contrary is also true.
  Sometimes the holiness is so little that the angels that are created through the renewal of the Torah have diminished power.
  They do not have the ability to receive the power to punishe the wicked with the sword and death.
  They only have the power to suppress the wicked and to bring fear into their hearts, but not to punish them with the sword and to remove them.
}





\subsubsection{Technology taboo}
The Redcor were the strongest supporters of the \hs{Vaimon technology taboo}. 











\subsection{Politics}





\subsubsection{\Banes}
The Redcor \hr{Redcor fight Banes}{knew that the \banes existed and fought them}. 






\subsubsection{\ClanTelcra}
\target{Redcor hate Telcra}
The Redcor kind of hate the \Telcras. 
They look down on the junior \vclan as unruly children, upstarts unworthy of the Vaimon name. 















\section{\Telcra}
\target{Telcra}
\index{\Telcra}
\Telcra{} is a \VaimonClan. 
It is the youngest \vclan, the only one founded after the \darkfall. 
It is \Iquinian{} (in name at least) and makes up one of the two branches of the \hs{Iquinian Church} (the other being the Redcor). 









\subsection{Culture}






\subsubsection{\Clerics{} and \templars}
\index{\cleric!\Telcra}
\index{\templar!\Telcra}
\index{\Telcra!\cleric}
\index{\Telcra!\templar}
Like the Redcor, \ClanTelcra{} is split into \clerics{} and \templars. 
But there is a relatively large gap between the two. 
This separation was there already from the beginning, because the \baccons{} didn't want to church to become too powerful. 
The \baccons{} wanted to keep their potential competitors scattered, so they could control them. 
So they made sure the \Telcra{} \clerics{} were relatively peaceful and the \templars{} relatively secular. 
And the fracturing of the \vclan has just made it worse. 

The \clerics{} run and maintain the \Telcra{} Church and tend to be very \Iquinian. 
The \templars{} can be overtly anti-\Iquinian{} and use \itzach{} right and left. 
The \clerics{} disapprove of this. 
Many \templars{} are more-or-less mercenaries, working in the highest-bidding \ishrah. 





\subsubsection{Demographics}
\target{Telcra demographics}
\ClanTelcra{} was the only \vclan to have an almost fifty-fifty split between \human{} and \scatha{} members. 
In terms of species, \Telcra{} was by far the most egalitarian of all \VaimonClans. 









\subsection{Geography}





\subsubsection{Castle Yeshimon}
\target{Zether control Yeshimon}
\ClanZether controlled Caste \hs{Yeshimon}, a holy site that was believed to be the \hr{Silqua died at Yeshimon}{place where Silqua died}. 
The place attracted many Iquinian pilgrims. 









\subsection{History}
\target{Telcra founded}
The \vclan was founded during the time of \hr{Tepharae}{\Tepharae}. 
Originally, the \bacconate{} had close ties to the Redcor. 
But the Redcor religion was frightening, and the common folk were not happy about accepting it. 
They wanted to make themselves less dependent on the Redcor. 

The Cabal had the great idea to create a new \Iquinian{} Church that would be less magical, less mystical and more down-to-earth. 
An \quo{Iquinianism light}. 
They also needed some new Vaimons that were not controlled by the Redcor. 
So they founded a new \VaimonClan, the \Telcra. 

And it worked great. 
The people adopted it.

But the Cabalists were being deceived by Sentinels (who had, of course, infiltrated the order). 
The \Telcra{} Church is a much weaker grip on the populace than the Redcor Church would have been. 

In the beginning the \vclan was centrally organized. 
But after the dissolution of \Tepharae{} the \vclan lost its central organization and became scattered and fractioned. 
In the days of the \bacconate, all major imperial lords kept a small but very professional \ishrah{} of \Telcras. 
After the fall of the \bacconate, the \vclan fell apart and most were killed. 
(Why were they killed? And by whom?) 

Now there are only a few, scattered \Telcras{} left, and few lords can muster a decent \ishrah. \Malcur only has a makeshift \ishrah{} of one or two \Telcras, some chaos sorcerers and hedge wizards, and Carzain \Shireyo. 





\subsubsection{Rose to power}
\Tepharae, together with \hr{Telcra}{\ClanTelcra}, \hr{Tepharae succeeds Ortaica}{rose to prominence after the fall of \Ortaica}.





\subsubsection{Passing of the \Human Age}
In terms of species, \Telcra{} was \hr{Telcra demographics}{the most egalitarian of all \VaimonClans}. 

The founding of a \VaimonClan full of \scathae{} was a symptom/consequence of the passing of the \quo{\hr{Human Age}{\Human{} Age}}, and also one of the causes it. 

\target{Telcra integrates Scathae}
One of the great successes of \ClanTelcra was integrating the \scathae. 
They preached that the \hr{Vaimon Caliphate oppressed Scathae}{\human supremacist Vaimons of the past} had been unjust oppressors and heretics, and that in truth, \scathae were the equals of \humans. 
They demonized previous Vaimons and rejected the validity of the \caliphate (at least at first).
The Redcor were not happy about that. 




\subsubsection{More popular than the Redcor}
\target{Telcra is more popular than Redcor}
The \Telcras had more popular appeal than the Redcor.
The \Telcras (\hr{Telcras believe in omnipotence}{who believed} in the \hr{Omnipotence of Iquin}{omnipotence of \Iquin}) preached that the One Light would triumph and everything would be good.
Their \iquin was \hr{Telcra forgiveness}{gentle and forgiving and peaceful}. 

The Redcor (\hr{Redcor do not believe in omnipotence}{who did not believe} in the \hr{Omnipotence of Iquin}{omnipotence of \Iquin}) preached doom and danger and asceticism, which was not popular. 
Their \iquin was \hr{Redcor sternness}{stern and embattled}. 

The Redcor were more realistic, but the masses chose hope over realism. 









\subsection{Philosophy}






\subsubsection{Forgiving}
\target{Telcra forgiveness}
\Telcra philosophy was fairly benevolent and forgiving. 

This philosophy was unrealistic in a sense, but \hr{Telcra is more popular than Redcor}{more popular} than the \hr{Redcor sternness}{stern Redcor beliefs}. 






\subsubsection{\Iquin{} and \Itzach}
\target{Telcras believe in omnipotence}
\ClanTelcra believed in the \hr{Omnipotence of Iquin}{omnipotence of \Iquin} (\hr{Redcor do not believe in omnipotence}{unlike the Redcor}. 
They believed that \itzach was no real threat and would always be overcome by \iquin.
They spoke highly of the value and power of pure faith. 
Therefore, they felt it was safe to invoke \itzach. 
According to their scriptures, \itzach existed at the mercy of \iquin as its slave. 
It was thus also given to the Vaimons, bound by the supremacy of \iquin to do their bidding. 
Channelling \itzach was not without risks, of course, but sometimes it was needed, and times of need you had to do things that would otherwise be sinful. 
Killing, for example, was normally a sin, but if you were fighting a just war, killing was acceptable if not required. 

This was one of the reasons why the Redcor and \Telcras disliked one another, and one reason why \hr{Telcra is more popular than Redcor}{\ClanTelcra gained more popularity with the people}.




\subsubsection{Non-rulership policy}
\hr{Redcor do not rule}{Like the Redcor}, \ClanTelcra{} has a rule that forbids them to try to rule other countries directly.
Still, under the \hr{Tepharae}{\Tepharin{} \Bacconate} they held quite a lot of political power. 
But after that \bacconate{} crumbled, so did the central organization of \ClanTelcra. 

The Redcor dislike the \pps{\Telcras} power and tried to antagonize them. 
Since the collapse of the \Telcra, the well-organized Redcor have been able to keep the scattered \Telcras{} from gaining too much power. 
The two \vclans \hr{Redcor hate Telcra}{kind of hate each other}. 









\subsection{Politics}
The Redcor resented them and saw them as rogues, not a true \vclan. 















\section{Vaimons}
\target{Vaimon}
\target{Vaimons}
\index{Vaimon}
The Vaimons are an order of \human{} mages, founded somewhere around the year 1 \IC{} by \hr{Silqua}{Silqua \Delaen} and \hs{Cordos Vaimon} (the first \VaimonCaliph), after whom it was named. Vaimon magic is based on the twin forces of \iquin{} and \nieur, and spells are cast by invoking the various \Archons{}. 

The Vaimons previously ruled a \hr{Vaimon Caliphate}{\VaimonCaliphate}, but it fell in \yic{Darkfall} in what is called the \quo{\Darkfall}. 

\also{Cordos Vaimon, Silqua, Redcor, Geican, Quaerin}









\subsection{Aesthetics}
Each \VaimonClan has a traditional \colour. There is no red \vclan, because red was the traditional \colour of the Vaimons' enemies back at the time when the \vclans were founded. 





\subsubsection{Middle-Eastern theme}
\target{Vaimon Middle-East}
The \VaimonCaliphate should have a Middle-Eastern theme, with domes and minarets and stuff. 









\subsection{\VClans}
\target{Vaimon clan}

The Vaimons are divided into a number of autonomous \quo{\vclans}. 
Originally there were six \vclans: 

\begin{description}
  \item[{\Zether}:] 
    Founded by \Zether Vaimon, son of Cordos and Silqua and the second \VaimonCaliph.
  \item[\hs{Redcor}:] 
    Founded by Rebecca Redcor, daughter of Cordos and Silqua.
  \item[{\Delaen}:] 
    Descended from Silqua's brothers, Arcan and Lestor.
  \item[{\Irgel}:] 
    Descended from Kerzah \Irgel, Cordos and Silqua's younger son.
  \item[\hs{Geican}:] 
    Founded by Tiraad Geican, son of Cordos and one of his other wives.
  \item[{Quaerin}:] 
    Founded by Norcah Quaerin, son of Cordos by a third wife.
\end{description}

Perhaps there was also a clad Ephrad.





\subsubsection{\Belzir's time}
By \Belzir's time, \hr{Vaimon Clans at Belzir's time}{there were only four clans left}.





\subsubsection{\Thirdbanewar}
By the time of the \thirdbanewar, only the \vclans Redcor and Geican survived, alongside a newcomer \vclan:

\begin{description}
  \item[\hr{Telcra}{\Telcra}:] 
    The only \vclan to have been founded after the
 \hr{Hundred Scourges}{\darkfall}. 
\end{description}

The \vclans \Delaen, Quaerin and \Zether had died out long before the \thirdbanewar. 

There existed also \quo{rogue} Vaimons, owing allegiance to no \vclan.









\subsection{Culture}





\subsubsection{Demographics}
Most of the \VaimonClans were \human{}-dominated. 
Some, like \ClanRedcor, were purely \human. 
Others, like \ClanGeican, admitted \scathaese{} members. 

\ClanTelcra{} was the only \vclan to have an almost fifty-fifty split between \human{} and \scatha{} members. 





\subsubsection{Domestic animals}
See also the section on \hr{Domestic animals}{domestic animals}. 





\subsubsection{Language}
\index{Vaimon!language}
\index{Archaic Vaimon (language)}
\index{Ancient Vaimon (language)}
\index{Modern Vaimon (language)}
\index{Redcor!language}
The Vaimon tongue is descended from that spoken in \hr{Imrath}{\Imrath}. 
It was once the official language of the \hr{Vaimon Caliphate}{\VaimonCaliphate} and is still spoken in \Redce{} and used by the Redcor. 
Previously used as an intercultural \emph{lingua franca}, it has been replaced by \Velcadian{} in recent centuries. 

The term \quo{Archaic Vaimon} or \quo{Ancient Vaimon} is used to describe older forms of the language, as spoken in the \VaimonCaliphate. 
This is contrasted to \quo{Modern Vaimon}, as spoken in \Redce. 
Modern Vaimon is sometimes called \quo{Redcor Vaimon} or simply the Redcor language, but the Redcor frown upon this terminology, insisting that their tongue is the true Vaimon tongue. 

The Redcor dialect is meant to resemble French, but Archaic Vaimon is meant to sound like Hebrew. 





\subsubsection{Nudity}
\target{Vaimon modesty}
The \vclans had very different ideals of modesty. 
\ClanZether was one of the least taboo-ridden. 
They had a proud tradition of making all their statues naked. 

The Redcor were much more prudish.
Their statues were always clothed.

The \Telcras inherited the Redcor prudishness.





\subsubsection{Prayers}
\target{prayers against disease}
The Vaimons have prayers against disease. 
This is necessary, because the \hr{Parasitic Archons}{\Archons{} are parasitic} and spread disease. 
In war, they need these prayers to ward off disease to prevent their armies from dying from it. 
(This causes the disease to hit random civilians instead, but helps keep the army clean.) 





\subsubsection{\Templars}
\index{\templar}
\target{Templar}
A \templar{} is a Vaimon warrior-mage and knight. 
Today, only \ClanRedcor maintains an order of \templars{}. 





\subsubsection{Technology taboo}
\target{Vaimon technology taboo}
The Vaimons, especially the Redcor, had a taboo against science and technology.
They used it, but they opposed any innovation and research.
Research that challenged traditionally held views were taboo.
It reminded the Vaimons of \Belzir's heresy, and of heathenism and other wickedness.









\subsection{Equipment}





\subsubsection{Archon Ward}
\target{Archon Ward}
\target{Archon Wards}
\index{Archon Ward}
A type of magical \armour made by the Vaimons during the time of the \hr{Vaimon Caliphate}{\VaimonCaliphate}. An Archon Ward consists of a headband, a necklace and a number of bracelets. Together, when activated by a skilled Vaimon, these can form an energy shield surrounding the user. 

Archon Wards were extremely rare and expensive even in the empire. Several \VaimonCaliphs are known to have worn them, but barely anyone else could afford them. 





\subsubsection{Sabres}
\index{sabre}
The Vaimons traditionally wield sabres. 
\quo{Sabre}, in this case, is a loose term covering a wide variety of lighter swords, usually single-edged and curved. 

Sabres are not very effective against \armoured opponents. 
Here, a Vaimon would use magic. 

All the different martial (and magical!) arts practiced by the Vaimons have names. 

\target{chandre}
\target{chatresse}
A medium-heavy sabre is called a \chandre. 
The art of wielding it is \chatresse. 
This is where \VizicarDurasRespina{} excelled. 





\subsubsection{Technology}
\index{technology!\VaimonCaliphate}
\hs{Technology} rose during the \VaimonCaliphate's time. 
But that was not anything the Vaimons could take credit for. 
They did not invent guns and Archon Wards and everything. 
These things were imported through Cabal channels. 





\subsubsection{\Truesilver}
\target{Truesilver}
\index{\truesilver}
\Truesilver{} is a powerful metal from which some old Vaimon weapons are made. 
It is not a naturally occurring mineral but an artificial alloy, the technique of whose making has been lost. 

\Truesilver{} is almost as strong as \dragonsteel, but significantly lighter. 

Its known components are \dragonsteel, iron and silver. 
There must also have been some lighter metals involved, but they are unknown. 







\subsection{Politics}





\subsubsection{\Ortaicans and \rethyaxes}
See the section about \hr{Ortaican-Vaimon relationship}{\Ortaican-Vaimon relationships}. 





\subsubsection{Reputation}
In the \hr{Scatha Age}{\Scatha Age}, Vaimons were admired and feared. 
They were larger-than-life figures, touched by \Iquin (or worse, \itzach). 









\subsection{\VaimonCalendar}
\target{Vaimon Calendar}
\index{\VaimonCalendar}
\index{\VC}
The calendar of the old \hr{Vaimon Caliphate}{\VaimonCaliphate}, still used by Vaimons and in most of \Velcad{}. 
\quo{$n$ \IC} denotes year number $n$ in the \ImperialCalendar, counting from the year when Cordos Vaimon was crowned \caliph (year 1 \IC{}). 

A year is 380 days long. 
%where year 1 \IC{} was the year when Cordos Vaimon was crowned \caliph. 

The \ImperialCalendar has sixteen months, dedicated to the sixteen \Sephiroth{}, and each week has eight days, named after the Vaimon founders. 

The end of the year is celebrated with the festival of \Camaire{} on the last day of \Gamishiel{}, midways between the winter solstice and the spring equinox. 
\also{days (\ImperialCalendar), months (\ImperialCalendar)}





\subsubsection{\Camaire}
\index{\Camaire}
In the \VaimonCalendar, \Camaire{} is the festival that marks the end of the year. It falls on the last day of the month of \Gamishiel{}, midways between the summer solstice and the spring equinox.





\subsubsection{Days of the week}
\index{days of the week (\VaimonCalendar)}
\index{\Corjin}
\index{\Zetherab}
\index{\Rebecab}
\index{\Arcab}
\index{\Norquin}
\index{\Tirjin}
\index{\Kerzab}
\index{\Siljin}
In the \ImperialCalendar, a week is eight days long. Each day is named after of the Vaimon founders. 
The days, in order, are:

\begin{enumerate}
  \item \Corjin (after Cordos Vaimon, the first \VaimonCaliph).
  \item \Zetherab (after \Zether Vaimon, son of Cordos and Silqua and the second \caliph).
  \item \Rebecab (after Rebecca Redcor, daughter of Cordos and Silqua).
  \item \Arcab (after Arcan \Delaen, Silqua's eldest brother).
  \item \Norquin (after Norcah Quaerin, son of Cordos and another wife).
  \item \Tirjin (after Tiraad Geican, son of Cordos and a third wife).
  \item \Kerzab (after Kerzah \Irgel, younger son of Cordos and Silqua).
  \item \Siljin (after \hr{Silqua}{Silqua \Delaen}, Cordos' wife). 
\end{enumerate}
\also{\ImperialCalendar, Vaimon}





\subsubsection{Months}
\index{months (\VaimonCalendar)}
\index{\Atzirah!month}
\index{\Feazirah!month}
\index{\Keshirah!month}
\index{\Razilah!month}
\index{\Barion!month}
\index{\Hapheron!month}
\index{\Izion!month}
\index{\Teshiron!month}
\index{\Cushed!month}
\index{\Hoshied!month}
\index{\Thimared!month}
\index{\Yemared!month}
\index{\Gamishiel!month}
\index{\Ishiel!month}
\index{\Omariel!month}
\index{\Yeziel!month}

The \ImperialCalendar{} has sixteen months, dedicated to the sixteen \sephiroth{}.
The four months of spring are \Atzirah{}, \Razilah, \Keshirah{} and \Feazirah{}. 
The summer months are \Barion{}, \Teshiron, \Izion{} and \Hapheron. 
The autumn months are \Thimared, \Yemared, \Cushed{} and \Hoshied. 
The winter months are \Omariel, \Yeziel, \Ishiel{} and \Gamishiel. 

Each month is 24 days long, split into three weeks of eight days each (beginning with \Corjin{} and ending with \Siljin). 
The exception is \Gamishiel{}, the last month of the year, which is only 20 days long. 
(\Gamishiel{} is the \sephirah{} of Sacrifice.) 
















\section{\VaimonCaliphate}
\target{Vaimon Caliphate}
\target{Vaimon age}
\index{Vaimon!\VaimonCaliphate}
The \VaimonCaliphate existed from the year \yic{Founding of the Vaimon Caliphate}, where it was founded by \hs{Cordos Vaimon}, and until the \hr{Hundred Scourges}{\darkfall} where it fell, during the reign of \hr{Belzir}{\Belzir}. 

It spanned much of \Azmith, if not most. 

\target{Vaimon Caliphate naivete}
Back then, people believed that the \Sephiroth{} and \hr{Iquinian angels}{angels} (\resphain) were noble and good, having defeated the evil \pdaemons{} and being well on their way to vanquishing the last remnants of evil in the world. 

But the truth was slowly revealed to the learned during the reigns of the last few \caliphs, including \hr{Vizicar}{\VizicarDurasRespina}. But Vizicar still very much believed in the faith of \iquin{} and did not discover much, only a suspicion that something is lurking beneath the surface somewhere.

It all collapses under \Belzir, who learns far more than she should.









\subsection{Culture}





\subsubsection{\VaimonCaliph}
The \VaimonCaliph was both a secular and a religious authority. 
He was addressed \quo{Your Magnificence}.









\subsection{Geography}





\subsubsection{Rainbow Palace}
\index{Rainbow Palace}
\target{Rainbow Palace}
The royal palace of the \VaimonCaliph in \hr{Shiin-Merodar}{\ShiinMerodar}. 
Now destroyed. 





\subsubsection{\ShiinMerodar}
\index{\ShiinMerodar}
\target{Shiin-Merodar}
Once the capital city of the \VaimonCaliphate, seat of the magnificent \hs{Rainbow Palace}, \Merodar has now sunk beneath the sea. It lies somewhere between \hr{Vidra}{\Vidra} and \hs{Ontephar}. 

There are only some islands left of the big place that once contained \ShiinMerodar. 
The place was destroyed in a magical catastrophe during the \hr{Hundred Scourges}{\darkfall}. 





\subsubsection{\Vymorja}
\target{Vymorja}
\index{\Vymorja}
\Vymorja was a city in the \VaimonCaliphate that became the site of heresy.
The Vaimons of \Vymorja studied demonology and the forbidden sorcery of the \ophidians.
They translated a number of \ophidian spells into Vaimon terms so that Vaimons could call upon them. 

\Vymorja was razed to the ground by the Iquinians, but the wisdom of the \Vymorjans survived.

\target{Vymorjan Chants}
The most well-known \Vymorjan spells were the set known as the \Vymorjan Chants, used to summon and control \hs{Night-Feasters}. 
There were three chants. 
The first chant summoned a Night-Feaster.
The second prevented the monster from attacking the caster. 
The third chant compelled the Feaster to obey the caster's commands. 









\subsection{History}





\subsubsection{Heretics}
\target{Vaimon heretics}
During the time of the \VaimonCaliphate there were some Vaimon heretics who rebelled against the beliefs of the \caliphate. 
They believed that \humans{} were gods and merely had to release their inner divinity through embracing a life of chaos and blasphemy. 
Some of these mystics were actually half-mad \hs{Scions}. 

One of these heretics was \hr{Iolivine}{\Iolivine}. 

Compare to Aleister Crowley, and some of the philosophy in \authorbook{Graham McNeill}{Fulgrim}. 





\subsubsection{The \Darkfall}
\index{\Darkfall}
Perhaps the \hr{Hundred Scourges}{\Darkfall} coincided with a terrible, bloody war between different \resphan factions. 
Perhaps this was even the fall of \Kezerad. 

Inspired by the album \bandalbum{Symphony X}{Paradise Lost}.









\subsection{Philosophy}





\subsubsection{\Human supremacist}
\target{Vaimon Caliphate oppressed Scathae}
The \VaimonCaliphate was a \human supremacist culture. 
They repressed \scathae. 

The \scathae rebelled many times. 
\hr{Scatha rebellion under Belzir}{The worst such rebellion happened during \ps{\Belzir} reign}. 

One of the great successes of \ClanTelcra was \hr{Telcra integrates Scathae}{integrating the \scathae}. 















\section{Magic}
\target{Vaimon magic}
Traditional Vaimon metaphysics tells that the Universe is governed by two basic forces, named \Iquin{} and \Itzach. \Iquin{} is translated \quo{Light}, and is considered gentle and preserving, whereas \Itzach{}, translated \quo{Darkness} or \quo{Shadow}, is seen as aggressive and destructive. 

Each Vaimon \vclan has their own interpretation of \Iquin{}-\Itzach{} magic theory. 

The \hs{Redcor} (and the \hs{Iquinian Church}) insist that \Itzach{} should be translated \quo{Shadow}, because, according to their religion, \Iquin{} existed before \Itzach{}, and \Itzach{} is a corruption of \Iquin{}, something secondary that exists only at the mercy of the Light.\footnote{This is comparable to the role of the Devil in certain Christian interpretations: God and Satan are not equally matched adversaries. Rather, God is seen as all-powerful, and Satan, with all his evil, exists only because God, in all his good, allows him to. Supposedly, this somehow makes sense.} The Geicans, on the other hand insist that \Itzach{} should be translated \quo{Darkness}. To them, \Itzach{} is not a feeble reflection of \Iquin{}, but a primal force, possibly even more primal than \Iquin{}. 





\subsection{\Archons}
In Vaimon metaphysics, \Archons{} were supernatural beings and forces of nature. 
The \Archons{} included the \hr{Sephirah}{\Sephiroth} and \hr{Qliphah}{\Kliffoth}, who could be invoked to cast magic.  

The \Archons of Vaimon metaphysics were sometimes worshipped like gods, but they were not gods. 
Especially Iquinians prided themselves on their Archons being worlds apart from the vulgar, earthly gods of some religions, and would take offense to the \Archons{} being labelled as \quo{gods}. 

The gods of many religions literally walked the earth and could be encountered, whereas the Archons were disembodied forces that could manifest their power but had no physical form. 

The Iquinians teach that the \Archons are godlike beings to be revered and worshipped. 
They preach submission to the \sephiroth, transforming oneself into a vessel for their will and their work. 
But \ClanGeican teaches that, powerful though they may be, the \Archons are not true intelligent creatures, and they can be dominated. 
The Vaimon must exercise his will and master the \Archons, compel them to his bidding. 
A competent Vaimon knows which \Archons and feats are within his ability and which are beyond him, and he knows to keep his mind free of the influence of dangerous \Archons. 






\subsubsection{Abyss}
\target{Vaimon Abyss}
The Abyss was an abstract, mental \quo{plane}. 
It separated \hr{Atziluth}{\Atziluth} from the dwelling places of the \qliphoth. 

It was unclear from the theory whether the \qliphoth dwelt \emph{in} the Abyss or beyond it.
But in order to invoke the \qliphoth, a Vaimon had to mentally \quo{cross} the Abyss. 

According to \hs{Iquinian mythology}, the \sephiroth created the Abyss in order to limit the \hr{Iquinian creation myth}{defilement of the world caused by the invading \qliphoth}. 
The Abyss acted as a \quo{moat} around \Atziluth. 

In reality, the Abyss had to do with the borders separating \Miith from \hr{Erebos}{\Erebos}:
The \hr{Crystal Sphere}{\CrystalSphere}. 





\subsubsection{\Empyrean}
\target{Empyrean}
\index{\empyrean}
In Vaimon metaphysics, the \quo{\empyrean} was the \quo{plane} that the \Archons{} were said to inhabit. 
The \empyrean{} was not a physical \hs{Realm} but a mental abstraction. 

\hr{Atziluth}{\Atziluth} was a part of the \empyrean. 





\subsubsection{Exact definition}
What exactly constitutes an \Archon?
It clearly includes \sephiroth and \qliphoth.
But what about \hr{Iquinian angels}{angels}? 
And \malachim?







\subsection{Cost}

Vaimon magic is more subtle than Chaos magic and warps only the soul, not the body\ldots{} mostly.

\Iquin{} brainwashes the channeller with the \quo{virtue} that the \Sephirah{} in question represents. The more the mage channels the \Sephirah{}, the more its virtue will be ingrained in his mind. So people who channel Iquin a lot tend to become zealously devoted to the Iquinian ideology (if not to the Church itself). 

Some magic schools have spells that extend one's life. 
These spells may have nasty side effects. 
(An example of this is \hr{Life drain}{life drain}, which gradually turns the caster into an undead Reaver.) 

To cast \hr{Iquin}{\Iquin} magic you must pray to the \sephiroth, which binds your soul to them and \hr{Parasitic Archons}{steals moments of your life span}. But at least the \sephiroth{} do it in an aesthetic manner, so it doesn't show immediately. 

But wait\ldots{} \hs{Vaimons} live \emph{longer} than regular people. How does that work? 

It must be because the \sephiroth{} hoard the life energy of \emph{all} worshippers, and the Vaimons tap into that pool. So they're not paying the cost themselves, everyone else is. 

When you draw power from the \qliphoth, you strengthen the connection from \Miith{} through \hr{Nyx}{\Nyx} to \hr{Erebos}{\Erebos}. 
Also, you run the risk of the Cabal or Sentinels finding you and getting rid of you. 
(The Cabal is not fond of unauthorized \hr{Itzach}{\itzach} channelling.)

The more you draw on a \qliphah, the more you bind your soul to it and the more it infects and corrupts your soul and body.
Hence mages who use \itzach are often feeble and sickly and plagued with infirmity, physical and psychic.
The more different \qliphoth you use, the worse these effects become (but the more formidable you become as a mage, because you gain a diversity of powers). 

What do the \hr{Resphan}{\resphain} do? They are inherently vampiric and parasitic in nature and must drink the blood and life-force of others (typically humans) to power their magic and their physical strength. 

Perhaps some Cabalists \hr{Lictors}{degenerate into wretches} as well.





\subsubsection{\Itzach: Pain and horror}
\target{Itzach pain}
The \qliphoth{} of the Dawn Circle is relatively harmless. 

But \qliphah-magic of the Darker Circles is painful for mortals. 
Both physically and psychologically. 

It is psychologically hard for a Vaimon to call upon and channel a \qliphah{}. 
When her mind reaches out to the nether gulfs of \Itzach, it instinctively recoils in fear and loathing. 
She gets a natural urge to flee from this inhuman pit of primal black horror and never gaze upon it again. 
She has to steel herself and block out everything except her goal, in order to keep her sanity. 

Even for strong, brave men like Carzain and Vizicar, it is hard and taxing for body and soul. 

Moreover, when channelling a \qliphah, the Vaimon often feels an illusory pain that somehow resembles the effect, she is trying to create with her magic.   
She then has to be able to endure this pain and hold the concentration and channel the power outwards. 
If she succumbs to the pain and loses her resolve and focus, the \qliphah{} will break control and attack her body, which can cause wounds springing open, nasty inner bleedings, and even death. 

Some philosophers see this as a kind of karma: 
\quo{%
  A Vaimon should not seek inflict on others what she has not first endured upon her own body.}

\Itzach{} Vaimons have devised optimized methods to achieve their desired results while minimizing the pain and risk to their own bodies and sanity. 

When a Vaimon is in pain and channelling a \qliphah, once in a while he will hear voices in her head. 
These voices do not speak clear words, but they seem to vaguely hint of things. 
They promise to make the pain go away if the Vaimon will only give in to them and let them into his soul and body. 
This is a trap. 
Letting the voices in will only bring madness and make the Vaimon lose all control of his \qliphoth. 
It is unknown if the \quo{voices} are actually the voices of \qliphoth{} or just imagination, but it is well-known that they are up to no good. 





\subsubsection{Parasitism}
Vaimon magic is parasitic, because it is powered by \hr{Parasitic Archons}{the life force stolen by the \Archons}. 





\subsubsection{Stigmata}
\target{Vaimon stigmata}
When a Vaimon expended a lot of energy this would take a toll on his body. 
Wounds would form\dash inner and outer. 
Some of these wounds never fully healed.
They remained as scars, called \emph{Vaimon stigmata}. 
Through their lifetimes, Vaimons typically accumulated these stigmata. 
Older Vaimons tended to have more. 

These stigmata were a kind of \hr{Decrepity}{decrepities}.








\subsection{Demography}
\target{Vaimon demography}
There are \hr{Rethyax demography}{more Vaimons than \rethyaxes}. 









\subsection{How to cast it}
Vaimon magic uses mostly fast, simple \quo{point-and-click} spells: 
You simply invoke the names of one or more \Archons, mentally visualize the effect you want, and perform a simple gesture such as pointing your finger. 
%Stricly speaking, neither gesture nor speech are actually needed. The name is a way of contacting the \Archon{}, which is not necessary if you are experienced and finely attuned to Iquin or Nieur. (Or is it? Maybe the invocation is always necessary.) 
The gesture is not really needed. 
It is merely a psychological aid to help shape the spell, which is unneeded if you have fine control over the \Archons. 
The invocation of the name may or may not be needed, I'm not sure. If the spell is difficult or requires much power, you may have to keep chanting the name(s) over and over. 
Vaimons can also combine into a circle to perform ritual magic. Such a circle is much less flexible and slower to react than a single caster, but can cast more powerful and complex magic. 





\subsubsection{\Shechinah{} and meditation}
\target{Shechinah}
\target{resonance}
\index{\shechinah}
\index{resonance}
The \shechinah, also called \quo{resonance} is a state of being connected and \quo{attuned} to the \Archons. 
Vaimons achieve resonance through meditation or prayer, usually practiced daily. 

Vaimons who use \Itzach{} will, before all others, call upon the \qliphah{} \hr{Kor-Rashad}{\KorRashad} to guide them through the Empyrean. 
Then some \sephiroth. 
\KorRashad{} acts to counterbalance the influence of the brainwashing \sephiroth{}, who in turn protect the Vaimon from the mind-consuming \qliphoth. 
Then the Dark Vaimon will invoke \hs{Thaid} and \hs{Thuin}. 
Then some more \Archons. 

The more \shechinah{} a Vaimon has, the easier it is to cast magic, and the cheaper (measured in personal energy) each spell is, and the greater effects you can accomplish. 
But every spell drains some \shechinah. 
Every spell has a minimum \shechinah{} level required to cast it, and big, heavy effects likewise. 

But the higher your \shechinah{}, the greater power the \Archons{} hold over your mind. 

An old an experienced Vaimon builds up a store of permanent \shechinah{} that cannot be lost. 

Commoners could also experience and feel the \shechinah, such as in \hr{Iquinian prayers}{prayer}. 









\subsection{Pros and cons}
Vaimon magic is faster and easier and safer to cast than \rethyactic{} magic. 
Chaos magic is more powerful, but only with plenty of \trope{PrepTime}{\quo{Prep Time}} to cast complex spells. 
Chaos is more versatile, but also more dangerous and volatile. 

On a battlefield, a \rethyactic{} \ps{\ishrah} place is far behind the lines, casting big and slow and terrible spells to rain down on the field. 
A Vaimon's place is out there in the fray, blasting away at the frontline wherever he can do the most harm. 

The Vaimons also have their shielding \sephiroth{} (\Barion, \Hoshied, \Teshiron{} and \Yeziel) which provide good protection against physical and magical attacks, giving them a further edge against other mages in close quarters. 

A Vaimon must spend both personal energy and \shechinah{} to cast magic. 
\Rethyaxes{} do not need \shechinah. 
On the other hand, a \rethyax{} must make explicit pacts with the dark gods and sometimes do favours, solve tasks and even undertake quests for them. 









\subsection{Spells}





\subsubsection{Nasty combat magic}
Vaimons need to have some really nasty combat spells. 
Compare to this:

\lyricswikipedia{http://en.wikipedia.org/wiki/Homunculus}{Homunculus}{
  In the visions, Zosimos mentions encountering a man who impales him with a sword, and then undergoes \quo{unendurable torment}, his eyes become blood, he spews forth his flesh, and changes into \quo{the opposite of himself, into a mutilated anthroparion, and he tore his flesh with his own teeth, and sank into himself}\ldots{}
}





\subsubsection{Flight and jumping}
Vaimons cannot fly, but they can make great powered leaps.
See the section on \hs{Flying magic}. 









\subsection{Sanity}
A Vaimon's sanity is constantly in danger of being invaded and brainwashed away by the \Archons. 
The \sephiroth{} pull the mind in a \quo{virtuous}, conservative, bigoted direction, while the \qliphoth{} pull in a chaotic, violent, manic direction. 

The Geicans believe, therefore, that the Vaimon must balance the two opposing influences and find a golden mean. 

\lyricsbs{Aleister Crowley}{%
  Liber Aleph vel XCI
  (Chapter 13: 
   De Somnis ($\delta$) Sequentia/%
   On Dreams (d) Continuation)
}{
  If then there be a Traitor in that Consciousness, how much the more is it necessary for thee to arise and extirpate him before he wholly infect thee with the divided Purpose which is the first Breach in that Fortress of the Soul whose Fall should bring it to the shapeless Ruin that is Choronzon!
}

\lyricsbs{Aleister Crowley}{%
  Liber Aleph vel XCI
  (Chapter 15: 
   De Via per Empyraeum/%
   On Travel through the Empyrean)
}{
  And therefore is Confusion or Terror in any such Practice an Error fearful indeed, bringing about Obsession, which is a temporary or even may it be a permanent Division of the Personality, or Insanity, and therefore a defeat most fatal and pernicious, a Surrender of the Soul to Choronzon.
}





\subsubsection{Failed Vaimons}
\target{Failed Vaimons}
Some people who try to become Vaimons fail and are driven halfway or wholly mad by the \Archons{} coursing through their mind and body. 
\hr{Roanne Deracille}{\Roanne{} \Deracille} is one such. 









\subsection{Uses}





\subsubsection{Healing}
\target{Vaimon healing}
\hs{Iquinian} Vaimons are some of the world's best healers. They believe it is because \hr{Iquin}{\Iquin} is the force of Light, Life, Creation and all that is good. 

But in reality, as it turns out, it is because the \sephiroth{} have plenty of lifeforce to go around, because they actively \hr{Parasitic Archons}{drain the lifeforce of worshippers}. 
Every prayer to the Light steals minutes or hours from your lifespan and binds your soul more tightly to \iquin{}. 
It also steals intelligence, free thinking and free will and binds you more tightly in the Shroud. 
And when you die, the \sephiroth{} can, if need be, dismantle and gobble up your very soul and use it as fuel for healing magic. 

Vaimons lived as long as other people.
One might think they would have shorter lifespans because the \sephiroth drained their lifeforce. 
But the Vaimons themselves took something back from the \sephiroth as well. 
They unknowingly siphoned the lifeforce and lifespan of the masses of Iquinian believers to keep themselves alive. 









\subsection{Visualizing the \Archon}
\target{Visualizing Archons}
Each \Archon{} has a distinct synesthetic feeling to it. 
It might be that of flying high above the ground, or a deep jungle, or the grinding of rocks, or whatever. 
It might also vary from person to person. 

Dark Vaimons imagine \hr{Kor-Rashad}{\KorRashad} as proclaiming their philosophy. 

\lyricsbs{Emperor}{Thus Spake the Nightspirit}{
  Close your eyes and gaze into\\
  this realm that I reveal.\\
  See where eternities are born.\\
  Close your eyes, behold the powers\\
  of the broken seal.\\
  See the liars bound in thorns.
  
  Fear, and you shall fall.\\
  Weakness suffocates your will.\\
  Dare, yet never fail.\\
  Wisdom guides the one,\\
  the strong who can defy\\
  death.
}

\lyricslimyaael{427806}{%
  \textbf{3) Beckon the grotesque.} 
  
  I've wondered lately why descriptive passages on magic in so many fantasy novels do nothing for me anymore. 
  There are doubtless multiple reasons, but I think part of it is that, even when the authors are writing about destructive magic or evil inhuman creatures like the Unseelie Court, they describe the effects of magic as beautiful, or pretty. 
  That tends in the direction of fluff if the author isn't careful. If she is, it'll still call up very similar pictures from a lot of other fantasy books.
  
  I've been thrilled and felt wonder from descriptions of the grotesque, however. 
  I still haven't managed to finish \emph{Perdido Street Station}, but the descriptions of New Crobuzon, especially the beetle-headed khepri, are a lot more intriguing than yet another scene of moonlit pools and silver wolves and unicorns. 
  And my candidate for most awe-inspiring magical talent I've read about this year isn't the king-and-the-land magic in \emph{The Fall of the Kings}, although it was beautifully described. 
  It's the ability to grow cocoons on one's palms and hatch insects from them that I read about in \emph{The Etched City}. 
  I'm also enjoying the three brothers nested in each other like Russian dolls from \emph{Someone Comes to Town, Someone Leaves Town}, although that's been slow reading for other reasons.

  Many fluffy magical systems that blur into each other across fantasy books share common touchstones\dash\quo{beautiful} animals like horses and wolves, images of light from moon and sun, natural elements like water and fire that we've been trained to admire, brilliant \colours. Replacing even a few of those touchstones may lead to the sense of the strange, the weird, the alienness that we don't understand and recoil from. Insects, disease, filth, blood, and mutated and decaying bodies are much less often terms of fluffy magic. Try beckoning the grotesque into your magical system and see what it does.
}








































