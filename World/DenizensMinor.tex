\chapter{Gods and Aliens}















\section{\Archon}
\target{Archon}
\index{\Archon{} (plural \Archons)}
In Vaimon metaphysics, \Archons{} are supernatural beings and forces of nature. 
Different classes of \Archons{} include the \hr{Sephiroth}{\Sephiroth} and \hr{Qliphah}{\qliphoth}, who can be invoked to cast magic. 

The \hr{Malach}{\Malachim} are also considered \Archons by some. 















\section{Catachthonians}
\target{Catachthonian}
\target{Catachthonians}
The Catachthonians were a race of subterranean monsters, mighty and feared. 















\section{Cosmic gods}
\target{Cosmic gods}
\target{Cosmic god}
\target{cosmic gods}
\target{cosmic god}
\target{Klatrymadon}
\index{cosmic gods}
I need to have some immensely powerful, mystic forces. Enigmatic gods whose motives are unknowable, but whose names are invoked in spells, and who will sometimes answer. Kind of super-\Qliphoth. 

They should be invoked early on in the story, side-by-side with \ps{\Ishnaruchaefir} name, to give him mythical status. 

Compare them to Klatrymadon and Zuranthus, Kur'oc and Gul-kor from the Bal-Sagoth mythology.

\lyricsbs{Bal-Sagoth}{Summoning the Guardians of the Astral Gate}{
Ka-kur-ra, I summon thee.\\
Zul'tekh Azor Vol-thoth.\\
Mighty Xuk'ul, arise.\\
Kur'oc, Gul-Kor, come forth.

The threshold looms, \\
(the star-way between dimensions stretches before me\ldots{}) \\
The Gate To That Which Lies Beyond yawns wide\ldots{} \\
Unspeakable forces gibber and pulsate in the Outer Darkness\ldots{} \\
Elder horrors dwell here, things which were ancient and revelled in sublime galactic malevolence when even Xuk'ul was naught but a bloated cosmic maggot, writhing and suckling at the breast of its amorphous mother\ldots{} \\
They-Who-Lurk-And-Breed-In-Limbo\ldots{} \\
the squamous sovereigns of the elder void!}

Perhaps the cosmic gods have a Nyarlathotep-like figure who is their representative. 









\subsection{A class above the \xss{} and \voyagers}
The \xss{}, \banelords{} and \voyagers{} are well above \dragons{} and \resphain{} in power, but they remain within the bounds of imagination\dash for the master races, not necessarily for mortals. 

But the cosmic gods are a class above them. They are as far above the \xss{} as the \xss{} are above \dragons{} and the \dragons{} are above \scathae. 

Comparing with the Cthulhu Mythos by H.P. Lovecraft and others:

\begin{itemize}
  \item Cosmic gods correspond to Outer Gods.
  \item Voyagers correspond to Elder Gods. 
  \item \XzaiShanns{} correspond to Great Old Ones. 
\end{itemize}








\subsection{Creating the \banes}
Possibly, the \banes{} were twisted by the design of a cruel cosmic god. See section \ref{Cosmic god creating the Banes}.









\subsection{True Death}
According to some myths, some cosmic gods were among \hr{Dead Universe}{those enemies who slew the universe}. 
They were agents of True Death. 


















\section{\Daemons}
\index{\daemon}
\Pdaemons{} are physical creatures from \Machai{} or other \chaotic{} realms. 





\subsection{Mighty \daemonic{} races}
Have some mighty races of \daemons{} that can only be contacted and bargained with, rarely bound. Compare to the \quo{star-spawn of Cthulhu} in the RPG \emph{Call of Cthulhu}. 





\subsection{Bat-like \daemon}
Have some minor \pdaemons{} with a mouth filled with a broad row of dagger-like teeth\dash a truly wicked smile. Inspired by Clive Barker's \emph{Hell's Event} (from \emph{Books of Blood II}). 

Also, it has batlike ears. 















\section{Gods}
\index{gods}









\subsection{\Human{} gods}
Have a race of \human{} gods. 
They might be called the \quo{Titans} or something like that. 

They are actually descendants of the \Kezeradi. 
They may or may not know of their own origins, and they may or may not still have ties to \Kezerad. 

Daxian and Isxae are the foremost among their number. 
The Imetric god Eoncos is also one of them. 







\subsection[Scathaese gods]{\Scathaese gods}
And remember to have some \Ortaican{} and \Shurco{} gods. 














\section{Gods Beneath}
\target{Gods Beneath}
\target{God Beneath}
\index{Gods Beneath}
\target{Masters of Negation}
\target{Master of Negation}
\index{Master of Negation}
\target{Gods Beneath and Masters of Negation}
The \quo{Gods Beneath}, also called the \quo{Masters of Negation}, were a race of ancient and monstrous gods. 
They were horrid, terrible forces that dwelt in the indescribable churning chaos at the planet's core, where they had gnawed and brooded since before the Realms were formed. 
They were spawned by the \hs{World-God} of \Miith after \hr{Voyagers slay the World-God of Miith}{it had been slain by the \voyagers}. 

It might have been the Masters of Negation that destroyed the \hs{Lords of the Deep} and the \hr{Shugul}{\shugul}. 









\subsection{Demographics}
The \noggyal \hs{mother-mass} was a God Beneath. 
The amorphous god \hr{Ubloth}{\Ubloth} was a fledgling of their kind. 








\subsection{Politics}





\subsubsection{Lords of the Deep}
See the section about \hr{Lords of the Deep and Masters of Negation}{the Lords of the Deep and the Masters of Negation}. 





\subsubsection{\Noggyaleth}
See the section about \hr{Noggyaleth and Masters of Negation}{\noggyaleth and the Masters of Negation}. 





\subsubsection{\Shugul}
See the section about \hr{Shugul and Masters of Negation}{\shugul and the Masters of Negation}. 










\subsection{Skills and powers}





\subsubsection{Cycles of strength}
\target{Masters of Negation cycle in strength}
The Masters were kept in check by the Heart of \Miith, so they could only rise periodically.

The Masters were only powerful once in a while, when the \hs{Dark Heart} was beating strongly.
In these periods they would resurge to destroy life.
In other periods they would be weak and lay dormant. 















\section{\Krakens}
\target{Kraken}
\index{\kraken{} (plural \krakens)}
The \krakens{} are native \Miithian{} gods. 
Billions of years ago they ruled the Realm. 
Then they became tired and went dormant. 















\section{\Maskim}
\target{Maskim}
\index{\Maskim{} (plural \Maskim)}
A group of supernatural beings in Rissitic metaphysics, considered dangerous and evil and sometimes invoked in curses. 
\quo{\Maskim{} take you} is a strong curse in Rissitic.















\section{\Moongods}
\target{Moon-god}
\target{Moon-gods}
\target{Moongod}
\target{Moongods}
\target{moon-god}
\target{moon-gods}
\target{moongod}
\target{moongods}
The \moongods were a race of gods that hailed from the moon of Visha. 
In ancient times they held dominion over \Miith as well.
They were worshipped by the \hr{shugul}{\shuguls} and \hr{glithid}{\glithids}.









\subsection{Biology}





\subsubsection{Nature}
The \moongods were mindless and lifeless husks of horror.
They were composed of massive death and rot and decay.
They were fragments of the \hr{Dead Universe}{cosmic death}, like the \hr{XS}{\xss} but lesser. 









\subsection{Physique}





\subsubsection{Appearance on \Miith and on Visha}
\target{Moongods on Miith and on Visha}
On \Miith the \moongods only appeared as thin shadows of their true selves.
On Visha they are infitely worse.









\subsection{Politics}





\subsubsection{\Caisith}
The \caisith once \hr{Ophidians invade Visha}{tried to settle Visha}.
The \caisith fled when they encountered the \moongods. 





\subsubsection{\Umbrae}
See the section about \hr{Umbrae and Moongods}{\umbrae and \moongods}. 









\subsection{Skills and powers}





\subsubsection{Destroying them}
In all of \ophidian history no \moongod was ever known to have been destroyed.















\section{\Voyagers}
\target{Voyager}
\target{Voyagers}
\index{\voyager}
The \voyagers{} were the ones who created the \banes, and possibly also the \nephilim{} and other \Miithian{} life. 

Compare them to \bandsong{Bal-Sagoth}{Voyagers Beneath the Mare Imbrium}.

See also: \bandsong{Bal-Sagoth}{As the Vortex Illumines the Crystalline Walls of Kor-Avul-Thaa}. 

\lyricsbalsagoth{The Scourge of the Fourth Celestial Host}{
  They possess power unparalleled\ldots{}\\
  Ageless, remorseless. Without pity or conscience.\\
  Manipulators of evolution on countless worlds.\\
  Gods of the stars\ldots{} the Celestial Host!
}

The \voyagers{} are quite alien, but they are still the most \human{} of the ancient races. 

\lyricstitle{\authorbook{\HPLovecraft}{At the Mountains of Madness}}{
  \tho{Whoever these creatures had been, they were men!}
}

Everything on \Miith{} is descended from the \voyagers. This is \hr{Why the Banes want Miith}{why the \banes{} want \Miith}!

\lyricsbalsagoth{Invocations Beyond the Outer-World Night}{
  The legacy of the First Ones, spawn of the Mera!
}

\citeauthorbook[p.138--139]{RobertEHoward:QueenoftheBlackCoast}{Robert E. Howard}{%
  Queen of the Black Coast
}{%
  Cast in the mold of humanity, they were distinctly not men. They were winged and of heroic proportions; not a branch on the mysterious stalk of evolution that culminated in man, but the ripe blossom on an alien tree, separate and apart from that stalk. Aside from their wings, in physical appearance they resembled man only as man in his highest form resembles the great apes. In spiritual, esthetic and intellectual development they were superior to man as man is superior to the gorilla. But when they reared their colossal city, man's primal ancestors had not yet risen from the slime of the primordial seas.
  
  \ldots
  
  But the lethal waters altered them even more horribly, from generation to more bestial generation. They who had been winged gods became pinioned demons, with all that remained of their ancestors' vast knowledge distorted and perverted and twisted into ghastly paths. As they had risen higher than mankind might dream, so they sank lower than man's maddest nightmares reach. They died fast, by cannibalism, and horrible feuds fought out in the murk of the midnight jungle. And at last among the lichen-grown ruins of their city only a single shape lurked, a stunted abhorrent perversion of nature.
}









\subsection{Physique}
The \voyagers{} are alien, but somehow beautiful. At least to \humans. This is because \humans{} are created to serve the \banes{}, who in turn were created by the \voyagers{} and patterned themselves after them.

They are vaguely humanoid, but only vaguely.
%, shining white with great gossamer wings. 
They have a body and a head with some sensory organs on it. Then they have multiple limbs radiating in all directions, like a starfish. Four of these are bigger than the rest, and these sort of resemble humanoid arms and legs. Some of their limbs end in what look like gossamer wings.

Compare them to the Navigator from the \emph{Dune} TV miniseries.

Their wings extend into mystic dimensions Beyond and are only half visible\dash they shimmer in and out of the dimensions that mortals can see. 
The rest of their bodies can also \quo{disappear} into these obscure dimensions. 

Compare to the Mi-Go, as described in \emph{Delta Green} (\emph{Call of Cthulhu} RPG). 









\subsection{Arsenal}





\subsubsection{Servitor races}
The \voyagers{} had hordes of servitor races that dwelt with them and served them (different servant races in different aeons). Some envision this mythical time as a utopian age. 





\subsubsection{Technology}
\target{Voyager technology}
\index{technology!\voyager}
The \voyagers{} possessed ultra-badass technology.





\subsubsection{Ultima Thule}
Maybe I should have a mystic place far to the north, near \Miith{}'s North Pole, where old \voyager{} cities, millions of years old, are buried beneath the ice. 

The \nagalords{} know many of the place's secrets, but they are not telling. 
The \nagae{} know of the place and can be persuaded to lead people there, but they will not enter themselves. They know the danger. 

What is the danger? The \bladedpeople? Or shoggoth-like creatures, like in \authorbook{\HPLovecraft}{At the Mountains of Madness}.

\lyricsbalsagoth{In Search of the Lost Cities of Antarctica}{
  Beneath the ice, the endless ice \\
  of Pangaea's (now) axial (eternally frozen) frontier, \\
  entombed for countless millions of years\ldots{} \\
  the lost cities of Antarctica!
  
  Secrets locked within the ice, the endless ice of Antarctica,\\
  'Neath the peak of Erebus the First Ones sleep, Lords of Pangaea,\\
  Cities lost within the night, the frozen night of Antarctica,\\
  Pre-Cambrian, the Voyagers, beyond the stars, Lords of Pangaea.
}

Some people see dreams visions of \voyager{} cities in their golden age. These cities surpass \emph{everything} that has ever existed on \Miith{}. \emph{Except} the glorious and terrible dark citadels of the \hr{XS}{\xss}\ldots{} and the halls of the \hs{cosmic gods}. Have \Ishnaruchaefir{} comment on this. 

\lyricsbalsagoth{In Search of the Lost Cities of Antarctica}{
  Once, the coruscating spires of [the \voyagers] here offered their splendour to the heavens. \\
  Now, those spires gleam no more, \\
  save in dreams of verdant plains, \\
  save in dreams of time-lost citadels. \\
  
  Legacy of a utopia lost, \\
  forever enshrined 'neath the ice\ldots{}
  
  Before the Nine Continents were formed from Pangaea's shattered surface\ldots{} \\
  Hewn from the Pre-Cambrian rock, \\
  behold this primordial metropolis!
}

Also comparable to Bal-Sagoth's Ultima Thule. 









\subsection{History}





\subsubsection{False Life}
According to some myths, the \hr{Voyagers are False Life}{\voyagers were False Life}. 









\subsection{History}





\subsubsection{Origin}
Perhaps the \voyagers{} were in turn created by an even mightier, more ancient cosmic race.

\lyricsbalsagoth{The Fallen Kingdoms of the Abyssal Plain}{
  Long ago, before the third of Earth's moons fell fiery from the star-seared sky, there were those whom we have come to call the First Ones. \\
  These men-who-were-not-men were the creations of the Mera, beings from the far reaches of the limitless cosmos, whose essence still flickers latently within the minds of all their disparate progeny.\\
  Praise the Mera, fathers of the First Ones, bondsmen of the K'laa, sworn foes of the Z'xulth!\\
  Sired in the great spawning vats beyond the fathomless deeps of the Pre-Cambrian sea, the First Ones throve.\\
  Those who were engineered to live on land duly constructed the grand Antarctic Megalopolis, ultimately becoming entangled in bitter conflicts with the hoary Serpent Kings before retreating into the subterrene depths of the vast inner world, whereas those First Ones that had chosen the embrace of the abyssal seas were the architects of vast and glorious submarine cities whose splendid spires and minarets towered proudly beneath the unfathomed waves.
}





\subsubsection{Craters and wrecks}
I should have some great craters and wrecks, at the sites where the \psp{\voyagers} great spaceships crashed down on \Miith{}, tens of thousands of years ago. 

Mystic power radiates from these craters, creating vast maelstroms of energy. Monsters and undead spirits feed on this energy, and so throng around the craters.

This is inspired by the vulcanic vent that Laura Daughtery discovers in the first episode of the TV series \emph{Surface}.





\subsubsection{Voyagers today}
\target{Voyagers today}
A few \voyagers{} survive on \Miith{}. They dwell in desolate-but-functional high-tech citadels in otherwise ruined cities. 

Compare to \authorbook{\HPLovecraft}{At the Mountains of Madness}.

Some of them function as remote, powerful \hs{cosmic gods}. Compare to the Elder Gods of the Cthulhu Mythos. 









\subsection{Name}
The \voyagers were also called by some the \quo{First Gods}, because they were the first god-like intelligent beings to come to \Miith and begin creating stuff. 















\section{World-Gods}
\target{World-Gods}
\target{World-God}
\index{World-God}
The \quo{World-Gods} were a race of cosmic gods.
See the section about the \hr{History of the World-Gods}{history of the World-Gods}.
















\section{\XzaiShanns}
\target{Thzan-Tzai}
\target{Xzai-Shann}
\target{XS}
\index{\xzaishann}
The \xzaishanns{} were a race of alien monsters or gods. 
Some of the dwelt on \Miith, especially \hr{Machai}{\Machai}. 
They were creatures of immense power, possibly incorporeal. 
They were served by horder upon hordes of minor \mdaemons{} and \mdaemons.

They are inspired by the alien invaders from \bandsong{Bal-Sagoth}{As the Vortex Illumines the Crystalline Walls of Kor-Avul-Thaa}, and the Great Old Ones from the Cthulhu Mythos by H.P. Lovecraft and others. 

The \xss{} had many \quo{lords} but no \quo{king}, nor even much of an organization. 
They were creatures of Chaos, remember. 

Each \xs was seen as the god of some particular portfolio. 

Some \draconian philosophers speculated that when the different individual \xss seemed to have different powers and specialization (and came to be seen as gods of some \quo{portfolio}), it was perhaps not a result of the gods' actual powers, but rather their interests.
Perhaps \NerranKoss was just a philosophical god with an interest in history and such matters. 
So when people asked him such questions, he was more likely to yield a useful answer than most other \xss.
And thus he became seen as a god of occult knowledge. 
It was unknown if this theory was true, but it was accepted by several. 








\subsection{Names}
I should have a pantheon of the \quo{Eldest Lords of the \XzaiShanns}. Their names might include: 

\begin{itemize}
  \item Abraloth. 
  \item \hr{Khoth-Sell}{\KhothSell}. 
  \item \hr{Kyaethem Chrei Az}{\KyaethemChreiAz}. 
  \item \hr{Naath-Kur-Ramalech}{\NaathKurRamalech}. 
  \item Niil-shacht.
  \item Uruzgal. 
  \item Vol-croth. 
  \item Yoggranath. 
\end{itemize}

Other \xss{} include:

\begin{itemize}
  \item \hr{Hoth-Nrul}{\HothNrul}. 
  \item \hr{Ubloth}{\Ubloth}.
  \item \hr{Yolbaoth}{\Yolbaoth}. 
\end{itemize}

See the section on \hr{Individual XS}{individual \xss}. 










\subsection{Biology}





\subsubsection{Collective beings}
Perhaps each \xs{} is actually a collective being, a hiveminded colony of many smaller \daemons. 
Compare to \hs{coral reefs}. 





\subsubsection{True Life}
According to some myths, the \xss were \hr{XS are True Life}{some of the last sickly, twisted, degenerate descedants of the last remnants of True Life}.









\subsection{Psychology}





\subsubsection{Speech}
\target{XS speech}
The \xss{} speak weird. 
Maybe in poems or koans. 

This is the closest thing a mortal (or immortal) mind can come to a translation of the \ps{\xss} alien thoughts into familiar concepts. 
Even the minds of great, wise immortals like \QuessanthIshnaruchaefir{} are horribly inadequate. 

The \xss{} speak in raw Aenigmata, raw Gnosis. 










\subsection{History}





\subsubsection{Empire}
The \xss ruled a vast interstellar empire spanning thousands of planets, perhaps millions. 
They knew secrets of dimensional travel that few races had achieved.
The \banes did not have such a dimensional \travelling skill. 
This technology was one of the reasons why the \xss had grown to be such a successful and powerful race. 




\subsubsection{Dwelling in \Miith and Beyond}
Some \xss dwelt on \Miith itself, such as \hr{Ubloth}{\Ubloth}, who dwelt near \Yormis.
The Shroud made them even more sluggish and slothful. 





\subsubsection{Death and slumber}
\target{XS slumber}
\target{Dead XS}
%The \xss{} are sleepy and tend to slumber dormant for thousands, if not millions of years. Possibly for \hs{astrological} reasons. 
The \xss{} are sleepy. They \quo{die} and lie dead and dreaming for thousands if not millions of years at a time. 
Possibly for \hr{Astrology}{astrological} reasons. 

Have some mythical references to the \hr{Dragons worship dead gods}{\quo{dead gods} whom the \dragons worship}.

Compare to Cthulhu from \authorbook{H.P. Lovecraft}{The Call of Cthulhu}, who sleeps until \quo{the stars are right}. 

They were already dead when \hr{Sethicus contacts XS}{\Sethicus contacted them}, but they were dreaming, which allowed her to communicate with them.





\subsubsection{Worshipped}
The \xss{} are worshipped by the \dragons{} and their allies. 

\lyricsbs{Hate Eternal}{Praise of the Almighty}{
  Praise this massive force of hate. \\
  Praise the strength of thee. \\
  Worship all they have become. \\
  I am one with thee. 
  
  I await the powerful entities \\
  to enact my destiny. \\
  I anticipate to obliviate. \\
  I must summon thee.
  
  Praise the almighty ones. \\
  Praise the strength of thee. \\
  Worship all they have become. \\
  Become one with thee. \\
  Praise the old darkest ones, \\
  thy mighty force of thee. \\
  Worship all they have become. \\
  Succumb all to thee.
  
  In awe I gaze at absolution. \\
  Staring through the haze of the blackened skies. \\
  In awe I gaze at pure perfection. \\
  Wandering through the haze of the blackened skies. 
}









\subsection{\Theratons}
\target{Theraton}
\index{\theraton}
The \theratons were spawn and servitors of the \xss. 
They were not \hs{True Life}, for no new True Life could be spawned after the Great Death, but they were still mighty gods. 

Some \theratons were spawned on \Miith and thus closer in nature and mind to \Miithian creatures. 
These had more \human and understandable minds and motivations than their \xsic sires. 

Possible examples of \theratons include:
\begin{itemize}
  \item \HothNrul. 
  \item \Ubloth. 
  \item \Yolbaoth. 
\end{itemize}










\subsection{\XzaiShannic \pdaemons}
There exist several species of \pdaemons{} that served the \xzaishanns, and can now be forced to serve a Chaos sorcerer. 

One such species is mostly humanoid, but with a wide mouth filled with a row of dagger-like teeth, and great bat-like ears. 
Perhaps bat-wings, too. 
Like the monster in Clive Barker's \emph{Hell's Event} (\emph{Books of Blood II}). 























\chapter{Other Intelligent Races}















\section{\Aryoth}
\target{Aryoth}
\target{Aryothim}
\index{\aryoth}
An elder race of humanoid giants. 
Related to the \nephilim, but immortal and far more powerful. 


Compare to the Thelomen Toblakai from \cite{StevenEriksonIanCameronEsslemont:MalazanBookoftheFallen}. 









\subsection{Biology}





\subsubsection{Diet}
\target{Aryoth diet}
The \aryothim did not eat souls per se, but they did require large amounts of freshly killed animal flesh to \hr{Immortal voracity}{sustain their immortal lives}. 
Some religious rituals performed during the killing (akin to halal slaughter) helped and made the flesh more nutritious.

The \aryothim \hr{Aryoth religion}{built a lot of religion on this}. 









\subsection{Culture}





\target{Aryoth inventors}
\index{technology!\aryoth}
The \quiljaaran{} often looked down on the \aryoth{} and considered them uncivilized barbarians due to their warlike tendencies. 
But they were actually great inventors. 
Where \hr{QJ philosophers}{the \quiljaaran{} cared more about philosophy and science} for its own sake, the \aryothim{} were more interested in practically applied technology. 





\subsubsection{Magic}
\target{Aryoth magic}
Traditional \aryoth sorcery drew from cryptic sources, possibly the \hs{Gods Beneath}. 





\subsubsection{Religion}
\target{Aryoth religion}
\target{Aryothim worship Gods Beneath}
The \aryothim worshipped the brutal and primitive and primal gods \hr{Malgaddon}{\Malgaddon} and \hr{Yagnathul}{\Yagnathul}, formless horrors which the dug forth from the chaos of Elder Night. 

The evil \aryoth gods were monstrosities with giant mouths, like the Dreamland Sentinels in \cite{MichaelNelson:FallofCthulhuIII}. 

They \hr{Aryoth diet}{had to eat a lot}. 
This became a religious thing. 






\subsubsection{Seamanship}
The \aryothim{} \hr{Aryoth seamanship}{were often great sailors}. 





\subsubsection{Technology}
The \aryothim \hr{Aryoth technology}{developed quite a lot of technology}. 

\target{Aryoth weapons}
They were fond of guns. 
They invented many kinds of guns and would often use them. 

\Aryoth{} weapons were physical, enhanced with magic, unlike \hr{QJ weapons}{\quiljaaran{} weapons}, which were based on magic from the start. 









\subsection{History}





\subsubsection{Origin}
See the section on \hr{Origin of Aryothim}{the origin of the \aryothim}. 





\subsubsection{Relation to \resphain}
\target{Resphan-Aryoth relationship}
The \resphain were descended from \aryothim to some extent. 
\hr{Semiza has Aryoth blood}{\Semiza had \aryoth blood}.





\subsubsection{Legacy}
In later ages, when they were rare, the \aryothim were still remembered in legends as terrible pre-\human demigods that once walked the earth and did battle against the \dragons, those horrors of screaming chaos, and their slithering reptilian spawn. 









\subsection{Physique}
They looked like \nephilim, but bigger and stronger. 

A \aryoth{} was bigger and physically stronger than a \quiljaar. 
But the \quiljaaran{} typically had more powerful magic, which evened the fight. 

\index{beard!\aryoth}
A \aryoth{} face was \human-like, but bestial-looking to \human{} eyes. 
They had tusks like boars, and lots of hair and beard (for men). 
Their faces sometimes gave associations of a male lion with its mane. 

\Aryoth{} women had \hr{Nephil breasts}{as many breasts as \nephil women}. 





\subsubsection{Size}
\target{Aryoth size}
An \aryoth male was up to 300 cm tall and very heavily built (\hr{Resphan vs Aryoth size}{compared to a \resphan} or a \human). 











\subsection{Politics}
The \aryothim{} were often allied with the \vorcanths. 
They often fought against the \quiljaaran. 





\subsubsection{\Nephilim}
\target{Aryothim and Nephilim}
The \aryothim would often rule over \nephilim.
The \nephil subjects would worship the \aryoth overlords, but also fear them. 
The \aryothim were brutal ape-gods that ate \nephil flesh.
(Compare to the wicked ape-god in \cite{RobertEHoward:TheIsleoftheEons}.)

The \aryothim were more advanced than the \nephilim, but also in a way more primitive.
They were greater, more epic. 
Larger-than-life.
Both physically, intellectually, magically and in terms of passions. 
They were a fierce, primal, barbaric people.





\subsubsection{\Ophidians}
See the section about \hr{Ophidians and Aryothim}{\ophidians and \aryothim}. 









\subsection{Psychology}
They were tough and belligerent warriors. 

Where the \ophidians were driven by cold philosophy and science and culture, the \aryothim were driven by bestial ferocity and passions. 
They were fierce barbarians. 
Compare them to Robert E. Howard's barbarians in stories such as \cite{RobertEHoward:TheValleyoftheLost}. 









\subsection{Skills and powers}





\subsubsection{Warfare}
\target{Aryoth warfare}
The \aryoth were a warrior people and waged many wars, especially \hr{Aryothim kill QJ}{against the \ophidians}.
They wielded many powerful weapons of war. 

They could call upon their \hr{Aryothim worship Gods Beneath}{Tartarean, \chthonian patrons} and made the earth crack open, then summoned up the tentacles of \hr{Noggyal}{\noggyaleth} from the deep to drag their foes screaming down into netherworld abysses of torment and horror. 

They used the sorcery of \Yagnathul to cause their enemies' flesh to decay and rot on their bones.
This was especially effective against the \ophidians who relied so much on \hr{Living machines}{organic technology} that even their walls and cities were living things.

Above all else, \Aryoth warriors used the fury of \Malgaddon and fought with rage and hatred and brute force and grim weapons: Gatling guns, swords and maces and axes. 









\subsection{Ape-gods of the \aryothim}
\target{Ape-gods}
The \quo{ape-gods} were very powerful \aryothim. 
They were bestial, primal, ape-like. 
Compare them to the Black God from \cite{RobertEHoward:RedShadows}.

\citeauthorbook[pp.43--71]{RobertEHoward:RedShadows}{Robert E. Howard}{Red Shadows}{
  Thrum, thrum, thrum, came the ceaseless monotone of the drums: war and death (they said); blood and lust; human sacrifice and human feast! The soul of Africa (said the drums); the spirit of the jungle; the chant of the gods of outer darkness, the gods that roar and gibber, the gods men knew when dawns were young, beast-eyed, gaping-mouthed, huge-bellied, bloody-handed, the Black Gods (sang the drums).

  ...

  There in front of him loomed a shape hideous and obscene--a black, formless thing, a grotesque parody of the human. Still, brooding, bloodstained, like the formless soul of Africa, the horror, the Black God.

  ...

  Kane gazed at the scene almost impersonally. Again, somewhere in his soul, dim primal deeps were stirring, age-old thought memories, veiled in the fogs of lost eons. He had been here before, thought Kane; he knew all this of old--the lurid flames beating back the sullen night, the bestial faces leering expectantly, and the god, the Black God, there in the shadows! Always the Black God, brooding back in the shadows. He had known the shouts, the frenzied chant of the worshipers, back there in the gray dawn of the world, the speech of the bellowing drums, the singing priests, the repellent, inflaming, all-pervading scent of freshly spilt blood. 

  ...

  Motionless lay the two at the feet of the Black God, and to Kane's dazed mind it seemed that the idol's great, inhuman eyes were fixed upon them with terrible, still laughter.

  ...

  I am everlasting (Kane thought the Black God said); I drink, no matter who rules; chiefs, slayers, wizards, they pass like the ghosts of dead men through the gray jungle; I stand, I rule; I am the soul of the jungle (said the Black God).

  ...

  There is wisdom in the shadows (brooded the drums), wisdom and magic; go into the darkness for wisdom; ancient magic shuns the light; we remember the lost ages (whispered the drums), ere man became wise and foolish; we remember the beast gods--the serpent gods and the ape gods and the nameless, the Black Gods, they who drank blood and whose voices roared through the shadowy hills, who feasted and lusted. The secrets of life and of death are theirs; we remember, we remember (sang the drums).

  ...

  He thought at first it was some blasphemous mockery of a man, for it went erect and was tall as a tall man. But it was inhumanly broad and thick, and its gigantic arms hung nearly to its misshapen feet. Then the moonlight smote full upon its bestial face, and Kane's mazed mind thought that the thing was the Black God coming out of the shadows, animated and blood-lusting. Then he saw that it was covered with hair, and he remembered the manlike thing dangling from the roof-pole in the native village. He looked at Gulka.

  ...

  Yet the Black God still reigned, thought Kane dizzily, brooding back in the shadows of this dark country, bestial, blood-lusting, caring naught who lived or died, so that he drank.
}















\section{\Cuezcans}
\target{Cuezcan}
\target{Cuezcans}
\index{\Cuezca}
\index{\Cuezca!\Cuezcan{} race}
The \cuezcans{} saw their civilization destroyed by the war between the \dragons{} and \banes{}, and they hold a grudge. 
They work to play the Cabal and Sentinels against each other, killing and destroying as much of each other as possible. 
In the end, the surviving \cuezcans{} want to revive their empire and destroy the Shroud. 

The \cuezcans{} were a race of feather-clad \saurians{}, closely related to \nycans. 
At the time of the \thirdbanewar, there were still close ties between the surviving \cuezcans{} and (certain) \nycans. 

\Cuezcans{} and \nycans{} were always tied together, but they fell out and grew apart near the end of the time of \Cuezca, before the Apocalypse. 
Most \nycans{} \hr{Nycans forgot Cuezcans}{since forgot the \cuezcans}. 















\section{\Glithids}
\target{Glithids}
\target{glithids}
\target{Glithid}
\target{glithid}
\index{\glithid}
The \glithids were a race of monstrous humanoids. 

Compare them to the ghouls or ghasts in the Cthulhu Mythos, in stories such as \cite{HPLovecraft:PickmansModel} and \cite{HPLovecraft:TheDreamQuestofUnknownKadath}.









\subsection{History}
They were bred in ancient time by the \hs{shugul}{\shuguls} to serve them as a slave race, millions of years before the \banewars. 
They were created on \Miith, engineered from some ancient primates (cousins of those primates that would later evolve into \nephilim). 

After the \hr{Ophidians drive out Shugul}{fall of the \shugul civilization}, the \glithids escaped into the \wylde and declined even further into savagery.
They now dwelt in the \wylde where they gibbered underneath the moons and gave obeisance to the grotesque \hs{Moon-god}{\moongods}. 









\subsection{Demographics}





\subsubsection{Lurking in the Beyond}
\Glithids lurked in the Beyond even in civilized places and cities. 
In places and times where the Shroud was thin, they could break through and attack humanoids.

Compare them to the azghouls from the RPG \cite{RPG:Kult}. 









\subsection{Name}
Also called \quo{gibbering ones}.









\subsection{Physique}





\subsubsection{Appearance}
A \glithid somewhat resembled a \human.
Its hairless skin was pale gray or brown.
They were thin and wiry.
A typical \glithid was smaller than a \human, but a few were very large. 





\subsubsection{Sounds}
The name \quo{\glithid} derived from the gibbering sounds they made, which sounded like \hypota{glau-glau-glau}, or with a bit of imagination \hypota{glith-glith-glith}. 









\subsection{Poltics}





\subsubsection{Sorcerers}
\Glithids were some of the horrors that dwelt in the \wylde and the Beyond. 
They could be summoned and compelled into service by sorcerers.
It was comparatively easy to enslave a \glithid because they had been designed as a slave race from the beginning.









\subsection{Psychology}
The \glithids were once highly civilized creatures that mastered the arts and the sciences, but they since declined into bestial things that could only howl and gibber. 
















\section{\Gnomphil}
\target{Gnomphil}
\index{\gnomphil}
The \gnomphilim were a race of simian \humanoids related to \nephilim.
They were of about the same size as \nephilim, but farther from \human form.
They had long fur and long, baboon-like faces (but no tail).
They had tribes and Stone Age technology.

They were well-adapted to cold climes.
They lived mostly in \UltimaThule and similar arctic regions, where the weather was too cold for the less hardy \humans and \scathae.

They worshipped \xss godlings. 

Compare them to the Gnophkeh from the Cthulhu Mythos. 















\section{\Human}
\target{Human}
\Miithian{} \humans{} are like Earth \humans{}. 
They are widespread especially in \Velcad{} and in the Far North. 

\Humans{} were originally created as a slave race by \Semiza-tachi. 
The \hr{Humans fail}{experiments failed}, and the test subjects were supposed to be killed off. 
But after \Thanatzil{} was slain the \humans{} escaped into the wild and bred true. 
The \resphain{} would later rediscover them and adopt them as their servant race. 









\subsection{Name}
As in English. The associated adjective is \emph{\human{}}. 





\subsection{Physique and metaphysique}
\subsubsection{Ethnic groups}
There are a number of races/ethnic groups of \humans{} on \Miith{}, who correspond to various ethnic groups on Earth. These include: 

\index{\Velcadians{} (race of \humans{})}
\index{Fraens{} (race of \humans{})}
\index{Kohons{} (race of \humans{})}
\index{Hoyds (race of \humans{})}
\begin{description}
  \item[{\Velcadians}] 
    have pale skin and look like the Germanic people of northern Europe (Germany, England, Scandinavia). They are the most common \human{} race, dominating most of \Velcad{} and the Northern Kingdoms. 
  \item[{Fraens}] 
    have somewhat dark skin and resemble people from southern Europe (Greece, Italy). They are common in the Imetrium, \Durcac and southern \Velcad{}. 
  \item[{Kohons}] 
    have dark brown or black skin and negroid features (corresponding to Africans/Negroes). They are common in \Durcac and the Far South. 
  \item[{Hoyds}] 
    have brown skin and look like Arabs. They are common in southeastern \Velcad{} (nations such as Geica) and the Orient. 
\end{description}





\subsubsection{Size}
An average \Miithian{} \human{} man is 170-175 cm tall and weighs 70 kg. A woman is 165-170 cm tall and weighs 55 kg. 





\subsubsection{\Humans suck}
\target{Humans suck}
\target{Humans are a failed slave race}
Being a failed slave race, \humans{} are sucky and measly creatures, and few of them are worth anything. 

At the time of the \hs{Murder of the Dawn}, \humans{} really sucked. 
They were smaller, weaker and more stupid than \nephilim. 
In lots of places they were the \ps{\nephilim}{} subordinates or slaves. 

The \emph{only} thing \humans{} had going for them (compared to \nephilim) was fertility. 
\Humans{} have $50\%$ women, so they breed much faster and can spread more. 

After the \resphain{} returned to assume control of the \humans, they have worked to improve them, by breeding and occult scientific experiments. 
Their efforts have met with some success, and the \humans{} of today are far better than the ones from \ps{\Merkyrah} time. 
They are still somewhat pathetic, though. 
Physically weaker than \scathae, and less well-organized. 
But more aggressive, and perhaps slightly more fertile. 

\lyricsbs{Monolith Deathcult}{Origin}{
  Risen from the seed of Enki\\
  with consent of the God-father Anu.\\
  We are the working race,\\
  created in the temples of SCH.RUPPAK.
  
  And God-slaves we are,\\
  suffering in the mines of our Masters.\\
  We populate the fields of E.DIN,\\
  the base of the Gods from Nibiru-Pha\"eton.
}

They have no culture of their own. 

\lyricsbs{Monolith Deathcult}{Deus Ex Machina}{
  Your history is not yours. \\
  I gave thee wisdom. \\
  I gave thee science \\
  and I delivered thee from bestiality.\\
}





\subsubsection{Purpose}
Unbeknownst to all, \hr{Purpose of Humanity}{\humanity{} had a dire purpose}. 









\subsection{Biology}
The average lifespan of a \human{} is 70 years, 90 in exceptional cases. 

\target{Humans are fertile}
\Humans{} are very fertile. 
More so than \scathae{} and \nephilim. 
This has allowed them to multiply a lot. 
It was probably what ensured their survival after \hr{Thanatzil dies}{\ps{\Thanatzil} fall}. 

\Humans{} can interbreed with \nephilim. 
The \human{} genes are dominant, so any children will tend to be predominantly \human. 





\subsubsection{\Demihumans}
\target{Demihuman}
There existed a number of \quo{\demihuman} races. 

\Demihumans were far more widespread before the \VaimonCaliphate. 
Back then they were not \quo{\demihumans}, but simply different variants of \humans.
There was no one race that was considered \quo{real} \humans, of which the others were deviations. 

The racist Vaimons exterminated many of the other \human races and subjugated or drove away the rest. 
Later, the race which the Vaimons represented came to be thought of as \quo{real} \humans and all the others as \demihumans. 

On other Realms than \Azmith, other \human variants dominated.

Have plenty of \demihuman slaves in Pelidor and stuff. 
Perhaps nobles (such as the Rungeran court and \ishrah) keep hot, exotic \demihuman girls as sex slaves. 

Races included:

\begin{gloss}
    
  \gitem{Standard \humans}
  \target{Men of Light}
    Standard \humans were called \quo{\truehumans}. 
    They were the \quo{Men of Light}, the Vaimon race, descended from Cordos Vaimon and Silqua. 
    More than any other \humans, they carried the genes inside them that connected them to the \hr{Lithrim}{\Lithrim} \matrix.
    
    The Men of Light \hr{Men of Light created}{originated on \Azmith}.
    Over the centuries, the Cabal slowly and covertly disseminated Men of Light to all other Realms where \humans lived, so that they would interbreed with the locals. 
    
    In time, the Men of Light interbred with other \demihuman races. 
    The \Lithrim gene was very dominant and quickly spread. 
    So in the days of the \thirdbanewar, almost all \humans carried \Lithrim inside them, even though they did not all look like Men of Light. 
    Only a few isolated \human bloodlines were \hr{Clean Humans}{clean}. 
    
  \gitem{\feere}
  \target{Feere}
    \Feeres resembled \sheomirs but were covered in fur all over.
    They were considered beautiful, in a bestial sort of way.
    They were mostly subjugated and kept as slaves. 
    They were more common than \sheomirs.
    
  \gitem{\glune}
    \Glunes were hairy, ugly and stupid.
    They were as large as True Men (where most \demihuman races were smaller). 
    They were much dumber than most \humans and mostly used as slave labour. 
    \Glunes did not survive well on their own, but they throve when ruled by other \humans.

  \gitem{\sheomir}
  \target{Sheomir}
    \Sheomirs had a fox-like furry tail and a stripe of fur running down the spine. 
    They were very rare in \Velcad, where they were often kept as exotic slaves (possibly sex slaves). 
    They were more common in the Imetrium.
    They are inspired by the Sheovins from \cite{NykiBlatchley:KaydanaandtheStaffofIshlun}. 

  \gitem{\tulan}
  \target{Tulan}
    \Tulans had skin with a reddish complexion, slightly sharp teeth, a pointed nose and face, and (for men especially) facial hair that looked a bit like whiskers. 
    Racists said they looked like rats. 
    They were the most common minority in Pelidor, where they were a subjugated lower class. 
    
    \hs{Rian} was a \tulan (as was Neina and her family).
 
  \gitem{Others}
    Some had tails or wings or horns. 
    
    Some had exotic skin \colours. 
    Some were stripy like a zebra. 
    \hs{Evith} was one of these. 

    Some \demihumans had tails like \nephilim.
    
    Some had feathers like \resphain.
    
    A few rare breeds even had wings like \resphain. 
    They could perform powered leaps with the wings, but they could not fly. 
    And the wings were fragile and did not regenerate. 
    Still, the winged \humans were considered by the \resphain to be the highest of all \humans because they resembled resphain. 
    
    Evith was a winged \demihuman. 
    And striped like a zebra. 
\end{gloss}




There were also \quo{\hr{Demiscatha}{\demiscathae}}.
The \scathae were more tolerant of their kin than \humans were. 





\subsubsection{\Resphan experiments on \humans}
The \resphain performed \hr{Resphan experiments on Humans}{experiments on \humans}. 









\subsection{Politics}





\subsubsection{Anti-\human racism}
\target{Anti-Human racism}
A few individuals began to really hate \humans after they learned that all \humans were connected to the loathsome \Lithrim. 

See also the sections about \hr{Anti-Scatha racism}{anti-\scatha racism} and how \hr{Scathae and Humans hate each other at first}{\scathae and \humans hated each other at first sight}. 

\citeauthorbook[p.231]{RobertEHoward:ChildrenoftheNight}{Robert E. Howard}{%
  Children of the Night%
}{
  But this changeling, this waif of darkness, this horror who bears the noble name of Ketrick, the brand of the serpent is upon him, and until he is destroyed there is no rest for me.
  Now that I know him for what he is, he pollutes the clean air and leaves the slime of the snake on the green earth.
  The sound of his lisping, hissing voice fills me with crawling horror and the sight of his slanted eyes inspires me with madness. 
  
  \ldots 
  
  And as my ancestors\dash as I, Aryara, destroyed the scum that writhed beneath our heels, so shall I, John O'Donnel, exterminate the reptilian thing, the monster bred of the snaky taint that slumbered so long unguessed in clean Saxon veins, the vestigial serpent-things left to taunt the Sons of Aryan.
  They say the blow I received affected my mind; I know it but opened my eyes.
  Mine anicnet enemy walks often on the moors alone, attracted, though he may not know it, by ancestral urgings.
  And on one of these lonely walks I shall meet him, and when I meet him, I will break his foul neck with my hands, as I, Aryara, broke the necks of foul night-things in the long, long ago.
}





\subsubsection{\Scathae}
\Scathae and \humans \hr{Scathae and Humans hate each other at first}{hated each other when they first met}.
After some centuries they slowly learned to accept one another. 









\subsection{Psychology}
Mankind is very Shrouded, \naive{} and stupid. 
Not quite as well-meaning and {innocent} as the \scathae, but not all that far from them.
That is why it is so effective when \hr{Lithrim arises}{their inner \Erebean{} darkness is finally released in the culmination of the \Morbus{} plan}. 





\subsubsection{Racial memory}
\target{Human racial memory}
\Humans possessed some degree of racial memory. 
This was because they were all, to some degree, connected to each other via \Lithrim.
Those who actively worshipped \Iquin were bound tighter to \Lithrim, but all other \humans were also tied to the loathsome god. 
Occasionally they would receive vague memories and emotions from the collective subconscious that was \Lithrim. 









\subsection{\Shapens}
\target{Shapen}
\index{\shapen}
\quo{The \shapens} was the term for \humans that had been reshaped into new forms by their \resphan masters to better serve specific purposes. 
The \resphain's citadels were full of \shapen servitors, each made for a single purpose.

The \resphain preferred to use \shapen rather than machines. 

Compare to the \quo{servitors} in \cite{GrahamMcNeill:Mechanicum}.

There were also \resphan \shapen. 
They were called \hr{Morphous}{Morphous}. 
They included the \hs{High Telepaths}.















\section{\Jinni}
\target{Jinn}
\index{\jinni}
The \jinn were a race of immortals.
They predated \humans and lived, among other things, in the lands southeast of \Durcac. 
\Secherdamon's cult encountered the \jinn and the humanoids who served and worshipped them, and merged with them.
He waged wars with the \jinn, but eventually struck alliances with many of them, and eventually came to rule over them.
The lords of the \jinn and the greatest among them were the ifrits.
Still, some \jinn opposed Nechsain and would not serve him. 
These \quo{evil} \jinn were horrors that haunted the desert, feared by all.

The \jinn were disembodied horrors of the void.
They haunted the empty deserts, howling. 

Make clear that the \jinn are horrible alien beings, like some Cthulhu monsters.

















\section{Lamia}
\target{Lamia}
\index{lamia}









\subsection{History}
Lamias were once created by the Cabal as servant monsters. 
They turned out to be inefficient, so the project was dropped, but many lamias escaped into the wild. 









\subsection{Physique}
Lamias were shapeshifting monsters. 
They had the innate ability to see and move through the Shroud to a fairly large extent and to change their shape. 
They were predators that hunted by stealth. 

A lamia's true shape was a slimy, fur-covered, amorphous thing with many large centipede-like legs. 

Lamias were monsters of cosmic horror, but they were some of the least powerful such monsters. 
In some regards they were more powerful than \humans, but in other regards they were weaker. 
A skilled \human warrior could kill a lamia if he could corner it. 

A lamia was typically considered a female creature. 
Lamias reproduced by parthenogenesis. 
They needed to drain lots of energy in order to reproduce. 
Sometimes they would transform into a sexy humanoid, seduce someone and drain the victim's life (sometimes killing them, but not always) to feed their young.
Because of this, many scholars wrongly assumed that lamias reproduced sexually. 
In truth, the young lamia was a clone of its mother. 
Any victims the mother may have drained were mere food, not mates. 

Lamias could change their size and mass, but only up to a certain maximum. 
A typical young lamia was about as massive as a \human child, 40 kg or so. 
They kept growing through their entire lives, so an old lamia could be huge and very dangerous.









\subsection{Psychology}
Since lamias were originally created as a servant race, there existed spells to summon lamias and compel them to obey. 
They were intelligent creatures and would fight against these spells. 
The mage would have to break the lamia by willpower or torture. 
With enough time and effort, a lamia could be molded into a quite obedient and loyal slave. 















\section{Lords of the Deep}
\target{Lord of the Deep}
\target{Lords of the Deep}
\target{Aboleth}
\target{Aboleths}
\index{Lord of the Deep}
The Lords of the Deep were a race of water-dwelling immortals.
They had dominated \Miith millions of years before the \banewars, before even the \shugul arose. 

Compare them to the Aboleths from \cite{RPG:DungeonsandDragons}.









\subsection{Demography}
One surviving Lord of the Deep was \hs{Mnoth}, who once ruled the half-men of \hs{Lom}.









\subsection{History}





\subsubsection{Destroyed by \noggyaleth}
The Lords of the Deep were eventually destroyed by the \hr{Noggyal}{\noggyaleth}, like \hr{Noggyaleth destroy civilizations}{other civilizations before them}. 
Only few Lords survived.





\subsubsection{Ruins}
\target{Aboleth ruins}
The Lords of the Deep left behind many ruins.
Their cities were all built beneath the sea, but over the course of millions of years some of the ruins ended up on dry land. 









\subsection{Physique}
A Lord of the Deep looked like an oblong fish-like creature with a three-way symmetry and several tentacles. 









\subsection{Politics}





\subsubsection{Masters of Negation}
\target{Lords of the Deep and Masters of Negation}
It might have been the \hs{Masters of Negation} that destroyed the civilization of the Lords of the Deep.





\subsubsection{\Shugul}
See the section about \hr{Shugul and Aboleths}{\shugul and the Lords of the Deep}. 









\subsection{Skills and powers}





\subsubsection{Transforming others into slimy things}
The Lords of the Deep could transform humanoids and other creatures into formless amoeboid slaves.
They did this by causing the victim's flesh and soul to regress back to the most primitive form of life.
This was an amoeboid thing not unlike a \noggyal. 
This amoeboid could be shaped and turned into a slave of the Lords of the Deep.

















\section[Meccaran]{\Meccaran}
\target{Meccaran}
\target{Meccara}
\index{\meccaran{} (plural \meccara{} or \meccarans)}
\Meccara{} are amphibian humanoids. They are widespread especially in the warmer climates of the South. 









\subsection{Get rid of them}
I should get rid of the \meccaran race and replace them with types of \demiscathae.
I am stretching realism by having so many races. 

See more in the section about \hr{Demiscatha}{\demiscathae}. 









\subsection{Name}
Singular \emph{\meccaran{}}, plural \emph{\meccara{}} or \emph{\meccarans{}}. %\emph{\Meccara} is generally used when referring to the race as a whole, while \emph{\meccarans} refer to a specific group of individuals. This grammar is Imetric. 
The word is Imetric. \emph{\Meccara{}} is Imetric declination, \emph{\meccarans{}} is English declination. I will use both forms synonymously. 
The associated adjective is \emph{\meccaran{}}. 









\subsection{Physique}
\Meccara{} look like humanoid frogs. They have large, strong hind legs and are fast and agile runners, leapers and swimmers. Their forearms are dextrous, but not very strong. In combat, \meccarans{} rely more on speed than on brute force. \Meccaran{} skin is smooth and no tougher than human skin. 

A \ps{\meccaran} mouth is filled with sharp teeth that curve backwards. Adapted (among other things) to catch fish, \meccaran{} teeth are effective for biting and holding. \Meccaran{} necks are short, however, so biting is difficult in combat. 

A average, full-grown \meccaran{} female weighs 60 kg and stands about 130 cm tall in her natual posture. The male is smaller than the female, about 50 kg and 120 cm in height. \Meccarans{} do not stand fully erect but in a \quo{crouched} position with the legs bent outwards. They can increase their height by up to one third by stretching out, but this is an unnatural position and can not be maintained for long. \Meccaran{} arms are short, only around 50 cm. 

\Meccaran{} vision is inferior to that of \humans. They are near-sighted and cannot see well at great distances (this is a result of adaptation to a life in dense forests). At close range they see as well as \humans. Their sense of smell is acute\dash not as good as that of a dog, but far better than a human's. Their other senses are like those of \humans. 

\Meccara{} have the ability to regenerate lost limbs. The time this takes depends on size: A lost hand or foot can be regrown in a month, an arm in four months and a leg in six months. The new limb is generally as good as the old one, but sometimes the regeneration is faulty, making the new limb smaller, weaker and/or less agile than the old one.\footnote{Make some kind of saving throw to avoid permanent attribute loss (and possible limping, if a leg grows back too short). If the same limb has been lost and regrown more than once, the chance of faulty regeneration increases each time (so it is generally not worth the risk to cut off the new limb and hope for a better one next time). A faulty limb cannot \quo{cured} by regular magical healing, since it not an injury but the natural state of the new limb. \quo{Fixing} a faulty limb is just as hard as it would be to strengthen the original limb.} At TL7 and above, limb transplants between individuals is generally easy. Non-fatal injuries to internal organs (even the brain) can also be healed. \Meccaran{} regeneration is slow, however, and not useful in combat. 

As a special feature, \meccara{} are left-handed by default. A minority ($20\%$) are right-handed. 









\subsection{Biology}
\Meccara{}, naturally evolved from large, freshwater-dwelling, predatory frogs, are carnivorous and semi-aquatic. They are able swimmers but cannot breathe water. They prefer subtropical to tropical climates and high humidity. Their natural habitats are jungles and swamps. They naturally live as hunters. Some of the more advanced \meccaran{} communities also raise livestock for food and tools, with farms to support the animals. (This is a science they have learned from other races. The naturally carnivorous \meccara{} would be unlikely to discover farming on their own.) 

%\Meccara{} prefer to bathe in fresh water regularly (once per day at least, more in hotter or drier areas). Drying out is painful, but not fatal. 

\Meccara{} must bathe in fresh water regularly. If a \meccaran{} cannot immerse (or at least splash) himself with fresh water every day, he will weaken and die. 

\Meccara{} have two genders and reproduce by external fertilization: The female lays her eggs in water and the male fertilizes them. A female can lay several dozen eggs at a time, most of which die. The eggs are spherical, 2-3 cm in diameter. After six months, the surviving eggs hatch into tadpoles. A hatchling tadpole is only few cm in length and has very little intelligence. Gradually they tadpoles develop legs and the ability to breathe air. After around five years, they are about 50 cm long, have legs as well as a tail, and are as intelligent as a \human{} child of two years or so. At this point they mostly land-living. After one or two more years, they lose the tail and the ability to breathe water. At the age of around 10, \meccara{} are sexually mature, about 80-90 cm tall and as intelligent as a \human{} teenager. They are full-grown adults around the age of 15. \Meccara{} live to be up to 50-60 years old, 70 in exceptional cases. The females outnumber the males, making up 60 percent of the population. 

%The \meccaran{} species is closely related to the Fitteran species. It is possible (albeit rare) for the two species to crossbreed. \Meccaran{}/Fitteran hybrids are larger than \meccara{} but more intelligent than Fittera. They are sterile, like mules, and usually considered ugly freaks by both species. There are, however, communities with \meccarans{} and Fitterans living together. In these communities, crossbreeds tend to be accepted as normal members of society. 

\Meccara{} are generally not monogamous and do not mate for life. Culturally, they have few sexual taboos and tend toward promiscuity\footnote{Promiscuity in the biological sense, meaning that any two individuals in a tribe may mate.}. Indeed, some tribes are known to indulge in collective sexual orgies, often under the influence of certain drugs brewed by their witch-doctors. Bi- and homosexuality is rather common and accepted in most cultures. 









\subsection{Psychology}
\Meccara{} tend to crave independence and freedom. As such, \meccaran{} tribes are rather loosely organized and laws are few and simple. \Meccara{} are curious and inquisitive by nature and have little fear of change. Some tribes have evolved into settled farmers and built towns or even cities, but their natural lifestyle is nomadic. 

\Meccaran{} adventureres are rather common. \Meccara{} tend to be very active and energetic, with laziness being seen as abnormal. 

All \meccara{} suffer from a mild dyslexia. The part of their brain that allows them to understand writing is not as well-developed as that of most races. They can learn to read and write, but they do it less well than other creatures of equal intelligence. Of course, dyslexia is not understood at TL3, so this is usually interpreted as stupidity, which leads to \meccarans{} sometimes being looked down on as barbarians. 









\subsection{Habitat}
\Meccara{} are most widespread in the tropical lands of Uzur and in the Far South. They are relatively common in the Imetrium and \Durcac and uncommon in \Velcad{}. They are rare in the North and East. Most \meccara{}-dominated communities lie in Uzur, but there are \meccaran{} tribes scattered across \Velcad{} and other places as well. 















\section{\Naiad}
\target{Naiad}
\index{\naiad}
\Naiads{} are water-dwelling \quo{spirits}. 
They are actually a kind of intelligent jellyfish with innate \hr{Telepathy}{telepathic} and telekinetic powers. 

They are traditionally referrec to as female, although this is biologically incorrect. 









\subsection{Name}
Singular \naiad, plural \emph{\naiads{}}. The word is originally Greek but the declination is English. The associated adjective is \emph{\naiad{}}, I guess. (Naiadese? Naiadean? Naiadic?) 









\subsection{Physique}
A \ps{\naiad}{} natural element is water. In the water, she appears as a barely visible patch of vaguely shimmering water. Closer inspection will reveal a network of long, spindly tendrils. These tendrils are the \ps{\naiad}{} actual body, a brain of some sort. In the water, a \naiad{} will typically spread out over an area 2-3 meters across. She is not massive and other creatures can pass through her body unharmed, perhaps without even noticing her. 

A \naiad{} must be surrounded by water to survive. On dry land, she must bring along water of her own using her telekinesis. To preserve her strength, she usually brings along only a small body of water. When \travelling over land, the easiest and most confortable form to assume is that of an amorphous amoeba that slithers along. With practice and effort, a \naiad{} is able to shape her watery body into a humanoid form (or sometimes a humanoid torso with an amoeboid lower body, which is easier to control than legs). \Naiads{} are not very strong, so they will rarely drag along much more than 10-20 liters of water. So if she assumes humanoid form, it will be that of a very small humanoid. 

Cold will numb and slow a \naiad{}. If frozen, she will go into torpor but not die. If she is frozen and then shattered, she will die. Fire damage will quickly kill a \naiad{} if it can boil or evaporate the water surrounding her. Electricity does no damage. Acid does damage if it directly hits her tendrils, but this is rarely fatal unless there is a \emph{lot} of acid. Death magic has its normal effect. 









\subsection{Biology}
\Naiads{} are actually asexual and reproduce by budding. 









% \subsection{Psychology}








% \subsection{Habitat}















\section{\Nephil}
\target{Nephil}
\index{\nephil}
The \nephilim were a race of primate humanoids. 









\subsection{Physique}
\target{Nephil breasts}
\Nephilim had tails like monkeys.

\Nephilic{} women had four breasts. 
The women were fewer than the men, so they had to give birth to more young at a time to compensate. 





\subsubsection{Bestial}
\target{Nephil beast-men}
\Humans saw the \nephilim as terribly frightening things; bestial and yet disturbingly \human-like. 

There were some myths that \humans were descended from \nephilim. 
The Vaimons denied this, but many \Ortaicans believed it. 

\citeauthorbook[p.292]{RobertEHoward:RoguesintheHouse}{Robert E. Howard}{%
  Rogues in the House%
}{
  He shuddered at the sight of the great black hands, thickly grown with hair that was almost furlike. The body was thick, broad, and stooped. The unnaturally wide shoulders had burst the scarlet gown, and on these shoulders Murilo noted the same thick growth of black hair. The face peering from the scarlet hood was utterly bestial, and yet Murilo realized that Nabonidus spoke truth when he said that Thak was not wholly a beast. There was something in the red murky eyes, something in the creature's clumsy posture, something in the whole appearance of the thing that set it apart from the truly animal. That monstrous body housed a brain and soul that were just budding awfully into something vaguely human. Murilo stood aghast as he recognized a faint and hideous kinship between his kind and that squatting monstrosity, and he was nauseated by a fleeting realization of the abysses of bellowing bestiality up through which humanity had painfully toiled.
}

\citeauthorbook[p.223]{RobertEHoward:ChildrenoftheNight}{Robert E. Howard}{%
  Children of the Night%
}{
  And about them clustered the\dash Things.
  \Humans they were, of a sort, though I did not consider them so.
  They were short and stocky, with broad heads too large for their scrawny bodies. 
  Their hair was snaky and stringy, their faces broad and square, with flat noses, hideously slanted eyes, a thin gash for a mouth, and pointed ears.
  They wore the skins of beasts, as did I, but these hides were but crudely dressed.
  They bore small bows and flint-tipped arrows, flint knives and cudgels.
  And they conversed in a speech as hideous as themselves, a hissing, reptilian speech that filled me with dread and loathing. 
}









\subsection{Biology}




\subsubsection{\Nephil/\human{} hybrids}
\Humans, being descended from \nephilim, can interbreed with them. 
These hybrids are called half-\humans{} or half-\nephilim{} depending on whom you ask. 
They resemble their mother more than their father. 
If the mother is a \nephil, then her half-breed children look like skinny, balding \nephilim{}, while those born of a \human{} mother look like big, burly, hairy \humans. 

Either way, the hybrids are sterile. 





\subsubsection{Sexuality}
There are fewer \nephilic{} females than males. 
They make up no more than $30\%$. 

\target{Nephil misogyny}
This means that wives and daughters are very valuable property. 

In some cultures, kings and lords could have several wives, but common men had to count themselves lucky if they could get just one wife (or just sex, for that matter). 

Gay sex is common. 
In some places it is traditional for a conquering man to ass-fuck his conquered enemies, to establish his dominance. 
It shows that they are no longer real men, but slaves, property. 
Like women. 









\subsection{History}
\subsubsection{\Ophidian{} servitude}
\target{Nephilim worship Ophidians}
Originally, before the time of \Kserasshana, the \nephilim{} served  \ophidians{} and worshipped them as gods. 

In fact, the \ophidians{} \hr{Ophidians create Nephilim}{had created them}. 





\subsubsection{Chariots}
Back in the day, the \nephilim{} used chariots in war, because the \nephilim{} were large, and they knew of no mounts that were both fast and big enough to carry them. So they rode chariots, drawn by horses or other things.





\subsubsection{\Aryothim{} appear}
At some point, mighty \nephilim{} turned themselves into \hr{Aryoth}{\aryothim}. 





\subsubsection{Enslaved}
In some places, \humans keep \nephilim{} as slaves. 

Some \human armies keep a number of \nephil \quo{ogres}. 
They look like \humans, but bigger, uglier and more monstrous. 
They have bestial faces with big-ass nasty teeth. 
Compare to the Ghouls in the game \cite{VideoGame:WarcraftIII} and the more monstrous Persians in the movie \cite{Movie:300}. 
Perhaps they are like the Blunderbores of the game \cite{VideoGame:DiabloII}: 
Giant-sized, wretched abominations, mentally warped and underdeveloped, and kept as slaves, beasts or worse. 
Some of them have had their arms amputated at the elbows, like the giants in a deleted scene in \cite{Movie:300}. This makes them unable to feed themselves in the wild should they flee, so they are forced to remain and live as slaves. 
When they march into battle, some of them have blades attached to their arms, like the monstrous executioner in \cite{Movie:300}.








\subsection{Ogre-Magi}
A certain group of \nephilic{} sorcerers are known to others as \quo{Ogre-Magi}. There are very few of them, but they are pretty powerful.









\subsection{Politics}





\subsubsection{\Aryothim}
See the section about \hr{Aryothim and Nephilim}{\aryothim and \nephilim}. 









\subsection{Psychology}





\subsubsection{Less affected by the Shroud}
\Nephilim, with their simple minds, were not as powerfully affected by the Shroud as were \scathae and \humans.
They were closer to nature and the animals. 





\subsubsection{No affinity with \saurians}
The \nephilim were mammals, and as such, they had difficulty building any mental rapport with \saurians.
This meant they could not easily tame \saurians and use them as beasts of burden.
This held their culture and technology back.
They could not develop much technology without the help of the \ophidians. 















\section{\Noggyal}
\target{Noggyal}
\target{Noggyaleth}
\target{Ghobal}
\target{Ghobaleth}
\index{\noggyal}
The \noggyaleth{} were a race of amorphous giant monsters. 
They were native to \Miith, but related to the \banes. 









\subsection{Name}
Singular \emph{\noggyal{}}, plural \emph{\noggyaleth{}}. 

The name is inspired by the \Qliphah{} Golab, in \Cabbalah mysticism.









\subsection{Physique}
\Noggyaleth looked shoggoth-like:
An amorphous, ever-twisting mass of slime that only vaguely resembled flesh.
A \noggyal was flexible and chaotic and could reshape its body at will.
Often they would be covered with eyes and other sensory organs.
They had little individuality but could join together to increase their intelligence.

They could can create pseudopods/tentacles when needed, or even halfway dissolve their own bodies. 
This sometimes made them appear like a writhing mass of dozens of worms. 
Especially when seen through the Shroud. 

They stank hideously.
Their smell was alien and unnatural, but yet familiar in a way that was so deeply disturbing that no mortal could bear to consider it.

The \noggyaleth were great burrowers.
With their acidic secretions they could burrow holes and tunnels through the ground as well as through the barriers between the Realms.

\citeauthorbook{LinCarter:TheNecronomiconTheDeeTranslation}{Lin Carter}{
  The Necronomicon: The Dee Translation (part I.VII.III)
}{
  And then, at length, there flashed upon my vision one glimpse of a depth and of an abomination more horrible than any that I had glimped before. I looked upon a foul black pit, with a carven rim of beslimed rock about it, all drowned in Plutonian gloom, litten only by the vile phosphorescence of the primal white jelly of the proto-Shoggoths\ldots{} and amidst the hideous slime and the obscene stench I saw the bubbling, quivering plastic horrors, those shuddering towers of gelatinous, liquescent filth, studed with naked and protruding and staring eyeballs\ldots{} and I shrieked, and fled, back down the pathways of space and time and dimension, knowing in that last, soul-blighting glimpse the nodding flowers that blossomed in the scummed shallows of that lake of bubbling filth\dash{}\emph{and shrieked, and fled, knowing at last where the Black Lotus bloomed, and upon what unspeakable slime it feeds.}
}

Compare them to:
\begin{itemize}
  \item The shoggoths in \cite{HPLovecraft:AttheMountainsofMadness}.
  \item The Dholes/Bholes in \cite{HPLovecraft:TheDreamQuestofUnknownKadath}. 
  \item To a lesser extent, the sandworms of \cite{FrankHerbert:Dune}.
  \item The beholders from \cite{RPG:DungeonsandDragons}.
\end{itemize}

\lyricstitle{Draft excerpt from the chapter \quo{What Slithers Beneath}}{
  And from out of the Stygian, subterranean depths of \Nyx, the Twilight Realm, his foe crawled forth. 
  Of behemoth proportions it was, and it brought the charnel stench of cosmic death and decay. 
  Wormlike, pallid gray in \colour, and gaping with a myriad hungry mouths, it slithered up to meet him. 
}

\citeauthorbook[p.133]{VengerSatanis:CthulhuCult}{Venger Satanis}{Cthulhu Cult}{
  Squirming Thing of the blackest reaches.\\
  Crawling Entity that we cannot know.\\
  Beneath the water you scream and whisper.
}









\subsection{Biology}
An idea is that the \noggyaleth{} feed on magical power, so \ps{\Teshrial} \noggyaleth{} are only truly formidable if the Sentinels show up with some great power. Or maybe they feed on \vertices. That might be evil. 

I need to think more on how exactly this works. 

They were asexual and reproduced by budding.





\subsubsection{Corrupting the planet}
\target{Noggyal corruption}
\target{Noggyal burrowing}
The \noggyaleth burrowed through the planet and attempted to gain control of it.
But it was hard for them on their own.
They were devious and cunning but \hr{Noggyaleth do not plan}{not cut out for long-term planning}.
In the time of the \ophidian empire, the \ophidians kept the \noggyaleth in check, so they accomplished nothing noteworthy. 
But even after the \ophidians had fallen, the \noggyaleth still achieved nothing noteworthy in their many thousands of years. 

When the \resphain came and allied with the \noggyaleth, things began to speed up. 

The \resphain knew of the \noggyaleth as monsters, allies and even pawns, but few suspected the big picture.
The \noggyaleth were really all one, and they were inextricably tied to the true nature of \Miith.
Compare to the sandworms of \cite{FrankHerbert:Dune}. 

\citeauthorbook[p.287]{DavidDrake:ThanCursetheDarkness}{David Drake}{%
  Than Curse the Darkness%
}{%
  \ta{Maybe they aren't gods at all, him and the others\ldots it and the others Alhazred wrote of.
  Call them cancers, spewed down on earth ages ago.
  Not life, surely, not even \emph{things}\dash but able to shape, to misshape things into a semblance of life and to grow and to grow and to grow.} 
  
  \ldots
  
  \ta{Into this earth, this very planet, if unchecked.}
  
  \ldots
  
  \ta{Not \quo{rule} the world,} she corrected.
  \ta{Rather \emph{become} the world.
  This thing, this seed awakened in the jungle by the actions of men more depraved and foolish than I can easily believe\ldots this existsence, unchecked, would permeate out world like mould through a loaf of bread, until the very planet became a ball of viscid slime hurtling around the sun and stretching tentacles towards Mars.}
}

The \banes had a plan to use the \noggyaleth to dig through the planet and build a big complex that would let them open the way to \Erebos. 

\citeauthorbook[p.210]{RPG:CallofCthulhu:BeyondtheMountainsofMadness}{%
  Charles and Janyce Engan%
}{%
  Beyond the Mountains of Madness%
}{
  For decades the resources of an entire world wwere turned to the task of building the two devices.
  The Great Lure was sculpted out of mountains; its components were shaped from the mighty and unseen forces that form reality itself.
  It was placed here\dash at the Pole of the World, at the center of the Earth's oldest energies\dash and set in operation to call mighty powers from the depths beyond.
  The Trap was built here too, in the shadow of the Lure\dash a device to hold the earthly form of such entities as might arrive in thrall.
  It was merely the centerpiece of a web of construction that dug deeply into the Earth's crust for hundreds of miles, anchoring and channeling all the power of nature to the cause. 
}





\subsubsection{Mother-mass}
\target{mother-mass}
There existed a great \quo{mother-mass}. 
It was the very first proto-\noggyal from which all others had budded. 
The mother-mass was the primordial slime from which all \Miithian{} life descended. 

Compare it to Ubbo-Sathla, a Great Old One from the Cthulhu Mythos. 
Ubbo-Sathla is the shoggoth mother-mass.
Also compare it to Abhoth from \cite{ClarkAshtonSmith:TheSevenGeases}. 

The mother-mass \hr{Noggyaleth were the first life}{was the first life on \Miith}. 
It was \hr{Voyagers create mother-mass}{tampered with by the \voyagers}, but it predated them. 

The mother-mass was not a fully physical thing. 
It was an insubstantial, formless, twisting mass. 
It was completely mindless. 
Physical \noggyaleth would sometimes bud from it. 

\hr{Sethicus}{\Sethicus} knew of the mother-mass.
In \hr{Sethican philosophy}{his mysticism}, he considered it to be a thing of \DaathKurZulNathla, the deepest plane of primal chaos. 

\citeauthorbook[p.45]{ClarkAshtonSmith:UbboSathla}{Clark Ashton Smith}{Ubbo-Sathla}{
  Through years and ages of the ophidian era [the soul of Zon Mezzamalech] returned, and was a thing that had not yet learned to think and dream and build.
  And the time came when there was no longer a continent, but only a vast, chaotic marsh, a sea of slime, without limit or horizon, that seethed with a blind writhing of amorphous vapours. 
  
  There, in the grey beginning of Earth, the formless mass that was Ubbo-Sathla reposed amid the slime and the vapors. Headless, without organs or members, it sloughed from its oozy sides, in a slow, ceaseless wave, the amoebic forms that were the archetypes of earthly life. Horrible it was, if there had been aught to apprehend the horror; and loathsome, if there had been any to feel loathing. About it, prone or tilted in the mire, there lay the mighty tablets of star-quarried stone that were writ with the inconceivable wisdom of the pre-mundane gods.
}









\subsection{History}





\subsubsection{Origin}
\target{Noggyaleth were the first life}
The \noggyaleth were native to \Miith.
The \noggyal \hs{mother-mass} was the very first life on \Miith. 
The \noggyaleth were truly ancient creatures, older than the \ophidians.

The mother-mass was the primordial slime from which all \Miithian{} life descended. 

The \noggyaleth were deeply and tightly connected to the Heart of \Miith, being the first life.
All other life, even \ophidians, somehow descended from the horrid \noggyaleth.

They embodied creative chaos, but not intelligence and planning.

The \noggyaleth might have been kin to the \xss.
Some \draconic myths described them as the spawn of \RuinSatha or \KyaethemChreiAz, but this is very vague and uncertain.





\subsubsection{\Voyager Age}
\target{Voyagers train Noggyaleth}
When \hr{Voyagers come to Miith}{the \voyagers came to \Miith}, they took the \noggyaleth and altered them to make a race of intelligent, powerful, versatile slaves.
They were originally completely mindless, but the \voyagers gave them intelligence.

The \voyagers trained the \noggyaleth and taught them many skills of the mind and body. 
In a sense, the \voyagers taught the \noggyaleth to be \hs{living machines}:
Stronger, more resilient and more versatile than any machine of metal. 

Perhaps the \noggyaleth rose up against the \voyagers and drove them out.
Or perhaps the \voyagers were driven out by the \xss and the \noggyaleth just remained.

In any case, the \voyagers did not leave willingly. 
It came to a great cataclysmic war. 

During the early time of the \voyagers, the life they had raised and shaped from the mother-mass had taken on diverse forms. 
Many of them were beautiful and advanced and full of love and life and light. 
Now came the cataclysmic war against the \voyagers. 
It was catastrophic for \Miith. 
The majority of multicellular life died, including most the beautiful forms. 
In order to survive this hell, the \voyager-shaped lifeforms degenerated into strong, resilient, brutal forms.
The surviving things were hideous hellish and monstrous like the post-apocalyptic world that had shaped them.
They became \noggyaleth. 
This apocalypse took thousands of years if not hundreds of thousands. 





\subsubsection{Destroying civilizations}
\target{Noggyaleth destroy civilizations}
The \noggyaleth, \hr{Noggyaleth were the first life}{being the oldest race on \Miith}, remembered the many elder races that had lived and died before the \ophidians came.
In fact, the \noggyaleth themselves had destroyed more than one civilization. 
This includes the \hs{Lords of the Deep}. 





\subsubsection{\Ophidian Age}
The \ophidians knew that the \noggyaleth existed. 
They did not know the \noggyaleth's true nature, although a few occultists suspected.
The \ophidians feared the \noggyaleth just as they did the \xss, but the \ophidian magic was mostly powerful enough to keep the chaotic, bestial \noggyaleth at bay or even destroy them.
In earlier days, \hr{Ophidian-Noggyal wars}{the \ophidians waged great wars against the \noggyaleth} and drove them underground.





\subsubsection{Role}
The \noggyaleth{} dwell in the depths below \Nyx. 
The \quo{bottom floor}. 
They slither and burrow in dimensions that exist parallel with the spires of \Nyx{} and gradually lead to the surface of \Erebos. 
(\hr{Nyx is above Erebos}{\Nyx{} exists in the skies high above \Erebos}.)
These dimensions are inaccessible and non-permeable to the \resphain{} and \banes. 
Only the \noggyaleth{} can cross them due to their exceptional burrowing abilities. 

It is their kind that have carved out \Erebos, leaving only a mass of twisted spires and no ground. 
The also dug out \Nyx. And now they are in the process of undermining \Miith{}, turning it into a dead husk, an empty shell. 

In the mystic gloom of the deep abyss underneath \Nyx, you can sometimes hear the writhing of the horrible \hr{Ghobal}{\noggyaleth}, or feel the tremours of their passing and their burrowing.
Once in a rare while you can feel a tower faintly trembling. 
This happens when a \noggyal{} violently collides with the \CrystalSphere{}\dash the monster scrapes the edge of the Sphere but fails to penetrate. 





\subsubsection{Burrowing through the Shroud}
The Cabal used \noggyaleth.
They were useful because they could drill through the Shroud. 

By burrowing through the crust of \Miith the \noggyaleth made the planet's barriers unstable. 
They worked to slowly undermine the \CrystalSphere and open a back door from \Miith to \Erebos.





\subsubsection{In \Malcur}
\ps{\Teshrial} monsters in \Malcur (see section \ref{Teshrial's creatures}) are \noggyaleth. 









\subsection{Politics}





\subsubsection{\Banes}
\target{Banes and Noggyaleth}
\Banes were related to \noggyaleth.
The \banes were descended from things analogous to the \noggyal \hs{mother-mass}, perhaps brought to \Erebos from \Miith by the \voyagers.

The \banekings learned of the \noggyaleth's existence.
They realized that there lay the path to their future.
If the two races could assimilate each other, they would be greater than their creators the \voyagers in all things.
They would possess the best of both worlds: 
Cold intelligence and wild creative chaos.
So the \banes forever longed and struggled to return to \Miith and rejoin the mother-mass. 

\target{Unification}
This was the Unification, the great goal of the Cabal. 

When the \banes destroyed the \voyagers, they quickly made plans to go to \Miith and unite with the \noggyaleth.
But the \ophidians, who were the heirs to the \voyagers on \Miith, would have none of that.

Individual \banes could not just merge with individual \noggyaleth.
The \baneking himself must come to \Miith and absorb into himself the great \hs{mother-mass} of the \noggyaleth, which lay seething and bubbling deep beneath the planet's surface, feeding directly on the Heart.
In a sense, the \noggyal mother-mass was an extension of the heart.

See also the section about \hr{Noggyal corruption}{how the \noggyaleth corrupted the planet}. 





\subsubsection{Masters of Negation}
\target{Noggyaleth and Masters of Negation}
It was the Masters of Negation that spawned the \noggyaleth, and the \noggyaleth continued to serve them. 






\subsubsection{\Resphain}
\target{Resphain and Noggyaleth}
The \resphain employed \noggyaleth, but they did not understand them.
They thought of the \noggyaleth as bestial, semi-intelligent things, tamed and subdued with the power of spells.
They did not suspect the true extent of the \noggyaleth's intelligence and power, nor their ties to the \banelords' long-term plan.
They assumed they had achieved power over the \noggyaleth, but in truth the \noggyaleth had more freedom and plans of their own than the \resphain realized.









\subsection{Psychology}





\subsubsection{Rumoured to be mindless}
The \resphain believed that \noggyaleth were mindless things that existed only to crawl, burrow and feed.

\citebandsong{Nile:Ithyphallic}{Nile}{
  Eat of the Dead
}{
  The highest fulfillment of man\\
  Is to become food for the crawling things\\
  That burrow and slither in human flesh\\
  Unceasing in mindless hunger\\
  Remorseless undefiled by reason\\
  The worms of the tomb they are pure

  Their purity elevates them\\
  Above the putrefying pride of our race

  The destiny of man is\\
  Merely to be\\
  The nourishment of the worm\\
  Yet their excrement bestows higher wisdom

  From decay arises new life\\
  Fill myself with that which rots\\
  And I shall be reborn

  By writhing upon my belly like a mindless worm\\
  I shall rise up in awareness of truth\\
  I gnaw upon my own decaying flesh\\
  And my mind is forever purged\\
  Of the corruption of faith
}





\subsubsection{Hivemind}
\target{Noggyal hivemind}
In reality, the \noggyaleth were more intelligent than that. 

\target{Noggyaleth do not plan}
They were devious and cunning but they seemed not cut out for long-term planning.
This was actually a ruse. 
The \noggyaleth were all part of a colossal and dreadful hivemind that dwelt deep beneath the earth. 
Almost no one knew of its existence. 
The \dragons only hypothesized about it.
\Sethicus did some research, but when he managed to catch a telepathic glimpse of the \noggyal hivemind he recoiled in horror.
He halted all research and never wrote or talked about it (except in furtive hints) because it was too awful to consider, even for him.
To think that this ancient foe of the \ophidian race was an immortal and unconquerable hivemind that had destroyed countless races before. 









\subsection{Skills and powers}





\subsubsection{Immunities}
\Noggyaleth were notoriously difficult to harm. 
Their morphic bodies were all but immune to all cutting, piercing and bludgeoning attacks.
They were resistant to fire and heat since they dwelt in magma underground. 
They were susceptible to cold, electricity, energy drain and death magic. 
Certain spells were designed specifically to harm or repel \noggyaleth. 





\subsubsection{Magic}
\Noggyaleth could actually use magic, but they almost never did.
The \hr{Noggyal hivemind}{\noggyal hivemind} did not feel it worth the effort.
Most of the time it was content to crush its foes by brute force. 

But once in a very rare while the \noggyaleth could channel horrible magic of unimaginable power. 
In all of \ophidian history this had happened less than half a dozen times, so these anomalies were explained away.
No one knew the \noggyaleth were capable of magic.
\Sethicus suspected, but did not know for sure. 





\subsubsection{Suspectible to spells}
\target{Controlling Noggyaleth}
The \ophidians knew spells that could banish or even control a \noggyal.
The control spells were very rarely used because they were difficult, unreliable and extremely risky. 
Only \Sethicus and a few of the greatest \dragons ever mastered them.

The \shugul were perhaps the only race that could enslave \noggyaleth, using their \hr{Shugul mind control}{dreaded mind control}.















\section[Nycan]{\Nycan}
\target{Nycan}
\target{Nycans}
\index{\nycan}
\Nycans{} are large predatory reptiles resembling real-world dinosaurs like Deinonychus or Velociraptor. 

This creature is based partially on the actual dinosaurs, partially on the Jurassic Park movies and partially pure fantasy. 

\Nycans{} are fierce predators that hunt in packs. 
They are highly intelligent and have \hr{Telepathy}{telepathic} abilities. 
They are found mostly in the Imetrium and Irokas. 
In the Imetrium, \nycans{} are domesticated and used as beasts of war. 









\subsection{Name}
Singular \emph{\nycan{}}, plural \emph{\nycans{}}. 
The word is Imetric but the declination is English. 
The associated adjective is \emph{\nycan{}}. 
%(The word is derived from \quo{nychus}, as in \quo{Deinonychus}. This is originally a Greek word for \quo{claw}.)









\subsection{Physique}
A \nycan{} is a slender, bipedal dinosaur with a long, stiff tail. Its forearms are long and strong and have sharp claws. Each hind leg has not one but two oversized claws. \Nycans{} are covered in feathers and \coloured in shades of brown, red and yellow.\footnote{There is evidence to show that some RL dinosaurs of this type were in fact feathered.} 

In combat, \nycans{} slash with the great claws on their hind legs if possible, and also attack with their foreclaws and bite. 

\Nycans{} are extremely effective predators. They are fast and agile runners and leapers and can sprint like cheetahs, reaching tremendous speeds over short distances. They are also highly intelligent, hunting in well-organized packs and displaying great cunning in their hunting behaviour. 

\Nycans{} have vision like that of \humans{} and hearing and smell like that of dogs. They are diurnal and do not see well in darkness. 

In the wild, \nycans{} are around 3 meters in length and weigh around 70 kg (like a \latinname{Deinonychus}). 
In captivity, beasts up to 7 meters long and weighing over a ton have been bred (like a \latinname{Utahraptor}). 
The female is slightly larger than the male. 

\target{Nycan endurance}
\Nycans{} are strong enough to carry riders. 
They can also sprint as fast as \hr{Relc}{\relcs} over short distances. 
But they are not suitable as mounts for longer rides. 
They lack the endurance of \relcs. 
They are built for explosive bursts of speed, not prolonged running. 






\subsubsection{Nycans are frightening}
\target{Nycans are frightening}
Describe how the \nycans{} are terrible and frightening, with their cold, reptilian eyes that know too much and hide an unnatural, inhuman intelligence. 

This is a toned-down version of \hr{Draconic appearance}{\draconic{} appearance}. 

Compare them to Sag'Churok and Gunth Mach, the two K'Chain Che'Malle that follow Redmask in \MalazanReapersGale.






\subsubsection{Magic}
\target{Nycan magic}
A few exceptionally advanced tribes/packs of \nycans{} have developed primitive magic and psionics: 
Telepathic attacks, telekinesis and maybe healing. 
Most other \nycans{} (especially tame ones, who are affected by the Shroud of Civilization) do not understand it and fear it. 









\subsection{Biology}
\Nycans{} are warm-blooded dinosaurs and a result of natural evolution. Only one species of \nycan{} is known to exist - they are so effective that they have driven all closely related species into extinction. 

A \nycan{} female lays a small cluster of one to five eggs. Equal numbers of males and females are born. \Nycans{} live in packs, each pack led by an alpha female. Males and females both participate in hunting and raising the young. They are not monogamous. 

A \nycan{} is sexually mature after 7-9 years. Their average lifespan is about 20 years in the wild and 30-35 years tame, although they can reach 50-60 years. 
%They can live up to 50 years tame, but rarely more than half that in the wild. 





\subsubsection{Races and breeds}
The Imetrians breed \nycans{} into a number of subraces and breeds. 


\index{Secca (plural Seccae)}
\index{Crycos (plural Crycoi)}
\index{Destran}
The Seccae (singular Secca) are the largest and the only \nycans{} strong enough to easily carry a rider. 
The Crycoi (singular Crycos) are the fastest, used as couriers (carrying letters) and as shock troopers. 
The Destrans (singular Destran) are bred for intelligence and sharp senses. 

The races are further divided into a number of \quo{breeds}, often named for the city, family or individual that breeds them or \quo{founded} the breed. 

\target{Mictzan}
\index{Mictzan}
Examples include the Mictzan breed of Destrans, named for the founder (an Imetric \scatha{} of Clictuan descent), and the Dorlinum breeds, of various races, bred in the Imetric city of Dorlinum. Especially renowned are the Dorlinum Secca, who are some of the largest and strongest \nycans{} known. 









\subsection{Psychology}
\Nycans{} are used to living in a pack and can be tamed and taught to recognize a non-\nycan{} as pack leader. Once tamed, \nycans{} are very loyal, and there is no risk of them \quo{going wild}. 

\Nycans{} are as intelligent as humanoids and can learn a wide variety of skills. They can learn to understand language well enough to understand simple sentences and maintain a rather large vocabulary of both simple and more abstract concepts. A skilled \hr{Nycaneer}{\nycaneer} can give his \nycans{} quite complex instructions. 





\subsubsection{Affected by the Shroud}
\Nycans, with their high intelligence and telepathy, were more strongly affected by the Shroud than dumber animals. 
This was one of the reasons why they did not dominate the world. 





\subsubsection{Communication}
Among themselves, \nycans{} communicate using \hs{telepathy}. 
This telepathy, combined with smell, allows them to detect emotions and thoughts in other creatures. 

They also commicate using sound. \Nycan{} voices are hoarse, screeching and high-pitched. They can scream (to warn fellows of danger), yelp (if in pain), hiss (to warn or intimidate foes) and purr (when friendly). 

The reason they use sound and body language, even though telepathy might be more effective, is the same reason why \humans{} use body language and not only speech: 
It's older. 
\Nycans{} and their ancestors have had sound and body language for countless millions of years, whereas telepathy is a relatively new evolutionary breakthrough, only a million years old or so. 









\subsection{Habitat}
\Nycans{} prefer warm climates and open grasslands. They are predators and prey on all sorts of animals. 

A millennium ago, \nycans{} were widespread over much of the Old Continent. Being very dangerous creatures who may attack livestock and people, the \nycans{} were hated and feared by most intelligent creatures, and as a result, they were hunted into extinction in most lands by humanoids armed with weapons and magic. 

The only place in the West where \nycans{} survived was the land that is now the Imetrium, and centuries later, the Imetrians discovered how to domesticate them. Tame \nycans{} are now an invaluable asset in Imetric society, taking the place of dogs in RL. \Nycans{} are stronger and more intelligent than dogs, however. In the military, \nycans{} are used as trackers, battlefield shocktroopers, and even assassins. 









\subsection{Arsenal}





\subsubsection{\Armour and weapons}
Imetric \nycans{} wear \armour and weapons when going into battle. 
They wear metal vambraces on the forearms which allow them to parry blows from swords and the like. 

And they wear big-ass metal blades covering the natural claws on their feet. 

Some wield enhanced metal foreclaws, or even \hr{Skekrathuin}{\skekrathuin}-like blades. 





\subsubsection{Magic}
There exist wise \nycans{} who know some magic. 
They have a hard time studying because they cannot easily learn to read and write, and as such have little in the way of culture. 
But some of them do learn some \quo{natural} magic. 





\subsubsection{Seeing into the Beyond}
The \pps{\nycans}{} telepathic abilities also let them see into the Beyond. 

\Nycans{} fear and hate the unnatural. 
This includes such things as undead and demons, as well as \quo{black} magic, but not all magic. 
As a general rule, anything that has a Horror Effect will qualify. 
When faced with \quo{unnatural} things, \nycans{} will become enraged and want to attack it, or flee, if the menace is perceived as too powerful or horrible. 
A \bane{}, for instance, will cause \nycans{} to flee, but they might stand and hiss at it from (what they perceive to be) a safe distance. 
Something with only a Slight Horror Effect, like a dark mage (such as a \Nieur{} channeller) or a Rissitic Dominus, will cause the \nycans{} to hiss threateningly, but not immediately attack. 
An undead creature (such as a Rissitic Immortal Priest) is likely to be savagely attacked. 
If the \nycans{} are tame and well-trained, a \nycaneer{} may be able to restrain them\footnote{This would require a skill roll of some kind.}, but the instinctive hatred is strong. 

%Note that what is \quo{unnatural} is not entirely instinctive. A Rissitic Dominus, for instance, has a Slight Horror Effect, but a \nycan{} raised by Rissitic trainers and accustomed to Rissitic culture and magic 









\subsection{History}
\target{Nycans forgot Cuezcans}
The \nycans{} were originally \hr{Ophidians breed}{shaped by the \ophidians} as beasts of war. 

The \nycans{} were originally closely tied to the \hr{Cuezcan}{\cuezcans}. 
But after the \CuezcanApocalypse, most \nycans{} grew \quo{\Wylde} and \quo{savage}.
They forgot the true history of their people(s). 









\subsection[Nycaneers]{\Nycaneers}
\target{Nycaneer}
\target{Nycaneers}
\index{\nycan!\nycaneer}
A \nycaneer{} (plural \emph{\nycaneers{}}) is a \scathaese{} \hr{Telepathy}{telepath} who communicates with \nycans{}. 

\Nycaneer ing is actually not really a special \quo{affinity} with \nycans, as is often believed. 
Here is the deal: 
\Scathae{} are \hr{Origin of Scathae}{genetically related to \nycans}. \Nycans{} are naturally telepathic. 
\Scathae{} also possess some innate telepathic ability, but it is latent in most, and the Shroud further oppresses it. 
A few \scathae{} show special talent for telepathy. 
Since their natural telepathic \quo{voice} and communication style is \nycan-like, they find themselves inherently able to communicate with \nycans{} to some extent. 
Telepaths who are in contact with \nycans{} from an early age will, by natural means, learn to understand a telepathic \nycan{} language and thus converse with them. 
These develop into \nycaneers. 

\Scathaese{} telepaths who do not encounter any \nycans{} in their youth, or who only learned telepathy as adults, will usually never learn \nycan{} tongues. 
If they later meet a \nycan, they will be unable to communicate with it, and as adults they can no longer easily pick up new languages. 
Such people will thus have no specific affinity with \nycans{} and therefore will not be considered \nycaneers. 

It should be noted that \nycans{} do not all speak the same telepathic language. 
These are as different as spoken languages. 
So a \nycaneer{} who can converse with one \nycan{} will not necessarily understand another. 
However, an experienced and socially ept \nycaneer{} will understand not only language but also many other aspects of \nycan{} behaviour and mannerisms. 
Using telepathy and this knowledge, a \nycaneer{} will be able to communicate with a strange \nycan{} to some limited extent, much like how two humanoids who share no language can still communicate using body language and tone of voice. (In contrast, a non-\nycaneer{} and a \nycan{} will be able to communicate about as well as a \human{} and a wild wolf, ie., extremely crudely.) 



\Nycaneers, with their extra-sensory perceptive abilities, develop into \vertices{} more often than other people. 

Only \scathae{} and \rachyth{} have the \nycaneer ing talent, since they are \hr{Origin of Scathae}{genetically related to \nycans}. 
The talent is found in varying degrees, but even a weak \nycaneer{} can develop his skill, eventually being able communicate with \nycans{} almost as well as he communicates with other humanoids. 

The Imetrium actively searches for young \nycaneers{} and maintains a formal \Nycaneer{} Academy. 

\index{\melda{} (plural \meldae)}
The \nycaneer{} that commands a given \nycan{} is that \ps{\nycan}{} \melda{} (Imetric word, plural \emph{\meldae{}}). 















\section{\Nymph}
\target{Nymph}
\target{Nymphs}
There are three races of \quo{\nymphs}: 
Dryads, \naiads{} and \sylphs. 

In poetry they are considered the \quo{daughters of Mother \Miith}. 
They are portrayed as ancient and benevolent, mourning and weeping whenever \Miith{} is at war and laid waste. 

Compare to the \quo{Daughters of Beulah} in \authorbook{William Blake}{The Four Zoas}. 

\WanderersInDarknessEmph is full of references to the sorrowing \nymphs. 
















\section[Shugul]{\Shugul}
\target{Shugul}
\target{shugul}
\index{\shugul}
The \shuguls were a race of semi-humanoid immortals. 









\subsection{History}
Also called \quo{\moonthings}. 









\subsection{Name}
They hailed from the moon of Visha and served the \moongods.
In the time before the \ophidians the \shugul came to \Miith and dominated it.
Later \hr{Ophidians drive out Shugul}{they were driven out by the \ophidians}. 

The \shugul bred the \hr{glithid}{\glithids} as a slave race.









\subsection{Physique}
A \shugul was a grotesque quasi-humanoid.
It had a corpulent body, three thick elephantine legs, three tentacled arms and a head with a long proboscis.
The head could turn all the way round, as many times as desired.
Their entire body had a three-way symmetry.

On Visha, their homeworld, gravity was lower.
Here they were tall, spindly, graceful creatures.
On \Miith, gravity squashed them into squat, grotesque and clumsy things. 

They spoke in very deep-pitched droning voices and had difficulty reproducing \humanoid sounds.

They were slow and physically weak and would often let themselves carry on litters or palanquins by their \glithid slaves. 
They were priests and mighty sorcerers, channelling the power of the \moongods.

Compare to the \quo{high priest not to be described} in \cite[p.444]{HPLovecraft:TheDreamQuestofUnknownKadath}.









\subsection{Politics}





\subsubsection{Lords of the Deep}
\target{Shugul and Aboleths}
The \shugul feared the \hs{Lords of the Deep}.
The Lords once enslaved the \shugul and nearly wiped them out. 
(Perhaps the Lords of the Deep even invaded Visha.) 

But the \shugul later learned many mysteries and much lore from the Lords and \hr{Aboleth ruins}{the ruins they left behind}.





\subsubsection{Masters of Negation}
\target{Shugul and Masters of Negation}
It might have been the \hs{Masters of Negation} that destroyed the civilization of the \shugul.









\subsection{Skills and powers}





\subsubsection{Immortality}
\Shugul were \hr{True immortal}{true immortals}. 
They would reform even if their bodies were destroyed. 





\subsubsection{Immunities}
A \shugul's body was made of alien, extra-\Miithian matter. 
It was difficult to harm the monster with conventional attacks. 
But they could be harmed by fire and heat. 
(Visha was a cold place.)





\subsubsection{Mind control}
\target{Shugul mind control}
The \shugul were most infamous for their powers of mind control, which were terrible.
Even immortals could fall under their sway. 
So they were feared by the \ophidians and \resphain.
They were the only race that could \hr{Controlling Noggyaleth}{enslave \noggyaleth}.

The \ophidians developed spells and talismans and symbiotes that protected against \shugul mind control, but an unprotected \ophidian was vulnerable and could be controlled.
Only \dragons were immune. 
















\section{Spider People}
Have a race of monstrous intelligent spider-like creatures that spin webs. 
Maybe merge them with the \hr{Weaver}{Weavers}. 

Compare to some of the wicked descriptions in \cite{Cracked:GeneticExperiments} (the spider/goat part). 
















\section{\Succubus}
\target{Succubus}
\target{Succubi}
\index{\succubus}
Maybe have a race of alien monsters that tempt people with someething they want, such as sex, and lure them out of the Shroud and into the Beyond. 
There the monsters feast on the victims's flesh and (maybe) souls. 

The monster shows people illusions, using people's dreams/fantasies/desires (sexual or otherwise) to lure them out, then manipulates their mind to draw them out through the Shroud and into the Beyond, where no one hears them scream.

Sometimes the \succubus{} will take the shape of a sex partner, seduce its victim, and then use sex to distort the victim's perception of reality and suck them into the Beyond. Then the \succubus{} transforms into a man-eating monster\dash suddenly, or gradually during sex. 

\target{Succubus sucking dick}
Maybe even have a vore-like sex scene where she devours her lover while they fuck\dash gradually sucking his life, blood and body out through his dick, leaving an empty husk. This is a super-powered version of Shereid's sperm-eating spell (see section \ref{Shereid's sperm-eating spell}).

%Compare to \authorbook{Clive Barker}{The Son of Celluloid} (\emph{Books of Blood III}). 

\lyricsbs{Clive Barker}{
  The Son of Celluloid (Books of Blood III)
}{
  You make me strong, looking at me that way. I need to be looked at, or I die. It's the natural state of illusions.
}















\section{\Vorcanth}
\target{Moon-Wolves}
\target{Vorcanth}
\index{\Vorcanth}
The \MoonWolves{}, the \quo{mystic wolves of the Frost-Moon}, were an ancient race of powerful, somewhat wolf-like creatures. 
They possess \human-level intelligence, but different, so they cannot easily communicate with humanoids. 
They are \Wylde{} creatures.

They are inspired by the song \bandsong{Bal-Sagoth}{Starfire Burning upon the Ice-Veiled Throne of Ultima Thule}, and by the Deragoth (Hounds of Darkness) in \cite{StevenEriksonIanCameronEsslemont:MalazanBookoftheFallen}. 

They are an ancient race, older than the \nephilim, and used to live side by side with the \ophidians. 
They used to be among the masters and rulers of the Beast Realm, but as the \dragons{} and \resphain{} have gained territory, the \moonwolves{} have declined, and there are now few of them left.







\subsection{Appearance}
A \vorcanth{} is a quadruped, six metres long (or that order of magnitude). 
They are covered in snowy white fur and look vaguely like wolves, and are typically likened to them, but they are not wolves and not closely related to them. 
A somewhat closer match in appearance are hyaenas. 
But even that is pretty far from the mark.

They are not entirely mammalian. 
They also have reptilian characteristics. 
They have snaky, reptilian tails. 

Underneath the fur they are covered in \armoured scales or plates. 
They have large \armour plates on the shoulders that kind of resemble the frills of \mulgrons. 

Compare them in appearance to the wolf-things in the movie \movie{Final Fantasy VII: Advent Children}. 

The impression they give is that of alien-looking creatures that look as if they've stepped out of the ancient pre-history. 
Compare them to Sag'Churok and Gunth Mach, the two K'Chain Che'Malle that follow Redmask in \MalazanReapersGale.





\subsubsection{Horrible}
\Vorcanths were dark, terrible, mysterious things like the Hounds of Tindalos from the Cthulhu Mythos. 
They were some of the horrors that lurked in the Realms Beyond and would sometimes prey on \Miithians. 
Even immortals were subject to their depredations. 
Even their \resphan allies never understood them very well. 









\subsection{Culture}





\subsubsection{Language}
\Vorcanths{} communicate using sound and body language. 
Most elders can understand \draconic{} and \resphan{} tongues. 
A few can even speak them. 

A few mages know telepathy as well. 









\subsection{History}





\subsubsection{Origin}
The \vorcanths were a terrible elder race, the terrifying predators of a barbaric, bestial age. 
They might even be the spawn of the horrid \hs{Gods Beneath} that gnawed and brooded in the churning chaos at the planet's core. 





\subsubsection{Enemies of \ophidians}
The \vorcanths{} were ancient rivals of the \ophidians. 
When the \dragons arose they waged war against the \vorcanths and almost wiped them out from \Miith. 
After this defeat the \vorcanths retreated to the moons of \Dun and Visha. 
After the \firstbanewar the \vorcanths invaded \Miith again and dominated much of the planet for a time. 

When the \dragons returned they made war again. 
In the \secondbanewar some \vorcanths allied with the \resphain.
This war saw most \vorcanths wiped out. 

After that they retreated to \Dun.





\subsubsection{On the moons}
\target{Moon-Wolves and the Moon}
The \vorcanths also dwelt on the moons of \Dun and Visha. 
They had their own wars and skirmishes on the moons; against one another and against other monsters such as the \hr{Shugul}{\shuguls}. 





\subsubsection{Few leaders left}
In Carzain's age, few of the great \vorcanth{} leaders (whom Ramiel-tachi knew from before they became \malachim) survived. 
Many had been killed in their various wars and conflicts. 
But new leaders had arisen, and some had grown very powerful. 









\subsection{Politics}
\subsubsection{Enmity with \dragons}
\target{Moon-Wolves dislike Dragons}
The \moonwolves{} dislike \dragons, their ancient rivals who almost wiped them out. They root for the \resphain\dash although they fear the \banes. 

They see Ramiel as a saviour of sorts.





\subsubsection{Association with Ramiel}
Ramiel is a friend of the \MoonWolves{}. In the past, Ramiel helped out one of the great, venerable alpha male leaders of the \moonwolves. As thanks, a pack of them chose to remain by his side as his personal allies, following him as their alpha.

But this was before Ramiel \hr{Ramiel becomes a Malach}{became a \Malach} and lost his memory. Since then, the wolves have \hr{Carzain dreams of Moon-Wolves}{tried to contact him in dreams}, but he has been entangled in the Shoud, and they have not been able to readily communicate with him. 

\target{Moon-Wolves help Ramiel in dreams}
The Shroud prevents them from just popping into the physical world. Also, if they did, the Sentinels and Cabal might hunt them down. Still, they come to his aid from time to time, in the world of dreams, at least. When he is besieged by monsters of his own imagination, at times he sees great white wolves that come and dispel or destroy the apparations. 

He senses that there is something he should know, but he doesn't understand it. But they awaken something in him, and they help him remember fragments of his past life. He realizes that the wolves are a vital clue in his quest to discover his past.

%He contacts them in dreams, and they help and guide him.
Perhaps one of his allied \moonwolves{} is sick or hurt, so the pack is questing for him, helping him and seeking his help in return. So they come to him in dreams, beckoning him to come to them. At last, he somehow manages to find the wounded wolf and help it. 

It pledges itself to him and becomes his permanent companion for the rest of the story. Ramiel now has his personal \moonwolf{} companion. At this point, he still does not understand the creatures and his link to them. He doesn't fully realize that until \hr{Ramiel's awakening in the temple}{his awakening in the temple}. 





\subsubsection{Worshippers}
There are some people who worship the \moonwolves{} as gods or demigods. Compare them to the wolf-worshipping Grey Swords and others in \cite{StevenEriksonIanCameronEsslemont:MalazanBookoftheFallen}.

\lyricsbs{Bal-Sagoth}{Naked Steel (The Warrior's Saga)}{'Neath the Moon-Wolf's gaze we shall slake our steel.}









\subsection{Skills and powers}





\subsubsection{Power level}
Physically the \vorcanths{} are extremely powerful. 
A \vorcanth{} is easily a match for several \resphain{} in close combat, and may even challenge a young or weak \dragon{}. 

\Vorcanths{} cannot fly. 
But to compensate, they are very fast runners, insanely fast jumpers and highly stealthy stalkers (utilizing their \hr{Vorcanth travel Beyond}{power to travel Beyond}). 

Their weakness is magic. 
Some \vorcanths{} do know magic, but it is rather primitive compared to that wielded by \dragons{}, \resphain{} and \quiljaaran{}. 
The \vorcanths{} are well aware of this, and it is surprising how often they manage to offset this disadvantage using stealth, cunning and speed. 





\subsubsection{Travelling in the Beyond}
\target{Vorcanth travel Beyond}
They have the ability to \quo{fade into the mist} and travel through hidden planes, journeying to secret \quo{folds} of the Realms, which the \dragons{} and others do not understand and where they cannot follow. 

Compare to the Hounds of Tindalos from a Cthulhu story by\ldots{} Frank Belknap Long, I believe. 























\chapter{Monsters}















\section{\Chimaera}
\target{Chimaera}
\index{\Chimaera}
A mythical creature, related to and possibly identical to the \hr{Malgryph}{\malgryph}. 

It was \hr{Urizeth researches Chimaera}{researched by \Urizeth}. 















\section{\Carth}
\target{Carth}
\index{\carth}
The \carths, also called \carth-serpents, were gigantic winged lindworms. 









\subsection{History}
The \carths were created by the \ophidians as mounts and beasts of war. 









\subsection{Politics}
The \carths also existed in \Nyx, where they were used by the \resphain as mounts. 















\section{Dark Young}
I should have a race forest monsters similar to the Dark Young of Shub-Niggurath in the RPG \emph{Call of Cthulhu}. 














\section{Flying Whales}
I should have a race of flying whale-like things. 
Compare to the windwhales in \cite{GlenCook:TheWhiteRose}. 

But they should be dark, grotesque and Cthulhu-esque. 














\section{\Lindworm}
\target{Lindworm}
\index{\lindworm}
\Lindworms were \dragon-like/snake-like animals. 
They were created by the \ophidians, who used them as mounts and beasts of war. 
They were long and agile and fairly intelligent. 














\section{\Malgryph}
\target{Malgryph}
\index{\malgryph}
A \malgryph was a mythological animal that looked like a giant nycan with feathered wings, a pair of backward-curving horns and the head and tail of a snake.
It was said to be wise, possessing secrets of sorcery.
It was occasionally used in heraldry. 
It was feared as a terribly destructive monster and a bringer of evil omen.
It was an ominous thing to have on your banner.

It was related to and possibly identical to the \hr{Chimaera}{\chimaera}. 

\target{Malgryph summoning}
There existed spells to conjure forth a \malgryph.
Not a \quo{real} \malgryph, of course. 
It was not a real, existing animal.
But chaos sorcerers could conjure daimonia and temporarily shape them in the form of a ghostly \malgryph that would fight for him for a time.
The \malgryph would be controlled by a \homunculus. 

It was a powerful, deadly spell of alienism.
But it was known only to \dragons and a few other sorcerers.
It was not found in regular \rethyactic textbooks.

See \emph{The Bible}, Isaiah 14:29.

\lyricsbible{Isaiah 14:29}{
  Rejoice not thou, whole Palestina, because the rod of him that smote thee is broken: for out of the serpent's root shall come forth a cockatrice, and his fruit [shall be] a fiery flying serpent.
}














\section{\NerasKirishgaith}
\target{Neras Kirish'gaith}
\target{Bladed People}
The \NerasKirishgaith{} are a race of mostrous semi-humanoid creatures with blades all over their bodies. Compare to the eponymous creatures from the anime \emph{Gilgamesh}. 

They might be \banes, but they might also be native \Miithians. 

Perhaps they have great, insect-like wings. Perhaps only some breeds have wings. 

They have hive-like societies with queens, warriors and drones. They mind-control humanoids and use the bodies of these to interact with humanoids. 

They are the ancient rivals of the \ophidians, descended from a group of terrible \quo{Progenitors}, who were some of the ones that created all life on \Miith{}, millions and millions of years ago. Compare to the Great Old Ones of H.P. Lovecraft's Cthulhu Mythos.















\section{Night-Feaster}
\target{Feasters in the Night}
\target{Night-Feaster}
\target{Night-Feasters}
\index{Night-Feaster}
The Night-Feasters, or Feasters in the Night, were a race of monsters that dwelt in the nearer layers of the Beyond. 
They sometimes broke through the Shroud and into the Shrouded Realms, especially in the \wylde. 









\subsection{History}





\subsubsection{Created by \ophidians}
\target{Origin of Night-Feasters}
The Night-Feasters, like many other creatures, were created by the \ophidians as living weapons. 
They bred true as a species. 
After the fall of the \ophidians' empire, they escaped into the \wylde.
The \ophidians lost control of them. 









\subsection{Name}
The Night-Feasters were called \emph{\feldraxes} by the \ophidians.









\subsection{Physique}
A Feaster was bat-like or gargoyle-like. 
It was vaguely humanoid, but only if you squinted right. 
Feasters were able and agile flyers. 

Some felt that a Feaster resembled a very small \umbra and conjectured that the two races were related. 
There was no further evidence to support this, though.
Feasters appeared to be native to \Miith whereas \umbrae originated from \Erebos, and Feasters had none of the powers that \umbrae did. 









\subsection{Politics}





\subsubsection{Summoned by sorcerers}
There existed spells that could summon Feasters. 
They were living weapons \hr{Origin of Night-Feasters}{designed by the \ophidians} for this purpose. 

It was also possible to affect the feaster's mind and encourage it to leave the summoner alone and attack his enemies. 
It was difficult to control the Feaster with any accuracy, though. 









\subsection{Psychology}





\subsubsection{Intelligence}
The Feasters were believed to be possessed of only high animal intelligence. 









\subsection{Skills and powers}





\subsubsection{Scream}
\target{Night-Feaster scream}
The Night-Feasters could rip asunder ears, brains and sanity with their screams, which resounded with the terrible music of the spheres which mortal ears could not endure to hear. 















\section{Ravening One}
\target{Ravening Ones}
\target{Ravening One}
\index{Ravening One}
The \quo{Ravening Ones} were a race of monsters that dwelt in the \wylde. 

A Ravening One could be summoned by sorcerers and compelled into servitude.









\subsection{History}





\subsubsection{Origin}
\target{Origin of Ravening Ones}
%The Ravening Ones might have evolved naturally, or they might be \xs-spawn.
The Ravening Ones were created by the \ophidians as living weapons.









\subsection{Name}
The Ravening Ones were called \emph{\rulyamoths} by the \ophidians.









\subsection{Physique}
A Ravening One was a massive, half-formed monstrosity. 
It had a number of feet with great bear-like claws.
It had a number of heads, each ending in a round lamprey-like mouth. 

A Ravening One was large, weighing about a ton. 









\subsection{Skills and powers}





\subsubsection{Intelligence}
A Ravening One possessed only moderate animal intelligence. 















\section{\Reptilecolossus}
\target{Reptile Colossus}
\index{\reptilecolossus}
The \reptilecolossi were a race of monsters artificially engineered by \Secherdamon. 
By the time of the \thirdbanewar he had amassed a small army of them (i.e., maybe 20--30). 

They were enormous, loathsome, bloated lizard/crocodile-like crawling things with many legs. 
They were part alive, part undead and part machine. 
They were walking tanks.
Their skins/carapaces had holes in them where \ophidian (or other) sorcerers could sit and attack with weapons or spells while being protected by the magic inherent in the monster. 

Each \reptilecolossus was animated by the enslaved soul of a powerful \resphan captive. 
(\Secherdamon kept \hr{Secherdamon's Resphan slaves}{many such slaves}.) 
This soul had to be permanently destroyed before the \reptilecolossus would die. 
One such slave was \hr{Sevestris}{\Sevestris}, \Sithiyacaan's beloved.















\section{\FireSalamander}
\target{Fire Salamander}
The great \firesalamanders were artificial creatures, manifestations of \hr{Satha}{\RuinSatha} (a \xs). 
They had semi-solid bodies made of pure alien fire.
Using some of the mightiest spells of \hr{Ruin Satha fire magic}{\draconian fire magic}, a mage could summon (or create) a \firesalamander and send it to fight for him. 

A \firesalamander took a form vaguely like a \dragon. 
See, they were formed by spells, and those spells were devised by \dragons. 
The \dragons envisioned their fiery warriors in their own image, so the \firesalamanders assumed a \dragon-like form. 















\section{\Werloc}
\target{Werloc}
\index{\werloc}
\Werlocs were semi-intelligent \scatha-like reptilian monsters. 
They dwelt in the \wylde (on the slopes of Mount \hr{Shrun}{\Shrun} among other places). 

\Werlocs were not strong, but they \traveled in great hordes and were quick, stealthy and cunning.






















 

\chapter{Animals}















\section{Birds}














\subsection{\Grulcan}
\target{Grulcan}
\target{Diatryma}
The \grulcans{} are a species of great flightless, predatory birds. 
They are inspired by real-world animals like \latinname{Brontornis} and \latinname{Gastornis} (\latinname{Diatryma}), who lived in the Eocene. 
Maybe these can be domesticated.

The name is inspired by the bird-monster Groth-Golka that appears in \authorbook{Robert E. Howard}{The Gods of Bal-Sagoth}. 

\Grulcans{} are not as fast as \hr{Relc}{\relcs}, but they are much more \manoeuvrable (\relcs{} turn slowly because of the tails). 















\section{Invertebrates}









\subsection{Skekkok}
A large scorpion-like creature that lives in the deserts of \Durcac. 

The name is an onomatopoeia for the clicking sounds the creatures make.

Compare to the scantids from \cite{VideoGame:Starcraft}. 















\section{Mammals}
\target{Saurian-dominated}
There are few large mammals on \Miith. 
\Miith{} is \saurian-dominated. 

The few large land mammals include hippopotami, hyaenas and white tigers (rare). 
Other mammals include small cats and the weasel family (carzains and wolverines). 









\begin{comment}
  \subsection{Buopoth}
  \target{buopoths}
  The buopoths are shy, forest-dwelling animals. 
  They are taken directly from some of \HPLovecraft's stories as a tribute/reference/\trope{ShoutOut}{shout-out}. 
  
  They look like elephants with smaller ears and bigger eyes. 
  Maybe they can be tamed. 
  
  \lyricstitle{\emph{Call of Cthulhu} RPG p.193}{
    \quo{%
      \ldots{} he had seen quaint lumbering buopoths come shyly from the wood to drink\ldots{}}
  }
\end{comment}













\subsection{Cats}
\target{cats}
Cats are an example of \hs{animals that can see into the Beyond}. 
They have great senses and know much of the true nature of the world, and are much more intelligent than humanoids tend to believe. 
However, a cat's mindset is quite alien, and it is difficult for humanoid \hr{Telepathy}{telepaths} to communicate well with a cat. 

Cats can see and move into the Beyond. 
That is why they are so skilled and stealthy hunters. 









\begin{comment}
  \subsection{Dogs and wolves}
  \target{dogs}
  \target{wolves}
  Wolves and other wild canines have good sight and \hr{animals that can see into the Beyond}{know quite a bit of the Beyond}. 
  %Especially the great and mighty \MoonWolves, of course (see section \ref{Moon-Wolves}). 
  They do not have a deep an insight as cats, but their minds are closer related to those of humanoids, so wolves are easier to communicate with.
  
  Tame dogs are much further removed from their \hr{Wild}{\Wylde} roots and more deeply entangled. 
  They see less than their \Wylde{} kin, and in certain regards they are more stupid.
\end{comment}















\begin{comment}
  \subsection{Gargantuan Beast}
  Have Gargantuan Beasts, like the ones in the game \emph{Diablo II}. 
  
  They look kind of like bears on two legs, but with a short neck and a flat face. 
  They are herbivorous and mostly peaceful. 
\end{comment}














\subsection{Hippopotamus}
\target{Hippopotamus}
\target{hippopotamus}
\index{hippopotamus}
The {hippopotamus} was the largest and most dangerous mammal in the world, as far as some people know.

Although there might be big whales that have the hippopotamus outclassed. 













\begin{comment}
  \subsection{Lion}
  \Miith{} has sabre-toothed lions. 
  
  They were once widespread all over \hr{Velcad}{\Velcad}, but have been hunted to near extinction by humanoids. 
  Now they are rare. 
  
  No longer a menace, they have since been romanticized and idealized (from the \hs{Vaimon age}) as a \quo{king of animals}. 
\end{comment}















\subsection{White tiger}
\index{white tiger}
\target{white tiger}
A species of tiger that lives in northern \hr{Velcad}{\Velcad} and the \hs{Northern Kingdoms}. 
It has a thick white fur. 

It is rare. 
Large \saurians{} are much more common. 















\section{Reptiles}









\subsection{\Brukath}
\target{Brukath}
\index{\brukath}
The largest species of sauropods on \Miith. 
Can exceed 45 metres in length and 200 tons in weight. 













\subsection{Caterpillar lizards}
I should have a race of caterpillar/lizard-like creatures that inhabit \hr{Machai}{\Machai} or the \hs{Veins}. 

Inspired by the drawings \quo{Curl-up} and \quo{House of stairs} by M.C. Escher. 

\lyricswikipedia{Curl-up}{Curl-up}{%
  Curl-up or Wentelteefje (original Dutch title) is a lithograph print by M. C. Escher which was first printed in November, 1951.
  
  This is the only work by Escher which consists largely of text. The text, which is written in Dutch, describes an imaginary species called Pedalternorotandomovens centroculatus articulosus, also known as \quo{wentelteefje} or \quo{rolpens}. He says this creature came into existence because of the absence in nature of wheel shaped, living creatures with the ability to roll themselves forward.
  
  The creature is elongated and \armoured with several keratinized joints. It has six legs, each with what appears to be a human foot. It has a disc-shaped head with a parrot-like beak and eyes on stalks on either side.
  
  It can either crawl over a variety of terrain with its six legs or press its beak to the ground and roll into a wheel shape. It can then roll, gaining acceleration by pushing with its legs. On slopes it can tuck its legs in and roll freely. This rolling can end in one of two ways; by abruptly unrolling in motion, which leaves the creature belly-up, or by braking to a stop with its legs and slowly unrolling backwards.
  
  The word wentelteefje is Dutch for French toast and is a contraction of \quo{wentel het eventjes}, which means \quo{turn it over briefly}. Rolpens is a dish made with chopped meat wrapped in a roll and then fired or baked. \quo{Een pens} means \quo{belly}, often used in the phrase beer-belly.
  
  There is a diagonal gap through the text containing an illustration showing the step by step process of the creature rolling into a wheel. This creature appears in two more prints completed later the same month, House of Stairs and House of Stairs II.
}















\subsection{\Corgorah}
\target{Corgorah}
\index{\corgorah{} (plural \corgoroth)}
\Gorgoroses{} are huge reptiles, terrible monsters from the \Wylde{}. 
A \gorgoros{} resembles a large theropod dinosaur like \latinname{Tyrannosaurus} or \latinname{Allosaurus}, but with far larger and stronger forearms ending in huge, wicked claws, like the Behemoths of the \emph{Heroes of Might and Magic} games. 

\Cortios{} are wild beasts, but they can be tamed, albeit with difficulty. 

Perhaps they are related to \dragons{} in some way. 

They have great mouths filled with tons of long, sharp, wicked teeth. Almost like a deep sea fish. 

Compare to Godzilla. 









\subsubsection{Name}
Singular \emph{\cortio{}}, plural \emph{\cortios{}}. 
This declination is English. 
The adjective is \emph{\cortio{}}. 

In Rissitic they are called \emph{\tashrek{}}. 









\subsubsection{Physique}
\Cortios{} are huge reptiles, similar to dinosaurs like Tyrannosaurus or Allosaurus. 









\subsubsection{Biology}








\subsubsection{Psychology}
\Cortios{} may be tamed, but they are difficult to control. They are savage, aggressive creatures and must be brutally dominated if they are to be kept loyal. 









\subsubsection{Habitat}
\Cortios{} are most common in \Durcac, southern Irokas and the Orient, but some exist in southern \Velcad{} and the Imetrium. 















\subsection{\Mezolisk}
\target{Mezolisk}
\index{\mezolisk}
Large crocodile-like or \dragon-like monster with great spikes. 

They are related to \ophidians. 
Like them, they can cool down and go cold-blooded. 

They are quite intelligent and can be tamed. 
The Rissitics use them as beasts of war. 
So do the Geicans and the people of the Orient. 
And even countries such as Runger. 

Also called \quo{dagger-drake}. 















\subsection{Ivory cobra}
\index{ivory cobra}
A cobra snake that is white in \colour and grows up to 150 cm in length. The ivory cobra is highly poisonous and its bite can easily kill a \human{} or \scatha{} (the teeth of an adult are strong enough to penetrate clothes and a \ps{\scatha} scales). 

The ivory cobra lives primarily in \Durcac. The animal is sacred to the Rissitics and a symbol of their religion and empire. 















\subsection{\Lotha}
\target{Lotha}
\index{\lotha}
A \lotha (plural \emph{\lothae{}}) was a medium-sized theropod, 7-10 metres long.
They could be tamed and used as mounts and beasts of war. 

The Rissitics \hr{Rissitic monsters}{used \lothae in war}.





\subsubsection{Psychology}
\target{Lothae are skittish}
\Lothae were somewhat skittish and easily frightened by \hr{Guns}{gunfire} and other things. 
They were not evolved to stand their ground against threats.















\subsection{\Mulgrons}
\target{Mulgron}
\target{Muroc}
\index{\mulgron}
A species of large ceratopsian dinosaurs. Similar to \latinname{Triceratops}.





\subsubsection{Psychology}
\target{Murocs are steady}
Some \saurians (such as \hr{Relcs are skittish}{\relcs} and \hr{Lothae are skittish}{\lothae}) were skittish and easily frightened by \hr{Guns}{gunfire} and other things. 
Not so \murocs. 
A \muroc was rock-solid and almost never panicked. 
It was a ten-metre colossus of muscles, \armour and horns. 

\Murocs were evolved not to flee from threats (they were slow), but to stand their ground, keep their calm and \emph{fight}. 
They were warrior animals. 
This made them great \hs{beasts of war}. 















\subsection{\Relc}
\index{\relc{} (plural \relcs)}
\target{Relc}
A herbivorous, quadrupedal \hr{Saurian}{saurian}. 
It around 5-6 metres long and has an elaborate crest on its head. 
Can be tamed and used as mounts or pack animals. 

\Relcs{} vary widely in \colour. 
Some are green or brown, some are zebra-striped in black and white. 

A person who rides a \relc{} is called a \relcer{}. 

As for \hr{Domestic animals}{domestication}: 
\Relcs were very much military animals.
Armies had them, but few civilians. 
Civilians would use other and slower animals.

\meta{%
  Similar to dinosaurs like \latinname{Saurolophus} or \latinname{Corythosaurus}, but smaller.}





\subsubsection{Psychology}
\target{Relcs are skittish}
\Relcs were somewhat skittish and easily frightened by \hr{Guns}{gunfire} and other things. 
They were herd animals.















\subsection{Sauropods}
\target{sauropod}
\target{sauropods}
\index{sauropod}
Sauropods are a group of \saurians{}. 
Species include the \brukath{} and \tondra. 

Some sauropods have an elephant-like trunk. 















\subsection{\Tondra}
\target{Tondra}
\index{\tondra}
A very large, quadrupedal, herbivorous \hr{Saurian}{\saurian} with a massive body and a long neck and tail. 
It is heavily built and \coloured in shades of white, beige and tan. 
One of the largest known \saurians{}, it can reach over 35 metres in length and weigh 100 tonnes. 

\Tondras{} were domesticated by the \hr{Vaimon Caliphate}{\VaimonCaliphate} and other cultures of their time, but no longer. 

\meta{%
  Compare to sauropod dinosaurs like \latinname{Brachiosaurus} or \latinname{Apatosaurus} (aka \latinname{Brontosaurus}). 
}















\subsection{\Varcal}
\target{varcal}
\index{\varcal}
A \varcal was a large theropod or \dragon-like monster.
It could be tamed and used as a beast of war.















\subsection{\Vreiiden}
\target{Wyverns}
\target{Vreiid}
\target{Vraiid}
\index{\vreiid}
Flying, \dragon-like reptiles. They are vicious, savage beasts. They have great mouths filled with rows of dagger-like teeth. 

The \vreiiden{} are a race of flying, reptilian monsters, like wyverns. They are savage and brutal creatures. They can be tamed to some extent, but it is difficult, and they'll never be very tame. They are used as mounts in \quo{dark} armies, like those of the Rissitics. 





\subsubsection{Mistaken for \dragons}
\Vreiid were rare and dangerous giant flying reptiles that resembled the \dragons of mythology. 
They were often referred to as \dragons by the people whom they occasionally preyed upon. 

A \vreiid was a very dangerous monster.
It was hideous and gave off a feeling of great age and evil. 
The fact that the giant creatures could fly made them even harder to defend against. 
But even so, a \vreiid had no magic and was not highly intelligent. 
They were fearsome things, but less fearsome than the \dragons described in mythology.

Some people suspected that the \vreiiden were not \dragons, and that there existed true \dragons far greater and more terrible than the \vreiiden. 

























\chapter{The Undead}















\section{Themes}
\target{Undead}
Remember to have armies of the undead.

Which side will have them? It would make most sense to have them on the Cabal side, mostly the province of the \banes, whose powers and very nature are connected with death, \hr{Entropy}{decay, defilement and parisitism} (like vampirism). 

But \hr{Rissitic pyramids}{Rissit has his pyramids}, which might house legions of the undead, waiting to be unleashed\ldots{}

I should have undead Revenants, like in Warcraft III. 

And Cold \hs{Wraiths}, that were once in Warcraft III. 

See also the section on \hs{Death}.

\lyricsbalsagoth{Arcana Antediluvia}{
  Down sixty fathoms, from stygian coral-clad tombs \\
  the pitiless abyssal sea disgorges its shambling mold-mottled dead,\\
  dank innards blackly acoil with nests of slithering things!\\
  Ghosts aglide upon the eldritch seas, \\
  unfathomed voyage to ascendancy.\\
  Traitorous blood, the surf roils red, \\
  churning crimson, thrice-cursed dead.
}







\subsection{Fallen civilizations of undead}
\target{Fallen civilizations of undead}
Have one or more fallen civilizations who were once great empires but now exist only as undead dwelling in ruined cities. 

Compare to the Tomb Kings from \emph{Warhammer}. 







\subsection{Monuments keeping the dead imprisoned}
See section \ref{Monuments keeping the dead imprisoned}.









\subsection{Undead guardians}
Maybe have undead guardians who are sacrificed and killed and have their souls and bodies bound in undeath to guard some object or place for eternity. 

Inspired by \authorbook{Poppy Z. Brite}{The Sixth Sentinel}. 









\subsection{Undead warrior/assassins}
Maybe I should have some creepy, disgusting undead warrior/assassins. 

Compare to the Autoj\"ager from the anime \emph{Trinity Blood}: \ta{They are corpses\ldots{} the corpses of vampires\ldots{}}












\begin{comment}
\section{Death Knight}
\target{Death Knight}
\index{Death Knight}
The Rissitic Death Knights are a kind of Wights. 

\subsection{Name}
\subsection{Physique}
\subsection{Biology}
\subsection{Psychology}
\subsection{Habitat}















\section{\Leech}
\target{Leech}
\index{\leech}
\Leeches{} are not to be confused with the similarly-named Liches, nor with regular leeches (non-capitalized), wormlike animals that live in swamps and the like and drink blood. \Leeches{} (capitalized) are a lesser form of \Reavers{} (see section \ref{\Reaver}): Intelligent undead who drain the life-force of others to survive. 

A \Reaver{} is created more-or-less voluntarily when a mage uses \hr{Life drain}{life-draining magic} extensively. A person cannot turn himself into a \Leech{}. \Leeches{} are created by \Reavers{} as lovers, companions or servants (or all three). Most \Leeches{} serve a \Reaver, and many dream of becoming \Reavers{} themselves. 







\subsection{Physique}
\Leeches{} look like living people of their race. There is nothing to put your finger on, but sensitive people will notice a certain savage, bestial aura about the \Leech. 

\Leeches{} have most of the powers of \Reavers{}, but weaker. They are supernaturally strong, fast and agile and can regenerate almost all wounds. They can see in almost total darkness and see auras around living creatures. A \Leech{} may take seemingly-mortal wounds yet rise again. If the brain is separated from the heart (usually by severing the head) or either brain or heart is smashed or torn apart, the \Leech{} is permanently destroyed. 

They also have the vulnerabilities of \Reavers{}, albeit to a lesser degree. Their vulnerability to wood is the same, but they are more resistant to sunlight. A \Leech{} will be destroyed in a few minutes if fully exposed to daylight, but normal shade, such as that provided by a broad-brimmed hat or umbrella, is enough to preserve them from harm. Even if exposed to sunlight, they can suffer it for up to a minute with no more than moderate pain and no visible damage. This makes them well-suited as agents for the \Reavers.

Unlike \Reavers, a \Leech{} does not drain life by spells but by drinking the blood of her victim. The victim must be alive or very recently killed for their blood to have any value to the \Leech. 

A \Leech{} retains all her former skills, including magic. But unlike \Reavers, \Leeches{} need not be mages, and most know no magic. 









\subsection{Biology}
To create a \Leech, a \Reaver{} must drain a victim to the brink of death and then cast a special spell on her, called the \quo{\Reaverz Kiss}, that transfers a portion of his life force to her, in effect causing her to drain some of his energy. The spell is also an enchantment that grants the \Leech-to-be a limited ability to drain life force without the use of spells, namely by drinking their blood. 

The \Leech-to-be will awaken with a tremendous thirst for blood. She can drink the blood of the \Reaver, but he is unlikely to allow her to sate her thirst on his own blood. Rather, he will send her to feed on mortals (perhaps keeping prisoners or servants ready, perhaps sending her off to hunt her own). 

At this point, the prospective \Leech{} has the supernatural powers of a \Leech{} to some extent and a great psychological craving for blood, but she is not fully one of the undead and not yet physically dependent on blood to survive. Resisting the desire to feed is difficult, but it can be done. If she refrains from drinking blood, the effects of the spell will gradually wear off. She will become sick and weak, but most likely she will survive, and after 5-7 days the spell has worn off completely and she will be back to normal. (In such a case, however, the \Reaver{} is likely to feel betrayed and attempt to reclaim or punish her.) 

If, on the other hand, the would-be-\Leech{} gives in to her desire and feeds on blood, she will find a great erotic pleasure in doing so, and she will find herself gradually transforming into a full-fledged \Leech{}. \Leeches{} can sustain their strength by drinking mortal blood, but they are not true \Reavers, and their method of draining life force is imperfect. Consequently, they cannot survive on this fare alone, but must also drink the blood of a \Reaver. At least once a month, the \Leech{} must feed on \Reaver{} blood or she will perish. Drinking mortal blood solidifies the \Leechz undead state, but drinking \Reaver{} blood cements it. A \Leech who has drunk \Reaver{} blood once (beyond the initial Kiss) can still fight the curse of undeath and return to life, although it becomes much harder, and even after having drunk twice the curse may be broken, although at this point magical healing and exorcism will be needed. After drinking \Reaver{} blood thrice after the Kiss, the \Leechz fate is sealed. She is now fully of the undead and can never return to life. 

As with a \Reaver, she needs not kill her victims, but unlike a \Reaver, a \Leech{} cannot absorb her victim's soul. The blood of another \Leech{} is particularly savoury and nourishing, but no substitute for \Reaver{} blood. 

The goal of many a \Leech{} is to become a \Reaver{} herself. To do this, she must learn spells of life-draining or acquire an enchanted item that duplicates the effect. This will let her drain true life force on her own, eliminating her need to feed on her master and thus eliminating her dependence on him. After draining life with magic for a while (usually a matter of months, since she will likely be feeding a lot), the \Leech{} will transform into a true \Reaver, with all the powers and weaknesses described in section \ref{\Reaver}. 

Like \Reavers{}, \Leeches{} do age, but they can arrest and reverse aging by drinking blood, thus living forever. 









\subsection{Psychology}
Some \Leeches{} love their \Reaver{} master and serve him willingly. \Reavers{} often turn their lovers into \Leeches{}. A \Reaver{} may have a single \Leech{} lover and treat her as an equal or near-equal, or he may have a whole harem of them vying for his affection. The \Reaver{} may, of course, also turn non-lovers into \Leeches, but only those people he believes he can trust. A \Reaver{} who truly cares for his \Leech{} might even teach her magic so that she may become a \Reaver{} herself. 

A \Reaver{} has no supernatural control over his \Leeches{}, beyond the fact that they need his blood to survive. Usually, he will dominate his \Leech{} servants through physical and magical power and sheer force of will. After all, the \Reaver{} will usually be not only much older and more experienced, but also a great mage, whereas most \Leeches{} know no magic. 

Still, a \Leech{} might hate her \Reaver{} master and plot against him. Of course, unless she also wants to end her own existence, she cannot simply kill him. But there are ways of dealing with this dependence:

First, a \Leech{} must drink \Reaver{} blood, but it needs not be that of the \Reaver{} that created her, so a disgruntled \Leech{} who encounters another \Reaver{} might defect. 

Second, a \Leech{} might be able to overthrow her master, keeping him as a prisoner on whom to feed at leisure. Usually, the \Leech{} will be no match for her \Reaver{} master (he will see to that), but she might be exceedingly cunning. Alternatively, if a \Reaver{} has several \Leeches{} under his control, they might band together to overthrow him. For this reason, a \Reaver{} who does not fully trust his \Leech{} slaves might play them against each other, turning them into rivals and enemies, ensuring they they do not ally against him. 

Third, the \Leech{} might learn of the fact that it is possible for her to become a \Reaver{} herself. If she manages to gain access to life-draining magic, she is likely to flee. 

To ensure her obedience, a \Reavers{} will often lie to his \Leech{} slave, convincing her that only his blood (not that of another \Reaver) will keep her alive and keeping her ignorant of how she might become a \Reaver. 

A \Reaver{} may create any number of \Leeches{} he desires, but each of them must drink his blood every month or die, and letting the \Leech{} drink his blood temporarily weakens the \Reaver, so he will not want too many mouths to feed. More importantly, a large group of \Leeches{} may mutiny against their master (as described above), so a \Reaver{} will not want to create more of them than he can control. 









\subsection{Habitat}
\Leeches{} always live near their \Reaver{} master (or they won't live long). But since they are less vulnerable to sunlight, they are more easily able to lead normal-seeming lives and not arouse suspicion. 
\end{comment}
















\section{Lich}
\target{Lich}
\index{Lich}
Liches (not to be confused with \Leeches, who are weaker undead) are arguably the most powerful type of undead. A Lich is a powerful mage who has transformed himself into one of the undead by means of an occult spell of terrible power. 

What characterizes the Lich is that it is fully self-sustained and immortal. The spell that creates the Lich opens a conduit to a source of dark energy somewhere in the Beyond. From now and evermore, this power sustains the Lich. Thus, the Lich needs to external power source. It requires no magical rituals to sustain it, nor must it feed on the life-force of others, like \Reavers{} do, but will sustain itself and exist forever.

A Lich is fully intelligent and retains all the skills and knowledge it possessed in life, including magic, and it will continue honing its skills and learning more magic throughout its immortal unlife. A mage must be of formidable skill and knowledge in order to become a Lich in the first place, and will grow only stronger as the centuries pass, so an old Lich is a mighty creature indeed. 









\subsection{Name}
Singular \emph{Lich}, plural \emph{Liches}. 









\subsection{Physique}
Liches retain the form they had in life\ldots{}









\subsection{\XulGann}
The \hr{Xul-Gann}{\XulGann} were a Rissitic form of \Liches. 















\begin{comment}
\section{\Reaver}
\target{\Reaver}
\index{\reaver}
Using dark magic, some people are able to \hr{Life drain}{drain the life force of others} to empower themselves. This is a very potent ability, but it carries a price: Repeated use of life-draining magic is addictive, physically as well as psychologically, and the mage will develop a growing craving for it. As the addiction grows, the mage will gradually transform into one of the undead. Such people are called \Reavers{}. 

\Reavers{} are usually powerful mages, but non-mages who rely on enchanted items to drain life force are also affected and may become \Reavers{}. 

Becoming a \Reaver{} is a gradual process. As a person succumbs to the addiction, his \Reaveric{} traits will grow stronger and more pronounced. The transformation is considered complete when the \Reaver{} can no longer sustain himself by natural means (food and drink) and must live off the life force of others alone. 

%\Reavers{} are people who 









\subsection{Name}
%\emph{\Reaver}, plural \emph{\Reavers} as in English. 
As in English. 
%The adjective is \emph{\Reaveric}. 









\subsection{Physique}
\Reavers{} look much like living people of their race. In time, their skin will grow somewhat pale, but they can still pass for normal people. Only people or creatures with special empathic skills will be able to detect the \Reaver{} at a glance. 

As a \Reaver{} grows in power, he will learn to drain more life force than he needs, storing it in his body to make himself supernaturally strong and fast. Powerful \Reavers{} are able to perform astonishing feats of superhuman agility and strength. Such power, however, is costly to maintain, and a \Reaver{} who wishes to remain strong must drain great amounts of life. \Reaver{} mages can use their drained life force to power their spells, making them very formidable spellcasters. 

\Reavers{} gain the ability to see in darkness, their vision unnaturally sharp even in minimal light. They cannot see in total darkness, but they see as well by starlight (under a clear sky) as in broad daylight. They also gain the ability to see the life force auras surrounding living creatures. This sense is similar to infravision, except that it shows life, not heat. It \emph{will} work in complete darkness and can see through up to 10-15 cm of earth, wood or stone or 1-2 cm of metal. This sense will \emph{not} detect undead or artificial constructs. It may or may not show alien creatures, depending on how alien they are. 

As a \Reaverz{} mastery grows, he can drain energy from opponents by a mere touch without having to cast an elaborate spell. Some \Reavers{} know spells that let them drain life through clothes or \armour, through a weapon or even at a distance. 

A \Reaverz{} greatest weakness (apart from his need to feed on life) is that he cannot abide the light of the Sun. Direct sunlight chars the \Reaverz{} flesh like fire. This vulnerability grows gradually as the \Reaver{} undergoes his transformation. A fully transformed \Reaver{} will die and crumble to ashes within a minute if exposed directly to full daylight (a few minutes if the he is large, like a \dragon{})\footnote{Only the body parts exposed to sunlight will burn. If the \Reaver{} is naked, his entire body will burn. But even if only the head is exposed, the \Reaver{} will still die when his head burns and crumbles.}. Light cast back from a strongly reflecting surface (such as a mirror or a lake) is almost as dangerous as direct sunlight. Light reflected from a bright surface (like snow, white marble or shiny metal) is less dangerous, wheras light cast back from dark surfaces (like wood or earth) is mostly harmless and can be endured for minutes with little harm. 

\Reavers{} mostly avoid going outside by day and travel only by night. If a \Reaver{} must travel by day, he will cover his entire body in clothing, using a hood, hat or mask to cover his head. There are subtle spells that offer some protection, but none are known that let a \Reaver{} withstand full daylight. A \Reaver{} unafraid of detection may use spells to cover himself in a cloud of darkness, providing further protection. 

Most non-sunlight (moonlight, bonfires etc.) is harmless to \Reavers{}. As for magical light, the spell description will sometimes state that the light works like sunlight. In this case, it will burn and can kill \Reavers{} as described above. 

\Reavers{} are vulnerable to weapons made of wood. Wooden weapons that hit bare skin or thin clothes will cause extra damage. Wood that strikes a \Reaver{} wearing \armour has no special effect. If a piece of wood pierces a part of the \Reaverz{} body, that body part will be paralyzed until the wood is removed, and numb for a while even then. If wood pierces the head or torso  (not necessarily through the heart), the \Reaverz{} entire body will be immobilized. This will work even with a wooden arrow with a metal head. 

\Reavers{} have the ability to heal almost any wound, reattach severed limbs or even regrow them from scrath. Burns, such as from sunlight, and wounds from wooden weapons can also be healed, albeit slower than most damage. Using powerful necromancy, \Reaver{} mages can sometimes even resurrect themselves after being killed and mutilated. This will leave the \Reaver{} drained and weak, however, and he must quickly feast upon great amounts of life or perish forever. If the \Reaverz{} body is burned (in fire or in sunlight), destroyed with acid or eaten and digested, it is destroyed beyond hope of resurrection. (Such a \Reaver{} might, however, still be raised as one of the incorporeal undead, such as a Wraith.) 

Certain animals can detect \Reavers{}. This includes all canines, all small cats (but not all large cats) and \nycans{}. They do this by a combination of smell and empathy. These animals will fear and hate the \Reaver{} and are likely to either attack or flee. 

It is possible for persons to learn the mystic, empathic skill of recognizing \Reavers{} and other undead. Imetric Paladins and the priests of \NishiS{}, certain Vaimon Templars and Clerics and the Ashenclaw knights of \KhothSell are all taught this ability. 









\subsection{Biology}
All races can become \Reavers{}. There are different ways to become a \Reaver{}, because there exist several varieties of life-draining magic. It is often said that \Reavers{} drink the blood of their victims, but in fact this is only one of several means of draining energy. %Some \Reavers{} know different ways to drain life, but some know only one. A \Reaver{} who only knows to drain life by drinking blood is called a Blood \Reaver{} by scholars. Other varieties are the Shadow \Reaver{} (using Rissitic Shadow magic), Nieur \Reaver{} (using Vaimon magic) and Chaos \Reaver{} (using \draconic{} Chaos magic). The different types of \Reavers{} will differ in their array of spells and skills, and likely also in culture and habits. 

The Rissitic \Ashenoch cannot become \Reavers{}. Using life-draining magic, the \Ashenoch will develop a physical and mental addiction, but they will not transform into undead or gain any of the \Reaveric{} traits described above, no matter how much energy they drain. 

Contrary to popular belief, those drained and killed by \Reavers{} do not rise as \Reavers{} themselves. A \Reaver{} cannot create other \Reavers{}. A person can only become a \Reaver{} through use and abuse of life-draining magic. However, it is possible for a \Reaver{} to create \Leeches{}, undead creatures similar to \Reavers{} but weaker (see section \ref{\Leech}). 

Unlike most undead, \Reavers{} do age, and their bodies will decay and weaken. In fact, \Reavers{} age much faster than the living and may age a decade in a single month. However, a \Reaver{} can use drained life force to rejuvenate himself, thus staying young and immortal potentielly forever. The \Reaver{} can regulate how much he wants to rejuvenate the surface of his body. Most \Reavers{} choose a certain age and maintain their appearance to fit this age. He may change this at any time by allowing himself to age or spending more power to rejuvenate himself. (This is not immediate, but may take several months.) Thus, regardless of his true age, a \Reaver{} may have the look of a youth or an ancient man. Whatever their skin looks like, however, all \Reavers{} make sure to keep their inner body healthy and strong, so the \quo{old man} \Reaver{} may be just as physically strong and agile as the \quo{young man} \Reaver{}. Only when a \Reaver{} is deprived of energy to drain will he begin to weaken. 

A \Reaver{} will die after 10-50 days of not feeding, depending on his strength and how much he exerts himself. But they prefer to feed every day. A \Reaver{} needs not kill his victim, but draining a victim provides much more energy for the \Reaver{} to absorb than merely draining her to the brink of death, because then the \Reaver{} is able to absorb her very soul. A victim thus absorbed is destroyed forever and can never be resurrected. 

\Reavers{} particularly savour the life force of \Leeches{}, or better yet, other \Reavers{}. (These are the only undead that can be drained.) Draining another of the undead is exceptionally nourishing and enjoyable to the \Reaver, but moreover, draining another undead to destruction and absorbing her will cause the \Reaver{} to gain a portion of her power, thus making himself permanently stronger. Occasionally, this has the side effect that the \Reaver{} will adopt some bits of the absorbed one's personality. 

A \Reaver{} cannot eat food nor drink normal drinks. He can swallow it, but his atrophied digestive system rejects it, but will have to regurgitate it soon after (up to ten minutes at most). 









\subsection{Psychology}
It is sometimes said that \Reavers{} can feel no pleasure or happiness and are forever tormented by their foul, unnatural state. It is true that there are some \Reavers{} who come to hate what they have become and succumb to self-loathing and depression. 

But \Reavers{} can actually feel lots of pleasure. Unlike most undead, \Reavers{} retain their sexual drive and can still have and enjoy sex in their undead state (although a fully transformed \Reaver{} is sterile). Draining energy is, to some extent, viewed as a sexual act, and \Reavers{} prefer to drain attractive people of their own race and the opposite sex (or whatever attracts them). Reavers can live off unintelligent animals if they must, but it is much less nourishing and distasteful to them. Draining the life of a beast gives none of the pleasure that comes with draining an intelligent creature. 

%As mentioned, they enjoy sex just as they did in life, and 

Most other passions from life remain in undeath as well. They can also taste food and drink, but this pleasure is marred by the fact that they'll have to vomit it up again. 

%Unlike most undead, \Reavers{} retain their sexual drive and can still have and enjoy sex in their undead state. Once the transformation into undead is complete the \Reaver{} is sterile, though. 

It is uncommon but not unheard of for \Reavers{} to have relationships with other \Reavers{}. More commonly, a \Reaver{} will create \Leeches{} to act as his companions and servants. 









\subsection{Habitat}
Some \Reavers{} live in civilization, where they find ways to avoid the sunlight and lead ordinary-looking lives. Others live removed from civilization and only seek out other people when they must feed. 









\subsection{Myths}
The story of the \Reaver{} is well-known throughout \Miith{}, and myths and superstition about them about in all lands. The vast majority of the storytellers who spread such myths have never met a \Reaver{} (though they may claim otherwise), so it should come as no surprise that much of what is \quo{known} about \Reavers{} is pure superstition. 

According to the tales, \Reavers{} cannot cross running water, cast neither shadow nor reflection and cannot enter a home without invitation. Garlic or a holy symbol of Good repels them. Wolves, rats and bats serve the \Reavers{}, and they can assume the form of any of these animals. 
 
\end{comment}
















\section{Wraiths}
\target{Wraiths}
\target{wraiths}
\index{wraith}
Have a race of black wraiths with blank faces, hair-like protuberances like the Protoss from \cite{VideoGame:Starcraft} and tails like Genie from Disney's \emph{Aladdin}. 

Maybe they are a type of \banes. 























\chapter{Plants and Other Life Forms}
\section{Coral reefs}
\target{coral reefs}
\index{coral reefs}
A coral reef is a collective organism, a colony made up of multitudes of smaller creatures. 
It is hiveminded and intelligent. 
But this intelligence is alien and malevolent to humanoids. 

The \nagae{} know the true nature and power of the coral reefs and fear them. 
Perhaps they worship them. 















\section{Trees}
\target{trees}
Remember to have some mysticism about trees. Trees are cool. They can live for thousands of years and accumulate wisdom and secrets. But they are cold, alien and terrifying. 

Trees hate civilized humanoids, and with good reason, for humanoids cut down trees and destroy the \Wylde{} to carve out their detestable, parasitic cities. 























