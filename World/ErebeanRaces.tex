\chapter{The \Erebean Races}
The \Erebean{} races were the \banes{} and other monsters native to \Erebos{} or \Nyx. 















\section{Atrocity}
\target{Atrocity}
\index{Atrocity}
The Atrocities were the great god-monsters of the \banes, the mighties weapons of the \SitraAchras. 
Each Atrocity was a colossal force and required immense forces and immense sacrifices to bring down. 

\Iscrafel's Shrouding spell was an attempt to destroy or banish the Atrocities and make sure that they would never be able to come to \Miith again. 

The \hr{Lithrim}{\Lithrim} was an Atrocity.















\section{\Bane}
\target{Bane}
\target{Banes}
\index{\banes}
The \banes{} are a terrible race of alien creatures. They are not native \Miith{} but come from the world of \Erebos, sometimes called the \Baneworld. 

The \banes{} were originally one of several races that warred for control of \Erebos. They won in the end by being the coldest, most ruthless and most efficient, almost machine-like. 

They have cruelly exploited their homeworld until \Erebos{} is now a dying husk, almost all its life energy drained away. The \banes{} know that they cannot evolve further while still tied to \Erebos\dash indeed, \Erebos{} and the \banes{} it are doomed to decay and extinction. So the \banes{} are searching for a new planet to leech from, so that they may survive and grow in power. (See also section \ref{Fighting for survival}.)

\citeauthorbook[p.137]{RobertEHoward:TheAltarandtheScorpion}{Robert E. Howard}{%
  The Altar and the Scorpion%
}{
  \ta{The real gods are dark and bloody!
    Remember my words when soon you lie on an ebon altar behind which broods a black shadow forever!
    Before you die you shall know the real gods, the powerful, the terrible gods, who came from forgotten worlds and lost realms of blackness.
    Who had their birth on frozen stars, and black suns brooding beyond the light of any stars!
    You shall know the brain shattering truth of that Unnamable One, to whose reality no earthly likeness may be given, but whose symbol is\dash the Black Shadow!}
    
    The girl ceased to cry, frozen, like the youth, into dazed silence.
    They sensed, behind these threats, a hideous and inhuman gulf of monstrous shadows.
}









\subsection{Biology}





\subsubsection{Entropy and True Death}
\index{Entropy}
\target{Entropy}
\target{Bane parasitism}
The \banes{} are a force of Entropy. 
This is a major theme. 
The \banes{} are inherently destructive and parasitic. 
They can sustain themselves and grow stronger only by feeding upon their own kind. 

This means that the \banes{} are doomed to stagnation and decline. 

The \banes{} are creatures of the cold, darkness and emptiness. 
They steal, absorb and swallow everything. 
This is unlike the \dragons, who \hr{Dragons radiate life}{radiate life and light alike}. 

\hr{Daggerrain}{\Daggerrain} and his master, the \hr{Voidbringer}{\Voidbringer}, understand this. 
They seek to improve their race and let the \bane{} people evolve, to achieve perfection. 
To do this, they must look outwards and find new sources of life, new cosmic \dweomers{} to drain, new power with which to infuse their race. 

This is why they have set their eyes on \Miith{}. 
\Miith{} has the \hr{Heart}{Heart of \Miith}, a powerful \dweomer{}, ultimately based on Chaos. 
Drawing on the life-giving power of the Heart, and working together with \hr{Semiza}{\ps{\Semiza}} sorcerers, \hr{Semiza designs Resphain}{they were able to create} the \hr{Resphan}{\resphain}, who were intended as the new generation of \banes, empowered with the creative force of the Chaotic Heart. 
The \hr{Satharioth}{\satharioth} took this a step forward and stole the Chaotic \xzaishannic{} power from the \dragons, making themselves even greater. They see themselves as the future of the \bane{} race, the heirs of the \bane{} legacy.

According to some myths, the \banes were \hr{Banes are True Death}{agents of the forces of True Death}.





\subsubsection{Life cycle}
\index{cannibalism!\banes}
\target{Bane cannibalism}
The \banes are cannibalistic. 
There are several tiers of them, from the lowly \banespawn{} through \lesserbanes{} to \banelords. 
\Banes{} can rise in the ranks only by devouring the souls of other creatures, including (mandatorily!) other \banes. 
New \banespawn{} are created by sacrificing a \lesserbane{} or a \banelord. 









\subsection{Equipment}





\subsubsection{Bane technology}
\target{Bane technology}
\index{technology!\bane}
The \banes{} command large reserves of spaceships and other technological artifacts salvaged from the \voyagers{} on \Erebos, and perhaps also other cultures whom the \banes{} have conquered. 

During the \hr{FBW}{\firstbanewar}, an important edge that the \banes{} had over the \Miithians was their superior technology. 

The \banes{} do not understand the technology. They are unable to create new things, or even replicate existing ones, and are hard pressed to just repair and maintain what they have. 
This is partly because the \voyagers{} were infinitely more advanced than their \bane{} spawn, and partly due to their being \hr{Entropy}{a force of Entropy, afflicted with stagnation and decay}. 
After all, 
\hr{Semiza}{\Semiza} had to 
\hr{Semiza designs Resphain}{help them design and create} their \resphain.

The \resphain are \hr{Resphan technology}{much more creative}.





\subsubsection{Space travel}
The \banes{} have spaceships. 
These are stolen from the \voyagers{} and tens of thousands of years old, if not millions. 

\Bane{} space travel is slow, partially because the \banes{} do not wholly understand the technology (including the navigational magic) and partially because the ships are old and damaged and the \banes{} can't repair them. 









\subsection{History}
\target{Bane history}





\subsubsection{World-God}
\target{World-God of Erebos}
\target{Voyagers slay the World-God of Erebos}
\Erebos was originally a \hs{World-God}, like \Miith.
Also \hr{Voyagers slay the World-God of Miith}{like \Miith}, the \voyagers invaded and eventually slew the World-God of \Erebos.
They chained the World-God's Dark Heart and created their own \dweomer on top of it.





\subsubsection{Origin of the \banes}
\target{Banes are created}
\target{Cosmic god creating the Banes}
\target{Cosmic god creates Banes}
\target{Evil Cosmic God creates Banes}
The Dark Heart of \Erebos rebelled against the invading \voyagers.
It took some of the \voyagers captive and corrupted them (like how Melkor created the Orcs in \cite{JRRTolkien:Silmarillion}. 
The World-God twisted took these creatures' love and creativity and twisted them for its own purposes.
They became the \banes. 
The undead World-God instilled the \banes with a neverending hatred against the \voyagers and all their creations. 
The World-God desired to destroy the \psp{\voyagers} civilization and corrupt their beloved children into a parasitic, nightmarish scourge to ravage the universe, destroying all in their path and leaving only Entropy behind. 

\target{Banes as an antibody}
In order to create this massive hatred, the World-God drew upon a vast cosmic principle, namely that of \hs{cannibalism}, the universal urge to betray, destroy and devour one's own kin, creator or parent. 
As such, the \banes{} are a physical manifestation of a cosmic force of destruction and devourment. 
In a sense, the \banes are a cosmic disease\ldots{} or cosmic antibodies, fighting against the infestation that is life.

Compare them to the Zerg Overmind from the game \cite{VideoGame:Starcraft}. 

\textbf{Note:} 
This was not a singular event that only happened because \Erebos was particularly nasty.
This is a necessary event that would have happened everywhere.
The \voyagers were doomed to decay into monstrosities. 
The \banes are what the \voyagers inevitably had to evolve into if the \voyagers were to continue their progress. 
This is a necessary consequence of the nature of the \hr{Necrocosmos}.
Later the \hr{Tiamat's hivemind is like the Banes}{same would happen to \Miith with \Tiamat's hivemind}.

\lyricsbalsagoth{The Sixth Adulation of His Chthonic Majesty}{
  I now know that there is something inestimably evil at large throughout the cosmos.\\
  It is a ravenous and pitiless storm which rages across the universe, permeating the very fabric of creation, existing simultaneously in all dimensions, wholly unconstrained by linear time.\\
  This force is the black, quasi-sentient mana which sustains such ageless revenants of the Z'xulth as the dread Dwellers in Eternal Shadow and the unspeakable They-Who-Lurk-And-Breed-In-Limbo.\\
  When beings whose essences are intrinsically malefic choose to embrace this darksome energy source, the resultant sinister symbiosis can be sublimely diabolical, as evidenced by the black blight that was the infamous pseudo-human sorcerer Lord Angsaar!
}





\subsubsection{Reconquest of \Erebos}
\target{Banes destroy Voyagers}
The \banes ultimately destroyed the \voyagers.

Then they waged wars of genocide and destroyed almost all life on \Erebos. 
Thus they destroyed the \voyagers' \dweomer and freed the Dark Heart of \Erebos. 
Alas, the Dark Heart had been irreversibly marred and could not be restored to what it was.
It had been afflicted with an incurable disease. 

This disease manifested itself as the \hr{Umbra}{\umbrae} and their spawn, the \hs{Nether Ones}.
They arose and began to pray on the \banes.
They could never destroy the \umbrae, for they were part of the World-God itself, born of its inner rot. 

\Erebos{} was not always a Realm of darkness. 
Once it had a sun that shone as brightly as any other. 
But now, with the \voyagers' \dweomer extinguished and only the rotting Dark Heart remaining, their sun had turned black and all life had gone out of \Erebos.
It was a dying husk.
\Erebos{} and the \banes{} were doomed to decay and extinction. 

Thus the World-God sent its \banes out in the universe.
It wanted to conquer a new world where its spawn could live on.
So the \banes{} were searching for a new planet to leech from, so that they may survive and grow in power. 
(See also section \ref{Fighting for survival}.)





\subsubsection{\Banes deplete the Heart of \Erebos}
\Erebos{} was not always a Realm of darkness. 
Once it had a sun that shone as brightly as any other. 
But for tens of thousands of years the \banes{} have sucked the life out of their \hr{Dweomer}{\dweomer}, the \hr{Heart of Erebos}{Heart of \Erebos}. 
And now their sun has turned black. 





\subsubsection{\Banes hunted by \umbrae}
\target{Banes and Umbrae}
The \umbrae were \hs{incarnations of death and horror}, and even the \banes feared them. 
This was the reason why the \banes wanted to invade \Miith so badly:
Their homeworld of \Erebos was slowly being devoured and laid waste by the \umbrae.
The \umbrae hunted the \banes, so the \banes had to flee their planet. 

Compare the \umbrae to the Dholes/Bholes from the Cthulhu Mythos, in how they burrow into a planet and destroy it from within.





\subsubsection{Interstellar empire}
\target{Interstellar Bane empire}
At the time of the \firstbanewar, the \banes already had an interstellar empire (just like the \ophidians did). 
They had been expanding for thousands of years, conquered dozens of planets, absorbed the local life and used it to evolve themselves, and then having moved on, leaving behind a barren, dead ruin, infested with wretched, degenerate survivors. 

\target{Why the Banes want Miith}
But \Miith was the important planet that they had been searching for all along. 
It had been settled and populated by the \voyagers, their own creators. 
It contained a Heart full of \voyager-tampered energy, and the \hr{Noggyal}{\noggyal} \hs{mother-mass}, which was the \quo{other half} that the \banes wanted. 

Fortunately for everyone else, the \banes had only primitive space travel, and they were un-creative.
So their interstellar empire could spread only slowly, and not far.
If the \bane swarm spread too far apart, the farthest parts withered and died.
They needed the \noggyal \hs{mother-mass} in order to truly expand their dominion. 





\subsubsection{\Firstbanewar}
The \banes{} \hr{First Banewar}{invaded \Miith}. 
They were defeated. 





\subsubsection{\Secondbanewar}
\hr{Resphan rebellion}{Some \resphain{} rebelled against \Merkyrah}. 
They won.
They summoned the \banes{} again. 
The \hr{Second Banewar}{\secondbanewar} began. 

Eventually \hr{End of the Second Banewar}{everyone lost}. 
The \banelords{} were gone again. 





\subsubsection{\Banelords{} imprisoned}
\index{\CrystalSphere}%
\Tiamat-tachi successfully banished the \banelords{} from \Miith{}. 
They were frozen in the icy \hr{Crystal Sphere}{\CrystalSphere}. 

\Semiza{} and \Thanatzil{} tried to free them but failed. 

The \resphain{} (just before the \Secondbanewar) managed to free some of them, but not all. 

The \banelords{} that were buried deepest (including \Daggerrain) have been frozen for ten thousand years. 

Compare to the devils in the manga \cite{NagaiGo:Devilman}. 

The \lesserbanes{} were also blocked out by the \CrystalSphere. 
They could not come to \Miith{} from \Erebos; however, many \lesserbanes{} remained frozen in the \CrystalSphere{} and could be brought to \Miith{} (such as by \hr{Banes possess Humans}{possessing a \human{} body}). 






\subsubsection{The return of the \banelords}
\target{Return of the Banelords}
\index{\CrystalSphere}%
As the Shroud \hr{unravelling}{unravelled}, the \CrystalSphere{} \hr{thawing}{began to thaw}. 
Some of the \banelords{} were able to break free of their icy prison and could now walk on Mith again. 
In the decades before the \hr{TBW}{\thirdbanewar}, the \banelords{} returned and resumed command of the Cabal, demanding that all \resphain{} bow to them. 
Not all \resphan{} lords were happy about this. 

The \dragons{} had long feared that the \banelords{} might return. 

\lyricsbalsagoth{Invocations Beyond the Outer-World Night}{
  But, it is here written that one day, when even the War of the Lexicon and the cataclysmic Great Chaos War have faded to naught but distant memory, a great conflict shall be waged between the forces of Order and the dread avatars of the Z'xulth. \\
  Vile fiends of the Outer Darkness, They-Who-Lurk-And-Breed-In-Limbo, the Dwellers in Eternal Shadow unleashed through The Gate to That Which Lies Beyond! \\
  The Black Galaxy disgorges its malignant horrors! \\
  Mankind shall suffer inestimably at the hands of these sinistrous black titans of maleficent Chaos!
}









\subsection{Name}
Singular \emph{\bane{}}, plural \emph{\banes{}}. 

The \bane{} race as a whole was also called \quo{the \Bane}. 

The name \quo{\bane} was \hr{Semiza names Banes}{invented by \Semiza}. 

The \dragons originally had \hr{Draconic names for Banes}{other names for the \banes}. 

\target{Sitra Achra}
\emph{\SitraAchras} was an alternate \resphan name for the \banes. 
It means approximately \quo{those from the ouside} in the \Resphan tongue 

\emph{\SitraAchra} is Hebrew for \quo{external forces}, \quo{other side}, \quo{side of evil}. 









\subsection{Personality}





\subsubsection{Hivemind}
The \banes had a hivemind. 
The overmind was the \baneking \Voidbringer.
There was only one \baneking. 





\subsubsection{We}
\Banelords, when talking, will always say \quo{we} instead of \quo{I}. 
This reflects their communal nature. 
\Banelords{} have little individualism and think about the collective instead. 

The only times they use \quo{I} is when referring specifically to the individual \banelord, usually for some physical purpose 
(as in \quo{I am not strong enough to counter this \dragon\ldots{}}). 









\subsection{Physique}





\subsubsection{Types of \banes}
The \SitraAchra race is split into multiple \quo{tiers} or \quo{castes}. 
These include: 
\target{Banelord}

\begin{description}
  \item[\Banekings] are the supreme lords of the \bane{} people. 
    Only one \baneking{} is known, namely the dread \hr{Voidbringer}{\Voidbringer}. 
  
  \item[\Banelords] are as powerful as gods or \dragons. 
    Only a few dozen \banelords{} have ever come to \Miith{}. 
    
  \item[\Greaterbanes] are tall, stately and fearsome. 
    Compare them to the Nazg\^ul from J.R.R. Tolkien's \emph{Lord of the Rings} or the Myrddraal from Robert Jordan's \emph{Wheel of Time}. 
    Also called \quo{\baneknights}. 
    
  \item[\Screamers] are flying monsters with bodies specially designed to combat \dragons.
    They are as intelligent as \lesserbanes. 
    
  \item[\Lesserbanes aka \stalkers] are short and wicked. 
    A \lesserbane{} is approximately as physically strong as a \resphan, but most \lesserbanes{} know no magic. 
    
    \citeauthorbook[p.50]{LeeClarkZumpe:PassagetoOblivion}{%
      Lee Clark Zumpe%
    }{%
      Passage to Oblivion%
    }{
      \ta{%
        Those are only shadowmen\dash pathetic minions of the overlords, the foot soldiers in an infinite army of godlike warriors.}
    }
    
  \item[\Banespawns] are crawling, worm-like vermin. 
\end{description}





\subsubsection{Brain}
\Banes{} have their brain in the torso rather than the head. 
They have some gill-like openings in the body that ventilate the brain. 

When they carry \armour, it is split-mail-like stuff made of overlapping plates, so as to let in some air so their brains can breathe. 





\subsubsection{Cold aura}
\target{Banes are cold}
\Banes{} are cold as death. Colder, in fact. Their body temperature is very low, below room temperature, near the freezing point. They spread deathly cold around them that seems to sap the life of creatures around them. 

This cold is one of the tell-tale signs that a \bane{} is nearby. When the \bane{} is submerged in \Nyx{} the cold cannot be felt, but as it surfaces, the cold spreads out like a draught. 





\subsubsection{Coming to \Miith through a \human}
\target{Banes possess Humans}
\Banes{} cannot come to \Miith{} directly. 
Instead, a \bane{} has to possess a \human{} body. 
(It must be a \human{} or \resphan. No other creatures will work.) 
The \bane{} takes the body and twists it into a \bane{} body. 
The \human{} is killed and \hr{Life drain}{his soul consumed} in the process. 

The victim needs not be a cool \human. 
It can equally well be a total loser. 
The \bane{} just needs the body as a shell. 
It then creates its own \bane{} body from it. 

Compare to the Agents from \emph{The Matrix} movies, or the parasitic plant from \authorbook{Clark Ashton Smith}{The Seed from the Sepulcher}. 





\subsubsection{Eat souls easily}
\target{Banes eat souls easily}
The perhaps most terrifying power of the \banes was their ability to \hr{soul-eating}{eat souls}. 
They needed no elaborate spells to do it. 
It was an inborn, natural ability of theirs. 
They could eat souls just by killing people\dash even powerful immortals such as \dragons and \resphain. 

Their \resphan allies knew this, and \hr{Resphain fear Banes}{it scared them shitless}. 

The \banes{} were terribly dangerous \trope{CosmicHorror}{Cosmic Horrors}. 





\subsubsection{Face}
A \bane{} has no face.
The front of its head is completely blank and smooth. 
This is very scary.

\Banelords{} have \armoured heads with frills, kind of like the Xenomorph Queens in the \movie{Alien} movies. 
Also compare to the Yautja Predators in the \movie{Predator} movies. 




\subsubsection{Fear rain}
Banes dislike rain and fear hail.
It is painful to their telekinetic sense.
They will retreat out of hail and not go out in it unless they absolutely must.






\subsubsection{Flying}
Greater \banes can fly using a number of long limbs resembling insect antennae with feeling-hairs on them.





\subsubsection{Giant form}
\Banelords{} are able to transform into a gigantic, monstrous \quo{combat form}\dash useful if you need to fight \dragons{} or other large monsters. 

The monstrous form looks bat- or ray-like. Like the \quo{Balrog} in the movie \emph{Eragon}. 





\subsubsection{Gravity}
\Banes were made of alien matter, like the mi-go from the Cthulhu Mythos. 
As such, the laws of nature affected them differently than \Miithian creatures. 
They were not held fast by gravity in the same way.
They were still affected by gravity, but they could utilize alien geometry and thus seemingly violate it. 

This means \lesserbanes could walk on ceilings, and they could run really fast through corridors by leaping from wall to wall. 
But this also had drawbacks. 
A \bane could not leap easily without something to leap onto\dash it would just drift off into space and lose any ability to \manoeuvre. 

Outdoors, \lesserbanes would walk on four legs. 
With only two legs they would have difficulty gaining proper speed while keeping their footing. 
Indoors, with walls on all sides, they could run securely on two legs. 





\subsubsection{Horror}
The \banes{} are a \trope{CosmicHorror}{Cosmic Horror}. 
Even the \resphain{} fear them. 
Even \lesserbanes{} are frightening and loathsome, and \resphain{} shy away from them.

When \humans{} see a \bane{} and especially its blank face, they see themselves reflected back at them in that empty visage. 
It is cruel and horrible, because it hints at the fact that \humans{} are descended from \banes. 
It conjures up vague feelings of hideous truths about yourself that you do \emph{not} want to face. 

The fear that the \banes{} invoke is meant to parallel, among other things, the fear of decay, boredom, hopelessness, decrepitude, illness, old age.
The fear of being alone and abandoned and left to wither unseen. 

\lyricsbs{Cradle of Filth}{Bathory Aria}{
  The Spirits have all but fled judgement.
  \\
  I rot, alone, insane,
  where the forest whispers puce laments for me
  from amidst the pine and wreathed wolfsbane
  beyond these walls, wherein condemned
  to the gloom of an austere tomb
  I pace with feral madness sent
  through the pale beams of a guiltless moon
  who, bereft of necrologies, thus
  commands creation over the earth. 
  \\
  Whilst I resign my lips to death,
  a slow cold kiss that chides rebirth. \\
  Though one last wish is bequeathed by fate. \\
  My beauty shalt wilt, unseen save for twin black eyes that shalt come to take my soul to peace or Hell for company.
  
  [Quoted words above are from Hammer Film's \quo{Countess Dracula} (1970). The singer is Imgrid Pitt, the actress who played the role of Elizabeth in that film.]
}

The \banelords were indescribably dreadful to behold. 

\citeauthorbook[p.257]{HPLovecraft:TheBlackTomeofAlsophocus}{H. P. Lovecraft}{%
  The Black Tome of Alsophocus%a
}{%
  The air was filled with the sound of their titterings and screamings as they danced obscenely and capered around me in a blasphemous ritual of depravity: and at the far end of the hall was the most terrifying sight of all, that dread black colossus of my visions, the inhabitant of the palace, Nyarlathotep.
  
  The Old One looked upon me intently, his gaze tearing at my soul and filling me with a horror so terrible that I screwed my eyes shut so as not to see that terrible visage of unnameable evil.
  Under that gaze my being began to melt away, as if it was being absorbed by some irresistible force.
  I was losing what little identity was left to me; my necromantic powers, which I now realized were as nothing compared to the powers of the inhabitant of this dark world, were stripped from me and scattered across the universe, never to be recovered. 
  
  Under that gaze my mind and soul were attacked from all sides by fear and loathing; I staggered as he tore at my being, peeling away my life layer by layer. 
  Sheer desperation took hold of me, but I was powerless to fight, unable to hold back the irrestible force that overwhelmed me.
}





\subsubsection{Limbs and posture}
\Lesserbanes had six limbs. 
The middle pair sits midway between the arms and legs. 
This pair can be used as extra arms or extra legs. 

The legs (both pairs) are sprawled out a bit, kind of like on a \meccaran. 
But a \bane{} has a small, slim, compact body where a \meccaran{} has a bigger belly. 

Maybe they have small, round, hooflike feet. 

In combat they draw their heads down between their shoulders to protect their sensory organs. 
In peacetime they hold their heads high for a better view.

\Greaterbanes had many insect-like legs, like the monster from \cite[p.55]{TanakaHirofumi:TheSecretMemoiroftheMissionary}. 





\subsubsection{Mouths in their hands}
\Banes{} have mouths in the palms of their hands. 
Their four fully opposable fingers are very rough and also act as a sort of mandibles or tongues. 

Maybe their telekinetic feelers are in the hands. 





\subsubsection{No bodies of their own}
\target{Banes have no bodies}
An idea:
\Banes had no bodies of their own.
They appeared in corporeal form by possessing a creature and mutating the host (permanently) into a \bane monster. 
Stalkers were made by possessing humans.
\Screamers were made from \lindworms or some other flying monster. 

There existed a type of might \bane angels created from \resphain.
Actually, this might be the purpose of the \hr{Neo-Resphan}{\neoresphain}. 

There also existed \ophidian \banes and even \draconian \banes, the most hideous and terrible of all. 

Compare to the monsters in the film \cite{Movie:IntheMouthofMadness}. 





\subsubsection{No bones}
\Banes had no bones.
Their bodies were flexible and could squeeze through very small openings (although they would have to leave any solid items behind). 





\subsubsection{Shrouded in darkness}
Because of their connection to \Nyx{} and \Erebos, when they appear in \Miith{} \banes{} are always \hs{Shrouded} in an unnatural aura of darkness. 

\lyricsbalsagoth{Six Keys to the Onyx Pyramid}{
The fiends seemed inexplicably to be an extension of the night, as if their misshapen bodies were actually somehow composed of the darkness itself. 
Even as I gazed directly at them, I found I could not truly focus on their stygian forms\ldots{} 
their bodies appearing to shimmer and shift like the ripples of a heat-haze upon an arid plain.}





\subsubsection{Telekinesis}
\target{Bane telekinesis}
\Banes{} have no sense of vision and cannot see. 
In its place they have a long-distance tactile sense. 
It is similar to bats' sonar sight, but based on telekinesis rather than sound. 

This sense lets them see perfectly in darkness, but water blurs it, and glass blocks it completely. 
Dense smoke or debris also blurs it. 
So do nets and grids and the like\dash their \quo{feelers} have a certain granularity. 

Their high sensitivity to touch is the reason why \banes{} shun the rain. 

The \banes{} native languages are kinetic, not sound-based. 
That is why they adopt somatic names like \quo{\Daggerrain}. 

Remember that even feeble \lesserbanes{} should have telekinetic powers. 





\subsubsection{Weaknesses}
\Banes{} have a certain fear of rough weather. 
Rain, snow and strong wind scares them, and they avoid it if possible. 
Hail and sandstorms are dreadfully painful and will scare them away. 

This is due to their tactile senses, which render them very sensitive to touch-based things. 








\subsection{Politics}





\subsubsection{\Umbrae}
See the section about \hr{Banes and Umbrae}{\banes and \umbrae}. 















% \begin{comment}
% \section{\Baneknight}
% The \banes{} most commonly summoned are the \baneknights{}. For this reason, they are typically referred to simply as \quo{\banes}. \Banespawn{} are rarely summoned, because they are difficult to communicate with and control, but a summoned \baneknight{} may sometimes bring along \banespawn{} as its own servants. 
% 
% In the following, \quo{\bane} will refer to the \baneknights{}. 
% 
% 
% 
% 
% 
% 
% 
% 
% 
% \subsection{Name}
% Singular \pronune{\bane}{BEJN}, plural \emph{\banes}. The word and grammar is English. 
% 
% 
% 
% 
% 
% 
% 
% 
% 
% \subsection{Physique}
% Seen from a distance, a \bane{} looks like a humanoid in a long, flowing robe, frayed and tattered at the edges. The robe is actually the creature's body, and the tattered egdes are small pseudopods used for walking. It has a head, torso and a number of arms (see below), but no legs, only a number of pseudopods or tentacles (each 20-50 cm long).  
% 
% Typically, a \bane{} will have two arms, but some have more, and they seem able to sprout new arms from their body as needed and retract them again when not needed. It is believed that growing additional arms and maintaining them is a psychically taxing, so that the \bane{} does not use it more than it has to. \Banes{} with only one arm have been sighted, but they typically maintain at least two. A newly sprouted arm will be weaker and less dextrous than normal\footnote{Perhaps a $50\%$ penalty to strength and a $30\%$ penalty to dexterity} and will grow to full strength gradually over a few minutes, perhaps 2-5 minutes. A \ps{\bane}{} arms do not have elbows; they are fully flexible and should perhaps be called tentacles. At the end an arm splits into a number (variable, but usually around four) of small \quo{fingers}. \Banes{} can also sprout new fingers and other appendages at will, and create bladed claws from their fingers. 
% 
% \Banes{} crawl only slowly on their tentacles. Maximum running speed is like a fast walk for a \human, and typical walking speed is like a slow walk. They can use magic to leap, fly or teleport, but this costs them valuable mana and casting time, so they only use it if necessary. 
% 
% \Banes{}' height varies from 170 to 250 cm. Their body is fully corporeal and solid and slightly denser than the flesh of most \Miithian{} creatures. Weight varies from 120 to 250 kg. 
% 
% The head has no recognizable features, including eyes. Sometimes you can feel the \quo{gaze} of the \bane{} upon you, but this is actually a psychic sensation of the \bane{} telepethically peering into your mind - a very unsettling experience, if not outright terrifying. The \ps{\bane}{} head does seem to contain vital organs, however, as a powerful blow to the head will sometimes kill the creature. 
% 
% The \ps{\bane}{} skin is tough and leathery on the body, head and major appendages. On the smaller or more recently generated limbs, the skin is more soft and supple. The creatures sometimes wear visible \armour, made of an alien metal and adorned with grotesque and hideous patterns and symbols. Some people have reported markings on a \ps{\bane}{} \armour resembling stylized drawings of known alien monsters. 
% 
% \Banes{} are \coloured in pure black, but wounds sometimes reveal blue-gray \quo{flesh} inside. A cut also causes a thick, pale white vapour to pour forth. This \quo{\baneblood} is heavier than air and will spill onto the ground. Large amounts of \baneblood{} spilt will leave a pale gray stain and kill vegetation in a small area. \baneblood{} smells foul but is not known to have any harmful effect on animal life. The blood may have magical properties if collected, but this is only speculation. 
% 
% In combat, \banes{} will typically strike in close combat with weapons. The weapons they typically wield are wicked swords and daggers, maces studded with spikes and blades, and (less often) strange weapons resembling a whip or flail. They seem to prefer close combat; if possible, a \bane{} will always strike its killing blow in melee. It is speculated that they \hr{Life drain}{drain life energy} from their opponents this way. \Banes{} are sometimes seen pulling weapons seemingly out of their bodies, including large weapons that it ought not be possible to conceal. If disarmed, a \bane{} will pull out a new weapon if it has one, or strike with its bare tentacles (forming claws at the end). A \bane{} will never pick up and use a \Miithian{} weapon, with the rare exception of some magical artifact. 
% 
% \Banes{} very much prefer to fight in close combat. All \banes{} are mages and can attack at a distance using magic, but they are reluctant to do so. They may do it in order to kill or subdue a fleeing victim if vitally necessary, but otherwise, \banes{} will only strike at range in order to defend themselves. They never wield ranged weapons such as bows. 
% 
% What senses do \banes{} have? They don't have smell and taste. Do they have vision and hearing? Do they have telekinetic and \hr{Telepathy}{telepathic} sense? Perhaps even more exotic senses? 
% 
% \Banes{} are somewhat resistant to heat/fire and cold attacks, and with their alien biology, they are immune to \Miithian{} poisons and diseases. Electricity and acid do full damage, as do attacks with physical weapons. They do not breathe, and as such, cannot be strangled or suffocate. 
% 
% 
% 
% 
% 
% 
% 
% 
% 
% \subsection{Biology}
% \Banes{} are from an alien world called \Erebos. Only a very few people from \Miith{} have ever visited \Erebos, so little is known of it. The Chronicler of Nom did not visit the \Baneworld{} (as far as is known), but he did communicate with \banes{} and learn something of it. He speaks of tremendous cities, both hideous and beautiful at the same time, of colossal monolithic towers and castles, and of massive armies of \banes{} and monsters fighting terrible wars, apparently against other \quo{nations} of \banes{} (nations presumably ruled by \banekings) as well as other monstrous races. He even speaks of dark temples where the \banes{} worship alien gods in grotesque rituals. 
% 
% \Banes{} do not eat, drink or breathe. It is unknown what they live on, if anything. It is believed that they \hr{Life drain}{drain life energy} from other creatures, for a summoned \bane{} will sometimes attack and kill living creatures, preferably intelligent creatures, for no apparent reason. 
% 
% \Banes{} do not seem to have genders. It is unknown how they reproduce, but the Book of Nom states that all \banes{} are born as \banespawn{} and slowly advance through the ranks. The Book seems to hint that the lower ranks of \banes{} (possibly only the Spawn) have finite life spans while the higher ranks are immortal. 
% 
% When slain, a \bane{} will dissipate into a pool of putrid \baneblood, leaving a dark and tainted stain of many square meters. On such a \bane{} \quo{tomb}, no wholesome vegetation will grow for decades (although it is sometimes infested by hideous, abnormal fungi and mosses), and there is a distinct atmosphere of unnatural evil. Seeing a \bane{} tomb has a Minor Horror Effect. 
% 
% 
% 
% 
% 
% 
% 
% 
% 
% \subsection{Psychology}
% Little is known about the \banes{}' mindset. They are alien creatures not like anything on \Miith{}. 
% 
% \Banes{} have no voices and cannot speak, but they can hear and understand speech.\footnote{Or can they? Maybe they can't hear, but use \hs{telepathy} to scan people's minds.} If a \bane{} needs to communicate, it will do so telepathically. \Banes{} encountered on \Miith{} are always able to understand one or more \Miithian{} languages, typically Kingstongue and/or Ancient Vaimon. It is believed that the majority of \banes{} know little to nothing of \Miith{}, and that those successfully summoned are always \quo{learned} \banes{} who have studied \Miith{} and its creatures. 
% 
% When summoned, a \bane{} must be bargained with. No magic is known for enslaving \banes{} and controlling them against their will, and attempts to do this typically result in the summoner dying a gruesome death. Typically, the \bane{} agrees to perform some service in exchange for a gift. Suitable gifts are live sacrifices (large numbers of intelligent creatures), arcane knowledge and magical items. Especially prized are items made to combat \banes{}. It is believed that the \banes{} covet these items partly to prevent others from using them against them, and partly to use them in their wars against fellow \banes{} on \Erebos. A \bane{} may also demand that the summoner perform some obscure magical ritual for it (typically refusing to reveal the purpose of it). Sometimes, a \bane{} will crave strange things in return, such as a large quantity of a particular mineral, or a specific person to be captured alive. 
% 
% Communicating with a \bane{} is a horrifying experience. People have reported that hearing the telepathic voice of a \bane{} inside your head feels like a terrible violation of the mind. One person described it as the equivalent of \quo{being pinned down, stripped naked and having someone carve their message in runes on your chest with a knife}. Such an experience will have a Moderate Horror Effect. That being said, \banes{} do not communicate much. They will not, for instance, comment, taunt nor threaten their opponents in combat. The only ones likely to ever hear the \ps{\bane}{} voice are the summoner and his allies. 
% 
% If asked questions, \banes{} will often refuse to reply. They have no recognizable concepts of politeness, so if asked a question it will or can not answer, the \bane{} will typically not explain but simple ignore it. Even if it deigns to answer, asking questions about the \banes{} themselves and their homeworld is risky business, as the \bane{} will tend to show rather than tell, telepathically showing images of the \baneworld{} and the \banes{}' life. Such an experience will have a Major to Extreme Horror Effect. (It should be noted that the \bane{} cannot use this as an attack. For the \bane{} to \quo{explain} something to you, you must be willing to accept the \quo{transfer}.) 
% 
% \Banes{} have no recognizable morals and principles. A \bane{} can be stealthy if ordered to, or has a reason to fear being discovered, but they will kill and destroy anyone and anything if they must, and sometimes will for no discernible reason. 
% 
% It is unknown how the \banes{} communicate among themselves. It might be telepathically, or using some strange senses that \Miithians do not have. Their own names cannot be translated into spoken languages. If it needs to communicate with \Miithian{} creatures, a \bane{} will sometimes adopt a spoken name for others to use, or allow its summoner to give it a name. Rissit Nechsain typically gives his \banes{} such names as Direfrost, Illwinter or Coldscar. 
% 
% \Banes{} wield great magical power, but it seems that they do not use it to its full effect. For example, they can attack at a distance using magic, but will not do so unless absolutely necessary. Some scholars believe that \bane{} magic is simply mana-expensive, so that they are not able to use it very often. Others believe that the \banes{} are bound by some alien code of \honour or religion (perhaps superstition) that restricts their actions. Yet others speculate that perhaps the \banes{} fear to use their magic because it is unnatural and frightening to them, like the way many \Miithian{} mages feel about their magic. 
% 
% 
% 
% 
% 
% 
% 
% 
% 
% \subsection{Habitat}
% \Banes{} may be found in any terrain. 
% 
% 
% 
% 
% 
% 
% 
% 
% 
% \subsection{Attributes}
% \begin{description}
%   \item[Horror effect:] 
%     Moderate to see the \bane{} close up, Minor if seen from a distance.
%     
%     Minor to see a \bane{} \quo{tomb} (see under Biology). 
%     
%     Moderate to hear the \ps{\bane}{} telepathic voice. 
%     
%     Major to Extreme to listen to a \bane{} explaining about the \baneworld. 
% \end{description} 
% \end{comment}















\section{\Flyingpolyps}
\target{Flying polyps}
\target{flying polyps}
\target{Flying polyp}
\target{flying polyp}
\index{\flyingpolyp}
There dwelt flying polyps beneath \Nyx and \Erebos, where they burrowed and slithered.

They were some of the \hr{Horrors of the Void}{horrors which even the \resphain feared}. 
Even the \banes feared these monsters.

Compare them to:
\begin{itemize}
  \item The flying polyps from \cite{HPLovecraft:TheShadowOutofTime}. 
  \item Beholders and illithids from \cite{RPG:DungeonsandDragons}.
  \item The Tyrant Worms of my older ideas.
\end{itemize}











\subsection{Biology}

An idea is that the \noggyaleth{} feed on magical power, so \ps{\Teshrial} \noggyaleth{} are only truly formidable if the Sentinels show up with some great power. Or maybe they feed on \vertices. That might be evil. 

I need to think more on how exactly this works. 

They were asexual and reproduced by budding.





\subsubsection{Habitat}
The \flyingpolyps dwell in the depths below \Nyx. 
The \quo{bottom floor}. 
They slither and burrow in dimensions that exist parallel with the spires of \Nyx and gradually lead to the surface of \Erebos. 
(\hr{Nyx is above Erebos}{\Nyx exists in the skies high above \Erebos}.)
These dimensions are inaccessible and non-permeable to the \resphain and \banes. 
Only the \flyingpolyps can cross them due to their exceptional burrowing abilities. 

It is their kind that have carved out \Erebos, leaving only a mass of twisted spires and no ground. 
The also dug out \Nyx. And now they are in the process of undermining \Miith{}, turning it into a dead husk, an empty shell. 

In the mystic gloom of the deep abyss underneath \Nyx, you can sometimes hear the writhing of the horrible \hr{Ghobal}{\noggyaleth}, or feel the tremours of their passing and their burrowing.
Once in a rare while you can feel a tower faintly trembling. 
This happens when a \noggyal{} violently collides with the \CrystalSphere{}\dash the monster scrapes the edge of the Sphere but fails to penetrate. 















\section{\Morkin}
\target{Morkin}
\index{\morkin}
The \morkins were a race of semi-humanoids that dwelt in \Nyx.

Compare them to the orcs of \cite{RPG:Warhammer}. 









\subsection{Biology}
They reproduced quickly and were voracious. 





\subsubsection{\Ozurians}
The \morkins might be related to the \ozurians. 
The \ozurians lay their eggs in \morkin bodies, and out of the eggs come \ozurians as well as \morkins. 









\subsection{History}
The \morkins were native to \Nyx.
When the first \resphain arrived, the \morkins had already dwelt in \Nyx a million years.
They were perfectly adapted to the place, where the \resphain were ill-adapted outsiders. 









\subsection{Imagery}
The \morkins were wicked and vicious and evil, like the akaanas in 









\subsection{Physique}
\Morkins were semi-humanoid.
They had webbed skin between their arms and legs, allowing them to glide on the wind.
They had very strong hind legs and could leap from a tower and glide to the next tower. 
They were also great climbers and could hunt in total darkness.









\subsection{Politics}





\subsubsection{\Resphain}
The \resphain hated and feared the \morkins.
The \morkins were one of the many monster races that plagued the \resphain. 









\subsection{Psychology}
They were semi-intelligent tool-users that formed tribes and hordes. 















\section{Nether Ones}
\target{Nether One}
\target{Nether Ones}
\index{Nether Ones}
The Nether Ones, also called \mothlain, were a race of monsters that dwelt in the \hs{Lower Halls} of \Nyx. 
They hated the \resphain and continually tried to invade the \hs{Upper Halls} where the \resphain dwelt. 
The \hr{Resphan Protector}{Protector} order was founded to combat the Nether Ones. 








\subsection{Name}
Singular \emph{\mothlan}, plural \emph{\mothlain}. 









\subsection{Politics}





\subsubsection{\Umbrae}
See the section about \hr{Umbrae and Nether Ones}{\umbrae and Nether Ones}. 


















\section{\Ophan}
\target{Ophan}
\index{\ophan}
\Ophanim were creatures that lived in \Nyx.
They were actually creations of the \banes that had migrated to \Nyx.
Here they came to be enslaved by the \resphain.
They were living chariots with wheels and lots of eyes.



















\section{\Ozurians}
\target{Ozurian}
\target{Ozurians}+)
\target{spider-people}
The \ozurians were a race of spider-centaur-like beings. 
They dwelt in \Nyx. 
Here they had been among the dominant races before the coming of the \resphain. 
They were highly intelligent and wielded powerful magic. 



















\section{\Screamer}
\target{Screamer}
\index{\Screamer}
\Screamers were a kind of \banes. 
Their forms were designed to better combat \dragons. 









\subsection{Biology}
\Screamers were regular \banes reshaped into special forms. 









\subsection{Physique}
A \screamer looked kind of like the Xenomorphs from \emph{Alien} and \emph{Alien versus Predator}. 
They had wings and could fly, like Zerg Mutalisks from \emph{Starcraft}. 
And great blades, like Zerg Hydralisks from \emph{Starcraft}. 
They were extremely fast and agile, built to evade a \dragon's melee and spell attacks.









\subsection{Psychology} 
\Screamers possessed humanoid intelligence, but they did not learn spells. 
They were supported by \lesserbane, \baneknight or \banelord sorcerers. 



















\section{Thorn Angel}
\target{Thorn Angel}
\index{Thorn Angel}
A race of creatures, probably not native to \Miith{}. 
There exist only a score or so of them. 
They form a single band. 
Their leader is \Hiothrex{}, and they all serve the Imetrium. 















\section{\Umbra}
\target{Umbra}
\index{\umbra}
A monster native to \Erebos. 









\subsection{Biology}






\subsubsection{Reproduction}
\target{Umbra reproduction}
No one knows how, or even if, \umbrae{} reproduce. 
It is unknown if they reproduce \hr{Wild Umbrae}{in the \Wylde{} on \Miith}, or if all \umbrae{} hail directly from \Erebos. 





\subsubsection{Soul-eating}
\target{Umbrae eat souls}
\Umbrae{} fed on lifeforce. 
They could \hr{soul-eating}{eat the souls} of other beings.
The energy drain of an \umbra was terribly dangerous, more so than any other known spell or attack. 
Even godlike creatures feared the touch of the \umbrae.

An \umbra drained energy from creatures, plants and even inanimate things.
It left death and ruin and decay and decrepitude in its wake.
Everything would become leprous, sickly and brittle and crumble to slime or dust.
All colours faded to a dead, gray emptiness without light or colour. 

Sometimes an \umbra might lash out with a pseudopod and strike some object.
The objects would not topple immediately from the force of the blow, for an \umbra's body was phantasmal and immaterial. 
But the objects struck would blacken and fade and become sickly and brittle and crumble under their own weight. 

Compare them to the titular monster in \cite{HPLovecraft:TheColourOutofSpace}. 

It was from the \umbrae{} that the \Merkyran{} rebels \hr{Rebels learn soul-eating from Umbrae}{reverse-engineered the technique} of eating souls. 









\subsection{Habitat}





\subsubsection{\Carcosa and \Oggra}
\Umbrae were often spotted in the \hr{Oggra}{Chasm of \Oggra} and near the towers of \hr{Carcosa}{\Carcosa}. 
Some \resphain believed that the \umbrae were spawned here.





\subsubsection{In the \Wylde}
\target{Wild Umbrae}
There live \umbrae{} in the \Wylde{} on \Miith. 
These are all beasts that were brought to \Miith{} by the \resphain{} or \banes{} but broke control and escaped. 
Or perhaps the descendants of such runaways, \hr{Umbra reproduction}{if that is possible}. 

In the \wylde in \Azmith one can occasionally see giant \umbrae soaring high above. 
Amorphous fearful shapes that cast a vast, ominous shadows.
Hence the name \quo{\umbra}. 









\subsection{History}





\subsubsection{Origin and relationship with \banes}
\target{Umbra origin}
The \umbrae{} originate from \Erebos{} where they prey on \banes. 
The \banes{} fight them, but they cannot destroy and exterminate them, for they races are tied to each other: 
The \umbrae{} are born from \banes{} that fail and mutate.

See, the \ps{\banes}{} \matrixx{} is tied to \FatherErebos{} and his power. 
The \umbrae, in turn, are an integral part of the \bane{} \matrixx. 
In a sense, the scourge of the \umbrae{} is a curse upon the \banes{} from \FatherErebos{} as a punishment for the \ps{\banes}{} betrayal of their homeworld. 
They are born enemies of the \banes{}, created by \quo{natural} processes to fight the \bane{} overpopulation (\hr{Umbra menace growing}{and later the \Merkyran{} overpopulation}). 





\subsubsection{Preying on \resphain}
Then, when \Nyx{} was created, the \umbrae{} came there. 
They soon began feeding on the \resphain, who had powerful souls and tasted like \banes. 
The \hr{Merkyrans fear Umbrae}{\Merkyrans{} feared them}. 









\subsection{Name}
Singular \emph{\umbra{}}, plural \emph{\umbrae{}}. 









\subsection{Physique and metaphysique}
\Umbrae are very much inspired by the picture \cite{Picture:GunnerRomantic:BermudaTaowls}. 

An \umbra was a titanic black horror, a hazy, unclear, shapeless cloud of thick smoky darkness. 
It was broad and flat, like a cloak of darkness, or perhaps like an flying manta ray or bat. 

\target{Umbra like bat}
They are capable of hanging completely still in the air.
They flap their wings very slowly and casually when at all. 
Their wide, flat body is thicker in the middle. 
They have a mouth on the front end with four jaws. 
These jaws can unfold to grasp and latch onto a victim, like the vampires in the movie \cite{Movie:BladeII}. 

\Umbrae can grow to humongous size.
Larger than \dragons (albeit less powerful). 
Total length and wingspan can be 50 metres or more. 
Only rare \umbrae grow to such huge size, though. 

An \umbra{} has a long, strong, whiplike tail. 
The manta-like taper to whip-thin appendages. 
The tail and whip-ends are razor-sharp and can be used as weapons in combat. 

They also have various other appendages on their back and belly. 
These are short and of obscure purpose. 

They have no separate \quo{back} and \quo{belly}. 
They can flip over at will. 

Like \banes, \umbrae{} have no eyes or other readily identifiable sensory organs. 

They have more than one vast mouth near the front. 
Sometimes when they open their mouths, pale gray ghostly light shines out.
Perhaps they have a breath weapon, like Godzilla's energy beam breath from the Millennium \emph{Godzilla} movies. 

\Umbrae{} fight by slashing with their razor-sharp wings, their mouths and their tails. 

\citeauthorbook[p.314]{RobertEHoward:TheValeofLostWomen}{Robert E. Howard}{%
  The Vale of Lost Women%
}{
  Now it hung directly over her, and her soul shrivelled and grew chill and small at the sight.
  Its wings were bat-like; but its body and the dim face that gazed down upon her were like nothing of sea or earth of air; she knew she looked upon ultimate horror, upon black cosmic foulness born in night-black gulfs beyond the reach of a madman's wildest dreams. 
}





\subsubsection{Dislike light}
\target{Umbrae dislike light}
\Umbrae disliked light. 
They would be less inclined to attack a brightly lit place. 
In Realms with day and night, \umbrae were sluggish in daylight and preferred to attack at night. 
An \umbra could still fight at full power in daylight if it had to, though. 





\subsubsection{Incarnations of death and horror}
\target{Incarnations of death and horror}
\target{incarnations of death and horror}
The \umbrae were incarnations of death and horror.





\subsubsection{Sounds}
\target{Umbra sounds}
\Umbrae{} emitted some very deep-pitched, mournful-sounding, droning howls. 
But loud. 
These sounds were felt (as vibrations) rather than heard. 









\subsection{Politics}





\subsubsection{\Banes}
See the section about \hr{Banes and Umbrae}{\banes and \umbrae}. 





\subsubsection{Cults}
A few \resphain and \Miithians worshipped the \umbrae (and \hr{Umbra gods}{their gods}) and derived monstrous power from them.





\subsubsection{\Moongods}
\target{Umbrae and Moongods}
There existed records (in \ophidian libraries) of an incident, thousands of years ago, when a titanic \umbra had attacked a \moongod. 
A tremendous battle ensued that devastated the countryside. 
Ultimately the \umbra was destroyed. 
The observers could not be certain whether or not the \moongod was hurt, but some claimed that the god did look smaller than before.





\subsubsection{Nether Ones}
\target{Umbrae and Nether Ones}
The \hs{Nether Ones} were spawn of the \umbrae. 
They, too, were \hs{incarnations of death and horror}. 





\subsubsection{\Umbra gods}
There existed some \hr{Umbra gods}{vast and terrible gods of the \umbrae}.









\subsection{Skills and powers}





\subsubsection{Cause agoraphobia}
An idea might be to have some character who develops agoraphobia (fear of open spaces) after being attacked by \hr{Umbra}{\umbrae} that swoop down from the sky. 
Possibly a \resphan{} in \Merkyrah. 

\lyricstitle{\emph{Call of Cthulhu RPG} p.50}{
  Agoraphobia: 
  \ta{%
    The sky is so wide, so heavy, so massive. It spreads into infinity with stars and clouds held up by who knows what. Monsters come from sky and space.}
}





\subsubsection{Darkness and light}
\Umbrae did not fear light, but they were most often found in darkness, for a number of reasons.

\begin{itemize}
  \item 
    Their native haunt was \Nyx, which was naturally dark.
  \item 
    In the Shrouded Realms it was very difficult to summon an \umbra in light but much easier in darkness, because the Shroud was weaker in darkness. 
  \item 
    An \umbra swallowed up light around it and spread an aura of unnatural darkness. 
    They left a trail of shadow, almost like the ink cloud of an octopus. 
\end{itemize}





\subsubsection{Fleeing}
\Umbrae very rarely fought to the death.
If an \umbra was overpowered in combat it would flee into the Beyond.





\subsubsection{Immunities}
An \umbra was extremely powerful, but it \emph{could} be harmed with normal weapons. 
The \umbra's own chief attack was an energy drain that it was almost impossible to defend against or be immune to.
It affected even inanimate objects. 





\subsubsection{Power and danger}
\target{Umbra power}
The \resphain{} tame the \umbrae{} and use them as beasts of war to great effect. 
\Umbrae{} are very powerful. 
Three \resphain{} on \umbrae{} are a serious threat to even a \dragon{} (\hr{Dragons vs Resphain in power}{normally it takes ten \resphain{} to bring down a \dragon}).

\Umbrae{} are alien monsters from \Erebos{} and must be controlled with occult, incomprehensible \bane{} magic. 
The \resphain{} utilize these spells, but they do not understand them. 
They only know that the spells work (most of the time), but not how or why. 
But this lack of understanding makes the control crude and brittle, liable to break. 
And \umbrae{} are vicious and aggressive. 
Often an \umbra{} has broken control and killed the \resphan{} who tried to ride it (\hr{Umbrae eat souls}{sometimes even permanently})\dash even a moment of broken control is enough for the deadly \umbra{} to kill its handler. 

For this reason, the \resphain{} fear them and do not use them in combat as much as they otherwise could. 
The \umbrae{} are a terrific weapon, but they are too dangerous. 

If an \umbra{} breaks control completely, it \emph{cannot} be brought back into the fold. 
It will escape \hr{Wild Umbrae}{into the \Wylde}. 





\subsubsection{Splitting apart}
Once in a while, when an \umbra was injured and maimed, a severed body part would take on life and fight on its own, as if it had become a new smaller \umbra. 
Apparently this limb was not quite a separate creature, though, for in a few instances the severed limb had been observed recombining with the main body to once more form a complete \umbra. 

If the \umbra fled from battle, all fragments would always flee together. 





\subsubsection{Zombies}
\target{Umbra zombies}
An \umbra would sometimes create zombies where it went. 
A mortal body that was drained by an \umbra but not quite drained enough to crumble to pieces might sometimes rise as a zombie, animated by a spark of \umbra power. 
This \umbra spark feeded on life force and could grow. 
The zombies defiled the land around them, sucking energy out of everything, like an \umbra.
When these zombies killed they infected their victims with the \umbra spark, and they in turn would rise as zombies.

Ultimately the \umbra would return to collect all its zombies. 
They contained refined \umbra energy on which the \umbra liked to feed.
It would drain all its zombies dry and in the process lay waste to the countryside, and maybe plant new zombies. 

Some believed that this formed a part of the \umbrae's reproductive cycle.

So people learned that when \umbra zombies appeared, they had to be destroyed quickly. 
If the zombies were killed the \umbra would not return. 
Otherwise the zombies would spread and ultimately attract an \umbra.

When a tower in \Nyx became infested with zombies, the \hr{Resphan Protector}{Protectors} would come in and destroy them.















\section{Weaver}
\target{Weaver}
\index{Weaver}
%An enormous monster, the shape of a dark cloud with a number of legs or tentacles. Looks vaguely like a giant spider. 

A Weaver is an alien monster from the \baneworld. They are enormous and extremely dangerous monsters, feared even by \dragons{}. 

%It looks like a great cloud of dark fog sprouting a number of legs and tentacles. Some find that a Weaver looks somewhat like a bloated giant spider. 

%\subsection{Name}
%\emph{Weaver} is English. There is no associated adjective. 
%As in English. 









\subsection{Physique}
A Weaver looks like a great cloud of dark fog sprouting a number of legs and tentacles. Some find that a Weaver looks somewhat like a bloated giant spider. 

They are huge in size, easily reaching 10 meters in diameter. Small Weavers down to 4 meters have been encountered. The average Weaver in Threll will be 7-8 meters in diameter, up to 10 at most. Larger one have been sighted in Nom. There are reliable reports of Weavers up to 30-40 meters in diameter, and a few explorers tell of behemoths growing as large as 100 meters. They are quite massive; a 10 meter Weaver weighs an estimated 20 tons, and the huge ones may weigh hundreds of tons. 

Weavers cannot fly or jump; they can only crawl, but they do this rather fast (slightly faster than a running \human{}). They are roughly symmetrical in all directions (having no front and back) and can move in any direction. 

A Weaver's central body is not visible because the creature excretes a cloud of opaque gray gas. This gas is somewhat lighter than air and constantly rises up from the creature like smoke; presumably exhaled as the Weaver breathes. The gas is poisonous if inhaled. 

A Weaver gives off a horrible stench, the reek of an unnatural and loathsome abomination. It can be smelled over a hundred meters away (depending on its size). The monster constantly emits hissing and groaning noises (believed to be incidental). 

Only the legs and tentacles are visible beyond the cloud of gas. The legs are segmented and somewhat spider- or insect-like, rhe tentacles soft and flexible and up to 20 meters long. Leg length up to is about $\fracs{1}{3}$ of the diameter, tentacle length is up to about $\fracs{3}{2}$ times the diameter. A Weaver has about 10-15 legs and 20-30 tentacles. The tentacles frequently sprout smaller sub-tentacles along the way. Some of the tentacles end in a sharp sword-like claw (these can be long, up to $\fracs{1}{10}$ diameter). The creature is covered in very tough, leathery skin with small spikes (centimeters in length) irregularly jutting out. The legs are more heavily \armoured than the tentacles. 

A strong wind (natural or artificial) may blow away the dark fog to reveal the body inside. It will become apparent that there is no central hub, only a mass of tentacles splitting off from and rejoining each other. The tentacles are much thicker near the centre of the creature (up to $\fracs{1}{5}$ of diameter). 

Scattered about on the creature's body are dozens of mouth-like orifices. The size of these holes varies from few centimeters to over a meter near the centre of the monster. The dark gas rises out of these holes, and it will shove captured victims into them. Once inside the Weaver, a victim will be slowly digested. This is a slow process, however; a swallowed victim is more likely to die from breathing the poisonous gas, or from suffocation. A swallowed victim can be rescued if (a part of) the monster is sliced open. 

In combat, a Weaver's primary weapon is to lash out with its tentacles. It will attempt to grab victims, pull them closer to its body, grab them with more tentacles, then pull them apart while slashing them with their claws. If the victims (or the parts of dismembered victims) are small enough, it will attempt to swallow them. The Weaver can also use its legs as bludgeoning weapons to kick or stomp. 

If the Weaver cannot easily grab its opponents with its tentacles, it will shoot its \quo{web} at them. This \quo{web}, from which the Weavers derive their name, is a mass of long strands of tough, sticky, pale white material. Each strand is 1-3 cm thick, and the Weaver can fire a score of them at a time towards the same target. The strands are fired from small holes in the tentacles (different from the mouth openings described above), and they have been known to fire them at ranges of 50 meters or more. The web is nearly impossible to tear, but it can be cut or burnt, and great cold causes it to freeze and shatter. A Weaver has a large supply of the web, and will generally never run out of it during a single battle. 

Weavers have some resistance to magic. Spells cast directly upon the creature tend to be absorbed and dispelled. Spells cast indirectly upon the monster (like a fireball) work normally. 

Slaying a Weaver is a monumental task. Once slain, the creature's body will thrash and convulse for a minute or so, then lay still. It will also exhaling the dark gas, but it still smells. After some hours, its flesh will dry out and crumble away. If the creature is cut up, a lot of web can be gathered (up to $\frac{1}{200}$ of the creature's weight). 









\subsection{Biology}
Almost nothing is known of the Weavers' biology. Presumably they eat the flesh of creatures they swallow. It is unknown how (or whether) they reproduce. It is assumed that the larger Weavers are older, but this is conjecture. 









\subsection{Psychology}
It is unknown how intelligent Weavers are. No one has ever successfully communicated with one, and they rarely show signs of intelligent behaviour, but some believe that they are extremely intelligent. Attempts to communicate with them using \hs{telepathy} usually result in serious damage to the telepath's mental health. 

Weavers are always aggressive when encountered and will attack pretty much every creature they detect. If its prey flees, the Weaver will pursue for a short time, but it will not move far from its original position. The Weaver itself will never flee, however; if pressed, it will fight to the death. 

Weavers are solitary; no one has encountered more than one at a time. It is unknown how and why they avoid each other and what would happen if two were to meet, but it is believed that each Weaver maintains a territory (this also explains why they rarely move far from the place they are encountered). 

\Banes{} greatly fear Weavers and will flee before them. 

Rarely, \banes{} may be encountered together with a Weaver, sometimes even hunting together. Such \banes{} can often be seen casting spells and performing strange rituals involving the Weaver. It is believed that these are religious rituals and that the renegade \banes{} serve and worship the Weaver. %These may be compulsion spells to keep the Weaver under their control, but it is also possible that the rites are of a religious nature and that the renegade \banes{} serve and worship the Weaver. 









\subsection{Habitat}
Not native to \Miith{}, the Weavers originate from \Erebos. Most likely, the Weavers on \Miith{} were summoned during the \banewar. On \Miith{}, they are known to exist only in Nom and Threll. All weavers encountered in Threll have been fairly small, no larger than 10 meters. In Nom, enormous specimens have been encountered. 

Where do they dwell? In caves or out in the open? In valleys or in mountains? 

The total number of Weavers is unknown, but Threll is estimated to home between 100 and 1000 of them. Nom is huge and unmapped, so no one can reliably estimate the number of Weavers there. 







\subsection{Weaver web}
\target{Weaver web}
\index{Weaver!Weaver web}
Weaver web may be gathered, during a fight or when the Weaver is dead. It dries in a couple of minutes, after which it is no longer (very) sticky. 

The chief use of Weaver web is that \banes{} fear it. \Banes{} seem to have a superstitious fear of Weavers and will flee from anything that smells like a Weaver, including web. One kg of web can be smelled by the \banes{} up to 30 meters away, and they will be hesitant to approach any closer than that. (Only \banespawn{} and \lesserbanes{} fall for this. \Banelords{} are not fooled. If led by a \banelord{}, \lesserbanes{} will attack, but will be at a disadvantage because they still fear the web.) 

The web dries up completely in a matter of days if left unprotected. Wrapped up and sealed, it can last up to a month, and preservation spells are known that will make it last several months. The older it is, the less it smells, and the less effective it is. When it dries up, it stops smelling and crumbles. 

The chief disadvantage of the Weaver web is that while it repels \banes{}, it also repels everyone else. The stench of it is loathsome and sickening, and animals and humanoids instinctively hate it. The easiest way to destroy the web is to burn it (it will produce some awful brown smoke, but then it will be gone).









\subsection{Weaver maggots}
\target{Weaver maggots}
\index{Weaver!Weaver maggots}
Weaver maggots are small creatures that crawl around on a Weaver and inside its mouths. They are gray, wormlike, 20-50 cm long (with thickness at the middle equal to about $\fracs{1}{3}$ of the length) and weigh 1-4 kg. They have dozens of small lumps scattered over their bodies which they use for slowly crawling about. A maggot has no top and bottom, it can crawl on any side. It has a front end with several (3-7) toothless mouths. It has tough, leathery skin covered by a thin layer of sticky slime, like a slug. They also exhale the dark gas, but in much smaller amounts, too little to cause more than coughing and irritation. 

The maggots are harmless, although they will try to eat organic matter that doesn't move. They look and smell revolting, however, so many people will feel a strong dislike of them and an urge to kill them. Killing them is easy; they can be crushed or cut in half. They can also be easily captured. There is no known use for them, and they rarely live more than a week in captivity, but nevertheless, researchers are sometimes willing to buy them to study. 

\index{cannibalism!Weaver maggots}
Every Weaver has plenty of maggots crawling about on it, 6-10 of them per ton of weight. When the Weaver is killed, the maggots will immediately begin to eat the flesh of their host. They are cannibalistic and will eat other dead maggots. Presumably, the maggots are a symbiotic or parasitic species, or possibly immature Weavers. 























