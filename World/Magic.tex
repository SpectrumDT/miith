\chapter{Magic}















\section{General}
\target{magic}
Magic exists in the \Miith{} universe and is an important part of it. 

\citeauthorbook{LinCarter:TheNecronomiconTheDeeTranslation}{Lin Carter}{
  The Necronomicon: The Dee Translation (part II.I)
}{
  Knowest thou this, that of all the arts and crafts and sciences whereunto may mortal men aspire, supremest and most potent of them all be the practice of Magic.
  Yet indeed, as \emph{Ibn Shoddathua} sayeth in his commentaries upon the Papyruses of Mum-Nath:
  Many are they who lust for the Mastery thereof, but few indeed are they who succeed therein.
  For the wise magician is the Master of Nature and the archpriest of all her Mysteries; at his command there openeth forth the Grave of Sod or the shutten Sepulchre of Stone, to admit forth they who slumber therein; before the utterance of his will shall storms becalm themselves, and floods retreat back into the secret fountains of the Deep, and conflagrations extinguish their fiery flames.
  
  Aye, and verily can he call down from beyond the stars That which abideth in the dark and freezing spaces of the Void, or forth from the Pit may he summon That which resideth in the black and frightgul abyss; spells and enchantments may be cast upon even the holiest of men or they that be purest of heart. 
  I say unto thee that such power may the accomplished Initiate command that nations shall grovel before his awesome might, and that the very Kings and Princes of the Earth shall flock to do him homande and obeisance.
  Even the very life of the Sorerer may be by his Art extended far beyond the ordinary limitations set upon mortal men, aye, and verily, for untold centuries mat he thrive, untouched by Time.
  For, behold! doth he not wield the keys of Life and Death? wherefore shall all mere mortal men exalt the Master thereof, and grovel at his feet, \emph{i\"a Nyarlathotep}! 
  The Wise Magician is a very mighty god.
}









\subsection{Cost}
\target{The cost of magic}
Magic always comes at a price\ldots{} but the currency can be stolen\ldots{}

\emph{All} magic, or at least all powerful magic, is powered by blood and life force. 

Different schools of magic carry different prices.

All these effects fall most strongly upon those mages of weaker mind. The more strong-willed mages are able to resist it better and tend to keep their personality more intact in the long run. 

If you practice two or more different kinds of magic, each with different side effects, these side effects may pull in different directions. This is the case with a Vaimon who uses both Iquin and Nieur: Iquin works to make him principled and virtuous, whereas Nieur works to make him savage and uninhibited. In such cases, these opposing forces may sometimes balance each other out, so that the mage can remain stable. On the other hand, they may also cause the mage to go schizophrenic or otherwise mad (perhaps even developing a genuine split personality) from the strain of having his mind pulled in two opposite directions. 






\subsubsection{Addictiveness}
One problem with magic is that it is addictive. Once you begin casting magic, you will have an urge to keep doing it, casting more spells and drawing more magical power. This addiction in itself is perhaps not so serious, but magic has other effects which are amplified by this compulsion to keep using it. 





\subsubsection{Overdose}
\target{Magic overdose}
When people use an overdose of magical power, their bodies suffer for it.
They get pain, bleedings\dash{}inner or outer\dash{}and scars; signs of disease and decrepitude.
They go around in pain and cough up blood.
This also happens to immortals (although they can heal more wounds than mortals can).

If you draw too much magical power, more than your skill and strength can handle, then you burn yourself up from the inside. 
Your brain, heart and blood burns to ash.

Compare to Beak in \cite{StevenErikson:ReapersGale} p. 757. 





\subsubsection{Humanoid sacrifice}
The sacrifice of living beings is an effective means of gathering magical energy that can be used in spells. 

In many cases, this is more effective if the victim in question is \quo{innocent}, ie., naive and idealistic, deeply entangled in the Shroud\dash preferably sexual virgins as well, since sexuality can be a road out of the Shroud. 
This way, you can shock the shit out of them in their very last moments, just before you kill them. When finally confronted with all the horror of the true universe, their shock is much more profound. 
This makes for a stronger \quo{jolt} in the Shroud (remember, the Shroud is maintained by people's beliefs), which releases more energy. 
This is especially useful in situations where the spell in question needs to re-weave the Shroud (as is the case with the summoning of \hr{Nith'dornazsh}{\Nithdornazsh} in \TwilightAngelRememberEmph).





\subsubsection{Wounds in the world}
Teleportation, portals, Realm travel and Shroud-weaving magic tear holes in the Shroud and leave bleeding wounds in the universe. 

Compare to the bleeding wounds in Kurald Emurlahn in the prologue to \MalazanReapersGale.










\subsection{Energy}
All magic works by reaching out with the mind and touching the world in ways not otherwise possible. 

In most cases, the mage does not perform all the effects of magic himself. 
Rather, with his spell he calls upon some external force or creature, binds it to his will and uses its power to affect the world. 
The Vaimons use the \Sephiroth{} and \Kliffoth{} for this purpose while Chaos mages call upon various \hr{Daemon}{\daemons}. 
Some of these forces are mindless or mostly mindless, while others are living creatures with more or less inteligence.

These external forces are normally unlimited (although they might be destroyed). 
But when casting magic, the mage must use some energy of his own to control the forces of magic, and a mortal mage's supply of energy is finite. 
As the mage expends energy, he will grow tired and exhausted and need to rest. 

Furthermore, in order to call upon the forces of the supernatural, the mage must spend some time attuning himself to these forces. 
This takes the form of meditation or prayer, where the mage contacts the occult forces and builds a connection to them. 
As he casts magic, his connection to the forces will become worn and decay, so every mage must periodically meditate to re-attune himself to the sources of his magic. 









\subsection{How to cast it}





\subsubsection{Circles of mages}
\target{circle of mages}
An \hr{Ishrah}{\ishrah}\dash a group of mages\dash can band together in a circle to perform spells together. In this way they can do more as a whole than as a group of individuals. A whole that is greater than the sum of the parts. 

Such circles need a name. How about \quo{concert}?

A circle of mages \hr{Circles are Matrices}{is a small, temporary \matrix} and obeys (or should obey) the rules for \matrices. 





\subsubsection{Laboratory magic}
Some magic is slow, taking hours or days, and may even require all sorts of material components. Such magic is best performed in a laboratory of some sort, and will here be called \quo{laboratory magic}. Laboratory magic typically includes all spells that are to have a permanent effect, including the creation of enchanted items. Major summonings are also often laboratory spells, as are major shape-changing spells (from \dragon{} to humanoid form). Major healings (for deep injuries or serious diseases) are also laboratory work. 

Laboratory work sometimes requires ingredients: Exotic minerals, plants or parts of creatures, or even living creatures. 





\subsubsection{Mortal vs. immortal magic}
It is difficult for Shrouded mortals to cast spells because the Shroud suppresses magic and the energies and natural forces on which it depends. 
Therefore, Shrouded mages must perform rituals and stuff in order to reach out to and touch those cosmic forces. 

People (mostly) free of the Shroud, such as the master races, can cast magic far easier, since they can readily see and perceive the energies they all around them. 





\subsubsection{Spellcasting aids}
Most magic, except laboratory magic, requires no items to be cast. But you can have items that help you cast spells. Typical spellcasting aids are staves, wands, rings and amulets. Such items should be made from special materials (exotic sorts of wood, occult metals or body parts of creatures) and enchanted with arcane sigils. Then they will help the mage channel his magical energy, so that he can cast stronger spells and expend less energy. It might also make it easier to contact the occult forces, so spells can be cast faster and with less risk of failure. 

There are also casting aids that are consumed and destroyed when the spell is cast. The most common example is the various herbs used in healing and in minor blessings and curses. (It should be noted, however, that sometimes that which is called \quo{herbal magic} is not magic at all, but simply natural medicine.) In most cases, these components are not strictly needed but serve to boost the spell's effect\dash{}or, in some cases, the spell boosts the herbs' natural effect. 









\subsection{Life force drain}
\target{Life drain}
It is possible to drain life force from others and use it to sustain yourself. 
This is actually what all animals do when they eat other animals or plants, but mages have more effective and nasty ways to do it. 

You can drain just a little force and leave the victim alive. 
Or you can drain the victim to death but let the soul escape. 
Most wicked of all, you can \hr{soul-eating}{\emph{consume} the soul itself}, permanently destroying the victim. 
This contributes to the \hr{Heart weakened}{weakening of the Heart of \Miith}. 
Some creatures (such as the \hr{Resphan}{\resphain}) \emph{live} by draining souls. 





\subsubsection{Voluntary sacrifice}
\target{Voluntary drain}
Life drain is more effective if the victim willingly lets herself get eaten. 
The \resphain{} employ this in their \hs{Communion}. 









\subsection{Mages}









\subsubsection{Learning magic}
It is sometimes said that to become a mage one must possess some unique, inborn talent or gift, but this is actually not true. The only character traits needed to learn magic are the intelligence to understand the theory of magic and the strength of will to bind the forces of the occult to your bidding. 

In addition, one must have access to teaching materials and a teacher. It is possible to learn magic by self-study from books and scrolls alone, but this is a slow, difficult and dangerous process\dash{}magic is dangerous work, and it is easy to hurt yourself if you don't know what you're doing. Similarly, it is possible to learn magic directly from a teacher with no written materials available\dash{}there are mages among pre-literate cultures, and their magic can be potent, if primitive. But learning from a teacher with books and scrolls available is the best and most common method. (Note that in order to use books one must of course be able to read, and very few people on \Miith{} are literate.) 

Learning magic is a long and difficult process. It usually takes years to learn to cast but the simplest spells and a decade or more to become a competent mage. Like all skills, magic is most easily learned at an early age. Many mages\dash{}and most of the skilled ones\dash{}were apprenticed as children, before the age of ten, and studied the art for at least fifteen years. 





\subsubsection{Losing one's magical ability}
There are only a few things that can permanently weaken or destroy a mage's ability to cast magic. These include brain damage, amnesia, insanity/mental illness and extreme shock/mental trauma. 

Most mages also use physical gestures to cast magic, so a mage who is suddenly maimed will find it harder to cast his spells. This is not an absolute hindrance, however, and in most cases you can learn to cast your spells anyway without relying on physical gestures. (There are exceptions, though. Some spells require a physical ritual, such as the drawing of occult symbols, and such spells cannot be cast if the ritual cannot be completed. Still, sometimes it is possible to have an assistant perform these parts of the ritual instead.)





\subsubsection{Mages vs. \vertices}
See the section on \hr{Vertices vs. mages}{\vertices vs mages}. 





\subsubsection{Magical strength}
Some mages are more powerful than others. There are a number of factors determining the magnitude of effects that a mage can accomplish with magic. These include: 

\begin{enumerate}
  \item Knowledge of spells. 
    Working magic is complex work, and knowing a clever spell will let you accomplish your work much faster, more effectively and with less expenditure of energy than a simple, naive spell. 
  \item Willpower. 
    Strength of will alone is important, as it lets you bind more power to your will. This one is open-ended and constrained solely by the will of the mage, but even so, even with infinite willpower there are limits to what you can do.
  \item Innate power of the soul. 
    This is a fuzzy concept as of yet, but the idea is that some creatures, even if their willpower is the same, have more mental \quo{muscle} and can handle more magic power than others. 
  \item   
    For a Chaos mage, self-\hs{Gnosis} is an important source of strength. 
\end{enumerate}

Every person has a certain magical \quo{power} or \quo{strength}. 
This works mostly like muscle strength: 
A person is born with a certain natural amount of power. 
This can be increased by training and experience. 

But power can also be stolen from others. 

And power can be gained by insight (or \hr{Madness}{madness}): 
Enlightenment, epiphany, revelation. 
This is kind of like lifting yet another layer of \hs{Shroud}: 
It reveals a deeper, truer world. 
Once you realize that this world exists, you can begin to interact with it and manipulate it. 

As a mage casts spells, he will expend energy and become tired. This is similar to physical fatigue, but not quite the same. 

Perhaps to regain your magical power you need to meditate and attune yourself to the mystic forces that you channel. In a religiously oriented magic theory, this meditation will take the form of prayer\dash{}to the \Sephiroth{} or to \hr{Rissit}{\RissitNechsain} or whoever. 









\subsection{Mages and society}





\subsubsection{Mage clans and rogue mages}
\target{Master races seek to control magic}
Many mages are organized in \ishroth, mage clans and other organizations. This serves the master races' purpose. Magic is necessary in the world, but dangerous, and the master races want to control it. 

\Nieur{} and Chaos mages are frowned upon (and sometimes actively hunted down) by the Cabal and Sentinels, unless they stay within established and controlled mage clans. The mage clans themselves hunt down defectors and runaways. \hr{Takestsha}{\Takestsha} employs this in \hr{Takestsha on the run}{her fake background story}.

\hr{Geica is embattled}{Geica is embattled by the master races} for this reason. 




\subsubsection{Mages are often nobles}
\target{Most mages are nobles}
Most mages are nobles. 
Commoners rarely get the chance to be educated and study the sciences, least of all magic. 





\subsubsection{Old mages sequester themselves}
Older mages have a tendency to lock themselves up in their towers.
There they continue their occult studies, hidden away from the world. 
It is a kind of tradition. 
This works well, because the world would rather see their heel than toe. 
People are happy to see them go, because old mages tend to be rather mad. 
And they can get pretty powerful, too. 





\subsubsection{Sorcerer-kings are uncommon}
\target{Sorcerer-kings}
\index{sorcerer-kings}
In the \hr{Scatha Age}{\Scatha Age}, in \Azmith, sorcerer-king were less common than one might think. 
In some \scathaese cultures, where the \hr{Ortaican religion}{\Ortaican religion} still dominated, \rethyax-dominated \hr{Baccon}{\baccons} still held power. 
But in Iquinian countries, sorcerer-kings were rare. 

\begin{prose}
  \ta{%
    Have you ever wondered why sorcerer-kings are so rare? 
    Mages are clearly superior to the common folk, in power, intellect and wisdom alike. 
    So why do they not rule \Miith{}? 
    Why do they, in most countries, bend the knee to mundane kings?
    
    The answer is: 
    Because we do not allow it. 
    We of the Cabal, and the Sentinels as well, fear that a sorcerer-king might become to powerful. 
    So we pull strings and see that they fail before they gain too much power, or pull them down if they do.}
\end{prose}

The organizations have also founded mage orders, with which they hope to keep future generations of mages under their control, or at least surveillance. 

Also, people from royal families and powerful nobles are rarely allowed to study magic, even if talented. 
The master races do not want them getting that kind of power. 





\subsubsection{Useful purpose}
Mages would be potentially useful for all sorts of menial labour: 

\lyricslimyaael{377933}{
  Wind and water mages would be in demand not only on ships, where some authors do put them, but to give crops good weather, provide pleasant days for large festivals, turn aside or dissipate storms, purify drinking water, move streams around or dam them, clear away this blasted fog, put this blasted fog in place so that it can blind the enemy or keep rival ships from sailing, calm this stream from flooding, coax that stream into flooding so that it can provide rich soil, make this hydraulic system work\ldots{} 
  There's lots and lots of uses. 
  Earth mages could restore soil, make it richer, help crops grow faster, call animals in for slaughter, prevent earthquakes and mudslides or clean up after them, insure that transported plants survive and acclimate to new soil, and do other things depending on what power you've given them.
}

They sometimes do. 
Especially the \hs{Rissitic} mages are well-integrated in society, and to a lesser degree the \hs{Imetric} ones. 
But they don't do as much honest work as they could, mostly because mages are snobs. 
They know they are highly valued, so they assume high places in society and shy away from menial labour. 

They do work on ships, though. 
\hr{Sea}{Travel by sea is fucking dangerous}. 









\subsection{Occultism/mysticism}
I should have a lot of magic that looks \quo{occult}, \quo{mystical}. 





\subsubsection{Astrology}
See the section on \hs{astrology}.





\subsubsection{Geometry}
\target{Occult geometry}
\target{occult geometry}
Have some occult geometry. 
\hr{Malcur}{\Malcur} is built using a lot of this.

Have some \emph{Feng-Shui}-like principles of building construction. 

Especially the Rissitics \hr{Rissitic architecture}{employ it}. 
The \Ortaicans{} also \hr{Ortaican architecture}{used it}. 
And \ps{\Ishnaruchaefir} \hs{glaive} is built with it. 





\subsubsection{Numerology}
Have some occult numerology. This could be connected to geometry, forming a unified whole of occult mathematics.





\subsubsection{Sexual mysticism}
\target{sexual mysticism}
And I should have stuff about sex and magic. 







\subsection{Suppressing magic}
It is possible to suppress magic. Suppression works on a person or an area and prevents the person, or everyone in the area, from casting magic of one or more types. 





\subsubsection{Suppression spells}
One way is by casting a suppression spell on one or more persons, thus suppressing any spellcasting on their part. There are two problem with such spells. One problem with such spells is that they must be continuously maintained by one or more mages. Another problem is that the suppression can be broken if the subject is a stronger or more crafty mage than the one holding the spell. 





\subsubsection{Suppression drugs}
\target{Witchbane}
Another way is using drugs. The most well-known of such are the toadstools called \quo{witchbane}. When eaten, drunk or injected into the veins, after being properly prepared and distilled (into a soup-like fluid), witchbane affects the victim's mind, making it difficult to think clearly and making spellcasting all but impossible. (Even if the afflicted victim manages to access his magical power, he is unlikely to be able to cast a coherent spell, but may be able to unleash some random magical mayhem.) 

The advantages of witchbane is that the toadstools are rather widespread and the concoction is easy to make. The drawback (if the victim must be kept alive and mostly unharmed) is that ingesting it in large quantities will cause permanent brain damage, potentially resulting in insanity, loss of intelligence or loss of motor skills (stuttering, uncontrollable shaking or paralysis). (At this time, no magical or mundane cures for any of these effects are known.) A victim can be pacified for at most 4-6 hours without risking serious permanent damage. (Of course, all these effects will affect anyone, whether mage or not.)

There are several species of witchbane toadstools, which can be more or less potent and more or less dangerous. 





\subsubsection{Inhibitors}
%A more potent method is using special items, called inhibitors. 
An inhibitor is a specially crafted and enchanted item made from exotic crystals and precious metals that suppresses spellcasting in an area. Some inhibitors cover only a small area, such as one person. These may take the form of collars or shackles. Others are strong enough to cover an entire room. These may have any form. 

The main advantage of inhibitors is that they are stable. An inhibitor functions constantly, with no operator necessary, until it is worn out. Typically they last many years. 

The disadvantage of inhibitors is that they are prohibitively expensive. Creating them is very difficult and time-consuming laboratory work, and the ingredients are rare and exotic. Another problem is that the inhibitor can never be turned off, which is inconvenient to the user. 

Inhibitors can be destroyed with physical force, but they are always enchanted to make them extremely durable, to prevent a prisoner from simply smashing them.










\subsection{Uses of magic}
Magic has many uses. 









\subsubsection{Magical healing}
\paragraph{Naturalist healing}
Magical healing might work by touching the recipient's \hs{Chaos} body, providing it with some energy and helping it to heal itself. In such cases, a portion of the energy comes from the healer and a portion comes from the recipient's own body. Such healing is rather easy, but crude, and cannot heal difficult injuries or badass diseases\dash things the body couldn't heal on its own. The healing might be assisted by herbs.

The Vaimons and most primitive cultures use naturalist healing. Most Chaos mages also use naturalist healing. This is because the theory of Chaos magic was developed by \dragons{}, and \dragons{} have formidable natural healing capabilities, so this kind of healing is very effective on them. For the \draconic{} Chaos mages, healing their servitors (\scathae{} or others) was never a high priority\ldots{}





\paragraph{Surgical healing}
Another approach is to use magic like surgery. Instead of relying on the body's natural ability to heal itself, the mage might rely on his own skill and knowledge to directly control the healing process. This requires the mage to know a lot about anatomy and surgery, so it is difficult to learn. The process itself is also difficult and consumes time and energy from the caster. The upside is that this healing is more sophisticated and can heal more grievous wounds and diseases. 

This kind of healing might also involve herbs or drugs, and the mage/doctor might use mundane surgical tools in conjunction with his spells. 

The Imetrians and Rissitics use surgical healing. Recently, some Vaimons in Geica have begun experimenting with it. 





\subsubsection{Magic in war}
Magic has a number of uses on the battlefield. 

A Vaimon can call down the elements to kill his opponents. 
He can also fly over walls or use Earth magic to break the walls. 

A Rissitic or Chaos mage can summon \hr{Daemon}{\daemons} or other creatures to his aid. 

In war, (mortal) mages would bombard the enemy army with lightning and rain of fire, all the while trying to dispel and negate the enemy sorcerers' attempts to do the same. 

Some mages could summon disease, drought and other disasters.

\citebandsong{KarlSanders:SaurianExorcisms}{Karl Sanders}{%
  Shira Gula Pazu%
}{%
  Hot winds burn me and smite the land of my ancestors.
  An evil south wind delivers my people to their doom.
  Our cities lie abandoned, our streets emptied, our brothers and sisters piled up as dead corpses.
  An evil wind smites out fields; our crops and vegetation are withered, diseased by the very air itself.
  An affliction is cast on our rivers; nothing swims anymore, our life-giving pure waters turned into poison.
  What have we done to bring this evil upon ourselves?
  And what shall we do when even the mighty Pazuzu has forsaken us?
}





\subsubsection{Magical transportation}
You can use magic for \travelling purposes in a number of ways. 



\paragraph{Super-speed}
  There are spells that can be cast on a creature to temporarily (or even permanently) increase its speed or endurance. 
  Such spells may come with a side effect: 
  After the effect wears off, the subject might experience extreme fatigue and/or weakness or even permanent damage, aging or death. 



\paragraph{Flight}
  \target{Flying magic}
  \target{Vaimon flight}
  A Vaimon cannot fly. 
  But with the help of \Atzirah a Vaimon can perform great powered jumps and walking-on-walls and stuff. 
  This is not easy, though. 
  
  Compare to the crazy acrobatics seen in Asian martial arts movies such as \cite{Movie:CrouchingTigerHiddenDragon}.
  
  The Imetrians have a similar spell. 
  It gives better control and \manoeuvrability but less speed, and it is more difficult to learn. 
  
  Rissitic and Chaos magic have no flying spells. 
  
  For immortals (such as \quiljaaran and \bezed \resphain), it was possible to fly with magic.
  But it was difficult to learn and taxing to do. 
  They only used it when they had to. 
  



\paragraph{Summoning}
  A Rissitic mage or Chaos mage might be able to conjure a creature and coerce it to carry him as a rider. 
  Such a creature might be able to run, swim, fly or perhaps even teleport. 



\paragraph{Teleportation}
  Chaos magic spells exist that teleport the mage from one teleporter to another. 
  There are several of these teleporters scattered across the world, but the secrets of their making have been lost. 





\subsubsection{Telepathy and empathy}
See the section \ref{Telepathy}: \quo{\hs{Telepathy}}





\subsubsection{Magical items}
\target{Magical items}
Items can be enchanted in many ways. These include:

\begin{enumerate}
  \item 
    Any kind of item can be made stronger, more durable. (This is quite simple.)
  \item 
    Weapons can be made to hit harder and more accurately. (This is quite complicated. I don't know how it works yet.)
  \item 
    Items can be enchanted to act as a focus for a spell, mystically as well as physically. An example might be a cannon made to fire fireballs or lightning bolts. Creating a cannon that can cast these spells on its own is extremely difficult, but you can make a cannon that can be operated by a mage, allowing him to cast spells through the cannon more powerful than he could otherwise cast. The physical shape of the cannot could also aid in shaping the spell, making it easier and safer for the mage. 
  \item 
    It is \emph{not} possible to make enchanted items with arbitrary superpowers. For instance, no \quo{Frying Pan of Ultimate Gourmet Cooking} or \quo{Comb of No Bad Hair Days Ever}.
  \item 
    You can enchant an item to carry a limited (small) number of \quo{charges} of a certain spell (or more than one). Examples are potions that are ingested to activate the spell. These may utilize the ordinary chemical properties of the ingredients in addiction to or in conjunction with the magic, to create more powerful effects (healing or whatever) than otherwise possible. 
\end{enumerate}

\paragraph{Making magical items}
Making a magical item is generally difficult and time-consuming laboratory work and often requires exotic ingredients. If a magical items is to be really effective, you cannot just enchant an existing item. Rather, it must be created from scratch, with the magic being addded from the beginning, weaving the magic in at every stage while it's being forged or assembled. 

Even creating a trivial magical item is tremendous work, so mages tend to make a few powerful ones rather than many little ones. The exception is one-off items like potions, which are relatively easy to make. 









\subsection{Visuals}
\target{Magic visuals}
This section gives ideas for how to describe the effects of magic. 

Immortal mages can unleash awesome powers. 

\lyricsbalsagoth{The Obsidian Crown Unbound}{
  And even as this transpired, the Emperor's Prime Sorcerer, emissary of the Imperial Court and master of those arts which speak to man in narcotic dreams from the darkest and most silent places, summoned forth that black potency which lay entwined in stygian tendrils within his mind\ldots{} an ireful power born of they who writhed upon the shores of Pangaea before man's progenitors ever erected their lofty spires to the restless skies.\\
  And yet Vyrgothia's Master Wizard, unrivalled Arch-Mage and adept of that lost Eastern order who journey beyond the boundaries of time and space upon those nebulous wings born of the sacred Azure Lotus, rose to meet this power which lapped at the periphery of his mind like a midnight tide, and stood firm against its insistent siren call.\\
  And upon that arid field of war, the sentinels of light and shadow spoke to each other in tongues dormant since the Third Moon fell burning from the heavens, and not sweet were the words they uttered.
}

There is dark art, despair and beauty. 

\lyricslimbonicart{Beyond the Candles Burning}{
  I am a dark star rising on the raveous bleaky sky,\\
  a black diamond slunning so deep within the night.\\
  Maliciously I dwell in a bluish shaded beam\\
  with a stonecold heart into the core of my being.
  
  Beyond the candles burning, beyond all minds eye.\\
  A vast emperic enigma awaits me as I die.\\
  In a graceful dance obscene, in a ring of fire,\\
  I obtain my majesty as flames caressing higher.
  
  Release my spirit, unleash my soul.\\
  From the darkest dungeon, oblivion calls.\\
  In the phallic halls of ancient forlorn\\
  a cold sanctuary in doom is born.
}

\lyricslimbonicart{Solace of the Shadows}{
  I require the solace of the shadows,\\
  so the night can be redeemed.\\
  As the winds of darkness whispers my name,\\
  a kiss of death I receive.\\
  Nocturnal enchanter, to thine art I yield.\\
  Within the candlelight a rapture is now revealed.
}





\subsubsection{Curses of destruction}
\target{Curses of destruction visuals}
\target{Draconic curses of destruction}

Remember that \hr{Ruin Satha fire magic}{\draconic/\rethyactic fire magic} is \hr{Ruin Satha and fire}{associated with the \xs \RuinSatha} and produces many-\coloured spiritual flames.

When a \dragon (such as \Ishnaruchaefir or \Nzessuacrith) kills, he does it by invoking \xss and channelling their destructive power.

\citebandsong{Nile:InTheirDarkenesShrines}{Nile}{
  Destruction of the Temple of the Enemies of Ra
}{
  The Fire of the Eye of Horus is Upon You\\
  Searching You Consuming You\\
  Setting you on Fire Burning you To Ashes

  Unemi, The Devouring Flame, Consumes You\\
  Sekhmet, The Blasting Immolation of the Desert, \\
  Maketh an End of You\\
  Xul Ur adjugeth you to Destruction\\
  Flame Fire Conflagration Pulverize You

  Your Souls Shades Bodies and Lives Shall Never Rise Up Again\\
  Your Heads Shall Never Rejoin your Bodies\\
  Even The Words of Power of The God Thoth\\
  The Lord of Spells\\
  Shall Never Enable you to Rise Again
}

Curses as a \dragon destroys his enemies in the names of the \xss:

\citebandsong{Nile:AnnihilationoftheWicked}{Nile}{
  Lashed to the Slave Stick
}{
  Ra Pronounceth the Formulae Against Thee.\\
  The Eye of Horus is Prepared to Attack Thee.\\
  Sekhmet Uttereth Words of Flame Against Thee \\
  and Pierceth Thy Breast.

  Abui, The Gods Who Burns the Dead.\\
  Shall Leave You Smoldering in Exile from the Netherworld.\\
  Abati, the Gorer, causes You to Howl like a Jackal in Anguish.
}

\citebandsong{Nile:Ithyphallic}{Nile}{
  Ithyphallic
}{
  Let the shades of my fathers curse their faces\\
  Let the eye of Sekhmet\\
  Send the violence of the sun down upon their heads\\
  Let searing torrents of fire descend upon their brow\\
  Let flames immolate their places of sleeping

  Let the eye of Sekhmet\\
  cause their hearts to burst into flames

  Let my curses be heard\\
  Let my will be as Menthu the bull\\
  Potent to create\\
  And savage to slay those whom I hath cursed\\
  Let my wrath be terrible\\
  And my vengeance unmerciful
}

\citebandsong{Nile:Ithyphallic}{Nile}{
  Laying Fire Upon Apep
}{
  Fire be upon thee Apep\\
  Ra maketh thee to burn\\
  Thou who art hateful unto him\\
  Ra pierceth thy head\\
  He cutteth through thy face\\
  Ra melteth thine countenance\\
  Lo your skull is crushed in his hand\\
  Thy bones are smashed in pieces

  Burn thou fiend\\
  Before the fire of the eye of Ra\\
  The hidden one hath overthrown thy words\\
  The gods have turned thy face backwards\\
  Thy skull is ripped from thy spine

  The lynx hath torn open thy breast\\
  The scorpion hath cast fetters upon thee\\
  Maat hath sent forth thy destruction\\
  Thou shalt burn

  [solo: Karl]

  The god Aker hath condemned thee to the flames

  Fire be upon thee Apep\\
  Thou enemy of Ra\\
  Let flames gnaw into thee\\
  And sear thy flesh\\
  Fall down Apep\\
  I hath set torch upon thee\\
  Taste thou death Apep\\
  The burning is upon you\\
  Thou art consumed\\
  I hath lain fire upon thee\\
  I hath smeared thy remains with excrement\\
  I hath spat on thin ashes\\
  Taste thou death
}

\target{Resphan curses of destruction}
Curses when the \resphan rebels devour the souls of their enemies:

\citebandsong{Nile:AnnihilationoftheWicked}{Nile}{
  Lashed to the Slave Stick
}{
  Your Corruptible Bodies Shall be Cut to Pieces.
  Your Souls Shall have No Existence.
  Ye Shall Never Again See Ra as He Journeyeth in the Hidden Land.
  The Doom of Ra is upon You.
}





\subsubsection{Dangers}
A warning about the dangers inherent in seeking magical power:

\citeauthorbook[p.175]{LinCarter:TheNecronomiconTheDeeTranslation}{Lin Carter}{
  The Necronomicon: The Dee Translation (part I.VII.III)
}{
  Aye, be thou warned, for in all such voyages and venturings of mind or soul or spirit there be very great and terrible dangers, by mortal men undreamt-of and unknown. 
  Beware then, lest thou penetrate too deeply into the blackest backward and depthless abysm of the womb of infinite time. 
  For beyond the very Beginning thereof, and on the Other Side thereof, there dwelleth That of which man suspecteth not; and there thou wilt find a strange and ominous Realm where hidden horrors lurk and naked Terror hunts unseen; which dim, uncanny bourn hath the seeming and the semblance of a pale, and grey, and indefinite shore, lapped by the sluggish waves of unmeasured and unthinkable Time.
  And it is even there, in an awful Light that is beyond all darkness, amidst a profound Silence that shrieketh beyond all sound, that \emph{They} slink and prowl in all their ghastliness, slavering with a loathsome and ana unspeakable hunger for all that is clean and whole and unsullied.
}




\subsubsection{Invocation of \daemons}
\target{Daemon invocation visuals}
Chaos sorcerers invoke \hr{Daemon}{\daemons}, or even the \xss. 

Read the section about the \hr{XS}{\xss}. 

\lyricsbalsagoth{%
  The Splendour of a Thousand Swords Gleaming Beneath the Blazon of the Hyperborean Empire% 
  \dash Part III: 
  Cry Havoc for Glory, and the Annihilation of the Titans of Chaos
}{
  Writhing tendrils of night-dark, coruscating energy lanced from the surface of the blade, entwining the King in a pulsating chrysalis of searing sorcerous power. \\
  His eyes shone deep crimson with an illuminatory radiance not born of this world, and forces which had lain dormant since before the fall of the Third Moon stirred at last from their aeons-old slumber\ldots{}
}

\lyricsbs{Nile}{What Can Be Safely Written}{
  It was him and his spawn that defeated the Elder Things,\\
  who had long possessed sovereignty of this world,\\
  before he descended on his gray and leathern wings\\
  through the upper gate opened by Yog-Sothoth.
}

\lyricsbs{Aeternus}{The Essence of the Elder}{
  Leaving the dismay of the world behind.\\
  Traveling in the essence of chaos.\\
  Time and matter dissolving about me.\\
  A journey beyond death.
  
  Your blood pulses with a thousand years of existence.
  
  Standing proud before those whose ageless blood\\
  flows through my veins.\\
  Their spiraling souls empower me.\\
  Visions of chaos.
  
  A union of blood, a union of souls.\\
  An endless bond sworn and upheld.
  
  I walk with you\\
  over the fields you bled red,\\
  through fires that finally devoured you,\\
  through your dreams and your nightmares.
}

\Tiamat{} and the \firstgendragons{} (who are \hr{Elder Dragons worshipped}{dead but still worshipped}) are invoked in Chaos magic. 
She was the one who made the original pacts with the \xss, so she must be invoked when one seeks aid from the \xss. 

\lyricsauthorbookpage{Anton Szandor LaVey}{The Satanic Bible}{116}{%
  In the name of Satan [\TyarithXserasshana?], 
  Ruler of the earth, 
  King of the world, 
  I command the forces of Darkness to bestow their Infernal power upon me.  
  Open wide the gates of Hell and come forth from the abyss in answer to your most Unholy names\ldots{}
}

\lyricsauthorbookpage{Anton Szandor LaVey}{The Satanic Bible}{147}{%
  \textbf{Invocation Employed Towards the Conjuration of Lust}
  
  Come forth, Oh great spawn of the abyss and make thy presence manifest. I have set my thoughts upon the blazing pinnacle which glows with the chosen lust of the moments of increase and grows fervent in the turgid swell. 

  Send forth that messenger of voluptuous delights, and let these obscene vistas of my dark desires take form in future deeds and doings. 
  
  From the sixth tower of Satan there shall come a sign which joineth with those saltes within, and as such will move the body of the flesh of my summoning. 
  
  I have gathered forth my symbols and prepare my garnishings of the is to be, and the image of my creation lurketh as a seething basilisk awaiting his release. 
  
  The vision shall become as reality and through the nourishment that my sacrifice giveth, the angles of the first dimension shall become the substance of the third. 
  
  Go out into the void of night (light of day) and pierce that mind that respondeth with thoughts which leadeth to paths of lewd abandon. 
  
  (Male) My rod is athrust! The penetrating force of my venom shall shatter the sanctity of that mind which is barren of lust; and as the seed falleth, so shall its vapours be spread within that reeling brain benumbing it to helplessness according to my will! In the name of the great god Pan, may my secret thoughts be marshalled into the movements of the flesh of that which I desire! 
  
  Shemhamforash! Hail Satan! 
  
  (Female) My loins are aflame! The dripping of the nectar from my eager cleft shall act as pollen to that slumbering brain, and the mind that feels not lust shall on a sudden reel with crazed impulse. And when my mighty surge is spent, new wanderings shall begin; and that flesh which I desire shall come to me. In the names of the great harlot of Babylon, and of Lilith, and of Hecate, may my lust be fulfilled! 
  
  Shemhamforash! Hail Satan!
  
  \textbf{Invocation Employed Towards the Conjuration of Destruction}
  
  Behold! The mighty voices of my vengeance smash the stillness of the air and stand as monoliths of wrath upon a plain of writhing serpents. I am become as a monstrous machine of annihilation to the festering fragments of the body of he (she) who would detain me. 

  It repenteth me not that my summons doth ride upon the blasting winds which multiply   the sting of my bitterness; And great black slimy shapes shall rise from brackish pits 
  and vomit forth their pustulence into his (her) puny brain. 
  
  I call upon the messengers of doom to slash with grim delight this victim I hath chosen. Silent is that voiceless bird that feeds upon the brain-pulp of him (her) who hath tormented me, and the agony of the is to be shall sustain itself in shrieks of pain, only to serve as signals of warning to those who would resent my being. 
  
  Oh come forth in the name of Abaddon and destroy him (her) whose name I giveth as a sign. 
  
  Oh great brothers of the night, thou who makest my place of comfort, who rideth out upon the hot winds of Hell, who dwelleth in the devil's fane; Move and appear! Present yourselves to him (her) who sustaineth the rottenness of the mind that moves the gibbering mouth that mocks the just and strong!; rend that gaggling tongue and close his (her) throat, Oh Kali! Pierce his (her) lungs with the stings of scorpions, Oh Sekhmet! Plunge his (her) substance into the dismal void, Oh mighty Dagon! 
  
  I thrust aloft the bifid barb of Hell and on its tines resplendently impaled my sacrifice through vengeance rests! 
  
  Shemhamforash! Hail Satan!
}





\subsubsection{Protection}
\Merkyran religious spell/ritual to ward off umbrae:

\citebandsong{Nile:Ithyphallic}{Nile}{
  Papyrus Containing the Spell to Preserve Its Possessor Against Attacks from He who is in the Water
}{
  Amun\\
  Lord of the gods\\
  Thou who art of the four rams heads upon thy neck\\
  Thou standest upon the spine of the crocodile fiends\\
  To thine sides are the dog headed apes\\
  The transformed spirits of the dawn

  Drive away from me the lions of the wastes\\
  The crocodiles which come forth from the river\\
  The bite of poisonous reptiles\\
  Which crawl forth from their holes

  Be driven back crocodile thou spawn of Set\\
  Move not by means of thy tail\\
  Work not thy feet and legs\\
  Open not thy mouth\\
  Let the water which is before thee\\
  Turn into a consuming fire

  I possess the spell to\\
  Preserve me from he who is in the water

  Thou whom the thirty seven gods didst make\\
  And whom the serpent of Ra didst put in chains\\
  Thou who wast fettered with links of iron\\
  In the presence of Ra\\
  Be driven back thou spawn of Set

  Drive away from me the lions of the wastes\\
  The crocodiles which come forth from the river\\
  The bite of poisonous reptiles\\
  Which crawl forth from their holes
}





\subsubsection{Spellwords}
\target{Spellwords}
Remember to have mystic spellwords and incantations. 
\quo{Forgotten spells not uttered in many aeons\ldots{}}

\Draconic spellwords included \hypota{neras gulja} and especially \hypota{zurra!}.

\lyricsbalsagoth{Shackled to the Trilithon of Kutulu}{
  Rise o' spawn of Chaos and elder night.\\
  With these words (and by the sign of Kish), I summon thee.\\
  Slumbering serpent, primal and serene,\\
  Great Old One, hearken to me!
  
  When the stars align in the Chaosphere, \\
  then the time of awakening shall be at hand!
}

\lyricsflnv{3}{
\ta{Zarrhahull aknesh zain taushaark!}

\ta{Aggrhul shai zain hoorghth.}}

\lyricsflnv{5}{
\ta{Necrain thanasthos eheirrheinnhenn rais!\\
    Deshtor neshan tharrh!\\
    Berrhell nerrhell rhul!\\
    Bezhul tharrh!\\
    Rais!\\
    Rais!
    
    Aalheen dorth haaarth zhhuuull!}
}

\lyricsflnv{10}{
\ta{Hierremm haarhzhuul haashthorn n herh thorr!\\
    Herrint sheen!
    
    Shemterrann!}
}

\lyricsbs{\FMFroideval}{Succubus (Volume 1)}{
  \ta{Hieram zaraoth, Desdemona! Hierem haarth!}
}





\subsubsection{Summoning magic}
\target{Summoning magic visuals}
\lyricsbs{Bal-Sagoth}{%
  As the Vortex Illumines the Crystalline Walls of Kor-Avul-Thaa%
}{
  The sky rent asunder. \\
  Black-winged devils surge forth from the void.\\
  A maelstrom of crimson fire burns above us.\\
  What carnage hast thou wrought?
  
  And beyond the Vortex, the churning black waters of the Void did disgorge the Dwellers in Eternal Shadow.\\ 
  And upon a horde of winged horrors, brandishing swords of ebon flame, they rode out from the gate. \\
  And a terible silence fell upon Kor-Avul-Thaa.
}





\subsubsection{Visualizing the \daemons}
\target{Visualizing Daemons}
Just \hr{Visualizing Archons}{like the \Archons}, the various \hr{Daemon}{\daemons} or classes of \daemons{} and gods feel differently. 

\lyricslimyaael{427806}{%
  \textbf{3) Beckon the grotesque.} 
  
  I've wondered lately why descriptive passages on magic in so many fantasy novels do nothing for me anymore. 
  There are doubtless multiple reasons, but I think part of it is that, even when the authors are writing about destructive magic or evil inhuman creatures like the Unseelie Court, they describe the effects of magic as beautiful, or pretty. 
  That tends in the direction of fluff if the author isn't careful. If she is, it'll still call up very similar pictures from a lot of other fantasy books.
  
  I've been thrilled and felt wonder from descriptions of the grotesque, however. 
  I still haven't managed to finish \emph{Perdido Street Station}, but the descriptions of New Crobuzon, especially the beetle-headed khepri, are a lot more intriguing than yet another scene of moonlit pools and silver wolves and unicorns. 
  And my candidate for most awe-inspiring magical talent I've read about this year isn't the king-and-the-land magic in \emph{The Fall of the Kings}, although it was beautifully described. 
  It's the ability to grow cocoons on one's palms and hatch insects from them that I read about in \emph{The Etched City}. 
  I'm also enjoying the three brothers nested in each other like Russian dolls from \emph{Someone Comes to Town, Someone Leaves Town}, although that's been slow reading for other reasons.

  Many fluffy magical systems that blur into each other across fantasy books share common touchstones\dash\quo{beautiful} animals like horses and wolves, images of light from moon and sun, natural elements like water and fire that we've been trained to admire, brilliant \colours. Replacing even a few of those touchstones may lead to the sense of the strange, the weird, the alienness that we don't understand and recoil from. Insects, disease, filth, blood, and mutated and decaying bodies are much less often terms of fluffy magic. Try beckoning the grotesque into your magical system and see what it does.
}















\section{Chaos magic}
\target{Chaos magic}




\subsection{Aenigmata and Gnosis}
\target{Aenigma}
\target{Aenigmata}
\target{Gnosis}
\index{Aenigma}
\index{Gnosis}
Chaos magic is based on certain \emph{Aenigmata} (singular \emph{Aenigma}). 
They are a kind of cosmic mysteries or \quo{equations} that must be understood and solved. 
The solution to an Aenigma is a \emph{Gnosis} (plural \emph{Gnoses}). 

To possess the Gnosis of an Aenigma is not a binary thing, but something gradual. 
One explores an Aenigma and gradually gains more Gnosis, but a perfect Gnosis is feasible only in the case of the simpler Aenigmata. 
Under the reigning axioms of Aenigma theory: 
\begin{enumerate}
  \item \ldots{} certain Aenigmata \emph{have} no Gnosis. 
  \item \ldots{} certain Aenigmata have a Gnosis, but this Gnosis can provably never be known to perfection. 
  \item \ldots{} and many more Aenigmata are suspected of being unknowable. 
\end{enumerate}

A Gnosis is something deep and intuitive, and can be told and taught only with difficulty. 
They are not simply bits of trivia but skills that must be studied, experienced and learned. 

\quo{Aenigma} is a broad term covering many things. 
Among other things, every living creature has a particular, unique Aenigma associated with it. 
This Aenigma can be compared to a DNA code (and indeed, the DNA code is one aspect of the Aenigma). 
To possess the Gnosis of a being's Aenigma is to have occult power over that creature. 

\index{name!true name (Gnosis)}
\index{true name}
The Gnosis of a person is sometimes poetically referred to as the person's \quo{true name}. 

Any Chaos mage worth his salt will have it as one of his life's goals to attain the Gnosis of his own Aenigma. 
Self-Gnosis is the key to much power. 

Similarly, one must possess at least some measure of Gnosis of a god or \hr{Daemon}{\daemons} in order to \hr{Chaos invocation}{invoke it}. 






\subsection{Cost}
A Chaos sorcerer must sacrifice blood\dash your own \emph{and} that of others!\dash to the \hr{Daemon}{\daemons}. See \hr{Telderain}{\Telderain} for an example. Much Chaos magic involves explicit pacts with dark powers, such as \hr{Rissit}{\RissitNechsain} or the \hs{Sentinels} and their masters.

If the sorcerer is not skilled or strong-willed enough to handle it, Chaos magic can take a terrible toll on his body and mind, leaving his mind maddened and his body twisted, deformed and sickened\dash rotting from the inside out due to the influence of unnatural, otherworldly power (\quo{unnatural} in the sense that it is alien to the world of the Shroud). Compare to Hannan Mosag and his K'risnan in \cite{StevenEriksonIanCameronEsslemont:MalazanBookoftheFallen}. Also, the disgusting priests in the movie \emph{300}.

Chaos tends to twist the mind into something more chaotic and bestial. It brings forth primal urges of lust, greed and aggression and encourages the mage to act on these instincts. As a result, a Chaos mage tends to become warped and mad with time. Most \dragons{} exhibit these traits due to practicing Chaos magic; indeed, these traits are so pervasive among their race that it is considered their natural state (although a \dragon{} might turn out differently if it refrained from using Chaos magic all the time), and people whose minds are twisted through Chaos magic are said to acquire \draconic{} minds. 





\subsubsection{Sacrifices}
A weak Chaos mage can enable himself to cast more powerful spells that he otherwise could by making sacrifices not just to the gods, but directly to the \hr{Daemon}{\daemons}. 
It does not have to be something of his own. 
You can feed the \daemons{} with the blood, flesh or even souls of others. 

This is not actual bargaining. 
Remember, the \daemons{} are barely sentient and certainly not intelligent. 
The real issue is that the mage must pay of his own energy to invoke the \daemons. 
You need not pay everything up front. 
You can have the \daemons{} do their task and then collect the rest of their reward from your body afterwards. 
But if you have something else to offer, you can compel the \daemons{} to feed on that instead, leaving your own body energy untapped. 
This means you can call in more or bigger \daemons{} than you could otherwise afford. 

If you call in too many \daemons, expecting to have something to offer them, and it turns out you cannot pay up, then they will take what they are owed from your body. 
This can kill the mage. 









\subsection{Gods and \daemons}
Chaos magic was based on pacts with gods. 
A mage would often learn spells and be given energy from one or more patron gods. 
In return, the mage would have to perform services or pay tribute (in the form of sacrifices) to the gods. 

The gods could bestow power and energy to the mage, or they could just grant knowledge.
Gods often preferred the latter, because it was cheaper for them.
(A god would not want to lose out on a deal by giving more energy to the mage than it got back.)
A gift of knowledge had a drawback, though: 
It was harder for the god to revoke. 
The mage could, in principle, take the knowledge and run away without repaying the god. 









\subsubsection{\Daemons}
\target{Daemon}
The \daemons were mythical \quo{creatures} that chaos mages summoned in order to cast magic. 

In reality, the \daemons were not true creatures.
They were insubstantial machines, robots or programs created by the \xss to serve them as tools. 

There were endless hordes/masses/swarms of the mindless, nameless \daemons. 

See also the sections on \hr{Daemon invocation visuals}{\daemon invocation visuals} and \hr{Visualizing Daemons}{visualizing \daemons}. 

I should have a big pantheon of \daemons{} from which to draw. 







\subsection{How to cast it}
Chaos sorcerers must invoke the names of \hr{Daemon}{\daemons} and dark gods when casting their magic. The amount of invocation depends on the power of the spell/entity compared to that of the sorcerer. An immensely powerful sorcerer, such as one of the great \dragons, can perform feats of tremendous magic without a word, and need invoke aloud only the greatest of \hs{cosmic gods} and \hr{XS}{\xzaishanns}. A lesser sorcerer will need to call out the names of lesser \daemons{} as well. 

\target{Invoking Daemons}
Every Chaos spell should invoke one or more \daemons{}. 
Sometimes \xss, sometimes just lesser \daemons. 
And it should invoke \Sethicus and \Tiamat. 

Chaos magic, compared to Vaimon magic, relies much more on words. 
A spell is a sentence in the \Draconic{} tongue which invokes the \daemon{} or \daemons{} you want and (superficially) describes the effect. 
Chaos magic also uses occult symbols. 
For simple spells, the symbol is usually already encarved on some item, such as a staff or amulet, or as a tattoo, and the item is used in the spellcasting. 
In more complex (and longer) spells, symbols are drawn while casting the spell, on the ground, on paper or on one's own body. 
Chaos magic spells are usually cast by a single caster, although they may be long, requiring a complex ritual. 




\subsubsection{Invocations}
\target{Invoking the XS}
\target{Chaos invocation}
Chaos sorcerers invoked \hr{Daemon}{\daemons} and gods to cast spells.
\quo{Gods} might be \hr{Taorthae}{\taorthae}, \dragons or even \hr{XS}{\xss} (\hr{Primordial}{\Primordials}). 

Some \hs{Gnosis} of a god or \daemon{} was required before you could invoke it. 
(For a \rethyax, this Gnosis came in the form of \hr{Arcanum}{\arcana}.)

\Tiamat{} and the \firstgendragons{} (who are \hr{Elder Dragons worshipped}{dead but still worshipped}) are invoked in Chaos magic. 
She was the one who made the original pacts with the \xss, so she must be invoked when one seeks aid from the \xss. 

See also the sections on \hr{Daemon invocation visuals}{\daemon invocation visuals} and \hr{Visualizing Daemons}{visualizing \daemons}. 

Read the section about the \hr{XS}{\xss}. 





\subsubsection{Magic types and \xss}
Spells and types of magic were associated with specific \xss.

\begin{description}
  \item[Fire magic:]
    \target{Ruin Satha fire magic}
    For example, fire magic was {associated with \RuinSatha}. 
    They would invoke him (with his descriptions and titles) whenever they had to cast such magic. 
    
    The fire magic of \RuinSatha was some of the most immediately destructive magic the \dragons had at their disposal.
    It created huge-ass flames of all sorts of \colours (depending on the specific spell and the effect you wanted). 
    These flames were not just physical, but spiritual as well. 
    
    Whenever such magic is cast, remember to look up the visuals associated with \hr{Curses of destruction visuals}{destructive magic}.
    
    \RuinSatha represented change through creation and destruction. 
    His flames could sometimes cause mutation, or cause new beings to spring to life. 
    Compare to the Chaos god Tzeentch from \cite{RPG:Warhammer40000}. 
    Also a bit like the All-Spark from \cite{Movie:Transformers2007}.
\end{description}









\subsection{Initiation}
\target{Chaos mage initiation}
Maybe have scenes where Chaos mages are initiated. 
Perhaps \hr{Moro Cornel}{Moro \Cornel}.

\lyricsbs{Hate Eternal}{Rising Legions of Black}{
  Mark thy masters wrath. \\
  The scrolls now entangled.\\
  I offer my blood in chants of disgust.\\
  Enter a dimension of hate.\\
  Rejoice in flaming circles.\\
  Temptation of ones blinding faith.
  
  Rings of fire engulfing the earth, now brought to a blaze.\\
  Bound by the shadows that dwell from within.\\
  Awaken the beasts now speaking in tongues, invoking despair.\\
  I am the grace of the rising legions of black.
}

\lyricsbs{Emperor}{Alsvartr}{
  Hark, O'Nightspirit,\\
  father of my dark self.\\
  From within this realm, wherein Thou dwelleth,\\
  by this lake of blood, from which we feed to breed,\\
  I call silently from Thy presence, as I lay this oath.
  
  May this night carry my will\\
  and may these old mountains\\
  forever remember this night.\\
  May the forest whisper my name\\
  and may the storm bring these words\\
  to the end of all worlds.
  
  May the wise moon be my witness\\
  as I swear on my \honour,\\
  in respect of my pride and darkness itself,\\
  that I shall rule by the blackest wisdom.
  
  O' Nightspirit!\\
  I am at one with thee.\\
  I am the eternal power.\\
  I am the Emperor.
}









\subsection{Meditation}
\target{Rethyax meditation}
\target{Guthac}
\target{Yothmac}
\index{Guthac}
\index{Yothmac}
When a \rethyax meditates he travels down into the mind, into arcane regions where his mind can reach the gods and \daemons.
He travels down the Onyx Stairs and past the Gate of the Animal Mind (through which only higher minds are able to pass). 
Here he passes the demigods Guthac (\Venus) and Yothmac (\Mars), who help him to open up his mind so that he can enter the Gulf of Obad-Sherah.









\subsection{Spells}
\begin{gloss}
  \gitem{khestni}
  \target{khestni}
  A \hr{True Draconic}{\TrueDraconic} word of power meaning \quo{die}. 
  As a spell, it kills. 
  \word{Khestni}, however, is not necessarily an instant-kill-spell, except when used on foes far weaker than the caster. 
  It is like a dagger's thrust: 
  Easy to defend against if you are prepared, but quick and easy, and very deadly in the right circumstances. 
  
  \word{Khestni} is not effective if used several times in a row against the same foe. 
  For some reason. 
  So just spamming \word{khestni} is not smart. 
\end{gloss}





\subsubsection{Summoning}
There were spells that could summon a \malgryph or a \firesalamander. 
The \firesalamander was the ultimate manifestation of \hr{Ruin Satha fire magic}{\draconian fire magic}. 





\subsubsection{Wraithworms}
There were spells that summoned wraithworms. 
Wraithworms were \daemons that took on the semblance of a skeletal snake. 
They were short-lived; they attacked once or a few times and then dissipated. 

Compare to the spell \quo{Bone Spirit} in the game \cite{VideoGame:DiabloII}. 















\section{\Matrix Theory}
Sidereal divination. 
The spiritual firmament and the art of interpreting it. 







\subsection{Astrology}
\target{Astrology}
\target{astrology}
\index{astrology}
\Matrices{} are connected with {astrology}. 
They can be observed in a way similar to how you can observe stars. 

An astromancer goes into a mystic trance and beholds a visualization of the status quo of the \matrices, their balance of power. 
This visualization takes the shape of a starry sky. 
Each \matrix{} is represented by a constellation in that sky. 
These constellations are given traditional names that symbolically refer to their associated \matrix. 
The \matrices{} and constellations are identified to such an extent that they are virtually synonymous in daily speech. 
The constellations only \emph{very} vaguely resemble the things they are named after; the names are poetic and traditional. 

By observing the subtle mystical signals the stars send out, one can gain a certain understanding of how the \matrices{} stand: 
Which \vertices{} are aligned with which \matrices, how much power does each \matrix{} have, etc. 
It is difficult to interpret, though, since \quo{balance of power} is not an easily quantifiable (meta)physical property. 

These mystic stars appear to blaze with a mighty fire\dash many-coloured, and at times cold-looking.

\target{Vertices in the sky}
Unlike physical constellations, the mystic constellations \emph{move} in the sky. 
They can shrink, grow, twist and writhe. 
The same goes for individual \vertices.
They have a position in the sky, indicating their position in one or more \matrices, but they can move around.  

\target{intersecting}
When two \matrices{} (or \vertices) combine their powers, they are said to \emph{intersect}.
But only if they work together in a way that is metaphysically measurable, such as by pooling their magical power for a joint spell. 





\subsubsection{Moons}
The \hs{Moons} have mystical power and are inhabited by sinister gods and creatures, such as the \hr{Moon-Wolves}{\moonwolves}.
See that section.









\subsection{\Matrix}
A single \vertex is rarely powerful enough to affect dramatic change on a great scale. 
So \vertices{} band together for power. 
They form a \matrixx, an organized group of \vertices{} bound together. 

A \matrix{} is also called a constellation. 

A \matrix{} is actually more than just a bunch of \vertices. 
A \matrix{} is a front-end for a \dweomer{} (or more than one \dweomer). 
It is an occult infrastructure that a whole group (or even a whole civilization) can draw on. 
It strengthens their magic and thus makes the group greater than the sum of its parts. 

A \matrix{} may contain smaller sub-\matrices. 

A \matrixx{} can \quo{eclipse} another \matrixx. 





\subsubsection{\Apex}
\target{Apex}
\index{\apex}
To be truly effective, a \matrixx{} must have an \apex, a leader \vertex. An \apex{} can be chosen unanimously, or several candidates may fight over the title. It is possible for a \vertex{} to force himself into a position of \apex, binding the rest of the \matrixx{} to his will (for a time, at least). 

\target{Cardinal point}
\index{\cardinalpoint}
Below the \apex, the important, pivotal \vertices{} of a matrix are called \cardinalpoints. 
For maximum effect, the \cardinalpoints must have the correct number and be correctly arranged (in accordance with \hs{occult geometry} and the \hr{Matrix formation}{formation of the \matrix}). 





\subsubsection{Circles of mages}
\target{Circles are Matrices}
A \hs{circle of mages} is {a small, temporary \matrix}. 
As such, it obeys (or should obey) the rules of \hr{Matrix formation}{\matrix formations}, \hs{occult geometry} and numerology. 





\subsubsection{\Draconic \matrices}
\target{Draconic Matrix}
\target{Draconic Matrices}
\target{Draconian Matrix}
\target{Draconian Matrices}
A \draconic \matrix required at least three or four core \vertices: 
\begin{enumerate}
  \item The Flame, representing the power of \RuinSatha. 
  \item The Skull, representing the power of \KhothSell. 
  \item The Hollow, representing the power of \NaathKurRamalech. 
  \item The Portal, representing the power of \Achamoth. 
\end{enumerate}






\subsubsection{Formations}
\target{Matrix formation}
A \matrix can assume a number of different formations. 
Each formation has different properties, in accordance with \hs{occult geometry} and numerology. 

Each formation has a different optimal number and arrangement of \hr{Cardinal point}{\cardinalpoints}. 





\subsubsection{List of \matrices}
The \matrices{} include (remember to give them all shapes and names!):

\begin{gloss}
  \gitemthe{Diamond} 
    \target{Diamond}
    The sub-\matrix that rules the Hydra. 
    
    \Apex: 
    Once \Nexagglachel.
    Later \Ishnaruchaefir.

    Members include \Rystessakhin.
    Later \Secherdamon became a member. 
  
  \gitemthe{Hydra} 
    \target{Hydra}
    \target{Pyre}
    The \dragons{} and their \xs-born power. 
    
    \Apex: 
    Once \TyarithXserasshana. 
    Later \Vizsherioch. 
    \Nexagglachel{} and \Secherdamon{} both strove for this position but never achieved it. 
    
    \index{Dagger, the}%
    There is also a \quo{\hs{Dagger}} position. 

  \gitemthe{Midnight Bat} 
    \target{Midnight Bat}
    \hr{Mystraacht Matrix}{\Mystraacht{} \matrix}. 
    Sub-\matrix{} of the Paths of Ice. 
    
    \Apex: 
    Once \Zachirah. 
    Then none, for a long while. 
    Ultimately Ramiel. 
    
    \CardinalPoints: 
    \Nathrach, \Shiaraid, Ramiel, \Dasteron, \Kishiel. 

  \gitemthe{Ouroboros}
    \target{Ouroboros}
    The \caisith{} \matrix{}. 
  
  \gitemthe{Paths of Ice} 
    The \banelords.
    The constellation gives associations of a thin sheet of ice (like the one covering a lake) that is slowly but surely cracking and breaking up. 
    The \quo{paths} are these cracks. 
    
    The Paths of Ice contain the Silver-Shining Rose and the other dynasty \matrices{} as sub-\matrices. 
    
    \Apex: 
    \Daggerrain. 
    
    \CardinalPoints: All \banelords. 
  
  \gitemthe{\Malgryph}
    \target{Malgryph constellation}
    Contains the stars representing \hr{Zaz}{\Zaz and \Urzaz}. 
    Was \hr{Urizeth researches Malgryph constellation}{researched by \Urizeth}. 
  
  \gitemthe{Orchid} 
    \target{Orchid}
    \target{Silver-Shining Rose}
    \CiriathSepher. 
    Sub-\matrix{} of the Paths of Ice. 
    Previously called the Silver-Shining Rose.\index{Silver-Shining Rose}
    
    \Apex:
    \Azraid. 
    
    \CardinalPoints: 
    \Teshrial{} is a \cardinalpoint{} in a sub-\matrix.
  
  \gitemthe{Salamander}
  
  \gitemthe{Torch}
  
  \gitem{Unnamed} 
    \Iquinian{} \matrix.
    Comprised of the sixteen \sephiroth. 
    
    \Apex: 
    None. The \sephiroth{} are equal in status. 
  
  \gitem{Unnamed} 
    Imetric \matrix. 
    
    \Apex: 
    Salacar. 
  
  \gitem{Unnamed} 
    Rissitic \matrix. 
    
    \Apex: 
    \HriistN{} (\Secherdamon). 
  
  \gitem{Unnamed} 
    \target{Vorcanth Matrix}
    \Vorcanth{} \matrix.
    Closely associated with \hs{Visha}. 
    
    \Apex: 
    Unknown. 
    Might be some great \vorcanth{} leader. 
\end{gloss}





\subsubsection{Visualizing a \matrixx}
\target{Visualizing a matrix}
In \hs{dreams} or using magic, it is possible to visualize a \matrixx{}. It can appear as a city. The \apex{} will be a castle or throne (possibly empty), the \vertices{} will be immense columns or statues, and the streets below will be filled with slaves, toiling as directed by the \vertices, swarming about like ants. 










\subsection{\Nexus}
\target{Nexus}
\index{\nexus}
A physical place with great power, where it is possible to manipulate the Web of the Realms, is called a \nexus. 





\subsubsection{Ley lines}
Perhaps there are ley lines of \nexus{} energy crisscrossing the planet and the universe. These form the key threads of the Web of Realms. 

\target{Myths of creatures beneath the earth}
There are \hs{myths} of creatures dwelling beneath the earth. Supposedly, their movement and supernatural power affects the surface world in a Feng Shui kind of way. These creatures might be \trueophidian{} lords or even more ancient monsters. 









\subsection{Scientific}
\Matrix theory is more scientific than, say \hr{Sethican philosophy}{\Sethican mysticism}. 
It was more universally accepted because it really \emph{is} more true. 
And because was was less horrible to think about than \Sethicus's \xs mysticism. 

The \dragons and the \resphain had very similar conceptions of \matrix theory. 









\subsection{\Vertex}
\target{Matrix}
\index{\matrix}
The Shroud is the phenomenon that makes creatures unable to see the true universe, seeing only a narrow slice of it. Thus, the Shroud is the force that creates and separates the illusory worlds which mortals inhabit. Only mighty creatures, like \dragons, \banes{} and \resphain, and those learned in occult lore, even know of the Shroud and can see it.

\target{Vertex}
\index{\vertex}
People with the power to see into and influence the Shroud are called \vertices. 
Ordinary people cannot see the Shroud and are slaves of it. 
Even those mages who know it exists must usually use magic to even see it, and use elaborate spells to affect it. 
One becomes a \vertex{} through force of will, through a refusal to believe in the illusion that one's brain tries to impose. 
Some \vertices{} are feeble and not even mages (such as Lica, the \quo{clairvoyant} girl introduced in \emph{\LicaBook}). 

\Vertices{} are those individuals who somehow reach outside the Shroud and as such are able to understand and influence the world, to some extent. 

Normal people who are not vertices are sometimes called \quo{specks} (as in \quo{specks of dust}). 





\subsubsection{Detecting \vertices}
\target{Detecting Vertices}
A person skilled in \matrix{} theory, or simply a mage or telepath, can detect \vertices{} when they come physically near. 
They emit \quo{vibrations} through the Web of Realms, like tremours that can be felt. 
These tremours are stronger for more powerful \vertices. 
And they are especially obvious within the \hs{Shrouded Realms} because these Realms are normally rigid and free of strong \vertex{} influence (whereas the Immortal Realms have \vertices{} all over the place). 
So when a mammoth \vertex{} such as \QuessanthIshnaruchaefir{} suddenly appears in \Azmith, it can be felt for miles away. 





\subsubsection{\Vertices vs. mages}
\target{Vertices vs. mages}
Not all mages are \vertices, although some are. 
A \vertex{} is (potentially) much more powerful than a mage, just like a king is more powerful than a sauropod. 









\subsection{Zenith and Nadir}
\target{Zenith}
\target{Nadir}
\index{Zenith}
\index{Nadir}
The \quo{Zenith} is the metaphysical \quo{place} where the \hr{Heart}{Heart of \Miith} waits. 
All \matrices{} strive to reach the Zenith and control the Heart. 

A Nadir is a \quo{low point} for a \matrix{}, constellation or \vertex. 
It is a state where the \matrix{} is abnormally weak. 
A \matrix{} can be forced into a Nadir by external forces (usually temporarily). 
Some \matrices{} have natural periods of Nadir for astrological or other reasons. 















\section{Psionics}
\target{Psionics}
\target{psionics}
\target{psionic}
\index{psionics}
Psionics is the art of reaching out with your \hs{extended soul} into the \hs{Beyond} and thus detect or affect other things in the world. 

The term \quo{psionics} is only known and used by a few researchers. 
Most people know only the concept \quo{\hs{magic}}. 

One of the possible uses of psionics is to contact otherworldly entities and enlist their aid. 
This manifests in \hs{invocations} and \hs{orisons}, which form the basis of most magic. 
So magic is based on psionics. 

Some people (including certain \hr{Ophidians}{\ophidians}) revile \hs{sorcery} (summoning-based magic) as evil and use only psionics. 









\subsection{Telepathy}
\target{Telepathy}
\target{telepathy}
A few creatures are naturally telepathic. 
This includes \hr{Nycan}{\nycans} and the rare \hr{Nycaneer}{\nycaneer} \scathae. 
Other creatures (immortal and mortal alike) can learn telepathy, but it is not a natural part of them nor their culture. 

Telepathy is a kind of \hs{psionics}. 
It allows you to communicate silently, mind-to-mind at a distance. 

The distance is still limited. 
The best telepaths can manage a range of a few kilometres. 

If the people communicating have no shared language, only pictures, sensations and vague emotions can be transmitted. This is slow and difficult to make sense of. 

Sending thoughts is easy. 
Reading them is much harder. 
Even a skilled telepath can only read surface thoughts and feelings, and only vaguely. 
Smooth communication requires that both parts be telepaths. 

The Imetrians have the most advanced telepathy. 
The \banes{} also have it, but they don't teach it to any but their trusted servitors. 















\section{Schools of magic}
Every self-respecting society has mages of some kind. But magic is not just magic. There are many different approaches to magic. Pretty much every civilization has its own magic theory, its own tradition of magic. 

But why does this diversity persist? Why do mages from all over the world not come together, exchange knowledge, do some research and find out which theory of magic is correct? Well, for several reasons: 

\begin{enumerate}
  \item Science is difficult. 
    Each theory has its pros and cons, and it is not always possible to \quo{combine} them. 
  \item Pride. 
    Often, scientists are very proud of their own theory, the one they were brought up with or may have helped develop. People want their own view to be correct. This makes them inflexible, irrational and unwilling to compromise and acknowledge possible flaws in their theory. 
  \item Distrust. 
    \Miith{} is not a friendly place where everyone gets along. Wars are fought between nations every day, and even in peace time there is the threat of war, old hatreds, racism, feelings of cultural superiority, and the fear of the unknown, all of which inhibit communication and the free exchange of scientific knowledge. 
  \item Secrecy. 
    Apart from the irrational \quo{distrust} above, many cultures have dreams of world domination. These are unwilling to share their knowledge, because knowledge is power, and they do not want to share power. 
  \item Religion. 
    Some magic is directly connected to some religion and weaves ties between the spellcaster and the gods of that pantheon. Typically, no one wants to be dependent upon the gods of another religion. And even in the cases where the magic is directly connected to specific gods, the theory is often intertwined with the dogmas and world view of some religion, which nonbelievers will be hesitant to accept. 
  \item Complexity. 
    The magical Universe is vast, dark and mysterious. At TL3, no civilization is close to understanding more than the tiniest fraction of it. The truth of how magic works in the Universe is perhaps unknowable, and at any rate far too complex to have been even glimpsed. But magic is a strange thing and manifests itself in many forms. 
  \item Fear. 
    This is a consequence of the fear of the unknown. The Universe of magic is vast and dark, and much magic will be horrible and frightening to the uninitiated. There is, of course, the very real risk that a spell or ritual might go horribly wrong and cause an explosion, transform the caster into an undead monster or accidentally conjure some demon from an alien world, but even apart from that, the supernatural tends to instill people with irrational fear and loathing unless they are very strong-willed or have been brought up to accept it. Therefore, most people will accept the magic of their own culture, but view most foreign magic as something foul and evil. 
\end{enumerate}

The result of all these factors is that every culture has its own magic, its own methods for casting spells. 

Schools of magic on \Miith{} include: 





\begin{gloss}
  
  
  
  
  \begin{comment}
    \paragraph{\Aryoth magic}
  \end{comment}
  \gitem{\Aryoth magic}
  See the section about \hr{Aryoth magic}{\aryoth magic}. 
  
  
  
  
  
  
  
  
  
  \begin{comment}
    \paragraph{\Bane magic}
  \end{comment}
  \gitem{\Bane magic}
  The original \bane{} magic drew its power directly from \Erebos{}. But since the conduit (vortex) to \Erebos{} has been sealed, they needed to find a new \hr{Dweomer}{\dweomer}. Aided by the \nephilic{} sorcerer \Semiza, they \hr{Origin of Qliphoth}{created the \Qliphoth}, who serve as conduits to \Erebos{} through which magical energy can be channelled. 
  
  (How were the \Qliphoth{} created? 
  From sacrificed \banelords{} or from enslaved \pdaemons{}/\mdaemons{}?)
  
  \Bane/\itzach{} magic sometimes works like defiler magic (from \emph{Dungeons and Dragons: Dark Sun}), sucking the life energy out of nearby creatures, leaving plants withered and dead. This is because the \ps{\banes}{} power is based on \hr{Entropy}{death, stagnation and parasitism}. is based on \hr{Entropy}{death, stagnation and parasitism}. 
  
  
  
  
  
  
  
  
  
  
  \begin{comment}
    \paragraph{Chaos magic}
  \end{comment}
  \gitem{\hs{Chaos magic}}
    As used by the \dragons.
  
  
  
  
  
  
  
  
  
  
  \begin{comment}
    \paragraph{Imetric magic}
  \end{comment}
  \gitem{\hs{Imetric magic}}
  
  
  
  
  
  
  
  
  
  
  \begin{comment}
    \paragraph{Naga magic}
  \end{comment}
  \gitem{\hr{Naga magic}{\Naga magic}}
  Many water-related spells were adopted from the \nagae. 
  Their incations were in the tongue of \hr{Nag}{Nag}.
  
  
  
  
  
  
  
  
  
  \begin{comment}\paragraph{\QuilJaaran magic}\end{comment}
  \gitem{\QuilJaaran{} magic}
  \target{QJ magic}
  \QuilJaaran{} magic was based on occult logic and philosophy, originally learned from some cosmic gods. 
  Or perhaps from the \voyagers. 
  
  \hs{Imetric magic} was very much inspired by the \quiljaaran{} tradition.
  
  
  
  
  
  
  
  
  
  
  \begin{comment}
    \paragraph{\Rethyactic magic}
  \end{comment}
  \gitem{\hr{Rethyax magic}{\Rethyactic magic}}
  
  
  
  
  
  
  
  
  
  
  \begin{comment}
    \paragraph{Rissitic magic}
  \end{comment}
  \gitem{\hr{Rissitic magic}{Rissitic magic}}
    The theory of the Body, Spirit and Shadow Worlds. 
  
  
  
  
  
  
  
  
  
  
  \begin{comment}
    \paragraph{Shamanistic magic}
  \end{comment}
  \gitem{\hs{Shamanistic magic}}
  
  
  
  
  
  
  
  
  
  
  \begin{comment}
    \paragraph{Vaimon magic}
  \end{comment}
  \gitem{\hs{Vaimon magic}}
    The theory of \hr{Iquin}{\iquin} and \hr{Itzach}{\itzach}.
 
\end{gloss}
 

























