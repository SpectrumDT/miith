
\chapter{The \Ophidian Peoples}
The \ophidian{} races include all the creatures descended from the old \ophidian{} people. 
The most powerful of these races are the \dragons, but the most widespread are the \scathae. 

This section describes the \ophidian{} races as well as political groups among them. 















\section{\Cregorr}
\target{Cregorr}
\index{\cregorr}
A \saurian{} humanoid race. 









\subsection{Physique}
A \cregorr{} looks like a medium-sized theropod (\latinname{Ceratosaurus} or \latinname{Megaraptor} or the like). 

In a sense, the \cregorrs{} are more sophisticated than \scathae, \nephilim{} and \humans{} because they were designed by the \ophidians{}  with all the superior science they possessed when they were at the top of their power. 
Among other things, \cregorrs{} can regenerate lost limbs. 








\subsection{Biology}
How long does a \cregorr{} live? 

\begin{enumerate}
  \item 
    They might live long because of their regenerative abilities and the way they were \quo{\hr{Origin of Cregorr}{intelligently designed}}. 
  
  \item 
    But they might also be short-lived because they live fast and \quo{burn out} quick. 
    After all, when you have a race of warrior servants, it is not important for them to have long lives. 
    It is more important that they mature and breed quickly. 
\end{enumerate}









\subsection{History}
The \cregorrs{} were \hr{Origin of Cregorr}{created by the \ophidians{} during the \firstbanewar}. 

After the \firstbanewar{} they escaped to \hr{Cregorr did not dominate}{live in the \wylde{} as barbarians}. 

Later they were \hr{Cregorr came to serve Dragons}{\quo{tamed} by the \dzraicchenosses{} and made to serve them}. 









\subsection{Politics}










\subsection{Psychology}





\subsubsection{Never Iquinian}
The \cregorrs{} are never \hs{Iquinian}. 
Their race is too old and bestial, their minds too simple and guile-less. 
Iquin, based on lies and asceticism and self-denial as it is, does not work on them. 
They do not understand it and cannot be made to believe in it. 

So in the Iquinian world, \cregorrs{} are mistrusted and hated. 
So they only live in the \wylde{} as barbarians. 





\subsubsection{Noble savages}
The \cregorrs{} have \xsic{} blood, so they have a primal fury inside them, \hr{Scatha fury}{just like the \scathae{} do}. 
But the \cregorrs{} are more liable to openly display their ferocity than are the \scathae. 

They are savage and barbaric, but with a certain \naivete{} and honesty. 
Their minds are not made to understand lies and intrigue, so they are honest and gullible. 
They are noble savages. 





\subsubsection{Religion}
\target{Cregorr worship XS}
Many \cregorrs{} worshipped the \xss, or the memory of them. 
They did not understand magic, but they understood their instincts, and somewhere inside them they had a craving towards some mighty beings which they felt must exist. 
They would seem to glimpse these gods in their \hs{dreams} or daydreams, or see hints of them in their own bodies and all around them. 
These beings were the \xss. 

The \cregorrs{} had simple minds, so they accepted their instincts at face value rather than try to cover them up in lies. 
So they worshipped these gods whom they saw in their dreams. 

They also believed that they themselves were descendants of these gods, in some distant past. 
They sort of felt the presence of their gods in their bodies. 
And this belief was true, in a sense. 

They understood little to nothing of sorcery and summoning, so they were rarely if ever able to actually commune with the \xss{} or cast any actual magic. 
But the veneration of the memory of the \xss{} triggered something instinctive inside them, which helped them access the well of Chaotic \xsic{} fury and power inside them. 
So their religious rituals (dances and orgies and whatnot) made them stronger. 

Their religion \emph{worked} for them, so they knew it must be real. 
So they kept on practicing it. 

One other factor worked in their favour: 
The simple \cregorr{} minds were almost unaffected by the \hr{Ophidians lose telepathy}{mind-rotting radiation that afflicted their \ophidian{} masters}. 















\section{\Dragon}
\target{Dragon}
\target{Dragons}
\index{\dragon}









\subsection{Name}
In the \Miith{} world, the word \quo{\dragon} is etymologically descended from \emph{\Draecchonosh{}} (plural \emph{\Draecchonosh{}}, which is a \draconic{} word of power with connotations of strength and ferocity. 









\subsection{Biology}
\index{technology!bio-technology}
\Dragons{} are descended from \naga{} lords who transformed themselves using bio-technology and magic, as described in section \ref{Origin of Dragons}. 

\Dragons{} reproduce by internal fertilization and lay eggs. They are very infertile. A mating only rarely causes impregnation. A female lays one to three eggs at a time (rarely four or five), but even under good conditions, only half of these eggs survive to hatch, and many of the young die early on. An average female can lay 10-15 eggs in her lifetime if she is sexually active thoughout her fertile life (of these, 2-4 young can be expected to survive). The female carries the eggs for slightly less than a year, and after she lays them they take about one and a half year to hatch. If a female is impregnated, she will know it a few (3-5) days after the mating (but even at this stage, miscarriages do occur).  
Psychologically \dragons{} have few or no sexual taboos, mating with whomever they choose. 

A \dragon{} is sexually mature at an age of about 200 years. 
\Dragons{} are immortal and can live forever until slain. 

\target{Dragons have three hearts}
\Dragons{} have three hearts. 





\subsubsection{Created from \ophidians}
The \dragons were originally created from \ophidians.
The first \dragons were born as \ophidians and later in life morphed into \dragons.
But in later ages of the world the technology to turn \ophidians into \dragons was lost.
Now new \dragons could only hatch from \dragon eggs.

See the section on \hr{Becoming a Dragon}{becoming a \dragon}.





\subsubsection{Demographics}
\target{Dragon demographics}
In the entire history of \Miith there has existed some thousands of \dragons. 

There were at most 2000 \dragons that went into \hs{Durance}. 
Of these, less than 1000 ever awoke. 
Some perished in Durance. 
(Or perhaps they achieved enlightenment and left their sleeping bodies. 
 And feasted on all the other sleeping souls in their tombs while they were at it.)
 
At the time of the \hs{Incursion}, there were about 1000 \dragons alive.
After the \hs{Shrouding}, there were 100. 
At the beginning of the \thirdbanewar, there were around 20 \dragons alive, plus up to 10 more \hr{Aloof Dragons}{aloof or \quo{dead but dreaming} ones}. 





\subsubsection{Diet}
\target{Draconic diet}
\Dragons{} are natural predators, used to eat anything that moves. But their eating habits differ from those of the \resphain{} (see section \ref{Resphan diet}). 

\Resphain{} are cannibals, eating \resphan{} meat alongside that of all other intelligent creatures they can catch. In contrast, \dragons, despite their chaotic nature and their tendency to war amongst themselves, have a universal taboo against eating the flesh of other \dragons{}, \ophidians{} or \rachyth. A cannibal \dragon{} would be seen as an abomination by his fellows, and would be hunted down and slain. 

Where \resphain{} love to eat enslaved creatures who willingly submit and let themselves eat, \dragons{} live for the thrill of the hunt. They want their prey to flee and fear for its life. When they catch their prey, they want to be able to taste its fear, pain and anger, for such feelings are manifestations of life itself, and \dragons{} eat life, whereas \resphain, in a sense, eat death. 





\subsubsection{Eggs and incubators}
\Draconian eggs were incubated in a special living machine. 
The incubator machine was a bloody monster that devoured living creatures body and soul in order to feed and strengthen the \draconian foetus. 
The machine looked like a grotesque, crooked tree or bush with long, thorn-clad metal branches. 

Compare to the sex machine in \cite{Anime:UrotsukidojiII}. 
Or the monster Mammut in \cite{StevenSavile:CurseoftheNecrarch}, composed of a metal skeleton and nine mutilated human bodies and souls.





\subsubsection{Living machines}
\target{Dragons are living machines}
In a sense, the \dragons were \hs{living machines} or cyborgs, designed by \Sethicus and other \ophidian scientist-sorcerers to be stronger and more resilient than any machine of metal. 





\subsubsection{Reproduction}
\target{No new Dragons were born}
\target{No new Dragons are born}
\Dragons were a species unto themselves. 
They could interbreed only with other \dragons, not \quiljaaran or \ophidians.
(But what about \Iurzmacul, father of \Ishnaruchaefir?)

\Dragons did not just give birth naturally like other creatures (even \resphain) did. 
For a pair of \dragons, to create an offspring was a long, very arduous and very arcane process. 
They must invoke the \xss and draw much power and life-energy from \KhothSell and from the Heart of \Miith, and deliberately and permanently invest much of their personal power in the child. 
Then the female would get pregnant and, later, lay an egg. 

In the age of the Shroud, this process was fraught with danger. 
The Heart was difficult to reach, so it took a lot of work and wasted energy to reach it and coax any new life out of it. 
It took more energy than usual, and the \draconic parents-to-be risked permanent bodily harm.
On top of that, it was uncertain. 
The egg might easily die without hatching, or the hatchling might be sickly and die young.
Most \dragons decided that the permanent investiture of personal power was not worth the risk, so they simply stopped having children.
After all, they were immortal, so they did not \emph{need} to replenish their race. 

The \dragons{} suffered greatly under the \hr{Heart weakened}{weakening of the Heart}. 
This is because \draconian{} souls were so powerful that it took a fucking lot of Heart energy to create one. 

Since the inception of the Shroud and the weakening of the Heart, almost no new \dragons were born.
Most \dragons alive in the \thirdbanewar were exceedingly old, being adult already when the first \resphain came to \Miith. 
In fact, \dragons were even longer-lived and slower-breeding than the \ophidians from which they were descended. 

But the \hr{Aloof Dragons}{aloof \dragons in \Machai} had \hr{Aloof Dragons do not care that no new Dragons are born}{stopped caring}.

It was sometimes said that the \dragons were a dying race.
But this was actually never true. 
Their number remained almost constant from the end of the \secondbanewar to the \thirdbanewar.
Remember, it took an assload of \resphain to take down just a single \dragon.






\subsubsection{Smelling the air}
\Dragons smell the air with their tongues just like snakes do. 
Most \ophidian-descended races do (but not \scathae or \cregorrs). 





\subsubsection{Starvation and undeath}
\target{Draconic starvation}
When \dragons hunger for energy, they starve in a more dramatic fashion \hr{Ophidian starvation}{than the \caisith do}. 
They become hateful and mute undead spectres, banshees. 
Over the course of thousands of years, these immortal spectres (of all \dragons who have ever starved (for spectres are difficult to destroy)) have merged into a hivemind, a loathsome god. 
Like the Forkrul Assail god in \cite{StevenErikson:TheCrippledGod}. 










\subsection{Culture}





\subsubsection{Aloof Elder \Dragons}
\target{Aloof Dragons}
Some \dragons have fled \Miith and dwell in \Machai. 
This is their \quo{second homeland}, because they draw so much of their power and their being from there.
So some are content to dwell in \Machai and concern themselves with \Machaic affairs and remain aloof from \Miithian affairs. 

Some had remained in their tombs, dead but dreaming\dash by choice or because they could not resurrect themselves. 
There lay lay, entombed with hordes of \ophidian retainers (who also lay dead but dreaming) and mortal servitors (often undead). 

But the aloof, retired \dragons still watched \Miith from the shadows.
They were still able to pay attention to what happened in the world around them to a degree, through their \daemons and \homunculi.


\citebandsong{Nile:AnnihilationoftheWicked}{Nile}{
  Von Unausspechlichen Kulten
}{
  I Hath Dreamed Black and Grim, Desolate Visions\\
  of the Pre-Human Serpent Folk \\
  and Communed with Long-dead Reptiles.\\
  Silently Watching Through the Ages in Cold, Curious Apathy.\\
  The Unending Sorrows and Suffering of an Abysmal \Human{}kind.
}

\target{Aloof Dragons do not care that no new Dragons are born}
Some of them knew that \hr{No new Dragons were born}{no new \dragons were born}.
But they had stopped caring. 
They were powerful and immortal and had the luxury to wait a few thousands or even tens of thousands of years between procreating. 
Many of them were over 20,000 years old and saw the war with the \resphain as a temporary nuisance. 

\Ishnaruchaefir and \Secherdamon were similarly old and would likely have thought the same, but the war came horribly close to them when some of their close family members (Nexagglachel and \hr{Ishnaruchaefir's sons die}{\Ishnaruchaefir's three sons}) were destroyed by the \resphain.

Later, in the \thirdbanewar, \Secherdamon or \Ishnaruchaefir had to convince the aloof ones that the war was more than a nuisance and that its conclusion would have great repercussions for all \dragons, even the ones who had fled to \Machai.

Late in the \thirdbanewar, Secherdamon or \Ishnaruchaefir finally \hr{Ishnaruchaefir leads Dragons to war in TBW}{manages to rouse} some of these aloof, retired \dragons.
Not all, but enough.

\target{Encountering Dragons in dreams}
It was possible to encounter the sleeping \dragons when \hs{dreaming}.
Like Cthulhu, they reached out to touch the minds of mortals.
And they could use mind control.





\subsubsection{Ancient homeland}
There exists \hr{Fallen Dragonland}{a place that once used to be the capitol and homeland of the proud \draconic{} empire, but now lies in ruins}.





\subsubsection{Architecture}
\target{Draconic architecture}
\index{architecture!\draconic}
The \dragons{} built huge, cyclopean edifices. 
Their buildings were often \quo{organic} and \quo{bestial} in shape, reminiscent of the carcasses of gargantuan animals and monsters. 
The bodies of the buildings were invariably vast, even bloated. 

They would decorate buildings with thin towers resembling spines, horns and antlers. 
These towers often served no useful purpose; they built them just because they could. 

An example is \hr{Nith'dornazsh}{\Nithdornazsh}. 

Contrast with \hr{QJ architecture}{\quiljaaran{} architecture} and \hr{Resphan architecture}{\resphan{} architecture}. 

\Draconian citadels were enormous. 
They sprouted huge, bulbous, misshapen spires.
They catered to the \xss and to the principles of Chaos. 
They were built in accordance with Chaotic occult geometry in order to more fully utilize the power of the \xss. 

The ancient \draconic{} buildings are built by \daemons{} under the \psp{\dragons} command. 

That is, if the buildings are not alive. The living buildings \emph{are} the \daemons.

See also the sections on \hr{Nyx}{\Nyx} and \hs{dark ancient cities}. 

\citeauthorbook{EdgarAllanPoe:ArthurGordonPym}{Edgar Allan Poe}{%
  The Narrative of Arthur Gordon Pym of Nantucket%
}{
  No more than a mile away through blasts of snow and wind, yet clearly visible, was the image of a titan tower\dash a lair of giants, \dragons, or some other fabulous and abhorrent creature, for it was far too great to have been erected by mere \human hands. 
  It soared above us, hidden beghind a swirling, freezing veil of ice and snow, taller than any medieval tower or citadel\ldots{} 
  There clung to this unholy edifice a hideous feeling of monstrousness, as if this were not something native to this earth, but an enormous, blasphemical Tower of Babel erected to mock God and all of His good works. 
}




\subsubsection{Dead gods}
\target{Elder Dragons}
\target{Elder Dragons worshipped}
The first great \dragonlords \hr{Elder Dragons die}{died at the end of the \firstbanewar}. 
But the heroic role they played in the war overshadowed their cruel tyranny. 
Today their corpses are \hr{Elder Dragons worshipped}{worshipped as dead gods}, close to the \xss{} in status. 

Kind of like how RL religions treat their \quo{prophets} with a veneration second only to that afforded the gods. 

\lyricsauthorbookpage{Graham McNeill}{False Gods}{313}{
  His father had become a carrion god who neither felt his subjects' pain nor cared for their fate. 
}

\lyricsauthorbookpage{Graham McNeill}{False Gods}{273}{
  I am Horus, forged of the Oldest Gods,\\
  I am he who gave way to Khaos.\\
  I am that great destroyer of all.\\
  I am he who did what seemed good to him,\\
  and set doom in the palace of my will.\\
  Mine is the fate of those who move along\\
  the serpentine path.
}

\Tiamat{} is invoked in Chaos magic. 
She was the one who made the original pacts with the \xss, so she must be invoked when one seeks aid from the \xss. 

\lyricsbs{Monolith Deathcult}{Den Ensomme Nordens Dronning}{
  Sleep, oh Majesty, surrounded by massive darkness. \\
  Her skin torn apart by sub-zero claws. \\
  Buried deep in thy ice dungeon. \\
  Sleep, oh Majesty, Lonely Queen of the North.
}





\subsubsection{Factions}
Remember to have fractions among the \dragons{}, and Sentinels in general. 
Their history must be as diverse and as bloody as that of the \resphain!

Speaking of which\ldots{} how much of the Sentinel organization is controlled by \dragons? There are also \rachyth, remember. And the \Baelzerach, they might be Sentinel-allied, too. And there are groups within the Sentinels who do not work for \dragons{} at all, but worship their own gods, and are just in because they want to get rid of the \banes{} and \resphain{} (this is a strong argument for any \scatha{} or \rachyth). 

And there are cults who seek to resurrect the \xzaishanns. Perhaps there are people with \xzaishannic{} blood. People, or descendants of people, who partook in \Tiamat-tachi's original ritual of summoning the \xzaishannic{} power. These people were meant as sacrifices, but a few of them survived and were imbued with \xzaishannic{} power. They fled and hid, but their descendats live on, as does their hatred for the \draecchonosh. 

And then there are the \ophidians{} and the \nagae{}, who may sometimes work with the Sentinels. 









\subsubsection{Etiquette}
\Dragons{} respect power. 
And they respect the confidence and assertiveness that stems from power. 
To get a \ps{\dragon}{} respect it is perfectly fine to brag and show off. 
You just have to have the inner game to back it up. 

Humility is not a virtue. 
It is seen as a sign of cowardice and weakness, and people who display it are looked down upon as insignificant prey and pawns. 

\hr{Dragon violence}{Physical violence} is a common part of \Draconic{} behaviour. 





\subsubsection{In art and mythology}
\target{Dragons in art}
\target{Myths about Dragons}
In mythology and art \dragons were described in many different ways and portrayed in many different forms. 
All their forms were reptilian and monstrous. 

According to many, \dragons were creatures of great horror and tremendous evil. 

Mortals had myths about \dragons. 

\citebandsong{Nile:Ithyphallic}{Nile}{
  What Can Be Safely Written
}{
  On the walls of lost cities\\
  And in the carvings of madmen\\
  Who have glimpsed him in their dreams\\
  Is his image delineated\\
  Within a tomb protected by great seals he lies in death\\
  Under the weight of the dark waters of the deep\\
  Yet he dreams still, and in his dreams continues to rule this world\\
  For his thoughts master the wills of lesser creatures
}

\quo{\Dragon wings} was a metaphor for ominous things. 

\citeauthorbook[p.328]{ClarkAshtonSmith:Xeethra}{Clark Ashton Smith}{Xeethra}{
  Then, in a latter autumn, it seemed that the stars looked disastrously upon Calyz.
  Murrain and blight and pestilence rode abroad as if on the wings of unseen \dragons.
}

See also the sections on \hr{Myths of vanquished monsters}{how \human heroes vanquished elder monsters} and on \hr{Iquinian myths about Dragons}{Iquinian myths about \dragons}. 





\subsubsection{Language}
\Dragons speak \hr{Draconic language}{\Draconic}. 

\target{True Draconic signifies emotion}
When \dragons felt the need to express great emotion, they would often switch to \TrueDraconic. 





\subsubsection{Music}
\target{Draconic music}
\index{music!\draconic}
\Draconic{} music is extremely complex and strange (both rhythmically, melodically and harmonically). 
You practically need to have a big, complex, \draconic{} brain in order to understand it. 
To mortal ears it often sounds like cacaphonous, chaotic noise. 





\subsubsection{Name}
\target{Draconic names}
All \dragons{} have an \quo{egg-name}. 
Many also have a taken name. 





\subsubsection{Non-cannibalism}
\target{Dragons do not eat Dragons}
\index{cannibalism!\Dragons}
The \dragons{} have one rock-solid and enforced tradition: 
You do not eat the flesh nor souls of other \dragons. 
You can fight and hurt them all you want, and killing the body is also OK. 
Even destroying souls is forgivable if for a good reason. 
But eating a fellow \dragon{} is an unforgivable atrocity. 

It is, however, acceptable to sacrifice a \ps{\dragon} soul to the \xss. 
\TyarithXserasshana{} \hr{Tiamat kills Hesod-Nerga}{did it to \HesodNerga}. 
\IrocasSecherdamon{} \hr{Secherdamon sacrifices Dragons}{did it to some of his rivals} during his rise to power. 









\subsubsection{Religion}
\target{Dragons worship dead gods}
The \dragons{} worshipped the corpses of dead gods. 
These included the \hr{Dead XS}{\xss that lay dead and dreaming}, but perhaps more interestingly it included the fallen \firstgendragons: 
\Sethicus, \Xserasshana and others. 

The younger \dragons originally \hr{Sethicus betrayed}{betrayed \Sethicus, rebelled against him and cast him down}. 
Later \Sethicus awoke and \hr{Sethicus dies}{was permanently killed} in the \firstbanewar, but his spirit survived in some form. 
Some \dragons would later realize that \Sethicus had been right all along and that they had been wrong to rebel against him. 
They began to worship him again, but now he was more aloof than ever. 
Some \dragons (such as \Ishnaruchaefir) were convinced that \Sethicus was still alive and could hear their calling, but he only rarely deigned to respond. 
From now on, the \dragons were a people forsaken by the gods they themselves had betrayed.
This made them bitter and cynical. 

When in deep emotion, \dragons would swear by the names of the \xss and the dead gods, including \Sethicus and \Tiamat. 

\ps{\Xserasshana} corpse now lay entombed in \hr{Baltherium}{\Baltherium}. 

The dead \dragons lived on in myth\dash and still did in the Age of the Shroud:

\citebandsong{Nile:Ithyphallic}{Nile}{
  What Can Be Safely Written
}{
  On the walls of lost cities\\
  And in the carvings of madmen\\
  Who have glimpsed him in their dreams\\
  Is his image delineated\\
  Within a tomb protected by great seals he lies in death\\
  Under the weight of the dark waters of the deep\\
  Yet he dreams still, and in his dreams continues to rule this world\\
  For his thoughts master the wills of lesser creatures
}

They also worship the \xss. 

\citebandsong{Nile:AnnihilationoftheWicked}{Nile}{
  Chapter of Obeisance Before Giving Breath to the Inert One in the Presence of the Crescent-Shaped Horns
}{
  Khensu Neter Hetef,\\
  who Possesseth Absolute Dominion \\
  over the Evil Spirits that Infest the Earth and Sky.\\
  He of the Silence of the Moon.\\
  Giver of Oracles. He that Must Forever Wax and Wane.\\
  Thou Art in Union with Thoth,\\
  the Excellent Tehuti of Truth and Time.\\
  Keeper of the Lunar Cycle,\\
  whose Hands are Able,\\
  whose Tongue is Mighty in Speech.\\
  Author of the Works of Knowledge.\\
  Writer of the Ancient Wisdom.\\
  Master of the Words of Power. 

  I am He Who Calleth Down Curses \\
  and Commandeth the Elements unto Darkness.\\
  I Hath Uttered the Hidden Words.\\
  Even unto the Divine Words which Art Written in the Book of Thoth.
}





\subsubsection{Rulership}
The \dragons{} are a chaotic people. 
Any central organization, such as monarchy, is unnatural and unstable for them. 
\hr{Tiamat}{\Tiamat} stayed in power as long as she did by brute force and effort, and her rule was never as absolute as she would have liked or later generations have made it sound. 





\subsubsection{Technology}
\target{Dragon living technology}
\Draconic technology focused much on \hr{Living machines}{living machines} and bio-technology. 
They used \hr{Symbiotes}{symbiotes} in combat. 
Read those sections. 

The \dragons are supposed to have a \quo{living} theme.
As opposed to the \resphain, who, \hr{Resphan dead technology}{with their metal/glass/crystal technology}, have a more \quo{dead}, \quo{artificial} theme. 

\hr{Dragons are living machines}{\Dragons were living machines}, remember. 

See also: 
\begin{enumerate}
  \item 
    The section about how \hr{Nexagglachel could not rebuild Ophidian civilization}{the awakened \dragons after the \firstbanewar were unable to rebuild their technological civilization}.
  \item 
    The section about how the \dragons \hr{Dragons repress technology}{repressed technology after their awakening}. 
\end{enumerate}






\subsubsection{Violence}
\target{Dragon violence}
Physical violence is a common part of \Draconic{} behaviour and social interaction. 
\Dragons{} will snap and lash out at one another. 
This is a sign of friendship, not animosity. 
Just staring coldly at each other is a sign of disrespect or hate. 

In the old days, before the \secondbanewar, one would often see \dragons{} fighting fiercely, even to the death. 

Non-lethal but bloody fights are a social custom among friends.
It is even expected. 
(Compare to how, among \humans, men tend to call each other rough names and mock-fight as a sign of friendship.)

\Dragons{} are quick to kill enemies and un-friends, and ferociously. 
There are many personal feuds, where enemies will try to \quo{gank} one another at sight. 
(Compare to multiplayer games like \cite{VideoGame:WorldofWarcraft}.)









\subsection{Physique}
A \ps{\dragon}{} natural form was that of a large reptilian quadruped with a long neck and tail. 
A \dragon{} walked on straight legs like a mammal, not with the legs spread outward like a lizard or crocodile. 
The tail was prehensile. 
A ridge of small spikes ran down the spine, from the head to the end of the tail. 

The forelegs and hindlegs were the same length. 
The hindclaws were larger and more bestial, whereas the foreclaws were more fine and dextrous, almost as fine as \human or \scatha hands. 

\Dragons{} have long, sharp teeth for biting and cutting, but also some flat teeth (at the back of the mouth) for chewing. 
They usually have a number of horns on the head, usually pointing backward. 
The shape and number of these horns varies a lot between individuals. 

\Dragons{} come in all \colours. 

\Dragons{} have good regenerative abilities. They heal wounds quickly even in combat and can heal wounds most creatures can't. They can't regrow limbs naturally, but with the aid of spells, a \dragon{} can reattach severed limbs or even regrow new ones. 





\subsubsection{Appearance}
\target{Draconic appearance}
With their immortality and Chaos-born power, \dragons{} radiate an impression of ancient, alien might. Their \ophidian{} eyes are cold, baleful and unfathomable. 

\lyricsbs{Steven Erikson}{Reaper's Gale}{
  \ldots{} we looked to the east, and there saw, rising vast and innumerable on the cloud-bound horizon, \dragons. Too large to comprehend, their heads above the Sun, their folded wings reaching down to cast a shadow that could swallow all of Drene. This was too much, too frightening [\ldots{}] for their dark eyes were upon us, an alien regard that drained the blood from our veins, the very iron from our swords and spears.
}

\citebandsong{KarlSanders:SaurianMeditation}{Karl Sanders}{
  Dreaming Through the Eyes of Serpents
}{
  After many trips with my son to the zoo, I noticed that crocodiles, monitors, snakes\dash{}and pretty much all the Reptiles\dash{}have this way of sometimes staying completely motionless when they want to with this deep, cold, timelessly penetrating glare. 
  It struck me as something like a trance state. It caused me to wonder what in the world would a snake meditate upon\ldots{}
}

\citeauthorbook[p.214]{HenryJVesterIII:TheResurrectionofKzadoolRa}{Henry J. Vester III}{
  The Resurrection of Kzadool-Ra
}{
  The being prokected an aura of incalculable age and wisdom, and of powers gained on worlds long lost in space and time.
}

\lyricsbible{Job 41:15--26}{
  [His] scales [are his] pride, shut up together [as with] a close seal. \\
  One is so near to another, that no air can come between them. \\
  They are joined one to another, they stick together, that they cannot be sundered. \\
  By his neesings a light doth shine, and his eyes [are] like the eyelids of the morning. \\
  Out of his mouth go burning lamps, [and] sparks of fire leap out. \\
  Out of his nostrils goeth smoke, as [out] of a seething pot or caldron. \\
  His breath kindleth coals, and a flame goeth out of his mouth. \\
  In his neck remaineth strength, and sorrow is turned into joy before him. \\
  The flakes of his flesh are joined together: they are firm in themselves; they cannot be moved. \\
  His heart is as firm as a stone; yea, as hard as a piece of the nether [millstone]. \\
  When he raiseth up himself, the mighty are afraid: by reason of breakings they purify themselves. \\
  The sword of him that layeth at him cannot hold: the spear, the dart, nor the habergeon. 
}

\citeauthorbook[p.360]{GrahamMcNeill:Mechanicum}{Graham McNeill}{Mechanicum}{
  Dalia's hands flew to her mouth and she cried out as she saw the Dragon's monstrous form.
  Ins hape it was half crawling beast, half loathsome bird, its scaled head immense and its tail twenty metres long.
  Its terrible winged body was covered with scales\ldots
  The light of devoured stars shone at its breast and malignant fire burned in its eyes.
}





\subsubsection{Armies of \dragons}
Only rarely has any \dragon been powerful and influential enough to command armies of \dragons. 
\Sethicus, \Nexagglachel and \Ishnaruchaefir were some of the only ones. 

\citeauthorbook[p.344]{ClarkAshtonSmith:TheDarkEidolon}{Clark Ashton Smith}{%
  The Dark Eidolon%
}{
  Yea, the undying worms of fire and darkness have come forth like an army at thy summons\ldots
}





\subsubsection{Metaphysical nature and feel}
\target{Dragons radiate life}
The \dragons{} are a force of Chaos: Destruction and creation alike. 
They are associated with light, fire, storm and lightning. 
They radiate life, energy and passion, but also fury and destructiveness. 
Like \hs{Nature}. 

This is in contrast to the \banes{} and \resphain, who are \hr{Bane parasitism}{dark and parasitic}. 

When mortals (or \resphain) look into the eyes of the great dragons they stare into the depths of unutterable cosmic evil, the blackest, most monstrous and alien abomination. 





\subsubsection{Morphing}
\target{Draconic morphing}
\Dragons were not fixed in shape.
They could change form quite freely, and they did.
They were twisting, morphing, formless horrors. 
They were born of Chaos, and Chaos filled them and gave them life. 
And they embodied Chaos. 

They often assumed \quo{classic} \draconian form: 
A giant reptilian quadruped with wings and a long neck and tail. 
But they could take on all sorts of other forms, including serpents or humanoids. 
They could also grow additional heads if they wanted to, thus taking on the form of a hydra. 

This is meant to help make my \dragons more grotesque and menacing and horrible. 





\subsubsection{Scales of metal}
\Draconic scales felt like metal. 
They were actually made of a super-durable quasi-organic metal-like material.
\hr{Dragons are living machines}{\Dragons were living machines}, remember. 





\subsubsection{Size}
\target{Dragon size}
A \dragon's length was approximately one quarter body, one quarter neck and one half tail.
Wingspan was about equal to total length.
Most \dragons were 12-20 metres long. 
The greatest \dragons could reach as much as 30 metres. 

A 25 metre \dragon weighed about as much as an \latinname{Allosaurus}. 





\subsubsection{Ward runes}
\index{ward rune}
When fighting in their true forms, \dragons{} do not wear \armour. 
That would not be economically feasible or practical for creatures of their size. 
But they do often wear \hs{ward runes}. 





\subsubsection{Weapons}
\target{Skekrathuin}
\index{\skekrathuin}
In battle, \dragons{} sometimes wield \skekrathuins{}\dash sword-like blades strapped to one's forearm. 
(Their hands are not made for holding weapons.)

\Dragons{} use their long, strong, prehensile tail a lot in combat. 
The tail might not be the \ps{\dragon} most powerful weapon, but it is certainly the swiftest and most dextrous. 

\target{Zrekklakh}
\index{\zrekklakh}
They often wear blades on their tails, called \zrekklakh{} (akin to \hr{Skekrathuin}{\skekrathuin}). 

They rarely wield firearms or other ranged weapons. 
They prefer to engage in \melee{} or use magic. 









\subsection{Psychology}
\Dragons{} are creatures of strong passions. They have the same emotions as \humans{} (love, hate, lust, pride\ldots{}), but on a larger scale. They are very individualistic creatures and, as a rule, consider their own needs and desires first and their fellows second.
\Dragons{} are often ruthless and arrogant, seeing humanoids as inferior, unworthy savages (though, of course, some \dragons{} are gentle and kind). 

\Dragons{} are proud and aggressive. They instinctively seek to dominate other creatures and do not welcome being given orders. This is the main reason why the \draconic{} kingdoms of Nom and Irokas have always been plagued by bloody wars (even more so than humanoid kingdoms): \Dragons{} are not genetically disposed to follow a King, so they tend to splinter and rebel. 

The \dragons{} have something of the \xs-born chaotic savagery in them, but they also have the cold, calculating \ophidian{} patience. 
Where \resphain{} and many mortals have a tendency to act cool and calm on the surface to hide the rage inside, a \dragon{} will often wear anger, aggression and physical violence as an outward persona, with the cold rational mind lurking beneath, observing and analyzing. 

\Dragons{} are not necessarily extremely evil just because they worship \xss. 
They just have a morality that is inhuman, superhuman and at times\ldots{} evil. 
One reason for this alien morality is their immortality and supreme power. 
It tends to erode your respect and regard for the lives of smaller, weaker, shorter-lived beings.




\subsubsection{Forgetfulness}
\target{Dragons have forgotten}
In the \hr{Ophidian golden age}{Golden Age of the \ophidian{} civilization},
the \dragons{} were even greater than they are today. 
But the \ophidian{} civilization \hr{Fall of the Ophidians}{fell}, and the surviving \ophidians{} fell into a long dark age. 
They forgot much of what they once knew, remembering only fragments of their tremendous science and magic.  

Many of the \daemons{} that once served as the \psp{\dragons}{} slaves have turned on them, and the \dragons{} have forgotten then spells to command them. 
Compare to the Azghouls of the RPG \emph{Kult}. 

All of \hr{Secherdamon's research}{\Secherdamon's research} is not just for discovering new things, but very much also for rediscovering the knowledge they once possessed. 

They wanted their golden age back.

\citeauthorbook[p.141]{RobertEHoward:TheCurseoftheGoldenSkull}{Robert E. Howard}{%
  The Curse of the Golden Skull%
}{
  Rotath's weird inhuman eyes smoldered with a terrible cold fire.
  A pageant of glory and splendor passed before his mind's eye.
  The acclaim of worshippers, the roar of silver trumpets, the whispering shadows of mighty and mystic temples where great wings swept unseen\dash then the intrigues, the onslaught of the invaders\dash death!
}





\subsubsection{Happiness in violence}
\target{Dragons find happiness in violence}
\target{Dragons love conflict}
The \dragons were able to find great ecstasy in violence and war and chaos.
This was a side-effect of the \xs taint in their blood.
But only true \dragons could experience this feeling.

\Caisith envied them this. 
That was one of the reasons why they tried so hard to become \dragons:
\Dragons could feel more happiness in this horrid world.

The \resphan \ketherain \hr{Dragons love conflict}{had only a pale illusion of this happiness}. 





\subsubsection{Madness}
Even \Sethicus and \Tiamat{} feared \hr{Sethicus maps the way to Machai}{the cosmic truths that they discovered}. 
And even in the age of the \thirdbanewar, the \dragons{} knew that they balanced on the edge of an abyss of \hr{Madness}{madness}. 

\lyricsbs{Limbonic Art}{The Yawning Abyss of Madness}{
  The yawning abyss of madness.\\
  A cryptic slaughter by hate.\\
  Darkness is the only survivor\\
  as evil dominion terminates.\\
  The yawning abyss of madness.
}

\lyricsbalsagoth{The Ghosts of Angkor Wat}{
  I have concluded that these
  perceived parallels and their possible significance carry me ever closer to
  the centre of this great global web, the strands of which I have been
  traversing in my long quest for enlightenment, and yet I now fear that the
  owner of this web has surely felt the tremblings I have caused along its
  delicate filaments, and may well feel compelled to investigate the
  disturbance\ldots{}
}

Some of them, such as \Ishnaruchaefir, met this fear with a \quo{Devil-may-care} daredevil attitude. 

The \psp{\dragons}{} use of \hr{XS}{\xsic} magic warps their minds and makes them more volatile, chaotic and emotional. 
They are more resistant to this than mortals because their minds are stronger, but on the other hand, the \dragons{} channel far more such magic than mortals do. 
Over the millennia this seeps deep into their minds and shapes their personalities. 
They live and breathe the \xs{} essence day and night. 





\subsubsection{Social intelligence}
\target{Dragons are not social}
\Dragons{} are fundamentally \emph{not} social creatures. 
They are fiercely independent and self-sufficient. 

This is probably the \psp{\dragons}{} greatest weakness against the \resphain, who because of their \hr{Resphain are social}{superior social skills} are better organized, whereas the \dragons{} tend to war and squabble amongst themselves. 
True, the \resphain{} war amongst themselves, too, but that is a direct consequence of the \hr{Curse}{\draconic{} blood in their veins}. 

It can also play to the \psp{\dragons}{} advantage, though. 
Unlike \resphain, \dragons{} are not vulnerable to social pressure and cannot be easily manipulated through taunts. 
A sneaky \dragon{} will understand the effectiveness of taunts against \resphain{} and use it against them. 

Because of their non-social nature, \dragonlords{} often keep \scathaese{} or \hr{QJ}{\quiljaaran} advisors and assistants to help them in social matters. 
\Secherdamon{} has \LocarPsyrex, and \Ishnaruchaefir{} has \Criseis. 

\target{Draconic sincerity}
\Dragons{} are liable to swing back and forth between brutally evil and genuinely compassionate and caring. 
They have strong emotions and tend to display them sincerely. 
Unlike the \resphain{}, who, being highly social creatures, are \hr{Resphan hypocrisy}{scheming and hypocritical}. 










\subsection{Skills and powers}





\subsubsection{Blood grants immortality}
\target{Dragon blood gives immortality}
It was known (in myth at least) that \draconian blood conveyed immortality.
Everyone knew that tidbit.

It was not necessarily the nice kind of immortality, though. 
It could be \hr{Psyrex's undeath}{undeath like \Psyrex's}.

The person who drank \draconian blood became addicted to it and must keep drinking it if he wanted to stay immortal. 

\Dragon blood worked best for \scathae, because \scathae were \hr{Scathae have potential for greatness}{designed with a potential for greatness in mind} (as \hr{Ortaican potential for greatness}{the \Ortaicans believed}). 

For \humans and others, drinking \draconian blood was very dangerous.
It had to be prepared with lots of special spells in order to be just drinkable.
Deriving any power from it required lots of more spells. 
An ordinary \human who drank \draconian blood would either die in horrible agony or mutate into a monster (and become mindless or raving mad in the process). 






\subsubsection{Dark knowledge}
\target{Dragons have dark knowledge}
\hr{Resphain and forbidden books}{Dark though the \pps{\resphain} books might be}, {the \dragons wielded even darker knowledge}. 

They were a truly ancient race.
The blood of the \xss was in their veins and they saw deep into the nature of the Cosmos.

Even \Criseis \hr{Criseis fears Dragons}{shuddered to think of it}. 







\subsubsection{Death and immortality}
\target{Draconic immortality}
\Dragons{} were \hs{True Immortals}. 
They were the \emph{first} True Immortals. 
They learned immortality though a pact with \hr{Khoth-Sell}{\KhothSell}, whom they worshipped as their goddess of death and immortality.

\hr{Sethican philosophy}{\Sethican philosophy} saw the \Draconic immortality as a great step forward for the spiritual development of the \ophidian race. 

\Draconian immortality was powerful. 
They have several \quo{levels of deadness}. 
A \dragon killed by normal means would retain quite some consciousness and even a minimum of sorcerous power. 
And even after its body was permanently destroyed and its magical power broken or eaten, the \dragon's consciousness might possibly still live on. 
It might have little power (at least in \Miith and in familiar Realms), but it could retain its memories and wisdom and would be free to journey to other worlds and build for itself a new life. 
This was no guarantee, however. 
It depended on the spiritual mastery of the individual \dragon (in accordance with \hr{Sethican philosophy}{\Sethican philosophy}). 

For example, \Sethicus \hr{Sethicus in the Threnody}{lived on in the Threnody} after his death. 
\Nexagglachel \hr{Nexagglachel lives on in Satharioth}{lived on inside the \satharioth}. 

Similarly, the \hr{Aloof Dragons}{\dragons dormant in Durance} were dead and entombed and could not rise on their own, but needed to be resurrected by external forces. 

\Draconic immortality was different from (and superior to) \hr{Ophidian immortality}{that of the \ophidians}, which was based on the shedding of skin. 

See also the section on \hr{Kinds of immortality}{different kinds of immortality}. 

\Draconic immortality was connected with \KhothSell, seen as the Ouroboros, the eternal serpent of immortality.

\citebandsong{KarlSanders:SaurianExorcisms}{Karl Sanders}{%
  Slavery Unto Nitokris%
}{%
  Become One with the Serpent, and the Serpent will Coil close about thee. 
  Cling to her bosom, and neither remorse nor sorrow nor despair for the weariness of existence shall trouble thee; for thou art beyond the reach of all these whilst wrapped in the spiritual narcotic of her Saurian Embrace.
}




\subsubsection{Durance \nexi}
Each \dragon in \hs{Durance} formed a powerful \nexus of energy. 
Many humanoid mages discovered these points of energy and built citadels and even cities there, unaware that their power flowed from a \dragon that lay entombed and sleeping beneath the earth. 





\subsubsection{Power compared to \resphain}
\target{Dragons vs Resphain in power}
In combat a \dragon is worth more than 10 \resphain. 
The average \dragon can take on at least 20 purebloods. 
But if the \resphain are \hr{Umbra power}{mounted on \umbrae} then it only takes an average of 6 of them.









\subsection{Becoming a \dragon}
\target{Becoming a Dragon}
\target{Shaeeroth ritual}
The \dragons{} were not born \dragons. 
\Ophidians became \dragons{} by imbibing \xsic{} \quo{blood} in a magical ritual. 
They would then die and be reborn as \dragons. 

Only the most powerful \ophidians{} could become \dragons. 

\citebandsong{DeathspellOmega:SiMonumentumRequiresCircumspice}{%
  Deathspell Omega
}{
  Blessed are the Dead Whiche Dye in the Lorde
}{
  Stare wide-eyed at this dense pitch boiling by the art divine\\
  Amniotic liquid of another kind\\
  That flesh and blood can not inherit the kingdom of God\\
  Behold the transformation, servant
  
  Like a malignant tumour and sudden growth of cancer divine\\
  A rebirth in putrefaction irreversible, \\
  corruption does not inherit uncorruption
}

\citebandsong{Nile:BlackSeedsofVengeance}{Nile}{
  The Black Flame
}{
  Open, For Me the Gates Shall Open\\
  Over the Fire of the Spirit, The Breath Drawn by the Gods. \\
  Arise Apophis Return, That I Might Return, \\
  Borne by the Flame Drawn by the Gods Who Clear the Way that I Might Pass.\\
  The Gods Which Sprang from the Drops of Blood \\
  which Dripped From the Phallus of Set\\
  That I might be Reborn\\
  For I am Khetti Satha Shemsu, Seneh Nekai\\
  And Will Become Set of a Million Years
  
  Akhu Amenti Hekau\\
  I shed My Burnt Skin and am Renewed
}

The process of becoming a \dragon involves lowering the walls of denial in one's mind and surrendering to the vast, cruel cosmos.
You have to realize that you are nothing next to the true forces of the universe.
Only then can you begin to acquire true power. 
Your ego and fears and wishes have to be broken down in order to be built up into something new and stronger.

\citebandsong{Nile:Ithyphallic}{Nile}{
  Language of the Shadows
}{
  Abandon hope\\
  And I shall become free\\
  And with freedom acquire emptiness

  With the mind cleansed and empty\\
  There is the void known as despair\\
  A gateway upon an emptiness endless and vast

  In despair the language of the shadows is intelligible\\
  In madness all sounds become articulate

  Terror and despair they guide me\\
  Into nightmares that follow one upon the other\\
  Like windblown grains of sand

  [solo: Dallas]

  I have become as the wastelands\\
  Of unending nothingness\\
  Now shall the night things\\
  Fill me with their whisperings\\
  And the shadows reveal their wisdom
}





\subsubsection{Price}
The early \dragons were fairly easy to create, back when \hr{Sethicus}{\Sethicus} was alive. 
Without \Sethicus and his gnosis it was much harder to create new \dragons.
Now they could only be created through terrible sacrifice and death and suffering. 















\section[Locul]{\Locul}
\target{Locul}
\index{\locul}
The \loculs{} were a race of \saurian{} humanoids.









\subsection{Physique}
\Loculs{} resembled \scathae{}, but smaller, smoother and less \armoured. 





\subsubsection{Weak}
\target{Locul weak}
\Loculs{} were not warriors.
They were weak and fearful. 
This made them \hr{Nephilim kill Locul}{easy prey for the marauding \nephilim}. 









\subsection{History}
The \loculs{} were created by the \ophidians{} as a servitor race.

After the \firstbanewar{} they continued to serve the \dragons{} and \quiljaaran. 

In the \hr{Aryothim kill QJ}{\aryoth-\quiljaaran{} wars} many \loculs{} were killed. 

Most of their \quiljaaran{} masters were killed, and the \loculs{} could not survive well on their own \hr{Locul slave mentality}{given their slave mentality}. 
Many more were killed by the \aryothim{} and \nephilim, who had an irrational hate of all reptilian races. 

Later the \dzraicchenosses{} took the \loculs{} and used them \hr{Origin of Scathae}{to breed the \scathae}. 
The surviving \loculs{} were displaced by the more aggressive \scathae{} who were now spreading all over \Miith. 

Eventually they died out, somewhere after the rise of the \dzraicchenosses{} and before the \secondbanewar. 





\subsubsection{\QuilJaaran{} miss them}
\target{QJ miss Locul}
After the \loculs{} had become extinct, many older \quiljaaran{} would mourn the loss. 
To the \nephilim{} who killed them, the \loculs{} were loathsome reptilian creeps.
But to the \quiljaaran{} they were gentle and lovable creatures, faithful and devoted. 
Much more lovable than the hard and fierce \scathae{} that replaced them. 

For millennia, some \quiljaaran{} elders wished to have the \loculs{} back. 
They were benevolent creatures that harmed no one. 
They were the children of a more enlightened age, according to the nostalgics. 









\subsection{Psychology}





\subsubsection{Friendly}
\target{Locul friendly}
The \loculs{} were friendly and trusting. 
Guile and aggression had been bred out of them. 
This made them \hr{Nephilim kill Locul}{easy prey for the marauding \nephilim}. 





\subsubsection{Slave mentality}
\target{Locul slave mentality}
The \loculs{} were bred as servitors. 
They were docile and had a slave mentality that made it difficult for them to act alone without a master. 















\section{\Nagae}
\target{Naga}
\target{Nagae}
\index{\naga{} (plural \nagae)}
\index{\nagalord{} (plural \nagalords{})}
\index{\naga{} (plural \nagae)!\nagalord{} (plural \nagalords{})}
\target{Vlekkesh'sala}
The \nagae{} are a race of sea-dwelling reptilian/ichthyic humanoids. 
They are long-lived (possibly immortal) and maintain their cities and kingdoms beneath the sea. They are the oldest humanoid species on \Miith{}. 

The more powerful of their race grow to huge size.
These \nagalords{} are called \nagalords{} in the language of Nag.









\subsection{Biology}
\Nagae{} live very long. A typical \naga{} has a lifespan of over 1000 years, and exceptional individuals may reach 2000 years or more. 





\subsubsection{Considered \demiscathae}
\target{Nagae considered Demiscathae}
Some thought of the \nagae as \hr{Demiscatha}{\demiscathae}.
Those who had actually seen and dealt with the \nagae knew better.
The \nagae were a far more ancient race than the \scathae. 
As old as the \dragons and older. 





\subsubsection{Habitat}
\Nagae preferred to live in the salt water of the oceans, but they sometimes came into rivers. 





\subsubsection{Immortality and shedding skin}
\target{Naga immortality}
\target{Nagae shed skin}
\Nagae were \hr{Lesser Immortality}{Lesser Immortals}
\Naga immortality worked similar to \hr{Ophidian immortality}{\ophidian immortality}

The \nagae shed their skin.
This was a part of their immortality. 

The Imetrians \hr{Imetric Naga skin clothes}{made clothes from shed \naga skin}. 





\subsubsection[Scatha/Naga hybrids]{\Scatha/\naga hybrids}
\target{Naga-Scatha hybrids}
\target{Scatha-Naga hybrids}
\target{Scatha-Naga hybrid}
\target{Naga hybrid}
In some coastal/island communities there dwell \scathae{} that interbreed with \nagae{}. 
The \nagae{} are more horrible, more primal kin to the \scathae, and the \scathae{} view them with fear, horror, loathing and awe. 

At times, children are born/hatched who are atavistic and look like monstrous \nagae. 
They are a kind of \hr{Demiscatha}{\demiscathae}. 

Half-\nagae walked more hunched-over than normal \scathae.
They had short legs and long tails. 
Their heads were strangely narrow and flat (in the vertical direction). 
Their bodies were flexible, and when they walked they seemed to writhe and wiggle in a repulsively fluid manner.
A particularly loathsome trait is their prehensile, snaking tails (alien to the \scathae, whose tails are rigid).
They were horrible to look at for a normal \scatha.

Compare to the Shake people in \cite[pp.360--361]{StevenErikson:ReapersGale}, or the people of Innsmouth in \authorbook{H.P. Lovecraft}{The Shadow Over Innsmouth}, or the people of Imboca in the movie \movie{Dagon}. 









\subsection{Culture}






\subsubsection{Magic}
\target{Naga magic}
\target{Spells in Nag}
Many water-related spells were adopted from the \nagae.
Their incations were in the tongue of \hr{Nag}{Nag}.
This gave \VizicarDurasRespina \hr{Vizicar fears Nag}{a fear of Nag}.






\subsubsection{Nag}
\target{Nag}
\index{Nag}
All \nagae{} in the waters near \Azmith{} belong to the kingdom of Nag and speak the language known as Nag. 

\Dragons{} can pronounce Nag flawlessly, and \scathae{} can learn it to some extent, but it is nearly unpronounceable to \humans{}. 
A few \nagae{} learn other tongues. 
A \naga{} encountered on the land is $20\%$ likely to know \CommonDraconic and $5\%$ likely to know Rissitic, Imetric or some ancient \scathaese{} tongue. 
A \naga{} is also $5\%$ likely to understand a bit of \Velcadian{} or Vaimon, but they can pronounce \human{} and Vaimon tongues only with great difficulty. 
(Those \nagae{} who do understand land-dweller tongues are usually the leaders and mages.) 






\subsubsection{Technology}
The \nagae relied mostly on biotechnology and sorcery (including biomancy/bio-magic) and not so much on metal and the like. 

\citeauthorbook[p.92--93 of 138]{KarlEdwardWagner:DarknessWeaves}{%
  Karl Edward Wagner%
}{%
  Darkness Weaves%
}{
  \ta{%
    In the eons before man walked the earth--when the sea was a vast, teeming wilderness of primitive life,
    its oceans far more immense than those of today--the race of creatures known to mankind as the
    Scylredi arose and flourished. Most of the continents we know today had not yet risen from the primeval
    sea, and only a few jungle-choked land masses stood out from the boundless seas of Elder Earth. The
    Scylredi lived beneath this ancient sea and created for themselves a civilization beyond man's wildest
    conception. Here in this very region they built their cities, for at that time all these islands lay upon the
    ocean floor.
    "They were a strange race, these creatures of awesome antiquity. Nothing on earth truly resembled
    them, even then. Were they some freak of evolution, a race from another world--or perhaps, like man,
    the result of some insane god's whimsy? Who can say at this distant age? The most ancient writings that I
    have studied are uncertain on so many points. But then, this earth has held many strange races about
    which mankind can only speculate, and all but a fragment of the secrets of prehuman history has been lost
    forever.
    
    Whatever their origin, the Scylredi were as gods themselves. They had control of powers both natural
    and supernatural. They used the great beasts of the primordial sea for their own purposes, controlling
    fantastic monsters known to mankind only through legend. With their knowledge of the physical sciences,
    they built great submarine seacraft--unearthly engines in which they \traveled the oceans and waged war
    with the other inhuman races of Elder Earth. That age was a far more violent world than the earth of our
    day, and there were many powerful forces the prehuman races must constantly contend against in the
    battle to survive. They were versed in the elder sorceries, as well--the secrets of the gulfs beyond our
    stars--and legend only hints at some of the hideous deeds that were committed by the Scylredi in their
    wars.
    
    Magnificent fortresses they raised--huge basalt structures that surpassed human imagination. The ruins
    of these great castles can be seen today--on hillsides where they have crumbled for millennia, ever since
    the waters receded from these islands. This very fortress, Dan-Legeh, is their creation. For the Scylredi,
    it is only a minor citadel, and built after their race had declined. It was an age of giants, and the Scylredi
    commanded both sorcery and science in their constant battle for supremacy in that prehistoric age of
    chaos.
    
    But as the centuries passed, their power slipped from them. Perhaps it was the shrinking of the great
    seas, or the cooling of the earth that caused their decline. It is recorded that there was a long period of
    horrific warfare between the Scylredi and some other race of elder beings. The conflict was waged with
    weapons of unimaginable power. Many of their colossal basalt castles were blasted into fused rubble,
    their gigantic seacraft destroyed, their fearsome servants annihilated, and the greater part of the Scylredi
    were killed. Both races lay near to extinction upon the termination of that war, and the scattered survivors
    were left to mourn amidst the ruins of their vanished civilizations.}
}








\subsection{History}
The \nagae{} are related to the \hr{Ophidians}{\ophidians}. 
In the very early beginning, the \ophidians{} and \nagae{} were one species. 
But eventually they grew apart. 

The \ophidians{} were reptiles, descended from amphibians who in turn had evolved from fish. 
One branch of the proto-\ophidians{} took to the sea. 
The ones who returned to the sea eventually degenerated into atavistic, fish-like forms. 
They came to resemble eels more than snakes. 
The sea-dwelling ones took up the worship of the \Krakens{} and became the \nagae{}.  

They thrived, and even survived the coming \firstbanewar{} relatively unscathed. 
But they stayed in the seas, and their technology remained comparatively low.








\subsection{Name}
Singular \emph{\naga{}}, plural \emph{\nagae{}}.%
\footnote{%
  The form \quo{\nagae} is not etymological justified, since \quo{\naga} is not Latin or Greek but from some Indian language, I believe. But this is the declination I use, because Latin grammar is cool.} 

The associated adjective is \emph{\naga{}}. 

\emph{Nag} one of a number of \naga{} nations. 
Its language is also \emph{Nag} and the adjective is \emph{Nagan}. 

Interesting tidbit: 
According to \DIBiggestSecret, the word \quo{Naga} means, in some Indian language, \quo{those who do not walk, but creep}. 

The \nagae were also called \quo{\ichthyans}. 





\subsubsection{Those who do not walk, but creep}
Interesting tidbit: 
According to \DIBiggestSecret, the word \quo{Naga} means, in some Indian language, \quo{those who do not walk, but creep}. 









\subsection{Physique and metaphysique}
A \naga{} is only vaguely humanoid, with an elongated snake- or eel-like body with a long tail and four limbs. 
They have what resembles a mix of fish and reptile characteristics. 

\Nagae{} are found in all kinds of \colours and patterns, but in Nag, various shades of green are most common. 

A typical \naga{} is 1.5 to 2 meters long and weighs 40-80 kg, but some are much larger, growing as long as 5 meters and weighing up to a ton. 
Their arms and legs are short. 
They can walk on land, but they are not very agile. 
In the water, on the other hand, they are very fast and \manoeuvrable, swimming with their legs and tail (rather like a seal). 

\Nagae{} can breathe both water and air, but they need to immerse themselves in water regularly. 
If they are kept out of the water for much more than 24 hours, they will dehydrate, weaken and die. 

\Nagae{} can regenerate lost limbs, but slowly. 
More slowly than \meccaran{} regeneration. 
An arm or leg takes about a year to regenerate. 

\index{technology!\naga}
\Nagae{} fight with weapons. 
Most of their weapons are daggers, spears and javelins. 
\Naga{} technology is low, so many of their weapons are primitive, made of stone or bone. 
The \nagae{} cannot forge iron or bronze, but some of their weapons are made from exotic metals unknown on the land. 
Occasionally, a \naga{} may be encountered wielding a weapon made by land-dwellers. 
(Of course, iron weapons will rust underwater, but \dragonsteel, \truesilver{} and certain enchanted weapons are immune to rust.) 

About \naga{} out of every twenty is a mage. 
\Naga{} magic is alien and very different from most land-dweller magic, but bears similary to \draconic{} magic. 





\subsubsection{Power over water and ice}
The \nagae{} wield great power over water and ice. They live near the poles in the summer and migrate to the equator in the winter. 

Some of the greatest \nagalords{} have frozen themselves into thrones of ice, where they lie asleep for thousands of years at a time. 





\subsubsection{A spear of ice}
Have a \naga{} who wields a spear of ice. 








\subsection{Psychology}







\subsection{Politics}
The \nagae{} work against the Cabal and Sentinels. They want to prevent either faction from controlling the Heart of \Miith{}. 

What is their end goal? Is it good or evil? 

But on the whole, they don't do much, \hr{Ophidians today}{like the \ophidians}. 

The \nagae{} use a \dweomer{} slightly different from that of the \dragons. 
Some fear that if the \banes{} conquer the \dragons, they will come after the \nagae{} and their \dweomer{} next. 
Others don't believe it, or believe that the \banes{} will never succeed. 
So they won't help in the war against them. 





\subsubsection{Coral reefs}
The \nagae{} know of the \hs{coral reefs} and their intelligence and power, and they fear them. 
Perhaps they worship them as gods. 





\subsubsection{Myths about them}
\target{Myths about Nagae}
There were myths about \quo{merfolk}: 
\Scatha-like creatures that dwelt in the sea and rivers. 
These were based on stories about sightings of \nagae. 

















\section{\Ophidians}
\target{Ophidian}
\target{Ophidians}
\target{QJ}
\index{\ophidian}
The \ophidians{} were an ancient immortal race of reptillian humanoids native to \Miith. 
They are the ancestors of \dragons, \quiljaaran{} and \nagae. 

Compare them to the Jaghut and the Forkrul Assail from \cite{StevenEriksonIanCameronEsslemont:MalazanBookoftheFallen}.

The \ophidians{} are an ancient race, having existed for many millions of years. 
They have seen the rise and fall of many civilizations of lesser beings (most of which destroyed themselves out of folly). 

They fulfilled a role as guardians of \Miith{}, and sometimes rulers. Occasionally, \ophidians{} would enthrone themselves as lords of the \nephilim{} or other lesser creatures\dash at times with evil intent. They may be something like the Jaghut in the \emph{Malazan Book of the Fallen} books. 

\target{Ophidian power source}
The \ophidians{} wielded\dash and wield\dash powerful magic. 
This magic is not of Chaos, but born of the native, natural power of \Miith{} and her Heart. 
It is similar to the \Wylde{} power that the \hr{Druids}{druids} use. 
The \hr{Kezeradi Iquin}{old incarnation of \iquin} used by the \hr{Kezerad}{\Kezeradi} was a modified, idealized version of this \dweomer{}.

In times of great need, such as when facing the \xzaishann, the \ophidians{} could also ask for help from a pantheon of aloof, mysterious \hs{cosmic gods}. 

The \dragons, \quiljaaran{} and \nagae{} are all descended from the \ophidians.
Those who have retained the old-style \ophidian{} are now called \quo{\trueophidians}. 

\lyricsbalsagoth{
  Into the Silent Chambers of the Sapphirean Throne (Sagas from the Antediluvian Scrolls)
}{
  Winged dragon coiled in thrice,\\
  bane of flame in shadowed ice.\\
  Flooded by the bloated Moon,\\
  the ivory worm now sleeps entombed.
}

They are intrinsically bound to \Miith{}, its life and future. Their blood is Life itself. And their venom is Death. 

Have some philosophy about how one bodily fluid from the \ophidians{} gives life while another takes life away. 









\subsection{Biology}
The \ophidians{} developed psionics and telekinesis before they developed hands. 
They had relatively advanced magic and psionics while they were still at the Stone Age stage with respect to tools. 
They learned immortality when they were at the Iron Age stage. 





\subsubsection{Castes: Imperials and Worms}
\target{Ophidian castes}
\target{Imperial Ophidian}
\target{Worm Ophidian}
\target{Ophidian Imperial}
\target{Ophidian Worm}
\target{Ophidian Worms}
The \ophidian race was divided into two castes: 
Imperials and Worms.

Imperial \ophidians were the great sorcerers, philosophers and rulers of their kind.
They were taller and looked regal.
All Imperials knew powerful sorcery.

The Worms were the lower class. 
They were smaller and looked loathsome and repellent.
Some Worms knew sorcery, but not all. 
Compare them to the monsters in \cite{RobertEHoward:TheShadowKingdom} and \cite{RobertEHoward:WormsoftheEarth}. 

Imperials made up 20--25 \% of the \ophidian population.

The two castes could interbreed.

Many times in \ophidian history did the Worm caste rise up in rebellion against the Imperials.
More than once was this system abolished (sometimes through the genocide of the entire aristocracy).
And each time the system somehow came back, with a new division of Imperials and Worms. 
The names might change, but the essence of the system remained.
Perhaps it was the natural equilibrium towards which \ophidian psychology naturally tended.





\subsubsection{Demographics}
At the time of the \thirdbanewar there were 1000-3000 \ophidians worldwide. 
\hr{Dark Crescent QJ}{Some of them worked for \Secherdamon}.





\subsubsection{Immortality}
\target{Ophidian immortality}
The \ophidian \hr{Imperial Ophidian}{Imperials} learned to focus \nexus essence inward to strengthen and heal their bodies, achieving long lives and even immortality. 
They shed their skin periodically to renew their youth and rebirth themselves into new life. 

The shed skin of an \ophidian{} contained some valuable essence. 
Usually the \ophidians would save the skins and distill the essence from them. 
However, this was a difficult process that required a laboratory. 
Occasionally the skins would be lost or stolen, and others would gain access to the essence stored in the skins. 

A regular \ophidian, not born to paragon-hood, can also heal his body, but his soul is not strong enough to do this effectively, so it requires a \emph{lot} of energy, and it gets more and more expensive as he grows older and accumulates \hs{decrepity}. 
An Imperial is stronger and more resilient by nature and is much more resistant to decrepity. 

\hr{Draconic immortality}{\Draconic{} immortality} is different from this. 

See also the section on \hr{Kinds of immortality}{different kinds of immortality}. 

Perhaps their immortality is psionic and not \hr{sorcery}{sorcerous} in nature. 





\subsubsection{Metabolism control}
See the section about how the \ophidians \hr{metabolism control}{evolved metabolism control} and strategic \hs{torpor}. 





\subsubsection{Parentage: \Voyagers and \xss}
\target{Ophidian parentage}
\target{Ophidians related to XS}
The \ophidians{} descend from both the \voyagers{} and the \xss, and as such represent a powerful combination of the two forces. 
Perhaps this is connected to \hs{sexual mysticism}, with \xsic{} \Chaos{} as their \quo{mother} and the \voyagers{} as their \quo{father}. 
Anyway, this combination of powers is what makes the \ophidians{} the greatest and mightiest race on \Miith{}, able to dominate the planet for millions of years. 

Even before \Sethicus, the \ophidians sort of knew they had \xs genes in them. 
They had myths and superstitions and taboos about their inner darkness. 
But they did not use it.
They \hr{Ophidian philosophy}{relied on the natural sciences} rather than mysticism. 

It was \Sethicus who \hr{Sethicus discovers Ophidian origin}{really understood the connection}. 





\subsubsection{Relationship with \XzaiShanns}
\target{Ophidians and XS}
The \ophidians{} are descended from \voyager-spawned beings, but they also had some amount of \xsic{} heritage in their blood. 
That was part of the reason why they became intelligent and grew to dominate \Miith{}: 
They had not only the \voyager-bred docility but also the \xsic{} aggression and evolutionary volatility. 

The \ophidians{} knew about the \xss{} and feared and shunned them. 
They also knew about their own \xsic{} blood, and they were not happy. 
To them, the \xss{} were not only outer \daemons, but also inner \daemons{} which they struggled against and fled from. 
They fought their inner Chaos using logic, self-control, meditation, philosophy and knowledge. 

Compare to the conflict between Chaos and Darkness in \cite{StevenEriksonIanCameronEsslemont:MalazanBookoftheFallen}. 





\subsubsection{Starvation}
\target{Ophidian starvation}
\target{Caisith starvation}
When a \caisith starves, he does not die.
He degenerates.
He loses his mind and slowly mutates into a mindless, slimy, crawling salamander-like lizard. 
It is considered a fate worse than death.
The madness and dementia are irreversible. 
\Caisith commonly kill themselves before this dementia sets in. 

\Dragons \hr{Draconic starvation}{starve in a more dramatic fashion}.









\subsection{Name}
\target{Names for Ophidians}
Their own name for themselves is \caisith (the name is both singular and plural). 
\quo{\Ophidian} is a name given to them by the mammalian races because of how they emulate snakes. 
Other names are Serpentines, Serpentine People, Serpent Folk or Serpent Men. 





\subsubsection{\QuilJaar}
Singular \emph{\quiljaar{}}, plural \emph{\quiljaaran{}}, adjective \emph{\quiljaaran{}}. 









\subsection{Culture}





\subsubsection{Architecture}
\target{Ophidian architecture}
\target{QJ architecture}
\index{architecture!\quiljaaran}
The \ophidians{} were fond of monumental architecture and often built enormous Cyclopean buildings and cities.
They engaged in great luxury, just because they could. 
Especially in their \hr{Ophidian Golden Age}{golden age}. 

In later ages (after the \firstbanewar), the \ophidians tended to build squat, functional and geometric houses and holds. 
Round, bulbous things; squares and rectangles, standing cylindical towers and semicylindrical longhouses. 
Always nice geometric forms when possible (sometimes in accordance with \hs{occult geometry}). 

Contrast with \hr{Draconic architecture}{\draconic{} architecture} and \hr{Resphan architecture}{\resphan{} architecture}. 





\subsubsection{Association with snakes}
\target{Ophidians and snakes}
Despite the name, and despite looking somewhat like snakes with legs, the \ophidians{} were not snakes, nor even closely related to snakes. 
But their culture idolized snakes. 
Snakes were seen as epitomizing virtues of patience, stealth and contemplative wisdom. 

Compare to how \humans{} in RL idolize lions for courage and owls for wisdom, even though this may be unfounded superstition. 

At any rate, snakes played a large role in \ophidian{} culture, and much of their aesthetics was patterned after a serpentine motif. 
Hence they conveyed a snake-like impression to other creatures, and hence the name \quo{\ophidian} survived. 
(They had other names for themselves, of course.)





\subsubsection{Basals, paracletes and hierarchs}
\target{Basal}
\target{Paraclete}
\target{Hierarch}
\index{basal}%
\index{paraclete}%
\index{hierarch}%
The \caisith had the unusual ability to share psychic energy. 
A band, tribe or nation of \caisith could pool their energy and form a \nexus. 
Some individuals would then take on the role of leaders (\apexes) and direct the energy as mages, while the rest supported the leaders by feeding them energy. 
Gradually a caste-like system emerged where different individuals were, from birth, better suited to one role than the other. 
The \apexes became nobles and the vassals became commoners. 

The lowest tier of \caisith were called \emph{basals}.
The second tier were called \emph{paracletes} and the upper tiers were called \emph{hierarchs}. 
(There could be multiple tiers of hierarchs above each another\dash e.g. a king supported by lords supported by knights.)

As a rough guideline, each \caisith liege needed 4-5 vassals in order to be effective.
So the population was made up of about 80\% basals, 16\% paracletes and 4\% hierarchs. 





\subsubsection{Dreadnoughts}
\target{Dreadnought}
\target{Dreadnoughts}
The Dreadnoughts were great war machines in the shape of giant reptiles.
Biomechanical constructs. 
They were used in the \caisiths's great wars.
Later they were \hr{Dragons use Dreadnoughts}{used by the \dragons}. 

To create these creatures, the \ophidians{} took inspiration (and at times actual genes) from various fierce creatures, including \nycans, \corgoroth{} and \vreiiden{} (many of which had themselves originally been designed and shaped by \ophidians). 





\subsubsection{Epitomes}
\target{Epitome}
\target{Epitomes}
\index{epitome}%
The \emph{epitomes} were the gods of the \caisith.
Each epitome was a non-sentient entity within the Ouroboros \nexus\dash a giant \daemon. 
The ignorant masses believed that the epitomes were sentient gods.
The more learned knew that the epitomes were non-sentient entities that could be invoked. 

Many epitomes were originally based on a living \caisith. 
These heroes were idolized in their lifetime, becoming powerful \hr{Hierarch}{hierarchs}. 
After death they were still idolized, and all this psychic energy flowing from \hr{Basal}{basals} and \hr{Paraclete}{paracletes} coalesced in the spiritual world and turned into great \daemons that could grant \quo{prayers} and effect magical \quo{miracles}. 





\subsubsection{Languages}
Tow of the great \caisith languages are \hs{Kush} and \hs{Saphyr}. 





\subsubsection{Nations}
In the later ages the \quiljaaran{} race became split into a number of \quo{nations}. 
Each nation had its own characteristic language, culture, facial/bodily traits and \colours. 

The nations included:
\begin{gloss}
  \gitemlink{Kush}

  \gitem{\KyanHweDin:} 
  The most populous nation (today, at least). 
  Their language sounds kind of like Chinese. 
  
  \gitem{\Okiriru:}
  They are dark gray in \colour and have glasses-like markings on the skin folds on their necks (like certain cobras). 
  Their language sounds like of like Japanese. 
\end{gloss}





\subsubsection{Riding mounts}
\target{QJ often ride}
\Ophidians often rode on mounts. 
They could only move fairly slowly on their own. 

Their favourite mounts included \mezolisks and \lindworms. 





\subsubsection{Science and philosophy}
\target{QJ philosophers}
\index{technology!\quiljaaran}
The \quiljaaran{} cared about philosophy and science for its own sake. 
Unlike the \aryothim, who \hr{Aryoth inventors}{cared more about technology}. 

\citeauthorbook[p.306]{ClarkAshtonSmith:TheSevenGeases}{Clark Ashton Smith}{%
  The Seven Geases%
}{
  \ldots and he came anon to the spacious caverns in which the serpent-men were busying themselves with a multitude of tasks.
  They walked lithely and sinuously erect on pre-mammalian members, their pied and hairless bodies bending with great suppleness.
  There was a loud and constant hissing of formulae as they went to and fro.
  Some were smelting the black nether ores; some were blowing molten obsidian into forms of flask and urn; some were measuring chemicals; others were decanting strange liquids and curious colloids.
}

\citeauthorbook[p.369]{ClarkAshtonSmith:TheFlowerWomen}{Clark Ashton Smith}{%
  The Flower-Women%
}{
  He crept forth very cautiously, and found himself in an immense, gloomily vaulted hall, whose windows were like the mouths of a deep cavern.
  The place was a sort of alchemy,a den of alien sorceries and abhorrent pharmaceutics.
  Everywhere, in the gloom, there were vats, cupels, furnaces, alembics and matrasses of unhuman form, bulking and towering colossally to the pigmy eyes of Maal Dweb.
  Close at hand, a monstrous cauldron fumed like a crater of black metal, its curving sides ascending far above the magician's head. 
  None of the Ispazars [lizard-men] was in sight; but, knowing that they might return at any moment, he hastened to make ready against them, feeling, for the first time in many years, the thrill of peril and expectation. 
}





\subsubsection{Seamanship}
In later ages, the \quiljaaran{} \hr{QJ seamanship}{were generally poor sailors}. 





\subsubsection{Slithering Ones}
The Slithering Ones were \caisith who had chosen to relinquish their humanoid form and take the forms of snakes.
They lived as ascetic hermits and philosophers and were revered for their wisdom. 





\subsubsection{Technology}
\target{QJ weapons}
In later ages, weapons and tools designed by the \quiljaaran{} were often based on magic from the ground up: 
Pistols that fired lightning bolts, etc. 
Unlike \hr{Aryoth weapons}{\aryoth{} weapons}, which were physical and enhanced with magic. 

\target{QJ symbiotes}
The \quiljaaran did not rely much on metal items, and they rarely wore clothes.
They used \hr{Symbiote}{symbiotes} instead. 
The \quiljaaran \hr{QJ morphing}{changed shape} to some extent, and they wanted their symbiotes to change with them. 





\subsubsection{\UzulKaya}
\target{Uzul-Kaya}
The \UzulKaya were an order of mystics among the \ophidians that conducted research into the Aenigmata of the \xss. 
They left behind a shrine in the city beneath \Yormis (later simply called the Shrine of the \UzulKaya). 









\subsection{History}

The \ophidians{} \hr{Ophidians evolved}{evolved from simple reptiles}. 
They evolved cunning and psionic abilities. 

They \hr{Early Ophidian culture}{quickly developed intelligence and culture}. 

They \hr{Ophidians as underdogs}{waged wars against the elder races}. 

They \hr{Ophidian civilization}{developed a great and mighty civilization}. 





\subsubsection{\QuilJaaran supremacy}
The \quiljaaran \hr{QJ supremacy}{rose to prominence after the \firstbanewar}, but lost it again. 





\subsubsection{Tried to rebuild}
\target{Noggyaleth plague Ophidians}
Throughout the ages, the \ophidians tried to rebuild their civilization, but always the \hr{Noggyal}{\noggyaleth} would ruin it.
Over the millennia the \ophidians lost many of their records to \noggyal attacks and other conflicts, and it was hard for them to replenish or replace them.
Their numbers were dwindling and there were few left who remembered the old \ophidian civilization. 

Mostly, the \ophidians remained underground in small groups.
There they worked on long-term plans and conducted research into science and sorcery. 
They still did at the time of the \thirdbanewar. 

Most \ophidians were affiliated with the Sentinels, but many had goals of their own as well.

The \ophidians learned the art of shapeshifting even before the Shroud came. 
They used it to infiltrate many things, including the \aryothim and possibly even the \resphain. 
This was a great asset for the Sentinels. 
An \ophidian was a much smaller \vertex than a \dragon, so it could more easily hide.
This would be impossible for a \dragon in the long run (even a master of disguise such as \Nzessuacrith). 

An \ophidian could masquerade as a \bezed. 
This was not widely advertised among the \resphain.
Few such spies existed, and much fewer had ever been caught. 
The \resphan leaders were afraid of the spies, but they were not willing to admit that they existed.
They did not want to cause a widespread panic (like what happened in Battlestar Galactica).





\subsubsection{\Saphyrae}
\hr{Saphyrae}{\Saphyrae} was the closest the \ophidians came to a new empire. 





\subsubsection{Shapeshifting and infiltration}
\target{Ophidian morphing}
\target{Ophidians infiltrate Cabal}
The \ophidians learned the art of shapeshifting even before the Shroud came. 
(The \dragons \hr{Draconic morphing}{developed this ability further}.)
They used it to infiltrate many things, including the \aryothim and even the Cabal. 
This was a great asset for the Sentinels. 
An \ophidian was a much smaller \vertex than a \dragon, so it could more easily hide.
This would be impossible for a \dragon in the long run (even a master of disguise such as \Nzessuacrith). 

An \ophidian could masquerade as a \bezed. 
This was not widely advertised among the \resphain.
Few such spies existed, and much fewer had ever been caught. 
The \resphan leaders were afraid of the spies, but they were not willing to admit that they existed.
They did not want to cause a widespread panic (like what happened in \cite{TV:BattlestarGalactica}).





\subsubsection{Degenerate relics}
\target{Degenerate Ophidians}
Some of the \ophidian \hr{Ophidian Worms}{Worms} of later ages had fallen from grace.
They were repellent relics of an older, darker, inhuman age. 
They were the degenerate, crippled, creeping remnants of a once proud and glorious civilization. 
Compare them to the half-humans of \cite{RobertEHoward:WormsoftheEarth}. 
But they were still terribly powerful.
They still had the power of \hr{Ophidian hypnotism}{hypnotism}. 

Many peoples had legends of the abhorrent serpentine people that dwelt in caves underneath the earth in the \wylde, said to be degenerate remnants of a race older than men or \scathae. 
Especially dreaded were their rare and terrible sorcerer-kings (Imperials).









\subsection{Physique}
\Ophidians were snake-like humanoids. 
They had two arms but no legs. 
Instead they slithered on a long snaky tail. 

They were taller and bigger than \humans{} and \scathae, but they were not so strong or fast. 

They could only creep slowly on their snake tails. 
They very often \hr{QJ often ride}{rode some kind of mount} to increase their mobility. 

But they had \hs{psionic} powers! 

\Ophidians looked not just serpent-like, but also \dragon-like. 
Their heads were much like those of \dragons (but without the horns). 
They had bony ridges on the top of the head and down along the spine and tail. 

They can extend the skin on their necks, like cobras. 





\subsubsection{Morphing}
\target{QJ morphing}
The \quiljaaran could morph their bodies \hr{Draconic morphing}{like \dragons}. 
Ergo \hr{QJ symbiotes}{they used symbiotes}. 

See also the section on \hr{Ophidian morphing}{\ophidian morphing}. 





\subsubsection{Undeath}
\target{Undead Ophidian}
\target{Undead Ophidians}
Some \ophidians had lain in \hs{Durance} and were revived.
They did not weather the Durance as well as their \draconian masters had done, for their \ophidian bodies were weaker than \draconian bodies.
They could not stand a million years entombed. 
So when they were woken, it was not to true life they returned.
Instead they rose to a horrific unlife as undead mummies. 
Their bodies were now weak and frail and decrepit. 
Some of them \hr{Ophidians in Yormis search for immortality}{did research to regain true life and immortality}.









\subsection{Politics}





\subsubsection{\Aryothim}
\target{Ophidians and Aryothim}
\target{Ophidians enslave Aryothim}
The \ophidians and \aryothim were ancestral enemies, ever since the \hr{Aryothim kill QJ}{\aryothim rebelled against their \ophidian creators}.
But \hr{Ophidian hypnotism}{\ophidian hypnotism} was powerful enough to sometimes enslave \aryothim. 

Many \ophidians hated \aryothim and enjoyed keeping some \aryothim as slaves, to show the world that the \aryoth race was always and would always be an \ophidian slave race. 





\subsubsection{Cults}
There are serpent cults that worship the old \ophidians. 
They are mostly aloof and merely seek wisdom. 
Only occasionally do they involve themselves in the \feud. 





\subsubsection{\Nephilim}
Maybe they kept \nephilim{} and \meccara{} as slaves or pets. 
They regarded them as clever animals that could learn to talk, but still just animals, nothing near their own level. 





\subsubsection{\Ophidians today}
\target{Ophidians today}
Perhaps the old, wise \ophidian{} lords and gods were drained and depleted after the harrowing war against the terrible \xzaishanns{} and went into dormancy. 
They are less violent than the wicked \draecchonosh, but cold, inhuman and alien. 

Perhaps many of them were slain by the nascent \draecchonosh. 

Perhaps the survivors sleep and dream in dark places beneath the earth. 

Perhaps they are weakened by the \dragon/\bane{} war, or the Shrouding. 

Maybe they just sleep and figure: 
\ta{Those \draecchonosh{} think they're so tough. 
  Let them deal with the \banes.} 
If the \banes{} were to conquer the \dragons, the \ophidians{} and \nagae{} \emph{might} be able to defeat them. 
They themselves believe they could. 
Others are less sure. 

Perhaps the \ophidians{} believe, \hr{Ishnaruchaefir chooses eternal war}{like \Ishnaruchaefir}, that eternal war is preferable to what would happen if one of the races won. 
They've seen what happened when the \draecchonosh{} won, after all, and it wasn't pretty.





\subsubsection{\Ophidian-\resphan connection}
Perhaps some of old \ophidians, repulsed by their \draconic{} brethren and their violent behaviour, have sided with the \resphain. Perhaps they helped found \Mystraacht. 

Perhaps \Ishnaruchaefir, being less evil and chaotic than many \dragons, has dealings with the traditional \ophidians{}, and with \Mystraacht.  





\subsubsection{Remnants of the \ophidian{} civilization}
The \scathae{} \hr{Origin of Scathae}{were created} with bits of the original serpent people, salvaged from \hr{Ophidian mummies}{mummies}. 

There exist ruins from the civilization of the serpent people. 

Compare to the serpent people from H.P. Lovecraft's stories such as \emph{The Nameless City}.

Perhaps they live on as a \quo{forgotten race}. Perhaps they live hidden among the \scathae{} and interbreeding with them, like the reptilian humanoids from \DIBiggestSecret. 





\subsubsection{Sentinels}
\target{Dark Crescent QJ}
In the \hs{Age of the Shroud}, many \quiljaaran were active members of the \hr{Sentinels}{Sentinels of \Miith}. 
Some of them worked for \hr{Secherdamon}{\IrocasSecherdamon}, possibly as part of his \hs{Dark Crescent}. 

At the time of the \thirdbanewar there were 1000-3000 \quiljaaran worldwide. 
100-200 of them worked for \Secherdamon.
10-15 of these were active on \Azmith.
5-6 of them were among the Rissitics. 

\target{QJ in Yormis}
Several \quiljaaran dwelt in or near \Yormis in disguise. 
Some of them were part of the Dark Crescent.
Others were independent.
They practiced their dark science and philosophy. 
Some worshipped the \xs (even \Ubloth) 









\subsection{Psychology}





\subsubsection{Apathy}
\target{QJ apathy}
In later ages, the \quiljaaran developed a tendency towards apathy. 
They were philosophical and non-aggressive and generally just wanted to be left in peace. 
This made it harder for them to withstand the \hr{Aryothim kill QJ}{attacks of the aggressive \aryothim}. 

\target{QJ not inventive}
Owing to their apathy, the \quiljaaran were not inventive. 
They had a lust for philosophy and knowledge, but they had a hard time putting it to practical use. 

\hr{Inventiveness}{Inventiveness is not something all creatures have}. 





\subsubsection{Evil}
Note that while the \ophidians{} are less chaotic and violent than \dragons, this does not mean that they are not evil. 
They are cold, calculating, emotionless and aloof. 
Being immortal, they possess inhuman patience and perspective, and tend to see the short lives of lesser creatures as expendable. 
\Dragons{} are likewise, but more violent and passionate. 

\lyricsbalsagoth{A Tale From the Deep Woods}{
  The orm-garth awaits me, darkly astir with ophidian malice\ldots{}
}

Some \trueophidians{} even hate all humanoids and \dragons{} and want to exterminate them, to make \Miith{} clean again.





\subsubsection{Magic}
Some \ophidian{} cultures rejected \hs{sorcery} (summoning-based magic) as evil. 
They used only \hs{psionics}. 
Their psionics were advanced and sophisticated, but lacked the brute force available in sorcery. 
That is why \hr{Sethicus}{\Sethicus} was able to defeat them. 

The {\quiljaaran} had \hr{QJ magic}{their own style of magic}. 





\subsubsection{Scientific disposition}
\target{Ophidian philosophy}
The old \ophidians{} were cool and dispassionate. 
They were scientists. 
They had no religion, only rational philosophy. 
They did, however, know of the existence of a number of powerful cosmic entities and \quo{\hs{gods}} and would sometimes make deals with them. 
But \hr{Ophidians and XS}{not with the \xss}, except for certain rebels and mavericks. 

The \ophidians were atheistic. 
They believed the universe was fundamentally devoid of meaning, purpose, reason and soul.
Life and thought existed in the universe, but only as \quo{passengers}, not as something inherent or immanent. 

As such, life was meaningless. 
The \ophidian philosophy was quite nihilistic. 

The \hr{Sethican philosophy}{philosophy of \Sethicus} renounced this world-view in favour of a more mystical one.





\subsubsection{Ouroboros}
They had a philosophical principle which they symbolized as the Worm Ouroboros. 
But this was not a god, just a metaphor. 

\index{cannibalism!\Ophidians}
Later, some \quo{fallen} \ophidians or \quiljaaran twisted the memory of this Ouroboros and turned it into a cannibal god that devours its own children (perhaps like the Hindu goddess Kali). 





\subsubsection{Slow lifestyle}
\target{Ophidians are slow}
The \ophidians were generally long-lived, slow-living and slothful, so they developed new ideas only slowly. 
Hence \hr{Early Ophidian culture}{their culture lasted for millions of years}.

\Sethicus \hr{Sethicus brought innovation}{sped things up}. 





\subsubsection{Telepathy}
\target{Ophidian telepathy}
Originally the \ophidians{} communicated using telepathy. 
This was how they evolved. 
They used sound and their voices only for artistic purposes, not for regular communication. 

\target{Ophidians discover spellwords}
But at some point, \ophidian{} sorcerers discovered a system of spellwords that used the voice to work sorcery. 
This would later evolve into the \hr{True Draconic}{\TrueDraconic} tongue. 

With this discovery, speech suddenly became more valuable. 
The \ophidians{} began to develop a spoken language. 

\hr{Sethicus}{\Sethicus} was a primus motor in the development of \TrueDraconic. 
It was connected to \hr{Sethican sound mysticism}{his mysticism regarding sound and words}. 

After the Shrouding, telepathy became much harder. 
The \ophidians could no longer hear one another's thoughts through the Shroud. 
They began to rely more on speech. 
Eventually they learned to use telepathy through the Shroud, but it was much harder now, so they only used it when they had to. 









\subsection{Skills and powers}





\subsubsection{Hypnotism}
\target{Ophidian hypnotism}
The \ophidians possessed the skill of hypnotism.
This was not an innate power but a magical (\hr{Psionics}{psionic}) technique. 
With their eyes they could mesmerize a victim, bring him under their mind control and force him to serve them, then forget all about it. 
This was an extremely useful ability in the Sentinels, and also helped the \ophidians \hr{Ophidians infiltrate Cabal}{infiltrate the Cabal} and spy on their enemies. 

This was not as powerful as \hr{Shugul mind control}{\shugul mind control}, but it was still potent. 
It could enslave powerful beings such as \aryothim, and sometimes even \resphain.
\Ophidians \hr{Ophidians enslave Aryothim}{enjoyed keeping \aryoth slaves}. 

\citeauthorbook[p.103]{RobertEHoward:TheScarletCitadel}{Robert E. Howard}{%
  The Scarlet Citadel%
}{
   \ta{The scaled people see what escapes the mortal eye,} answered Pelias, cryptically. 
   \ta{You see my fleshly guise; he saw my naked soul.}
}





\subsubsection{Possession}
\target{Ophidians possess corpses}
\target{Ophidian possession}
\Ophidians were \quo{\hr{lesser immortals}}: 
They died permanently if their physical bodies were killed. 
To protect themselves, some \ophidians chose to let their bodies lie guarded in a tomb while they sent their mind out in the world.
Using sorcery an \ophidian necromancer could project his soul into a corpse and animate it. 
He could walk around in the corpse, use it to talk, fight or even cast spells, and if the body was destroyed his soul would be drawn back to his own body.
(The sorcerer might use magic to make the corpse look more presentable.)

There were some drawbacks to this method:
\begin{enumerate}
  \item 
    The caster's own body would lie unconscious and vulnerable.
  \item 
    The range was limited.
    If the possessed body moved too far away from the caster's body it would become harder to control, and eventually the caster's mind would be jerked out of the vessel and back to his own body.
\end{enumerate}


The body had to be a corpse. 
The possession spell would kill a living victim, and it would be much harder. 
It was easier to kill the victim beforehand. 

Mortal sorcerers could also learn this spell, but it was hard.





\subsubsection{Warfare}
\target{Ophidian warfare}
The \ophidians liked to see themselves as a peaceful race, but they still wielded many terrible weapons of war.

They could call upon the \hs{Night-Feasters} (whom \hr{Origin of Night-Feasters}{they created}) to attack with \hr{Night-Feaster scream}{their screams} and their talons.

They used magical diseases and chemical weapons that spread uncontrollably and melted bodies to plasma, which would then serve as sustenance and body-mass for the \ophidians' living war machines. 

They could use mind control to destroy their enemies' morale and turn them against each other, or even enslave them.

They could release the fire of \hr{Ruin Satha}{\RuinSatha} to consume legions. 
And they could open channels to \RuinSatha's throne of chaos, whose destructive winds of madness disintegrated matter and dissolved the fabric of the reality itself. 









\subsection{Story ideas}
\subsubsection{Sleeping \ophidians}
Someone, on a journey through the world (probably the \Wylde{}), encounters one or more sleeping \ophidians{}. 
Probably in a cave beneath the earth, or in an ancient, abandoned temple or palace. 

They communicate with the \ophidians{} in dreams. 
Or perhaps the humanoids merely feel the \psp{\ophidians} dreams, almost being sucked into them because of their mental great power. 

The humanoids might be \Shilred-tachi, or perhaps Carzain-tachi in some later story. 









\subsection{Undeath}
\target{Ophidian mummies}
When the \hs{Durance} began, great numbers of \ophidians who had sided with the \dragons ended up joining them in Durance, voluntarily or by force. 
They became entombed as mummies.

In later ages, more \ophidians would let themselves mummify when they died.
Even \Sethicus did not know everthing there was to know about death and immortality. 
They believed that if a dead Lesser Immortal was mummified using the correct spells and prayers to \KhothSell, it would be enabled to rise again at a later date as one of the undead. 
This was a lesser form of True Immortality. 

When some \dragons awoke from Durance, some of their servitors also awoke. 
\Sethicus awakened many of them to fight in his armies in the \firstbanewar. 
Perhaps even most, also taking most of the \ophidians that were supposed to serve those \dragons who were still in Durance. 

The \ophidian mummies \hr{Undead Ophidian}{arose to undeath}.

But not all awoke.
It proved very difficult to awaken the \ophidian mummies. 
So many kept sleeping. 
They might potentially awaken later.

\hr{Thessulax}{\Thessulax} had \hr{Thessulax's mummies}{many such mummies in her tomb}. 
There \hr{Mummies in Yormis}{were a few in \Yormis}. 

Compare to the entombed saurians in \cite{Nile:InTheirDarkenesShrines}. 





\subsubsection{Worshipped}
\target{Ophidian mummy worship}
The mummies were worshipped by some. 
Though they lay dormant, the dead Elder \ophidians could still impart some wisdom unto their worshippers. 
















\section[Scatha]{\Scatha}
\target{Scatha}
\target{Scathae}
\index{\scatha{} (plural \scathae)}
The \scathae{} are reptilian humanoids. They are one of the dominant humanoid species on \Miith{}, especially common in the Imetrium and in the Rissitic Dominion. 

\Scathae{} are very common and widespread. %In a sense, they are the 'humans' of \Miith{}. 









\subsection{Name}
Singular \emph{\scatha{}}, plural \emph{\scathae{}}. 
This grammar is Imetric. 
The associated adjective is \emph{\scathaese{}}. 

A male \scatha{} is called a \dax. 
A female is called a \sphyle. 









\subsection{Physique}
\Scathae{} are large, bipedal reptiles. 
They have a tail and do not stand fully erect. 
They have five fingers on each hand and four toes on each foot. 

Males are generally dark in \colour whereas females have brighter scales. Males are recognizable by two bony ridges on the top of the head (one over each ear). Females have a single ridge in the middle of the head, generally larger than that of the males. 

In combat, \scathae{} use weapons, such as swords or spears. They have good eyesight and make good archers or gunners. 





\subsubsection{Compared to \humans}

A \scatha{} is slightly stronger than a \human and significantly tougher. 
But they are not very dextrous nor fast. 
They have thick scales, especially on the back, shoulders and overarms, which provide protection in combat. 
Their teeth and nails are small, like those of \humans, and not useful in combat. 

A \scatha's posture gave him a number of pros and cons in combat compared to \humans. 
A \scatha was a much longer target than a \human and could thus be easier to strike in \melee.
Especially the forward-protruding head was a vulnerable target. 
The posture with a forward-jutting head made it hard for a \scatha to use a shield. 
The tail was likewise vulnerable, but could also be used as an additional striking limb (mostly in unarmed combat).

To compensate for some of these flaws, \scathae had hard scales and strong arms. 
Their legs were also long and strong and could deliver nasty kicks. 





\subsubsection{Posture}
\Scathae did not stand fully erect. 
Under normal circumstances they held their back and tail at an angle of 30 to 35 degrees from the ground (i.e., closer to horizontal than vertical). 
With some difficulty they could rear up to an angle of around 60 degrees. 

When running, a \scatha would lean forward and hold his back and tail parallel to the ground. 





\subsubsection{Potential for greatness}
\target{Scathae have potential for greatness}
\hr{Origin of Scathae}{The \scathae were created by the \dragons} using bio-magic learned from the \xss.
The process of creating the \scathae was very much inspired by the way the \dragons had created themselves.
The \scathae were designed as \dragons in miniature.

As such, \scathae genuinely did have more individual potential for greatness than \humans did.
\hr{Humans suck}{\Humans were worthwhile only in a big mass}.
Each \scatha could reach great spiritual advancement and become a demigod.
Look at \hr{Criseis}{\Criseis} and \hr{Psyrex}{\LocarPsyrex}. 

This was also why \hr{Dragon blood gives immortality}{drinking \draconian blood} worked better for \scathae than for \humans. 





\subsubsection{Senses}

%\Scathaese{} eyes are covered by a fully transparent but solid membrane that protects from sandstorms and the like. The outermost layer of this membrane is shed once every two months or so (similar to how a snake sheds its skin). They also have normal eyelids over this. As a special feature, 
\Scathae{} blink upwards with their lower eyelids, unlike \humans, who blink downwards with their upper eyelids. 
A \scatha{} has a long tongue and can lick his own eyes. 

\Scathae{} are diurnal and do not see well in the dark. 

\Scathaese{} senses are like those of \humans, except the sense of touch, which is rather dull (due to their thick scales). 

\target{Scathaese colour vision}
\Scathae{} have \colour vision different from that of \humans. They can distinguish between some nuances that \humans{} cannot tell apart, and vice versa. \hr{Curiet's colour vision}{This is brought up} by \hs{Curiet Serpentin}.





\subsubsection{Size}
An average adult \scatha{} is 2 meters long, 150-160 cm tall and weighs 70--80 kg. 
Males and females are the same size. 
The tail accounts for about $40\%$ of the length and is not prehensile. 

As they age, \scathae{} continue to grow in height and length, but not much in weight. Very old \scathae{} tend to be long and gaunt. 









\subsection{Biology}
%\Scathae{} are warm-blooded reptilians. They are Archosaurs, related to dinosaurs and birds. The species is a result of natural evolution and has a few closely related (but unintelligent) species. They prefer temperate to subtropical climates with low humidity. They are omnivores with feeding habits similar to those of \humans{}. (Today, most \scathae{} are civilized and live off farming.) The \scathae{} evolved in the sandy deserts of the East, and the violent sandstorms are one of the reasons why they developed their thick, protective scales and eye-membranes. 

\Scathae{} are warm-blooded reptilians. They did not evolve naturally, but were the creations of the early \dragonlords{} who mixed \dragon{} and \naga{} genes with those of land-dwelling reptiles (including \nycans{}). They are herbivores. 

\Scathae{} have two genders and reproduce by internal fertilization. The female lays a small cluster of eggs (typically one or two). Males and females are born in equal numbers. 

\Scathae{} have a life expenctancy of around 75 years at TL3. Ancient \scathae{} of up to 100 years are exceptional but not unheard of. They reach sexual maturity in their early teens. They are usually considered adult around the age of 20 (varying depending on culture). 

Among \scathae{}, males and females are very similar, physically as well as psychologically, and the genders usually equal in society. Because of this, homo- and bisexuality is rather common. \Scathae{} are usually monogamous, but they may or may not mate for life.  





\subsubsection{\Demiscathae}
\target{Demiscatha}
There existed a number of \quo{\demiscatha} races. 

\begin{gloss}
  \gitem{\hr{Scatha-Naga hybrids}{\scatha/\naga hybrids}}
    These hybrids, such as the ones in the Imetrium, were considered \demiscathae by outsiders. 
  
  \gitem{\hr{Naga}{\naga}}
    Some \hr{Nagae considered Demiscathae}{thought of the \nagae as \demiscathae}.
    Those who had actually seen and dealt with the \nagae knew better.
\end{gloss}

There were also \quo{\hr{Demihuman}{\demihumans}}.

The \scathae were more tolerant of their different kin than the \humans were, but after millennia of interbreeding there had emerged a race of \quo{standard} \scathae, and other variants came to be called \quo{\demiscathae}. 

Rethink the idea of the different \scatha ethnicities!

Some \scathae had fins or sails on their backs. 





\subsubsection{\Meccara}
\target{Meccaran Scathae}
I should get rid of the \meccara and replace them with various \demiscatha types. 
Some races of \demiscathae could be small and jungle-dwelling, and some could have regenerative powers. 
Some could even be \emph{named} \quo{\meccara}. 

The \meccara were considered a lower, less intelligent and less evolved form of \scathae. 
They were certainly smaller, weaker and more technologically primitive.
They were a degenerate, atavistic strain.

Compare to \cite{RobertEHoward:WormsoftheEarth}. 





\subsubsection{\Scatha/\naga hybrids}
\Scathae could interbreed with \nagae, \hr{Scatha-Naga hybrids}{giving rise to hybrid children}. 









\subsection{Psychology}
The \scathae{} are naturally evolved from animals that were both predators and prey. Where many other creatures of the same areas developed great size and horns or spines to defend themselves, the proto-\scathae{} instead developed pack tactics, intelligence and hands. 

As the descendants of pack-living prey animals, \scathae{} have a strong sense of commitment to their fellows. This makes the typical \scatha{} loyal and patriotic, but also xenophobic. \Scathae{} are fiercely devoted to their community (family, clan, nation), slow to trust outsiders and zealous about punishing betrayers and criminals. 

\Scathae{} value security and a well-ordered life. They prefer to know their place in the pack/community and function well in a class system. \Scathae{} make good craftsmen and soldiers. They have little craving for excitement and rarely go out to \quo{adventure}. \Scathaese{} adventurers are typically on some kind of quest for their community, rather than simply seeking fame and wealth. 





\subsubsection{Empathy}
\Scathae were much better organized than \humans and better at functioning as a group. 
They had a limited empathy which made them work better together. 

Compare to the telepathic Ant-kinden in \cite{AdrianTchaikovsky:ShadowsoftheApt}. 





\subsubsection{Ferocity}
\target{Scatha fury}
The \scathae{} actually have a deep-seated, bestial, primal fury and ferocity. 
It is tempered and held in abeyance by cold, rational \ophidian{} genes and millennia of breeding and conditioning, but it still lurks underneath the surface and can erupt in full flame. 
After all, the \scathae{} do have \xsic{} blood. 
They were created by the \dzraicchenosses{} in (a shadow of) their own image. 

Most of the time, this ferocity is sublimated into some other lust. 
Often religious or ideological fervour. 

Sometimes, \hr{Imetrian coldness}{as with the Imetrians}, the ferocity manifests as a cold, merciless fanaticism. 





\subsubsection{Societies}
\Scathaese{} societies tend to be more collective and less monarchical than \human{} ones. 
\Scathae{} do not have the same hunger for personal power, but tend to be more social and pro-community, so it is easier for them to form stable systems of many-man-rule. 
Such as \hr{Bacconate}{\bacconates}. 





\subsubsection{\XzaiShanns and aspects of the mind}
Different aspects of the \scathaese mind \hr{Primordials and the Scathaese mind}{were associated with different \xss}.
There was some truth in this. 
(See the section.)









\subsection{Demographics}
\subsubsection{Habitat}
\Scathae{} are among the most widespread humanoids on \Miith{}. 
They are common in all lands except the Northern Kingdoms, but especially dominant in the Imetrium, \Durcac and Irokas. 





\subsubsection{Subraces}
The \scathaese{} race can be divided into three major subraces: 

\begin{gloss}
  \gitem[\Tassians]{\Tassian}
    \target{Tassian}
    The most widespread subrace. 
    Blue scales. 
    Common in the Imetrium and southern \Velcad{}
    Peoples include the \hr{Masthenon}{\Masthenon} and \hr{Ortaica}{\Ortaicans}. 
  \gitem[\Mekriis]{\Mekrii}
    \target{Mekrii}
    Red or reddish brown scales.
    Most common in \Durcac and the Orient. 
    Peoples include the \hr{Shurco}{\Shurco} and \hs{Rissitics}. 
  \gitem[\Lois]{\Loi}
    \target{Loi}
    Green or greenish brown (very rarely black).
    Most common in the north and the \Serplands. 
\end{gloss}










\subsection{History}
The \scathae{} were \hr{Origin of Scathae}{created to serve the \dragons}. 

Despite this, they were not slaves to the same degree as \humans. 
They were bred from \naga{} stock and as such, they were free citizens to a higher degree than \humans{} ever were. 
They had more capability to think for themselves, and they understood more of the world around them, and of their masters. 
After the \hr{Shrouding}{\SecondShrouding}, though, they became more enslaved, as their minds succumbed to the Shroud.

Perhaps they were created to replace the old \hr{Ophidian humanoids}{\ophidian{} humanoids} when they became extinct. 





\subsubsection{The first \scathae}
The first generation of \scathae{} to be created were more powerful and had more \draconian{} and \xsic{} blood than modern \scathae. A few of these prototypes survive, including \hr{Criseis}{\Criseis} and \hr{Psyrex}{\LocarPsyrex}. 









\subsection{\Tsutoras}
\target{Troglodyte}
\target{Tsu-Tora}
\index{\tsutora}
\index{\troglodyte}
The \tsutoras, also called \troglodytes, were a degenerate subrace of \demiscathae (or several superficially similar subraces) that dwelt in caves underground. 
They were often found worshipping the \daemonic{} monsters of \hs{Kai Leng}. 

To the eyes of other \scathae, the \troglodytes{} were frightening and loathsome abominations. 
They remind the \scathae of some primal, primitive aspect of their own nature and origin, something which the \scathae would rather forget. 

Some subraces of \tsutoras were very large and imposing, and much stronger than an average \scatha. 
Some \tsutoras had exotic features such as a neck frill, camouflage, a long tongue, a prehensile tail or even regenerative powers. 

Compare them to the Tu-Tara from the game \cite{VideoGame:Torchlight}. 









\subsection{Politics}





\subsubsection{Anti-\scatha racism}
\target{Anti-Scatha racism}
Some \human racists saw the \scathae as loathsome reptilian monsters. 

See also the sections about \hr{Anti-Human racism}{anti-\human racism} and how \hr{Scathae and Humans hate each other at first}{\scathae and \humans hated each other at first sight}. 

\citeauthorbook[p.225]{RobertEHoward:ChildrenoftheNight}{Robert E. Howard}{%
  Children of the Night%
}{
  \ldots they were, in truth, people of night and darkness and the ancient horror-ridden shadows of bygone ages.
  For these creatures were very old, and they represented an outworn age.
  They had once overrun and possessed this land, and they had been driven into hidign and obscurity by the dark, fierce little Picts with whom we contested now, and who hated and loathed them as savagely as did we.
}





\subsubsection{\Humans}
\Scathae and \humans \hr{Scathae and Humans hate each other at first}{hated each other when they first met}.
After some centuries they slowly learned to accept one another. 















\section{Wyrms}
\target{Wyrm}
\target{Wyrms}
\index{Wyrm}
Some \ophidians were not powerful enough to become \dragons.
But they were able to become lesser \dragon{}like creatures called Wyrms. 



























