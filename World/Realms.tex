\chapter{The Realms}
This is about individual Realms. 
For general information on how the Realms work, see the section on \hs{Realms}. 















\section{\Draconian Realms}
There existed a bunch of Realms full of the \dragons' old cities and machines. 
Many of these Realms later became \wylde ruins. 
Such realms were called \emph{Pandaemonia} (sing. \emph{Pandaemonium}). 

Many of the \dragons' slave monsters still existed, but they could not be controlled because no living \dragons knew their passwords, or because they were imprinted to obey only specific living \dragons which were now dead. 















\section{\Erebos}
\target{Erebos}
\index{\Erebos}
Deep in the vastness of space, many thousand light years from \Miith{}, lies the sinister world of \Erebos{}\dash called the Nightmare Dimension by the \caisith. 
Thousands of years before the awakening of the \dragons{}, \Erebos{} had been visited by the \voyagers{}. 
There they \hr{Banes are created}{created the \banes}. 

Whenever I write about \Erebos, be sure to also read about \hr{Nyx}{\Nyx}.









\subsection{Dark mirror of \Nyx}
\Erebos was the dark mirror of \Nyx. 
The two worlds were connected because of the Dark Heart and the \hr{World-God of Erebos}{World-God} that lay dead and dreaming.









\subsection{Denizens}
Denizens of \Erebos included:
\begin{itemize}
  \item \Banes.
  \item \Flyingpolyps.
  \item \Noggyaleth (maybe).
  \item \Screamers.
  \item \Umbrae.
  \item Weavers.
\end{itemize}










\subsection{Scenery}
On \Erebos, the sky is pitch black, but down in the depths beneath the twisted spires and towers, that stand thousands of metres tall, there is a dimly luminescent mist down in the depths, which sometimes lights up the place a little bit with its ghastly, spectral glow. 

It is a world of iron shards and rotting flesh and rivers of chemical poison. 

In the brooding darkness of the skies dwell the \hr{Umbra}{\umbrae} and other horrors. 

Perhaps there are flying fortresses drifting around up there, remnants of the \hr{Voyager}{\voyagers} or an extinct \erebean{} civilization that was destroyed by the \banes. 

\lyricsbible{Job 10:21--22}{
  Before I go [whence] I shall not return, [even] to the land of darkness and the shadow of death; \\
  A land of darkness, as darkness [itself; and] of the shadow of death, without any order, and [where] the light [is] as darkness.
}

\citeauthorbook[p.184]{HPLovecraft:TheWhispererinDarkness}{\HPLovecraft}{%
  The Whisperer in Darkness
}{
  The sun shines there no brighter than a star, but the beings needs no light.
  They have other ,subtler senses, and put no windows in their great houses and temples.
  Light even hurts and hampers and confuses them, for it does not exists at all in the black cosmos outside time and space where they came from originally.
  To visit Yuggoth would drive any weak man mad\dash yet I am going there.
  The black rivers of pitch that flow under those mysterious Cyclopean bridges\dash things built by some elder race extinct and forgotten before the beings came to Yuggoth from the ultimate voids\dash ought to be enough to make any man a Dante or Poe if he can keep sane long enough to tell what he has seen.
}





\subsubsection{Burrowers beneath}
In the mystic gloom of the deep abyss, you can sometimes hear the writhing of the horrible \hr{Ghobal}{\ghobaleth} and/or \hr{Flying polyps}{\flyingpolyps}, or feel the tremours of their passing and their burrowing.









\subsection{History}
See the section on the \hr{Bane history}{history of the \banes}. 









\subsection{The Heart of \Erebos}
\target{Heart of Erebos}
\target{Erebos undead}
Perhaps \Erebos{} is, in some sense, an undead planet, a vampire and scavenger.

The Heart of \Erebos{} still beats, but the blood it pumps is ashen and all but lifeless. 
That is why the \banes{} invade \Miith{}: 
They seek the Heart of \Miith{} so that they may live on. 















\section{\KaiLeng, the Underworld}
\target{Kai Leng}
\target{Kai-Leng}
\target{KaiLeng}
\index{\KaiLeng}
\KaiLeng{} was the underworld of \Miith{}. 
It was a \hs{Chthonic Realm}, accessible from \Azmith{} through deep tunnels. 









\subsection{Entrances}
\KaiLeng was accessible from \Azmith{} through deep tunnels. 
One system of such tunnels lay underneath Mount \hr{Shrun}{\Shrun} near \hr{Yormis}{\Yormis}. 









\subsection{Inhabitants}
In \KaiLeng dwell monsters and Great Old Ones, perhaps \xss. 

Perhaps the Tyrant Worms. 
Perhaps these are somehow related to the \hr{Ghobal}{\ghobaleth}. 
Perhaps they were engineered by the \voyagers, like the \banes{} were. 
Or perhaps the Tyrant Worms were indigenous to \Miith{}, and the \voyagers{} created cheap copies of them on \Erebos, which evolved into \ghobaleth. 

There also dwell \hr{Troglodyte}{\troglodytes} that worship these monsters. 

\KaiLeng{} contains relics of the \psp{\voyagers} civilization, and of the \xss. 





\subsubsection{Gods}
Gods that dwelt in \KaiLeng include \hr{Yolbaoth}{\Yolbaoth} and \hr{Ubloth}{\Ubloth}. 









\subsection{Someone explores \KaiLeng}
Have a scene where someone explores \KaiLeng. 

\lyricsbalsagoth{Invocations Beyond the Outer-World Night}{
  These darkling subterrene dominions, astir with strange and terrible beings, sired by entities whose genesis was far beyond the nighted void of our own outer-world! \\
  The legacy of the First Ones, spawn of the Mera!
}

Someone delves deep and finds ancient \voyager{} artifacts, connected to the origin of all life, and of the \hr{Heart}{Heart of \Miith}, and the \hs{Sun}. 

\lyricsbalsagoth{Invocations Beyond the Outer-World Night}{
  Behold, a vast plasma-fueled crystalline illuminatory orb\ldots{} \\
  a vril-sun rising!\\
  And marvel at the colossal terra-forming machines of the First Ones!
  
  Far, far beneath the surface of this coruscating sphere, at the very heart of our mysterious globe, lies the true path to man's dark destiny beyond the heavens\ldots{}
}















\section{Labyrinth}
See the \hs{Labyrinth} section.















\section{\Machai}
\target{Machai}
\index{\Machai}
\Machai was originally a Realm of \Miith. 
After the \hr{Shrouding}{\Shrouding}, \Machai was torn apart and became split into three Realms. 

\Machai was always closer to the Realm of the \xss than the rest of \Miith. 
So it was inhabited by strange and exotic creatures and monsters. 

After (or before) the \hs{Draconian Supremacy} began, \Tiamat and many of her fellow \dragons moved into \Machai and made it their base. 

After the \secondbanewar and the \Shrouding, the three fragments of Machai came to be considered \quo{Immortal Realms}. 
They were dominated by the Sentinels. 









\subsection{Fantastic landscape}
Some \hs{myths} paint \Machai{} as Hell, but it actually isn't. 
It is a diverse and exotic Realm, terrible at times, but also beautiful and fascinating. 
It features such things as:

\begin{itemize}
  \item Rivers that run with liquid metal.
  \item Flying trees. These float at a constant altitude and are huge. You can build houses and towns in them. 
  \item Storms and geysers of fire. 
  \item Flying islands.
  \item Enourmous pyramids and domes. 
  \item Moon-like craters and rocks.
\end{itemize}

Compare to \authorbook{Clark Ashton Smith}{The Door to Saturn}, and the middle-illustrations of \bandalbum{Limbonic Art}{Ad Noctum - Dynasty of Death} and \bandalbum{Limbonic Art}{The Ultimate Death Worship}. 

Compare the strange lifeforms here to the things in the movie \cite{Movie:Evolution}. 

\citeauthorbook[p.57--58]{RobertEHoward:TheMirrorsofTuzunThune}{Robert E. Howard}{%
  The Mirrors of Tuzun Thune%
}{
  Gray fogs obscured the vision, great billows of mist, ever heaving and changing like the ghost of a great river; through these fogs Kull caught swift fleeting visions of horror and strangeness; beasts and men moved there and shapes neither men nor beasts; great exotic blossoms glowed through the grayness; tall tropic trees towered high over reeking swamps, where reptilian monsters wallowed and bellowed; the sky was ghastly with flying dragons and the restless seas rocked and roared and beat endlessly along with muddy beaches.
  Man was not, yet man was the dream of the gods and strange were the nightmare forms that glided through the noisome jungles.
  Battle and onslaught were there, and frightful love.
  Death was there, for Life and Death go hand in hand. 
  Across the slimy beaches of the world sounded the bellowing of the monsters, and incredible shapes loomed through the steaming curtain of the incessant rain.
}











\subsection{\Nom}
\target{Dragonland}
\target{Nom}
\target{Fallen Dragonland}
\index{\Nom}
\Nom is a place that once used to be the capitol and homeland of the proud \draconic{} empire. 
Perhaps it's a whole Realm, or at least a pocket Realm (like \hr{Nyx}{\Nyx}). 





\subsubsection{Description}
Now \Nom lies in ruins, a dead necropolis haunted by monsters and the ghosts of dead \dragons. 
Perhaps there are monsters who once served the \dragons{} and now roam free as the \dragons{} have \hr{Dragons have forgotten}{forgotten how to command them}. 

There should be huge rocks that jut up like claws coming out of the earth. 
Compare to certain Limbonic Art album art pieces, or the realm of \Juujinkai{} in the anime \emph{\Urotsukidouji}. 

Nom was a nightmare land where the Shroud was thin and Chaos seeped through from the Beyond to distort the fabric of reality.
There were burnt and decayed ruins of enormous buildings.
There were grotesque plants and fungi\dash bloated and sickly, some of them amorphous and fleshlike and full of eyes and mouths. 
These were the mutated remnants of the living buildings. 

Compare to the Dreamlands as shown in \cite{MichaelNelson:FallofCthulhuII}. 





\subsubsection{History}
\target{Sethicus invades Nom}
\Nom was once the site of a great city of the \hr{Shugul}{\shugul}.
From here they ruled the world and worshipped their \hr{Moongods}{\moongods}.
\hr{Sethicus}{\Sethicus} waged war against the \shugul, and when his armies were strong enough he invaded \Nom.
At the suggestion of \hr{Achamoth}{\Achamoth} he vanquished the \shugul and conquered their colossal citadels, which had stood for millions of years.
The \shugul temples were vast and reached far out into the Beyond.
\Sethicus used them foundations when he built his own enormous temple-city of \hr{Baltherium}{\Baltherium}.

\Sethicus wanted to drain the mighty dweomers that he had seized from the \shugul.
He even went so far as to imprison some of the \moongods themselves and use them as living, raging \dweomers. 
By draining the power of these gods \Sethicus was able to raise the mystic and marvellous \Baltherium. 

All of \Baltherium was built up around a gigantic stairway-tower that, when completed, would form a permanent portal to \RuinSatha's burning throne of chaos at the centre of the universe. 
This would give \Sethicus access to unlimited power and insight. 
But the tower was never completed. 
After \Sethicus went into \hs{Durance} \Baltherium was soon overrun by grotesque horrors. 
The \dragons abandoned \Baltherium and all of \Nom.

When \Sethicus died there was no one strong enough to subdue the captive \moongods.
They began to slowly break free. 
They were malevolent and immensely powerful.





\subsubsection{\Baltherium}
\target{Dathka}
\target{Baltherium}
\index{\Baltherium}
Here we find \Baltherium: 
A colossal, cyclopean palace/temple where \Tiamat{} and her \firstgendragons{} once dwelt. 
Now it serves as their necropolis. 

\Baltherium had great mythical status.
It was known to be older and even more mysterious than \hr{Haamon}{\Haamon}\dash more powerful and more dreaded.

Compare to:

\begin{itemize}
  \item Kadath in \cite{HPLovecraft:TheDreamQuestofUnknownKadath}.
  \item R'lyeh in \cite{HPLovecraft:TheCallofCthulhu}. 
  \item The Antarctic city in \cite{HPLovecraft:AttheMountainsofMadness}. 
  \item Starvald Demelain in \cite{StevenErikson:ReapersGale}. 
\end{itemize}

\citebandsong{DeathspellOmega:SiMonumentumRequiresCircumspice}{%
  Deathspell Omega
}{
  Carnal Malefactor
}{
  Below the lid of a vast rounded monument\\
  Trickling of gristly vestiges and whacked hopes\\
  Enhanced by the horrible excess of fetid exhalation\\
  And uterine strangulation by the wreaths\\
  Of the herds astray, arid in despair, blessed\\
  With dilated flakes of fire, slowly wafting down\ldots{}
}

The tomb of \Sethicus: 

\citebandsong{Nile:AnnihilationoftheWicked}{Nile}{
  Annihilation of the Wicked
}{
  The Dominion of Seker.\\
  Barren Desert of Eternal Night.\\
  Shunned by Ra.\\
  Behind the Gate Aha-Neteru.\\
  The Wastelands of Seker.\\
  Eldest Lord of Impenetrable Blackness.\\
  Death God of Memphis.\\
  He of the Darkness and Decay of the Tomb.\\
  He of Rosetau, the Mouth of the Passage to the Underworld.\\
  Closely Guarded by Terrible Serpents\\
  who Careth Not for His Own Cult of Worshippers.
  
  Seker, Ancient and Dead,\\
  Primeval Master of the World Below,\\
  Remaineth Unwitnessed, Unseen.\\
  Hidden in His Secret Chamber.\\
  His Primitive Graven Image like as a Hawk-headed Man.\\
  Shrouded and Swathed in Tomb Wrappings.\\
  Standing Between a Pair of Wings \\
  which Issue Forth from the Back of a Monstrous Serpent,\\
  Having Two Heads, Having Two Necks \\
  and Whose Tail Terminates in a \human Skull.
}





\subsubsection{Knowledge}
There was much arcane knowledge hidden here in the \draconic/\ophidian ruins. 
Some brave seekers would come seeking it. 
But it was guarded. 

\lyricsbalsagoth{Unfettering the Hoary Sentinels of Karnak}{
  The Coptic papyrus states that, upon the walls of the pyramids and the temple were inscribed the mysteries of science, astronomy, geometry and physics; inscriptions of unknown peoples and lost civilizations whose lore was carved into the stone to preserve it from the ravages of the great deluge.\\
  The surviving knowledge of long forgotten antediluvian races!\\
  Aye, prudent Surid, heeding the warnings of his priests, erected certain repositories of long forgotten knowledge to withstand the first great flood, and then an all-consuming fire which was prophesied would come from the sky.\\
  Masoudi, in the tenth century, described automata; titanic guardians of stone and metal which were placed to guard the treasures and the entombed lore, and which were tasked to destroy all those deemed unworthy, all those who dared enter the chambers unbidden.\\
  I see them!\\
  The hoary sentinels of Karnak are unfettered!\\
  Rising from their sandy tombs to smite the intruder, the raider and the interloper with righteous fury!\\
  And what is this\ldots{} was there once a glimmer of life within the sightless stone eyes of the Theban guardian?\\
  Does the silent watcher of Giza even now descend from its granite dais to once more stalk the shifting sands on carven claws?
}





\subsubsection{Undead machines}
The \draconic{} necropolises were protected and maintained by \hs{undead machines}. 










\subsection{Blood Red Sun}
\target{Blood Red Sun of Machai}
In some parts of \Machai, the Sun looks monstrous and strange. 
It is not only huge, filling most of the sky; it is also a dark blood red \colour. It is dreadful to behold. 

Compare to the \hr{Black stars of Nyx}{black stars of \Nyx}. 

This is not all of \Machai, though. Perhaps this only applies in the \hr{Fallen Dragonland}{fallen \dragon-land}. 









\subsection{Living buildings}
\target{Living buildings of Machai}
Buildings in \Machai{} are colossal. But where \hr{City of Nyx}{the edifices of \Nyx} are tall, skeletal and dead, the structures on \Machai{} are bloated and squat. They resemble grotesque, overgrown living creatures\dash which is exactly what they are!

Perhaps the buildings are worshipped as mindless demigods by the lesser beings of \Machai. Kind of like the living Zerg buildings in the game \cite{VideoGame:Starcraft}. 

\lyricsbalsagoth{Beneath the Crimson Vaults of Cydonia}{
  Colossal shapes etched against the moons. \\
  Supine obeisance 'fore the mound.\\
  Accursed fiends hail the Slitherer. \\
  Abhorrent jaws drooling lunacy.
}









\subsection{Moons}
\Machai{} has its own evil moons.

\lyricsbalsagoth{Beneath the Crimson Vaults of Cydonia}{
  Phobos, Deimos! \\
  The moons' rays liquefied in these blood red pyramids.\\
  In the shrines of abomination, black tongues rapt with blasphemy.\\
  Chaosphere, watchtowers, genesis, Cydonia\ldots{}\\
  The Abyss yawns wide!
}









\subsection{Stephen Marley's description of Hell}
Here is a description of the Chinese Hell.

\lyricstitle{\authorbook{Stephen Marley}{Mortal Mask} p.247}{
  Red-robed Yen-Lo, Lord of the Sad Dead, took his ease on a red lacquer throne as he surveyed his ten regions of afflication.
  In the hell of desecrators and cannibals, yellow-eyed demons and black dogs drove the condemned souls into a river of boiling blood. 
  In the hell of dismemberment, the guilty hung in bits and pieces from hooks and chains.
  In the upside-down hell, the damned were suspended by their heels from the ceiling, their condition reflecting their scale of values in life.
}















\section{Mirage Asylum}
\target{Mirage Asylum}
The Mirage Asylum is \ps{\Ishnaruchaefir} citadel. 

\Ishnaruchaefir{} uses the Asylum to explore and research the far reaches of the cosmos. 
Remember, he has connections to the \hs{cosmic gods}, forces beyond the heavens. 
Unlike, for instance, \Secherdamon, who is obsessed with \Machai{} and the \xss. 









\subsection{Culture}





\subsubsection{\Dragons lie and think}
\target{Ishnaruchaefir lies thinking in Mirage Asylum}
When \Ishnaruchaefir and the other \dragons lie still and think in the Mirage Asylum, they reach out into the Beyond with their minds.
Here they not only think and speculate and theorize, but also practice their magic and do their research and experiments.
Often you can see storms of power around them as physical symptoms and manifestations of their otherworldly magical experiments.
The inhabitants of the Asylum see this as proof of the \dragons' terrible divinity and cower down and worship them.





\subsubsection{Language}
The Mirage Asylum was almost completely isolated from the rest of the world from its creation till its destruction. 
So naturally, the population developed their own language: 
Issikulik. 
Based on Greenlandic (Kalaallisut). 





\subsubsection{Population and daily life}
The Asylum had a population of around 1000 \scathae. 

The Asylum is self-providing. 
There are some large \quo{gardens} where exotic plants grow and exotic animals feed. 
These are farmed by the civilian inhabitants. 
Compare them to the legendary \quo{Hanging Gardens of Babylon}. 

Only \Iscrafel, \hr{Criseis}{\Criseis} and \hr{Najarod}{\Najarod} possessed the Gnosis required to open the portals that let them leave or enter the Asylum. 
And they seldom did. 

\Iscrafel often left the Asylum to hunt. 









\subsection{History}





\subsubsection{Origin}
The Mirage Asylum was built on fragments of \Ishnaruchaefir's old tomb in which he had slept for a million years.

\target{Mirage Asylum became a hideout}
After the Shrouding, \Ishnaruchaefir tore the tomb loose and cast it out \hr{Mirage Asylum orbit}{to orbit \Miith}. 
He also used Shroud spells to hide it and thus create a permanent hideout. 
A side effect of this was that the place became twisted into its current form. 





\subsubsection{Destroyed}
At some point in the \thirdbanewar, the Asylym was \hr{Mirage Asylum destroyed}{breached and destroyed}. 

The Asylum was protected by the Shroud. 
It was only because the Shroud was \hs{unravelling} that it was possible for the \resphain{} to breach it. 









\subsection{Nature}





\subsubsection{Alive}
\target{Mirage Asylum lives}
The Mirage Asylum was a living structure (like \hr{Living machines}{living machines}). 
It looked sort of like a humongous conch. 

There were great crystalline growths that somehow grew from the living flesh of the Asylum. 
These crystals were vital to the economy and ecology of the place. 
Compare them to horns, nails or teeth: 
Dead mineral structures that grow from a living body. 





\subsubsection{Appearance}
The Mirage Asylum was like a half-open castle ruin or space hulk drifting afloat in the vast, empty void of space. 

Its architecture was insane, featuring all sorts of impossible geometry. 
It floated free in space with barely any walls. 

The Asylum had no central \quo{body}. 
It was composed entirely of winding stairways and bridges and branches and the occasional big bulging hub. 
It was sort of like a tree, but cyclic, often doubling back upon itself. 

Compare it to the Arcane Sanctuary from \emph{Diablo II}. 
Also compare to the cover of \cite{LimbonicArt:InAbhorrenceDementia}.
And the art of M.C. Escher.

See also the sections on \hr{Resphan architecture}{\resphan architecture} and \hs{dark ancient cities}. 





\subsubsection{Energy}
The Mirage Asylum was located near a very energy-rich conflux point.
This let \Iscrafel dwell undetected and also let him feed on the many monsters that flocked to this watering hole. 





\subsubsection{Gravity}
The gravity on the Asylum always pointed \quo{down} towards the surface of the Asylum. 
There were plenty of places where one could go all the way around some narrow branch, but on all sides the gravity would point straight down. 

The Asylum had lower gravity than \Miith. 
This was not the gravity of the Asylum itself, but that of \Miith. 

\target{Mirage Asylum orbit}
The place orbited \Miith at a quite low orbit. 
\Miith was never visible from the Asylum. 
\Miith was always \quo{down} under the ground. 
Said \quo{ground} twisted around.
This was a result of the extremely strange geometry that abounded in the Asylum. 
The Asylum was \hr{Mirage Asylum became a hideout}{twisted into a hideout} for \Ishnaruchaefir. 
\Miith was not visible from the Asylum because it always lay \quo{underground}, and similarly, the Asylum was not visible from \Miith. 





\subsubsection{Shroud nature}
The Asylum was so twisted and insane because it was close to one of the sources of the Shroud itself. 
\Ishnaruchaefir was \hr{Ishnaruchaefir maintains the Shroud}{actively maintaining the Shroud} from within the Mirage Asylum. 
He had made it extra twisted. 





\subsubsection{Sky}
The Mirage Asylum drifted afloat in the vast, empty void of space. 
Stars and nebulae were visible above. 

What about the Sun?
Was there day and night?
From where did the place get its energy?









\subsection{Name and significance}
The name is taken from \bandsong{Limbonic Art}{Deathtrip to a Mirage Asylum}. 





\subsubsection{Reputation}
The Mirage Asylum had a reputation as a dark, formless throne of evil. 

\citeauthorbook[p.251]{HPLovecraft:TheBlackTomeofAlsophocus}{H. P. Lovecraft}{%
  The Black Tome of Alsophocus%
}{%
  \quo{%
    Nyarlathotep [\Ishnaruchaefir] rules in Sharnoth, beyond space and timeM in his gignatic ebony palace he awaits his second coming, served by his minions he broods and festers in blackest night.
    Let none meddle with spells and enchantements concerning him, for he is quick to trap the unwary.
    Let the ignorant beware, heed the \emph{Black Tome}, for terrible indeed is the wrath of Nyarlathotep.}
}





\subsubsection{Symbolic meaning of the name}
\target{Mirage Asylum symbolism}
\ps{\Ishnaruchaefir} choice of the name \quo{Mirage Asylum} is self-deprecating. 
\quo{Mirage} because it is an illusion of peace and isolation behin which he hides to avoid having to deal with his own self, his emotions and the world. 
And \quo{Asylum} because he sort of sees himself as a dangerous madman who must be sequestered and hidden away from the world, for both his sake and the world's. 
He has explored the universe to gain insight, but in a sense he has also fled out into the outer universe to escape having to face difficult questions and answers about himself. 
He likes to tell himself that he wants to learn the answers to the \hs{Aenigmata} of the universe in order to put his own Aenigma into a context and thus understand it better. 
But perhaps that is just an excuse, procrastination. 









\subsection{Politics}





\subsubsection{\Resphain resent it}
\target{Resphain resent Mirage Asylum}
\Ishnaruchaefir hid in his Mirage Asylum, a small secret Realm which only he knew how to find.
It was one of his most valuable resources.
It was how he has kept himself hidden and alive all these millennia.

The \resphain hated him for it. 
They saw it as cowardice. 
If he would only come out of his hole and fight like an honourable warrior, the \resphain could have dealt with him millennia ago (or so many felt). 

Of course he knew this, which was exactly why he kept his Asylum so secret and guarded.





\subsubsection{Threatened by horrors}
The Mirage Asylum is located in a place far removed from the beaten paths of \Miith, so the \resphain and other \Miithians cannot find it.
But this also means the Asylum is partially outside the protective \hs{Palisades}.
This means various mindless or intelligent creatures of the void can enter and attack them.

So the Asylum is regularly attacked by \hr{Horrors of the Void}{horrors of the void}.
Then the inhabitants need the magic that flows from the \dragons and their blood in order to repel the invaders and defend their home.
In especially bad cases, when large swarms of horrors attack, the \dragons themselves must arise and fight.
The inhabitants know this, so they worship the \dragons as their protectors.
And they willingly surrender some of their own as soul sacrifices to the \dragons when needed. 

The \dragons also feast on those horrors of the void when they can.
The depredations of the horrors is a necessary evil.

A pro of the Asylum's this is that they can grow exotic plants and things by drawing on some occult, dark energy streams that flow in from the void (but are blocked inside the Palisades), and combining these with the life-giving energy that flows from the Heart.









\subsection{Scenes}
Maybe have a scene where someone is dropped into the Asylum. 
They explore the place and lose tons of sanity points. 
It also gives some insight into how warped \ps{\Ishnaruchaefir} mind must be. 

















\section{Moons}
\target{Moons}
\target{moons}
\index{Moons}
Remember that the moons have their own Realms. 
The \moonwolves{} live there, as do other things. 

The moons play a large role in the \feud. There are \nexi{} there.

They have \hr{Astrology}{mystic astrological properties}. 

The Moon-Realms are \quo{sort of} part of the Realm of \Miith{}, although commonly not considered as such in everyday speech. 
They are still connected to the Heart of \Miith{}. 

Only at certain special times and in special places can one travel between \Azmith{} and the moons. 

One or both moons dwell great, white, loathsome maggots. 
Compare to the vision of the moon shown in \cite{KarlEdwardWagner:Bloodstone} when the eerie priestesses use their magic. 







\subsection[Dun]{\Dun} 
\target{Dun}
\index{\Dun}
\index{Gray Moon}
\Dun, called the Gray Moon, was the larger of \Miith{}'s two moons. 

It was well-known in mythology and legend that the planets and moons were worlds with their own inhabitants. 
There might dwell special \demihuman and \demiscatha races on the two Lunar Realms.

\Dun{} was a world of its own, full of life. 
Unlike \hs{Visha}, which was mostly desolate wasteland. 

Remember that the moons had low gravity. 





\subsubsection{Astrology}
In \hs{astrology}, \Dun{} is considered mostly benevolent. 





\subsubsection{Astronomy}
\Dun{} is about the same size as Earth's Moon. It is closer to \Miith{} than Earth's Moon is to Earth, so it appears larger in the sky. \Dun{} has a dark gray \colour. 

\Dun{} is large enough to cause a solar eclipse. It is too large to cause the \squo{ring} effect known from solar eclipses on Earth, however. \Dun{} is larger than the Sun in the sky, so during an eclipse, the Sun is completely swallowed. 

\Dun{} itself is eclipsed when it passes behind the shadow of \Miith{}. This is called a \Dun{} eclipse.\index{\Dun{}!\Dun{} eclipse} 

\Dun{} circles \Miith{} once every $23.5$ days, roughly corresponding to a month of the \hr{Vaimon Calendar}{\VaimonCalendar}. 

\Dun and Visha were both completely full on the very first day of the year 88 years before the beginning of the \hr{Runger war}{Pelidor-Runger war}. 

The \hr{Vorcanth}{\vorcanths} originally came from \Dun, but their realm was destroyed and they were driven out. 
Compare to \cite{HPLovecraft:TheDoomThatCametoSarnath}.  









\subsection{Visha}
\target{Visha}
\index{Visha}
\index{Pale Moon}
Visha, called the Pale Moon, is the smaller of \Miith{}'s two moons. 

It was well-known in mythology and legend that the planets and moons were worlds with their own inhabitants. 
There might dwell special \demihuman and \demiscatha races on the two Lunar Realms.

Where \Dun{} was teeming with life, Visha was mostly wasteland (in the \hs{Age of the Shroud} at least).
Visha's Realm was a mostly tranquil but eerie place. 
Haunted by terrible predators, the \vorcanths. 
It was also inhabited by ghosts, cruel \hs{cosmic gods} and perhaps the corpses of some \hr{Dead XS}{dead \xss}. 
The \hr{moongods}{\moongods} and the \hr{shugul}{\shuguls} hailed from Visha. 

Remember that the moons had low gravity. 

\citeauthorbook[p.175]{LinCarter:TheNecronomiconTheDeeTranslation}{Lin Carter}{
  The Necronomicon: The Dee Translation (part I.VII.III)
}{
  Aye, be thou warned, for in all such voyages and venturings of mind or soul or spirit there be very great and terrible dangers, by mortal men undreamt-of and unknown. 
  Beware then, lest thou penetrate too deeply into the blackest backward and depthless abysm of the womb of infinite time. 
  For beyond the very Beginning thereof, and on the Other Side thereof,there dwelleth That of which man suspecteth not; and there thou wilt find a strange and ominous Realm where hidden horrors lurk and naked Terror hunts unseen; which dim, uncanny bourn hath the seeming and the semblance of a pale, and grey, and indefinite shore, lapped by the sluggish waves of unmeasured and unthinkable Time.
  And it is eve there, in an awful Light that is beyond all darkness, amidst a profound Silence that shieketh beyond all sound, that \emph{They} slink and prowl in all their ghastliness, slavering with a loathsome and ana unspeakable hunger for all that is clean and whole and unsullied.
}

\lyricslimbonicart{Moon in the Scorpio}{
  A mirror blank ocean above me decoy.\\
  Superior forces that heal or destroy.\\
  Take me astray into the moonlight above
  through twilight eyes as a spectre shadow.
  
  In an atmosphere supreme\\
  forces dwell in domancy.\\
  The essence of its spirit is evil,\\
  as a curse upon thy name.
  
  Midnight is the shepherd of mysterious powers\\
  and moving shadows in the corner of the eye.\\
  Moon's blazing intuition\\
  contains what death requires.
  
  Behold the sky above \\
  when the moon is in the Scorpio.\\
  A cold bleak light
}





\subsubsection{Astrology}
Astrologically, Visha is considered malevolent and a bringer of ill omens. 

Visha is closely associated with the \hr{Vorcanth Matrix}{\vorcanth{} \matrix}. 





\subsubsection{Astronomy}
Visha is only half the diameter of \Dun{}. 
It is also farther away. 

Visha is not large enough to eclipse the Sun. It is much smaller, so when Visha moves in front of the Sun, it is visible as a dark hole in the Sun. This phenomenon is known as a \squo{Sunhole}\index{Sunhole}. 

Visha also sometimes casts a shadow on \Dun{}. This is called a \squo{\Dun{} hole}.\index{\Dun{}!\Dun{} hole}\index{Visha!\Dun{} hole} 

Visha itself is eclipsed when it passes behind the shadow of \Dun{} or \Miith{} iself. This is called a Visha eclipse\index{Visha!Visha eclipse}. 

Visha circles \Miith{} once every 50 days, roughly two months in the \hr{Vaimon Calendar}{\VaimonCalendar}.





\subsubsection{\Vorcanths{} and wolves}
The \hr{Vorcanth}{\vorcanths} dwelt on Visha. 

In symbolism, Visha was associated with wolves. 
The \quo{Mystic Wolves of the Frost-Moon} were well-known in mythology, but very few people knew what the \vorcanths{} were really like, or even that they existed. 













\section{\Neevrai}
\index{\Neevrai}
\target{Neevrai}
\Neevrai was a \hs{Telluric Realm} of \Miith, existing in parallel with \Azmith. 
It was the resting place of the \hs{Ark}. 









\subsection{Importance}
\Neevrai was generally considered a Realm of lesser importance by the immortals. 
Much less important than \Azmith. 

Unbeknownst to most, \Neevrai was the resting place of the \hr{Sethicus tomb}{tomb of \Sethicus}, which his followers thought long lost. 
This tomb would later become the \hs{Ark}. 









\subsection{Nations}
\Neevrai was dominated by a number of tribes or nations, each patterned after a type of animal. 
Both \humans and \scathae followed this custom. 

Compare to the kindens from \cite{AdrianTchaikovsky:ShadowsoftheApt}. 





\subsubsection{Crocodile Tribe}
The Crocodile tribe were \scathae.
They were a warrior race, like the Mantis-kinden from \cite{AdrianTchaikovsky:ShadowsoftheApt}. 
They were the most heroic of the tribes and eventually led the war against the evil White Foxes. 





\subsubsection{Skaven}
I should have an evil race of degenerate \humans that infest the world like a plague. 
They were a rot, a plague, a blight, an infection of evil. 

See also the section about \hr{Lictor}{Lictor-like humanoids}.

Compare to the Skaven from \cite{RPG:Warhammer}. 





\subsubsection{White Fox Kingdom}
The White Foxes were \humans. 
They had a kingdom ruled by a czar.
They were evil. 

The Foxes had a religion that was a variant of \hs{Iquinianism}. 
They dominated part of \Neevrai and waged wars of conquest to bring all of the Realm under their heel. 

Compare them to the Wasp Empire from \cite{AdrianTchaikovsky:ShadowsoftheApt}. 
















\section{\Nithdornazsh}
\target{Nithdornazsh}
\target{Nith'dornazsh}
\index{\Nithdornazsh}
\Nithdornazsh{} was the ancient \draconic{} fortress of the \dragonking{} \Nexagglachel. 

It lay near where \Malcur lies today, because it was built upon the ruins of the same \hr{Wild}{\Wylde}{} fortress of which \Malcur's foundations are also a branch. 

\Nithdornazsh{} has been sealed off from \Miith{} for 10,000 years, maybe 20,000. It's so old that even most Sentinels and Cabalists don't know about it, but \Malcur is connected to \Nithdornazsh{} and has always been. Because of the Shroud, it is hard to use this connection for anything. 

At the time of \TwilightAngelRememberEmph, \HriistD{} has a plan to bring \Nithdornazsh{} back. See, after the death of \Nexagglachel, \Nithdornazsh{} fell into disuse and disrepair\dash all \dragons{} abandoned the place because of the painful memories of the king's fall. In fact, it has withered and died. But its soul lingers and can be reborn. It just needs to feed. Resurrecting a living fortress has very rarely been done, so no one suspects that this is what \HriistD{} is up to. 

For \HriistD, this project is not only a part of the war, but also a prestige project in \honour of the memory of his brother. 

The gambit to bring \Nithdornazsh{} to \Miith{} is a blatant breach of the \charade\dash the unspoken mutual agreement that the Sentinel-Cabal conflict should be kept underground and hidden from mortal eyes. It is uncharacteristic of \HriistD{}, who usually does not cheat this much. But for the sake of his brother, he is willing to do a lot. 

The whole \Nithdornazsh{} gambit is \ps{\HriistD}{} personal project, secret and hidden from his fellow Sentinels. 
Officially, he merely intends to invade Pelidor\dash partially to gain a foothold in central \Velcad{}, and partially as a part of his plan to awaken the \Haskelek{} (more about the Haskelek in \CarzainWithRedcorBook). 
One of their targets is the \hs{Ghost Tower} in northern Pelidor, which is a potent \nexus{} point in its own right. 
\Nzessuacrith{} is personally responsible for this last part. 

But unbeknownst to \Nzessuacrith, \HriistD{} has his own ideas and intends to use her attack as a distraction to further his own plan. He wants to resurrect \Nithdornazsh{} in \Malcur, transforming \Malcur into a reincarnation of the \draconic{} fortress. 









\subsection{Appearance}
\Nithdornazsh{} was cyclopean, gigantic. 
It was built to accomodate some of the vastest \dragons{} that ever lived. 
It conformed to the aesthetics of \hr{Draconic architecture}{\draconic{} architecture}. 

In the centre of the city there lay a temple, in which was kept a powerful relic: 
A mummified claw of \TyarithXserasshana. 

When it was summoned \Nithdornazsh{} was guarded by \hs{undead machines}. 

See also the sections on \hr{Resphan architecture}{\resphan architecture} and \hs{dark ancient cities}. 





\subsubsection{End of the world}
In \Nithdornazsh, when it arose in \Malcur, there were \quo{holes} where one could see out of the Shroud, into the Beyond, to the \quo{end of the world}, where the recognizable world gave way to unshaped chaos.
This was a \hr{XS}{\xsic} Realm. 

\citetitle[p.45--47]{RHCharles:BookofEnoch}{The Book of Enoch XVIII.11--}{
  And I saw a deep abyss, with columns of heavenly fire, and among them I saw columnds of fire fall, which were beyond measure alike towards the height and towards the depth.
  \\
  And beyond that abyss I saw a place which had no firmament of the heaven above, and no firmly founded earth beneath it: there was no water upon it, and no birds, but it was a waste and horrible place. 
  \\
  I saw there seven stars like great burning mountains, and to me, when I inquired regarding them, 
  \\
  The angel said: 
  This place is the end of heaven and earth: this has become a prison for the stars and the host of heaven.
  
  \ldots
  
  And I proceeded to where things were chaotic.
  \\
  And I saw there something horrible:
  I saw neither a heaven above for a firmly founded earth, but a place chaotic and horrible.
  \\
  And there I saw seven stars of the heaven bound together in it, like great mountains and burning with fire.
  \\
  \ldots
  \\
  And from thence I wence to another place which was still more horrible than the former, and I saw a horrible thing: a great fire there which burnt and blazed, and the place was cleft as far as the abyss, being full of the great descending columns of fire; neither its extent or magnitude could I see, not could I conjecture. 
}









\subsection{History}
\target{Nithdornazsh was Nexagglachel's tomb}
\Nithdornazsh was the old tomb of \Nexagglachel. 

When \Nexagglachel awakened he turned \Nexagglachel into his citadel. 

Out of respect for him it became deserted after his death. 
It became a necropolis in his memory, instead of being taken over and used for some other purpose. 
Thus the \resphain forgot about it. 
It was deep in \draconic territory and seemed to have little strategic significance, so the \resphain felt they had better things to do that attempt to conquer or raid it.
Until, finally, \Secherdamon felt it was time to revive it. 















\section{\Nyx}
\target{Nyx}
\index{\Nyx}

The purpose of \Nyx{} is to serve as a conduit between \Erebos{} and \Miith{}. 
\Nyx{} was initially connected to \Erebos{}, but the \dragons{} successfully sealed it off from \Erebos{}. 
Today the \banes{} use \Nyx{} as a base of operations and a place to hide away from the \dragons{}. 



\lyricsduana{thisistheend}{This is the End}{
  I dream of a host cockroaches \\
  crawling, swarming over charred flesh \\
  up bitter blackend walls \\
  in sewer pipes now purgd \\
  the cries of many amplifd \\
  then swallowd by silence \\
  echoes of a lost world \\
  of a foul smoke that lingers like stale breath \\
  of steel/glass/concrete fusion-statues \\
  rising high in a pale and infinite twilight \\
  shining beacons like bleachd bones 
}

Whenever I write about \Nyx, be sure to also read about \hr{Erebos}{\Erebos}.









\subsection{Demographics}
Each tower is about 300 m in diameter.
That is about 70.000 m${}^2$.
The towers cover about 1\% of the surface of \Nyx on average.

A tower can support about 100 \humans.
Each 100 \humans can support about one \resphan. 
This means one \resphan per 7 km${}^2$.

These densities varied from period to period. 
When technology was higher, the \resphain could better fight back monsters and thus claim a greater area.









\subsection{Denizens}
Denizens of \Nyx included:
\begin{itemize}
  \item \Flyingpolyps.
  \item \Noggyaleth (maybe).
  \item \Ophanim. 
  \item \Resphain. 
  \item \Umbrae.
  \item Weavers.
\end{itemize}





\subsubsection{The worms that walk}
There were ghastly things that dwelt in the spires not claimed by the \resphain, and even on the uninhabited levels of the towers the \resphain otherwise called their own.
These things were rarely seen. 
It was unknown quite how dangerous they were, but allegedly they had been known to kill \resphain, so everyone feared them. 

Things have learned to walk which ought to crawl. 

\citeauthorbook[p.166]{RamseyCampbell:TheInhabitantoftheLake}{Ramsey Campbell}{%
  The Inhabitant of the Lake
}{
  There are other things, too\dash the race \quo{of which Vulthoom is merely a child}\dash the source of vampires\dash and the pale dead things which walk black cities on the dark side of the moon...
}









\subsection{Ecology}





\subsubsection{Energy source}
\Nyx was connected to \Erebos at the bottom and to \Miith near the top.
\Nyxian plants got most of their physical energy from their roots, and some small amount from sunlight. 
They got some amount of lifeforce from the Heart of \Miith. 
That was why they grew upwards: 
They needed physical energy from the bottom and lifeforce from the top. 
(They could not live on the lifeforce of the \hr{Heart of Erebos}{Dark Heart of \Erebos}, for it was an undead husk and could support no wholesome life.)





\subsubsection{Fauna and flora}
\target{Nyxian plants}
There were many plants in \Nyx. 
They grew upwards to escape from the deep. 

The plants were full of liquid. 
\Resphain and other \Nyxian life got most of their water from plant juices, and a little from rain water.

The plants were frequently overharvested and died. 
Then the \resphain and \humans had to venture deeper down into the lower levels in search of food. 

Plants were \hr{Plants in Merkyran religion}{important} to the \Merkyran religion. 

The \resphain used magic to keep the plants alive.
They hunted herbivorous animals, partially to spare the previous plants and partially for meat to eat. 

Sometimes the \resphain ventured into abandoned towers in search of food. 
This could be very profitable, but also very dangerous.

The \resphain cultivated parasitic plants that grew on the bigger plants and sucked nourishment from them.
Some of these parasite plants were very tasy and nourishing to humanoids.

There were wild animals for the \resphain and \humans to hunt. 





\subsubsection{Natural resources}
\target{Resources in Nyx}
\Nyx was poor in natural resources. 
There was no iron or other high-quality metals.
The tower walls themselves were made of exotic metals that required high technology to forge.

There was no coal or gunpowder.

There were bronze mines in the deep. 
\Humans would often be put to labour in these mines, with a few \resphan warriors serving as escort to protect the \humans from monsters.

There was noble metals and gemstones, and horn for bows. 









\subsection{Endless dark city}
\target{City of Nyx}
\Nyx{}, created as a shadow of \Erebos, is a world of immense, dead, decaying cities. 
Buildings of stone and metal reach thousands of metres up into the sky. 
At times a \traveller finds himself on a ledge or causeway, in which case the ground might be hundreds or thousands of metres below\dash the buildings seeming to go on forever both up and down. 

The buildings are mostly empty and decaying, and the whole city feels like a necropolis. 
Monstrous scavengers lurk in the corners, and you may encounter the occasional \bane{} or \resphan{} or a \human{} slave. 
There are many buildings who were once inhabited by \resphain{} but have since been abandoned as their population dwindled. 
\Nyx{} is full of dangerous monsters, always ready to reclaim an undefended tower when the \resphain{} leave. 

The buildings in \Nyx{} are tall, spindly, emaciated. 
And dead; macabre corpses or skeletons. 
Compare them to the bloated \hr{Living buildings of Machai}{living buildings of \Machai}.

An average tower was about 300 m in diameter and had 20 habitable levels and another 20 levels below that which were semi-safe to go to but not habitable.
An average level was 10 m tall. 

\target{Hold}
\index{hold}
The towers were clustered together in holds. 
Each hold could consist of 5-30 towers. 
The holds stood as islands.
Between the holds were great ocean-like voids of empty space. 
Within a hold, each tower stood about 3-4 tower-widths from its neighbour. 
The towers took up about 5\% of the total land area of \Nyx.

See also the sections on \hr{Resphan architecture}{\resphan architecture} and \hs{dark ancient cities}. 
For areas and population, see the section about \hr{Resphan demographics}{\resphan demographics}.

In the time of \Merkyrah, \Nyx was shining and beautiful. 
It was \hr{Resphain devastate Nyx}{devastated in the rebellion}, then \hr{Nyx rebuilt}{rebuilt under \Morcariel} and finally \hr{Ishnaruchaefir devastates Nyx}{devastated again by \Ishnaruchaefir}. 

\lyricsbs{Arcane Wisdom}{Misanthropic Horror Magnified}{
  Senseless oblivious towers,\\
  monuments to absurdity.\\
  Oceans of ignorance.\\
  Humanity at its cruel best! 
  
  When all noble values are (shamelessly) inverted, \\
  and brainwashed minds endure.\\
  When animal essence is forgotten. \\
  Misanthropic horror\ldots{} magnified.
  
  Decadence is all I see. \\
  Triviality lies before me. \\
  Nihilism\dash the only solution. \\
  Misanthropy\dash the final escapism. 
  
  Your world sinks to vomiting proportions. \\
  Visions of decay spring everywhere. \\
  Entire lost civilizations \\
  succumb to a \quo{worldly} oddity. 
  
  Woe and torment be behind me\ldots{} \\
  the maze of such reality. 
}

\lyricsbs{Exmortem}{Grand Dome of Destruction}{
  Naked chambers so cold and grim.\\
  A last gasp for air.\\
  A smell of funerals to come.\\
  Icecold Ugliness.
  
  Here I saw the Lord of Death,\\
  and his eyes flashed with rage.
  
  Gruesome Icons. Demonic Tokens. \\
  Images of a defuct future.\\
  Funeral fests. Nocturnal Chill.\\
  A mirror of the underworld.
}

\Nyx was older than the \resphain.
It was clearly originally built by alien monsters.

\citeauthorbook[p.178,199]{RPG:CallofCthulhu:BeyondtheMountainsofMadness}{%
  Charles and Janyce Engan%
}{%
  Beyond the Mountains of Madness%
}{
  The first impressions explorers receive when standing in the City are ones of age and alienness, and a sort of dumb bleak despair.
  The eternal wailing song of the high peaks underlines the dreariness and oppression of this huge and ancient monument.
  The stones are huge, dark and heavy; doorways and chambers are wrongly sized for \human eyes and \human bodies.
  The ruddy light of the sun makes shadows thick and leaches out any semblance of hue.
  
  \ldots
  
  Beyond the valley, behind the great roiling wall of blackness and storm clouds, rise the fantastic spires and peaks of the Western Range.
  Even the mighty Mountains of Madness pale in comparison\dash these new mountains are higher and more angular still, reaching upward into the heavens like slate-black fangs.
  Their highest peaks shine with the clear brilliance of sunlight unmuted by the thickness of the air.
  It seems as if the mighty range thrusts up into the very edge of space itself. 
  
  The most alarming thing about the newly visible peaks is not their height but a terrible sense of design in their arrangement.
  Beyond the curve of the storm, the mountainsides erupt stark and black and bare, without the caves and encrustations that mark the eastern range\dash but the tallest, closest peaks have about them a disturbing regularitym an evenness of spacing and of shape that suggests a malignant and powerful guiding hand.
  They rise about the cloudy maelstrom like gigantic fingers, cupping in between them as if to guard it from harm\dash or to crush it. 
  
  \ldots 
  The sense of pattern and structure in the cliffs and spires is profoundly unnerving\dash as if some terrible purpose would be revealed if the viewers could only understand. 
}





\subsubsection{Chimneys}
Some places in \Nyx{} there are tall chimneys spewing out yellow and brown smog. 

The yellow is more repugnant than the black that is otherwise prevalent in \Nyx. 
The black is at least pure darkness. 
The yellow, on the other hand, is pure poison, rot, disease, corruption. 





\subsubsection{Dust-like moss}
The walls of \Nyxian{} buildings are often covered in a layer of what looks like coarse dust or ash. 
It is actually alive, a moss-like thing. 





\subsubsection{It was once nicer}
\Nyx{} was once far more civilized. 
Back in \ps{\Merkyrah}{} days, and even after that. 
\Nyx{} was not a ruin, but beautiful and thriving.

Back then, there were many \resphain, and they had tons of slaves and could maintain the gigantic city. 
But after the \secondbanewar{} and the \resphanwars, the \resphain{} and their slaves were decimated. 
There were not enough of them to maintain the enormous cities, and they fell into disrepair.
Today \Nyx{} is a decaying ruin, overrun by hideous monsters. 
Only small parts of the humongous buildings are inhabited. 









\subsection{Geography}





\subsubsection[The Abysses of Uluor]{The Abysses of \Ullor}
\target{safe zone}
Only the tops of the spires were inhabitable. 
This top layer was called the \quo{safe zone} or \quo{safe belt}. 
This layer comprised \Nyx proper.
The deeps below, called the Abysses of \Ulorr, were monster-haunted and too dangerous to venture into. 

In \Nyx, you can sometimes feel (or imagine that you feel), infinitely far below the endless city of steel and stone, a living planet. 
Or, rather, a once-living planet, now writhing in its last convulsions. 
\FatherErebos{} was once alive and vibrant, savage and full of energy like \Miith{}. 
But the \banes{} conquered the planet and changed that. 
With their endless greed and cruelty they have sucked all life out of their homeworld. 
All that is left of their creator is a tortured, enslaved, withered husk, weeping in its endless torment and grief, weeping over the betrayal by its own children. 

\Nyx{} is not a living world and has no heart. 
It is merely an artificial shadow world built on top of \Erebos. 
The Abysses of \Ullor were what separated \Nyx from \Erebos. 

Occasionally, when people are cast into \Nyx, they find that the earth is covered in bones and corpses\dash faces and skulls grin back with their toothy mouths and empty eyes, yet the images are so flighty and dreamlike, the faces flowing into one another, that the viewer doesn't recognize any, nor is he even sure what species the skulls were. 

\target{Nyx is above Erebos}
\Nyx{} actually (sort of) exists in the skies high above \Erebos. 
\Erebos{} is full of huge towers, many thousands of metres tall. 
The towers are thicker nearer the ground, but up in \Nyx{} they are spindly, emaciated, skeletal. 

The deep below \Nyx{} continues all the way down to \Erebos, but it is monster-infested and dangerous and not a viable route for anything, not even legions of \banes. 
The \CrystalSphere{} makes it even worse. 





\subsubsection{The Chasm of \Oggra and the \Hyardes Towers}
See the sections about \hr{Oggra}{\Oggra} and \hr{Hyardes}{\Hyardes}. 





\subsubsection{Dead towers}
It would seem that the \glowmoss had some sort of symbiotic relationship with \resphain and other \quo{wholesome} beings. 
For the \resphain noticed that whenever a tower was taken over by \hs{Nether Ones} or \hr{Umbra zombies}{\umbra zombies} or some other awful menace, the \glowmoss would die and the tower would become dark.

There were attempts to revive dead towers, but without luck. 





\subsubsection{Lower Halls}
\target{Lower Halls}
\index{Lower Halls}
The Lower Halls were the regions below where \resphain did not dwell.
The \hs{Nether Ones} dwelt here.





\subsubsection{Upper Halls}
\target{Upper Halls}
\index{Upper Halls}
The Upper Halls were those levels where the \resphain lived.
In most places the Upper Halls were the top levels of the towers, but there were a few very high towers that had very high levels where the \resphain did not dare go.








\subsection{Holds}





\begin{gloss}



  \gitem{\Cathedon}
  \target{Cathedon}
  The capital hold of \hr{Merkyrah}{\Merkyrah}, and later of \CiriathSepher. 
  It lay near \Hyardes. 



  \gitem[Hyardes]{\Hyardes}
  Once the capital hold of \hr{Tarcharos}{\Tarcharos}; later a haunted necropolis.
  It was known to all \resphain that the \resphan race had originated in \Tarcharos, although the exact tower was unknown. 
  (It was really \hr{Jazerubel}{\Jazerubel}.)



  \gitem{\Surammas}
  \target{Surammas}
  The capital hold of \Mystraacht. 



\end{gloss}









\subsection{Illumination}
The \resphan{} safe zone is lit mostly by the flashes of lightning from \hr{Thunder in Nyx}{the thunderclouds far below}. 
There is also starlight from above, \hr{Sky in Nyx}{but no moon or sun}. 
Not since the \hs{Murder of the Dawn}, anyway. 





\subsubsection{\Glowmoss}
\target{Glowmoss}
\target{Glow-moss}
\index{\glowmoss}
When the \resphain{} needed light, they used a luminescent moss- or coral-like lifeform that was native to \Nyx. 

The \glowmoss{} grew in the deep, drawing its nourishment from the thunderstorms. 
The \resphain{} had slaves that went down to harvest it. 
It was then put into glass lamps. 

It gave off a dull blue light. 

When eaten, \glowmoss was a strong but dangerous narcotic, easily fatal to mortals and very unhealthy to immortals. 





\subsubsection{Lighting of the towers}
\Nyx was dark, but the \resphan buildings were well-lit. 
In the days of \Merkyrah as well as afterwards. 
This was because \hr{Umbrae dislike light}{\umbrae disliked light}, so a brightly lit building was slightly less likely to be attacked by an \umbra.

The interiors of buildings were usually beautifully decorated with \hr{Resphan crystal technology}{crystal} and \hr{Glowmoss}{\glowmoss} as well as darker things. 

For visuals, see the section on \hr{Beauty of dark ancient cities}{the beauty of dark ancient cities}. 









\subsection{Monsters}
\target{Nyx monsters}
\Nyx{} was full of nightmarish horrors. 
The \resphain{} prey on them and eat them. 
In the days of \Merkyrah, the \resphain{} hunted some monsters to extinction and thinned the populations of others. 
But in the millennia after the \hs{Murder of the Dawn}, most of the monster populations (that were not completely extinct) recovered, now that the \resphain{} themselves were declining in population for \hr{Heart weakened}{other reasons}. 





\subsubsection{Dark gods}
\target{Gods in Nyx}
There dwelt some dark gods in \Nyx{} that were independent of the \banes. 
\Daggerrain{} tried to keep them out of his pocket dimension, but he could not stop them all, and some slipped in. 

Some of these gods came to be \hr{Early diabolist Resphain}{worshipped by the early non-\Merkyran{} \resphan{} tribes}. 

Some of them were later \hr{Bael'Zerach diabolism}{worshipped by some of the \Baelzerach}. 





\subsubsection{Lurkers in the deep}
There dwell monsters in the deep underneath the cities of \Nyx. 
The most well-known of these are the \mothlain, also called \hr{Nether One}{Nether Ones}. 

There are also other monsters that lurk in the deep. 
No one (except maybe the \hr{Banelord}{\banelords}) know what these monsters eat to live. 





\subsubsection{Spider-like monsters}
In \Nyx{} and/or \Erebos, I should have a monster that looks like a giant, monstrously misshapen black spider. 

Inspired by the cracks and blotches in the display on Jeppe's cell phone. (Jeppe is a gamer whom I met at Bjarke Lassen's workshop in February 2008.) 









% \subsection{\Similuth}
% \Similuth{} is the part of \Nyx{} that the \resphain{} inhabit. 
% \Merkyrah{} lay here. 









\subsection{Parasitic Realm}
\target{Nyx is a parasite Realm}
\Nyx{} is a parasitic sub-Realm. 
In a sense, it is a cancerous growth on \Miith{} that cannot sustain itself but must suck nourishment from the rest of \Miith{}. 





\subsubsection{\Nyx{} was killed}
\target{Nyx was killed}
Perhaps it was different under \hr{Merkyrah}{\Merkyrah}, which lay in \Nyx. 
Perhaps the rebels had to somehow \quo{kill} the Realm in order to establish the connection to \Erebos{} that they needed, so that they could draw on \Erebean{} power and summon \banes{} and monsters. 
Perhaps this was the \hs{Murder of the Dawn}. 




\subsubsection{It is hard to feel the Heart}
\target{Nyx is far from the Heart}
\Nyx{} is farther removed from the Heart of \Miith{} than the rest of \Miith{}. 
The Heart cannot be felt to the same extent in \Nyx{} as in \hr{Tembrae}{\Tembrae} and its fragments. 
Therefore, \Nyx{} feels barren and dead to some. 

\lyricsdimmuborgir{Stormbl\aa{}st}{
  Undring og angst samler seg i natten,\\
  i m\oe{}rket som ruver om spiret.\\
  For ingen dag kan veien hit.\\
  Intet lys kan luske frem.
}









\subsection{Portals to \Azmith}
\target{Portals to Nyx}
There were regions in \Nyx where the Realm was close to that of \Azmith. 
These regions functioned as portals where it was possible to cross over between the two Realms. 









\subsection{Sky}
\target{Sky in Nyx}
The sky above \Nyx{} similar to that above \Azmith.
But there is no sun and no moons. 
And you can see the black stars of \Erebos. 





\subsubsection{Black stars}
\target{Black stars of Nyx}
\target{black stars}
\index{\Nyx!black stars}
Some of the stars in the \Nyxian{} sky are black. But you can clearly sense that they are there. Somehow they stand out from the blackness of the sky, by virtue of their unearthly radiance. 

Compare to the various stories in \authorbook{Robert W. Chambers}{The King in Yellow}.

\lyricsbs{Robert W. Chambers}{The Repairer of Reputations}{
  \ta{%
    I\ldots{} wept and laughed and trembled with a horror which at times assails me yet. 
    
    This is the thing that troubles me, for I cannot forget Carcosa where black stars hang in the heavens; where the shadows of men's thoughts lengthen in the afternoon, when the twin suns sink into the Lake of Hali; and my mind will bear forever the memory of the Pallid Mask.%
  }
}

The black stars are a characteristic of \Nyx{} and \Erebos. They are powerful and terrible energy sources that \quo{shine} through from \Erebos{} with their dark light. 

When you pierce the Shroud into \Nyx, the black stars always become visible. They are often one of the first things you notice\dash so strong is the menacing power they exude\dash and they are horrible to behold. 

The black stars are mentioned in \emph{\hr{Wanderers in Darkness}{\WanderersInDarkness}}. 







\subsection{Statues of caterpillars and wormsirens}
\Nyx is full of statues of vast caterpillar-like monsters. It is believed that they are likenesses of the extinct founders and builders of \Nyx, called the Ancient Nyxians. 

It is suspected by some that the \hs{wormsirens} are degenerate descendants of the caterpillar-like Ancient Nyxians. 










\subsection{Towers}
Every inhabited tower in \Nyx had a name. 





\begin{gloss}



  \gitem{\Jazerubel}
  \target{Jazerubel}
  The tower in \hr{Hyardes}{\Hyardes} where the first \resphan and \nephil refugees were deposited.
  \Semiza lay entombed in the cellars of this tower. 



  \gitem{\Lamaruch}
  \target{Lamaruch}
  An important \Mystraacht tower.



  \gitem{\Roshmal}
  \target{Roshmal}
  An important \Mystraacht tower.



  \gitem{\Sherem}
  \target{Sherem}
  A tower in \Merkyrah where \hr{Monara}{\Monara}'s rebels once dwelt.



  \gitem{\Shaiphon}
  \target{Shaiphon}
  The mighty tower where \hr{Azraid}{\Azraid} dwelt.



  \gitem{\Tebethal}
  \target{Tebethal}
  The tower in which \hr{Teshrial}{\Teshrial} and \hr{Menessiaraid}{\Menessiaraid} dwelt. 



  \gitem{\Tirunad}
  \target{Tirunad}
  \Tirunad was the tower in \hr{Cathedon}{\Cathedon} in \hr{Merkyrah}{\Merkyrah} in which lay the chamber of \hr{Hoshiabalon}{\Hoshiabalon} where the \hr{Merkyran God}{One God} allegedly \hr{Berugiel's revelation}{revealed his words} to \hr{Berugiel}{\Berugiel}.



  \gitem{\Tzarubal}
  \target{Tzarubal}
  An important \Mystraacht tower.



\end{gloss}









\subsection{Weather}





\subsubsection{Thunder}
\target{Thunder in Nyx}
Thunder is common in \Nyx.
It is seen not above, but in the deeps below the towers where the \resphain live. 
\hr{Glowmoss}{\Glowmoss} gets its nourishment from the thunder. 















\section{The Realm of the Deep}
\target{Deep Realm}
The \quo{Realms of the Deep}, also known as the \hr{Aquatic Realm}{Aquatic or Pelagic Realms}, were the seas and the underseas, where the \nagalords{} and the \nagae{} dwelt.

\lyricsbs{Ancient Rites}{Het Verdronken Land van Saeftinge}{
  Here one can hear \\
  the call of the sea, \\
  while a deadwhite moonlight \\
  is creating the ultimate unlight. \\
  Or at night, or at night\ldots{} 
  
  O sad and beautiful night, \\
  filled with melancholy. \\
  When the silent dark waters \\
  are inviting the lonely souls. 
  
  Of mourning lost ones\ldots{} like me. \\
  Of mourning lost ones\ldots{} like me. \\
  Like me.
}















\section{The Sea}
\target{The Sea}
\target{Sea}









\subsection{Denizens}





\subsubsection{Myths about merfolk}
There were \hr{Myths about Nagae}{myths about merfolk}.
They were really \nagae.









\subsection{In ancient times}
In ancient times there was plenty of ships, seafaring and naval battles. 

The \hs{Deep Realm} and the \hr{Fragments of Tembrae}{fragments of \Tembrae} are closely intertwined.

Even in the time of the great \dragon-\resphan{} wars there was lots of seafaring and ships. 
The seas, as well as the air above them, were dangerous and treacherous, even for \dragons. 
There dwell cruel gods. 
Possibly the \hr{Kraken}{\krakens}, rivals of the \hr{XS}{\xss}.

\Dragons{} and \resphain{} could fly above the seas, but it was hard and exhausting to fly long distances, especially through the often fierce ocean winds. 
Besides, \dragons{} and \resphain{} were built with the \emph{ability} to fly, but they were not built for a \emph{life} in the air. 
So it was infeasible to fly all the way across the sea, or to fly around patrolling for longer periods of time, or fly to and from naval battles without landing. 

So it was easier and safer to sail \emph{on} the Sea (in sufficiently badass ships) than it was to fly above it for longer periods. 
So they did. 
Of course, in battles they often flied around to fight, but after the battles they would settle down on their ships. 

Besides, many immortals could not fly: 
\QuilJaaran, \aryothim, \vorcanths, \bezedeth. 

The master races, of course, built super-powered mega-ships, forged by great sorcery from exotic materials and nigh-indestructible. They could be humongous in size, several hundred metres long. 

The \resphain used ships a lot.
Sailors were often \bezedeth, since they could not fly.
Purebloods had great pride in their wings, so they did not like to admit the fact that they were dependent on ships, and so they did not become professional sailors.

The \dragons depended less on ships because \dragons were great swimmers, so if they tired of flying they could just swim.







\subsection{Circumventing the sea}
If you're cool, you can take a route through the Beyond around the seas and bypass them entirely, so you don't have to cross the sea at all. 
This is the long way around, though. 

An an example, one might go through \ps{\QuessanthIshnaruchaefir} \hs{Mirage Asylum}. 









\subsection{Placating the powers of the sea}
When sailing on the sea, you need to have priests/mages along to placate gods and spirits and keep at bay sea monsters and \Wylde{}-related natural disasters (storms and stuff). 
All cultures have some kind of sea gods to help them cross the sea. 

At times, humanoid sacrifices are needed to placate the wicked spirits of the \hs{Deep Realm}.

Remember to have gruesome things lurking in the deeps. 

\lyricsbalsagoth{Atlantis Ascendant}{This terror in the Astral Seas\ldots{}}









\subsection{Ships and navies}
In the \hs{Shrouded Realms}, mortal navies tended to be small. 
\hr{Wood is precious}{Wood was a precious resource}. 

Warships would be enchanted to protect them from fire and other calamities. 
A big ship was too precious to lose, so it had to be protected well. 
They were too expensive to let them go to waste. 

The mightiest ships were indestructible dreadnoughts, hundreds if not thousands of years old. 















\section{\Tembrae}
\target{Tembrae}
In the days of the \secondbanewar, \Tembrae{} was the \quo{main} Realm, the principal part of \Miith{}. 
\Machai{}, \Nyx{} and the Deep Realm were its neighbouring Realms. 








\subsection{Fragmentation}
\target{Fragments of Tembrae}
After the \SecondShrouding{}, \Tembrae{} was fragmented into a number of smaller Realms. 
\Azmith{} is one of them. 

\lyricsxkcd{240}{
  I had a dream that I met a girl in a dying world. 
  
  It was all coming apart. \\
  Hairline cracks in reality widened to yawning chasms. \
  Everything was going dark and light all at once, and there was a sound like breaking waves rising into a piercing scream at the edge of hearing. \\
  I knew we didn't have long together. 
}









\subsection{Less land than before}
There is not as much habitable land in \Tembrae{} (i.e., all the fragments combined) than there used to be. 
This is because \hr{Heart weakened}{the Heart is weakened} and cannot easily restore the land after all the destruction wrought on it. 
For example, the entire Realm that once housed \hr{Cuezca}{\Cuezca} is now uninhabitable, dreary wasteland. 















\section{The Voids Between the Worlds}
\target{Horrors of the Void}
The voids between planets were filled by all sorts of horrors.
When the \voyagers settled \Miith hundreds of millions of years ago, they \hr{Voyagers erect Palisades}{erected dimensional barriers around the Realms of \Miith} to keep it safe from these horrors.

In a sense, the \xss were some of these \quo{horrors} that the \voyagers feared.
They were some of the greatest \quo{horrors}.

The voids between the Realms were fucking dangerous. 
There existed blind, mindless, slavering things of which even the \resphain lived in fear. 
Whenever \resphain had to travel between the Realms, they could not just travel through the Beyond.
They only \travelled along well-known pathways and had to have magic to keep them safe (more cosmic versions of the \eidola and \wylde charms that mortals used).

According to some myths, the horrors were \hr{Horrors are False Death}{manifestations of rot and decay, of False Death}. 

\citetitle[Lords of Cthul]{Misc:Monsterpocalypse}{Monsterpocalypse}{
  Beyond the veil of our own universe exist myriad dimensions teeming with unknown threats. 
  For eons, practitioners of the occult have dared to peel back that fragile layer that separates our world from a vast realm of darkness to glimpse the ancient powers that lurk within. 
  From time to time, those horrors have slipped through the void.
}

\citebandsong{DarkEmpire:DistantTides}{Dark Empire}{Distant Tides}{
  The deep of the gaping void\\
  Will surely swallow you whole\\
  Return is no option, only death\\
  By the shadows of the dark
}

\citebandsong{BlindGuardian:IFTOS}{Blind Guardian}{I'm Alive}{
  Outside they say death is waiting\\
  But it creeps down through the shaft\\
  Finds pleasure in our helpless fear\\
  Fills empty rooms with morbid thought\\
  They've locked the door and hold the key\\
  Sitting beside you\\
  When silent screams\\
  Changing my mind and dreams\\
  Oh, it's never ending
}

\citemovie{IntheMouthofMadness}{In the Mouth of Madness}{
  Trent stood at the edge of the rip, stared into the illimitable gulf of the unknown\ldots{} the stygian world yawning blackly beyond.
  Trent's eyes refused to close.
  He did not shriek, but the hideous, unholy abominations shrieked for him.
  As in the same second he saw them spill and tumble upward out of an enormous, carrion black pit choked with the gleaming white bones of countless unhallowed centuries; he began to back away from the rip as the army fo unspeakable figures twilit by the glow from the bottomless pit came pouring at him towards our world. 
}









\subsection{Travel between Realms}





\subsubsection{Only at certain points}
It is not feasible to travel between Realms at any arbitrary point. 
It is only safe to travel from the Immortal Realms into the Shrouded Realms and back at certain special points. 
You have to be aligned with a \matrix{} that has a connection/alignment with some \nexus{} point in the Shrouded Realm in question. 

In other places, the voids between the worlds are full of monsters and storms, making them terribly dangerous even for immortals. 
Only really badass immortals (like \QuessanthIshnaruchaefir) dare to travel the voids. 
Most immortals travel along the safe \quo{caravan routes} mapped out by the \matrices. 





\subsubsection{Submerging and surfacing}
\target{submerging}
\target{surfacing}
\index{submerging}
\index{surfacing}
It is possible to stray close to and far from the border between Realms. 
To \quo{surface} is to move closer to a given Realm, and to \quo{submerge} is to move further away from it. 
Surfacing and submerging is often seen relative to the Shrouded Realms. 

In order to see into the other world you have to move close to the border, and that makes you easier to detect. 

The Shrouded Realms are much smaller and narrower than the Immortal Realms, because of the Shroud. 
When one moves \quo{sideways} through the Shroud in a Shrouded Realm, one gradually submerges into an Immortal Realm. 

The Immortal Realms are full of monsters. 
When you submerge out of a Shrouded Realm and into an Immortal Realm, you risk running into the monsters that dwell there. 
The monsters cannot easily enter the Shrouded Realms because the Shroud keeps them out. 
They are bound to their homes because their Shrouded minds are accustomed to living there. 
Just like mortals, they cannot see into other Realms. 

But if some hapless mortal from a Shrouded Realm ventures outside, the monsters are free to eat him. 














\section{The \Wylde}
\target{Wild}
\target{Wylde}
%\subsection{\Wylde{} versus civilization}
The \Wylde{} is the uncharted wilderness between cities and villages, inhabited by dangerous beasts and monsters. 

Humanoids dwelling in the wild are labelled as savages and barbarians. 

A subtheme of the whole story is that of \Wylde{} vs. civilization. Neither of them is \quo{good}, both are evil in their own way. 

The \Wylde{} is the true, natural state of the world (the Realm of Beasts at least). 
It is a state of some sort of chaotic balance, the state to which the Realm always seeks to revert. 
But the \Wylde{} is also cruel and violent, a world of bloodshed, conflict and competition. 
Creatures of the \Wylde{} may live in packs or even tribes, but these groups are xenophobic of outsiders, and oppressive and merciless towards insiders. 

Civilization, ie., towns, cities and farmland, is an abomination in the eyes of \hs{Nature}. 
Farming is a parasitic process that sucks life out of the land, humiliating and enslaving the proud, \Wylde{} land to serve the lowly but arrogant humanoids. 
Sensitive souls can sometimes feel the suffering of the land as it cries out in anguish against its tormentors. 

For the above reason, farmland must be left barren occasionally to recuperate. 
If farmed too hard, it dies, letting nothing grow. 
Dead land, when left alone for a while, will gradually return to life as a \Wylde{} desert-like area, and may later become fertile again. 
Perhaps the deserts of the South and Orient are a result of over-farming. 

Civilized creatures are an abomination, outsiders, anathema to the world. 
They feel at home in their cities\dash the blunted, dumbed down, tortured, enslaved world that they created. 
But in the \Wylde{}, they feel out of place, unwelcome, hated. 
Like a unnatural disease that the world is striving\dash rightfully\dash to expel.

People fear the \wylde.
The \wylde is fucking dangerous.
It is a place of twisting, crawling chaos, where nightmarish horrors\dash{}giant animals and worse\dash{}lurk and feed on hapless \travelers.









\subsection{Creatures}





\subsubsection{Animals see into the Beyond}
\target{animals that can see into the Beyond}
Some of the more intelligent animals are less entangled in the Shroud than humanoids and can see deeper into the Beyond. 
Examples include \hs{cats} and \hr{Nycan}{\nycans}.  





\subsubsection{Beasts, monsters and plants}
Remember to have flying monsters in the \Wylde{}! 
\hr{Vreiid}{\Vreiiden}, giant bats, giant birds of prey, pterosaurs\ldots{} 

There needs to be a race of quadrupedal, theropod-like reptiles. 
They are bigger than \nycans, smaller than \cortios{} and can be tamed. 
The Rissitics use them. 





\subsubsection{Gods of the \wylde}
\target{Wylde gods}
The \wylde was ruled by many gruesome, nameless things, such as the \hr{Moongods}{\moongods} and the \hs{Gods Beneath}. 
The few humanoids who knew about their existence dared not think about them, for they were too terrible to consider.





\subsubsection{Horrors}
Supernatural horrors dwelt in the \wylde. 
They included:
\begin{itemize}
  \item The \hs{Feasters in the Night}. 
  \item \hr{Glithid}{\Glithids}.
  \item The \hs{Ravening Ones}.
\end{itemize}






\subsubsection{Humongous creatures}
Have some truly humongous prehistoric creatures. 
Many times larger than even \dragons. 
Hundreds of metres long. 

Before the \ophidians{} came, they ruled. 
Now their indestructible bones remain. 

Perhaps these no longer exist. 
Perhaps they only exist in the seas now. 

Have buildings and things made from the bones of gargantuan monsters. 
Like in the movies \emph{Pitch Black} or \emph{Red Sonja}. 

The bones of the bigger of these creatures are so large that they can be used to construct palaces that even \dragons{} can dwell in. 

One reason why many of these are extinct is that \hr{Heart weakened}{the Heart of \Miith{} is weakened} and therefore has a harder time supporting these massive, powerful beings. 
Smaller creatures are cheaper and easier to keep alive for the Heart. 





\subsubsection{Trees}
Remember to have some mysticism about \hs{trees}. Trees are cool. 





\subsubsection{Degeneration}
Berserkers are prone to degenerating into beastly forms if they, in their mind, embrace their \Wylde{} power too much. See section \ref{The price of madness}. 









\subsection{Dark, unexplored places}
\target{Unexplored places}
Have many dark, unexplored \quo{here-there-be-\dragons} places in the \wylde. 
Even in \Velcad. 
In the \thirdbanewar period as well as in earlier periods. 
Compare to places from the Cthulhu Mythos:

\begin{itemize}
  \item The Vale of Pnoth.
  \item The Forest of Zoogs.
  \item The Peaks of Throk.
  \item The Vaults of Zin.
  \item The Tower of Koth. 
  \item Kadath in the Cold Waste.
\end{itemize}

Among other things, have a dark valley of naked, black basaltic pillars, inhabited by Gug-like monsters. 
And have places where the \quiljaaran live. 
The Serpent Men were known from legends and feared. 

See also the section on the \hs{dark universe}.









\subsection{History}
\target{History of the Wylde}
In the \hr{Ophidian Golden Age}{\ophidian Golden Age}, before \Sethicus, the \wylde had been contained and driven back.
Most of the planet had been taken under control made civilized. 

Then, in the \hr{Dragon wars under Sethicus}{bloody \dragon wars in \Sethicus' time}, the influx of destructive \xs energy caused the \wylde to grow again.
The \dragons tore open dimensional wounds in the world. 
These wounds were holes in the Heart of \Miith, through which the diseased essence of the \hs{Dark Heart} could bleed through. 
This gave power to the \hs{Gods Beneath} and the \hr{Noggyaleth}{\noggyaleth}.
This was something that \Sethicus never quite understood.

When \Sethicus and the \dragons went into \hs{Durance}, the \ophidians found that \hr{Ophidians cannot contain the Wylde}{they could not contain the \wylde}. 
It had become to powerful to control and contain.
They could only flee into their cities. 

\Miith never recovered. 
Never again did it become a world full of civilization. 









\subsection{Imagery}





\subsubsection{Horrors}
Horrors dwelt in the \wylde. 
Woods and stuff were dark and haunted and morbid. 
See also the section on the \hr{Horrors of the Void}{horrors of the void}. 

\citeauthorbook[p.105]{RobertEHoward:WormsoftheEarth}{Robert E. Howard}{%
  Worms of the Earth%
}{
  What have I known but the lone winds of the fens, the dready fire of cold sunsets, the whispering of the marsh grasses?\dash the faces that blink up at me in the waters of the meres, the foot-pad of night-things in the gloom, the glimmer of red eyes, the grisly murmur of nameless beings in the night!
  
  \ldots 
  
  That night the king went across the dark desolation of the moors with the silent werewoman.
  The night was thick and still as if the land lay in ancient slumber.
  The stars blinked vaguely, mere points of red struggling through the unbreathing gloom.
  Their gleam was dimmer than the glitter in the eyes of the woman who glided beside the king.
  Strange thoughts shook Bran, vague, titanic, primeval.
  Tonight ancestral linkings with these slumbering fens stirred in his soul and troubled him with the fantasmal, eon-veiled shapes of monstrous dreams.
  The vast age of his race was borne upon him\ldots{}
  Yet his race likewise had been invaders, and there was an older race than his\dash a race whose beginnings lay lost and hidden back beyond the dark oblivion of antiquity. 
}





\subsubsection{The majesty of Nature}
%\subsection{The land itself}
\target{Nature}
\target{Majesty of nature}
Be sure to fill \Miith{} up with vast mountains, valleys, raging rivers, deep clefts, haunted swamps, dense forests with trees over a hundred metres tall. Between these, even armies of hundreds of thousands of men cannot help but feel insignificant, dwarfed by the vastness and majesty of the world that surrounds them\dash a world they cannot hope to conquer or even move. 

The \hs{Mask of Civilization} can hide Nature's terrifying cruelty, but not its awe-inspiring majesty and power. 

\citeauthorbook[p.70]{RobertEHoward:KullUntitledDraft}{Robert E. Howard}{%
  Untitled Draft%
}{
  Along the broad white streets of Valusia swept the king and his horsemen, out through the suburbs with their spacious estates and lordly palaces; on and on until the golden spires and sapphirean towers of Valusia were but a silver shimmer in the distance and the green hills of Zalgara loomed majestically before them. 
  
  \ldots 
  But Kull walked apart, beyond the glow of the campfires to gaze out across the mystic vistas of crag and valley.
  The slopes were softened by verdure and foliage, the vales deepening into shadowy realms of magic, the hills standing out bold and clear in the silver of the moon.
  The hills of Zalgara had always keld a fascination for Kull.
  They brought to his mind the mountains of Atlantis whose snowy heights he had scaled as a youth, ere he fared forth into the great world to write his name across the stars and make an ancient throne his seat.
  
  Yet there was a difference. 
  The crags of Atlantis rose stark and gaunt; her cliffs were barren and rugged.
  The mountains of Atlantis were brutal and terrible with youth, even as Kull.
  Age had not softened their might.
  The hills of Zalgara rose up like ancient gods but green groves and waving verdure laughed upon their shoulders and cliffs and their outline was soft and flowing.
  Age\dash age\dash thought Kull; many drifting centuries had worn away their craggy splendor; they were mellow and beautiful with antiquity.
  Ancient mountains dreaming of bygone kings whose careless feet had trod their sward.
}





\subsubsection{The world eaten by maggots}
At times, people can look into the Beyond and see the world as if alive. 
It is a decaying piece of carrion, crawling with horrid, bloated maggots. 

These maggots are actually caused by civilization. 
They are a physical manifestation of the corruption caused by humanoids and their parasitic leeching, a symptom of the disease that is humanoids. 









\subsection{Maintaining the Mask of Civilization}
See section \ref{Maintaining the Mask of Civilization}









\subsection{People connected to the \wylde}





\subsubsection{Berserkers}
\target{Berserkers}
I need to same some berserker people who become superhumanly strong by channelling the raw power of their inner, \chaotic{} self. 

Maybe these people are halfway lycanthropes. They don't entirely shapeshift into animals, but in their enraged state they have shed their humanity so much that they almost appear like beastly half-men. 

\KarsaOrlong{} might be one of these: A mighty warrior from a \cregorr{} tribe, only recently turned to Rissitism. 





\subsubsection{Druids}
\target{Druids}
Druids are a special order of nature mages who channel the power of the \Wylde{}. 





\subsubsection{\Rangers: Hunters, pathfinders, explorers}
Some humanoids are \rangers: Hunters, explorers and pathfinders who have learned to live in a sort of pact with the \Wylde{}. They are in a special kind of contact with their inner, primal, \chaotic{} self, and as such can move in the \Wylde{} more safely: They understand the \Wylde{} better, and the \Wylde{} does not actively hate them. 

Actually, as a \ranger{} you can sometimes travel through the \Wylde{} \emph{faster} than on a road. This is because the \Wylde{} is more malleable and can be bent and reshaped through willpower and cunning. 

But ordinary folk sometimes hate the \rangers{}, seeing them as strange, half-savage outsiders. Compare them to the Wolfbrothers in \emph{Wheel of Time}. 

\Nycaneers{} all have some \ranger{} talent. Ilcas Northstar is a \ranger. 

\Rangers{} are prone to degenerating into beastly forms if they, in their mind, embrace their \Wylde{} power too much. See section \ref{The price of madness}. 

Compare this with the Wolfbrothers from Robert Jordan's \emph{Wheel of Time}, who run the risk of losing their humanity, goin mad and ending up running with the wolves.







\subsection{Travelling through the \Wylde}
\target{Travelling through the Wylde}
Most humanoids, when they had to travel through the \wylde, used roads, as described in the section on \hr{Roads through the Wild}{roads through the \wylde}. 

When mortal humanoids had to travel through the \wylde itself, without roads, they needed magical protection to ward away the predatory monsters and inimical magic of the \wylde. 

In the \hs{Iquinian religion}, at least two \sephiroth were specifically charged with safeguarding the sancta of civilization and keeping the \wylde at bay.
These were invoked and/or prayed to in all \wylde scenes.
Other religions had their own gods dedicated to protecting people from the \wylde. 
People walking in the \wylde almost always carried \wylde talismans blessed by these \sephiroth or other divine beings.

Preferably, people would have Vaimons and assistant priests with them (see the section on the \hs{Iquinian clerical hierarchy}).
They would pray, sing hymns to \Iquin and Silqua and burn incenses and carry sacred totems.
All this is to keep the \wylde at bay.
The common soldiers would join in the songs and sing the chorus lines.
(Have some songs like \quo{Gregoriansk Datalogi}.)

\citebandsong{Nile:Ithyphallic}{Nile}{
  Papyrus Containing the Spell to Preserve Its Possessor Against Attacks from He who is in the Water
}{
  Amun\\
  Lord of the gods\\
  Thou who art of the four rams heads upon thy neck\\
  Thou standest upon the spine of the crocodile fiends\\
  To thine sides are the dog headed apes\\
  The transformed spirits of the dawn

  Drive away from me the lions of the wastes\\
  The crocodiles which come forth from the river\\
  The bite of poisonous reptiles\\
  Which crawl forth from their holes

  Be driven back crocodile thou spawn of Set\\
  Move not by means of thy tail\\
  Work not thy feet and legs\\
  Open not thy mouth\\
  Let the water which is before thee\\
  Turn into a consuming fire

  I possess the spell to\\
  Preserve me from he who is in the water

  Thou whom the thirty seven gods didst make\\
  And whom the serpent of Ra didst put in chains\\
  Thou who wast fettered with links of iron\\
  In the presence of Ra\\
  Be driven back thou spawn of Set

  Drive away from me the lions of the wastes\\
  The crocodiles which come forth from the river\\
  The bite of poisonous reptiles\\
  Which crawl forth from their holes
}









\subsection{Terrain types}





\subsubsection{Deserts}
Have giant worms in the desert. Not quite as big as the ones from \authorseries{Frank Herbert}{Dune}, but a clear reference. 





\subsubsection{Jungles}
\target{Jungle}
Have deep, dark, humid jungles. 
Mystic and exotic. 

Compare to the film \cite{Movie:IceAge:III}. 





\subsubsection{The sea}
See section \ref{The Sea}









\subsection{Wildfog}
\index{\wildfog}
Areas of \Wylde{} are sometimes shrouded in \wildfog, a fog-like substance that obscures vision. 
(\Wildfog{} is not regular fog. It is not made of water and may be found even in deserts.) 










\subsection{\Wylde border}
See the section on \hr{Wylde border}{\wylde borders and \eidola}. 







































