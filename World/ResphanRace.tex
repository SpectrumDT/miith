\chapter{The \Resphain}

















\section{\Baelzerach}
\target{BZ}
\target{Baelzerach}
\target{Bael'zerach}
\target{Bael'Zerach}
The \daemonic{} \resphain{} who dwell in \Machai{}. 
They have severed all ties to the \banes. 
\Baelzerach{} is called a dynasty, but this is a bit of a misnomer, since they have no central organization and are split into a number of independent tribes. 

Sometimes they wage war against the \dragons, sometimes they work together. 

The \Baelzerach{} have reshaped their bodies into \daemonic{} forms, with fiery red skin, horns, wings, tails and/or goat-like legs. 
The \resviel{} are more \human-looking than their males, tho. 

\Ishnaruchaefir played a part in the founding of \Baelzerach{} and has a place in their mythology as a legendary figure, half hero and half Devil. 

The \Baelzerach{} enjoy to hunt, kill and eat in the \draconic{} manner (see section \ref{Draconic diet}), rather than \hr{Resphan diet}{eating submissive slaves like other \resphain{} do}. 
This is one of the things that causes other \resphain{} to label them as barbarians. 

\Baelzerach{} alone of the great dynasties has no \satharioth{} among its numbers, since they branched off from \KiriathSepher{} before \hr{Origin of Satharioth}{the \satharioth{} were created}. 
They do have \ketherain, since there has been interbreeding among the dynasties, but they are usually not referred to as such among the \Baelzerach, who reject the notion of a \KiriathSepher{}-based aristocracy. 









\subsection{\Daemoniacs}
\target{Daemoniac}
\index{\daemoniac}
Some \Baelzerach became \daemoniacs.
By using \draconian sorcery they have absorbed \daemons into their bodies and have mutated into monstrous demonic humanoids.
A few \daemoniacs have become extremely powerful.

Most \daemoniacs make a living as raiders or pirates.









\subsection{History}





\subsubsection{Absorbing tribes}
\target{Bael'Zerach absorbs tribes}
After the \hs{Murder of the Dawn}, some \Baelzerach{} rebels left the other dynasties and went out to join the \hr{Early Resphan tribes}{early \resphan{} tribes}. 
They mixed with the tribes and merged with them. 
Later all these tribes came to be considered \Baelzerach{} by the dynasties. 









\subsection{Politics}





\subsubsection{Dark gods}
\target{Bael'Zerach diabolism}
Some \Baelzerach{} worshipped the \hr{Gods in Nyx}{\xss{} and other dark gods that dwelt in \Nyx}, just like \hr{Early Resphan diabolism}{other \resphan{} tribes had done before them}. 















\section{\CiriathSepher}
\target{Ciriath-Sepher}
\target{CS}
\index{\CiriathSepher}
\KiriathSepher{} was the oldest and most traditional of the \resphan{} factions. 
They were loyal to the \banes. 

\target{High Lord of Kiriath-Sepher}
%They are ruled by a High Lord (like a king). 
The first (and only known) High Lord of \KiriathSepher{} was \hr{Azraid}{\Azraid}. 









\subsection{Aesthetics}
The \CiriathSepher{} traditionally prefer to dress in bright \colours: 
White, yellow, orange, silver, gold, bronze. 

They are the \quo{lightest} of the dynasties. 
They were the ones who devised the idea of marketing \Iquin{} as \quo{the Light}. 
(The \Kezeradi{}, who created \Iquin, were aesthetically darker, and they never called their \dweomer{} \quo{the Light}.) 

%They typically dress in white, silver and golden \colours. 
The other factions criticize them for having adopted too much of the \Merkyran{} imagery and aesthetics, and perhaps even their philosophy. This is a result of \hr{Azraid adopts Merkyran imagery}{\ps{\Azraid}{} policy}.





\subsubsection{The Silver Starlight Rose}
\target{CS symbolism}
One of the symbols of \CiriathSepher{} is the Silver Starlight Rose. 
It is a stylized rose-like flower shining silver like a star. 
Its thorns are daggers and swords. 

The Rose has symbolic value (for \hr{Mystraacht symbolism}{unlike \Mystraacht}, the \CiriathSepher{} \emph{do} believe in symbolism). 
It symbolizes beauty and strength; sophistication and art, but also martial power. 









\subsection{Culture}
The \CiriathSepher{} were the most social, diplomatic and extroverted of the dynasties (albeit not necessarily the most free-thinking). 





\subsubsection{Crime and punishment}
\target{CS punishment}
The \CiriathSepher favoured punishing offenders by hitting their reputation. 
A criminal might never be punished directly, but word of his crime would be spread, and everyone in good society would shun and condemn him. 
He would lose reputation and thus influence. 
This was a quite horrible punishment, especially because the crime could be remembered forever and never be forgotten. 
It also favoured the popular and socially skilled, who might be able to dodge such a defamation, whereas the friendless where struck especially hard. 
This was seen as a good thing by the snobbish \CiriathSepher. 

Contrast with \hr{Mystraacht punishment}{\Mystraacht punishments}. 





\subsubsection{The Dance}
\target{Dance}
\index{Dance, the}
\quo{The Dance} was the \CiriathSepher{} name for their delicate and intricate system of etiquette. 
It permeated (and very nearly governed) the life of every \resphan of \CiriathSepher{}. 





\subsubsection{Education}
\target{Ciriath-Sepher education}
\CiriathSepher{} had one central school where all children were educated. 
After all, there were never many children at a time. 

\Mystraacht, on the other hand, \hr{Mystraacht education}{raised each child separately}. 






\subsubsection{Government}
\CiriathSepher{} was sort of a constitutional monarchy. 
The High Lord had a lot of power, but the system ran fine without him, and he could not just decide everything. 
There was a \quo{council} or some such that could override him in many cases. 
They could not depose him, though, nor override his choice of successor. 

\target{CS order of succession}
\hr{Azraid}{\Azraid} made sure that \CiriathSepher{} had a very clear order of succession in case he were to die. 
That way, outsiders knew that \CiriathSepher{} would not be plunged into chaos if someone were to assassinate \Azraid, and insiders knew that \emph{they} would not be able to grab the throne by killing \Azraid. 

People also knew that \Azraid{} was not indispensable. 
After all, it was \hr{Daggerrain}{\Daggerrain} and not he who was the supreme leader of the Cabal. 

Among other things, this had the advantage that \Azraid{} could relatively easily meet with outsiders (even Sentinels) to negotiate. 
They knew that trying to assassinate him was perhaps not worth the effort. 

Furthermore, {\Azraid} purposefully \hr{Azraid obscure}{kept his role obscure}. 

The first in line to inherit \ps{\Azraid} throne was \hr{Harbeth}{\Harbeth}\ldots{} and \hr{Harbeth is Azraid's heir}{no one wanted that to happen}. 
\hr{Zereth}{\Zereth} was one of the next in line. 

\target{CS heirs die}
Not everything turned out according to \ps{\Azraid} \trope{XanatosRoulette}{Xanatos Roulette}. 
The various heirs got killed and destroyed during \SentinelsofMithEmph. 
Among other things, \hr{Harbeth dies}{\Harbeth{} died}. 

\Azraid{} was not happy. 
This endangered the future of \CiriathSepher. 
But \Azraid{} continued with his plan anyway. 
He was perfectly willing to sacrifice \CiriathSepher{} to save \resphan-kind. 

Besides, \hr{Azraid protects Ramiel}{he had Ramiel}. 





\subsubsection{Philosophy}
The \KiriathSepher{} were very lawful and concerned with laws, \honour, etiquette (the \quo{\hs{Dance}}), tradition, principles, art, games/sports and everything that showed off how advanced and well-developed a culture they were. 
They saw themselves as a bastion of righteousness and civilization. 
This put them in stark contrast to the \Mystraacht. 





\subsubsection{Taboos around the newly-revived}
In \CiriathSepher, it was considered taboo to look on a recently dead and newly-revived \resphan.
Only close friends and family members were allowed to see the revived until he had healed. 

This was because the \CiriathSepher placed so great emphasis on appearances, propriety and dignity. 
It is comparable to the nudity and sex taboos in many RL cultures. 









\subsection{Politics}





\subsubsection{\Azraid}
See the section about \hr{Azraid and Ciriath-Sepher}{\Azraid and \CiriathSepher}.















\section{\Kezerad}
\target{Kezerad}
\target{Kezeradi}
A \resphan{} faction who, out of moral scruples, defied the \banes{} and created their own kingdom, supposedly one based on justice and good. 









\subsection{Aesthetics}
Their traditional \colours are gold and bronze. 
Or are they? 
\Kezerad{} shouldn't be too bright. 
Remember, I am trying to subvert the \quo{light is good, dark is evil} trope. 







\subsection{Arsenal}





\subsubsection{\Beacons}
\target{Kezeradi beacons}
\target{Beacons of Kezerad}
The four \Beacons of \Kezerad were the cornerstones of the \Kezeradi \dweomer, \hr{Kezeradi Iquin}{\iquin}. 
They were a foundation upon which the \Kezeradi built their enlightened civilization.
When \iquin was corrupted by the Cabal, the four \beacons became the four \hs{elements} of the \sephiroth. 

After the fall of \Kezerad, the survivors would forever mourn the loss of their beautiful \beacons. 
Nowadays they were forced to draw their magic from \itzach (which was horrible because it invoked the \banes) or from the corrupted \iquin (which was worse because they knew how evil a thing \iquin really was). 

The \Kezeradi knew much about how evil \iquin was and suspected more.
They often tried to warn other \resphain about this, but other \resphain would reject them, thinking that the \Kezeradi merely mourned their own loss and saw everything non-\Kezeradi as an evil corruption that would destroy the world.
They were doom-prophets and not taken seriously. 





\subsubsection{Telepathy}
\target{Kezeradi telepathy}
Back in the day, the \Kezeradi{} shared a \hr{Telepathy}{telepathic} bond and were all intimately bound to each other. They even shared this bond, to some extent, with their \human{} and \nephilic{} subjects. This empathy is one of the reasons why \Kezerad{} was such an enlightened realm. 

This could also be a weakness in war, since the bond meant that they shared pain, so the massacre of one \Kezeradi{} village or city would create a mental backlash that could demoralize the rest. (Compare with the destruction of the planet Alderaan in \emph{Star Wars IV: A New Hope} and Obi-Wan Kenobi's remark that: \ta{It is as if a million voices suddenly cried out in pain, and then fell silent.})

And, even worse, when the \Kezeradi{} lords were captured and transformed into the horrid \Sephiroth, the bond persisted, giving the \Sephiroth{} great mental power over the remaining people of \Kezerad. The invaders had planned this well indeed. 

After the fall of the \Kezeradi{} civilization, the conquerors utilized the psychic bond to hunt down the remaining \Kezeradi{} and eradicate them. In order to survive, those who escaped had to deaden their telepathy and block out all empathic feelings from their minds. 
(Compare this with the Protoss Dark Templar from the game \cite{VideoGame:Starcraft}, who allegedly severed their nerve endings to forever off themselves off from the telepathy of the Protoss race.) 
This gradually turned the survivors into bitter husks, desperately longing for intimacy but knowing that to give in to feelings is to be destroyed. 

%The \Kezeradi{} still have a deep connection to the \Sephiroth, since the \Sephiroth{} are each forged around a core of a \Kezeradi{} soul. See, back in the day, the \Kezeradi{} shared a telepathic bond and were all intimately bound to each other. 







\subsection{History}
\subsubsection{Inception}
\target{Founding of Kezerad}
\Kezerad{} was formed shortly after the \Merkyran{} rebellion by \resphain{} disgruntled with \ps{\Azraid}{} methods, and \ps{\Zachirah}{} even more. 

They acknowledged that \Merkyrah{} was bad, but felt that this new thing they had become was worse still. 
They wanted to learn from both ways and build a better society. 
They wanted to be good again, not evil. 

\citebandsong{DeathspellOmega:FasIteMaledictiinIgnemAeternum}{%
  Deathspell Omega
}{
  The Repellent Scars of Abandon and Election
}{
  Nothing of what man can know, to this end, \\
  could be evaded without degradation, without sin. \\
  Is it no burden to bear \\
  the repellent scars of abandon, of election?\\
  It leaves but a state of supplication and deserted expanses, \\
  an absorption into despair.
}

Even if this new way may be \quo{true}, even if it is the \quo{\hr{Resphan purpose}{purpose}} for which the \resphain{} were created, it is still morally wrong. 

\citebandsong{DeathspellOmega:FasIteMaledictiinIgnemAeternum}{%
  Deathspell Omega
}{
  The Repellent Scars of Abandon and Election
}{
  The existence of things cannot enclose \\
  the death which it brings to me.
}

\Sithiyacaan, gripped by \hr{Curse}{\NexagglachelsCurse}, began to feel self-destructive. 

\citebandsong{DeathspellOmega:FasIteMaledictiinIgnemAeternum}{%
  Deathspell Omega
}{
  The Repellent Scars of Abandon and Election
}{
  The existence is itself projected into my death, \\
  and it is my death which encloses it. \\
  Am I deranged?
}

They split off from the Cabalist dynasties and built a beautiful, happy, almost utopian kingdom of \resphain, \humans, \nephilim{} and possibly other creatures. 
They created their own pure and good \dweomer{}, \iquin, which was less dependent on \Erebos{} and instead drew upon the natural energy of \Miith{} and its Heart. 
Perhaps they had help from elder \ophidians{} and/or other wise creatures in establishing this. 

The foundation of \Kezerad{} must have been \emph{after} the birth of the \satharioth, but \emph{before} the \hr{Malach project}{\Malach{} project}. 
See, \hr{Eryal}{\Eryal} was one of the founders of \Kezerad{} after \hr{Shiaraid and Eryal driven apart}{her falling-out with \Shiaraid} but before they both became \malachim. 





\subsubsection{The fall of \Kezerad}
\Kezerad{} \hr{Fall of Kezerad}{was destroyed}. 





\subsubsection{\Kezeradi{} today}
There are surviving \Kezeradi. They are in an uneasy alliance with the \hr{Cuezcan}{\cuezcans}, since both want to see the \bane{} faction destroyed. \Sanyor, the \scathaese{} chaos sorcerer who is Curwen's second-in-command but will betray him, is a \cuezcan{} in disguise and working to free the \Sephiroth. 

The \Kezeradi{} have an image of \quo{fallen angels} about them. Not \quo{fallen} in the usual sense of having turned to evil, but in the sense that they were once an idealistic people, believing in good and beauty, but have become hardened, bitter and disillusioned. They look angelic, but harrowed: Bright-\coloured skin and great feathered wings, but the wings are tattered and torn, their skin deathly pale or blotched and dis\coloured, and their once beatific visages are grim, contorted from millennia of grief, pain and rage. 





\subsubsection{\Sithiyacaan: The last \Kezeradi{} prince}
\hr{Sithiyacaan}{\Sithiyacaan} is a great hero who is the last surviving \Kezeradi{} lord. 





\subsubsection{New \Kezerad}
Some surviving \Kezeradi{} have built a new \Kezerad, a new beautiful, harmonious kingdom. It is much smaller than the original \Kezerad, but good. Once in a while they are able to rescue someone and bring them there. It's almost a Tanelorn-like place (as in Michael Moorcock's stories). This is one of the silver linings of the story. 

\hr{Sithiyacaan}{\Sithiyacaan} knows about the place, but he does not know its location, and he can never go there. It would make him remember too much, and the \Sephiroth{} would gain access to his mind and learn the place's location and come to destroy it. 

This causes him great distress. 









\subsection{Politics}





\subsubsection{\TiphredSerah}
During the \hr{Resphan Wars}{\resphanwars} the \TiphredSerah{}, being \hr{Tiphred-Serah free-thinking}{the most free-thinking of the dynasties}, were the ones \hr{Tiphred-Serah and Kezerad ally}{most liable to ally with \Kezerad}. 















\section{Lawbringers}
\target{Lawbringers}
\target{Lawbringer}
\index{Lawbringer}
The Lawbringers were a group of \resphain that had been reshaped into terrible half-\resphan beings. 

All Lawbringers were pureblood \resphain or \resviel. 









\subsection{Demographics}
The names of the Lawbringers include:
\begin{itemize}
  \item \Dolsharra (deceased).
  \item Ithaarit (the only female Lawbringer and the most powerful of them, after \Dolsharra).
  \item Mazdrach.
  \item Ozruth.
  \item Sacbal. 
\end{itemize}










\subsection{History}
After the \hs{Murder of the Dawn} but before the \hs{Shrouding} there were those \resphain who felt that their people had gone too far, become too evil, too corrupt, too decadent. 
\hr{Dolsharra}{\Dolsharra} was the leader of one such group. 
The members of this group were often politically powerful \resphain.
They laid their plans to influence their race and enforce more law and order. 

Then, before their plans could come to fruition, the Shrouding came. 
The insurgents almost died.
In their despair \Dolsharra prayed to the \SitraAchras for help. 
A \banelord (possibly \Daggerrain) contacted them. 
It told them to keep their minds open to the \SitraAchras and to keep praying, and they would be saved. 

They complied.
In the end most of them survived, but the \banelord had twisted their minds and bodies and turned them into its own grim police. 

\target{Dolsharra dies}
\Dolsharra, who was by far the strongest of will, protested and refused to serve as the \banelord's robot slave. 
The \banelord destroyed him effortlessly, to set an example. 
The others submitted and accepted their fate.
They became Lawbringers. 









\subsection{Physique}
A Lawbringer looked like a normal \resphan, but with unnaturally pale, grayish skin.
Their skin almost resembled that of \bezedeth (although they were winged and clearly purebloods), but more pallid, corpselike. 

Their faces always looked grim and tormented. 
Often they were wrinkly and harrowed. 
Some Lawbringers were always weeping. 









\subsection{Politics}
The Lawbringers were the pawns of the \banelords.
They were tasked with enforcing ancient laws, keeping the \resphan population in place and culling the weak \resphain in order to keep the race strong. 

They kept the rest in line, but they did not rule. 
They preserved a set of ancient laws and answered to no one, but they made no attempt to seize more power. 





\subsubsection{Culling the weak}
The Lawbringers were also responsible for culling the weak elements in the \resphan empire. 
They hunted down those \resphain who were succumbing to a malady of the body or dementia of the mind. 
There were many of those. 

Many such diseases were curable or could go away of their own. 
But the afflicted had to survive the depredations of the zealous Lawbringers until she got her health back.

This is a story idea:
A \resvil develops some weakness, and then fights desperately to regain her health and fend off the Lawbringers. 









\subsection{Psychology}
The Lawbringers were superpowered things with twisted minds. 
They were the slaves of the \banelords. 















\section{\Malachim}
\target{Malach}
%In Vaimon metaphysics, the \Malachim{} are a class of \Archons{} that may incarnate as \humans{}. An incarnation of a \Malach{} is called a Scion. 
The \Malachim{} are \resphan{} lords who has left their \resphan{} bodies to incarnate again and again as \humans{}. 

The names of the \Malachim{} include: 
\begin{itemize}
  \item \hr{Eryal}{\Eryal}.
  \item \hr{Ishicah}{\Ishicah}.
  \item Nelchael.
  \item \hs{Ramiel}.
  \item Sachiel.
  \item \hr{Shiaraid}{\Shiaraid}.
  \item Two more. 
\end{itemize}










\subsection{Biology}






\subsubsection{Immortality}
\target{Malach immortality}
The \Malach{} model of immortality through rebirth is intended as an improved form of the \hr{Ophidian immortality}{\ophidian{} shedding of skin}. It's an attempt to achieve immortality without stagnation. It's also inspired by \hr{Draconic immortality}{\ps{\KhothSell}{} project of \draconic{} immortality}. 

They derive sexual power from this, too, because whenever a \malach{} is born anew, he gains some new, improved lifeforce from his parents. Compare this with the Xenomorphs of the \emph{Alien} movies, who improve themselves by stealing the genes of their hosts. 

Many women die in giving birth to a Scion, because the \malach{} \hr{Life drain}{drains} too much lifeforce from its host. 

See also the section on \hr{Kinds of immortality}{different kinds of immortality}. 





\subsubsection{\Kenosis{} and \Apotheosis}
\target{Kenosis}
\target{Apotheosis}
\index{\kenosis}
\index{\apotheosis}
Vaimon metaphysics holds that a Scion is created when a divine \ps{\Malach}{} descends to \Miith{} and incarnates in a \human{} body. 
The \Malach{} willingly gives up its divinity in order to walk on \Miith{}. 

\quo{\Kenosis} is the term for the reconciliation and between the mortal \human{} and the divine \malach. 
For a Scion, to achieve \kenosis{} is to become at peace with your nature and thus gain full access to the memories of your recent incarnations (to the extent that these memories carry over). 

\quo{\Apotheosis} is that which all Scions must strive for: 
The re-realization of the divine potential that lies dormant within the Scion. 

Only by \Kenosis{} and \Apotheosis{} in combination can the divine \Malach{} and the mortal Scion be fully reconciled and in harmony/balance/union. 










\subsection{History}





\subsubsection{Origin}
Some of the \Malachim, including Ramiel, were of the \satharioth. 

The \malachim{} were created as part of a \hr{Malach project}{secret project}.
But somewhere the process went wrong, and the \Malachim{} all had their memories of their previous lives as \resphain{} erased. 

Unbeknownst to all, the project was \hr{Azraid masterminds Malachim}{secretly monitored by \Azraid}. 





\subsubsection{Amnesia}
They now recall only scattered fragments of their old lives, typically in fever-like dreams and \deajvus. 





\subsubsection{Captive \malach}
\target{Captive Malach}
Unbeknownst to anyone, \Azraid{} and one of his inner circles of researchers had, at some point, managed to capture a \malach{} or two. 
These they kept hidden and used as study objects. 
\Azraid{} wanted to unlock the secrets of the \ps{\malachim} creation and utilize that knowledge in his own \hr{Neo-Resphan}{\neoresphan} project. 

Compare to the anime \cite{Anime:NeonGenesisEvangelion}, where NERV and SEELE create the Evangelia as clones of a captive Angel. 









\subsection{Name}
\emph{Neshamah} is a Hebrew word that, in \Cabbalah, refers to one of the \quo{highest} parts of the \human soul.









\subsection{Scions}
\target{Scion}
\target{Scions}
\index{Scion}
A \human{} incarnation of a \Malach{} is called a Scion. 
Each Scion typically retains some memories of his previous incarnations, but how this works differs from \Malach{} to \Malach. 

For example, \hs{Ramiel} tends to be born with no memories at all, and then have his \emph{previous} incarnation \quo{awaken} later, triggered by some massively emotional event. (\hr{Carzain}{Carzain \Shireyo}, a Scion of Ramiel, has his predecessor, \hr{Vizicar}{\VizicarDurasRespina}, awaken when Carzain fights his first battle to the death and kills a man.) 
From that point, the Scion has a split personality with two distinct persons inhabiting the same body. If the two personalities get along, they will absorb traits of each other and eventually merge into a single character. If the two do not get along, the Scion will degenerate into a mood-swinging maniac.

\Shiaraid{} remembers a lot of sensations and emotions, but no concrete thoughts and words. 

Other \Malachim{} work differently. 

A Scion does not necessarily remember his true name, but the Vaimons possess the knowledge to research a Scion's mind and ascertain his \Malach{} identity. 





\subsubsection{Conjuctions}
Perhaps there are astrological reasons that determine when and where the Scions incarnate. 
Perhaps there are big Conjuctions that cause many Scions to incarnate shortly after each other, so their lives overlap. 
These times tend to have lots of \vertex/\matrix{} activity, too. 





\subsubsection{Scions are often only children}
\target{Scions are often only children}
Scions often have no siblings. 
They may have older siblings, but rarely younger siblings. 
It takes a great toll on the mother to give birth to so powerful a soul, so if it does not kill her outright it tends to render her barren, spent. 










\subsection{Skills and powers}





\subsubsection{Binding souls}
\target{Malachim binding souls}
\target{Malachim bind souls}
All \malachim{} had the power to capture the souls of others (including, but not limited to, people they slew in combat) to themselves and bind them in a \hr{Carcer}{\carcer}.

This was a more powerful version of the \hr{Resphan vampirism}{vampiric powers of all \resphain}.

This ability was extra powerful among \sathariah{} \malachim.

For the amnesiac \malachim, the presence of these bound souls in their \carcer could be traumatic. 

\citebandsong{BlindGuardian:IFTOS}{Blind Guardian}{Imaginations from the Other Side}{
  Where are these silent faces\\
  I took them all\\
  They all went away\\
  Now you're alone\\
  To turn out every light so deep in me\\
  Hold on, too late
  
  Will I ever see them back again?\\
  Or did they all die by my hand? \\
  Or were they killed by the old evil ghost\\
  Who had taken\ldots{}\\
  The ocean of all my dreams\\
  which were worth to keep\\
  Deep inside my heart\\
  I wish I vould get them back\\
  From the everflow\\
  Before they'll fade away
}









\subsection{Vaimon view}
In Vaimon metaphysics, the \Malachim{} are considered a class of \Archons{}. 
















\section[Merkyrah]{\Merkyrah}
\target{Good Resphan Empire}
\target{Merkyrah}









\subsection{Culture}





\subsubsection{Age as status}
\target{Age in Merkyrah}
In \Merkyrah age was a criterion of status. 
Elders were considered wise and wielded much power.
The rebellion was to a large extent a rebellion of the youth.
Hence stems part of the reason for the later \hr{Resphan age taboo}{age taboo}.





\subsubsection{\Bezedeth}
\target{Bezed status in Merkyrah}
In \Merkyrah, \bezedeth had an understandable reputation for being cowards, compared to the True Immortal purebloods. 
They were scorned by the \Merkyran church. 
Being mortal they were condemned to oblivion, while the purebloods would receive the mercy of God and live forever. 

This made them very bitter, so when the rebels came and preached to them the \bezedeth were \hr{Bezedeth rebel against Merkyrah}{all too eager to rebel against \Merkyrah}.

\citebandsong{Nile:InTheirDarkenesShrines}{Nile}{
  The Blessed Dead
}{
  Looked Down Upong With Scorn\\
  We Work the Fields of the Masters\\
  And Share Not the Bounty of the Black Earth

  Destitute Servile Cast Out\\
  Affording No Tomb\\
  We Shall Be Buried\\
  Unprepared in the Sand

  We Shall Never Be The Blessed Dead

  Scorned By Asar\\
  Condemned at the Weighing of the Heart\\
  We are Exiled from the Netherworld\\
  Serpents fall Upon us Dragging us Away\\
  Ammitt Who Teareth the Wicked to Pieces

  Pale Shades of the UnBlessed Dead\\
  None Shall Enter Without the Knowledge\\
  Of the Magickal Formulas\\
  Which is Given to Few to Possess

  Not for Us to Sekhet Aaru\\
  Our Souls Will be Cut to Pieces with Sharp Knives\\
  Tortured Devoured\\
  Consumed in Everlasting Flames

  We Shall Never Be The Blessed Dead
}





\subsubsection{Breeding creatures}
The \Merkyrans{} breed some creatures to serve them\dash or perhaps they \emph{created} them, using the remains of \hr{Bane technology}{\bane/\voyager{} technology}, in the very beginning, before they lost all their memory. 

These creatures are beautiful and benevolent, shaped as such by the good, \hr{Old Good Iquin}{\iquin-based magic} of the \resphain{}. 

Maybe they look like Faerie Dragons or Wisps from \emph{Warcraft III}, or maybe something with gossamer wings.

But as the rebels' evil seeps out and permeates their society and their world, these creatures become twisted and evil monstrosities.

\lyricsduana{Chrysalis}{chrysallis}{
  and she weeps at the fragile beauty \\
  that warps the darkness \\
  and captures her \\
  once again in the web of life
}





\subsubsection{Demographics}
\target{Merkyran demography}
There were relatively many \resphain{} in \Merkyrah. 
More than one per 100 mortals. 
This was before the devastating \secondbanewar{} and \resphanwars, and before the \hr{Heart}{Heart of \Miith} became \hr{Heart weakened}{so badly weakened}. 





\subsubsection{Fallen ones}
\target{Early fallen Resphain}
\target{Early Resphan fallen ones}
\target{Merkyran fallen ones}
But there were some \quo{fallen} \resphain. 
These were unable to suppress the darkness within them and fully embrace their religion. 
These live either in secret, pretending piety while feeling blasphemy inside, or as overt outsiders and criminals, or even mafia. 

Many of the fallen ones, perhaps most, were \quo{\hr{Ashenblood}{\ashenbloods}}, born of \human{} mothers. 
In \Merkyrah, they were outsiders because they killed their mothers and were born with blood on their hands. 
This was a visible reminder of the \resphan{} \quo{original sin}, which the \Merkyrans{} feared. 
Therefore the Church cannot abide the sight of them, and they are expelled and shunned. 

Some of them learned to channel \hr{Itzach}{\itzach}, which was forbidden. 
They liked \nieur, because it was more suited to their nature and felt more true, more \resphan-like than the false Light of \hr{Old Good Iquin}{\Esheram} that the church forced down their throat.

Some of these are full of \trope{Wangst}{Wangst}, feeling guilt and shame over their wickedness and weakness, their inability to rise above their base, bestial nature. 
Others embrace the darkness and chaos and becomes violent, cruel miscreants. 
These often form criminal gangs or cults, where they have sexual orgies and stuff. 

A few fallen ones are powerful individuals who choose this path for themselves, or who gain strength through adversity. 
But most of them are weaklings who are simply to feeble of mind to suppress their inner darkness and conform to their religion. They are a pretty sorry bunch, and most are very unhappy. 
They use sex (and violence and other decadence) in an attempt to fill the devouring void within them, but they don't succeed. 
They suffer, like gothics or emos. 





\subsubsection{Mortality and cannibalism taboo}
\target{Merkyrans do not eat souls}
\target{Merkyran cannibalism taboo}
Before the return of \Semiza, the \resphain did not know how to eat \resphan souls. 
(This included \emph{all} \resphain, not only \Merkyrah.)

This meant that the \resphain lost their immortality. 
They lived only a few centuries (up to 300 years, or 400 in extreme cases). 
As they ages they became weak and hungry.
Their bodies showed age just like \humans.
In the end they died of hunger for immortal flesh and blood and power and souls.

\Merkyrah developed a taboo against eating \resphan flesh. 
They saw it as a sin leading to impurity and suffering and madness and evil. 
(Eating \resphan flesh \hr{Madness from eating Resphan flesh}{could lead to madness} if not done properly.)

The \Merkyrans did not know how to create \hr{Life-seed}{life-seeds}, so they died if their bodies were destroyed. 
It was customary to toss the bodies of slain enemies into the deep so they could not revive. 
Almost no one had ever returned from being tossed in the deep, so all \resphain were deadly afraid of it.
Your own dead were interred in crypts. 
(Fortunately there was plenty of room. Lots of unused chambers that could be used as crypts.)





\subsubsection{Prisons}
\target{Merkyrah prison}
\target{Merkyran prison}
\Merkyrah had prisons where they kept \resphan criminals.
The \Merkyrans did not know how to kill a \resphan, except by starvation. 
The worst punishment for a \resphan was to be thrown in jail without food. 
This could be for years. 
Many would succumb and die permanently from lack of food. 
The longer the sentence, the greater the risk of death. 
This was the intent. 

Prisoners had their wings severed.
Without food, they could not regrow them. 
Some especially vile criminals were dismembered even more. 





\subsubsection{Sex and marriage}
\target{Merkyran monogamy}
\Merkyrah{} practiced monogamous marriage.
Sex outside marriage was a sin and a crime. 
(They were pretty crazy.) 





\subsubsection{Technology}
\target{Merkyran technology}
\index{technology!\Merkyrah}
The \Merkyrans{} had low technology.
They were limited by the \hr{Resources in Nyx}{scarcity of natural resources in \Nyx}.
They were at a bronze age level. 

They did have a few artifacts of \hr{Bane technology}{\bane{} technology}. 
They did not understand these, so they tended to worship them as religious relics\dash or shun them as devices of evil. 

Only later did the \resphain{} \hr{Resphan technology}{learn about the lost science}. 





\subsubsection{Titles}
\target{Resphan Warlord}
\target{Orator}
\target{First Orator}
\target{Thearch}
Among \Merkyrah and related cultures, a tribe or nation was traditionally led by a triumvirate of leaders: 
\begin{gloss}
  \gitem{Warlord} A commander of warriors.
  \gitem{First Orator} A speaker, politician, lawgiver and administrator.
  \gitem{Thearch} A high priest and mage. 
\end{gloss}









\subsection{History}





\subsubsection{\Resphain orphaned}
The \resphain{} were successfully born. 
But most people around them were killed. 
They ended up raised by some \nephilic{} commoners who really had little conception of what was going on. 
So the young \resphain{} were never told of their legacy. 

\Daggerrain{} had calculated with the possibility that \Thanatzil{} might fail, so he had installed in the \resphain{} an instinctive knowledge of their true nature. 
But he had underestimated the weakness of the \nephilic/\human/\resphan{} mind, and how abhorrent the truth would seem to the abandoned, hapless \resphain{} alone in the world. 

They repressed the truth. 
Their collective denial drew upon the cosmic barriers already woven in the \firstbanewar{} and used threads of these to create a sort of Shroud that prevented the \banelords{} from communicating with them. 





\subsubsection{Development of culture}
In this haze of delusion and denial they developed a culture of their own. 
At first they lived as wretched barbarian scavengers amid the \hr{City of Nyx}{endless decaying ruins of \Nyx}, its spires towering enormous and dark and frightening around them. 
They feared and fled from the many monsters that dwelt in \Nyx. 
But they had some magic, and that gave them a bit of an edge. 

Meanwhile, \ps{\Daggerrain}{} plans were set back a thousand years or more while he waited for the \resphain{} to rediscover the truth\dash as he knew that they must, sooner or later.

It took them hundreds of years, but eventually they built kingdoms and empires.
The greatest of these empires was \Merkyrah.




\subsubsection{\Resphain were lost and afraid}
After \ps{\Thanatzil} fall, the \resphain{} were lost and afraid. 

\lyricsbs{Emperor}{Grey}{
  when all is dark\\
  there are no points of reference\\
  and we no longer navigate by the stars\\
  we just end up somewhere\\
  \ldots{}nowhere\ldots{}
}

They longed for a belief to cling to, something to give meaning to their frightening lives. 





\subsubsection{Created as a Shroud anomaly}
\Merkyrah{} was founded due to a Shroud anomaly. 
The \banes{} tried to open the way from \Erebos{} through \Nyx{} into \Tembrae. 
It went awry. 
There was a great implosion, and all the \resphain{} (as well as a lot of \nephilim, who lived, and some \humans{}, who all died) were sucked into \Nyx. 

Here they now had to build their new lives. 





\subsubsection{\Semiza helped them}
In the beginning, \hr{Semiza helps fugitives establish themselves in Nyx}{the fugitives in \Nyx were aided by \Semiza}. 





\subsubsection{Repressive religion won}
After the exile into \Nyx, the early \resphain{} quickly splintered into many warring tribes. 
They had an instinctive bloodthirst and a craving for fratricide and cannibalism. 
But they did not have the technology to \hr{soul-eating}{eat souls}, so they could not draw much advantage from all this cannibalism. 

Ergo, \resphain{} who followed their warlike nature had no particular advantage over those who lived in denial. 
So after centuries or millennia it came to be that some fractions rose to prominence with a very ascetic and \resphan-nature-denying religion. 

These formed \Merkyrah. 
Of the many \resphan{} kingdoms, \Merkyrah became the greatest. 





\subsubsection{The \Hoshiabalon Letters}
\target{Berugiel's revelation}
\target{Hoshiabalon}
\index{\Hoshiabalon}
The \resphan \hr{Berugiel}{\Berugiel} claimed that the \hr{Merkyran God}{One God} revealed his words to him in the chamber of \hr{Hoshiabalon}{\Hoshiabalon} in the tower of \hr{Tirunad}{\Tirunad}. 

The words were written in the mystic \quo{\Umaric} tongue (which may be fictional) in aethereal flaming letters. 
Only those who were wise and pure of heart could see these letters. 
The language was incomprehensible, so you had to meditate on the letters and interpret them. 
Attempts to transcribe the letters were only partially successful.

The letters were actually a forgery. 
They were not created by the One God.
Rather, they were the result of a spell cast by \Berugiel. 
Into the spell he poured all his feelings and thoughts and fears and ideology. 
It was a powerful spell which drained him almost dry.
He died less than a year later.
But he had powerful friends who carried on his deception. 
Later generations forgot the deceit, and his legend became accepted as truth. 

The rebels \hr{Rebels expose Berugiel}{exposed the \Hoshiabalon writings as a fraud}. 





\subsubsection{Monster-hunting}
\target{Merkyran hunters}
In the beginning, \Nyx{}\dash being a bastard child of \Erebos{} and \Miith\dash was full of nightmarish horrors and monsters. 
But over the course of many centuries, the \Merkyrans{} and other \resphain{} pushed back these monsters, and even exterminated some of them. 

Compare to the later \human{} \hr{Myths of vanquished monsters}{myths of Iquinian heroes vanquishing inhuman Elder Races and monsters}. 

The \resphain{} ate these monsters.
They cooked them using religious magic. 
They did not understand the principles back then, but their arcane cooking released some soul-power from the monsters, which the \resphain{} devoured. 
This was very healthy for them and helped them sate their \hr{Resphan vampirism}{vampiric thirst} and thus grow and stay alive. 

This was actually part of \ps{\Daggerrain} master plan. 
He knew from the start that the \resphain, having \bane{} blood in their veins, would be vampiric/cannibalistic parasites, so he prepared a Realm full of nice, nutritious monsters into which he could plunk his \resphain. 
Thus they could live there for some centuries or millennia and grow big, strong and plentiful. 

In \Merkyrah, \quo{hunter} was a common, respected and very valuable profession in society. 
\Nyx{} had little possibility to support agriculture, so they had to live off hunting. 

Near the end of the time of \Merkyrah, the dependence on monster-hunting became a problem.
The \resphain{} had hunted several species into extinction, and severely thinned the populations of many of the rest.
At least, the more manageable ones. 
The most dangerous and/or less edible monsters were still around and posed as great a threat as ever. 
This meant that near the end, hunting had become sparse. 
To find good edible game the \resphan{} hunters had to venture far away from their safe home towers (either out or \emph{\hr{safe zone}{down}}), and they had to go through large stretches of land that was almost devoid of game suitable for \resphan{} consumption but full of dangerous monsters for whom \emph{\resphain{}} were attractive prey (such as the dreaded \hr{Umbra}{\umbrae}). 

This scarcity of prey meant that the \resphain{} were beginning to suffer from malnutrition. 
Their health was failing and their lifespans dropping. 
They were becoming frail and weak, which only made hunting more difficult and compounded the problem. 

The monster populations \hr{Nyx monsters}{since recovered}. 





\subsubsection{Golden age}
The \resphain were always divided into many nations, tribes, religions and languages. 
\Merkyrah eventually became dominant and conquered many of its neighbours. 

\Merkyrah existed for thousands of years. 
It had a time of greatness, where most of the warring tribes paid tribute to great \Merkyrah. 
It was a mighty and glorious empire.
Compare it to the empire of the elves in \cite{RPG:Warhammer}. 





\subsubsection{Sorcerers reign}
For thousands of years, \Nyx was terrorized by dark sorcerers. 
\Merkyrah fought against them, for they knew that sorcery was evil. 
They \Merkyrans used their \quo{white} magic against the sorcerers' \quo{black} magic. 

The sorcerers' names, might and evil deeds lived on in history and legend long after their time had ended. 
Compare with the lists of famous mages in \cite{JackVance:RhialtotheMarvelous}
The greatest of them was \hr{Sartheron}{\Sartheron}. 

In the last centuries before the \hs{Delving}, the power of \Merkyrah and other ascetic religions had grown while the power of sorcerers had waned. 
Some believed that the time of darkness and fear and the reign of sorcerers was ending and that this would be a new age of light and reason where honest warfare would dominate over black magic. 
Some believed that \hr{Sartheron}{\Sartheron} and his ilk were the last dying remnants of a bygone age.





\subsubsection{Decline}
Then \Merkyrah fell into decline.
It lost power, and the tribes stole much territory. 
At the time of the \hs{Delving}, \Merkyrah was weak, stagnant and bogged down in endless, pointless wars against the tribes. 

Compare to the gloomy dystopia of Hastur in \cite{JamesBlish:MoreLight}. 





\subsubsection{The Heart is strained}
\target{Merkyrah strained the Heart}
At the end of \ps{\Merkyrah} time, the \hr{Heart}{Heart of \Miith} was being strained. 
Right after the \firstbanewar, \hr{Heart after First Banewar}{the Heart had been going strong}, but now, thousands of years later, there were hundreds of thousands of \resphain{} alive, which was more than the Heart could safely support. 

As a result, everyone's health was deteriorating\dash especially the \resphain{} themselves, but also other creatures, including those in \Tembrae{} that knew nothing of the \resphain. 
The \dragons{} \hr{Dragons wonder why the Heart is weak}{wondered about this}, but could not figure it out. 

\target{Umbra menace growing}
This Heart-weakening had another side-effect, which was nastier from a \resphan{} point-of-view:
It brought more \umbrae. 
This was also \hr{Umbra origin}{how the \umbrae{} had originated on \Erebos}: 
When the life-giving \dweomer{} was being wrung dry by \bane{} wickedness, it would retaliate by pumping out \bane-eating monsters. 
Now \umbrae{} were multiplying like crazy, and it seemed there was nothing the hapless \Merkyrans{} (and other \resphan{} tribes) could do about it. 





\subsubsection{\Daggerrain knew it was time}
At the time shortly before the \quo{\hs{Delving}}, \Daggerrain{} knew that \Merkyrah{} had played its role, and that he must set things in motion to bring his \resphain{} back into the fold. 
The reasons for this included (but were not limited to):

\begin{itemize}
  \item 
    The \resphan{} population was nearing its natural ceiling and \hr{Merkyrah strained the Heart}{straining the Heart of \Miith}. 
  \item 
    The \resphain{} were having trouble sustaining themselves by \hr{Merkyran hunters}{hunting monsters}. 
  \item 
    The \hr{Umbra menace}{\umbra{} menace} was growing. 
\end{itemize}





\subsubsection{All \resphain were ignorant}
At this time all \resphan tribes and nations (not only \Merkyrah) were deluded and ignorant of their true nature and heritage. 
They all had false beliefs, but each had small pieces of truth.
By studying the various tribes and their belief systems one could narrow it down and piece together the true picture. 





\subsubsection{\Damiarch and \Netzachirah doubt}
\Damiarch and \Netzachirah and \Nathrach and \Kezrabal were malcontents.
They lived inside the system and had success, but they were dissatisfied. 
They doubted the \Merkyran faith and wanted more\dash a greater meaning, a greater truth.

They were active in the wars against the rival tribes. 
On expeditions they would learn a little of each enemy tribe's beliefs and come a bit closer to an understanding of the truth. 

From some of the savages they learned that the \resphain had not always been the masters of \Nyx. 
The \hs{spider-people} had ruled \Nyx for a million years before the first \resphan was made. 










\subsection{Infinite city}
\target{Merkyrah is one huge city}
\Merkyrah{} is one huge, infinite city. 
It is based on \Erebos{} and beautified by the good \resphain. 

Beyond the borders of the civilized \Merkyrah{} lie more builings, but there are empty, haunted ruins, shunned by all. 
By venturing beyond these ruins one can gradually transit into \Erebos\dash if you're really lucky. 
\Nyx{} is a shadow of \Erebos, remember. 





\subsubsection{Cathedon}
The capital citadel of \Merkyrah{} is Cathedon. 
(The name is taken from \authorbook{William Blake}{The Four Zoas}.) 











\subsection{Philosophy and religion}
\target{Merkyran myths}





\subsubsection{Black stars are feared}
\target{Merkyran myths of black stars}
The \Merkyrans{} had myths about black stars. 
They were said to be terrible wounds in the sky, through which the hideous outer void bled all its evil and corruption. 

These were based on a memory of the \hs{black stars} of \Erebos. 





\subsubsection{Childishness}
\target{Merkyrah is childish}
In a sense, the \resphain of \Merkyrah were like children: 
Immature, ignorant, irresponsible, unaware of their own true nature and potential.
At times they were happy in their ignorance, but it also made them helpless and afraid. 

In the rebellion, the \resphain matured and became responsible, knowing adults. 
They learned to cope with the grim, cruel world around them instead of just building walls to hide behind. 
They had to face hideous truths. 
In some ways this made them less happy, but it also gave more meaning and purpose and direction to their existence, and they were able to find happiness in this striving. 
In contrast, their old religion was a \naive, childish, irresponsible, pointless belief system of dreams and illusions. 

This is a way for me to insert some more \quo{serious}, \quo{literary} material into my story, and so cater to critics and snobs. 





\subsubsection{Creation myth}
\target{Merkyran creation myth}
The \hr{Merkyran God}{One God} created the towers and made them rise and aspire towards the \hr{Merkyran heaven}{heavens} to reach his glory.
The towers were ladders of God, and the highest levels were the most holy. 
Only noble higher creatures like \resphain were allowed to dwell in the upper levels. 

\hr{Umbra}{\Umbrae} were fell creatures of the Outer Darkness whom God had banished to the deep. 

He created \glowmoss as a reminder that his eternal light held dominion over all things, even in the darkened deep below.

God \hr{Berugiel's revelation}{revealed his words} to \hr{Berugiel}{\Berugiel} in \hr{Hoshiabalon}{\Hoshiabalon}.





\subsubsection{Decay}
Perhaps their religion causes decay. 

\lyricsbs{Vital Remains}{Descent Into Hell}{
  Trapped behind the false moral confines of Christianity.\\
  Your self-destructive belief system is a festering cancer.\\
  It should have become extinct a long time ago.
}

Perhaps the \Merkyran{} priests are poisoned by their own lying religion, degenerating into crippled, sickly ancients. 
Compare to the priests in the movie \emph{300}. 





\subsubsection{Eschaton}
\target{Merkyran eschatology}
One day the holiest spirits of the One God would descend from their place in the Crown.
They would conquer and consume all the Wyrms.
Before their shining visages the darkness would dissipate.
The towers would all become pure and bright and beautiful and full of life all the way down.





\subsubsection{God and \Esheram}
\target{Good Resphain and God}
\target{Merkyran God}
\target{Merkyran god}
The \Merkyrans invented a false god whom they called the One God. 
This god did come into metaphysical existence, but as a phenomenon rather than an actual sentient being. 
It was based on the \resphain's own fear of recognizing their true nature. 
Whenever they suspected their true nature, they felt the horror of the \banes, so they fled from the revelation and repressed it. 

Their religion became based on asceticism, fear, guilt, duty, humility, obedience and submission. 

\target{Old Good Iquin}
\target{Esheram}
The \Merkyrans built a \dweomer for themselves, which they associated with their god. 
It was just \hs{front-end} for \itzach, and drew all of its energy from \Nyx{} and the \banelords, but it was much more user-friendly and encapsulated all the evil and \hs{Entropy} of \itzach. 
They called it \Esheram, \quo{the Light}. 
It was a precursor of the later \iquin.
It was filtered through the Shroud and encased in all the \ps{\resphain}{} denial and idealism, which made it feel good and benevolent and concealed its true, destructive, vampiric, parasitic nature. 
The \Merkyran filter suppressed the \Nyxian{} aspect of their power and kept it in check, like a mini-Shroud, but it lay latent just underneath the surface. 

They knew about the existence of \itzach, but forbade its use. 
Only the \hr{Early Resphan fallen ones}{fallen ones} used it. 

The One God dwelt among the stars in the sky. 
It was he who had \hr{Merkyran creation myth}{created the towers and the \glowmoss}. 
He mastered the thunder in the deep, and the grim monsters fled from his visage. 





\subsubsection{Heaven}
\target{Merkyran heaven}
The \Merkyran \quo{heaven} was a bright place full of clean and shining towers (for they feared the decay and decrepitude). 
Glowing plants grow everywhere in all the colours of the rainbow (for vegetation was a scarce resource in \Nyx, and colours were rare and exotic). 
There were happy and smiling \resphain and \humans everywhere (for \Nyx was sparsely populated and they feared loneliness).
There were gentle warm winds that held everyone up and prevented them from falling into the deep (for everyone dreaded falling into the deep). 

In heaven you were so far above the deep that you could barely glimpse the darkness below, and no monsters could reach you.

The \Merkyrans believed that the heavenly towers reached all the way down.
Hence the heavenly towers existed all around them. 
They were just made of subtle spiritual matter and could not be seen nor touched by sinful bodily eyes. 
These towers extended all the way up to the heavens and the top of the world; up to the Crown that was the One God. 

The \quo{Crown} was both God himself and the place at the top of the heavens where God dwelt. 
The Crown symbolized God's nature as a god of law and rulership.
He was a strict god who cast his foes down into the deep. 





\subsubsection{\Iod, \iai and \iath}
In \Merkyran metaphysics a \resphan had three essential components: 
\begin{enumerate}
  \item \Iod, the physical body. 
  \item \Iai, the lower soul.
  \item \Iath, the higher soul. 
\end{enumerate}

\Iai could be destroyed or consumed (although \hr{Merkyrans do not eat souls}{consuming a \resphan's \iai was a sin}). 
\Iath, they believed, was immortal and could never be destroyed. 
(The rebels later \hr{Rebels learn soul-eating}{learned how to eat \iath}.)

Mortals and \ashenbloods also had \iod and \iai, but only true immortals possessed \iath. 





\subsubsection{Myths about \resphan nature}
Ignorant of the truth, the \Merkyrans{} fabricated creation myths. 
One myth claimed that they were created by the Light as angels of good, and that it was their very nature to be good and noble. 
Another, more truthful myth hinted that dark powers had had a hand in their creation, or that their people had committed some terrible crime early on in their existence. 

Whatever the explanation, there prevailed the notion that the \resphan{} race was somehow afflicted by a grievous original sin for which they must atone. 
This made them zealous, fanatic in their crusade to spread what they considered the ways of good. 
With this (vaguely defined) \quo{evil} hanging as a shadow over their history, they must strictly discipline themselves to suppress and overcome this inner evil. 
This made them repressive and intolerant, and they came to expect the same perfection from others as they allegedly strove to achieve for themselves. 

As a matter of course, they saw themselves as destined to rule as enlightened rulers over \humans{} and \nephilim{} at the very least. 

\hr{Myth of Wyrm invasion}{One myth} held that the \resphain had once ruled all of \Nyx and had been driven out by the \hr{Merkyran Wyrms}{Wyrms} and saved by \hr{Merkyran Thanatzil}{\Thanatzil}. 

(This was based on a warped memory of the \aryoth empires that were destroyed by the return of the \dragons.) 





\subsubsection{Plants}
\target{Plants in Merkyran religion}
Plants were important to the \Merkyran religion because the \hr{Nyxian plants}{sparse \Nyxian flora} was vital to the \resphain's survival. 





\subsubsection{Repressiveness}
The \Merkyrans were obsessed with religion, morality, sin, shame, guilt, atonement and duty. Officially they preached humility and shame over their dark past (of which almost nothing was known, and false myths proliferated), but underneath the \facade{} they grew arrogant and domineering, addicted to control and power, the power to control other people's emotions and thoughts. 
Almost like \authorbook{George Orwell}{1984}. 

The \Merkyrans{} disliked the body all kinds of sex and pleasure. 
They preach asceticism and flagellation.

They dyed their black skin white or otherwise bright as a signal that they had repented their original sin and were striving to change, reform.

\Merkyrah{} was just, in a sense. 
But it was also repressive and conservative and conformist. 
Free-thinkers were frowned upon. 







\subsubsection{Search for God}
\target{Search for God}
Once, the \Merkyran{} church experimented with journeying to the innermost reaches of their mind to search for \quo{God}. 
But they found a terrifying, frightening emptiness and recoiled. 

They rationalized this, interpreting that God does not want to be searched for. 
He wants to be blindly believed in. 





\subsubsection{\Thanatzil}
\target{Merkyran Thanatzil}
The \Merkyrans{} had a twisted myth about \Thanatzil. 
It was pretty far from the truth. 
The rebels who found \Semiza{} learned the true story of what happened to \Thanatzil. 

\target{Myth of Wyrm invasion}
One myth held that the \resphain had once built great empires and ruled all the world.
They dwelt in all towers of \Nyx, in all levels, all the way down. 
But they grew sinful and arrogant and decadent, and in their wickedness they awoke the evil \hr{Merkyran Wyrms}{Wyrms}. 
Given power from the \resphain's wicked deeds and thoughts, the Wyrms rose up and destroyed the \resphan empires. 
The Wyrms brought with them the darkness. 
The \resphain were forced to flee ever upward to escape the rising darkness.
But one hero arose who was still faithful to the One God: 
\Thanatzil. 
He gathered his people around him and told them to flee upwards.
He stayed below and sacrificed his life to halt the onslaught of the Wyrms. 
It worked, but at a terrible price:
All \resphain except \Thanatzil's family perished, and the deeps became the dominion of the Wyrms. 
But \Thanatzil's descendants were able to rebuild their race. 






\subsubsection{Towers}
\Merkyran theologians were divided on the issue of what happened at the bottom of the towers.
\begin{enumerate}
  \item 
    Some said that the towers were infinite and continued forever.
  \item
    Some said that the towers were all connected to an infinite and flat bottom floor.
  \item
    Some said that the towers' bottoms were just like their tops: 
    They tapered to a tip and hung in midair.
\end{enumerate}





\subsubsection{Voices of God}
\target{Voices of God}
%They believed to hear the voice of God, but it was really just their own imagination, given shape by the Shoud.
There was a group of fallen \resphain{} who remembered the truth. 
They died, but stuck around as ghosts. 
They were quite weak, but from time to time they were able to communicate with the living \resphain{} in dreams. 

The \Merkyrans{} believed such visitations to be the \quo{voice of God}, but in reality it was just the echoes of a bunch of powerless ghosts. 

Perhaps they were imprisoned by the same Shroud that encased \iquin{} and gave it a sheen of benevolence. 

The \quo{voices} kept the truth hidden because they hated it and wanted their people to remain pure, innocent and good. 

Perhaps the oldest \resphain, who today hold positions as high priests, knew the truth deep down. 
They knew that their religion was a lie and that their god did not exist. 
They created the religion as a means to suppress the truth and keep the \ps{\resphain}{} minds occupied and pacified. 

Perhaps they are the ones faking the \quo{voice of God}. 
Or perhaps they are in league with the ghosts. 

Perhaps the high priests did not initially know the truth, having blocked it out. 
But then they \hr{Search for God}{searched for God}. 
They discovered that there was no God, only a vast, consuming darkness (a vision of \Erebos). 
They realized the truth and decided that they must hide it. 





\subsubsection{Wyrms}
\target{Merkyran myths of Dragons}
\target{Merkyran Wyrms}
The Wyrms were a mythical race of \dragon-like monsters.
They were said to dwell in the deep and symbolized all sorts of evil. 
According to \Merkyran myth the \hr{Myth of Wyrm invasion}{Wyrms had once driven the \resphain from their ancestral lands}.









\subsection{Politics}





\subsubsection{\Umbra{} menace}
\target{Merkyrans fear Umbrae}
\target{Umbra menace}
\target{Resphain fear Umbrae}
The \Merkyrans{} were preyed on by \umbrae. 
They greatly feared the \umbrae. 
In fact, a major pillar of their religion was that it was supposed to protect them from the horrid \umbrae{}. 

When \umbrae{} attacked, the \resphain{} would seek refuge in their churches, which were somewhat protected by magic (but still not entirely safe). 

\citebandsong{Ihsahn:TheAdversary}{Ihsahn}{Will You Love Me Now?}{
  And they gathered\\
  in their halls of justice,\\
  halls of mirrors,\\
  halls of echoes.
  
  And they gathered\\
  in their houses of worship,\\
  within the walls of the unspoken,\\
  sheltered from the rain.
}

These churches should be a bit scary and life-denying. 
Like the book \emph{The House With No Windows}. 

Interestingly, the \umbrae{} by far preferred to attack \resphain. 
They would very rarely attack \nephilim{} and \humans, so the mortals had little to fear from the monsters. 
They still feared them, though. 
Partly because their religion told them to, and partly because it was horrifying to see the gods you worshipped get eaten by monsters. 

Remember also that \resphain{} were weaker in the days of \Merkyrah{} than they later became.

Near the end of the time of \Merkyrah, the \hr{Umbra menace growing}{\umbra{} menace was growing} because \hr{Merkyrah strained the Heart}{\Merkyrah{} strained the Heart of \Miith}. 

Later the rebels would \hr{Resphain learn to command Umbrae}{learn to command the \umbrae}. 

Pureblood \resphain were, if anything, more terrified of \umbrae than \bezedeth. 
The \bezedeth \hr{Ashenblood lesser immortality}{were only Lesser Immortals}, so they knew plenty of things that could kill them. 
Purebloods were True Immortals and usually feared nothing, so the thought of a monster that could permanently destroy them filled them with dread. 





\subsubsection{\Umbra{} siege}
The \hr{Umbra menace}{\umbra{} menace} had another consequence: 
It shaped the dwellings and cities of the \resphain. 
\Nyx{} was a vast Realm, and the \resphain{} would like nothing more than to spread out and conquer it all. 

But the \umbrae{} were a problem. 
See, whenever a small group of \resphain{} were out in the open, it would attract \umbrae.
This made it hard for them to start new settlements. 
They had to build large, compact cities so they could easily summon large numbers of \resphan{} warriors to defend themselves against \umbra{} attacks. 
This kept all \resphain\dash \Merkyrans{} and \hr{Resphan tribes}{tribesmen} alike\dash isolated in small clumps of \Nyx, spreading only slowly and with great difficulty. 

These claustrophobic living conditions were one of the reasons why the different \resphan{} tribes fought and hated each other so much. 
If they disagreed on something, they could not simply go different ways. 
They had to live near each other. 





\subsubsection{Other \resphan{} tribes}
\target{Resphan tribes}
\target{Early Resphan tribes}
\Merkyrah{} did not rule all \resphain{} in \Nyx. 
It lay at war with some other tribes. 
Compare to Deepgate and the Heshette in \authorbook{Alan Campbell}{Scar Night}. 

But \resphain{} were \hr{immortality}{immortal}. 
When they were killed in war they just came back to life a bit later (in the same body, if possible). 

\target{Early diabolist Resphain}
\target{Early Resphan diabolism}
These tribes worshipped dark gods: 
\XzaiShanns{} or cosmic gods that had taken up residence in \Nyx{}, or just had some tenuous connection there. 
Some of those gods \hr{Gods in Nyx}{still lived in \Nyx{} thousands of years later}. 

Some of these tribes were later \hr{Bael'Zerach absorbs tribes}{absorbed into \Baelzerach}. 
Other tribes \hr{Awakening}{Awakened} on their own after hearing the word from the dynasties. 

The rest of the tribes who did not \hr{Awakening}{Awaken} were no match for the Awakened \resphain{}, so they were hunted to extinction and eaten. 









\subsection{Skills and powers}





\subsubsection{Magic}
\target{Merkyran magic}
\target{Theurge}
\target{theurge}
The \Merkyrans had only weak power. 
Their mages were theurges who practiced the magic of \hr{Esheram}{\Esheram}. 

\target{Merkyran sorcery taboo}
All other magic was named black sorcery and forbidden.
Sorcery could grant power and health and longevity, but it brought bloodthirst and madness and thus had to be forbidden and eradicated. 

Not all \resphain were taught magic.
Some were taught basic magic. 
Only few ever became true theurges.















\section{Morphous}
\target{Morphous}
\index{Morphous}
The Morphous were reshaped \resphain\dash the \resphan equivalent of the \human \hr{Shapen}{\shapens}. 

Morphous were made possible by \bane sorceries. 

Most Morphous were \bezedeth who hungered for the strength of purebloods. 









\subsection{Physique}
After being transformed into a Morphous, the \resphan would look normal at first.
But after some centuries the body would begin to mutate and become monstrous. 









\subsection{Psychology}
Morphous tended to go mad when they become to old.
They also became more powerful as they grew older. 

Often, old Morphous would be killed before they grew too powerful and too insane. 









\subsection{Types}





\subsubsection{Bloodhunter}
Bloodhunters were Morphous \resphain.
They were strong and fast fighters.
Even if they were ashenblooded, they could still fly, on some strange grotesque wings. 

No Bloodhunter had ever lived more than 2000 years after being Morphed. 
They were killed in battle. 
They were bloodthirsty and craved battle. 





\subsubsection{Channeller}
Channellers were Morphous \resphain who could channel great amounts of mystical energy to themselves and other allies nearby. 
They were highly treasured both in combat and in peace time. 

The Channeller himself was not extremely powerful.
He could not use all the extra energy himself.





\subsubsection{High Telepaths}
\target{High Telepath}
\target{High Telepaths}
A High Telepath was a Morphous \resphan{} who specialized in telepathy. 
They worked as messengers. 

The High Telepaths were crippled, misshapen wretches with oversized heads and thin, frail bodies.
Their limbs would being to atrophy, as if they were lepers or sclerotics. 
After perhaps 1000 years they could no longer walk.
After 2000 years they could not eat.

After 3000 years they could not move at all and must be fed intravenously.
To this end, they would be connected bodily to one or more \shapen who would act as their mouths. 
They would be tied together with a gross umbilical cord. 

Their numbers included \hs{Shessirar} and \hs{Kizmath}. 

Compare them to the Navigators from \cite{FrankHerbert:Dune} and the Word Bearer supplicants from \cite{BenCounter:BattlefortheAbyss}.















\section[Mystraacht]{\Mystraacht}
\target{Mystraacht}
The fiercest, most chaotic, most belligerent of the \resphain{} (barring the \hr{BZ}{\Baelzerach}). Ramiel belongs to them. 









\subsection{Aesthetics}
The \Mystraacht{} traditionally prefer dark \colours with a \quo{violent} feel to them: 
Black, red (ranging from blood red to fiery orange), sinister purple, deathly gray. 
They glorify their martial prowess and often dress in the trappings of war: 
Metal \armour, or robes designed to remind of \armour. 

\index{beard!\Mystraacht}
\Mystraacht{} men wear beards more often than other \resphain. 
Beards give a masculine, \quo{bestial} image which the \Mystraacht{} like. 
Besides, \Zachirah{} did it and everyone else jumped on the bandwagon. 





\subsubsection{Black Bat}
\target{Mystraacht symbolism}
One of the dynasty's symbols is a black bat with blood-red fangs, eyes, claws and ears. Some of their warriors wear bat-like masks. 

Contrary to what some believe, the black bat has no deep metaphorical significance. 
\hr{CS symbolism}{Unlike the \CiriathSepher}, the \Mystraacht{} do not believe in metaphors and deep meaning. 
They believe in power and fear. 
The bat was simply chosen because it looked physically impressive and could be used to inspire fear.

\lyricsbalsagoth{
  The Splendour of a Thousand Swords Gleaming Beneath the Blazon of the Hyperborean Empire 
  - Part III: 
  Cry Havoc for Glory, and the Annihilation of the Titans of Chaos
}{
  \ldots{} none who gaze in awe beyond the mists and are blessed to behold it shall ever forget the splendour of a thousand swords gleaming beneath the blazon of [the Black Bat of \Mystraacht].
}





\subsubsection{Halls of iron}
\Mystraacht{} citadels tend to be dark, made of iron or the like. 
They are stark and martial, adorned with blades, spikes and trophies of war, but not frivolous luxury (unlike the more foppish \KiriathSepher). 








\subsection{Culture}
\target{Mystraacht philosophy}
The \Mystraacht{} are the most overtly evil of the \resphan{} dynasties. 
They openly embrace both of their heritages: 
The \SitraAchra legacy, with all its parasitism, betrayal and patricide, and the \chaotic{} legacy, with all its violence and savagery. 

In contrast to the \KiriathSepher, who are \hr{Dance}{concerned with etiquette}, fashion and principles, the \Mystraacht{} are much more chaotic and take a pragmatic approach: 
If it works, then it is good. 
If it fails, then it is ill. 

\target{Mystraacht not diplomatic}
But they were also less diplomatic than the other dynasties because of their prideful, violent disposition, so they had a hard time forming alliances. 

\hr{Zachirah}{\Zachirah} formulated the \Mystraacht{} ideology along these lines: 

\lyricsdimmuborgir{Det Nye Riket}{
  V\aa rt hat skal vinne.\\
  V\aa r ondskap skal gro.\\
  A feste seg i unge sjeler.
  
  Den siste krig skal vi vinne,\\
  og de godes blod skal falle som regn.\\
  Deres korte sjeler skal samles.
  
  Vi skal r\aa de over kaos og evig natt.\\
  Vi skal glemme de kvinnelig vikante m\oe dre\\
  og utslette alt.
  
  Et rike skal reise seg\\
  i asken av brennte hjem.\\
  Det er kun en herre hersker.\\
  Vi heller deg, Satan, de sterkes konge.
  
  Din tid er kommet.
}

They see themselves as more \quo{pure}, more true to their legacy of Darkness and Chaos than the other dynasties. 

\lyricsdimmuborgir{Progenies of the Great Apocalypse}{
  We, who not deny the animal of our nature. \\
  We, who yearn to preserve our liberation. \\
  We, who face darkness in our hearts with a solemn fire. \\
  We, who aspire to the truth and pursue it's strength.
  
  Are we not the undisputed prodigy of warfare, \\
  fearing all the mediocrity that they possess? \\
  Should we not hunt the bastards down with our might, \\
  reinforce and claim the throne that is rightfully ours?
  
  Consider the god we could be without the grace. \\
  Once and for all. \\
  Diminish the sub principle and leave it's toxic trace. \\
  Once and for all.
}





\subsubsection{Crime and punishment}
\target{Mystraacht punishment}
The \Mystraacht favoured physical punishment: 
A criminal would be beaten, flogged, maimed or even killed (temporarily in case of a pureblood \resphan). 
After the punishment the crime was considered atoned and the offender's honour was restored. 

Contrast with \hr{CS punishment}{\CiriathSepher punishments}. 





\subsubsection{\Daemon{} summoning}
\target{Mystraacht summon Daemons}
They embrace and endorse Chaos much more overtly than the other dynasties. 
They conjure and bind \daemons{}.
The other dynasties do not approve of this, for several reasons: 
Principles, envy, fear of \ps{\Mystraacht}{} growing power. 

Compare to the story from the game \cite{VideoGame:Diablo} about the Warlord of Blood, a Vizjerei mage who summoned demons and became too powerful and too mad and evil, so the other mages banded together to take him down. 

Maybe this was the reason for \hs{Ramiel's fall from grace}.



\lyricstitle{\emph{Call of Cthulhu} RPG p.118}{
They [the Serpent People] built black basalt cities and fought wars, all in the Permian aera or before. 
They were great sorcerers and scientists, and devoted much energy to calling forth dreadful \daemons{} and brewing potent poisons.}





\subsubsection{Education}
\target{Mystraacht education}
\Mystraacht{} had every new child raised locally where its parents lived. 
They did not feel it worth the effort to gather the children in a school, since there were so few of them anyway. 

\CiriathSepher, on the other hand, \hr{Ciriath-Sepher education}{had one central school where all children were educated}. 





\subsubsection{Government}
\Mystraacht{} was \emph{de facto} a sort of elective monarchy. 
\Zachirah{} assumed Overlordship not just by being the founder, but also by being the biggest, baddest and sexiest of the lot. 

When \hr{Zachirah dies}{he died}, there was no mechanism in place for choosing a successor, so the belligerent \Mystraacht{} squabbled amongst themselves for thousands of years. 

Eventually \hr{Dasteron becomes Overlord}{\Dasteron{} made himself Overlord} by gathering enough support to beat down those who disagreed. 
This had taken more than a thousand years of concentrated work. 





\subsubsection{Knights of the Void}
\target{Dark knights of Mystraacht}
\target{Knights of the Void}
\target{Knight of the Void}
\index{Knight of the Void}
The Knights of the Void were an order of superpowered warriors among the \Mystraacht. 
They were warrior-mages who had pledged themselves to the Other Side, the \SitraAchras. 

Compare them to the \hr{Dark Crescent Knights}{the Dark Crescent knights} and the Lords of Negation in \FLuneNoire. 

Maybe they resemble Revenants from the game \emph{Warcraft III}. 

The dark knights sometimes wore intimidating metal masks wrought in the shapes of bats or monsters. 

\citeauthorbook[p.15]{RPG:Warhammer40000:Necrons}{Andy Chambers et al.}{Codex: Necrons}{
  \textbf{Gaze of Flame:} 
  Flickering witch-fires blaze from the metal death mask of the Necron Lord, chilling the very heart of those who look upon it, stealing away their strength and crushing their courage. 
  
  \ldots
  
  \textbf{Nightmare Shroud:} 
  The worst fears are summoned from the pits of nightmare and thrust into the minds of all those near the Necron Lord,
  Palpable waves of horror radiate from the metal-skinned monster, and all who look upon it will find their courage tested to the very limit. 
}





\subsubsection{Less snobbish}
The \Mystraacht{} are less snobbish than the \KiriathSepher. 
They don't hate the \ashenblooded{} so much. 
They respect people who are strong and capable, not just people of high birth. 





\subsubsection{Sex and gender roles}
\target{Mystraacht sexuality}
\Mystraacht{} \resviel{} are more promiscuous than those of \CiriathSepher{} and \TiphredSerah. 
The \Mystraacht{} believe in the freedom to act on one's desires. 
Sexuality should be embraced wildly and without inhibition. 

Sometimes they have sex in public. 
Often to show off. 

\target{Mystraacht amazons}
Many \resphan{} societies had a \hr{Resviel do not fight}{principle that \resviel{} should not fight in physical combat}. 
\Mystraacht{} had no such rule. 
Many \Mystraacht{} \resviel{} fought as \quo{amazons}. 





\subsubsection{Warrior ethos}
They idolize and glorify their martial strength. 
They see physical and magical prowess as an ideal. 
It signifies being in touch with your true nature as a child of Chaos and Darkness. 

\target{Mystraacht trial by combat}
This also means that \quo{trial by combat} is an accepted means of settling disputes among the \Mystraacht. 
Such fights are to the death, but not \hr{Immortality}{annihilation}. 

Unlike \KiriathSepher, \Mystraacht{} has no law against killing another \resphan, as long as it happens in open, honourable combat. 

\lyricsbs{Exmortem}{Terror Mundi}{
  War emperors.\\
  Apocalyptic death knights.\\
  Cries of battle.\\
  Death before dishonour.
  
  Stressed with wicked intolerance,\\
  blessed with winds of chaos.\\
  Unconquerable demons.\\
  Lick the bones of war.\\
  Release the fury.
  
  Terror Mundi.
}

\lyricsbs{Exmortem}{Gruesome Icons}{
  Ride the storm of mayhem.\\
  Fly with the wings of the damned.\\
  Blackwinged Deathknight.\\
  Your will is supreme.
  
  Praise the fall of [\Merkyrah]. \\
  I'll show you the way to obscurity.\\
  Images of defunct future.\\
  A mirror of the underworld.
}

\lyricsbs{Hate Eternal}{To Know Our Enemies}{
  Prepared for death by this code of \honour, our weapon is our soul.\\
  Chivalrous in our stand with valor.\\
  Herald by many, but feared by all, revered in our essence.\\
  Gallant in our fight we will conquer.
  
  We shall command! \\
  We shall command!
  
  With the mind we control our bodies, even if we rule no longer.\\
  In unification we shall learn. \\
  You can not dare abolish our strengths. With wisdom we shall lead. \\
  It is the way of the warriors.
  
  Propagate the masses, with the embodiment of who we are.\\
  Iconoclastic new age. We shall transform.
}

\lyricsbs{Hate Eternal}{It Is Our Will}{
  It is our will to conquer formidable foes. \\
  we will not concede. We shall not concede. \\
  To call upon all of our strengths. \\
  You will not halt us. You will not halt us.\\
  It is our destiny, it is ever so powerful,\\
  This mass that haunts us.\\
  We shall overcome all.\\
  Rise above we shall, rise above we shall.\\
  Rise above on our road of descent.
  
  To embrace the whole of our power. \\
  We will not concede. We shall not concede.\\
  To overcome hordes of the blind.\\
  You will not halt us. You will not halt us.\\
  To thrust ourselves in destiny. \\
  We will not concede. We shall not concede.\\
  To avenge the fallen spirits.\\ 
  You will not halt us. You will not halt us. 
}

\lyricsbs{Vital Remains}{The Night has a Thousand Eyes}{
  We rose from the Earth and fell from the Heavens, \\
  exalted saints of flesh and will. \\
  Far into the opaque silk that is the night. \\
  We are the provenance of fear 
  and the heralds of the profane. \\
  
  Call us fiends (oh, the apostasy), \\
  call us demons (oh, the apostasy), \\
  but we are just wolves in our right, \\
  hunting and feasting on the human breed, \\
  so infantile and yet so ripe. 
  
  Let us prey.
}









\subsection{History}





\subsubsection{Dominance}
For a long time during the \resphanwars, \hr{Mystraacht dominance during Resphan Wars}{\Mystraacht{} was the most powerful of the dynasties}. 





\subsubsection{Fall}
But \hr{Mystraacht betrayed}{then they fell}. 









\subsection{Politics}





\subsubsection{\Banes}
Ostensibly they are loyal to the \SitraAchra, but in secret they have their own \matrixx{} and are plotting against their creators. 





\subsubsection{Leadership}
\target{Overlord}
They once had a Overlord, but now they have only Princes.

Currently, the \Mystraacht{} are somewhat without direction, and have been stagnating for some thousand years. They need an \apex{} to their \matrixx, but all their worthy candidates fight amongst themselves and antagonize each other. 







\subsubsection{Ramiel's return}
\target{Mystraacht Matrix}
But one day, Ramiel returns to claim the throne of \Mystraacht. 

\lyricsbalsagoth{Dreaming of Atlantean Spires}{
  The Topaz Throne is beckoning,\\
  the jewelled sword awaits my graps.\\
  The Dreaming Gods now grimly brood\\
  in the silence of Atlantean Spires.
}

Ramiel has always desired the \Mystraacht{} throne. 
Before he became a \malach, however, there was another \Mystraacht{} lord who outranked him. 
He has been slain and has no obvious heir. 

Ramiel has sinister dreams about the \Mystraacht{} \matrix{} and the throne. 
Insert a scene with Vizicar/Carzain dreaming about it. 
Really Bal-Sagoth-esque. 

\lyricsbalsagoth{
  Into the Silent Chambers of the Sapphirean Throne 
  (Sagas from the Antediluvian Scrolls)
}{
  Torches glow in silver cressets\\
  in the Temple of the Serpent.
}

Do the \Mystraacht{} have \ophidian{} connections?















\section{\NeoResphan}
\target{Neo-Resphan}
\index{\neoresphan}
The \neoresphain were designed to be the next, improved generation of the \resphan race. 
They were an experiment by \hr{Azraid}{\Azraid} and other visionary scientists. 
They were meant to be the successors of \resphain. 
They are an attempt to perfect the \resphan{} race, \hr{Resphan purpose}{as is their purpose}. 

%Who are they? Are they the very first batch of experimental \resphain, or are they a new, stronger variant? 
Perhaps they even rival the \satharioth{} in power? 

This project was, in a sense, the Cabal parallel to \ps{\Secherdamon} \Vizsherioch-project. 









\subsection{Appearance}
\target{Neo-Resphan appearance}

\begin{itemize}
  \item 
    Maybe they are \bane-like, with semi-blank \bane-like faces. 
    
    Compare to the woman on the cover of \bandalbum{Hour of Penance}{The Vile Conception}.
    
  \item 
    Maybe they can change back and forth between \human-like and \bane-like forms. 
    In their \bane-like form they are hideous and loathsome to behold even for \resphain. 
    
  \item 
    Maybe they resembles Spawn from \cite{ToddMcFarlane:Spawn}, 
    or a Xenomorph from \cite{Movie:Alien}, 
    or Venom from \cite{StanLeeSteveDitko:SpiderMan}. 
    
  \item 
    Alternately, they might look like the angels in the Bible: 
    With six wings, and four faces (the three of which are animal faces).
    And iron teeth and brass nails (or a metal that resembles brass in \colour). 
    And covered with eyes. 
    And ten horns (that may have eyes and even mouths on them). 
    And four arms (each arm under a wing), where each arm has its own face and wings. 
    See especially Daniel 7, Revelation 4, Ezekiel 1 and more. 
    Look up \quo{Hierarchy of angels} on Wikipedia. 
  
  \item 
    Maybe they shine and burn with white or black fire. 
    Beautiful and terrible. 
  
  \item 
    \target{Neo-Resphain resemble Noggyaleth}
    Maybe they are covered with eyes like \noggyaleth.
    Maybe they have other \noggyal-like traits. 
    Maybe \hr{Neo-Resphain and Noggyaleth}{\noggyal matter was used to create them}. 
  
  \item 
    Maybe they are \dragon-like. 
\end{itemize}



\citeauthorbook[\quo{Death and Punishment}, p.243--245]{%
  AlanUnterman:TheKabbalisticTradition%
}{%
  Alan Unterman%
}{%
  The Kabbalistic Tradition%
}{
  He immediately opens his eyes and sees an angel whose width is from one end of the world to the other.
  From the soles of his feet to the crown of his head, he is full of eyes, his garments are fire, his clothing is fire, he is entirely fire and he has a knife in his hand on the end of which a bitter drop is suspended.
  From this drop the man's body will become putrid, and from this his face will turn green. 
}









\subsection{Biology}





\subsubsection{Connection to \bane possession}
\Banes \hr{Banes have no bodies}{had no bodies and had to possess the bodies of others}. 
Perhaps a \neoresphan was a \resphan more-or-less possessed by \banes.






\subsubsection{Mature state of \resphain}
The tranformation from \resphan to \neoresphan was a metamorphosis, like what insects and amphibians do. 
In a sense, the \resphain were neotenic, like axolotls: 
They had the natural potential to metamorph into a more powerful mature form (the \neoresphan form), but they were inhibited and unable to leave their immature, undeveloped \human-like forms. 
The \neoresphan project was intended to bring the \resphain out of their childhood, allow them to break their weak shells.





\subsubsection{\Noggyal connection}
\target{Neo-Resphain and Noggyaleth}
Maybe \noggyal matter was used to create the \neoresphain. 
The \banes \hr{Banes and Noggyaleth}{wanted to merge with the \noggyaleth}, remember.
The \neoresphain with \noggyal genes might have been a step in this direction. 

Maybe the \neoresphain \hr{Neo-Resphain resemble Noggyaleth}{physically resembled \noggyaleth} to reflect this. 









\subsection{History}
The \neoresphain{} were developed during the course of {\SentinelsofMithEmph}. 
\hr{Teshrial}{\Teshrial} partook in \hr{Teshrial's experimental weapon}{some \neoresphan-related experiments} in \hr{Teshrial's quest}{his quest} to slay \QuessanthIshnaruchaefir. 

\hr{Azraid}{\Azraid} led and oversaw the \neoresphan project, but \hr{Azraid never became Neo-Resphan}{he never underwent the treatment himself}. 

At the end of {\SentinelsofMithEmph}, after \hr{Daggerrain falls}{the fall of \Daggerrain}, they were almost ready. 
\Azraid{} predicted that they would be the future of their people, potentially more powerful than \banes, \resphain{} or \satharioth. 





\subsubsection{Conspiracy to take over the world}
\target{Neo-Resphan conspiracy}
There was a sinister conspiracy of scientists and extremists among the \resphain.
They wanted the horrible, super-powered \neoresphain to take over the world, replacing the normal \resphain. 
\Azraid was the sinister mastermind behind this conspiracy, but very secretly.
Everyone suspected \Azraid, but no one knew for certain, and everyone feared to act against the sinister High Lord. 
No one was certain what his role was\ldots{} in anything. 

Some \resphain who learned of the plot might ponder whether the \resphain truly had the right and the destiny to inherit and rule the world, \hr{Resphan purpose}{as they believed was their purpose}. 
Some argued that the \resphain had no such right. 
They were scum who had been responsible for the devastation of \Miith and the \hr{Heart weakened}{withering of the Heart}.
From such a point of view, it was only just and right for the \neoresphain to destroy the regular \resphain and take over. 

Compare this to the scenes in \cite{CodyGoodfellow:RadiantDawn} where \humans defend their \quo{right} as a species to continue to dominate the planet. 
Except the other way around. 









\subsection{Skills and powers}
\target{Neo-Resphan powers}





\subsubsection{Invulnerability}
\NeoResphain are almost invulnerable.
Compare them to the monsters from \cite{CodyGoodfellow:RadiantDawn}. 





\subsubsection{Vampire powers}
\target{increased vampire powers}
The \neoresphain{} have increased vampire powers and can drain life energy from living targets at a distance. 
(\hr{Resphan vampirism}{\Resphain{} are vampiric}, remember.) 















\section{\Resphan}
\target{Resphan}
\index{\resphan{} (plural \resphain)}
The \resphain{} are the progenitors of \humans{}, created to be the heirs of the \SitraAchra. 
They were bred from \nephilic{} stock, endowed with \erebean{} minds and infused with the life-giving Chaotic energy from the Heart of \Miith. 









\subsection{Name}
Singular \emph{\resphan{}}, plural \emph{\resphain{}}. The adjective is \emph{\resphan{}}. 

The word \quo{\resphan} is also used to refer specifically to the male of the species. The female is called a \resvil, plural \resviel. 









\subsection{Biology}





\subsubsection{Blood}
Drinking \resphan blood is very healthy for \humans.
It heals wounds and rejuvenates and extends lifespan.

But it is also addictive.
Psychologically at first.
Repeated consumption over short time leads to physical addiction.
Once badly addicted, a \human will need it regularly or she will die.

Drinking \resphan blood also has a psychological effect of making the drinker instinctively devoted to the \resphan in question.
He or she will effectively fall in love and will feel an urge to serve him.





\subsubsection{Demographics}
\target{Resphan demographics}
At the time of the \hs{Incursion}, the \resphain invaded with an army of 130,000. 
At that time the total \resphan population was 220,000.
They needed a population of 20-30 million \humans to support this many \resphain.
At that time the inhabited part of \Nyx was as large as Europe plus more.

The race had an overweight of males. 
Only about $25\%$ of all \resphain were female. 





\subsubsection{Failed \resphain{} and \humans}
Perhaps some of the early attempts at creating \resphain{} and \humans{} resulted in horrid abominations. 
Some of them still exist, used by the \resphain{} as cattle or slaves. 
Others have survived in the \hr{Wild}{\Wylde}. 





\subsubsection{Healing}
\target{Ketherain heal faster}
A \resphan could heal almost any injury given time, except \hr{Decrepity}{decrepities}. 
The higher \quo{tiers} of \resphain (\satharioth and \ketherain) healed faster than the lower tiers (\thelyadeth and \bezedeth). 





\subsubsection{Lifespan and mortality}
\target{Resphan lifespan}
\target{Resphan immortality}
\target{Ashenblood lesser immortality}
\Resphain had a natural lifespan of 200-300 years. 
After this they would weaken and die of old age like \humans. 

By consuming the essence of \humans and similar lesser beings, a \resphan could grow in power and extend his lifespan to 300-500 years. 
But throughout his life, he would gradually accumulate \hr{Decrepity}{decrepity} and his hunger would grow so that he needed more and more essence each year to stay alive. 
If deprived of essence he would grow even more decrepit in both body and mind (in the form of madness, dementia and deformity) and ultimately die a horrible death. 

By consuming the essence of \resphain and other greater beings, a \resphan could grow even more powerful and live up to 500-1000 years. Such \hr{Cannibal Resphain}{cannibal \resphain} saw their hunger grow quickly. 
Consuming \resphan essence was immediately addictive, and an addict deprived of essence would grow mad and decrepit and die very soon. 

This led to most \resphain becoming preoccupied with seeking out ever more essence in order to keep themselves alive, which ended up \hr{Resphan economy}{forming the basis of their entire economy}.

A minority of \resphain refused to consume \human essence.
These ascetics were weaker than their vampiric kin and lived shorter lives, but they were also happier and more peaceful of mind. 
They died quickly and gracefully and were not subject to the madness and dementia that plagued their vampiric kin. 

Only the \satharioth (and perhaps a few other special individuals) \hr{Sathariah immortality}{could truly live forever}. 

A few of the more powerful pureblood \resphain earned the power of \hs{True Immortality} (the ability to reform after having their physical body destroyed). 

See also the section on \hr{Kinds of immortality}{different kinds of immortality}. 






\subsubsection{Procreation}
\target{Resphan procreation}
\Resphain{} can be born of \resvil{}, \human{} or \nephilic{} mothers, but the father is always a \resphan. 

The children born of a union between a \resphan{} and a mortal (\human{} or \nephil) woman will always be a \resphan, and as full-blooded as the child of a \resphan{} and a \resvil. 

For a male \resphan, orgasm and ejaculation carried with it a permanent loss of essence. 
Some \resphain chose to remain celibate or practiced having sex without ejaculating. 
Most \resphain chose to accept the loss of essence in exchange for the pleasure of orgasm and ejaculation. 
Each orgasm would leave him slightly weakened and his lifespan shortened. 

Even when a \resphan had sex with a \resvil and ejaculated, there was no guarantee that she would conceive. 
A \resvil could only become pregnant voluntarily, never by accident. 
Conception required a permanent investment of essence from the mother, and even then it was not guaranteed to succeed. 
If she successfully conceived and gave birth, she would be permanently weakened and her lifespan shortened.
Hence, despite potentially having lots of sex, \resviel conceived very rarely, and some never reproduced. 
(Unlike mortal women, a \resvil{} remains fertile throughout her entire life. 
Her body grows new eggs periodically, so she can bear a theoretically unbounded number of children.) 

A mortal man cannot impregnate a \resvil; her body will accept nothing less than \resphan{} seed. 

A \resphan's seed could impregnate a mortal woman.
This happened rarely and usually as an accident, but it did happen. 

A mortal woman who gives birth to a \resphan{} child will invariably die in childbirth as the parasitic \resphan{} sucks all life-force out of her. 
She will also suffer while carrying the wicked child\dash pain, nightmares, hallucination and possibly madness. 

When it is discovered that a mortal woman bears a \resphan{} child, she will sometimes be eaten. 
This is a great \honour for her. 
Some women escape or are released. 
They give birth to \hr{Beuzed}{\bezed} children. 





\subsubsection{Purebloods and \ashenbloods}
\target{Pureblood Resphain}
\target{Ashenblood}
The \resphain{} are divided into four social classes:
\begin{gloss}
  \gitem[\satharioth]{\sathariah}
  The highest of \resphan{} nobility, with the stolen blood of \Nexagglachel{} in their veins. 
  See the \hr{Sathariah}{main section about the \satharioth}. 
  
  \gitem[\ketherain]{\ketheran}
  \target{Ketheran}
  Higher nobility; the descendants of \satharioth. 
  They, too, carry the \draconian{} blood, but diluted.   
  
  \gitem[\ruistheleth]{\ruisthel}
  \target{Ruisthel}
  Lower nobility. 
  They are descendants of the \satharioth, but born before they became \satharioth. 
  As such, they carry no \draconian{} blood, but still stand above the \thelyadeth. 
  
  The word comes from a root \quo{ruis} meaning \quo{old, original, venerable}\dash also found in \quo{\hr{Ruishagh}{\Ruishagh}}. 
  
  \gitem[\thelyadeth]{\thelyad}
  \target{Thelyad}
  Full-blooded \resphain, but with no \sathariah{} blood. 
  The \ketherain{} sometimes disparagingly call them \quo{plainbloods}. 
  
  \gitem{pureblood}
  Collective term for all full-blooded \resphain{}, born of a \resvil{} mother. 
  Only purebloods have wings. 
  
  \gitem[\gessurim]{\gessur}
  \target{Gessur}
  Full-blooded \resphain, but born of a \hr{Yurid}{\yurid} mother. 
  They were considered below \thelyadeth{} in status, but still above \bezedeth. 
  As purebloods they counted as members of their father's dynasty. 
  
  \gitem[\bezedeth]{\bezed}
  \target{Bezed}
  \target{Beuzed}
  Half-breeds, born of a \resphan{} father and a \human{} or \nephilic{} mother. 
  They were considered inferior by their full-blooded brethren. 
  They lacked wings.
  They had pale white skin instead of brown.
  Their hair and feathers were white or gray (instead of the wide variety of colours that normal \resphain exhibit). 
  They were smaller and weaker than purebloods. 
  They are disparagingly called \quo{\ashenbloods}\dash a reference to how they kill their mothers during birth. 
  
  \Human-born \resphain{} are short and skinny. 
  
  \index{beard!\bezed}
  \Nephil-born \resphain{} are broader, heavier and more hairy, but not necessarily taller. 
  If male, they will have plenty of beard. 
  
  The \bezedeth{} are particularly reviled because they take up space in the \hr{Matrix}{\matrix} and consume precious \hs{Heart} energy but will never be able to give anything back to the \resphan{} race, because they are sterile. 
  
  A \bezed{} \resvil{} can never become pregnant. 
  \Bezed{} males can sometimes impregnate a mortal woman (never a \resvil), but the child is usually stillborn and often deformed. 
\end{gloss}





\subsubsection{Relation to \aryothim}
\hr{Resphan-Aryoth relationship}{The \resphain were descended from \aryothim to some extent}.





\subsubsection{Sexual maturity}
\target{Resphan sexual maturity}
\Resphain grew and matured half as fast as \humans. 
They became sexually mature somewhere in their 20s and were fully developed at about age 40. 

\Thanatzil was a special case. 
He matured in less than ten years. 









\subsection{Culture}





\subsubsection{Architecture}
\target{Resphan architecture}
\index{architecture!\resphan}
The \resphain{} liked tall, spindly towers with bridges and causeways. 

Very unlike the bulky, bloated edifices of \hr{Draconic architecture}{\draconic{} architecture} and \hr{QJ architecture}{\quiljaaran{} architecture}. 

See also the sections on \hr{Nyx}{\Nyx} and \hs{dark ancient cities}. 





\subsubsection{Bloodwine}
\target{Bloodwine}
The \resphain{} drink a class of beverages called blood wines. 
They are sort of wine-like, but with the blood of intelligent creatures distilled into them, and with life force bound in it. 
It is brewed with some special stuff that keeps the blood from coagulating. 

One of the more popular types of bloodwine is \ethylshe. 

All \resphan{} purebloods are expected to know stuff about wine. 
It is a culture thing. 
It shows you know culture and fine arts and stuff. 
It's a way of signalling status. 

Bloodwine with \resphan{} blood is the best kind, surpassed only by bloodwine with \draconic{} blood. 





\subsubsection{Cannibalism}
\target{Cannibal Resphain}
\target{Resphan cannibals}
In order to grow in power and \hr{Resphan lifespan}{extend their lives}, some \resphain took to consuming the essence of other \resphain and other greater beings. 
These cannibal \resphain grew much more powerful (physicall and magically) than their brethren, but at a price. 
Eating \resphan essence was instantly addictive, and the withdrawal symptoms included madness and mutations (\hr{Decrepity}{decrepity}). 
Their thirst for essence would grow with time, and they would need to consume more and more \resphain to stay alive. 

Consuming \resphan essence caused the cannibal's skin to sport black spots. 
These spots would grow as he consumed more essence, eventually turning his skin pure black. 
This was essentially a kind of decrepity that could not easily be concealed. 

Obviously such cannibals were a menace to other \resphain around them, so many \resphan cultures outlawed the practice and hunted down and killed cannibals whenever possible. 
In other cultures, an elite of cannibals rose to power and became the rulers. 
These cannibal rulers would invariably lay down laws to restrict and limit cannibalism in order to ensure social stability. 
\Mystraacht was one such culture. 

Some cannibals became \hr{Resphan pirates}{outlaw pirates}. 





\subsubsection{Citadels, demesnes and manors}
The \resphain{} have most of their citadels in \hr{Nyx}{\Nyx}, where they feel most at home are are in contact with their roots and their \hr{Dweomer}{\dweomers}. 

They also have \emph{demesnes} (estates with farmland). 
These are not in \Nyx{} but in \hr{Azmith}{\Azmith} and other \hr{Tembrae}{\Tembraean} Realms, since \hr{Nyx is a parasite Realm}{\Nyx{} has no fertile farming land} and must steal from other Realms. 
Each demesne has a central manor from which the \resphain{} rule. 







\subsubsection{Economy}
\target{Resphan economy}
\target{Jal}
\index{\jal}
Each \resphan, once he began consuming human essence, \hr{Resphan lifespan}{would need ever more and more essence to stay alive}. 
This ended up dominating the entire \resphan economy. 
The \jal\dash a unit used to measure essence\dash became their currecy. 
Everything came to be priced and measured in \jal. 
Every old \resphan's life became a struggle to secure enough essence to stay alive.
Every young \resphan's life became a struggle to build a career and an income that would allow him to keep himself alive in old age. 

Moreover, \resphain could only \hr{Resphan procreation}{procreate by sacrificing some of their essence}. 
A parent would be permanently weaker and require more \jal to stay alive. 
This discouraged them from having children. 
This was a terrible Prisoner's Dilemma: 
The well-being of their civilizations as a whole relied on new \resphain being born and the old \resphain dying when they no longer pulled their weight. 
But the \quo{old guard} refused to die and used their entrenched position to drag the entire \resphan race down in stagnation. 

\target{Resphan farms}
The \resphain built great farms all across the Realms where they would raise food for themselves and their captive \humans.
They also built mines and factories to supply their empires. 
The agriculture and industry were tended by slave \humans and \nephilim, downtrodden and controlled in the most totalitarian manner. 
These \humans and \nephilim were mentally starved because of their miserable lives, and so their essence was feeble and barely edible. 
Those \humans intended for \resphan consumption were bred elsewhere\dash the great farms existed mostly to feed this \human livestock. 

Some \resphain\dash especially \hr{Cannibal Resphain}{cannibals}\dash instead lived as outlaws and pirates, raiding farms instead of tending their own.
These pirate \resphain often looked down on their \quo{farmer} kin and saw themselves as a superior culture of warriors and predators.





\subsubsection{Hierophant, Truespeaker and Blood Warlord}
\target{Hierophant}
\target{Truespeaker}
\target{Blood Warlord}
The highest-ranking \resphan titles, at least in \Merkyrah, were that of Hierophant, Truespeaker and Blood Warlord. 





\subsubsection{The Host}
\target{Host}
\index{Host, the}
The Host was the combined army of all \resphan Dynasties.





\subsubsection{Language}
The \hr{Resphan language}{\Resphan{} language} (spoken by most \resphain, but not all) is based on Hebrew. 

\Mystraacht, \CiriathSepher, \TiphredSerah{} and \Kezerad{} all speak the same language, but in different dialects. 
So do some \Baelzerach, but some of them speak completely different languages. 

It is worth noting that this language uses a Spanish-style \quo{rolling} R. 





\subsubsection{Morality}
Some \resphain{} believe that they truly are the noble and good angels that they style themselves to be. They see the \dragons{} and their spawn as dangerous, vicious savages, and themselves (and, perhaps, their \SitraAchra sires) as a superior civilization with an inborn right to rule. So they truly believe that what they are doing is morally right. Remember, \trope{UtopiaJustifiesTheMeans}{Utopia Justifies the Means}.

Others are more amoral in their outlook and simply see the \secretwar{} as a struggle for survival between two peoples who cannot and will not coexist. (See section \ref{Fighting for survival}.)





\subsubsection{Myths of vanquished monsters}
\target{Resphain vanquish monsters}
The \resphain had their own legendary tales of vanquished monsters. 
Before the coming of the \resphain, \Miith was dominated by monstrous, cruel and incredibly ancient empires of the \dragons and \ophidians. 
The \resphain came to \Miith and began pushing back these Elder monsters to claim the world.
\Miith was their birthright because they were the superior race; more beautiful and perfect, unlike these bestial and hideous monsters that inhabited \Miith before them. 

See also the \hr{Myths of vanquished monsters}{Iquinian myths of vanquished monsters}. 





\subsubsection{Piracy}
\target{Resphan pirates}
Some \resphain\dash especially \hr{Cannibal Resphain}{cannibals}\dash instead lived as outlaws and pirates, raiding the \hr{Resphan farms}{farms of other \resphain} instead of tending their own.
These pirate \resphain often looked down on their \quo{farmer} kin and saw themselves as a superior culture of warriors and predators.





\subsubsection{Protectors}
\target{Resphan Protector}
\index{Protector}
The \resphan Protectors were a special order of secret warriors.
They were tasked with hunting down invasive elements that lay hiding among the \resphain.
Threats such as the \hs{Nether Ones} and \hr{Umbra zombies}{\umbra zombies}.

Compare to the Makai Knights from \cite{TV:Garo}. 

One such Protectress was \hr{Essaryn}{\Essaryn}. 

Every Protector had a homunculus in her body. 
Compare to the mannikin in \cite{RobertBloch:TheMannikin}.
The homunculus was a reshaped \hr{Nether One}{\mothlan}.
It gave her supernatural strength and resilience and the ability to detect \mothlain in disguise.
But the homunculus grew and grew until one day it took over control of its host. 
Before that happened, a \hs{Lawbringer} would come and slay the Protector.





\subsubsection{The Purpose}
\target{Resphan purpose}
The \resphan{} race had a big purpose: 
To improve themselves, to evolve, and finally to attain perfection. 

\target{Resphain grow stronger}
And they did. 
Their numbers might continue to dwindle, but the ones who survived were the strongest, and they kept growing stronger. 

It helped that \iquin{} gave them a lot of power. 

\target{Resphan experiments on Humans}
The \resphain practice a lot of eugenics.
They want to improve and purify their race. 
To this end, they perform a lot of experiments on \humans:
Breeding and sorcery and medicine. 
They hope to find results from \human research that will carry over to the \resphain themselves. 
Compare to the Nazis and their race theory. 
Especially \Mystraacht had some strict ideals of physical purity, strength and health, tying in with their macho warrior ideology.

A few of the innermost Cabalists knew about the deeper, more hidden purpose: 
The \hs{Unification} of the \baneking \Voidbringer and the \noggyal \hs{mother-mass}.
Or at least they knew the name \quo{Unification}. 
Only the very top level, such as \Azraid, knew what it meant.
Even Ramiel did not learn this until late. 





\subsubsection{Religion}
\target{Resphan religion}
The \resphain had almost no religion.
They did not form any formal religious beliefs because they feared to think about it.
If they began to ponder the nature of the universe and the meaning of life, they would have to factor in their own nature, and thus their own origin.
They knew their origin: They were engineered by the \banes.
Try as they might, they could not escape that fact.

They revered the Lord of the \SitraAchra, but not much.
Some few took to worshipping the \SitraAchras as gods, but most could not bring themselves to do that.
They \hr{Resphain fear Banes}{feared the \banes} and did not like to talk or even think about them, for the \banes were horrible monstrosities.
They served the \banelords in name, but in their everyday lives the \resphain would prefer to pretend the \banelords did not exist.
They certainly had no desire to worship them.

\target{Other Side}
\index{Other Side, the}
They called the \SitraAchra \quo{the Other Side} in order to distance themselves from it and avoid having to think about what it truly meant. 

So the \resphain shied away from the subject of the meaning of life.
Therefore they never developed much in the way of religion and mythology and philosophy.

They had some kind of psychological need for religion, like mortals did, but they filled that need with superstition and informal magibabble such as invoking their \matrices as \quo{gods}, as if the \matrices could be prayed to.
They saw their \matrices and the Heart as something great and powerful to swear and curse by. 
They knew the \matrices did not answer prayers, but like some atheists do, they sometimes succumbed to the psychological urge to pray.
Besides, they \emph{knew} how the \matrices worked (sort of), so it could not form the basis of a real religion. 
(Religions are based on ignorance and fiction and mumbo-jumbo.)







\subsubsection{Sex and gender roles}
The \resviel{} are highly prized for their scarcity and because they are the key to propagating their race. 
There is a lot of chivalry in the \resphan{} cultures (although some dynasties more than others). 

A \resvil{} must always be kept as healthy as possible, because it makes her more fertile. 
Therefore, it is customary at meals that a \resvil{} always be offered the best part of the body that is eaten. 

\Merkyrah{} \hr{Merkyran monogamy}{practiced monogamous marriage}, but that was crazy, so pretty much all later \resphan{} dynasties and tribes abolished that practice in favour of polyandry or promiscuity.

In most \resphan{} societies, a \resvil{} had a duty to do her utmost to get pregnant. 
There was great status in getting pregnant and bearing children. 
During her fertile periods, a \resvil{} was expected to have as much sex as is reasonably possible, but also required to be selective, so as not to risk letting an inferior \resphan{} breed. 
So she should have sex with as high-quality \resphain{} as possible. 
She held the key to her race's future in her womb, and it was her duty to do her best to become pregnant. 
(Outside her fertile periods she was more free to have sex as she chose.)

\target{Resviel do not fight}
For this reason, \resviel{} were also usually not expected to fight as warriors. 
They should support the \resphain{} in war, but not fight on the frontline. 
They were too valuable for that. 
\Mystraacht{} abolished this principle, though, and \hr{Mystraacht amazons}{had plenty of \quo{amazon} \resviel}. 





\subsubsection{Vampire lords}
\target{Resphan vampire lords}
\target{Vampire Lord}
\target{Vampire Lords}
Some \ashenblood \resphain lived among mortals as vampires. 
They learned to live on the blood of mortals in a special way. 

They discovered that with certain spells, a mortal could drink \resphan blood and become a super-powered undead, a \quo{\reaver}. 
A \reaver \quo{sort of} counted as an immortal, so a \bezed \resphan could survive by drinking the blood of \reavers (but leaving them their souls) and devouring the souls of mortals. 
The \reavers, in turn, were hopelessly addicted to \resphan blood. 
They had to regularly drink the blood of the \resphan who had sired them or they would die within a matter of months or weeks. 
Only their sire's blood could sustain them, no other \resphan's. 
This meant the \reavers would do anything to keep their sire alive. 
The sire also had great psychological power over his \reavers. 
They were practically his slaves. 

The \reavers also had to kill mortals and drink their blood to survive. 

Only a \bezed could create \reavers from his blood. 
The spell would kill any mortal that attempted to drink the blood of a pure \resphan.

Compare them to the Vampire Counts from \cite{RPG:Warhammer}. 

Vampire lords would sometimes set themselves up as lords or gods, lording over mortal servitors. 
Examples include:

\begin{itemize}
  \item 
    The gods of \hr{Uruthar}{Uruthar}. 
  \item 
    The \quo{Cold Wraiths} that lorded over mortals in the realm of \hr{Thule}{\UltimaThule}. 
\end{itemize}







\subsubsection[Yurideth]{\Yurideth}
\target{Yurid}
\index{\yurid}
The \resphan{} dynasties kept some captive \resviel{} as sex/breeding slaves. 
They were called \yurideth{} (singular \yurid{}) and were universally looked down on with contempt. 
They were slave whores who had lost ownership of their cunts and wombs, otherwise a \resvil{}'s most prized possession.
They were kept imprisoned and subdued with spells so they could escape or defend themselves. 
They were raped continuously by \resphain{} deemed fertile. The purpose was to get the slaves to breed. 
Every dynasty wanted to breed more children. 

The word \emph{shiphchah} or \emph{shifchah} is Hebrew for \quo{maidservant} or \quo{slave girl}. 

When it was detected that a \yurid{} was pregnant, she was quickly sedated and sent into a sorcerous sleep, to prevent her from harming herself and the baby. 

The \yurideth{} hated their fate and would often try hard to spite their captors by ensuring that they do not bear any children. Some of them got their spirits broken, however, and ended up as wretched, servile things that willingly tried to please their captors. 
(These were considered repellent abominations by their fellow \resphain{}, including their captors. Such submission was unworthy of a \resvil{}. A \resvil{} was supposed to guard and protect and take care of her womb with dignity.)

\Resphain{} born as the children of \yurideth{} were considered to have lower status than \thelyadeth. They were called \hr{Gessur}{\gessurim} (singular \hr{Gessur}{\gessur}). 

The Cabalist dynasties only openly admit to having \yurideth{} captured from \Kezerad{} and \Baelzerach. 
But in reality, they also have \yurideth{} captured from each other, kept secret. 
Many of these have lived as \yurideth{} ever since before the days of the Cabal, but some have been kidnapped in the time since then, when the dynasties were officially allies. This is extremely secret. 
(Although everyone suspects that it is going on.)

As part of the \hs{Consolidation} (the big process of forming the Cabal and allying the three Cabalist factions), every dynasty released several \yurideth{} and let them return to their families. They told the other dynasties that they had now released all of their Cabalist \yurideth{}, but they all lied and kept some in their dungeons/harems.
At the time of the \thirdbanewar, there were some \resviel{} alive who carried deep scars from having spend hundreds of even thousands of years in slavery as \yurideth{}.

\Yurideth{} are kept in seraglios. A seraglio is often also equipped as a debauched pleasure chamber. Or torture chamber, for the unfortunate \yurid{}. The slave whores have no rights, remember. \Resphain{} with sadistic tendencies to go them to inflict all the torture on them that they cannot inflict on free \resviel{}. Some \resphain{} have a nasty habit of taking it out on a \yurid{} whenever they have some grudge against a \resvil{}.

Each dynasty has about a dozen \yurideth{}. It is hard to keep the slaves healthy, for many reasons (among other things, the spells used to sedate them are unhealthy), so they tend to die after some few centuries. 

Some \yurideth{} are maimed, with spells cast on them to suppress their regeneration. 
Cutting off the wings is particularly common, since it is highly demeaning and de-\resphan{}izing. It marks them as less than pureblood \resphain{}, little better than \bezedeth{} or even mortals.

But a \yurid{} still had hope. 
If she escaped, she could heal all her physical wounds (including regrow her wings) and perhaps come to terms with her psychological wounds. 
\Resviel{} were strong. 









\subsection{Equipment}
\target{Resphan equipment}
See also the general section on \hs{technology}. 
(Really. Do it.)





\subsubsection{\Bane sorcery and forbidden books}
\target{Resphain and forbidden books}
The \resphain possessed a number of \quo{forbidden books}, pertaining to the \SitraAchras and their magic or other dark cosmic phenomena. 
These were dark and mysterious works of a blasphemous nature, horrible and repulsive because they went against the world-view that the \resphain liked to maintain. 

Especially, everything that had to do with the \SitraAchras was considered \quo{blasphemous} and \quo{banned}. 
\hr{Resphain fear Banes}{The \resphain feared the \banes} and hated to talk about them or remember the fact that the \banes existed.

The \resphain had a number of texts on \SitraAchra sorcery.
They could cause madness even in \resphain.

The books were not literally forbidden. 
It was not illegal to read them, but it was considered in very bad taste. 
No respectable \resphan would sully himself with such dark sorcery. 

The poem \WanderersInDarknessEmph was one such \quo{forbidden} text. 

But dark though the \pps{\resphain} books might be, \hr{Dragons have dark knowledge}{the \dragons possessed even darker knowledge}. 





\subsubsection{Crystal technology}
\index{technology!crystal technology|see{crystal}}
\index{crystal}
\target{Resphan crystal technology}
\target{Resphan crystals}
\target{Glass armour}
The \resphain{} had technology based on crystal. 
They wore \armour made of glass onyx and other crystals. 





\subsubsection{Monsters}
The \resphain{} tame monsters and use them as beasts of war. 
These include the \hr{Umbra}{\umbrae}. 

And also a species of crawling, somewhat lobster-like things (Cthulhu horror!). 
These things crawl on the sides of the tall buildings of \Nyx. e
The \resphain{} outfit them with cannons, like in the movie \cite{Movie:D-War}. 





\subsubsection{Technology}
\target{Resphan technology}
\index{technology!\resphan}
The \resphain{} inherited much of the \hr{Bane technology}{\ps{\banes}{} technological artifacts and knowledge}. 

They had them \hr{Merkyran technology}{already in \Merkyrah}. 
Back then they didn't know what to do with it. 
In later ages they learned much more. 

The \resphain{} were inherently much more creative than their \bane{} sires and were able to create new inventions. 
But at the time of the \thirdbanewar they were still far from the level of the \voyagers, and there was still much of their stolen \voyager{} technology that they did not understand. 

Some pieces of technology the \resphain used included:
\begin{itemize}
  \item \hr{Glow-moss}{\Glowmoss}.
  \item \hs{Graph-glass}.
\end{itemize}

\target{Resphan dead technology}
\Resphan technology focused much on metal, glass and crystal.
They are supposed to have a \quo{dead}, \quo{artificial} theme.
As opposed to the \dragons, who, \hr{Dragon living technology}{with their living technology}, have a more \quo{living} theme. 






\subsubsection{Weapons}
\target{Resphan weapons}
A lot of \resphain{} dual-wield weapons. 
But not huge-ass weapons like \senain{} or \belthradeth. 
They are too long and unwieldy. 

See also the section on \hr{Resphan equipment}{\resphan equipment}. 

\Resphan swords and other melee weapons were covered with fields of shimmering energy.
Like forceblades from \emph{GURPS} and vibro-blades from \emph{Rifts}.
In all sorts of fantastic \colours. 

Remember that different weapons are connected to the different Paths of \hr{Resphan martial arts}{\resphan martial arts}. 



\begin{gloss}
  \gitem[\belthradeth]{\belthrad}
  \target{Belthrad}
  \index{\belthrad}
  The \belthrad{} (plural \belthradeth) is a class of sword, favoured by \Mystraacht. 
  A \belthrad{} is shorter and broader than a \hr{Senaan}{\senaan}. 
  It has a shorter range, but is faster and packs a meaner bite. 
  It is considered a less sophisticated and more brutal weapon. 
  
  The blades \hr{Ascaril}{\Ascaril} and \hr{Scaleron}{\Scaleron} were \belthradeth. 
  
  
  
  \gitem{guns}
  \target{Ghijed}
  \index{\ghijed}%
  They often wield pistols. 
  A typical \resphan{} pistol is the \ghijed{} (plural \ghijedeth). 
  
  
  
  \gitem[\kilghain]{\kilghan}
  \target{Kilghan}
  \index{\kilghan}%
  \Resphain{} almost never use conventional shields. 
  It is not effective. 
  Instead, they wear vambraces on their wings, called \kilghain{} (singular \kilghan). 
  These are made of metal or crystal and used to parry attacks. 
  They come in many shapes and sizes.
  Some are narrow ridges on the very edge of the wing, others are big things covering the entire wing. 
  (Like the metal wings seen on the art for \cite{SymphonyX:ParadiseLost}.) 
  
  
  
  \gitem[\ruthiel]{\ruthil}
  A \ruthil{} (plural \ruthiel) was a \resphan{} sword, designed by \TiphredSerah. 
  They were shorter than \senain{} and ligher and slimmer than \belthradeth. 
  They were especially favoured by \resviel, who usually relied on speed and skill rather than brute force. 
  
  \Ruthiel{} were often dual-wielded or used with a sidearm such as a pistol. 
  
  
  
  \gitem[\senain]{\senaan}
  \target{Senaan}
  \index{\senaan}
  A \senaan{} (plural \senain) is a \resphan{} sword, designed by \CiriathSepher. 
  It is extremely long and slim, because it must reach further than the \resphan{} wielder's wings, and it must be usable against large opponents such as \dragons. 
  It has a long haft. 
  It resembles a Japanese \zanbatou, or almost spear or naginata. 
  
  The \senaan{} is probably the biggest \resphan{} weapon that can still be called a sword. 
  There exist even longer weapons that resemble lances or halberds more than swords. 
\end{gloss}









\subsection{Physique}





\subsubsection{Appearance}
\Resphain{} look like \humans, but taller and more beautiful, more perfect. 
Most markedly, they sprout a pair of great feathered wings on their backs. 
On some \resphain{} these wings are tattered and torn. 
(Why? 
Obvious candidates for this are the \Kezeradi{} survivors, and maybe Ramiel, if he is able to assume \resphan{} form at times, before he has regained his full memory (and, with it, his full perfection).)

\Resphan skin was normally dark brown. 
\Resphan hair and feathers varied widely and could take many colours: Black, gray, brown, red, white, yellow or even orange.
Many dyed their hair (most commonly in white or silvery \colours, occasionally blues or purples). 





\subsubsection{Beards}
\index{beard!\resphain}
Male \resphain could grow beards. 
In \Mystraacht most men wore beards. 
In \CiriathSepher most males shaved them off. 

\Nephil-born \hr{Ashenblood}{\ashenbloods} had much more body hair than other \resphain.





\subsubsection{Beauty}
To \human{} eyes, the \resphain{} seem superhumanly perfect and beautiful\dash which they are, for they are the superbeings of whom \humans{} are but a shallow copy. 
Moreover, \humans{} were designed as slaves of the \resphain, so they are genetically disposed to love, admire and worship the \resphain. 
A \resphan{} visage inspires instinctive feelings of love and subservience in a \human. 

To \nephilim, \resphain{} appear mighty, godlike, larger-than-life, but not necessary glorious and lovable. 
Rather, their beauty is a cold, alien, frightening kind. 
Still, they have a certain alluring aura of power and mystery, 
\Nephilim{} (and other creatures) may serve \resphain{} out of fear or awe, but not out of the same instinctive love and admiration that \humans{} do. 





\subsubsection{Ethnicity}
In terms of facial traits the \resphain{} most closely resembled East Asians such as the Japanese. 





\subsubsection{Power}
It was known that \satharioth were more powerful than other purebloods, and that purebloods were more powerful than \bezedeth. 

It was commonly believed that \ketherain were more powerful than \thelyadeth, but this was a dubious assertion.


See also the section on \hr{Dragons vs Resphain in power}{\dragons versus \resphain} and about \hr{Umbra power}{\umbra power}.






\subsubsection{Size}
\target{Resphan size}
An adult male full-blooded \resphan{} is typically 300 cm tall. 

The \resviel{} are shorter. 
200-250 cm on average. 
They are physically not as strong as the \resphain, but faster and more agile. 

\Bezedeth{} are smaller than purebloods.

\target{Resphan vs Aryoth size}
\hr{Aryoth size}{Compared to an \aryoth}, a \resphan is as tall or taller, but much less massive. 








\subsection{Politics}





\subsubsection{\Banes}
The Cabalist dynasties of \KiriathSepher, \TiphredSerah{} and \Mystraacht{} served the Lords of the \SitraAchras, but they preferred not to admit it. 
They occasionally swear by the \banelords, but rarely. 
They were hesitant to admit that there existed those even mightier and darker than they. 
The vain \KiriathSepher{} most of all.

\target{Resphain fear Banes}
Moreover, the \resphain feared the \banes. 
\Resphain feared \banes because \hr{Banes eat souls easily}{\banes could eat souls very easily}. 
That was what they \emph{did}. 
A few \lesserbanes were, on average, no match for a \resphan. 
But on the off chance that the \banes were to overpower the \resphan, they would be able to eat him with no trouble at all. 
The \resphain knew this, and it made them uneasy and fearful around \banes 
Even a single \lesserbane was enough to badly unsettle a \resphan. 
The \resphain recognized the Banes as the Cosmic Horrors they were.

This also allows me to have more \trope{CosmicHorror}{Cosmic Horror} feeling in the story, even from an immortal POV.

A consequence of this was that any books and the like pertaining to the \SitraAchras and their magic, and other dark cosmic phenomena, \hr{Resphain and forbidden books}{were considered \quo{forbidden}}. 

Note: 
The correct term was \quo{Lords of the \SitraAchras}. 
\quo{\Banelord} was considered a crude, rude term. 
By the \resphain, that is. 
The \banes themselves did not care. 





\subsubsection{Dynasties}
The \resphan{} nobility is divided into five great dynasties: 
\KiriathSepher, \TiphredSerah, \Mystraacht, \Kezerad{} and \Baelzerach. 

The dynasties of \KiriathSepher{} and \TiphredSerah{} are loyal to the \SitraAchras. 
\Mystraacht{} claim loyalty but secretly plot against the \SitraAchras. 
\Kezerad{} and \Baelzerach{} have forsaken their creators entirely. 

Some of the rebels are the \satharioth{} who drank the blood of \Nexagglachel{} and inherited his greed, his ambition, and \hr{Curse}{above all his hatred of the \banes}. 
These \resphain{} see themselves as true \Miithians, the ultimate heirs to the legacy of \Miith{} and \Erebos{} alike, possessing both \nephilic, \draconic{} and \bane blood. 

The dynasties count as members only pureblood \resphain{}. 
The \ashenblooded{} commoners are not members of the dynasties, although most of them are in service to one dynasty. 





\subsubsection{\Noggyaleth}
See the section on \hr{Resphain and Noggyaleth}{\resphain and \noggyaleth}. 









\subsection{Psychology}





\subsubsection{Delight in conflict}
\target{Resphain love conflict}
\Resphan psychology differed from \human psychology.
In real life it is often asserted that all true happiness derives from \quo{love}, and that all other desires are hollow and unfulfilling in the end.
This may or may not be true in RL, but it was not true for the \resphain.

\Resphain derived genuine pleasure from conflict, from living out \quo{negative} emotions such as anger, hate and pride.
To some extent, they \emph{enjoyed} anger and pain.
It was an affirmation that they were alive and active.
Pain reminded them of their bravery and daring.
Anger reminded them of their will, their goals and motivation.

See also the section on how \hr{Races love war}{the immortals loved war}.

\textbf{Alternative idea:} 
\target{Ketherain love conflict}
\Resphain normally had \human-like motivations.
But the \satharioth and \ketherain had a \draconian taint (and hence a \xs taint) in their minds that drew them towards chaos and violence and hate and war. 
The voice promised them happiness, but it lied.
Only \dragons could \hr{Dragons love conflict}{find true happiness in violence and hate and chaos}. 

\hr{Conspiracy against the Human Race}{Such was the horror of the Shroud.}





\subsubsection{Driven temporarily mad}
The \resphain were driven to a crazed fury during the War of Awakening (the rebellion against \Merkyrah).
They were mad with bloodlust and with the desire for change.
It is what made them so insane and violent.
It is what made even sensible and compassionate \resphain commit terrible, bloody atrocities.
It is what caused them to embrace such a twisted \trope{ReligionOfEvil}{Religion of Evil} and wage a war of genocide against their own people and their hapless servitors.
Their true, chaotic power brought a terrible rush.
Only later as they got more used to their powers did they learn to get over this rush and still maintain a level head.
Many came to repent their evil deeds during the rebellion.
Many of these repenting ones ended up joining \Kezerad.





\subsubsection{Possessiveness}
\target{Resphain are possessive}
Male \resphain{} are always possessive of their women. 
Every \resphan{} wants to \emph{own} his \resviel{} and thus control their pussies and wombs. 
But it never happens. 
The \resviel{} are too strong to let themselves dominate. 

Except in the case of \Zachirah, who was actually able to keep \hr{Zachirah's slave Resviel}{slave \resviel}. 
That was one of the reasons he was so admired. 





\subsubsection{Racial memory}
\target{Resphan racial memory}
The \resphain{} have inherited some degree of racial memory from their \SitraAchra forebears. 
At least, the oldest, greatest, most pureblooded of the \resphain{} have. 
They remember vague impressions of the \hr{Voyager}{\voyagers}. 





\subsubsection{Social intelligence}
\target{Resphain are social}
\Resphain{} are social creatures, very much conscious of social things such as etiquette and fashion. 
(One particularly strong manifestation of this was the \CiriathSepher{} \quo{\hs{Dance}}.)
They are highly empathic and good at understanding emotions. 
But they are also vulnerable to social pressure, and their pride and fear of social rejection can be used against them by a clever manipulator. 

Their social intelligence allows the \resphain{} to form a highly organized and efficient society. 
This is probably their most important edge in their war against the \dragons: 
They are better organized than \hr{Dragons are not social}{less social \dragons}. 

\target{Resphan hypocrisy}
Their social nature also makes the \resphain{} very scheming and hypocritical, as opposed to the \dragons, who are \hr{Draconic sincerity}{more sincere}. 





\subsubsection{Taboos}
\target{Resphan age taboo}
The \resphain, \hr{Resphain are social}{being highly social beings}, are obsessed with etiquette and taboos. 
Some taboos that many \resphan{} cultures share include: 
\begin{itemize}
  \item 
    You do not talk about birth and bloodline directly. 
    At least, not your own, and not that of those of lower status. 
    You subcommunicate \quo{\ashenblood}, but you don't say it. 
    (It's OK to talk about those of higher blood, though.)
  \item 
    Likewise with age. 
    Age gives some kind of status, but it's the unspoken kind, not the kind you bring up in conversation. 
    (\hr{Age in Merkyrah}{This was different in \Merkyrah}.)
  \item 
    You do not praise yourself or call yourself by titles (such as \quo{\hr{Ketheran}{\ketheran}}). 
    But it is OK to praise others, especially those of higher status than yourself. 
    So a high-ranking \resphan{} will often keep a \quo{herald} around to introduce him with all his due pomp and ceremony. 
    \subitem
      The \Mystraacht{} do not have this rule. 
      They flaunt their greatness. 
      They revel in pride and violence. 
      This is actually one of the points of conflict between \Mystraacht{} (who \quo{embrace their nature}) and \KiriathSepher, who are sophisticated and foppish. 
\end{itemize}





\subsubsection{Wings and their role}
\target{Resphan wing body language}
Pureblood \resphain{} have wings. 
These play a major role in their body language. 

\begin{itemize}
  \item 
    Wings folded across the back (so the tips cross) is modest and polite. 
  \item 
    Wings hanging loose from the shoulders is casual. 
  \item 
    Wings spread wide signifies pride, or very-casualness. 
    It is rude except with inferiors or close friends. 
  \item 
    Wings half-spread and raised is threatening. 
\end{itemize}

They use special chairs with holes for wings, and special wide sofas with wing-rests so you can spread them wide (casual use only). 

\Resphan{} wings are stronger and more durable than they look. 
In combat they may be swung as weapons. 
In war a \resphan{} might wear bladed shins on the wings, and maybe even full metal plating over them. 
Compare to the illustrations for the album \bandalbum{Symphony X}{Paradise Lost}. 

\target{Resphain enjoy flying}
The \resphain{} enjoy flying. 
Where mortals, especially those of high status, often disdain walking and resort to riding on beasts or in carriages, \resphain{} love using their wings. 
The wings symbolize their superhuman status, and they are proud of their ability to fly. 
Especially because only purebloods have them. 

\target{Resphan wingspan is important}
Wingspan is an important aspect when measuring how attractive or impressive a \resphan{} looks. 
Just as important as height. 









\subsection{Servants and slaves}
The mortal underlings of a \resphan dynasty were called \hedrim, singular \hedor. 
The \hedrim[\Mystraacht], for example, were the \hedrim that served \Mystraacht. 

There are some sickly, degenerate \humans{} who work as the \ps{\resphain}{} slaves. 
They do menial labour in foundries, mines, the bowels of ships and the like, far out of the sight of their masters, who do not want to see these ugly wretches. 

There are also some \quo{elite} slaves, bred to be beautiful and healthy, and groomed and nurtured to preserve them. They serve the \ps{\resphain}{} immediate needs as manservants, butlers and sex slaves. 
These upper slaves sometimes command slaves of their own. 





\subsubsection{Livery}
\target{Resphan slave livery}
\CiriathSepher{} slaves were dressed nicely in clothes that identify their owner through heraldry and symbolism. 

\Mystraacht{} slaves, on the other hand, were often naked, wearing only a metal collar to mark their owner. 
The \Mystraacht{} were fond of very direct displays of dominance. 





\subsubsection{\Naorim}
\target{Resphan food slaves}
\target{Naor}
\index{\naor}
The \naorim{} (singular: \naor) are a high caste of slaves. 
They exist to be eaten by the \resphain. 

The food slaves are considered holy among other slaves and have very high status. 
They are religiously dedicated to what they see as a sacred duty to their living gods, and are happy to die and serve the \resphain{}, to be devoured with body and soul by their beloved masters. 

\target{Communion}
\index{Communion}
The \naorim{} are taught that being eaten is something sacred. 
It is ritualized and called the Communion. 
They are told that when they are eaten, they will \quo{become one} with their \resphan{} gods and live on forever inside the \resphain, inseparable from them in body and soul. 
This is sort of true\ldots{}

\target{carver}
\target{Gelveir}
\index{\gelveir}
\index{carver}
The Communion subject is killed by a specially trained \emph{carver} (a \human) with a special ritual dagger called a \gelveir. 

The Communion rite is symbolically based on \hr{Thanatzil must die}{\ps{\Thanatzil} sacrifice}: 
\quo{%
  The legendary Messiah of the \resphain{}, who gave up his life that his people might live.}

It is customary for the \naor{} chosen for Communion to sometimes be honoured with a special boon on her or his last night: 
To have sex with a \resphan{} or \resvil. 
This will be the slave's first and only sex ever, and therefore something special and precious.  

\target{epitaph}
\index{epitaph}
At the Communion, the \naor{} is expected to recite an epitaph, praising her masters and asking them to accept her sacrifice. 
By tradition, the \naor{} should compose her own epitaph, so it is personal and sincere and shows her personal dedication and love for her masters. 









\subsection{Skills and powers}





\subsubsection{Foreign languages}
\target{Resphain speak poor Draconic}
The \resphain \hr{Resphain learn Miithian languages}{tried to learn \Miithian languages when they came to \Tembrae}. 

Only very few \resphain ever mastered the \Draconic tongue. 
Partially because it was very hard, and partially because it was hard to find anyone willing to teach them. 
No \dragon ever taught a \resphan language. 
But the \resphain were able to find a few \quiljaaran who spoke \Draconic and could be persuaded or coerced into teaching. 
These \quiljaaran did not speak perfect \Draconic, though. 
And no \resphan ever learned \TrueDraconic. 






\subsubsection{Martial arts}
\target{Resphan martial arts}
\target{Paths}
\index{Paths}
The \resphain{} practiced a number of martial arts. 
The most common one was a system called the \quo{Paths}. 
It was a supernatural martial art, combining physical body training, weapon skill, psychometabolism and sorcery. 
As the plural name suggests, the \quo{Paths} was split into a number of specializations.

There were three Paths: Darkness, Ice and Light. 

Note that despite the similar terminology, the Darkness/Light distinction has nothing to do with the distiction of \iquin/Light and \itzach/Darkness. 

See also the section on \hr{Weapons}{weapons} for more martial arts. 


\begin{description}
  \item[\Rumicor, the Path of Darkness:]
    \target{Rumicor}
    \target{Path of Darkness}
    \index{Paths!Path of Darkness}
    Focused on stealth, misdirection, subterfuge, mental attacks, illusion and necromancy. 
    
    Favoured by \TiphredSerah. 
    
    
    
  \item[\Eshethicor, the Path of Ice:]
    \target{Eshethicor}
    \target{Path of Ice}
    \index{Paths!Path of Ice}
    Focused on control, perfection, accuracy and a cool overview. 
    
    Favoured by \CiriathSepher. 
    
    
    
  \item[\Shabacora, the Path of Light:]
    \target{Shabacora}
    \target{Path of Light}
    \index{Paths!Path of Light}
    Focused on fire, lightning, ferocity and all-out-attack. 
    
    Favoured by \Mystraacht{} (despite how \Mystraacht{} aesthetics otherwise had a very \quo{dark} theme).
\end{description}










\subsection{Vampirism and cannibalism}
\index{cannibalism!\resphain}
\target{Resphan vampirism}
\target{Resphan parasitism}
\target{Resphan cannibalism}
\target{Resphan diet}
\Resphain{} are vampiric creatures. They must consume the blood, flesh, \hr{Life drain}{life-force} and \hr{soul-eating}{souls} of other creatures in order to sustain themselves and hold at bay the devouring \hr{Entropy}{Entropy} within them, a curse which they inherited from their \SitraAchra sires.

The \hr{Merkyrah}{\Merkyrans} refused this, and that made them weak, easily overcome by the \hr{Resphan rebellion}{rebels}. 

Most \Resphain{} happily eat \human, \scatha{} or even \resphan{} flesh\dash or \dragon{} flesh, whenever they can get their hands on it. They gain life energy this way, especially if it's powerful creatures like other \resphain{} or \dragons. 

The \resphain{} are naturally cannibals who crave the flesh of their own kind. 
\Draconian{} blood may be more potent, but the most delicious treat a \resphan{} knows is the blood and flesh of another \resphan{}. 
The mightier the better. 
Ideally, a \hr{Sathariah}{\sathariah}.

Like their \SitraAchra sires, the \resphain{} can grow in power especially by consuming the flesh and souls other \resphain. 
That is \hr{Cannibal Malachim}{what the \malachim{} did}.

They can live on mortal souls for a while, but in order to survive in the long run they must consume other immortals. 
This means that the \resphain{} are a parasitic race who must always expand, lest they be forced to turn to feeding upon themselves. 
They are a destructive, invasive species. 
That is why the \feud{} \hr{Fighting for survival}{can never be resolved}. 
In the millennia after the \hs{Incursion}, the \resphain{} were unable to expand as much as they needed to, and their population dwindled as a result. 
The Cabalist dynasties waged wars against \Kezerad{} and \Baelzerach{} in order to sate their hunger for immortal souls. 




\subsubsection{\Bezedeth need less sustenance}
\Bezedeth were smaller and weaker than purebloods.
They had less lifeforce. 
This had the benefit that they did not need quite as much lifeforce to survive as purebloods. 
They could survive for a long time on mortal souls alone and only needed to consume immortal power once in a while or they would weaken.

Even if deprived of immortal sustenance for long, they would not die quickly but weaken, fall into torpor and slowly wither away over a course of decades.
Some would set themselves up as rulers of mortals\dash \hr{Resphan vampire lords}{immortal vampire lords}.





\subsubsection{Madness}
\target{Madness from eating Resphan flesh}
Eating \resphan flesh was dangerous. 
If the flesh was not prepared with the proper spells, eating it could lead to madness. 
Fragments of the eaten \resphan's soul could seep into the body of the eater. 
If these fragments were allowed to roam free they could cause great psychic harm before the mind's immune system beat them down.





\subsubsection{Power and hunger}
\target{Resphan power and hunger}
The mightier a \resphan{} is, the hungrier he is, and the more he must feed. Otherwise he risks losing power, permanently. 

The \Malachim{} do not lose power permanently if they fail to feed. That is one of their superpowers. 




\subsubsection{\Resphan{} captives}
The evil \resphain{} keep enemy \resphain{} captive to eat. 
They are bound with spells so they cannot escape, and then killed, eaten and allowed to reincarnate over and over. 

It's more nourishing to eat the soul, of course, but \resphan{} meat alone is pretty tasty, too. 





\subsubsection{Sexual connotations}
Some of them develop a \quo{vore} fetish, take a sexual delight in eating flesh. Some take this to masochistic extremes, letting their own flesh be eaten and then using magic to regrow it. (The magic to regrow limbs is extremely expensive luxury. A \resphan{} needs to sacrifice something like three \humans{} just to regenerate his little finger.) 

Occasionally, a \resphan{} will bloodlet himself and drink his own blood as a kind of masturbation.

















\section{\Satharioth}
\target{Satharioth}
\target{Sathariah}
The \satharioth{} are a group of exalted \resphan{} lords, empowered with \draconic{} blood \hr{Fall of Nexagglachel}{stolen from \Nexagglachel}. 

Ramiel and \Shiaraid{} belong to this group.










\subsection{Biology}
\subsubsection{Attempts to create more \satharioth}
Since the inception of the original \satharioth, there have been attempts at creating more of them. 
All have failed, for a number of reasons:

\begin{itemize}
  \item 
    The \satharioth{} and \banelords{} keep the technology a secret because they don't want competition. 
  \item
    To create new \satharioth{} you would need to consume an \uber-powerful \vertex{}. 
    \Nexagglachel{} was one of the greatest, most formidable \dragons{} that ever existed. 
    Few of his like exist today, and none of those are stupid enough to let themselves capture. 
\end{itemize}





\subsubsection{Fragments of \Nexagglachel}
\target{Fragments of Nexagglachel}
When he was killed, \hr{Nexagglachel}{\Nexagglachel} was symbolically dismembered, and each \sathariah{} consumed a part of his body: 

\begin{itemize}
  \item \Azraid: Brain.
  \item Ramiel: A claw. 
  \item \Shiaraid: \hr{Shiaraid's sexuality}{The part that let him endure pain}. 
  \item \Zachirah: Some sexual organ. 
\end{itemize}





\subsubsection{Immortality}
\target{Sathariah immortality}
The \satharioth did not age. 
They did not \hr{Resphan lifespan}{automatically accumulate decrepity like normal \resphain did}.
While their hunger for essence was great, it remained constant (or grew slowly in tandem with the individual's power).
It did not grow on its own. 
This meant that an old \sathariah could sustain himself on vastly less essence than a regular \resphan of similar age. 
So the \satharioth could live for many thousands of years. 









\subsection{Culture}





\subsubsection{Social status}
\target{Sathariah social status}
The \satharioth had very high social status. 
They were the top of the \resphan aristocracy in all dynasties (except \Baelzerach, which had no \satharioth). 

This has several reasons:
\begin{itemize}
  \item 
    The \resphain who became \satharioth were high-ranking, influential ones to begin with.
  \item 
    \Satharioth were physically, mentally and metaphysically powerful, which made it easy for them to gain political power as well.
  \item 
    The \resphain had an \hr{Races love war}{inborn psychological tendency to idolize the fierce and violent}. 
    The \satharioth \emph{embodied} the fierce and violent, so the \resphain loved them. 
\end{itemize}












\subsection{Demographics}
Originally there were eleven (\hr{Satharioth die}{surviving}) \satharioth{}, of which three were \resviel. 
\begin{gloss}
  \gitem{\KiriathSepher} (4)
  \begin{enumerate}
    \item \hr{Azraid}{\Azraid} (alive). 
    \item \hr{Harbeth}{\Harbeth} (alive). 
    \item \hr{Mehaloch}{\Mehaloch} (slain in the Incursion). 
    \item \hr{Morcariel}{\Morcariel} (slain in the Incursion). 
  \end{enumerate}
  \gitem{\Kezerad} (2)
  \begin{enumerate}
    \item \hr{Sithiyacaan}{\Sithiyacaan} (missing but alive). 
  \end{enumerate}
  \gitem{\Mystraacht} (4)
  \begin{enumerate}
    \item \hs{Ramiel}  (\Malach). 
    \item \hr{Shiaraid}{\Shiaraid} (\Malach). 
    \item \hr{Zachirah}{\Zachirah} (\hr{Zachirah dies}{slain by treachery} in the Incursion). 
  \end{enumerate}
  \gitem{\TiphredSerah} (3)
  \begin{enumerate}
    \item \hr{Dorzand}{\Dorzand} (?). 
    \item \hr{Ishicah}{\Ishicah} (\Malach, \hr{Ishicah enslaved}{destroyed in \Ortaican{} captivity}). 
    \item \hr{Quelthah}{\Quelthah} (\hr{Quelthah dies}{slain by \Ishnaruchaefir} in the Incursion). 
  \end{enumerate}
\end{gloss}









\subsection{Physique}
The \satharioth{} are terrible to behold. 

\lyricstitle{\SETolltheHounds{} p.108}{
  Through the gate\ldots{} he saw \emph{him} [Anomander Rake] approaching. 
  From the below city. 
  His forearms sheathed in black glistening scales, his bared chest made a thing of natural \armour. 
  The blood of Tiam ran riot through, fired to life by the conflation of chaotic sorcery, and his eyes flowed with ferocious will. 
}











\subsection{Psychology}





\subsubsection{\NexagglachelsCurse}
%\subsubsection{Hatred of the \banes}
\target{Curse}
\target{Nexagglachel's curse}
\target{Satharioth hate Banes}

\Nexagglachel \hr{Nexagglachel lives on in Satharioth}{did not perish but lived on inside the \satharioth}. 

The \satharioth{} and \ketherain{} harboured a deep-seated, irrational hatred of their \bane{} masters. 
The reason for this was twofold:

\begin{enumerate}
  \item 
    It was a a part of their heritage from the \banes, who were \hr{Bane cannibalism}{cannibalistic and patricidal by nature}. 
  \item
    It was amplified by the soul of \Nexagglachel. 
    He was \hr{Nexagglachel lives on in Satharioth}{dead, but he lived on in some form} in the blood of the \satharioth. 
    It is hinted that he \hr{Nexagglachel sacrifices himself}{willingly sacrificed himself} in order to \hr{Nexagglachel makes Satharioth hate Banes}{sow hatred} between the \resphain{} and their \bane{} progenitors.
\end{enumerate}

This hatred made the \satharioth{} \hr{Satharioth betray Banes}{betray the \banes} and other \resphain{} several times. 

The \resphain{} still fear \Nexagglachel{} and his corrupting influence. 
Perhaps they perform religious rituals to keep him at bay, so they can freely use his power. 
Compare to the Egyptian rituals used to pacify the \dragon{} Apep.





\subsubsection{Perhaps \Nexagglachel{} is \ps{\Daggerrain}{} blind spot}
\target{Nexagglachel is Daggerrain's blind spot}
But \ps{\Nexagglachel}{} ghost is clever, and he eludes and manipulates them. 
Perhaps he is \hr{Daggerrain's blind spot}{\ps{\Daggerrain}{} blind spot}, the factor that \Daggerrain{} never managed to understand and enter into his calculations. 
Or perhaps the entire \feud{} is actually very much a contest of sneakiness between \Nexagglachel{} and \Daggerrain, with the \dragonlord{} hiding from \Daggerrain, using the knowledge of the \banes{} gained from merging with the \resphan{} soul to hide from the \banelord{}, gain insight into his plans and subtly subvert them. 
This may also be what eventually allows \hr{Ramiel betrays Banes}{Ramiel to betray \Daggerrain{} and overthrow him}.





\subsubsection{The curse causes them to descend into madness}
\target{Madness of the Curse}
The curse causes the \satharioth{} to slowly descend into madness and dementia. 

\lyricslimbonicart{Dynasty of Death}{
  In the dark caves of oblivion\\
  bad blood rises from the Mega Therion [\Nexagglachel].\\
  From vast stalactite halls so undivine,\\
  through the misty corridors of time.\\
  As one lives one shall die.\\
  Ad noctum.
}

\lyricslimbonicart{The Yawning Abyss of Madness}{
  Again I drift the halls of wondering. \\
  The black castle of solitude. \\
  On the very edge of sanity \\
  in mental cryogenic interludes. \\
  I have slipped into the seventh, \\
  the seventh circle of Hell, \\
  in realms where deadly shadows \\
  infest every cell. 
  
  Internal ceremonies in ritual death. \\
  External bleedings for the demon of madness. \\
  Hide from the torture of the dazzling light. \\
  The demolition voice shall speak tonight. 
  
  While I'm staring down into the darkest pit, \\
  an ocean black as the night, \\
  so infinite deep and consuming, \\
  it swallows all life force with might. 
  
  Again I drift the halls of wondering, \\
  as I focus for the darkness to come. \\
  In anguish minds uplift the conquering \\
  to cross the line of death beyond.
  
  An abstract reality and bottomless insanity. \\
  To search for the powers to please\\
  the subconscious spirit of disease. \\
  Time found no remedy, \\
  cause winds of darkness was stealing me. 
  
  The yawning abyss of madness. \\
  A cryptic slaughter by hate. \\
  Darkness is the only survivor \\
  as evil dominion terminates. \\
  The yawning abyss of madness.
}

Some of them despair at the curse. 

\target{Satharioth despair at the curse}
\lyricsbs{Emperor}{Grey}{
  where lights are dim and shades of black are grey\\
  from the moment of arrival we are led astray\\
  with nothing but a distant cry from deep within a soul\\
  a wordless voice to guide us on the way
  
  desperately we name the voice and make the cries our own\\
  as if to deny the fact that we are all alone\\
  in solitude we mingle, disillusioned we fall prey\\
  where lights are dim and shades of black are grey\\
  I can always live another day\ldots{}
}





\subsubsection{Dreaming of \ophidian{} eyes}
\target{Satharioth dream of Ophidian eyes}
The \satharioth dreamt of \ophidian eyes. 
Frozen eyes. 
Eyes of black ice. 
Full of cold hatred. 

They were the eyes of \Nexagglachel, staring cold at them. 
Whenever they closed their eyes they found him staring back at them, and they remembered \hr{Nexagglachel's hatred}{his promise to destroy them}. 




\subsubsection{The curse lives on}
\target{Curse lives on}
Even after \ps{\Nexagglachel} short-term goal is accomplished and \hr{Daggerrain falls}{\Daggerrain{} falls}, the curse lives on. 
\Nexagglachel{} still hates the \satharioth{} and all the \resphan{} people, and still wants to see them suffer and destroy themselves. 















\section{\Tarcharos}
\target{Tarcharos}
\index{\Tarcharos}
\Tarcharos was the first great \resphan empire, before even \hr{Merkyrah}{\Merkyrah}.
It was ruled by evil sorcerers who summoned forth dark powers. 
The capital of \Tarcharos was \hr{Hyardes}{\Hyardes Hold}. 









\subsection{History}
\target{Resphain invoked Umbra gods}
Finally the sorcerer-lords of \Tarcharos attempted to make pacts with, or even control, the fell \hr{Umbra}{\umbrae} of the deep. 
In their folly they called upon \Norganthus and \Tzyaragoth, hoping to use their power against their enemies. 

Their sinister plan was foiled by the sacrifice of a great number of valiant heroes.
In the end, the sorcerers' summoning backfired.
The black gods of madness were freed from the deeps for a short time and caused a terrible catastrophe.
Many towers were laid waste, including a vast and glorious citadel. 
This laid waste to \Hyardes, the vast and glorious capital hold of \Tarcharos, and created the Chasm of \hr{Oggra}{\Oggra}.









\subsection{Geography}





\subsubsection{The Chasm of \Oggra}
\target{Oggra}
\index{\Oggra}
The Chasm of \Oggra was a vast abyss that cut through the \resphan-inhabited cities of \Nyx (the areas that were once \Merkyrah). 
It was an empty area with no towers in it.
The mists in this are were darker and more menacing than elsewhere, and the deep somehow seemed even deeper beneath it (hence the name \quo{Chasm}). 
The place was feared and shunned by the \resphain. 
It was thousands of metres across. 





\subsubsection{\Hyardes Hold}
\target{Carcosa}
\target{Hyardes}
\index{\Carcosa}
\index{\Hyardes}
In the middle of the Chasm of \Oggra lay \Carcosa, also called the Towers of \Hyardes or the \Hyardes Towers. 
This was a collection of especially exotic and ghastly-looking towers. 
It was once the glorious capital of \Tarcharos.
It was laid waste in the fall of \Tarcharos.

It was known to all \resphain that the \resphan race had originated in \Tarcharos, although the exact tower was unknown. 
(It was really \hr{Jazerubel}{\Jazerubel}.)

No \resphain dwelt here after the fall of \Tarcharos. 
The spires of \Carcosa were extremely tall, appearing to tower over all other spires (although it was hard to tell). 
Eerily, the towers seemed to rise \emph{behind} the \hs{black stars} rather than in front of them. 

\Umbrae were often spotted in \Oggra, especially near \Carcosa.
Some believed that the \umbrae were spawned deep below in \Oggra or in the lower levels of \Carcosa. 















\section{\TiphredSerah}
A \resphan{} faction loyal to the \banes. 









\subsection{Aesthetics}

The \TiphredSerah{} often dressed like vampires, with robes and high collars. 
Their traditional \colours were blues, violets and purples.

The \TiphredSerah{} had a tradition for lacquering their fingernails in their dynasty \colours. 









\subsection{Culture}
The \TiphredSerah{} were more introverted and secretive than the other dynasties. 
They were schemers, in a colder, more rational, more long-term-thinking way than the \CiriathSepher{} with their petty intrigues or the \Mystraacht{} with their open and brutal animosity. 

They were sneaky, ninja-like guys. 

\target{Tiphred-Serah free-thinking}
They were also the most free-thinking of the dynasties. 

One could argue that they were nerds. 





\subsubsection{Government}
Their rulers were unknown, but they were represented by a Speaker. 
Several of their leaders were \resviel. 









\subsection{History}





\subsubsection{\ResphanWars}
\TiphredSerah \hr{TiphredSerah in the Resphan Wars}{got their butts kicked} in the \resphanwars. 





\subsubsection{Inception of the Cabal}
It was people from \TiphredSerah{} who \hr{Tiphred-Serah form Cabal}{founded the Cabal} and \hr{Tiphred-Serah formulate Unspoken Covenant}{helped formulate} the \hs{Unspoken Covenant}. 









\subsection{Politics}





\subsubsection{\Kezerad}
During the \hr{Resphan Wars}{\resphanwars} the \TiphredSerah{}, being \hr{Tiphred-Serah free-thinking}{the most free-thinking of the dynasties}, were the ones \hr{Tiphred-Serah and Kezerad ally}{most liable to ally with \Kezerad}. 

























